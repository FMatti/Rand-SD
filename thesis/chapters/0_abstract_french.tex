\chapter*{Résumé}
\label{chp:0-resume}

Nous développons une famille de méthodes utilisées pour calculer approximativement
la distribution des valeurs propres -- la densité spectrale -- de grandes matrices symétriques. Ces méthodes sont basées
sur l'expansion polynomiale d'une fonction lisse en combinaison avec une estimation probabiliste
de la trace, une factorisation probabiliste de rang bas, ou les deux simultanément.\\

D'abord, nous présentons une introduction et une vue d'ensemble des procédures
utilisées dans la littérature pour calculer la densité spectrale des matrices. Ensuite, nous montrons
comment certaines fonctions matricielles peuvent être calculées efficacement sur
la base de leur expansion de Chebyshev en calculant les transformées en cosinus
discrètes. L'utilisation de l'estimateur stochastique de trace de Hutchinson
mène à un premier algorithme d'approximation de la densité spectrale : la
méthode Delta-Gauss-Chebyshev. Ensuite, une analyse de la structure de la
fonction matricielle impliquée dans le calcul motive l'utilisation d'une factorisation de
Nystr\"om de rang bas pour réduire la dimensionnalité du problème, ce qui
nous conduit à la méthode de Nystr\"om-Chebychev. Pour contourner l'inefficacité
de l'une et le manque de robustesse de l'autre méthode, elles sont combinées en
un troisième algorithme appelé méthode Nystr\"om-Chebychev++.\\

Les techniques employées dans ces méthodes sont motivées et introduites de manière
rigoureuse. Nous présentons plusieurs stratégies d'implémentation pour améliorer
la vitesse de calcul, la précision et la stabilité, et nous donnons une analyse
théorique de chaque méthode. Dans diverses expériences, l'analyse de nos algorithmes
est confirmée numériquement et leur efficacité est comparée à d'autres méthodes
conventionnelles.\\

\textbf{Mots-clés :} Densité spectrale, expansion de Chebyshev,
transformée en cosinus discrète, estimation probabiliste de trace,
Hutchinson's estimator, factorisation probabiliste de rang bas, approximation de Nystr\"om,
fonctions matricielles, matrices dépendantes des paramètres

