\chapter*{Résumé}
\label{chp:0-resume}

Nous étudions une famille de méthodes utilisées pour calculer approximativement
la densité spectrale de grandes matrices symétriques. Ces méthodes sont basées
sur l'expansion polynomiale d'une fonction lisse en combinaison avec une estimation randomisé
de la trace, une approximation randomisé de bas rang, ou les deux simultanément.\\

D'abord, nous présentons une introduction et un aperçu des procédures
utilisées pour calculer la densité spectrale des matrices. Ensuite, nous montrons
comment certaines fonctions matricielles peuvent être calculées efficacement sur
la base de leur expansion de Chebyshev -- le schéma d'interpolation de choix pour
nos méthodes. Un premier algorithme est proposé, basé sur l'estimateur stochastique
de trace de Hutchinson. Ensuite, l'utilisation d'une approximation de bas rang
est motivée et un second algorithme basé sur l'approximationde Nystr\"om est présenté.
Enfin, une méthode hybride basée sur une combinaison des deux techniques mentionnées
précédemment est discutée.\\

Les techniques employées dans ces méthodes sont motivées et introduites de manière
rigoureuse. Nous présentons plusieurs stratégies d'implémentation pour améliorer
la vitesse de calcul, la précision et la stabilité, et nous donnons une analyse
théorique de chaque méthode. Dans diverses expériences, l'analyse de nos algorithmes
est confirmée numériquement et leur efficacité est comparée à d'autres méthodes
conventionnelles.
