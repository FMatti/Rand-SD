\chapter{Variance-reduced trace estimation}
\label{chp:4-nystromchebyshev}

In \refsec{sec:2-chebyshev-stochastic-trace-estimation} we have introduced the
classical approach for estimating the trace of a matrix using matrix-vector
products. A drawback of this method is the rather slow, reciprocal decrease of
the variance with \glsfirst{num-hutchinson-queries}. Variance-reduced trace estimators
try to improve on this. They usually take a \enquote{hybrid} approach, combining
low-rank factorization with stochastic trace estimation.\\

This chapter will take the methods we have discussed in \refchp{chp:2-chebyshev}
and \refchp{chp:3-nystrom} combine them to an improved trace estimator which
we call the \glsfirst{NCPP} method.

%%%%%%%%%%%%%%%%%%%%%%%%%%%%%%%%%%%%%%%%%%%%%%%%%%%%%%%%%%%%%%%%%%%%%%%%%%%%%%%%

\section{Fundamentals of variance-reduced trace estimation}
\label{sec:4-nystromchebyshev-hybrid}

The initial idea from \cite{lin2017randomized} was to perform a low-rank
factorization before the stochastic trace estimation to achieve a
steeper decrease of the estimation error as the number of computed
matrix-vector products increase. Recent theoretical developments confirm the
efficacy of this approach \cite{meyer2021hutch,persson2022hutch}.
We will first briefly discuss these results which are valid 
for constant matrices.\\

\subsection{Constant matrices}
\label{subsec:4-nystromchebyshev-reduction-constant-matrices}

Due to the linearity of the trace, we can decompose the trace of any symmetric
matrix $\mtx{B} \in \mathbb{R}^{n \times n}$ into two parts using another matrix
$\widehat{\mtx{B}} \in \mathbb{R}^{n \times n}$
\begin{equation}
    \Tr(\mtx{B}) = \Tr(\widehat{\mtx{B}}) + \Tr(\mtx{B} - \widehat{\mtx{B}}).
    \label{equ:4-nystromchebyshev-trace-decomposition}
\end{equation}
If we manage to find a matrix $\widehat{\mtx{B}}$, such that the trace of
$\widehat{\mtx{B}}$ can be computed efficiently and
$\lVert \mtx{B} - \widehat{\mtx{B}} \rVert _F$ is small,
then we can compute the first term in \refequ{equ:4-nystromchebyshev-trace-decomposition}
exactly and the second term can be approximated well using
the \gls{num-hutchinson-queries}-query Girard-Hutchinson
trace estimator $\Hutch_{n_{\Psi}}$ \refequ{equ:2-chebyshev-DGC-hutchionson-estimator}
due to its small Frobenius norm \refequ{equ:2-chebyshev-hutchinson-mse}.
That is,
\begin{equation}
    \Tr^{++}(\mtx{B}) = \Tr(\widehat{\mtx{B}}) + \Hutch_{n_{\Psi}}(\mtx{\Delta}),
    \label{equ:4-nystromchebyshev-hutch-pp}
\end{equation}
with the residual $\mtx{\Delta} = \mtx{B} - \widehat{\mtx{B}}$, will be an excellent
approximation of $\Tr(\mtx{B})$.\\

In \refsec{subsec:3-nystrom-factorization-constant-matrices} we have introduced
some ways in which a matrix $\widehat{\mtx{B}}$ satisfying the above mentioned
criteria can be constructed. For instance, for a symmetric \gls{PSD} matrix $\mtx{B} \in \mathbb{R}^{n \times n}$
of rank $r \ll n$, we can compute a factorization of the form
$\mtx{B} = \mtx{V}_1 \mtx{\Sigma}_1 \mtx{V}_1^{\top}$ \refequ{equ:3-nystrom-eigenvalue-decoposition}.
Since $\mtx{V}_1$ is orthonormal, $\Tr(\mtx{B}) = \Tr(\mtx{\Sigma}_1)$ by the
cyclic property of the trace, which is easy to compute. On the other hand,
$\mtx{\Delta} = \mtx{B} - \widehat{\mtx{B}} = \mtx{0}$ which implies that the
Girard-Hutchinson estimator will trivially be exact. Thus, \refequ{equ:4-nystromchebyshev-hutch-pp}
would yield an exact estimate. However, this factorization is -- in general -- prohibitively expensive
to compute.\\

This is where randomized low-rank factorizations come into play.
One such example is a factorization of the form \refequ{equ:3-nystrom-RSVD}
which is employed in the Hutch++ algorithm \cite[algorithm~1]{meyer2021hutch}.
For \gls{PSD} matrices this algorithm was shown to give
an estimate whose relative deviation from the actual trace is at most
$\varepsilon$ by only using $\mathcal{O}(\varepsilon^{-1})$ matrix-vector multiplications,
with high probability. The key to this conclusion is the following theorem \cite[theorem~1]{meyer2021hutch}.
\begin{theorem}{Variance-reduced trace estimation}{4-nystromchebyshev-trace-correction}
    Suppose $\mtx{B} \in \mathbb{R}^{n \times n}$ is symmetric \gls{PSD}. Let $\widehat{\mtx{B}}$ and
    $\mtx{\Delta}$ be any matrices such that
    \begin{equation}
        \begin{cases}
            \Tr(\mtx{B}) = \Tr(\widehat{\mtx{B}}) + \Tr(\mtx{\Delta}), \\
            \lVert \mtx{\Delta} \rVert _F \leq 2 \lVert \mtx{B} - \mtx{B}_{n_{\Omega}} \rVert _F.
        \end{cases}
    \end{equation}
    where $\mtx{B}_{n_{\Omega}}$ is the best rank-$n_{\Omega}$ approximation to $\mtx{B}$.
    For fixed constants $c_1, c_2 > 0$, if \gls{num-hutchinson-queries} $> c_1\log(1/\delta)$, then with probability $\geq 1 - \delta$,
    \begin{equation}
        |\Tr(\mtx{B}) - \Tr^{++}(\mtx{B})| \leq 2 c_2 \sqrt{\frac{\log(1/\delta)}{n_{\Psi} n_{\Omega}}} \Tr(\mtx{B}).
    \end{equation}
    In particular, if $n_{\Omega}=n_{\Psi}=\mathcal{O}\left( \sqrt{\log(1/\delta)}/ \varepsilon + \log(1/\delta) \right)$, $\Tr^{++}(\mtx{B})$ is a $(1 \pm \varepsilon)$ error approximation to $\Tr(\mtx{B})$.
\end{theorem}

In \cite{meyer2021hutch} this theorem is applied to a low-rank factorization of
the form \refequ{equ:3-nystrom-RSVD} to prove the $\mathcal{O}(\varepsilon^{-1})$-dependence
requirement on the number of matrix-vector multiplications to achieve the relative
approximation error $\varepsilon$, which lead to the ubiquitous Hutch++ algorithm.
\cite{lin2017randomized} uses and \cite{persson2022hutch} refines an analogous procedure for the
case of the Nystr\"om approximation \refequ{equ:3-nystrom-nystrom}. In \cite[theorem~3.4]{persson2022hutch}
it is shown that similarly to the Hutch++ algorithm, the following result holds:
\begin{theorem}{Error of the Nystr\"om++ trace estimator}{4-nystromchebyshev-nystrom-pp}
    If the trace estimator \refequ{equ:4-nystromchebyshev-hutch-pp} based
    on the Nystr\"om approximation \refequ{equ:3-nystrom-nystrom} is computed
    with \gls{sketch-size} $=$ \gls{num-hutchinson-queries} $= \mathcal{O}(\sqrt{\log(1/\delta)}/\varepsilon + \log(1/\delta))$
    and $\delta \in (0, 1/2)$, then
    \begin{equation}
        |\Tr(\mtx{B}) - \Tr^{++}(\mtx{B})| \leq \varepsilon |\Tr(\mtx{B})|
    \end{equation}
    holds with probability $\geq 1-\delta$.
\end{theorem}

\subsection{Parameter-dependent matrices}
\label{subsec:4-nystromchebyshev-reduction-parametrized-matrices}

Similarly to the techniques introduced in \refchp{chp:2-chebyshev} and \refchp{chp:3-nystrom},
also the variance-reduced trace estimation can be extended to the case where
the trace of a matrix $\mtx{B}(t) \in \mathbb{R}^{n \times n}$ which depends continuously on a parameter $t$
needs to be computed.\\

With a matrix $\widehat{\mtx{B}}(t)$, the residual $\mtx{\Delta}(t) = \mtx{B}(t) - \widehat{\mtx{B}}(t)$,
and the parameter-dependent \gls{num-hutchinson-queries}-query Girard-Hutchinson estimator $\Hutch_{n_{\Psi}}$
\refequ{equ:2-chebyshev-DGC-hutchionson-estimator-parameter}, we define the estimator 
\begin{equation}
    \Tr^{++}(\mtx{B}(t)) = \Tr(\widehat{\mtx{B}}(t)) + \Hutch_{n_{\Psi}}(\mtx{\Delta}(t)).
    \label{equ:4-nystromchebyshev-parameter-hutch-pp}
\end{equation}
Here, the dependence of $\widehat{\mtx{B}}(t)$, and therefore also $\mtx{\Delta}(t)$,
on a certain number \gls{sketch-size} $\in \mathbb{N}$, usually the \glsfirst{sketch-size}
of the \glsfirst{sketching-matrix}, is implicitly assumed.
Using the result from \reflem{lem:2-chebyshev-parameter-hutchinson},
we can -- under certain conditions -- derive an analogous result
to \refthm{thm:4-nystromchebyshev-nystrom-pp} for any trace estimate of the form
\refequ{equ:4-nystromchebyshev-hutch-pp} in the parameter-dependent case.

\begin{theorem}{Variance-reduced parameter-dependent trace estimation}{4-nystromchebyshev-trace-correction-parameter-dependent}
    Suppose $\mtx{B}(t) \in \mathbb{R}^{n \times n}$ is symmetric \gls{PSD}
    and continuous in $t \in [a, b]$. Let $\widehat{\mtx{B}}(t)$ and
    $\mtx{\Delta}(t)$ be any
    symmetric \gls{PSD} and continuous matrices such that
    \begin{equation}
        \begin{cases}
            \Tr(\mtx{B}(t)) = \Tr(\widehat{\mtx{B}}(t)) + \Tr(\mtx{\Delta}(t)); \\
            \int_{a}^{b} \lVert \mtx{\Delta}(t) \rVert _F \mathrm{d}t \leq c_{\Omega} \frac{1}{\sqrt{n_{\Omega}}} \int_{a}^{b} \Tr(\mtx{B}(t)) \mathrm{d}t.
        \end{cases}
    \end{equation}
    for some constant $c_{\Omega} \geq 0$. Then, for a fixed constant $c \geq 0$ and $\delta \in (0, e^{-1})$, with probability $\geq 1 - \delta$,
    \begin{equation}
        \int_{a}^{b} |\Tr(\mtx{B}(t)) - \Tr^{++}(\mtx{B}(t))| \mathrm{d}t \leq c \frac{\log(1/\delta)}{\sqrt{n_{\Psi} n_{\Omega}}} \int_{a}^{b} \Tr(\mtx{B}(t)) \mathrm{d}t.
    \end{equation}
    In particular, if \gls{sketch-size} $=$ \gls{num-hutchinson-queries} $=\mathcal{O}\left( \log(1/\delta) / \varepsilon \right)$,
    then $\Tr^{++}(\mtx{B}(t))$ is an $\varepsilon$-error approximation of $\Tr(\mtx{B}(t))$
    in the relative $L^1$-norm.
\end{theorem}

\begin{proof}
    We may directly bound
    \begin{align*}
        &\int_{a}^{b} |\Tr(\mtx{B}(t)) - \Tr^{++}(\mtx{B}(t))| \mathrm{d}t \notag \\
        &= \int_{a}^{b} |\Tr(\mtx{\Delta}(t)) - \Hutch_{n_{\Psi}}(\mtx{\Delta}(t))| \mathrm{d}t && \text{(definition of $\Tr^{++}$ and linearity of $\Tr$)}  \notag \\
        &\leq c_{\Psi} \frac{\log(1/\delta)}{\sqrt{n_{\Psi}}} \int_{a}^{b} \lVert \mtx{\Delta}(t) \rVert _F \mathrm{d}t && \text{(using \reflem{lem:2-chebyshev-parameter-hutchinson})} \notag \\
        &= c_{\Psi} c_{\Omega} \frac{\log(1/\delta)}{\sqrt{n_{\Psi} n_{\Omega}}} \int_{a}^{b} \Tr(\mtx{B}(t)) \mathrm{d}t && \text{(assumption on $\mtx{\Delta}(t)$)} \notag \\
    \end{align*}
    Identifying $c=c_{\Psi} c_{\Omega}$, we get the desired result with probability $\geq 1 - \delta$.
\end{proof}

When compared to constant matrices (\refthm{thm:4-nystromchebyshev-trace-correction}),
the parameter-dependent case (\refthm{thm:4-nystromchebyshev-trace-correction-parameter-dependent})
requires the residual $\mtx{\Delta}(t)$ to be bounded by
(\gls{sketch-size})$^{-1/2}$ times the $L^1$-norm of the
trace of $\mtx{B}(t)$, instead of the best approximation error.
We also require an additional factor of $\sqrt{\log(1/\delta)}$ when choosing
\gls{sketch-size} and \gls{num-hutchinson-queries} such that we can achieve
an $\varepsilon$ error.\\

For the parameter-dependent Nystr\"om approximation in particular \refequ{equ:3-nystrom-nystrom-parameter},
we can show that it is an approximation which satisfies the conditions
of \refthm{thm:4-nystromchebyshev-trace-correction-parameter-dependent}. The key is to use the following
lemma from \cite{he2023parameter} which guarantees the desired
convergence property of the parameter-dependent Nystr\"om approximation.

\begin{theorem}{$L^1$-error of the parameter-dependent Nystr\"om++ trace estimator}{4-nystromchebyshev-final}
    The parameter-dependent Nystr\"om++ computed with
    $n_{\Omega} = n_{\Psi} = \mathcal{O}\left( \log(1/\delta) / \varepsilon \right)$
    and even $n_{\Omega} \geq 8 \log(1/\delta)$,
    satisfies for any symmetric \gls{PSD} matrix $\mtx{B}(t) \in \mathbb{R}^{n \times n}$ which continuously
    depends on $t \in [a, b]$, and $\delta \in (0, e^{-1})$, with probability
    $\geq 1 - \delta$
    \begin{equation}
        \int_{a}^{b} |\Tr(\mtx{B}(t)) - \Tr^{++}(\mtx{B}(t))| \mathrm{d}t \leq \varepsilon \int_{a}^{b}\Tr(\mtx{B}(t)) \mathrm{d}t
    \end{equation}
\end{theorem}

\begin{proof}
    According to \refthm{thm:4-nystromchebyshev-trace-correction-parameter-dependent}
    it is enough to verify that, with the linearity of the trace, the Nystr\"om
    approximation $\widehat{\mtx{B}}(t)$ of $\mtx{B}(t)$ satisfies 
    \begin{equation}
        \Tr(\mtx{B}(t)) = \Tr(\widehat{\mtx{B}}(t) + \mtx{B}(t) - \widehat{\mtx{B}}(t)) = \Tr(\widehat{\mtx{B}}(t)) + \Tr(\mtx{\Delta}(t))
    \end{equation}
    for all $t \in [a, b]$ and with probability $\geq 1 - \delta$
    \begin{equation}
        \int_{a}^{b} \lVert \mtx{\Delta}(t) \rVert _F \mathrm{d}t = \int_{a}^{b} \lVert \mtx{B}(t) - \widehat{\mtx{B}}(t) \rVert _F \mathrm{d}t \leq c_{\Omega} \frac{1}{\sqrt{n_{\Omega}}} \int_{a}^{b} \Tr(\mtx{B}(t)) \mathrm{d}t.
    \end{equation}
    The latter follows from \reflem{lem:3-nystrom-parameter-nystrom}. Finally,
    the choice of \gls{sketch-size} and \gls{num-hutchinson-queries} follows
    from \refthm{thm:4-nystromchebyshev-trace-correction-parameter-dependent}
    and \reflem{lem:3-nystrom-parameter-nystrom}.
\end{proof}

Compared to constant matrices (\refthm{thm:4-nystromchebyshev-nystrom-pp}),
we require an additional factor of $\sqrt{\log(1/\delta)}$ in the choice of
\gls{sketch-size} and \gls{num-hutchinson-queries}, and the choice of $\delta$
is slightly more restricted.

%%%%%%%%%%%%%%%%%%%%%%%%%%%%%%%%%%%%%%%%%%%%%%%%%%%%%%%%%%%%%%%%%%%%%%%%%%%%%%%%

\section{The Nystr\"om-Chebyshev++ method}
\label{sec:4-nystromchebyshev-nystromchebyshev-pp}

Taking the algorithmic developments from \refchp{chp:2-chebyshev} and \refchp{chp:3-nystrom},
we can easily combine them to a powerful hybrid method which we call the \glsfirst{NCPP}
method.

As mentioned at the start of this chapter, this method improves
upon the \gls{NC} method by correcting it with an estimate of the trace of the
residual $\mtx{\Delta}(t) = g_{\sigma}^{(m)}(t\mtx{I}_n - \mtx{A}) - \widehat{g}_{\sigma}^{(m)}(t\mtx{I}_n - \mtx{A})$
of the Nystr\"om approximation. To this purpose, the \gls{num-hutchinson-queries}-query
Girard-Hutchinson estimator $\Hutch_{n_{\Psi}}$
\refequ{equ:2-chebyshev-DGC-hutchionson-estimator} is employed as follows:
\begin{align}
    \breve{\phi}_{\sigma}^{(m)}(t)
    %&= \Tr(\widehat{g}_{\sigma}^{(m)}(t\mtx{I}_n - \mtx{A}))
    %- \Hutch_{n_{\Psi}}(\underbrace{g_{\sigma}^{(m)}(t\mtx{I}_n - \mtx{A}) + \widehat{g}_{\sigma}^{(m)}(t\mtx{I}_n - \mtx{A})}_{=\mtx{\Delta}(t)}) \notag \\
    &= \Tr(\widehat{g}_{\sigma}^{(m)}(t\mtx{I}_n - \mtx{A}))
    + \Hutch_{n_{\Psi}}(\mtx{\Delta}(t)) \notag \\
    &= \Tr(\widehat{g}_{\sigma}^{(m)}(t\mtx{I}_n - \mtx{A}))
    + \Hutch_{n_{\Psi}}(g_{\sigma}^{(m)}(t\mtx{I}_n - \mtx{A}) - \widehat{g}_{\sigma}^{(m)}(t\mtx{I}_n - \mtx{A})) \notag \\
    &= \underbrace{\Tr(\widehat{g}_{\sigma}^{(m)}(t\mtx{I}_n - \mtx{A}))}_{=\widehat{\phi}_{\sigma}^{(m)}(t)}
    + \underbrace{\Hutch_{n_{\Psi}}(g_{\sigma}^{(m)}(t\mtx{I}_n - \mtx{A}))}_{=\widetilde{\phi}_{\sigma}^{(m)}(t)}
    - \Hutch_{n_{\Psi}}(\widehat{g}_{\sigma}^{(m)}(t\mtx{I}_n - \mtx{A})).
    \label{equ:4-nystromchebyshev-spectral-density-decomposition}
\end{align}
It turns out that two of the three terms appearing in \refequ{equ:4-nystromchebyshev-spectral-density-decomposition}
are already quite familiar to us: In \refchp{chp:2-chebyshev} we have seen
how to compute $\widetilde{\phi}_{\sigma}^{(m)}$ with the \gls{DGC} method, whereas
in \refchp{chp:3-nystrom}, $\widehat{\phi}_{\sigma}^{(m)}$ is computed with the \gls{NC}
method. Only the last term is new. Using the standard Gaussian \glsfirst{random-matrix}
from the \gls{DGC} method, the \glsfirst{sketching-matrix} from the \gls{NC} method,
and the definition of the Nystr\"om approximation
$\widehat{g}_{\sigma}^{(m)}(t\mtx{I}_n - \mtx{A})$ \refequ{equ:3-nystrom-nystrom-smoothing-kernel},
we may rewrite it as
\begin{align*}
    &\Hutch_{n_{\Psi}}(\widehat{g}_{\sigma}^{(m)}(t\mtx{I}_n - \mtx{A}))\\
    &=\frac{1}{n_{\Psi}}\Tr\big(
        (\underbrace{\mtx{\Psi}^{\top} g_{\sigma}^{(m)}(t\mtx{I}_n - \mtx{A}) \mtx{\Omega}}_{=\mtx{L}_1(t)^{\top}})
        (\underbrace{\mtx{\Omega}^{\top} g_{\sigma}^{(m)}(t\mtx{I}_n - \mtx{A}) \mtx{\Omega}}_{=\mtx{K}_1(t)})^{\dagger}
        (\underbrace{\mtx{\Omega}^{\top} g_{\sigma}^{(m)}(t\mtx{I}_n - \mtx{A}) \mtx{\Psi}}_{=\mtx{L}_1(t)})
    \big).
\end{align*}
Notice how the involved matrices $\mtx{L}_1(t) \in \mathbb{R}^{n_{\Omega} \times n_{\Psi}}$ and $\mtx{K}_1(t) \in \mathbb{R}^{n_{\Omega} \times n_{\Omega}}$
again have a form in which they can be expanded efficiently and for all $t$ simultaneously,
as we have seen in \refsec{sec:2-chebyshev-delta-gauss-chebyshev}.\\

The implementation of this new method is similar to the \gls{DGC} and \gls{NC} methods (\refalg{alg:2-chebyshev-DGC} and \refalg{alg:3-nystrom-nystrom-chebyshev}).
Although from \refequ{equ:4-nystromchebyshev-spectral-density-decomposition} we
could see that the result of the \glsfirst{NCPP} method could be obtained by
combining the results of the \gls{NC} and \gls{DGC} methods, doing so is not
efficient in practice. The pseudocode for the \glsfirst{NCPP} method is given
in \refalg{alg:4-nystromchebyshev-nystrom-chebyshev-pp}.

\begin{algo}{Nystr\"om-Chebyshev++ method}{4-nystromchebyshev-nystrom-chebyshev-pp}
    \hspace*{\algorithmicindent} \textbf{Input:} Symmetric matrix $\mtx{A} \in \mathbb{R}^{n \times n}$, evaluation points $\{t_i\}_{i=1}^{n_t}$ \\
\hspace*{\algorithmicindent} \textbf{Parameters:} \Glsfirst{sketch-size}, \glsfirst{num-hutchinson-queries}, \glsfirst{chebyshev-degree} \\
\hspace*{\algorithmicindent} \textbf{Output:} Approximate evaluations of the spectral density $\{\widetilde{\phi}_{\sigma}^m(t_i)\}_{i=1}^{n_t}$
\begin{algorithmic}[1]
    \State Compute $\{\mu_l(t_i)\}_{l=0}^m$ for all $t_i$ using \refalg{alg:2-chebyshev-chebyshev-expansion}
    \State Compute $\{\nu_l(t_i)\}_{l=0}^{2m}$ for all $t_i$ using \refalg{alg:3-nystrom-chebyshev-exponentiation}
    \State Generate \glsfirst{sketching-matrix} $\in \mathbb{R}^{n \times n_{\Omega}}$ and \glsfirst{random-matrix} $\in \mathbb{R}^{n \times n_{\Psi}}$
    \State Initialize $[\mtx{V}_1, \mtx{V}_2, \mtx{V}_3] \gets [\mtx{0}_{n \times n_{\Omega}}, \mtx{\mtx{\Omega}}, \mtx{0}_{n \times n_{\Omega}}]$
    \State Initialize $[\mtx{W}_1, \mtx{W}_2, \mtx{W}_3] \gets [\mtx{0}_{n \times n_{\Psi}}, \mtx{\Psi}, \mtx{0}_{n \times n_{\Psi}}]$
    \State Initialize $[\mtx{K}_1(t_i), \mtx{K}_2(t_i)] \gets [\mtx{0}_{n_{\Omega} \times n_{\Omega}}, \mtx{0}_{n_{\Omega} \times n_{\Omega}}]$ for all $t_i$
    \State Initialize $[l_1(t_i), \mtx{L}_2(t_i)] \gets [0, \mtx{0}_{n_{\Omega} \times n_{\Psi}}]$ for all $t_i$
    \State Set ${\phi}_{\sigma}^m(t_i) \gets 0$ for $i=1,\dots,n_t$
    \For {$l = 0, \dots, 2m$}
      \State $\mtx{X} \gets \mtx{\mtx{\Omega}}^{\top} \mtx{V}_2$
      \State $\mtx{Y} \gets \mtx{\mtx{\Omega}}^{\top} \mtx{W}_2$
      \State $z \gets \Tr(\mtx{\Psi}^{\top} \mtx{W}_2$)
      \For {$i = 1, \dots, n_t$}
        \If {$l \leq m$}
            \State $\mtx{K}_1(t_i) \gets \mtx{K}_1(t_i) + \mu_l(t_i) \mtx{X}$
            \State $l_1(t_i) \gets l_1(t_i) + \nu_l(t_i) z$
        \EndIf
        \State $\mtx{K}_2(t_i) \gets \mtx{K}_2(t_i) + \nu_l(t_i) \mtx{X}$
        \State $\mtx{L}_2(t_i) \gets \mtx{L}_2(t_i) + \mu_l(t_i) \mtx{Y}$
      \EndFor
      \State $\mtx{V}_3 \gets (2 - \delta_{l0}) \mtx{A} \mtx{V}_2 - \mtx{V}_1$ \Comment{Chebyshev recurrence \refequ{equ:2-chebyshev-chebyshev-recursion}}
      \State $\mtx{V}_1 \gets \mtx{V}_2, \mtx{V}_2 \gets \mtx{V}_3$
      \State $\mtx{W}_3 \gets (2 - \delta_{l0}) \mtx{A} \mtx{W}_2 - \mtx{W}_1$ \Comment{Chebyshev recurrence \refequ{equ:2-chebyshev-chebyshev-recursion}}
      \State $\mtx{W}_1 \gets \mtx{W}_2, \mtx{W}_2 \gets \mtx{W}_3$
    \EndFor
    \For {$i = 1, \dots, n_t$}
      \State $\widetilde{\phi}_{\sigma}^m(t_i) \gets \Tr\left( \mtx{K}_1(t_i)^{\dagger}\mtx{K}_2(t_i) \right) + \frac{1}{n_{\Psi}} \left( l_1(t_i) + \Tr\left( \mtx{L}_2(t_i)^{\top} \mtx{K}_1(t_i)^{\dagger} \mtx{L}_2(t_i) \right)  \right) $
    \EndFor
\end{algorithmic}

\end{algo}

With the cost of a matrix-vector product denoted by
$c(n)$, and supposing we allocate the random vectors equally
to the low-rank approximation and the trace estimation, i.e. $n_{\Omega} \approx n_{\Psi}$,
we determine the computational complexity of the \gls{NCPP}
method to be $\mathcal{O}(m \log(m) n_t + m n_{\Omega}^2 n + m n_t n_{\Omega}^2 +  m c(n) n_{\Omega} + n_t n_{\Omega}^3)$, with
$\mathcal{O}(m n_t + n n_{\Omega} + n_{\Omega}^2 n_t)$ required additional storage.\\

It is not hard to extend this method to the other low-rank approximations
we have mentioned in \refsec{subsec:3-nystrom-other-low-rank}.

%%%%%%%%%%%%%%%%%%%%%%%%%%%%%%%%%%%%%%%%%%%%%%%%%%%%%%%%%%%%%%%%%%%%%%%%%%%%%%%%

\subsection{Implementation details}
\label{subsec:4-nystromchebyshev-implementation-details}

All of the implementation details for the \gls{DGC} method (\refsec{subsec:2-chebyshev-implementation-details})
and \gls{NC} method (\refsec{subsec:4-nystromchebyshev-implementation-details})
can be directly translated to the \gls{NCPP} method.\\

An interesting and useful observation, which we can make in \refequ{equ:3-nystrom-converted-generalized-eigenvalue-problem},
is that by identifying
\begin{equation}
    \mtx{D}(t) = \mtx{W}_1(t) \mtx{\Gamma}_1(t)^{-1/2} \mtx{X}(t),
    \label{equ:4-nystromchebyshev-generalized-eigenvector}
\end{equation}
which contains, unlike suggested in \cite[algorithm~4]{lin2017randomized},
not the generalized eigenvectors from \refequ{equ:3-nystrom-low-rank-eigenvalue-problem},
we can compute
\begin{equation}
    \mtx{\Xi}(t) = \mtx{D}(t)^{\top} (\mtx{\Omega}^{\top} (g_{\sigma}^{(m)}(t\mtx{I}_n - \mtx{A}))^2 \mtx{\Omega}) \mtx{D}(t),
    \label{equ:4-nystromchebyshev-generalized-eigenvector-xi}
\end{equation}
in other words, the matrix whose trace we used in \refsec{subsec:3-nystrom-implementation-details}
to form $\widehat{\phi}_{\sigma}^{(m)}(t)$ from the Nystr\"om approximation of
$\widehat{g}_{\sigma}^{(m)}(t\mtx{I}_n - \mtx{A})$.
For consistency between all the terms in \refequ{equ:4-nystromchebyshev-spectral-density-decomposition},
it is crucial to compute the correction term $\Hutch_{n_{\Psi}}(\widehat{g}_{\sigma}^{(m)}(t\mtx{I}_n - \mtx{A}))$
using the same Girard-Hutchinson estimator, i.e. the same \gls{random-matrix}, which was already used to compute the
second term $\widetilde{\phi}_{\sigma}^{(m)}(t)$, i.e.
\begin{equation}
    \Hutch_{n_{\Psi}}(\mtx{\Xi}(t)) = \frac{1}{n_{\Psi}} \Tr(\mtx{\Psi}^{\top} \mtx{\Xi}(t) \mtx{\Psi}).
\end{equation}
This can be done quickly and consistently by reusing $\mtx{D}(t)$ from \refalg{alg:3-nystrom-eigenvalue-problem},
since by \refequ{equ:4-nystromchebyshev-generalized-eigenvector-xi} and the cyclic property of the trace
\begin{align*}
    \Tr(\mtx{\Xi}(t))
    &= \Tr(\mtx{D}(t)^{\top} (\mtx{\Omega}^{\top} (g_{\sigma}^{(m)}(t\mtx{I}_n - \mtx{A}))^2 \mtx{\Omega}) \mtx{D}(t)) \notag \\
    &= \Tr( g_{\sigma}^{(m)}(t\mtx{I}_n - \mtx{A}) \mtx{\Omega} \mtx{D}(t) \mtx{D}(t)^{\top} \mtx{\Omega}^{\top} g_{\sigma}^{(m)}(t\mtx{I}_n - \mtx{A})) \notag \\
    &= \mathbb{E}\bigg[\frac{1}{n_{\Psi}} \Tr( (\underbrace{\mtx{\Psi}^{\top} g_{\sigma}^{(m)}(t\mtx{I}_n - \mtx{A}) \mtx{\Omega}}_{=\mtx{L}_1(t)^{\top}})
                                               (\mtx{D}(t) \mtx{D}(t)^{\top})
                                               (\underbrace{\mtx{\Omega}^{\top} g_{\sigma}^{(m)}(t\mtx{I}_n - \mtx{A}) \mtx{\Psi}}_{=\mtx{L}_1(t)})) \bigg].
    \label{equ:4-nystromchebyshev-generalized-eigenvector-trace}
\end{align*}

%%%%%%%%%%%%%%%%%%%%%%%%%%%%%%%%%%%%%%%%%%%%%%%%%%%%%%%%%%%%%%%%%%%%%%%%%%%%%%%%

\subsection{Theoretical analysis}
\label{subsec:4-nystromchebyshev-analysis}

%From \refthm{thm:4-nystromchebyshev-final} it is possible
%to directly conclude that with probability $\geq 1-\delta$,
%\begin{equation}
%    \lVert \phi_{\sigma} - \breve{\phi}_{\sigma} \rVert _1 \leq \varepsilon,
%    \label{equ:4-nystromchebyshev-spectral-density-error}
%\end{equation}
%if we run the \gls{NCPP} method with \gls{sketch-size} $=$ \gls{num-hutchinson-queries} $= \mathcal{O}(\log(1/\delta)/\varepsilon)$
%and assume the Chebyshev expansion to be exact. Due to the fact that in general the Chebyshev
%expansion is not exact and, hence, does not necessarily preserve the \gls{PSD} structure of a matrix function,
%a result similar to \refthm{thm:2-delta-gauss-chebyshev} for the \gls{DGC} method
%is out of reach for the \gls{NCPP} method for now.

Similarly to \refsec{subsec:3-nystrom-theoretical-analysis}, we can again consider
the shifted spectral density $\underline{g}_{\sigma} = g_{\sigma} + \rho$. For
sufficiently large \gls{shift} $\geq 0$ we may then guarantee that $\underline{g}_{\sigma}^{(m)}(t\mtx{I}_n - \mtx{A})$
is symmetric \gls{PSD}. Therefore, \refthm{thm:4-nystromchebyshev-final} can be
employed to get the following result.

\begin{theorem}{$L^1$-error of Nystr\"om-Chebyshev++ method with shift}{4-nystromchebyshev-nystromchebyshev-method}
    Let $\breve{\underline{\phi}}_{\sigma}^{(m)}$ be the result from running
    \refalg{alg:4-nystromchebyshev-nystrom-chebyshev-pp} on a symmetric matrix
    $\mtx{A} \in \mathbb{R}^{n \times n}$ with its spectrum contained in $[-1, 1]$
    using a shifted Gaussian smoothing kernel $\underline{g}_{\sigma} = g_{\sigma} + \rho$
    and with the parameters \gls{smoothing-parameter} $>0$, \gls{chebyshev-degree} $\in \mathbb{N}$, and
    \gls{num-hutchinson-queries} $+$ \gls{sketch-size} $=\mathcal{O}(\log(1/\delta)/\varepsilon)$
    with even \gls{sketch-size} $\geq 8 \log(1/\delta)$.
    If \gls{shift} $\geq \frac{\sqrt{2}}{n\sigma^2} (1 + \sigma)^{-m}$, then
    for $\delta \in (0, e^{-1})$ with probability $\geq 1-\delta$
    \begin{equation}
        \lVert \underline{\phi}_{\sigma} - \breve{\underline{\phi}}_{\sigma}^{(m)} \rVert _1
        \leq (1 + \varepsilon)  \frac{2\sqrt{2}}{\sigma^2} (1 + \sigma)^{-m} + \varepsilon(1 + 2 n \rho).
    \end{equation}
\end{theorem}

\begin{proof}
    With the choice of \gls{shift} in the assumptions of the theorem, we
    can conclude from \reflem{lem:2-chebyshev-error} that the expansion of the
    shifted smoothing kernel $\underline{g}_{\sigma}^{(m)}$ is non-negative.
    Therefore, the parameter-dependent matrix $\underline{g}_{\sigma}^{(m)}(t\mtx{I} - \mtx{A})$
    is symmetric \gls{PSD}, and \refthm{thm:4-nystromchebyshev-final} can be
    applied.\\

    Thus, under the conditions of the theorem, we can conclude
    \begin{align*}
        &\lVert \underline{\phi}_{\sigma} - \breve{\underline{\phi}}_{\sigma}^{(m)} \rVert _1 \notag \\
        &\leq \lVert \underline{\phi}_{\sigma} - \underline{\phi}_{\sigma}^{(m)} \rVert _1 
        + \lVert \underline{\phi}_{\sigma}^{(m)} - \breve{\underline{\phi}}_{\sigma}^{(m)} \rVert _1 && \text{(triangle inequality)} \notag \\
        &\leq \lVert \underline{\phi}_{\sigma} - \underline{\phi}_{\sigma}^{(m)} \rVert _1
        + \varepsilon \lVert \underline{\phi}_{\sigma}^{(m)} \rVert _1 && \text{(\refthm{thm:4-nystromchebyshev-final})} \notag \\
        &\leq (1 + \varepsilon) \lVert \underline{\phi}_{\sigma} - \underline{\phi}_{\sigma}^{(m)} \rVert _1
        + \varepsilon \lVert \underline{\phi}_{\sigma} \rVert _1 && \text{(triangle inequality)} \notag \\
        %&\leq \varepsilon \sqrt{\frac{8}{\pi}} \frac{e^{\frac{c^2}{2}}}{c \sigma^2}  (1 + c \sigma)^{-m} + \varepsilon (1 + 2 \rho) 
        %+ \sqrt{\frac{8}{\pi}} \frac{e^{\frac{c^2}{2}}}{c \sigma^2}  (1 + c \sigma)^{-m} && \text{(\reflem{lem:2-chebyshev-error} and $\lVert \phi_{\sigma} \rVert _1=1$)} \notag \\
        &= (1 + \varepsilon) \frac{2\sqrt{2}}{\sigma^2} (1 + \sigma)^{-m} + \varepsilon(1 + 2 n \rho) && \text{(\reflem{lem:2-chebyshev-error} and $\lVert \phi_{\sigma} \rVert _1=1$)}
    \end{align*}
    with probability $\geq 1-\delta$.
\end{proof}

%These results now allow us to state an error guarantee for the output of the
%\gls{NCPP} algorithm.
%\begin{theorem}{Convergence of Nystr\"om-Chebyshev++ method}{4-nystromchebyshev-spectral-density}
%    The \gls{NCPP} method with $n_{\Omega} = n_{\Psi} = \mathcal{O}\left( \frac{\sqrt{\log(2/\delta)}}{\varepsilon} + \log(1/\delta) \right)$
%    satisfies with probability $1 - \delta$
%    \begin{equation}
%        \lVert \phi_{\sigma}(t) - \breve{\phi}_{\sigma}^{(m)}(t) \rVert _1 \leq.
%    \end{equation}
%\end{theorem}

%\Refthm{thm:4-nystromchebyshev-trace-correction-parameter-dependent} allows us to almost immediately
%also conclude $1/\varepsilon$ result for the higher order approximations
%\refequ{equ:3-nystrom-trace-generalization}.
%
%For RSVD ($k=2$), we use \cite[theorem~9.1]{halko2011finding}
%\begin{equation}
%    \lVert (\mtx{I}_n - \Pi_{\mtx{B}\mtx{\Omega}}) \mtx{B} \rVert _F \leq \lVert \mtx{\Lambda}_2 \rVert _F + \lVert \mtx{\Lambda}_2 \mtx{\Omega}_2 \mtx{\Omega}_1^{\dagger} \rVert _F.
%\end{equation}
%Result from parameter-dependent Nyström++
%\begin{equation}
%    %\mathbb{E}^n_{\Omega}\left[ \lVert \mtx{\Omega}_1^{\dagger} \rVert _F^2 \right] \leq C
%    \mathbb{E}^{n_{\Omega}/2} \left[ \lVert \mtx{\Lambda}_2 \mtx{\Omega}_2 \mtx{\Omega}_1^{\dagger} \rVert _F \right] \leq C(\sqrt{n_{\Omega}} \lVert \mtx{\Lambda}_2 \rVert _2 + \lVert \mtx{\Lambda}_2 \rVert _F),% \leq 2 C \cdot \frac{\operatorname{Tr}(\mtx{B})}{\sqrt{n_{\Omega}}}
%\end{equation}
%where $n_{\Omega}$ hence
%\begin{equation}
%    \mathbb{E}^{n_{\Omega}/2} \left[ \lVert (\mtx{I}_n - \Pi_{\mtx{B}\mtx{\Omega}}) \mtx{B} \rVert _F \right]
%    \leq (1 + 2C) \cdot \frac{\operatorname{Tr}(\mtx{B})}{\sqrt{n_{\Omega}}}.
%\end{equation}
%By Markov's and Minkowski's inequalities we have for $n_{\Omega} \geq 2 \log(1/\delta)$ with probability $\geq 1 - \delta$
%\begin{equation}
%    \int_{a}^{b}\lVert  (\mtx{I}_n - \Pi_{\mtx{B}(t)\mtx{\Omega}}) \mtx{B}(t) \rVert _F \mathrm{d}t \leq (1 + 2C) e \int_{a}^{b}\frac{\operatorname{Tr}(\mtx{B}(t))}{\sqrt{n_{\Omega}}} \mathrm{d}t
%\end{equation}
%
%For Nystr\"om with one subspace iteration ($k=3$) we apply \cite[lemma~5.2]{tropp2023randomized}
%to directly conclude from the RSVD case.
%
%Have result for all $k \geq 1$, but issue with implementation, since we do not
%have access to column sketch (whose QR-factorization is crucial for stability).
