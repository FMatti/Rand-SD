\chapter{The Nystr\"om-Chebyshev++ method}
\label{chp:4-nystromchebyshev}

In \refsec{sec:2-chebyshev-stochastic-trace-estimation} we have introduced the
classical approach for estimating the trace of a matrix using matrix-vector
products. A drawback of this method is the rather slow, reciprocal decrease of
the \gls{MSE} with \gls{num-hutchinson-queries}. Variance reduced trace estimators
try to improve on this. They usually take a \enquote{hybrid} approach, combining
low-rank approximation with stochastic trace estimation.\\

This chapter will take the methods we have discussed in \refchp{chp:2-chebyshev}
and \refchp{chp:3-nystrom} combine them to a variance reduced method which
we call the \glsfirst{NCPP} method.

%%%%%%%%%%%%%%%%%%%%%%%%%%%%%%%%%%%%%%%%%%%%%%%%%%%%%%%%%%%%%%%%%%%%%%%%%%%%%%%%

\section{Variance reduced trace estimation}
\label{sec:4-nystromchebyshev-hybrid}

The initial idea from \cite{lin2017randomized} was to combine low-rank approximation
with stochastic trace estimation to cause the \gls{MSE}, which -- for unbiased estimators
-- is synonymous with the variance, to decrease quadratically with the square
of the reciprocal of the number of matrix-vector products.
Recent theoretical developments confirm the
efficacy of this approach \cite{meyer2021hutch,persson2022hutch}.\\

The hope is that by first computing a low-rank approximation $\widehat{\mtx{B}} \approx \mtx{B}$
we can capture as much of the contribution of the dominant eigenvalues to the
trace of $\mtx{B}$ as possible.

\todo{Explain intuition and results behind Hutch++, and show it is also possible with Nystrom++}

for which we can directly compute the trace. Subsequently, the
Hutchinson's trace estimator \refequ{equ:2-chebyshev-DGC-hutchionson-estimator}
is used to estimate the trace of the residual
\begin{equation}
    \mtx{\Delta} = \mtx{B} - \widehat{\mtx{B}},
    \label{equ:4-nystromchebyshev-residual}
\end{equation}
such that the final trace estimate is
\begin{equation}
    \Tr(\mtx{B}) \approx \Tr(\widehat{\mtx{B}}) + \Hutch(\mtx{\Delta}).
    \label{equ:4-nystromchebyshev-hutch-pp}
\end{equation}
If we are able to compute a good low-rank approximation, the Frobenius norm
of the residual $\mtx{\Delta}$ will be small, and, hence, the \gls{MSE}
of the Hutchinson \refequ{equ:2-chebyshev-DGC-hutchinson-variance} also.

%%%%%%%%%%%%%%%%%%%%%%%%%%%%%%%%%%%%%%%%%%%%%%%%%%%%%%%%%%%%%%%%%%%%%%%%%%%%%%%%

\section{The Nystr\"om-Chebyshev++ method}
\label{sec:4-nystromchebyshev-nystromchebyshev-pp}

The remainder is 

Taking the developments from \refchp{chp:2-chebyshev} and \refchp{chp:3-nystrom},
we can combine them to a powerful hybrid method which we call the \glsfirst{NCPP}
method.

\todo{Better explain how residual is estimated.}

Its implementations is similar to the \gls{NC} method (\refalg{alg:3-nystrom-nystrom-chebyshev}).
The pseudocode for the \glsfirst{NCPP} is given in \refalg{alg:4-nystromchebyshev-nystrom-chebyshev-pp}.

\begin{algo}{Nystr\"om-Chebyshev++ method}{4-nystromchebyshev-nystrom-chebyshev-pp}
    \hspace*{\algorithmicindent} \textbf{Input:} Symmetric matrix $\mtx{A} \in \mathbb{R}^{n \times n}$, evaluation points $\{t_i\}_{i=1}^{n_t}$ \\
\hspace*{\algorithmicindent} \textbf{Parameters:} \Glsfirst{sketch-size}, \glsfirst{num-hutchinson-queries}, \glsfirst{chebyshev-degree} \\
\hspace*{\algorithmicindent} \textbf{Output:} Approximate evaluations of the spectral density $\{\widetilde{\phi}_{\sigma}^m(t_i)\}_{i=1}^{n_t}$
\begin{algorithmic}[1]
    \State Compute $\{\mu_l(t_i)\}_{l=0}^m$ for all $t_i$ using \refalg{alg:2-chebyshev-chebyshev-expansion}
    \State Compute $\{\nu_l(t_i)\}_{l=0}^{2m}$ for all $t_i$ using \refalg{alg:3-nystrom-chebyshev-exponentiation}
    \State Generate \glsfirst{sketching-matrix} $\in \mathbb{R}^{n \times n_{\Omega}}$ and \glsfirst{random-matrix} $\in \mathbb{R}^{n \times n_{\Psi}}$
    \State Initialize $[\mtx{V}_1, \mtx{V}_2, \mtx{V}_3] \gets [\mtx{0}_{n \times n_{\Omega}}, \mtx{\mtx{\Omega}}, \mtx{0}_{n \times n_{\Omega}}]$
    \State Initialize $[\mtx{W}_1, \mtx{W}_2, \mtx{W}_3] \gets [\mtx{0}_{n \times n_{\Psi}}, \mtx{\Psi}, \mtx{0}_{n \times n_{\Psi}}]$
    \State Initialize $[\mtx{K}_1(t_i), \mtx{K}_2(t_i)] \gets [\mtx{0}_{n_{\Omega} \times n_{\Omega}}, \mtx{0}_{n_{\Omega} \times n_{\Omega}}]$ for all $t_i$
    \State Initialize $[l_1(t_i), \mtx{L}_2(t_i)] \gets [0, \mtx{0}_{n_{\Omega} \times n_{\Psi}}]$ for all $t_i$
    \State Set ${\phi}_{\sigma}^m(t_i) \gets 0$ for $i=1,\dots,n_t$
    \For {$l = 0, \dots, 2m$}
      \State $\mtx{X} \gets \mtx{\mtx{\Omega}}^{\top} \mtx{V}_2$
      \State $\mtx{Y} \gets \mtx{\mtx{\Omega}}^{\top} \mtx{W}_2$
      \State $z \gets \Tr(\mtx{\Psi}^{\top} \mtx{W}_2$)
      \For {$i = 1, \dots, n_t$}
        \If {$l \leq m$}
            \State $\mtx{K}_1(t_i) \gets \mtx{K}_1(t_i) + \mu_l(t_i) \mtx{X}$
            \State $l_1(t_i) \gets l_1(t_i) + \nu_l(t_i) z$
        \EndIf
        \State $\mtx{K}_2(t_i) \gets \mtx{K}_2(t_i) + \nu_l(t_i) \mtx{X}$
        \State $\mtx{L}_2(t_i) \gets \mtx{L}_2(t_i) + \mu_l(t_i) \mtx{Y}$
      \EndFor
      \State $\mtx{V}_3 \gets (2 - \delta_{l0}) \mtx{A} \mtx{V}_2 - \mtx{V}_1$ \Comment{Chebyshev recurrence \refequ{equ:2-chebyshev-chebyshev-recursion}}
      \State $\mtx{V}_1 \gets \mtx{V}_2, \mtx{V}_2 \gets \mtx{V}_3$
      \State $\mtx{W}_3 \gets (2 - \delta_{l0}) \mtx{A} \mtx{W}_2 - \mtx{W}_1$ \Comment{Chebyshev recurrence \refequ{equ:2-chebyshev-chebyshev-recursion}}
      \State $\mtx{W}_1 \gets \mtx{W}_2, \mtx{W}_2 \gets \mtx{W}_3$
    \EndFor
    \For {$i = 1, \dots, n_t$}
      \State $\widetilde{\phi}_{\sigma}^m(t_i) \gets \Tr\left( \mtx{K}_1(t_i)^{\dagger}\mtx{K}_2(t_i) \right) + \frac{1}{n_{\Psi}} \left( l_1(t_i) + \Tr\left( \mtx{L}_2(t_i)^{\top} \mtx{K}_1(t_i)^{\dagger} \mtx{L}_2(t_i) \right)  \right) $
    \EndFor
\end{algorithmic}

\end{algo}

With the cost of a matrix-vector product denoted by
$c(n)$, and supposing we allocate the random vectors equally
to the low-rank approximation and the trace estimation, i.e. $n_{\mtx{\Omega}} \approx n_{\mtx{\Psi}}$,
we determine the computational complexity of the \gls{NCPP}
method to be $\mathcal{O}(m \log(m) n_t + m n_{\mtx{\Omega}}^2 n + m n_t n_{\mtx{\Omega}}^2 +  m c(n) n_{\mtx{\Omega}} + n_t n_{\mtx{\Omega}}^3)$, with
$\mathcal{O}(n n_{\mtx{\Omega}} + n_{\mtx{\Omega}}^2 n_t + m n_t)$ required additional storage.\\

%%%%%%%%%%%%%%%%%%%%%%%%%%%%%%%%%%%%%%%%%%%%%%%%%%%%%%%%%%%%%%%%%%%%%%%%%%%%%%%%

\subsection{Implementation details}
\label{subsec:4-nystromchebyshev-implementation-details}

\todo{
In at the end of algorithm \refalg{alg:3-nystrom-eigenvalue-problem} we
can additionally compute the generalized eigenvectors \refequ{equ:3-nystrom-generalized-eigenvector}
to make solving this easier

\begin{equation}
    \Tr(\mtx{L}_2(t_i)^{\top} \mtx{K}_1(t_i)^{\dagger} \mtx{L}_2(t_i))
        = \Tr(\Omega \dots )
    \label{equ:4-nystromchebyshev-eigenvectors}
\end{equation}
}

It is not hard to extend this method to the other low-rank approximations
we have mentioned in \refsec{subsec:3-nystrom-other-low-rank}.

%%%%%%%%%%%%%%%%%%%%%%%%%%%%%%%%%%%%%%%%%%%%%%%%%%%%%%%%%%%%%%%%%%%%%%%%%%%%%%%%

\subsection{Theoretical analysis}
\label{subsec:4-nystromchebyshev-analysis}

Using the result from \reflem{lem:2-chebyshev-parameter-hutchinson},
we can -- under certain conditions -- derive an analogous result
to \cite{meyer2021hutch} for any trace estimate of the form
\refequ{equ:4-nystromchebyshev-hutch-pp} in the parameter-dependent case.

\begin{theorem}{Trace correction}{4-nystromchebyshev-trace-correction}
    Suppose $\mtx{B}(t) \in \mathbb{R}^{n \times n}$ is symmetric positive definite
    and continuous in $t \in [a, b]$. Let $\widehat{\mtx{B}}(t)$ and
    $\{\mtx{\Delta}_{n_{\mtx{\Omega}}}(t)\}_{n_{\mtx{\Omega}} \geq 1}$ be any
    symmetric positive definite and continuous matrices in $t \in [a, b]$ such that
    \begin{equation}
        \begin{cases}
            \Tr(\mtx{B}(t)) = \Tr(\widehat{\mtx{B}}(t)) + \Tr(\mtx{\Delta}_{n_{\mtx{\Omega}}}(t)); \\
            \int_{a}^{b} \lVert \mtx{\Delta}_{n_{\mtx{\Omega}}}(t) \rVert _F \mathrm{d}t \leq C_{\mtx{\Omega}} \frac{1}{\sqrt{n_{\mtx{\Omega}}}} \int_{a}^{b} \Tr(\mtx{B}(t)) \mathrm{d}t.
        \end{cases}
    \end{equation}
    For a fixed constant $C$ and with probability $\geq 1 - \delta$, $Z(t) = \Tr(\widehat{\mtx{B}}(t)) + \Hutch_l(\mtx{\Delta}_{n_{\mtx{\Omega}}}(t))$ satisfies:
    \begin{equation}
        \int_{a}^{b} |Z(t) - \Tr(\mtx{\Delta}_{n_{\mtx{\Omega}}}(t))| \mathrm{d}t \leq C \sqrt{\frac{\log(2/\delta)}{n_{\mtx{\Psi}} n_{\mtx{\Omega}}}} \int_{a}^{b} \Tr(\mtx{B}(t)) \mathrm{d}t
    \end{equation}
    In particular, if $n_{\mtx{\Omega}}=n_{\mtx{\Psi}}=\mathcal{O}\left( \frac{\sqrt{\log(2/\delta)}}{\varepsilon} \right)$, $Z(t)$ is a $(1 \pm \varepsilon)$ error approximation to $\Tr(\mtx{B}(t))$.
\end{theorem}

\begin{proof}
    We may directly evaluate
    \begin{align}
        \int_{a}^{b} |Z(t) - \Tr(\mtx{B}(t))| \mathrm{d}t
        &= \int_{a}^{b} |\Hutch_l(\mtx{\Delta}_{n_{\mtx{\Omega}}}(t)) - \Tr(\mtx{\Delta}_{n_{\mtx{\Omega}}}(t))| \mathrm{d}t && \text{(by definition of $Z(t)$)} \\
        &= C_{\mtx{\Psi}} \sqrt{\frac{\log(2/\delta)}{n_{\mtx{\Psi}}}} \int_{a}^{b} \lVert \mtx{\Delta}_{n_{\mtx{\Omega}}}(t) \rVert _F \mathrm{d}t && \text{(using \reflem{lem:2-chebyshev-parameter-hutchinson})} \\
        &= C_{\mtx{\Psi}} C_{\mtx{\Omega}} \sqrt{\frac{\log(2/\delta)}{n_{\mtx{\Psi}} n_{\mtx{\Omega}}}} \int_{a}^{b} \Tr(\mtx{B}(t)) \mathrm{d}t && \text{(assumption on $\mtx{\Delta}_{n_{\mtx{\Omega}}}(t)$)} \\
    \end{align}
    Taking $C=C_{\mtx{\Psi}} C_{\mtx{\Omega}}$, we get the desired result with probability $\geq 1 - \delta$.
\end{proof}

For the Nystr\"om approximation in particular \refequ{equ:3-nystrom-nystrom},
we can show that it is an approximation which satisfies the conditions
in \refthm{thm:4-nystromchebyshev-trace-correction}. The key is to use the following
lemma \textcolor{red}{[cite in preparation]} which guarantees the desired
convergence property of the parameter-dependent Nystr\"om approximation to the


the Nystr\"om approximation converges as \gls{sketch-size} grows.
This is captured in the following lemma \textcolor{red}{[cite in preparation]}.
\todo{define Nystr\"om++}

\begin{lemma}{Parameter-dependent Nystr\"om approximation}{4-nystromchebyshev-parameter-nystrom}
    Let $\mtx{B}(t) \in \mathbb{R}^{n \times n}$ symmetric positive definite and continuous in $t \in [a, b]$. Then the parameter-dependent Nystr\"om approximation
    \begin{equation}
        \widehat{\mtx{B}}(t) = (\mtx{B}(t) \mtx{\Omega}) (\mtx{\Omega}^{\top} \mtx{B}(t) \mtx{\Omega})^{\dagger} (\mtx{B}(t) \mtx{\Omega})^{\top}
    \end{equation}
    with $\mtx{\Omega} \in \mathbb{R}^{n \times n_{\mtx{\Omega}}}$ a standard Gaussian matrix, $n_{\mtx{\Omega}} \geq 8 \log(1/\delta)$, and constant $C_{\mtx{\Omega}}$, satisfies with probability $\geq 1 - \delta$
    \begin{equation}
        \int_{a}^{b} \lVert \mtx{B}(t) - \widehat{\mtx{B}}(t) \rVert _F \mathrm{d}t \leq C_{\mtx{\Omega}} \frac{1}{\sqrt{n_{\mtx{\Omega}}}} \int_{a}^{b} \Tr(\mtx{B}(t)) \mathrm{d}t.
    \end{equation}
\end{lemma}

\begin{theorem}{Parameter-dependent Nystr\"om++ estimator}{4-nystromchebyshev-final}
    The parameter-dependent Nystr\"om++ computed with
    $n_{\mtx{\Omega}} = n_{\mtx{\Psi}} = \mathcal{O}\left( \frac{\sqrt{\log(2/\delta)}}{\varepsilon} + \log(1/\delta) \right)$
    for any symmetric positive definite matrix $\mtx{B}(t)$ which continuously
    depends on $t \in [a, b]$ satisfies, with probability
    $1 - \delta$
    \begin{equation}
        (1 - \varepsilon) \int_{a}^{b}\Tr(\mtx{B}(t)) \mathrm{d}t \leq \int_{a}^{b}\Tr^{n++}(\mtx{B}(t)) \mathrm{d}t \leq  (1 + \varepsilon) \int_{a}^{b}\Tr(\mtx{B}(t)) \mathrm{d}t
    \end{equation}
\end{theorem}

\begin{proof}
    According to \refthm{thm:4-nystromchebyshev-trace-correction} it is enough to verify that the Nystr\"om approximation $\widehat{\mtx{B}}(t)$ of $\mtx{B}(t)$ satisfies
    \begin{equation}
        \Tr(\mtx{B}(t)) = \Tr(\widehat{\mtx{B}}(t) + \mtx{B}(t) - \widehat{\mtx{B}}(t)) = \Tr(\widehat{\mtx{B}}(t)) + \Tr(\mtx{\Delta}(t))
    \end{equation}
    for all $t \in [a, b]$ and with probability $\geq 1 - \delta$
    \begin{equation}
        \int_{a}^{b} \lVert \mtx{\Delta}_{n_{\mtx{\Omega}}}(t) \rVert _F \mathrm{d}t = \int_{a}^{b} \lVert \mtx{B}(t) - \widehat{\mtx{B}}(t) \rVert _F \mathrm{d}t \leq C_{\mtx{\Omega}} \frac{1}{\sqrt{n_{\mtx{\Omega}}}} \int_{a}^{b} \Tr(\mtx{B}(t)) \mathrm{d}t.
    \end{equation}
    The latter follows from \reflem{lem:4-nystromchebyshev-parameter-nystrom}. Finally, the choices of \gls{sketch-size} and \gls{num-hutchinson-queries} follow from \refthm{thm:4-nystromchebyshev-trace-correction} and \reflem{lem:4-nystromchebyshev-parameter-nystrom}.
\end{proof}

These results now allow us to state an error guarantee for the output of the
\gls{NCPP} algorithm.
\begin{theorem}{Final}{4-nystromchebyshev-spectral-density}
    The \gls{NCPP} method with $n_{\mtx{\Omega}} = n_{\mtx{\Psi}} = \mathcal{O}\left( \frac{\sqrt{\log(2/\delta)}}{\varepsilon} + \log(1/\delta) \right)$
    \todo{condition on $m$} satisfies with probability $1 - \delta$
    \begin{equation}
        (1 - \epsilon) \int_{a}^{b} \phi_{\sigma}(t) \mathrm{d}t \leq \int_{a}^{b} \widetilde{\phi}_{\sigma}(t) \mathrm{d}t \leq (1 + \epsilon) \int_{a}^{b} \phi_{\sigma}(t) \mathrm{d}t.
    \end{equation}
\end{theorem}

\Refthm{thm:4-nystromchebyshev-trace-correction} allows us to almost immediately
also conclude $1/\varepsilon$ result for the higher order approximations
\refequ{equ:3-nystrom-trace-generalization}.

For RSVD ($k=2$), we use \cite[theorem~9.1]{halko2011finding}
\begin{equation}
    \lVert (\mtx{I} - \Pi_{\mtx{B}\mtx{\Omega}}) \mtx{B} \rVert _F \leq \lVert \mtx{\Lambda}_2 \rVert _F + \lVert \mtx{\Lambda}_2 \mtx{\Omega}_2 \mtx{\Omega}_1^{\dagger} \rVert _F.
\end{equation}
Result from parameter-dependent Nyström++ \todo{[cite preprint]}
\begin{equation}
    %\mathbb{E}^n_{\mtx{\Omega}}\left[ \lVert \mtx{\Omega}_1^{\dagger} \rVert _F^2 \right] \leq C
    \mathbb{E}^{n_{\mtx{\Omega}}/2} \left[ \lVert \mtx{\Lambda}_2 \mtx{\Omega}_2 \mtx{\Omega}_1^{\dagger} \rVert _F \right] \leq C(\sqrt{n_{\mtx{\Omega}}} \lVert \mtx{\Lambda}_2 \rVert _2 + \lVert \mtx{\Lambda}_2 \rVert _F),% \leq 2 C \cdot \frac{\operatorname{Tr}(\mtx{B})}{\sqrt{n_{\mtx{\Omega}}}}
\end{equation}
where $n_{\mtx{\Omega}}$ hence
\begin{equation}
    \mathbb{E}^{n_{\mtx{\Omega}}/2} \left[ \lVert (\mtx{I} - \Pi_{\mtx{B}\mtx{\Omega}}) \mtx{B} \rVert _F \right]
    \leq (1 + 2C) \cdot \frac{\operatorname{Tr}(\mtx{B})}{\sqrt{n_{\mtx{\Omega}}}}.
\end{equation}
By Markov's and Minkowski's inequalities we have for $n_{\mtx{\Omega}} \geq 2 \log(1/\delta)$ with probability $\geq 1 - \delta$
\begin{equation}
    \int_{a}^{b}\lVert  (\mtx{I} - \Pi_{\mtx{B}(t)\mtx{\Omega}}) \mtx{B}(t) \rVert _F \mathrm{d}t \leq (1 + 2C) e \int_{a}^{b}\frac{\operatorname{Tr}(\mtx{B}(t))}{\sqrt{n_{\mtx{\Omega}}}} \mathrm{d}t
\end{equation}

For Nystr\"om with one subspace iteration ($k=3$) we apply \cite[lemma~5.2]{tropp2023randomized}
to directly conclude from the RSVD case.

Have result for all $k \geq 1$, but issue with implementation, since we do not
have access to column sketch (whose QR-factorization is crucial for stability).

\todo{Put together all errors (Nyström, Chebyshev, Hutchinson)}
