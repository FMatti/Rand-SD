\chapter{The Nystr\"om-Chebyshev++ method}
\label{chp:4-nystromchebyshev}

In \refsec{sec:2-chebyshev-stochastic-trace-estimation} we have introduced the
classical approach for estimating the trace of a matrix using matrix-vector
products. A drawback of this method is the rather slow, reciprocal decrease of
the \gls{MSE} with \gls{num-hutchinson-queries}. Variance reduced trace estimators
try to improve on this. They usually take a \enquote{hybrid} approach, combining
low-rank approximation with stochastic trace estimation.\\

This chapter will take the methods we have discussed in \refchp{chp:2-chebyshev}
and \refchp{chp:3-nystrom} combine them to a variance reduced method which
we call the \glsfirst{NCPP} method.

%%%%%%%%%%%%%%%%%%%%%%%%%%%%%%%%%%%%%%%%%%%%%%%%%%%%%%%%%%%%%%%%%%%%%%%%%%%%%%%%

\section{Variance reduced trace estimation}
\label{sec:4-nystromchebyshev-hybrid}

The initial idea from \cite{lin2017randomized} was to combine low-rank approximation
with stochastic trace estimation to cause the \gls{MSE}, which -- for unbiased estimators
-- is synonymous with the variance, to decrease quadratically with the square
of the reciprocal of the number of matrix-vector products.
Recent theoretical developments confirm the
efficacy of this approach \cite{meyer2021hutch,persson2022hutch}.
We will first briefly discuss these results which are valid 
for constant matrices.\\

\subsection{Constant matrices}
\label{subsec:4-nystromchebyshev-reduction-constant-matrices}

Due to the linearity of the trace of a matrix, for any symmetric matrix $\mtx{B} \in \mathbb{R}^{n \times n}$, we
can decompose its trace into two parts using another matrix $\widehat{\mtx{B}} \in \mathbb{R}^{n \times n}$
\begin{equation}
    \Tr(\mtx{B}) = \Tr(\widehat{\mtx{B}}) + \Tr(\mtx{B} - \widehat{\mtx{B}}).
    \label{equ:4-nystromchebyshev-trace-decomposition}
\end{equation}
If we manage to find a matrix $\widehat{\mtx{B}}$, such that the trace of
$\widehat{\mtx{B}}$ can be computed efficiently and
$\lVert \mtx{B} - \widehat{\mtx{B}} \rVert _F$ is small,
then the first term in \refequ{equ:4-nystromchebyshev-trace-decomposition} can
be computed exactly and the second term can be approximated well using Hutchinson's
trace estimator $\Hutch$ \refequ{equ:2-chebyshev-DGC-hutchionson-estimator}
due to its small Frobenius norm \refequ{equ:2-chebyshev-hutchinson-mse}.
That is,
\begin{equation}
    \Tr^{++}(\mtx{B}) = \Tr(\widehat{\mtx{B}}) + \Hutch(\mtx{\Delta})
    \label{equ:4-nystromchebyshev-hutch-pp}
\end{equation}
with the residual $\mtx{\Delta} = \mtx{B} - \widehat{\mtx{B}}$ can be an excellent
approximation.\\

In \refsec{subsec:3-nystrom-factorization-constant-matrices} we have introduced
some ways in which a matrix $\widehat{\mtx{B}}$ satisfying the above mentioned
criteria can be constructed. For instance, for a matrix $\mtx{B} \in \mathbb{R}^{n \times n}$
of rank $r \ll n$, we can theoretically take a factorization of the form
$\mtx{B} = \mtx{V}_1 \mtx{\Sigma}_1 \mtx{V}_1^{\top}$ \refequ{equ:3-nystrom-eigenvalue-decoposition}.
Since $\mtx{V}_1$ is orthonormal, $\Tr(\mtx{B}) = \Tr(\mtx{\Sigma}_1)$ by the
cyclic property of the trace, which is immediate to compute. On the other hand,
$\mtx{\Delta} = \mtx{B} - \widehat{\mtx{B}} = \mtx{0}$ which implies that the
Hutchinson's estimator will be exact. Thus, \refequ{equ:4-nystromchebyshev-hutch-pp}
would be exact. However, this factorization is -- in general -- prohibitly expensive
to compute. That is where randomized low-rank factorizations come into play.\\

A first such algorithm is the Hutch++ algorithm \cite[algorithm~1]{meyer2021hutch}, which
uses a low-rank factorization of the form \refequ{equ:3-nystrom-RSVD} for
$\widehat{mtx{B}}$. For positive semi-definite matrices it was shown to give
an estimate whose relative deviation from the actual trace is at most
$\epsilon$ by only using $\mathcal{O}(\epsilon^{-1})$ matrix-vector multiplications,
with high probability \cite[theorem~1]{meyer2021hutch}. Similarly, \cite{lin2017randomized}
uses and \cite{persson2022hutch} refines the same procedure when using
the Nystr\"om approximation \refequ{equ:3-nystrom-nystrom}. In \cite[theorem~3.4]{persson2022hutch}
it is shown that similarly to the Hutch++ algorithm, the following statement holds:
\begin{theorem}{Nystr\"om++}{4-nystromchebyshev-nystrom-pp}
    If the trace estimator \refequ{equ:4-nystromchebyshev-hutch-pp} based
    on the Nystr\"om approximation is computed
    with \gls{sketch-size}$ = $\gls{num-hutchinson-queries}$ = \mathcal{O}(\sqrt{\log(1/\delta)}/\varepsilon + \log(1/\delta))$
    and $\delta \in (0, 1/2)$
    \begin{equation}
        |\Tr^{++}(\mtx{B}) - \Tr(\mtx{B})| \leq \varepsilon |\Tr(\mtx{B})|
    \end{equation}
    with probability at least $1-\delta$.
\end{theorem}

\subsection{Parameter-dependent matrices}
\label{subsec:4-nystromchebyshev-reduction-parametrized-matrices}

Using the result from \reflem{lem:2-chebyshev-parameter-hutchinson},
we can -- under certain conditions -- derive an analogous result
to \refthm{thm:4-nystromchebyshev-nystrom-pp} for any trace estimate of the form
\refequ{equ:4-nystromchebyshev-hutch-pp} in the parameter-dependent case.

\begin{theorem}{Trace correction}{4-nystromchebyshev-trace-correction}
    Suppose $\mtx{B}(t) \in \mathbb{R}^{n \times n}$ is symmetric positive definite
    and continuous in $t \in [a, b]$. Let $\widehat{\mtx{B}}(t)$ and
    $\{\mtx{\Delta}_{n_{\Omega}}(t)\}_{n_{\Omega} \geq 1}$ be any
    symmetric positive definite and continuous matrices in $t \in [a, b]$ such that
    \begin{equation}
        \begin{cases}
            \Tr(\mtx{B}(t)) = \Tr(\widehat{\mtx{B}}(t)) + \Tr(\mtx{\Delta}_{n_{\Omega}}(t)); \\
            \int_{a}^{b} \lVert \mtx{\Delta}_{n_{\Omega}}(t) \rVert _F \mathrm{d}t \leq C_{\Omega} \frac{1}{\sqrt{n_{\Omega}}} \int_{a}^{b} \Tr(\mtx{B}(t)) \mathrm{d}t.
        \end{cases}
    \end{equation}
    For a fixed constant $C$ and with probability $\geq 1 - \delta$, $\Tr^{++}(t) = \Tr(\widehat{\mtx{B}}(t)) + \Hutch_{n_{\Psi}}(\mtx{\Delta}_{n_{\Omega}}(t))$ satisfies:
    \begin{equation}
        \int_{a}^{b} |\Tr^{++}(t) - \Tr(\mtx{\Delta}_{n_{\Omega}}(t))| \mathrm{d}t \leq C \sqrt{\frac{\log(2/\delta)}{n_{\Psi} n_{\Omega}}} \int_{a}^{b} \Tr(\mtx{B}(t)) \mathrm{d}t
    \end{equation}
    In particular, if $n_{\Omega}=n_{\Psi}=\mathcal{O}\left( \frac{\sqrt{\log(2/\delta)}}{\varepsilon} \right)$, $\Tr^{++}(t)$ is a $(1 \pm \varepsilon)$ error approximation to $\Tr(\mtx{B}(t))$.
\end{theorem}

\begin{proof}
    We may directly evaluate
    \begin{align}
        \int_{a}^{b} |\Tr^{++}(t) - \Tr(\mtx{B}(t))| \mathrm{d}t
        &= \int_{a}^{b} |\Hutch_{n_{\Psi}}(\mtx{\Delta}_{n_{\Omega}}(t)) - \Tr(\mtx{\Delta}_{n_{\Omega}}(t))| \mathrm{d}t && \text{(definition $\Tr^{++}(t)$)} \\
        &= C_{\Psi} \sqrt{\frac{\log(2/\delta)}{n_{\Psi}}} \int_{a}^{b} \lVert \mtx{\Delta}_{n_{\Omega}}(t) \rVert _F \mathrm{d}t && \text{(using \reflem{lem:2-chebyshev-parameter-hutchinson})} \\
        &= C_{\Psi} C_{\Omega} \sqrt{\frac{\log(2/\delta)}{n_{\Psi} n_{\Omega}}} \int_{a}^{b} \Tr(\mtx{B}(t)) \mathrm{d}t && \text{(assumption $\mtx{\Delta}_{n_{\Omega}}(t)$)} \\
    \end{align}
    Taking $C=C_{\Psi} C_{\Omega}$, we get the desired result with probability $\geq 1 - \delta$.
\end{proof}

For the Nystr\"om approximation in particular \refequ{equ:3-nystrom-nystrom},
we can show that it is an approximation which satisfies the conditions
in \refthm{thm:4-nystromchebyshev-trace-correction}. The key is to use the following
lemma \textcolor{red}{[cite in preparation]} which guarantees the desired
convergence property of the parameter-dependent Nystr\"om approximation.

\begin{lemma}{Parameter-dependent Nystr\"om approximation}{4-nystromchebyshev-parameter-nystrom}
    Let $\mtx{B}(t) \in \mathbb{R}^{n \times n}$ symmetric positive definite and continuous in $t \in [a, b]$. Then the parameter-dependent Nystr\"om approximation
    \begin{equation}
        \widehat{\mtx{B}}(t) = (\mtx{B}(t) \mtx{\Omega}) (\mtx{\Omega}^{\top} \mtx{B}(t) \mtx{\Omega})^{\dagger} (\mtx{B}(t) \mtx{\Omega})^{\top}
    \end{equation}
    with $\mtx{\Omega} \in \mathbb{R}^{n \times n_{\Omega}}$ a standard Gaussian matrix, $n_{\Omega} \geq 8 \log(1/\delta)$, and constant $C_{\Omega}$, satisfies with probability $\geq 1 - \delta$
    \begin{equation}
        \int_{a}^{b} \lVert \mtx{B}(t) - \widehat{\mtx{B}}(t) \rVert _F \mathrm{d}t \leq C_{\Omega} \frac{1}{\sqrt{n_{\Omega}}} \int_{a}^{b} \Tr(\mtx{B}(t)) \mathrm{d}t.
    \end{equation}
\end{lemma}

\begin{theorem}{Parameter-dependent Nystr\"om++ estimator}{4-nystromchebyshev-final}
    The parameter-dependent Nystr\"om++ computed with
    $n_{\Omega} = n_{\Psi} = \mathcal{O}\left( \frac{\sqrt{\log(2/\delta)}}{\varepsilon} + \log(1/\delta) \right)$
    for any symmetric positive definite matrix $\mtx{B}(t)$ which continuously
    depends on $t \in [a, b]$ satisfies, with probability
    $1 - \delta$
    \begin{equation}
        (1 - \varepsilon) \int_{a}^{b}\Tr(\mtx{B}(t)) \mathrm{d}t \leq \int_{a}^{b}\Tr^{n++}(\mtx{B}(t)) \mathrm{d}t \leq  (1 + \varepsilon) \int_{a}^{b}\Tr(\mtx{B}(t)) \mathrm{d}t
    \end{equation}
\end{theorem}

\begin{proof}
    According to \refthm{thm:4-nystromchebyshev-trace-correction} it is enough to verify that the Nystr\"om approximation $\widehat{\mtx{B}}(t)$ of $\mtx{B}(t)$ satisfies
    \begin{equation}
        \Tr(\mtx{B}(t)) = \Tr(\widehat{\mtx{B}}(t) + \mtx{B}(t) - \widehat{\mtx{B}}(t)) = \Tr(\widehat{\mtx{B}}(t)) + \Tr(\mtx{\Delta}(t))
    \end{equation}
    for all $t \in [a, b]$ and with probability $\geq 1 - \delta$
    \begin{equation}
        \int_{a}^{b} \lVert \mtx{\Delta}_{n_{\Omega}}(t) \rVert _F \mathrm{d}t = \int_{a}^{b} \lVert \mtx{B}(t) - \widehat{\mtx{B}}(t) \rVert _F \mathrm{d}t \leq C_{\Omega} \frac{1}{\sqrt{n_{\Omega}}} \int_{a}^{b} \Tr(\mtx{B}(t)) \mathrm{d}t.
    \end{equation}
    The latter follows from \reflem{lem:4-nystromchebyshev-parameter-nystrom}. Finally, the choices of \gls{sketch-size} and \gls{num-hutchinson-queries} follow from \refthm{thm:4-nystromchebyshev-trace-correction} and \reflem{lem:4-nystromchebyshev-parameter-nystrom}.
\end{proof}

%%%%%%%%%%%%%%%%%%%%%%%%%%%%%%%%%%%%%%%%%%%%%%%%%%%%%%%%%%%%%%%%%%%%%%%%%%%%%%%%

\section{The Nystr\"om-Chebyshev++ method}
\label{sec:4-nystromchebyshev-nystromchebyshev-pp}

Taking the algorithmic developments from \refchp{chp:2-chebyshev} and \refchp{chp:3-nystrom},
we can easily combine them to a powerful hybrid method which we call the \glsfirst{NCPP}
method.

As introduced in \refsec{sec:4-nystromchebyshev-hybrid}, this method improves
upon the \gls{NC} method with an additional estimation of the trace of the
residual $\mtx{\Delta}(t) = g_{\sigma}^m(t\mtx{I} - \mtx{A}) - \widehat{g}_{\sigma}^m(t\mtx{I} - \mtx{A})$
from the Nystr\"om approximation using the Hutchinson's estimator
\refequ{equ:2-chebyshev-DGC-hutchionson-estimator} as follows
\begin{align}
    \breve{\phi}_{\sigma}^m(t)
    %&= \Tr(\widehat{g}_{\sigma}^m(t\mtx{I} - \mtx{A}))
    %- \Hutch_{n_{\Psi}}(\underbrace{g_{\sigma}^m(t\mtx{I} - \mtx{A}) + \widehat{g}_{\sigma}^m(t\mtx{I} - \mtx{A})}_{=\mtx{\Delta}(t)}) \notag \\
    &= \Tr(\widehat{g}_{\sigma}^m(t\mtx{I} - \mtx{A}))
    + \Hutch_{n_{\Psi}}(g_{\sigma}^m(t\mtx{I} - \mtx{A}) - \widehat{g}_{\sigma}^m(t\mtx{I} - \mtx{A})) \notag \\
    &= \underbrace{\Tr(\widehat{g}_{\sigma}^m(t\mtx{I} - \mtx{A}))}_{=\widehat{\phi}_{\sigma}^m(t)}
    + \underbrace{\Hutch_{n_{\Psi}}(g_{\sigma}^m(t\mtx{I} - \mtx{A}))}_{=\widetilde{\phi}_{\sigma}^m(t)}
    - \Hutch_{n_{\Psi}}(\widehat{g}_{\sigma}^m(t\mtx{I} - \mtx{A})).
    \label{equ:4-nystromchebyshev-spectral-density-decomposition}
\end{align}
It turns out that two of the three terms appearing in \refequ{equ:4-nystromchebyshev-spectral-density-decomposition}
are already quite familiar to us: In \refchp{chp:2-chebyshev} we have seen
how to compute $\widetilde{\phi}_{\sigma}^m(t)$ with the \gls{DGC} method, whereas
in \refchp{chp:3-nystrom}, $\widehat{\phi}_{\sigma}^m(t)$ is computed with the \gls{NC}
method. Only the last term is new. Using the standard Gaussian \glsfirst{random-matrix},
the \glsfirst{sketching-matrix}, and the definition of the Nystr\"om approximation
$\widehat{g}_{\sigma}^m(t\mtx{I} - \mtx{A})$ \refequ{equ:3-nystrom-nystrom-smoothing-kernel}
we may rewrite it as
\begin{equation}
    % \Hutch_{n_{\Psi}}(\widehat{g}_{\sigma}^m(t\mtx{I} - \mtx{A}))
    \frac{1}{n_{\Psi}}\Tr\big(
        (\underbrace{\mtx{\Psi}^{\top} g_{\sigma}^m(t\mtx{I} - \mtx{A}) \mtx{\Omega}}_{=\mtx{L}_2(t)^{\top}})
        (\underbrace{\mtx{\Omega}^{\top} g_{\sigma}^m(t\mtx{I} - \mtx{A}) \mtx{\Omega}}_{=\mtx{K}_1(t)})^{\dagger}
        (\underbrace{\mtx{\Omega}^{\top} g_{\sigma}^m(t\mtx{I} - \mtx{A}) \mtx{\Psi}}_{=\mtx{L}_2(t)})
      ).
\end{equation}
Notice how all the involved objects $l_1(t) \in \mathbb{R}$, $\mtx{L}_2(t) \in \mathbb{R}^{n_{\Omega} \times n_{\Psi}}$, and $\mtx{K}_1(t) \in \mathbb{R}^{n_{\Omega} \times n_{\Omega}}$
again have a form in which they can be expanded efficiently and for all $t$ simultaneously.\\

The implementation of this new method is similar to the \gls{DGC} and \gls{NC} methods (\refalg{alg:2-chebyshev-DGC} and \refalg{alg:3-nystrom-nystrom-chebyshev}).
The pseudocode for the \glsfirst{NCPP} is given in \refalg{alg:4-nystromchebyshev-nystrom-chebyshev-pp}.

\begin{algo}{Nystr\"om-Chebyshev++ method}{4-nystromchebyshev-nystrom-chebyshev-pp}
    \hspace*{\algorithmicindent} \textbf{Input:} Symmetric matrix $\mtx{A} \in \mathbb{R}^{n \times n}$, evaluation points $\{t_i\}_{i=1}^{n_t}$ \\
\hspace*{\algorithmicindent} \textbf{Parameters:} \Glsfirst{sketch-size}, \glsfirst{num-hutchinson-queries}, \glsfirst{chebyshev-degree} \\
\hspace*{\algorithmicindent} \textbf{Output:} Approximate evaluations of the spectral density $\{\widetilde{\phi}_{\sigma}^m(t_i)\}_{i=1}^{n_t}$
\begin{algorithmic}[1]
    \State Compute $\{\mu_l(t_i)\}_{l=0}^m$ for all $t_i$ using \refalg{alg:2-chebyshev-chebyshev-expansion}
    \State Compute $\{\nu_l(t_i)\}_{l=0}^{2m}$ for all $t_i$ using \refalg{alg:3-nystrom-chebyshev-exponentiation}
    \State Generate \glsfirst{sketching-matrix} $\in \mathbb{R}^{n \times n_{\Omega}}$ and \glsfirst{random-matrix} $\in \mathbb{R}^{n \times n_{\Psi}}$
    \State Initialize $[\mtx{V}_1, \mtx{V}_2, \mtx{V}_3] \gets [\mtx{0}_{n \times n_{\Omega}}, \mtx{\mtx{\Omega}}, \mtx{0}_{n \times n_{\Omega}}]$
    \State Initialize $[\mtx{W}_1, \mtx{W}_2, \mtx{W}_3] \gets [\mtx{0}_{n \times n_{\Psi}}, \mtx{\Psi}, \mtx{0}_{n \times n_{\Psi}}]$
    \State Initialize $[\mtx{K}_1(t_i), \mtx{K}_2(t_i)] \gets [\mtx{0}_{n_{\Omega} \times n_{\Omega}}, \mtx{0}_{n_{\Omega} \times n_{\Omega}}]$ for all $t_i$
    \State Initialize $[l_1(t_i), \mtx{L}_2(t_i)] \gets [0, \mtx{0}_{n_{\Omega} \times n_{\Psi}}]$ for all $t_i$
    \State Set ${\phi}_{\sigma}^m(t_i) \gets 0$ for $i=1,\dots,n_t$
    \For {$l = 0, \dots, 2m$}
      \State $\mtx{X} \gets \mtx{\mtx{\Omega}}^{\top} \mtx{V}_2$
      \State $\mtx{Y} \gets \mtx{\mtx{\Omega}}^{\top} \mtx{W}_2$
      \State $z \gets \Tr(\mtx{\Psi}^{\top} \mtx{W}_2$)
      \For {$i = 1, \dots, n_t$}
        \If {$l \leq m$}
            \State $\mtx{K}_1(t_i) \gets \mtx{K}_1(t_i) + \mu_l(t_i) \mtx{X}$
            \State $l_1(t_i) \gets l_1(t_i) + \nu_l(t_i) z$
        \EndIf
        \State $\mtx{K}_2(t_i) \gets \mtx{K}_2(t_i) + \nu_l(t_i) \mtx{X}$
        \State $\mtx{L}_2(t_i) \gets \mtx{L}_2(t_i) + \mu_l(t_i) \mtx{Y}$
      \EndFor
      \State $\mtx{V}_3 \gets (2 - \delta_{l0}) \mtx{A} \mtx{V}_2 - \mtx{V}_1$ \Comment{Chebyshev recurrence \refequ{equ:2-chebyshev-chebyshev-recursion}}
      \State $\mtx{V}_1 \gets \mtx{V}_2, \mtx{V}_2 \gets \mtx{V}_3$
      \State $\mtx{W}_3 \gets (2 - \delta_{l0}) \mtx{A} \mtx{W}_2 - \mtx{W}_1$ \Comment{Chebyshev recurrence \refequ{equ:2-chebyshev-chebyshev-recursion}}
      \State $\mtx{W}_1 \gets \mtx{W}_2, \mtx{W}_2 \gets \mtx{W}_3$
    \EndFor
    \For {$i = 1, \dots, n_t$}
      \State $\widetilde{\phi}_{\sigma}^m(t_i) \gets \Tr\left( \mtx{K}_1(t_i)^{\dagger}\mtx{K}_2(t_i) \right) + \frac{1}{n_{\Psi}} \left( l_1(t_i) + \Tr\left( \mtx{L}_2(t_i)^{\top} \mtx{K}_1(t_i)^{\dagger} \mtx{L}_2(t_i) \right)  \right) $
    \EndFor
\end{algorithmic}

\end{algo}

With the cost of a matrix-vector product denoted by
$c(n)$, and supposing we allocate the random vectors equally
to the low-rank approximation and the trace estimation, i.e. $n_{\Omega} \approx n_{\Psi}$,
we determine the computational complexity of the \gls{NCPP}
method to be $\mathcal{O}(m \log(m) n_t + m n_{\Omega}^2 n + m n_t n_{\Omega}^2 +  m c(n) n_{\Omega} + n_t n_{\Omega}^3)$, with
$\mathcal{O}(n n_{\Omega} + n_{\Omega}^2 n_t + m n_t)$ required additional storage.\\

It is not hard to extend this method to the other low-rank approximations
we have mentioned in \refsec{subsec:3-nystrom-other-low-rank}.

%%%%%%%%%%%%%%%%%%%%%%%%%%%%%%%%%%%%%%%%%%%%%%%%%%%%%%%%%%%%%%%%%%%%%%%%%%%%%%%%

\subsection{Implementation details}
\label{subsec:4-nystromchebyshev-implementation-details}

All of the implementation details for the \gls{DGC} method (\refsec{subsec:2-chebyshev-implementation-details})
and \gls{NC} method (\refsec{subsec:4-nystromchebyshev-implementation-details})
can be directly transfered to the \gls{NCPP} method.\\

An interesting observation we can take from \refequ{equ:3-nystrom-converted-generalized-eigenvalue-problem}
is that by identifying
\begin{equation}
    \mtx{D} = \mtx{W}_1 \mtx{\Gamma}_1^{-1/2} \mtx{X},
    \label{equ:4-nystromchebyshev-generalized-eigenvector}
\end{equation}
we can quickly and consistently compute
\begin{equation}
    \mtx{\Xi} = \mtx{D}^{\top} (\mtx{\Omega}^{\top} (g_{\sigma}^m(t\mtx{I} - \mtx{A}))^2 \mtx{\Omega}) \mtx{D}.
    \label{equ:4-nystromchebyshev-generalized-eigenvector-xi}
\end{equation}
%Unlike claimed in \cite[algorithm~4]{lin2017randomized}, $\mtx{D}$ is not -- in general --
%a generalized eigenvector of \refequ{equ:3-nystrom-low-rank-eigenvalue-problem}.
Hence, if we use the strategy from \refsec{subsec:3-nystrom-implementation-details}
to compute the first term on \reflin{lin:4-nystromchebyshev-nystrom-pp} in \refalg{alg:4-nystromchebyshev-nystrom-chebyshev-pp},
we can reuse its result to quickly compute an approximation of the last term as
\begin{align}
    \Tr(\mtx{\Xi})
    &= \Tr(\mtx{D}^{\top} (\mtx{\Omega}^{\top} (g_{\sigma}^m(t\mtx{I} - \mtx{A}))^2 \mtx{\Omega}) \mtx{D}) \notag \\
    &= \Tr( g_{\sigma}^m(t\mtx{I} - \mtx{A}) \mtx{\Omega} \mtx{D} \mtx{D}^{\top} \mtx{\Omega}^{\top} g_{\sigma}^m(t\mtx{I} - \mtx{A})) \notag \\
    &= \mathbb{E}\bigg[\frac{1}{n_{\Psi}} \Tr( (\underbrace{\mtx{\Psi}^{\top} g_{\sigma}^m(t\mtx{I} - \mtx{A}) \mtx{\Omega}}_{=\mtx{L}_2(t)^{\top}})
                                               (\mtx{D} \mtx{D}^{\top})
                                               (\underbrace{\mtx{\Omega}^{\top} g_{\sigma}^m(t\mtx{I} - \mtx{A}) \mtx{\Psi}}_{=\mtx{L}_2(t)})) \bigg].
    \label{equ:4-nystromchebyshev-generalized-eigenvector-trace}
\end{align}

%%%%%%%%%%%%%%%%%%%%%%%%%%%%%%%%%%%%%%%%%%%%%%%%%%%%%%%%%%%%%%%%%%%%%%%%%%%%%%%%

%\subsection{Theoretical analysis}
%\label{subsec:4-nystromchebyshev-analysis}
%
%
%These results now allow us to state an error guarantee for the output of the
%\gls{NCPP} algorithm.
%\begin{theorem}{Convergence of Nystr\"om-Chebyshev++ method}{4-nystromchebyshev-spectral-density}
%    The \gls{NCPP} method with $n_{\Omega} = n_{\Psi} = \mathcal{O}\left( \frac{\sqrt{\log(2/\delta)}}{\varepsilon} + \log(1/\delta) \right)$
%    satisfies with probability $1 - \delta$
%    \begin{equation}
%        \lVert \phi_{\sigma}(t) - \breve{\phi}_{\sigma}^m(t) \rVert _1 \leq.
%    \end{equation}
%\end{theorem}

%\Refthm{thm:4-nystromchebyshev-trace-correction} allows us to almost immediately
%also conclude $1/\varepsilon$ result for the higher order approximations
%\refequ{equ:3-nystrom-trace-generalization}.
%
%For RSVD ($k=2$), we use \cite[theorem~9.1]{halko2011finding}
%\begin{equation}
%    \lVert (\mtx{I} - \Pi_{\mtx{B}\mtx{\Omega}}) \mtx{B} \rVert _F \leq \lVert \mtx{\Lambda}_2 \rVert _F + \lVert \mtx{\Lambda}_2 \mtx{\Omega}_2 \mtx{\Omega}_1^{\dagger} \rVert _F.
%\end{equation}
%Result from parameter-dependent Nyström++
%\begin{equation}
%    %\mathbb{E}^n_{\Omega}\left[ \lVert \mtx{\Omega}_1^{\dagger} \rVert _F^2 \right] \leq C
%    \mathbb{E}^{n_{\Omega}/2} \left[ \lVert \mtx{\Lambda}_2 \mtx{\Omega}_2 \mtx{\Omega}_1^{\dagger} \rVert _F \right] \leq C(\sqrt{n_{\Omega}} \lVert \mtx{\Lambda}_2 \rVert _2 + \lVert \mtx{\Lambda}_2 \rVert _F),% \leq 2 C \cdot \frac{\operatorname{Tr}(\mtx{B})}{\sqrt{n_{\Omega}}}
%\end{equation}
%where $n_{\Omega}$ hence
%\begin{equation}
%    \mathbb{E}^{n_{\Omega}/2} \left[ \lVert (\mtx{I} - \Pi_{\mtx{B}\mtx{\Omega}}) \mtx{B} \rVert _F \right]
%    \leq (1 + 2C) \cdot \frac{\operatorname{Tr}(\mtx{B})}{\sqrt{n_{\Omega}}}.
%\end{equation}
%By Markov's and Minkowski's inequalities we have for $n_{\Omega} \geq 2 \log(1/\delta)$ with probability $\geq 1 - \delta$
%\begin{equation}
%    \int_{a}^{b}\lVert  (\mtx{I} - \Pi_{\mtx{B}(t)\mtx{\Omega}}) \mtx{B}(t) \rVert _F \mathrm{d}t \leq (1 + 2C) e \int_{a}^{b}\frac{\operatorname{Tr}(\mtx{B}(t))}{\sqrt{n_{\Omega}}} \mathrm{d}t
%\end{equation}
%
%For Nystr\"om with one subspace iteration ($k=3$) we apply \cite[lemma~5.2]{tropp2023randomized}
%to directly conclude from the RSVD case.
%
%Have result for all $k \geq 1$, but issue with implementation, since we do not
%have access to column sketch (whose QR-factorization is crucial for stability).
