\chapter*{Summary}
\label{chp:0-summary}

We study a family of methods used to approximately compute the spectral density
of large symmetric matrices. These methods are based on the polynomial expansion
of a smoothing function in combination with either a randomized trace estimation,
a randomized low-rank approximation, or both simultaneously.\\

Initially, we give an introduction and overview of procedures which are used
to compute the spectral density of matrices. Then, we proceed to showing how
some matrix functions can efficiently be computed based on
their Chebyshev expansion by computing discrete cosine transforms.
The use of the Hutchinson's stochastic trace estimator leads to a first
algorithm for approximating the spectral density, the Delta-Gauss-Chebyshev method.
Subsequently, an analysis of the structure of the involved matrix function
motivates the usage of a Nystr\"om low-rank factorization for reducing the
dimensionality of the problem, which leads us to the Nystr\"om-Chebyshev method.
To circumvent the inefficiency of one and the lack of robustness of the other
method they are combined into a third algorithm called the
Nystr\"om-Chebyshev++ method.\\

The techniques employed in these methods are motivated and introduced in a
rigorous manner. We present multiple implementation strategies for improved
computational speed, accuracy, and stability, and give a theoretical analysis
of each method. In various experiments the analysis of our algorithms is
numerically confirmed and their effectiveness is compared to other
conventionally used methods.\\

\textbf{Keywords:} Spectral density, Chebyshev expansion, stochastic trace estimation,
Hutchinson's estimator, randomized low-rank factorization, Nystr\"om approximation,
matrix functions, parameter dependent matrices
