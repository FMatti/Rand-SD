\chapter{Numerical experiments}
\label{chp:5-experiments}

In the previous chapters we have introduced multiple methods for approximating the
spectral density of a symmetric matrix $\mtx{A} \in \mathbb{R}^{n \times n}$. Now,
our goal is to compare these methods with each other and with other related methods
in terms of their accuracy and speed. In order to do so, we apply these algorithms
in multiple scenarios. We first consider a model problem from density functional theory
\cite{lin2017randomized} and discuss the convergence properties of our algorithms
in this setting (\refsec{sec:5-experiments-density-function}). Subsequently,
we show that the developed methods are also effective for other choices of
kernels, for example the Lorentzian kernel (\refsec{sec:5-experiments-haydock-method}).
Finally, we test the methods on various other matrices which we have found to
be commonly used in literature (\refsec{sec:5-experiments-various-matrices}).\\

The accuracy is measured in terms of the discrete relative $L^1$-error of the approximated
spectral density $\widetilde{\phi}_{\sigma}^{(m)}$ (also denoted $\widehat{\phi}_{\sigma}^{(m)}$ and
$\breve{\phi}_{\sigma}^{(m)}$) from the spectral density $\phi_{\sigma}$
which we obtain using standard eigenvalue solvers\footnote{\url{https://numpy.org/doc/stable/reference/generated/numpy.linalg.eigvalsh.html}}:
\begin{equation}
    \frac{\sum_{i=1}^{n_t} |\widetilde{\phi}_{\sigma}^{(m)}(t_i) - \phi_{\sigma}(t_i)|}{\sum_{i=1}^{n_t} |\phi_{\sigma}(t_i)|}.
    \label{equ:5-experiments-L1-error}
\end{equation}
We always use $n_t=100$ evenly spaced evaluation points which cover the whole spectrum of
$\mtx{A}$. The choice of this metric can be justified by the fact that this
error roughly corresponds to the midpoint quadrature rule \cite[chapter~9.2.1]{quarteroni2007numerical}
applied to the $L^1$-norm, for which our theoretical results hold.

%%%%%%%%%%%%%%%%%%%%%%%%%%%%%%%%%%%%%%%%%%%%%%%%%%%%%%%%%%%%%%%%%%%%%%%%%%%%%%%%

\section{Model problem from density functional theory}
\label{sec:5-experiments-density-function}

For our first example, we assemble the matrix which arises from the second order
finite difference discretization of the differential operator
\begin{equation}
    \mathcal{A} u(\vct{x}) = - \Delta u(\vct{x}) + V(\vct{x}) u(\vct{x})
    \label{equ:5-experiments-electronic-hamiltonian}
\end{equation}
for a uniform mesh of size $h=0.6$. The potential $V$ results from a
lattice whose primitive cell is of side-length $L=6$ and in whose center the
charge
\begin{equation}
    \alpha \exp(-\frac{\lVert \vct{x} \rVert _2^2}{ 2 \beta^2 })
    \label{equ:5-experiments-gaussian-cell}
\end{equation}
with $\alpha = 4$, $\beta = 2$ is located. The computational domain is chosen
to span $c \in \mathbb{N}$ primitive cells in every spatial dimension, hence, yielding
discretization matrices which are growing in size with $c$. In our experiments
we consider the three-dimensional case, but for visualization purposes, we
illustrate the potential in \reffig{fig:5-experiments-periodic-gaussian-well}
in two dimensions.\\

\begin{figure}[ht]
    \begin{subfigure}[b]{0.32\columnwidth}
        %% Creator: Matplotlib, PGF backend
%%
%% To include the figure in your LaTeX document, write
%%   \input{<filename>.pgf}
%%
%% Make sure the required packages are loaded in your preamble
%%   \usepackage{pgf}
%%
%% Also ensure that all the required font packages are loaded; for instance,
%% the lmodern package is sometimes necessary when using math font.
%%   \usepackage{lmodern}
%%
%% Figures using additional raster images can only be included by \input if
%% they are in the same directory as the main LaTeX file. For loading figures
%% from other directories you can use the `import` package
%%   \usepackage{import}
%%
%% and then include the figures with
%%   \import{<path to file>}{<filename>.pgf}
%%
%% Matplotlib used the following preamble
%%   \def\mathdefault#1{#1}
%%   \everymath=\expandafter{\the\everymath\displaystyle}
%%   
%%   \usepackage{fontspec}
%%   \setmainfont{DejaVuSerif.ttf}[Path=\detokenize{C:/Users/fabio/Documents/Work/MasterThesis/Rand-SD/.venv/Lib/site-packages/matplotlib/mpl-data/fonts/ttf/}]
%%   \setsansfont{DejaVuSans.ttf}[Path=\detokenize{C:/Users/fabio/Documents/Work/MasterThesis/Rand-SD/.venv/Lib/site-packages/matplotlib/mpl-data/fonts/ttf/}]
%%   \setmonofont{DejaVuSansMono.ttf}[Path=\detokenize{C:/Users/fabio/Documents/Work/MasterThesis/Rand-SD/.venv/Lib/site-packages/matplotlib/mpl-data/fonts/ttf/}]
%%   \makeatletter\@ifpackageloaded{underscore}{}{\usepackage[strings]{underscore}}\makeatother
%%
\begingroup%
\makeatletter%
\begin{pgfpicture}%
\pgfpathrectangle{\pgfpointorigin}{\pgfqpoint{1.979770in}{1.831897in}}%
\pgfusepath{use as bounding box, clip}%
\begin{pgfscope}%
\pgfsetbuttcap%
\pgfsetmiterjoin%
\definecolor{currentfill}{rgb}{1.000000,1.000000,1.000000}%
\pgfsetfillcolor{currentfill}%
\pgfsetlinewidth{0.000000pt}%
\definecolor{currentstroke}{rgb}{1.000000,1.000000,1.000000}%
\pgfsetstrokecolor{currentstroke}%
\pgfsetdash{}{0pt}%
\pgfpathmoveto{\pgfqpoint{0.000000in}{0.000000in}}%
\pgfpathlineto{\pgfqpoint{1.979770in}{0.000000in}}%
\pgfpathlineto{\pgfqpoint{1.979770in}{1.831897in}}%
\pgfpathlineto{\pgfqpoint{0.000000in}{1.831897in}}%
\pgfpathlineto{\pgfqpoint{0.000000in}{0.000000in}}%
\pgfpathclose%
\pgfusepath{fill}%
\end{pgfscope}%
\begin{pgfscope}%
\pgfsetbuttcap%
\pgfsetmiterjoin%
\definecolor{currentfill}{rgb}{1.000000,1.000000,1.000000}%
\pgfsetfillcolor{currentfill}%
\pgfsetlinewidth{0.000000pt}%
\definecolor{currentstroke}{rgb}{0.000000,0.000000,0.000000}%
\pgfsetstrokecolor{currentstroke}%
\pgfsetstrokeopacity{0.000000}%
\pgfsetdash{}{0pt}%
\pgfpathmoveto{\pgfqpoint{0.285588in}{0.331635in}}%
\pgfpathlineto{\pgfqpoint{1.835588in}{0.331635in}}%
\pgfpathlineto{\pgfqpoint{1.835588in}{1.679135in}}%
\pgfpathlineto{\pgfqpoint{0.285588in}{1.679135in}}%
\pgfpathlineto{\pgfqpoint{0.285588in}{0.331635in}}%
\pgfpathclose%
\pgfusepath{fill}%
\end{pgfscope}%
\begin{pgfscope}%
\pgfpathrectangle{\pgfqpoint{0.285588in}{0.331635in}}{\pgfqpoint{1.550000in}{1.347500in}}%
\pgfusepath{clip}%
\pgfsetbuttcap%
\pgfsetroundjoin%
\definecolor{currentfill}{rgb}{0.993545,0.862859,0.619299}%
\pgfsetfillcolor{currentfill}%
\pgfsetlinewidth{0.000000pt}%
\definecolor{currentstroke}{rgb}{0.000000,0.000000,0.000000}%
\pgfsetstrokecolor{currentstroke}%
\pgfsetdash{}{0pt}%
\pgfpathmoveto{\pgfqpoint{1.021446in}{0.859592in}}%
\pgfpathlineto{\pgfqpoint{1.037103in}{0.857015in}}%
\pgfpathlineto{\pgfqpoint{1.052759in}{0.855727in}}%
\pgfpathlineto{\pgfqpoint{1.068416in}{0.855727in}}%
\pgfpathlineto{\pgfqpoint{1.084072in}{0.857015in}}%
\pgfpathlineto{\pgfqpoint{1.099729in}{0.859592in}}%
\pgfpathlineto{\pgfqpoint{1.111384in}{0.862468in}}%
\pgfpathlineto{\pgfqpoint{1.115385in}{0.863541in}}%
\pgfpathlineto{\pgfqpoint{1.131042in}{0.869130in}}%
\pgfpathlineto{\pgfqpoint{1.146609in}{0.876079in}}%
\pgfpathlineto{\pgfqpoint{1.146699in}{0.876123in}}%
\pgfpathlineto{\pgfqpoint{1.162355in}{0.885324in}}%
\pgfpathlineto{\pgfqpoint{1.168741in}{0.889691in}}%
\pgfpathlineto{\pgfqpoint{1.178012in}{0.896776in}}%
\pgfpathlineto{\pgfqpoint{1.185519in}{0.903302in}}%
\pgfpathlineto{\pgfqpoint{1.193668in}{0.911362in}}%
\pgfpathlineto{\pgfqpoint{1.198691in}{0.916913in}}%
\pgfpathlineto{\pgfqpoint{1.209275in}{0.930524in}}%
\pgfpathlineto{\pgfqpoint{1.209325in}{0.930601in}}%
\pgfpathlineto{\pgfqpoint{1.217319in}{0.944135in}}%
\pgfpathlineto{\pgfqpoint{1.223748in}{0.957746in}}%
\pgfpathlineto{\pgfqpoint{1.224981in}{0.961225in}}%
\pgfpathlineto{\pgfqpoint{1.228290in}{0.971357in}}%
\pgfpathlineto{\pgfqpoint{1.231254in}{0.984968in}}%
\pgfpathlineto{\pgfqpoint{1.232735in}{0.998579in}}%
\pgfpathlineto{\pgfqpoint{1.232735in}{1.012191in}}%
\pgfpathlineto{\pgfqpoint{1.231254in}{1.025802in}}%
\pgfpathlineto{\pgfqpoint{1.228290in}{1.039413in}}%
\pgfpathlineto{\pgfqpoint{1.224981in}{1.049545in}}%
\pgfpathlineto{\pgfqpoint{1.223748in}{1.053024in}}%
\pgfpathlineto{\pgfqpoint{1.217319in}{1.066635in}}%
\pgfpathlineto{\pgfqpoint{1.209325in}{1.080169in}}%
\pgfpathlineto{\pgfqpoint{1.209275in}{1.080246in}}%
\pgfpathlineto{\pgfqpoint{1.198691in}{1.093857in}}%
\pgfpathlineto{\pgfqpoint{1.193668in}{1.099408in}}%
\pgfpathlineto{\pgfqpoint{1.185519in}{1.107468in}}%
\pgfpathlineto{\pgfqpoint{1.178012in}{1.113994in}}%
\pgfpathlineto{\pgfqpoint{1.168741in}{1.121079in}}%
\pgfpathlineto{\pgfqpoint{1.162355in}{1.125446in}}%
\pgfpathlineto{\pgfqpoint{1.146699in}{1.134647in}}%
\pgfpathlineto{\pgfqpoint{1.146609in}{1.134691in}}%
\pgfpathlineto{\pgfqpoint{1.131042in}{1.141640in}}%
\pgfpathlineto{\pgfqpoint{1.115385in}{1.147229in}}%
\pgfpathlineto{\pgfqpoint{1.111384in}{1.148302in}}%
\pgfpathlineto{\pgfqpoint{1.099729in}{1.151178in}}%
\pgfpathlineto{\pgfqpoint{1.084072in}{1.153755in}}%
\pgfpathlineto{\pgfqpoint{1.068416in}{1.155043in}}%
\pgfpathlineto{\pgfqpoint{1.052759in}{1.155043in}}%
\pgfpathlineto{\pgfqpoint{1.037103in}{1.153755in}}%
\pgfpathlineto{\pgfqpoint{1.021446in}{1.151178in}}%
\pgfpathlineto{\pgfqpoint{1.009791in}{1.148302in}}%
\pgfpathlineto{\pgfqpoint{1.005790in}{1.147229in}}%
\pgfpathlineto{\pgfqpoint{0.990133in}{1.141640in}}%
\pgfpathlineto{\pgfqpoint{0.974566in}{1.134691in}}%
\pgfpathlineto{\pgfqpoint{0.974476in}{1.134647in}}%
\pgfpathlineto{\pgfqpoint{0.958820in}{1.125446in}}%
\pgfpathlineto{\pgfqpoint{0.952434in}{1.121079in}}%
\pgfpathlineto{\pgfqpoint{0.943163in}{1.113994in}}%
\pgfpathlineto{\pgfqpoint{0.935656in}{1.107468in}}%
\pgfpathlineto{\pgfqpoint{0.927507in}{1.099408in}}%
\pgfpathlineto{\pgfqpoint{0.922484in}{1.093857in}}%
\pgfpathlineto{\pgfqpoint{0.911900in}{1.080246in}}%
\pgfpathlineto{\pgfqpoint{0.911850in}{1.080169in}}%
\pgfpathlineto{\pgfqpoint{0.903856in}{1.066635in}}%
\pgfpathlineto{\pgfqpoint{0.897427in}{1.053024in}}%
\pgfpathlineto{\pgfqpoint{0.896194in}{1.049545in}}%
\pgfpathlineto{\pgfqpoint{0.892885in}{1.039413in}}%
\pgfpathlineto{\pgfqpoint{0.889921in}{1.025802in}}%
\pgfpathlineto{\pgfqpoint{0.888440in}{1.012191in}}%
\pgfpathlineto{\pgfqpoint{0.888440in}{0.998579in}}%
\pgfpathlineto{\pgfqpoint{0.889921in}{0.984968in}}%
\pgfpathlineto{\pgfqpoint{0.892885in}{0.971357in}}%
\pgfpathlineto{\pgfqpoint{0.896194in}{0.961225in}}%
\pgfpathlineto{\pgfqpoint{0.897427in}{0.957746in}}%
\pgfpathlineto{\pgfqpoint{0.903856in}{0.944135in}}%
\pgfpathlineto{\pgfqpoint{0.911850in}{0.930601in}}%
\pgfpathlineto{\pgfqpoint{0.911900in}{0.930524in}}%
\pgfpathlineto{\pgfqpoint{0.922484in}{0.916913in}}%
\pgfpathlineto{\pgfqpoint{0.927507in}{0.911362in}}%
\pgfpathlineto{\pgfqpoint{0.935656in}{0.903302in}}%
\pgfpathlineto{\pgfqpoint{0.943163in}{0.896776in}}%
\pgfpathlineto{\pgfqpoint{0.952434in}{0.889691in}}%
\pgfpathlineto{\pgfqpoint{0.958820in}{0.885324in}}%
\pgfpathlineto{\pgfqpoint{0.974476in}{0.876123in}}%
\pgfpathlineto{\pgfqpoint{0.974566in}{0.876079in}}%
\pgfpathlineto{\pgfqpoint{0.990133in}{0.869130in}}%
\pgfpathlineto{\pgfqpoint{1.005790in}{0.863541in}}%
\pgfpathlineto{\pgfqpoint{1.009791in}{0.862468in}}%
\pgfpathlineto{\pgfqpoint{1.021446in}{0.859592in}}%
\pgfpathclose%
\pgfusepath{fill}%
\end{pgfscope}%
\begin{pgfscope}%
\pgfpathrectangle{\pgfqpoint{0.285588in}{0.331635in}}{\pgfqpoint{1.550000in}{1.347500in}}%
\pgfusepath{clip}%
\pgfsetbuttcap%
\pgfsetroundjoin%
\definecolor{currentfill}{rgb}{0.993326,0.602275,0.414390}%
\pgfsetfillcolor{currentfill}%
\pgfsetlinewidth{0.000000pt}%
\definecolor{currentstroke}{rgb}{0.000000,0.000000,0.000000}%
\pgfsetstrokecolor{currentstroke}%
\pgfsetdash{}{0pt}%
\pgfpathmoveto{\pgfqpoint{1.005790in}{0.725680in}}%
\pgfpathlineto{\pgfqpoint{1.021446in}{0.723372in}}%
\pgfpathlineto{\pgfqpoint{1.037103in}{0.721833in}}%
\pgfpathlineto{\pgfqpoint{1.052759in}{0.721064in}}%
\pgfpathlineto{\pgfqpoint{1.068416in}{0.721064in}}%
\pgfpathlineto{\pgfqpoint{1.084072in}{0.721833in}}%
\pgfpathlineto{\pgfqpoint{1.099729in}{0.723372in}}%
\pgfpathlineto{\pgfqpoint{1.115385in}{0.725680in}}%
\pgfpathlineto{\pgfqpoint{1.118838in}{0.726357in}}%
\pgfpathlineto{\pgfqpoint{1.131042in}{0.728800in}}%
\pgfpathlineto{\pgfqpoint{1.146699in}{0.732717in}}%
\pgfpathlineto{\pgfqpoint{1.162355in}{0.737419in}}%
\pgfpathlineto{\pgfqpoint{1.169646in}{0.739968in}}%
\pgfpathlineto{\pgfqpoint{1.178012in}{0.742972in}}%
\pgfpathlineto{\pgfqpoint{1.193668in}{0.749386in}}%
\pgfpathlineto{\pgfqpoint{1.202789in}{0.753579in}}%
\pgfpathlineto{\pgfqpoint{1.209325in}{0.756685in}}%
\pgfpathlineto{\pgfqpoint{1.224981in}{0.764921in}}%
\pgfpathlineto{\pgfqpoint{1.228925in}{0.767191in}}%
\pgfpathlineto{\pgfqpoint{1.240638in}{0.774198in}}%
\pgfpathlineto{\pgfqpoint{1.250806in}{0.780802in}}%
\pgfpathlineto{\pgfqpoint{1.256295in}{0.784530in}}%
\pgfpathlineto{\pgfqpoint{1.269820in}{0.794413in}}%
\pgfpathlineto{\pgfqpoint{1.271951in}{0.796052in}}%
\pgfpathlineto{\pgfqpoint{1.286541in}{0.808024in}}%
\pgfpathlineto{\pgfqpoint{1.287608in}{0.808951in}}%
\pgfpathlineto{\pgfqpoint{1.301379in}{0.821635in}}%
\pgfpathlineto{\pgfqpoint{1.303264in}{0.823488in}}%
\pgfpathlineto{\pgfqpoint{1.314632in}{0.835246in}}%
\pgfpathlineto{\pgfqpoint{1.318921in}{0.840018in}}%
\pgfpathlineto{\pgfqpoint{1.326517in}{0.848857in}}%
\pgfpathlineto{\pgfqpoint{1.334577in}{0.859040in}}%
\pgfpathlineto{\pgfqpoint{1.337188in}{0.862468in}}%
\pgfpathlineto{\pgfqpoint{1.346662in}{0.876079in}}%
\pgfpathlineto{\pgfqpoint{1.350234in}{0.881761in}}%
\pgfpathlineto{\pgfqpoint{1.355058in}{0.889691in}}%
\pgfpathlineto{\pgfqpoint{1.362435in}{0.903302in}}%
\pgfpathlineto{\pgfqpoint{1.365891in}{0.910574in}}%
\pgfpathlineto{\pgfqpoint{1.368823in}{0.916913in}}%
\pgfpathlineto{\pgfqpoint{1.374232in}{0.930524in}}%
\pgfpathlineto{\pgfqpoint{1.378737in}{0.944135in}}%
\pgfpathlineto{\pgfqpoint{1.381547in}{0.954745in}}%
\pgfpathlineto{\pgfqpoint{1.382326in}{0.957746in}}%
\pgfpathlineto{\pgfqpoint{1.384981in}{0.971357in}}%
\pgfpathlineto{\pgfqpoint{1.386751in}{0.984968in}}%
\pgfpathlineto{\pgfqpoint{1.387635in}{0.998579in}}%
\pgfpathlineto{\pgfqpoint{1.387635in}{1.012191in}}%
\pgfpathlineto{\pgfqpoint{1.386751in}{1.025802in}}%
\pgfpathlineto{\pgfqpoint{1.384981in}{1.039413in}}%
\pgfpathlineto{\pgfqpoint{1.382326in}{1.053024in}}%
\pgfpathlineto{\pgfqpoint{1.381547in}{1.056025in}}%
\pgfpathlineto{\pgfqpoint{1.378737in}{1.066635in}}%
\pgfpathlineto{\pgfqpoint{1.374232in}{1.080246in}}%
\pgfpathlineto{\pgfqpoint{1.368823in}{1.093857in}}%
\pgfpathlineto{\pgfqpoint{1.365891in}{1.100196in}}%
\pgfpathlineto{\pgfqpoint{1.362435in}{1.107468in}}%
\pgfpathlineto{\pgfqpoint{1.355058in}{1.121079in}}%
\pgfpathlineto{\pgfqpoint{1.350234in}{1.129009in}}%
\pgfpathlineto{\pgfqpoint{1.346662in}{1.134691in}}%
\pgfpathlineto{\pgfqpoint{1.337188in}{1.148302in}}%
\pgfpathlineto{\pgfqpoint{1.334577in}{1.151730in}}%
\pgfpathlineto{\pgfqpoint{1.326517in}{1.161913in}}%
\pgfpathlineto{\pgfqpoint{1.318921in}{1.170752in}}%
\pgfpathlineto{\pgfqpoint{1.314632in}{1.175524in}}%
\pgfpathlineto{\pgfqpoint{1.303264in}{1.187282in}}%
\pgfpathlineto{\pgfqpoint{1.301379in}{1.189135in}}%
\pgfpathlineto{\pgfqpoint{1.287608in}{1.201819in}}%
\pgfpathlineto{\pgfqpoint{1.286541in}{1.202746in}}%
\pgfpathlineto{\pgfqpoint{1.271951in}{1.214718in}}%
\pgfpathlineto{\pgfqpoint{1.269820in}{1.216357in}}%
\pgfpathlineto{\pgfqpoint{1.256295in}{1.226240in}}%
\pgfpathlineto{\pgfqpoint{1.250806in}{1.229968in}}%
\pgfpathlineto{\pgfqpoint{1.240638in}{1.236572in}}%
\pgfpathlineto{\pgfqpoint{1.228925in}{1.243579in}}%
\pgfpathlineto{\pgfqpoint{1.224981in}{1.245849in}}%
\pgfpathlineto{\pgfqpoint{1.209325in}{1.254085in}}%
\pgfpathlineto{\pgfqpoint{1.202789in}{1.257191in}}%
\pgfpathlineto{\pgfqpoint{1.193668in}{1.261384in}}%
\pgfpathlineto{\pgfqpoint{1.178012in}{1.267798in}}%
\pgfpathlineto{\pgfqpoint{1.169646in}{1.270802in}}%
\pgfpathlineto{\pgfqpoint{1.162355in}{1.273351in}}%
\pgfpathlineto{\pgfqpoint{1.146699in}{1.278053in}}%
\pgfpathlineto{\pgfqpoint{1.131042in}{1.281970in}}%
\pgfpathlineto{\pgfqpoint{1.118838in}{1.284413in}}%
\pgfpathlineto{\pgfqpoint{1.115385in}{1.285090in}}%
\pgfpathlineto{\pgfqpoint{1.099729in}{1.287398in}}%
\pgfpathlineto{\pgfqpoint{1.084072in}{1.288937in}}%
\pgfpathlineto{\pgfqpoint{1.068416in}{1.289706in}}%
\pgfpathlineto{\pgfqpoint{1.052759in}{1.289706in}}%
\pgfpathlineto{\pgfqpoint{1.037103in}{1.288937in}}%
\pgfpathlineto{\pgfqpoint{1.021446in}{1.287398in}}%
\pgfpathlineto{\pgfqpoint{1.005790in}{1.285090in}}%
\pgfpathlineto{\pgfqpoint{1.002337in}{1.284413in}}%
\pgfpathlineto{\pgfqpoint{0.990133in}{1.281970in}}%
\pgfpathlineto{\pgfqpoint{0.974476in}{1.278053in}}%
\pgfpathlineto{\pgfqpoint{0.958820in}{1.273351in}}%
\pgfpathlineto{\pgfqpoint{0.951529in}{1.270802in}}%
\pgfpathlineto{\pgfqpoint{0.943163in}{1.267798in}}%
\pgfpathlineto{\pgfqpoint{0.927507in}{1.261384in}}%
\pgfpathlineto{\pgfqpoint{0.918386in}{1.257191in}}%
\pgfpathlineto{\pgfqpoint{0.911850in}{1.254085in}}%
\pgfpathlineto{\pgfqpoint{0.896194in}{1.245849in}}%
\pgfpathlineto{\pgfqpoint{0.892250in}{1.243579in}}%
\pgfpathlineto{\pgfqpoint{0.880537in}{1.236572in}}%
\pgfpathlineto{\pgfqpoint{0.870369in}{1.229968in}}%
\pgfpathlineto{\pgfqpoint{0.864880in}{1.226240in}}%
\pgfpathlineto{\pgfqpoint{0.851355in}{1.216357in}}%
\pgfpathlineto{\pgfqpoint{0.849224in}{1.214718in}}%
\pgfpathlineto{\pgfqpoint{0.834634in}{1.202746in}}%
\pgfpathlineto{\pgfqpoint{0.833567in}{1.201819in}}%
\pgfpathlineto{\pgfqpoint{0.819796in}{1.189135in}}%
\pgfpathlineto{\pgfqpoint{0.817911in}{1.187282in}}%
\pgfpathlineto{\pgfqpoint{0.806543in}{1.175524in}}%
\pgfpathlineto{\pgfqpoint{0.802254in}{1.170752in}}%
\pgfpathlineto{\pgfqpoint{0.794658in}{1.161913in}}%
\pgfpathlineto{\pgfqpoint{0.786598in}{1.151730in}}%
\pgfpathlineto{\pgfqpoint{0.783987in}{1.148302in}}%
\pgfpathlineto{\pgfqpoint{0.774513in}{1.134691in}}%
\pgfpathlineto{\pgfqpoint{0.770941in}{1.129009in}}%
\pgfpathlineto{\pgfqpoint{0.766117in}{1.121079in}}%
\pgfpathlineto{\pgfqpoint{0.758740in}{1.107468in}}%
\pgfpathlineto{\pgfqpoint{0.755284in}{1.100196in}}%
\pgfpathlineto{\pgfqpoint{0.752352in}{1.093857in}}%
\pgfpathlineto{\pgfqpoint{0.746943in}{1.080246in}}%
\pgfpathlineto{\pgfqpoint{0.742438in}{1.066635in}}%
\pgfpathlineto{\pgfqpoint{0.739628in}{1.056025in}}%
\pgfpathlineto{\pgfqpoint{0.738849in}{1.053024in}}%
\pgfpathlineto{\pgfqpoint{0.736194in}{1.039413in}}%
\pgfpathlineto{\pgfqpoint{0.734424in}{1.025802in}}%
\pgfpathlineto{\pgfqpoint{0.733540in}{1.012191in}}%
\pgfpathlineto{\pgfqpoint{0.733540in}{0.998579in}}%
\pgfpathlineto{\pgfqpoint{0.734424in}{0.984968in}}%
\pgfpathlineto{\pgfqpoint{0.736194in}{0.971357in}}%
\pgfpathlineto{\pgfqpoint{0.738849in}{0.957746in}}%
\pgfpathlineto{\pgfqpoint{0.739628in}{0.954745in}}%
\pgfpathlineto{\pgfqpoint{0.742438in}{0.944135in}}%
\pgfpathlineto{\pgfqpoint{0.746943in}{0.930524in}}%
\pgfpathlineto{\pgfqpoint{0.752352in}{0.916913in}}%
\pgfpathlineto{\pgfqpoint{0.755284in}{0.910574in}}%
\pgfpathlineto{\pgfqpoint{0.758740in}{0.903302in}}%
\pgfpathlineto{\pgfqpoint{0.766117in}{0.889691in}}%
\pgfpathlineto{\pgfqpoint{0.770941in}{0.881761in}}%
\pgfpathlineto{\pgfqpoint{0.774513in}{0.876079in}}%
\pgfpathlineto{\pgfqpoint{0.783987in}{0.862468in}}%
\pgfpathlineto{\pgfqpoint{0.786598in}{0.859040in}}%
\pgfpathlineto{\pgfqpoint{0.794658in}{0.848857in}}%
\pgfpathlineto{\pgfqpoint{0.802254in}{0.840018in}}%
\pgfpathlineto{\pgfqpoint{0.806543in}{0.835246in}}%
\pgfpathlineto{\pgfqpoint{0.817911in}{0.823488in}}%
\pgfpathlineto{\pgfqpoint{0.819796in}{0.821635in}}%
\pgfpathlineto{\pgfqpoint{0.833567in}{0.808951in}}%
\pgfpathlineto{\pgfqpoint{0.834634in}{0.808024in}}%
\pgfpathlineto{\pgfqpoint{0.849224in}{0.796052in}}%
\pgfpathlineto{\pgfqpoint{0.851355in}{0.794413in}}%
\pgfpathlineto{\pgfqpoint{0.864880in}{0.784530in}}%
\pgfpathlineto{\pgfqpoint{0.870369in}{0.780802in}}%
\pgfpathlineto{\pgfqpoint{0.880537in}{0.774198in}}%
\pgfpathlineto{\pgfqpoint{0.892250in}{0.767191in}}%
\pgfpathlineto{\pgfqpoint{0.896194in}{0.764921in}}%
\pgfpathlineto{\pgfqpoint{0.911850in}{0.756685in}}%
\pgfpathlineto{\pgfqpoint{0.918386in}{0.753579in}}%
\pgfpathlineto{\pgfqpoint{0.927507in}{0.749386in}}%
\pgfpathlineto{\pgfqpoint{0.943163in}{0.742972in}}%
\pgfpathlineto{\pgfqpoint{0.951529in}{0.739968in}}%
\pgfpathlineto{\pgfqpoint{0.958820in}{0.737419in}}%
\pgfpathlineto{\pgfqpoint{0.974476in}{0.732717in}}%
\pgfpathlineto{\pgfqpoint{0.990133in}{0.728800in}}%
\pgfpathlineto{\pgfqpoint{1.002337in}{0.726357in}}%
\pgfpathlineto{\pgfqpoint{1.005790in}{0.725680in}}%
\pgfpathclose%
\pgfpathmoveto{\pgfqpoint{1.009791in}{0.862468in}}%
\pgfpathlineto{\pgfqpoint{1.005790in}{0.863541in}}%
\pgfpathlineto{\pgfqpoint{0.990133in}{0.869130in}}%
\pgfpathlineto{\pgfqpoint{0.974566in}{0.876079in}}%
\pgfpathlineto{\pgfqpoint{0.974476in}{0.876123in}}%
\pgfpathlineto{\pgfqpoint{0.958820in}{0.885324in}}%
\pgfpathlineto{\pgfqpoint{0.952434in}{0.889691in}}%
\pgfpathlineto{\pgfqpoint{0.943163in}{0.896776in}}%
\pgfpathlineto{\pgfqpoint{0.935656in}{0.903302in}}%
\pgfpathlineto{\pgfqpoint{0.927507in}{0.911362in}}%
\pgfpathlineto{\pgfqpoint{0.922484in}{0.916913in}}%
\pgfpathlineto{\pgfqpoint{0.911900in}{0.930524in}}%
\pgfpathlineto{\pgfqpoint{0.911850in}{0.930601in}}%
\pgfpathlineto{\pgfqpoint{0.903856in}{0.944135in}}%
\pgfpathlineto{\pgfqpoint{0.897427in}{0.957746in}}%
\pgfpathlineto{\pgfqpoint{0.896194in}{0.961225in}}%
\pgfpathlineto{\pgfqpoint{0.892885in}{0.971357in}}%
\pgfpathlineto{\pgfqpoint{0.889921in}{0.984968in}}%
\pgfpathlineto{\pgfqpoint{0.888440in}{0.998579in}}%
\pgfpathlineto{\pgfqpoint{0.888440in}{1.012191in}}%
\pgfpathlineto{\pgfqpoint{0.889921in}{1.025802in}}%
\pgfpathlineto{\pgfqpoint{0.892885in}{1.039413in}}%
\pgfpathlineto{\pgfqpoint{0.896194in}{1.049545in}}%
\pgfpathlineto{\pgfqpoint{0.897427in}{1.053024in}}%
\pgfpathlineto{\pgfqpoint{0.903856in}{1.066635in}}%
\pgfpathlineto{\pgfqpoint{0.911850in}{1.080169in}}%
\pgfpathlineto{\pgfqpoint{0.911900in}{1.080246in}}%
\pgfpathlineto{\pgfqpoint{0.922484in}{1.093857in}}%
\pgfpathlineto{\pgfqpoint{0.927507in}{1.099408in}}%
\pgfpathlineto{\pgfqpoint{0.935656in}{1.107468in}}%
\pgfpathlineto{\pgfqpoint{0.943163in}{1.113994in}}%
\pgfpathlineto{\pgfqpoint{0.952434in}{1.121079in}}%
\pgfpathlineto{\pgfqpoint{0.958820in}{1.125446in}}%
\pgfpathlineto{\pgfqpoint{0.974476in}{1.134647in}}%
\pgfpathlineto{\pgfqpoint{0.974566in}{1.134691in}}%
\pgfpathlineto{\pgfqpoint{0.990133in}{1.141640in}}%
\pgfpathlineto{\pgfqpoint{1.005790in}{1.147229in}}%
\pgfpathlineto{\pgfqpoint{1.009791in}{1.148302in}}%
\pgfpathlineto{\pgfqpoint{1.021446in}{1.151178in}}%
\pgfpathlineto{\pgfqpoint{1.037103in}{1.153755in}}%
\pgfpathlineto{\pgfqpoint{1.052759in}{1.155043in}}%
\pgfpathlineto{\pgfqpoint{1.068416in}{1.155043in}}%
\pgfpathlineto{\pgfqpoint{1.084072in}{1.153755in}}%
\pgfpathlineto{\pgfqpoint{1.099729in}{1.151178in}}%
\pgfpathlineto{\pgfqpoint{1.111384in}{1.148302in}}%
\pgfpathlineto{\pgfqpoint{1.115385in}{1.147229in}}%
\pgfpathlineto{\pgfqpoint{1.131042in}{1.141640in}}%
\pgfpathlineto{\pgfqpoint{1.146609in}{1.134691in}}%
\pgfpathlineto{\pgfqpoint{1.146699in}{1.134647in}}%
\pgfpathlineto{\pgfqpoint{1.162355in}{1.125446in}}%
\pgfpathlineto{\pgfqpoint{1.168741in}{1.121079in}}%
\pgfpathlineto{\pgfqpoint{1.178012in}{1.113994in}}%
\pgfpathlineto{\pgfqpoint{1.185519in}{1.107468in}}%
\pgfpathlineto{\pgfqpoint{1.193668in}{1.099408in}}%
\pgfpathlineto{\pgfqpoint{1.198691in}{1.093857in}}%
\pgfpathlineto{\pgfqpoint{1.209275in}{1.080246in}}%
\pgfpathlineto{\pgfqpoint{1.209325in}{1.080169in}}%
\pgfpathlineto{\pgfqpoint{1.217319in}{1.066635in}}%
\pgfpathlineto{\pgfqpoint{1.223748in}{1.053024in}}%
\pgfpathlineto{\pgfqpoint{1.224981in}{1.049545in}}%
\pgfpathlineto{\pgfqpoint{1.228290in}{1.039413in}}%
\pgfpathlineto{\pgfqpoint{1.231254in}{1.025802in}}%
\pgfpathlineto{\pgfqpoint{1.232735in}{1.012191in}}%
\pgfpathlineto{\pgfqpoint{1.232735in}{0.998579in}}%
\pgfpathlineto{\pgfqpoint{1.231254in}{0.984968in}}%
\pgfpathlineto{\pgfqpoint{1.228290in}{0.971357in}}%
\pgfpathlineto{\pgfqpoint{1.224981in}{0.961225in}}%
\pgfpathlineto{\pgfqpoint{1.223748in}{0.957746in}}%
\pgfpathlineto{\pgfqpoint{1.217319in}{0.944135in}}%
\pgfpathlineto{\pgfqpoint{1.209325in}{0.930601in}}%
\pgfpathlineto{\pgfqpoint{1.209275in}{0.930524in}}%
\pgfpathlineto{\pgfqpoint{1.198691in}{0.916913in}}%
\pgfpathlineto{\pgfqpoint{1.193668in}{0.911362in}}%
\pgfpathlineto{\pgfqpoint{1.185519in}{0.903302in}}%
\pgfpathlineto{\pgfqpoint{1.178012in}{0.896776in}}%
\pgfpathlineto{\pgfqpoint{1.168741in}{0.889691in}}%
\pgfpathlineto{\pgfqpoint{1.162355in}{0.885324in}}%
\pgfpathlineto{\pgfqpoint{1.146699in}{0.876123in}}%
\pgfpathlineto{\pgfqpoint{1.146609in}{0.876079in}}%
\pgfpathlineto{\pgfqpoint{1.131042in}{0.869130in}}%
\pgfpathlineto{\pgfqpoint{1.115385in}{0.863541in}}%
\pgfpathlineto{\pgfqpoint{1.111384in}{0.862468in}}%
\pgfpathlineto{\pgfqpoint{1.099729in}{0.859592in}}%
\pgfpathlineto{\pgfqpoint{1.084072in}{0.857015in}}%
\pgfpathlineto{\pgfqpoint{1.068416in}{0.855727in}}%
\pgfpathlineto{\pgfqpoint{1.052759in}{0.855727in}}%
\pgfpathlineto{\pgfqpoint{1.037103in}{0.857015in}}%
\pgfpathlineto{\pgfqpoint{1.021446in}{0.859592in}}%
\pgfpathlineto{\pgfqpoint{1.009791in}{0.862468in}}%
\pgfpathclose%
\pgfusepath{fill}%
\end{pgfscope}%
\begin{pgfscope}%
\pgfpathrectangle{\pgfqpoint{0.285588in}{0.331635in}}{\pgfqpoint{1.550000in}{1.347500in}}%
\pgfusepath{clip}%
\pgfsetbuttcap%
\pgfsetroundjoin%
\definecolor{currentfill}{rgb}{0.921884,0.341098,0.377376}%
\pgfsetfillcolor{currentfill}%
\pgfsetlinewidth{0.000000pt}%
\definecolor{currentstroke}{rgb}{0.000000,0.000000,0.000000}%
\pgfsetstrokecolor{currentstroke}%
\pgfsetdash{}{0pt}%
\pgfpathmoveto{\pgfqpoint{0.974476in}{0.617467in}}%
\pgfpathlineto{\pgfqpoint{0.990133in}{0.614007in}}%
\pgfpathlineto{\pgfqpoint{1.005790in}{0.611240in}}%
\pgfpathlineto{\pgfqpoint{1.021446in}{0.609165in}}%
\pgfpathlineto{\pgfqpoint{1.037103in}{0.607783in}}%
\pgfpathlineto{\pgfqpoint{1.052759in}{0.607092in}}%
\pgfpathlineto{\pgfqpoint{1.068416in}{0.607092in}}%
\pgfpathlineto{\pgfqpoint{1.084072in}{0.607783in}}%
\pgfpathlineto{\pgfqpoint{1.099729in}{0.609165in}}%
\pgfpathlineto{\pgfqpoint{1.115385in}{0.611240in}}%
\pgfpathlineto{\pgfqpoint{1.131042in}{0.614007in}}%
\pgfpathlineto{\pgfqpoint{1.146699in}{0.617467in}}%
\pgfpathlineto{\pgfqpoint{1.146704in}{0.617468in}}%
\pgfpathlineto{\pgfqpoint{1.162355in}{0.621556in}}%
\pgfpathlineto{\pgfqpoint{1.178012in}{0.626329in}}%
\pgfpathlineto{\pgfqpoint{1.191649in}{0.631079in}}%
\pgfpathlineto{\pgfqpoint{1.193668in}{0.631779in}}%
\pgfpathlineto{\pgfqpoint{1.209325in}{0.637851in}}%
\pgfpathlineto{\pgfqpoint{1.224981in}{0.644600in}}%
\pgfpathlineto{\pgfqpoint{1.225174in}{0.644691in}}%
\pgfpathlineto{\pgfqpoint{1.240638in}{0.651971in}}%
\pgfpathlineto{\pgfqpoint{1.252987in}{0.658302in}}%
\pgfpathlineto{\pgfqpoint{1.256295in}{0.660006in}}%
\pgfpathlineto{\pgfqpoint{1.271951in}{0.668685in}}%
\pgfpathlineto{\pgfqpoint{1.277396in}{0.671913in}}%
\pgfpathlineto{\pgfqpoint{1.287608in}{0.678030in}}%
\pgfpathlineto{\pgfqpoint{1.299366in}{0.685524in}}%
\pgfpathlineto{\pgfqpoint{1.303264in}{0.688046in}}%
\pgfpathlineto{\pgfqpoint{1.318921in}{0.698750in}}%
\pgfpathlineto{\pgfqpoint{1.319457in}{0.699135in}}%
\pgfpathlineto{\pgfqpoint{1.334577in}{0.710197in}}%
\pgfpathlineto{\pgfqpoint{1.337910in}{0.712746in}}%
\pgfpathlineto{\pgfqpoint{1.350234in}{0.722409in}}%
\pgfpathlineto{\pgfqpoint{1.355075in}{0.726357in}}%
\pgfpathlineto{\pgfqpoint{1.365891in}{0.735443in}}%
\pgfpathlineto{\pgfqpoint{1.371096in}{0.739968in}}%
\pgfpathlineto{\pgfqpoint{1.381547in}{0.749371in}}%
\pgfpathlineto{\pgfqpoint{1.386088in}{0.753579in}}%
\pgfpathlineto{\pgfqpoint{1.397204in}{0.764293in}}%
\pgfpathlineto{\pgfqpoint{1.400136in}{0.767191in}}%
\pgfpathlineto{\pgfqpoint{1.412860in}{0.780335in}}%
\pgfpathlineto{\pgfqpoint{1.413303in}{0.780802in}}%
\pgfpathlineto{\pgfqpoint{1.425616in}{0.794413in}}%
\pgfpathlineto{\pgfqpoint{1.428517in}{0.797801in}}%
\pgfpathlineto{\pgfqpoint{1.437137in}{0.808024in}}%
\pgfpathlineto{\pgfqpoint{1.444173in}{0.816901in}}%
\pgfpathlineto{\pgfqpoint{1.447887in}{0.821635in}}%
\pgfpathlineto{\pgfqpoint{1.457870in}{0.835246in}}%
\pgfpathlineto{\pgfqpoint{1.459830in}{0.838122in}}%
\pgfpathlineto{\pgfqpoint{1.467112in}{0.848857in}}%
\pgfpathlineto{\pgfqpoint{1.475486in}{0.862301in}}%
\pgfpathlineto{\pgfqpoint{1.475591in}{0.862468in}}%
\pgfpathlineto{\pgfqpoint{1.483354in}{0.876079in}}%
\pgfpathlineto{\pgfqpoint{1.490339in}{0.889691in}}%
\pgfpathlineto{\pgfqpoint{1.491143in}{0.891446in}}%
\pgfpathlineto{\pgfqpoint{1.496607in}{0.903302in}}%
\pgfpathlineto{\pgfqpoint{1.502098in}{0.916913in}}%
\pgfpathlineto{\pgfqpoint{1.506800in}{0.930519in}}%
\pgfpathlineto{\pgfqpoint{1.506801in}{0.930524in}}%
\pgfpathlineto{\pgfqpoint{1.510781in}{0.944135in}}%
\pgfpathlineto{\pgfqpoint{1.513964in}{0.957746in}}%
\pgfpathlineto{\pgfqpoint{1.516350in}{0.971357in}}%
\pgfpathlineto{\pgfqpoint{1.517941in}{0.984968in}}%
\pgfpathlineto{\pgfqpoint{1.518736in}{0.998579in}}%
\pgfpathlineto{\pgfqpoint{1.518736in}{1.012191in}}%
\pgfpathlineto{\pgfqpoint{1.517941in}{1.025802in}}%
\pgfpathlineto{\pgfqpoint{1.516350in}{1.039413in}}%
\pgfpathlineto{\pgfqpoint{1.513964in}{1.053024in}}%
\pgfpathlineto{\pgfqpoint{1.510781in}{1.066635in}}%
\pgfpathlineto{\pgfqpoint{1.506801in}{1.080246in}}%
\pgfpathlineto{\pgfqpoint{1.506800in}{1.080251in}}%
\pgfpathlineto{\pgfqpoint{1.502098in}{1.093857in}}%
\pgfpathlineto{\pgfqpoint{1.496607in}{1.107468in}}%
\pgfpathlineto{\pgfqpoint{1.491143in}{1.119324in}}%
\pgfpathlineto{\pgfqpoint{1.490339in}{1.121079in}}%
\pgfpathlineto{\pgfqpoint{1.483354in}{1.134691in}}%
\pgfpathlineto{\pgfqpoint{1.475591in}{1.148302in}}%
\pgfpathlineto{\pgfqpoint{1.475486in}{1.148469in}}%
\pgfpathlineto{\pgfqpoint{1.467112in}{1.161913in}}%
\pgfpathlineto{\pgfqpoint{1.459830in}{1.172648in}}%
\pgfpathlineto{\pgfqpoint{1.457870in}{1.175524in}}%
\pgfpathlineto{\pgfqpoint{1.447887in}{1.189135in}}%
\pgfpathlineto{\pgfqpoint{1.444173in}{1.193869in}}%
\pgfpathlineto{\pgfqpoint{1.437137in}{1.202746in}}%
\pgfpathlineto{\pgfqpoint{1.428517in}{1.212969in}}%
\pgfpathlineto{\pgfqpoint{1.425616in}{1.216357in}}%
\pgfpathlineto{\pgfqpoint{1.413303in}{1.229968in}}%
\pgfpathlineto{\pgfqpoint{1.412860in}{1.230435in}}%
\pgfpathlineto{\pgfqpoint{1.400136in}{1.243579in}}%
\pgfpathlineto{\pgfqpoint{1.397204in}{1.246477in}}%
\pgfpathlineto{\pgfqpoint{1.386088in}{1.257191in}}%
\pgfpathlineto{\pgfqpoint{1.381547in}{1.261399in}}%
\pgfpathlineto{\pgfqpoint{1.371096in}{1.270802in}}%
\pgfpathlineto{\pgfqpoint{1.365891in}{1.275327in}}%
\pgfpathlineto{\pgfqpoint{1.355075in}{1.284413in}}%
\pgfpathlineto{\pgfqpoint{1.350234in}{1.288361in}}%
\pgfpathlineto{\pgfqpoint{1.337910in}{1.298024in}}%
\pgfpathlineto{\pgfqpoint{1.334577in}{1.300573in}}%
\pgfpathlineto{\pgfqpoint{1.319457in}{1.311635in}}%
\pgfpathlineto{\pgfqpoint{1.318921in}{1.312020in}}%
\pgfpathlineto{\pgfqpoint{1.303264in}{1.322724in}}%
\pgfpathlineto{\pgfqpoint{1.299366in}{1.325246in}}%
\pgfpathlineto{\pgfqpoint{1.287608in}{1.332740in}}%
\pgfpathlineto{\pgfqpoint{1.277396in}{1.338857in}}%
\pgfpathlineto{\pgfqpoint{1.271951in}{1.342085in}}%
\pgfpathlineto{\pgfqpoint{1.256295in}{1.350764in}}%
\pgfpathlineto{\pgfqpoint{1.252987in}{1.352468in}}%
\pgfpathlineto{\pgfqpoint{1.240638in}{1.358799in}}%
\pgfpathlineto{\pgfqpoint{1.225174in}{1.366079in}}%
\pgfpathlineto{\pgfqpoint{1.224981in}{1.366170in}}%
\pgfpathlineto{\pgfqpoint{1.209325in}{1.372919in}}%
\pgfpathlineto{\pgfqpoint{1.193668in}{1.378991in}}%
\pgfpathlineto{\pgfqpoint{1.191649in}{1.379691in}}%
\pgfpathlineto{\pgfqpoint{1.178012in}{1.384441in}}%
\pgfpathlineto{\pgfqpoint{1.162355in}{1.389214in}}%
\pgfpathlineto{\pgfqpoint{1.146704in}{1.393302in}}%
\pgfpathlineto{\pgfqpoint{1.146699in}{1.393303in}}%
\pgfpathlineto{\pgfqpoint{1.131042in}{1.396763in}}%
\pgfpathlineto{\pgfqpoint{1.115385in}{1.399530in}}%
\pgfpathlineto{\pgfqpoint{1.099729in}{1.401605in}}%
\pgfpathlineto{\pgfqpoint{1.084072in}{1.402987in}}%
\pgfpathlineto{\pgfqpoint{1.068416in}{1.403678in}}%
\pgfpathlineto{\pgfqpoint{1.052759in}{1.403678in}}%
\pgfpathlineto{\pgfqpoint{1.037103in}{1.402987in}}%
\pgfpathlineto{\pgfqpoint{1.021446in}{1.401605in}}%
\pgfpathlineto{\pgfqpoint{1.005790in}{1.399530in}}%
\pgfpathlineto{\pgfqpoint{0.990133in}{1.396763in}}%
\pgfpathlineto{\pgfqpoint{0.974476in}{1.393303in}}%
\pgfpathlineto{\pgfqpoint{0.974471in}{1.393302in}}%
\pgfpathlineto{\pgfqpoint{0.958820in}{1.389214in}}%
\pgfpathlineto{\pgfqpoint{0.943163in}{1.384441in}}%
\pgfpathlineto{\pgfqpoint{0.929526in}{1.379691in}}%
\pgfpathlineto{\pgfqpoint{0.927507in}{1.378991in}}%
\pgfpathlineto{\pgfqpoint{0.911850in}{1.372919in}}%
\pgfpathlineto{\pgfqpoint{0.896194in}{1.366170in}}%
\pgfpathlineto{\pgfqpoint{0.896001in}{1.366079in}}%
\pgfpathlineto{\pgfqpoint{0.880537in}{1.358799in}}%
\pgfpathlineto{\pgfqpoint{0.868188in}{1.352468in}}%
\pgfpathlineto{\pgfqpoint{0.864880in}{1.350764in}}%
\pgfpathlineto{\pgfqpoint{0.849224in}{1.342085in}}%
\pgfpathlineto{\pgfqpoint{0.843779in}{1.338857in}}%
\pgfpathlineto{\pgfqpoint{0.833567in}{1.332740in}}%
\pgfpathlineto{\pgfqpoint{0.821809in}{1.325246in}}%
\pgfpathlineto{\pgfqpoint{0.817911in}{1.322724in}}%
\pgfpathlineto{\pgfqpoint{0.802254in}{1.312020in}}%
\pgfpathlineto{\pgfqpoint{0.801718in}{1.311635in}}%
\pgfpathlineto{\pgfqpoint{0.786598in}{1.300573in}}%
\pgfpathlineto{\pgfqpoint{0.783265in}{1.298024in}}%
\pgfpathlineto{\pgfqpoint{0.770941in}{1.288361in}}%
\pgfpathlineto{\pgfqpoint{0.766100in}{1.284413in}}%
\pgfpathlineto{\pgfqpoint{0.755284in}{1.275327in}}%
\pgfpathlineto{\pgfqpoint{0.750079in}{1.270802in}}%
\pgfpathlineto{\pgfqpoint{0.739628in}{1.261399in}}%
\pgfpathlineto{\pgfqpoint{0.735087in}{1.257191in}}%
\pgfpathlineto{\pgfqpoint{0.723971in}{1.246477in}}%
\pgfpathlineto{\pgfqpoint{0.721039in}{1.243579in}}%
\pgfpathlineto{\pgfqpoint{0.708315in}{1.230435in}}%
\pgfpathlineto{\pgfqpoint{0.707872in}{1.229968in}}%
\pgfpathlineto{\pgfqpoint{0.695559in}{1.216357in}}%
\pgfpathlineto{\pgfqpoint{0.692658in}{1.212969in}}%
\pgfpathlineto{\pgfqpoint{0.684038in}{1.202746in}}%
\pgfpathlineto{\pgfqpoint{0.677002in}{1.193869in}}%
\pgfpathlineto{\pgfqpoint{0.673288in}{1.189135in}}%
\pgfpathlineto{\pgfqpoint{0.663305in}{1.175524in}}%
\pgfpathlineto{\pgfqpoint{0.661345in}{1.172648in}}%
\pgfpathlineto{\pgfqpoint{0.654063in}{1.161913in}}%
\pgfpathlineto{\pgfqpoint{0.645689in}{1.148469in}}%
\pgfpathlineto{\pgfqpoint{0.645584in}{1.148302in}}%
\pgfpathlineto{\pgfqpoint{0.637821in}{1.134691in}}%
\pgfpathlineto{\pgfqpoint{0.630836in}{1.121079in}}%
\pgfpathlineto{\pgfqpoint{0.630032in}{1.119324in}}%
\pgfpathlineto{\pgfqpoint{0.624568in}{1.107468in}}%
\pgfpathlineto{\pgfqpoint{0.619077in}{1.093857in}}%
\pgfpathlineto{\pgfqpoint{0.614375in}{1.080251in}}%
\pgfpathlineto{\pgfqpoint{0.614374in}{1.080246in}}%
\pgfpathlineto{\pgfqpoint{0.610394in}{1.066635in}}%
\pgfpathlineto{\pgfqpoint{0.607211in}{1.053024in}}%
\pgfpathlineto{\pgfqpoint{0.604825in}{1.039413in}}%
\pgfpathlineto{\pgfqpoint{0.603234in}{1.025802in}}%
\pgfpathlineto{\pgfqpoint{0.602439in}{1.012191in}}%
\pgfpathlineto{\pgfqpoint{0.602439in}{0.998579in}}%
\pgfpathlineto{\pgfqpoint{0.603234in}{0.984968in}}%
\pgfpathlineto{\pgfqpoint{0.604825in}{0.971357in}}%
\pgfpathlineto{\pgfqpoint{0.607211in}{0.957746in}}%
\pgfpathlineto{\pgfqpoint{0.610394in}{0.944135in}}%
\pgfpathlineto{\pgfqpoint{0.614374in}{0.930524in}}%
\pgfpathlineto{\pgfqpoint{0.614375in}{0.930519in}}%
\pgfpathlineto{\pgfqpoint{0.619077in}{0.916913in}}%
\pgfpathlineto{\pgfqpoint{0.624568in}{0.903302in}}%
\pgfpathlineto{\pgfqpoint{0.630032in}{0.891446in}}%
\pgfpathlineto{\pgfqpoint{0.630836in}{0.889691in}}%
\pgfpathlineto{\pgfqpoint{0.637821in}{0.876079in}}%
\pgfpathlineto{\pgfqpoint{0.645584in}{0.862468in}}%
\pgfpathlineto{\pgfqpoint{0.645689in}{0.862301in}}%
\pgfpathlineto{\pgfqpoint{0.654063in}{0.848857in}}%
\pgfpathlineto{\pgfqpoint{0.661345in}{0.838122in}}%
\pgfpathlineto{\pgfqpoint{0.663305in}{0.835246in}}%
\pgfpathlineto{\pgfqpoint{0.673288in}{0.821635in}}%
\pgfpathlineto{\pgfqpoint{0.677002in}{0.816901in}}%
\pgfpathlineto{\pgfqpoint{0.684038in}{0.808024in}}%
\pgfpathlineto{\pgfqpoint{0.692658in}{0.797801in}}%
\pgfpathlineto{\pgfqpoint{0.695559in}{0.794413in}}%
\pgfpathlineto{\pgfqpoint{0.707872in}{0.780802in}}%
\pgfpathlineto{\pgfqpoint{0.708315in}{0.780335in}}%
\pgfpathlineto{\pgfqpoint{0.721039in}{0.767191in}}%
\pgfpathlineto{\pgfqpoint{0.723971in}{0.764293in}}%
\pgfpathlineto{\pgfqpoint{0.735087in}{0.753579in}}%
\pgfpathlineto{\pgfqpoint{0.739628in}{0.749371in}}%
\pgfpathlineto{\pgfqpoint{0.750079in}{0.739968in}}%
\pgfpathlineto{\pgfqpoint{0.755284in}{0.735443in}}%
\pgfpathlineto{\pgfqpoint{0.766100in}{0.726357in}}%
\pgfpathlineto{\pgfqpoint{0.770941in}{0.722409in}}%
\pgfpathlineto{\pgfqpoint{0.783265in}{0.712746in}}%
\pgfpathlineto{\pgfqpoint{0.786598in}{0.710197in}}%
\pgfpathlineto{\pgfqpoint{0.801718in}{0.699135in}}%
\pgfpathlineto{\pgfqpoint{0.802254in}{0.698750in}}%
\pgfpathlineto{\pgfqpoint{0.817911in}{0.688046in}}%
\pgfpathlineto{\pgfqpoint{0.821809in}{0.685524in}}%
\pgfpathlineto{\pgfqpoint{0.833567in}{0.678030in}}%
\pgfpathlineto{\pgfqpoint{0.843779in}{0.671913in}}%
\pgfpathlineto{\pgfqpoint{0.849224in}{0.668685in}}%
\pgfpathlineto{\pgfqpoint{0.864880in}{0.660006in}}%
\pgfpathlineto{\pgfqpoint{0.868188in}{0.658302in}}%
\pgfpathlineto{\pgfqpoint{0.880537in}{0.651971in}}%
\pgfpathlineto{\pgfqpoint{0.896001in}{0.644691in}}%
\pgfpathlineto{\pgfqpoint{0.896194in}{0.644600in}}%
\pgfpathlineto{\pgfqpoint{0.911850in}{0.637851in}}%
\pgfpathlineto{\pgfqpoint{0.927507in}{0.631779in}}%
\pgfpathlineto{\pgfqpoint{0.929526in}{0.631079in}}%
\pgfpathlineto{\pgfqpoint{0.943163in}{0.626329in}}%
\pgfpathlineto{\pgfqpoint{0.958820in}{0.621556in}}%
\pgfpathlineto{\pgfqpoint{0.974471in}{0.617468in}}%
\pgfpathlineto{\pgfqpoint{0.974476in}{0.617467in}}%
\pgfpathclose%
\pgfpathmoveto{\pgfqpoint{1.002337in}{0.726357in}}%
\pgfpathlineto{\pgfqpoint{0.990133in}{0.728800in}}%
\pgfpathlineto{\pgfqpoint{0.974476in}{0.732717in}}%
\pgfpathlineto{\pgfqpoint{0.958820in}{0.737419in}}%
\pgfpathlineto{\pgfqpoint{0.951529in}{0.739968in}}%
\pgfpathlineto{\pgfqpoint{0.943163in}{0.742972in}}%
\pgfpathlineto{\pgfqpoint{0.927507in}{0.749386in}}%
\pgfpathlineto{\pgfqpoint{0.918386in}{0.753579in}}%
\pgfpathlineto{\pgfqpoint{0.911850in}{0.756685in}}%
\pgfpathlineto{\pgfqpoint{0.896194in}{0.764921in}}%
\pgfpathlineto{\pgfqpoint{0.892250in}{0.767191in}}%
\pgfpathlineto{\pgfqpoint{0.880537in}{0.774198in}}%
\pgfpathlineto{\pgfqpoint{0.870369in}{0.780802in}}%
\pgfpathlineto{\pgfqpoint{0.864880in}{0.784530in}}%
\pgfpathlineto{\pgfqpoint{0.851355in}{0.794413in}}%
\pgfpathlineto{\pgfqpoint{0.849224in}{0.796052in}}%
\pgfpathlineto{\pgfqpoint{0.834634in}{0.808024in}}%
\pgfpathlineto{\pgfqpoint{0.833567in}{0.808951in}}%
\pgfpathlineto{\pgfqpoint{0.819796in}{0.821635in}}%
\pgfpathlineto{\pgfqpoint{0.817911in}{0.823488in}}%
\pgfpathlineto{\pgfqpoint{0.806543in}{0.835246in}}%
\pgfpathlineto{\pgfqpoint{0.802254in}{0.840018in}}%
\pgfpathlineto{\pgfqpoint{0.794658in}{0.848857in}}%
\pgfpathlineto{\pgfqpoint{0.786598in}{0.859040in}}%
\pgfpathlineto{\pgfqpoint{0.783987in}{0.862468in}}%
\pgfpathlineto{\pgfqpoint{0.774513in}{0.876079in}}%
\pgfpathlineto{\pgfqpoint{0.770941in}{0.881761in}}%
\pgfpathlineto{\pgfqpoint{0.766117in}{0.889691in}}%
\pgfpathlineto{\pgfqpoint{0.758740in}{0.903302in}}%
\pgfpathlineto{\pgfqpoint{0.755284in}{0.910574in}}%
\pgfpathlineto{\pgfqpoint{0.752352in}{0.916913in}}%
\pgfpathlineto{\pgfqpoint{0.746943in}{0.930524in}}%
\pgfpathlineto{\pgfqpoint{0.742438in}{0.944135in}}%
\pgfpathlineto{\pgfqpoint{0.739628in}{0.954745in}}%
\pgfpathlineto{\pgfqpoint{0.738849in}{0.957746in}}%
\pgfpathlineto{\pgfqpoint{0.736194in}{0.971357in}}%
\pgfpathlineto{\pgfqpoint{0.734424in}{0.984968in}}%
\pgfpathlineto{\pgfqpoint{0.733540in}{0.998579in}}%
\pgfpathlineto{\pgfqpoint{0.733540in}{1.012191in}}%
\pgfpathlineto{\pgfqpoint{0.734424in}{1.025802in}}%
\pgfpathlineto{\pgfqpoint{0.736194in}{1.039413in}}%
\pgfpathlineto{\pgfqpoint{0.738849in}{1.053024in}}%
\pgfpathlineto{\pgfqpoint{0.739628in}{1.056025in}}%
\pgfpathlineto{\pgfqpoint{0.742438in}{1.066635in}}%
\pgfpathlineto{\pgfqpoint{0.746943in}{1.080246in}}%
\pgfpathlineto{\pgfqpoint{0.752352in}{1.093857in}}%
\pgfpathlineto{\pgfqpoint{0.755284in}{1.100196in}}%
\pgfpathlineto{\pgfqpoint{0.758740in}{1.107468in}}%
\pgfpathlineto{\pgfqpoint{0.766117in}{1.121079in}}%
\pgfpathlineto{\pgfqpoint{0.770941in}{1.129009in}}%
\pgfpathlineto{\pgfqpoint{0.774513in}{1.134691in}}%
\pgfpathlineto{\pgfqpoint{0.783987in}{1.148302in}}%
\pgfpathlineto{\pgfqpoint{0.786598in}{1.151730in}}%
\pgfpathlineto{\pgfqpoint{0.794658in}{1.161913in}}%
\pgfpathlineto{\pgfqpoint{0.802254in}{1.170752in}}%
\pgfpathlineto{\pgfqpoint{0.806543in}{1.175524in}}%
\pgfpathlineto{\pgfqpoint{0.817911in}{1.187282in}}%
\pgfpathlineto{\pgfqpoint{0.819796in}{1.189135in}}%
\pgfpathlineto{\pgfqpoint{0.833567in}{1.201819in}}%
\pgfpathlineto{\pgfqpoint{0.834634in}{1.202746in}}%
\pgfpathlineto{\pgfqpoint{0.849224in}{1.214718in}}%
\pgfpathlineto{\pgfqpoint{0.851355in}{1.216357in}}%
\pgfpathlineto{\pgfqpoint{0.864880in}{1.226240in}}%
\pgfpathlineto{\pgfqpoint{0.870369in}{1.229968in}}%
\pgfpathlineto{\pgfqpoint{0.880537in}{1.236572in}}%
\pgfpathlineto{\pgfqpoint{0.892250in}{1.243579in}}%
\pgfpathlineto{\pgfqpoint{0.896194in}{1.245849in}}%
\pgfpathlineto{\pgfqpoint{0.911850in}{1.254085in}}%
\pgfpathlineto{\pgfqpoint{0.918386in}{1.257191in}}%
\pgfpathlineto{\pgfqpoint{0.927507in}{1.261384in}}%
\pgfpathlineto{\pgfqpoint{0.943163in}{1.267798in}}%
\pgfpathlineto{\pgfqpoint{0.951529in}{1.270802in}}%
\pgfpathlineto{\pgfqpoint{0.958820in}{1.273351in}}%
\pgfpathlineto{\pgfqpoint{0.974476in}{1.278053in}}%
\pgfpathlineto{\pgfqpoint{0.990133in}{1.281970in}}%
\pgfpathlineto{\pgfqpoint{1.002337in}{1.284413in}}%
\pgfpathlineto{\pgfqpoint{1.005790in}{1.285090in}}%
\pgfpathlineto{\pgfqpoint{1.021446in}{1.287398in}}%
\pgfpathlineto{\pgfqpoint{1.037103in}{1.288937in}}%
\pgfpathlineto{\pgfqpoint{1.052759in}{1.289706in}}%
\pgfpathlineto{\pgfqpoint{1.068416in}{1.289706in}}%
\pgfpathlineto{\pgfqpoint{1.084072in}{1.288937in}}%
\pgfpathlineto{\pgfqpoint{1.099729in}{1.287398in}}%
\pgfpathlineto{\pgfqpoint{1.115385in}{1.285090in}}%
\pgfpathlineto{\pgfqpoint{1.118838in}{1.284413in}}%
\pgfpathlineto{\pgfqpoint{1.131042in}{1.281970in}}%
\pgfpathlineto{\pgfqpoint{1.146699in}{1.278053in}}%
\pgfpathlineto{\pgfqpoint{1.162355in}{1.273351in}}%
\pgfpathlineto{\pgfqpoint{1.169646in}{1.270802in}}%
\pgfpathlineto{\pgfqpoint{1.178012in}{1.267798in}}%
\pgfpathlineto{\pgfqpoint{1.193668in}{1.261384in}}%
\pgfpathlineto{\pgfqpoint{1.202789in}{1.257191in}}%
\pgfpathlineto{\pgfqpoint{1.209325in}{1.254085in}}%
\pgfpathlineto{\pgfqpoint{1.224981in}{1.245849in}}%
\pgfpathlineto{\pgfqpoint{1.228925in}{1.243579in}}%
\pgfpathlineto{\pgfqpoint{1.240638in}{1.236572in}}%
\pgfpathlineto{\pgfqpoint{1.250806in}{1.229968in}}%
\pgfpathlineto{\pgfqpoint{1.256295in}{1.226240in}}%
\pgfpathlineto{\pgfqpoint{1.269820in}{1.216357in}}%
\pgfpathlineto{\pgfqpoint{1.271951in}{1.214718in}}%
\pgfpathlineto{\pgfqpoint{1.286541in}{1.202746in}}%
\pgfpathlineto{\pgfqpoint{1.287608in}{1.201819in}}%
\pgfpathlineto{\pgfqpoint{1.301379in}{1.189135in}}%
\pgfpathlineto{\pgfqpoint{1.303264in}{1.187282in}}%
\pgfpathlineto{\pgfqpoint{1.314632in}{1.175524in}}%
\pgfpathlineto{\pgfqpoint{1.318921in}{1.170752in}}%
\pgfpathlineto{\pgfqpoint{1.326517in}{1.161913in}}%
\pgfpathlineto{\pgfqpoint{1.334577in}{1.151730in}}%
\pgfpathlineto{\pgfqpoint{1.337188in}{1.148302in}}%
\pgfpathlineto{\pgfqpoint{1.346662in}{1.134691in}}%
\pgfpathlineto{\pgfqpoint{1.350234in}{1.129009in}}%
\pgfpathlineto{\pgfqpoint{1.355058in}{1.121079in}}%
\pgfpathlineto{\pgfqpoint{1.362435in}{1.107468in}}%
\pgfpathlineto{\pgfqpoint{1.365891in}{1.100196in}}%
\pgfpathlineto{\pgfqpoint{1.368823in}{1.093857in}}%
\pgfpathlineto{\pgfqpoint{1.374232in}{1.080246in}}%
\pgfpathlineto{\pgfqpoint{1.378737in}{1.066635in}}%
\pgfpathlineto{\pgfqpoint{1.381547in}{1.056025in}}%
\pgfpathlineto{\pgfqpoint{1.382326in}{1.053024in}}%
\pgfpathlineto{\pgfqpoint{1.384981in}{1.039413in}}%
\pgfpathlineto{\pgfqpoint{1.386751in}{1.025802in}}%
\pgfpathlineto{\pgfqpoint{1.387635in}{1.012191in}}%
\pgfpathlineto{\pgfqpoint{1.387635in}{0.998579in}}%
\pgfpathlineto{\pgfqpoint{1.386751in}{0.984968in}}%
\pgfpathlineto{\pgfqpoint{1.384981in}{0.971357in}}%
\pgfpathlineto{\pgfqpoint{1.382326in}{0.957746in}}%
\pgfpathlineto{\pgfqpoint{1.381547in}{0.954745in}}%
\pgfpathlineto{\pgfqpoint{1.378737in}{0.944135in}}%
\pgfpathlineto{\pgfqpoint{1.374232in}{0.930524in}}%
\pgfpathlineto{\pgfqpoint{1.368823in}{0.916913in}}%
\pgfpathlineto{\pgfqpoint{1.365891in}{0.910574in}}%
\pgfpathlineto{\pgfqpoint{1.362435in}{0.903302in}}%
\pgfpathlineto{\pgfqpoint{1.355058in}{0.889691in}}%
\pgfpathlineto{\pgfqpoint{1.350234in}{0.881761in}}%
\pgfpathlineto{\pgfqpoint{1.346662in}{0.876079in}}%
\pgfpathlineto{\pgfqpoint{1.337188in}{0.862468in}}%
\pgfpathlineto{\pgfqpoint{1.334577in}{0.859040in}}%
\pgfpathlineto{\pgfqpoint{1.326517in}{0.848857in}}%
\pgfpathlineto{\pgfqpoint{1.318921in}{0.840018in}}%
\pgfpathlineto{\pgfqpoint{1.314632in}{0.835246in}}%
\pgfpathlineto{\pgfqpoint{1.303264in}{0.823488in}}%
\pgfpathlineto{\pgfqpoint{1.301379in}{0.821635in}}%
\pgfpathlineto{\pgfqpoint{1.287608in}{0.808951in}}%
\pgfpathlineto{\pgfqpoint{1.286541in}{0.808024in}}%
\pgfpathlineto{\pgfqpoint{1.271951in}{0.796052in}}%
\pgfpathlineto{\pgfqpoint{1.269820in}{0.794413in}}%
\pgfpathlineto{\pgfqpoint{1.256295in}{0.784530in}}%
\pgfpathlineto{\pgfqpoint{1.250806in}{0.780802in}}%
\pgfpathlineto{\pgfqpoint{1.240638in}{0.774198in}}%
\pgfpathlineto{\pgfqpoint{1.228925in}{0.767191in}}%
\pgfpathlineto{\pgfqpoint{1.224981in}{0.764921in}}%
\pgfpathlineto{\pgfqpoint{1.209325in}{0.756685in}}%
\pgfpathlineto{\pgfqpoint{1.202789in}{0.753579in}}%
\pgfpathlineto{\pgfqpoint{1.193668in}{0.749386in}}%
\pgfpathlineto{\pgfqpoint{1.178012in}{0.742972in}}%
\pgfpathlineto{\pgfqpoint{1.169646in}{0.739968in}}%
\pgfpathlineto{\pgfqpoint{1.162355in}{0.737419in}}%
\pgfpathlineto{\pgfqpoint{1.146699in}{0.732717in}}%
\pgfpathlineto{\pgfqpoint{1.131042in}{0.728800in}}%
\pgfpathlineto{\pgfqpoint{1.118838in}{0.726357in}}%
\pgfpathlineto{\pgfqpoint{1.115385in}{0.725680in}}%
\pgfpathlineto{\pgfqpoint{1.099729in}{0.723372in}}%
\pgfpathlineto{\pgfqpoint{1.084072in}{0.721833in}}%
\pgfpathlineto{\pgfqpoint{1.068416in}{0.721064in}}%
\pgfpathlineto{\pgfqpoint{1.052759in}{0.721064in}}%
\pgfpathlineto{\pgfqpoint{1.037103in}{0.721833in}}%
\pgfpathlineto{\pgfqpoint{1.021446in}{0.723372in}}%
\pgfpathlineto{\pgfqpoint{1.005790in}{0.725680in}}%
\pgfpathlineto{\pgfqpoint{1.002337in}{0.726357in}}%
\pgfpathclose%
\pgfusepath{fill}%
\end{pgfscope}%
\begin{pgfscope}%
\pgfpathrectangle{\pgfqpoint{0.285588in}{0.331635in}}{\pgfqpoint{1.550000in}{1.347500in}}%
\pgfusepath{clip}%
\pgfsetbuttcap%
\pgfsetroundjoin%
\definecolor{currentfill}{rgb}{0.709962,0.212797,0.477201}%
\pgfsetfillcolor{currentfill}%
\pgfsetlinewidth{0.000000pt}%
\definecolor{currentstroke}{rgb}{0.000000,0.000000,0.000000}%
\pgfsetstrokecolor{currentstroke}%
\pgfsetdash{}{0pt}%
\pgfpathmoveto{\pgfqpoint{1.037103in}{0.467441in}}%
\pgfpathlineto{\pgfqpoint{1.052759in}{0.466410in}}%
\pgfpathlineto{\pgfqpoint{1.068416in}{0.466410in}}%
\pgfpathlineto{\pgfqpoint{1.084072in}{0.467441in}}%
\pgfpathlineto{\pgfqpoint{1.086388in}{0.467746in}}%
\pgfpathlineto{\pgfqpoint{1.099729in}{0.469358in}}%
\pgfpathlineto{\pgfqpoint{1.115385in}{0.472197in}}%
\pgfpathlineto{\pgfqpoint{1.131042in}{0.475983in}}%
\pgfpathlineto{\pgfqpoint{1.146699in}{0.480717in}}%
\pgfpathlineto{\pgfqpoint{1.148469in}{0.481357in}}%
\pgfpathlineto{\pgfqpoint{1.162355in}{0.486032in}}%
\pgfpathlineto{\pgfqpoint{1.178012in}{0.492181in}}%
\pgfpathlineto{\pgfqpoint{1.184239in}{0.494968in}}%
\pgfpathlineto{\pgfqpoint{1.193668in}{0.498938in}}%
\pgfpathlineto{\pgfqpoint{1.209325in}{0.506340in}}%
\pgfpathlineto{\pgfqpoint{1.213608in}{0.508579in}}%
\pgfpathlineto{\pgfqpoint{1.224981in}{0.514225in}}%
\pgfpathlineto{\pgfqpoint{1.239603in}{0.522191in}}%
\pgfpathlineto{\pgfqpoint{1.240638in}{0.522730in}}%
\pgfpathlineto{\pgfqpoint{1.256295in}{0.531569in}}%
\pgfpathlineto{\pgfqpoint{1.263251in}{0.535802in}}%
\pgfpathlineto{\pgfqpoint{1.271951in}{0.540905in}}%
\pgfpathlineto{\pgfqpoint{1.285499in}{0.549413in}}%
\pgfpathlineto{\pgfqpoint{1.287608in}{0.550698in}}%
\pgfpathlineto{\pgfqpoint{1.303264in}{0.560825in}}%
\pgfpathlineto{\pgfqpoint{1.306483in}{0.563024in}}%
\pgfpathlineto{\pgfqpoint{1.318921in}{0.571321in}}%
\pgfpathlineto{\pgfqpoint{1.326495in}{0.576635in}}%
\pgfpathlineto{\pgfqpoint{1.334577in}{0.582205in}}%
\pgfpathlineto{\pgfqpoint{1.345729in}{0.590246in}}%
\pgfpathlineto{\pgfqpoint{1.350234in}{0.593454in}}%
\pgfpathlineto{\pgfqpoint{1.364264in}{0.603857in}}%
\pgfpathlineto{\pgfqpoint{1.365891in}{0.605054in}}%
\pgfpathlineto{\pgfqpoint{1.381547in}{0.616991in}}%
\pgfpathlineto{\pgfqpoint{1.382154in}{0.617468in}}%
\pgfpathlineto{\pgfqpoint{1.397204in}{0.629266in}}%
\pgfpathlineto{\pgfqpoint{1.399455in}{0.631079in}}%
\pgfpathlineto{\pgfqpoint{1.412860in}{0.641901in}}%
\pgfpathlineto{\pgfqpoint{1.416236in}{0.644691in}}%
\pgfpathlineto{\pgfqpoint{1.428517in}{0.654901in}}%
\pgfpathlineto{\pgfqpoint{1.432529in}{0.658302in}}%
\pgfpathlineto{\pgfqpoint{1.444173in}{0.668275in}}%
\pgfpathlineto{\pgfqpoint{1.448357in}{0.671913in}}%
\pgfpathlineto{\pgfqpoint{1.459830in}{0.682036in}}%
\pgfpathlineto{\pgfqpoint{1.463741in}{0.685524in}}%
\pgfpathlineto{\pgfqpoint{1.475486in}{0.696200in}}%
\pgfpathlineto{\pgfqpoint{1.478695in}{0.699135in}}%
\pgfpathlineto{\pgfqpoint{1.491143in}{0.710789in}}%
\pgfpathlineto{\pgfqpoint{1.493229in}{0.712746in}}%
\pgfpathlineto{\pgfqpoint{1.506800in}{0.725829in}}%
\pgfpathlineto{\pgfqpoint{1.507349in}{0.726357in}}%
\pgfpathlineto{\pgfqpoint{1.521080in}{0.739968in}}%
\pgfpathlineto{\pgfqpoint{1.522456in}{0.741382in}}%
\pgfpathlineto{\pgfqpoint{1.534423in}{0.753579in}}%
\pgfpathlineto{\pgfqpoint{1.538113in}{0.757496in}}%
\pgfpathlineto{\pgfqpoint{1.547363in}{0.767191in}}%
\pgfpathlineto{\pgfqpoint{1.553769in}{0.774217in}}%
\pgfpathlineto{\pgfqpoint{1.559882in}{0.780802in}}%
\pgfpathlineto{\pgfqpoint{1.569426in}{0.791614in}}%
\pgfpathlineto{\pgfqpoint{1.571955in}{0.794413in}}%
\pgfpathlineto{\pgfqpoint{1.583604in}{0.808024in}}%
\pgfpathlineto{\pgfqpoint{1.585082in}{0.809857in}}%
\pgfpathlineto{\pgfqpoint{1.594869in}{0.821635in}}%
\pgfpathlineto{\pgfqpoint{1.600739in}{0.829198in}}%
\pgfpathlineto{\pgfqpoint{1.605607in}{0.835246in}}%
\pgfpathlineto{\pgfqpoint{1.615775in}{0.848857in}}%
\pgfpathlineto{\pgfqpoint{1.616396in}{0.849757in}}%
\pgfpathlineto{\pgfqpoint{1.625558in}{0.862468in}}%
\pgfpathlineto{\pgfqpoint{1.632052in}{0.872356in}}%
\pgfpathlineto{\pgfqpoint{1.634628in}{0.876079in}}%
\pgfpathlineto{\pgfqpoint{1.643142in}{0.889691in}}%
\pgfpathlineto{\pgfqpoint{1.647709in}{0.897888in}}%
\pgfpathlineto{\pgfqpoint{1.650914in}{0.903302in}}%
\pgfpathlineto{\pgfqpoint{1.657988in}{0.916913in}}%
\pgfpathlineto{\pgfqpoint{1.663365in}{0.928984in}}%
\pgfpathlineto{\pgfqpoint{1.664102in}{0.930524in}}%
\pgfpathlineto{\pgfqpoint{1.669548in}{0.944135in}}%
\pgfpathlineto{\pgfqpoint{1.673902in}{0.957746in}}%
\pgfpathlineto{\pgfqpoint{1.677167in}{0.971357in}}%
\pgfpathlineto{\pgfqpoint{1.679022in}{0.982955in}}%
\pgfpathlineto{\pgfqpoint{1.679372in}{0.984968in}}%
\pgfpathlineto{\pgfqpoint{1.680558in}{0.998579in}}%
\pgfpathlineto{\pgfqpoint{1.680558in}{1.012191in}}%
\pgfpathlineto{\pgfqpoint{1.679372in}{1.025802in}}%
\pgfpathlineto{\pgfqpoint{1.679022in}{1.027815in}}%
\pgfpathlineto{\pgfqpoint{1.677167in}{1.039413in}}%
\pgfpathlineto{\pgfqpoint{1.673902in}{1.053024in}}%
\pgfpathlineto{\pgfqpoint{1.669548in}{1.066635in}}%
\pgfpathlineto{\pgfqpoint{1.664102in}{1.080246in}}%
\pgfpathlineto{\pgfqpoint{1.663365in}{1.081786in}}%
\pgfpathlineto{\pgfqpoint{1.657988in}{1.093857in}}%
\pgfpathlineto{\pgfqpoint{1.650914in}{1.107468in}}%
\pgfpathlineto{\pgfqpoint{1.647709in}{1.112882in}}%
\pgfpathlineto{\pgfqpoint{1.643142in}{1.121079in}}%
\pgfpathlineto{\pgfqpoint{1.634628in}{1.134691in}}%
\pgfpathlineto{\pgfqpoint{1.632052in}{1.138414in}}%
\pgfpathlineto{\pgfqpoint{1.625558in}{1.148302in}}%
\pgfpathlineto{\pgfqpoint{1.616396in}{1.161013in}}%
\pgfpathlineto{\pgfqpoint{1.615775in}{1.161913in}}%
\pgfpathlineto{\pgfqpoint{1.605607in}{1.175524in}}%
\pgfpathlineto{\pgfqpoint{1.600739in}{1.181572in}}%
\pgfpathlineto{\pgfqpoint{1.594869in}{1.189135in}}%
\pgfpathlineto{\pgfqpoint{1.585082in}{1.200913in}}%
\pgfpathlineto{\pgfqpoint{1.583604in}{1.202746in}}%
\pgfpathlineto{\pgfqpoint{1.571955in}{1.216357in}}%
\pgfpathlineto{\pgfqpoint{1.569426in}{1.219156in}}%
\pgfpathlineto{\pgfqpoint{1.559882in}{1.229968in}}%
\pgfpathlineto{\pgfqpoint{1.553769in}{1.236553in}}%
\pgfpathlineto{\pgfqpoint{1.547363in}{1.243579in}}%
\pgfpathlineto{\pgfqpoint{1.538113in}{1.253274in}}%
\pgfpathlineto{\pgfqpoint{1.534423in}{1.257191in}}%
\pgfpathlineto{\pgfqpoint{1.522456in}{1.269388in}}%
\pgfpathlineto{\pgfqpoint{1.521080in}{1.270802in}}%
\pgfpathlineto{\pgfqpoint{1.507349in}{1.284413in}}%
\pgfpathlineto{\pgfqpoint{1.506800in}{1.284941in}}%
\pgfpathlineto{\pgfqpoint{1.493229in}{1.298024in}}%
\pgfpathlineto{\pgfqpoint{1.491143in}{1.299981in}}%
\pgfpathlineto{\pgfqpoint{1.478695in}{1.311635in}}%
\pgfpathlineto{\pgfqpoint{1.475486in}{1.314570in}}%
\pgfpathlineto{\pgfqpoint{1.463741in}{1.325246in}}%
\pgfpathlineto{\pgfqpoint{1.459830in}{1.328734in}}%
\pgfpathlineto{\pgfqpoint{1.448357in}{1.338857in}}%
\pgfpathlineto{\pgfqpoint{1.444173in}{1.342495in}}%
\pgfpathlineto{\pgfqpoint{1.432529in}{1.352468in}}%
\pgfpathlineto{\pgfqpoint{1.428517in}{1.355869in}}%
\pgfpathlineto{\pgfqpoint{1.416236in}{1.366079in}}%
\pgfpathlineto{\pgfqpoint{1.412860in}{1.368869in}}%
\pgfpathlineto{\pgfqpoint{1.399455in}{1.379691in}}%
\pgfpathlineto{\pgfqpoint{1.397204in}{1.381504in}}%
\pgfpathlineto{\pgfqpoint{1.382154in}{1.393302in}}%
\pgfpathlineto{\pgfqpoint{1.381547in}{1.393779in}}%
\pgfpathlineto{\pgfqpoint{1.365891in}{1.405716in}}%
\pgfpathlineto{\pgfqpoint{1.364264in}{1.406913in}}%
\pgfpathlineto{\pgfqpoint{1.350234in}{1.417316in}}%
\pgfpathlineto{\pgfqpoint{1.345729in}{1.420524in}}%
\pgfpathlineto{\pgfqpoint{1.334577in}{1.428565in}}%
\pgfpathlineto{\pgfqpoint{1.326495in}{1.434135in}}%
\pgfpathlineto{\pgfqpoint{1.318921in}{1.439449in}}%
\pgfpathlineto{\pgfqpoint{1.306483in}{1.447746in}}%
\pgfpathlineto{\pgfqpoint{1.303264in}{1.449945in}}%
\pgfpathlineto{\pgfqpoint{1.287608in}{1.460072in}}%
\pgfpathlineto{\pgfqpoint{1.285499in}{1.461357in}}%
\pgfpathlineto{\pgfqpoint{1.271951in}{1.469865in}}%
\pgfpathlineto{\pgfqpoint{1.263251in}{1.474968in}}%
\pgfpathlineto{\pgfqpoint{1.256295in}{1.479201in}}%
\pgfpathlineto{\pgfqpoint{1.240638in}{1.488040in}}%
\pgfpathlineto{\pgfqpoint{1.239603in}{1.488579in}}%
\pgfpathlineto{\pgfqpoint{1.224981in}{1.496545in}}%
\pgfpathlineto{\pgfqpoint{1.213608in}{1.502191in}}%
\pgfpathlineto{\pgfqpoint{1.209325in}{1.504430in}}%
\pgfpathlineto{\pgfqpoint{1.193668in}{1.511832in}}%
\pgfpathlineto{\pgfqpoint{1.184239in}{1.515802in}}%
\pgfpathlineto{\pgfqpoint{1.178012in}{1.518589in}}%
\pgfpathlineto{\pgfqpoint{1.162355in}{1.524738in}}%
\pgfpathlineto{\pgfqpoint{1.148469in}{1.529413in}}%
\pgfpathlineto{\pgfqpoint{1.146699in}{1.530053in}}%
\pgfpathlineto{\pgfqpoint{1.131042in}{1.534787in}}%
\pgfpathlineto{\pgfqpoint{1.115385in}{1.538573in}}%
\pgfpathlineto{\pgfqpoint{1.099729in}{1.541412in}}%
\pgfpathlineto{\pgfqpoint{1.086388in}{1.543024in}}%
\pgfpathlineto{\pgfqpoint{1.084072in}{1.543329in}}%
\pgfpathlineto{\pgfqpoint{1.068416in}{1.544360in}}%
\pgfpathlineto{\pgfqpoint{1.052759in}{1.544360in}}%
\pgfpathlineto{\pgfqpoint{1.037103in}{1.543329in}}%
\pgfpathlineto{\pgfqpoint{1.034787in}{1.543024in}}%
\pgfpathlineto{\pgfqpoint{1.021446in}{1.541412in}}%
\pgfpathlineto{\pgfqpoint{1.005790in}{1.538573in}}%
\pgfpathlineto{\pgfqpoint{0.990133in}{1.534787in}}%
\pgfpathlineto{\pgfqpoint{0.974476in}{1.530053in}}%
\pgfpathlineto{\pgfqpoint{0.972706in}{1.529413in}}%
\pgfpathlineto{\pgfqpoint{0.958820in}{1.524738in}}%
\pgfpathlineto{\pgfqpoint{0.943163in}{1.518589in}}%
\pgfpathlineto{\pgfqpoint{0.936936in}{1.515802in}}%
\pgfpathlineto{\pgfqpoint{0.927507in}{1.511832in}}%
\pgfpathlineto{\pgfqpoint{0.911850in}{1.504430in}}%
\pgfpathlineto{\pgfqpoint{0.907567in}{1.502191in}}%
\pgfpathlineto{\pgfqpoint{0.896194in}{1.496545in}}%
\pgfpathlineto{\pgfqpoint{0.881572in}{1.488579in}}%
\pgfpathlineto{\pgfqpoint{0.880537in}{1.488040in}}%
\pgfpathlineto{\pgfqpoint{0.864880in}{1.479201in}}%
\pgfpathlineto{\pgfqpoint{0.857924in}{1.474968in}}%
\pgfpathlineto{\pgfqpoint{0.849224in}{1.469865in}}%
\pgfpathlineto{\pgfqpoint{0.835676in}{1.461357in}}%
\pgfpathlineto{\pgfqpoint{0.833567in}{1.460072in}}%
\pgfpathlineto{\pgfqpoint{0.817911in}{1.449945in}}%
\pgfpathlineto{\pgfqpoint{0.814692in}{1.447746in}}%
\pgfpathlineto{\pgfqpoint{0.802254in}{1.439449in}}%
\pgfpathlineto{\pgfqpoint{0.794680in}{1.434135in}}%
\pgfpathlineto{\pgfqpoint{0.786598in}{1.428565in}}%
\pgfpathlineto{\pgfqpoint{0.775446in}{1.420524in}}%
\pgfpathlineto{\pgfqpoint{0.770941in}{1.417316in}}%
\pgfpathlineto{\pgfqpoint{0.756911in}{1.406913in}}%
\pgfpathlineto{\pgfqpoint{0.755284in}{1.405716in}}%
\pgfpathlineto{\pgfqpoint{0.739628in}{1.393779in}}%
\pgfpathlineto{\pgfqpoint{0.739021in}{1.393302in}}%
\pgfpathlineto{\pgfqpoint{0.723971in}{1.381504in}}%
\pgfpathlineto{\pgfqpoint{0.721720in}{1.379691in}}%
\pgfpathlineto{\pgfqpoint{0.708315in}{1.368869in}}%
\pgfpathlineto{\pgfqpoint{0.704939in}{1.366079in}}%
\pgfpathlineto{\pgfqpoint{0.692658in}{1.355869in}}%
\pgfpathlineto{\pgfqpoint{0.688646in}{1.352468in}}%
\pgfpathlineto{\pgfqpoint{0.677002in}{1.342495in}}%
\pgfpathlineto{\pgfqpoint{0.672818in}{1.338857in}}%
\pgfpathlineto{\pgfqpoint{0.661345in}{1.328734in}}%
\pgfpathlineto{\pgfqpoint{0.657434in}{1.325246in}}%
\pgfpathlineto{\pgfqpoint{0.645689in}{1.314570in}}%
\pgfpathlineto{\pgfqpoint{0.642480in}{1.311635in}}%
\pgfpathlineto{\pgfqpoint{0.630032in}{1.299981in}}%
\pgfpathlineto{\pgfqpoint{0.627946in}{1.298024in}}%
\pgfpathlineto{\pgfqpoint{0.614375in}{1.284941in}}%
\pgfpathlineto{\pgfqpoint{0.613826in}{1.284413in}}%
\pgfpathlineto{\pgfqpoint{0.600095in}{1.270802in}}%
\pgfpathlineto{\pgfqpoint{0.598719in}{1.269388in}}%
\pgfpathlineto{\pgfqpoint{0.586752in}{1.257191in}}%
\pgfpathlineto{\pgfqpoint{0.583062in}{1.253274in}}%
\pgfpathlineto{\pgfqpoint{0.573812in}{1.243579in}}%
\pgfpathlineto{\pgfqpoint{0.567406in}{1.236553in}}%
\pgfpathlineto{\pgfqpoint{0.561293in}{1.229968in}}%
\pgfpathlineto{\pgfqpoint{0.551749in}{1.219156in}}%
\pgfpathlineto{\pgfqpoint{0.549220in}{1.216357in}}%
\pgfpathlineto{\pgfqpoint{0.537571in}{1.202746in}}%
\pgfpathlineto{\pgfqpoint{0.536093in}{1.200913in}}%
\pgfpathlineto{\pgfqpoint{0.526306in}{1.189135in}}%
\pgfpathlineto{\pgfqpoint{0.520436in}{1.181572in}}%
\pgfpathlineto{\pgfqpoint{0.515568in}{1.175524in}}%
\pgfpathlineto{\pgfqpoint{0.505400in}{1.161913in}}%
\pgfpathlineto{\pgfqpoint{0.504779in}{1.161013in}}%
\pgfpathlineto{\pgfqpoint{0.495617in}{1.148302in}}%
\pgfpathlineto{\pgfqpoint{0.489123in}{1.138414in}}%
\pgfpathlineto{\pgfqpoint{0.486547in}{1.134691in}}%
\pgfpathlineto{\pgfqpoint{0.478033in}{1.121079in}}%
\pgfpathlineto{\pgfqpoint{0.473466in}{1.112882in}}%
\pgfpathlineto{\pgfqpoint{0.470261in}{1.107468in}}%
\pgfpathlineto{\pgfqpoint{0.463187in}{1.093857in}}%
\pgfpathlineto{\pgfqpoint{0.457810in}{1.081786in}}%
\pgfpathlineto{\pgfqpoint{0.457073in}{1.080246in}}%
\pgfpathlineto{\pgfqpoint{0.451627in}{1.066635in}}%
\pgfpathlineto{\pgfqpoint{0.447273in}{1.053024in}}%
\pgfpathlineto{\pgfqpoint{0.444008in}{1.039413in}}%
\pgfpathlineto{\pgfqpoint{0.442153in}{1.027815in}}%
\pgfpathlineto{\pgfqpoint{0.441803in}{1.025802in}}%
\pgfpathlineto{\pgfqpoint{0.440617in}{1.012191in}}%
\pgfpathlineto{\pgfqpoint{0.440617in}{0.998579in}}%
\pgfpathlineto{\pgfqpoint{0.441803in}{0.984968in}}%
\pgfpathlineto{\pgfqpoint{0.442153in}{0.982955in}}%
\pgfpathlineto{\pgfqpoint{0.444008in}{0.971357in}}%
\pgfpathlineto{\pgfqpoint{0.447273in}{0.957746in}}%
\pgfpathlineto{\pgfqpoint{0.451627in}{0.944135in}}%
\pgfpathlineto{\pgfqpoint{0.457073in}{0.930524in}}%
\pgfpathlineto{\pgfqpoint{0.457810in}{0.928984in}}%
\pgfpathlineto{\pgfqpoint{0.463187in}{0.916913in}}%
\pgfpathlineto{\pgfqpoint{0.470261in}{0.903302in}}%
\pgfpathlineto{\pgfqpoint{0.473466in}{0.897888in}}%
\pgfpathlineto{\pgfqpoint{0.478033in}{0.889691in}}%
\pgfpathlineto{\pgfqpoint{0.486547in}{0.876079in}}%
\pgfpathlineto{\pgfqpoint{0.489123in}{0.872356in}}%
\pgfpathlineto{\pgfqpoint{0.495617in}{0.862468in}}%
\pgfpathlineto{\pgfqpoint{0.504779in}{0.849757in}}%
\pgfpathlineto{\pgfqpoint{0.505400in}{0.848857in}}%
\pgfpathlineto{\pgfqpoint{0.515568in}{0.835246in}}%
\pgfpathlineto{\pgfqpoint{0.520436in}{0.829198in}}%
\pgfpathlineto{\pgfqpoint{0.526306in}{0.821635in}}%
\pgfpathlineto{\pgfqpoint{0.536093in}{0.809857in}}%
\pgfpathlineto{\pgfqpoint{0.537571in}{0.808024in}}%
\pgfpathlineto{\pgfqpoint{0.549220in}{0.794413in}}%
\pgfpathlineto{\pgfqpoint{0.551749in}{0.791614in}}%
\pgfpathlineto{\pgfqpoint{0.561293in}{0.780802in}}%
\pgfpathlineto{\pgfqpoint{0.567406in}{0.774217in}}%
\pgfpathlineto{\pgfqpoint{0.573812in}{0.767191in}}%
\pgfpathlineto{\pgfqpoint{0.583062in}{0.757496in}}%
\pgfpathlineto{\pgfqpoint{0.586752in}{0.753579in}}%
\pgfpathlineto{\pgfqpoint{0.598719in}{0.741382in}}%
\pgfpathlineto{\pgfqpoint{0.600095in}{0.739968in}}%
\pgfpathlineto{\pgfqpoint{0.613826in}{0.726357in}}%
\pgfpathlineto{\pgfqpoint{0.614375in}{0.725829in}}%
\pgfpathlineto{\pgfqpoint{0.627946in}{0.712746in}}%
\pgfpathlineto{\pgfqpoint{0.630032in}{0.710789in}}%
\pgfpathlineto{\pgfqpoint{0.642480in}{0.699135in}}%
\pgfpathlineto{\pgfqpoint{0.645689in}{0.696200in}}%
\pgfpathlineto{\pgfqpoint{0.657434in}{0.685524in}}%
\pgfpathlineto{\pgfqpoint{0.661345in}{0.682036in}}%
\pgfpathlineto{\pgfqpoint{0.672818in}{0.671913in}}%
\pgfpathlineto{\pgfqpoint{0.677002in}{0.668275in}}%
\pgfpathlineto{\pgfqpoint{0.688646in}{0.658302in}}%
\pgfpathlineto{\pgfqpoint{0.692658in}{0.654901in}}%
\pgfpathlineto{\pgfqpoint{0.704939in}{0.644691in}}%
\pgfpathlineto{\pgfqpoint{0.708315in}{0.641901in}}%
\pgfpathlineto{\pgfqpoint{0.721720in}{0.631079in}}%
\pgfpathlineto{\pgfqpoint{0.723971in}{0.629266in}}%
\pgfpathlineto{\pgfqpoint{0.739021in}{0.617468in}}%
\pgfpathlineto{\pgfqpoint{0.739628in}{0.616991in}}%
\pgfpathlineto{\pgfqpoint{0.755284in}{0.605054in}}%
\pgfpathlineto{\pgfqpoint{0.756911in}{0.603857in}}%
\pgfpathlineto{\pgfqpoint{0.770941in}{0.593454in}}%
\pgfpathlineto{\pgfqpoint{0.775446in}{0.590246in}}%
\pgfpathlineto{\pgfqpoint{0.786598in}{0.582205in}}%
\pgfpathlineto{\pgfqpoint{0.794680in}{0.576635in}}%
\pgfpathlineto{\pgfqpoint{0.802254in}{0.571321in}}%
\pgfpathlineto{\pgfqpoint{0.814692in}{0.563024in}}%
\pgfpathlineto{\pgfqpoint{0.817911in}{0.560825in}}%
\pgfpathlineto{\pgfqpoint{0.833567in}{0.550698in}}%
\pgfpathlineto{\pgfqpoint{0.835676in}{0.549413in}}%
\pgfpathlineto{\pgfqpoint{0.849224in}{0.540905in}}%
\pgfpathlineto{\pgfqpoint{0.857924in}{0.535802in}}%
\pgfpathlineto{\pgfqpoint{0.864880in}{0.531569in}}%
\pgfpathlineto{\pgfqpoint{0.880537in}{0.522730in}}%
\pgfpathlineto{\pgfqpoint{0.881572in}{0.522191in}}%
\pgfpathlineto{\pgfqpoint{0.896194in}{0.514225in}}%
\pgfpathlineto{\pgfqpoint{0.907567in}{0.508579in}}%
\pgfpathlineto{\pgfqpoint{0.911850in}{0.506340in}}%
\pgfpathlineto{\pgfqpoint{0.927507in}{0.498938in}}%
\pgfpathlineto{\pgfqpoint{0.936936in}{0.494968in}}%
\pgfpathlineto{\pgfqpoint{0.943163in}{0.492181in}}%
\pgfpathlineto{\pgfqpoint{0.958820in}{0.486032in}}%
\pgfpathlineto{\pgfqpoint{0.972706in}{0.481357in}}%
\pgfpathlineto{\pgfqpoint{0.974476in}{0.480717in}}%
\pgfpathlineto{\pgfqpoint{0.990133in}{0.475983in}}%
\pgfpathlineto{\pgfqpoint{1.005790in}{0.472197in}}%
\pgfpathlineto{\pgfqpoint{1.021446in}{0.469358in}}%
\pgfpathlineto{\pgfqpoint{1.034787in}{0.467746in}}%
\pgfpathlineto{\pgfqpoint{1.037103in}{0.467441in}}%
\pgfpathclose%
\pgfpathmoveto{\pgfqpoint{0.974471in}{0.617468in}}%
\pgfpathlineto{\pgfqpoint{0.958820in}{0.621556in}}%
\pgfpathlineto{\pgfqpoint{0.943163in}{0.626329in}}%
\pgfpathlineto{\pgfqpoint{0.929526in}{0.631079in}}%
\pgfpathlineto{\pgfqpoint{0.927507in}{0.631779in}}%
\pgfpathlineto{\pgfqpoint{0.911850in}{0.637851in}}%
\pgfpathlineto{\pgfqpoint{0.896194in}{0.644600in}}%
\pgfpathlineto{\pgfqpoint{0.896001in}{0.644691in}}%
\pgfpathlineto{\pgfqpoint{0.880537in}{0.651971in}}%
\pgfpathlineto{\pgfqpoint{0.868188in}{0.658302in}}%
\pgfpathlineto{\pgfqpoint{0.864880in}{0.660006in}}%
\pgfpathlineto{\pgfqpoint{0.849224in}{0.668685in}}%
\pgfpathlineto{\pgfqpoint{0.843779in}{0.671913in}}%
\pgfpathlineto{\pgfqpoint{0.833567in}{0.678030in}}%
\pgfpathlineto{\pgfqpoint{0.821809in}{0.685524in}}%
\pgfpathlineto{\pgfqpoint{0.817911in}{0.688046in}}%
\pgfpathlineto{\pgfqpoint{0.802254in}{0.698750in}}%
\pgfpathlineto{\pgfqpoint{0.801718in}{0.699135in}}%
\pgfpathlineto{\pgfqpoint{0.786598in}{0.710197in}}%
\pgfpathlineto{\pgfqpoint{0.783265in}{0.712746in}}%
\pgfpathlineto{\pgfqpoint{0.770941in}{0.722409in}}%
\pgfpathlineto{\pgfqpoint{0.766100in}{0.726357in}}%
\pgfpathlineto{\pgfqpoint{0.755284in}{0.735443in}}%
\pgfpathlineto{\pgfqpoint{0.750079in}{0.739968in}}%
\pgfpathlineto{\pgfqpoint{0.739628in}{0.749371in}}%
\pgfpathlineto{\pgfqpoint{0.735087in}{0.753579in}}%
\pgfpathlineto{\pgfqpoint{0.723971in}{0.764293in}}%
\pgfpathlineto{\pgfqpoint{0.721039in}{0.767191in}}%
\pgfpathlineto{\pgfqpoint{0.708315in}{0.780335in}}%
\pgfpathlineto{\pgfqpoint{0.707872in}{0.780802in}}%
\pgfpathlineto{\pgfqpoint{0.695559in}{0.794413in}}%
\pgfpathlineto{\pgfqpoint{0.692658in}{0.797801in}}%
\pgfpathlineto{\pgfqpoint{0.684038in}{0.808024in}}%
\pgfpathlineto{\pgfqpoint{0.677002in}{0.816901in}}%
\pgfpathlineto{\pgfqpoint{0.673288in}{0.821635in}}%
\pgfpathlineto{\pgfqpoint{0.663305in}{0.835246in}}%
\pgfpathlineto{\pgfqpoint{0.661345in}{0.838122in}}%
\pgfpathlineto{\pgfqpoint{0.654063in}{0.848857in}}%
\pgfpathlineto{\pgfqpoint{0.645689in}{0.862301in}}%
\pgfpathlineto{\pgfqpoint{0.645584in}{0.862468in}}%
\pgfpathlineto{\pgfqpoint{0.637821in}{0.876079in}}%
\pgfpathlineto{\pgfqpoint{0.630836in}{0.889691in}}%
\pgfpathlineto{\pgfqpoint{0.630032in}{0.891446in}}%
\pgfpathlineto{\pgfqpoint{0.624568in}{0.903302in}}%
\pgfpathlineto{\pgfqpoint{0.619077in}{0.916913in}}%
\pgfpathlineto{\pgfqpoint{0.614375in}{0.930519in}}%
\pgfpathlineto{\pgfqpoint{0.614374in}{0.930524in}}%
\pgfpathlineto{\pgfqpoint{0.610394in}{0.944135in}}%
\pgfpathlineto{\pgfqpoint{0.607211in}{0.957746in}}%
\pgfpathlineto{\pgfqpoint{0.604825in}{0.971357in}}%
\pgfpathlineto{\pgfqpoint{0.603234in}{0.984968in}}%
\pgfpathlineto{\pgfqpoint{0.602439in}{0.998579in}}%
\pgfpathlineto{\pgfqpoint{0.602439in}{1.012191in}}%
\pgfpathlineto{\pgfqpoint{0.603234in}{1.025802in}}%
\pgfpathlineto{\pgfqpoint{0.604825in}{1.039413in}}%
\pgfpathlineto{\pgfqpoint{0.607211in}{1.053024in}}%
\pgfpathlineto{\pgfqpoint{0.610394in}{1.066635in}}%
\pgfpathlineto{\pgfqpoint{0.614374in}{1.080246in}}%
\pgfpathlineto{\pgfqpoint{0.614375in}{1.080251in}}%
\pgfpathlineto{\pgfqpoint{0.619077in}{1.093857in}}%
\pgfpathlineto{\pgfqpoint{0.624568in}{1.107468in}}%
\pgfpathlineto{\pgfqpoint{0.630032in}{1.119324in}}%
\pgfpathlineto{\pgfqpoint{0.630836in}{1.121079in}}%
\pgfpathlineto{\pgfqpoint{0.637821in}{1.134691in}}%
\pgfpathlineto{\pgfqpoint{0.645584in}{1.148302in}}%
\pgfpathlineto{\pgfqpoint{0.645689in}{1.148469in}}%
\pgfpathlineto{\pgfqpoint{0.654063in}{1.161913in}}%
\pgfpathlineto{\pgfqpoint{0.661345in}{1.172648in}}%
\pgfpathlineto{\pgfqpoint{0.663305in}{1.175524in}}%
\pgfpathlineto{\pgfqpoint{0.673288in}{1.189135in}}%
\pgfpathlineto{\pgfqpoint{0.677002in}{1.193869in}}%
\pgfpathlineto{\pgfqpoint{0.684038in}{1.202746in}}%
\pgfpathlineto{\pgfqpoint{0.692658in}{1.212969in}}%
\pgfpathlineto{\pgfqpoint{0.695559in}{1.216357in}}%
\pgfpathlineto{\pgfqpoint{0.707872in}{1.229968in}}%
\pgfpathlineto{\pgfqpoint{0.708315in}{1.230435in}}%
\pgfpathlineto{\pgfqpoint{0.721039in}{1.243579in}}%
\pgfpathlineto{\pgfqpoint{0.723971in}{1.246477in}}%
\pgfpathlineto{\pgfqpoint{0.735087in}{1.257191in}}%
\pgfpathlineto{\pgfqpoint{0.739628in}{1.261399in}}%
\pgfpathlineto{\pgfqpoint{0.750079in}{1.270802in}}%
\pgfpathlineto{\pgfqpoint{0.755284in}{1.275327in}}%
\pgfpathlineto{\pgfqpoint{0.766100in}{1.284413in}}%
\pgfpathlineto{\pgfqpoint{0.770941in}{1.288361in}}%
\pgfpathlineto{\pgfqpoint{0.783265in}{1.298024in}}%
\pgfpathlineto{\pgfqpoint{0.786598in}{1.300573in}}%
\pgfpathlineto{\pgfqpoint{0.801718in}{1.311635in}}%
\pgfpathlineto{\pgfqpoint{0.802254in}{1.312020in}}%
\pgfpathlineto{\pgfqpoint{0.817911in}{1.322724in}}%
\pgfpathlineto{\pgfqpoint{0.821809in}{1.325246in}}%
\pgfpathlineto{\pgfqpoint{0.833567in}{1.332740in}}%
\pgfpathlineto{\pgfqpoint{0.843779in}{1.338857in}}%
\pgfpathlineto{\pgfqpoint{0.849224in}{1.342085in}}%
\pgfpathlineto{\pgfqpoint{0.864880in}{1.350764in}}%
\pgfpathlineto{\pgfqpoint{0.868188in}{1.352468in}}%
\pgfpathlineto{\pgfqpoint{0.880537in}{1.358799in}}%
\pgfpathlineto{\pgfqpoint{0.896001in}{1.366079in}}%
\pgfpathlineto{\pgfqpoint{0.896194in}{1.366170in}}%
\pgfpathlineto{\pgfqpoint{0.911850in}{1.372919in}}%
\pgfpathlineto{\pgfqpoint{0.927507in}{1.378991in}}%
\pgfpathlineto{\pgfqpoint{0.929526in}{1.379691in}}%
\pgfpathlineto{\pgfqpoint{0.943163in}{1.384441in}}%
\pgfpathlineto{\pgfqpoint{0.958820in}{1.389214in}}%
\pgfpathlineto{\pgfqpoint{0.974471in}{1.393302in}}%
\pgfpathlineto{\pgfqpoint{0.974476in}{1.393303in}}%
\pgfpathlineto{\pgfqpoint{0.990133in}{1.396763in}}%
\pgfpathlineto{\pgfqpoint{1.005790in}{1.399530in}}%
\pgfpathlineto{\pgfqpoint{1.021446in}{1.401605in}}%
\pgfpathlineto{\pgfqpoint{1.037103in}{1.402987in}}%
\pgfpathlineto{\pgfqpoint{1.052759in}{1.403678in}}%
\pgfpathlineto{\pgfqpoint{1.068416in}{1.403678in}}%
\pgfpathlineto{\pgfqpoint{1.084072in}{1.402987in}}%
\pgfpathlineto{\pgfqpoint{1.099729in}{1.401605in}}%
\pgfpathlineto{\pgfqpoint{1.115385in}{1.399530in}}%
\pgfpathlineto{\pgfqpoint{1.131042in}{1.396763in}}%
\pgfpathlineto{\pgfqpoint{1.146699in}{1.393303in}}%
\pgfpathlineto{\pgfqpoint{1.146704in}{1.393302in}}%
\pgfpathlineto{\pgfqpoint{1.162355in}{1.389214in}}%
\pgfpathlineto{\pgfqpoint{1.178012in}{1.384441in}}%
\pgfpathlineto{\pgfqpoint{1.191649in}{1.379691in}}%
\pgfpathlineto{\pgfqpoint{1.193668in}{1.378991in}}%
\pgfpathlineto{\pgfqpoint{1.209325in}{1.372919in}}%
\pgfpathlineto{\pgfqpoint{1.224981in}{1.366170in}}%
\pgfpathlineto{\pgfqpoint{1.225174in}{1.366079in}}%
\pgfpathlineto{\pgfqpoint{1.240638in}{1.358799in}}%
\pgfpathlineto{\pgfqpoint{1.252987in}{1.352468in}}%
\pgfpathlineto{\pgfqpoint{1.256295in}{1.350764in}}%
\pgfpathlineto{\pgfqpoint{1.271951in}{1.342085in}}%
\pgfpathlineto{\pgfqpoint{1.277396in}{1.338857in}}%
\pgfpathlineto{\pgfqpoint{1.287608in}{1.332740in}}%
\pgfpathlineto{\pgfqpoint{1.299366in}{1.325246in}}%
\pgfpathlineto{\pgfqpoint{1.303264in}{1.322724in}}%
\pgfpathlineto{\pgfqpoint{1.318921in}{1.312020in}}%
\pgfpathlineto{\pgfqpoint{1.319457in}{1.311635in}}%
\pgfpathlineto{\pgfqpoint{1.334577in}{1.300573in}}%
\pgfpathlineto{\pgfqpoint{1.337910in}{1.298024in}}%
\pgfpathlineto{\pgfqpoint{1.350234in}{1.288361in}}%
\pgfpathlineto{\pgfqpoint{1.355075in}{1.284413in}}%
\pgfpathlineto{\pgfqpoint{1.365891in}{1.275327in}}%
\pgfpathlineto{\pgfqpoint{1.371096in}{1.270802in}}%
\pgfpathlineto{\pgfqpoint{1.381547in}{1.261399in}}%
\pgfpathlineto{\pgfqpoint{1.386088in}{1.257191in}}%
\pgfpathlineto{\pgfqpoint{1.397204in}{1.246477in}}%
\pgfpathlineto{\pgfqpoint{1.400136in}{1.243579in}}%
\pgfpathlineto{\pgfqpoint{1.412860in}{1.230435in}}%
\pgfpathlineto{\pgfqpoint{1.413303in}{1.229968in}}%
\pgfpathlineto{\pgfqpoint{1.425616in}{1.216357in}}%
\pgfpathlineto{\pgfqpoint{1.428517in}{1.212969in}}%
\pgfpathlineto{\pgfqpoint{1.437137in}{1.202746in}}%
\pgfpathlineto{\pgfqpoint{1.444173in}{1.193869in}}%
\pgfpathlineto{\pgfqpoint{1.447887in}{1.189135in}}%
\pgfpathlineto{\pgfqpoint{1.457870in}{1.175524in}}%
\pgfpathlineto{\pgfqpoint{1.459830in}{1.172648in}}%
\pgfpathlineto{\pgfqpoint{1.467112in}{1.161913in}}%
\pgfpathlineto{\pgfqpoint{1.475486in}{1.148469in}}%
\pgfpathlineto{\pgfqpoint{1.475591in}{1.148302in}}%
\pgfpathlineto{\pgfqpoint{1.483354in}{1.134691in}}%
\pgfpathlineto{\pgfqpoint{1.490339in}{1.121079in}}%
\pgfpathlineto{\pgfqpoint{1.491143in}{1.119324in}}%
\pgfpathlineto{\pgfqpoint{1.496607in}{1.107468in}}%
\pgfpathlineto{\pgfqpoint{1.502098in}{1.093857in}}%
\pgfpathlineto{\pgfqpoint{1.506800in}{1.080251in}}%
\pgfpathlineto{\pgfqpoint{1.506801in}{1.080246in}}%
\pgfpathlineto{\pgfqpoint{1.510781in}{1.066635in}}%
\pgfpathlineto{\pgfqpoint{1.513964in}{1.053024in}}%
\pgfpathlineto{\pgfqpoint{1.516350in}{1.039413in}}%
\pgfpathlineto{\pgfqpoint{1.517941in}{1.025802in}}%
\pgfpathlineto{\pgfqpoint{1.518736in}{1.012191in}}%
\pgfpathlineto{\pgfqpoint{1.518736in}{0.998579in}}%
\pgfpathlineto{\pgfqpoint{1.517941in}{0.984968in}}%
\pgfpathlineto{\pgfqpoint{1.516350in}{0.971357in}}%
\pgfpathlineto{\pgfqpoint{1.513964in}{0.957746in}}%
\pgfpathlineto{\pgfqpoint{1.510781in}{0.944135in}}%
\pgfpathlineto{\pgfqpoint{1.506801in}{0.930524in}}%
\pgfpathlineto{\pgfqpoint{1.506800in}{0.930519in}}%
\pgfpathlineto{\pgfqpoint{1.502098in}{0.916913in}}%
\pgfpathlineto{\pgfqpoint{1.496607in}{0.903302in}}%
\pgfpathlineto{\pgfqpoint{1.491143in}{0.891446in}}%
\pgfpathlineto{\pgfqpoint{1.490339in}{0.889691in}}%
\pgfpathlineto{\pgfqpoint{1.483354in}{0.876079in}}%
\pgfpathlineto{\pgfqpoint{1.475591in}{0.862468in}}%
\pgfpathlineto{\pgfqpoint{1.475486in}{0.862301in}}%
\pgfpathlineto{\pgfqpoint{1.467112in}{0.848857in}}%
\pgfpathlineto{\pgfqpoint{1.459830in}{0.838122in}}%
\pgfpathlineto{\pgfqpoint{1.457870in}{0.835246in}}%
\pgfpathlineto{\pgfqpoint{1.447887in}{0.821635in}}%
\pgfpathlineto{\pgfqpoint{1.444173in}{0.816901in}}%
\pgfpathlineto{\pgfqpoint{1.437137in}{0.808024in}}%
\pgfpathlineto{\pgfqpoint{1.428517in}{0.797801in}}%
\pgfpathlineto{\pgfqpoint{1.425616in}{0.794413in}}%
\pgfpathlineto{\pgfqpoint{1.413303in}{0.780802in}}%
\pgfpathlineto{\pgfqpoint{1.412860in}{0.780335in}}%
\pgfpathlineto{\pgfqpoint{1.400136in}{0.767191in}}%
\pgfpathlineto{\pgfqpoint{1.397204in}{0.764293in}}%
\pgfpathlineto{\pgfqpoint{1.386088in}{0.753579in}}%
\pgfpathlineto{\pgfqpoint{1.381547in}{0.749371in}}%
\pgfpathlineto{\pgfqpoint{1.371096in}{0.739968in}}%
\pgfpathlineto{\pgfqpoint{1.365891in}{0.735443in}}%
\pgfpathlineto{\pgfqpoint{1.355075in}{0.726357in}}%
\pgfpathlineto{\pgfqpoint{1.350234in}{0.722409in}}%
\pgfpathlineto{\pgfqpoint{1.337910in}{0.712746in}}%
\pgfpathlineto{\pgfqpoint{1.334577in}{0.710197in}}%
\pgfpathlineto{\pgfqpoint{1.319457in}{0.699135in}}%
\pgfpathlineto{\pgfqpoint{1.318921in}{0.698750in}}%
\pgfpathlineto{\pgfqpoint{1.303264in}{0.688046in}}%
\pgfpathlineto{\pgfqpoint{1.299366in}{0.685524in}}%
\pgfpathlineto{\pgfqpoint{1.287608in}{0.678030in}}%
\pgfpathlineto{\pgfqpoint{1.277396in}{0.671913in}}%
\pgfpathlineto{\pgfqpoint{1.271951in}{0.668685in}}%
\pgfpathlineto{\pgfqpoint{1.256295in}{0.660006in}}%
\pgfpathlineto{\pgfqpoint{1.252987in}{0.658302in}}%
\pgfpathlineto{\pgfqpoint{1.240638in}{0.651971in}}%
\pgfpathlineto{\pgfqpoint{1.225174in}{0.644691in}}%
\pgfpathlineto{\pgfqpoint{1.224981in}{0.644600in}}%
\pgfpathlineto{\pgfqpoint{1.209325in}{0.637851in}}%
\pgfpathlineto{\pgfqpoint{1.193668in}{0.631779in}}%
\pgfpathlineto{\pgfqpoint{1.191649in}{0.631079in}}%
\pgfpathlineto{\pgfqpoint{1.178012in}{0.626329in}}%
\pgfpathlineto{\pgfqpoint{1.162355in}{0.621556in}}%
\pgfpathlineto{\pgfqpoint{1.146704in}{0.617468in}}%
\pgfpathlineto{\pgfqpoint{1.146699in}{0.617467in}}%
\pgfpathlineto{\pgfqpoint{1.131042in}{0.614007in}}%
\pgfpathlineto{\pgfqpoint{1.115385in}{0.611240in}}%
\pgfpathlineto{\pgfqpoint{1.099729in}{0.609165in}}%
\pgfpathlineto{\pgfqpoint{1.084072in}{0.607783in}}%
\pgfpathlineto{\pgfqpoint{1.068416in}{0.607092in}}%
\pgfpathlineto{\pgfqpoint{1.052759in}{0.607092in}}%
\pgfpathlineto{\pgfqpoint{1.037103in}{0.607783in}}%
\pgfpathlineto{\pgfqpoint{1.021446in}{0.609165in}}%
\pgfpathlineto{\pgfqpoint{1.005790in}{0.611240in}}%
\pgfpathlineto{\pgfqpoint{0.990133in}{0.614007in}}%
\pgfpathlineto{\pgfqpoint{0.974476in}{0.617467in}}%
\pgfpathlineto{\pgfqpoint{0.974471in}{0.617468in}}%
\pgfpathclose%
\pgfusepath{fill}%
\end{pgfscope}%
\begin{pgfscope}%
\pgfpathrectangle{\pgfqpoint{0.285588in}{0.331635in}}{\pgfqpoint{1.550000in}{1.347500in}}%
\pgfusepath{clip}%
\pgfsetbuttcap%
\pgfsetroundjoin%
\definecolor{currentfill}{rgb}{0.481929,0.136891,0.507989}%
\pgfsetfillcolor{currentfill}%
\pgfsetlinewidth{0.000000pt}%
\definecolor{currentstroke}{rgb}{0.000000,0.000000,0.000000}%
\pgfsetstrokecolor{currentstroke}%
\pgfsetdash{}{0pt}%
\pgfpathmoveto{\pgfqpoint{0.802254in}{0.331635in}}%
\pgfpathlineto{\pgfqpoint{0.817911in}{0.331635in}}%
\pgfpathlineto{\pgfqpoint{0.833567in}{0.331635in}}%
\pgfpathlineto{\pgfqpoint{0.849224in}{0.331635in}}%
\pgfpathlineto{\pgfqpoint{0.864880in}{0.331635in}}%
\pgfpathlineto{\pgfqpoint{0.880537in}{0.331635in}}%
\pgfpathlineto{\pgfqpoint{0.896194in}{0.331635in}}%
\pgfpathlineto{\pgfqpoint{0.911850in}{0.331635in}}%
\pgfpathlineto{\pgfqpoint{0.927507in}{0.331635in}}%
\pgfpathlineto{\pgfqpoint{0.943163in}{0.331635in}}%
\pgfpathlineto{\pgfqpoint{0.958820in}{0.331635in}}%
\pgfpathlineto{\pgfqpoint{0.974476in}{0.331635in}}%
\pgfpathlineto{\pgfqpoint{0.990133in}{0.331635in}}%
\pgfpathlineto{\pgfqpoint{1.005790in}{0.331635in}}%
\pgfpathlineto{\pgfqpoint{1.021446in}{0.331635in}}%
\pgfpathlineto{\pgfqpoint{1.037103in}{0.331635in}}%
\pgfpathlineto{\pgfqpoint{1.052759in}{0.331635in}}%
\pgfpathlineto{\pgfqpoint{1.068416in}{0.331635in}}%
\pgfpathlineto{\pgfqpoint{1.084072in}{0.331635in}}%
\pgfpathlineto{\pgfqpoint{1.099729in}{0.331635in}}%
\pgfpathlineto{\pgfqpoint{1.115385in}{0.331635in}}%
\pgfpathlineto{\pgfqpoint{1.131042in}{0.331635in}}%
\pgfpathlineto{\pgfqpoint{1.146699in}{0.331635in}}%
\pgfpathlineto{\pgfqpoint{1.162355in}{0.331635in}}%
\pgfpathlineto{\pgfqpoint{1.178012in}{0.331635in}}%
\pgfpathlineto{\pgfqpoint{1.193668in}{0.331635in}}%
\pgfpathlineto{\pgfqpoint{1.209325in}{0.331635in}}%
\pgfpathlineto{\pgfqpoint{1.224981in}{0.331635in}}%
\pgfpathlineto{\pgfqpoint{1.240638in}{0.331635in}}%
\pgfpathlineto{\pgfqpoint{1.256295in}{0.331635in}}%
\pgfpathlineto{\pgfqpoint{1.271951in}{0.331635in}}%
\pgfpathlineto{\pgfqpoint{1.287608in}{0.331635in}}%
\pgfpathlineto{\pgfqpoint{1.303264in}{0.331635in}}%
\pgfpathlineto{\pgfqpoint{1.318921in}{0.331635in}}%
\pgfpathlineto{\pgfqpoint{1.325916in}{0.331635in}}%
\pgfpathlineto{\pgfqpoint{1.326710in}{0.345246in}}%
\pgfpathlineto{\pgfqpoint{1.329086in}{0.358857in}}%
\pgfpathlineto{\pgfqpoint{1.333018in}{0.372468in}}%
\pgfpathlineto{\pgfqpoint{1.334577in}{0.376356in}}%
\pgfpathlineto{\pgfqpoint{1.338344in}{0.386079in}}%
\pgfpathlineto{\pgfqpoint{1.345033in}{0.399691in}}%
\pgfpathlineto{\pgfqpoint{1.350234in}{0.408485in}}%
\pgfpathlineto{\pgfqpoint{1.352992in}{0.413302in}}%
\pgfpathlineto{\pgfqpoint{1.362044in}{0.426913in}}%
\pgfpathlineto{\pgfqpoint{1.365891in}{0.432024in}}%
\pgfpathlineto{\pgfqpoint{1.372100in}{0.440524in}}%
\pgfpathlineto{\pgfqpoint{1.381547in}{0.452205in}}%
\pgfpathlineto{\pgfqpoint{1.383068in}{0.454135in}}%
\pgfpathlineto{\pgfqpoint{1.394778in}{0.467746in}}%
\pgfpathlineto{\pgfqpoint{1.397204in}{0.470363in}}%
\pgfpathlineto{\pgfqpoint{1.407159in}{0.481357in}}%
\pgfpathlineto{\pgfqpoint{1.412860in}{0.487272in}}%
\pgfpathlineto{\pgfqpoint{1.420132in}{0.494968in}}%
\pgfpathlineto{\pgfqpoint{1.428517in}{0.503385in}}%
\pgfpathlineto{\pgfqpoint{1.433607in}{0.508579in}}%
\pgfpathlineto{\pgfqpoint{1.444173in}{0.518889in}}%
\pgfpathlineto{\pgfqpoint{1.447513in}{0.522191in}}%
\pgfpathlineto{\pgfqpoint{1.459830in}{0.533918in}}%
\pgfpathlineto{\pgfqpoint{1.461790in}{0.535802in}}%
\pgfpathlineto{\pgfqpoint{1.475486in}{0.548564in}}%
\pgfpathlineto{\pgfqpoint{1.476392in}{0.549413in}}%
\pgfpathlineto{\pgfqpoint{1.491143in}{0.562896in}}%
\pgfpathlineto{\pgfqpoint{1.491283in}{0.563024in}}%
\pgfpathlineto{\pgfqpoint{1.506443in}{0.576635in}}%
\pgfpathlineto{\pgfqpoint{1.506800in}{0.576950in}}%
\pgfpathlineto{\pgfqpoint{1.521854in}{0.590246in}}%
\pgfpathlineto{\pgfqpoint{1.522456in}{0.590772in}}%
\pgfpathlineto{\pgfqpoint{1.537508in}{0.603857in}}%
\pgfpathlineto{\pgfqpoint{1.538113in}{0.604380in}}%
\pgfpathlineto{\pgfqpoint{1.553407in}{0.617468in}}%
\pgfpathlineto{\pgfqpoint{1.553769in}{0.617778in}}%
\pgfpathlineto{\pgfqpoint{1.569426in}{0.630958in}}%
\pgfpathlineto{\pgfqpoint{1.569573in}{0.631079in}}%
\pgfpathlineto{\pgfqpoint{1.585082in}{0.643904in}}%
\pgfpathlineto{\pgfqpoint{1.586058in}{0.644691in}}%
\pgfpathlineto{\pgfqpoint{1.600739in}{0.656598in}}%
\pgfpathlineto{\pgfqpoint{1.602906in}{0.658302in}}%
\pgfpathlineto{\pgfqpoint{1.616396in}{0.669010in}}%
\pgfpathlineto{\pgfqpoint{1.620193in}{0.671913in}}%
\pgfpathlineto{\pgfqpoint{1.632052in}{0.681099in}}%
\pgfpathlineto{\pgfqpoint{1.638027in}{0.685524in}}%
\pgfpathlineto{\pgfqpoint{1.647709in}{0.692814in}}%
\pgfpathlineto{\pgfqpoint{1.656562in}{0.699135in}}%
\pgfpathlineto{\pgfqpoint{1.663365in}{0.704091in}}%
\pgfpathlineto{\pgfqpoint{1.676012in}{0.712746in}}%
\pgfpathlineto{\pgfqpoint{1.679022in}{0.714855in}}%
\pgfpathlineto{\pgfqpoint{1.694678in}{0.725035in}}%
\pgfpathlineto{\pgfqpoint{1.696898in}{0.726357in}}%
\pgfpathlineto{\pgfqpoint{1.710335in}{0.734570in}}%
\pgfpathlineto{\pgfqpoint{1.720112in}{0.739968in}}%
\pgfpathlineto{\pgfqpoint{1.725992in}{0.743313in}}%
\pgfpathlineto{\pgfqpoint{1.741648in}{0.751182in}}%
\pgfpathlineto{\pgfqpoint{1.747188in}{0.753579in}}%
\pgfpathlineto{\pgfqpoint{1.757305in}{0.758101in}}%
\pgfpathlineto{\pgfqpoint{1.772961in}{0.763916in}}%
\pgfpathlineto{\pgfqpoint{1.784146in}{0.767191in}}%
\pgfpathlineto{\pgfqpoint{1.788618in}{0.768546in}}%
\pgfpathlineto{\pgfqpoint{1.804274in}{0.771964in}}%
\pgfpathlineto{\pgfqpoint{1.819931in}{0.774030in}}%
\pgfpathlineto{\pgfqpoint{1.835588in}{0.774721in}}%
\pgfpathlineto{\pgfqpoint{1.835588in}{0.780802in}}%
\pgfpathlineto{\pgfqpoint{1.835588in}{0.794413in}}%
\pgfpathlineto{\pgfqpoint{1.835588in}{0.808024in}}%
\pgfpathlineto{\pgfqpoint{1.835588in}{0.821635in}}%
\pgfpathlineto{\pgfqpoint{1.835588in}{0.835246in}}%
\pgfpathlineto{\pgfqpoint{1.835588in}{0.848857in}}%
\pgfpathlineto{\pgfqpoint{1.835588in}{0.862468in}}%
\pgfpathlineto{\pgfqpoint{1.835588in}{0.876079in}}%
\pgfpathlineto{\pgfqpoint{1.835588in}{0.889691in}}%
\pgfpathlineto{\pgfqpoint{1.835588in}{0.903302in}}%
\pgfpathlineto{\pgfqpoint{1.835588in}{0.916913in}}%
\pgfpathlineto{\pgfqpoint{1.835588in}{0.930524in}}%
\pgfpathlineto{\pgfqpoint{1.835588in}{0.944135in}}%
\pgfpathlineto{\pgfqpoint{1.835588in}{0.957746in}}%
\pgfpathlineto{\pgfqpoint{1.835588in}{0.971357in}}%
\pgfpathlineto{\pgfqpoint{1.835588in}{0.984968in}}%
\pgfpathlineto{\pgfqpoint{1.835588in}{0.998579in}}%
\pgfpathlineto{\pgfqpoint{1.835588in}{1.012191in}}%
\pgfpathlineto{\pgfqpoint{1.835588in}{1.025802in}}%
\pgfpathlineto{\pgfqpoint{1.835588in}{1.039413in}}%
\pgfpathlineto{\pgfqpoint{1.835588in}{1.053024in}}%
\pgfpathlineto{\pgfqpoint{1.835588in}{1.066635in}}%
\pgfpathlineto{\pgfqpoint{1.835588in}{1.080246in}}%
\pgfpathlineto{\pgfqpoint{1.835588in}{1.093857in}}%
\pgfpathlineto{\pgfqpoint{1.835588in}{1.107468in}}%
\pgfpathlineto{\pgfqpoint{1.835588in}{1.121079in}}%
\pgfpathlineto{\pgfqpoint{1.835588in}{1.134691in}}%
\pgfpathlineto{\pgfqpoint{1.835588in}{1.148302in}}%
\pgfpathlineto{\pgfqpoint{1.835588in}{1.161913in}}%
\pgfpathlineto{\pgfqpoint{1.835588in}{1.175524in}}%
\pgfpathlineto{\pgfqpoint{1.835588in}{1.189135in}}%
\pgfpathlineto{\pgfqpoint{1.835588in}{1.202746in}}%
\pgfpathlineto{\pgfqpoint{1.835588in}{1.216357in}}%
\pgfpathlineto{\pgfqpoint{1.835588in}{1.229968in}}%
\pgfpathlineto{\pgfqpoint{1.835588in}{1.236049in}}%
\pgfpathlineto{\pgfqpoint{1.819931in}{1.236740in}}%
\pgfpathlineto{\pgfqpoint{1.804274in}{1.238806in}}%
\pgfpathlineto{\pgfqpoint{1.788618in}{1.242224in}}%
\pgfpathlineto{\pgfqpoint{1.784146in}{1.243579in}}%
\pgfpathlineto{\pgfqpoint{1.772961in}{1.246854in}}%
\pgfpathlineto{\pgfqpoint{1.757305in}{1.252669in}}%
\pgfpathlineto{\pgfqpoint{1.747188in}{1.257191in}}%
\pgfpathlineto{\pgfqpoint{1.741648in}{1.259588in}}%
\pgfpathlineto{\pgfqpoint{1.725992in}{1.267457in}}%
\pgfpathlineto{\pgfqpoint{1.720112in}{1.270802in}}%
\pgfpathlineto{\pgfqpoint{1.710335in}{1.276200in}}%
\pgfpathlineto{\pgfqpoint{1.696898in}{1.284413in}}%
\pgfpathlineto{\pgfqpoint{1.694678in}{1.285735in}}%
\pgfpathlineto{\pgfqpoint{1.679022in}{1.295915in}}%
\pgfpathlineto{\pgfqpoint{1.676012in}{1.298024in}}%
\pgfpathlineto{\pgfqpoint{1.663365in}{1.306679in}}%
\pgfpathlineto{\pgfqpoint{1.656562in}{1.311635in}}%
\pgfpathlineto{\pgfqpoint{1.647709in}{1.317956in}}%
\pgfpathlineto{\pgfqpoint{1.638027in}{1.325246in}}%
\pgfpathlineto{\pgfqpoint{1.632052in}{1.329671in}}%
\pgfpathlineto{\pgfqpoint{1.620193in}{1.338857in}}%
\pgfpathlineto{\pgfqpoint{1.616396in}{1.341760in}}%
\pgfpathlineto{\pgfqpoint{1.602906in}{1.352468in}}%
\pgfpathlineto{\pgfqpoint{1.600739in}{1.354172in}}%
\pgfpathlineto{\pgfqpoint{1.586058in}{1.366079in}}%
\pgfpathlineto{\pgfqpoint{1.585082in}{1.366866in}}%
\pgfpathlineto{\pgfqpoint{1.569573in}{1.379691in}}%
\pgfpathlineto{\pgfqpoint{1.569426in}{1.379812in}}%
\pgfpathlineto{\pgfqpoint{1.553769in}{1.392992in}}%
\pgfpathlineto{\pgfqpoint{1.553407in}{1.393302in}}%
\pgfpathlineto{\pgfqpoint{1.538113in}{1.406390in}}%
\pgfpathlineto{\pgfqpoint{1.537508in}{1.406913in}}%
\pgfpathlineto{\pgfqpoint{1.522456in}{1.419998in}}%
\pgfpathlineto{\pgfqpoint{1.521854in}{1.420524in}}%
\pgfpathlineto{\pgfqpoint{1.506800in}{1.433820in}}%
\pgfpathlineto{\pgfqpoint{1.506443in}{1.434135in}}%
\pgfpathlineto{\pgfqpoint{1.491283in}{1.447746in}}%
\pgfpathlineto{\pgfqpoint{1.491143in}{1.447874in}}%
\pgfpathlineto{\pgfqpoint{1.476392in}{1.461357in}}%
\pgfpathlineto{\pgfqpoint{1.475486in}{1.462206in}}%
\pgfpathlineto{\pgfqpoint{1.461790in}{1.474968in}}%
\pgfpathlineto{\pgfqpoint{1.459830in}{1.476852in}}%
\pgfpathlineto{\pgfqpoint{1.447513in}{1.488579in}}%
\pgfpathlineto{\pgfqpoint{1.444173in}{1.491881in}}%
\pgfpathlineto{\pgfqpoint{1.433607in}{1.502191in}}%
\pgfpathlineto{\pgfqpoint{1.428517in}{1.507385in}}%
\pgfpathlineto{\pgfqpoint{1.420132in}{1.515802in}}%
\pgfpathlineto{\pgfqpoint{1.412860in}{1.523498in}}%
\pgfpathlineto{\pgfqpoint{1.407159in}{1.529413in}}%
\pgfpathlineto{\pgfqpoint{1.397204in}{1.540407in}}%
\pgfpathlineto{\pgfqpoint{1.394778in}{1.543024in}}%
\pgfpathlineto{\pgfqpoint{1.383068in}{1.556635in}}%
\pgfpathlineto{\pgfqpoint{1.381547in}{1.558565in}}%
\pgfpathlineto{\pgfqpoint{1.372100in}{1.570246in}}%
\pgfpathlineto{\pgfqpoint{1.365891in}{1.578746in}}%
\pgfpathlineto{\pgfqpoint{1.362044in}{1.583857in}}%
\pgfpathlineto{\pgfqpoint{1.352992in}{1.597468in}}%
\pgfpathlineto{\pgfqpoint{1.350234in}{1.602285in}}%
\pgfpathlineto{\pgfqpoint{1.345033in}{1.611079in}}%
\pgfpathlineto{\pgfqpoint{1.338344in}{1.624691in}}%
\pgfpathlineto{\pgfqpoint{1.334577in}{1.634414in}}%
\pgfpathlineto{\pgfqpoint{1.333018in}{1.638302in}}%
\pgfpathlineto{\pgfqpoint{1.329086in}{1.651913in}}%
\pgfpathlineto{\pgfqpoint{1.326710in}{1.665524in}}%
\pgfpathlineto{\pgfqpoint{1.325916in}{1.679135in}}%
\pgfpathlineto{\pgfqpoint{1.318921in}{1.679135in}}%
\pgfpathlineto{\pgfqpoint{1.303264in}{1.679135in}}%
\pgfpathlineto{\pgfqpoint{1.287608in}{1.679135in}}%
\pgfpathlineto{\pgfqpoint{1.271951in}{1.679135in}}%
\pgfpathlineto{\pgfqpoint{1.256295in}{1.679135in}}%
\pgfpathlineto{\pgfqpoint{1.240638in}{1.679135in}}%
\pgfpathlineto{\pgfqpoint{1.224981in}{1.679135in}}%
\pgfpathlineto{\pgfqpoint{1.209325in}{1.679135in}}%
\pgfpathlineto{\pgfqpoint{1.193668in}{1.679135in}}%
\pgfpathlineto{\pgfqpoint{1.178012in}{1.679135in}}%
\pgfpathlineto{\pgfqpoint{1.162355in}{1.679135in}}%
\pgfpathlineto{\pgfqpoint{1.146699in}{1.679135in}}%
\pgfpathlineto{\pgfqpoint{1.131042in}{1.679135in}}%
\pgfpathlineto{\pgfqpoint{1.115385in}{1.679135in}}%
\pgfpathlineto{\pgfqpoint{1.099729in}{1.679135in}}%
\pgfpathlineto{\pgfqpoint{1.084072in}{1.679135in}}%
\pgfpathlineto{\pgfqpoint{1.068416in}{1.679135in}}%
\pgfpathlineto{\pgfqpoint{1.052759in}{1.679135in}}%
\pgfpathlineto{\pgfqpoint{1.037103in}{1.679135in}}%
\pgfpathlineto{\pgfqpoint{1.021446in}{1.679135in}}%
\pgfpathlineto{\pgfqpoint{1.005790in}{1.679135in}}%
\pgfpathlineto{\pgfqpoint{0.990133in}{1.679135in}}%
\pgfpathlineto{\pgfqpoint{0.974476in}{1.679135in}}%
\pgfpathlineto{\pgfqpoint{0.958820in}{1.679135in}}%
\pgfpathlineto{\pgfqpoint{0.943163in}{1.679135in}}%
\pgfpathlineto{\pgfqpoint{0.927507in}{1.679135in}}%
\pgfpathlineto{\pgfqpoint{0.911850in}{1.679135in}}%
\pgfpathlineto{\pgfqpoint{0.896194in}{1.679135in}}%
\pgfpathlineto{\pgfqpoint{0.880537in}{1.679135in}}%
\pgfpathlineto{\pgfqpoint{0.864880in}{1.679135in}}%
\pgfpathlineto{\pgfqpoint{0.849224in}{1.679135in}}%
\pgfpathlineto{\pgfqpoint{0.833567in}{1.679135in}}%
\pgfpathlineto{\pgfqpoint{0.817911in}{1.679135in}}%
\pgfpathlineto{\pgfqpoint{0.802254in}{1.679135in}}%
\pgfpathlineto{\pgfqpoint{0.795259in}{1.679135in}}%
\pgfpathlineto{\pgfqpoint{0.794465in}{1.665524in}}%
\pgfpathlineto{\pgfqpoint{0.792089in}{1.651913in}}%
\pgfpathlineto{\pgfqpoint{0.788157in}{1.638302in}}%
\pgfpathlineto{\pgfqpoint{0.786598in}{1.634414in}}%
\pgfpathlineto{\pgfqpoint{0.782831in}{1.624691in}}%
\pgfpathlineto{\pgfqpoint{0.776142in}{1.611079in}}%
\pgfpathlineto{\pgfqpoint{0.770941in}{1.602285in}}%
\pgfpathlineto{\pgfqpoint{0.768183in}{1.597468in}}%
\pgfpathlineto{\pgfqpoint{0.759131in}{1.583857in}}%
\pgfpathlineto{\pgfqpoint{0.755284in}{1.578746in}}%
\pgfpathlineto{\pgfqpoint{0.749075in}{1.570246in}}%
\pgfpathlineto{\pgfqpoint{0.739628in}{1.558565in}}%
\pgfpathlineto{\pgfqpoint{0.738107in}{1.556635in}}%
\pgfpathlineto{\pgfqpoint{0.726397in}{1.543024in}}%
\pgfpathlineto{\pgfqpoint{0.723971in}{1.540407in}}%
\pgfpathlineto{\pgfqpoint{0.714016in}{1.529413in}}%
\pgfpathlineto{\pgfqpoint{0.708315in}{1.523498in}}%
\pgfpathlineto{\pgfqpoint{0.701043in}{1.515802in}}%
\pgfpathlineto{\pgfqpoint{0.692658in}{1.507385in}}%
\pgfpathlineto{\pgfqpoint{0.687568in}{1.502191in}}%
\pgfpathlineto{\pgfqpoint{0.677002in}{1.491881in}}%
\pgfpathlineto{\pgfqpoint{0.673662in}{1.488579in}}%
\pgfpathlineto{\pgfqpoint{0.661345in}{1.476852in}}%
\pgfpathlineto{\pgfqpoint{0.659385in}{1.474968in}}%
\pgfpathlineto{\pgfqpoint{0.645689in}{1.462206in}}%
\pgfpathlineto{\pgfqpoint{0.644783in}{1.461357in}}%
\pgfpathlineto{\pgfqpoint{0.630032in}{1.447874in}}%
\pgfpathlineto{\pgfqpoint{0.629892in}{1.447746in}}%
\pgfpathlineto{\pgfqpoint{0.614732in}{1.434135in}}%
\pgfpathlineto{\pgfqpoint{0.614375in}{1.433820in}}%
\pgfpathlineto{\pgfqpoint{0.599321in}{1.420524in}}%
\pgfpathlineto{\pgfqpoint{0.598719in}{1.419998in}}%
\pgfpathlineto{\pgfqpoint{0.583667in}{1.406913in}}%
\pgfpathlineto{\pgfqpoint{0.583062in}{1.406390in}}%
\pgfpathlineto{\pgfqpoint{0.567768in}{1.393302in}}%
\pgfpathlineto{\pgfqpoint{0.567406in}{1.392992in}}%
\pgfpathlineto{\pgfqpoint{0.551749in}{1.379812in}}%
\pgfpathlineto{\pgfqpoint{0.551602in}{1.379691in}}%
\pgfpathlineto{\pgfqpoint{0.536093in}{1.366866in}}%
\pgfpathlineto{\pgfqpoint{0.535117in}{1.366079in}}%
\pgfpathlineto{\pgfqpoint{0.520436in}{1.354172in}}%
\pgfpathlineto{\pgfqpoint{0.518269in}{1.352468in}}%
\pgfpathlineto{\pgfqpoint{0.504779in}{1.341760in}}%
\pgfpathlineto{\pgfqpoint{0.500982in}{1.338857in}}%
\pgfpathlineto{\pgfqpoint{0.489123in}{1.329671in}}%
\pgfpathlineto{\pgfqpoint{0.483148in}{1.325246in}}%
\pgfpathlineto{\pgfqpoint{0.473466in}{1.317956in}}%
\pgfpathlineto{\pgfqpoint{0.464613in}{1.311635in}}%
\pgfpathlineto{\pgfqpoint{0.457810in}{1.306679in}}%
\pgfpathlineto{\pgfqpoint{0.445163in}{1.298024in}}%
\pgfpathlineto{\pgfqpoint{0.442153in}{1.295915in}}%
\pgfpathlineto{\pgfqpoint{0.426497in}{1.285735in}}%
\pgfpathlineto{\pgfqpoint{0.424277in}{1.284413in}}%
\pgfpathlineto{\pgfqpoint{0.410840in}{1.276200in}}%
\pgfpathlineto{\pgfqpoint{0.401063in}{1.270802in}}%
\pgfpathlineto{\pgfqpoint{0.395183in}{1.267457in}}%
\pgfpathlineto{\pgfqpoint{0.379527in}{1.259588in}}%
\pgfpathlineto{\pgfqpoint{0.373987in}{1.257191in}}%
\pgfpathlineto{\pgfqpoint{0.363870in}{1.252669in}}%
\pgfpathlineto{\pgfqpoint{0.348214in}{1.246854in}}%
\pgfpathlineto{\pgfqpoint{0.337029in}{1.243579in}}%
\pgfpathlineto{\pgfqpoint{0.332557in}{1.242224in}}%
\pgfpathlineto{\pgfqpoint{0.316901in}{1.238806in}}%
\pgfpathlineto{\pgfqpoint{0.301244in}{1.236740in}}%
\pgfpathlineto{\pgfqpoint{0.285588in}{1.236049in}}%
\pgfpathlineto{\pgfqpoint{0.285588in}{1.229968in}}%
\pgfpathlineto{\pgfqpoint{0.285588in}{1.216357in}}%
\pgfpathlineto{\pgfqpoint{0.285588in}{1.202746in}}%
\pgfpathlineto{\pgfqpoint{0.285588in}{1.189135in}}%
\pgfpathlineto{\pgfqpoint{0.285588in}{1.175524in}}%
\pgfpathlineto{\pgfqpoint{0.285588in}{1.161913in}}%
\pgfpathlineto{\pgfqpoint{0.285588in}{1.148302in}}%
\pgfpathlineto{\pgfqpoint{0.285588in}{1.134691in}}%
\pgfpathlineto{\pgfqpoint{0.285588in}{1.121079in}}%
\pgfpathlineto{\pgfqpoint{0.285588in}{1.107468in}}%
\pgfpathlineto{\pgfqpoint{0.285588in}{1.093857in}}%
\pgfpathlineto{\pgfqpoint{0.285588in}{1.080246in}}%
\pgfpathlineto{\pgfqpoint{0.285588in}{1.066635in}}%
\pgfpathlineto{\pgfqpoint{0.285588in}{1.053024in}}%
\pgfpathlineto{\pgfqpoint{0.285588in}{1.039413in}}%
\pgfpathlineto{\pgfqpoint{0.285588in}{1.025802in}}%
\pgfpathlineto{\pgfqpoint{0.285588in}{1.012191in}}%
\pgfpathlineto{\pgfqpoint{0.285588in}{0.998579in}}%
\pgfpathlineto{\pgfqpoint{0.285588in}{0.984968in}}%
\pgfpathlineto{\pgfqpoint{0.285588in}{0.971357in}}%
\pgfpathlineto{\pgfqpoint{0.285588in}{0.957746in}}%
\pgfpathlineto{\pgfqpoint{0.285588in}{0.944135in}}%
\pgfpathlineto{\pgfqpoint{0.285588in}{0.930524in}}%
\pgfpathlineto{\pgfqpoint{0.285588in}{0.916913in}}%
\pgfpathlineto{\pgfqpoint{0.285588in}{0.903302in}}%
\pgfpathlineto{\pgfqpoint{0.285588in}{0.889691in}}%
\pgfpathlineto{\pgfqpoint{0.285588in}{0.876079in}}%
\pgfpathlineto{\pgfqpoint{0.285588in}{0.862468in}}%
\pgfpathlineto{\pgfqpoint{0.285588in}{0.848857in}}%
\pgfpathlineto{\pgfqpoint{0.285588in}{0.835246in}}%
\pgfpathlineto{\pgfqpoint{0.285588in}{0.821635in}}%
\pgfpathlineto{\pgfqpoint{0.285588in}{0.808024in}}%
\pgfpathlineto{\pgfqpoint{0.285588in}{0.794413in}}%
\pgfpathlineto{\pgfqpoint{0.285588in}{0.780802in}}%
\pgfpathlineto{\pgfqpoint{0.285588in}{0.774721in}}%
\pgfpathlineto{\pgfqpoint{0.301244in}{0.774030in}}%
\pgfpathlineto{\pgfqpoint{0.316901in}{0.771964in}}%
\pgfpathlineto{\pgfqpoint{0.332557in}{0.768546in}}%
\pgfpathlineto{\pgfqpoint{0.337029in}{0.767191in}}%
\pgfpathlineto{\pgfqpoint{0.348214in}{0.763916in}}%
\pgfpathlineto{\pgfqpoint{0.363870in}{0.758101in}}%
\pgfpathlineto{\pgfqpoint{0.373987in}{0.753579in}}%
\pgfpathlineto{\pgfqpoint{0.379527in}{0.751182in}}%
\pgfpathlineto{\pgfqpoint{0.395183in}{0.743313in}}%
\pgfpathlineto{\pgfqpoint{0.401063in}{0.739968in}}%
\pgfpathlineto{\pgfqpoint{0.410840in}{0.734570in}}%
\pgfpathlineto{\pgfqpoint{0.424277in}{0.726357in}}%
\pgfpathlineto{\pgfqpoint{0.426497in}{0.725035in}}%
\pgfpathlineto{\pgfqpoint{0.442153in}{0.714855in}}%
\pgfpathlineto{\pgfqpoint{0.445163in}{0.712746in}}%
\pgfpathlineto{\pgfqpoint{0.457810in}{0.704091in}}%
\pgfpathlineto{\pgfqpoint{0.464613in}{0.699135in}}%
\pgfpathlineto{\pgfqpoint{0.473466in}{0.692814in}}%
\pgfpathlineto{\pgfqpoint{0.483148in}{0.685524in}}%
\pgfpathlineto{\pgfqpoint{0.489123in}{0.681099in}}%
\pgfpathlineto{\pgfqpoint{0.500982in}{0.671913in}}%
\pgfpathlineto{\pgfqpoint{0.504779in}{0.669010in}}%
\pgfpathlineto{\pgfqpoint{0.518269in}{0.658302in}}%
\pgfpathlineto{\pgfqpoint{0.520436in}{0.656598in}}%
\pgfpathlineto{\pgfqpoint{0.535117in}{0.644691in}}%
\pgfpathlineto{\pgfqpoint{0.536093in}{0.643904in}}%
\pgfpathlineto{\pgfqpoint{0.551602in}{0.631079in}}%
\pgfpathlineto{\pgfqpoint{0.551749in}{0.630958in}}%
\pgfpathlineto{\pgfqpoint{0.567406in}{0.617778in}}%
\pgfpathlineto{\pgfqpoint{0.567768in}{0.617468in}}%
\pgfpathlineto{\pgfqpoint{0.583062in}{0.604380in}}%
\pgfpathlineto{\pgfqpoint{0.583667in}{0.603857in}}%
\pgfpathlineto{\pgfqpoint{0.598719in}{0.590772in}}%
\pgfpathlineto{\pgfqpoint{0.599321in}{0.590246in}}%
\pgfpathlineto{\pgfqpoint{0.614375in}{0.576950in}}%
\pgfpathlineto{\pgfqpoint{0.614732in}{0.576635in}}%
\pgfpathlineto{\pgfqpoint{0.629892in}{0.563024in}}%
\pgfpathlineto{\pgfqpoint{0.630032in}{0.562896in}}%
\pgfpathlineto{\pgfqpoint{0.644783in}{0.549413in}}%
\pgfpathlineto{\pgfqpoint{0.645689in}{0.548564in}}%
\pgfpathlineto{\pgfqpoint{0.659385in}{0.535802in}}%
\pgfpathlineto{\pgfqpoint{0.661345in}{0.533918in}}%
\pgfpathlineto{\pgfqpoint{0.673662in}{0.522191in}}%
\pgfpathlineto{\pgfqpoint{0.677002in}{0.518889in}}%
\pgfpathlineto{\pgfqpoint{0.687568in}{0.508579in}}%
\pgfpathlineto{\pgfqpoint{0.692658in}{0.503385in}}%
\pgfpathlineto{\pgfqpoint{0.701043in}{0.494968in}}%
\pgfpathlineto{\pgfqpoint{0.708315in}{0.487272in}}%
\pgfpathlineto{\pgfqpoint{0.714016in}{0.481357in}}%
\pgfpathlineto{\pgfqpoint{0.723971in}{0.470363in}}%
\pgfpathlineto{\pgfqpoint{0.726397in}{0.467746in}}%
\pgfpathlineto{\pgfqpoint{0.738107in}{0.454135in}}%
\pgfpathlineto{\pgfqpoint{0.739628in}{0.452205in}}%
\pgfpathlineto{\pgfqpoint{0.749075in}{0.440524in}}%
\pgfpathlineto{\pgfqpoint{0.755284in}{0.432024in}}%
\pgfpathlineto{\pgfqpoint{0.759131in}{0.426913in}}%
\pgfpathlineto{\pgfqpoint{0.768183in}{0.413302in}}%
\pgfpathlineto{\pgfqpoint{0.770941in}{0.408485in}}%
\pgfpathlineto{\pgfqpoint{0.776142in}{0.399691in}}%
\pgfpathlineto{\pgfqpoint{0.782831in}{0.386079in}}%
\pgfpathlineto{\pgfqpoint{0.786598in}{0.376356in}}%
\pgfpathlineto{\pgfqpoint{0.788157in}{0.372468in}}%
\pgfpathlineto{\pgfqpoint{0.792089in}{0.358857in}}%
\pgfpathlineto{\pgfqpoint{0.794465in}{0.345246in}}%
\pgfpathlineto{\pgfqpoint{0.795259in}{0.331635in}}%
\pgfpathlineto{\pgfqpoint{0.802254in}{0.331635in}}%
\pgfpathclose%
\pgfpathmoveto{\pgfqpoint{1.034787in}{0.467746in}}%
\pgfpathlineto{\pgfqpoint{1.021446in}{0.469358in}}%
\pgfpathlineto{\pgfqpoint{1.005790in}{0.472197in}}%
\pgfpathlineto{\pgfqpoint{0.990133in}{0.475983in}}%
\pgfpathlineto{\pgfqpoint{0.974476in}{0.480717in}}%
\pgfpathlineto{\pgfqpoint{0.972706in}{0.481357in}}%
\pgfpathlineto{\pgfqpoint{0.958820in}{0.486032in}}%
\pgfpathlineto{\pgfqpoint{0.943163in}{0.492181in}}%
\pgfpathlineto{\pgfqpoint{0.936936in}{0.494968in}}%
\pgfpathlineto{\pgfqpoint{0.927507in}{0.498938in}}%
\pgfpathlineto{\pgfqpoint{0.911850in}{0.506340in}}%
\pgfpathlineto{\pgfqpoint{0.907567in}{0.508579in}}%
\pgfpathlineto{\pgfqpoint{0.896194in}{0.514225in}}%
\pgfpathlineto{\pgfqpoint{0.881572in}{0.522191in}}%
\pgfpathlineto{\pgfqpoint{0.880537in}{0.522730in}}%
\pgfpathlineto{\pgfqpoint{0.864880in}{0.531569in}}%
\pgfpathlineto{\pgfqpoint{0.857924in}{0.535802in}}%
\pgfpathlineto{\pgfqpoint{0.849224in}{0.540905in}}%
\pgfpathlineto{\pgfqpoint{0.835676in}{0.549413in}}%
\pgfpathlineto{\pgfqpoint{0.833567in}{0.550698in}}%
\pgfpathlineto{\pgfqpoint{0.817911in}{0.560825in}}%
\pgfpathlineto{\pgfqpoint{0.814692in}{0.563024in}}%
\pgfpathlineto{\pgfqpoint{0.802254in}{0.571321in}}%
\pgfpathlineto{\pgfqpoint{0.794680in}{0.576635in}}%
\pgfpathlineto{\pgfqpoint{0.786598in}{0.582205in}}%
\pgfpathlineto{\pgfqpoint{0.775446in}{0.590246in}}%
\pgfpathlineto{\pgfqpoint{0.770941in}{0.593454in}}%
\pgfpathlineto{\pgfqpoint{0.756911in}{0.603857in}}%
\pgfpathlineto{\pgfqpoint{0.755284in}{0.605054in}}%
\pgfpathlineto{\pgfqpoint{0.739628in}{0.616991in}}%
\pgfpathlineto{\pgfqpoint{0.739021in}{0.617468in}}%
\pgfpathlineto{\pgfqpoint{0.723971in}{0.629266in}}%
\pgfpathlineto{\pgfqpoint{0.721720in}{0.631079in}}%
\pgfpathlineto{\pgfqpoint{0.708315in}{0.641901in}}%
\pgfpathlineto{\pgfqpoint{0.704939in}{0.644691in}}%
\pgfpathlineto{\pgfqpoint{0.692658in}{0.654901in}}%
\pgfpathlineto{\pgfqpoint{0.688646in}{0.658302in}}%
\pgfpathlineto{\pgfqpoint{0.677002in}{0.668275in}}%
\pgfpathlineto{\pgfqpoint{0.672818in}{0.671913in}}%
\pgfpathlineto{\pgfqpoint{0.661345in}{0.682036in}}%
\pgfpathlineto{\pgfqpoint{0.657434in}{0.685524in}}%
\pgfpathlineto{\pgfqpoint{0.645689in}{0.696200in}}%
\pgfpathlineto{\pgfqpoint{0.642480in}{0.699135in}}%
\pgfpathlineto{\pgfqpoint{0.630032in}{0.710789in}}%
\pgfpathlineto{\pgfqpoint{0.627946in}{0.712746in}}%
\pgfpathlineto{\pgfqpoint{0.614375in}{0.725829in}}%
\pgfpathlineto{\pgfqpoint{0.613826in}{0.726357in}}%
\pgfpathlineto{\pgfqpoint{0.600095in}{0.739968in}}%
\pgfpathlineto{\pgfqpoint{0.598719in}{0.741382in}}%
\pgfpathlineto{\pgfqpoint{0.586752in}{0.753579in}}%
\pgfpathlineto{\pgfqpoint{0.583062in}{0.757496in}}%
\pgfpathlineto{\pgfqpoint{0.573812in}{0.767191in}}%
\pgfpathlineto{\pgfqpoint{0.567406in}{0.774217in}}%
\pgfpathlineto{\pgfqpoint{0.561293in}{0.780802in}}%
\pgfpathlineto{\pgfqpoint{0.551749in}{0.791614in}}%
\pgfpathlineto{\pgfqpoint{0.549220in}{0.794413in}}%
\pgfpathlineto{\pgfqpoint{0.537571in}{0.808024in}}%
\pgfpathlineto{\pgfqpoint{0.536093in}{0.809857in}}%
\pgfpathlineto{\pgfqpoint{0.526306in}{0.821635in}}%
\pgfpathlineto{\pgfqpoint{0.520436in}{0.829198in}}%
\pgfpathlineto{\pgfqpoint{0.515568in}{0.835246in}}%
\pgfpathlineto{\pgfqpoint{0.505400in}{0.848857in}}%
\pgfpathlineto{\pgfqpoint{0.504779in}{0.849757in}}%
\pgfpathlineto{\pgfqpoint{0.495617in}{0.862468in}}%
\pgfpathlineto{\pgfqpoint{0.489123in}{0.872356in}}%
\pgfpathlineto{\pgfqpoint{0.486547in}{0.876079in}}%
\pgfpathlineto{\pgfqpoint{0.478033in}{0.889691in}}%
\pgfpathlineto{\pgfqpoint{0.473466in}{0.897888in}}%
\pgfpathlineto{\pgfqpoint{0.470261in}{0.903302in}}%
\pgfpathlineto{\pgfqpoint{0.463187in}{0.916913in}}%
\pgfpathlineto{\pgfqpoint{0.457810in}{0.928984in}}%
\pgfpathlineto{\pgfqpoint{0.457073in}{0.930524in}}%
\pgfpathlineto{\pgfqpoint{0.451627in}{0.944135in}}%
\pgfpathlineto{\pgfqpoint{0.447273in}{0.957746in}}%
\pgfpathlineto{\pgfqpoint{0.444008in}{0.971357in}}%
\pgfpathlineto{\pgfqpoint{0.442153in}{0.982955in}}%
\pgfpathlineto{\pgfqpoint{0.441803in}{0.984968in}}%
\pgfpathlineto{\pgfqpoint{0.440617in}{0.998579in}}%
\pgfpathlineto{\pgfqpoint{0.440617in}{1.012191in}}%
\pgfpathlineto{\pgfqpoint{0.441803in}{1.025802in}}%
\pgfpathlineto{\pgfqpoint{0.442153in}{1.027815in}}%
\pgfpathlineto{\pgfqpoint{0.444008in}{1.039413in}}%
\pgfpathlineto{\pgfqpoint{0.447273in}{1.053024in}}%
\pgfpathlineto{\pgfqpoint{0.451627in}{1.066635in}}%
\pgfpathlineto{\pgfqpoint{0.457073in}{1.080246in}}%
\pgfpathlineto{\pgfqpoint{0.457810in}{1.081786in}}%
\pgfpathlineto{\pgfqpoint{0.463187in}{1.093857in}}%
\pgfpathlineto{\pgfqpoint{0.470261in}{1.107468in}}%
\pgfpathlineto{\pgfqpoint{0.473466in}{1.112882in}}%
\pgfpathlineto{\pgfqpoint{0.478033in}{1.121079in}}%
\pgfpathlineto{\pgfqpoint{0.486547in}{1.134691in}}%
\pgfpathlineto{\pgfqpoint{0.489123in}{1.138414in}}%
\pgfpathlineto{\pgfqpoint{0.495617in}{1.148302in}}%
\pgfpathlineto{\pgfqpoint{0.504779in}{1.161013in}}%
\pgfpathlineto{\pgfqpoint{0.505400in}{1.161913in}}%
\pgfpathlineto{\pgfqpoint{0.515568in}{1.175524in}}%
\pgfpathlineto{\pgfqpoint{0.520436in}{1.181572in}}%
\pgfpathlineto{\pgfqpoint{0.526306in}{1.189135in}}%
\pgfpathlineto{\pgfqpoint{0.536093in}{1.200913in}}%
\pgfpathlineto{\pgfqpoint{0.537571in}{1.202746in}}%
\pgfpathlineto{\pgfqpoint{0.549220in}{1.216357in}}%
\pgfpathlineto{\pgfqpoint{0.551749in}{1.219156in}}%
\pgfpathlineto{\pgfqpoint{0.561293in}{1.229968in}}%
\pgfpathlineto{\pgfqpoint{0.567406in}{1.236553in}}%
\pgfpathlineto{\pgfqpoint{0.573812in}{1.243579in}}%
\pgfpathlineto{\pgfqpoint{0.583062in}{1.253274in}}%
\pgfpathlineto{\pgfqpoint{0.586752in}{1.257191in}}%
\pgfpathlineto{\pgfqpoint{0.598719in}{1.269388in}}%
\pgfpathlineto{\pgfqpoint{0.600095in}{1.270802in}}%
\pgfpathlineto{\pgfqpoint{0.613826in}{1.284413in}}%
\pgfpathlineto{\pgfqpoint{0.614375in}{1.284941in}}%
\pgfpathlineto{\pgfqpoint{0.627946in}{1.298024in}}%
\pgfpathlineto{\pgfqpoint{0.630032in}{1.299981in}}%
\pgfpathlineto{\pgfqpoint{0.642480in}{1.311635in}}%
\pgfpathlineto{\pgfqpoint{0.645689in}{1.314570in}}%
\pgfpathlineto{\pgfqpoint{0.657434in}{1.325246in}}%
\pgfpathlineto{\pgfqpoint{0.661345in}{1.328734in}}%
\pgfpathlineto{\pgfqpoint{0.672818in}{1.338857in}}%
\pgfpathlineto{\pgfqpoint{0.677002in}{1.342495in}}%
\pgfpathlineto{\pgfqpoint{0.688646in}{1.352468in}}%
\pgfpathlineto{\pgfqpoint{0.692658in}{1.355869in}}%
\pgfpathlineto{\pgfqpoint{0.704939in}{1.366079in}}%
\pgfpathlineto{\pgfqpoint{0.708315in}{1.368869in}}%
\pgfpathlineto{\pgfqpoint{0.721720in}{1.379691in}}%
\pgfpathlineto{\pgfqpoint{0.723971in}{1.381504in}}%
\pgfpathlineto{\pgfqpoint{0.739021in}{1.393302in}}%
\pgfpathlineto{\pgfqpoint{0.739628in}{1.393779in}}%
\pgfpathlineto{\pgfqpoint{0.755284in}{1.405716in}}%
\pgfpathlineto{\pgfqpoint{0.756911in}{1.406913in}}%
\pgfpathlineto{\pgfqpoint{0.770941in}{1.417316in}}%
\pgfpathlineto{\pgfqpoint{0.775446in}{1.420524in}}%
\pgfpathlineto{\pgfqpoint{0.786598in}{1.428565in}}%
\pgfpathlineto{\pgfqpoint{0.794680in}{1.434135in}}%
\pgfpathlineto{\pgfqpoint{0.802254in}{1.439449in}}%
\pgfpathlineto{\pgfqpoint{0.814692in}{1.447746in}}%
\pgfpathlineto{\pgfqpoint{0.817911in}{1.449945in}}%
\pgfpathlineto{\pgfqpoint{0.833567in}{1.460072in}}%
\pgfpathlineto{\pgfqpoint{0.835676in}{1.461357in}}%
\pgfpathlineto{\pgfqpoint{0.849224in}{1.469865in}}%
\pgfpathlineto{\pgfqpoint{0.857924in}{1.474968in}}%
\pgfpathlineto{\pgfqpoint{0.864880in}{1.479201in}}%
\pgfpathlineto{\pgfqpoint{0.880537in}{1.488040in}}%
\pgfpathlineto{\pgfqpoint{0.881572in}{1.488579in}}%
\pgfpathlineto{\pgfqpoint{0.896194in}{1.496545in}}%
\pgfpathlineto{\pgfqpoint{0.907567in}{1.502191in}}%
\pgfpathlineto{\pgfqpoint{0.911850in}{1.504430in}}%
\pgfpathlineto{\pgfqpoint{0.927507in}{1.511832in}}%
\pgfpathlineto{\pgfqpoint{0.936936in}{1.515802in}}%
\pgfpathlineto{\pgfqpoint{0.943163in}{1.518589in}}%
\pgfpathlineto{\pgfqpoint{0.958820in}{1.524738in}}%
\pgfpathlineto{\pgfqpoint{0.972706in}{1.529413in}}%
\pgfpathlineto{\pgfqpoint{0.974476in}{1.530053in}}%
\pgfpathlineto{\pgfqpoint{0.990133in}{1.534787in}}%
\pgfpathlineto{\pgfqpoint{1.005790in}{1.538573in}}%
\pgfpathlineto{\pgfqpoint{1.021446in}{1.541412in}}%
\pgfpathlineto{\pgfqpoint{1.034787in}{1.543024in}}%
\pgfpathlineto{\pgfqpoint{1.037103in}{1.543329in}}%
\pgfpathlineto{\pgfqpoint{1.052759in}{1.544360in}}%
\pgfpathlineto{\pgfqpoint{1.068416in}{1.544360in}}%
\pgfpathlineto{\pgfqpoint{1.084072in}{1.543329in}}%
\pgfpathlineto{\pgfqpoint{1.086388in}{1.543024in}}%
\pgfpathlineto{\pgfqpoint{1.099729in}{1.541412in}}%
\pgfpathlineto{\pgfqpoint{1.115385in}{1.538573in}}%
\pgfpathlineto{\pgfqpoint{1.131042in}{1.534787in}}%
\pgfpathlineto{\pgfqpoint{1.146699in}{1.530053in}}%
\pgfpathlineto{\pgfqpoint{1.148469in}{1.529413in}}%
\pgfpathlineto{\pgfqpoint{1.162355in}{1.524738in}}%
\pgfpathlineto{\pgfqpoint{1.178012in}{1.518589in}}%
\pgfpathlineto{\pgfqpoint{1.184239in}{1.515802in}}%
\pgfpathlineto{\pgfqpoint{1.193668in}{1.511832in}}%
\pgfpathlineto{\pgfqpoint{1.209325in}{1.504430in}}%
\pgfpathlineto{\pgfqpoint{1.213608in}{1.502191in}}%
\pgfpathlineto{\pgfqpoint{1.224981in}{1.496545in}}%
\pgfpathlineto{\pgfqpoint{1.239603in}{1.488579in}}%
\pgfpathlineto{\pgfqpoint{1.240638in}{1.488040in}}%
\pgfpathlineto{\pgfqpoint{1.256295in}{1.479201in}}%
\pgfpathlineto{\pgfqpoint{1.263251in}{1.474968in}}%
\pgfpathlineto{\pgfqpoint{1.271951in}{1.469865in}}%
\pgfpathlineto{\pgfqpoint{1.285499in}{1.461357in}}%
\pgfpathlineto{\pgfqpoint{1.287608in}{1.460072in}}%
\pgfpathlineto{\pgfqpoint{1.303264in}{1.449945in}}%
\pgfpathlineto{\pgfqpoint{1.306483in}{1.447746in}}%
\pgfpathlineto{\pgfqpoint{1.318921in}{1.439449in}}%
\pgfpathlineto{\pgfqpoint{1.326495in}{1.434135in}}%
\pgfpathlineto{\pgfqpoint{1.334577in}{1.428565in}}%
\pgfpathlineto{\pgfqpoint{1.345729in}{1.420524in}}%
\pgfpathlineto{\pgfqpoint{1.350234in}{1.417316in}}%
\pgfpathlineto{\pgfqpoint{1.364264in}{1.406913in}}%
\pgfpathlineto{\pgfqpoint{1.365891in}{1.405716in}}%
\pgfpathlineto{\pgfqpoint{1.381547in}{1.393779in}}%
\pgfpathlineto{\pgfqpoint{1.382154in}{1.393302in}}%
\pgfpathlineto{\pgfqpoint{1.397204in}{1.381504in}}%
\pgfpathlineto{\pgfqpoint{1.399455in}{1.379691in}}%
\pgfpathlineto{\pgfqpoint{1.412860in}{1.368869in}}%
\pgfpathlineto{\pgfqpoint{1.416236in}{1.366079in}}%
\pgfpathlineto{\pgfqpoint{1.428517in}{1.355869in}}%
\pgfpathlineto{\pgfqpoint{1.432529in}{1.352468in}}%
\pgfpathlineto{\pgfqpoint{1.444173in}{1.342495in}}%
\pgfpathlineto{\pgfqpoint{1.448357in}{1.338857in}}%
\pgfpathlineto{\pgfqpoint{1.459830in}{1.328734in}}%
\pgfpathlineto{\pgfqpoint{1.463741in}{1.325246in}}%
\pgfpathlineto{\pgfqpoint{1.475486in}{1.314570in}}%
\pgfpathlineto{\pgfqpoint{1.478695in}{1.311635in}}%
\pgfpathlineto{\pgfqpoint{1.491143in}{1.299981in}}%
\pgfpathlineto{\pgfqpoint{1.493229in}{1.298024in}}%
\pgfpathlineto{\pgfqpoint{1.506800in}{1.284941in}}%
\pgfpathlineto{\pgfqpoint{1.507349in}{1.284413in}}%
\pgfpathlineto{\pgfqpoint{1.521080in}{1.270802in}}%
\pgfpathlineto{\pgfqpoint{1.522456in}{1.269388in}}%
\pgfpathlineto{\pgfqpoint{1.534423in}{1.257191in}}%
\pgfpathlineto{\pgfqpoint{1.538113in}{1.253274in}}%
\pgfpathlineto{\pgfqpoint{1.547363in}{1.243579in}}%
\pgfpathlineto{\pgfqpoint{1.553769in}{1.236553in}}%
\pgfpathlineto{\pgfqpoint{1.559882in}{1.229968in}}%
\pgfpathlineto{\pgfqpoint{1.569426in}{1.219156in}}%
\pgfpathlineto{\pgfqpoint{1.571955in}{1.216357in}}%
\pgfpathlineto{\pgfqpoint{1.583604in}{1.202746in}}%
\pgfpathlineto{\pgfqpoint{1.585082in}{1.200913in}}%
\pgfpathlineto{\pgfqpoint{1.594869in}{1.189135in}}%
\pgfpathlineto{\pgfqpoint{1.600739in}{1.181572in}}%
\pgfpathlineto{\pgfqpoint{1.605607in}{1.175524in}}%
\pgfpathlineto{\pgfqpoint{1.615775in}{1.161913in}}%
\pgfpathlineto{\pgfqpoint{1.616396in}{1.161013in}}%
\pgfpathlineto{\pgfqpoint{1.625558in}{1.148302in}}%
\pgfpathlineto{\pgfqpoint{1.632052in}{1.138414in}}%
\pgfpathlineto{\pgfqpoint{1.634628in}{1.134691in}}%
\pgfpathlineto{\pgfqpoint{1.643142in}{1.121079in}}%
\pgfpathlineto{\pgfqpoint{1.647709in}{1.112882in}}%
\pgfpathlineto{\pgfqpoint{1.650914in}{1.107468in}}%
\pgfpathlineto{\pgfqpoint{1.657988in}{1.093857in}}%
\pgfpathlineto{\pgfqpoint{1.663365in}{1.081786in}}%
\pgfpathlineto{\pgfqpoint{1.664102in}{1.080246in}}%
\pgfpathlineto{\pgfqpoint{1.669548in}{1.066635in}}%
\pgfpathlineto{\pgfqpoint{1.673902in}{1.053024in}}%
\pgfpathlineto{\pgfqpoint{1.677167in}{1.039413in}}%
\pgfpathlineto{\pgfqpoint{1.679022in}{1.027815in}}%
\pgfpathlineto{\pgfqpoint{1.679372in}{1.025802in}}%
\pgfpathlineto{\pgfqpoint{1.680558in}{1.012191in}}%
\pgfpathlineto{\pgfqpoint{1.680558in}{0.998579in}}%
\pgfpathlineto{\pgfqpoint{1.679372in}{0.984968in}}%
\pgfpathlineto{\pgfqpoint{1.679022in}{0.982955in}}%
\pgfpathlineto{\pgfqpoint{1.677167in}{0.971357in}}%
\pgfpathlineto{\pgfqpoint{1.673902in}{0.957746in}}%
\pgfpathlineto{\pgfqpoint{1.669548in}{0.944135in}}%
\pgfpathlineto{\pgfqpoint{1.664102in}{0.930524in}}%
\pgfpathlineto{\pgfqpoint{1.663365in}{0.928984in}}%
\pgfpathlineto{\pgfqpoint{1.657988in}{0.916913in}}%
\pgfpathlineto{\pgfqpoint{1.650914in}{0.903302in}}%
\pgfpathlineto{\pgfqpoint{1.647709in}{0.897888in}}%
\pgfpathlineto{\pgfqpoint{1.643142in}{0.889691in}}%
\pgfpathlineto{\pgfqpoint{1.634628in}{0.876079in}}%
\pgfpathlineto{\pgfqpoint{1.632052in}{0.872356in}}%
\pgfpathlineto{\pgfqpoint{1.625558in}{0.862468in}}%
\pgfpathlineto{\pgfqpoint{1.616396in}{0.849757in}}%
\pgfpathlineto{\pgfqpoint{1.615775in}{0.848857in}}%
\pgfpathlineto{\pgfqpoint{1.605607in}{0.835246in}}%
\pgfpathlineto{\pgfqpoint{1.600739in}{0.829198in}}%
\pgfpathlineto{\pgfqpoint{1.594869in}{0.821635in}}%
\pgfpathlineto{\pgfqpoint{1.585082in}{0.809857in}}%
\pgfpathlineto{\pgfqpoint{1.583604in}{0.808024in}}%
\pgfpathlineto{\pgfqpoint{1.571955in}{0.794413in}}%
\pgfpathlineto{\pgfqpoint{1.569426in}{0.791614in}}%
\pgfpathlineto{\pgfqpoint{1.559882in}{0.780802in}}%
\pgfpathlineto{\pgfqpoint{1.553769in}{0.774217in}}%
\pgfpathlineto{\pgfqpoint{1.547363in}{0.767191in}}%
\pgfpathlineto{\pgfqpoint{1.538113in}{0.757496in}}%
\pgfpathlineto{\pgfqpoint{1.534423in}{0.753579in}}%
\pgfpathlineto{\pgfqpoint{1.522456in}{0.741382in}}%
\pgfpathlineto{\pgfqpoint{1.521080in}{0.739968in}}%
\pgfpathlineto{\pgfqpoint{1.507349in}{0.726357in}}%
\pgfpathlineto{\pgfqpoint{1.506800in}{0.725829in}}%
\pgfpathlineto{\pgfqpoint{1.493229in}{0.712746in}}%
\pgfpathlineto{\pgfqpoint{1.491143in}{0.710789in}}%
\pgfpathlineto{\pgfqpoint{1.478695in}{0.699135in}}%
\pgfpathlineto{\pgfqpoint{1.475486in}{0.696200in}}%
\pgfpathlineto{\pgfqpoint{1.463741in}{0.685524in}}%
\pgfpathlineto{\pgfqpoint{1.459830in}{0.682036in}}%
\pgfpathlineto{\pgfqpoint{1.448357in}{0.671913in}}%
\pgfpathlineto{\pgfqpoint{1.444173in}{0.668275in}}%
\pgfpathlineto{\pgfqpoint{1.432529in}{0.658302in}}%
\pgfpathlineto{\pgfqpoint{1.428517in}{0.654901in}}%
\pgfpathlineto{\pgfqpoint{1.416236in}{0.644691in}}%
\pgfpathlineto{\pgfqpoint{1.412860in}{0.641901in}}%
\pgfpathlineto{\pgfqpoint{1.399455in}{0.631079in}}%
\pgfpathlineto{\pgfqpoint{1.397204in}{0.629266in}}%
\pgfpathlineto{\pgfqpoint{1.382154in}{0.617468in}}%
\pgfpathlineto{\pgfqpoint{1.381547in}{0.616991in}}%
\pgfpathlineto{\pgfqpoint{1.365891in}{0.605054in}}%
\pgfpathlineto{\pgfqpoint{1.364264in}{0.603857in}}%
\pgfpathlineto{\pgfqpoint{1.350234in}{0.593454in}}%
\pgfpathlineto{\pgfqpoint{1.345729in}{0.590246in}}%
\pgfpathlineto{\pgfqpoint{1.334577in}{0.582205in}}%
\pgfpathlineto{\pgfqpoint{1.326495in}{0.576635in}}%
\pgfpathlineto{\pgfqpoint{1.318921in}{0.571321in}}%
\pgfpathlineto{\pgfqpoint{1.306483in}{0.563024in}}%
\pgfpathlineto{\pgfqpoint{1.303264in}{0.560825in}}%
\pgfpathlineto{\pgfqpoint{1.287608in}{0.550698in}}%
\pgfpathlineto{\pgfqpoint{1.285499in}{0.549413in}}%
\pgfpathlineto{\pgfqpoint{1.271951in}{0.540905in}}%
\pgfpathlineto{\pgfqpoint{1.263251in}{0.535802in}}%
\pgfpathlineto{\pgfqpoint{1.256295in}{0.531569in}}%
\pgfpathlineto{\pgfqpoint{1.240638in}{0.522730in}}%
\pgfpathlineto{\pgfqpoint{1.239603in}{0.522191in}}%
\pgfpathlineto{\pgfqpoint{1.224981in}{0.514225in}}%
\pgfpathlineto{\pgfqpoint{1.213608in}{0.508579in}}%
\pgfpathlineto{\pgfqpoint{1.209325in}{0.506340in}}%
\pgfpathlineto{\pgfqpoint{1.193668in}{0.498938in}}%
\pgfpathlineto{\pgfqpoint{1.184239in}{0.494968in}}%
\pgfpathlineto{\pgfqpoint{1.178012in}{0.492181in}}%
\pgfpathlineto{\pgfqpoint{1.162355in}{0.486032in}}%
\pgfpathlineto{\pgfqpoint{1.148469in}{0.481357in}}%
\pgfpathlineto{\pgfqpoint{1.146699in}{0.480717in}}%
\pgfpathlineto{\pgfqpoint{1.131042in}{0.475983in}}%
\pgfpathlineto{\pgfqpoint{1.115385in}{0.472197in}}%
\pgfpathlineto{\pgfqpoint{1.099729in}{0.469358in}}%
\pgfpathlineto{\pgfqpoint{1.086388in}{0.467746in}}%
\pgfpathlineto{\pgfqpoint{1.084072in}{0.467441in}}%
\pgfpathlineto{\pgfqpoint{1.068416in}{0.466410in}}%
\pgfpathlineto{\pgfqpoint{1.052759in}{0.466410in}}%
\pgfpathlineto{\pgfqpoint{1.037103in}{0.467441in}}%
\pgfpathlineto{\pgfqpoint{1.034787in}{0.467746in}}%
\pgfpathclose%
\pgfusepath{fill}%
\end{pgfscope}%
\begin{pgfscope}%
\pgfpathrectangle{\pgfqpoint{0.285588in}{0.331635in}}{\pgfqpoint{1.550000in}{1.347500in}}%
\pgfusepath{clip}%
\pgfsetbuttcap%
\pgfsetroundjoin%
\definecolor{currentfill}{rgb}{0.252220,0.059415,0.453248}%
\pgfsetfillcolor{currentfill}%
\pgfsetlinewidth{0.000000pt}%
\definecolor{currentstroke}{rgb}{0.000000,0.000000,0.000000}%
\pgfsetstrokecolor{currentstroke}%
\pgfsetdash{}{0pt}%
\pgfpathmoveto{\pgfqpoint{0.598719in}{0.331635in}}%
\pgfpathlineto{\pgfqpoint{0.614375in}{0.331635in}}%
\pgfpathlineto{\pgfqpoint{0.630032in}{0.331635in}}%
\pgfpathlineto{\pgfqpoint{0.645689in}{0.331635in}}%
\pgfpathlineto{\pgfqpoint{0.661345in}{0.331635in}}%
\pgfpathlineto{\pgfqpoint{0.677002in}{0.331635in}}%
\pgfpathlineto{\pgfqpoint{0.692658in}{0.331635in}}%
\pgfpathlineto{\pgfqpoint{0.708315in}{0.331635in}}%
\pgfpathlineto{\pgfqpoint{0.723971in}{0.331635in}}%
\pgfpathlineto{\pgfqpoint{0.739628in}{0.331635in}}%
\pgfpathlineto{\pgfqpoint{0.755284in}{0.331635in}}%
\pgfpathlineto{\pgfqpoint{0.770941in}{0.331635in}}%
\pgfpathlineto{\pgfqpoint{0.786598in}{0.331635in}}%
\pgfpathlineto{\pgfqpoint{0.795259in}{0.331635in}}%
\pgfpathlineto{\pgfqpoint{0.794465in}{0.345246in}}%
\pgfpathlineto{\pgfqpoint{0.792089in}{0.358857in}}%
\pgfpathlineto{\pgfqpoint{0.788157in}{0.372468in}}%
\pgfpathlineto{\pgfqpoint{0.786598in}{0.376356in}}%
\pgfpathlineto{\pgfqpoint{0.782831in}{0.386079in}}%
\pgfpathlineto{\pgfqpoint{0.776142in}{0.399691in}}%
\pgfpathlineto{\pgfqpoint{0.770941in}{0.408485in}}%
\pgfpathlineto{\pgfqpoint{0.768183in}{0.413302in}}%
\pgfpathlineto{\pgfqpoint{0.759131in}{0.426913in}}%
\pgfpathlineto{\pgfqpoint{0.755284in}{0.432024in}}%
\pgfpathlineto{\pgfqpoint{0.749075in}{0.440524in}}%
\pgfpathlineto{\pgfqpoint{0.739628in}{0.452205in}}%
\pgfpathlineto{\pgfqpoint{0.738107in}{0.454135in}}%
\pgfpathlineto{\pgfqpoint{0.726397in}{0.467746in}}%
\pgfpathlineto{\pgfqpoint{0.723971in}{0.470363in}}%
\pgfpathlineto{\pgfqpoint{0.714016in}{0.481357in}}%
\pgfpathlineto{\pgfqpoint{0.708315in}{0.487272in}}%
\pgfpathlineto{\pgfqpoint{0.701043in}{0.494968in}}%
\pgfpathlineto{\pgfqpoint{0.692658in}{0.503385in}}%
\pgfpathlineto{\pgfqpoint{0.687568in}{0.508579in}}%
\pgfpathlineto{\pgfqpoint{0.677002in}{0.518889in}}%
\pgfpathlineto{\pgfqpoint{0.673662in}{0.522191in}}%
\pgfpathlineto{\pgfqpoint{0.661345in}{0.533918in}}%
\pgfpathlineto{\pgfqpoint{0.659385in}{0.535802in}}%
\pgfpathlineto{\pgfqpoint{0.645689in}{0.548564in}}%
\pgfpathlineto{\pgfqpoint{0.644783in}{0.549413in}}%
\pgfpathlineto{\pgfqpoint{0.630032in}{0.562896in}}%
\pgfpathlineto{\pgfqpoint{0.629892in}{0.563024in}}%
\pgfpathlineto{\pgfqpoint{0.614732in}{0.576635in}}%
\pgfpathlineto{\pgfqpoint{0.614375in}{0.576950in}}%
\pgfpathlineto{\pgfqpoint{0.599321in}{0.590246in}}%
\pgfpathlineto{\pgfqpoint{0.598719in}{0.590772in}}%
\pgfpathlineto{\pgfqpoint{0.583667in}{0.603857in}}%
\pgfpathlineto{\pgfqpoint{0.583062in}{0.604380in}}%
\pgfpathlineto{\pgfqpoint{0.567768in}{0.617468in}}%
\pgfpathlineto{\pgfqpoint{0.567406in}{0.617778in}}%
\pgfpathlineto{\pgfqpoint{0.551749in}{0.630958in}}%
\pgfpathlineto{\pgfqpoint{0.551602in}{0.631079in}}%
\pgfpathlineto{\pgfqpoint{0.536093in}{0.643904in}}%
\pgfpathlineto{\pgfqpoint{0.535117in}{0.644691in}}%
\pgfpathlineto{\pgfqpoint{0.520436in}{0.656598in}}%
\pgfpathlineto{\pgfqpoint{0.518269in}{0.658302in}}%
\pgfpathlineto{\pgfqpoint{0.504779in}{0.669010in}}%
\pgfpathlineto{\pgfqpoint{0.500982in}{0.671913in}}%
\pgfpathlineto{\pgfqpoint{0.489123in}{0.681099in}}%
\pgfpathlineto{\pgfqpoint{0.483148in}{0.685524in}}%
\pgfpathlineto{\pgfqpoint{0.473466in}{0.692814in}}%
\pgfpathlineto{\pgfqpoint{0.464613in}{0.699135in}}%
\pgfpathlineto{\pgfqpoint{0.457810in}{0.704091in}}%
\pgfpathlineto{\pgfqpoint{0.445163in}{0.712746in}}%
\pgfpathlineto{\pgfqpoint{0.442153in}{0.714855in}}%
\pgfpathlineto{\pgfqpoint{0.426497in}{0.725035in}}%
\pgfpathlineto{\pgfqpoint{0.424277in}{0.726357in}}%
\pgfpathlineto{\pgfqpoint{0.410840in}{0.734570in}}%
\pgfpathlineto{\pgfqpoint{0.401063in}{0.739968in}}%
\pgfpathlineto{\pgfqpoint{0.395183in}{0.743313in}}%
\pgfpathlineto{\pgfqpoint{0.379527in}{0.751182in}}%
\pgfpathlineto{\pgfqpoint{0.373987in}{0.753579in}}%
\pgfpathlineto{\pgfqpoint{0.363870in}{0.758101in}}%
\pgfpathlineto{\pgfqpoint{0.348214in}{0.763916in}}%
\pgfpathlineto{\pgfqpoint{0.337029in}{0.767191in}}%
\pgfpathlineto{\pgfqpoint{0.332557in}{0.768546in}}%
\pgfpathlineto{\pgfqpoint{0.316901in}{0.771964in}}%
\pgfpathlineto{\pgfqpoint{0.301244in}{0.774030in}}%
\pgfpathlineto{\pgfqpoint{0.285588in}{0.774721in}}%
\pgfpathlineto{\pgfqpoint{0.285588in}{0.767191in}}%
\pgfpathlineto{\pgfqpoint{0.285588in}{0.753579in}}%
\pgfpathlineto{\pgfqpoint{0.285588in}{0.739968in}}%
\pgfpathlineto{\pgfqpoint{0.285588in}{0.726357in}}%
\pgfpathlineto{\pgfqpoint{0.285588in}{0.712746in}}%
\pgfpathlineto{\pgfqpoint{0.285588in}{0.699135in}}%
\pgfpathlineto{\pgfqpoint{0.285588in}{0.685524in}}%
\pgfpathlineto{\pgfqpoint{0.285588in}{0.671913in}}%
\pgfpathlineto{\pgfqpoint{0.285588in}{0.658302in}}%
\pgfpathlineto{\pgfqpoint{0.285588in}{0.644691in}}%
\pgfpathlineto{\pgfqpoint{0.285588in}{0.631079in}}%
\pgfpathlineto{\pgfqpoint{0.285588in}{0.617468in}}%
\pgfpathlineto{\pgfqpoint{0.285588in}{0.603857in}}%
\pgfpathlineto{\pgfqpoint{0.285588in}{0.591635in}}%
\pgfpathlineto{\pgfqpoint{0.301244in}{0.591096in}}%
\pgfpathlineto{\pgfqpoint{0.309486in}{0.590246in}}%
\pgfpathlineto{\pgfqpoint{0.316901in}{0.589464in}}%
\pgfpathlineto{\pgfqpoint{0.332557in}{0.586729in}}%
\pgfpathlineto{\pgfqpoint{0.348214in}{0.582941in}}%
\pgfpathlineto{\pgfqpoint{0.363870in}{0.578141in}}%
\pgfpathlineto{\pgfqpoint{0.367942in}{0.576635in}}%
\pgfpathlineto{\pgfqpoint{0.379527in}{0.572250in}}%
\pgfpathlineto{\pgfqpoint{0.395183in}{0.565382in}}%
\pgfpathlineto{\pgfqpoint{0.399932in}{0.563024in}}%
\pgfpathlineto{\pgfqpoint{0.410840in}{0.557462in}}%
\pgfpathlineto{\pgfqpoint{0.425115in}{0.549413in}}%
\pgfpathlineto{\pgfqpoint{0.426497in}{0.548610in}}%
\pgfpathlineto{\pgfqpoint{0.442153in}{0.538664in}}%
\pgfpathlineto{\pgfqpoint{0.446337in}{0.535802in}}%
\pgfpathlineto{\pgfqpoint{0.457810in}{0.527676in}}%
\pgfpathlineto{\pgfqpoint{0.465082in}{0.522191in}}%
\pgfpathlineto{\pgfqpoint{0.473466in}{0.515604in}}%
\pgfpathlineto{\pgfqpoint{0.481936in}{0.508579in}}%
\pgfpathlineto{\pgfqpoint{0.489123in}{0.502331in}}%
\pgfpathlineto{\pgfqpoint{0.497203in}{0.494968in}}%
\pgfpathlineto{\pgfqpoint{0.504779in}{0.487679in}}%
\pgfpathlineto{\pgfqpoint{0.511089in}{0.481357in}}%
\pgfpathlineto{\pgfqpoint{0.520436in}{0.471384in}}%
\pgfpathlineto{\pgfqpoint{0.523728in}{0.467746in}}%
\pgfpathlineto{\pgfqpoint{0.535169in}{0.454135in}}%
\pgfpathlineto{\pgfqpoint{0.536093in}{0.452934in}}%
\pgfpathlineto{\pgfqpoint{0.545351in}{0.440524in}}%
\pgfpathlineto{\pgfqpoint{0.551749in}{0.431041in}}%
\pgfpathlineto{\pgfqpoint{0.554462in}{0.426913in}}%
\pgfpathlineto{\pgfqpoint{0.562362in}{0.413302in}}%
\pgfpathlineto{\pgfqpoint{0.567406in}{0.403230in}}%
\pgfpathlineto{\pgfqpoint{0.569138in}{0.399691in}}%
\pgfpathlineto{\pgfqpoint{0.574660in}{0.386079in}}%
\pgfpathlineto{\pgfqpoint{0.579017in}{0.372468in}}%
\pgfpathlineto{\pgfqpoint{0.582163in}{0.358857in}}%
\pgfpathlineto{\pgfqpoint{0.583062in}{0.352411in}}%
\pgfpathlineto{\pgfqpoint{0.584040in}{0.345246in}}%
\pgfpathlineto{\pgfqpoint{0.584660in}{0.331635in}}%
\pgfpathlineto{\pgfqpoint{0.598719in}{0.331635in}}%
\pgfpathclose%
\pgfpathmoveto{\pgfqpoint{1.334577in}{0.331635in}}%
\pgfpathlineto{\pgfqpoint{1.350234in}{0.331635in}}%
\pgfpathlineto{\pgfqpoint{1.365891in}{0.331635in}}%
\pgfpathlineto{\pgfqpoint{1.381547in}{0.331635in}}%
\pgfpathlineto{\pgfqpoint{1.397204in}{0.331635in}}%
\pgfpathlineto{\pgfqpoint{1.412860in}{0.331635in}}%
\pgfpathlineto{\pgfqpoint{1.428517in}{0.331635in}}%
\pgfpathlineto{\pgfqpoint{1.444173in}{0.331635in}}%
\pgfpathlineto{\pgfqpoint{1.459830in}{0.331635in}}%
\pgfpathlineto{\pgfqpoint{1.475486in}{0.331635in}}%
\pgfpathlineto{\pgfqpoint{1.491143in}{0.331635in}}%
\pgfpathlineto{\pgfqpoint{1.506800in}{0.331635in}}%
\pgfpathlineto{\pgfqpoint{1.522456in}{0.331635in}}%
\pgfpathlineto{\pgfqpoint{1.536515in}{0.331635in}}%
\pgfpathlineto{\pgfqpoint{1.537135in}{0.345246in}}%
\pgfpathlineto{\pgfqpoint{1.538113in}{0.352411in}}%
\pgfpathlineto{\pgfqpoint{1.539012in}{0.358857in}}%
\pgfpathlineto{\pgfqpoint{1.542158in}{0.372468in}}%
\pgfpathlineto{\pgfqpoint{1.546515in}{0.386079in}}%
\pgfpathlineto{\pgfqpoint{1.552037in}{0.399691in}}%
\pgfpathlineto{\pgfqpoint{1.553769in}{0.403230in}}%
\pgfpathlineto{\pgfqpoint{1.558813in}{0.413302in}}%
\pgfpathlineto{\pgfqpoint{1.566713in}{0.426913in}}%
\pgfpathlineto{\pgfqpoint{1.569426in}{0.431041in}}%
\pgfpathlineto{\pgfqpoint{1.575824in}{0.440524in}}%
\pgfpathlineto{\pgfqpoint{1.585082in}{0.452934in}}%
\pgfpathlineto{\pgfqpoint{1.586006in}{0.454135in}}%
\pgfpathlineto{\pgfqpoint{1.597447in}{0.467746in}}%
\pgfpathlineto{\pgfqpoint{1.600739in}{0.471384in}}%
\pgfpathlineto{\pgfqpoint{1.610086in}{0.481357in}}%
\pgfpathlineto{\pgfqpoint{1.616396in}{0.487679in}}%
\pgfpathlineto{\pgfqpoint{1.623972in}{0.494968in}}%
\pgfpathlineto{\pgfqpoint{1.632052in}{0.502331in}}%
\pgfpathlineto{\pgfqpoint{1.639239in}{0.508579in}}%
\pgfpathlineto{\pgfqpoint{1.647709in}{0.515604in}}%
\pgfpathlineto{\pgfqpoint{1.656093in}{0.522191in}}%
\pgfpathlineto{\pgfqpoint{1.663365in}{0.527676in}}%
\pgfpathlineto{\pgfqpoint{1.674838in}{0.535802in}}%
\pgfpathlineto{\pgfqpoint{1.679022in}{0.538664in}}%
\pgfpathlineto{\pgfqpoint{1.694678in}{0.548610in}}%
\pgfpathlineto{\pgfqpoint{1.696060in}{0.549413in}}%
\pgfpathlineto{\pgfqpoint{1.710335in}{0.557462in}}%
\pgfpathlineto{\pgfqpoint{1.721243in}{0.563024in}}%
\pgfpathlineto{\pgfqpoint{1.725992in}{0.565382in}}%
\pgfpathlineto{\pgfqpoint{1.741648in}{0.572250in}}%
\pgfpathlineto{\pgfqpoint{1.753233in}{0.576635in}}%
\pgfpathlineto{\pgfqpoint{1.757305in}{0.578141in}}%
\pgfpathlineto{\pgfqpoint{1.772961in}{0.582941in}}%
\pgfpathlineto{\pgfqpoint{1.788618in}{0.586729in}}%
\pgfpathlineto{\pgfqpoint{1.804274in}{0.589464in}}%
\pgfpathlineto{\pgfqpoint{1.811689in}{0.590246in}}%
\pgfpathlineto{\pgfqpoint{1.819931in}{0.591096in}}%
\pgfpathlineto{\pgfqpoint{1.835588in}{0.591635in}}%
\pgfpathlineto{\pgfqpoint{1.835588in}{0.603857in}}%
\pgfpathlineto{\pgfqpoint{1.835588in}{0.617468in}}%
\pgfpathlineto{\pgfqpoint{1.835588in}{0.631079in}}%
\pgfpathlineto{\pgfqpoint{1.835588in}{0.644691in}}%
\pgfpathlineto{\pgfqpoint{1.835588in}{0.658302in}}%
\pgfpathlineto{\pgfqpoint{1.835588in}{0.671913in}}%
\pgfpathlineto{\pgfqpoint{1.835588in}{0.685524in}}%
\pgfpathlineto{\pgfqpoint{1.835588in}{0.699135in}}%
\pgfpathlineto{\pgfqpoint{1.835588in}{0.712746in}}%
\pgfpathlineto{\pgfqpoint{1.835588in}{0.726357in}}%
\pgfpathlineto{\pgfqpoint{1.835588in}{0.739968in}}%
\pgfpathlineto{\pgfqpoint{1.835588in}{0.753579in}}%
\pgfpathlineto{\pgfqpoint{1.835588in}{0.767191in}}%
\pgfpathlineto{\pgfqpoint{1.835588in}{0.774721in}}%
\pgfpathlineto{\pgfqpoint{1.819931in}{0.774030in}}%
\pgfpathlineto{\pgfqpoint{1.804274in}{0.771964in}}%
\pgfpathlineto{\pgfqpoint{1.788618in}{0.768546in}}%
\pgfpathlineto{\pgfqpoint{1.784146in}{0.767191in}}%
\pgfpathlineto{\pgfqpoint{1.772961in}{0.763916in}}%
\pgfpathlineto{\pgfqpoint{1.757305in}{0.758101in}}%
\pgfpathlineto{\pgfqpoint{1.747188in}{0.753579in}}%
\pgfpathlineto{\pgfqpoint{1.741648in}{0.751182in}}%
\pgfpathlineto{\pgfqpoint{1.725992in}{0.743313in}}%
\pgfpathlineto{\pgfqpoint{1.720112in}{0.739968in}}%
\pgfpathlineto{\pgfqpoint{1.710335in}{0.734570in}}%
\pgfpathlineto{\pgfqpoint{1.696898in}{0.726357in}}%
\pgfpathlineto{\pgfqpoint{1.694678in}{0.725035in}}%
\pgfpathlineto{\pgfqpoint{1.679022in}{0.714855in}}%
\pgfpathlineto{\pgfqpoint{1.676012in}{0.712746in}}%
\pgfpathlineto{\pgfqpoint{1.663365in}{0.704091in}}%
\pgfpathlineto{\pgfqpoint{1.656562in}{0.699135in}}%
\pgfpathlineto{\pgfqpoint{1.647709in}{0.692814in}}%
\pgfpathlineto{\pgfqpoint{1.638027in}{0.685524in}}%
\pgfpathlineto{\pgfqpoint{1.632052in}{0.681099in}}%
\pgfpathlineto{\pgfqpoint{1.620193in}{0.671913in}}%
\pgfpathlineto{\pgfqpoint{1.616396in}{0.669010in}}%
\pgfpathlineto{\pgfqpoint{1.602906in}{0.658302in}}%
\pgfpathlineto{\pgfqpoint{1.600739in}{0.656598in}}%
\pgfpathlineto{\pgfqpoint{1.586058in}{0.644691in}}%
\pgfpathlineto{\pgfqpoint{1.585082in}{0.643904in}}%
\pgfpathlineto{\pgfqpoint{1.569573in}{0.631079in}}%
\pgfpathlineto{\pgfqpoint{1.569426in}{0.630958in}}%
\pgfpathlineto{\pgfqpoint{1.553769in}{0.617778in}}%
\pgfpathlineto{\pgfqpoint{1.553407in}{0.617468in}}%
\pgfpathlineto{\pgfqpoint{1.538113in}{0.604380in}}%
\pgfpathlineto{\pgfqpoint{1.537508in}{0.603857in}}%
\pgfpathlineto{\pgfqpoint{1.522456in}{0.590772in}}%
\pgfpathlineto{\pgfqpoint{1.521854in}{0.590246in}}%
\pgfpathlineto{\pgfqpoint{1.506800in}{0.576950in}}%
\pgfpathlineto{\pgfqpoint{1.506443in}{0.576635in}}%
\pgfpathlineto{\pgfqpoint{1.491283in}{0.563024in}}%
\pgfpathlineto{\pgfqpoint{1.491143in}{0.562896in}}%
\pgfpathlineto{\pgfqpoint{1.476392in}{0.549413in}}%
\pgfpathlineto{\pgfqpoint{1.475486in}{0.548564in}}%
\pgfpathlineto{\pgfqpoint{1.461790in}{0.535802in}}%
\pgfpathlineto{\pgfqpoint{1.459830in}{0.533918in}}%
\pgfpathlineto{\pgfqpoint{1.447513in}{0.522191in}}%
\pgfpathlineto{\pgfqpoint{1.444173in}{0.518889in}}%
\pgfpathlineto{\pgfqpoint{1.433607in}{0.508579in}}%
\pgfpathlineto{\pgfqpoint{1.428517in}{0.503385in}}%
\pgfpathlineto{\pgfqpoint{1.420132in}{0.494968in}}%
\pgfpathlineto{\pgfqpoint{1.412860in}{0.487272in}}%
\pgfpathlineto{\pgfqpoint{1.407159in}{0.481357in}}%
\pgfpathlineto{\pgfqpoint{1.397204in}{0.470363in}}%
\pgfpathlineto{\pgfqpoint{1.394778in}{0.467746in}}%
\pgfpathlineto{\pgfqpoint{1.383068in}{0.454135in}}%
\pgfpathlineto{\pgfqpoint{1.381547in}{0.452205in}}%
\pgfpathlineto{\pgfqpoint{1.372100in}{0.440524in}}%
\pgfpathlineto{\pgfqpoint{1.365891in}{0.432024in}}%
\pgfpathlineto{\pgfqpoint{1.362044in}{0.426913in}}%
\pgfpathlineto{\pgfqpoint{1.352992in}{0.413302in}}%
\pgfpathlineto{\pgfqpoint{1.350234in}{0.408485in}}%
\pgfpathlineto{\pgfqpoint{1.345033in}{0.399691in}}%
\pgfpathlineto{\pgfqpoint{1.338344in}{0.386079in}}%
\pgfpathlineto{\pgfqpoint{1.334577in}{0.376356in}}%
\pgfpathlineto{\pgfqpoint{1.333018in}{0.372468in}}%
\pgfpathlineto{\pgfqpoint{1.329086in}{0.358857in}}%
\pgfpathlineto{\pgfqpoint{1.326710in}{0.345246in}}%
\pgfpathlineto{\pgfqpoint{1.325916in}{0.331635in}}%
\pgfpathlineto{\pgfqpoint{1.334577in}{0.331635in}}%
\pgfpathclose%
\pgfpathmoveto{\pgfqpoint{0.301244in}{1.236740in}}%
\pgfpathlineto{\pgfqpoint{0.316901in}{1.238806in}}%
\pgfpathlineto{\pgfqpoint{0.332557in}{1.242224in}}%
\pgfpathlineto{\pgfqpoint{0.337029in}{1.243579in}}%
\pgfpathlineto{\pgfqpoint{0.348214in}{1.246854in}}%
\pgfpathlineto{\pgfqpoint{0.363870in}{1.252669in}}%
\pgfpathlineto{\pgfqpoint{0.373987in}{1.257191in}}%
\pgfpathlineto{\pgfqpoint{0.379527in}{1.259588in}}%
\pgfpathlineto{\pgfqpoint{0.395183in}{1.267457in}}%
\pgfpathlineto{\pgfqpoint{0.401063in}{1.270802in}}%
\pgfpathlineto{\pgfqpoint{0.410840in}{1.276200in}}%
\pgfpathlineto{\pgfqpoint{0.424277in}{1.284413in}}%
\pgfpathlineto{\pgfqpoint{0.426497in}{1.285735in}}%
\pgfpathlineto{\pgfqpoint{0.442153in}{1.295915in}}%
\pgfpathlineto{\pgfqpoint{0.445163in}{1.298024in}}%
\pgfpathlineto{\pgfqpoint{0.457810in}{1.306679in}}%
\pgfpathlineto{\pgfqpoint{0.464613in}{1.311635in}}%
\pgfpathlineto{\pgfqpoint{0.473466in}{1.317956in}}%
\pgfpathlineto{\pgfqpoint{0.483148in}{1.325246in}}%
\pgfpathlineto{\pgfqpoint{0.489123in}{1.329671in}}%
\pgfpathlineto{\pgfqpoint{0.500982in}{1.338857in}}%
\pgfpathlineto{\pgfqpoint{0.504779in}{1.341760in}}%
\pgfpathlineto{\pgfqpoint{0.518269in}{1.352468in}}%
\pgfpathlineto{\pgfqpoint{0.520436in}{1.354172in}}%
\pgfpathlineto{\pgfqpoint{0.535117in}{1.366079in}}%
\pgfpathlineto{\pgfqpoint{0.536093in}{1.366866in}}%
\pgfpathlineto{\pgfqpoint{0.551602in}{1.379691in}}%
\pgfpathlineto{\pgfqpoint{0.551749in}{1.379812in}}%
\pgfpathlineto{\pgfqpoint{0.567406in}{1.392992in}}%
\pgfpathlineto{\pgfqpoint{0.567768in}{1.393302in}}%
\pgfpathlineto{\pgfqpoint{0.583062in}{1.406390in}}%
\pgfpathlineto{\pgfqpoint{0.583667in}{1.406913in}}%
\pgfpathlineto{\pgfqpoint{0.598719in}{1.419998in}}%
\pgfpathlineto{\pgfqpoint{0.599321in}{1.420524in}}%
\pgfpathlineto{\pgfqpoint{0.614375in}{1.433820in}}%
\pgfpathlineto{\pgfqpoint{0.614732in}{1.434135in}}%
\pgfpathlineto{\pgfqpoint{0.629892in}{1.447746in}}%
\pgfpathlineto{\pgfqpoint{0.630032in}{1.447874in}}%
\pgfpathlineto{\pgfqpoint{0.644783in}{1.461357in}}%
\pgfpathlineto{\pgfqpoint{0.645689in}{1.462206in}}%
\pgfpathlineto{\pgfqpoint{0.659385in}{1.474968in}}%
\pgfpathlineto{\pgfqpoint{0.661345in}{1.476852in}}%
\pgfpathlineto{\pgfqpoint{0.673662in}{1.488579in}}%
\pgfpathlineto{\pgfqpoint{0.677002in}{1.491881in}}%
\pgfpathlineto{\pgfqpoint{0.687568in}{1.502191in}}%
\pgfpathlineto{\pgfqpoint{0.692658in}{1.507385in}}%
\pgfpathlineto{\pgfqpoint{0.701043in}{1.515802in}}%
\pgfpathlineto{\pgfqpoint{0.708315in}{1.523498in}}%
\pgfpathlineto{\pgfqpoint{0.714016in}{1.529413in}}%
\pgfpathlineto{\pgfqpoint{0.723971in}{1.540407in}}%
\pgfpathlineto{\pgfqpoint{0.726397in}{1.543024in}}%
\pgfpathlineto{\pgfqpoint{0.738107in}{1.556635in}}%
\pgfpathlineto{\pgfqpoint{0.739628in}{1.558565in}}%
\pgfpathlineto{\pgfqpoint{0.749075in}{1.570246in}}%
\pgfpathlineto{\pgfqpoint{0.755284in}{1.578746in}}%
\pgfpathlineto{\pgfqpoint{0.759131in}{1.583857in}}%
\pgfpathlineto{\pgfqpoint{0.768183in}{1.597468in}}%
\pgfpathlineto{\pgfqpoint{0.770941in}{1.602285in}}%
\pgfpathlineto{\pgfqpoint{0.776142in}{1.611079in}}%
\pgfpathlineto{\pgfqpoint{0.782831in}{1.624691in}}%
\pgfpathlineto{\pgfqpoint{0.786598in}{1.634414in}}%
\pgfpathlineto{\pgfqpoint{0.788157in}{1.638302in}}%
\pgfpathlineto{\pgfqpoint{0.792089in}{1.651913in}}%
\pgfpathlineto{\pgfqpoint{0.794465in}{1.665524in}}%
\pgfpathlineto{\pgfqpoint{0.795259in}{1.679135in}}%
\pgfpathlineto{\pgfqpoint{0.786598in}{1.679135in}}%
\pgfpathlineto{\pgfqpoint{0.770941in}{1.679135in}}%
\pgfpathlineto{\pgfqpoint{0.755284in}{1.679135in}}%
\pgfpathlineto{\pgfqpoint{0.739628in}{1.679135in}}%
\pgfpathlineto{\pgfqpoint{0.723971in}{1.679135in}}%
\pgfpathlineto{\pgfqpoint{0.708315in}{1.679135in}}%
\pgfpathlineto{\pgfqpoint{0.692658in}{1.679135in}}%
\pgfpathlineto{\pgfqpoint{0.677002in}{1.679135in}}%
\pgfpathlineto{\pgfqpoint{0.661345in}{1.679135in}}%
\pgfpathlineto{\pgfqpoint{0.645689in}{1.679135in}}%
\pgfpathlineto{\pgfqpoint{0.630032in}{1.679135in}}%
\pgfpathlineto{\pgfqpoint{0.614375in}{1.679135in}}%
\pgfpathlineto{\pgfqpoint{0.598719in}{1.679135in}}%
\pgfpathlineto{\pgfqpoint{0.584660in}{1.679135in}}%
\pgfpathlineto{\pgfqpoint{0.584040in}{1.665524in}}%
\pgfpathlineto{\pgfqpoint{0.583062in}{1.658359in}}%
\pgfpathlineto{\pgfqpoint{0.582163in}{1.651913in}}%
\pgfpathlineto{\pgfqpoint{0.579017in}{1.638302in}}%
\pgfpathlineto{\pgfqpoint{0.574660in}{1.624691in}}%
\pgfpathlineto{\pgfqpoint{0.569138in}{1.611079in}}%
\pgfpathlineto{\pgfqpoint{0.567406in}{1.607540in}}%
\pgfpathlineto{\pgfqpoint{0.562362in}{1.597468in}}%
\pgfpathlineto{\pgfqpoint{0.554462in}{1.583857in}}%
\pgfpathlineto{\pgfqpoint{0.551749in}{1.579729in}}%
\pgfpathlineto{\pgfqpoint{0.545351in}{1.570246in}}%
\pgfpathlineto{\pgfqpoint{0.536093in}{1.557836in}}%
\pgfpathlineto{\pgfqpoint{0.535169in}{1.556635in}}%
\pgfpathlineto{\pgfqpoint{0.523728in}{1.543024in}}%
\pgfpathlineto{\pgfqpoint{0.520436in}{1.539386in}}%
\pgfpathlineto{\pgfqpoint{0.511089in}{1.529413in}}%
\pgfpathlineto{\pgfqpoint{0.504779in}{1.523091in}}%
\pgfpathlineto{\pgfqpoint{0.497203in}{1.515802in}}%
\pgfpathlineto{\pgfqpoint{0.489123in}{1.508439in}}%
\pgfpathlineto{\pgfqpoint{0.481936in}{1.502191in}}%
\pgfpathlineto{\pgfqpoint{0.473466in}{1.495166in}}%
\pgfpathlineto{\pgfqpoint{0.465082in}{1.488579in}}%
\pgfpathlineto{\pgfqpoint{0.457810in}{1.483094in}}%
\pgfpathlineto{\pgfqpoint{0.446337in}{1.474968in}}%
\pgfpathlineto{\pgfqpoint{0.442153in}{1.472106in}}%
\pgfpathlineto{\pgfqpoint{0.426497in}{1.462160in}}%
\pgfpathlineto{\pgfqpoint{0.425115in}{1.461357in}}%
\pgfpathlineto{\pgfqpoint{0.410840in}{1.453308in}}%
\pgfpathlineto{\pgfqpoint{0.399932in}{1.447746in}}%
\pgfpathlineto{\pgfqpoint{0.395183in}{1.445388in}}%
\pgfpathlineto{\pgfqpoint{0.379527in}{1.438520in}}%
\pgfpathlineto{\pgfqpoint{0.367942in}{1.434135in}}%
\pgfpathlineto{\pgfqpoint{0.363870in}{1.432629in}}%
\pgfpathlineto{\pgfqpoint{0.348214in}{1.427829in}}%
\pgfpathlineto{\pgfqpoint{0.332557in}{1.424041in}}%
\pgfpathlineto{\pgfqpoint{0.316901in}{1.421306in}}%
\pgfpathlineto{\pgfqpoint{0.309486in}{1.420524in}}%
\pgfpathlineto{\pgfqpoint{0.301244in}{1.419674in}}%
\pgfpathlineto{\pgfqpoint{0.285588in}{1.419135in}}%
\pgfpathlineto{\pgfqpoint{0.285588in}{1.406913in}}%
\pgfpathlineto{\pgfqpoint{0.285588in}{1.393302in}}%
\pgfpathlineto{\pgfqpoint{0.285588in}{1.379691in}}%
\pgfpathlineto{\pgfqpoint{0.285588in}{1.366079in}}%
\pgfpathlineto{\pgfqpoint{0.285588in}{1.352468in}}%
\pgfpathlineto{\pgfqpoint{0.285588in}{1.338857in}}%
\pgfpathlineto{\pgfqpoint{0.285588in}{1.325246in}}%
\pgfpathlineto{\pgfqpoint{0.285588in}{1.311635in}}%
\pgfpathlineto{\pgfqpoint{0.285588in}{1.298024in}}%
\pgfpathlineto{\pgfqpoint{0.285588in}{1.284413in}}%
\pgfpathlineto{\pgfqpoint{0.285588in}{1.270802in}}%
\pgfpathlineto{\pgfqpoint{0.285588in}{1.257191in}}%
\pgfpathlineto{\pgfqpoint{0.285588in}{1.243579in}}%
\pgfpathlineto{\pgfqpoint{0.285588in}{1.236049in}}%
\pgfpathlineto{\pgfqpoint{0.301244in}{1.236740in}}%
\pgfpathclose%
\pgfpathmoveto{\pgfqpoint{1.788618in}{1.242224in}}%
\pgfpathlineto{\pgfqpoint{1.804274in}{1.238806in}}%
\pgfpathlineto{\pgfqpoint{1.819931in}{1.236740in}}%
\pgfpathlineto{\pgfqpoint{1.835588in}{1.236049in}}%
\pgfpathlineto{\pgfqpoint{1.835588in}{1.243579in}}%
\pgfpathlineto{\pgfqpoint{1.835588in}{1.257191in}}%
\pgfpathlineto{\pgfqpoint{1.835588in}{1.270802in}}%
\pgfpathlineto{\pgfqpoint{1.835588in}{1.284413in}}%
\pgfpathlineto{\pgfqpoint{1.835588in}{1.298024in}}%
\pgfpathlineto{\pgfqpoint{1.835588in}{1.311635in}}%
\pgfpathlineto{\pgfqpoint{1.835588in}{1.325246in}}%
\pgfpathlineto{\pgfqpoint{1.835588in}{1.338857in}}%
\pgfpathlineto{\pgfqpoint{1.835588in}{1.352468in}}%
\pgfpathlineto{\pgfqpoint{1.835588in}{1.366079in}}%
\pgfpathlineto{\pgfqpoint{1.835588in}{1.379691in}}%
\pgfpathlineto{\pgfqpoint{1.835588in}{1.393302in}}%
\pgfpathlineto{\pgfqpoint{1.835588in}{1.406913in}}%
\pgfpathlineto{\pgfqpoint{1.835588in}{1.419135in}}%
\pgfpathlineto{\pgfqpoint{1.819931in}{1.419674in}}%
\pgfpathlineto{\pgfqpoint{1.811689in}{1.420524in}}%
\pgfpathlineto{\pgfqpoint{1.804274in}{1.421306in}}%
\pgfpathlineto{\pgfqpoint{1.788618in}{1.424041in}}%
\pgfpathlineto{\pgfqpoint{1.772961in}{1.427829in}}%
\pgfpathlineto{\pgfqpoint{1.757305in}{1.432629in}}%
\pgfpathlineto{\pgfqpoint{1.753233in}{1.434135in}}%
\pgfpathlineto{\pgfqpoint{1.741648in}{1.438520in}}%
\pgfpathlineto{\pgfqpoint{1.725992in}{1.445388in}}%
\pgfpathlineto{\pgfqpoint{1.721243in}{1.447746in}}%
\pgfpathlineto{\pgfqpoint{1.710335in}{1.453308in}}%
\pgfpathlineto{\pgfqpoint{1.696060in}{1.461357in}}%
\pgfpathlineto{\pgfqpoint{1.694678in}{1.462160in}}%
\pgfpathlineto{\pgfqpoint{1.679022in}{1.472106in}}%
\pgfpathlineto{\pgfqpoint{1.674838in}{1.474968in}}%
\pgfpathlineto{\pgfqpoint{1.663365in}{1.483094in}}%
\pgfpathlineto{\pgfqpoint{1.656093in}{1.488579in}}%
\pgfpathlineto{\pgfqpoint{1.647709in}{1.495166in}}%
\pgfpathlineto{\pgfqpoint{1.639239in}{1.502191in}}%
\pgfpathlineto{\pgfqpoint{1.632052in}{1.508439in}}%
\pgfpathlineto{\pgfqpoint{1.623972in}{1.515802in}}%
\pgfpathlineto{\pgfqpoint{1.616396in}{1.523091in}}%
\pgfpathlineto{\pgfqpoint{1.610086in}{1.529413in}}%
\pgfpathlineto{\pgfqpoint{1.600739in}{1.539386in}}%
\pgfpathlineto{\pgfqpoint{1.597447in}{1.543024in}}%
\pgfpathlineto{\pgfqpoint{1.586006in}{1.556635in}}%
\pgfpathlineto{\pgfqpoint{1.585082in}{1.557836in}}%
\pgfpathlineto{\pgfqpoint{1.575824in}{1.570246in}}%
\pgfpathlineto{\pgfqpoint{1.569426in}{1.579729in}}%
\pgfpathlineto{\pgfqpoint{1.566713in}{1.583857in}}%
\pgfpathlineto{\pgfqpoint{1.558813in}{1.597468in}}%
\pgfpathlineto{\pgfqpoint{1.553769in}{1.607540in}}%
\pgfpathlineto{\pgfqpoint{1.552037in}{1.611079in}}%
\pgfpathlineto{\pgfqpoint{1.546515in}{1.624691in}}%
\pgfpathlineto{\pgfqpoint{1.542158in}{1.638302in}}%
\pgfpathlineto{\pgfqpoint{1.539012in}{1.651913in}}%
\pgfpathlineto{\pgfqpoint{1.538113in}{1.658359in}}%
\pgfpathlineto{\pgfqpoint{1.537135in}{1.665524in}}%
\pgfpathlineto{\pgfqpoint{1.536515in}{1.679135in}}%
\pgfpathlineto{\pgfqpoint{1.522456in}{1.679135in}}%
\pgfpathlineto{\pgfqpoint{1.506800in}{1.679135in}}%
\pgfpathlineto{\pgfqpoint{1.491143in}{1.679135in}}%
\pgfpathlineto{\pgfqpoint{1.475486in}{1.679135in}}%
\pgfpathlineto{\pgfqpoint{1.459830in}{1.679135in}}%
\pgfpathlineto{\pgfqpoint{1.444173in}{1.679135in}}%
\pgfpathlineto{\pgfqpoint{1.428517in}{1.679135in}}%
\pgfpathlineto{\pgfqpoint{1.412860in}{1.679135in}}%
\pgfpathlineto{\pgfqpoint{1.397204in}{1.679135in}}%
\pgfpathlineto{\pgfqpoint{1.381547in}{1.679135in}}%
\pgfpathlineto{\pgfqpoint{1.365891in}{1.679135in}}%
\pgfpathlineto{\pgfqpoint{1.350234in}{1.679135in}}%
\pgfpathlineto{\pgfqpoint{1.334577in}{1.679135in}}%
\pgfpathlineto{\pgfqpoint{1.325916in}{1.679135in}}%
\pgfpathlineto{\pgfqpoint{1.326710in}{1.665524in}}%
\pgfpathlineto{\pgfqpoint{1.329086in}{1.651913in}}%
\pgfpathlineto{\pgfqpoint{1.333018in}{1.638302in}}%
\pgfpathlineto{\pgfqpoint{1.334577in}{1.634414in}}%
\pgfpathlineto{\pgfqpoint{1.338344in}{1.624691in}}%
\pgfpathlineto{\pgfqpoint{1.345033in}{1.611079in}}%
\pgfpathlineto{\pgfqpoint{1.350234in}{1.602285in}}%
\pgfpathlineto{\pgfqpoint{1.352992in}{1.597468in}}%
\pgfpathlineto{\pgfqpoint{1.362044in}{1.583857in}}%
\pgfpathlineto{\pgfqpoint{1.365891in}{1.578746in}}%
\pgfpathlineto{\pgfqpoint{1.372100in}{1.570246in}}%
\pgfpathlineto{\pgfqpoint{1.381547in}{1.558565in}}%
\pgfpathlineto{\pgfqpoint{1.383068in}{1.556635in}}%
\pgfpathlineto{\pgfqpoint{1.394778in}{1.543024in}}%
\pgfpathlineto{\pgfqpoint{1.397204in}{1.540407in}}%
\pgfpathlineto{\pgfqpoint{1.407159in}{1.529413in}}%
\pgfpathlineto{\pgfqpoint{1.412860in}{1.523498in}}%
\pgfpathlineto{\pgfqpoint{1.420132in}{1.515802in}}%
\pgfpathlineto{\pgfqpoint{1.428517in}{1.507385in}}%
\pgfpathlineto{\pgfqpoint{1.433607in}{1.502191in}}%
\pgfpathlineto{\pgfqpoint{1.444173in}{1.491881in}}%
\pgfpathlineto{\pgfqpoint{1.447513in}{1.488579in}}%
\pgfpathlineto{\pgfqpoint{1.459830in}{1.476852in}}%
\pgfpathlineto{\pgfqpoint{1.461790in}{1.474968in}}%
\pgfpathlineto{\pgfqpoint{1.475486in}{1.462206in}}%
\pgfpathlineto{\pgfqpoint{1.476392in}{1.461357in}}%
\pgfpathlineto{\pgfqpoint{1.491143in}{1.447874in}}%
\pgfpathlineto{\pgfqpoint{1.491283in}{1.447746in}}%
\pgfpathlineto{\pgfqpoint{1.506443in}{1.434135in}}%
\pgfpathlineto{\pgfqpoint{1.506800in}{1.433820in}}%
\pgfpathlineto{\pgfqpoint{1.521854in}{1.420524in}}%
\pgfpathlineto{\pgfqpoint{1.522456in}{1.419998in}}%
\pgfpathlineto{\pgfqpoint{1.537508in}{1.406913in}}%
\pgfpathlineto{\pgfqpoint{1.538113in}{1.406390in}}%
\pgfpathlineto{\pgfqpoint{1.553407in}{1.393302in}}%
\pgfpathlineto{\pgfqpoint{1.553769in}{1.392992in}}%
\pgfpathlineto{\pgfqpoint{1.569426in}{1.379812in}}%
\pgfpathlineto{\pgfqpoint{1.569573in}{1.379691in}}%
\pgfpathlineto{\pgfqpoint{1.585082in}{1.366866in}}%
\pgfpathlineto{\pgfqpoint{1.586058in}{1.366079in}}%
\pgfpathlineto{\pgfqpoint{1.600739in}{1.354172in}}%
\pgfpathlineto{\pgfqpoint{1.602906in}{1.352468in}}%
\pgfpathlineto{\pgfqpoint{1.616396in}{1.341760in}}%
\pgfpathlineto{\pgfqpoint{1.620193in}{1.338857in}}%
\pgfpathlineto{\pgfqpoint{1.632052in}{1.329671in}}%
\pgfpathlineto{\pgfqpoint{1.638027in}{1.325246in}}%
\pgfpathlineto{\pgfqpoint{1.647709in}{1.317956in}}%
\pgfpathlineto{\pgfqpoint{1.656562in}{1.311635in}}%
\pgfpathlineto{\pgfqpoint{1.663365in}{1.306679in}}%
\pgfpathlineto{\pgfqpoint{1.676012in}{1.298024in}}%
\pgfpathlineto{\pgfqpoint{1.679022in}{1.295915in}}%
\pgfpathlineto{\pgfqpoint{1.694678in}{1.285735in}}%
\pgfpathlineto{\pgfqpoint{1.696898in}{1.284413in}}%
\pgfpathlineto{\pgfqpoint{1.710335in}{1.276200in}}%
\pgfpathlineto{\pgfqpoint{1.720112in}{1.270802in}}%
\pgfpathlineto{\pgfqpoint{1.725992in}{1.267457in}}%
\pgfpathlineto{\pgfqpoint{1.741648in}{1.259588in}}%
\pgfpathlineto{\pgfqpoint{1.747188in}{1.257191in}}%
\pgfpathlineto{\pgfqpoint{1.757305in}{1.252669in}}%
\pgfpathlineto{\pgfqpoint{1.772961in}{1.246854in}}%
\pgfpathlineto{\pgfqpoint{1.784146in}{1.243579in}}%
\pgfpathlineto{\pgfqpoint{1.788618in}{1.242224in}}%
\pgfpathclose%
\pgfusepath{fill}%
\end{pgfscope}%
\begin{pgfscope}%
\pgfpathrectangle{\pgfqpoint{0.285588in}{0.331635in}}{\pgfqpoint{1.550000in}{1.347500in}}%
\pgfusepath{clip}%
\pgfsetbuttcap%
\pgfsetroundjoin%
\definecolor{currentfill}{rgb}{0.048062,0.036607,0.150327}%
\pgfsetfillcolor{currentfill}%
\pgfsetlinewidth{0.000000pt}%
\definecolor{currentstroke}{rgb}{0.000000,0.000000,0.000000}%
\pgfsetstrokecolor{currentstroke}%
\pgfsetdash{}{0pt}%
\pgfpathmoveto{\pgfqpoint{0.301244in}{0.331635in}}%
\pgfpathlineto{\pgfqpoint{0.316901in}{0.331635in}}%
\pgfpathlineto{\pgfqpoint{0.332557in}{0.331635in}}%
\pgfpathlineto{\pgfqpoint{0.348214in}{0.331635in}}%
\pgfpathlineto{\pgfqpoint{0.363870in}{0.331635in}}%
\pgfpathlineto{\pgfqpoint{0.379527in}{0.331635in}}%
\pgfpathlineto{\pgfqpoint{0.395183in}{0.331635in}}%
\pgfpathlineto{\pgfqpoint{0.410840in}{0.331635in}}%
\pgfpathlineto{\pgfqpoint{0.426497in}{0.331635in}}%
\pgfpathlineto{\pgfqpoint{0.442153in}{0.331635in}}%
\pgfpathlineto{\pgfqpoint{0.457810in}{0.331635in}}%
\pgfpathlineto{\pgfqpoint{0.473466in}{0.331635in}}%
\pgfpathlineto{\pgfqpoint{0.489123in}{0.331635in}}%
\pgfpathlineto{\pgfqpoint{0.504779in}{0.331635in}}%
\pgfpathlineto{\pgfqpoint{0.520436in}{0.331635in}}%
\pgfpathlineto{\pgfqpoint{0.536093in}{0.331635in}}%
\pgfpathlineto{\pgfqpoint{0.551749in}{0.331635in}}%
\pgfpathlineto{\pgfqpoint{0.567406in}{0.331635in}}%
\pgfpathlineto{\pgfqpoint{0.583062in}{0.331635in}}%
\pgfpathlineto{\pgfqpoint{0.584660in}{0.331635in}}%
\pgfpathlineto{\pgfqpoint{0.584040in}{0.345246in}}%
\pgfpathlineto{\pgfqpoint{0.583062in}{0.352411in}}%
\pgfpathlineto{\pgfqpoint{0.582163in}{0.358857in}}%
\pgfpathlineto{\pgfqpoint{0.579017in}{0.372468in}}%
\pgfpathlineto{\pgfqpoint{0.574660in}{0.386079in}}%
\pgfpathlineto{\pgfqpoint{0.569138in}{0.399691in}}%
\pgfpathlineto{\pgfqpoint{0.567406in}{0.403230in}}%
\pgfpathlineto{\pgfqpoint{0.562362in}{0.413302in}}%
\pgfpathlineto{\pgfqpoint{0.554462in}{0.426913in}}%
\pgfpathlineto{\pgfqpoint{0.551749in}{0.431041in}}%
\pgfpathlineto{\pgfqpoint{0.545351in}{0.440524in}}%
\pgfpathlineto{\pgfqpoint{0.536093in}{0.452934in}}%
\pgfpathlineto{\pgfqpoint{0.535169in}{0.454135in}}%
\pgfpathlineto{\pgfqpoint{0.523728in}{0.467746in}}%
\pgfpathlineto{\pgfqpoint{0.520436in}{0.471384in}}%
\pgfpathlineto{\pgfqpoint{0.511089in}{0.481357in}}%
\pgfpathlineto{\pgfqpoint{0.504779in}{0.487679in}}%
\pgfpathlineto{\pgfqpoint{0.497203in}{0.494968in}}%
\pgfpathlineto{\pgfqpoint{0.489123in}{0.502331in}}%
\pgfpathlineto{\pgfqpoint{0.481936in}{0.508579in}}%
\pgfpathlineto{\pgfqpoint{0.473466in}{0.515604in}}%
\pgfpathlineto{\pgfqpoint{0.465082in}{0.522191in}}%
\pgfpathlineto{\pgfqpoint{0.457810in}{0.527676in}}%
\pgfpathlineto{\pgfqpoint{0.446337in}{0.535802in}}%
\pgfpathlineto{\pgfqpoint{0.442153in}{0.538664in}}%
\pgfpathlineto{\pgfqpoint{0.426497in}{0.548610in}}%
\pgfpathlineto{\pgfqpoint{0.425115in}{0.549413in}}%
\pgfpathlineto{\pgfqpoint{0.410840in}{0.557462in}}%
\pgfpathlineto{\pgfqpoint{0.399932in}{0.563024in}}%
\pgfpathlineto{\pgfqpoint{0.395183in}{0.565382in}}%
\pgfpathlineto{\pgfqpoint{0.379527in}{0.572250in}}%
\pgfpathlineto{\pgfqpoint{0.367942in}{0.576635in}}%
\pgfpathlineto{\pgfqpoint{0.363870in}{0.578141in}}%
\pgfpathlineto{\pgfqpoint{0.348214in}{0.582941in}}%
\pgfpathlineto{\pgfqpoint{0.332557in}{0.586729in}}%
\pgfpathlineto{\pgfqpoint{0.316901in}{0.589464in}}%
\pgfpathlineto{\pgfqpoint{0.309486in}{0.590246in}}%
\pgfpathlineto{\pgfqpoint{0.301244in}{0.591096in}}%
\pgfpathlineto{\pgfqpoint{0.285588in}{0.591635in}}%
\pgfpathlineto{\pgfqpoint{0.285588in}{0.590246in}}%
\pgfpathlineto{\pgfqpoint{0.285588in}{0.576635in}}%
\pgfpathlineto{\pgfqpoint{0.285588in}{0.563024in}}%
\pgfpathlineto{\pgfqpoint{0.285588in}{0.549413in}}%
\pgfpathlineto{\pgfqpoint{0.285588in}{0.535802in}}%
\pgfpathlineto{\pgfqpoint{0.285588in}{0.522191in}}%
\pgfpathlineto{\pgfqpoint{0.285588in}{0.508579in}}%
\pgfpathlineto{\pgfqpoint{0.285588in}{0.494968in}}%
\pgfpathlineto{\pgfqpoint{0.285588in}{0.481357in}}%
\pgfpathlineto{\pgfqpoint{0.285588in}{0.467746in}}%
\pgfpathlineto{\pgfqpoint{0.285588in}{0.454135in}}%
\pgfpathlineto{\pgfqpoint{0.285588in}{0.440524in}}%
\pgfpathlineto{\pgfqpoint{0.285588in}{0.426913in}}%
\pgfpathlineto{\pgfqpoint{0.285588in}{0.413302in}}%
\pgfpathlineto{\pgfqpoint{0.285588in}{0.399691in}}%
\pgfpathlineto{\pgfqpoint{0.285588in}{0.386079in}}%
\pgfpathlineto{\pgfqpoint{0.285588in}{0.372468in}}%
\pgfpathlineto{\pgfqpoint{0.285588in}{0.358857in}}%
\pgfpathlineto{\pgfqpoint{0.285588in}{0.345246in}}%
\pgfpathlineto{\pgfqpoint{0.285588in}{0.331635in}}%
\pgfpathlineto{\pgfqpoint{0.301244in}{0.331635in}}%
\pgfpathclose%
\pgfpathmoveto{\pgfqpoint{1.538113in}{0.331635in}}%
\pgfpathlineto{\pgfqpoint{1.553769in}{0.331635in}}%
\pgfpathlineto{\pgfqpoint{1.569426in}{0.331635in}}%
\pgfpathlineto{\pgfqpoint{1.585082in}{0.331635in}}%
\pgfpathlineto{\pgfqpoint{1.600739in}{0.331635in}}%
\pgfpathlineto{\pgfqpoint{1.616396in}{0.331635in}}%
\pgfpathlineto{\pgfqpoint{1.632052in}{0.331635in}}%
\pgfpathlineto{\pgfqpoint{1.647709in}{0.331635in}}%
\pgfpathlineto{\pgfqpoint{1.663365in}{0.331635in}}%
\pgfpathlineto{\pgfqpoint{1.679022in}{0.331635in}}%
\pgfpathlineto{\pgfqpoint{1.694678in}{0.331635in}}%
\pgfpathlineto{\pgfqpoint{1.710335in}{0.331635in}}%
\pgfpathlineto{\pgfqpoint{1.725992in}{0.331635in}}%
\pgfpathlineto{\pgfqpoint{1.741648in}{0.331635in}}%
\pgfpathlineto{\pgfqpoint{1.757305in}{0.331635in}}%
\pgfpathlineto{\pgfqpoint{1.772961in}{0.331635in}}%
\pgfpathlineto{\pgfqpoint{1.788618in}{0.331635in}}%
\pgfpathlineto{\pgfqpoint{1.804274in}{0.331635in}}%
\pgfpathlineto{\pgfqpoint{1.819931in}{0.331635in}}%
\pgfpathlineto{\pgfqpoint{1.835588in}{0.331635in}}%
\pgfpathlineto{\pgfqpoint{1.835588in}{0.345246in}}%
\pgfpathlineto{\pgfqpoint{1.835588in}{0.358857in}}%
\pgfpathlineto{\pgfqpoint{1.835588in}{0.372468in}}%
\pgfpathlineto{\pgfqpoint{1.835588in}{0.386079in}}%
\pgfpathlineto{\pgfqpoint{1.835588in}{0.399691in}}%
\pgfpathlineto{\pgfqpoint{1.835588in}{0.413302in}}%
\pgfpathlineto{\pgfqpoint{1.835588in}{0.426913in}}%
\pgfpathlineto{\pgfqpoint{1.835588in}{0.440524in}}%
\pgfpathlineto{\pgfqpoint{1.835588in}{0.454135in}}%
\pgfpathlineto{\pgfqpoint{1.835588in}{0.467746in}}%
\pgfpathlineto{\pgfqpoint{1.835588in}{0.481357in}}%
\pgfpathlineto{\pgfqpoint{1.835588in}{0.494968in}}%
\pgfpathlineto{\pgfqpoint{1.835588in}{0.508579in}}%
\pgfpathlineto{\pgfqpoint{1.835588in}{0.522191in}}%
\pgfpathlineto{\pgfqpoint{1.835588in}{0.535802in}}%
\pgfpathlineto{\pgfqpoint{1.835588in}{0.549413in}}%
\pgfpathlineto{\pgfqpoint{1.835588in}{0.563024in}}%
\pgfpathlineto{\pgfqpoint{1.835588in}{0.576635in}}%
\pgfpathlineto{\pgfqpoint{1.835588in}{0.590246in}}%
\pgfpathlineto{\pgfqpoint{1.835588in}{0.591635in}}%
\pgfpathlineto{\pgfqpoint{1.819931in}{0.591096in}}%
\pgfpathlineto{\pgfqpoint{1.811689in}{0.590246in}}%
\pgfpathlineto{\pgfqpoint{1.804274in}{0.589464in}}%
\pgfpathlineto{\pgfqpoint{1.788618in}{0.586729in}}%
\pgfpathlineto{\pgfqpoint{1.772961in}{0.582941in}}%
\pgfpathlineto{\pgfqpoint{1.757305in}{0.578141in}}%
\pgfpathlineto{\pgfqpoint{1.753233in}{0.576635in}}%
\pgfpathlineto{\pgfqpoint{1.741648in}{0.572250in}}%
\pgfpathlineto{\pgfqpoint{1.725992in}{0.565382in}}%
\pgfpathlineto{\pgfqpoint{1.721243in}{0.563024in}}%
\pgfpathlineto{\pgfqpoint{1.710335in}{0.557462in}}%
\pgfpathlineto{\pgfqpoint{1.696060in}{0.549413in}}%
\pgfpathlineto{\pgfqpoint{1.694678in}{0.548610in}}%
\pgfpathlineto{\pgfqpoint{1.679022in}{0.538664in}}%
\pgfpathlineto{\pgfqpoint{1.674838in}{0.535802in}}%
\pgfpathlineto{\pgfqpoint{1.663365in}{0.527676in}}%
\pgfpathlineto{\pgfqpoint{1.656093in}{0.522191in}}%
\pgfpathlineto{\pgfqpoint{1.647709in}{0.515604in}}%
\pgfpathlineto{\pgfqpoint{1.639239in}{0.508579in}}%
\pgfpathlineto{\pgfqpoint{1.632052in}{0.502331in}}%
\pgfpathlineto{\pgfqpoint{1.623972in}{0.494968in}}%
\pgfpathlineto{\pgfqpoint{1.616396in}{0.487679in}}%
\pgfpathlineto{\pgfqpoint{1.610086in}{0.481357in}}%
\pgfpathlineto{\pgfqpoint{1.600739in}{0.471384in}}%
\pgfpathlineto{\pgfqpoint{1.597447in}{0.467746in}}%
\pgfpathlineto{\pgfqpoint{1.586006in}{0.454135in}}%
\pgfpathlineto{\pgfqpoint{1.585082in}{0.452934in}}%
\pgfpathlineto{\pgfqpoint{1.575824in}{0.440524in}}%
\pgfpathlineto{\pgfqpoint{1.569426in}{0.431041in}}%
\pgfpathlineto{\pgfqpoint{1.566713in}{0.426913in}}%
\pgfpathlineto{\pgfqpoint{1.558813in}{0.413302in}}%
\pgfpathlineto{\pgfqpoint{1.553769in}{0.403230in}}%
\pgfpathlineto{\pgfqpoint{1.552037in}{0.399691in}}%
\pgfpathlineto{\pgfqpoint{1.546515in}{0.386079in}}%
\pgfpathlineto{\pgfqpoint{1.542158in}{0.372468in}}%
\pgfpathlineto{\pgfqpoint{1.539012in}{0.358857in}}%
\pgfpathlineto{\pgfqpoint{1.538113in}{0.352411in}}%
\pgfpathlineto{\pgfqpoint{1.537135in}{0.345246in}}%
\pgfpathlineto{\pgfqpoint{1.536515in}{0.331635in}}%
\pgfpathlineto{\pgfqpoint{1.538113in}{0.331635in}}%
\pgfpathclose%
\pgfpathmoveto{\pgfqpoint{0.301244in}{1.419674in}}%
\pgfpathlineto{\pgfqpoint{0.309486in}{1.420524in}}%
\pgfpathlineto{\pgfqpoint{0.316901in}{1.421306in}}%
\pgfpathlineto{\pgfqpoint{0.332557in}{1.424041in}}%
\pgfpathlineto{\pgfqpoint{0.348214in}{1.427829in}}%
\pgfpathlineto{\pgfqpoint{0.363870in}{1.432629in}}%
\pgfpathlineto{\pgfqpoint{0.367942in}{1.434135in}}%
\pgfpathlineto{\pgfqpoint{0.379527in}{1.438520in}}%
\pgfpathlineto{\pgfqpoint{0.395183in}{1.445388in}}%
\pgfpathlineto{\pgfqpoint{0.399932in}{1.447746in}}%
\pgfpathlineto{\pgfqpoint{0.410840in}{1.453308in}}%
\pgfpathlineto{\pgfqpoint{0.425115in}{1.461357in}}%
\pgfpathlineto{\pgfqpoint{0.426497in}{1.462160in}}%
\pgfpathlineto{\pgfqpoint{0.442153in}{1.472106in}}%
\pgfpathlineto{\pgfqpoint{0.446337in}{1.474968in}}%
\pgfpathlineto{\pgfqpoint{0.457810in}{1.483094in}}%
\pgfpathlineto{\pgfqpoint{0.465082in}{1.488579in}}%
\pgfpathlineto{\pgfqpoint{0.473466in}{1.495166in}}%
\pgfpathlineto{\pgfqpoint{0.481936in}{1.502191in}}%
\pgfpathlineto{\pgfqpoint{0.489123in}{1.508439in}}%
\pgfpathlineto{\pgfqpoint{0.497203in}{1.515802in}}%
\pgfpathlineto{\pgfqpoint{0.504779in}{1.523091in}}%
\pgfpathlineto{\pgfqpoint{0.511089in}{1.529413in}}%
\pgfpathlineto{\pgfqpoint{0.520436in}{1.539386in}}%
\pgfpathlineto{\pgfqpoint{0.523728in}{1.543024in}}%
\pgfpathlineto{\pgfqpoint{0.535169in}{1.556635in}}%
\pgfpathlineto{\pgfqpoint{0.536093in}{1.557836in}}%
\pgfpathlineto{\pgfqpoint{0.545351in}{1.570246in}}%
\pgfpathlineto{\pgfqpoint{0.551749in}{1.579729in}}%
\pgfpathlineto{\pgfqpoint{0.554462in}{1.583857in}}%
\pgfpathlineto{\pgfqpoint{0.562362in}{1.597468in}}%
\pgfpathlineto{\pgfqpoint{0.567406in}{1.607540in}}%
\pgfpathlineto{\pgfqpoint{0.569138in}{1.611079in}}%
\pgfpathlineto{\pgfqpoint{0.574660in}{1.624691in}}%
\pgfpathlineto{\pgfqpoint{0.579017in}{1.638302in}}%
\pgfpathlineto{\pgfqpoint{0.582163in}{1.651913in}}%
\pgfpathlineto{\pgfqpoint{0.583062in}{1.658359in}}%
\pgfpathlineto{\pgfqpoint{0.584040in}{1.665524in}}%
\pgfpathlineto{\pgfqpoint{0.584660in}{1.679135in}}%
\pgfpathlineto{\pgfqpoint{0.583062in}{1.679135in}}%
\pgfpathlineto{\pgfqpoint{0.567406in}{1.679135in}}%
\pgfpathlineto{\pgfqpoint{0.551749in}{1.679135in}}%
\pgfpathlineto{\pgfqpoint{0.536093in}{1.679135in}}%
\pgfpathlineto{\pgfqpoint{0.520436in}{1.679135in}}%
\pgfpathlineto{\pgfqpoint{0.504779in}{1.679135in}}%
\pgfpathlineto{\pgfqpoint{0.489123in}{1.679135in}}%
\pgfpathlineto{\pgfqpoint{0.473466in}{1.679135in}}%
\pgfpathlineto{\pgfqpoint{0.457810in}{1.679135in}}%
\pgfpathlineto{\pgfqpoint{0.442153in}{1.679135in}}%
\pgfpathlineto{\pgfqpoint{0.426497in}{1.679135in}}%
\pgfpathlineto{\pgfqpoint{0.410840in}{1.679135in}}%
\pgfpathlineto{\pgfqpoint{0.395183in}{1.679135in}}%
\pgfpathlineto{\pgfqpoint{0.379527in}{1.679135in}}%
\pgfpathlineto{\pgfqpoint{0.363870in}{1.679135in}}%
\pgfpathlineto{\pgfqpoint{0.348214in}{1.679135in}}%
\pgfpathlineto{\pgfqpoint{0.332557in}{1.679135in}}%
\pgfpathlineto{\pgfqpoint{0.316901in}{1.679135in}}%
\pgfpathlineto{\pgfqpoint{0.301244in}{1.679135in}}%
\pgfpathlineto{\pgfqpoint{0.285588in}{1.679135in}}%
\pgfpathlineto{\pgfqpoint{0.285588in}{1.665524in}}%
\pgfpathlineto{\pgfqpoint{0.285588in}{1.651913in}}%
\pgfpathlineto{\pgfqpoint{0.285588in}{1.638302in}}%
\pgfpathlineto{\pgfqpoint{0.285588in}{1.624691in}}%
\pgfpathlineto{\pgfqpoint{0.285588in}{1.611079in}}%
\pgfpathlineto{\pgfqpoint{0.285588in}{1.597468in}}%
\pgfpathlineto{\pgfqpoint{0.285588in}{1.583857in}}%
\pgfpathlineto{\pgfqpoint{0.285588in}{1.570246in}}%
\pgfpathlineto{\pgfqpoint{0.285588in}{1.556635in}}%
\pgfpathlineto{\pgfqpoint{0.285588in}{1.543024in}}%
\pgfpathlineto{\pgfqpoint{0.285588in}{1.529413in}}%
\pgfpathlineto{\pgfqpoint{0.285588in}{1.515802in}}%
\pgfpathlineto{\pgfqpoint{0.285588in}{1.502191in}}%
\pgfpathlineto{\pgfqpoint{0.285588in}{1.488579in}}%
\pgfpathlineto{\pgfqpoint{0.285588in}{1.474968in}}%
\pgfpathlineto{\pgfqpoint{0.285588in}{1.461357in}}%
\pgfpathlineto{\pgfqpoint{0.285588in}{1.447746in}}%
\pgfpathlineto{\pgfqpoint{0.285588in}{1.434135in}}%
\pgfpathlineto{\pgfqpoint{0.285588in}{1.420524in}}%
\pgfpathlineto{\pgfqpoint{0.285588in}{1.419135in}}%
\pgfpathlineto{\pgfqpoint{0.301244in}{1.419674in}}%
\pgfpathclose%
\pgfpathmoveto{\pgfqpoint{1.819931in}{1.419674in}}%
\pgfpathlineto{\pgfqpoint{1.835588in}{1.419135in}}%
\pgfpathlineto{\pgfqpoint{1.835588in}{1.420524in}}%
\pgfpathlineto{\pgfqpoint{1.835588in}{1.434135in}}%
\pgfpathlineto{\pgfqpoint{1.835588in}{1.447746in}}%
\pgfpathlineto{\pgfqpoint{1.835588in}{1.461357in}}%
\pgfpathlineto{\pgfqpoint{1.835588in}{1.474968in}}%
\pgfpathlineto{\pgfqpoint{1.835588in}{1.488579in}}%
\pgfpathlineto{\pgfqpoint{1.835588in}{1.502191in}}%
\pgfpathlineto{\pgfqpoint{1.835588in}{1.515802in}}%
\pgfpathlineto{\pgfqpoint{1.835588in}{1.529413in}}%
\pgfpathlineto{\pgfqpoint{1.835588in}{1.543024in}}%
\pgfpathlineto{\pgfqpoint{1.835588in}{1.556635in}}%
\pgfpathlineto{\pgfqpoint{1.835588in}{1.570246in}}%
\pgfpathlineto{\pgfqpoint{1.835588in}{1.583857in}}%
\pgfpathlineto{\pgfqpoint{1.835588in}{1.597468in}}%
\pgfpathlineto{\pgfqpoint{1.835588in}{1.611079in}}%
\pgfpathlineto{\pgfqpoint{1.835588in}{1.624691in}}%
\pgfpathlineto{\pgfqpoint{1.835588in}{1.638302in}}%
\pgfpathlineto{\pgfqpoint{1.835588in}{1.651913in}}%
\pgfpathlineto{\pgfqpoint{1.835588in}{1.665524in}}%
\pgfpathlineto{\pgfqpoint{1.835588in}{1.679135in}}%
\pgfpathlineto{\pgfqpoint{1.819931in}{1.679135in}}%
\pgfpathlineto{\pgfqpoint{1.804274in}{1.679135in}}%
\pgfpathlineto{\pgfqpoint{1.788618in}{1.679135in}}%
\pgfpathlineto{\pgfqpoint{1.772961in}{1.679135in}}%
\pgfpathlineto{\pgfqpoint{1.757305in}{1.679135in}}%
\pgfpathlineto{\pgfqpoint{1.741648in}{1.679135in}}%
\pgfpathlineto{\pgfqpoint{1.725992in}{1.679135in}}%
\pgfpathlineto{\pgfqpoint{1.710335in}{1.679135in}}%
\pgfpathlineto{\pgfqpoint{1.694678in}{1.679135in}}%
\pgfpathlineto{\pgfqpoint{1.679022in}{1.679135in}}%
\pgfpathlineto{\pgfqpoint{1.663365in}{1.679135in}}%
\pgfpathlineto{\pgfqpoint{1.647709in}{1.679135in}}%
\pgfpathlineto{\pgfqpoint{1.632052in}{1.679135in}}%
\pgfpathlineto{\pgfqpoint{1.616396in}{1.679135in}}%
\pgfpathlineto{\pgfqpoint{1.600739in}{1.679135in}}%
\pgfpathlineto{\pgfqpoint{1.585082in}{1.679135in}}%
\pgfpathlineto{\pgfqpoint{1.569426in}{1.679135in}}%
\pgfpathlineto{\pgfqpoint{1.553769in}{1.679135in}}%
\pgfpathlineto{\pgfqpoint{1.538113in}{1.679135in}}%
\pgfpathlineto{\pgfqpoint{1.536515in}{1.679135in}}%
\pgfpathlineto{\pgfqpoint{1.537135in}{1.665524in}}%
\pgfpathlineto{\pgfqpoint{1.538113in}{1.658359in}}%
\pgfpathlineto{\pgfqpoint{1.539012in}{1.651913in}}%
\pgfpathlineto{\pgfqpoint{1.542158in}{1.638302in}}%
\pgfpathlineto{\pgfqpoint{1.546515in}{1.624691in}}%
\pgfpathlineto{\pgfqpoint{1.552037in}{1.611079in}}%
\pgfpathlineto{\pgfqpoint{1.553769in}{1.607540in}}%
\pgfpathlineto{\pgfqpoint{1.558813in}{1.597468in}}%
\pgfpathlineto{\pgfqpoint{1.566713in}{1.583857in}}%
\pgfpathlineto{\pgfqpoint{1.569426in}{1.579729in}}%
\pgfpathlineto{\pgfqpoint{1.575824in}{1.570246in}}%
\pgfpathlineto{\pgfqpoint{1.585082in}{1.557836in}}%
\pgfpathlineto{\pgfqpoint{1.586006in}{1.556635in}}%
\pgfpathlineto{\pgfqpoint{1.597447in}{1.543024in}}%
\pgfpathlineto{\pgfqpoint{1.600739in}{1.539386in}}%
\pgfpathlineto{\pgfqpoint{1.610086in}{1.529413in}}%
\pgfpathlineto{\pgfqpoint{1.616396in}{1.523091in}}%
\pgfpathlineto{\pgfqpoint{1.623972in}{1.515802in}}%
\pgfpathlineto{\pgfqpoint{1.632052in}{1.508439in}}%
\pgfpathlineto{\pgfqpoint{1.639239in}{1.502191in}}%
\pgfpathlineto{\pgfqpoint{1.647709in}{1.495166in}}%
\pgfpathlineto{\pgfqpoint{1.656093in}{1.488579in}}%
\pgfpathlineto{\pgfqpoint{1.663365in}{1.483094in}}%
\pgfpathlineto{\pgfqpoint{1.674838in}{1.474968in}}%
\pgfpathlineto{\pgfqpoint{1.679022in}{1.472106in}}%
\pgfpathlineto{\pgfqpoint{1.694678in}{1.462160in}}%
\pgfpathlineto{\pgfqpoint{1.696060in}{1.461357in}}%
\pgfpathlineto{\pgfqpoint{1.710335in}{1.453308in}}%
\pgfpathlineto{\pgfqpoint{1.721243in}{1.447746in}}%
\pgfpathlineto{\pgfqpoint{1.725992in}{1.445388in}}%
\pgfpathlineto{\pgfqpoint{1.741648in}{1.438520in}}%
\pgfpathlineto{\pgfqpoint{1.753233in}{1.434135in}}%
\pgfpathlineto{\pgfqpoint{1.757305in}{1.432629in}}%
\pgfpathlineto{\pgfqpoint{1.772961in}{1.427829in}}%
\pgfpathlineto{\pgfqpoint{1.788618in}{1.424041in}}%
\pgfpathlineto{\pgfqpoint{1.804274in}{1.421306in}}%
\pgfpathlineto{\pgfqpoint{1.811689in}{1.420524in}}%
\pgfpathlineto{\pgfqpoint{1.819931in}{1.419674in}}%
\pgfpathclose%
\pgfusepath{fill}%
\end{pgfscope}%
\begin{pgfscope}%
\pgfsetbuttcap%
\pgfsetroundjoin%
\definecolor{currentfill}{rgb}{0.000000,0.000000,0.000000}%
\pgfsetfillcolor{currentfill}%
\pgfsetlinewidth{0.803000pt}%
\definecolor{currentstroke}{rgb}{0.000000,0.000000,0.000000}%
\pgfsetstrokecolor{currentstroke}%
\pgfsetdash{}{0pt}%
\pgfsys@defobject{currentmarker}{\pgfqpoint{0.000000in}{-0.048611in}}{\pgfqpoint{0.000000in}{0.000000in}}{%
\pgfpathmoveto{\pgfqpoint{0.000000in}{0.000000in}}%
\pgfpathlineto{\pgfqpoint{0.000000in}{-0.048611in}}%
\pgfusepath{stroke,fill}%
}%
\begin{pgfscope}%
\pgfsys@transformshift{0.285588in}{0.331635in}%
\pgfsys@useobject{currentmarker}{}%
\end{pgfscope}%
\end{pgfscope}%
\begin{pgfscope}%
\definecolor{textcolor}{rgb}{0.000000,0.000000,0.000000}%
\pgfsetstrokecolor{textcolor}%
\pgfsetfillcolor{textcolor}%
\pgftext[x=0.285588in,y=0.234413in,,top]{\color{textcolor}{\sffamily\fontsize{10.000000}{12.000000}\selectfont\catcode`\^=\active\def^{\ifmmode\sp\else\^{}\fi}\catcode`\%=\active\def%{\%}0}}%
\end{pgfscope}%
\begin{pgfscope}%
\pgfsetbuttcap%
\pgfsetroundjoin%
\definecolor{currentfill}{rgb}{0.000000,0.000000,0.000000}%
\pgfsetfillcolor{currentfill}%
\pgfsetlinewidth{0.803000pt}%
\definecolor{currentstroke}{rgb}{0.000000,0.000000,0.000000}%
\pgfsetstrokecolor{currentstroke}%
\pgfsetdash{}{0pt}%
\pgfsys@defobject{currentmarker}{\pgfqpoint{0.000000in}{-0.048611in}}{\pgfqpoint{0.000000in}{0.000000in}}{%
\pgfpathmoveto{\pgfqpoint{0.000000in}{0.000000in}}%
\pgfpathlineto{\pgfqpoint{0.000000in}{-0.048611in}}%
\pgfusepath{stroke,fill}%
}%
\begin{pgfscope}%
\pgfsys@transformshift{0.802254in}{0.331635in}%
\pgfsys@useobject{currentmarker}{}%
\end{pgfscope}%
\end{pgfscope}%
\begin{pgfscope}%
\definecolor{textcolor}{rgb}{0.000000,0.000000,0.000000}%
\pgfsetstrokecolor{textcolor}%
\pgfsetfillcolor{textcolor}%
\pgftext[x=0.802254in,y=0.234413in,,top]{\color{textcolor}{\sffamily\fontsize{10.000000}{12.000000}\selectfont\catcode`\^=\active\def^{\ifmmode\sp\else\^{}\fi}\catcode`\%=\active\def%{\%}2}}%
\end{pgfscope}%
\begin{pgfscope}%
\pgfsetbuttcap%
\pgfsetroundjoin%
\definecolor{currentfill}{rgb}{0.000000,0.000000,0.000000}%
\pgfsetfillcolor{currentfill}%
\pgfsetlinewidth{0.803000pt}%
\definecolor{currentstroke}{rgb}{0.000000,0.000000,0.000000}%
\pgfsetstrokecolor{currentstroke}%
\pgfsetdash{}{0pt}%
\pgfsys@defobject{currentmarker}{\pgfqpoint{0.000000in}{-0.048611in}}{\pgfqpoint{0.000000in}{0.000000in}}{%
\pgfpathmoveto{\pgfqpoint{0.000000in}{0.000000in}}%
\pgfpathlineto{\pgfqpoint{0.000000in}{-0.048611in}}%
\pgfusepath{stroke,fill}%
}%
\begin{pgfscope}%
\pgfsys@transformshift{1.318921in}{0.331635in}%
\pgfsys@useobject{currentmarker}{}%
\end{pgfscope}%
\end{pgfscope}%
\begin{pgfscope}%
\definecolor{textcolor}{rgb}{0.000000,0.000000,0.000000}%
\pgfsetstrokecolor{textcolor}%
\pgfsetfillcolor{textcolor}%
\pgftext[x=1.318921in,y=0.234413in,,top]{\color{textcolor}{\sffamily\fontsize{10.000000}{12.000000}\selectfont\catcode`\^=\active\def^{\ifmmode\sp\else\^{}\fi}\catcode`\%=\active\def%{\%}4}}%
\end{pgfscope}%
\begin{pgfscope}%
\pgfsetbuttcap%
\pgfsetroundjoin%
\definecolor{currentfill}{rgb}{0.000000,0.000000,0.000000}%
\pgfsetfillcolor{currentfill}%
\pgfsetlinewidth{0.803000pt}%
\definecolor{currentstroke}{rgb}{0.000000,0.000000,0.000000}%
\pgfsetstrokecolor{currentstroke}%
\pgfsetdash{}{0pt}%
\pgfsys@defobject{currentmarker}{\pgfqpoint{0.000000in}{-0.048611in}}{\pgfqpoint{0.000000in}{0.000000in}}{%
\pgfpathmoveto{\pgfqpoint{0.000000in}{0.000000in}}%
\pgfpathlineto{\pgfqpoint{0.000000in}{-0.048611in}}%
\pgfusepath{stroke,fill}%
}%
\begin{pgfscope}%
\pgfsys@transformshift{1.835588in}{0.331635in}%
\pgfsys@useobject{currentmarker}{}%
\end{pgfscope}%
\end{pgfscope}%
\begin{pgfscope}%
\definecolor{textcolor}{rgb}{0.000000,0.000000,0.000000}%
\pgfsetstrokecolor{textcolor}%
\pgfsetfillcolor{textcolor}%
\pgftext[x=1.835587in,y=0.234413in,,top]{\color{textcolor}{\sffamily\fontsize{10.000000}{12.000000}\selectfont\catcode`\^=\active\def^{\ifmmode\sp\else\^{}\fi}\catcode`\%=\active\def%{\%}6}}%
\end{pgfscope}%
\begin{pgfscope}%
\pgfsetbuttcap%
\pgfsetroundjoin%
\definecolor{currentfill}{rgb}{0.000000,0.000000,0.000000}%
\pgfsetfillcolor{currentfill}%
\pgfsetlinewidth{0.803000pt}%
\definecolor{currentstroke}{rgb}{0.000000,0.000000,0.000000}%
\pgfsetstrokecolor{currentstroke}%
\pgfsetdash{}{0pt}%
\pgfsys@defobject{currentmarker}{\pgfqpoint{-0.048611in}{0.000000in}}{\pgfqpoint{-0.000000in}{0.000000in}}{%
\pgfpathmoveto{\pgfqpoint{-0.000000in}{0.000000in}}%
\pgfpathlineto{\pgfqpoint{-0.048611in}{0.000000in}}%
\pgfusepath{stroke,fill}%
}%
\begin{pgfscope}%
\pgfsys@transformshift{0.285588in}{0.331635in}%
\pgfsys@useobject{currentmarker}{}%
\end{pgfscope}%
\end{pgfscope}%
\begin{pgfscope}%
\definecolor{textcolor}{rgb}{0.000000,0.000000,0.000000}%
\pgfsetstrokecolor{textcolor}%
\pgfsetfillcolor{textcolor}%
\pgftext[x=0.100000in, y=0.278873in, left, base]{\color{textcolor}{\sffamily\fontsize{10.000000}{12.000000}\selectfont\catcode`\^=\active\def^{\ifmmode\sp\else\^{}\fi}\catcode`\%=\active\def%{\%}0}}%
\end{pgfscope}%
\begin{pgfscope}%
\pgfsetbuttcap%
\pgfsetroundjoin%
\definecolor{currentfill}{rgb}{0.000000,0.000000,0.000000}%
\pgfsetfillcolor{currentfill}%
\pgfsetlinewidth{0.803000pt}%
\definecolor{currentstroke}{rgb}{0.000000,0.000000,0.000000}%
\pgfsetstrokecolor{currentstroke}%
\pgfsetdash{}{0pt}%
\pgfsys@defobject{currentmarker}{\pgfqpoint{-0.048611in}{0.000000in}}{\pgfqpoint{-0.000000in}{0.000000in}}{%
\pgfpathmoveto{\pgfqpoint{-0.000000in}{0.000000in}}%
\pgfpathlineto{\pgfqpoint{-0.048611in}{0.000000in}}%
\pgfusepath{stroke,fill}%
}%
\begin{pgfscope}%
\pgfsys@transformshift{0.285588in}{0.780802in}%
\pgfsys@useobject{currentmarker}{}%
\end{pgfscope}%
\end{pgfscope}%
\begin{pgfscope}%
\definecolor{textcolor}{rgb}{0.000000,0.000000,0.000000}%
\pgfsetstrokecolor{textcolor}%
\pgfsetfillcolor{textcolor}%
\pgftext[x=0.100000in, y=0.728040in, left, base]{\color{textcolor}{\sffamily\fontsize{10.000000}{12.000000}\selectfont\catcode`\^=\active\def^{\ifmmode\sp\else\^{}\fi}\catcode`\%=\active\def%{\%}2}}%
\end{pgfscope}%
\begin{pgfscope}%
\pgfsetbuttcap%
\pgfsetroundjoin%
\definecolor{currentfill}{rgb}{0.000000,0.000000,0.000000}%
\pgfsetfillcolor{currentfill}%
\pgfsetlinewidth{0.803000pt}%
\definecolor{currentstroke}{rgb}{0.000000,0.000000,0.000000}%
\pgfsetstrokecolor{currentstroke}%
\pgfsetdash{}{0pt}%
\pgfsys@defobject{currentmarker}{\pgfqpoint{-0.048611in}{0.000000in}}{\pgfqpoint{-0.000000in}{0.000000in}}{%
\pgfpathmoveto{\pgfqpoint{-0.000000in}{0.000000in}}%
\pgfpathlineto{\pgfqpoint{-0.048611in}{0.000000in}}%
\pgfusepath{stroke,fill}%
}%
\begin{pgfscope}%
\pgfsys@transformshift{0.285588in}{1.229968in}%
\pgfsys@useobject{currentmarker}{}%
\end{pgfscope}%
\end{pgfscope}%
\begin{pgfscope}%
\definecolor{textcolor}{rgb}{0.000000,0.000000,0.000000}%
\pgfsetstrokecolor{textcolor}%
\pgfsetfillcolor{textcolor}%
\pgftext[x=0.100000in, y=1.177207in, left, base]{\color{textcolor}{\sffamily\fontsize{10.000000}{12.000000}\selectfont\catcode`\^=\active\def^{\ifmmode\sp\else\^{}\fi}\catcode`\%=\active\def%{\%}4}}%
\end{pgfscope}%
\begin{pgfscope}%
\pgfsetbuttcap%
\pgfsetroundjoin%
\definecolor{currentfill}{rgb}{0.000000,0.000000,0.000000}%
\pgfsetfillcolor{currentfill}%
\pgfsetlinewidth{0.803000pt}%
\definecolor{currentstroke}{rgb}{0.000000,0.000000,0.000000}%
\pgfsetstrokecolor{currentstroke}%
\pgfsetdash{}{0pt}%
\pgfsys@defobject{currentmarker}{\pgfqpoint{-0.048611in}{0.000000in}}{\pgfqpoint{-0.000000in}{0.000000in}}{%
\pgfpathmoveto{\pgfqpoint{-0.000000in}{0.000000in}}%
\pgfpathlineto{\pgfqpoint{-0.048611in}{0.000000in}}%
\pgfusepath{stroke,fill}%
}%
\begin{pgfscope}%
\pgfsys@transformshift{0.285588in}{1.679135in}%
\pgfsys@useobject{currentmarker}{}%
\end{pgfscope}%
\end{pgfscope}%
\begin{pgfscope}%
\definecolor{textcolor}{rgb}{0.000000,0.000000,0.000000}%
\pgfsetstrokecolor{textcolor}%
\pgfsetfillcolor{textcolor}%
\pgftext[x=0.100000in, y=1.626373in, left, base]{\color{textcolor}{\sffamily\fontsize{10.000000}{12.000000}\selectfont\catcode`\^=\active\def^{\ifmmode\sp\else\^{}\fi}\catcode`\%=\active\def%{\%}6}}%
\end{pgfscope}%
\begin{pgfscope}%
\pgfsetrectcap%
\pgfsetmiterjoin%
\pgfsetlinewidth{0.803000pt}%
\definecolor{currentstroke}{rgb}{0.000000,0.000000,0.000000}%
\pgfsetstrokecolor{currentstroke}%
\pgfsetdash{}{0pt}%
\pgfpathmoveto{\pgfqpoint{0.285588in}{0.331635in}}%
\pgfpathlineto{\pgfqpoint{0.285588in}{1.679135in}}%
\pgfusepath{stroke}%
\end{pgfscope}%
\begin{pgfscope}%
\pgfsetrectcap%
\pgfsetmiterjoin%
\pgfsetlinewidth{0.803000pt}%
\definecolor{currentstroke}{rgb}{0.000000,0.000000,0.000000}%
\pgfsetstrokecolor{currentstroke}%
\pgfsetdash{}{0pt}%
\pgfpathmoveto{\pgfqpoint{1.835588in}{0.331635in}}%
\pgfpathlineto{\pgfqpoint{1.835588in}{1.679135in}}%
\pgfusepath{stroke}%
\end{pgfscope}%
\begin{pgfscope}%
\pgfsetrectcap%
\pgfsetmiterjoin%
\pgfsetlinewidth{0.803000pt}%
\definecolor{currentstroke}{rgb}{0.000000,0.000000,0.000000}%
\pgfsetstrokecolor{currentstroke}%
\pgfsetdash{}{0pt}%
\pgfpathmoveto{\pgfqpoint{0.285588in}{0.331635in}}%
\pgfpathlineto{\pgfqpoint{1.835588in}{0.331635in}}%
\pgfusepath{stroke}%
\end{pgfscope}%
\begin{pgfscope}%
\pgfsetrectcap%
\pgfsetmiterjoin%
\pgfsetlinewidth{0.803000pt}%
\definecolor{currentstroke}{rgb}{0.000000,0.000000,0.000000}%
\pgfsetstrokecolor{currentstroke}%
\pgfsetdash{}{0pt}%
\pgfpathmoveto{\pgfqpoint{0.285588in}{1.679135in}}%
\pgfpathlineto{\pgfqpoint{1.835588in}{1.679135in}}%
\pgfusepath{stroke}%
\end{pgfscope}%
\end{pgfpicture}%
\makeatother%
\endgroup%

        \caption{$c=1$}
        \label{fig:5-experiments-periodic-gaussian-well-1}
    \end{subfigure}
    \begin{subfigure}[b]{0.32\columnwidth}
        %% Creator: Matplotlib, PGF backend
%%
%% To include the figure in your LaTeX document, write
%%   \input{<filename>.pgf}
%%
%% Make sure the required packages are loaded in your preamble
%%   \usepackage{pgf}
%%
%% Also ensure that all the required font packages are loaded; for instance,
%% the lmodern package is sometimes necessary when using math font.
%%   \usepackage{lmodern}
%%
%% Figures using additional raster images can only be included by \input if
%% they are in the same directory as the main LaTeX file. For loading figures
%% from other directories you can use the `import` package
%%   \usepackage{import}
%%
%% and then include the figures with
%%   \import{<path to file>}{<filename>.pgf}
%%
%% Matplotlib used the following preamble
%%   \def\mathdefault#1{#1}
%%   \everymath=\expandafter{\the\everymath\displaystyle}
%%   
%%   \usepackage{fontspec}
%%   \setmainfont{DejaVuSerif.ttf}[Path=\detokenize{C:/Users/fabio/Documents/Work/MasterThesis/Rand-SD/.venv/Lib/site-packages/matplotlib/mpl-data/fonts/ttf/}]
%%   \setsansfont{DejaVuSans.ttf}[Path=\detokenize{C:/Users/fabio/Documents/Work/MasterThesis/Rand-SD/.venv/Lib/site-packages/matplotlib/mpl-data/fonts/ttf/}]
%%   \setmonofont{DejaVuSansMono.ttf}[Path=\detokenize{C:/Users/fabio/Documents/Work/MasterThesis/Rand-SD/.venv/Lib/site-packages/matplotlib/mpl-data/fonts/ttf/}]
%%   \makeatletter\@ifpackageloaded{underscore}{}{\usepackage[strings]{underscore}}\makeatother
%%
\begingroup%
\makeatletter%
\begin{pgfpicture}%
\pgfpathrectangle{\pgfpointorigin}{\pgfqpoint{2.023953in}{1.779135in}}%
\pgfusepath{use as bounding box, clip}%
\begin{pgfscope}%
\pgfsetbuttcap%
\pgfsetmiterjoin%
\definecolor{currentfill}{rgb}{1.000000,1.000000,1.000000}%
\pgfsetfillcolor{currentfill}%
\pgfsetlinewidth{0.000000pt}%
\definecolor{currentstroke}{rgb}{1.000000,1.000000,1.000000}%
\pgfsetstrokecolor{currentstroke}%
\pgfsetdash{}{0pt}%
\pgfpathmoveto{\pgfqpoint{0.000000in}{0.000000in}}%
\pgfpathlineto{\pgfqpoint{2.023953in}{0.000000in}}%
\pgfpathlineto{\pgfqpoint{2.023953in}{1.779135in}}%
\pgfpathlineto{\pgfqpoint{0.000000in}{1.779135in}}%
\pgfpathlineto{\pgfqpoint{0.000000in}{0.000000in}}%
\pgfpathclose%
\pgfusepath{fill}%
\end{pgfscope}%
\begin{pgfscope}%
\pgfsetbuttcap%
\pgfsetmiterjoin%
\definecolor{currentfill}{rgb}{1.000000,1.000000,1.000000}%
\pgfsetfillcolor{currentfill}%
\pgfsetlinewidth{0.000000pt}%
\definecolor{currentstroke}{rgb}{0.000000,0.000000,0.000000}%
\pgfsetstrokecolor{currentstroke}%
\pgfsetstrokeopacity{0.000000}%
\pgfsetdash{}{0pt}%
\pgfpathmoveto{\pgfqpoint{0.373953in}{0.331635in}}%
\pgfpathlineto{\pgfqpoint{1.923953in}{0.331635in}}%
\pgfpathlineto{\pgfqpoint{1.923953in}{1.679135in}}%
\pgfpathlineto{\pgfqpoint{0.373953in}{1.679135in}}%
\pgfpathlineto{\pgfqpoint{0.373953in}{0.331635in}}%
\pgfpathclose%
\pgfusepath{fill}%
\end{pgfscope}%
\begin{pgfscope}%
\pgfpathrectangle{\pgfqpoint{0.373953in}{0.331635in}}{\pgfqpoint{1.550000in}{1.347500in}}%
\pgfusepath{clip}%
\pgfsetbuttcap%
\pgfsetroundjoin%
\definecolor{currentfill}{rgb}{0.993545,0.862859,0.619299}%
\pgfsetfillcolor{currentfill}%
\pgfsetlinewidth{0.000000pt}%
\definecolor{currentstroke}{rgb}{0.000000,0.000000,0.000000}%
\pgfsetstrokecolor{currentstroke}%
\pgfsetdash{}{0pt}%
\pgfpathmoveto{\pgfqpoint{0.734054in}{0.597840in}}%
\pgfpathlineto{\pgfqpoint{0.749710in}{0.594489in}}%
\pgfpathlineto{\pgfqpoint{0.765367in}{0.593819in}}%
\pgfpathlineto{\pgfqpoint{0.781024in}{0.595830in}}%
\pgfpathlineto{\pgfqpoint{0.796680in}{0.600522in}}%
\pgfpathlineto{\pgfqpoint{0.803786in}{0.603857in}}%
\pgfpathlineto{\pgfqpoint{0.812337in}{0.608724in}}%
\pgfpathlineto{\pgfqpoint{0.823668in}{0.617468in}}%
\pgfpathlineto{\pgfqpoint{0.827993in}{0.621790in}}%
\pgfpathlineto{\pgfqpoint{0.835437in}{0.631079in}}%
\pgfpathlineto{\pgfqpoint{0.843005in}{0.644691in}}%
\pgfpathlineto{\pgfqpoint{0.843650in}{0.646775in}}%
\pgfpathlineto{\pgfqpoint{0.846675in}{0.658302in}}%
\pgfpathlineto{\pgfqpoint{0.847390in}{0.671913in}}%
\pgfpathlineto{\pgfqpoint{0.845246in}{0.685524in}}%
\pgfpathlineto{\pgfqpoint{0.843650in}{0.689878in}}%
\pgfpathlineto{\pgfqpoint{0.839642in}{0.699135in}}%
\pgfpathlineto{\pgfqpoint{0.830389in}{0.712746in}}%
\pgfpathlineto{\pgfqpoint{0.827993in}{0.715351in}}%
\pgfpathlineto{\pgfqpoint{0.815333in}{0.726357in}}%
\pgfpathlineto{\pgfqpoint{0.812337in}{0.728440in}}%
\pgfpathlineto{\pgfqpoint{0.796680in}{0.736484in}}%
\pgfpathlineto{\pgfqpoint{0.786032in}{0.739968in}}%
\pgfpathlineto{\pgfqpoint{0.781024in}{0.741356in}}%
\pgfpathlineto{\pgfqpoint{0.765367in}{0.743220in}}%
\pgfpathlineto{\pgfqpoint{0.749710in}{0.742598in}}%
\pgfpathlineto{\pgfqpoint{0.736451in}{0.739968in}}%
\pgfpathlineto{\pgfqpoint{0.734054in}{0.739407in}}%
\pgfpathlineto{\pgfqpoint{0.718397in}{0.732829in}}%
\pgfpathlineto{\pgfqpoint{0.707712in}{0.726357in}}%
\pgfpathlineto{\pgfqpoint{0.702741in}{0.722597in}}%
\pgfpathlineto{\pgfqpoint{0.692682in}{0.712746in}}%
\pgfpathlineto{\pgfqpoint{0.687084in}{0.705312in}}%
\pgfpathlineto{\pgfqpoint{0.683248in}{0.699135in}}%
\pgfpathlineto{\pgfqpoint{0.677850in}{0.685524in}}%
\pgfpathlineto{\pgfqpoint{0.675538in}{0.671913in}}%
\pgfpathlineto{\pgfqpoint{0.676309in}{0.658302in}}%
\pgfpathlineto{\pgfqpoint{0.680163in}{0.644691in}}%
\pgfpathlineto{\pgfqpoint{0.687084in}{0.631122in}}%
\pgfpathlineto{\pgfqpoint{0.687111in}{0.631079in}}%
\pgfpathlineto{\pgfqpoint{0.699186in}{0.617468in}}%
\pgfpathlineto{\pgfqpoint{0.702741in}{0.614378in}}%
\pgfpathlineto{\pgfqpoint{0.718397in}{0.603880in}}%
\pgfpathlineto{\pgfqpoint{0.718447in}{0.603857in}}%
\pgfpathlineto{\pgfqpoint{0.734054in}{0.597840in}}%
\pgfpathclose%
\pgfpathmoveto{\pgfqpoint{1.501226in}{0.600522in}}%
\pgfpathlineto{\pgfqpoint{1.516882in}{0.595830in}}%
\pgfpathlineto{\pgfqpoint{1.532539in}{0.593819in}}%
\pgfpathlineto{\pgfqpoint{1.548195in}{0.594489in}}%
\pgfpathlineto{\pgfqpoint{1.563852in}{0.597840in}}%
\pgfpathlineto{\pgfqpoint{1.579459in}{0.603857in}}%
\pgfpathlineto{\pgfqpoint{1.579508in}{0.603880in}}%
\pgfpathlineto{\pgfqpoint{1.595165in}{0.614378in}}%
\pgfpathlineto{\pgfqpoint{1.598720in}{0.617468in}}%
\pgfpathlineto{\pgfqpoint{1.610795in}{0.631079in}}%
\pgfpathlineto{\pgfqpoint{1.610822in}{0.631122in}}%
\pgfpathlineto{\pgfqpoint{1.617743in}{0.644691in}}%
\pgfpathlineto{\pgfqpoint{1.621597in}{0.658302in}}%
\pgfpathlineto{\pgfqpoint{1.622368in}{0.671913in}}%
\pgfpathlineto{\pgfqpoint{1.620056in}{0.685524in}}%
\pgfpathlineto{\pgfqpoint{1.614658in}{0.699135in}}%
\pgfpathlineto{\pgfqpoint{1.610822in}{0.705312in}}%
\pgfpathlineto{\pgfqpoint{1.605223in}{0.712746in}}%
\pgfpathlineto{\pgfqpoint{1.595165in}{0.722597in}}%
\pgfpathlineto{\pgfqpoint{1.590194in}{0.726357in}}%
\pgfpathlineto{\pgfqpoint{1.579508in}{0.732829in}}%
\pgfpathlineto{\pgfqpoint{1.563852in}{0.739407in}}%
\pgfpathlineto{\pgfqpoint{1.561454in}{0.739968in}}%
\pgfpathlineto{\pgfqpoint{1.548195in}{0.742598in}}%
\pgfpathlineto{\pgfqpoint{1.532539in}{0.743220in}}%
\pgfpathlineto{\pgfqpoint{1.516882in}{0.741356in}}%
\pgfpathlineto{\pgfqpoint{1.511873in}{0.739968in}}%
\pgfpathlineto{\pgfqpoint{1.501226in}{0.736484in}}%
\pgfpathlineto{\pgfqpoint{1.485569in}{0.728440in}}%
\pgfpathlineto{\pgfqpoint{1.482573in}{0.726357in}}%
\pgfpathlineto{\pgfqpoint{1.469913in}{0.715351in}}%
\pgfpathlineto{\pgfqpoint{1.467517in}{0.712746in}}%
\pgfpathlineto{\pgfqpoint{1.458264in}{0.699135in}}%
\pgfpathlineto{\pgfqpoint{1.454256in}{0.689878in}}%
\pgfpathlineto{\pgfqpoint{1.452660in}{0.685524in}}%
\pgfpathlineto{\pgfqpoint{1.450516in}{0.671913in}}%
\pgfpathlineto{\pgfqpoint{1.451231in}{0.658302in}}%
\pgfpathlineto{\pgfqpoint{1.454256in}{0.646775in}}%
\pgfpathlineto{\pgfqpoint{1.454901in}{0.644691in}}%
\pgfpathlineto{\pgfqpoint{1.462468in}{0.631079in}}%
\pgfpathlineto{\pgfqpoint{1.469913in}{0.621790in}}%
\pgfpathlineto{\pgfqpoint{1.474238in}{0.617468in}}%
\pgfpathlineto{\pgfqpoint{1.485569in}{0.608724in}}%
\pgfpathlineto{\pgfqpoint{1.494120in}{0.603857in}}%
\pgfpathlineto{\pgfqpoint{1.501226in}{0.600522in}}%
\pgfpathclose%
\pgfpathmoveto{\pgfqpoint{0.749710in}{1.268172in}}%
\pgfpathlineto{\pgfqpoint{0.765367in}{1.267550in}}%
\pgfpathlineto{\pgfqpoint{0.781024in}{1.269414in}}%
\pgfpathlineto{\pgfqpoint{0.786032in}{1.270802in}}%
\pgfpathlineto{\pgfqpoint{0.796680in}{1.274286in}}%
\pgfpathlineto{\pgfqpoint{0.812337in}{1.282330in}}%
\pgfpathlineto{\pgfqpoint{0.815333in}{1.284413in}}%
\pgfpathlineto{\pgfqpoint{0.827993in}{1.295419in}}%
\pgfpathlineto{\pgfqpoint{0.830389in}{1.298024in}}%
\pgfpathlineto{\pgfqpoint{0.839642in}{1.311635in}}%
\pgfpathlineto{\pgfqpoint{0.843650in}{1.320892in}}%
\pgfpathlineto{\pgfqpoint{0.845246in}{1.325246in}}%
\pgfpathlineto{\pgfqpoint{0.847390in}{1.338857in}}%
\pgfpathlineto{\pgfqpoint{0.846675in}{1.352468in}}%
\pgfpathlineto{\pgfqpoint{0.843650in}{1.363995in}}%
\pgfpathlineto{\pgfqpoint{0.843005in}{1.366079in}}%
\pgfpathlineto{\pgfqpoint{0.835437in}{1.379691in}}%
\pgfpathlineto{\pgfqpoint{0.827993in}{1.388980in}}%
\pgfpathlineto{\pgfqpoint{0.823668in}{1.393302in}}%
\pgfpathlineto{\pgfqpoint{0.812337in}{1.402046in}}%
\pgfpathlineto{\pgfqpoint{0.803786in}{1.406913in}}%
\pgfpathlineto{\pgfqpoint{0.796680in}{1.410248in}}%
\pgfpathlineto{\pgfqpoint{0.781024in}{1.414940in}}%
\pgfpathlineto{\pgfqpoint{0.765367in}{1.416951in}}%
\pgfpathlineto{\pgfqpoint{0.749710in}{1.416281in}}%
\pgfpathlineto{\pgfqpoint{0.734054in}{1.412930in}}%
\pgfpathlineto{\pgfqpoint{0.718447in}{1.406913in}}%
\pgfpathlineto{\pgfqpoint{0.718397in}{1.406890in}}%
\pgfpathlineto{\pgfqpoint{0.702741in}{1.396392in}}%
\pgfpathlineto{\pgfqpoint{0.699186in}{1.393302in}}%
\pgfpathlineto{\pgfqpoint{0.687111in}{1.379691in}}%
\pgfpathlineto{\pgfqpoint{0.687084in}{1.379648in}}%
\pgfpathlineto{\pgfqpoint{0.680163in}{1.366079in}}%
\pgfpathlineto{\pgfqpoint{0.676309in}{1.352468in}}%
\pgfpathlineto{\pgfqpoint{0.675538in}{1.338857in}}%
\pgfpathlineto{\pgfqpoint{0.677850in}{1.325246in}}%
\pgfpathlineto{\pgfqpoint{0.683248in}{1.311635in}}%
\pgfpathlineto{\pgfqpoint{0.687084in}{1.305458in}}%
\pgfpathlineto{\pgfqpoint{0.692682in}{1.298024in}}%
\pgfpathlineto{\pgfqpoint{0.702741in}{1.288173in}}%
\pgfpathlineto{\pgfqpoint{0.707712in}{1.284413in}}%
\pgfpathlineto{\pgfqpoint{0.718397in}{1.277941in}}%
\pgfpathlineto{\pgfqpoint{0.734054in}{1.271363in}}%
\pgfpathlineto{\pgfqpoint{0.736451in}{1.270802in}}%
\pgfpathlineto{\pgfqpoint{0.749710in}{1.268172in}}%
\pgfpathclose%
\pgfpathmoveto{\pgfqpoint{1.516882in}{1.269414in}}%
\pgfpathlineto{\pgfqpoint{1.532539in}{1.267550in}}%
\pgfpathlineto{\pgfqpoint{1.548195in}{1.268172in}}%
\pgfpathlineto{\pgfqpoint{1.561454in}{1.270802in}}%
\pgfpathlineto{\pgfqpoint{1.563852in}{1.271363in}}%
\pgfpathlineto{\pgfqpoint{1.579508in}{1.277941in}}%
\pgfpathlineto{\pgfqpoint{1.590194in}{1.284413in}}%
\pgfpathlineto{\pgfqpoint{1.595165in}{1.288173in}}%
\pgfpathlineto{\pgfqpoint{1.605223in}{1.298024in}}%
\pgfpathlineto{\pgfqpoint{1.610822in}{1.305458in}}%
\pgfpathlineto{\pgfqpoint{1.614658in}{1.311635in}}%
\pgfpathlineto{\pgfqpoint{1.620056in}{1.325246in}}%
\pgfpathlineto{\pgfqpoint{1.622368in}{1.338857in}}%
\pgfpathlineto{\pgfqpoint{1.621597in}{1.352468in}}%
\pgfpathlineto{\pgfqpoint{1.617743in}{1.366079in}}%
\pgfpathlineto{\pgfqpoint{1.610822in}{1.379648in}}%
\pgfpathlineto{\pgfqpoint{1.610795in}{1.379691in}}%
\pgfpathlineto{\pgfqpoint{1.598720in}{1.393302in}}%
\pgfpathlineto{\pgfqpoint{1.595165in}{1.396392in}}%
\pgfpathlineto{\pgfqpoint{1.579508in}{1.406890in}}%
\pgfpathlineto{\pgfqpoint{1.579459in}{1.406913in}}%
\pgfpathlineto{\pgfqpoint{1.563852in}{1.412930in}}%
\pgfpathlineto{\pgfqpoint{1.548195in}{1.416281in}}%
\pgfpathlineto{\pgfqpoint{1.532539in}{1.416951in}}%
\pgfpathlineto{\pgfqpoint{1.516882in}{1.414940in}}%
\pgfpathlineto{\pgfqpoint{1.501226in}{1.410248in}}%
\pgfpathlineto{\pgfqpoint{1.494120in}{1.406913in}}%
\pgfpathlineto{\pgfqpoint{1.485569in}{1.402046in}}%
\pgfpathlineto{\pgfqpoint{1.474238in}{1.393302in}}%
\pgfpathlineto{\pgfqpoint{1.469913in}{1.388980in}}%
\pgfpathlineto{\pgfqpoint{1.462468in}{1.379691in}}%
\pgfpathlineto{\pgfqpoint{1.454901in}{1.366079in}}%
\pgfpathlineto{\pgfqpoint{1.454256in}{1.363995in}}%
\pgfpathlineto{\pgfqpoint{1.451231in}{1.352468in}}%
\pgfpathlineto{\pgfqpoint{1.450516in}{1.338857in}}%
\pgfpathlineto{\pgfqpoint{1.452660in}{1.325246in}}%
\pgfpathlineto{\pgfqpoint{1.454256in}{1.320892in}}%
\pgfpathlineto{\pgfqpoint{1.458264in}{1.311635in}}%
\pgfpathlineto{\pgfqpoint{1.467517in}{1.298024in}}%
\pgfpathlineto{\pgfqpoint{1.469913in}{1.295419in}}%
\pgfpathlineto{\pgfqpoint{1.482573in}{1.284413in}}%
\pgfpathlineto{\pgfqpoint{1.485569in}{1.282330in}}%
\pgfpathlineto{\pgfqpoint{1.501226in}{1.274286in}}%
\pgfpathlineto{\pgfqpoint{1.511873in}{1.270802in}}%
\pgfpathlineto{\pgfqpoint{1.516882in}{1.269414in}}%
\pgfpathclose%
\pgfusepath{fill}%
\end{pgfscope}%
\begin{pgfscope}%
\pgfpathrectangle{\pgfqpoint{0.373953in}{0.331635in}}{\pgfqpoint{1.550000in}{1.347500in}}%
\pgfusepath{clip}%
\pgfsetbuttcap%
\pgfsetroundjoin%
\definecolor{currentfill}{rgb}{0.993326,0.602275,0.414390}%
\pgfsetfillcolor{currentfill}%
\pgfsetlinewidth{0.000000pt}%
\definecolor{currentstroke}{rgb}{0.000000,0.000000,0.000000}%
\pgfsetstrokecolor{currentstroke}%
\pgfsetdash{}{0pt}%
\pgfpathmoveto{\pgfqpoint{0.718397in}{0.532207in}}%
\pgfpathlineto{\pgfqpoint{0.734054in}{0.528713in}}%
\pgfpathlineto{\pgfqpoint{0.749710in}{0.526773in}}%
\pgfpathlineto{\pgfqpoint{0.765367in}{0.526385in}}%
\pgfpathlineto{\pgfqpoint{0.781024in}{0.527549in}}%
\pgfpathlineto{\pgfqpoint{0.796680in}{0.530266in}}%
\pgfpathlineto{\pgfqpoint{0.812337in}{0.534538in}}%
\pgfpathlineto{\pgfqpoint{0.815758in}{0.535802in}}%
\pgfpathlineto{\pgfqpoint{0.827993in}{0.540572in}}%
\pgfpathlineto{\pgfqpoint{0.843650in}{0.548293in}}%
\pgfpathlineto{\pgfqpoint{0.845552in}{0.549413in}}%
\pgfpathlineto{\pgfqpoint{0.859306in}{0.558172in}}%
\pgfpathlineto{\pgfqpoint{0.865884in}{0.563024in}}%
\pgfpathlineto{\pgfqpoint{0.874963in}{0.570446in}}%
\pgfpathlineto{\pgfqpoint{0.881704in}{0.576635in}}%
\pgfpathlineto{\pgfqpoint{0.890620in}{0.585964in}}%
\pgfpathlineto{\pgfqpoint{0.894358in}{0.590246in}}%
\pgfpathlineto{\pgfqpoint{0.904462in}{0.603857in}}%
\pgfpathlineto{\pgfqpoint{0.906276in}{0.606849in}}%
\pgfpathlineto{\pgfqpoint{0.912311in}{0.617468in}}%
\pgfpathlineto{\pgfqpoint{0.918235in}{0.631079in}}%
\pgfpathlineto{\pgfqpoint{0.921933in}{0.643352in}}%
\pgfpathlineto{\pgfqpoint{0.922320in}{0.644691in}}%
\pgfpathlineto{\pgfqpoint{0.924518in}{0.658302in}}%
\pgfpathlineto{\pgfqpoint{0.924957in}{0.671913in}}%
\pgfpathlineto{\pgfqpoint{0.923639in}{0.685524in}}%
\pgfpathlineto{\pgfqpoint{0.921933in}{0.693085in}}%
\pgfpathlineto{\pgfqpoint{0.920512in}{0.699135in}}%
\pgfpathlineto{\pgfqpoint{0.915502in}{0.712746in}}%
\pgfpathlineto{\pgfqpoint{0.908662in}{0.726357in}}%
\pgfpathlineto{\pgfqpoint{0.906276in}{0.730139in}}%
\pgfpathlineto{\pgfqpoint{0.899652in}{0.739968in}}%
\pgfpathlineto{\pgfqpoint{0.890620in}{0.751109in}}%
\pgfpathlineto{\pgfqpoint{0.888430in}{0.753579in}}%
\pgfpathlineto{\pgfqpoint{0.874963in}{0.766714in}}%
\pgfpathlineto{\pgfqpoint{0.874415in}{0.767191in}}%
\pgfpathlineto{\pgfqpoint{0.859306in}{0.778898in}}%
\pgfpathlineto{\pgfqpoint{0.856465in}{0.780802in}}%
\pgfpathlineto{\pgfqpoint{0.843650in}{0.788654in}}%
\pgfpathlineto{\pgfqpoint{0.832344in}{0.794413in}}%
\pgfpathlineto{\pgfqpoint{0.827993in}{0.796487in}}%
\pgfpathlineto{\pgfqpoint{0.812337in}{0.802433in}}%
\pgfpathlineto{\pgfqpoint{0.796680in}{0.806789in}}%
\pgfpathlineto{\pgfqpoint{0.789721in}{0.808024in}}%
\pgfpathlineto{\pgfqpoint{0.781024in}{0.809507in}}%
\pgfpathlineto{\pgfqpoint{0.765367in}{0.810653in}}%
\pgfpathlineto{\pgfqpoint{0.749710in}{0.810271in}}%
\pgfpathlineto{\pgfqpoint{0.734054in}{0.808360in}}%
\pgfpathlineto{\pgfqpoint{0.732514in}{0.808024in}}%
\pgfpathlineto{\pgfqpoint{0.718397in}{0.804810in}}%
\pgfpathlineto{\pgfqpoint{0.702741in}{0.799659in}}%
\pgfpathlineto{\pgfqpoint{0.690525in}{0.794413in}}%
\pgfpathlineto{\pgfqpoint{0.687084in}{0.792836in}}%
\pgfpathlineto{\pgfqpoint{0.671428in}{0.784052in}}%
\pgfpathlineto{\pgfqpoint{0.666502in}{0.780802in}}%
\pgfpathlineto{\pgfqpoint{0.655771in}{0.773051in}}%
\pgfpathlineto{\pgfqpoint{0.648652in}{0.767191in}}%
\pgfpathlineto{\pgfqpoint{0.640115in}{0.759298in}}%
\pgfpathlineto{\pgfqpoint{0.634534in}{0.753579in}}%
\pgfpathlineto{\pgfqpoint{0.624458in}{0.741622in}}%
\pgfpathlineto{\pgfqpoint{0.623170in}{0.739968in}}%
\pgfpathlineto{\pgfqpoint{0.614289in}{0.726357in}}%
\pgfpathlineto{\pgfqpoint{0.608801in}{0.715720in}}%
\pgfpathlineto{\pgfqpoint{0.607348in}{0.712746in}}%
\pgfpathlineto{\pgfqpoint{0.602433in}{0.699135in}}%
\pgfpathlineto{\pgfqpoint{0.599309in}{0.685524in}}%
\pgfpathlineto{\pgfqpoint{0.597970in}{0.671913in}}%
\pgfpathlineto{\pgfqpoint{0.598416in}{0.658302in}}%
\pgfpathlineto{\pgfqpoint{0.600648in}{0.644691in}}%
\pgfpathlineto{\pgfqpoint{0.604667in}{0.631079in}}%
\pgfpathlineto{\pgfqpoint{0.608801in}{0.621368in}}%
\pgfpathlineto{\pgfqpoint{0.610552in}{0.617468in}}%
\pgfpathlineto{\pgfqpoint{0.618495in}{0.603857in}}%
\pgfpathlineto{\pgfqpoint{0.624458in}{0.595542in}}%
\pgfpathlineto{\pgfqpoint{0.628562in}{0.590246in}}%
\pgfpathlineto{\pgfqpoint{0.640115in}{0.577590in}}%
\pgfpathlineto{\pgfqpoint{0.641080in}{0.576635in}}%
\pgfpathlineto{\pgfqpoint{0.655771in}{0.563863in}}%
\pgfpathlineto{\pgfqpoint{0.656870in}{0.563024in}}%
\pgfpathlineto{\pgfqpoint{0.671428in}{0.552980in}}%
\pgfpathlineto{\pgfqpoint{0.677520in}{0.549413in}}%
\pgfpathlineto{\pgfqpoint{0.687084in}{0.544229in}}%
\pgfpathlineto{\pgfqpoint{0.702741in}{0.537323in}}%
\pgfpathlineto{\pgfqpoint{0.707227in}{0.535802in}}%
\pgfpathlineto{\pgfqpoint{0.718397in}{0.532207in}}%
\pgfpathclose%
\pgfpathmoveto{\pgfqpoint{0.718447in}{0.603857in}}%
\pgfpathlineto{\pgfqpoint{0.718397in}{0.603880in}}%
\pgfpathlineto{\pgfqpoint{0.702741in}{0.614378in}}%
\pgfpathlineto{\pgfqpoint{0.699186in}{0.617468in}}%
\pgfpathlineto{\pgfqpoint{0.687111in}{0.631079in}}%
\pgfpathlineto{\pgfqpoint{0.687084in}{0.631122in}}%
\pgfpathlineto{\pgfqpoint{0.680163in}{0.644691in}}%
\pgfpathlineto{\pgfqpoint{0.676309in}{0.658302in}}%
\pgfpathlineto{\pgfqpoint{0.675538in}{0.671913in}}%
\pgfpathlineto{\pgfqpoint{0.677850in}{0.685524in}}%
\pgfpathlineto{\pgfqpoint{0.683248in}{0.699135in}}%
\pgfpathlineto{\pgfqpoint{0.687084in}{0.705312in}}%
\pgfpathlineto{\pgfqpoint{0.692682in}{0.712746in}}%
\pgfpathlineto{\pgfqpoint{0.702741in}{0.722597in}}%
\pgfpathlineto{\pgfqpoint{0.707712in}{0.726357in}}%
\pgfpathlineto{\pgfqpoint{0.718397in}{0.732829in}}%
\pgfpathlineto{\pgfqpoint{0.734054in}{0.739407in}}%
\pgfpathlineto{\pgfqpoint{0.736451in}{0.739968in}}%
\pgfpathlineto{\pgfqpoint{0.749710in}{0.742598in}}%
\pgfpathlineto{\pgfqpoint{0.765367in}{0.743220in}}%
\pgfpathlineto{\pgfqpoint{0.781024in}{0.741356in}}%
\pgfpathlineto{\pgfqpoint{0.786032in}{0.739968in}}%
\pgfpathlineto{\pgfqpoint{0.796680in}{0.736484in}}%
\pgfpathlineto{\pgfqpoint{0.812337in}{0.728440in}}%
\pgfpathlineto{\pgfqpoint{0.815333in}{0.726357in}}%
\pgfpathlineto{\pgfqpoint{0.827993in}{0.715351in}}%
\pgfpathlineto{\pgfqpoint{0.830389in}{0.712746in}}%
\pgfpathlineto{\pgfqpoint{0.839642in}{0.699135in}}%
\pgfpathlineto{\pgfqpoint{0.843650in}{0.689878in}}%
\pgfpathlineto{\pgfqpoint{0.845246in}{0.685524in}}%
\pgfpathlineto{\pgfqpoint{0.847390in}{0.671913in}}%
\pgfpathlineto{\pgfqpoint{0.846675in}{0.658302in}}%
\pgfpathlineto{\pgfqpoint{0.843650in}{0.646775in}}%
\pgfpathlineto{\pgfqpoint{0.843005in}{0.644691in}}%
\pgfpathlineto{\pgfqpoint{0.835437in}{0.631079in}}%
\pgfpathlineto{\pgfqpoint{0.827993in}{0.621790in}}%
\pgfpathlineto{\pgfqpoint{0.823668in}{0.617468in}}%
\pgfpathlineto{\pgfqpoint{0.812337in}{0.608724in}}%
\pgfpathlineto{\pgfqpoint{0.803786in}{0.603857in}}%
\pgfpathlineto{\pgfqpoint{0.796680in}{0.600522in}}%
\pgfpathlineto{\pgfqpoint{0.781024in}{0.595830in}}%
\pgfpathlineto{\pgfqpoint{0.765367in}{0.593819in}}%
\pgfpathlineto{\pgfqpoint{0.749710in}{0.594489in}}%
\pgfpathlineto{\pgfqpoint{0.734054in}{0.597840in}}%
\pgfpathlineto{\pgfqpoint{0.718447in}{0.603857in}}%
\pgfpathclose%
\pgfpathmoveto{\pgfqpoint{1.485569in}{0.534538in}}%
\pgfpathlineto{\pgfqpoint{1.501226in}{0.530266in}}%
\pgfpathlineto{\pgfqpoint{1.516882in}{0.527549in}}%
\pgfpathlineto{\pgfqpoint{1.532539in}{0.526385in}}%
\pgfpathlineto{\pgfqpoint{1.548195in}{0.526773in}}%
\pgfpathlineto{\pgfqpoint{1.563852in}{0.528713in}}%
\pgfpathlineto{\pgfqpoint{1.579508in}{0.532207in}}%
\pgfpathlineto{\pgfqpoint{1.590679in}{0.535802in}}%
\pgfpathlineto{\pgfqpoint{1.595165in}{0.537323in}}%
\pgfpathlineto{\pgfqpoint{1.610822in}{0.544229in}}%
\pgfpathlineto{\pgfqpoint{1.620386in}{0.549413in}}%
\pgfpathlineto{\pgfqpoint{1.626478in}{0.552980in}}%
\pgfpathlineto{\pgfqpoint{1.641036in}{0.563024in}}%
\pgfpathlineto{\pgfqpoint{1.642135in}{0.563863in}}%
\pgfpathlineto{\pgfqpoint{1.656826in}{0.576635in}}%
\pgfpathlineto{\pgfqpoint{1.657791in}{0.577590in}}%
\pgfpathlineto{\pgfqpoint{1.669344in}{0.590246in}}%
\pgfpathlineto{\pgfqpoint{1.673448in}{0.595542in}}%
\pgfpathlineto{\pgfqpoint{1.679411in}{0.603857in}}%
\pgfpathlineto{\pgfqpoint{1.687354in}{0.617468in}}%
\pgfpathlineto{\pgfqpoint{1.689104in}{0.621368in}}%
\pgfpathlineto{\pgfqpoint{1.693239in}{0.631079in}}%
\pgfpathlineto{\pgfqpoint{1.697258in}{0.644691in}}%
\pgfpathlineto{\pgfqpoint{1.699490in}{0.658302in}}%
\pgfpathlineto{\pgfqpoint{1.699936in}{0.671913in}}%
\pgfpathlineto{\pgfqpoint{1.698597in}{0.685524in}}%
\pgfpathlineto{\pgfqpoint{1.695472in}{0.699135in}}%
\pgfpathlineto{\pgfqpoint{1.690558in}{0.712746in}}%
\pgfpathlineto{\pgfqpoint{1.689104in}{0.715720in}}%
\pgfpathlineto{\pgfqpoint{1.683617in}{0.726357in}}%
\pgfpathlineto{\pgfqpoint{1.674736in}{0.739968in}}%
\pgfpathlineto{\pgfqpoint{1.673448in}{0.741622in}}%
\pgfpathlineto{\pgfqpoint{1.663372in}{0.753579in}}%
\pgfpathlineto{\pgfqpoint{1.657791in}{0.759298in}}%
\pgfpathlineto{\pgfqpoint{1.649254in}{0.767191in}}%
\pgfpathlineto{\pgfqpoint{1.642135in}{0.773051in}}%
\pgfpathlineto{\pgfqpoint{1.631404in}{0.780802in}}%
\pgfpathlineto{\pgfqpoint{1.626478in}{0.784052in}}%
\pgfpathlineto{\pgfqpoint{1.610822in}{0.792836in}}%
\pgfpathlineto{\pgfqpoint{1.607381in}{0.794413in}}%
\pgfpathlineto{\pgfqpoint{1.595165in}{0.799659in}}%
\pgfpathlineto{\pgfqpoint{1.579508in}{0.804810in}}%
\pgfpathlineto{\pgfqpoint{1.565392in}{0.808024in}}%
\pgfpathlineto{\pgfqpoint{1.563852in}{0.808360in}}%
\pgfpathlineto{\pgfqpoint{1.548195in}{0.810271in}}%
\pgfpathlineto{\pgfqpoint{1.532539in}{0.810653in}}%
\pgfpathlineto{\pgfqpoint{1.516882in}{0.809507in}}%
\pgfpathlineto{\pgfqpoint{1.508185in}{0.808024in}}%
\pgfpathlineto{\pgfqpoint{1.501226in}{0.806789in}}%
\pgfpathlineto{\pgfqpoint{1.485569in}{0.802433in}}%
\pgfpathlineto{\pgfqpoint{1.469913in}{0.796487in}}%
\pgfpathlineto{\pgfqpoint{1.465562in}{0.794413in}}%
\pgfpathlineto{\pgfqpoint{1.454256in}{0.788654in}}%
\pgfpathlineto{\pgfqpoint{1.441441in}{0.780802in}}%
\pgfpathlineto{\pgfqpoint{1.438599in}{0.778898in}}%
\pgfpathlineto{\pgfqpoint{1.423491in}{0.767191in}}%
\pgfpathlineto{\pgfqpoint{1.422943in}{0.766714in}}%
\pgfpathlineto{\pgfqpoint{1.409475in}{0.753579in}}%
\pgfpathlineto{\pgfqpoint{1.407286in}{0.751109in}}%
\pgfpathlineto{\pgfqpoint{1.398254in}{0.739968in}}%
\pgfpathlineto{\pgfqpoint{1.391630in}{0.730139in}}%
\pgfpathlineto{\pgfqpoint{1.389244in}{0.726357in}}%
\pgfpathlineto{\pgfqpoint{1.382404in}{0.712746in}}%
\pgfpathlineto{\pgfqpoint{1.377393in}{0.699135in}}%
\pgfpathlineto{\pgfqpoint{1.375973in}{0.693085in}}%
\pgfpathlineto{\pgfqpoint{1.374267in}{0.685524in}}%
\pgfpathlineto{\pgfqpoint{1.372949in}{0.671913in}}%
\pgfpathlineto{\pgfqpoint{1.373388in}{0.658302in}}%
\pgfpathlineto{\pgfqpoint{1.375586in}{0.644691in}}%
\pgfpathlineto{\pgfqpoint{1.375973in}{0.643352in}}%
\pgfpathlineto{\pgfqpoint{1.379670in}{0.631079in}}%
\pgfpathlineto{\pgfqpoint{1.385595in}{0.617468in}}%
\pgfpathlineto{\pgfqpoint{1.391630in}{0.606849in}}%
\pgfpathlineto{\pgfqpoint{1.393443in}{0.603857in}}%
\pgfpathlineto{\pgfqpoint{1.403547in}{0.590246in}}%
\pgfpathlineto{\pgfqpoint{1.407286in}{0.585964in}}%
\pgfpathlineto{\pgfqpoint{1.416202in}{0.576635in}}%
\pgfpathlineto{\pgfqpoint{1.422943in}{0.570446in}}%
\pgfpathlineto{\pgfqpoint{1.432022in}{0.563024in}}%
\pgfpathlineto{\pgfqpoint{1.438599in}{0.558172in}}%
\pgfpathlineto{\pgfqpoint{1.452354in}{0.549413in}}%
\pgfpathlineto{\pgfqpoint{1.454256in}{0.548293in}}%
\pgfpathlineto{\pgfqpoint{1.469913in}{0.540572in}}%
\pgfpathlineto{\pgfqpoint{1.482148in}{0.535802in}}%
\pgfpathlineto{\pgfqpoint{1.485569in}{0.534538in}}%
\pgfpathclose%
\pgfpathmoveto{\pgfqpoint{1.494120in}{0.603857in}}%
\pgfpathlineto{\pgfqpoint{1.485569in}{0.608724in}}%
\pgfpathlineto{\pgfqpoint{1.474238in}{0.617468in}}%
\pgfpathlineto{\pgfqpoint{1.469913in}{0.621790in}}%
\pgfpathlineto{\pgfqpoint{1.462468in}{0.631079in}}%
\pgfpathlineto{\pgfqpoint{1.454901in}{0.644691in}}%
\pgfpathlineto{\pgfqpoint{1.454256in}{0.646775in}}%
\pgfpathlineto{\pgfqpoint{1.451231in}{0.658302in}}%
\pgfpathlineto{\pgfqpoint{1.450516in}{0.671913in}}%
\pgfpathlineto{\pgfqpoint{1.452660in}{0.685524in}}%
\pgfpathlineto{\pgfqpoint{1.454256in}{0.689878in}}%
\pgfpathlineto{\pgfqpoint{1.458264in}{0.699135in}}%
\pgfpathlineto{\pgfqpoint{1.467517in}{0.712746in}}%
\pgfpathlineto{\pgfqpoint{1.469913in}{0.715351in}}%
\pgfpathlineto{\pgfqpoint{1.482573in}{0.726357in}}%
\pgfpathlineto{\pgfqpoint{1.485569in}{0.728440in}}%
\pgfpathlineto{\pgfqpoint{1.501226in}{0.736484in}}%
\pgfpathlineto{\pgfqpoint{1.511873in}{0.739968in}}%
\pgfpathlineto{\pgfqpoint{1.516882in}{0.741356in}}%
\pgfpathlineto{\pgfqpoint{1.532539in}{0.743220in}}%
\pgfpathlineto{\pgfqpoint{1.548195in}{0.742598in}}%
\pgfpathlineto{\pgfqpoint{1.561454in}{0.739968in}}%
\pgfpathlineto{\pgfqpoint{1.563852in}{0.739407in}}%
\pgfpathlineto{\pgfqpoint{1.579508in}{0.732829in}}%
\pgfpathlineto{\pgfqpoint{1.590194in}{0.726357in}}%
\pgfpathlineto{\pgfqpoint{1.595165in}{0.722597in}}%
\pgfpathlineto{\pgfqpoint{1.605223in}{0.712746in}}%
\pgfpathlineto{\pgfqpoint{1.610822in}{0.705312in}}%
\pgfpathlineto{\pgfqpoint{1.614658in}{0.699135in}}%
\pgfpathlineto{\pgfqpoint{1.620056in}{0.685524in}}%
\pgfpathlineto{\pgfqpoint{1.622368in}{0.671913in}}%
\pgfpathlineto{\pgfqpoint{1.621597in}{0.658302in}}%
\pgfpathlineto{\pgfqpoint{1.617743in}{0.644691in}}%
\pgfpathlineto{\pgfqpoint{1.610822in}{0.631122in}}%
\pgfpathlineto{\pgfqpoint{1.610795in}{0.631079in}}%
\pgfpathlineto{\pgfqpoint{1.598720in}{0.617468in}}%
\pgfpathlineto{\pgfqpoint{1.595165in}{0.614378in}}%
\pgfpathlineto{\pgfqpoint{1.579508in}{0.603880in}}%
\pgfpathlineto{\pgfqpoint{1.579459in}{0.603857in}}%
\pgfpathlineto{\pgfqpoint{1.563852in}{0.597840in}}%
\pgfpathlineto{\pgfqpoint{1.548195in}{0.594489in}}%
\pgfpathlineto{\pgfqpoint{1.532539in}{0.593819in}}%
\pgfpathlineto{\pgfqpoint{1.516882in}{0.595830in}}%
\pgfpathlineto{\pgfqpoint{1.501226in}{0.600522in}}%
\pgfpathlineto{\pgfqpoint{1.494120in}{0.603857in}}%
\pgfpathclose%
\pgfpathmoveto{\pgfqpoint{0.734054in}{1.202410in}}%
\pgfpathlineto{\pgfqpoint{0.749710in}{1.200499in}}%
\pgfpathlineto{\pgfqpoint{0.765367in}{1.200117in}}%
\pgfpathlineto{\pgfqpoint{0.781024in}{1.201263in}}%
\pgfpathlineto{\pgfqpoint{0.789721in}{1.202746in}}%
\pgfpathlineto{\pgfqpoint{0.796680in}{1.203981in}}%
\pgfpathlineto{\pgfqpoint{0.812337in}{1.208337in}}%
\pgfpathlineto{\pgfqpoint{0.827993in}{1.214283in}}%
\pgfpathlineto{\pgfqpoint{0.832344in}{1.216357in}}%
\pgfpathlineto{\pgfqpoint{0.843650in}{1.222116in}}%
\pgfpathlineto{\pgfqpoint{0.856465in}{1.229968in}}%
\pgfpathlineto{\pgfqpoint{0.859306in}{1.231872in}}%
\pgfpathlineto{\pgfqpoint{0.874415in}{1.243579in}}%
\pgfpathlineto{\pgfqpoint{0.874963in}{1.244056in}}%
\pgfpathlineto{\pgfqpoint{0.888430in}{1.257191in}}%
\pgfpathlineto{\pgfqpoint{0.890620in}{1.259661in}}%
\pgfpathlineto{\pgfqpoint{0.899652in}{1.270802in}}%
\pgfpathlineto{\pgfqpoint{0.906276in}{1.280631in}}%
\pgfpathlineto{\pgfqpoint{0.908662in}{1.284413in}}%
\pgfpathlineto{\pgfqpoint{0.915502in}{1.298024in}}%
\pgfpathlineto{\pgfqpoint{0.920512in}{1.311635in}}%
\pgfpathlineto{\pgfqpoint{0.921933in}{1.317685in}}%
\pgfpathlineto{\pgfqpoint{0.923639in}{1.325246in}}%
\pgfpathlineto{\pgfqpoint{0.924957in}{1.338857in}}%
\pgfpathlineto{\pgfqpoint{0.924518in}{1.352468in}}%
\pgfpathlineto{\pgfqpoint{0.922320in}{1.366079in}}%
\pgfpathlineto{\pgfqpoint{0.921933in}{1.367418in}}%
\pgfpathlineto{\pgfqpoint{0.918235in}{1.379691in}}%
\pgfpathlineto{\pgfqpoint{0.912311in}{1.393302in}}%
\pgfpathlineto{\pgfqpoint{0.906276in}{1.403921in}}%
\pgfpathlineto{\pgfqpoint{0.904462in}{1.406913in}}%
\pgfpathlineto{\pgfqpoint{0.894358in}{1.420524in}}%
\pgfpathlineto{\pgfqpoint{0.890620in}{1.424806in}}%
\pgfpathlineto{\pgfqpoint{0.881704in}{1.434135in}}%
\pgfpathlineto{\pgfqpoint{0.874963in}{1.440324in}}%
\pgfpathlineto{\pgfqpoint{0.865884in}{1.447746in}}%
\pgfpathlineto{\pgfqpoint{0.859306in}{1.452598in}}%
\pgfpathlineto{\pgfqpoint{0.845552in}{1.461357in}}%
\pgfpathlineto{\pgfqpoint{0.843650in}{1.462477in}}%
\pgfpathlineto{\pgfqpoint{0.827993in}{1.470198in}}%
\pgfpathlineto{\pgfqpoint{0.815758in}{1.474968in}}%
\pgfpathlineto{\pgfqpoint{0.812337in}{1.476232in}}%
\pgfpathlineto{\pgfqpoint{0.796680in}{1.480504in}}%
\pgfpathlineto{\pgfqpoint{0.781024in}{1.483221in}}%
\pgfpathlineto{\pgfqpoint{0.765367in}{1.484385in}}%
\pgfpathlineto{\pgfqpoint{0.749710in}{1.483997in}}%
\pgfpathlineto{\pgfqpoint{0.734054in}{1.482057in}}%
\pgfpathlineto{\pgfqpoint{0.718397in}{1.478563in}}%
\pgfpathlineto{\pgfqpoint{0.707227in}{1.474968in}}%
\pgfpathlineto{\pgfqpoint{0.702741in}{1.473447in}}%
\pgfpathlineto{\pgfqpoint{0.687084in}{1.466541in}}%
\pgfpathlineto{\pgfqpoint{0.677520in}{1.461357in}}%
\pgfpathlineto{\pgfqpoint{0.671428in}{1.457790in}}%
\pgfpathlineto{\pgfqpoint{0.656870in}{1.447746in}}%
\pgfpathlineto{\pgfqpoint{0.655771in}{1.446907in}}%
\pgfpathlineto{\pgfqpoint{0.641080in}{1.434135in}}%
\pgfpathlineto{\pgfqpoint{0.640115in}{1.433180in}}%
\pgfpathlineto{\pgfqpoint{0.628562in}{1.420524in}}%
\pgfpathlineto{\pgfqpoint{0.624458in}{1.415228in}}%
\pgfpathlineto{\pgfqpoint{0.618495in}{1.406913in}}%
\pgfpathlineto{\pgfqpoint{0.610552in}{1.393302in}}%
\pgfpathlineto{\pgfqpoint{0.608801in}{1.389402in}}%
\pgfpathlineto{\pgfqpoint{0.604667in}{1.379691in}}%
\pgfpathlineto{\pgfqpoint{0.600648in}{1.366079in}}%
\pgfpathlineto{\pgfqpoint{0.598416in}{1.352468in}}%
\pgfpathlineto{\pgfqpoint{0.597970in}{1.338857in}}%
\pgfpathlineto{\pgfqpoint{0.599309in}{1.325246in}}%
\pgfpathlineto{\pgfqpoint{0.602433in}{1.311635in}}%
\pgfpathlineto{\pgfqpoint{0.607348in}{1.298024in}}%
\pgfpathlineto{\pgfqpoint{0.608801in}{1.295050in}}%
\pgfpathlineto{\pgfqpoint{0.614289in}{1.284413in}}%
\pgfpathlineto{\pgfqpoint{0.623170in}{1.270802in}}%
\pgfpathlineto{\pgfqpoint{0.624458in}{1.269148in}}%
\pgfpathlineto{\pgfqpoint{0.634534in}{1.257191in}}%
\pgfpathlineto{\pgfqpoint{0.640115in}{1.251472in}}%
\pgfpathlineto{\pgfqpoint{0.648652in}{1.243579in}}%
\pgfpathlineto{\pgfqpoint{0.655771in}{1.237719in}}%
\pgfpathlineto{\pgfqpoint{0.666502in}{1.229968in}}%
\pgfpathlineto{\pgfqpoint{0.671428in}{1.226718in}}%
\pgfpathlineto{\pgfqpoint{0.687084in}{1.217934in}}%
\pgfpathlineto{\pgfqpoint{0.690525in}{1.216357in}}%
\pgfpathlineto{\pgfqpoint{0.702741in}{1.211111in}}%
\pgfpathlineto{\pgfqpoint{0.718397in}{1.205960in}}%
\pgfpathlineto{\pgfqpoint{0.732514in}{1.202746in}}%
\pgfpathlineto{\pgfqpoint{0.734054in}{1.202410in}}%
\pgfpathclose%
\pgfpathmoveto{\pgfqpoint{0.736451in}{1.270802in}}%
\pgfpathlineto{\pgfqpoint{0.734054in}{1.271363in}}%
\pgfpathlineto{\pgfqpoint{0.718397in}{1.277941in}}%
\pgfpathlineto{\pgfqpoint{0.707712in}{1.284413in}}%
\pgfpathlineto{\pgfqpoint{0.702741in}{1.288173in}}%
\pgfpathlineto{\pgfqpoint{0.692682in}{1.298024in}}%
\pgfpathlineto{\pgfqpoint{0.687084in}{1.305458in}}%
\pgfpathlineto{\pgfqpoint{0.683248in}{1.311635in}}%
\pgfpathlineto{\pgfqpoint{0.677850in}{1.325246in}}%
\pgfpathlineto{\pgfqpoint{0.675538in}{1.338857in}}%
\pgfpathlineto{\pgfqpoint{0.676309in}{1.352468in}}%
\pgfpathlineto{\pgfqpoint{0.680163in}{1.366079in}}%
\pgfpathlineto{\pgfqpoint{0.687084in}{1.379648in}}%
\pgfpathlineto{\pgfqpoint{0.687111in}{1.379691in}}%
\pgfpathlineto{\pgfqpoint{0.699186in}{1.393302in}}%
\pgfpathlineto{\pgfqpoint{0.702741in}{1.396392in}}%
\pgfpathlineto{\pgfqpoint{0.718397in}{1.406890in}}%
\pgfpathlineto{\pgfqpoint{0.718447in}{1.406913in}}%
\pgfpathlineto{\pgfqpoint{0.734054in}{1.412930in}}%
\pgfpathlineto{\pgfqpoint{0.749710in}{1.416281in}}%
\pgfpathlineto{\pgfqpoint{0.765367in}{1.416951in}}%
\pgfpathlineto{\pgfqpoint{0.781024in}{1.414940in}}%
\pgfpathlineto{\pgfqpoint{0.796680in}{1.410248in}}%
\pgfpathlineto{\pgfqpoint{0.803786in}{1.406913in}}%
\pgfpathlineto{\pgfqpoint{0.812337in}{1.402046in}}%
\pgfpathlineto{\pgfqpoint{0.823668in}{1.393302in}}%
\pgfpathlineto{\pgfqpoint{0.827993in}{1.388980in}}%
\pgfpathlineto{\pgfqpoint{0.835437in}{1.379691in}}%
\pgfpathlineto{\pgfqpoint{0.843005in}{1.366079in}}%
\pgfpathlineto{\pgfqpoint{0.843650in}{1.363995in}}%
\pgfpathlineto{\pgfqpoint{0.846675in}{1.352468in}}%
\pgfpathlineto{\pgfqpoint{0.847390in}{1.338857in}}%
\pgfpathlineto{\pgfqpoint{0.845246in}{1.325246in}}%
\pgfpathlineto{\pgfqpoint{0.843650in}{1.320892in}}%
\pgfpathlineto{\pgfqpoint{0.839642in}{1.311635in}}%
\pgfpathlineto{\pgfqpoint{0.830389in}{1.298024in}}%
\pgfpathlineto{\pgfqpoint{0.827993in}{1.295419in}}%
\pgfpathlineto{\pgfqpoint{0.815333in}{1.284413in}}%
\pgfpathlineto{\pgfqpoint{0.812337in}{1.282330in}}%
\pgfpathlineto{\pgfqpoint{0.796680in}{1.274286in}}%
\pgfpathlineto{\pgfqpoint{0.786032in}{1.270802in}}%
\pgfpathlineto{\pgfqpoint{0.781024in}{1.269414in}}%
\pgfpathlineto{\pgfqpoint{0.765367in}{1.267550in}}%
\pgfpathlineto{\pgfqpoint{0.749710in}{1.268172in}}%
\pgfpathlineto{\pgfqpoint{0.736451in}{1.270802in}}%
\pgfpathclose%
\pgfpathmoveto{\pgfqpoint{1.516882in}{1.201263in}}%
\pgfpathlineto{\pgfqpoint{1.532539in}{1.200117in}}%
\pgfpathlineto{\pgfqpoint{1.548195in}{1.200499in}}%
\pgfpathlineto{\pgfqpoint{1.563852in}{1.202410in}}%
\pgfpathlineto{\pgfqpoint{1.565392in}{1.202746in}}%
\pgfpathlineto{\pgfqpoint{1.579508in}{1.205960in}}%
\pgfpathlineto{\pgfqpoint{1.595165in}{1.211111in}}%
\pgfpathlineto{\pgfqpoint{1.607381in}{1.216357in}}%
\pgfpathlineto{\pgfqpoint{1.610822in}{1.217934in}}%
\pgfpathlineto{\pgfqpoint{1.626478in}{1.226718in}}%
\pgfpathlineto{\pgfqpoint{1.631404in}{1.229968in}}%
\pgfpathlineto{\pgfqpoint{1.642135in}{1.237719in}}%
\pgfpathlineto{\pgfqpoint{1.649254in}{1.243579in}}%
\pgfpathlineto{\pgfqpoint{1.657791in}{1.251472in}}%
\pgfpathlineto{\pgfqpoint{1.663372in}{1.257191in}}%
\pgfpathlineto{\pgfqpoint{1.673448in}{1.269148in}}%
\pgfpathlineto{\pgfqpoint{1.674736in}{1.270802in}}%
\pgfpathlineto{\pgfqpoint{1.683617in}{1.284413in}}%
\pgfpathlineto{\pgfqpoint{1.689104in}{1.295050in}}%
\pgfpathlineto{\pgfqpoint{1.690558in}{1.298024in}}%
\pgfpathlineto{\pgfqpoint{1.695472in}{1.311635in}}%
\pgfpathlineto{\pgfqpoint{1.698597in}{1.325246in}}%
\pgfpathlineto{\pgfqpoint{1.699936in}{1.338857in}}%
\pgfpathlineto{\pgfqpoint{1.699490in}{1.352468in}}%
\pgfpathlineto{\pgfqpoint{1.697258in}{1.366079in}}%
\pgfpathlineto{\pgfqpoint{1.693239in}{1.379691in}}%
\pgfpathlineto{\pgfqpoint{1.689104in}{1.389402in}}%
\pgfpathlineto{\pgfqpoint{1.687354in}{1.393302in}}%
\pgfpathlineto{\pgfqpoint{1.679411in}{1.406913in}}%
\pgfpathlineto{\pgfqpoint{1.673448in}{1.415228in}}%
\pgfpathlineto{\pgfqpoint{1.669344in}{1.420524in}}%
\pgfpathlineto{\pgfqpoint{1.657791in}{1.433180in}}%
\pgfpathlineto{\pgfqpoint{1.656826in}{1.434135in}}%
\pgfpathlineto{\pgfqpoint{1.642135in}{1.446907in}}%
\pgfpathlineto{\pgfqpoint{1.641036in}{1.447746in}}%
\pgfpathlineto{\pgfqpoint{1.626478in}{1.457790in}}%
\pgfpathlineto{\pgfqpoint{1.620386in}{1.461357in}}%
\pgfpathlineto{\pgfqpoint{1.610822in}{1.466541in}}%
\pgfpathlineto{\pgfqpoint{1.595165in}{1.473447in}}%
\pgfpathlineto{\pgfqpoint{1.590679in}{1.474968in}}%
\pgfpathlineto{\pgfqpoint{1.579508in}{1.478563in}}%
\pgfpathlineto{\pgfqpoint{1.563852in}{1.482057in}}%
\pgfpathlineto{\pgfqpoint{1.548195in}{1.483997in}}%
\pgfpathlineto{\pgfqpoint{1.532539in}{1.484385in}}%
\pgfpathlineto{\pgfqpoint{1.516882in}{1.483221in}}%
\pgfpathlineto{\pgfqpoint{1.501226in}{1.480504in}}%
\pgfpathlineto{\pgfqpoint{1.485569in}{1.476232in}}%
\pgfpathlineto{\pgfqpoint{1.482148in}{1.474968in}}%
\pgfpathlineto{\pgfqpoint{1.469913in}{1.470198in}}%
\pgfpathlineto{\pgfqpoint{1.454256in}{1.462477in}}%
\pgfpathlineto{\pgfqpoint{1.452354in}{1.461357in}}%
\pgfpathlineto{\pgfqpoint{1.438599in}{1.452598in}}%
\pgfpathlineto{\pgfqpoint{1.432022in}{1.447746in}}%
\pgfpathlineto{\pgfqpoint{1.422943in}{1.440324in}}%
\pgfpathlineto{\pgfqpoint{1.416202in}{1.434135in}}%
\pgfpathlineto{\pgfqpoint{1.407286in}{1.424806in}}%
\pgfpathlineto{\pgfqpoint{1.403547in}{1.420524in}}%
\pgfpathlineto{\pgfqpoint{1.393443in}{1.406913in}}%
\pgfpathlineto{\pgfqpoint{1.391630in}{1.403921in}}%
\pgfpathlineto{\pgfqpoint{1.385595in}{1.393302in}}%
\pgfpathlineto{\pgfqpoint{1.379670in}{1.379691in}}%
\pgfpathlineto{\pgfqpoint{1.375973in}{1.367418in}}%
\pgfpathlineto{\pgfqpoint{1.375586in}{1.366079in}}%
\pgfpathlineto{\pgfqpoint{1.373388in}{1.352468in}}%
\pgfpathlineto{\pgfqpoint{1.372949in}{1.338857in}}%
\pgfpathlineto{\pgfqpoint{1.374267in}{1.325246in}}%
\pgfpathlineto{\pgfqpoint{1.375973in}{1.317685in}}%
\pgfpathlineto{\pgfqpoint{1.377393in}{1.311635in}}%
\pgfpathlineto{\pgfqpoint{1.382404in}{1.298024in}}%
\pgfpathlineto{\pgfqpoint{1.389244in}{1.284413in}}%
\pgfpathlineto{\pgfqpoint{1.391630in}{1.280631in}}%
\pgfpathlineto{\pgfqpoint{1.398254in}{1.270802in}}%
\pgfpathlineto{\pgfqpoint{1.407286in}{1.259661in}}%
\pgfpathlineto{\pgfqpoint{1.409475in}{1.257191in}}%
\pgfpathlineto{\pgfqpoint{1.422943in}{1.244056in}}%
\pgfpathlineto{\pgfqpoint{1.423491in}{1.243579in}}%
\pgfpathlineto{\pgfqpoint{1.438599in}{1.231872in}}%
\pgfpathlineto{\pgfqpoint{1.441441in}{1.229968in}}%
\pgfpathlineto{\pgfqpoint{1.454256in}{1.222116in}}%
\pgfpathlineto{\pgfqpoint{1.465562in}{1.216357in}}%
\pgfpathlineto{\pgfqpoint{1.469913in}{1.214283in}}%
\pgfpathlineto{\pgfqpoint{1.485569in}{1.208337in}}%
\pgfpathlineto{\pgfqpoint{1.501226in}{1.203981in}}%
\pgfpathlineto{\pgfqpoint{1.508185in}{1.202746in}}%
\pgfpathlineto{\pgfqpoint{1.516882in}{1.201263in}}%
\pgfpathclose%
\pgfpathmoveto{\pgfqpoint{1.511873in}{1.270802in}}%
\pgfpathlineto{\pgfqpoint{1.501226in}{1.274286in}}%
\pgfpathlineto{\pgfqpoint{1.485569in}{1.282330in}}%
\pgfpathlineto{\pgfqpoint{1.482573in}{1.284413in}}%
\pgfpathlineto{\pgfqpoint{1.469913in}{1.295419in}}%
\pgfpathlineto{\pgfqpoint{1.467517in}{1.298024in}}%
\pgfpathlineto{\pgfqpoint{1.458264in}{1.311635in}}%
\pgfpathlineto{\pgfqpoint{1.454256in}{1.320892in}}%
\pgfpathlineto{\pgfqpoint{1.452660in}{1.325246in}}%
\pgfpathlineto{\pgfqpoint{1.450516in}{1.338857in}}%
\pgfpathlineto{\pgfqpoint{1.451231in}{1.352468in}}%
\pgfpathlineto{\pgfqpoint{1.454256in}{1.363995in}}%
\pgfpathlineto{\pgfqpoint{1.454901in}{1.366079in}}%
\pgfpathlineto{\pgfqpoint{1.462468in}{1.379691in}}%
\pgfpathlineto{\pgfqpoint{1.469913in}{1.388980in}}%
\pgfpathlineto{\pgfqpoint{1.474238in}{1.393302in}}%
\pgfpathlineto{\pgfqpoint{1.485569in}{1.402046in}}%
\pgfpathlineto{\pgfqpoint{1.494120in}{1.406913in}}%
\pgfpathlineto{\pgfqpoint{1.501226in}{1.410248in}}%
\pgfpathlineto{\pgfqpoint{1.516882in}{1.414940in}}%
\pgfpathlineto{\pgfqpoint{1.532539in}{1.416951in}}%
\pgfpathlineto{\pgfqpoint{1.548195in}{1.416281in}}%
\pgfpathlineto{\pgfqpoint{1.563852in}{1.412930in}}%
\pgfpathlineto{\pgfqpoint{1.579459in}{1.406913in}}%
\pgfpathlineto{\pgfqpoint{1.579508in}{1.406890in}}%
\pgfpathlineto{\pgfqpoint{1.595165in}{1.396392in}}%
\pgfpathlineto{\pgfqpoint{1.598720in}{1.393302in}}%
\pgfpathlineto{\pgfqpoint{1.610795in}{1.379691in}}%
\pgfpathlineto{\pgfqpoint{1.610822in}{1.379648in}}%
\pgfpathlineto{\pgfqpoint{1.617743in}{1.366079in}}%
\pgfpathlineto{\pgfqpoint{1.621597in}{1.352468in}}%
\pgfpathlineto{\pgfqpoint{1.622368in}{1.338857in}}%
\pgfpathlineto{\pgfqpoint{1.620056in}{1.325246in}}%
\pgfpathlineto{\pgfqpoint{1.614658in}{1.311635in}}%
\pgfpathlineto{\pgfqpoint{1.610822in}{1.305458in}}%
\pgfpathlineto{\pgfqpoint{1.605223in}{1.298024in}}%
\pgfpathlineto{\pgfqpoint{1.595165in}{1.288173in}}%
\pgfpathlineto{\pgfqpoint{1.590194in}{1.284413in}}%
\pgfpathlineto{\pgfqpoint{1.579508in}{1.277941in}}%
\pgfpathlineto{\pgfqpoint{1.563852in}{1.271363in}}%
\pgfpathlineto{\pgfqpoint{1.561454in}{1.270802in}}%
\pgfpathlineto{\pgfqpoint{1.548195in}{1.268172in}}%
\pgfpathlineto{\pgfqpoint{1.532539in}{1.267550in}}%
\pgfpathlineto{\pgfqpoint{1.516882in}{1.269414in}}%
\pgfpathlineto{\pgfqpoint{1.511873in}{1.270802in}}%
\pgfpathclose%
\pgfusepath{fill}%
\end{pgfscope}%
\begin{pgfscope}%
\pgfpathrectangle{\pgfqpoint{0.373953in}{0.331635in}}{\pgfqpoint{1.550000in}{1.347500in}}%
\pgfusepath{clip}%
\pgfsetbuttcap%
\pgfsetroundjoin%
\definecolor{currentfill}{rgb}{0.921884,0.341098,0.377376}%
\pgfsetfillcolor{currentfill}%
\pgfsetlinewidth{0.000000pt}%
\definecolor{currentstroke}{rgb}{0.000000,0.000000,0.000000}%
\pgfsetstrokecolor{currentstroke}%
\pgfsetdash{}{0pt}%
\pgfpathmoveto{\pgfqpoint{0.702741in}{0.478963in}}%
\pgfpathlineto{\pgfqpoint{0.718397in}{0.474497in}}%
\pgfpathlineto{\pgfqpoint{0.734054in}{0.471408in}}%
\pgfpathlineto{\pgfqpoint{0.749710in}{0.469693in}}%
\pgfpathlineto{\pgfqpoint{0.765367in}{0.469351in}}%
\pgfpathlineto{\pgfqpoint{0.781024in}{0.470379in}}%
\pgfpathlineto{\pgfqpoint{0.796680in}{0.472781in}}%
\pgfpathlineto{\pgfqpoint{0.812337in}{0.476558in}}%
\pgfpathlineto{\pgfqpoint{0.826922in}{0.481357in}}%
\pgfpathlineto{\pgfqpoint{0.827993in}{0.481706in}}%
\pgfpathlineto{\pgfqpoint{0.843650in}{0.488092in}}%
\pgfpathlineto{\pgfqpoint{0.857594in}{0.494968in}}%
\pgfpathlineto{\pgfqpoint{0.859306in}{0.495821in}}%
\pgfpathlineto{\pgfqpoint{0.874963in}{0.504833in}}%
\pgfpathlineto{\pgfqpoint{0.880719in}{0.508579in}}%
\pgfpathlineto{\pgfqpoint{0.890620in}{0.515222in}}%
\pgfpathlineto{\pgfqpoint{0.900018in}{0.522191in}}%
\pgfpathlineto{\pgfqpoint{0.906276in}{0.527064in}}%
\pgfpathlineto{\pgfqpoint{0.916651in}{0.535802in}}%
\pgfpathlineto{\pgfqpoint{0.921933in}{0.540565in}}%
\pgfpathlineto{\pgfqpoint{0.931186in}{0.549413in}}%
\pgfpathlineto{\pgfqpoint{0.937589in}{0.556105in}}%
\pgfpathlineto{\pgfqpoint{0.943952in}{0.563024in}}%
\pgfpathlineto{\pgfqpoint{0.953246in}{0.574325in}}%
\pgfpathlineto{\pgfqpoint{0.955106in}{0.576635in}}%
\pgfpathlineto{\pgfqpoint{0.964722in}{0.590246in}}%
\pgfpathlineto{\pgfqpoint{0.968902in}{0.597238in}}%
\pgfpathlineto{\pgfqpoint{0.972859in}{0.603857in}}%
\pgfpathlineto{\pgfqpoint{0.979492in}{0.617468in}}%
\pgfpathlineto{\pgfqpoint{0.984559in}{0.631077in}}%
\pgfpathlineto{\pgfqpoint{0.984560in}{0.631079in}}%
\pgfpathlineto{\pgfqpoint{0.988177in}{0.644691in}}%
\pgfpathlineto{\pgfqpoint{0.990185in}{0.658302in}}%
\pgfpathlineto{\pgfqpoint{0.990587in}{0.671913in}}%
\pgfpathlineto{\pgfqpoint{0.989382in}{0.685524in}}%
\pgfpathlineto{\pgfqpoint{0.986570in}{0.699135in}}%
\pgfpathlineto{\pgfqpoint{0.984559in}{0.705350in}}%
\pgfpathlineto{\pgfqpoint{0.982222in}{0.712746in}}%
\pgfpathlineto{\pgfqpoint{0.976371in}{0.726357in}}%
\pgfpathlineto{\pgfqpoint{0.968955in}{0.739968in}}%
\pgfpathlineto{\pgfqpoint{0.968902in}{0.740049in}}%
\pgfpathlineto{\pgfqpoint{0.960106in}{0.753579in}}%
\pgfpathlineto{\pgfqpoint{0.953246in}{0.762630in}}%
\pgfpathlineto{\pgfqpoint{0.949720in}{0.767191in}}%
\pgfpathlineto{\pgfqpoint{0.937810in}{0.780802in}}%
\pgfpathlineto{\pgfqpoint{0.937589in}{0.781031in}}%
\pgfpathlineto{\pgfqpoint{0.924189in}{0.794413in}}%
\pgfpathlineto{\pgfqpoint{0.921933in}{0.796494in}}%
\pgfpathlineto{\pgfqpoint{0.908670in}{0.808024in}}%
\pgfpathlineto{\pgfqpoint{0.906276in}{0.809985in}}%
\pgfpathlineto{\pgfqpoint{0.890884in}{0.821635in}}%
\pgfpathlineto{\pgfqpoint{0.890620in}{0.821827in}}%
\pgfpathlineto{\pgfqpoint{0.874963in}{0.832181in}}%
\pgfpathlineto{\pgfqpoint{0.869717in}{0.835246in}}%
\pgfpathlineto{\pgfqpoint{0.859306in}{0.841210in}}%
\pgfpathlineto{\pgfqpoint{0.843743in}{0.848857in}}%
\pgfpathlineto{\pgfqpoint{0.843650in}{0.848903in}}%
\pgfpathlineto{\pgfqpoint{0.827993in}{0.855350in}}%
\pgfpathlineto{\pgfqpoint{0.812337in}{0.860436in}}%
\pgfpathlineto{\pgfqpoint{0.803829in}{0.862468in}}%
\pgfpathlineto{\pgfqpoint{0.796680in}{0.864216in}}%
\pgfpathlineto{\pgfqpoint{0.781024in}{0.866661in}}%
\pgfpathlineto{\pgfqpoint{0.765367in}{0.867709in}}%
\pgfpathlineto{\pgfqpoint{0.749710in}{0.867360in}}%
\pgfpathlineto{\pgfqpoint{0.734054in}{0.865614in}}%
\pgfpathlineto{\pgfqpoint{0.718397in}{0.862469in}}%
\pgfpathlineto{\pgfqpoint{0.718395in}{0.862468in}}%
\pgfpathlineto{\pgfqpoint{0.702741in}{0.858063in}}%
\pgfpathlineto{\pgfqpoint{0.687084in}{0.852297in}}%
\pgfpathlineto{\pgfqpoint{0.679471in}{0.848857in}}%
\pgfpathlineto{\pgfqpoint{0.671428in}{0.845223in}}%
\pgfpathlineto{\pgfqpoint{0.655771in}{0.836863in}}%
\pgfpathlineto{\pgfqpoint{0.653114in}{0.835246in}}%
\pgfpathlineto{\pgfqpoint{0.640115in}{0.827167in}}%
\pgfpathlineto{\pgfqpoint{0.632155in}{0.821635in}}%
\pgfpathlineto{\pgfqpoint{0.624458in}{0.816068in}}%
\pgfpathlineto{\pgfqpoint{0.614280in}{0.808024in}}%
\pgfpathlineto{\pgfqpoint{0.608801in}{0.803432in}}%
\pgfpathlineto{\pgfqpoint{0.598750in}{0.794413in}}%
\pgfpathlineto{\pgfqpoint{0.593145in}{0.788973in}}%
\pgfpathlineto{\pgfqpoint{0.585129in}{0.780802in}}%
\pgfpathlineto{\pgfqpoint{0.577488in}{0.772195in}}%
\pgfpathlineto{\pgfqpoint{0.573179in}{0.767191in}}%
\pgfpathlineto{\pgfqpoint{0.562812in}{0.753579in}}%
\pgfpathlineto{\pgfqpoint{0.561832in}{0.752091in}}%
\pgfpathlineto{\pgfqpoint{0.553922in}{0.739968in}}%
\pgfpathlineto{\pgfqpoint{0.546576in}{0.726357in}}%
\pgfpathlineto{\pgfqpoint{0.546175in}{0.725426in}}%
\pgfpathlineto{\pgfqpoint{0.540655in}{0.712746in}}%
\pgfpathlineto{\pgfqpoint{0.536310in}{0.699135in}}%
\pgfpathlineto{\pgfqpoint{0.533548in}{0.685524in}}%
\pgfpathlineto{\pgfqpoint{0.532364in}{0.671913in}}%
\pgfpathlineto{\pgfqpoint{0.532759in}{0.658302in}}%
\pgfpathlineto{\pgfqpoint{0.534731in}{0.644691in}}%
\pgfpathlineto{\pgfqpoint{0.538284in}{0.631079in}}%
\pgfpathlineto{\pgfqpoint{0.543422in}{0.617468in}}%
\pgfpathlineto{\pgfqpoint{0.546175in}{0.611854in}}%
\pgfpathlineto{\pgfqpoint{0.550055in}{0.603857in}}%
\pgfpathlineto{\pgfqpoint{0.558177in}{0.590246in}}%
\pgfpathlineto{\pgfqpoint{0.561832in}{0.585045in}}%
\pgfpathlineto{\pgfqpoint{0.567805in}{0.576635in}}%
\pgfpathlineto{\pgfqpoint{0.577488in}{0.564759in}}%
\pgfpathlineto{\pgfqpoint{0.578945in}{0.563024in}}%
\pgfpathlineto{\pgfqpoint{0.591685in}{0.549413in}}%
\pgfpathlineto{\pgfqpoint{0.593145in}{0.547983in}}%
\pgfpathlineto{\pgfqpoint{0.606221in}{0.535802in}}%
\pgfpathlineto{\pgfqpoint{0.608801in}{0.533558in}}%
\pgfpathlineto{\pgfqpoint{0.622813in}{0.522191in}}%
\pgfpathlineto{\pgfqpoint{0.624458in}{0.520921in}}%
\pgfpathlineto{\pgfqpoint{0.640115in}{0.509846in}}%
\pgfpathlineto{\pgfqpoint{0.642110in}{0.508579in}}%
\pgfpathlineto{\pgfqpoint{0.655771in}{0.500162in}}%
\pgfpathlineto{\pgfqpoint{0.665445in}{0.494968in}}%
\pgfpathlineto{\pgfqpoint{0.671428in}{0.491791in}}%
\pgfpathlineto{\pgfqpoint{0.687084in}{0.484730in}}%
\pgfpathlineto{\pgfqpoint{0.696283in}{0.481357in}}%
\pgfpathlineto{\pgfqpoint{0.702741in}{0.478963in}}%
\pgfpathclose%
\pgfpathmoveto{\pgfqpoint{0.707227in}{0.535802in}}%
\pgfpathlineto{\pgfqpoint{0.702741in}{0.537323in}}%
\pgfpathlineto{\pgfqpoint{0.687084in}{0.544229in}}%
\pgfpathlineto{\pgfqpoint{0.677520in}{0.549413in}}%
\pgfpathlineto{\pgfqpoint{0.671428in}{0.552980in}}%
\pgfpathlineto{\pgfqpoint{0.656870in}{0.563024in}}%
\pgfpathlineto{\pgfqpoint{0.655771in}{0.563863in}}%
\pgfpathlineto{\pgfqpoint{0.641080in}{0.576635in}}%
\pgfpathlineto{\pgfqpoint{0.640115in}{0.577590in}}%
\pgfpathlineto{\pgfqpoint{0.628562in}{0.590246in}}%
\pgfpathlineto{\pgfqpoint{0.624458in}{0.595542in}}%
\pgfpathlineto{\pgfqpoint{0.618495in}{0.603857in}}%
\pgfpathlineto{\pgfqpoint{0.610552in}{0.617468in}}%
\pgfpathlineto{\pgfqpoint{0.608801in}{0.621368in}}%
\pgfpathlineto{\pgfqpoint{0.604667in}{0.631079in}}%
\pgfpathlineto{\pgfqpoint{0.600648in}{0.644691in}}%
\pgfpathlineto{\pgfqpoint{0.598416in}{0.658302in}}%
\pgfpathlineto{\pgfqpoint{0.597970in}{0.671913in}}%
\pgfpathlineto{\pgfqpoint{0.599309in}{0.685524in}}%
\pgfpathlineto{\pgfqpoint{0.602433in}{0.699135in}}%
\pgfpathlineto{\pgfqpoint{0.607348in}{0.712746in}}%
\pgfpathlineto{\pgfqpoint{0.608801in}{0.715720in}}%
\pgfpathlineto{\pgfqpoint{0.614289in}{0.726357in}}%
\pgfpathlineto{\pgfqpoint{0.623170in}{0.739968in}}%
\pgfpathlineto{\pgfqpoint{0.624458in}{0.741622in}}%
\pgfpathlineto{\pgfqpoint{0.634534in}{0.753579in}}%
\pgfpathlineto{\pgfqpoint{0.640115in}{0.759298in}}%
\pgfpathlineto{\pgfqpoint{0.648652in}{0.767191in}}%
\pgfpathlineto{\pgfqpoint{0.655771in}{0.773051in}}%
\pgfpathlineto{\pgfqpoint{0.666502in}{0.780802in}}%
\pgfpathlineto{\pgfqpoint{0.671428in}{0.784052in}}%
\pgfpathlineto{\pgfqpoint{0.687084in}{0.792836in}}%
\pgfpathlineto{\pgfqpoint{0.690525in}{0.794413in}}%
\pgfpathlineto{\pgfqpoint{0.702741in}{0.799659in}}%
\pgfpathlineto{\pgfqpoint{0.718397in}{0.804810in}}%
\pgfpathlineto{\pgfqpoint{0.732514in}{0.808024in}}%
\pgfpathlineto{\pgfqpoint{0.734054in}{0.808360in}}%
\pgfpathlineto{\pgfqpoint{0.749710in}{0.810271in}}%
\pgfpathlineto{\pgfqpoint{0.765367in}{0.810653in}}%
\pgfpathlineto{\pgfqpoint{0.781024in}{0.809507in}}%
\pgfpathlineto{\pgfqpoint{0.789721in}{0.808024in}}%
\pgfpathlineto{\pgfqpoint{0.796680in}{0.806789in}}%
\pgfpathlineto{\pgfqpoint{0.812337in}{0.802433in}}%
\pgfpathlineto{\pgfqpoint{0.827993in}{0.796487in}}%
\pgfpathlineto{\pgfqpoint{0.832344in}{0.794413in}}%
\pgfpathlineto{\pgfqpoint{0.843650in}{0.788654in}}%
\pgfpathlineto{\pgfqpoint{0.856465in}{0.780802in}}%
\pgfpathlineto{\pgfqpoint{0.859306in}{0.778898in}}%
\pgfpathlineto{\pgfqpoint{0.874415in}{0.767191in}}%
\pgfpathlineto{\pgfqpoint{0.874963in}{0.766714in}}%
\pgfpathlineto{\pgfqpoint{0.888430in}{0.753579in}}%
\pgfpathlineto{\pgfqpoint{0.890620in}{0.751109in}}%
\pgfpathlineto{\pgfqpoint{0.899652in}{0.739968in}}%
\pgfpathlineto{\pgfqpoint{0.906276in}{0.730139in}}%
\pgfpathlineto{\pgfqpoint{0.908662in}{0.726357in}}%
\pgfpathlineto{\pgfqpoint{0.915502in}{0.712746in}}%
\pgfpathlineto{\pgfqpoint{0.920512in}{0.699135in}}%
\pgfpathlineto{\pgfqpoint{0.921933in}{0.693085in}}%
\pgfpathlineto{\pgfqpoint{0.923639in}{0.685524in}}%
\pgfpathlineto{\pgfqpoint{0.924957in}{0.671913in}}%
\pgfpathlineto{\pgfqpoint{0.924518in}{0.658302in}}%
\pgfpathlineto{\pgfqpoint{0.922320in}{0.644691in}}%
\pgfpathlineto{\pgfqpoint{0.921933in}{0.643352in}}%
\pgfpathlineto{\pgfqpoint{0.918235in}{0.631079in}}%
\pgfpathlineto{\pgfqpoint{0.912311in}{0.617468in}}%
\pgfpathlineto{\pgfqpoint{0.906276in}{0.606849in}}%
\pgfpathlineto{\pgfqpoint{0.904462in}{0.603857in}}%
\pgfpathlineto{\pgfqpoint{0.894358in}{0.590246in}}%
\pgfpathlineto{\pgfqpoint{0.890620in}{0.585964in}}%
\pgfpathlineto{\pgfqpoint{0.881704in}{0.576635in}}%
\pgfpathlineto{\pgfqpoint{0.874963in}{0.570446in}}%
\pgfpathlineto{\pgfqpoint{0.865884in}{0.563024in}}%
\pgfpathlineto{\pgfqpoint{0.859306in}{0.558172in}}%
\pgfpathlineto{\pgfqpoint{0.845552in}{0.549413in}}%
\pgfpathlineto{\pgfqpoint{0.843650in}{0.548293in}}%
\pgfpathlineto{\pgfqpoint{0.827993in}{0.540572in}}%
\pgfpathlineto{\pgfqpoint{0.815758in}{0.535802in}}%
\pgfpathlineto{\pgfqpoint{0.812337in}{0.534538in}}%
\pgfpathlineto{\pgfqpoint{0.796680in}{0.530266in}}%
\pgfpathlineto{\pgfqpoint{0.781024in}{0.527549in}}%
\pgfpathlineto{\pgfqpoint{0.765367in}{0.526385in}}%
\pgfpathlineto{\pgfqpoint{0.749710in}{0.526773in}}%
\pgfpathlineto{\pgfqpoint{0.734054in}{0.528713in}}%
\pgfpathlineto{\pgfqpoint{0.718397in}{0.532207in}}%
\pgfpathlineto{\pgfqpoint{0.707227in}{0.535802in}}%
\pgfpathclose%
\pgfpathmoveto{\pgfqpoint{1.485569in}{0.476558in}}%
\pgfpathlineto{\pgfqpoint{1.501226in}{0.472781in}}%
\pgfpathlineto{\pgfqpoint{1.516882in}{0.470379in}}%
\pgfpathlineto{\pgfqpoint{1.532539in}{0.469351in}}%
\pgfpathlineto{\pgfqpoint{1.548195in}{0.469693in}}%
\pgfpathlineto{\pgfqpoint{1.563852in}{0.471408in}}%
\pgfpathlineto{\pgfqpoint{1.579508in}{0.474497in}}%
\pgfpathlineto{\pgfqpoint{1.595165in}{0.478963in}}%
\pgfpathlineto{\pgfqpoint{1.601623in}{0.481357in}}%
\pgfpathlineto{\pgfqpoint{1.610822in}{0.484730in}}%
\pgfpathlineto{\pgfqpoint{1.626478in}{0.491791in}}%
\pgfpathlineto{\pgfqpoint{1.632461in}{0.494968in}}%
\pgfpathlineto{\pgfqpoint{1.642135in}{0.500162in}}%
\pgfpathlineto{\pgfqpoint{1.655796in}{0.508579in}}%
\pgfpathlineto{\pgfqpoint{1.657791in}{0.509846in}}%
\pgfpathlineto{\pgfqpoint{1.673448in}{0.520921in}}%
\pgfpathlineto{\pgfqpoint{1.675092in}{0.522191in}}%
\pgfpathlineto{\pgfqpoint{1.689104in}{0.533558in}}%
\pgfpathlineto{\pgfqpoint{1.691685in}{0.535802in}}%
\pgfpathlineto{\pgfqpoint{1.704761in}{0.547983in}}%
\pgfpathlineto{\pgfqpoint{1.706221in}{0.549413in}}%
\pgfpathlineto{\pgfqpoint{1.718961in}{0.563024in}}%
\pgfpathlineto{\pgfqpoint{1.720418in}{0.564759in}}%
\pgfpathlineto{\pgfqpoint{1.730100in}{0.576635in}}%
\pgfpathlineto{\pgfqpoint{1.736074in}{0.585045in}}%
\pgfpathlineto{\pgfqpoint{1.739729in}{0.590246in}}%
\pgfpathlineto{\pgfqpoint{1.747851in}{0.603857in}}%
\pgfpathlineto{\pgfqpoint{1.751731in}{0.611854in}}%
\pgfpathlineto{\pgfqpoint{1.754484in}{0.617468in}}%
\pgfpathlineto{\pgfqpoint{1.759621in}{0.631079in}}%
\pgfpathlineto{\pgfqpoint{1.763175in}{0.644691in}}%
\pgfpathlineto{\pgfqpoint{1.765147in}{0.658302in}}%
\pgfpathlineto{\pgfqpoint{1.765542in}{0.671913in}}%
\pgfpathlineto{\pgfqpoint{1.764358in}{0.685524in}}%
\pgfpathlineto{\pgfqpoint{1.761596in}{0.699135in}}%
\pgfpathlineto{\pgfqpoint{1.757251in}{0.712746in}}%
\pgfpathlineto{\pgfqpoint{1.751731in}{0.725426in}}%
\pgfpathlineto{\pgfqpoint{1.751330in}{0.726357in}}%
\pgfpathlineto{\pgfqpoint{1.743983in}{0.739968in}}%
\pgfpathlineto{\pgfqpoint{1.736074in}{0.752091in}}%
\pgfpathlineto{\pgfqpoint{1.735094in}{0.753579in}}%
\pgfpathlineto{\pgfqpoint{1.724726in}{0.767191in}}%
\pgfpathlineto{\pgfqpoint{1.720418in}{0.772195in}}%
\pgfpathlineto{\pgfqpoint{1.712777in}{0.780802in}}%
\pgfpathlineto{\pgfqpoint{1.704761in}{0.788973in}}%
\pgfpathlineto{\pgfqpoint{1.699156in}{0.794413in}}%
\pgfpathlineto{\pgfqpoint{1.689104in}{0.803432in}}%
\pgfpathlineto{\pgfqpoint{1.683625in}{0.808024in}}%
\pgfpathlineto{\pgfqpoint{1.673448in}{0.816068in}}%
\pgfpathlineto{\pgfqpoint{1.665750in}{0.821635in}}%
\pgfpathlineto{\pgfqpoint{1.657791in}{0.827167in}}%
\pgfpathlineto{\pgfqpoint{1.644792in}{0.835246in}}%
\pgfpathlineto{\pgfqpoint{1.642135in}{0.836863in}}%
\pgfpathlineto{\pgfqpoint{1.626478in}{0.845223in}}%
\pgfpathlineto{\pgfqpoint{1.618435in}{0.848857in}}%
\pgfpathlineto{\pgfqpoint{1.610822in}{0.852297in}}%
\pgfpathlineto{\pgfqpoint{1.595165in}{0.858063in}}%
\pgfpathlineto{\pgfqpoint{1.579511in}{0.862468in}}%
\pgfpathlineto{\pgfqpoint{1.579508in}{0.862469in}}%
\pgfpathlineto{\pgfqpoint{1.563852in}{0.865614in}}%
\pgfpathlineto{\pgfqpoint{1.548195in}{0.867360in}}%
\pgfpathlineto{\pgfqpoint{1.532539in}{0.867709in}}%
\pgfpathlineto{\pgfqpoint{1.516882in}{0.866661in}}%
\pgfpathlineto{\pgfqpoint{1.501226in}{0.864216in}}%
\pgfpathlineto{\pgfqpoint{1.494077in}{0.862468in}}%
\pgfpathlineto{\pgfqpoint{1.485569in}{0.860436in}}%
\pgfpathlineto{\pgfqpoint{1.469913in}{0.855350in}}%
\pgfpathlineto{\pgfqpoint{1.454256in}{0.848903in}}%
\pgfpathlineto{\pgfqpoint{1.454163in}{0.848857in}}%
\pgfpathlineto{\pgfqpoint{1.438599in}{0.841210in}}%
\pgfpathlineto{\pgfqpoint{1.428188in}{0.835246in}}%
\pgfpathlineto{\pgfqpoint{1.422943in}{0.832181in}}%
\pgfpathlineto{\pgfqpoint{1.407286in}{0.821827in}}%
\pgfpathlineto{\pgfqpoint{1.407022in}{0.821635in}}%
\pgfpathlineto{\pgfqpoint{1.391630in}{0.809985in}}%
\pgfpathlineto{\pgfqpoint{1.389236in}{0.808024in}}%
\pgfpathlineto{\pgfqpoint{1.375973in}{0.796494in}}%
\pgfpathlineto{\pgfqpoint{1.373717in}{0.794413in}}%
\pgfpathlineto{\pgfqpoint{1.360317in}{0.781031in}}%
\pgfpathlineto{\pgfqpoint{1.360096in}{0.780802in}}%
\pgfpathlineto{\pgfqpoint{1.348186in}{0.767191in}}%
\pgfpathlineto{\pgfqpoint{1.344660in}{0.762630in}}%
\pgfpathlineto{\pgfqpoint{1.337800in}{0.753579in}}%
\pgfpathlineto{\pgfqpoint{1.329003in}{0.740049in}}%
\pgfpathlineto{\pgfqpoint{1.328951in}{0.739968in}}%
\pgfpathlineto{\pgfqpoint{1.321534in}{0.726357in}}%
\pgfpathlineto{\pgfqpoint{1.315684in}{0.712746in}}%
\pgfpathlineto{\pgfqpoint{1.313347in}{0.705350in}}%
\pgfpathlineto{\pgfqpoint{1.311336in}{0.699135in}}%
\pgfpathlineto{\pgfqpoint{1.308524in}{0.685524in}}%
\pgfpathlineto{\pgfqpoint{1.307319in}{0.671913in}}%
\pgfpathlineto{\pgfqpoint{1.307720in}{0.658302in}}%
\pgfpathlineto{\pgfqpoint{1.309729in}{0.644691in}}%
\pgfpathlineto{\pgfqpoint{1.313346in}{0.631079in}}%
\pgfpathlineto{\pgfqpoint{1.313347in}{0.631077in}}%
\pgfpathlineto{\pgfqpoint{1.318414in}{0.617468in}}%
\pgfpathlineto{\pgfqpoint{1.325047in}{0.603857in}}%
\pgfpathlineto{\pgfqpoint{1.329003in}{0.597238in}}%
\pgfpathlineto{\pgfqpoint{1.333184in}{0.590246in}}%
\pgfpathlineto{\pgfqpoint{1.342800in}{0.576635in}}%
\pgfpathlineto{\pgfqpoint{1.344660in}{0.574325in}}%
\pgfpathlineto{\pgfqpoint{1.353954in}{0.563024in}}%
\pgfpathlineto{\pgfqpoint{1.360317in}{0.556105in}}%
\pgfpathlineto{\pgfqpoint{1.366720in}{0.549413in}}%
\pgfpathlineto{\pgfqpoint{1.375973in}{0.540565in}}%
\pgfpathlineto{\pgfqpoint{1.381255in}{0.535802in}}%
\pgfpathlineto{\pgfqpoint{1.391630in}{0.527064in}}%
\pgfpathlineto{\pgfqpoint{1.397887in}{0.522191in}}%
\pgfpathlineto{\pgfqpoint{1.407286in}{0.515222in}}%
\pgfpathlineto{\pgfqpoint{1.417187in}{0.508579in}}%
\pgfpathlineto{\pgfqpoint{1.422943in}{0.504833in}}%
\pgfpathlineto{\pgfqpoint{1.438599in}{0.495821in}}%
\pgfpathlineto{\pgfqpoint{1.440312in}{0.494968in}}%
\pgfpathlineto{\pgfqpoint{1.454256in}{0.488092in}}%
\pgfpathlineto{\pgfqpoint{1.469913in}{0.481706in}}%
\pgfpathlineto{\pgfqpoint{1.470984in}{0.481357in}}%
\pgfpathlineto{\pgfqpoint{1.485569in}{0.476558in}}%
\pgfpathclose%
\pgfpathmoveto{\pgfqpoint{1.482148in}{0.535802in}}%
\pgfpathlineto{\pgfqpoint{1.469913in}{0.540572in}}%
\pgfpathlineto{\pgfqpoint{1.454256in}{0.548293in}}%
\pgfpathlineto{\pgfqpoint{1.452354in}{0.549413in}}%
\pgfpathlineto{\pgfqpoint{1.438599in}{0.558172in}}%
\pgfpathlineto{\pgfqpoint{1.432022in}{0.563024in}}%
\pgfpathlineto{\pgfqpoint{1.422943in}{0.570446in}}%
\pgfpathlineto{\pgfqpoint{1.416202in}{0.576635in}}%
\pgfpathlineto{\pgfqpoint{1.407286in}{0.585964in}}%
\pgfpathlineto{\pgfqpoint{1.403547in}{0.590246in}}%
\pgfpathlineto{\pgfqpoint{1.393443in}{0.603857in}}%
\pgfpathlineto{\pgfqpoint{1.391630in}{0.606849in}}%
\pgfpathlineto{\pgfqpoint{1.385595in}{0.617468in}}%
\pgfpathlineto{\pgfqpoint{1.379670in}{0.631079in}}%
\pgfpathlineto{\pgfqpoint{1.375973in}{0.643352in}}%
\pgfpathlineto{\pgfqpoint{1.375586in}{0.644691in}}%
\pgfpathlineto{\pgfqpoint{1.373388in}{0.658302in}}%
\pgfpathlineto{\pgfqpoint{1.372949in}{0.671913in}}%
\pgfpathlineto{\pgfqpoint{1.374267in}{0.685524in}}%
\pgfpathlineto{\pgfqpoint{1.375973in}{0.693085in}}%
\pgfpathlineto{\pgfqpoint{1.377393in}{0.699135in}}%
\pgfpathlineto{\pgfqpoint{1.382404in}{0.712746in}}%
\pgfpathlineto{\pgfqpoint{1.389244in}{0.726357in}}%
\pgfpathlineto{\pgfqpoint{1.391630in}{0.730139in}}%
\pgfpathlineto{\pgfqpoint{1.398254in}{0.739968in}}%
\pgfpathlineto{\pgfqpoint{1.407286in}{0.751109in}}%
\pgfpathlineto{\pgfqpoint{1.409475in}{0.753579in}}%
\pgfpathlineto{\pgfqpoint{1.422943in}{0.766714in}}%
\pgfpathlineto{\pgfqpoint{1.423491in}{0.767191in}}%
\pgfpathlineto{\pgfqpoint{1.438599in}{0.778898in}}%
\pgfpathlineto{\pgfqpoint{1.441441in}{0.780802in}}%
\pgfpathlineto{\pgfqpoint{1.454256in}{0.788654in}}%
\pgfpathlineto{\pgfqpoint{1.465562in}{0.794413in}}%
\pgfpathlineto{\pgfqpoint{1.469913in}{0.796487in}}%
\pgfpathlineto{\pgfqpoint{1.485569in}{0.802433in}}%
\pgfpathlineto{\pgfqpoint{1.501226in}{0.806789in}}%
\pgfpathlineto{\pgfqpoint{1.508185in}{0.808024in}}%
\pgfpathlineto{\pgfqpoint{1.516882in}{0.809507in}}%
\pgfpathlineto{\pgfqpoint{1.532539in}{0.810653in}}%
\pgfpathlineto{\pgfqpoint{1.548195in}{0.810271in}}%
\pgfpathlineto{\pgfqpoint{1.563852in}{0.808360in}}%
\pgfpathlineto{\pgfqpoint{1.565392in}{0.808024in}}%
\pgfpathlineto{\pgfqpoint{1.579508in}{0.804810in}}%
\pgfpathlineto{\pgfqpoint{1.595165in}{0.799659in}}%
\pgfpathlineto{\pgfqpoint{1.607381in}{0.794413in}}%
\pgfpathlineto{\pgfqpoint{1.610822in}{0.792836in}}%
\pgfpathlineto{\pgfqpoint{1.626478in}{0.784052in}}%
\pgfpathlineto{\pgfqpoint{1.631404in}{0.780802in}}%
\pgfpathlineto{\pgfqpoint{1.642135in}{0.773051in}}%
\pgfpathlineto{\pgfqpoint{1.649254in}{0.767191in}}%
\pgfpathlineto{\pgfqpoint{1.657791in}{0.759298in}}%
\pgfpathlineto{\pgfqpoint{1.663372in}{0.753579in}}%
\pgfpathlineto{\pgfqpoint{1.673448in}{0.741622in}}%
\pgfpathlineto{\pgfqpoint{1.674736in}{0.739968in}}%
\pgfpathlineto{\pgfqpoint{1.683617in}{0.726357in}}%
\pgfpathlineto{\pgfqpoint{1.689104in}{0.715720in}}%
\pgfpathlineto{\pgfqpoint{1.690558in}{0.712746in}}%
\pgfpathlineto{\pgfqpoint{1.695472in}{0.699135in}}%
\pgfpathlineto{\pgfqpoint{1.698597in}{0.685524in}}%
\pgfpathlineto{\pgfqpoint{1.699936in}{0.671913in}}%
\pgfpathlineto{\pgfqpoint{1.699490in}{0.658302in}}%
\pgfpathlineto{\pgfqpoint{1.697258in}{0.644691in}}%
\pgfpathlineto{\pgfqpoint{1.693239in}{0.631079in}}%
\pgfpathlineto{\pgfqpoint{1.689104in}{0.621368in}}%
\pgfpathlineto{\pgfqpoint{1.687354in}{0.617468in}}%
\pgfpathlineto{\pgfqpoint{1.679411in}{0.603857in}}%
\pgfpathlineto{\pgfqpoint{1.673448in}{0.595542in}}%
\pgfpathlineto{\pgfqpoint{1.669344in}{0.590246in}}%
\pgfpathlineto{\pgfqpoint{1.657791in}{0.577590in}}%
\pgfpathlineto{\pgfqpoint{1.656826in}{0.576635in}}%
\pgfpathlineto{\pgfqpoint{1.642135in}{0.563863in}}%
\pgfpathlineto{\pgfqpoint{1.641036in}{0.563024in}}%
\pgfpathlineto{\pgfqpoint{1.626478in}{0.552980in}}%
\pgfpathlineto{\pgfqpoint{1.620386in}{0.549413in}}%
\pgfpathlineto{\pgfqpoint{1.610822in}{0.544229in}}%
\pgfpathlineto{\pgfqpoint{1.595165in}{0.537323in}}%
\pgfpathlineto{\pgfqpoint{1.590679in}{0.535802in}}%
\pgfpathlineto{\pgfqpoint{1.579508in}{0.532207in}}%
\pgfpathlineto{\pgfqpoint{1.563852in}{0.528713in}}%
\pgfpathlineto{\pgfqpoint{1.548195in}{0.526773in}}%
\pgfpathlineto{\pgfqpoint{1.532539in}{0.526385in}}%
\pgfpathlineto{\pgfqpoint{1.516882in}{0.527549in}}%
\pgfpathlineto{\pgfqpoint{1.501226in}{0.530266in}}%
\pgfpathlineto{\pgfqpoint{1.485569in}{0.534538in}}%
\pgfpathlineto{\pgfqpoint{1.482148in}{0.535802in}}%
\pgfpathclose%
\pgfpathmoveto{\pgfqpoint{0.718397in}{1.148301in}}%
\pgfpathlineto{\pgfqpoint{0.734054in}{1.145156in}}%
\pgfpathlineto{\pgfqpoint{0.749710in}{1.143410in}}%
\pgfpathlineto{\pgfqpoint{0.765367in}{1.143061in}}%
\pgfpathlineto{\pgfqpoint{0.781024in}{1.144109in}}%
\pgfpathlineto{\pgfqpoint{0.796680in}{1.146554in}}%
\pgfpathlineto{\pgfqpoint{0.803829in}{1.148302in}}%
\pgfpathlineto{\pgfqpoint{0.812337in}{1.150334in}}%
\pgfpathlineto{\pgfqpoint{0.827993in}{1.155420in}}%
\pgfpathlineto{\pgfqpoint{0.843650in}{1.161867in}}%
\pgfpathlineto{\pgfqpoint{0.843743in}{1.161913in}}%
\pgfpathlineto{\pgfqpoint{0.859306in}{1.169560in}}%
\pgfpathlineto{\pgfqpoint{0.869717in}{1.175524in}}%
\pgfpathlineto{\pgfqpoint{0.874963in}{1.178589in}}%
\pgfpathlineto{\pgfqpoint{0.890620in}{1.188943in}}%
\pgfpathlineto{\pgfqpoint{0.890884in}{1.189135in}}%
\pgfpathlineto{\pgfqpoint{0.906276in}{1.200785in}}%
\pgfpathlineto{\pgfqpoint{0.908670in}{1.202746in}}%
\pgfpathlineto{\pgfqpoint{0.921933in}{1.214276in}}%
\pgfpathlineto{\pgfqpoint{0.924189in}{1.216357in}}%
\pgfpathlineto{\pgfqpoint{0.937589in}{1.229739in}}%
\pgfpathlineto{\pgfqpoint{0.937810in}{1.229968in}}%
\pgfpathlineto{\pgfqpoint{0.949720in}{1.243579in}}%
\pgfpathlineto{\pgfqpoint{0.953246in}{1.248140in}}%
\pgfpathlineto{\pgfqpoint{0.960106in}{1.257191in}}%
\pgfpathlineto{\pgfqpoint{0.968902in}{1.270721in}}%
\pgfpathlineto{\pgfqpoint{0.968955in}{1.270802in}}%
\pgfpathlineto{\pgfqpoint{0.976371in}{1.284413in}}%
\pgfpathlineto{\pgfqpoint{0.982222in}{1.298024in}}%
\pgfpathlineto{\pgfqpoint{0.984559in}{1.305420in}}%
\pgfpathlineto{\pgfqpoint{0.986570in}{1.311635in}}%
\pgfpathlineto{\pgfqpoint{0.989382in}{1.325246in}}%
\pgfpathlineto{\pgfqpoint{0.990587in}{1.338857in}}%
\pgfpathlineto{\pgfqpoint{0.990185in}{1.352468in}}%
\pgfpathlineto{\pgfqpoint{0.988177in}{1.366079in}}%
\pgfpathlineto{\pgfqpoint{0.984560in}{1.379691in}}%
\pgfpathlineto{\pgfqpoint{0.984559in}{1.379693in}}%
\pgfpathlineto{\pgfqpoint{0.979492in}{1.393302in}}%
\pgfpathlineto{\pgfqpoint{0.972859in}{1.406913in}}%
\pgfpathlineto{\pgfqpoint{0.968902in}{1.413532in}}%
\pgfpathlineto{\pgfqpoint{0.964722in}{1.420524in}}%
\pgfpathlineto{\pgfqpoint{0.955106in}{1.434135in}}%
\pgfpathlineto{\pgfqpoint{0.953246in}{1.436445in}}%
\pgfpathlineto{\pgfqpoint{0.943952in}{1.447746in}}%
\pgfpathlineto{\pgfqpoint{0.937589in}{1.454665in}}%
\pgfpathlineto{\pgfqpoint{0.931186in}{1.461357in}}%
\pgfpathlineto{\pgfqpoint{0.921933in}{1.470205in}}%
\pgfpathlineto{\pgfqpoint{0.916651in}{1.474968in}}%
\pgfpathlineto{\pgfqpoint{0.906276in}{1.483706in}}%
\pgfpathlineto{\pgfqpoint{0.900018in}{1.488579in}}%
\pgfpathlineto{\pgfqpoint{0.890620in}{1.495548in}}%
\pgfpathlineto{\pgfqpoint{0.880719in}{1.502191in}}%
\pgfpathlineto{\pgfqpoint{0.874963in}{1.505937in}}%
\pgfpathlineto{\pgfqpoint{0.859306in}{1.514949in}}%
\pgfpathlineto{\pgfqpoint{0.857594in}{1.515802in}}%
\pgfpathlineto{\pgfqpoint{0.843650in}{1.522678in}}%
\pgfpathlineto{\pgfqpoint{0.827993in}{1.529064in}}%
\pgfpathlineto{\pgfqpoint{0.826922in}{1.529413in}}%
\pgfpathlineto{\pgfqpoint{0.812337in}{1.534212in}}%
\pgfpathlineto{\pgfqpoint{0.796680in}{1.537989in}}%
\pgfpathlineto{\pgfqpoint{0.781024in}{1.540391in}}%
\pgfpathlineto{\pgfqpoint{0.765367in}{1.541419in}}%
\pgfpathlineto{\pgfqpoint{0.749710in}{1.541077in}}%
\pgfpathlineto{\pgfqpoint{0.734054in}{1.539362in}}%
\pgfpathlineto{\pgfqpoint{0.718397in}{1.536273in}}%
\pgfpathlineto{\pgfqpoint{0.702741in}{1.531807in}}%
\pgfpathlineto{\pgfqpoint{0.696283in}{1.529413in}}%
\pgfpathlineto{\pgfqpoint{0.687084in}{1.526040in}}%
\pgfpathlineto{\pgfqpoint{0.671428in}{1.518979in}}%
\pgfpathlineto{\pgfqpoint{0.665445in}{1.515802in}}%
\pgfpathlineto{\pgfqpoint{0.655771in}{1.510608in}}%
\pgfpathlineto{\pgfqpoint{0.642110in}{1.502191in}}%
\pgfpathlineto{\pgfqpoint{0.640115in}{1.500924in}}%
\pgfpathlineto{\pgfqpoint{0.624458in}{1.489849in}}%
\pgfpathlineto{\pgfqpoint{0.622813in}{1.488579in}}%
\pgfpathlineto{\pgfqpoint{0.608801in}{1.477212in}}%
\pgfpathlineto{\pgfqpoint{0.606221in}{1.474968in}}%
\pgfpathlineto{\pgfqpoint{0.593145in}{1.462787in}}%
\pgfpathlineto{\pgfqpoint{0.591685in}{1.461357in}}%
\pgfpathlineto{\pgfqpoint{0.578945in}{1.447746in}}%
\pgfpathlineto{\pgfqpoint{0.577488in}{1.446011in}}%
\pgfpathlineto{\pgfqpoint{0.567805in}{1.434135in}}%
\pgfpathlineto{\pgfqpoint{0.561832in}{1.425725in}}%
\pgfpathlineto{\pgfqpoint{0.558177in}{1.420524in}}%
\pgfpathlineto{\pgfqpoint{0.550055in}{1.406913in}}%
\pgfpathlineto{\pgfqpoint{0.546175in}{1.398916in}}%
\pgfpathlineto{\pgfqpoint{0.543422in}{1.393302in}}%
\pgfpathlineto{\pgfqpoint{0.538284in}{1.379691in}}%
\pgfpathlineto{\pgfqpoint{0.534731in}{1.366079in}}%
\pgfpathlineto{\pgfqpoint{0.532759in}{1.352468in}}%
\pgfpathlineto{\pgfqpoint{0.532364in}{1.338857in}}%
\pgfpathlineto{\pgfqpoint{0.533548in}{1.325246in}}%
\pgfpathlineto{\pgfqpoint{0.536310in}{1.311635in}}%
\pgfpathlineto{\pgfqpoint{0.540655in}{1.298024in}}%
\pgfpathlineto{\pgfqpoint{0.546175in}{1.285344in}}%
\pgfpathlineto{\pgfqpoint{0.546576in}{1.284413in}}%
\pgfpathlineto{\pgfqpoint{0.553922in}{1.270802in}}%
\pgfpathlineto{\pgfqpoint{0.561832in}{1.258679in}}%
\pgfpathlineto{\pgfqpoint{0.562812in}{1.257191in}}%
\pgfpathlineto{\pgfqpoint{0.573179in}{1.243579in}}%
\pgfpathlineto{\pgfqpoint{0.577488in}{1.238575in}}%
\pgfpathlineto{\pgfqpoint{0.585129in}{1.229968in}}%
\pgfpathlineto{\pgfqpoint{0.593145in}{1.221797in}}%
\pgfpathlineto{\pgfqpoint{0.598750in}{1.216357in}}%
\pgfpathlineto{\pgfqpoint{0.608801in}{1.207338in}}%
\pgfpathlineto{\pgfqpoint{0.614280in}{1.202746in}}%
\pgfpathlineto{\pgfqpoint{0.624458in}{1.194702in}}%
\pgfpathlineto{\pgfqpoint{0.632155in}{1.189135in}}%
\pgfpathlineto{\pgfqpoint{0.640115in}{1.183603in}}%
\pgfpathlineto{\pgfqpoint{0.653114in}{1.175524in}}%
\pgfpathlineto{\pgfqpoint{0.655771in}{1.173907in}}%
\pgfpathlineto{\pgfqpoint{0.671428in}{1.165547in}}%
\pgfpathlineto{\pgfqpoint{0.679471in}{1.161913in}}%
\pgfpathlineto{\pgfqpoint{0.687084in}{1.158473in}}%
\pgfpathlineto{\pgfqpoint{0.702741in}{1.152707in}}%
\pgfpathlineto{\pgfqpoint{0.718395in}{1.148302in}}%
\pgfpathlineto{\pgfqpoint{0.718397in}{1.148301in}}%
\pgfpathclose%
\pgfpathmoveto{\pgfqpoint{0.732514in}{1.202746in}}%
\pgfpathlineto{\pgfqpoint{0.718397in}{1.205960in}}%
\pgfpathlineto{\pgfqpoint{0.702741in}{1.211111in}}%
\pgfpathlineto{\pgfqpoint{0.690525in}{1.216357in}}%
\pgfpathlineto{\pgfqpoint{0.687084in}{1.217934in}}%
\pgfpathlineto{\pgfqpoint{0.671428in}{1.226718in}}%
\pgfpathlineto{\pgfqpoint{0.666502in}{1.229968in}}%
\pgfpathlineto{\pgfqpoint{0.655771in}{1.237719in}}%
\pgfpathlineto{\pgfqpoint{0.648652in}{1.243579in}}%
\pgfpathlineto{\pgfqpoint{0.640115in}{1.251472in}}%
\pgfpathlineto{\pgfqpoint{0.634534in}{1.257191in}}%
\pgfpathlineto{\pgfqpoint{0.624458in}{1.269148in}}%
\pgfpathlineto{\pgfqpoint{0.623170in}{1.270802in}}%
\pgfpathlineto{\pgfqpoint{0.614289in}{1.284413in}}%
\pgfpathlineto{\pgfqpoint{0.608801in}{1.295050in}}%
\pgfpathlineto{\pgfqpoint{0.607348in}{1.298024in}}%
\pgfpathlineto{\pgfqpoint{0.602433in}{1.311635in}}%
\pgfpathlineto{\pgfqpoint{0.599309in}{1.325246in}}%
\pgfpathlineto{\pgfqpoint{0.597970in}{1.338857in}}%
\pgfpathlineto{\pgfqpoint{0.598416in}{1.352468in}}%
\pgfpathlineto{\pgfqpoint{0.600648in}{1.366079in}}%
\pgfpathlineto{\pgfqpoint{0.604667in}{1.379691in}}%
\pgfpathlineto{\pgfqpoint{0.608801in}{1.389402in}}%
\pgfpathlineto{\pgfqpoint{0.610552in}{1.393302in}}%
\pgfpathlineto{\pgfqpoint{0.618495in}{1.406913in}}%
\pgfpathlineto{\pgfqpoint{0.624458in}{1.415228in}}%
\pgfpathlineto{\pgfqpoint{0.628562in}{1.420524in}}%
\pgfpathlineto{\pgfqpoint{0.640115in}{1.433180in}}%
\pgfpathlineto{\pgfqpoint{0.641080in}{1.434135in}}%
\pgfpathlineto{\pgfqpoint{0.655771in}{1.446907in}}%
\pgfpathlineto{\pgfqpoint{0.656870in}{1.447746in}}%
\pgfpathlineto{\pgfqpoint{0.671428in}{1.457790in}}%
\pgfpathlineto{\pgfqpoint{0.677520in}{1.461357in}}%
\pgfpathlineto{\pgfqpoint{0.687084in}{1.466541in}}%
\pgfpathlineto{\pgfqpoint{0.702741in}{1.473447in}}%
\pgfpathlineto{\pgfqpoint{0.707227in}{1.474968in}}%
\pgfpathlineto{\pgfqpoint{0.718397in}{1.478563in}}%
\pgfpathlineto{\pgfqpoint{0.734054in}{1.482057in}}%
\pgfpathlineto{\pgfqpoint{0.749710in}{1.483997in}}%
\pgfpathlineto{\pgfqpoint{0.765367in}{1.484385in}}%
\pgfpathlineto{\pgfqpoint{0.781024in}{1.483221in}}%
\pgfpathlineto{\pgfqpoint{0.796680in}{1.480504in}}%
\pgfpathlineto{\pgfqpoint{0.812337in}{1.476232in}}%
\pgfpathlineto{\pgfqpoint{0.815758in}{1.474968in}}%
\pgfpathlineto{\pgfqpoint{0.827993in}{1.470198in}}%
\pgfpathlineto{\pgfqpoint{0.843650in}{1.462477in}}%
\pgfpathlineto{\pgfqpoint{0.845552in}{1.461357in}}%
\pgfpathlineto{\pgfqpoint{0.859306in}{1.452598in}}%
\pgfpathlineto{\pgfqpoint{0.865884in}{1.447746in}}%
\pgfpathlineto{\pgfqpoint{0.874963in}{1.440324in}}%
\pgfpathlineto{\pgfqpoint{0.881704in}{1.434135in}}%
\pgfpathlineto{\pgfqpoint{0.890620in}{1.424806in}}%
\pgfpathlineto{\pgfqpoint{0.894358in}{1.420524in}}%
\pgfpathlineto{\pgfqpoint{0.904462in}{1.406913in}}%
\pgfpathlineto{\pgfqpoint{0.906276in}{1.403921in}}%
\pgfpathlineto{\pgfqpoint{0.912311in}{1.393302in}}%
\pgfpathlineto{\pgfqpoint{0.918235in}{1.379691in}}%
\pgfpathlineto{\pgfqpoint{0.921933in}{1.367418in}}%
\pgfpathlineto{\pgfqpoint{0.922320in}{1.366079in}}%
\pgfpathlineto{\pgfqpoint{0.924518in}{1.352468in}}%
\pgfpathlineto{\pgfqpoint{0.924957in}{1.338857in}}%
\pgfpathlineto{\pgfqpoint{0.923639in}{1.325246in}}%
\pgfpathlineto{\pgfqpoint{0.921933in}{1.317685in}}%
\pgfpathlineto{\pgfqpoint{0.920512in}{1.311635in}}%
\pgfpathlineto{\pgfqpoint{0.915502in}{1.298024in}}%
\pgfpathlineto{\pgfqpoint{0.908662in}{1.284413in}}%
\pgfpathlineto{\pgfqpoint{0.906276in}{1.280631in}}%
\pgfpathlineto{\pgfqpoint{0.899652in}{1.270802in}}%
\pgfpathlineto{\pgfqpoint{0.890620in}{1.259661in}}%
\pgfpathlineto{\pgfqpoint{0.888430in}{1.257191in}}%
\pgfpathlineto{\pgfqpoint{0.874963in}{1.244056in}}%
\pgfpathlineto{\pgfqpoint{0.874415in}{1.243579in}}%
\pgfpathlineto{\pgfqpoint{0.859306in}{1.231872in}}%
\pgfpathlineto{\pgfqpoint{0.856465in}{1.229968in}}%
\pgfpathlineto{\pgfqpoint{0.843650in}{1.222116in}}%
\pgfpathlineto{\pgfqpoint{0.832344in}{1.216357in}}%
\pgfpathlineto{\pgfqpoint{0.827993in}{1.214283in}}%
\pgfpathlineto{\pgfqpoint{0.812337in}{1.208337in}}%
\pgfpathlineto{\pgfqpoint{0.796680in}{1.203981in}}%
\pgfpathlineto{\pgfqpoint{0.789721in}{1.202746in}}%
\pgfpathlineto{\pgfqpoint{0.781024in}{1.201263in}}%
\pgfpathlineto{\pgfqpoint{0.765367in}{1.200117in}}%
\pgfpathlineto{\pgfqpoint{0.749710in}{1.200499in}}%
\pgfpathlineto{\pgfqpoint{0.734054in}{1.202410in}}%
\pgfpathlineto{\pgfqpoint{0.732514in}{1.202746in}}%
\pgfpathclose%
\pgfpathmoveto{\pgfqpoint{1.501226in}{1.146554in}}%
\pgfpathlineto{\pgfqpoint{1.516882in}{1.144109in}}%
\pgfpathlineto{\pgfqpoint{1.532539in}{1.143061in}}%
\pgfpathlineto{\pgfqpoint{1.548195in}{1.143410in}}%
\pgfpathlineto{\pgfqpoint{1.563852in}{1.145156in}}%
\pgfpathlineto{\pgfqpoint{1.579508in}{1.148301in}}%
\pgfpathlineto{\pgfqpoint{1.579511in}{1.148302in}}%
\pgfpathlineto{\pgfqpoint{1.595165in}{1.152707in}}%
\pgfpathlineto{\pgfqpoint{1.610822in}{1.158473in}}%
\pgfpathlineto{\pgfqpoint{1.618435in}{1.161913in}}%
\pgfpathlineto{\pgfqpoint{1.626478in}{1.165547in}}%
\pgfpathlineto{\pgfqpoint{1.642135in}{1.173907in}}%
\pgfpathlineto{\pgfqpoint{1.644792in}{1.175524in}}%
\pgfpathlineto{\pgfqpoint{1.657791in}{1.183603in}}%
\pgfpathlineto{\pgfqpoint{1.665750in}{1.189135in}}%
\pgfpathlineto{\pgfqpoint{1.673448in}{1.194702in}}%
\pgfpathlineto{\pgfqpoint{1.683625in}{1.202746in}}%
\pgfpathlineto{\pgfqpoint{1.689104in}{1.207338in}}%
\pgfpathlineto{\pgfqpoint{1.699156in}{1.216357in}}%
\pgfpathlineto{\pgfqpoint{1.704761in}{1.221797in}}%
\pgfpathlineto{\pgfqpoint{1.712777in}{1.229968in}}%
\pgfpathlineto{\pgfqpoint{1.720418in}{1.238575in}}%
\pgfpathlineto{\pgfqpoint{1.724726in}{1.243579in}}%
\pgfpathlineto{\pgfqpoint{1.735094in}{1.257191in}}%
\pgfpathlineto{\pgfqpoint{1.736074in}{1.258679in}}%
\pgfpathlineto{\pgfqpoint{1.743983in}{1.270802in}}%
\pgfpathlineto{\pgfqpoint{1.751330in}{1.284413in}}%
\pgfpathlineto{\pgfqpoint{1.751731in}{1.285344in}}%
\pgfpathlineto{\pgfqpoint{1.757251in}{1.298024in}}%
\pgfpathlineto{\pgfqpoint{1.761596in}{1.311635in}}%
\pgfpathlineto{\pgfqpoint{1.764358in}{1.325246in}}%
\pgfpathlineto{\pgfqpoint{1.765542in}{1.338857in}}%
\pgfpathlineto{\pgfqpoint{1.765147in}{1.352468in}}%
\pgfpathlineto{\pgfqpoint{1.763175in}{1.366079in}}%
\pgfpathlineto{\pgfqpoint{1.759621in}{1.379691in}}%
\pgfpathlineto{\pgfqpoint{1.754484in}{1.393302in}}%
\pgfpathlineto{\pgfqpoint{1.751731in}{1.398916in}}%
\pgfpathlineto{\pgfqpoint{1.747851in}{1.406913in}}%
\pgfpathlineto{\pgfqpoint{1.739729in}{1.420524in}}%
\pgfpathlineto{\pgfqpoint{1.736074in}{1.425725in}}%
\pgfpathlineto{\pgfqpoint{1.730100in}{1.434135in}}%
\pgfpathlineto{\pgfqpoint{1.720418in}{1.446011in}}%
\pgfpathlineto{\pgfqpoint{1.718961in}{1.447746in}}%
\pgfpathlineto{\pgfqpoint{1.706221in}{1.461357in}}%
\pgfpathlineto{\pgfqpoint{1.704761in}{1.462787in}}%
\pgfpathlineto{\pgfqpoint{1.691685in}{1.474968in}}%
\pgfpathlineto{\pgfqpoint{1.689104in}{1.477212in}}%
\pgfpathlineto{\pgfqpoint{1.675092in}{1.488579in}}%
\pgfpathlineto{\pgfqpoint{1.673448in}{1.489849in}}%
\pgfpathlineto{\pgfqpoint{1.657791in}{1.500924in}}%
\pgfpathlineto{\pgfqpoint{1.655796in}{1.502191in}}%
\pgfpathlineto{\pgfqpoint{1.642135in}{1.510608in}}%
\pgfpathlineto{\pgfqpoint{1.632461in}{1.515802in}}%
\pgfpathlineto{\pgfqpoint{1.626478in}{1.518979in}}%
\pgfpathlineto{\pgfqpoint{1.610822in}{1.526040in}}%
\pgfpathlineto{\pgfqpoint{1.601623in}{1.529413in}}%
\pgfpathlineto{\pgfqpoint{1.595165in}{1.531807in}}%
\pgfpathlineto{\pgfqpoint{1.579508in}{1.536273in}}%
\pgfpathlineto{\pgfqpoint{1.563852in}{1.539362in}}%
\pgfpathlineto{\pgfqpoint{1.548195in}{1.541077in}}%
\pgfpathlineto{\pgfqpoint{1.532539in}{1.541419in}}%
\pgfpathlineto{\pgfqpoint{1.516882in}{1.540391in}}%
\pgfpathlineto{\pgfqpoint{1.501226in}{1.537989in}}%
\pgfpathlineto{\pgfqpoint{1.485569in}{1.534212in}}%
\pgfpathlineto{\pgfqpoint{1.470984in}{1.529413in}}%
\pgfpathlineto{\pgfqpoint{1.469913in}{1.529064in}}%
\pgfpathlineto{\pgfqpoint{1.454256in}{1.522678in}}%
\pgfpathlineto{\pgfqpoint{1.440312in}{1.515802in}}%
\pgfpathlineto{\pgfqpoint{1.438599in}{1.514949in}}%
\pgfpathlineto{\pgfqpoint{1.422943in}{1.505937in}}%
\pgfpathlineto{\pgfqpoint{1.417187in}{1.502191in}}%
\pgfpathlineto{\pgfqpoint{1.407286in}{1.495548in}}%
\pgfpathlineto{\pgfqpoint{1.397887in}{1.488579in}}%
\pgfpathlineto{\pgfqpoint{1.391630in}{1.483706in}}%
\pgfpathlineto{\pgfqpoint{1.381255in}{1.474968in}}%
\pgfpathlineto{\pgfqpoint{1.375973in}{1.470205in}}%
\pgfpathlineto{\pgfqpoint{1.366720in}{1.461357in}}%
\pgfpathlineto{\pgfqpoint{1.360317in}{1.454665in}}%
\pgfpathlineto{\pgfqpoint{1.353954in}{1.447746in}}%
\pgfpathlineto{\pgfqpoint{1.344660in}{1.436445in}}%
\pgfpathlineto{\pgfqpoint{1.342800in}{1.434135in}}%
\pgfpathlineto{\pgfqpoint{1.333184in}{1.420524in}}%
\pgfpathlineto{\pgfqpoint{1.329003in}{1.413532in}}%
\pgfpathlineto{\pgfqpoint{1.325047in}{1.406913in}}%
\pgfpathlineto{\pgfqpoint{1.318414in}{1.393302in}}%
\pgfpathlineto{\pgfqpoint{1.313347in}{1.379693in}}%
\pgfpathlineto{\pgfqpoint{1.313346in}{1.379691in}}%
\pgfpathlineto{\pgfqpoint{1.309729in}{1.366079in}}%
\pgfpathlineto{\pgfqpoint{1.307720in}{1.352468in}}%
\pgfpathlineto{\pgfqpoint{1.307319in}{1.338857in}}%
\pgfpathlineto{\pgfqpoint{1.308524in}{1.325246in}}%
\pgfpathlineto{\pgfqpoint{1.311336in}{1.311635in}}%
\pgfpathlineto{\pgfqpoint{1.313347in}{1.305420in}}%
\pgfpathlineto{\pgfqpoint{1.315684in}{1.298024in}}%
\pgfpathlineto{\pgfqpoint{1.321534in}{1.284413in}}%
\pgfpathlineto{\pgfqpoint{1.328951in}{1.270802in}}%
\pgfpathlineto{\pgfqpoint{1.329003in}{1.270721in}}%
\pgfpathlineto{\pgfqpoint{1.337800in}{1.257191in}}%
\pgfpathlineto{\pgfqpoint{1.344660in}{1.248140in}}%
\pgfpathlineto{\pgfqpoint{1.348186in}{1.243579in}}%
\pgfpathlineto{\pgfqpoint{1.360096in}{1.229968in}}%
\pgfpathlineto{\pgfqpoint{1.360317in}{1.229739in}}%
\pgfpathlineto{\pgfqpoint{1.373717in}{1.216357in}}%
\pgfpathlineto{\pgfqpoint{1.375973in}{1.214276in}}%
\pgfpathlineto{\pgfqpoint{1.389236in}{1.202746in}}%
\pgfpathlineto{\pgfqpoint{1.391630in}{1.200785in}}%
\pgfpathlineto{\pgfqpoint{1.407022in}{1.189135in}}%
\pgfpathlineto{\pgfqpoint{1.407286in}{1.188943in}}%
\pgfpathlineto{\pgfqpoint{1.422943in}{1.178589in}}%
\pgfpathlineto{\pgfqpoint{1.428188in}{1.175524in}}%
\pgfpathlineto{\pgfqpoint{1.438599in}{1.169560in}}%
\pgfpathlineto{\pgfqpoint{1.454163in}{1.161913in}}%
\pgfpathlineto{\pgfqpoint{1.454256in}{1.161867in}}%
\pgfpathlineto{\pgfqpoint{1.469913in}{1.155420in}}%
\pgfpathlineto{\pgfqpoint{1.485569in}{1.150334in}}%
\pgfpathlineto{\pgfqpoint{1.494077in}{1.148302in}}%
\pgfpathlineto{\pgfqpoint{1.501226in}{1.146554in}}%
\pgfpathclose%
\pgfpathmoveto{\pgfqpoint{1.508185in}{1.202746in}}%
\pgfpathlineto{\pgfqpoint{1.501226in}{1.203981in}}%
\pgfpathlineto{\pgfqpoint{1.485569in}{1.208337in}}%
\pgfpathlineto{\pgfqpoint{1.469913in}{1.214283in}}%
\pgfpathlineto{\pgfqpoint{1.465562in}{1.216357in}}%
\pgfpathlineto{\pgfqpoint{1.454256in}{1.222116in}}%
\pgfpathlineto{\pgfqpoint{1.441441in}{1.229968in}}%
\pgfpathlineto{\pgfqpoint{1.438599in}{1.231872in}}%
\pgfpathlineto{\pgfqpoint{1.423491in}{1.243579in}}%
\pgfpathlineto{\pgfqpoint{1.422943in}{1.244056in}}%
\pgfpathlineto{\pgfqpoint{1.409475in}{1.257191in}}%
\pgfpathlineto{\pgfqpoint{1.407286in}{1.259661in}}%
\pgfpathlineto{\pgfqpoint{1.398254in}{1.270802in}}%
\pgfpathlineto{\pgfqpoint{1.391630in}{1.280631in}}%
\pgfpathlineto{\pgfqpoint{1.389244in}{1.284413in}}%
\pgfpathlineto{\pgfqpoint{1.382404in}{1.298024in}}%
\pgfpathlineto{\pgfqpoint{1.377393in}{1.311635in}}%
\pgfpathlineto{\pgfqpoint{1.375973in}{1.317685in}}%
\pgfpathlineto{\pgfqpoint{1.374267in}{1.325246in}}%
\pgfpathlineto{\pgfqpoint{1.372949in}{1.338857in}}%
\pgfpathlineto{\pgfqpoint{1.373388in}{1.352468in}}%
\pgfpathlineto{\pgfqpoint{1.375586in}{1.366079in}}%
\pgfpathlineto{\pgfqpoint{1.375973in}{1.367418in}}%
\pgfpathlineto{\pgfqpoint{1.379670in}{1.379691in}}%
\pgfpathlineto{\pgfqpoint{1.385595in}{1.393302in}}%
\pgfpathlineto{\pgfqpoint{1.391630in}{1.403921in}}%
\pgfpathlineto{\pgfqpoint{1.393443in}{1.406913in}}%
\pgfpathlineto{\pgfqpoint{1.403547in}{1.420524in}}%
\pgfpathlineto{\pgfqpoint{1.407286in}{1.424806in}}%
\pgfpathlineto{\pgfqpoint{1.416202in}{1.434135in}}%
\pgfpathlineto{\pgfqpoint{1.422943in}{1.440324in}}%
\pgfpathlineto{\pgfqpoint{1.432022in}{1.447746in}}%
\pgfpathlineto{\pgfqpoint{1.438599in}{1.452598in}}%
\pgfpathlineto{\pgfqpoint{1.452354in}{1.461357in}}%
\pgfpathlineto{\pgfqpoint{1.454256in}{1.462477in}}%
\pgfpathlineto{\pgfqpoint{1.469913in}{1.470198in}}%
\pgfpathlineto{\pgfqpoint{1.482148in}{1.474968in}}%
\pgfpathlineto{\pgfqpoint{1.485569in}{1.476232in}}%
\pgfpathlineto{\pgfqpoint{1.501226in}{1.480504in}}%
\pgfpathlineto{\pgfqpoint{1.516882in}{1.483221in}}%
\pgfpathlineto{\pgfqpoint{1.532539in}{1.484385in}}%
\pgfpathlineto{\pgfqpoint{1.548195in}{1.483997in}}%
\pgfpathlineto{\pgfqpoint{1.563852in}{1.482057in}}%
\pgfpathlineto{\pgfqpoint{1.579508in}{1.478563in}}%
\pgfpathlineto{\pgfqpoint{1.590679in}{1.474968in}}%
\pgfpathlineto{\pgfqpoint{1.595165in}{1.473447in}}%
\pgfpathlineto{\pgfqpoint{1.610822in}{1.466541in}}%
\pgfpathlineto{\pgfqpoint{1.620386in}{1.461357in}}%
\pgfpathlineto{\pgfqpoint{1.626478in}{1.457790in}}%
\pgfpathlineto{\pgfqpoint{1.641036in}{1.447746in}}%
\pgfpathlineto{\pgfqpoint{1.642135in}{1.446907in}}%
\pgfpathlineto{\pgfqpoint{1.656826in}{1.434135in}}%
\pgfpathlineto{\pgfqpoint{1.657791in}{1.433180in}}%
\pgfpathlineto{\pgfqpoint{1.669344in}{1.420524in}}%
\pgfpathlineto{\pgfqpoint{1.673448in}{1.415228in}}%
\pgfpathlineto{\pgfqpoint{1.679411in}{1.406913in}}%
\pgfpathlineto{\pgfqpoint{1.687354in}{1.393302in}}%
\pgfpathlineto{\pgfqpoint{1.689104in}{1.389402in}}%
\pgfpathlineto{\pgfqpoint{1.693239in}{1.379691in}}%
\pgfpathlineto{\pgfqpoint{1.697258in}{1.366079in}}%
\pgfpathlineto{\pgfqpoint{1.699490in}{1.352468in}}%
\pgfpathlineto{\pgfqpoint{1.699936in}{1.338857in}}%
\pgfpathlineto{\pgfqpoint{1.698597in}{1.325246in}}%
\pgfpathlineto{\pgfqpoint{1.695472in}{1.311635in}}%
\pgfpathlineto{\pgfqpoint{1.690558in}{1.298024in}}%
\pgfpathlineto{\pgfqpoint{1.689104in}{1.295050in}}%
\pgfpathlineto{\pgfqpoint{1.683617in}{1.284413in}}%
\pgfpathlineto{\pgfqpoint{1.674736in}{1.270802in}}%
\pgfpathlineto{\pgfqpoint{1.673448in}{1.269148in}}%
\pgfpathlineto{\pgfqpoint{1.663372in}{1.257191in}}%
\pgfpathlineto{\pgfqpoint{1.657791in}{1.251472in}}%
\pgfpathlineto{\pgfqpoint{1.649254in}{1.243579in}}%
\pgfpathlineto{\pgfqpoint{1.642135in}{1.237719in}}%
\pgfpathlineto{\pgfqpoint{1.631404in}{1.229968in}}%
\pgfpathlineto{\pgfqpoint{1.626478in}{1.226718in}}%
\pgfpathlineto{\pgfqpoint{1.610822in}{1.217934in}}%
\pgfpathlineto{\pgfqpoint{1.607381in}{1.216357in}}%
\pgfpathlineto{\pgfqpoint{1.595165in}{1.211111in}}%
\pgfpathlineto{\pgfqpoint{1.579508in}{1.205960in}}%
\pgfpathlineto{\pgfqpoint{1.565392in}{1.202746in}}%
\pgfpathlineto{\pgfqpoint{1.563852in}{1.202410in}}%
\pgfpathlineto{\pgfqpoint{1.548195in}{1.200499in}}%
\pgfpathlineto{\pgfqpoint{1.532539in}{1.200117in}}%
\pgfpathlineto{\pgfqpoint{1.516882in}{1.201263in}}%
\pgfpathlineto{\pgfqpoint{1.508185in}{1.202746in}}%
\pgfpathclose%
\pgfusepath{fill}%
\end{pgfscope}%
\begin{pgfscope}%
\pgfpathrectangle{\pgfqpoint{0.373953in}{0.331635in}}{\pgfqpoint{1.550000in}{1.347500in}}%
\pgfusepath{clip}%
\pgfsetbuttcap%
\pgfsetroundjoin%
\definecolor{currentfill}{rgb}{0.709962,0.212797,0.477201}%
\pgfsetfillcolor{currentfill}%
\pgfsetlinewidth{0.000000pt}%
\definecolor{currentstroke}{rgb}{0.000000,0.000000,0.000000}%
\pgfsetstrokecolor{currentstroke}%
\pgfsetdash{}{0pt}%
\pgfpathmoveto{\pgfqpoint{0.749710in}{0.399531in}}%
\pgfpathlineto{\pgfqpoint{0.765367in}{0.398990in}}%
\pgfpathlineto{\pgfqpoint{0.772136in}{0.399691in}}%
\pgfpathlineto{\pgfqpoint{0.781024in}{0.400466in}}%
\pgfpathlineto{\pgfqpoint{0.796680in}{0.403653in}}%
\pgfpathlineto{\pgfqpoint{0.812337in}{0.408664in}}%
\pgfpathlineto{\pgfqpoint{0.822989in}{0.413302in}}%
\pgfpathlineto{\pgfqpoint{0.827993in}{0.415228in}}%
\pgfpathlineto{\pgfqpoint{0.843650in}{0.422813in}}%
\pgfpathlineto{\pgfqpoint{0.850702in}{0.426913in}}%
\pgfpathlineto{\pgfqpoint{0.859306in}{0.431496in}}%
\pgfpathlineto{\pgfqpoint{0.873880in}{0.440524in}}%
\pgfpathlineto{\pgfqpoint{0.874963in}{0.441155in}}%
\pgfpathlineto{\pgfqpoint{0.890620in}{0.451432in}}%
\pgfpathlineto{\pgfqpoint{0.894348in}{0.454135in}}%
\pgfpathlineto{\pgfqpoint{0.906276in}{0.462480in}}%
\pgfpathlineto{\pgfqpoint{0.913215in}{0.467746in}}%
\pgfpathlineto{\pgfqpoint{0.921933in}{0.474261in}}%
\pgfpathlineto{\pgfqpoint{0.930843in}{0.481357in}}%
\pgfpathlineto{\pgfqpoint{0.937589in}{0.486748in}}%
\pgfpathlineto{\pgfqpoint{0.947410in}{0.494968in}}%
\pgfpathlineto{\pgfqpoint{0.953246in}{0.499957in}}%
\pgfpathlineto{\pgfqpoint{0.963040in}{0.508579in}}%
\pgfpathlineto{\pgfqpoint{0.968902in}{0.513941in}}%
\pgfpathlineto{\pgfqpoint{0.977813in}{0.522191in}}%
\pgfpathlineto{\pgfqpoint{0.984559in}{0.528788in}}%
\pgfpathlineto{\pgfqpoint{0.991771in}{0.535802in}}%
\pgfpathlineto{\pgfqpoint{1.000216in}{0.544629in}}%
\pgfpathlineto{\pgfqpoint{1.004908in}{0.549413in}}%
\pgfpathlineto{\pgfqpoint{1.015872in}{0.561650in}}%
\pgfpathlineto{\pgfqpoint{1.017162in}{0.563024in}}%
\pgfpathlineto{\pgfqpoint{1.028652in}{0.576635in}}%
\pgfpathlineto{\pgfqpoint{1.031529in}{0.580534in}}%
\pgfpathlineto{\pgfqpoint{1.039241in}{0.590246in}}%
\pgfpathlineto{\pgfqpoint{1.047185in}{0.602070in}}%
\pgfpathlineto{\pgfqpoint{1.048516in}{0.603857in}}%
\pgfpathlineto{\pgfqpoint{1.056824in}{0.617468in}}%
\pgfpathlineto{\pgfqpoint{1.062842in}{0.630365in}}%
\pgfpathlineto{\pgfqpoint{1.063226in}{0.631079in}}%
\pgfpathlineto{\pgfqpoint{1.068338in}{0.644691in}}%
\pgfpathlineto{\pgfqpoint{1.071176in}{0.658302in}}%
\pgfpathlineto{\pgfqpoint{1.071743in}{0.671913in}}%
\pgfpathlineto{\pgfqpoint{1.070041in}{0.685524in}}%
\pgfpathlineto{\pgfqpoint{1.066067in}{0.699135in}}%
\pgfpathlineto{\pgfqpoint{1.062842in}{0.706184in}}%
\pgfpathlineto{\pgfqpoint{1.060243in}{0.712746in}}%
\pgfpathlineto{\pgfqpoint{1.052916in}{0.726357in}}%
\pgfpathlineto{\pgfqpoint{1.047185in}{0.734797in}}%
\pgfpathlineto{\pgfqpoint{1.044022in}{0.739968in}}%
\pgfpathlineto{\pgfqpoint{1.034026in}{0.753579in}}%
\pgfpathlineto{\pgfqpoint{1.031529in}{0.756521in}}%
\pgfpathlineto{\pgfqpoint{1.023102in}{0.767191in}}%
\pgfpathlineto{\pgfqpoint{1.015872in}{0.775282in}}%
\pgfpathlineto{\pgfqpoint{1.011170in}{0.780802in}}%
\pgfpathlineto{\pgfqpoint{1.000216in}{0.792424in}}%
\pgfpathlineto{\pgfqpoint{0.998389in}{0.794413in}}%
\pgfpathlineto{\pgfqpoint{0.984836in}{0.808024in}}%
\pgfpathlineto{\pgfqpoint{0.984559in}{0.808286in}}%
\pgfpathlineto{\pgfqpoint{0.970516in}{0.821635in}}%
\pgfpathlineto{\pgfqpoint{0.968902in}{0.823099in}}%
\pgfpathlineto{\pgfqpoint{0.955341in}{0.835246in}}%
\pgfpathlineto{\pgfqpoint{0.953246in}{0.837068in}}%
\pgfpathlineto{\pgfqpoint{0.939273in}{0.848857in}}%
\pgfpathlineto{\pgfqpoint{0.937589in}{0.850260in}}%
\pgfpathlineto{\pgfqpoint{0.922234in}{0.862468in}}%
\pgfpathlineto{\pgfqpoint{0.921933in}{0.862709in}}%
\pgfpathlineto{\pgfqpoint{0.906276in}{0.874492in}}%
\pgfpathlineto{\pgfqpoint{0.903988in}{0.876079in}}%
\pgfpathlineto{\pgfqpoint{0.890620in}{0.885603in}}%
\pgfpathlineto{\pgfqpoint{0.884270in}{0.889691in}}%
\pgfpathlineto{\pgfqpoint{0.874963in}{0.895976in}}%
\pgfpathlineto{\pgfqpoint{0.862690in}{0.903302in}}%
\pgfpathlineto{\pgfqpoint{0.859306in}{0.905473in}}%
\pgfpathlineto{\pgfqpoint{0.843650in}{0.914163in}}%
\pgfpathlineto{\pgfqpoint{0.837702in}{0.916913in}}%
\pgfpathlineto{\pgfqpoint{0.827993in}{0.921894in}}%
\pgfpathlineto{\pgfqpoint{0.812337in}{0.928265in}}%
\pgfpathlineto{\pgfqpoint{0.804789in}{0.930524in}}%
\pgfpathlineto{\pgfqpoint{0.796680in}{0.933327in}}%
\pgfpathlineto{\pgfqpoint{0.781024in}{0.936783in}}%
\pgfpathlineto{\pgfqpoint{0.765367in}{0.938263in}}%
\pgfpathlineto{\pgfqpoint{0.749710in}{0.937769in}}%
\pgfpathlineto{\pgfqpoint{0.734054in}{0.935302in}}%
\pgfpathlineto{\pgfqpoint{0.718397in}{0.930858in}}%
\pgfpathlineto{\pgfqpoint{0.717576in}{0.930524in}}%
\pgfpathlineto{\pgfqpoint{0.702741in}{0.925293in}}%
\pgfpathlineto{\pgfqpoint{0.687084in}{0.918070in}}%
\pgfpathlineto{\pgfqpoint{0.685029in}{0.916913in}}%
\pgfpathlineto{\pgfqpoint{0.671428in}{0.910007in}}%
\pgfpathlineto{\pgfqpoint{0.660256in}{0.903302in}}%
\pgfpathlineto{\pgfqpoint{0.655771in}{0.900801in}}%
\pgfpathlineto{\pgfqpoint{0.640115in}{0.890812in}}%
\pgfpathlineto{\pgfqpoint{0.638534in}{0.889691in}}%
\pgfpathlineto{\pgfqpoint{0.624458in}{0.880159in}}%
\pgfpathlineto{\pgfqpoint{0.618955in}{0.876079in}}%
\pgfpathlineto{\pgfqpoint{0.608801in}{0.868738in}}%
\pgfpathlineto{\pgfqpoint{0.600734in}{0.862468in}}%
\pgfpathlineto{\pgfqpoint{0.593145in}{0.856604in}}%
\pgfpathlineto{\pgfqpoint{0.583655in}{0.848857in}}%
\pgfpathlineto{\pgfqpoint{0.577488in}{0.843760in}}%
\pgfpathlineto{\pgfqpoint{0.567570in}{0.835246in}}%
\pgfpathlineto{\pgfqpoint{0.561832in}{0.830173in}}%
\pgfpathlineto{\pgfqpoint{0.552376in}{0.821635in}}%
\pgfpathlineto{\pgfqpoint{0.546175in}{0.815770in}}%
\pgfpathlineto{\pgfqpoint{0.538013in}{0.808024in}}%
\pgfpathlineto{\pgfqpoint{0.530519in}{0.800445in}}%
\pgfpathlineto{\pgfqpoint{0.524461in}{0.794413in}}%
\pgfpathlineto{\pgfqpoint{0.514862in}{0.784043in}}%
\pgfpathlineto{\pgfqpoint{0.511753in}{0.780802in}}%
\pgfpathlineto{\pgfqpoint{0.499932in}{0.767191in}}%
\pgfpathlineto{\pgfqpoint{0.499205in}{0.766249in}}%
\pgfpathlineto{\pgfqpoint{0.488820in}{0.753579in}}%
\pgfpathlineto{\pgfqpoint{0.483549in}{0.746099in}}%
\pgfpathlineto{\pgfqpoint{0.478833in}{0.739968in}}%
\pgfpathlineto{\pgfqpoint{0.470108in}{0.726357in}}%
\pgfpathlineto{\pgfqpoint{0.467892in}{0.722006in}}%
\pgfpathlineto{\pgfqpoint{0.462558in}{0.712746in}}%
\pgfpathlineto{\pgfqpoint{0.456793in}{0.699135in}}%
\pgfpathlineto{\pgfqpoint{0.453128in}{0.685524in}}%
\pgfpathlineto{\pgfqpoint{0.452236in}{0.677797in}}%
\pgfpathlineto{\pgfqpoint{0.451429in}{0.671913in}}%
\pgfpathlineto{\pgfqpoint{0.452052in}{0.658302in}}%
\pgfpathlineto{\pgfqpoint{0.452236in}{0.657495in}}%
\pgfpathlineto{\pgfqpoint{0.454698in}{0.644691in}}%
\pgfpathlineto{\pgfqpoint{0.459413in}{0.631079in}}%
\pgfpathlineto{\pgfqpoint{0.466229in}{0.617468in}}%
\pgfpathlineto{\pgfqpoint{0.467892in}{0.614905in}}%
\pgfpathlineto{\pgfqpoint{0.474240in}{0.603857in}}%
\pgfpathlineto{\pgfqpoint{0.483549in}{0.590714in}}%
\pgfpathlineto{\pgfqpoint{0.483852in}{0.590246in}}%
\pgfpathlineto{\pgfqpoint{0.494201in}{0.576635in}}%
\pgfpathlineto{\pgfqpoint{0.499205in}{0.570888in}}%
\pgfpathlineto{\pgfqpoint{0.505657in}{0.563024in}}%
\pgfpathlineto{\pgfqpoint{0.514862in}{0.552990in}}%
\pgfpathlineto{\pgfqpoint{0.518026in}{0.549413in}}%
\pgfpathlineto{\pgfqpoint{0.530519in}{0.536518in}}%
\pgfpathlineto{\pgfqpoint{0.531201in}{0.535802in}}%
\pgfpathlineto{\pgfqpoint{0.545124in}{0.522191in}}%
\pgfpathlineto{\pgfqpoint{0.546175in}{0.521216in}}%
\pgfpathlineto{\pgfqpoint{0.559869in}{0.508579in}}%
\pgfpathlineto{\pgfqpoint{0.561832in}{0.506836in}}%
\pgfpathlineto{\pgfqpoint{0.575483in}{0.494968in}}%
\pgfpathlineto{\pgfqpoint{0.577488in}{0.493262in}}%
\pgfpathlineto{\pgfqpoint{0.592024in}{0.481357in}}%
\pgfpathlineto{\pgfqpoint{0.593145in}{0.480443in}}%
\pgfpathlineto{\pgfqpoint{0.608801in}{0.468340in}}%
\pgfpathlineto{\pgfqpoint{0.609625in}{0.467746in}}%
\pgfpathlineto{\pgfqpoint{0.624458in}{0.456886in}}%
\pgfpathlineto{\pgfqpoint{0.628573in}{0.454135in}}%
\pgfpathlineto{\pgfqpoint{0.640115in}{0.446132in}}%
\pgfpathlineto{\pgfqpoint{0.649160in}{0.440524in}}%
\pgfpathlineto{\pgfqpoint{0.655771in}{0.436173in}}%
\pgfpathlineto{\pgfqpoint{0.671428in}{0.427176in}}%
\pgfpathlineto{\pgfqpoint{0.671966in}{0.426913in}}%
\pgfpathlineto{\pgfqpoint{0.687084in}{0.418820in}}%
\pgfpathlineto{\pgfqpoint{0.699792in}{0.413302in}}%
\pgfpathlineto{\pgfqpoint{0.702741in}{0.411855in}}%
\pgfpathlineto{\pgfqpoint{0.718397in}{0.405930in}}%
\pgfpathlineto{\pgfqpoint{0.734054in}{0.401831in}}%
\pgfpathlineto{\pgfqpoint{0.748782in}{0.399691in}}%
\pgfpathlineto{\pgfqpoint{0.749710in}{0.399531in}}%
\pgfpathclose%
\pgfpathmoveto{\pgfqpoint{0.696283in}{0.481357in}}%
\pgfpathlineto{\pgfqpoint{0.687084in}{0.484730in}}%
\pgfpathlineto{\pgfqpoint{0.671428in}{0.491791in}}%
\pgfpathlineto{\pgfqpoint{0.665445in}{0.494968in}}%
\pgfpathlineto{\pgfqpoint{0.655771in}{0.500162in}}%
\pgfpathlineto{\pgfqpoint{0.642110in}{0.508579in}}%
\pgfpathlineto{\pgfqpoint{0.640115in}{0.509846in}}%
\pgfpathlineto{\pgfqpoint{0.624458in}{0.520921in}}%
\pgfpathlineto{\pgfqpoint{0.622813in}{0.522191in}}%
\pgfpathlineto{\pgfqpoint{0.608801in}{0.533558in}}%
\pgfpathlineto{\pgfqpoint{0.606221in}{0.535802in}}%
\pgfpathlineto{\pgfqpoint{0.593145in}{0.547983in}}%
\pgfpathlineto{\pgfqpoint{0.591685in}{0.549413in}}%
\pgfpathlineto{\pgfqpoint{0.578945in}{0.563024in}}%
\pgfpathlineto{\pgfqpoint{0.577488in}{0.564759in}}%
\pgfpathlineto{\pgfqpoint{0.567805in}{0.576635in}}%
\pgfpathlineto{\pgfqpoint{0.561832in}{0.585045in}}%
\pgfpathlineto{\pgfqpoint{0.558177in}{0.590246in}}%
\pgfpathlineto{\pgfqpoint{0.550055in}{0.603857in}}%
\pgfpathlineto{\pgfqpoint{0.546175in}{0.611854in}}%
\pgfpathlineto{\pgfqpoint{0.543422in}{0.617468in}}%
\pgfpathlineto{\pgfqpoint{0.538284in}{0.631079in}}%
\pgfpathlineto{\pgfqpoint{0.534731in}{0.644691in}}%
\pgfpathlineto{\pgfqpoint{0.532759in}{0.658302in}}%
\pgfpathlineto{\pgfqpoint{0.532364in}{0.671913in}}%
\pgfpathlineto{\pgfqpoint{0.533548in}{0.685524in}}%
\pgfpathlineto{\pgfqpoint{0.536310in}{0.699135in}}%
\pgfpathlineto{\pgfqpoint{0.540655in}{0.712746in}}%
\pgfpathlineto{\pgfqpoint{0.546175in}{0.725426in}}%
\pgfpathlineto{\pgfqpoint{0.546576in}{0.726357in}}%
\pgfpathlineto{\pgfqpoint{0.553922in}{0.739968in}}%
\pgfpathlineto{\pgfqpoint{0.561832in}{0.752091in}}%
\pgfpathlineto{\pgfqpoint{0.562812in}{0.753579in}}%
\pgfpathlineto{\pgfqpoint{0.573179in}{0.767191in}}%
\pgfpathlineto{\pgfqpoint{0.577488in}{0.772195in}}%
\pgfpathlineto{\pgfqpoint{0.585129in}{0.780802in}}%
\pgfpathlineto{\pgfqpoint{0.593145in}{0.788973in}}%
\pgfpathlineto{\pgfqpoint{0.598750in}{0.794413in}}%
\pgfpathlineto{\pgfqpoint{0.608801in}{0.803432in}}%
\pgfpathlineto{\pgfqpoint{0.614280in}{0.808024in}}%
\pgfpathlineto{\pgfqpoint{0.624458in}{0.816068in}}%
\pgfpathlineto{\pgfqpoint{0.632155in}{0.821635in}}%
\pgfpathlineto{\pgfqpoint{0.640115in}{0.827167in}}%
\pgfpathlineto{\pgfqpoint{0.653114in}{0.835246in}}%
\pgfpathlineto{\pgfqpoint{0.655771in}{0.836863in}}%
\pgfpathlineto{\pgfqpoint{0.671428in}{0.845223in}}%
\pgfpathlineto{\pgfqpoint{0.679471in}{0.848857in}}%
\pgfpathlineto{\pgfqpoint{0.687084in}{0.852297in}}%
\pgfpathlineto{\pgfqpoint{0.702741in}{0.858063in}}%
\pgfpathlineto{\pgfqpoint{0.718395in}{0.862468in}}%
\pgfpathlineto{\pgfqpoint{0.718397in}{0.862469in}}%
\pgfpathlineto{\pgfqpoint{0.734054in}{0.865614in}}%
\pgfpathlineto{\pgfqpoint{0.749710in}{0.867360in}}%
\pgfpathlineto{\pgfqpoint{0.765367in}{0.867709in}}%
\pgfpathlineto{\pgfqpoint{0.781024in}{0.866661in}}%
\pgfpathlineto{\pgfqpoint{0.796680in}{0.864216in}}%
\pgfpathlineto{\pgfqpoint{0.803829in}{0.862468in}}%
\pgfpathlineto{\pgfqpoint{0.812337in}{0.860436in}}%
\pgfpathlineto{\pgfqpoint{0.827993in}{0.855350in}}%
\pgfpathlineto{\pgfqpoint{0.843650in}{0.848903in}}%
\pgfpathlineto{\pgfqpoint{0.843743in}{0.848857in}}%
\pgfpathlineto{\pgfqpoint{0.859306in}{0.841210in}}%
\pgfpathlineto{\pgfqpoint{0.869717in}{0.835246in}}%
\pgfpathlineto{\pgfqpoint{0.874963in}{0.832181in}}%
\pgfpathlineto{\pgfqpoint{0.890620in}{0.821827in}}%
\pgfpathlineto{\pgfqpoint{0.890884in}{0.821635in}}%
\pgfpathlineto{\pgfqpoint{0.906276in}{0.809985in}}%
\pgfpathlineto{\pgfqpoint{0.908670in}{0.808024in}}%
\pgfpathlineto{\pgfqpoint{0.921933in}{0.796494in}}%
\pgfpathlineto{\pgfqpoint{0.924189in}{0.794413in}}%
\pgfpathlineto{\pgfqpoint{0.937589in}{0.781031in}}%
\pgfpathlineto{\pgfqpoint{0.937810in}{0.780802in}}%
\pgfpathlineto{\pgfqpoint{0.949720in}{0.767191in}}%
\pgfpathlineto{\pgfqpoint{0.953246in}{0.762630in}}%
\pgfpathlineto{\pgfqpoint{0.960106in}{0.753579in}}%
\pgfpathlineto{\pgfqpoint{0.968902in}{0.740049in}}%
\pgfpathlineto{\pgfqpoint{0.968955in}{0.739968in}}%
\pgfpathlineto{\pgfqpoint{0.976371in}{0.726357in}}%
\pgfpathlineto{\pgfqpoint{0.982222in}{0.712746in}}%
\pgfpathlineto{\pgfqpoint{0.984559in}{0.705350in}}%
\pgfpathlineto{\pgfqpoint{0.986570in}{0.699135in}}%
\pgfpathlineto{\pgfqpoint{0.989382in}{0.685524in}}%
\pgfpathlineto{\pgfqpoint{0.990587in}{0.671913in}}%
\pgfpathlineto{\pgfqpoint{0.990185in}{0.658302in}}%
\pgfpathlineto{\pgfqpoint{0.988177in}{0.644691in}}%
\pgfpathlineto{\pgfqpoint{0.984560in}{0.631079in}}%
\pgfpathlineto{\pgfqpoint{0.984559in}{0.631077in}}%
\pgfpathlineto{\pgfqpoint{0.979492in}{0.617468in}}%
\pgfpathlineto{\pgfqpoint{0.972859in}{0.603857in}}%
\pgfpathlineto{\pgfqpoint{0.968902in}{0.597238in}}%
\pgfpathlineto{\pgfqpoint{0.964722in}{0.590246in}}%
\pgfpathlineto{\pgfqpoint{0.955106in}{0.576635in}}%
\pgfpathlineto{\pgfqpoint{0.953246in}{0.574325in}}%
\pgfpathlineto{\pgfqpoint{0.943952in}{0.563024in}}%
\pgfpathlineto{\pgfqpoint{0.937589in}{0.556105in}}%
\pgfpathlineto{\pgfqpoint{0.931186in}{0.549413in}}%
\pgfpathlineto{\pgfqpoint{0.921933in}{0.540565in}}%
\pgfpathlineto{\pgfqpoint{0.916651in}{0.535802in}}%
\pgfpathlineto{\pgfqpoint{0.906276in}{0.527064in}}%
\pgfpathlineto{\pgfqpoint{0.900018in}{0.522191in}}%
\pgfpathlineto{\pgfqpoint{0.890620in}{0.515222in}}%
\pgfpathlineto{\pgfqpoint{0.880719in}{0.508579in}}%
\pgfpathlineto{\pgfqpoint{0.874963in}{0.504833in}}%
\pgfpathlineto{\pgfqpoint{0.859306in}{0.495821in}}%
\pgfpathlineto{\pgfqpoint{0.857594in}{0.494968in}}%
\pgfpathlineto{\pgfqpoint{0.843650in}{0.488092in}}%
\pgfpathlineto{\pgfqpoint{0.827993in}{0.481706in}}%
\pgfpathlineto{\pgfqpoint{0.826922in}{0.481357in}}%
\pgfpathlineto{\pgfqpoint{0.812337in}{0.476558in}}%
\pgfpathlineto{\pgfqpoint{0.796680in}{0.472781in}}%
\pgfpathlineto{\pgfqpoint{0.781024in}{0.470379in}}%
\pgfpathlineto{\pgfqpoint{0.765367in}{0.469351in}}%
\pgfpathlineto{\pgfqpoint{0.749710in}{0.469693in}}%
\pgfpathlineto{\pgfqpoint{0.734054in}{0.471408in}}%
\pgfpathlineto{\pgfqpoint{0.718397in}{0.474497in}}%
\pgfpathlineto{\pgfqpoint{0.702741in}{0.478963in}}%
\pgfpathlineto{\pgfqpoint{0.696283in}{0.481357in}}%
\pgfpathclose%
\pgfpathmoveto{\pgfqpoint{1.532539in}{0.398990in}}%
\pgfpathlineto{\pgfqpoint{1.548195in}{0.399531in}}%
\pgfpathlineto{\pgfqpoint{1.549123in}{0.399691in}}%
\pgfpathlineto{\pgfqpoint{1.563852in}{0.401831in}}%
\pgfpathlineto{\pgfqpoint{1.579508in}{0.405930in}}%
\pgfpathlineto{\pgfqpoint{1.595165in}{0.411855in}}%
\pgfpathlineto{\pgfqpoint{1.598114in}{0.413302in}}%
\pgfpathlineto{\pgfqpoint{1.610822in}{0.418820in}}%
\pgfpathlineto{\pgfqpoint{1.625940in}{0.426913in}}%
\pgfpathlineto{\pgfqpoint{1.626478in}{0.427176in}}%
\pgfpathlineto{\pgfqpoint{1.642135in}{0.436173in}}%
\pgfpathlineto{\pgfqpoint{1.648746in}{0.440524in}}%
\pgfpathlineto{\pgfqpoint{1.657791in}{0.446132in}}%
\pgfpathlineto{\pgfqpoint{1.669333in}{0.454135in}}%
\pgfpathlineto{\pgfqpoint{1.673448in}{0.456886in}}%
\pgfpathlineto{\pgfqpoint{1.688281in}{0.467746in}}%
\pgfpathlineto{\pgfqpoint{1.689104in}{0.468340in}}%
\pgfpathlineto{\pgfqpoint{1.704761in}{0.480443in}}%
\pgfpathlineto{\pgfqpoint{1.705882in}{0.481357in}}%
\pgfpathlineto{\pgfqpoint{1.720418in}{0.493262in}}%
\pgfpathlineto{\pgfqpoint{1.722423in}{0.494968in}}%
\pgfpathlineto{\pgfqpoint{1.736074in}{0.506836in}}%
\pgfpathlineto{\pgfqpoint{1.738037in}{0.508579in}}%
\pgfpathlineto{\pgfqpoint{1.751731in}{0.521216in}}%
\pgfpathlineto{\pgfqpoint{1.752782in}{0.522191in}}%
\pgfpathlineto{\pgfqpoint{1.766704in}{0.535802in}}%
\pgfpathlineto{\pgfqpoint{1.767387in}{0.536518in}}%
\pgfpathlineto{\pgfqpoint{1.779880in}{0.549413in}}%
\pgfpathlineto{\pgfqpoint{1.783044in}{0.552990in}}%
\pgfpathlineto{\pgfqpoint{1.792249in}{0.563024in}}%
\pgfpathlineto{\pgfqpoint{1.798700in}{0.570888in}}%
\pgfpathlineto{\pgfqpoint{1.803705in}{0.576635in}}%
\pgfpathlineto{\pgfqpoint{1.814054in}{0.590246in}}%
\pgfpathlineto{\pgfqpoint{1.814357in}{0.590714in}}%
\pgfpathlineto{\pgfqpoint{1.823666in}{0.603857in}}%
\pgfpathlineto{\pgfqpoint{1.830014in}{0.614905in}}%
\pgfpathlineto{\pgfqpoint{1.831677in}{0.617468in}}%
\pgfpathlineto{\pgfqpoint{1.838493in}{0.631079in}}%
\pgfpathlineto{\pgfqpoint{1.843207in}{0.644691in}}%
\pgfpathlineto{\pgfqpoint{1.845670in}{0.657495in}}%
\pgfpathlineto{\pgfqpoint{1.845854in}{0.658302in}}%
\pgfpathlineto{\pgfqpoint{1.846476in}{0.671913in}}%
\pgfpathlineto{\pgfqpoint{1.845670in}{0.677797in}}%
\pgfpathlineto{\pgfqpoint{1.844778in}{0.685524in}}%
\pgfpathlineto{\pgfqpoint{1.841113in}{0.699135in}}%
\pgfpathlineto{\pgfqpoint{1.835348in}{0.712746in}}%
\pgfpathlineto{\pgfqpoint{1.830014in}{0.722006in}}%
\pgfpathlineto{\pgfqpoint{1.827797in}{0.726357in}}%
\pgfpathlineto{\pgfqpoint{1.819073in}{0.739968in}}%
\pgfpathlineto{\pgfqpoint{1.814357in}{0.746099in}}%
\pgfpathlineto{\pgfqpoint{1.809085in}{0.753579in}}%
\pgfpathlineto{\pgfqpoint{1.798700in}{0.766249in}}%
\pgfpathlineto{\pgfqpoint{1.797974in}{0.767191in}}%
\pgfpathlineto{\pgfqpoint{1.786152in}{0.780802in}}%
\pgfpathlineto{\pgfqpoint{1.783044in}{0.784043in}}%
\pgfpathlineto{\pgfqpoint{1.773444in}{0.794413in}}%
\pgfpathlineto{\pgfqpoint{1.767387in}{0.800445in}}%
\pgfpathlineto{\pgfqpoint{1.759893in}{0.808024in}}%
\pgfpathlineto{\pgfqpoint{1.751731in}{0.815770in}}%
\pgfpathlineto{\pgfqpoint{1.745530in}{0.821635in}}%
\pgfpathlineto{\pgfqpoint{1.736074in}{0.830173in}}%
\pgfpathlineto{\pgfqpoint{1.730336in}{0.835246in}}%
\pgfpathlineto{\pgfqpoint{1.720418in}{0.843760in}}%
\pgfpathlineto{\pgfqpoint{1.714250in}{0.848857in}}%
\pgfpathlineto{\pgfqpoint{1.704761in}{0.856604in}}%
\pgfpathlineto{\pgfqpoint{1.697172in}{0.862468in}}%
\pgfpathlineto{\pgfqpoint{1.689104in}{0.868738in}}%
\pgfpathlineto{\pgfqpoint{1.678950in}{0.876079in}}%
\pgfpathlineto{\pgfqpoint{1.673448in}{0.880159in}}%
\pgfpathlineto{\pgfqpoint{1.659372in}{0.889691in}}%
\pgfpathlineto{\pgfqpoint{1.657791in}{0.890812in}}%
\pgfpathlineto{\pgfqpoint{1.642135in}{0.900801in}}%
\pgfpathlineto{\pgfqpoint{1.637649in}{0.903302in}}%
\pgfpathlineto{\pgfqpoint{1.626478in}{0.910007in}}%
\pgfpathlineto{\pgfqpoint{1.612877in}{0.916913in}}%
\pgfpathlineto{\pgfqpoint{1.610822in}{0.918070in}}%
\pgfpathlineto{\pgfqpoint{1.595165in}{0.925293in}}%
\pgfpathlineto{\pgfqpoint{1.580330in}{0.930524in}}%
\pgfpathlineto{\pgfqpoint{1.579508in}{0.930858in}}%
\pgfpathlineto{\pgfqpoint{1.563852in}{0.935302in}}%
\pgfpathlineto{\pgfqpoint{1.548195in}{0.937769in}}%
\pgfpathlineto{\pgfqpoint{1.532539in}{0.938263in}}%
\pgfpathlineto{\pgfqpoint{1.516882in}{0.936783in}}%
\pgfpathlineto{\pgfqpoint{1.501226in}{0.933327in}}%
\pgfpathlineto{\pgfqpoint{1.493117in}{0.930524in}}%
\pgfpathlineto{\pgfqpoint{1.485569in}{0.928265in}}%
\pgfpathlineto{\pgfqpoint{1.469913in}{0.921894in}}%
\pgfpathlineto{\pgfqpoint{1.460204in}{0.916913in}}%
\pgfpathlineto{\pgfqpoint{1.454256in}{0.914163in}}%
\pgfpathlineto{\pgfqpoint{1.438599in}{0.905473in}}%
\pgfpathlineto{\pgfqpoint{1.435216in}{0.903302in}}%
\pgfpathlineto{\pgfqpoint{1.422943in}{0.895976in}}%
\pgfpathlineto{\pgfqpoint{1.413635in}{0.889691in}}%
\pgfpathlineto{\pgfqpoint{1.407286in}{0.885603in}}%
\pgfpathlineto{\pgfqpoint{1.393917in}{0.876079in}}%
\pgfpathlineto{\pgfqpoint{1.391630in}{0.874492in}}%
\pgfpathlineto{\pgfqpoint{1.375973in}{0.862709in}}%
\pgfpathlineto{\pgfqpoint{1.375671in}{0.862468in}}%
\pgfpathlineto{\pgfqpoint{1.360317in}{0.850260in}}%
\pgfpathlineto{\pgfqpoint{1.358632in}{0.848857in}}%
\pgfpathlineto{\pgfqpoint{1.344660in}{0.837068in}}%
\pgfpathlineto{\pgfqpoint{1.342564in}{0.835246in}}%
\pgfpathlineto{\pgfqpoint{1.329003in}{0.823099in}}%
\pgfpathlineto{\pgfqpoint{1.327390in}{0.821635in}}%
\pgfpathlineto{\pgfqpoint{1.313347in}{0.808286in}}%
\pgfpathlineto{\pgfqpoint{1.313070in}{0.808024in}}%
\pgfpathlineto{\pgfqpoint{1.299517in}{0.794413in}}%
\pgfpathlineto{\pgfqpoint{1.297690in}{0.792424in}}%
\pgfpathlineto{\pgfqpoint{1.286736in}{0.780802in}}%
\pgfpathlineto{\pgfqpoint{1.282034in}{0.775282in}}%
\pgfpathlineto{\pgfqpoint{1.274804in}{0.767191in}}%
\pgfpathlineto{\pgfqpoint{1.266377in}{0.756521in}}%
\pgfpathlineto{\pgfqpoint{1.263880in}{0.753579in}}%
\pgfpathlineto{\pgfqpoint{1.253883in}{0.739968in}}%
\pgfpathlineto{\pgfqpoint{1.250721in}{0.734797in}}%
\pgfpathlineto{\pgfqpoint{1.244990in}{0.726357in}}%
\pgfpathlineto{\pgfqpoint{1.237663in}{0.712746in}}%
\pgfpathlineto{\pgfqpoint{1.235064in}{0.706184in}}%
\pgfpathlineto{\pgfqpoint{1.231839in}{0.699135in}}%
\pgfpathlineto{\pgfqpoint{1.227865in}{0.685524in}}%
\pgfpathlineto{\pgfqpoint{1.226162in}{0.671913in}}%
\pgfpathlineto{\pgfqpoint{1.226730in}{0.658302in}}%
\pgfpathlineto{\pgfqpoint{1.229568in}{0.644691in}}%
\pgfpathlineto{\pgfqpoint{1.234680in}{0.631079in}}%
\pgfpathlineto{\pgfqpoint{1.235064in}{0.630365in}}%
\pgfpathlineto{\pgfqpoint{1.241081in}{0.617468in}}%
\pgfpathlineto{\pgfqpoint{1.249390in}{0.603857in}}%
\pgfpathlineto{\pgfqpoint{1.250721in}{0.602070in}}%
\pgfpathlineto{\pgfqpoint{1.258664in}{0.590246in}}%
\pgfpathlineto{\pgfqpoint{1.266377in}{0.580534in}}%
\pgfpathlineto{\pgfqpoint{1.269254in}{0.576635in}}%
\pgfpathlineto{\pgfqpoint{1.280744in}{0.563024in}}%
\pgfpathlineto{\pgfqpoint{1.282034in}{0.561650in}}%
\pgfpathlineto{\pgfqpoint{1.292998in}{0.549413in}}%
\pgfpathlineto{\pgfqpoint{1.297690in}{0.544629in}}%
\pgfpathlineto{\pgfqpoint{1.306135in}{0.535802in}}%
\pgfpathlineto{\pgfqpoint{1.313347in}{0.528788in}}%
\pgfpathlineto{\pgfqpoint{1.320093in}{0.522191in}}%
\pgfpathlineto{\pgfqpoint{1.329003in}{0.513941in}}%
\pgfpathlineto{\pgfqpoint{1.334866in}{0.508579in}}%
\pgfpathlineto{\pgfqpoint{1.344660in}{0.499957in}}%
\pgfpathlineto{\pgfqpoint{1.350496in}{0.494968in}}%
\pgfpathlineto{\pgfqpoint{1.360317in}{0.486748in}}%
\pgfpathlineto{\pgfqpoint{1.367063in}{0.481357in}}%
\pgfpathlineto{\pgfqpoint{1.375973in}{0.474261in}}%
\pgfpathlineto{\pgfqpoint{1.384690in}{0.467746in}}%
\pgfpathlineto{\pgfqpoint{1.391630in}{0.462480in}}%
\pgfpathlineto{\pgfqpoint{1.403558in}{0.454135in}}%
\pgfpathlineto{\pgfqpoint{1.407286in}{0.451432in}}%
\pgfpathlineto{\pgfqpoint{1.422943in}{0.441155in}}%
\pgfpathlineto{\pgfqpoint{1.424026in}{0.440524in}}%
\pgfpathlineto{\pgfqpoint{1.438599in}{0.431496in}}%
\pgfpathlineto{\pgfqpoint{1.447204in}{0.426913in}}%
\pgfpathlineto{\pgfqpoint{1.454256in}{0.422813in}}%
\pgfpathlineto{\pgfqpoint{1.469913in}{0.415228in}}%
\pgfpathlineto{\pgfqpoint{1.474917in}{0.413302in}}%
\pgfpathlineto{\pgfqpoint{1.485569in}{0.408664in}}%
\pgfpathlineto{\pgfqpoint{1.501226in}{0.403653in}}%
\pgfpathlineto{\pgfqpoint{1.516882in}{0.400466in}}%
\pgfpathlineto{\pgfqpoint{1.525770in}{0.399691in}}%
\pgfpathlineto{\pgfqpoint{1.532539in}{0.398990in}}%
\pgfpathclose%
\pgfpathmoveto{\pgfqpoint{1.470984in}{0.481357in}}%
\pgfpathlineto{\pgfqpoint{1.469913in}{0.481706in}}%
\pgfpathlineto{\pgfqpoint{1.454256in}{0.488092in}}%
\pgfpathlineto{\pgfqpoint{1.440312in}{0.494968in}}%
\pgfpathlineto{\pgfqpoint{1.438599in}{0.495821in}}%
\pgfpathlineto{\pgfqpoint{1.422943in}{0.504833in}}%
\pgfpathlineto{\pgfqpoint{1.417187in}{0.508579in}}%
\pgfpathlineto{\pgfqpoint{1.407286in}{0.515222in}}%
\pgfpathlineto{\pgfqpoint{1.397887in}{0.522191in}}%
\pgfpathlineto{\pgfqpoint{1.391630in}{0.527064in}}%
\pgfpathlineto{\pgfqpoint{1.381255in}{0.535802in}}%
\pgfpathlineto{\pgfqpoint{1.375973in}{0.540565in}}%
\pgfpathlineto{\pgfqpoint{1.366720in}{0.549413in}}%
\pgfpathlineto{\pgfqpoint{1.360317in}{0.556105in}}%
\pgfpathlineto{\pgfqpoint{1.353954in}{0.563024in}}%
\pgfpathlineto{\pgfqpoint{1.344660in}{0.574325in}}%
\pgfpathlineto{\pgfqpoint{1.342800in}{0.576635in}}%
\pgfpathlineto{\pgfqpoint{1.333184in}{0.590246in}}%
\pgfpathlineto{\pgfqpoint{1.329003in}{0.597238in}}%
\pgfpathlineto{\pgfqpoint{1.325047in}{0.603857in}}%
\pgfpathlineto{\pgfqpoint{1.318414in}{0.617468in}}%
\pgfpathlineto{\pgfqpoint{1.313347in}{0.631077in}}%
\pgfpathlineto{\pgfqpoint{1.313346in}{0.631079in}}%
\pgfpathlineto{\pgfqpoint{1.309729in}{0.644691in}}%
\pgfpathlineto{\pgfqpoint{1.307720in}{0.658302in}}%
\pgfpathlineto{\pgfqpoint{1.307319in}{0.671913in}}%
\pgfpathlineto{\pgfqpoint{1.308524in}{0.685524in}}%
\pgfpathlineto{\pgfqpoint{1.311336in}{0.699135in}}%
\pgfpathlineto{\pgfqpoint{1.313347in}{0.705350in}}%
\pgfpathlineto{\pgfqpoint{1.315684in}{0.712746in}}%
\pgfpathlineto{\pgfqpoint{1.321534in}{0.726357in}}%
\pgfpathlineto{\pgfqpoint{1.328951in}{0.739968in}}%
\pgfpathlineto{\pgfqpoint{1.329003in}{0.740049in}}%
\pgfpathlineto{\pgfqpoint{1.337800in}{0.753579in}}%
\pgfpathlineto{\pgfqpoint{1.344660in}{0.762630in}}%
\pgfpathlineto{\pgfqpoint{1.348186in}{0.767191in}}%
\pgfpathlineto{\pgfqpoint{1.360096in}{0.780802in}}%
\pgfpathlineto{\pgfqpoint{1.360317in}{0.781031in}}%
\pgfpathlineto{\pgfqpoint{1.373717in}{0.794413in}}%
\pgfpathlineto{\pgfqpoint{1.375973in}{0.796494in}}%
\pgfpathlineto{\pgfqpoint{1.389236in}{0.808024in}}%
\pgfpathlineto{\pgfqpoint{1.391630in}{0.809985in}}%
\pgfpathlineto{\pgfqpoint{1.407022in}{0.821635in}}%
\pgfpathlineto{\pgfqpoint{1.407286in}{0.821827in}}%
\pgfpathlineto{\pgfqpoint{1.422943in}{0.832181in}}%
\pgfpathlineto{\pgfqpoint{1.428188in}{0.835246in}}%
\pgfpathlineto{\pgfqpoint{1.438599in}{0.841210in}}%
\pgfpathlineto{\pgfqpoint{1.454163in}{0.848857in}}%
\pgfpathlineto{\pgfqpoint{1.454256in}{0.848903in}}%
\pgfpathlineto{\pgfqpoint{1.469913in}{0.855350in}}%
\pgfpathlineto{\pgfqpoint{1.485569in}{0.860436in}}%
\pgfpathlineto{\pgfqpoint{1.494077in}{0.862468in}}%
\pgfpathlineto{\pgfqpoint{1.501226in}{0.864216in}}%
\pgfpathlineto{\pgfqpoint{1.516882in}{0.866661in}}%
\pgfpathlineto{\pgfqpoint{1.532539in}{0.867709in}}%
\pgfpathlineto{\pgfqpoint{1.548195in}{0.867360in}}%
\pgfpathlineto{\pgfqpoint{1.563852in}{0.865614in}}%
\pgfpathlineto{\pgfqpoint{1.579508in}{0.862469in}}%
\pgfpathlineto{\pgfqpoint{1.579511in}{0.862468in}}%
\pgfpathlineto{\pgfqpoint{1.595165in}{0.858063in}}%
\pgfpathlineto{\pgfqpoint{1.610822in}{0.852297in}}%
\pgfpathlineto{\pgfqpoint{1.618435in}{0.848857in}}%
\pgfpathlineto{\pgfqpoint{1.626478in}{0.845223in}}%
\pgfpathlineto{\pgfqpoint{1.642135in}{0.836863in}}%
\pgfpathlineto{\pgfqpoint{1.644792in}{0.835246in}}%
\pgfpathlineto{\pgfqpoint{1.657791in}{0.827167in}}%
\pgfpathlineto{\pgfqpoint{1.665750in}{0.821635in}}%
\pgfpathlineto{\pgfqpoint{1.673448in}{0.816068in}}%
\pgfpathlineto{\pgfqpoint{1.683625in}{0.808024in}}%
\pgfpathlineto{\pgfqpoint{1.689104in}{0.803432in}}%
\pgfpathlineto{\pgfqpoint{1.699156in}{0.794413in}}%
\pgfpathlineto{\pgfqpoint{1.704761in}{0.788973in}}%
\pgfpathlineto{\pgfqpoint{1.712777in}{0.780802in}}%
\pgfpathlineto{\pgfqpoint{1.720418in}{0.772195in}}%
\pgfpathlineto{\pgfqpoint{1.724726in}{0.767191in}}%
\pgfpathlineto{\pgfqpoint{1.735094in}{0.753579in}}%
\pgfpathlineto{\pgfqpoint{1.736074in}{0.752091in}}%
\pgfpathlineto{\pgfqpoint{1.743983in}{0.739968in}}%
\pgfpathlineto{\pgfqpoint{1.751330in}{0.726357in}}%
\pgfpathlineto{\pgfqpoint{1.751731in}{0.725426in}}%
\pgfpathlineto{\pgfqpoint{1.757251in}{0.712746in}}%
\pgfpathlineto{\pgfqpoint{1.761596in}{0.699135in}}%
\pgfpathlineto{\pgfqpoint{1.764358in}{0.685524in}}%
\pgfpathlineto{\pgfqpoint{1.765542in}{0.671913in}}%
\pgfpathlineto{\pgfqpoint{1.765147in}{0.658302in}}%
\pgfpathlineto{\pgfqpoint{1.763175in}{0.644691in}}%
\pgfpathlineto{\pgfqpoint{1.759621in}{0.631079in}}%
\pgfpathlineto{\pgfqpoint{1.754484in}{0.617468in}}%
\pgfpathlineto{\pgfqpoint{1.751731in}{0.611854in}}%
\pgfpathlineto{\pgfqpoint{1.747851in}{0.603857in}}%
\pgfpathlineto{\pgfqpoint{1.739729in}{0.590246in}}%
\pgfpathlineto{\pgfqpoint{1.736074in}{0.585045in}}%
\pgfpathlineto{\pgfqpoint{1.730100in}{0.576635in}}%
\pgfpathlineto{\pgfqpoint{1.720418in}{0.564759in}}%
\pgfpathlineto{\pgfqpoint{1.718961in}{0.563024in}}%
\pgfpathlineto{\pgfqpoint{1.706221in}{0.549413in}}%
\pgfpathlineto{\pgfqpoint{1.704761in}{0.547983in}}%
\pgfpathlineto{\pgfqpoint{1.691685in}{0.535802in}}%
\pgfpathlineto{\pgfqpoint{1.689104in}{0.533558in}}%
\pgfpathlineto{\pgfqpoint{1.675092in}{0.522191in}}%
\pgfpathlineto{\pgfqpoint{1.673448in}{0.520921in}}%
\pgfpathlineto{\pgfqpoint{1.657791in}{0.509846in}}%
\pgfpathlineto{\pgfqpoint{1.655796in}{0.508579in}}%
\pgfpathlineto{\pgfqpoint{1.642135in}{0.500162in}}%
\pgfpathlineto{\pgfqpoint{1.632461in}{0.494968in}}%
\pgfpathlineto{\pgfqpoint{1.626478in}{0.491791in}}%
\pgfpathlineto{\pgfqpoint{1.610822in}{0.484730in}}%
\pgfpathlineto{\pgfqpoint{1.601623in}{0.481357in}}%
\pgfpathlineto{\pgfqpoint{1.595165in}{0.478963in}}%
\pgfpathlineto{\pgfqpoint{1.579508in}{0.474497in}}%
\pgfpathlineto{\pgfqpoint{1.563852in}{0.471408in}}%
\pgfpathlineto{\pgfqpoint{1.548195in}{0.469693in}}%
\pgfpathlineto{\pgfqpoint{1.532539in}{0.469351in}}%
\pgfpathlineto{\pgfqpoint{1.516882in}{0.470379in}}%
\pgfpathlineto{\pgfqpoint{1.501226in}{0.472781in}}%
\pgfpathlineto{\pgfqpoint{1.485569in}{0.476558in}}%
\pgfpathlineto{\pgfqpoint{1.470984in}{0.481357in}}%
\pgfpathclose%
\pgfpathmoveto{\pgfqpoint{0.718397in}{1.079912in}}%
\pgfpathlineto{\pgfqpoint{0.734054in}{1.075468in}}%
\pgfpathlineto{\pgfqpoint{0.749710in}{1.073001in}}%
\pgfpathlineto{\pgfqpoint{0.765367in}{1.072507in}}%
\pgfpathlineto{\pgfqpoint{0.781024in}{1.073987in}}%
\pgfpathlineto{\pgfqpoint{0.796680in}{1.077443in}}%
\pgfpathlineto{\pgfqpoint{0.804789in}{1.080246in}}%
\pgfpathlineto{\pgfqpoint{0.812337in}{1.082505in}}%
\pgfpathlineto{\pgfqpoint{0.827993in}{1.088876in}}%
\pgfpathlineto{\pgfqpoint{0.837702in}{1.093857in}}%
\pgfpathlineto{\pgfqpoint{0.843650in}{1.096607in}}%
\pgfpathlineto{\pgfqpoint{0.859306in}{1.105297in}}%
\pgfpathlineto{\pgfqpoint{0.862690in}{1.107468in}}%
\pgfpathlineto{\pgfqpoint{0.874963in}{1.114794in}}%
\pgfpathlineto{\pgfqpoint{0.884270in}{1.121079in}}%
\pgfpathlineto{\pgfqpoint{0.890620in}{1.125167in}}%
\pgfpathlineto{\pgfqpoint{0.903988in}{1.134691in}}%
\pgfpathlineto{\pgfqpoint{0.906276in}{1.136278in}}%
\pgfpathlineto{\pgfqpoint{0.921933in}{1.148061in}}%
\pgfpathlineto{\pgfqpoint{0.922234in}{1.148302in}}%
\pgfpathlineto{\pgfqpoint{0.937589in}{1.160510in}}%
\pgfpathlineto{\pgfqpoint{0.939273in}{1.161913in}}%
\pgfpathlineto{\pgfqpoint{0.953246in}{1.173702in}}%
\pgfpathlineto{\pgfqpoint{0.955341in}{1.175524in}}%
\pgfpathlineto{\pgfqpoint{0.968902in}{1.187671in}}%
\pgfpathlineto{\pgfqpoint{0.970516in}{1.189135in}}%
\pgfpathlineto{\pgfqpoint{0.984559in}{1.202484in}}%
\pgfpathlineto{\pgfqpoint{0.984836in}{1.202746in}}%
\pgfpathlineto{\pgfqpoint{0.998389in}{1.216357in}}%
\pgfpathlineto{\pgfqpoint{1.000216in}{1.218346in}}%
\pgfpathlineto{\pgfqpoint{1.011170in}{1.229968in}}%
\pgfpathlineto{\pgfqpoint{1.015872in}{1.235488in}}%
\pgfpathlineto{\pgfqpoint{1.023102in}{1.243579in}}%
\pgfpathlineto{\pgfqpoint{1.031529in}{1.254249in}}%
\pgfpathlineto{\pgfqpoint{1.034026in}{1.257191in}}%
\pgfpathlineto{\pgfqpoint{1.044022in}{1.270802in}}%
\pgfpathlineto{\pgfqpoint{1.047185in}{1.275973in}}%
\pgfpathlineto{\pgfqpoint{1.052916in}{1.284413in}}%
\pgfpathlineto{\pgfqpoint{1.060243in}{1.298024in}}%
\pgfpathlineto{\pgfqpoint{1.062842in}{1.304586in}}%
\pgfpathlineto{\pgfqpoint{1.066067in}{1.311635in}}%
\pgfpathlineto{\pgfqpoint{1.070041in}{1.325246in}}%
\pgfpathlineto{\pgfqpoint{1.071743in}{1.338857in}}%
\pgfpathlineto{\pgfqpoint{1.071176in}{1.352468in}}%
\pgfpathlineto{\pgfqpoint{1.068338in}{1.366079in}}%
\pgfpathlineto{\pgfqpoint{1.063226in}{1.379691in}}%
\pgfpathlineto{\pgfqpoint{1.062842in}{1.380405in}}%
\pgfpathlineto{\pgfqpoint{1.056824in}{1.393302in}}%
\pgfpathlineto{\pgfqpoint{1.048516in}{1.406913in}}%
\pgfpathlineto{\pgfqpoint{1.047185in}{1.408700in}}%
\pgfpathlineto{\pgfqpoint{1.039241in}{1.420524in}}%
\pgfpathlineto{\pgfqpoint{1.031529in}{1.430236in}}%
\pgfpathlineto{\pgfqpoint{1.028652in}{1.434135in}}%
\pgfpathlineto{\pgfqpoint{1.017162in}{1.447746in}}%
\pgfpathlineto{\pgfqpoint{1.015872in}{1.449120in}}%
\pgfpathlineto{\pgfqpoint{1.004908in}{1.461357in}}%
\pgfpathlineto{\pgfqpoint{1.000216in}{1.466141in}}%
\pgfpathlineto{\pgfqpoint{0.991771in}{1.474968in}}%
\pgfpathlineto{\pgfqpoint{0.984559in}{1.481982in}}%
\pgfpathlineto{\pgfqpoint{0.977813in}{1.488579in}}%
\pgfpathlineto{\pgfqpoint{0.968902in}{1.496829in}}%
\pgfpathlineto{\pgfqpoint{0.963040in}{1.502191in}}%
\pgfpathlineto{\pgfqpoint{0.953246in}{1.510813in}}%
\pgfpathlineto{\pgfqpoint{0.947410in}{1.515802in}}%
\pgfpathlineto{\pgfqpoint{0.937589in}{1.524022in}}%
\pgfpathlineto{\pgfqpoint{0.930843in}{1.529413in}}%
\pgfpathlineto{\pgfqpoint{0.921933in}{1.536509in}}%
\pgfpathlineto{\pgfqpoint{0.913215in}{1.543024in}}%
\pgfpathlineto{\pgfqpoint{0.906276in}{1.548290in}}%
\pgfpathlineto{\pgfqpoint{0.894348in}{1.556635in}}%
\pgfpathlineto{\pgfqpoint{0.890620in}{1.559338in}}%
\pgfpathlineto{\pgfqpoint{0.874963in}{1.569615in}}%
\pgfpathlineto{\pgfqpoint{0.873880in}{1.570246in}}%
\pgfpathlineto{\pgfqpoint{0.859306in}{1.579274in}}%
\pgfpathlineto{\pgfqpoint{0.850702in}{1.583857in}}%
\pgfpathlineto{\pgfqpoint{0.843650in}{1.587957in}}%
\pgfpathlineto{\pgfqpoint{0.827993in}{1.595542in}}%
\pgfpathlineto{\pgfqpoint{0.822989in}{1.597468in}}%
\pgfpathlineto{\pgfqpoint{0.812337in}{1.602106in}}%
\pgfpathlineto{\pgfqpoint{0.796680in}{1.607117in}}%
\pgfpathlineto{\pgfqpoint{0.781024in}{1.610304in}}%
\pgfpathlineto{\pgfqpoint{0.772136in}{1.611079in}}%
\pgfpathlineto{\pgfqpoint{0.765367in}{1.611780in}}%
\pgfpathlineto{\pgfqpoint{0.749710in}{1.611239in}}%
\pgfpathlineto{\pgfqpoint{0.748782in}{1.611079in}}%
\pgfpathlineto{\pgfqpoint{0.734054in}{1.608939in}}%
\pgfpathlineto{\pgfqpoint{0.718397in}{1.604840in}}%
\pgfpathlineto{\pgfqpoint{0.702741in}{1.598915in}}%
\pgfpathlineto{\pgfqpoint{0.699792in}{1.597468in}}%
\pgfpathlineto{\pgfqpoint{0.687084in}{1.591950in}}%
\pgfpathlineto{\pgfqpoint{0.671966in}{1.583857in}}%
\pgfpathlineto{\pgfqpoint{0.671428in}{1.583594in}}%
\pgfpathlineto{\pgfqpoint{0.655771in}{1.574597in}}%
\pgfpathlineto{\pgfqpoint{0.649160in}{1.570246in}}%
\pgfpathlineto{\pgfqpoint{0.640115in}{1.564638in}}%
\pgfpathlineto{\pgfqpoint{0.628573in}{1.556635in}}%
\pgfpathlineto{\pgfqpoint{0.624458in}{1.553884in}}%
\pgfpathlineto{\pgfqpoint{0.609625in}{1.543024in}}%
\pgfpathlineto{\pgfqpoint{0.608801in}{1.542430in}}%
\pgfpathlineto{\pgfqpoint{0.593145in}{1.530327in}}%
\pgfpathlineto{\pgfqpoint{0.592024in}{1.529413in}}%
\pgfpathlineto{\pgfqpoint{0.577488in}{1.517508in}}%
\pgfpathlineto{\pgfqpoint{0.575483in}{1.515802in}}%
\pgfpathlineto{\pgfqpoint{0.561832in}{1.503934in}}%
\pgfpathlineto{\pgfqpoint{0.559869in}{1.502191in}}%
\pgfpathlineto{\pgfqpoint{0.546175in}{1.489554in}}%
\pgfpathlineto{\pgfqpoint{0.545124in}{1.488579in}}%
\pgfpathlineto{\pgfqpoint{0.531201in}{1.474968in}}%
\pgfpathlineto{\pgfqpoint{0.530519in}{1.474252in}}%
\pgfpathlineto{\pgfqpoint{0.518026in}{1.461357in}}%
\pgfpathlineto{\pgfqpoint{0.514862in}{1.457780in}}%
\pgfpathlineto{\pgfqpoint{0.505657in}{1.447746in}}%
\pgfpathlineto{\pgfqpoint{0.499205in}{1.439882in}}%
\pgfpathlineto{\pgfqpoint{0.494201in}{1.434135in}}%
\pgfpathlineto{\pgfqpoint{0.483852in}{1.420524in}}%
\pgfpathlineto{\pgfqpoint{0.483549in}{1.420056in}}%
\pgfpathlineto{\pgfqpoint{0.474240in}{1.406913in}}%
\pgfpathlineto{\pgfqpoint{0.467892in}{1.395865in}}%
\pgfpathlineto{\pgfqpoint{0.466229in}{1.393302in}}%
\pgfpathlineto{\pgfqpoint{0.459413in}{1.379691in}}%
\pgfpathlineto{\pgfqpoint{0.454698in}{1.366079in}}%
\pgfpathlineto{\pgfqpoint{0.452236in}{1.353275in}}%
\pgfpathlineto{\pgfqpoint{0.452052in}{1.352468in}}%
\pgfpathlineto{\pgfqpoint{0.451429in}{1.338857in}}%
\pgfpathlineto{\pgfqpoint{0.452236in}{1.332973in}}%
\pgfpathlineto{\pgfqpoint{0.453128in}{1.325246in}}%
\pgfpathlineto{\pgfqpoint{0.456793in}{1.311635in}}%
\pgfpathlineto{\pgfqpoint{0.462558in}{1.298024in}}%
\pgfpathlineto{\pgfqpoint{0.467892in}{1.288764in}}%
\pgfpathlineto{\pgfqpoint{0.470108in}{1.284413in}}%
\pgfpathlineto{\pgfqpoint{0.478833in}{1.270802in}}%
\pgfpathlineto{\pgfqpoint{0.483549in}{1.264671in}}%
\pgfpathlineto{\pgfqpoint{0.488820in}{1.257191in}}%
\pgfpathlineto{\pgfqpoint{0.499205in}{1.244521in}}%
\pgfpathlineto{\pgfqpoint{0.499932in}{1.243579in}}%
\pgfpathlineto{\pgfqpoint{0.511753in}{1.229968in}}%
\pgfpathlineto{\pgfqpoint{0.514862in}{1.226727in}}%
\pgfpathlineto{\pgfqpoint{0.524461in}{1.216357in}}%
\pgfpathlineto{\pgfqpoint{0.530519in}{1.210325in}}%
\pgfpathlineto{\pgfqpoint{0.538013in}{1.202746in}}%
\pgfpathlineto{\pgfqpoint{0.546175in}{1.195000in}}%
\pgfpathlineto{\pgfqpoint{0.552376in}{1.189135in}}%
\pgfpathlineto{\pgfqpoint{0.561832in}{1.180597in}}%
\pgfpathlineto{\pgfqpoint{0.567570in}{1.175524in}}%
\pgfpathlineto{\pgfqpoint{0.577488in}{1.167010in}}%
\pgfpathlineto{\pgfqpoint{0.583655in}{1.161913in}}%
\pgfpathlineto{\pgfqpoint{0.593145in}{1.154166in}}%
\pgfpathlineto{\pgfqpoint{0.600734in}{1.148302in}}%
\pgfpathlineto{\pgfqpoint{0.608801in}{1.142032in}}%
\pgfpathlineto{\pgfqpoint{0.618955in}{1.134691in}}%
\pgfpathlineto{\pgfqpoint{0.624458in}{1.130611in}}%
\pgfpathlineto{\pgfqpoint{0.638534in}{1.121079in}}%
\pgfpathlineto{\pgfqpoint{0.640115in}{1.119958in}}%
\pgfpathlineto{\pgfqpoint{0.655771in}{1.109969in}}%
\pgfpathlineto{\pgfqpoint{0.660256in}{1.107468in}}%
\pgfpathlineto{\pgfqpoint{0.671428in}{1.100763in}}%
\pgfpathlineto{\pgfqpoint{0.685029in}{1.093857in}}%
\pgfpathlineto{\pgfqpoint{0.687084in}{1.092700in}}%
\pgfpathlineto{\pgfqpoint{0.702741in}{1.085477in}}%
\pgfpathlineto{\pgfqpoint{0.717576in}{1.080246in}}%
\pgfpathlineto{\pgfqpoint{0.718397in}{1.079912in}}%
\pgfpathclose%
\pgfpathmoveto{\pgfqpoint{0.718395in}{1.148302in}}%
\pgfpathlineto{\pgfqpoint{0.702741in}{1.152707in}}%
\pgfpathlineto{\pgfqpoint{0.687084in}{1.158473in}}%
\pgfpathlineto{\pgfqpoint{0.679471in}{1.161913in}}%
\pgfpathlineto{\pgfqpoint{0.671428in}{1.165547in}}%
\pgfpathlineto{\pgfqpoint{0.655771in}{1.173907in}}%
\pgfpathlineto{\pgfqpoint{0.653114in}{1.175524in}}%
\pgfpathlineto{\pgfqpoint{0.640115in}{1.183603in}}%
\pgfpathlineto{\pgfqpoint{0.632155in}{1.189135in}}%
\pgfpathlineto{\pgfqpoint{0.624458in}{1.194702in}}%
\pgfpathlineto{\pgfqpoint{0.614280in}{1.202746in}}%
\pgfpathlineto{\pgfqpoint{0.608801in}{1.207338in}}%
\pgfpathlineto{\pgfqpoint{0.598750in}{1.216357in}}%
\pgfpathlineto{\pgfqpoint{0.593145in}{1.221797in}}%
\pgfpathlineto{\pgfqpoint{0.585129in}{1.229968in}}%
\pgfpathlineto{\pgfqpoint{0.577488in}{1.238575in}}%
\pgfpathlineto{\pgfqpoint{0.573179in}{1.243579in}}%
\pgfpathlineto{\pgfqpoint{0.562812in}{1.257191in}}%
\pgfpathlineto{\pgfqpoint{0.561832in}{1.258679in}}%
\pgfpathlineto{\pgfqpoint{0.553922in}{1.270802in}}%
\pgfpathlineto{\pgfqpoint{0.546576in}{1.284413in}}%
\pgfpathlineto{\pgfqpoint{0.546175in}{1.285344in}}%
\pgfpathlineto{\pgfqpoint{0.540655in}{1.298024in}}%
\pgfpathlineto{\pgfqpoint{0.536310in}{1.311635in}}%
\pgfpathlineto{\pgfqpoint{0.533548in}{1.325246in}}%
\pgfpathlineto{\pgfqpoint{0.532364in}{1.338857in}}%
\pgfpathlineto{\pgfqpoint{0.532759in}{1.352468in}}%
\pgfpathlineto{\pgfqpoint{0.534731in}{1.366079in}}%
\pgfpathlineto{\pgfqpoint{0.538284in}{1.379691in}}%
\pgfpathlineto{\pgfqpoint{0.543422in}{1.393302in}}%
\pgfpathlineto{\pgfqpoint{0.546175in}{1.398916in}}%
\pgfpathlineto{\pgfqpoint{0.550055in}{1.406913in}}%
\pgfpathlineto{\pgfqpoint{0.558177in}{1.420524in}}%
\pgfpathlineto{\pgfqpoint{0.561832in}{1.425725in}}%
\pgfpathlineto{\pgfqpoint{0.567805in}{1.434135in}}%
\pgfpathlineto{\pgfqpoint{0.577488in}{1.446011in}}%
\pgfpathlineto{\pgfqpoint{0.578945in}{1.447746in}}%
\pgfpathlineto{\pgfqpoint{0.591685in}{1.461357in}}%
\pgfpathlineto{\pgfqpoint{0.593145in}{1.462787in}}%
\pgfpathlineto{\pgfqpoint{0.606221in}{1.474968in}}%
\pgfpathlineto{\pgfqpoint{0.608801in}{1.477212in}}%
\pgfpathlineto{\pgfqpoint{0.622813in}{1.488579in}}%
\pgfpathlineto{\pgfqpoint{0.624458in}{1.489849in}}%
\pgfpathlineto{\pgfqpoint{0.640115in}{1.500924in}}%
\pgfpathlineto{\pgfqpoint{0.642110in}{1.502191in}}%
\pgfpathlineto{\pgfqpoint{0.655771in}{1.510608in}}%
\pgfpathlineto{\pgfqpoint{0.665445in}{1.515802in}}%
\pgfpathlineto{\pgfqpoint{0.671428in}{1.518979in}}%
\pgfpathlineto{\pgfqpoint{0.687084in}{1.526040in}}%
\pgfpathlineto{\pgfqpoint{0.696283in}{1.529413in}}%
\pgfpathlineto{\pgfqpoint{0.702741in}{1.531807in}}%
\pgfpathlineto{\pgfqpoint{0.718397in}{1.536273in}}%
\pgfpathlineto{\pgfqpoint{0.734054in}{1.539362in}}%
\pgfpathlineto{\pgfqpoint{0.749710in}{1.541077in}}%
\pgfpathlineto{\pgfqpoint{0.765367in}{1.541419in}}%
\pgfpathlineto{\pgfqpoint{0.781024in}{1.540391in}}%
\pgfpathlineto{\pgfqpoint{0.796680in}{1.537989in}}%
\pgfpathlineto{\pgfqpoint{0.812337in}{1.534212in}}%
\pgfpathlineto{\pgfqpoint{0.826922in}{1.529413in}}%
\pgfpathlineto{\pgfqpoint{0.827993in}{1.529064in}}%
\pgfpathlineto{\pgfqpoint{0.843650in}{1.522678in}}%
\pgfpathlineto{\pgfqpoint{0.857594in}{1.515802in}}%
\pgfpathlineto{\pgfqpoint{0.859306in}{1.514949in}}%
\pgfpathlineto{\pgfqpoint{0.874963in}{1.505937in}}%
\pgfpathlineto{\pgfqpoint{0.880719in}{1.502191in}}%
\pgfpathlineto{\pgfqpoint{0.890620in}{1.495548in}}%
\pgfpathlineto{\pgfqpoint{0.900018in}{1.488579in}}%
\pgfpathlineto{\pgfqpoint{0.906276in}{1.483706in}}%
\pgfpathlineto{\pgfqpoint{0.916651in}{1.474968in}}%
\pgfpathlineto{\pgfqpoint{0.921933in}{1.470205in}}%
\pgfpathlineto{\pgfqpoint{0.931186in}{1.461357in}}%
\pgfpathlineto{\pgfqpoint{0.937589in}{1.454665in}}%
\pgfpathlineto{\pgfqpoint{0.943952in}{1.447746in}}%
\pgfpathlineto{\pgfqpoint{0.953246in}{1.436445in}}%
\pgfpathlineto{\pgfqpoint{0.955106in}{1.434135in}}%
\pgfpathlineto{\pgfqpoint{0.964722in}{1.420524in}}%
\pgfpathlineto{\pgfqpoint{0.968902in}{1.413532in}}%
\pgfpathlineto{\pgfqpoint{0.972859in}{1.406913in}}%
\pgfpathlineto{\pgfqpoint{0.979492in}{1.393302in}}%
\pgfpathlineto{\pgfqpoint{0.984559in}{1.379693in}}%
\pgfpathlineto{\pgfqpoint{0.984560in}{1.379691in}}%
\pgfpathlineto{\pgfqpoint{0.988177in}{1.366079in}}%
\pgfpathlineto{\pgfqpoint{0.990185in}{1.352468in}}%
\pgfpathlineto{\pgfqpoint{0.990587in}{1.338857in}}%
\pgfpathlineto{\pgfqpoint{0.989382in}{1.325246in}}%
\pgfpathlineto{\pgfqpoint{0.986570in}{1.311635in}}%
\pgfpathlineto{\pgfqpoint{0.984559in}{1.305420in}}%
\pgfpathlineto{\pgfqpoint{0.982222in}{1.298024in}}%
\pgfpathlineto{\pgfqpoint{0.976371in}{1.284413in}}%
\pgfpathlineto{\pgfqpoint{0.968955in}{1.270802in}}%
\pgfpathlineto{\pgfqpoint{0.968902in}{1.270721in}}%
\pgfpathlineto{\pgfqpoint{0.960106in}{1.257191in}}%
\pgfpathlineto{\pgfqpoint{0.953246in}{1.248140in}}%
\pgfpathlineto{\pgfqpoint{0.949720in}{1.243579in}}%
\pgfpathlineto{\pgfqpoint{0.937810in}{1.229968in}}%
\pgfpathlineto{\pgfqpoint{0.937589in}{1.229739in}}%
\pgfpathlineto{\pgfqpoint{0.924189in}{1.216357in}}%
\pgfpathlineto{\pgfqpoint{0.921933in}{1.214276in}}%
\pgfpathlineto{\pgfqpoint{0.908670in}{1.202746in}}%
\pgfpathlineto{\pgfqpoint{0.906276in}{1.200785in}}%
\pgfpathlineto{\pgfqpoint{0.890884in}{1.189135in}}%
\pgfpathlineto{\pgfqpoint{0.890620in}{1.188943in}}%
\pgfpathlineto{\pgfqpoint{0.874963in}{1.178589in}}%
\pgfpathlineto{\pgfqpoint{0.869717in}{1.175524in}}%
\pgfpathlineto{\pgfqpoint{0.859306in}{1.169560in}}%
\pgfpathlineto{\pgfqpoint{0.843743in}{1.161913in}}%
\pgfpathlineto{\pgfqpoint{0.843650in}{1.161867in}}%
\pgfpathlineto{\pgfqpoint{0.827993in}{1.155420in}}%
\pgfpathlineto{\pgfqpoint{0.812337in}{1.150334in}}%
\pgfpathlineto{\pgfqpoint{0.803829in}{1.148302in}}%
\pgfpathlineto{\pgfqpoint{0.796680in}{1.146554in}}%
\pgfpathlineto{\pgfqpoint{0.781024in}{1.144109in}}%
\pgfpathlineto{\pgfqpoint{0.765367in}{1.143061in}}%
\pgfpathlineto{\pgfqpoint{0.749710in}{1.143410in}}%
\pgfpathlineto{\pgfqpoint{0.734054in}{1.145156in}}%
\pgfpathlineto{\pgfqpoint{0.718397in}{1.148301in}}%
\pgfpathlineto{\pgfqpoint{0.718395in}{1.148302in}}%
\pgfpathclose%
\pgfpathmoveto{\pgfqpoint{1.501226in}{1.077443in}}%
\pgfpathlineto{\pgfqpoint{1.516882in}{1.073987in}}%
\pgfpathlineto{\pgfqpoint{1.532539in}{1.072507in}}%
\pgfpathlineto{\pgfqpoint{1.548195in}{1.073001in}}%
\pgfpathlineto{\pgfqpoint{1.563852in}{1.075468in}}%
\pgfpathlineto{\pgfqpoint{1.579508in}{1.079912in}}%
\pgfpathlineto{\pgfqpoint{1.580330in}{1.080246in}}%
\pgfpathlineto{\pgfqpoint{1.595165in}{1.085477in}}%
\pgfpathlineto{\pgfqpoint{1.610822in}{1.092700in}}%
\pgfpathlineto{\pgfqpoint{1.612877in}{1.093857in}}%
\pgfpathlineto{\pgfqpoint{1.626478in}{1.100763in}}%
\pgfpathlineto{\pgfqpoint{1.637649in}{1.107468in}}%
\pgfpathlineto{\pgfqpoint{1.642135in}{1.109969in}}%
\pgfpathlineto{\pgfqpoint{1.657791in}{1.119958in}}%
\pgfpathlineto{\pgfqpoint{1.659372in}{1.121079in}}%
\pgfpathlineto{\pgfqpoint{1.673448in}{1.130611in}}%
\pgfpathlineto{\pgfqpoint{1.678950in}{1.134691in}}%
\pgfpathlineto{\pgfqpoint{1.689104in}{1.142032in}}%
\pgfpathlineto{\pgfqpoint{1.697172in}{1.148302in}}%
\pgfpathlineto{\pgfqpoint{1.704761in}{1.154166in}}%
\pgfpathlineto{\pgfqpoint{1.714250in}{1.161913in}}%
\pgfpathlineto{\pgfqpoint{1.720418in}{1.167010in}}%
\pgfpathlineto{\pgfqpoint{1.730336in}{1.175524in}}%
\pgfpathlineto{\pgfqpoint{1.736074in}{1.180597in}}%
\pgfpathlineto{\pgfqpoint{1.745530in}{1.189135in}}%
\pgfpathlineto{\pgfqpoint{1.751731in}{1.195000in}}%
\pgfpathlineto{\pgfqpoint{1.759893in}{1.202746in}}%
\pgfpathlineto{\pgfqpoint{1.767387in}{1.210325in}}%
\pgfpathlineto{\pgfqpoint{1.773444in}{1.216357in}}%
\pgfpathlineto{\pgfqpoint{1.783044in}{1.226727in}}%
\pgfpathlineto{\pgfqpoint{1.786152in}{1.229968in}}%
\pgfpathlineto{\pgfqpoint{1.797974in}{1.243579in}}%
\pgfpathlineto{\pgfqpoint{1.798700in}{1.244521in}}%
\pgfpathlineto{\pgfqpoint{1.809085in}{1.257191in}}%
\pgfpathlineto{\pgfqpoint{1.814357in}{1.264671in}}%
\pgfpathlineto{\pgfqpoint{1.819073in}{1.270802in}}%
\pgfpathlineto{\pgfqpoint{1.827797in}{1.284413in}}%
\pgfpathlineto{\pgfqpoint{1.830014in}{1.288764in}}%
\pgfpathlineto{\pgfqpoint{1.835348in}{1.298024in}}%
\pgfpathlineto{\pgfqpoint{1.841113in}{1.311635in}}%
\pgfpathlineto{\pgfqpoint{1.844778in}{1.325246in}}%
\pgfpathlineto{\pgfqpoint{1.845670in}{1.332973in}}%
\pgfpathlineto{\pgfqpoint{1.846476in}{1.338857in}}%
\pgfpathlineto{\pgfqpoint{1.845854in}{1.352468in}}%
\pgfpathlineto{\pgfqpoint{1.845670in}{1.353275in}}%
\pgfpathlineto{\pgfqpoint{1.843207in}{1.366079in}}%
\pgfpathlineto{\pgfqpoint{1.838493in}{1.379691in}}%
\pgfpathlineto{\pgfqpoint{1.831677in}{1.393302in}}%
\pgfpathlineto{\pgfqpoint{1.830014in}{1.395865in}}%
\pgfpathlineto{\pgfqpoint{1.823666in}{1.406913in}}%
\pgfpathlineto{\pgfqpoint{1.814357in}{1.420056in}}%
\pgfpathlineto{\pgfqpoint{1.814054in}{1.420524in}}%
\pgfpathlineto{\pgfqpoint{1.803705in}{1.434135in}}%
\pgfpathlineto{\pgfqpoint{1.798700in}{1.439882in}}%
\pgfpathlineto{\pgfqpoint{1.792249in}{1.447746in}}%
\pgfpathlineto{\pgfqpoint{1.783044in}{1.457780in}}%
\pgfpathlineto{\pgfqpoint{1.779880in}{1.461357in}}%
\pgfpathlineto{\pgfqpoint{1.767387in}{1.474252in}}%
\pgfpathlineto{\pgfqpoint{1.766704in}{1.474968in}}%
\pgfpathlineto{\pgfqpoint{1.752782in}{1.488579in}}%
\pgfpathlineto{\pgfqpoint{1.751731in}{1.489554in}}%
\pgfpathlineto{\pgfqpoint{1.738037in}{1.502191in}}%
\pgfpathlineto{\pgfqpoint{1.736074in}{1.503934in}}%
\pgfpathlineto{\pgfqpoint{1.722423in}{1.515802in}}%
\pgfpathlineto{\pgfqpoint{1.720418in}{1.517508in}}%
\pgfpathlineto{\pgfqpoint{1.705882in}{1.529413in}}%
\pgfpathlineto{\pgfqpoint{1.704761in}{1.530327in}}%
\pgfpathlineto{\pgfqpoint{1.689104in}{1.542430in}}%
\pgfpathlineto{\pgfqpoint{1.688281in}{1.543024in}}%
\pgfpathlineto{\pgfqpoint{1.673448in}{1.553884in}}%
\pgfpathlineto{\pgfqpoint{1.669333in}{1.556635in}}%
\pgfpathlineto{\pgfqpoint{1.657791in}{1.564638in}}%
\pgfpathlineto{\pgfqpoint{1.648746in}{1.570246in}}%
\pgfpathlineto{\pgfqpoint{1.642135in}{1.574597in}}%
\pgfpathlineto{\pgfqpoint{1.626478in}{1.583594in}}%
\pgfpathlineto{\pgfqpoint{1.625940in}{1.583857in}}%
\pgfpathlineto{\pgfqpoint{1.610822in}{1.591950in}}%
\pgfpathlineto{\pgfqpoint{1.598114in}{1.597468in}}%
\pgfpathlineto{\pgfqpoint{1.595165in}{1.598915in}}%
\pgfpathlineto{\pgfqpoint{1.579508in}{1.604840in}}%
\pgfpathlineto{\pgfqpoint{1.563852in}{1.608939in}}%
\pgfpathlineto{\pgfqpoint{1.549123in}{1.611079in}}%
\pgfpathlineto{\pgfqpoint{1.548195in}{1.611239in}}%
\pgfpathlineto{\pgfqpoint{1.532539in}{1.611780in}}%
\pgfpathlineto{\pgfqpoint{1.525770in}{1.611079in}}%
\pgfpathlineto{\pgfqpoint{1.516882in}{1.610304in}}%
\pgfpathlineto{\pgfqpoint{1.501226in}{1.607117in}}%
\pgfpathlineto{\pgfqpoint{1.485569in}{1.602106in}}%
\pgfpathlineto{\pgfqpoint{1.474917in}{1.597468in}}%
\pgfpathlineto{\pgfqpoint{1.469913in}{1.595542in}}%
\pgfpathlineto{\pgfqpoint{1.454256in}{1.587957in}}%
\pgfpathlineto{\pgfqpoint{1.447204in}{1.583857in}}%
\pgfpathlineto{\pgfqpoint{1.438599in}{1.579274in}}%
\pgfpathlineto{\pgfqpoint{1.424026in}{1.570246in}}%
\pgfpathlineto{\pgfqpoint{1.422943in}{1.569615in}}%
\pgfpathlineto{\pgfqpoint{1.407286in}{1.559338in}}%
\pgfpathlineto{\pgfqpoint{1.403558in}{1.556635in}}%
\pgfpathlineto{\pgfqpoint{1.391630in}{1.548290in}}%
\pgfpathlineto{\pgfqpoint{1.384690in}{1.543024in}}%
\pgfpathlineto{\pgfqpoint{1.375973in}{1.536509in}}%
\pgfpathlineto{\pgfqpoint{1.367063in}{1.529413in}}%
\pgfpathlineto{\pgfqpoint{1.360317in}{1.524022in}}%
\pgfpathlineto{\pgfqpoint{1.350496in}{1.515802in}}%
\pgfpathlineto{\pgfqpoint{1.344660in}{1.510813in}}%
\pgfpathlineto{\pgfqpoint{1.334866in}{1.502191in}}%
\pgfpathlineto{\pgfqpoint{1.329003in}{1.496829in}}%
\pgfpathlineto{\pgfqpoint{1.320093in}{1.488579in}}%
\pgfpathlineto{\pgfqpoint{1.313347in}{1.481982in}}%
\pgfpathlineto{\pgfqpoint{1.306135in}{1.474968in}}%
\pgfpathlineto{\pgfqpoint{1.297690in}{1.466141in}}%
\pgfpathlineto{\pgfqpoint{1.292998in}{1.461357in}}%
\pgfpathlineto{\pgfqpoint{1.282034in}{1.449120in}}%
\pgfpathlineto{\pgfqpoint{1.280744in}{1.447746in}}%
\pgfpathlineto{\pgfqpoint{1.269254in}{1.434135in}}%
\pgfpathlineto{\pgfqpoint{1.266377in}{1.430236in}}%
\pgfpathlineto{\pgfqpoint{1.258664in}{1.420524in}}%
\pgfpathlineto{\pgfqpoint{1.250721in}{1.408700in}}%
\pgfpathlineto{\pgfqpoint{1.249390in}{1.406913in}}%
\pgfpathlineto{\pgfqpoint{1.241081in}{1.393302in}}%
\pgfpathlineto{\pgfqpoint{1.235064in}{1.380405in}}%
\pgfpathlineto{\pgfqpoint{1.234680in}{1.379691in}}%
\pgfpathlineto{\pgfqpoint{1.229568in}{1.366079in}}%
\pgfpathlineto{\pgfqpoint{1.226730in}{1.352468in}}%
\pgfpathlineto{\pgfqpoint{1.226162in}{1.338857in}}%
\pgfpathlineto{\pgfqpoint{1.227865in}{1.325246in}}%
\pgfpathlineto{\pgfqpoint{1.231839in}{1.311635in}}%
\pgfpathlineto{\pgfqpoint{1.235064in}{1.304586in}}%
\pgfpathlineto{\pgfqpoint{1.237663in}{1.298024in}}%
\pgfpathlineto{\pgfqpoint{1.244990in}{1.284413in}}%
\pgfpathlineto{\pgfqpoint{1.250721in}{1.275973in}}%
\pgfpathlineto{\pgfqpoint{1.253883in}{1.270802in}}%
\pgfpathlineto{\pgfqpoint{1.263880in}{1.257191in}}%
\pgfpathlineto{\pgfqpoint{1.266377in}{1.254249in}}%
\pgfpathlineto{\pgfqpoint{1.274804in}{1.243579in}}%
\pgfpathlineto{\pgfqpoint{1.282034in}{1.235488in}}%
\pgfpathlineto{\pgfqpoint{1.286736in}{1.229968in}}%
\pgfpathlineto{\pgfqpoint{1.297690in}{1.218346in}}%
\pgfpathlineto{\pgfqpoint{1.299517in}{1.216357in}}%
\pgfpathlineto{\pgfqpoint{1.313070in}{1.202746in}}%
\pgfpathlineto{\pgfqpoint{1.313347in}{1.202484in}}%
\pgfpathlineto{\pgfqpoint{1.327390in}{1.189135in}}%
\pgfpathlineto{\pgfqpoint{1.329003in}{1.187671in}}%
\pgfpathlineto{\pgfqpoint{1.342564in}{1.175524in}}%
\pgfpathlineto{\pgfqpoint{1.344660in}{1.173702in}}%
\pgfpathlineto{\pgfqpoint{1.358632in}{1.161913in}}%
\pgfpathlineto{\pgfqpoint{1.360317in}{1.160510in}}%
\pgfpathlineto{\pgfqpoint{1.375671in}{1.148302in}}%
\pgfpathlineto{\pgfqpoint{1.375973in}{1.148061in}}%
\pgfpathlineto{\pgfqpoint{1.391630in}{1.136278in}}%
\pgfpathlineto{\pgfqpoint{1.393917in}{1.134691in}}%
\pgfpathlineto{\pgfqpoint{1.407286in}{1.125167in}}%
\pgfpathlineto{\pgfqpoint{1.413635in}{1.121079in}}%
\pgfpathlineto{\pgfqpoint{1.422943in}{1.114794in}}%
\pgfpathlineto{\pgfqpoint{1.435216in}{1.107468in}}%
\pgfpathlineto{\pgfqpoint{1.438599in}{1.105297in}}%
\pgfpathlineto{\pgfqpoint{1.454256in}{1.096607in}}%
\pgfpathlineto{\pgfqpoint{1.460204in}{1.093857in}}%
\pgfpathlineto{\pgfqpoint{1.469913in}{1.088876in}}%
\pgfpathlineto{\pgfqpoint{1.485569in}{1.082505in}}%
\pgfpathlineto{\pgfqpoint{1.493117in}{1.080246in}}%
\pgfpathlineto{\pgfqpoint{1.501226in}{1.077443in}}%
\pgfpathclose%
\pgfpathmoveto{\pgfqpoint{1.494077in}{1.148302in}}%
\pgfpathlineto{\pgfqpoint{1.485569in}{1.150334in}}%
\pgfpathlineto{\pgfqpoint{1.469913in}{1.155420in}}%
\pgfpathlineto{\pgfqpoint{1.454256in}{1.161867in}}%
\pgfpathlineto{\pgfqpoint{1.454163in}{1.161913in}}%
\pgfpathlineto{\pgfqpoint{1.438599in}{1.169560in}}%
\pgfpathlineto{\pgfqpoint{1.428188in}{1.175524in}}%
\pgfpathlineto{\pgfqpoint{1.422943in}{1.178589in}}%
\pgfpathlineto{\pgfqpoint{1.407286in}{1.188943in}}%
\pgfpathlineto{\pgfqpoint{1.407022in}{1.189135in}}%
\pgfpathlineto{\pgfqpoint{1.391630in}{1.200785in}}%
\pgfpathlineto{\pgfqpoint{1.389236in}{1.202746in}}%
\pgfpathlineto{\pgfqpoint{1.375973in}{1.214276in}}%
\pgfpathlineto{\pgfqpoint{1.373717in}{1.216357in}}%
\pgfpathlineto{\pgfqpoint{1.360317in}{1.229739in}}%
\pgfpathlineto{\pgfqpoint{1.360096in}{1.229968in}}%
\pgfpathlineto{\pgfqpoint{1.348186in}{1.243579in}}%
\pgfpathlineto{\pgfqpoint{1.344660in}{1.248140in}}%
\pgfpathlineto{\pgfqpoint{1.337800in}{1.257191in}}%
\pgfpathlineto{\pgfqpoint{1.329003in}{1.270721in}}%
\pgfpathlineto{\pgfqpoint{1.328951in}{1.270802in}}%
\pgfpathlineto{\pgfqpoint{1.321534in}{1.284413in}}%
\pgfpathlineto{\pgfqpoint{1.315684in}{1.298024in}}%
\pgfpathlineto{\pgfqpoint{1.313347in}{1.305420in}}%
\pgfpathlineto{\pgfqpoint{1.311336in}{1.311635in}}%
\pgfpathlineto{\pgfqpoint{1.308524in}{1.325246in}}%
\pgfpathlineto{\pgfqpoint{1.307319in}{1.338857in}}%
\pgfpathlineto{\pgfqpoint{1.307720in}{1.352468in}}%
\pgfpathlineto{\pgfqpoint{1.309729in}{1.366079in}}%
\pgfpathlineto{\pgfqpoint{1.313346in}{1.379691in}}%
\pgfpathlineto{\pgfqpoint{1.313347in}{1.379693in}}%
\pgfpathlineto{\pgfqpoint{1.318414in}{1.393302in}}%
\pgfpathlineto{\pgfqpoint{1.325047in}{1.406913in}}%
\pgfpathlineto{\pgfqpoint{1.329003in}{1.413532in}}%
\pgfpathlineto{\pgfqpoint{1.333184in}{1.420524in}}%
\pgfpathlineto{\pgfqpoint{1.342800in}{1.434135in}}%
\pgfpathlineto{\pgfqpoint{1.344660in}{1.436445in}}%
\pgfpathlineto{\pgfqpoint{1.353954in}{1.447746in}}%
\pgfpathlineto{\pgfqpoint{1.360317in}{1.454665in}}%
\pgfpathlineto{\pgfqpoint{1.366720in}{1.461357in}}%
\pgfpathlineto{\pgfqpoint{1.375973in}{1.470205in}}%
\pgfpathlineto{\pgfqpoint{1.381255in}{1.474968in}}%
\pgfpathlineto{\pgfqpoint{1.391630in}{1.483706in}}%
\pgfpathlineto{\pgfqpoint{1.397887in}{1.488579in}}%
\pgfpathlineto{\pgfqpoint{1.407286in}{1.495548in}}%
\pgfpathlineto{\pgfqpoint{1.417187in}{1.502191in}}%
\pgfpathlineto{\pgfqpoint{1.422943in}{1.505937in}}%
\pgfpathlineto{\pgfqpoint{1.438599in}{1.514949in}}%
\pgfpathlineto{\pgfqpoint{1.440312in}{1.515802in}}%
\pgfpathlineto{\pgfqpoint{1.454256in}{1.522678in}}%
\pgfpathlineto{\pgfqpoint{1.469913in}{1.529064in}}%
\pgfpathlineto{\pgfqpoint{1.470984in}{1.529413in}}%
\pgfpathlineto{\pgfqpoint{1.485569in}{1.534212in}}%
\pgfpathlineto{\pgfqpoint{1.501226in}{1.537989in}}%
\pgfpathlineto{\pgfqpoint{1.516882in}{1.540391in}}%
\pgfpathlineto{\pgfqpoint{1.532539in}{1.541419in}}%
\pgfpathlineto{\pgfqpoint{1.548195in}{1.541077in}}%
\pgfpathlineto{\pgfqpoint{1.563852in}{1.539362in}}%
\pgfpathlineto{\pgfqpoint{1.579508in}{1.536273in}}%
\pgfpathlineto{\pgfqpoint{1.595165in}{1.531807in}}%
\pgfpathlineto{\pgfqpoint{1.601623in}{1.529413in}}%
\pgfpathlineto{\pgfqpoint{1.610822in}{1.526040in}}%
\pgfpathlineto{\pgfqpoint{1.626478in}{1.518979in}}%
\pgfpathlineto{\pgfqpoint{1.632461in}{1.515802in}}%
\pgfpathlineto{\pgfqpoint{1.642135in}{1.510608in}}%
\pgfpathlineto{\pgfqpoint{1.655796in}{1.502191in}}%
\pgfpathlineto{\pgfqpoint{1.657791in}{1.500924in}}%
\pgfpathlineto{\pgfqpoint{1.673448in}{1.489849in}}%
\pgfpathlineto{\pgfqpoint{1.675092in}{1.488579in}}%
\pgfpathlineto{\pgfqpoint{1.689104in}{1.477212in}}%
\pgfpathlineto{\pgfqpoint{1.691685in}{1.474968in}}%
\pgfpathlineto{\pgfqpoint{1.704761in}{1.462787in}}%
\pgfpathlineto{\pgfqpoint{1.706221in}{1.461357in}}%
\pgfpathlineto{\pgfqpoint{1.718961in}{1.447746in}}%
\pgfpathlineto{\pgfqpoint{1.720418in}{1.446011in}}%
\pgfpathlineto{\pgfqpoint{1.730100in}{1.434135in}}%
\pgfpathlineto{\pgfqpoint{1.736074in}{1.425725in}}%
\pgfpathlineto{\pgfqpoint{1.739729in}{1.420524in}}%
\pgfpathlineto{\pgfqpoint{1.747851in}{1.406913in}}%
\pgfpathlineto{\pgfqpoint{1.751731in}{1.398916in}}%
\pgfpathlineto{\pgfqpoint{1.754484in}{1.393302in}}%
\pgfpathlineto{\pgfqpoint{1.759621in}{1.379691in}}%
\pgfpathlineto{\pgfqpoint{1.763175in}{1.366079in}}%
\pgfpathlineto{\pgfqpoint{1.765147in}{1.352468in}}%
\pgfpathlineto{\pgfqpoint{1.765542in}{1.338857in}}%
\pgfpathlineto{\pgfqpoint{1.764358in}{1.325246in}}%
\pgfpathlineto{\pgfqpoint{1.761596in}{1.311635in}}%
\pgfpathlineto{\pgfqpoint{1.757251in}{1.298024in}}%
\pgfpathlineto{\pgfqpoint{1.751731in}{1.285344in}}%
\pgfpathlineto{\pgfqpoint{1.751330in}{1.284413in}}%
\pgfpathlineto{\pgfqpoint{1.743983in}{1.270802in}}%
\pgfpathlineto{\pgfqpoint{1.736074in}{1.258679in}}%
\pgfpathlineto{\pgfqpoint{1.735094in}{1.257191in}}%
\pgfpathlineto{\pgfqpoint{1.724726in}{1.243579in}}%
\pgfpathlineto{\pgfqpoint{1.720418in}{1.238575in}}%
\pgfpathlineto{\pgfqpoint{1.712777in}{1.229968in}}%
\pgfpathlineto{\pgfqpoint{1.704761in}{1.221797in}}%
\pgfpathlineto{\pgfqpoint{1.699156in}{1.216357in}}%
\pgfpathlineto{\pgfqpoint{1.689104in}{1.207338in}}%
\pgfpathlineto{\pgfqpoint{1.683625in}{1.202746in}}%
\pgfpathlineto{\pgfqpoint{1.673448in}{1.194702in}}%
\pgfpathlineto{\pgfqpoint{1.665750in}{1.189135in}}%
\pgfpathlineto{\pgfqpoint{1.657791in}{1.183603in}}%
\pgfpathlineto{\pgfqpoint{1.644792in}{1.175524in}}%
\pgfpathlineto{\pgfqpoint{1.642135in}{1.173907in}}%
\pgfpathlineto{\pgfqpoint{1.626478in}{1.165547in}}%
\pgfpathlineto{\pgfqpoint{1.618435in}{1.161913in}}%
\pgfpathlineto{\pgfqpoint{1.610822in}{1.158473in}}%
\pgfpathlineto{\pgfqpoint{1.595165in}{1.152707in}}%
\pgfpathlineto{\pgfqpoint{1.579511in}{1.148302in}}%
\pgfpathlineto{\pgfqpoint{1.579508in}{1.148301in}}%
\pgfpathlineto{\pgfqpoint{1.563852in}{1.145156in}}%
\pgfpathlineto{\pgfqpoint{1.548195in}{1.143410in}}%
\pgfpathlineto{\pgfqpoint{1.532539in}{1.143061in}}%
\pgfpathlineto{\pgfqpoint{1.516882in}{1.144109in}}%
\pgfpathlineto{\pgfqpoint{1.501226in}{1.146554in}}%
\pgfpathlineto{\pgfqpoint{1.494077in}{1.148302in}}%
\pgfpathclose%
\pgfusepath{fill}%
\end{pgfscope}%
\begin{pgfscope}%
\pgfpathrectangle{\pgfqpoint{0.373953in}{0.331635in}}{\pgfqpoint{1.550000in}{1.347500in}}%
\pgfusepath{clip}%
\pgfsetbuttcap%
\pgfsetroundjoin%
\definecolor{currentfill}{rgb}{0.481929,0.136891,0.507989}%
\pgfsetfillcolor{currentfill}%
\pgfsetlinewidth{0.000000pt}%
\definecolor{currentstroke}{rgb}{0.000000,0.000000,0.000000}%
\pgfsetstrokecolor{currentstroke}%
\pgfsetdash{}{0pt}%
\pgfpathmoveto{\pgfqpoint{0.640115in}{0.331635in}}%
\pgfpathlineto{\pgfqpoint{0.655771in}{0.331635in}}%
\pgfpathlineto{\pgfqpoint{0.671428in}{0.331635in}}%
\pgfpathlineto{\pgfqpoint{0.687084in}{0.331635in}}%
\pgfpathlineto{\pgfqpoint{0.702741in}{0.331635in}}%
\pgfpathlineto{\pgfqpoint{0.718397in}{0.331635in}}%
\pgfpathlineto{\pgfqpoint{0.734054in}{0.331635in}}%
\pgfpathlineto{\pgfqpoint{0.749710in}{0.331635in}}%
\pgfpathlineto{\pgfqpoint{0.765367in}{0.331635in}}%
\pgfpathlineto{\pgfqpoint{0.781024in}{0.331635in}}%
\pgfpathlineto{\pgfqpoint{0.796680in}{0.331635in}}%
\pgfpathlineto{\pgfqpoint{0.812337in}{0.331635in}}%
\pgfpathlineto{\pgfqpoint{0.827993in}{0.331635in}}%
\pgfpathlineto{\pgfqpoint{0.843650in}{0.331635in}}%
\pgfpathlineto{\pgfqpoint{0.859306in}{0.331635in}}%
\pgfpathlineto{\pgfqpoint{0.874963in}{0.331635in}}%
\pgfpathlineto{\pgfqpoint{0.890620in}{0.331635in}}%
\pgfpathlineto{\pgfqpoint{0.894062in}{0.331635in}}%
\pgfpathlineto{\pgfqpoint{0.895623in}{0.345246in}}%
\pgfpathlineto{\pgfqpoint{0.900238in}{0.358857in}}%
\pgfpathlineto{\pgfqpoint{0.906276in}{0.369827in}}%
\pgfpathlineto{\pgfqpoint{0.907640in}{0.372468in}}%
\pgfpathlineto{\pgfqpoint{0.917158in}{0.386079in}}%
\pgfpathlineto{\pgfqpoint{0.921933in}{0.391631in}}%
\pgfpathlineto{\pgfqpoint{0.928505in}{0.399691in}}%
\pgfpathlineto{\pgfqpoint{0.937589in}{0.409307in}}%
\pgfpathlineto{\pgfqpoint{0.941216in}{0.413302in}}%
\pgfpathlineto{\pgfqpoint{0.953246in}{0.425214in}}%
\pgfpathlineto{\pgfqpoint{0.954918in}{0.426913in}}%
\pgfpathlineto{\pgfqpoint{0.968902in}{0.440090in}}%
\pgfpathlineto{\pgfqpoint{0.969358in}{0.440524in}}%
\pgfpathlineto{\pgfqpoint{0.984382in}{0.454135in}}%
\pgfpathlineto{\pgfqpoint{0.984559in}{0.454291in}}%
\pgfpathlineto{\pgfqpoint{0.999916in}{0.467746in}}%
\pgfpathlineto{\pgfqpoint{1.000216in}{0.468006in}}%
\pgfpathlineto{\pgfqpoint{1.015872in}{0.481296in}}%
\pgfpathlineto{\pgfqpoint{1.015947in}{0.481357in}}%
\pgfpathlineto{\pgfqpoint{1.031529in}{0.494113in}}%
\pgfpathlineto{\pgfqpoint{1.032640in}{0.494968in}}%
\pgfpathlineto{\pgfqpoint{1.047185in}{0.506368in}}%
\pgfpathlineto{\pgfqpoint{1.050273in}{0.508579in}}%
\pgfpathlineto{\pgfqpoint{1.062842in}{0.517882in}}%
\pgfpathlineto{\pgfqpoint{1.069438in}{0.522191in}}%
\pgfpathlineto{\pgfqpoint{1.078498in}{0.528388in}}%
\pgfpathlineto{\pgfqpoint{1.091383in}{0.535802in}}%
\pgfpathlineto{\pgfqpoint{1.094155in}{0.537496in}}%
\pgfpathlineto{\pgfqpoint{1.109812in}{0.544928in}}%
\pgfpathlineto{\pgfqpoint{1.123506in}{0.549413in}}%
\pgfpathlineto{\pgfqpoint{1.125468in}{0.550103in}}%
\pgfpathlineto{\pgfqpoint{1.141125in}{0.552895in}}%
\pgfpathlineto{\pgfqpoint{1.156781in}{0.552895in}}%
\pgfpathlineto{\pgfqpoint{1.172438in}{0.550103in}}%
\pgfpathlineto{\pgfqpoint{1.174400in}{0.549413in}}%
\pgfpathlineto{\pgfqpoint{1.188094in}{0.544928in}}%
\pgfpathlineto{\pgfqpoint{1.203751in}{0.537496in}}%
\pgfpathlineto{\pgfqpoint{1.206523in}{0.535802in}}%
\pgfpathlineto{\pgfqpoint{1.219407in}{0.528388in}}%
\pgfpathlineto{\pgfqpoint{1.228467in}{0.522191in}}%
\pgfpathlineto{\pgfqpoint{1.235064in}{0.517882in}}%
\pgfpathlineto{\pgfqpoint{1.247633in}{0.508579in}}%
\pgfpathlineto{\pgfqpoint{1.250721in}{0.506368in}}%
\pgfpathlineto{\pgfqpoint{1.265266in}{0.494968in}}%
\pgfpathlineto{\pgfqpoint{1.266377in}{0.494113in}}%
\pgfpathlineto{\pgfqpoint{1.281959in}{0.481357in}}%
\pgfpathlineto{\pgfqpoint{1.282034in}{0.481296in}}%
\pgfpathlineto{\pgfqpoint{1.297690in}{0.468006in}}%
\pgfpathlineto{\pgfqpoint{1.297990in}{0.467746in}}%
\pgfpathlineto{\pgfqpoint{1.313347in}{0.454291in}}%
\pgfpathlineto{\pgfqpoint{1.313524in}{0.454135in}}%
\pgfpathlineto{\pgfqpoint{1.328548in}{0.440524in}}%
\pgfpathlineto{\pgfqpoint{1.329003in}{0.440090in}}%
\pgfpathlineto{\pgfqpoint{1.342988in}{0.426913in}}%
\pgfpathlineto{\pgfqpoint{1.344660in}{0.425214in}}%
\pgfpathlineto{\pgfqpoint{1.356689in}{0.413302in}}%
\pgfpathlineto{\pgfqpoint{1.360317in}{0.409307in}}%
\pgfpathlineto{\pgfqpoint{1.369401in}{0.399691in}}%
\pgfpathlineto{\pgfqpoint{1.375973in}{0.391631in}}%
\pgfpathlineto{\pgfqpoint{1.380748in}{0.386079in}}%
\pgfpathlineto{\pgfqpoint{1.390266in}{0.372468in}}%
\pgfpathlineto{\pgfqpoint{1.391630in}{0.369827in}}%
\pgfpathlineto{\pgfqpoint{1.397667in}{0.358857in}}%
\pgfpathlineto{\pgfqpoint{1.402283in}{0.345246in}}%
\pgfpathlineto{\pgfqpoint{1.403843in}{0.331635in}}%
\pgfpathlineto{\pgfqpoint{1.407286in}{0.331635in}}%
\pgfpathlineto{\pgfqpoint{1.422943in}{0.331635in}}%
\pgfpathlineto{\pgfqpoint{1.438599in}{0.331635in}}%
\pgfpathlineto{\pgfqpoint{1.454256in}{0.331635in}}%
\pgfpathlineto{\pgfqpoint{1.469913in}{0.331635in}}%
\pgfpathlineto{\pgfqpoint{1.485569in}{0.331635in}}%
\pgfpathlineto{\pgfqpoint{1.501226in}{0.331635in}}%
\pgfpathlineto{\pgfqpoint{1.516882in}{0.331635in}}%
\pgfpathlineto{\pgfqpoint{1.532539in}{0.331635in}}%
\pgfpathlineto{\pgfqpoint{1.548195in}{0.331635in}}%
\pgfpathlineto{\pgfqpoint{1.563852in}{0.331635in}}%
\pgfpathlineto{\pgfqpoint{1.579508in}{0.331635in}}%
\pgfpathlineto{\pgfqpoint{1.595165in}{0.331635in}}%
\pgfpathlineto{\pgfqpoint{1.610822in}{0.331635in}}%
\pgfpathlineto{\pgfqpoint{1.626478in}{0.331635in}}%
\pgfpathlineto{\pgfqpoint{1.642135in}{0.331635in}}%
\pgfpathlineto{\pgfqpoint{1.657791in}{0.331635in}}%
\pgfpathlineto{\pgfqpoint{1.669038in}{0.331635in}}%
\pgfpathlineto{\pgfqpoint{1.670652in}{0.345246in}}%
\pgfpathlineto{\pgfqpoint{1.673448in}{0.353195in}}%
\pgfpathlineto{\pgfqpoint{1.675306in}{0.358857in}}%
\pgfpathlineto{\pgfqpoint{1.682570in}{0.372468in}}%
\pgfpathlineto{\pgfqpoint{1.689104in}{0.381546in}}%
\pgfpathlineto{\pgfqpoint{1.692182in}{0.386079in}}%
\pgfpathlineto{\pgfqpoint{1.703538in}{0.399691in}}%
\pgfpathlineto{\pgfqpoint{1.704761in}{0.400949in}}%
\pgfpathlineto{\pgfqpoint{1.716207in}{0.413302in}}%
\pgfpathlineto{\pgfqpoint{1.720418in}{0.417386in}}%
\pgfpathlineto{\pgfqpoint{1.729913in}{0.426913in}}%
\pgfpathlineto{\pgfqpoint{1.736074in}{0.432643in}}%
\pgfpathlineto{\pgfqpoint{1.744382in}{0.440524in}}%
\pgfpathlineto{\pgfqpoint{1.751731in}{0.447150in}}%
\pgfpathlineto{\pgfqpoint{1.759441in}{0.454135in}}%
\pgfpathlineto{\pgfqpoint{1.767387in}{0.461123in}}%
\pgfpathlineto{\pgfqpoint{1.775005in}{0.467746in}}%
\pgfpathlineto{\pgfqpoint{1.783044in}{0.474654in}}%
\pgfpathlineto{\pgfqpoint{1.791079in}{0.481357in}}%
\pgfpathlineto{\pgfqpoint{1.798700in}{0.487745in}}%
\pgfpathlineto{\pgfqpoint{1.807766in}{0.494968in}}%
\pgfpathlineto{\pgfqpoint{1.814357in}{0.500324in}}%
\pgfpathlineto{\pgfqpoint{1.825315in}{0.508579in}}%
\pgfpathlineto{\pgfqpoint{1.830014in}{0.512240in}}%
\pgfpathlineto{\pgfqpoint{1.844222in}{0.522191in}}%
\pgfpathlineto{\pgfqpoint{1.845670in}{0.523254in}}%
\pgfpathlineto{\pgfqpoint{1.861327in}{0.533126in}}%
\pgfpathlineto{\pgfqpoint{1.866541in}{0.535802in}}%
\pgfpathlineto{\pgfqpoint{1.876983in}{0.541482in}}%
\pgfpathlineto{\pgfqpoint{1.892640in}{0.547797in}}%
\pgfpathlineto{\pgfqpoint{1.899153in}{0.549413in}}%
\pgfpathlineto{\pgfqpoint{1.908296in}{0.551843in}}%
\pgfpathlineto{\pgfqpoint{1.923953in}{0.553246in}}%
\pgfpathlineto{\pgfqpoint{1.923953in}{0.563024in}}%
\pgfpathlineto{\pgfqpoint{1.923953in}{0.576635in}}%
\pgfpathlineto{\pgfqpoint{1.923953in}{0.590246in}}%
\pgfpathlineto{\pgfqpoint{1.923953in}{0.603857in}}%
\pgfpathlineto{\pgfqpoint{1.923953in}{0.617468in}}%
\pgfpathlineto{\pgfqpoint{1.923953in}{0.631079in}}%
\pgfpathlineto{\pgfqpoint{1.923953in}{0.644691in}}%
\pgfpathlineto{\pgfqpoint{1.923953in}{0.658302in}}%
\pgfpathlineto{\pgfqpoint{1.923953in}{0.671913in}}%
\pgfpathlineto{\pgfqpoint{1.923953in}{0.685524in}}%
\pgfpathlineto{\pgfqpoint{1.923953in}{0.699135in}}%
\pgfpathlineto{\pgfqpoint{1.923953in}{0.712746in}}%
\pgfpathlineto{\pgfqpoint{1.923953in}{0.726357in}}%
\pgfpathlineto{\pgfqpoint{1.923953in}{0.739968in}}%
\pgfpathlineto{\pgfqpoint{1.923953in}{0.753579in}}%
\pgfpathlineto{\pgfqpoint{1.923953in}{0.767191in}}%
\pgfpathlineto{\pgfqpoint{1.923953in}{0.780802in}}%
\pgfpathlineto{\pgfqpoint{1.923953in}{0.783795in}}%
\pgfpathlineto{\pgfqpoint{1.908296in}{0.785152in}}%
\pgfpathlineto{\pgfqpoint{1.892640in}{0.789164in}}%
\pgfpathlineto{\pgfqpoint{1.880021in}{0.794413in}}%
\pgfpathlineto{\pgfqpoint{1.876983in}{0.795598in}}%
\pgfpathlineto{\pgfqpoint{1.861327in}{0.803873in}}%
\pgfpathlineto{\pgfqpoint{1.854941in}{0.808024in}}%
\pgfpathlineto{\pgfqpoint{1.845670in}{0.813738in}}%
\pgfpathlineto{\pgfqpoint{1.834608in}{0.821635in}}%
\pgfpathlineto{\pgfqpoint{1.830014in}{0.824788in}}%
\pgfpathlineto{\pgfqpoint{1.816312in}{0.835246in}}%
\pgfpathlineto{\pgfqpoint{1.814357in}{0.836700in}}%
\pgfpathlineto{\pgfqpoint{1.799200in}{0.848857in}}%
\pgfpathlineto{\pgfqpoint{1.798700in}{0.849253in}}%
\pgfpathlineto{\pgfqpoint{1.783044in}{0.862314in}}%
\pgfpathlineto{\pgfqpoint{1.782865in}{0.862468in}}%
\pgfpathlineto{\pgfqpoint{1.767387in}{0.875819in}}%
\pgfpathlineto{\pgfqpoint{1.767089in}{0.876079in}}%
\pgfpathlineto{\pgfqpoint{1.751801in}{0.889691in}}%
\pgfpathlineto{\pgfqpoint{1.751731in}{0.889756in}}%
\pgfpathlineto{\pgfqpoint{1.737058in}{0.903302in}}%
\pgfpathlineto{\pgfqpoint{1.736074in}{0.904268in}}%
\pgfpathlineto{\pgfqpoint{1.722962in}{0.916913in}}%
\pgfpathlineto{\pgfqpoint{1.720418in}{0.919597in}}%
\pgfpathlineto{\pgfqpoint{1.709717in}{0.930524in}}%
\pgfpathlineto{\pgfqpoint{1.704761in}{0.936259in}}%
\pgfpathlineto{\pgfqpoint{1.697632in}{0.944135in}}%
\pgfpathlineto{\pgfqpoint{1.689104in}{0.955336in}}%
\pgfpathlineto{\pgfqpoint{1.687156in}{0.957746in}}%
\pgfpathlineto{\pgfqpoint{1.678607in}{0.971357in}}%
\pgfpathlineto{\pgfqpoint{1.673448in}{0.983262in}}%
\pgfpathlineto{\pgfqpoint{1.672654in}{0.984968in}}%
\pgfpathlineto{\pgfqpoint{1.669443in}{0.998579in}}%
\pgfpathlineto{\pgfqpoint{1.669443in}{1.012191in}}%
\pgfpathlineto{\pgfqpoint{1.672654in}{1.025802in}}%
\pgfpathlineto{\pgfqpoint{1.673448in}{1.027508in}}%
\pgfpathlineto{\pgfqpoint{1.678607in}{1.039413in}}%
\pgfpathlineto{\pgfqpoint{1.687156in}{1.053024in}}%
\pgfpathlineto{\pgfqpoint{1.689104in}{1.055434in}}%
\pgfpathlineto{\pgfqpoint{1.697632in}{1.066635in}}%
\pgfpathlineto{\pgfqpoint{1.704761in}{1.074511in}}%
\pgfpathlineto{\pgfqpoint{1.709717in}{1.080246in}}%
\pgfpathlineto{\pgfqpoint{1.720418in}{1.091173in}}%
\pgfpathlineto{\pgfqpoint{1.722962in}{1.093857in}}%
\pgfpathlineto{\pgfqpoint{1.736074in}{1.106502in}}%
\pgfpathlineto{\pgfqpoint{1.737058in}{1.107468in}}%
\pgfpathlineto{\pgfqpoint{1.751731in}{1.121014in}}%
\pgfpathlineto{\pgfqpoint{1.751801in}{1.121079in}}%
\pgfpathlineto{\pgfqpoint{1.767089in}{1.134691in}}%
\pgfpathlineto{\pgfqpoint{1.767387in}{1.134951in}}%
\pgfpathlineto{\pgfqpoint{1.782865in}{1.148302in}}%
\pgfpathlineto{\pgfqpoint{1.783044in}{1.148456in}}%
\pgfpathlineto{\pgfqpoint{1.798700in}{1.161517in}}%
\pgfpathlineto{\pgfqpoint{1.799200in}{1.161913in}}%
\pgfpathlineto{\pgfqpoint{1.814357in}{1.174070in}}%
\pgfpathlineto{\pgfqpoint{1.816312in}{1.175524in}}%
\pgfpathlineto{\pgfqpoint{1.830014in}{1.185982in}}%
\pgfpathlineto{\pgfqpoint{1.834608in}{1.189135in}}%
\pgfpathlineto{\pgfqpoint{1.845670in}{1.197032in}}%
\pgfpathlineto{\pgfqpoint{1.854941in}{1.202746in}}%
\pgfpathlineto{\pgfqpoint{1.861327in}{1.206897in}}%
\pgfpathlineto{\pgfqpoint{1.876983in}{1.215172in}}%
\pgfpathlineto{\pgfqpoint{1.880021in}{1.216357in}}%
\pgfpathlineto{\pgfqpoint{1.892640in}{1.221606in}}%
\pgfpathlineto{\pgfqpoint{1.908296in}{1.225618in}}%
\pgfpathlineto{\pgfqpoint{1.923953in}{1.226975in}}%
\pgfpathlineto{\pgfqpoint{1.923953in}{1.229968in}}%
\pgfpathlineto{\pgfqpoint{1.923953in}{1.243579in}}%
\pgfpathlineto{\pgfqpoint{1.923953in}{1.257191in}}%
\pgfpathlineto{\pgfqpoint{1.923953in}{1.270802in}}%
\pgfpathlineto{\pgfqpoint{1.923953in}{1.284413in}}%
\pgfpathlineto{\pgfqpoint{1.923953in}{1.298024in}}%
\pgfpathlineto{\pgfqpoint{1.923953in}{1.311635in}}%
\pgfpathlineto{\pgfqpoint{1.923953in}{1.325246in}}%
\pgfpathlineto{\pgfqpoint{1.923953in}{1.338857in}}%
\pgfpathlineto{\pgfqpoint{1.923953in}{1.352468in}}%
\pgfpathlineto{\pgfqpoint{1.923953in}{1.366079in}}%
\pgfpathlineto{\pgfqpoint{1.923953in}{1.379691in}}%
\pgfpathlineto{\pgfqpoint{1.923953in}{1.393302in}}%
\pgfpathlineto{\pgfqpoint{1.923953in}{1.406913in}}%
\pgfpathlineto{\pgfqpoint{1.923953in}{1.420524in}}%
\pgfpathlineto{\pgfqpoint{1.923953in}{1.434135in}}%
\pgfpathlineto{\pgfqpoint{1.923953in}{1.447746in}}%
\pgfpathlineto{\pgfqpoint{1.923953in}{1.457524in}}%
\pgfpathlineto{\pgfqpoint{1.908296in}{1.458927in}}%
\pgfpathlineto{\pgfqpoint{1.899153in}{1.461357in}}%
\pgfpathlineto{\pgfqpoint{1.892640in}{1.462973in}}%
\pgfpathlineto{\pgfqpoint{1.876983in}{1.469288in}}%
\pgfpathlineto{\pgfqpoint{1.866541in}{1.474968in}}%
\pgfpathlineto{\pgfqpoint{1.861327in}{1.477644in}}%
\pgfpathlineto{\pgfqpoint{1.845670in}{1.487516in}}%
\pgfpathlineto{\pgfqpoint{1.844222in}{1.488579in}}%
\pgfpathlineto{\pgfqpoint{1.830014in}{1.498530in}}%
\pgfpathlineto{\pgfqpoint{1.825315in}{1.502191in}}%
\pgfpathlineto{\pgfqpoint{1.814357in}{1.510446in}}%
\pgfpathlineto{\pgfqpoint{1.807766in}{1.515802in}}%
\pgfpathlineto{\pgfqpoint{1.798700in}{1.523025in}}%
\pgfpathlineto{\pgfqpoint{1.791079in}{1.529413in}}%
\pgfpathlineto{\pgfqpoint{1.783044in}{1.536116in}}%
\pgfpathlineto{\pgfqpoint{1.775005in}{1.543024in}}%
\pgfpathlineto{\pgfqpoint{1.767387in}{1.549647in}}%
\pgfpathlineto{\pgfqpoint{1.759441in}{1.556635in}}%
\pgfpathlineto{\pgfqpoint{1.751731in}{1.563620in}}%
\pgfpathlineto{\pgfqpoint{1.744382in}{1.570246in}}%
\pgfpathlineto{\pgfqpoint{1.736074in}{1.578127in}}%
\pgfpathlineto{\pgfqpoint{1.729913in}{1.583857in}}%
\pgfpathlineto{\pgfqpoint{1.720418in}{1.593384in}}%
\pgfpathlineto{\pgfqpoint{1.716207in}{1.597468in}}%
\pgfpathlineto{\pgfqpoint{1.704761in}{1.609821in}}%
\pgfpathlineto{\pgfqpoint{1.703538in}{1.611079in}}%
\pgfpathlineto{\pgfqpoint{1.692182in}{1.624691in}}%
\pgfpathlineto{\pgfqpoint{1.689104in}{1.629224in}}%
\pgfpathlineto{\pgfqpoint{1.682570in}{1.638302in}}%
\pgfpathlineto{\pgfqpoint{1.675306in}{1.651913in}}%
\pgfpathlineto{\pgfqpoint{1.673448in}{1.657575in}}%
\pgfpathlineto{\pgfqpoint{1.670652in}{1.665524in}}%
\pgfpathlineto{\pgfqpoint{1.669038in}{1.679135in}}%
\pgfpathlineto{\pgfqpoint{1.657791in}{1.679135in}}%
\pgfpathlineto{\pgfqpoint{1.642135in}{1.679135in}}%
\pgfpathlineto{\pgfqpoint{1.626478in}{1.679135in}}%
\pgfpathlineto{\pgfqpoint{1.610822in}{1.679135in}}%
\pgfpathlineto{\pgfqpoint{1.595165in}{1.679135in}}%
\pgfpathlineto{\pgfqpoint{1.579508in}{1.679135in}}%
\pgfpathlineto{\pgfqpoint{1.563852in}{1.679135in}}%
\pgfpathlineto{\pgfqpoint{1.548195in}{1.679135in}}%
\pgfpathlineto{\pgfqpoint{1.532539in}{1.679135in}}%
\pgfpathlineto{\pgfqpoint{1.516882in}{1.679135in}}%
\pgfpathlineto{\pgfqpoint{1.501226in}{1.679135in}}%
\pgfpathlineto{\pgfqpoint{1.485569in}{1.679135in}}%
\pgfpathlineto{\pgfqpoint{1.469913in}{1.679135in}}%
\pgfpathlineto{\pgfqpoint{1.454256in}{1.679135in}}%
\pgfpathlineto{\pgfqpoint{1.438599in}{1.679135in}}%
\pgfpathlineto{\pgfqpoint{1.422943in}{1.679135in}}%
\pgfpathlineto{\pgfqpoint{1.407286in}{1.679135in}}%
\pgfpathlineto{\pgfqpoint{1.403843in}{1.679135in}}%
\pgfpathlineto{\pgfqpoint{1.402283in}{1.665524in}}%
\pgfpathlineto{\pgfqpoint{1.397667in}{1.651913in}}%
\pgfpathlineto{\pgfqpoint{1.391630in}{1.640943in}}%
\pgfpathlineto{\pgfqpoint{1.390266in}{1.638302in}}%
\pgfpathlineto{\pgfqpoint{1.380748in}{1.624691in}}%
\pgfpathlineto{\pgfqpoint{1.375973in}{1.619139in}}%
\pgfpathlineto{\pgfqpoint{1.369401in}{1.611079in}}%
\pgfpathlineto{\pgfqpoint{1.360317in}{1.601463in}}%
\pgfpathlineto{\pgfqpoint{1.356689in}{1.597468in}}%
\pgfpathlineto{\pgfqpoint{1.344660in}{1.585556in}}%
\pgfpathlineto{\pgfqpoint{1.342988in}{1.583857in}}%
\pgfpathlineto{\pgfqpoint{1.329003in}{1.570680in}}%
\pgfpathlineto{\pgfqpoint{1.328548in}{1.570246in}}%
\pgfpathlineto{\pgfqpoint{1.313524in}{1.556635in}}%
\pgfpathlineto{\pgfqpoint{1.313347in}{1.556479in}}%
\pgfpathlineto{\pgfqpoint{1.297990in}{1.543024in}}%
\pgfpathlineto{\pgfqpoint{1.297690in}{1.542764in}}%
\pgfpathlineto{\pgfqpoint{1.282034in}{1.529474in}}%
\pgfpathlineto{\pgfqpoint{1.281959in}{1.529413in}}%
\pgfpathlineto{\pgfqpoint{1.266377in}{1.516657in}}%
\pgfpathlineto{\pgfqpoint{1.265266in}{1.515802in}}%
\pgfpathlineto{\pgfqpoint{1.250721in}{1.504402in}}%
\pgfpathlineto{\pgfqpoint{1.247633in}{1.502191in}}%
\pgfpathlineto{\pgfqpoint{1.235064in}{1.492888in}}%
\pgfpathlineto{\pgfqpoint{1.228467in}{1.488579in}}%
\pgfpathlineto{\pgfqpoint{1.219407in}{1.482382in}}%
\pgfpathlineto{\pgfqpoint{1.206523in}{1.474968in}}%
\pgfpathlineto{\pgfqpoint{1.203751in}{1.473274in}}%
\pgfpathlineto{\pgfqpoint{1.188094in}{1.465842in}}%
\pgfpathlineto{\pgfqpoint{1.174400in}{1.461357in}}%
\pgfpathlineto{\pgfqpoint{1.172438in}{1.460667in}}%
\pgfpathlineto{\pgfqpoint{1.156781in}{1.457875in}}%
\pgfpathlineto{\pgfqpoint{1.141125in}{1.457875in}}%
\pgfpathlineto{\pgfqpoint{1.125468in}{1.460667in}}%
\pgfpathlineto{\pgfqpoint{1.123506in}{1.461357in}}%
\pgfpathlineto{\pgfqpoint{1.109812in}{1.465842in}}%
\pgfpathlineto{\pgfqpoint{1.094155in}{1.473274in}}%
\pgfpathlineto{\pgfqpoint{1.091383in}{1.474968in}}%
\pgfpathlineto{\pgfqpoint{1.078498in}{1.482382in}}%
\pgfpathlineto{\pgfqpoint{1.069438in}{1.488579in}}%
\pgfpathlineto{\pgfqpoint{1.062842in}{1.492888in}}%
\pgfpathlineto{\pgfqpoint{1.050273in}{1.502191in}}%
\pgfpathlineto{\pgfqpoint{1.047185in}{1.504402in}}%
\pgfpathlineto{\pgfqpoint{1.032640in}{1.515802in}}%
\pgfpathlineto{\pgfqpoint{1.031529in}{1.516657in}}%
\pgfpathlineto{\pgfqpoint{1.015947in}{1.529413in}}%
\pgfpathlineto{\pgfqpoint{1.015872in}{1.529474in}}%
\pgfpathlineto{\pgfqpoint{1.000216in}{1.542764in}}%
\pgfpathlineto{\pgfqpoint{0.999916in}{1.543024in}}%
\pgfpathlineto{\pgfqpoint{0.984559in}{1.556479in}}%
\pgfpathlineto{\pgfqpoint{0.984382in}{1.556635in}}%
\pgfpathlineto{\pgfqpoint{0.969358in}{1.570246in}}%
\pgfpathlineto{\pgfqpoint{0.968902in}{1.570680in}}%
\pgfpathlineto{\pgfqpoint{0.954918in}{1.583857in}}%
\pgfpathlineto{\pgfqpoint{0.953246in}{1.585556in}}%
\pgfpathlineto{\pgfqpoint{0.941216in}{1.597468in}}%
\pgfpathlineto{\pgfqpoint{0.937589in}{1.601463in}}%
\pgfpathlineto{\pgfqpoint{0.928505in}{1.611079in}}%
\pgfpathlineto{\pgfqpoint{0.921933in}{1.619139in}}%
\pgfpathlineto{\pgfqpoint{0.917158in}{1.624691in}}%
\pgfpathlineto{\pgfqpoint{0.907640in}{1.638302in}}%
\pgfpathlineto{\pgfqpoint{0.906276in}{1.640943in}}%
\pgfpathlineto{\pgfqpoint{0.900238in}{1.651913in}}%
\pgfpathlineto{\pgfqpoint{0.895623in}{1.665524in}}%
\pgfpathlineto{\pgfqpoint{0.894062in}{1.679135in}}%
\pgfpathlineto{\pgfqpoint{0.890620in}{1.679135in}}%
\pgfpathlineto{\pgfqpoint{0.874963in}{1.679135in}}%
\pgfpathlineto{\pgfqpoint{0.859306in}{1.679135in}}%
\pgfpathlineto{\pgfqpoint{0.843650in}{1.679135in}}%
\pgfpathlineto{\pgfqpoint{0.827993in}{1.679135in}}%
\pgfpathlineto{\pgfqpoint{0.812337in}{1.679135in}}%
\pgfpathlineto{\pgfqpoint{0.796680in}{1.679135in}}%
\pgfpathlineto{\pgfqpoint{0.781024in}{1.679135in}}%
\pgfpathlineto{\pgfqpoint{0.765367in}{1.679135in}}%
\pgfpathlineto{\pgfqpoint{0.749710in}{1.679135in}}%
\pgfpathlineto{\pgfqpoint{0.734054in}{1.679135in}}%
\pgfpathlineto{\pgfqpoint{0.718397in}{1.679135in}}%
\pgfpathlineto{\pgfqpoint{0.702741in}{1.679135in}}%
\pgfpathlineto{\pgfqpoint{0.687084in}{1.679135in}}%
\pgfpathlineto{\pgfqpoint{0.671428in}{1.679135in}}%
\pgfpathlineto{\pgfqpoint{0.655771in}{1.679135in}}%
\pgfpathlineto{\pgfqpoint{0.640115in}{1.679135in}}%
\pgfpathlineto{\pgfqpoint{0.628868in}{1.679135in}}%
\pgfpathlineto{\pgfqpoint{0.627253in}{1.665524in}}%
\pgfpathlineto{\pgfqpoint{0.624458in}{1.657575in}}%
\pgfpathlineto{\pgfqpoint{0.622600in}{1.651913in}}%
\pgfpathlineto{\pgfqpoint{0.615336in}{1.638302in}}%
\pgfpathlineto{\pgfqpoint{0.608801in}{1.629224in}}%
\pgfpathlineto{\pgfqpoint{0.605723in}{1.624691in}}%
\pgfpathlineto{\pgfqpoint{0.594368in}{1.611079in}}%
\pgfpathlineto{\pgfqpoint{0.593145in}{1.609821in}}%
\pgfpathlineto{\pgfqpoint{0.581699in}{1.597468in}}%
\pgfpathlineto{\pgfqpoint{0.577488in}{1.593384in}}%
\pgfpathlineto{\pgfqpoint{0.567992in}{1.583857in}}%
\pgfpathlineto{\pgfqpoint{0.561832in}{1.578127in}}%
\pgfpathlineto{\pgfqpoint{0.553523in}{1.570246in}}%
\pgfpathlineto{\pgfqpoint{0.546175in}{1.563620in}}%
\pgfpathlineto{\pgfqpoint{0.538465in}{1.556635in}}%
\pgfpathlineto{\pgfqpoint{0.530519in}{1.549647in}}%
\pgfpathlineto{\pgfqpoint{0.522900in}{1.543024in}}%
\pgfpathlineto{\pgfqpoint{0.514862in}{1.536116in}}%
\pgfpathlineto{\pgfqpoint{0.506827in}{1.529413in}}%
\pgfpathlineto{\pgfqpoint{0.499205in}{1.523025in}}%
\pgfpathlineto{\pgfqpoint{0.490140in}{1.515802in}}%
\pgfpathlineto{\pgfqpoint{0.483549in}{1.510446in}}%
\pgfpathlineto{\pgfqpoint{0.472590in}{1.502191in}}%
\pgfpathlineto{\pgfqpoint{0.467892in}{1.498530in}}%
\pgfpathlineto{\pgfqpoint{0.453684in}{1.488579in}}%
\pgfpathlineto{\pgfqpoint{0.452236in}{1.487516in}}%
\pgfpathlineto{\pgfqpoint{0.436579in}{1.477644in}}%
\pgfpathlineto{\pgfqpoint{0.431364in}{1.474968in}}%
\pgfpathlineto{\pgfqpoint{0.420923in}{1.469288in}}%
\pgfpathlineto{\pgfqpoint{0.405266in}{1.462973in}}%
\pgfpathlineto{\pgfqpoint{0.398752in}{1.461357in}}%
\pgfpathlineto{\pgfqpoint{0.389609in}{1.458927in}}%
\pgfpathlineto{\pgfqpoint{0.373953in}{1.457524in}}%
\pgfpathlineto{\pgfqpoint{0.373953in}{1.447746in}}%
\pgfpathlineto{\pgfqpoint{0.373953in}{1.434135in}}%
\pgfpathlineto{\pgfqpoint{0.373953in}{1.420524in}}%
\pgfpathlineto{\pgfqpoint{0.373953in}{1.406913in}}%
\pgfpathlineto{\pgfqpoint{0.373953in}{1.393302in}}%
\pgfpathlineto{\pgfqpoint{0.373953in}{1.379691in}}%
\pgfpathlineto{\pgfqpoint{0.373953in}{1.366079in}}%
\pgfpathlineto{\pgfqpoint{0.373953in}{1.352468in}}%
\pgfpathlineto{\pgfqpoint{0.373953in}{1.338857in}}%
\pgfpathlineto{\pgfqpoint{0.373953in}{1.325246in}}%
\pgfpathlineto{\pgfqpoint{0.373953in}{1.311635in}}%
\pgfpathlineto{\pgfqpoint{0.373953in}{1.298024in}}%
\pgfpathlineto{\pgfqpoint{0.373953in}{1.284413in}}%
\pgfpathlineto{\pgfqpoint{0.373953in}{1.270802in}}%
\pgfpathlineto{\pgfqpoint{0.373953in}{1.257191in}}%
\pgfpathlineto{\pgfqpoint{0.373953in}{1.243579in}}%
\pgfpathlineto{\pgfqpoint{0.373953in}{1.229968in}}%
\pgfpathlineto{\pgfqpoint{0.373953in}{1.226975in}}%
\pgfpathlineto{\pgfqpoint{0.389609in}{1.225618in}}%
\pgfpathlineto{\pgfqpoint{0.405266in}{1.221606in}}%
\pgfpathlineto{\pgfqpoint{0.417885in}{1.216357in}}%
\pgfpathlineto{\pgfqpoint{0.420923in}{1.215172in}}%
\pgfpathlineto{\pgfqpoint{0.436579in}{1.206897in}}%
\pgfpathlineto{\pgfqpoint{0.442965in}{1.202746in}}%
\pgfpathlineto{\pgfqpoint{0.452236in}{1.197032in}}%
\pgfpathlineto{\pgfqpoint{0.463298in}{1.189135in}}%
\pgfpathlineto{\pgfqpoint{0.467892in}{1.185982in}}%
\pgfpathlineto{\pgfqpoint{0.481594in}{1.175524in}}%
\pgfpathlineto{\pgfqpoint{0.483549in}{1.174070in}}%
\pgfpathlineto{\pgfqpoint{0.498706in}{1.161913in}}%
\pgfpathlineto{\pgfqpoint{0.499205in}{1.161517in}}%
\pgfpathlineto{\pgfqpoint{0.514862in}{1.148456in}}%
\pgfpathlineto{\pgfqpoint{0.515041in}{1.148302in}}%
\pgfpathlineto{\pgfqpoint{0.530519in}{1.134951in}}%
\pgfpathlineto{\pgfqpoint{0.530817in}{1.134691in}}%
\pgfpathlineto{\pgfqpoint{0.546105in}{1.121079in}}%
\pgfpathlineto{\pgfqpoint{0.546175in}{1.121014in}}%
\pgfpathlineto{\pgfqpoint{0.560848in}{1.107468in}}%
\pgfpathlineto{\pgfqpoint{0.561832in}{1.106502in}}%
\pgfpathlineto{\pgfqpoint{0.574944in}{1.093857in}}%
\pgfpathlineto{\pgfqpoint{0.577488in}{1.091173in}}%
\pgfpathlineto{\pgfqpoint{0.588189in}{1.080246in}}%
\pgfpathlineto{\pgfqpoint{0.593145in}{1.074511in}}%
\pgfpathlineto{\pgfqpoint{0.600274in}{1.066635in}}%
\pgfpathlineto{\pgfqpoint{0.608801in}{1.055434in}}%
\pgfpathlineto{\pgfqpoint{0.610750in}{1.053024in}}%
\pgfpathlineto{\pgfqpoint{0.619299in}{1.039413in}}%
\pgfpathlineto{\pgfqpoint{0.624458in}{1.027508in}}%
\pgfpathlineto{\pgfqpoint{0.625252in}{1.025802in}}%
\pgfpathlineto{\pgfqpoint{0.628463in}{1.012191in}}%
\pgfpathlineto{\pgfqpoint{0.628463in}{0.998579in}}%
\pgfpathlineto{\pgfqpoint{0.625252in}{0.984968in}}%
\pgfpathlineto{\pgfqpoint{0.624458in}{0.983262in}}%
\pgfpathlineto{\pgfqpoint{0.619299in}{0.971357in}}%
\pgfpathlineto{\pgfqpoint{0.610750in}{0.957746in}}%
\pgfpathlineto{\pgfqpoint{0.608801in}{0.955336in}}%
\pgfpathlineto{\pgfqpoint{0.600274in}{0.944135in}}%
\pgfpathlineto{\pgfqpoint{0.593145in}{0.936259in}}%
\pgfpathlineto{\pgfqpoint{0.588189in}{0.930524in}}%
\pgfpathlineto{\pgfqpoint{0.577488in}{0.919597in}}%
\pgfpathlineto{\pgfqpoint{0.574944in}{0.916913in}}%
\pgfpathlineto{\pgfqpoint{0.561832in}{0.904268in}}%
\pgfpathlineto{\pgfqpoint{0.560848in}{0.903302in}}%
\pgfpathlineto{\pgfqpoint{0.546175in}{0.889756in}}%
\pgfpathlineto{\pgfqpoint{0.546105in}{0.889691in}}%
\pgfpathlineto{\pgfqpoint{0.530817in}{0.876079in}}%
\pgfpathlineto{\pgfqpoint{0.530519in}{0.875819in}}%
\pgfpathlineto{\pgfqpoint{0.515041in}{0.862468in}}%
\pgfpathlineto{\pgfqpoint{0.514862in}{0.862314in}}%
\pgfpathlineto{\pgfqpoint{0.499205in}{0.849253in}}%
\pgfpathlineto{\pgfqpoint{0.498706in}{0.848857in}}%
\pgfpathlineto{\pgfqpoint{0.483549in}{0.836700in}}%
\pgfpathlineto{\pgfqpoint{0.481594in}{0.835246in}}%
\pgfpathlineto{\pgfqpoint{0.467892in}{0.824788in}}%
\pgfpathlineto{\pgfqpoint{0.463298in}{0.821635in}}%
\pgfpathlineto{\pgfqpoint{0.452236in}{0.813738in}}%
\pgfpathlineto{\pgfqpoint{0.442965in}{0.808024in}}%
\pgfpathlineto{\pgfqpoint{0.436579in}{0.803873in}}%
\pgfpathlineto{\pgfqpoint{0.420923in}{0.795598in}}%
\pgfpathlineto{\pgfqpoint{0.417885in}{0.794413in}}%
\pgfpathlineto{\pgfqpoint{0.405266in}{0.789164in}}%
\pgfpathlineto{\pgfqpoint{0.389609in}{0.785152in}}%
\pgfpathlineto{\pgfqpoint{0.373953in}{0.783795in}}%
\pgfpathlineto{\pgfqpoint{0.373953in}{0.780802in}}%
\pgfpathlineto{\pgfqpoint{0.373953in}{0.767191in}}%
\pgfpathlineto{\pgfqpoint{0.373953in}{0.753579in}}%
\pgfpathlineto{\pgfqpoint{0.373953in}{0.739968in}}%
\pgfpathlineto{\pgfqpoint{0.373953in}{0.726357in}}%
\pgfpathlineto{\pgfqpoint{0.373953in}{0.712746in}}%
\pgfpathlineto{\pgfqpoint{0.373953in}{0.699135in}}%
\pgfpathlineto{\pgfqpoint{0.373953in}{0.685524in}}%
\pgfpathlineto{\pgfqpoint{0.373953in}{0.671913in}}%
\pgfpathlineto{\pgfqpoint{0.373953in}{0.658302in}}%
\pgfpathlineto{\pgfqpoint{0.373953in}{0.644691in}}%
\pgfpathlineto{\pgfqpoint{0.373953in}{0.631079in}}%
\pgfpathlineto{\pgfqpoint{0.373953in}{0.617468in}}%
\pgfpathlineto{\pgfqpoint{0.373953in}{0.603857in}}%
\pgfpathlineto{\pgfqpoint{0.373953in}{0.590246in}}%
\pgfpathlineto{\pgfqpoint{0.373953in}{0.576635in}}%
\pgfpathlineto{\pgfqpoint{0.373953in}{0.563024in}}%
\pgfpathlineto{\pgfqpoint{0.373953in}{0.553246in}}%
\pgfpathlineto{\pgfqpoint{0.389609in}{0.551843in}}%
\pgfpathlineto{\pgfqpoint{0.398752in}{0.549413in}}%
\pgfpathlineto{\pgfqpoint{0.405266in}{0.547797in}}%
\pgfpathlineto{\pgfqpoint{0.420923in}{0.541482in}}%
\pgfpathlineto{\pgfqpoint{0.431364in}{0.535802in}}%
\pgfpathlineto{\pgfqpoint{0.436579in}{0.533126in}}%
\pgfpathlineto{\pgfqpoint{0.452236in}{0.523254in}}%
\pgfpathlineto{\pgfqpoint{0.453684in}{0.522191in}}%
\pgfpathlineto{\pgfqpoint{0.467892in}{0.512240in}}%
\pgfpathlineto{\pgfqpoint{0.472590in}{0.508579in}}%
\pgfpathlineto{\pgfqpoint{0.483549in}{0.500324in}}%
\pgfpathlineto{\pgfqpoint{0.490140in}{0.494968in}}%
\pgfpathlineto{\pgfqpoint{0.499205in}{0.487745in}}%
\pgfpathlineto{\pgfqpoint{0.506827in}{0.481357in}}%
\pgfpathlineto{\pgfqpoint{0.514862in}{0.474654in}}%
\pgfpathlineto{\pgfqpoint{0.522900in}{0.467746in}}%
\pgfpathlineto{\pgfqpoint{0.530519in}{0.461123in}}%
\pgfpathlineto{\pgfqpoint{0.538465in}{0.454135in}}%
\pgfpathlineto{\pgfqpoint{0.546175in}{0.447150in}}%
\pgfpathlineto{\pgfqpoint{0.553523in}{0.440524in}}%
\pgfpathlineto{\pgfqpoint{0.561832in}{0.432643in}}%
\pgfpathlineto{\pgfqpoint{0.567992in}{0.426913in}}%
\pgfpathlineto{\pgfqpoint{0.577488in}{0.417386in}}%
\pgfpathlineto{\pgfqpoint{0.581699in}{0.413302in}}%
\pgfpathlineto{\pgfqpoint{0.593145in}{0.400949in}}%
\pgfpathlineto{\pgfqpoint{0.594368in}{0.399691in}}%
\pgfpathlineto{\pgfqpoint{0.605723in}{0.386079in}}%
\pgfpathlineto{\pgfqpoint{0.608801in}{0.381546in}}%
\pgfpathlineto{\pgfqpoint{0.615336in}{0.372468in}}%
\pgfpathlineto{\pgfqpoint{0.622600in}{0.358857in}}%
\pgfpathlineto{\pgfqpoint{0.624458in}{0.353195in}}%
\pgfpathlineto{\pgfqpoint{0.627253in}{0.345246in}}%
\pgfpathlineto{\pgfqpoint{0.628868in}{0.331635in}}%
\pgfpathlineto{\pgfqpoint{0.640115in}{0.331635in}}%
\pgfpathclose%
\pgfpathmoveto{\pgfqpoint{0.748782in}{0.399691in}}%
\pgfpathlineto{\pgfqpoint{0.734054in}{0.401831in}}%
\pgfpathlineto{\pgfqpoint{0.718397in}{0.405930in}}%
\pgfpathlineto{\pgfqpoint{0.702741in}{0.411855in}}%
\pgfpathlineto{\pgfqpoint{0.699792in}{0.413302in}}%
\pgfpathlineto{\pgfqpoint{0.687084in}{0.418820in}}%
\pgfpathlineto{\pgfqpoint{0.671966in}{0.426913in}}%
\pgfpathlineto{\pgfqpoint{0.671428in}{0.427176in}}%
\pgfpathlineto{\pgfqpoint{0.655771in}{0.436173in}}%
\pgfpathlineto{\pgfqpoint{0.649160in}{0.440524in}}%
\pgfpathlineto{\pgfqpoint{0.640115in}{0.446132in}}%
\pgfpathlineto{\pgfqpoint{0.628573in}{0.454135in}}%
\pgfpathlineto{\pgfqpoint{0.624458in}{0.456886in}}%
\pgfpathlineto{\pgfqpoint{0.609625in}{0.467746in}}%
\pgfpathlineto{\pgfqpoint{0.608801in}{0.468340in}}%
\pgfpathlineto{\pgfqpoint{0.593145in}{0.480443in}}%
\pgfpathlineto{\pgfqpoint{0.592024in}{0.481357in}}%
\pgfpathlineto{\pgfqpoint{0.577488in}{0.493262in}}%
\pgfpathlineto{\pgfqpoint{0.575483in}{0.494968in}}%
\pgfpathlineto{\pgfqpoint{0.561832in}{0.506836in}}%
\pgfpathlineto{\pgfqpoint{0.559869in}{0.508579in}}%
\pgfpathlineto{\pgfqpoint{0.546175in}{0.521216in}}%
\pgfpathlineto{\pgfqpoint{0.545124in}{0.522191in}}%
\pgfpathlineto{\pgfqpoint{0.531201in}{0.535802in}}%
\pgfpathlineto{\pgfqpoint{0.530519in}{0.536518in}}%
\pgfpathlineto{\pgfqpoint{0.518026in}{0.549413in}}%
\pgfpathlineto{\pgfqpoint{0.514862in}{0.552990in}}%
\pgfpathlineto{\pgfqpoint{0.505657in}{0.563024in}}%
\pgfpathlineto{\pgfqpoint{0.499205in}{0.570888in}}%
\pgfpathlineto{\pgfqpoint{0.494201in}{0.576635in}}%
\pgfpathlineto{\pgfqpoint{0.483852in}{0.590246in}}%
\pgfpathlineto{\pgfqpoint{0.483549in}{0.590714in}}%
\pgfpathlineto{\pgfqpoint{0.474240in}{0.603857in}}%
\pgfpathlineto{\pgfqpoint{0.467892in}{0.614905in}}%
\pgfpathlineto{\pgfqpoint{0.466229in}{0.617468in}}%
\pgfpathlineto{\pgfqpoint{0.459413in}{0.631079in}}%
\pgfpathlineto{\pgfqpoint{0.454698in}{0.644691in}}%
\pgfpathlineto{\pgfqpoint{0.452236in}{0.657495in}}%
\pgfpathlineto{\pgfqpoint{0.452052in}{0.658302in}}%
\pgfpathlineto{\pgfqpoint{0.451429in}{0.671913in}}%
\pgfpathlineto{\pgfqpoint{0.452236in}{0.677797in}}%
\pgfpathlineto{\pgfqpoint{0.453128in}{0.685524in}}%
\pgfpathlineto{\pgfqpoint{0.456793in}{0.699135in}}%
\pgfpathlineto{\pgfqpoint{0.462558in}{0.712746in}}%
\pgfpathlineto{\pgfqpoint{0.467892in}{0.722006in}}%
\pgfpathlineto{\pgfqpoint{0.470108in}{0.726357in}}%
\pgfpathlineto{\pgfqpoint{0.478833in}{0.739968in}}%
\pgfpathlineto{\pgfqpoint{0.483549in}{0.746099in}}%
\pgfpathlineto{\pgfqpoint{0.488820in}{0.753579in}}%
\pgfpathlineto{\pgfqpoint{0.499205in}{0.766249in}}%
\pgfpathlineto{\pgfqpoint{0.499932in}{0.767191in}}%
\pgfpathlineto{\pgfqpoint{0.511753in}{0.780802in}}%
\pgfpathlineto{\pgfqpoint{0.514862in}{0.784043in}}%
\pgfpathlineto{\pgfqpoint{0.524461in}{0.794413in}}%
\pgfpathlineto{\pgfqpoint{0.530519in}{0.800445in}}%
\pgfpathlineto{\pgfqpoint{0.538013in}{0.808024in}}%
\pgfpathlineto{\pgfqpoint{0.546175in}{0.815770in}}%
\pgfpathlineto{\pgfqpoint{0.552376in}{0.821635in}}%
\pgfpathlineto{\pgfqpoint{0.561832in}{0.830173in}}%
\pgfpathlineto{\pgfqpoint{0.567570in}{0.835246in}}%
\pgfpathlineto{\pgfqpoint{0.577488in}{0.843760in}}%
\pgfpathlineto{\pgfqpoint{0.583655in}{0.848857in}}%
\pgfpathlineto{\pgfqpoint{0.593145in}{0.856604in}}%
\pgfpathlineto{\pgfqpoint{0.600734in}{0.862468in}}%
\pgfpathlineto{\pgfqpoint{0.608801in}{0.868738in}}%
\pgfpathlineto{\pgfqpoint{0.618955in}{0.876079in}}%
\pgfpathlineto{\pgfqpoint{0.624458in}{0.880159in}}%
\pgfpathlineto{\pgfqpoint{0.638534in}{0.889691in}}%
\pgfpathlineto{\pgfqpoint{0.640115in}{0.890812in}}%
\pgfpathlineto{\pgfqpoint{0.655771in}{0.900801in}}%
\pgfpathlineto{\pgfqpoint{0.660256in}{0.903302in}}%
\pgfpathlineto{\pgfqpoint{0.671428in}{0.910007in}}%
\pgfpathlineto{\pgfqpoint{0.685029in}{0.916913in}}%
\pgfpathlineto{\pgfqpoint{0.687084in}{0.918070in}}%
\pgfpathlineto{\pgfqpoint{0.702741in}{0.925293in}}%
\pgfpathlineto{\pgfqpoint{0.717576in}{0.930524in}}%
\pgfpathlineto{\pgfqpoint{0.718397in}{0.930858in}}%
\pgfpathlineto{\pgfqpoint{0.734054in}{0.935302in}}%
\pgfpathlineto{\pgfqpoint{0.749710in}{0.937769in}}%
\pgfpathlineto{\pgfqpoint{0.765367in}{0.938263in}}%
\pgfpathlineto{\pgfqpoint{0.781024in}{0.936783in}}%
\pgfpathlineto{\pgfqpoint{0.796680in}{0.933327in}}%
\pgfpathlineto{\pgfqpoint{0.804789in}{0.930524in}}%
\pgfpathlineto{\pgfqpoint{0.812337in}{0.928265in}}%
\pgfpathlineto{\pgfqpoint{0.827993in}{0.921894in}}%
\pgfpathlineto{\pgfqpoint{0.837702in}{0.916913in}}%
\pgfpathlineto{\pgfqpoint{0.843650in}{0.914163in}}%
\pgfpathlineto{\pgfqpoint{0.859306in}{0.905473in}}%
\pgfpathlineto{\pgfqpoint{0.862690in}{0.903302in}}%
\pgfpathlineto{\pgfqpoint{0.874963in}{0.895976in}}%
\pgfpathlineto{\pgfqpoint{0.884270in}{0.889691in}}%
\pgfpathlineto{\pgfqpoint{0.890620in}{0.885603in}}%
\pgfpathlineto{\pgfqpoint{0.903988in}{0.876079in}}%
\pgfpathlineto{\pgfqpoint{0.906276in}{0.874492in}}%
\pgfpathlineto{\pgfqpoint{0.921933in}{0.862709in}}%
\pgfpathlineto{\pgfqpoint{0.922234in}{0.862468in}}%
\pgfpathlineto{\pgfqpoint{0.937589in}{0.850260in}}%
\pgfpathlineto{\pgfqpoint{0.939273in}{0.848857in}}%
\pgfpathlineto{\pgfqpoint{0.953246in}{0.837068in}}%
\pgfpathlineto{\pgfqpoint{0.955341in}{0.835246in}}%
\pgfpathlineto{\pgfqpoint{0.968902in}{0.823099in}}%
\pgfpathlineto{\pgfqpoint{0.970516in}{0.821635in}}%
\pgfpathlineto{\pgfqpoint{0.984559in}{0.808286in}}%
\pgfpathlineto{\pgfqpoint{0.984836in}{0.808024in}}%
\pgfpathlineto{\pgfqpoint{0.998389in}{0.794413in}}%
\pgfpathlineto{\pgfqpoint{1.000216in}{0.792424in}}%
\pgfpathlineto{\pgfqpoint{1.011170in}{0.780802in}}%
\pgfpathlineto{\pgfqpoint{1.015872in}{0.775282in}}%
\pgfpathlineto{\pgfqpoint{1.023102in}{0.767191in}}%
\pgfpathlineto{\pgfqpoint{1.031529in}{0.756521in}}%
\pgfpathlineto{\pgfqpoint{1.034026in}{0.753579in}}%
\pgfpathlineto{\pgfqpoint{1.044022in}{0.739968in}}%
\pgfpathlineto{\pgfqpoint{1.047185in}{0.734797in}}%
\pgfpathlineto{\pgfqpoint{1.052916in}{0.726357in}}%
\pgfpathlineto{\pgfqpoint{1.060243in}{0.712746in}}%
\pgfpathlineto{\pgfqpoint{1.062842in}{0.706184in}}%
\pgfpathlineto{\pgfqpoint{1.066067in}{0.699135in}}%
\pgfpathlineto{\pgfqpoint{1.070041in}{0.685524in}}%
\pgfpathlineto{\pgfqpoint{1.071743in}{0.671913in}}%
\pgfpathlineto{\pgfqpoint{1.071176in}{0.658302in}}%
\pgfpathlineto{\pgfqpoint{1.068338in}{0.644691in}}%
\pgfpathlineto{\pgfqpoint{1.063226in}{0.631079in}}%
\pgfpathlineto{\pgfqpoint{1.062842in}{0.630365in}}%
\pgfpathlineto{\pgfqpoint{1.056824in}{0.617468in}}%
\pgfpathlineto{\pgfqpoint{1.048516in}{0.603857in}}%
\pgfpathlineto{\pgfqpoint{1.047185in}{0.602070in}}%
\pgfpathlineto{\pgfqpoint{1.039241in}{0.590246in}}%
\pgfpathlineto{\pgfqpoint{1.031529in}{0.580534in}}%
\pgfpathlineto{\pgfqpoint{1.028652in}{0.576635in}}%
\pgfpathlineto{\pgfqpoint{1.017162in}{0.563024in}}%
\pgfpathlineto{\pgfqpoint{1.015872in}{0.561650in}}%
\pgfpathlineto{\pgfqpoint{1.004908in}{0.549413in}}%
\pgfpathlineto{\pgfqpoint{1.000216in}{0.544629in}}%
\pgfpathlineto{\pgfqpoint{0.991771in}{0.535802in}}%
\pgfpathlineto{\pgfqpoint{0.984559in}{0.528788in}}%
\pgfpathlineto{\pgfqpoint{0.977813in}{0.522191in}}%
\pgfpathlineto{\pgfqpoint{0.968902in}{0.513941in}}%
\pgfpathlineto{\pgfqpoint{0.963040in}{0.508579in}}%
\pgfpathlineto{\pgfqpoint{0.953246in}{0.499957in}}%
\pgfpathlineto{\pgfqpoint{0.947410in}{0.494968in}}%
\pgfpathlineto{\pgfqpoint{0.937589in}{0.486748in}}%
\pgfpathlineto{\pgfqpoint{0.930843in}{0.481357in}}%
\pgfpathlineto{\pgfqpoint{0.921933in}{0.474261in}}%
\pgfpathlineto{\pgfqpoint{0.913215in}{0.467746in}}%
\pgfpathlineto{\pgfqpoint{0.906276in}{0.462480in}}%
\pgfpathlineto{\pgfqpoint{0.894348in}{0.454135in}}%
\pgfpathlineto{\pgfqpoint{0.890620in}{0.451432in}}%
\pgfpathlineto{\pgfqpoint{0.874963in}{0.441155in}}%
\pgfpathlineto{\pgfqpoint{0.873880in}{0.440524in}}%
\pgfpathlineto{\pgfqpoint{0.859306in}{0.431496in}}%
\pgfpathlineto{\pgfqpoint{0.850702in}{0.426913in}}%
\pgfpathlineto{\pgfqpoint{0.843650in}{0.422813in}}%
\pgfpathlineto{\pgfqpoint{0.827993in}{0.415228in}}%
\pgfpathlineto{\pgfqpoint{0.822989in}{0.413302in}}%
\pgfpathlineto{\pgfqpoint{0.812337in}{0.408664in}}%
\pgfpathlineto{\pgfqpoint{0.796680in}{0.403653in}}%
\pgfpathlineto{\pgfqpoint{0.781024in}{0.400466in}}%
\pgfpathlineto{\pgfqpoint{0.772136in}{0.399691in}}%
\pgfpathlineto{\pgfqpoint{0.765367in}{0.398990in}}%
\pgfpathlineto{\pgfqpoint{0.749710in}{0.399531in}}%
\pgfpathlineto{\pgfqpoint{0.748782in}{0.399691in}}%
\pgfpathclose%
\pgfpathmoveto{\pgfqpoint{1.525770in}{0.399691in}}%
\pgfpathlineto{\pgfqpoint{1.516882in}{0.400466in}}%
\pgfpathlineto{\pgfqpoint{1.501226in}{0.403653in}}%
\pgfpathlineto{\pgfqpoint{1.485569in}{0.408664in}}%
\pgfpathlineto{\pgfqpoint{1.474917in}{0.413302in}}%
\pgfpathlineto{\pgfqpoint{1.469913in}{0.415228in}}%
\pgfpathlineto{\pgfqpoint{1.454256in}{0.422813in}}%
\pgfpathlineto{\pgfqpoint{1.447204in}{0.426913in}}%
\pgfpathlineto{\pgfqpoint{1.438599in}{0.431496in}}%
\pgfpathlineto{\pgfqpoint{1.424026in}{0.440524in}}%
\pgfpathlineto{\pgfqpoint{1.422943in}{0.441155in}}%
\pgfpathlineto{\pgfqpoint{1.407286in}{0.451432in}}%
\pgfpathlineto{\pgfqpoint{1.403558in}{0.454135in}}%
\pgfpathlineto{\pgfqpoint{1.391630in}{0.462480in}}%
\pgfpathlineto{\pgfqpoint{1.384690in}{0.467746in}}%
\pgfpathlineto{\pgfqpoint{1.375973in}{0.474261in}}%
\pgfpathlineto{\pgfqpoint{1.367063in}{0.481357in}}%
\pgfpathlineto{\pgfqpoint{1.360317in}{0.486748in}}%
\pgfpathlineto{\pgfqpoint{1.350496in}{0.494968in}}%
\pgfpathlineto{\pgfqpoint{1.344660in}{0.499957in}}%
\pgfpathlineto{\pgfqpoint{1.334866in}{0.508579in}}%
\pgfpathlineto{\pgfqpoint{1.329003in}{0.513941in}}%
\pgfpathlineto{\pgfqpoint{1.320093in}{0.522191in}}%
\pgfpathlineto{\pgfqpoint{1.313347in}{0.528788in}}%
\pgfpathlineto{\pgfqpoint{1.306135in}{0.535802in}}%
\pgfpathlineto{\pgfqpoint{1.297690in}{0.544629in}}%
\pgfpathlineto{\pgfqpoint{1.292998in}{0.549413in}}%
\pgfpathlineto{\pgfqpoint{1.282034in}{0.561650in}}%
\pgfpathlineto{\pgfqpoint{1.280744in}{0.563024in}}%
\pgfpathlineto{\pgfqpoint{1.269254in}{0.576635in}}%
\pgfpathlineto{\pgfqpoint{1.266377in}{0.580534in}}%
\pgfpathlineto{\pgfqpoint{1.258664in}{0.590246in}}%
\pgfpathlineto{\pgfqpoint{1.250721in}{0.602070in}}%
\pgfpathlineto{\pgfqpoint{1.249390in}{0.603857in}}%
\pgfpathlineto{\pgfqpoint{1.241081in}{0.617468in}}%
\pgfpathlineto{\pgfqpoint{1.235064in}{0.630365in}}%
\pgfpathlineto{\pgfqpoint{1.234680in}{0.631079in}}%
\pgfpathlineto{\pgfqpoint{1.229568in}{0.644691in}}%
\pgfpathlineto{\pgfqpoint{1.226730in}{0.658302in}}%
\pgfpathlineto{\pgfqpoint{1.226162in}{0.671913in}}%
\pgfpathlineto{\pgfqpoint{1.227865in}{0.685524in}}%
\pgfpathlineto{\pgfqpoint{1.231839in}{0.699135in}}%
\pgfpathlineto{\pgfqpoint{1.235064in}{0.706184in}}%
\pgfpathlineto{\pgfqpoint{1.237663in}{0.712746in}}%
\pgfpathlineto{\pgfqpoint{1.244990in}{0.726357in}}%
\pgfpathlineto{\pgfqpoint{1.250721in}{0.734797in}}%
\pgfpathlineto{\pgfqpoint{1.253883in}{0.739968in}}%
\pgfpathlineto{\pgfqpoint{1.263880in}{0.753579in}}%
\pgfpathlineto{\pgfqpoint{1.266377in}{0.756521in}}%
\pgfpathlineto{\pgfqpoint{1.274804in}{0.767191in}}%
\pgfpathlineto{\pgfqpoint{1.282034in}{0.775282in}}%
\pgfpathlineto{\pgfqpoint{1.286736in}{0.780802in}}%
\pgfpathlineto{\pgfqpoint{1.297690in}{0.792424in}}%
\pgfpathlineto{\pgfqpoint{1.299517in}{0.794413in}}%
\pgfpathlineto{\pgfqpoint{1.313070in}{0.808024in}}%
\pgfpathlineto{\pgfqpoint{1.313347in}{0.808286in}}%
\pgfpathlineto{\pgfqpoint{1.327390in}{0.821635in}}%
\pgfpathlineto{\pgfqpoint{1.329003in}{0.823099in}}%
\pgfpathlineto{\pgfqpoint{1.342564in}{0.835246in}}%
\pgfpathlineto{\pgfqpoint{1.344660in}{0.837068in}}%
\pgfpathlineto{\pgfqpoint{1.358632in}{0.848857in}}%
\pgfpathlineto{\pgfqpoint{1.360317in}{0.850260in}}%
\pgfpathlineto{\pgfqpoint{1.375671in}{0.862468in}}%
\pgfpathlineto{\pgfqpoint{1.375973in}{0.862709in}}%
\pgfpathlineto{\pgfqpoint{1.391630in}{0.874492in}}%
\pgfpathlineto{\pgfqpoint{1.393917in}{0.876079in}}%
\pgfpathlineto{\pgfqpoint{1.407286in}{0.885603in}}%
\pgfpathlineto{\pgfqpoint{1.413635in}{0.889691in}}%
\pgfpathlineto{\pgfqpoint{1.422943in}{0.895976in}}%
\pgfpathlineto{\pgfqpoint{1.435216in}{0.903302in}}%
\pgfpathlineto{\pgfqpoint{1.438599in}{0.905473in}}%
\pgfpathlineto{\pgfqpoint{1.454256in}{0.914163in}}%
\pgfpathlineto{\pgfqpoint{1.460204in}{0.916913in}}%
\pgfpathlineto{\pgfqpoint{1.469913in}{0.921894in}}%
\pgfpathlineto{\pgfqpoint{1.485569in}{0.928265in}}%
\pgfpathlineto{\pgfqpoint{1.493117in}{0.930524in}}%
\pgfpathlineto{\pgfqpoint{1.501226in}{0.933327in}}%
\pgfpathlineto{\pgfqpoint{1.516882in}{0.936783in}}%
\pgfpathlineto{\pgfqpoint{1.532539in}{0.938263in}}%
\pgfpathlineto{\pgfqpoint{1.548195in}{0.937769in}}%
\pgfpathlineto{\pgfqpoint{1.563852in}{0.935302in}}%
\pgfpathlineto{\pgfqpoint{1.579508in}{0.930858in}}%
\pgfpathlineto{\pgfqpoint{1.580330in}{0.930524in}}%
\pgfpathlineto{\pgfqpoint{1.595165in}{0.925293in}}%
\pgfpathlineto{\pgfqpoint{1.610822in}{0.918070in}}%
\pgfpathlineto{\pgfqpoint{1.612877in}{0.916913in}}%
\pgfpathlineto{\pgfqpoint{1.626478in}{0.910007in}}%
\pgfpathlineto{\pgfqpoint{1.637649in}{0.903302in}}%
\pgfpathlineto{\pgfqpoint{1.642135in}{0.900801in}}%
\pgfpathlineto{\pgfqpoint{1.657791in}{0.890812in}}%
\pgfpathlineto{\pgfqpoint{1.659372in}{0.889691in}}%
\pgfpathlineto{\pgfqpoint{1.673448in}{0.880159in}}%
\pgfpathlineto{\pgfqpoint{1.678950in}{0.876079in}}%
\pgfpathlineto{\pgfqpoint{1.689104in}{0.868738in}}%
\pgfpathlineto{\pgfqpoint{1.697172in}{0.862468in}}%
\pgfpathlineto{\pgfqpoint{1.704761in}{0.856604in}}%
\pgfpathlineto{\pgfqpoint{1.714250in}{0.848857in}}%
\pgfpathlineto{\pgfqpoint{1.720418in}{0.843760in}}%
\pgfpathlineto{\pgfqpoint{1.730336in}{0.835246in}}%
\pgfpathlineto{\pgfqpoint{1.736074in}{0.830173in}}%
\pgfpathlineto{\pgfqpoint{1.745530in}{0.821635in}}%
\pgfpathlineto{\pgfqpoint{1.751731in}{0.815770in}}%
\pgfpathlineto{\pgfqpoint{1.759893in}{0.808024in}}%
\pgfpathlineto{\pgfqpoint{1.767387in}{0.800445in}}%
\pgfpathlineto{\pgfqpoint{1.773444in}{0.794413in}}%
\pgfpathlineto{\pgfqpoint{1.783044in}{0.784043in}}%
\pgfpathlineto{\pgfqpoint{1.786152in}{0.780802in}}%
\pgfpathlineto{\pgfqpoint{1.797974in}{0.767191in}}%
\pgfpathlineto{\pgfqpoint{1.798700in}{0.766249in}}%
\pgfpathlineto{\pgfqpoint{1.809085in}{0.753579in}}%
\pgfpathlineto{\pgfqpoint{1.814357in}{0.746099in}}%
\pgfpathlineto{\pgfqpoint{1.819073in}{0.739968in}}%
\pgfpathlineto{\pgfqpoint{1.827797in}{0.726357in}}%
\pgfpathlineto{\pgfqpoint{1.830014in}{0.722006in}}%
\pgfpathlineto{\pgfqpoint{1.835348in}{0.712746in}}%
\pgfpathlineto{\pgfqpoint{1.841113in}{0.699135in}}%
\pgfpathlineto{\pgfqpoint{1.844778in}{0.685524in}}%
\pgfpathlineto{\pgfqpoint{1.845670in}{0.677797in}}%
\pgfpathlineto{\pgfqpoint{1.846476in}{0.671913in}}%
\pgfpathlineto{\pgfqpoint{1.845854in}{0.658302in}}%
\pgfpathlineto{\pgfqpoint{1.845670in}{0.657495in}}%
\pgfpathlineto{\pgfqpoint{1.843207in}{0.644691in}}%
\pgfpathlineto{\pgfqpoint{1.838493in}{0.631079in}}%
\pgfpathlineto{\pgfqpoint{1.831677in}{0.617468in}}%
\pgfpathlineto{\pgfqpoint{1.830014in}{0.614905in}}%
\pgfpathlineto{\pgfqpoint{1.823666in}{0.603857in}}%
\pgfpathlineto{\pgfqpoint{1.814357in}{0.590714in}}%
\pgfpathlineto{\pgfqpoint{1.814054in}{0.590246in}}%
\pgfpathlineto{\pgfqpoint{1.803705in}{0.576635in}}%
\pgfpathlineto{\pgfqpoint{1.798700in}{0.570888in}}%
\pgfpathlineto{\pgfqpoint{1.792249in}{0.563024in}}%
\pgfpathlineto{\pgfqpoint{1.783044in}{0.552990in}}%
\pgfpathlineto{\pgfqpoint{1.779880in}{0.549413in}}%
\pgfpathlineto{\pgfqpoint{1.767387in}{0.536518in}}%
\pgfpathlineto{\pgfqpoint{1.766704in}{0.535802in}}%
\pgfpathlineto{\pgfqpoint{1.752782in}{0.522191in}}%
\pgfpathlineto{\pgfqpoint{1.751731in}{0.521216in}}%
\pgfpathlineto{\pgfqpoint{1.738037in}{0.508579in}}%
\pgfpathlineto{\pgfqpoint{1.736074in}{0.506836in}}%
\pgfpathlineto{\pgfqpoint{1.722423in}{0.494968in}}%
\pgfpathlineto{\pgfqpoint{1.720418in}{0.493262in}}%
\pgfpathlineto{\pgfqpoint{1.705882in}{0.481357in}}%
\pgfpathlineto{\pgfqpoint{1.704761in}{0.480443in}}%
\pgfpathlineto{\pgfqpoint{1.689104in}{0.468340in}}%
\pgfpathlineto{\pgfqpoint{1.688281in}{0.467746in}}%
\pgfpathlineto{\pgfqpoint{1.673448in}{0.456886in}}%
\pgfpathlineto{\pgfqpoint{1.669333in}{0.454135in}}%
\pgfpathlineto{\pgfqpoint{1.657791in}{0.446132in}}%
\pgfpathlineto{\pgfqpoint{1.648746in}{0.440524in}}%
\pgfpathlineto{\pgfqpoint{1.642135in}{0.436173in}}%
\pgfpathlineto{\pgfqpoint{1.626478in}{0.427176in}}%
\pgfpathlineto{\pgfqpoint{1.625940in}{0.426913in}}%
\pgfpathlineto{\pgfqpoint{1.610822in}{0.418820in}}%
\pgfpathlineto{\pgfqpoint{1.598114in}{0.413302in}}%
\pgfpathlineto{\pgfqpoint{1.595165in}{0.411855in}}%
\pgfpathlineto{\pgfqpoint{1.579508in}{0.405930in}}%
\pgfpathlineto{\pgfqpoint{1.563852in}{0.401831in}}%
\pgfpathlineto{\pgfqpoint{1.549123in}{0.399691in}}%
\pgfpathlineto{\pgfqpoint{1.548195in}{0.399531in}}%
\pgfpathlineto{\pgfqpoint{1.532539in}{0.398990in}}%
\pgfpathlineto{\pgfqpoint{1.525770in}{0.399691in}}%
\pgfpathclose%
\pgfpathmoveto{\pgfqpoint{1.105155in}{0.794413in}}%
\pgfpathlineto{\pgfqpoint{1.094155in}{0.799491in}}%
\pgfpathlineto{\pgfqpoint{1.079672in}{0.808024in}}%
\pgfpathlineto{\pgfqpoint{1.078498in}{0.808681in}}%
\pgfpathlineto{\pgfqpoint{1.062842in}{0.819141in}}%
\pgfpathlineto{\pgfqpoint{1.059554in}{0.821635in}}%
\pgfpathlineto{\pgfqpoint{1.047185in}{0.830642in}}%
\pgfpathlineto{\pgfqpoint{1.041409in}{0.835246in}}%
\pgfpathlineto{\pgfqpoint{1.031529in}{0.842914in}}%
\pgfpathlineto{\pgfqpoint{1.024334in}{0.848857in}}%
\pgfpathlineto{\pgfqpoint{1.015872in}{0.855762in}}%
\pgfpathlineto{\pgfqpoint{1.007975in}{0.862468in}}%
\pgfpathlineto{\pgfqpoint{1.000216in}{0.869078in}}%
\pgfpathlineto{\pgfqpoint{0.992162in}{0.876079in}}%
\pgfpathlineto{\pgfqpoint{0.984559in}{0.882825in}}%
\pgfpathlineto{\pgfqpoint{0.976845in}{0.889691in}}%
\pgfpathlineto{\pgfqpoint{0.968902in}{0.897047in}}%
\pgfpathlineto{\pgfqpoint{0.962066in}{0.903302in}}%
\pgfpathlineto{\pgfqpoint{0.953246in}{0.911891in}}%
\pgfpathlineto{\pgfqpoint{0.947950in}{0.916913in}}%
\pgfpathlineto{\pgfqpoint{0.937589in}{0.927665in}}%
\pgfpathlineto{\pgfqpoint{0.934720in}{0.930524in}}%
\pgfpathlineto{\pgfqpoint{0.922688in}{0.944135in}}%
\pgfpathlineto{\pgfqpoint{0.921933in}{0.945155in}}%
\pgfpathlineto{\pgfqpoint{0.912117in}{0.957746in}}%
\pgfpathlineto{\pgfqpoint{0.906276in}{0.967309in}}%
\pgfpathlineto{\pgfqpoint{0.903635in}{0.971357in}}%
\pgfpathlineto{\pgfqpoint{0.897558in}{0.984968in}}%
\pgfpathlineto{\pgfqpoint{0.894454in}{0.998579in}}%
\pgfpathlineto{\pgfqpoint{0.894454in}{1.012191in}}%
\pgfpathlineto{\pgfqpoint{0.897558in}{1.025802in}}%
\pgfpathlineto{\pgfqpoint{0.903635in}{1.039413in}}%
\pgfpathlineto{\pgfqpoint{0.906276in}{1.043461in}}%
\pgfpathlineto{\pgfqpoint{0.912117in}{1.053024in}}%
\pgfpathlineto{\pgfqpoint{0.921933in}{1.065615in}}%
\pgfpathlineto{\pgfqpoint{0.922688in}{1.066635in}}%
\pgfpathlineto{\pgfqpoint{0.934720in}{1.080246in}}%
\pgfpathlineto{\pgfqpoint{0.937589in}{1.083105in}}%
\pgfpathlineto{\pgfqpoint{0.947950in}{1.093857in}}%
\pgfpathlineto{\pgfqpoint{0.953246in}{1.098879in}}%
\pgfpathlineto{\pgfqpoint{0.962066in}{1.107468in}}%
\pgfpathlineto{\pgfqpoint{0.968902in}{1.113723in}}%
\pgfpathlineto{\pgfqpoint{0.976845in}{1.121079in}}%
\pgfpathlineto{\pgfqpoint{0.984559in}{1.127945in}}%
\pgfpathlineto{\pgfqpoint{0.992162in}{1.134691in}}%
\pgfpathlineto{\pgfqpoint{1.000216in}{1.141692in}}%
\pgfpathlineto{\pgfqpoint{1.007975in}{1.148302in}}%
\pgfpathlineto{\pgfqpoint{1.015872in}{1.155008in}}%
\pgfpathlineto{\pgfqpoint{1.024334in}{1.161913in}}%
\pgfpathlineto{\pgfqpoint{1.031529in}{1.167856in}}%
\pgfpathlineto{\pgfqpoint{1.041409in}{1.175524in}}%
\pgfpathlineto{\pgfqpoint{1.047185in}{1.180128in}}%
\pgfpathlineto{\pgfqpoint{1.059554in}{1.189135in}}%
\pgfpathlineto{\pgfqpoint{1.062842in}{1.191629in}}%
\pgfpathlineto{\pgfqpoint{1.078498in}{1.202089in}}%
\pgfpathlineto{\pgfqpoint{1.079672in}{1.202746in}}%
\pgfpathlineto{\pgfqpoint{1.094155in}{1.211279in}}%
\pgfpathlineto{\pgfqpoint{1.105155in}{1.216357in}}%
\pgfpathlineto{\pgfqpoint{1.109812in}{1.218653in}}%
\pgfpathlineto{\pgfqpoint{1.125468in}{1.223936in}}%
\pgfpathlineto{\pgfqpoint{1.141125in}{1.226635in}}%
\pgfpathlineto{\pgfqpoint{1.156781in}{1.226635in}}%
\pgfpathlineto{\pgfqpoint{1.172438in}{1.223936in}}%
\pgfpathlineto{\pgfqpoint{1.188094in}{1.218653in}}%
\pgfpathlineto{\pgfqpoint{1.192751in}{1.216357in}}%
\pgfpathlineto{\pgfqpoint{1.203751in}{1.211279in}}%
\pgfpathlineto{\pgfqpoint{1.218234in}{1.202746in}}%
\pgfpathlineto{\pgfqpoint{1.219407in}{1.202089in}}%
\pgfpathlineto{\pgfqpoint{1.235064in}{1.191629in}}%
\pgfpathlineto{\pgfqpoint{1.238352in}{1.189135in}}%
\pgfpathlineto{\pgfqpoint{1.250721in}{1.180128in}}%
\pgfpathlineto{\pgfqpoint{1.256496in}{1.175524in}}%
\pgfpathlineto{\pgfqpoint{1.266377in}{1.167856in}}%
\pgfpathlineto{\pgfqpoint{1.273572in}{1.161913in}}%
\pgfpathlineto{\pgfqpoint{1.282034in}{1.155008in}}%
\pgfpathlineto{\pgfqpoint{1.289931in}{1.148302in}}%
\pgfpathlineto{\pgfqpoint{1.297690in}{1.141692in}}%
\pgfpathlineto{\pgfqpoint{1.305744in}{1.134691in}}%
\pgfpathlineto{\pgfqpoint{1.313347in}{1.127945in}}%
\pgfpathlineto{\pgfqpoint{1.321061in}{1.121079in}}%
\pgfpathlineto{\pgfqpoint{1.329003in}{1.113723in}}%
\pgfpathlineto{\pgfqpoint{1.335840in}{1.107468in}}%
\pgfpathlineto{\pgfqpoint{1.344660in}{1.098879in}}%
\pgfpathlineto{\pgfqpoint{1.349956in}{1.093857in}}%
\pgfpathlineto{\pgfqpoint{1.360317in}{1.083105in}}%
\pgfpathlineto{\pgfqpoint{1.363186in}{1.080246in}}%
\pgfpathlineto{\pgfqpoint{1.375218in}{1.066635in}}%
\pgfpathlineto{\pgfqpoint{1.375973in}{1.065615in}}%
\pgfpathlineto{\pgfqpoint{1.385789in}{1.053024in}}%
\pgfpathlineto{\pgfqpoint{1.391630in}{1.043461in}}%
\pgfpathlineto{\pgfqpoint{1.394271in}{1.039413in}}%
\pgfpathlineto{\pgfqpoint{1.400348in}{1.025802in}}%
\pgfpathlineto{\pgfqpoint{1.403452in}{1.012191in}}%
\pgfpathlineto{\pgfqpoint{1.403452in}{0.998579in}}%
\pgfpathlineto{\pgfqpoint{1.400348in}{0.984968in}}%
\pgfpathlineto{\pgfqpoint{1.394271in}{0.971357in}}%
\pgfpathlineto{\pgfqpoint{1.391630in}{0.967309in}}%
\pgfpathlineto{\pgfqpoint{1.385789in}{0.957746in}}%
\pgfpathlineto{\pgfqpoint{1.375973in}{0.945155in}}%
\pgfpathlineto{\pgfqpoint{1.375218in}{0.944135in}}%
\pgfpathlineto{\pgfqpoint{1.363186in}{0.930524in}}%
\pgfpathlineto{\pgfqpoint{1.360317in}{0.927665in}}%
\pgfpathlineto{\pgfqpoint{1.349956in}{0.916913in}}%
\pgfpathlineto{\pgfqpoint{1.344660in}{0.911891in}}%
\pgfpathlineto{\pgfqpoint{1.335840in}{0.903302in}}%
\pgfpathlineto{\pgfqpoint{1.329003in}{0.897047in}}%
\pgfpathlineto{\pgfqpoint{1.321061in}{0.889691in}}%
\pgfpathlineto{\pgfqpoint{1.313347in}{0.882825in}}%
\pgfpathlineto{\pgfqpoint{1.305744in}{0.876079in}}%
\pgfpathlineto{\pgfqpoint{1.297690in}{0.869078in}}%
\pgfpathlineto{\pgfqpoint{1.289931in}{0.862468in}}%
\pgfpathlineto{\pgfqpoint{1.282034in}{0.855762in}}%
\pgfpathlineto{\pgfqpoint{1.273572in}{0.848857in}}%
\pgfpathlineto{\pgfqpoint{1.266377in}{0.842914in}}%
\pgfpathlineto{\pgfqpoint{1.256496in}{0.835246in}}%
\pgfpathlineto{\pgfqpoint{1.250721in}{0.830642in}}%
\pgfpathlineto{\pgfqpoint{1.238352in}{0.821635in}}%
\pgfpathlineto{\pgfqpoint{1.235064in}{0.819141in}}%
\pgfpathlineto{\pgfqpoint{1.219407in}{0.808681in}}%
\pgfpathlineto{\pgfqpoint{1.218234in}{0.808024in}}%
\pgfpathlineto{\pgfqpoint{1.203751in}{0.799491in}}%
\pgfpathlineto{\pgfqpoint{1.192751in}{0.794413in}}%
\pgfpathlineto{\pgfqpoint{1.188094in}{0.792117in}}%
\pgfpathlineto{\pgfqpoint{1.172438in}{0.786834in}}%
\pgfpathlineto{\pgfqpoint{1.156781in}{0.784135in}}%
\pgfpathlineto{\pgfqpoint{1.141125in}{0.784135in}}%
\pgfpathlineto{\pgfqpoint{1.125468in}{0.786834in}}%
\pgfpathlineto{\pgfqpoint{1.109812in}{0.792117in}}%
\pgfpathlineto{\pgfqpoint{1.105155in}{0.794413in}}%
\pgfpathclose%
\pgfpathmoveto{\pgfqpoint{0.717576in}{1.080246in}}%
\pgfpathlineto{\pgfqpoint{0.702741in}{1.085477in}}%
\pgfpathlineto{\pgfqpoint{0.687084in}{1.092700in}}%
\pgfpathlineto{\pgfqpoint{0.685029in}{1.093857in}}%
\pgfpathlineto{\pgfqpoint{0.671428in}{1.100763in}}%
\pgfpathlineto{\pgfqpoint{0.660256in}{1.107468in}}%
\pgfpathlineto{\pgfqpoint{0.655771in}{1.109969in}}%
\pgfpathlineto{\pgfqpoint{0.640115in}{1.119958in}}%
\pgfpathlineto{\pgfqpoint{0.638534in}{1.121079in}}%
\pgfpathlineto{\pgfqpoint{0.624458in}{1.130611in}}%
\pgfpathlineto{\pgfqpoint{0.618955in}{1.134691in}}%
\pgfpathlineto{\pgfqpoint{0.608801in}{1.142032in}}%
\pgfpathlineto{\pgfqpoint{0.600734in}{1.148302in}}%
\pgfpathlineto{\pgfqpoint{0.593145in}{1.154166in}}%
\pgfpathlineto{\pgfqpoint{0.583655in}{1.161913in}}%
\pgfpathlineto{\pgfqpoint{0.577488in}{1.167010in}}%
\pgfpathlineto{\pgfqpoint{0.567570in}{1.175524in}}%
\pgfpathlineto{\pgfqpoint{0.561832in}{1.180597in}}%
\pgfpathlineto{\pgfqpoint{0.552376in}{1.189135in}}%
\pgfpathlineto{\pgfqpoint{0.546175in}{1.195000in}}%
\pgfpathlineto{\pgfqpoint{0.538013in}{1.202746in}}%
\pgfpathlineto{\pgfqpoint{0.530519in}{1.210325in}}%
\pgfpathlineto{\pgfqpoint{0.524461in}{1.216357in}}%
\pgfpathlineto{\pgfqpoint{0.514862in}{1.226727in}}%
\pgfpathlineto{\pgfqpoint{0.511753in}{1.229968in}}%
\pgfpathlineto{\pgfqpoint{0.499932in}{1.243579in}}%
\pgfpathlineto{\pgfqpoint{0.499205in}{1.244521in}}%
\pgfpathlineto{\pgfqpoint{0.488820in}{1.257191in}}%
\pgfpathlineto{\pgfqpoint{0.483549in}{1.264671in}}%
\pgfpathlineto{\pgfqpoint{0.478833in}{1.270802in}}%
\pgfpathlineto{\pgfqpoint{0.470108in}{1.284413in}}%
\pgfpathlineto{\pgfqpoint{0.467892in}{1.288764in}}%
\pgfpathlineto{\pgfqpoint{0.462558in}{1.298024in}}%
\pgfpathlineto{\pgfqpoint{0.456793in}{1.311635in}}%
\pgfpathlineto{\pgfqpoint{0.453128in}{1.325246in}}%
\pgfpathlineto{\pgfqpoint{0.452236in}{1.332973in}}%
\pgfpathlineto{\pgfqpoint{0.451429in}{1.338857in}}%
\pgfpathlineto{\pgfqpoint{0.452052in}{1.352468in}}%
\pgfpathlineto{\pgfqpoint{0.452236in}{1.353275in}}%
\pgfpathlineto{\pgfqpoint{0.454698in}{1.366079in}}%
\pgfpathlineto{\pgfqpoint{0.459413in}{1.379691in}}%
\pgfpathlineto{\pgfqpoint{0.466229in}{1.393302in}}%
\pgfpathlineto{\pgfqpoint{0.467892in}{1.395865in}}%
\pgfpathlineto{\pgfqpoint{0.474240in}{1.406913in}}%
\pgfpathlineto{\pgfqpoint{0.483549in}{1.420056in}}%
\pgfpathlineto{\pgfqpoint{0.483852in}{1.420524in}}%
\pgfpathlineto{\pgfqpoint{0.494201in}{1.434135in}}%
\pgfpathlineto{\pgfqpoint{0.499205in}{1.439882in}}%
\pgfpathlineto{\pgfqpoint{0.505657in}{1.447746in}}%
\pgfpathlineto{\pgfqpoint{0.514862in}{1.457780in}}%
\pgfpathlineto{\pgfqpoint{0.518026in}{1.461357in}}%
\pgfpathlineto{\pgfqpoint{0.530519in}{1.474252in}}%
\pgfpathlineto{\pgfqpoint{0.531201in}{1.474968in}}%
\pgfpathlineto{\pgfqpoint{0.545124in}{1.488579in}}%
\pgfpathlineto{\pgfqpoint{0.546175in}{1.489554in}}%
\pgfpathlineto{\pgfqpoint{0.559869in}{1.502191in}}%
\pgfpathlineto{\pgfqpoint{0.561832in}{1.503934in}}%
\pgfpathlineto{\pgfqpoint{0.575483in}{1.515802in}}%
\pgfpathlineto{\pgfqpoint{0.577488in}{1.517508in}}%
\pgfpathlineto{\pgfqpoint{0.592024in}{1.529413in}}%
\pgfpathlineto{\pgfqpoint{0.593145in}{1.530327in}}%
\pgfpathlineto{\pgfqpoint{0.608801in}{1.542430in}}%
\pgfpathlineto{\pgfqpoint{0.609625in}{1.543024in}}%
\pgfpathlineto{\pgfqpoint{0.624458in}{1.553884in}}%
\pgfpathlineto{\pgfqpoint{0.628573in}{1.556635in}}%
\pgfpathlineto{\pgfqpoint{0.640115in}{1.564638in}}%
\pgfpathlineto{\pgfqpoint{0.649160in}{1.570246in}}%
\pgfpathlineto{\pgfqpoint{0.655771in}{1.574597in}}%
\pgfpathlineto{\pgfqpoint{0.671428in}{1.583594in}}%
\pgfpathlineto{\pgfqpoint{0.671966in}{1.583857in}}%
\pgfpathlineto{\pgfqpoint{0.687084in}{1.591950in}}%
\pgfpathlineto{\pgfqpoint{0.699792in}{1.597468in}}%
\pgfpathlineto{\pgfqpoint{0.702741in}{1.598915in}}%
\pgfpathlineto{\pgfqpoint{0.718397in}{1.604840in}}%
\pgfpathlineto{\pgfqpoint{0.734054in}{1.608939in}}%
\pgfpathlineto{\pgfqpoint{0.748782in}{1.611079in}}%
\pgfpathlineto{\pgfqpoint{0.749710in}{1.611239in}}%
\pgfpathlineto{\pgfqpoint{0.765367in}{1.611780in}}%
\pgfpathlineto{\pgfqpoint{0.772136in}{1.611079in}}%
\pgfpathlineto{\pgfqpoint{0.781024in}{1.610304in}}%
\pgfpathlineto{\pgfqpoint{0.796680in}{1.607117in}}%
\pgfpathlineto{\pgfqpoint{0.812337in}{1.602106in}}%
\pgfpathlineto{\pgfqpoint{0.822989in}{1.597468in}}%
\pgfpathlineto{\pgfqpoint{0.827993in}{1.595542in}}%
\pgfpathlineto{\pgfqpoint{0.843650in}{1.587957in}}%
\pgfpathlineto{\pgfqpoint{0.850702in}{1.583857in}}%
\pgfpathlineto{\pgfqpoint{0.859306in}{1.579274in}}%
\pgfpathlineto{\pgfqpoint{0.873880in}{1.570246in}}%
\pgfpathlineto{\pgfqpoint{0.874963in}{1.569615in}}%
\pgfpathlineto{\pgfqpoint{0.890620in}{1.559338in}}%
\pgfpathlineto{\pgfqpoint{0.894348in}{1.556635in}}%
\pgfpathlineto{\pgfqpoint{0.906276in}{1.548290in}}%
\pgfpathlineto{\pgfqpoint{0.913215in}{1.543024in}}%
\pgfpathlineto{\pgfqpoint{0.921933in}{1.536509in}}%
\pgfpathlineto{\pgfqpoint{0.930843in}{1.529413in}}%
\pgfpathlineto{\pgfqpoint{0.937589in}{1.524022in}}%
\pgfpathlineto{\pgfqpoint{0.947410in}{1.515802in}}%
\pgfpathlineto{\pgfqpoint{0.953246in}{1.510813in}}%
\pgfpathlineto{\pgfqpoint{0.963040in}{1.502191in}}%
\pgfpathlineto{\pgfqpoint{0.968902in}{1.496829in}}%
\pgfpathlineto{\pgfqpoint{0.977813in}{1.488579in}}%
\pgfpathlineto{\pgfqpoint{0.984559in}{1.481982in}}%
\pgfpathlineto{\pgfqpoint{0.991771in}{1.474968in}}%
\pgfpathlineto{\pgfqpoint{1.000216in}{1.466141in}}%
\pgfpathlineto{\pgfqpoint{1.004908in}{1.461357in}}%
\pgfpathlineto{\pgfqpoint{1.015872in}{1.449120in}}%
\pgfpathlineto{\pgfqpoint{1.017162in}{1.447746in}}%
\pgfpathlineto{\pgfqpoint{1.028652in}{1.434135in}}%
\pgfpathlineto{\pgfqpoint{1.031529in}{1.430236in}}%
\pgfpathlineto{\pgfqpoint{1.039241in}{1.420524in}}%
\pgfpathlineto{\pgfqpoint{1.047185in}{1.408700in}}%
\pgfpathlineto{\pgfqpoint{1.048516in}{1.406913in}}%
\pgfpathlineto{\pgfqpoint{1.056824in}{1.393302in}}%
\pgfpathlineto{\pgfqpoint{1.062842in}{1.380405in}}%
\pgfpathlineto{\pgfqpoint{1.063226in}{1.379691in}}%
\pgfpathlineto{\pgfqpoint{1.068338in}{1.366079in}}%
\pgfpathlineto{\pgfqpoint{1.071176in}{1.352468in}}%
\pgfpathlineto{\pgfqpoint{1.071743in}{1.338857in}}%
\pgfpathlineto{\pgfqpoint{1.070041in}{1.325246in}}%
\pgfpathlineto{\pgfqpoint{1.066067in}{1.311635in}}%
\pgfpathlineto{\pgfqpoint{1.062842in}{1.304586in}}%
\pgfpathlineto{\pgfqpoint{1.060243in}{1.298024in}}%
\pgfpathlineto{\pgfqpoint{1.052916in}{1.284413in}}%
\pgfpathlineto{\pgfqpoint{1.047185in}{1.275973in}}%
\pgfpathlineto{\pgfqpoint{1.044022in}{1.270802in}}%
\pgfpathlineto{\pgfqpoint{1.034026in}{1.257191in}}%
\pgfpathlineto{\pgfqpoint{1.031529in}{1.254249in}}%
\pgfpathlineto{\pgfqpoint{1.023102in}{1.243579in}}%
\pgfpathlineto{\pgfqpoint{1.015872in}{1.235488in}}%
\pgfpathlineto{\pgfqpoint{1.011170in}{1.229968in}}%
\pgfpathlineto{\pgfqpoint{1.000216in}{1.218346in}}%
\pgfpathlineto{\pgfqpoint{0.998389in}{1.216357in}}%
\pgfpathlineto{\pgfqpoint{0.984836in}{1.202746in}}%
\pgfpathlineto{\pgfqpoint{0.984559in}{1.202484in}}%
\pgfpathlineto{\pgfqpoint{0.970516in}{1.189135in}}%
\pgfpathlineto{\pgfqpoint{0.968902in}{1.187671in}}%
\pgfpathlineto{\pgfqpoint{0.955341in}{1.175524in}}%
\pgfpathlineto{\pgfqpoint{0.953246in}{1.173702in}}%
\pgfpathlineto{\pgfqpoint{0.939273in}{1.161913in}}%
\pgfpathlineto{\pgfqpoint{0.937589in}{1.160510in}}%
\pgfpathlineto{\pgfqpoint{0.922234in}{1.148302in}}%
\pgfpathlineto{\pgfqpoint{0.921933in}{1.148061in}}%
\pgfpathlineto{\pgfqpoint{0.906276in}{1.136278in}}%
\pgfpathlineto{\pgfqpoint{0.903988in}{1.134691in}}%
\pgfpathlineto{\pgfqpoint{0.890620in}{1.125167in}}%
\pgfpathlineto{\pgfqpoint{0.884270in}{1.121079in}}%
\pgfpathlineto{\pgfqpoint{0.874963in}{1.114794in}}%
\pgfpathlineto{\pgfqpoint{0.862690in}{1.107468in}}%
\pgfpathlineto{\pgfqpoint{0.859306in}{1.105297in}}%
\pgfpathlineto{\pgfqpoint{0.843650in}{1.096607in}}%
\pgfpathlineto{\pgfqpoint{0.837702in}{1.093857in}}%
\pgfpathlineto{\pgfqpoint{0.827993in}{1.088876in}}%
\pgfpathlineto{\pgfqpoint{0.812337in}{1.082505in}}%
\pgfpathlineto{\pgfqpoint{0.804789in}{1.080246in}}%
\pgfpathlineto{\pgfqpoint{0.796680in}{1.077443in}}%
\pgfpathlineto{\pgfqpoint{0.781024in}{1.073987in}}%
\pgfpathlineto{\pgfqpoint{0.765367in}{1.072507in}}%
\pgfpathlineto{\pgfqpoint{0.749710in}{1.073001in}}%
\pgfpathlineto{\pgfqpoint{0.734054in}{1.075468in}}%
\pgfpathlineto{\pgfqpoint{0.718397in}{1.079912in}}%
\pgfpathlineto{\pgfqpoint{0.717576in}{1.080246in}}%
\pgfpathclose%
\pgfpathmoveto{\pgfqpoint{1.493117in}{1.080246in}}%
\pgfpathlineto{\pgfqpoint{1.485569in}{1.082505in}}%
\pgfpathlineto{\pgfqpoint{1.469913in}{1.088876in}}%
\pgfpathlineto{\pgfqpoint{1.460204in}{1.093857in}}%
\pgfpathlineto{\pgfqpoint{1.454256in}{1.096607in}}%
\pgfpathlineto{\pgfqpoint{1.438599in}{1.105297in}}%
\pgfpathlineto{\pgfqpoint{1.435216in}{1.107468in}}%
\pgfpathlineto{\pgfqpoint{1.422943in}{1.114794in}}%
\pgfpathlineto{\pgfqpoint{1.413635in}{1.121079in}}%
\pgfpathlineto{\pgfqpoint{1.407286in}{1.125167in}}%
\pgfpathlineto{\pgfqpoint{1.393917in}{1.134691in}}%
\pgfpathlineto{\pgfqpoint{1.391630in}{1.136278in}}%
\pgfpathlineto{\pgfqpoint{1.375973in}{1.148061in}}%
\pgfpathlineto{\pgfqpoint{1.375671in}{1.148302in}}%
\pgfpathlineto{\pgfqpoint{1.360317in}{1.160510in}}%
\pgfpathlineto{\pgfqpoint{1.358632in}{1.161913in}}%
\pgfpathlineto{\pgfqpoint{1.344660in}{1.173702in}}%
\pgfpathlineto{\pgfqpoint{1.342564in}{1.175524in}}%
\pgfpathlineto{\pgfqpoint{1.329003in}{1.187671in}}%
\pgfpathlineto{\pgfqpoint{1.327390in}{1.189135in}}%
\pgfpathlineto{\pgfqpoint{1.313347in}{1.202484in}}%
\pgfpathlineto{\pgfqpoint{1.313070in}{1.202746in}}%
\pgfpathlineto{\pgfqpoint{1.299517in}{1.216357in}}%
\pgfpathlineto{\pgfqpoint{1.297690in}{1.218346in}}%
\pgfpathlineto{\pgfqpoint{1.286736in}{1.229968in}}%
\pgfpathlineto{\pgfqpoint{1.282034in}{1.235488in}}%
\pgfpathlineto{\pgfqpoint{1.274804in}{1.243579in}}%
\pgfpathlineto{\pgfqpoint{1.266377in}{1.254249in}}%
\pgfpathlineto{\pgfqpoint{1.263880in}{1.257191in}}%
\pgfpathlineto{\pgfqpoint{1.253883in}{1.270802in}}%
\pgfpathlineto{\pgfqpoint{1.250721in}{1.275973in}}%
\pgfpathlineto{\pgfqpoint{1.244990in}{1.284413in}}%
\pgfpathlineto{\pgfqpoint{1.237663in}{1.298024in}}%
\pgfpathlineto{\pgfqpoint{1.235064in}{1.304586in}}%
\pgfpathlineto{\pgfqpoint{1.231839in}{1.311635in}}%
\pgfpathlineto{\pgfqpoint{1.227865in}{1.325246in}}%
\pgfpathlineto{\pgfqpoint{1.226162in}{1.338857in}}%
\pgfpathlineto{\pgfqpoint{1.226730in}{1.352468in}}%
\pgfpathlineto{\pgfqpoint{1.229568in}{1.366079in}}%
\pgfpathlineto{\pgfqpoint{1.234680in}{1.379691in}}%
\pgfpathlineto{\pgfqpoint{1.235064in}{1.380405in}}%
\pgfpathlineto{\pgfqpoint{1.241081in}{1.393302in}}%
\pgfpathlineto{\pgfqpoint{1.249390in}{1.406913in}}%
\pgfpathlineto{\pgfqpoint{1.250721in}{1.408700in}}%
\pgfpathlineto{\pgfqpoint{1.258664in}{1.420524in}}%
\pgfpathlineto{\pgfqpoint{1.266377in}{1.430236in}}%
\pgfpathlineto{\pgfqpoint{1.269254in}{1.434135in}}%
\pgfpathlineto{\pgfqpoint{1.280744in}{1.447746in}}%
\pgfpathlineto{\pgfqpoint{1.282034in}{1.449120in}}%
\pgfpathlineto{\pgfqpoint{1.292998in}{1.461357in}}%
\pgfpathlineto{\pgfqpoint{1.297690in}{1.466141in}}%
\pgfpathlineto{\pgfqpoint{1.306135in}{1.474968in}}%
\pgfpathlineto{\pgfqpoint{1.313347in}{1.481982in}}%
\pgfpathlineto{\pgfqpoint{1.320093in}{1.488579in}}%
\pgfpathlineto{\pgfqpoint{1.329003in}{1.496829in}}%
\pgfpathlineto{\pgfqpoint{1.334866in}{1.502191in}}%
\pgfpathlineto{\pgfqpoint{1.344660in}{1.510813in}}%
\pgfpathlineto{\pgfqpoint{1.350496in}{1.515802in}}%
\pgfpathlineto{\pgfqpoint{1.360317in}{1.524022in}}%
\pgfpathlineto{\pgfqpoint{1.367063in}{1.529413in}}%
\pgfpathlineto{\pgfqpoint{1.375973in}{1.536509in}}%
\pgfpathlineto{\pgfqpoint{1.384690in}{1.543024in}}%
\pgfpathlineto{\pgfqpoint{1.391630in}{1.548290in}}%
\pgfpathlineto{\pgfqpoint{1.403558in}{1.556635in}}%
\pgfpathlineto{\pgfqpoint{1.407286in}{1.559338in}}%
\pgfpathlineto{\pgfqpoint{1.422943in}{1.569615in}}%
\pgfpathlineto{\pgfqpoint{1.424026in}{1.570246in}}%
\pgfpathlineto{\pgfqpoint{1.438599in}{1.579274in}}%
\pgfpathlineto{\pgfqpoint{1.447204in}{1.583857in}}%
\pgfpathlineto{\pgfqpoint{1.454256in}{1.587957in}}%
\pgfpathlineto{\pgfqpoint{1.469913in}{1.595542in}}%
\pgfpathlineto{\pgfqpoint{1.474917in}{1.597468in}}%
\pgfpathlineto{\pgfqpoint{1.485569in}{1.602106in}}%
\pgfpathlineto{\pgfqpoint{1.501226in}{1.607117in}}%
\pgfpathlineto{\pgfqpoint{1.516882in}{1.610304in}}%
\pgfpathlineto{\pgfqpoint{1.525770in}{1.611079in}}%
\pgfpathlineto{\pgfqpoint{1.532539in}{1.611780in}}%
\pgfpathlineto{\pgfqpoint{1.548195in}{1.611239in}}%
\pgfpathlineto{\pgfqpoint{1.549123in}{1.611079in}}%
\pgfpathlineto{\pgfqpoint{1.563852in}{1.608939in}}%
\pgfpathlineto{\pgfqpoint{1.579508in}{1.604840in}}%
\pgfpathlineto{\pgfqpoint{1.595165in}{1.598915in}}%
\pgfpathlineto{\pgfqpoint{1.598114in}{1.597468in}}%
\pgfpathlineto{\pgfqpoint{1.610822in}{1.591950in}}%
\pgfpathlineto{\pgfqpoint{1.625940in}{1.583857in}}%
\pgfpathlineto{\pgfqpoint{1.626478in}{1.583594in}}%
\pgfpathlineto{\pgfqpoint{1.642135in}{1.574597in}}%
\pgfpathlineto{\pgfqpoint{1.648746in}{1.570246in}}%
\pgfpathlineto{\pgfqpoint{1.657791in}{1.564638in}}%
\pgfpathlineto{\pgfqpoint{1.669333in}{1.556635in}}%
\pgfpathlineto{\pgfqpoint{1.673448in}{1.553884in}}%
\pgfpathlineto{\pgfqpoint{1.688281in}{1.543024in}}%
\pgfpathlineto{\pgfqpoint{1.689104in}{1.542430in}}%
\pgfpathlineto{\pgfqpoint{1.704761in}{1.530327in}}%
\pgfpathlineto{\pgfqpoint{1.705882in}{1.529413in}}%
\pgfpathlineto{\pgfqpoint{1.720418in}{1.517508in}}%
\pgfpathlineto{\pgfqpoint{1.722423in}{1.515802in}}%
\pgfpathlineto{\pgfqpoint{1.736074in}{1.503934in}}%
\pgfpathlineto{\pgfqpoint{1.738037in}{1.502191in}}%
\pgfpathlineto{\pgfqpoint{1.751731in}{1.489554in}}%
\pgfpathlineto{\pgfqpoint{1.752782in}{1.488579in}}%
\pgfpathlineto{\pgfqpoint{1.766704in}{1.474968in}}%
\pgfpathlineto{\pgfqpoint{1.767387in}{1.474252in}}%
\pgfpathlineto{\pgfqpoint{1.779880in}{1.461357in}}%
\pgfpathlineto{\pgfqpoint{1.783044in}{1.457780in}}%
\pgfpathlineto{\pgfqpoint{1.792249in}{1.447746in}}%
\pgfpathlineto{\pgfqpoint{1.798700in}{1.439882in}}%
\pgfpathlineto{\pgfqpoint{1.803705in}{1.434135in}}%
\pgfpathlineto{\pgfqpoint{1.814054in}{1.420524in}}%
\pgfpathlineto{\pgfqpoint{1.814357in}{1.420056in}}%
\pgfpathlineto{\pgfqpoint{1.823666in}{1.406913in}}%
\pgfpathlineto{\pgfqpoint{1.830014in}{1.395865in}}%
\pgfpathlineto{\pgfqpoint{1.831677in}{1.393302in}}%
\pgfpathlineto{\pgfqpoint{1.838493in}{1.379691in}}%
\pgfpathlineto{\pgfqpoint{1.843207in}{1.366079in}}%
\pgfpathlineto{\pgfqpoint{1.845670in}{1.353275in}}%
\pgfpathlineto{\pgfqpoint{1.845854in}{1.352468in}}%
\pgfpathlineto{\pgfqpoint{1.846476in}{1.338857in}}%
\pgfpathlineto{\pgfqpoint{1.845670in}{1.332973in}}%
\pgfpathlineto{\pgfqpoint{1.844778in}{1.325246in}}%
\pgfpathlineto{\pgfqpoint{1.841113in}{1.311635in}}%
\pgfpathlineto{\pgfqpoint{1.835348in}{1.298024in}}%
\pgfpathlineto{\pgfqpoint{1.830014in}{1.288764in}}%
\pgfpathlineto{\pgfqpoint{1.827797in}{1.284413in}}%
\pgfpathlineto{\pgfqpoint{1.819073in}{1.270802in}}%
\pgfpathlineto{\pgfqpoint{1.814357in}{1.264671in}}%
\pgfpathlineto{\pgfqpoint{1.809085in}{1.257191in}}%
\pgfpathlineto{\pgfqpoint{1.798700in}{1.244521in}}%
\pgfpathlineto{\pgfqpoint{1.797974in}{1.243579in}}%
\pgfpathlineto{\pgfqpoint{1.786152in}{1.229968in}}%
\pgfpathlineto{\pgfqpoint{1.783044in}{1.226727in}}%
\pgfpathlineto{\pgfqpoint{1.773444in}{1.216357in}}%
\pgfpathlineto{\pgfqpoint{1.767387in}{1.210325in}}%
\pgfpathlineto{\pgfqpoint{1.759893in}{1.202746in}}%
\pgfpathlineto{\pgfqpoint{1.751731in}{1.195000in}}%
\pgfpathlineto{\pgfqpoint{1.745530in}{1.189135in}}%
\pgfpathlineto{\pgfqpoint{1.736074in}{1.180597in}}%
\pgfpathlineto{\pgfqpoint{1.730336in}{1.175524in}}%
\pgfpathlineto{\pgfqpoint{1.720418in}{1.167010in}}%
\pgfpathlineto{\pgfqpoint{1.714250in}{1.161913in}}%
\pgfpathlineto{\pgfqpoint{1.704761in}{1.154166in}}%
\pgfpathlineto{\pgfqpoint{1.697172in}{1.148302in}}%
\pgfpathlineto{\pgfqpoint{1.689104in}{1.142032in}}%
\pgfpathlineto{\pgfqpoint{1.678950in}{1.134691in}}%
\pgfpathlineto{\pgfqpoint{1.673448in}{1.130611in}}%
\pgfpathlineto{\pgfqpoint{1.659372in}{1.121079in}}%
\pgfpathlineto{\pgfqpoint{1.657791in}{1.119958in}}%
\pgfpathlineto{\pgfqpoint{1.642135in}{1.109969in}}%
\pgfpathlineto{\pgfqpoint{1.637649in}{1.107468in}}%
\pgfpathlineto{\pgfqpoint{1.626478in}{1.100763in}}%
\pgfpathlineto{\pgfqpoint{1.612877in}{1.093857in}}%
\pgfpathlineto{\pgfqpoint{1.610822in}{1.092700in}}%
\pgfpathlineto{\pgfqpoint{1.595165in}{1.085477in}}%
\pgfpathlineto{\pgfqpoint{1.580330in}{1.080246in}}%
\pgfpathlineto{\pgfqpoint{1.579508in}{1.079912in}}%
\pgfpathlineto{\pgfqpoint{1.563852in}{1.075468in}}%
\pgfpathlineto{\pgfqpoint{1.548195in}{1.073001in}}%
\pgfpathlineto{\pgfqpoint{1.532539in}{1.072507in}}%
\pgfpathlineto{\pgfqpoint{1.516882in}{1.073987in}}%
\pgfpathlineto{\pgfqpoint{1.501226in}{1.077443in}}%
\pgfpathlineto{\pgfqpoint{1.493117in}{1.080246in}}%
\pgfpathclose%
\pgfusepath{fill}%
\end{pgfscope}%
\begin{pgfscope}%
\pgfpathrectangle{\pgfqpoint{0.373953in}{0.331635in}}{\pgfqpoint{1.550000in}{1.347500in}}%
\pgfusepath{clip}%
\pgfsetbuttcap%
\pgfsetroundjoin%
\definecolor{currentfill}{rgb}{0.252220,0.059415,0.453248}%
\pgfsetfillcolor{currentfill}%
\pgfsetlinewidth{0.000000pt}%
\definecolor{currentstroke}{rgb}{0.000000,0.000000,0.000000}%
\pgfsetstrokecolor{currentstroke}%
\pgfsetdash{}{0pt}%
\pgfpathmoveto{\pgfqpoint{0.530519in}{0.331635in}}%
\pgfpathlineto{\pgfqpoint{0.546175in}{0.331635in}}%
\pgfpathlineto{\pgfqpoint{0.561832in}{0.331635in}}%
\pgfpathlineto{\pgfqpoint{0.577488in}{0.331635in}}%
\pgfpathlineto{\pgfqpoint{0.593145in}{0.331635in}}%
\pgfpathlineto{\pgfqpoint{0.608801in}{0.331635in}}%
\pgfpathlineto{\pgfqpoint{0.624458in}{0.331635in}}%
\pgfpathlineto{\pgfqpoint{0.628868in}{0.331635in}}%
\pgfpathlineto{\pgfqpoint{0.627253in}{0.345246in}}%
\pgfpathlineto{\pgfqpoint{0.624458in}{0.353195in}}%
\pgfpathlineto{\pgfqpoint{0.622600in}{0.358857in}}%
\pgfpathlineto{\pgfqpoint{0.615336in}{0.372468in}}%
\pgfpathlineto{\pgfqpoint{0.608801in}{0.381546in}}%
\pgfpathlineto{\pgfqpoint{0.605723in}{0.386079in}}%
\pgfpathlineto{\pgfqpoint{0.594368in}{0.399691in}}%
\pgfpathlineto{\pgfqpoint{0.593145in}{0.400949in}}%
\pgfpathlineto{\pgfqpoint{0.581699in}{0.413302in}}%
\pgfpathlineto{\pgfqpoint{0.577488in}{0.417386in}}%
\pgfpathlineto{\pgfqpoint{0.567992in}{0.426913in}}%
\pgfpathlineto{\pgfqpoint{0.561832in}{0.432643in}}%
\pgfpathlineto{\pgfqpoint{0.553523in}{0.440524in}}%
\pgfpathlineto{\pgfqpoint{0.546175in}{0.447150in}}%
\pgfpathlineto{\pgfqpoint{0.538465in}{0.454135in}}%
\pgfpathlineto{\pgfqpoint{0.530519in}{0.461123in}}%
\pgfpathlineto{\pgfqpoint{0.522900in}{0.467746in}}%
\pgfpathlineto{\pgfqpoint{0.514862in}{0.474654in}}%
\pgfpathlineto{\pgfqpoint{0.506827in}{0.481357in}}%
\pgfpathlineto{\pgfqpoint{0.499205in}{0.487745in}}%
\pgfpathlineto{\pgfqpoint{0.490140in}{0.494968in}}%
\pgfpathlineto{\pgfqpoint{0.483549in}{0.500324in}}%
\pgfpathlineto{\pgfqpoint{0.472590in}{0.508579in}}%
\pgfpathlineto{\pgfqpoint{0.467892in}{0.512240in}}%
\pgfpathlineto{\pgfqpoint{0.453684in}{0.522191in}}%
\pgfpathlineto{\pgfqpoint{0.452236in}{0.523254in}}%
\pgfpathlineto{\pgfqpoint{0.436579in}{0.533126in}}%
\pgfpathlineto{\pgfqpoint{0.431364in}{0.535802in}}%
\pgfpathlineto{\pgfqpoint{0.420923in}{0.541482in}}%
\pgfpathlineto{\pgfqpoint{0.405266in}{0.547797in}}%
\pgfpathlineto{\pgfqpoint{0.398752in}{0.549413in}}%
\pgfpathlineto{\pgfqpoint{0.389609in}{0.551843in}}%
\pgfpathlineto{\pgfqpoint{0.373953in}{0.553246in}}%
\pgfpathlineto{\pgfqpoint{0.373953in}{0.549413in}}%
\pgfpathlineto{\pgfqpoint{0.373953in}{0.535802in}}%
\pgfpathlineto{\pgfqpoint{0.373953in}{0.522191in}}%
\pgfpathlineto{\pgfqpoint{0.373953in}{0.508579in}}%
\pgfpathlineto{\pgfqpoint{0.373953in}{0.494968in}}%
\pgfpathlineto{\pgfqpoint{0.373953in}{0.481357in}}%
\pgfpathlineto{\pgfqpoint{0.373953in}{0.467746in}}%
\pgfpathlineto{\pgfqpoint{0.373953in}{0.461560in}}%
\pgfpathlineto{\pgfqpoint{0.389609in}{0.460471in}}%
\pgfpathlineto{\pgfqpoint{0.405266in}{0.457249in}}%
\pgfpathlineto{\pgfqpoint{0.414570in}{0.454135in}}%
\pgfpathlineto{\pgfqpoint{0.420923in}{0.451905in}}%
\pgfpathlineto{\pgfqpoint{0.436579in}{0.444479in}}%
\pgfpathlineto{\pgfqpoint{0.443364in}{0.440524in}}%
\pgfpathlineto{\pgfqpoint{0.452236in}{0.435027in}}%
\pgfpathlineto{\pgfqpoint{0.463541in}{0.426913in}}%
\pgfpathlineto{\pgfqpoint{0.467892in}{0.423523in}}%
\pgfpathlineto{\pgfqpoint{0.479649in}{0.413302in}}%
\pgfpathlineto{\pgfqpoint{0.483549in}{0.409519in}}%
\pgfpathlineto{\pgfqpoint{0.492883in}{0.399691in}}%
\pgfpathlineto{\pgfqpoint{0.499205in}{0.391977in}}%
\pgfpathlineto{\pgfqpoint{0.503755in}{0.386079in}}%
\pgfpathlineto{\pgfqpoint{0.512297in}{0.372468in}}%
\pgfpathlineto{\pgfqpoint{0.514862in}{0.366946in}}%
\pgfpathlineto{\pgfqpoint{0.518444in}{0.358857in}}%
\pgfpathlineto{\pgfqpoint{0.522150in}{0.345246in}}%
\pgfpathlineto{\pgfqpoint{0.523403in}{0.331635in}}%
\pgfpathlineto{\pgfqpoint{0.530519in}{0.331635in}}%
\pgfpathclose%
\pgfpathmoveto{\pgfqpoint{0.906276in}{0.331635in}}%
\pgfpathlineto{\pgfqpoint{0.921933in}{0.331635in}}%
\pgfpathlineto{\pgfqpoint{0.937589in}{0.331635in}}%
\pgfpathlineto{\pgfqpoint{0.953246in}{0.331635in}}%
\pgfpathlineto{\pgfqpoint{0.968902in}{0.331635in}}%
\pgfpathlineto{\pgfqpoint{0.984559in}{0.331635in}}%
\pgfpathlineto{\pgfqpoint{0.999425in}{0.331635in}}%
\pgfpathlineto{\pgfqpoint{1.000216in}{0.340413in}}%
\pgfpathlineto{\pgfqpoint{1.000672in}{0.345246in}}%
\pgfpathlineto{\pgfqpoint{1.004478in}{0.358857in}}%
\pgfpathlineto{\pgfqpoint{1.010644in}{0.372468in}}%
\pgfpathlineto{\pgfqpoint{1.015872in}{0.381021in}}%
\pgfpathlineto{\pgfqpoint{1.019134in}{0.386079in}}%
\pgfpathlineto{\pgfqpoint{1.029915in}{0.399691in}}%
\pgfpathlineto{\pgfqpoint{1.031529in}{0.401440in}}%
\pgfpathlineto{\pgfqpoint{1.043250in}{0.413302in}}%
\pgfpathlineto{\pgfqpoint{1.047185in}{0.416875in}}%
\pgfpathlineto{\pgfqpoint{1.059327in}{0.426913in}}%
\pgfpathlineto{\pgfqpoint{1.062842in}{0.429593in}}%
\pgfpathlineto{\pgfqpoint{1.078498in}{0.440113in}}%
\pgfpathlineto{\pgfqpoint{1.079228in}{0.440524in}}%
\pgfpathlineto{\pgfqpoint{1.094155in}{0.448412in}}%
\pgfpathlineto{\pgfqpoint{1.107937in}{0.454135in}}%
\pgfpathlineto{\pgfqpoint{1.109812in}{0.454879in}}%
\pgfpathlineto{\pgfqpoint{1.125468in}{0.459120in}}%
\pgfpathlineto{\pgfqpoint{1.141125in}{0.461287in}}%
\pgfpathlineto{\pgfqpoint{1.156781in}{0.461287in}}%
\pgfpathlineto{\pgfqpoint{1.172438in}{0.459120in}}%
\pgfpathlineto{\pgfqpoint{1.188094in}{0.454879in}}%
\pgfpathlineto{\pgfqpoint{1.189968in}{0.454135in}}%
\pgfpathlineto{\pgfqpoint{1.203751in}{0.448412in}}%
\pgfpathlineto{\pgfqpoint{1.218677in}{0.440524in}}%
\pgfpathlineto{\pgfqpoint{1.219407in}{0.440113in}}%
\pgfpathlineto{\pgfqpoint{1.235064in}{0.429593in}}%
\pgfpathlineto{\pgfqpoint{1.238579in}{0.426913in}}%
\pgfpathlineto{\pgfqpoint{1.250721in}{0.416875in}}%
\pgfpathlineto{\pgfqpoint{1.254656in}{0.413302in}}%
\pgfpathlineto{\pgfqpoint{1.266377in}{0.401440in}}%
\pgfpathlineto{\pgfqpoint{1.267990in}{0.399691in}}%
\pgfpathlineto{\pgfqpoint{1.278771in}{0.386079in}}%
\pgfpathlineto{\pgfqpoint{1.282034in}{0.381021in}}%
\pgfpathlineto{\pgfqpoint{1.287262in}{0.372468in}}%
\pgfpathlineto{\pgfqpoint{1.293427in}{0.358857in}}%
\pgfpathlineto{\pgfqpoint{1.297234in}{0.345246in}}%
\pgfpathlineto{\pgfqpoint{1.297690in}{0.340413in}}%
\pgfpathlineto{\pgfqpoint{1.298481in}{0.331635in}}%
\pgfpathlineto{\pgfqpoint{1.313347in}{0.331635in}}%
\pgfpathlineto{\pgfqpoint{1.329003in}{0.331635in}}%
\pgfpathlineto{\pgfqpoint{1.344660in}{0.331635in}}%
\pgfpathlineto{\pgfqpoint{1.360317in}{0.331635in}}%
\pgfpathlineto{\pgfqpoint{1.375973in}{0.331635in}}%
\pgfpathlineto{\pgfqpoint{1.391630in}{0.331635in}}%
\pgfpathlineto{\pgfqpoint{1.403843in}{0.331635in}}%
\pgfpathlineto{\pgfqpoint{1.402283in}{0.345246in}}%
\pgfpathlineto{\pgfqpoint{1.397667in}{0.358857in}}%
\pgfpathlineto{\pgfqpoint{1.391630in}{0.369827in}}%
\pgfpathlineto{\pgfqpoint{1.390266in}{0.372468in}}%
\pgfpathlineto{\pgfqpoint{1.380748in}{0.386079in}}%
\pgfpathlineto{\pgfqpoint{1.375973in}{0.391631in}}%
\pgfpathlineto{\pgfqpoint{1.369401in}{0.399691in}}%
\pgfpathlineto{\pgfqpoint{1.360317in}{0.409307in}}%
\pgfpathlineto{\pgfqpoint{1.356689in}{0.413302in}}%
\pgfpathlineto{\pgfqpoint{1.344660in}{0.425214in}}%
\pgfpathlineto{\pgfqpoint{1.342988in}{0.426913in}}%
\pgfpathlineto{\pgfqpoint{1.329003in}{0.440090in}}%
\pgfpathlineto{\pgfqpoint{1.328548in}{0.440524in}}%
\pgfpathlineto{\pgfqpoint{1.313524in}{0.454135in}}%
\pgfpathlineto{\pgfqpoint{1.313347in}{0.454291in}}%
\pgfpathlineto{\pgfqpoint{1.297990in}{0.467746in}}%
\pgfpathlineto{\pgfqpoint{1.297690in}{0.468006in}}%
\pgfpathlineto{\pgfqpoint{1.282034in}{0.481296in}}%
\pgfpathlineto{\pgfqpoint{1.281959in}{0.481357in}}%
\pgfpathlineto{\pgfqpoint{1.266377in}{0.494113in}}%
\pgfpathlineto{\pgfqpoint{1.265266in}{0.494968in}}%
\pgfpathlineto{\pgfqpoint{1.250721in}{0.506368in}}%
\pgfpathlineto{\pgfqpoint{1.247633in}{0.508579in}}%
\pgfpathlineto{\pgfqpoint{1.235064in}{0.517882in}}%
\pgfpathlineto{\pgfqpoint{1.228467in}{0.522191in}}%
\pgfpathlineto{\pgfqpoint{1.219407in}{0.528388in}}%
\pgfpathlineto{\pgfqpoint{1.206523in}{0.535802in}}%
\pgfpathlineto{\pgfqpoint{1.203751in}{0.537496in}}%
\pgfpathlineto{\pgfqpoint{1.188094in}{0.544928in}}%
\pgfpathlineto{\pgfqpoint{1.174400in}{0.549413in}}%
\pgfpathlineto{\pgfqpoint{1.172438in}{0.550103in}}%
\pgfpathlineto{\pgfqpoint{1.156781in}{0.552895in}}%
\pgfpathlineto{\pgfqpoint{1.141125in}{0.552895in}}%
\pgfpathlineto{\pgfqpoint{1.125468in}{0.550103in}}%
\pgfpathlineto{\pgfqpoint{1.123506in}{0.549413in}}%
\pgfpathlineto{\pgfqpoint{1.109812in}{0.544928in}}%
\pgfpathlineto{\pgfqpoint{1.094155in}{0.537496in}}%
\pgfpathlineto{\pgfqpoint{1.091383in}{0.535802in}}%
\pgfpathlineto{\pgfqpoint{1.078498in}{0.528388in}}%
\pgfpathlineto{\pgfqpoint{1.069438in}{0.522191in}}%
\pgfpathlineto{\pgfqpoint{1.062842in}{0.517882in}}%
\pgfpathlineto{\pgfqpoint{1.050273in}{0.508579in}}%
\pgfpathlineto{\pgfqpoint{1.047185in}{0.506368in}}%
\pgfpathlineto{\pgfqpoint{1.032640in}{0.494968in}}%
\pgfpathlineto{\pgfqpoint{1.031529in}{0.494113in}}%
\pgfpathlineto{\pgfqpoint{1.015947in}{0.481357in}}%
\pgfpathlineto{\pgfqpoint{1.015872in}{0.481296in}}%
\pgfpathlineto{\pgfqpoint{1.000216in}{0.468006in}}%
\pgfpathlineto{\pgfqpoint{0.999916in}{0.467746in}}%
\pgfpathlineto{\pgfqpoint{0.984559in}{0.454291in}}%
\pgfpathlineto{\pgfqpoint{0.984382in}{0.454135in}}%
\pgfpathlineto{\pgfqpoint{0.969358in}{0.440524in}}%
\pgfpathlineto{\pgfqpoint{0.968902in}{0.440090in}}%
\pgfpathlineto{\pgfqpoint{0.954918in}{0.426913in}}%
\pgfpathlineto{\pgfqpoint{0.953246in}{0.425214in}}%
\pgfpathlineto{\pgfqpoint{0.941216in}{0.413302in}}%
\pgfpathlineto{\pgfqpoint{0.937589in}{0.409307in}}%
\pgfpathlineto{\pgfqpoint{0.928505in}{0.399691in}}%
\pgfpathlineto{\pgfqpoint{0.921933in}{0.391631in}}%
\pgfpathlineto{\pgfqpoint{0.917158in}{0.386079in}}%
\pgfpathlineto{\pgfqpoint{0.907640in}{0.372468in}}%
\pgfpathlineto{\pgfqpoint{0.906276in}{0.369827in}}%
\pgfpathlineto{\pgfqpoint{0.900238in}{0.358857in}}%
\pgfpathlineto{\pgfqpoint{0.895623in}{0.345246in}}%
\pgfpathlineto{\pgfqpoint{0.894062in}{0.331635in}}%
\pgfpathlineto{\pgfqpoint{0.906276in}{0.331635in}}%
\pgfpathclose%
\pgfpathmoveto{\pgfqpoint{1.673448in}{0.331635in}}%
\pgfpathlineto{\pgfqpoint{1.689104in}{0.331635in}}%
\pgfpathlineto{\pgfqpoint{1.704761in}{0.331635in}}%
\pgfpathlineto{\pgfqpoint{1.720418in}{0.331635in}}%
\pgfpathlineto{\pgfqpoint{1.736074in}{0.331635in}}%
\pgfpathlineto{\pgfqpoint{1.751731in}{0.331635in}}%
\pgfpathlineto{\pgfqpoint{1.767387in}{0.331635in}}%
\pgfpathlineto{\pgfqpoint{1.774503in}{0.331635in}}%
\pgfpathlineto{\pgfqpoint{1.775756in}{0.345246in}}%
\pgfpathlineto{\pgfqpoint{1.779461in}{0.358857in}}%
\pgfpathlineto{\pgfqpoint{1.783044in}{0.366946in}}%
\pgfpathlineto{\pgfqpoint{1.785609in}{0.372468in}}%
\pgfpathlineto{\pgfqpoint{1.794150in}{0.386079in}}%
\pgfpathlineto{\pgfqpoint{1.798700in}{0.391977in}}%
\pgfpathlineto{\pgfqpoint{1.805023in}{0.399691in}}%
\pgfpathlineto{\pgfqpoint{1.814357in}{0.409519in}}%
\pgfpathlineto{\pgfqpoint{1.818257in}{0.413302in}}%
\pgfpathlineto{\pgfqpoint{1.830014in}{0.423523in}}%
\pgfpathlineto{\pgfqpoint{1.834365in}{0.426913in}}%
\pgfpathlineto{\pgfqpoint{1.845670in}{0.435027in}}%
\pgfpathlineto{\pgfqpoint{1.854542in}{0.440524in}}%
\pgfpathlineto{\pgfqpoint{1.861327in}{0.444479in}}%
\pgfpathlineto{\pgfqpoint{1.876983in}{0.451905in}}%
\pgfpathlineto{\pgfqpoint{1.883336in}{0.454135in}}%
\pgfpathlineto{\pgfqpoint{1.892640in}{0.457249in}}%
\pgfpathlineto{\pgfqpoint{1.908296in}{0.460471in}}%
\pgfpathlineto{\pgfqpoint{1.923953in}{0.461560in}}%
\pgfpathlineto{\pgfqpoint{1.923953in}{0.467746in}}%
\pgfpathlineto{\pgfqpoint{1.923953in}{0.481357in}}%
\pgfpathlineto{\pgfqpoint{1.923953in}{0.494968in}}%
\pgfpathlineto{\pgfqpoint{1.923953in}{0.508579in}}%
\pgfpathlineto{\pgfqpoint{1.923953in}{0.522191in}}%
\pgfpathlineto{\pgfqpoint{1.923953in}{0.535802in}}%
\pgfpathlineto{\pgfqpoint{1.923953in}{0.549413in}}%
\pgfpathlineto{\pgfqpoint{1.923953in}{0.553246in}}%
\pgfpathlineto{\pgfqpoint{1.908296in}{0.551843in}}%
\pgfpathlineto{\pgfqpoint{1.899153in}{0.549413in}}%
\pgfpathlineto{\pgfqpoint{1.892640in}{0.547797in}}%
\pgfpathlineto{\pgfqpoint{1.876983in}{0.541482in}}%
\pgfpathlineto{\pgfqpoint{1.866541in}{0.535802in}}%
\pgfpathlineto{\pgfqpoint{1.861327in}{0.533126in}}%
\pgfpathlineto{\pgfqpoint{1.845670in}{0.523254in}}%
\pgfpathlineto{\pgfqpoint{1.844222in}{0.522191in}}%
\pgfpathlineto{\pgfqpoint{1.830014in}{0.512240in}}%
\pgfpathlineto{\pgfqpoint{1.825315in}{0.508579in}}%
\pgfpathlineto{\pgfqpoint{1.814357in}{0.500324in}}%
\pgfpathlineto{\pgfqpoint{1.807766in}{0.494968in}}%
\pgfpathlineto{\pgfqpoint{1.798700in}{0.487745in}}%
\pgfpathlineto{\pgfqpoint{1.791079in}{0.481357in}}%
\pgfpathlineto{\pgfqpoint{1.783044in}{0.474654in}}%
\pgfpathlineto{\pgfqpoint{1.775005in}{0.467746in}}%
\pgfpathlineto{\pgfqpoint{1.767387in}{0.461123in}}%
\pgfpathlineto{\pgfqpoint{1.759441in}{0.454135in}}%
\pgfpathlineto{\pgfqpoint{1.751731in}{0.447150in}}%
\pgfpathlineto{\pgfqpoint{1.744382in}{0.440524in}}%
\pgfpathlineto{\pgfqpoint{1.736074in}{0.432643in}}%
\pgfpathlineto{\pgfqpoint{1.729913in}{0.426913in}}%
\pgfpathlineto{\pgfqpoint{1.720418in}{0.417386in}}%
\pgfpathlineto{\pgfqpoint{1.716207in}{0.413302in}}%
\pgfpathlineto{\pgfqpoint{1.704761in}{0.400949in}}%
\pgfpathlineto{\pgfqpoint{1.703538in}{0.399691in}}%
\pgfpathlineto{\pgfqpoint{1.692182in}{0.386079in}}%
\pgfpathlineto{\pgfqpoint{1.689104in}{0.381546in}}%
\pgfpathlineto{\pgfqpoint{1.682570in}{0.372468in}}%
\pgfpathlineto{\pgfqpoint{1.675306in}{0.358857in}}%
\pgfpathlineto{\pgfqpoint{1.673448in}{0.353195in}}%
\pgfpathlineto{\pgfqpoint{1.670652in}{0.345246in}}%
\pgfpathlineto{\pgfqpoint{1.669038in}{0.331635in}}%
\pgfpathlineto{\pgfqpoint{1.673448in}{0.331635in}}%
\pgfpathclose%
\pgfpathmoveto{\pgfqpoint{0.389609in}{0.785152in}}%
\pgfpathlineto{\pgfqpoint{0.405266in}{0.789164in}}%
\pgfpathlineto{\pgfqpoint{0.417885in}{0.794413in}}%
\pgfpathlineto{\pgfqpoint{0.420923in}{0.795598in}}%
\pgfpathlineto{\pgfqpoint{0.436579in}{0.803873in}}%
\pgfpathlineto{\pgfqpoint{0.442965in}{0.808024in}}%
\pgfpathlineto{\pgfqpoint{0.452236in}{0.813738in}}%
\pgfpathlineto{\pgfqpoint{0.463298in}{0.821635in}}%
\pgfpathlineto{\pgfqpoint{0.467892in}{0.824788in}}%
\pgfpathlineto{\pgfqpoint{0.481594in}{0.835246in}}%
\pgfpathlineto{\pgfqpoint{0.483549in}{0.836700in}}%
\pgfpathlineto{\pgfqpoint{0.498706in}{0.848857in}}%
\pgfpathlineto{\pgfqpoint{0.499205in}{0.849253in}}%
\pgfpathlineto{\pgfqpoint{0.514862in}{0.862314in}}%
\pgfpathlineto{\pgfqpoint{0.515041in}{0.862468in}}%
\pgfpathlineto{\pgfqpoint{0.530519in}{0.875819in}}%
\pgfpathlineto{\pgfqpoint{0.530817in}{0.876079in}}%
\pgfpathlineto{\pgfqpoint{0.546105in}{0.889691in}}%
\pgfpathlineto{\pgfqpoint{0.546175in}{0.889756in}}%
\pgfpathlineto{\pgfqpoint{0.560848in}{0.903302in}}%
\pgfpathlineto{\pgfqpoint{0.561832in}{0.904268in}}%
\pgfpathlineto{\pgfqpoint{0.574944in}{0.916913in}}%
\pgfpathlineto{\pgfqpoint{0.577488in}{0.919597in}}%
\pgfpathlineto{\pgfqpoint{0.588189in}{0.930524in}}%
\pgfpathlineto{\pgfqpoint{0.593145in}{0.936259in}}%
\pgfpathlineto{\pgfqpoint{0.600274in}{0.944135in}}%
\pgfpathlineto{\pgfqpoint{0.608801in}{0.955336in}}%
\pgfpathlineto{\pgfqpoint{0.610750in}{0.957746in}}%
\pgfpathlineto{\pgfqpoint{0.619299in}{0.971357in}}%
\pgfpathlineto{\pgfqpoint{0.624458in}{0.983262in}}%
\pgfpathlineto{\pgfqpoint{0.625252in}{0.984968in}}%
\pgfpathlineto{\pgfqpoint{0.628463in}{0.998579in}}%
\pgfpathlineto{\pgfqpoint{0.628463in}{1.012191in}}%
\pgfpathlineto{\pgfqpoint{0.625252in}{1.025802in}}%
\pgfpathlineto{\pgfqpoint{0.624458in}{1.027508in}}%
\pgfpathlineto{\pgfqpoint{0.619299in}{1.039413in}}%
\pgfpathlineto{\pgfqpoint{0.610750in}{1.053024in}}%
\pgfpathlineto{\pgfqpoint{0.608801in}{1.055434in}}%
\pgfpathlineto{\pgfqpoint{0.600274in}{1.066635in}}%
\pgfpathlineto{\pgfqpoint{0.593145in}{1.074511in}}%
\pgfpathlineto{\pgfqpoint{0.588189in}{1.080246in}}%
\pgfpathlineto{\pgfqpoint{0.577488in}{1.091173in}}%
\pgfpathlineto{\pgfqpoint{0.574944in}{1.093857in}}%
\pgfpathlineto{\pgfqpoint{0.561832in}{1.106502in}}%
\pgfpathlineto{\pgfqpoint{0.560848in}{1.107468in}}%
\pgfpathlineto{\pgfqpoint{0.546175in}{1.121014in}}%
\pgfpathlineto{\pgfqpoint{0.546105in}{1.121079in}}%
\pgfpathlineto{\pgfqpoint{0.530817in}{1.134691in}}%
\pgfpathlineto{\pgfqpoint{0.530519in}{1.134951in}}%
\pgfpathlineto{\pgfqpoint{0.515041in}{1.148302in}}%
\pgfpathlineto{\pgfqpoint{0.514862in}{1.148456in}}%
\pgfpathlineto{\pgfqpoint{0.499205in}{1.161517in}}%
\pgfpathlineto{\pgfqpoint{0.498706in}{1.161913in}}%
\pgfpathlineto{\pgfqpoint{0.483549in}{1.174070in}}%
\pgfpathlineto{\pgfqpoint{0.481594in}{1.175524in}}%
\pgfpathlineto{\pgfqpoint{0.467892in}{1.185982in}}%
\pgfpathlineto{\pgfqpoint{0.463298in}{1.189135in}}%
\pgfpathlineto{\pgfqpoint{0.452236in}{1.197032in}}%
\pgfpathlineto{\pgfqpoint{0.442965in}{1.202746in}}%
\pgfpathlineto{\pgfqpoint{0.436579in}{1.206897in}}%
\pgfpathlineto{\pgfqpoint{0.420923in}{1.215172in}}%
\pgfpathlineto{\pgfqpoint{0.417885in}{1.216357in}}%
\pgfpathlineto{\pgfqpoint{0.405266in}{1.221606in}}%
\pgfpathlineto{\pgfqpoint{0.389609in}{1.225618in}}%
\pgfpathlineto{\pgfqpoint{0.373953in}{1.226975in}}%
\pgfpathlineto{\pgfqpoint{0.373953in}{1.216357in}}%
\pgfpathlineto{\pgfqpoint{0.373953in}{1.202746in}}%
\pgfpathlineto{\pgfqpoint{0.373953in}{1.189135in}}%
\pgfpathlineto{\pgfqpoint{0.373953in}{1.175524in}}%
\pgfpathlineto{\pgfqpoint{0.373953in}{1.161913in}}%
\pgfpathlineto{\pgfqpoint{0.373953in}{1.148302in}}%
\pgfpathlineto{\pgfqpoint{0.373953in}{1.135378in}}%
\pgfpathlineto{\pgfqpoint{0.384050in}{1.134691in}}%
\pgfpathlineto{\pgfqpoint{0.389609in}{1.134294in}}%
\pgfpathlineto{\pgfqpoint{0.405266in}{1.130985in}}%
\pgfpathlineto{\pgfqpoint{0.420923in}{1.125625in}}%
\pgfpathlineto{\pgfqpoint{0.430761in}{1.121079in}}%
\pgfpathlineto{\pgfqpoint{0.436579in}{1.118243in}}%
\pgfpathlineto{\pgfqpoint{0.452236in}{1.108871in}}%
\pgfpathlineto{\pgfqpoint{0.454249in}{1.107468in}}%
\pgfpathlineto{\pgfqpoint{0.467892in}{1.097279in}}%
\pgfpathlineto{\pgfqpoint{0.472002in}{1.093857in}}%
\pgfpathlineto{\pgfqpoint{0.483549in}{1.083302in}}%
\pgfpathlineto{\pgfqpoint{0.486632in}{1.080246in}}%
\pgfpathlineto{\pgfqpoint{0.498733in}{1.066635in}}%
\pgfpathlineto{\pgfqpoint{0.499205in}{1.066000in}}%
\pgfpathlineto{\pgfqpoint{0.508279in}{1.053024in}}%
\pgfpathlineto{\pgfqpoint{0.514862in}{1.041042in}}%
\pgfpathlineto{\pgfqpoint{0.515717in}{1.039413in}}%
\pgfpathlineto{\pgfqpoint{0.520596in}{1.025802in}}%
\pgfpathlineto{\pgfqpoint{0.523089in}{1.012191in}}%
\pgfpathlineto{\pgfqpoint{0.523089in}{0.998579in}}%
\pgfpathlineto{\pgfqpoint{0.520596in}{0.984968in}}%
\pgfpathlineto{\pgfqpoint{0.515717in}{0.971357in}}%
\pgfpathlineto{\pgfqpoint{0.514862in}{0.969728in}}%
\pgfpathlineto{\pgfqpoint{0.508279in}{0.957746in}}%
\pgfpathlineto{\pgfqpoint{0.499205in}{0.944770in}}%
\pgfpathlineto{\pgfqpoint{0.498733in}{0.944135in}}%
\pgfpathlineto{\pgfqpoint{0.486632in}{0.930524in}}%
\pgfpathlineto{\pgfqpoint{0.483549in}{0.927468in}}%
\pgfpathlineto{\pgfqpoint{0.472002in}{0.916913in}}%
\pgfpathlineto{\pgfqpoint{0.467892in}{0.913491in}}%
\pgfpathlineto{\pgfqpoint{0.454249in}{0.903302in}}%
\pgfpathlineto{\pgfqpoint{0.452236in}{0.901899in}}%
\pgfpathlineto{\pgfqpoint{0.436579in}{0.892527in}}%
\pgfpathlineto{\pgfqpoint{0.430761in}{0.889691in}}%
\pgfpathlineto{\pgfqpoint{0.420923in}{0.885145in}}%
\pgfpathlineto{\pgfqpoint{0.405266in}{0.879785in}}%
\pgfpathlineto{\pgfqpoint{0.389609in}{0.876476in}}%
\pgfpathlineto{\pgfqpoint{0.384050in}{0.876079in}}%
\pgfpathlineto{\pgfqpoint{0.373953in}{0.875392in}}%
\pgfpathlineto{\pgfqpoint{0.373953in}{0.862468in}}%
\pgfpathlineto{\pgfqpoint{0.373953in}{0.848857in}}%
\pgfpathlineto{\pgfqpoint{0.373953in}{0.835246in}}%
\pgfpathlineto{\pgfqpoint{0.373953in}{0.821635in}}%
\pgfpathlineto{\pgfqpoint{0.373953in}{0.808024in}}%
\pgfpathlineto{\pgfqpoint{0.373953in}{0.794413in}}%
\pgfpathlineto{\pgfqpoint{0.373953in}{0.783795in}}%
\pgfpathlineto{\pgfqpoint{0.389609in}{0.785152in}}%
\pgfpathclose%
\pgfpathmoveto{\pgfqpoint{1.109812in}{0.792117in}}%
\pgfpathlineto{\pgfqpoint{1.125468in}{0.786834in}}%
\pgfpathlineto{\pgfqpoint{1.141125in}{0.784135in}}%
\pgfpathlineto{\pgfqpoint{1.156781in}{0.784135in}}%
\pgfpathlineto{\pgfqpoint{1.172438in}{0.786834in}}%
\pgfpathlineto{\pgfqpoint{1.188094in}{0.792117in}}%
\pgfpathlineto{\pgfqpoint{1.192751in}{0.794413in}}%
\pgfpathlineto{\pgfqpoint{1.203751in}{0.799491in}}%
\pgfpathlineto{\pgfqpoint{1.218234in}{0.808024in}}%
\pgfpathlineto{\pgfqpoint{1.219407in}{0.808681in}}%
\pgfpathlineto{\pgfqpoint{1.235064in}{0.819141in}}%
\pgfpathlineto{\pgfqpoint{1.238352in}{0.821635in}}%
\pgfpathlineto{\pgfqpoint{1.250721in}{0.830642in}}%
\pgfpathlineto{\pgfqpoint{1.256496in}{0.835246in}}%
\pgfpathlineto{\pgfqpoint{1.266377in}{0.842914in}}%
\pgfpathlineto{\pgfqpoint{1.273572in}{0.848857in}}%
\pgfpathlineto{\pgfqpoint{1.282034in}{0.855762in}}%
\pgfpathlineto{\pgfqpoint{1.289931in}{0.862468in}}%
\pgfpathlineto{\pgfqpoint{1.297690in}{0.869078in}}%
\pgfpathlineto{\pgfqpoint{1.305744in}{0.876079in}}%
\pgfpathlineto{\pgfqpoint{1.313347in}{0.882825in}}%
\pgfpathlineto{\pgfqpoint{1.321061in}{0.889691in}}%
\pgfpathlineto{\pgfqpoint{1.329003in}{0.897047in}}%
\pgfpathlineto{\pgfqpoint{1.335840in}{0.903302in}}%
\pgfpathlineto{\pgfqpoint{1.344660in}{0.911891in}}%
\pgfpathlineto{\pgfqpoint{1.349956in}{0.916913in}}%
\pgfpathlineto{\pgfqpoint{1.360317in}{0.927665in}}%
\pgfpathlineto{\pgfqpoint{1.363186in}{0.930524in}}%
\pgfpathlineto{\pgfqpoint{1.375218in}{0.944135in}}%
\pgfpathlineto{\pgfqpoint{1.375973in}{0.945155in}}%
\pgfpathlineto{\pgfqpoint{1.385789in}{0.957746in}}%
\pgfpathlineto{\pgfqpoint{1.391630in}{0.967309in}}%
\pgfpathlineto{\pgfqpoint{1.394271in}{0.971357in}}%
\pgfpathlineto{\pgfqpoint{1.400348in}{0.984968in}}%
\pgfpathlineto{\pgfqpoint{1.403452in}{0.998579in}}%
\pgfpathlineto{\pgfqpoint{1.403452in}{1.012191in}}%
\pgfpathlineto{\pgfqpoint{1.400348in}{1.025802in}}%
\pgfpathlineto{\pgfqpoint{1.394271in}{1.039413in}}%
\pgfpathlineto{\pgfqpoint{1.391630in}{1.043461in}}%
\pgfpathlineto{\pgfqpoint{1.385789in}{1.053024in}}%
\pgfpathlineto{\pgfqpoint{1.375973in}{1.065615in}}%
\pgfpathlineto{\pgfqpoint{1.375218in}{1.066635in}}%
\pgfpathlineto{\pgfqpoint{1.363186in}{1.080246in}}%
\pgfpathlineto{\pgfqpoint{1.360317in}{1.083105in}}%
\pgfpathlineto{\pgfqpoint{1.349956in}{1.093857in}}%
\pgfpathlineto{\pgfqpoint{1.344660in}{1.098879in}}%
\pgfpathlineto{\pgfqpoint{1.335840in}{1.107468in}}%
\pgfpathlineto{\pgfqpoint{1.329003in}{1.113723in}}%
\pgfpathlineto{\pgfqpoint{1.321061in}{1.121079in}}%
\pgfpathlineto{\pgfqpoint{1.313347in}{1.127945in}}%
\pgfpathlineto{\pgfqpoint{1.305744in}{1.134691in}}%
\pgfpathlineto{\pgfqpoint{1.297690in}{1.141692in}}%
\pgfpathlineto{\pgfqpoint{1.289931in}{1.148302in}}%
\pgfpathlineto{\pgfqpoint{1.282034in}{1.155008in}}%
\pgfpathlineto{\pgfqpoint{1.273572in}{1.161913in}}%
\pgfpathlineto{\pgfqpoint{1.266377in}{1.167856in}}%
\pgfpathlineto{\pgfqpoint{1.256496in}{1.175524in}}%
\pgfpathlineto{\pgfqpoint{1.250721in}{1.180128in}}%
\pgfpathlineto{\pgfqpoint{1.238352in}{1.189135in}}%
\pgfpathlineto{\pgfqpoint{1.235064in}{1.191629in}}%
\pgfpathlineto{\pgfqpoint{1.219407in}{1.202089in}}%
\pgfpathlineto{\pgfqpoint{1.218234in}{1.202746in}}%
\pgfpathlineto{\pgfqpoint{1.203751in}{1.211279in}}%
\pgfpathlineto{\pgfqpoint{1.192751in}{1.216357in}}%
\pgfpathlineto{\pgfqpoint{1.188094in}{1.218653in}}%
\pgfpathlineto{\pgfqpoint{1.172438in}{1.223936in}}%
\pgfpathlineto{\pgfqpoint{1.156781in}{1.226635in}}%
\pgfpathlineto{\pgfqpoint{1.141125in}{1.226635in}}%
\pgfpathlineto{\pgfqpoint{1.125468in}{1.223936in}}%
\pgfpathlineto{\pgfqpoint{1.109812in}{1.218653in}}%
\pgfpathlineto{\pgfqpoint{1.105155in}{1.216357in}}%
\pgfpathlineto{\pgfqpoint{1.094155in}{1.211279in}}%
\pgfpathlineto{\pgfqpoint{1.079672in}{1.202746in}}%
\pgfpathlineto{\pgfqpoint{1.078498in}{1.202089in}}%
\pgfpathlineto{\pgfqpoint{1.062842in}{1.191629in}}%
\pgfpathlineto{\pgfqpoint{1.059554in}{1.189135in}}%
\pgfpathlineto{\pgfqpoint{1.047185in}{1.180128in}}%
\pgfpathlineto{\pgfqpoint{1.041409in}{1.175524in}}%
\pgfpathlineto{\pgfqpoint{1.031529in}{1.167856in}}%
\pgfpathlineto{\pgfqpoint{1.024334in}{1.161913in}}%
\pgfpathlineto{\pgfqpoint{1.015872in}{1.155008in}}%
\pgfpathlineto{\pgfqpoint{1.007975in}{1.148302in}}%
\pgfpathlineto{\pgfqpoint{1.000216in}{1.141692in}}%
\pgfpathlineto{\pgfqpoint{0.992162in}{1.134691in}}%
\pgfpathlineto{\pgfqpoint{0.984559in}{1.127945in}}%
\pgfpathlineto{\pgfqpoint{0.976845in}{1.121079in}}%
\pgfpathlineto{\pgfqpoint{0.968902in}{1.113723in}}%
\pgfpathlineto{\pgfqpoint{0.962066in}{1.107468in}}%
\pgfpathlineto{\pgfqpoint{0.953246in}{1.098879in}}%
\pgfpathlineto{\pgfqpoint{0.947950in}{1.093857in}}%
\pgfpathlineto{\pgfqpoint{0.937589in}{1.083105in}}%
\pgfpathlineto{\pgfqpoint{0.934720in}{1.080246in}}%
\pgfpathlineto{\pgfqpoint{0.922688in}{1.066635in}}%
\pgfpathlineto{\pgfqpoint{0.921933in}{1.065615in}}%
\pgfpathlineto{\pgfqpoint{0.912117in}{1.053024in}}%
\pgfpathlineto{\pgfqpoint{0.906276in}{1.043461in}}%
\pgfpathlineto{\pgfqpoint{0.903635in}{1.039413in}}%
\pgfpathlineto{\pgfqpoint{0.897558in}{1.025802in}}%
\pgfpathlineto{\pgfqpoint{0.894454in}{1.012191in}}%
\pgfpathlineto{\pgfqpoint{0.894454in}{0.998579in}}%
\pgfpathlineto{\pgfqpoint{0.897558in}{0.984968in}}%
\pgfpathlineto{\pgfqpoint{0.903635in}{0.971357in}}%
\pgfpathlineto{\pgfqpoint{0.906276in}{0.967309in}}%
\pgfpathlineto{\pgfqpoint{0.912117in}{0.957746in}}%
\pgfpathlineto{\pgfqpoint{0.921933in}{0.945155in}}%
\pgfpathlineto{\pgfqpoint{0.922688in}{0.944135in}}%
\pgfpathlineto{\pgfqpoint{0.934720in}{0.930524in}}%
\pgfpathlineto{\pgfqpoint{0.937589in}{0.927665in}}%
\pgfpathlineto{\pgfqpoint{0.947950in}{0.916913in}}%
\pgfpathlineto{\pgfqpoint{0.953246in}{0.911891in}}%
\pgfpathlineto{\pgfqpoint{0.962066in}{0.903302in}}%
\pgfpathlineto{\pgfqpoint{0.968902in}{0.897047in}}%
\pgfpathlineto{\pgfqpoint{0.976845in}{0.889691in}}%
\pgfpathlineto{\pgfqpoint{0.984559in}{0.882825in}}%
\pgfpathlineto{\pgfqpoint{0.992162in}{0.876079in}}%
\pgfpathlineto{\pgfqpoint{1.000216in}{0.869078in}}%
\pgfpathlineto{\pgfqpoint{1.007975in}{0.862468in}}%
\pgfpathlineto{\pgfqpoint{1.015872in}{0.855762in}}%
\pgfpathlineto{\pgfqpoint{1.024334in}{0.848857in}}%
\pgfpathlineto{\pgfqpoint{1.031529in}{0.842914in}}%
\pgfpathlineto{\pgfqpoint{1.041409in}{0.835246in}}%
\pgfpathlineto{\pgfqpoint{1.047185in}{0.830642in}}%
\pgfpathlineto{\pgfqpoint{1.059554in}{0.821635in}}%
\pgfpathlineto{\pgfqpoint{1.062842in}{0.819141in}}%
\pgfpathlineto{\pgfqpoint{1.078498in}{0.808681in}}%
\pgfpathlineto{\pgfqpoint{1.079672in}{0.808024in}}%
\pgfpathlineto{\pgfqpoint{1.094155in}{0.799491in}}%
\pgfpathlineto{\pgfqpoint{1.105155in}{0.794413in}}%
\pgfpathlineto{\pgfqpoint{1.109812in}{0.792117in}}%
\pgfpathclose%
\pgfpathmoveto{\pgfqpoint{1.138029in}{0.876079in}}%
\pgfpathlineto{\pgfqpoint{1.125468in}{0.877864in}}%
\pgfpathlineto{\pgfqpoint{1.109812in}{0.882221in}}%
\pgfpathlineto{\pgfqpoint{1.094155in}{0.888529in}}%
\pgfpathlineto{\pgfqpoint{1.091916in}{0.889691in}}%
\pgfpathlineto{\pgfqpoint{1.078498in}{0.897025in}}%
\pgfpathlineto{\pgfqpoint{1.068826in}{0.903302in}}%
\pgfpathlineto{\pgfqpoint{1.062842in}{0.907470in}}%
\pgfpathlineto{\pgfqpoint{1.050899in}{0.916913in}}%
\pgfpathlineto{\pgfqpoint{1.047185in}{0.920141in}}%
\pgfpathlineto{\pgfqpoint{1.036323in}{0.930524in}}%
\pgfpathlineto{\pgfqpoint{1.031529in}{0.935726in}}%
\pgfpathlineto{\pgfqpoint{1.024309in}{0.944135in}}%
\pgfpathlineto{\pgfqpoint{1.015872in}{0.955800in}}%
\pgfpathlineto{\pgfqpoint{1.014536in}{0.957746in}}%
\pgfpathlineto{\pgfqpoint{1.007280in}{0.971357in}}%
\pgfpathlineto{\pgfqpoint{1.002268in}{0.984968in}}%
\pgfpathlineto{\pgfqpoint{1.000216in}{0.995888in}}%
\pgfpathlineto{\pgfqpoint{0.999732in}{0.998579in}}%
\pgfpathlineto{\pgfqpoint{0.999732in}{1.012191in}}%
\pgfpathlineto{\pgfqpoint{1.000216in}{1.014882in}}%
\pgfpathlineto{\pgfqpoint{1.002268in}{1.025802in}}%
\pgfpathlineto{\pgfqpoint{1.007280in}{1.039413in}}%
\pgfpathlineto{\pgfqpoint{1.014536in}{1.053024in}}%
\pgfpathlineto{\pgfqpoint{1.015872in}{1.054970in}}%
\pgfpathlineto{\pgfqpoint{1.024309in}{1.066635in}}%
\pgfpathlineto{\pgfqpoint{1.031529in}{1.075044in}}%
\pgfpathlineto{\pgfqpoint{1.036323in}{1.080246in}}%
\pgfpathlineto{\pgfqpoint{1.047185in}{1.090629in}}%
\pgfpathlineto{\pgfqpoint{1.050899in}{1.093857in}}%
\pgfpathlineto{\pgfqpoint{1.062842in}{1.103300in}}%
\pgfpathlineto{\pgfqpoint{1.068826in}{1.107468in}}%
\pgfpathlineto{\pgfqpoint{1.078498in}{1.113745in}}%
\pgfpathlineto{\pgfqpoint{1.091916in}{1.121079in}}%
\pgfpathlineto{\pgfqpoint{1.094155in}{1.122241in}}%
\pgfpathlineto{\pgfqpoint{1.109812in}{1.128549in}}%
\pgfpathlineto{\pgfqpoint{1.125468in}{1.132906in}}%
\pgfpathlineto{\pgfqpoint{1.138029in}{1.134691in}}%
\pgfpathlineto{\pgfqpoint{1.141125in}{1.135111in}}%
\pgfpathlineto{\pgfqpoint{1.156781in}{1.135111in}}%
\pgfpathlineto{\pgfqpoint{1.159877in}{1.134691in}}%
\pgfpathlineto{\pgfqpoint{1.172438in}{1.132906in}}%
\pgfpathlineto{\pgfqpoint{1.188094in}{1.128549in}}%
\pgfpathlineto{\pgfqpoint{1.203751in}{1.122241in}}%
\pgfpathlineto{\pgfqpoint{1.205989in}{1.121079in}}%
\pgfpathlineto{\pgfqpoint{1.219407in}{1.113745in}}%
\pgfpathlineto{\pgfqpoint{1.229080in}{1.107468in}}%
\pgfpathlineto{\pgfqpoint{1.235064in}{1.103300in}}%
\pgfpathlineto{\pgfqpoint{1.247007in}{1.093857in}}%
\pgfpathlineto{\pgfqpoint{1.250721in}{1.090629in}}%
\pgfpathlineto{\pgfqpoint{1.261583in}{1.080246in}}%
\pgfpathlineto{\pgfqpoint{1.266377in}{1.075044in}}%
\pgfpathlineto{\pgfqpoint{1.273597in}{1.066635in}}%
\pgfpathlineto{\pgfqpoint{1.282034in}{1.054970in}}%
\pgfpathlineto{\pgfqpoint{1.283370in}{1.053024in}}%
\pgfpathlineto{\pgfqpoint{1.290626in}{1.039413in}}%
\pgfpathlineto{\pgfqpoint{1.295638in}{1.025802in}}%
\pgfpathlineto{\pgfqpoint{1.297690in}{1.014882in}}%
\pgfpathlineto{\pgfqpoint{1.298174in}{1.012191in}}%
\pgfpathlineto{\pgfqpoint{1.298174in}{0.998579in}}%
\pgfpathlineto{\pgfqpoint{1.297690in}{0.995888in}}%
\pgfpathlineto{\pgfqpoint{1.295638in}{0.984968in}}%
\pgfpathlineto{\pgfqpoint{1.290626in}{0.971357in}}%
\pgfpathlineto{\pgfqpoint{1.283370in}{0.957746in}}%
\pgfpathlineto{\pgfqpoint{1.282034in}{0.955800in}}%
\pgfpathlineto{\pgfqpoint{1.273597in}{0.944135in}}%
\pgfpathlineto{\pgfqpoint{1.266377in}{0.935726in}}%
\pgfpathlineto{\pgfqpoint{1.261583in}{0.930524in}}%
\pgfpathlineto{\pgfqpoint{1.250721in}{0.920141in}}%
\pgfpathlineto{\pgfqpoint{1.247007in}{0.916913in}}%
\pgfpathlineto{\pgfqpoint{1.235064in}{0.907470in}}%
\pgfpathlineto{\pgfqpoint{1.229080in}{0.903302in}}%
\pgfpathlineto{\pgfqpoint{1.219407in}{0.897025in}}%
\pgfpathlineto{\pgfqpoint{1.205989in}{0.889691in}}%
\pgfpathlineto{\pgfqpoint{1.203751in}{0.888529in}}%
\pgfpathlineto{\pgfqpoint{1.188094in}{0.882221in}}%
\pgfpathlineto{\pgfqpoint{1.172438in}{0.877864in}}%
\pgfpathlineto{\pgfqpoint{1.159877in}{0.876079in}}%
\pgfpathlineto{\pgfqpoint{1.156781in}{0.875659in}}%
\pgfpathlineto{\pgfqpoint{1.141125in}{0.875659in}}%
\pgfpathlineto{\pgfqpoint{1.138029in}{0.876079in}}%
\pgfpathclose%
\pgfpathmoveto{\pgfqpoint{1.892640in}{0.789164in}}%
\pgfpathlineto{\pgfqpoint{1.908296in}{0.785152in}}%
\pgfpathlineto{\pgfqpoint{1.923953in}{0.783795in}}%
\pgfpathlineto{\pgfqpoint{1.923953in}{0.794413in}}%
\pgfpathlineto{\pgfqpoint{1.923953in}{0.808024in}}%
\pgfpathlineto{\pgfqpoint{1.923953in}{0.821635in}}%
\pgfpathlineto{\pgfqpoint{1.923953in}{0.835246in}}%
\pgfpathlineto{\pgfqpoint{1.923953in}{0.848857in}}%
\pgfpathlineto{\pgfqpoint{1.923953in}{0.862468in}}%
\pgfpathlineto{\pgfqpoint{1.923953in}{0.875392in}}%
\pgfpathlineto{\pgfqpoint{1.913856in}{0.876079in}}%
\pgfpathlineto{\pgfqpoint{1.908296in}{0.876476in}}%
\pgfpathlineto{\pgfqpoint{1.892640in}{0.879785in}}%
\pgfpathlineto{\pgfqpoint{1.876983in}{0.885145in}}%
\pgfpathlineto{\pgfqpoint{1.867145in}{0.889691in}}%
\pgfpathlineto{\pgfqpoint{1.861327in}{0.892527in}}%
\pgfpathlineto{\pgfqpoint{1.845670in}{0.901899in}}%
\pgfpathlineto{\pgfqpoint{1.843657in}{0.903302in}}%
\pgfpathlineto{\pgfqpoint{1.830014in}{0.913491in}}%
\pgfpathlineto{\pgfqpoint{1.825904in}{0.916913in}}%
\pgfpathlineto{\pgfqpoint{1.814357in}{0.927468in}}%
\pgfpathlineto{\pgfqpoint{1.811274in}{0.930524in}}%
\pgfpathlineto{\pgfqpoint{1.799173in}{0.944135in}}%
\pgfpathlineto{\pgfqpoint{1.798700in}{0.944770in}}%
\pgfpathlineto{\pgfqpoint{1.789627in}{0.957746in}}%
\pgfpathlineto{\pgfqpoint{1.783044in}{0.969728in}}%
\pgfpathlineto{\pgfqpoint{1.782188in}{0.971357in}}%
\pgfpathlineto{\pgfqpoint{1.777310in}{0.984968in}}%
\pgfpathlineto{\pgfqpoint{1.774817in}{0.998579in}}%
\pgfpathlineto{\pgfqpoint{1.774817in}{1.012191in}}%
\pgfpathlineto{\pgfqpoint{1.777310in}{1.025802in}}%
\pgfpathlineto{\pgfqpoint{1.782188in}{1.039413in}}%
\pgfpathlineto{\pgfqpoint{1.783044in}{1.041042in}}%
\pgfpathlineto{\pgfqpoint{1.789627in}{1.053024in}}%
\pgfpathlineto{\pgfqpoint{1.798700in}{1.066000in}}%
\pgfpathlineto{\pgfqpoint{1.799173in}{1.066635in}}%
\pgfpathlineto{\pgfqpoint{1.811274in}{1.080246in}}%
\pgfpathlineto{\pgfqpoint{1.814357in}{1.083302in}}%
\pgfpathlineto{\pgfqpoint{1.825904in}{1.093857in}}%
\pgfpathlineto{\pgfqpoint{1.830014in}{1.097279in}}%
\pgfpathlineto{\pgfqpoint{1.843657in}{1.107468in}}%
\pgfpathlineto{\pgfqpoint{1.845670in}{1.108871in}}%
\pgfpathlineto{\pgfqpoint{1.861327in}{1.118243in}}%
\pgfpathlineto{\pgfqpoint{1.867145in}{1.121079in}}%
\pgfpathlineto{\pgfqpoint{1.876983in}{1.125625in}}%
\pgfpathlineto{\pgfqpoint{1.892640in}{1.130985in}}%
\pgfpathlineto{\pgfqpoint{1.908296in}{1.134294in}}%
\pgfpathlineto{\pgfqpoint{1.913856in}{1.134691in}}%
\pgfpathlineto{\pgfqpoint{1.923953in}{1.135378in}}%
\pgfpathlineto{\pgfqpoint{1.923953in}{1.148302in}}%
\pgfpathlineto{\pgfqpoint{1.923953in}{1.161913in}}%
\pgfpathlineto{\pgfqpoint{1.923953in}{1.175524in}}%
\pgfpathlineto{\pgfqpoint{1.923953in}{1.189135in}}%
\pgfpathlineto{\pgfqpoint{1.923953in}{1.202746in}}%
\pgfpathlineto{\pgfqpoint{1.923953in}{1.216357in}}%
\pgfpathlineto{\pgfqpoint{1.923953in}{1.226975in}}%
\pgfpathlineto{\pgfqpoint{1.908296in}{1.225618in}}%
\pgfpathlineto{\pgfqpoint{1.892640in}{1.221606in}}%
\pgfpathlineto{\pgfqpoint{1.880021in}{1.216357in}}%
\pgfpathlineto{\pgfqpoint{1.876983in}{1.215172in}}%
\pgfpathlineto{\pgfqpoint{1.861327in}{1.206897in}}%
\pgfpathlineto{\pgfqpoint{1.854941in}{1.202746in}}%
\pgfpathlineto{\pgfqpoint{1.845670in}{1.197032in}}%
\pgfpathlineto{\pgfqpoint{1.834608in}{1.189135in}}%
\pgfpathlineto{\pgfqpoint{1.830014in}{1.185982in}}%
\pgfpathlineto{\pgfqpoint{1.816312in}{1.175524in}}%
\pgfpathlineto{\pgfqpoint{1.814357in}{1.174070in}}%
\pgfpathlineto{\pgfqpoint{1.799200in}{1.161913in}}%
\pgfpathlineto{\pgfqpoint{1.798700in}{1.161517in}}%
\pgfpathlineto{\pgfqpoint{1.783044in}{1.148456in}}%
\pgfpathlineto{\pgfqpoint{1.782865in}{1.148302in}}%
\pgfpathlineto{\pgfqpoint{1.767387in}{1.134951in}}%
\pgfpathlineto{\pgfqpoint{1.767089in}{1.134691in}}%
\pgfpathlineto{\pgfqpoint{1.751801in}{1.121079in}}%
\pgfpathlineto{\pgfqpoint{1.751731in}{1.121014in}}%
\pgfpathlineto{\pgfqpoint{1.737058in}{1.107468in}}%
\pgfpathlineto{\pgfqpoint{1.736074in}{1.106502in}}%
\pgfpathlineto{\pgfqpoint{1.722962in}{1.093857in}}%
\pgfpathlineto{\pgfqpoint{1.720418in}{1.091173in}}%
\pgfpathlineto{\pgfqpoint{1.709717in}{1.080246in}}%
\pgfpathlineto{\pgfqpoint{1.704761in}{1.074511in}}%
\pgfpathlineto{\pgfqpoint{1.697632in}{1.066635in}}%
\pgfpathlineto{\pgfqpoint{1.689104in}{1.055434in}}%
\pgfpathlineto{\pgfqpoint{1.687156in}{1.053024in}}%
\pgfpathlineto{\pgfqpoint{1.678607in}{1.039413in}}%
\pgfpathlineto{\pgfqpoint{1.673448in}{1.027508in}}%
\pgfpathlineto{\pgfqpoint{1.672654in}{1.025802in}}%
\pgfpathlineto{\pgfqpoint{1.669443in}{1.012191in}}%
\pgfpathlineto{\pgfqpoint{1.669443in}{0.998579in}}%
\pgfpathlineto{\pgfqpoint{1.672654in}{0.984968in}}%
\pgfpathlineto{\pgfqpoint{1.673448in}{0.983262in}}%
\pgfpathlineto{\pgfqpoint{1.678607in}{0.971357in}}%
\pgfpathlineto{\pgfqpoint{1.687156in}{0.957746in}}%
\pgfpathlineto{\pgfqpoint{1.689104in}{0.955336in}}%
\pgfpathlineto{\pgfqpoint{1.697632in}{0.944135in}}%
\pgfpathlineto{\pgfqpoint{1.704761in}{0.936259in}}%
\pgfpathlineto{\pgfqpoint{1.709717in}{0.930524in}}%
\pgfpathlineto{\pgfqpoint{1.720418in}{0.919597in}}%
\pgfpathlineto{\pgfqpoint{1.722962in}{0.916913in}}%
\pgfpathlineto{\pgfqpoint{1.736074in}{0.904268in}}%
\pgfpathlineto{\pgfqpoint{1.737058in}{0.903302in}}%
\pgfpathlineto{\pgfqpoint{1.751731in}{0.889756in}}%
\pgfpathlineto{\pgfqpoint{1.751801in}{0.889691in}}%
\pgfpathlineto{\pgfqpoint{1.767089in}{0.876079in}}%
\pgfpathlineto{\pgfqpoint{1.767387in}{0.875819in}}%
\pgfpathlineto{\pgfqpoint{1.782865in}{0.862468in}}%
\pgfpathlineto{\pgfqpoint{1.783044in}{0.862314in}}%
\pgfpathlineto{\pgfqpoint{1.798700in}{0.849253in}}%
\pgfpathlineto{\pgfqpoint{1.799200in}{0.848857in}}%
\pgfpathlineto{\pgfqpoint{1.814357in}{0.836700in}}%
\pgfpathlineto{\pgfqpoint{1.816312in}{0.835246in}}%
\pgfpathlineto{\pgfqpoint{1.830014in}{0.824788in}}%
\pgfpathlineto{\pgfqpoint{1.834608in}{0.821635in}}%
\pgfpathlineto{\pgfqpoint{1.845670in}{0.813738in}}%
\pgfpathlineto{\pgfqpoint{1.854941in}{0.808024in}}%
\pgfpathlineto{\pgfqpoint{1.861327in}{0.803873in}}%
\pgfpathlineto{\pgfqpoint{1.876983in}{0.795598in}}%
\pgfpathlineto{\pgfqpoint{1.880021in}{0.794413in}}%
\pgfpathlineto{\pgfqpoint{1.892640in}{0.789164in}}%
\pgfpathclose%
\pgfpathmoveto{\pgfqpoint{0.389609in}{1.458927in}}%
\pgfpathlineto{\pgfqpoint{0.398752in}{1.461357in}}%
\pgfpathlineto{\pgfqpoint{0.405266in}{1.462973in}}%
\pgfpathlineto{\pgfqpoint{0.420923in}{1.469288in}}%
\pgfpathlineto{\pgfqpoint{0.431364in}{1.474968in}}%
\pgfpathlineto{\pgfqpoint{0.436579in}{1.477644in}}%
\pgfpathlineto{\pgfqpoint{0.452236in}{1.487516in}}%
\pgfpathlineto{\pgfqpoint{0.453684in}{1.488579in}}%
\pgfpathlineto{\pgfqpoint{0.467892in}{1.498530in}}%
\pgfpathlineto{\pgfqpoint{0.472590in}{1.502191in}}%
\pgfpathlineto{\pgfqpoint{0.483549in}{1.510446in}}%
\pgfpathlineto{\pgfqpoint{0.490140in}{1.515802in}}%
\pgfpathlineto{\pgfqpoint{0.499205in}{1.523025in}}%
\pgfpathlineto{\pgfqpoint{0.506827in}{1.529413in}}%
\pgfpathlineto{\pgfqpoint{0.514862in}{1.536116in}}%
\pgfpathlineto{\pgfqpoint{0.522900in}{1.543024in}}%
\pgfpathlineto{\pgfqpoint{0.530519in}{1.549647in}}%
\pgfpathlineto{\pgfqpoint{0.538465in}{1.556635in}}%
\pgfpathlineto{\pgfqpoint{0.546175in}{1.563620in}}%
\pgfpathlineto{\pgfqpoint{0.553523in}{1.570246in}}%
\pgfpathlineto{\pgfqpoint{0.561832in}{1.578127in}}%
\pgfpathlineto{\pgfqpoint{0.567992in}{1.583857in}}%
\pgfpathlineto{\pgfqpoint{0.577488in}{1.593384in}}%
\pgfpathlineto{\pgfqpoint{0.581699in}{1.597468in}}%
\pgfpathlineto{\pgfqpoint{0.593145in}{1.609821in}}%
\pgfpathlineto{\pgfqpoint{0.594368in}{1.611079in}}%
\pgfpathlineto{\pgfqpoint{0.605723in}{1.624691in}}%
\pgfpathlineto{\pgfqpoint{0.608801in}{1.629224in}}%
\pgfpathlineto{\pgfqpoint{0.615336in}{1.638302in}}%
\pgfpathlineto{\pgfqpoint{0.622600in}{1.651913in}}%
\pgfpathlineto{\pgfqpoint{0.624458in}{1.657575in}}%
\pgfpathlineto{\pgfqpoint{0.627253in}{1.665524in}}%
\pgfpathlineto{\pgfqpoint{0.628868in}{1.679135in}}%
\pgfpathlineto{\pgfqpoint{0.624458in}{1.679135in}}%
\pgfpathlineto{\pgfqpoint{0.608801in}{1.679135in}}%
\pgfpathlineto{\pgfqpoint{0.593145in}{1.679135in}}%
\pgfpathlineto{\pgfqpoint{0.577488in}{1.679135in}}%
\pgfpathlineto{\pgfqpoint{0.561832in}{1.679135in}}%
\pgfpathlineto{\pgfqpoint{0.546175in}{1.679135in}}%
\pgfpathlineto{\pgfqpoint{0.530519in}{1.679135in}}%
\pgfpathlineto{\pgfqpoint{0.523403in}{1.679135in}}%
\pgfpathlineto{\pgfqpoint{0.522150in}{1.665524in}}%
\pgfpathlineto{\pgfqpoint{0.518444in}{1.651913in}}%
\pgfpathlineto{\pgfqpoint{0.514862in}{1.643824in}}%
\pgfpathlineto{\pgfqpoint{0.512297in}{1.638302in}}%
\pgfpathlineto{\pgfqpoint{0.503755in}{1.624691in}}%
\pgfpathlineto{\pgfqpoint{0.499205in}{1.618793in}}%
\pgfpathlineto{\pgfqpoint{0.492883in}{1.611079in}}%
\pgfpathlineto{\pgfqpoint{0.483549in}{1.601251in}}%
\pgfpathlineto{\pgfqpoint{0.479649in}{1.597468in}}%
\pgfpathlineto{\pgfqpoint{0.467892in}{1.587247in}}%
\pgfpathlineto{\pgfqpoint{0.463541in}{1.583857in}}%
\pgfpathlineto{\pgfqpoint{0.452236in}{1.575743in}}%
\pgfpathlineto{\pgfqpoint{0.443364in}{1.570246in}}%
\pgfpathlineto{\pgfqpoint{0.436579in}{1.566291in}}%
\pgfpathlineto{\pgfqpoint{0.420923in}{1.558865in}}%
\pgfpathlineto{\pgfqpoint{0.414570in}{1.556635in}}%
\pgfpathlineto{\pgfqpoint{0.405266in}{1.553521in}}%
\pgfpathlineto{\pgfqpoint{0.389609in}{1.550299in}}%
\pgfpathlineto{\pgfqpoint{0.373953in}{1.549210in}}%
\pgfpathlineto{\pgfqpoint{0.373953in}{1.543024in}}%
\pgfpathlineto{\pgfqpoint{0.373953in}{1.529413in}}%
\pgfpathlineto{\pgfqpoint{0.373953in}{1.515802in}}%
\pgfpathlineto{\pgfqpoint{0.373953in}{1.502191in}}%
\pgfpathlineto{\pgfqpoint{0.373953in}{1.488579in}}%
\pgfpathlineto{\pgfqpoint{0.373953in}{1.474968in}}%
\pgfpathlineto{\pgfqpoint{0.373953in}{1.461357in}}%
\pgfpathlineto{\pgfqpoint{0.373953in}{1.457524in}}%
\pgfpathlineto{\pgfqpoint{0.389609in}{1.458927in}}%
\pgfpathclose%
\pgfpathmoveto{\pgfqpoint{1.125468in}{1.460667in}}%
\pgfpathlineto{\pgfqpoint{1.141125in}{1.457875in}}%
\pgfpathlineto{\pgfqpoint{1.156781in}{1.457875in}}%
\pgfpathlineto{\pgfqpoint{1.172438in}{1.460667in}}%
\pgfpathlineto{\pgfqpoint{1.174400in}{1.461357in}}%
\pgfpathlineto{\pgfqpoint{1.188094in}{1.465842in}}%
\pgfpathlineto{\pgfqpoint{1.203751in}{1.473274in}}%
\pgfpathlineto{\pgfqpoint{1.206523in}{1.474968in}}%
\pgfpathlineto{\pgfqpoint{1.219407in}{1.482382in}}%
\pgfpathlineto{\pgfqpoint{1.228467in}{1.488579in}}%
\pgfpathlineto{\pgfqpoint{1.235064in}{1.492888in}}%
\pgfpathlineto{\pgfqpoint{1.247633in}{1.502191in}}%
\pgfpathlineto{\pgfqpoint{1.250721in}{1.504402in}}%
\pgfpathlineto{\pgfqpoint{1.265266in}{1.515802in}}%
\pgfpathlineto{\pgfqpoint{1.266377in}{1.516657in}}%
\pgfpathlineto{\pgfqpoint{1.281959in}{1.529413in}}%
\pgfpathlineto{\pgfqpoint{1.282034in}{1.529474in}}%
\pgfpathlineto{\pgfqpoint{1.297690in}{1.542764in}}%
\pgfpathlineto{\pgfqpoint{1.297990in}{1.543024in}}%
\pgfpathlineto{\pgfqpoint{1.313347in}{1.556479in}}%
\pgfpathlineto{\pgfqpoint{1.313524in}{1.556635in}}%
\pgfpathlineto{\pgfqpoint{1.328548in}{1.570246in}}%
\pgfpathlineto{\pgfqpoint{1.329003in}{1.570680in}}%
\pgfpathlineto{\pgfqpoint{1.342988in}{1.583857in}}%
\pgfpathlineto{\pgfqpoint{1.344660in}{1.585556in}}%
\pgfpathlineto{\pgfqpoint{1.356689in}{1.597468in}}%
\pgfpathlineto{\pgfqpoint{1.360317in}{1.601463in}}%
\pgfpathlineto{\pgfqpoint{1.369401in}{1.611079in}}%
\pgfpathlineto{\pgfqpoint{1.375973in}{1.619139in}}%
\pgfpathlineto{\pgfqpoint{1.380748in}{1.624691in}}%
\pgfpathlineto{\pgfqpoint{1.390266in}{1.638302in}}%
\pgfpathlineto{\pgfqpoint{1.391630in}{1.640943in}}%
\pgfpathlineto{\pgfqpoint{1.397667in}{1.651913in}}%
\pgfpathlineto{\pgfqpoint{1.402283in}{1.665524in}}%
\pgfpathlineto{\pgfqpoint{1.403843in}{1.679135in}}%
\pgfpathlineto{\pgfqpoint{1.391630in}{1.679135in}}%
\pgfpathlineto{\pgfqpoint{1.375973in}{1.679135in}}%
\pgfpathlineto{\pgfqpoint{1.360317in}{1.679135in}}%
\pgfpathlineto{\pgfqpoint{1.344660in}{1.679135in}}%
\pgfpathlineto{\pgfqpoint{1.329003in}{1.679135in}}%
\pgfpathlineto{\pgfqpoint{1.313347in}{1.679135in}}%
\pgfpathlineto{\pgfqpoint{1.298481in}{1.679135in}}%
\pgfpathlineto{\pgfqpoint{1.297690in}{1.670357in}}%
\pgfpathlineto{\pgfqpoint{1.297234in}{1.665524in}}%
\pgfpathlineto{\pgfqpoint{1.293427in}{1.651913in}}%
\pgfpathlineto{\pgfqpoint{1.287262in}{1.638302in}}%
\pgfpathlineto{\pgfqpoint{1.282034in}{1.629749in}}%
\pgfpathlineto{\pgfqpoint{1.278771in}{1.624691in}}%
\pgfpathlineto{\pgfqpoint{1.267990in}{1.611079in}}%
\pgfpathlineto{\pgfqpoint{1.266377in}{1.609330in}}%
\pgfpathlineto{\pgfqpoint{1.254656in}{1.597468in}}%
\pgfpathlineto{\pgfqpoint{1.250721in}{1.593895in}}%
\pgfpathlineto{\pgfqpoint{1.238579in}{1.583857in}}%
\pgfpathlineto{\pgfqpoint{1.235064in}{1.581177in}}%
\pgfpathlineto{\pgfqpoint{1.219407in}{1.570657in}}%
\pgfpathlineto{\pgfqpoint{1.218677in}{1.570246in}}%
\pgfpathlineto{\pgfqpoint{1.203751in}{1.562358in}}%
\pgfpathlineto{\pgfqpoint{1.189968in}{1.556635in}}%
\pgfpathlineto{\pgfqpoint{1.188094in}{1.555891in}}%
\pgfpathlineto{\pgfqpoint{1.172438in}{1.551650in}}%
\pgfpathlineto{\pgfqpoint{1.156781in}{1.549483in}}%
\pgfpathlineto{\pgfqpoint{1.141125in}{1.549483in}}%
\pgfpathlineto{\pgfqpoint{1.125468in}{1.551650in}}%
\pgfpathlineto{\pgfqpoint{1.109812in}{1.555891in}}%
\pgfpathlineto{\pgfqpoint{1.107937in}{1.556635in}}%
\pgfpathlineto{\pgfqpoint{1.094155in}{1.562358in}}%
\pgfpathlineto{\pgfqpoint{1.079228in}{1.570246in}}%
\pgfpathlineto{\pgfqpoint{1.078498in}{1.570657in}}%
\pgfpathlineto{\pgfqpoint{1.062842in}{1.581177in}}%
\pgfpathlineto{\pgfqpoint{1.059327in}{1.583857in}}%
\pgfpathlineto{\pgfqpoint{1.047185in}{1.593895in}}%
\pgfpathlineto{\pgfqpoint{1.043250in}{1.597468in}}%
\pgfpathlineto{\pgfqpoint{1.031529in}{1.609330in}}%
\pgfpathlineto{\pgfqpoint{1.029915in}{1.611079in}}%
\pgfpathlineto{\pgfqpoint{1.019134in}{1.624691in}}%
\pgfpathlineto{\pgfqpoint{1.015872in}{1.629749in}}%
\pgfpathlineto{\pgfqpoint{1.010644in}{1.638302in}}%
\pgfpathlineto{\pgfqpoint{1.004478in}{1.651913in}}%
\pgfpathlineto{\pgfqpoint{1.000672in}{1.665524in}}%
\pgfpathlineto{\pgfqpoint{1.000216in}{1.670357in}}%
\pgfpathlineto{\pgfqpoint{0.999425in}{1.679135in}}%
\pgfpathlineto{\pgfqpoint{0.984559in}{1.679135in}}%
\pgfpathlineto{\pgfqpoint{0.968902in}{1.679135in}}%
\pgfpathlineto{\pgfqpoint{0.953246in}{1.679135in}}%
\pgfpathlineto{\pgfqpoint{0.937589in}{1.679135in}}%
\pgfpathlineto{\pgfqpoint{0.921933in}{1.679135in}}%
\pgfpathlineto{\pgfqpoint{0.906276in}{1.679135in}}%
\pgfpathlineto{\pgfqpoint{0.894062in}{1.679135in}}%
\pgfpathlineto{\pgfqpoint{0.895623in}{1.665524in}}%
\pgfpathlineto{\pgfqpoint{0.900238in}{1.651913in}}%
\pgfpathlineto{\pgfqpoint{0.906276in}{1.640943in}}%
\pgfpathlineto{\pgfqpoint{0.907640in}{1.638302in}}%
\pgfpathlineto{\pgfqpoint{0.917158in}{1.624691in}}%
\pgfpathlineto{\pgfqpoint{0.921933in}{1.619139in}}%
\pgfpathlineto{\pgfqpoint{0.928505in}{1.611079in}}%
\pgfpathlineto{\pgfqpoint{0.937589in}{1.601463in}}%
\pgfpathlineto{\pgfqpoint{0.941216in}{1.597468in}}%
\pgfpathlineto{\pgfqpoint{0.953246in}{1.585556in}}%
\pgfpathlineto{\pgfqpoint{0.954918in}{1.583857in}}%
\pgfpathlineto{\pgfqpoint{0.968902in}{1.570680in}}%
\pgfpathlineto{\pgfqpoint{0.969358in}{1.570246in}}%
\pgfpathlineto{\pgfqpoint{0.984382in}{1.556635in}}%
\pgfpathlineto{\pgfqpoint{0.984559in}{1.556479in}}%
\pgfpathlineto{\pgfqpoint{0.999916in}{1.543024in}}%
\pgfpathlineto{\pgfqpoint{1.000216in}{1.542764in}}%
\pgfpathlineto{\pgfqpoint{1.015872in}{1.529474in}}%
\pgfpathlineto{\pgfqpoint{1.015947in}{1.529413in}}%
\pgfpathlineto{\pgfqpoint{1.031529in}{1.516657in}}%
\pgfpathlineto{\pgfqpoint{1.032640in}{1.515802in}}%
\pgfpathlineto{\pgfqpoint{1.047185in}{1.504402in}}%
\pgfpathlineto{\pgfqpoint{1.050273in}{1.502191in}}%
\pgfpathlineto{\pgfqpoint{1.062842in}{1.492888in}}%
\pgfpathlineto{\pgfqpoint{1.069438in}{1.488579in}}%
\pgfpathlineto{\pgfqpoint{1.078498in}{1.482382in}}%
\pgfpathlineto{\pgfqpoint{1.091383in}{1.474968in}}%
\pgfpathlineto{\pgfqpoint{1.094155in}{1.473274in}}%
\pgfpathlineto{\pgfqpoint{1.109812in}{1.465842in}}%
\pgfpathlineto{\pgfqpoint{1.123506in}{1.461357in}}%
\pgfpathlineto{\pgfqpoint{1.125468in}{1.460667in}}%
\pgfpathclose%
\pgfpathmoveto{\pgfqpoint{1.908296in}{1.458927in}}%
\pgfpathlineto{\pgfqpoint{1.923953in}{1.457524in}}%
\pgfpathlineto{\pgfqpoint{1.923953in}{1.461357in}}%
\pgfpathlineto{\pgfqpoint{1.923953in}{1.474968in}}%
\pgfpathlineto{\pgfqpoint{1.923953in}{1.488579in}}%
\pgfpathlineto{\pgfqpoint{1.923953in}{1.502191in}}%
\pgfpathlineto{\pgfqpoint{1.923953in}{1.515802in}}%
\pgfpathlineto{\pgfqpoint{1.923953in}{1.529413in}}%
\pgfpathlineto{\pgfqpoint{1.923953in}{1.543024in}}%
\pgfpathlineto{\pgfqpoint{1.923953in}{1.549210in}}%
\pgfpathlineto{\pgfqpoint{1.908296in}{1.550299in}}%
\pgfpathlineto{\pgfqpoint{1.892640in}{1.553521in}}%
\pgfpathlineto{\pgfqpoint{1.883336in}{1.556635in}}%
\pgfpathlineto{\pgfqpoint{1.876983in}{1.558865in}}%
\pgfpathlineto{\pgfqpoint{1.861327in}{1.566291in}}%
\pgfpathlineto{\pgfqpoint{1.854542in}{1.570246in}}%
\pgfpathlineto{\pgfqpoint{1.845670in}{1.575743in}}%
\pgfpathlineto{\pgfqpoint{1.834365in}{1.583857in}}%
\pgfpathlineto{\pgfqpoint{1.830014in}{1.587247in}}%
\pgfpathlineto{\pgfqpoint{1.818257in}{1.597468in}}%
\pgfpathlineto{\pgfqpoint{1.814357in}{1.601251in}}%
\pgfpathlineto{\pgfqpoint{1.805023in}{1.611079in}}%
\pgfpathlineto{\pgfqpoint{1.798700in}{1.618793in}}%
\pgfpathlineto{\pgfqpoint{1.794150in}{1.624691in}}%
\pgfpathlineto{\pgfqpoint{1.785609in}{1.638302in}}%
\pgfpathlineto{\pgfqpoint{1.783044in}{1.643824in}}%
\pgfpathlineto{\pgfqpoint{1.779461in}{1.651913in}}%
\pgfpathlineto{\pgfqpoint{1.775756in}{1.665524in}}%
\pgfpathlineto{\pgfqpoint{1.774503in}{1.679135in}}%
\pgfpathlineto{\pgfqpoint{1.767387in}{1.679135in}}%
\pgfpathlineto{\pgfqpoint{1.751731in}{1.679135in}}%
\pgfpathlineto{\pgfqpoint{1.736074in}{1.679135in}}%
\pgfpathlineto{\pgfqpoint{1.720418in}{1.679135in}}%
\pgfpathlineto{\pgfqpoint{1.704761in}{1.679135in}}%
\pgfpathlineto{\pgfqpoint{1.689104in}{1.679135in}}%
\pgfpathlineto{\pgfqpoint{1.673448in}{1.679135in}}%
\pgfpathlineto{\pgfqpoint{1.669038in}{1.679135in}}%
\pgfpathlineto{\pgfqpoint{1.670652in}{1.665524in}}%
\pgfpathlineto{\pgfqpoint{1.673448in}{1.657575in}}%
\pgfpathlineto{\pgfqpoint{1.675306in}{1.651913in}}%
\pgfpathlineto{\pgfqpoint{1.682570in}{1.638302in}}%
\pgfpathlineto{\pgfqpoint{1.689104in}{1.629224in}}%
\pgfpathlineto{\pgfqpoint{1.692182in}{1.624691in}}%
\pgfpathlineto{\pgfqpoint{1.703538in}{1.611079in}}%
\pgfpathlineto{\pgfqpoint{1.704761in}{1.609821in}}%
\pgfpathlineto{\pgfqpoint{1.716207in}{1.597468in}}%
\pgfpathlineto{\pgfqpoint{1.720418in}{1.593384in}}%
\pgfpathlineto{\pgfqpoint{1.729913in}{1.583857in}}%
\pgfpathlineto{\pgfqpoint{1.736074in}{1.578127in}}%
\pgfpathlineto{\pgfqpoint{1.744382in}{1.570246in}}%
\pgfpathlineto{\pgfqpoint{1.751731in}{1.563620in}}%
\pgfpathlineto{\pgfqpoint{1.759441in}{1.556635in}}%
\pgfpathlineto{\pgfqpoint{1.767387in}{1.549647in}}%
\pgfpathlineto{\pgfqpoint{1.775005in}{1.543024in}}%
\pgfpathlineto{\pgfqpoint{1.783044in}{1.536116in}}%
\pgfpathlineto{\pgfqpoint{1.791079in}{1.529413in}}%
\pgfpathlineto{\pgfqpoint{1.798700in}{1.523025in}}%
\pgfpathlineto{\pgfqpoint{1.807766in}{1.515802in}}%
\pgfpathlineto{\pgfqpoint{1.814357in}{1.510446in}}%
\pgfpathlineto{\pgfqpoint{1.825315in}{1.502191in}}%
\pgfpathlineto{\pgfqpoint{1.830014in}{1.498530in}}%
\pgfpathlineto{\pgfqpoint{1.844222in}{1.488579in}}%
\pgfpathlineto{\pgfqpoint{1.845670in}{1.487516in}}%
\pgfpathlineto{\pgfqpoint{1.861327in}{1.477644in}}%
\pgfpathlineto{\pgfqpoint{1.866541in}{1.474968in}}%
\pgfpathlineto{\pgfqpoint{1.876983in}{1.469288in}}%
\pgfpathlineto{\pgfqpoint{1.892640in}{1.462973in}}%
\pgfpathlineto{\pgfqpoint{1.899153in}{1.461357in}}%
\pgfpathlineto{\pgfqpoint{1.908296in}{1.458927in}}%
\pgfpathclose%
\pgfusepath{fill}%
\end{pgfscope}%
\begin{pgfscope}%
\pgfpathrectangle{\pgfqpoint{0.373953in}{0.331635in}}{\pgfqpoint{1.550000in}{1.347500in}}%
\pgfusepath{clip}%
\pgfsetbuttcap%
\pgfsetroundjoin%
\definecolor{currentfill}{rgb}{0.048062,0.036607,0.150327}%
\pgfsetfillcolor{currentfill}%
\pgfsetlinewidth{0.000000pt}%
\definecolor{currentstroke}{rgb}{0.000000,0.000000,0.000000}%
\pgfsetstrokecolor{currentstroke}%
\pgfsetdash{}{0pt}%
\pgfpathmoveto{\pgfqpoint{0.389609in}{0.331635in}}%
\pgfpathlineto{\pgfqpoint{0.405266in}{0.331635in}}%
\pgfpathlineto{\pgfqpoint{0.420923in}{0.331635in}}%
\pgfpathlineto{\pgfqpoint{0.436579in}{0.331635in}}%
\pgfpathlineto{\pgfqpoint{0.452236in}{0.331635in}}%
\pgfpathlineto{\pgfqpoint{0.467892in}{0.331635in}}%
\pgfpathlineto{\pgfqpoint{0.483549in}{0.331635in}}%
\pgfpathlineto{\pgfqpoint{0.499205in}{0.331635in}}%
\pgfpathlineto{\pgfqpoint{0.514862in}{0.331635in}}%
\pgfpathlineto{\pgfqpoint{0.523403in}{0.331635in}}%
\pgfpathlineto{\pgfqpoint{0.522150in}{0.345246in}}%
\pgfpathlineto{\pgfqpoint{0.518444in}{0.358857in}}%
\pgfpathlineto{\pgfqpoint{0.514862in}{0.366946in}}%
\pgfpathlineto{\pgfqpoint{0.512297in}{0.372468in}}%
\pgfpathlineto{\pgfqpoint{0.503755in}{0.386079in}}%
\pgfpathlineto{\pgfqpoint{0.499205in}{0.391977in}}%
\pgfpathlineto{\pgfqpoint{0.492883in}{0.399691in}}%
\pgfpathlineto{\pgfqpoint{0.483549in}{0.409519in}}%
\pgfpathlineto{\pgfqpoint{0.479649in}{0.413302in}}%
\pgfpathlineto{\pgfqpoint{0.467892in}{0.423523in}}%
\pgfpathlineto{\pgfqpoint{0.463541in}{0.426913in}}%
\pgfpathlineto{\pgfqpoint{0.452236in}{0.435027in}}%
\pgfpathlineto{\pgfqpoint{0.443364in}{0.440524in}}%
\pgfpathlineto{\pgfqpoint{0.436579in}{0.444479in}}%
\pgfpathlineto{\pgfqpoint{0.420923in}{0.451905in}}%
\pgfpathlineto{\pgfqpoint{0.414570in}{0.454135in}}%
\pgfpathlineto{\pgfqpoint{0.405266in}{0.457249in}}%
\pgfpathlineto{\pgfqpoint{0.389609in}{0.460471in}}%
\pgfpathlineto{\pgfqpoint{0.373953in}{0.461560in}}%
\pgfpathlineto{\pgfqpoint{0.373953in}{0.454135in}}%
\pgfpathlineto{\pgfqpoint{0.373953in}{0.440524in}}%
\pgfpathlineto{\pgfqpoint{0.373953in}{0.426913in}}%
\pgfpathlineto{\pgfqpoint{0.373953in}{0.413302in}}%
\pgfpathlineto{\pgfqpoint{0.373953in}{0.399691in}}%
\pgfpathlineto{\pgfqpoint{0.373953in}{0.386079in}}%
\pgfpathlineto{\pgfqpoint{0.373953in}{0.372468in}}%
\pgfpathlineto{\pgfqpoint{0.373953in}{0.358857in}}%
\pgfpathlineto{\pgfqpoint{0.373953in}{0.345246in}}%
\pgfpathlineto{\pgfqpoint{0.373953in}{0.331635in}}%
\pgfpathlineto{\pgfqpoint{0.389609in}{0.331635in}}%
\pgfpathclose%
\pgfpathmoveto{\pgfqpoint{1.000216in}{0.331635in}}%
\pgfpathlineto{\pgfqpoint{1.015872in}{0.331635in}}%
\pgfpathlineto{\pgfqpoint{1.031529in}{0.331635in}}%
\pgfpathlineto{\pgfqpoint{1.047185in}{0.331635in}}%
\pgfpathlineto{\pgfqpoint{1.062842in}{0.331635in}}%
\pgfpathlineto{\pgfqpoint{1.078498in}{0.331635in}}%
\pgfpathlineto{\pgfqpoint{1.094155in}{0.331635in}}%
\pgfpathlineto{\pgfqpoint{1.109812in}{0.331635in}}%
\pgfpathlineto{\pgfqpoint{1.125468in}{0.331635in}}%
\pgfpathlineto{\pgfqpoint{1.141125in}{0.331635in}}%
\pgfpathlineto{\pgfqpoint{1.156781in}{0.331635in}}%
\pgfpathlineto{\pgfqpoint{1.172438in}{0.331635in}}%
\pgfpathlineto{\pgfqpoint{1.188094in}{0.331635in}}%
\pgfpathlineto{\pgfqpoint{1.203751in}{0.331635in}}%
\pgfpathlineto{\pgfqpoint{1.219407in}{0.331635in}}%
\pgfpathlineto{\pgfqpoint{1.235064in}{0.331635in}}%
\pgfpathlineto{\pgfqpoint{1.250721in}{0.331635in}}%
\pgfpathlineto{\pgfqpoint{1.266377in}{0.331635in}}%
\pgfpathlineto{\pgfqpoint{1.282034in}{0.331635in}}%
\pgfpathlineto{\pgfqpoint{1.297690in}{0.331635in}}%
\pgfpathlineto{\pgfqpoint{1.298481in}{0.331635in}}%
\pgfpathlineto{\pgfqpoint{1.297690in}{0.340413in}}%
\pgfpathlineto{\pgfqpoint{1.297234in}{0.345246in}}%
\pgfpathlineto{\pgfqpoint{1.293427in}{0.358857in}}%
\pgfpathlineto{\pgfqpoint{1.287262in}{0.372468in}}%
\pgfpathlineto{\pgfqpoint{1.282034in}{0.381021in}}%
\pgfpathlineto{\pgfqpoint{1.278771in}{0.386079in}}%
\pgfpathlineto{\pgfqpoint{1.267990in}{0.399691in}}%
\pgfpathlineto{\pgfqpoint{1.266377in}{0.401440in}}%
\pgfpathlineto{\pgfqpoint{1.254656in}{0.413302in}}%
\pgfpathlineto{\pgfqpoint{1.250721in}{0.416875in}}%
\pgfpathlineto{\pgfqpoint{1.238579in}{0.426913in}}%
\pgfpathlineto{\pgfqpoint{1.235064in}{0.429593in}}%
\pgfpathlineto{\pgfqpoint{1.219407in}{0.440113in}}%
\pgfpathlineto{\pgfqpoint{1.218677in}{0.440524in}}%
\pgfpathlineto{\pgfqpoint{1.203751in}{0.448412in}}%
\pgfpathlineto{\pgfqpoint{1.189968in}{0.454135in}}%
\pgfpathlineto{\pgfqpoint{1.188094in}{0.454879in}}%
\pgfpathlineto{\pgfqpoint{1.172438in}{0.459120in}}%
\pgfpathlineto{\pgfqpoint{1.156781in}{0.461287in}}%
\pgfpathlineto{\pgfqpoint{1.141125in}{0.461287in}}%
\pgfpathlineto{\pgfqpoint{1.125468in}{0.459120in}}%
\pgfpathlineto{\pgfqpoint{1.109812in}{0.454879in}}%
\pgfpathlineto{\pgfqpoint{1.107937in}{0.454135in}}%
\pgfpathlineto{\pgfqpoint{1.094155in}{0.448412in}}%
\pgfpathlineto{\pgfqpoint{1.079228in}{0.440524in}}%
\pgfpathlineto{\pgfqpoint{1.078498in}{0.440113in}}%
\pgfpathlineto{\pgfqpoint{1.062842in}{0.429593in}}%
\pgfpathlineto{\pgfqpoint{1.059327in}{0.426913in}}%
\pgfpathlineto{\pgfqpoint{1.047185in}{0.416875in}}%
\pgfpathlineto{\pgfqpoint{1.043250in}{0.413302in}}%
\pgfpathlineto{\pgfqpoint{1.031529in}{0.401440in}}%
\pgfpathlineto{\pgfqpoint{1.029915in}{0.399691in}}%
\pgfpathlineto{\pgfqpoint{1.019134in}{0.386079in}}%
\pgfpathlineto{\pgfqpoint{1.015872in}{0.381021in}}%
\pgfpathlineto{\pgfqpoint{1.010644in}{0.372468in}}%
\pgfpathlineto{\pgfqpoint{1.004478in}{0.358857in}}%
\pgfpathlineto{\pgfqpoint{1.000672in}{0.345246in}}%
\pgfpathlineto{\pgfqpoint{1.000216in}{0.340413in}}%
\pgfpathlineto{\pgfqpoint{0.999425in}{0.331635in}}%
\pgfpathlineto{\pgfqpoint{1.000216in}{0.331635in}}%
\pgfpathclose%
\pgfpathmoveto{\pgfqpoint{1.783044in}{0.331635in}}%
\pgfpathlineto{\pgfqpoint{1.798700in}{0.331635in}}%
\pgfpathlineto{\pgfqpoint{1.814357in}{0.331635in}}%
\pgfpathlineto{\pgfqpoint{1.830014in}{0.331635in}}%
\pgfpathlineto{\pgfqpoint{1.845670in}{0.331635in}}%
\pgfpathlineto{\pgfqpoint{1.861327in}{0.331635in}}%
\pgfpathlineto{\pgfqpoint{1.876983in}{0.331635in}}%
\pgfpathlineto{\pgfqpoint{1.892640in}{0.331635in}}%
\pgfpathlineto{\pgfqpoint{1.908296in}{0.331635in}}%
\pgfpathlineto{\pgfqpoint{1.923953in}{0.331635in}}%
\pgfpathlineto{\pgfqpoint{1.923953in}{0.345246in}}%
\pgfpathlineto{\pgfqpoint{1.923953in}{0.358857in}}%
\pgfpathlineto{\pgfqpoint{1.923953in}{0.372468in}}%
\pgfpathlineto{\pgfqpoint{1.923953in}{0.386079in}}%
\pgfpathlineto{\pgfqpoint{1.923953in}{0.399691in}}%
\pgfpathlineto{\pgfqpoint{1.923953in}{0.413302in}}%
\pgfpathlineto{\pgfqpoint{1.923953in}{0.426913in}}%
\pgfpathlineto{\pgfqpoint{1.923953in}{0.440524in}}%
\pgfpathlineto{\pgfqpoint{1.923953in}{0.454135in}}%
\pgfpathlineto{\pgfqpoint{1.923953in}{0.461560in}}%
\pgfpathlineto{\pgfqpoint{1.908296in}{0.460471in}}%
\pgfpathlineto{\pgfqpoint{1.892640in}{0.457249in}}%
\pgfpathlineto{\pgfqpoint{1.883336in}{0.454135in}}%
\pgfpathlineto{\pgfqpoint{1.876983in}{0.451905in}}%
\pgfpathlineto{\pgfqpoint{1.861327in}{0.444479in}}%
\pgfpathlineto{\pgfqpoint{1.854542in}{0.440524in}}%
\pgfpathlineto{\pgfqpoint{1.845670in}{0.435027in}}%
\pgfpathlineto{\pgfqpoint{1.834365in}{0.426913in}}%
\pgfpathlineto{\pgfqpoint{1.830014in}{0.423523in}}%
\pgfpathlineto{\pgfqpoint{1.818257in}{0.413302in}}%
\pgfpathlineto{\pgfqpoint{1.814357in}{0.409519in}}%
\pgfpathlineto{\pgfqpoint{1.805023in}{0.399691in}}%
\pgfpathlineto{\pgfqpoint{1.798700in}{0.391977in}}%
\pgfpathlineto{\pgfqpoint{1.794150in}{0.386079in}}%
\pgfpathlineto{\pgfqpoint{1.785609in}{0.372468in}}%
\pgfpathlineto{\pgfqpoint{1.783044in}{0.366946in}}%
\pgfpathlineto{\pgfqpoint{1.779461in}{0.358857in}}%
\pgfpathlineto{\pgfqpoint{1.775756in}{0.345246in}}%
\pgfpathlineto{\pgfqpoint{1.774503in}{0.331635in}}%
\pgfpathlineto{\pgfqpoint{1.783044in}{0.331635in}}%
\pgfpathclose%
\pgfpathmoveto{\pgfqpoint{0.384050in}{0.876079in}}%
\pgfpathlineto{\pgfqpoint{0.389609in}{0.876476in}}%
\pgfpathlineto{\pgfqpoint{0.405266in}{0.879785in}}%
\pgfpathlineto{\pgfqpoint{0.420923in}{0.885145in}}%
\pgfpathlineto{\pgfqpoint{0.430761in}{0.889691in}}%
\pgfpathlineto{\pgfqpoint{0.436579in}{0.892527in}}%
\pgfpathlineto{\pgfqpoint{0.452236in}{0.901899in}}%
\pgfpathlineto{\pgfqpoint{0.454249in}{0.903302in}}%
\pgfpathlineto{\pgfqpoint{0.467892in}{0.913491in}}%
\pgfpathlineto{\pgfqpoint{0.472002in}{0.916913in}}%
\pgfpathlineto{\pgfqpoint{0.483549in}{0.927468in}}%
\pgfpathlineto{\pgfqpoint{0.486632in}{0.930524in}}%
\pgfpathlineto{\pgfqpoint{0.498733in}{0.944135in}}%
\pgfpathlineto{\pgfqpoint{0.499205in}{0.944770in}}%
\pgfpathlineto{\pgfqpoint{0.508279in}{0.957746in}}%
\pgfpathlineto{\pgfqpoint{0.514862in}{0.969728in}}%
\pgfpathlineto{\pgfqpoint{0.515717in}{0.971357in}}%
\pgfpathlineto{\pgfqpoint{0.520596in}{0.984968in}}%
\pgfpathlineto{\pgfqpoint{0.523089in}{0.998579in}}%
\pgfpathlineto{\pgfqpoint{0.523089in}{1.012191in}}%
\pgfpathlineto{\pgfqpoint{0.520596in}{1.025802in}}%
\pgfpathlineto{\pgfqpoint{0.515717in}{1.039413in}}%
\pgfpathlineto{\pgfqpoint{0.514862in}{1.041042in}}%
\pgfpathlineto{\pgfqpoint{0.508279in}{1.053024in}}%
\pgfpathlineto{\pgfqpoint{0.499205in}{1.066000in}}%
\pgfpathlineto{\pgfqpoint{0.498733in}{1.066635in}}%
\pgfpathlineto{\pgfqpoint{0.486632in}{1.080246in}}%
\pgfpathlineto{\pgfqpoint{0.483549in}{1.083302in}}%
\pgfpathlineto{\pgfqpoint{0.472002in}{1.093857in}}%
\pgfpathlineto{\pgfqpoint{0.467892in}{1.097279in}}%
\pgfpathlineto{\pgfqpoint{0.454249in}{1.107468in}}%
\pgfpathlineto{\pgfqpoint{0.452236in}{1.108871in}}%
\pgfpathlineto{\pgfqpoint{0.436579in}{1.118243in}}%
\pgfpathlineto{\pgfqpoint{0.430761in}{1.121079in}}%
\pgfpathlineto{\pgfqpoint{0.420923in}{1.125625in}}%
\pgfpathlineto{\pgfqpoint{0.405266in}{1.130985in}}%
\pgfpathlineto{\pgfqpoint{0.389609in}{1.134294in}}%
\pgfpathlineto{\pgfqpoint{0.384050in}{1.134691in}}%
\pgfpathlineto{\pgfqpoint{0.373953in}{1.135378in}}%
\pgfpathlineto{\pgfqpoint{0.373953in}{1.134691in}}%
\pgfpathlineto{\pgfqpoint{0.373953in}{1.121079in}}%
\pgfpathlineto{\pgfqpoint{0.373953in}{1.107468in}}%
\pgfpathlineto{\pgfqpoint{0.373953in}{1.093857in}}%
\pgfpathlineto{\pgfqpoint{0.373953in}{1.080246in}}%
\pgfpathlineto{\pgfqpoint{0.373953in}{1.066635in}}%
\pgfpathlineto{\pgfqpoint{0.373953in}{1.053024in}}%
\pgfpathlineto{\pgfqpoint{0.373953in}{1.039413in}}%
\pgfpathlineto{\pgfqpoint{0.373953in}{1.025802in}}%
\pgfpathlineto{\pgfqpoint{0.373953in}{1.012191in}}%
\pgfpathlineto{\pgfqpoint{0.373953in}{0.998579in}}%
\pgfpathlineto{\pgfqpoint{0.373953in}{0.984968in}}%
\pgfpathlineto{\pgfqpoint{0.373953in}{0.971357in}}%
\pgfpathlineto{\pgfqpoint{0.373953in}{0.957746in}}%
\pgfpathlineto{\pgfqpoint{0.373953in}{0.944135in}}%
\pgfpathlineto{\pgfqpoint{0.373953in}{0.930524in}}%
\pgfpathlineto{\pgfqpoint{0.373953in}{0.916913in}}%
\pgfpathlineto{\pgfqpoint{0.373953in}{0.903302in}}%
\pgfpathlineto{\pgfqpoint{0.373953in}{0.889691in}}%
\pgfpathlineto{\pgfqpoint{0.373953in}{0.876079in}}%
\pgfpathlineto{\pgfqpoint{0.373953in}{0.875392in}}%
\pgfpathlineto{\pgfqpoint{0.384050in}{0.876079in}}%
\pgfpathclose%
\pgfpathmoveto{\pgfqpoint{1.141125in}{0.875659in}}%
\pgfpathlineto{\pgfqpoint{1.156781in}{0.875659in}}%
\pgfpathlineto{\pgfqpoint{1.159877in}{0.876079in}}%
\pgfpathlineto{\pgfqpoint{1.172438in}{0.877864in}}%
\pgfpathlineto{\pgfqpoint{1.188094in}{0.882221in}}%
\pgfpathlineto{\pgfqpoint{1.203751in}{0.888529in}}%
\pgfpathlineto{\pgfqpoint{1.205989in}{0.889691in}}%
\pgfpathlineto{\pgfqpoint{1.219407in}{0.897025in}}%
\pgfpathlineto{\pgfqpoint{1.229080in}{0.903302in}}%
\pgfpathlineto{\pgfqpoint{1.235064in}{0.907470in}}%
\pgfpathlineto{\pgfqpoint{1.247007in}{0.916913in}}%
\pgfpathlineto{\pgfqpoint{1.250721in}{0.920141in}}%
\pgfpathlineto{\pgfqpoint{1.261583in}{0.930524in}}%
\pgfpathlineto{\pgfqpoint{1.266377in}{0.935726in}}%
\pgfpathlineto{\pgfqpoint{1.273597in}{0.944135in}}%
\pgfpathlineto{\pgfqpoint{1.282034in}{0.955800in}}%
\pgfpathlineto{\pgfqpoint{1.283370in}{0.957746in}}%
\pgfpathlineto{\pgfqpoint{1.290626in}{0.971357in}}%
\pgfpathlineto{\pgfqpoint{1.295638in}{0.984968in}}%
\pgfpathlineto{\pgfqpoint{1.297690in}{0.995888in}}%
\pgfpathlineto{\pgfqpoint{1.298174in}{0.998579in}}%
\pgfpathlineto{\pgfqpoint{1.298174in}{1.012191in}}%
\pgfpathlineto{\pgfqpoint{1.297690in}{1.014882in}}%
\pgfpathlineto{\pgfqpoint{1.295638in}{1.025802in}}%
\pgfpathlineto{\pgfqpoint{1.290626in}{1.039413in}}%
\pgfpathlineto{\pgfqpoint{1.283370in}{1.053024in}}%
\pgfpathlineto{\pgfqpoint{1.282034in}{1.054970in}}%
\pgfpathlineto{\pgfqpoint{1.273597in}{1.066635in}}%
\pgfpathlineto{\pgfqpoint{1.266377in}{1.075044in}}%
\pgfpathlineto{\pgfqpoint{1.261583in}{1.080246in}}%
\pgfpathlineto{\pgfqpoint{1.250721in}{1.090629in}}%
\pgfpathlineto{\pgfqpoint{1.247007in}{1.093857in}}%
\pgfpathlineto{\pgfqpoint{1.235064in}{1.103300in}}%
\pgfpathlineto{\pgfqpoint{1.229080in}{1.107468in}}%
\pgfpathlineto{\pgfqpoint{1.219407in}{1.113745in}}%
\pgfpathlineto{\pgfqpoint{1.205989in}{1.121079in}}%
\pgfpathlineto{\pgfqpoint{1.203751in}{1.122241in}}%
\pgfpathlineto{\pgfqpoint{1.188094in}{1.128549in}}%
\pgfpathlineto{\pgfqpoint{1.172438in}{1.132906in}}%
\pgfpathlineto{\pgfqpoint{1.159877in}{1.134691in}}%
\pgfpathlineto{\pgfqpoint{1.156781in}{1.135111in}}%
\pgfpathlineto{\pgfqpoint{1.141125in}{1.135111in}}%
\pgfpathlineto{\pgfqpoint{1.138029in}{1.134691in}}%
\pgfpathlineto{\pgfqpoint{1.125468in}{1.132906in}}%
\pgfpathlineto{\pgfqpoint{1.109812in}{1.128549in}}%
\pgfpathlineto{\pgfqpoint{1.094155in}{1.122241in}}%
\pgfpathlineto{\pgfqpoint{1.091916in}{1.121079in}}%
\pgfpathlineto{\pgfqpoint{1.078498in}{1.113745in}}%
\pgfpathlineto{\pgfqpoint{1.068826in}{1.107468in}}%
\pgfpathlineto{\pgfqpoint{1.062842in}{1.103300in}}%
\pgfpathlineto{\pgfqpoint{1.050899in}{1.093857in}}%
\pgfpathlineto{\pgfqpoint{1.047185in}{1.090629in}}%
\pgfpathlineto{\pgfqpoint{1.036323in}{1.080246in}}%
\pgfpathlineto{\pgfqpoint{1.031529in}{1.075044in}}%
\pgfpathlineto{\pgfqpoint{1.024309in}{1.066635in}}%
\pgfpathlineto{\pgfqpoint{1.015872in}{1.054970in}}%
\pgfpathlineto{\pgfqpoint{1.014536in}{1.053024in}}%
\pgfpathlineto{\pgfqpoint{1.007280in}{1.039413in}}%
\pgfpathlineto{\pgfqpoint{1.002268in}{1.025802in}}%
\pgfpathlineto{\pgfqpoint{1.000216in}{1.014882in}}%
\pgfpathlineto{\pgfqpoint{0.999732in}{1.012191in}}%
\pgfpathlineto{\pgfqpoint{0.999732in}{0.998579in}}%
\pgfpathlineto{\pgfqpoint{1.000216in}{0.995888in}}%
\pgfpathlineto{\pgfqpoint{1.002268in}{0.984968in}}%
\pgfpathlineto{\pgfqpoint{1.007280in}{0.971357in}}%
\pgfpathlineto{\pgfqpoint{1.014536in}{0.957746in}}%
\pgfpathlineto{\pgfqpoint{1.015872in}{0.955800in}}%
\pgfpathlineto{\pgfqpoint{1.024309in}{0.944135in}}%
\pgfpathlineto{\pgfqpoint{1.031529in}{0.935726in}}%
\pgfpathlineto{\pgfqpoint{1.036323in}{0.930524in}}%
\pgfpathlineto{\pgfqpoint{1.047185in}{0.920141in}}%
\pgfpathlineto{\pgfqpoint{1.050899in}{0.916913in}}%
\pgfpathlineto{\pgfqpoint{1.062842in}{0.907470in}}%
\pgfpathlineto{\pgfqpoint{1.068826in}{0.903302in}}%
\pgfpathlineto{\pgfqpoint{1.078498in}{0.897025in}}%
\pgfpathlineto{\pgfqpoint{1.091916in}{0.889691in}}%
\pgfpathlineto{\pgfqpoint{1.094155in}{0.888529in}}%
\pgfpathlineto{\pgfqpoint{1.109812in}{0.882221in}}%
\pgfpathlineto{\pgfqpoint{1.125468in}{0.877864in}}%
\pgfpathlineto{\pgfqpoint{1.138029in}{0.876079in}}%
\pgfpathlineto{\pgfqpoint{1.141125in}{0.875659in}}%
\pgfpathclose%
\pgfpathmoveto{\pgfqpoint{1.923953in}{0.875392in}}%
\pgfpathlineto{\pgfqpoint{1.923953in}{0.876079in}}%
\pgfpathlineto{\pgfqpoint{1.923953in}{0.889691in}}%
\pgfpathlineto{\pgfqpoint{1.923953in}{0.903302in}}%
\pgfpathlineto{\pgfqpoint{1.923953in}{0.916913in}}%
\pgfpathlineto{\pgfqpoint{1.923953in}{0.930524in}}%
\pgfpathlineto{\pgfqpoint{1.923953in}{0.944135in}}%
\pgfpathlineto{\pgfqpoint{1.923953in}{0.957746in}}%
\pgfpathlineto{\pgfqpoint{1.923953in}{0.971357in}}%
\pgfpathlineto{\pgfqpoint{1.923953in}{0.984968in}}%
\pgfpathlineto{\pgfqpoint{1.923953in}{0.998579in}}%
\pgfpathlineto{\pgfqpoint{1.923953in}{1.012191in}}%
\pgfpathlineto{\pgfqpoint{1.923953in}{1.025802in}}%
\pgfpathlineto{\pgfqpoint{1.923953in}{1.039413in}}%
\pgfpathlineto{\pgfqpoint{1.923953in}{1.053024in}}%
\pgfpathlineto{\pgfqpoint{1.923953in}{1.066635in}}%
\pgfpathlineto{\pgfqpoint{1.923953in}{1.080246in}}%
\pgfpathlineto{\pgfqpoint{1.923953in}{1.093857in}}%
\pgfpathlineto{\pgfqpoint{1.923953in}{1.107468in}}%
\pgfpathlineto{\pgfqpoint{1.923953in}{1.121079in}}%
\pgfpathlineto{\pgfqpoint{1.923953in}{1.134691in}}%
\pgfpathlineto{\pgfqpoint{1.923953in}{1.135378in}}%
\pgfpathlineto{\pgfqpoint{1.913856in}{1.134691in}}%
\pgfpathlineto{\pgfqpoint{1.908296in}{1.134294in}}%
\pgfpathlineto{\pgfqpoint{1.892640in}{1.130985in}}%
\pgfpathlineto{\pgfqpoint{1.876983in}{1.125625in}}%
\pgfpathlineto{\pgfqpoint{1.867145in}{1.121079in}}%
\pgfpathlineto{\pgfqpoint{1.861327in}{1.118243in}}%
\pgfpathlineto{\pgfqpoint{1.845670in}{1.108871in}}%
\pgfpathlineto{\pgfqpoint{1.843657in}{1.107468in}}%
\pgfpathlineto{\pgfqpoint{1.830014in}{1.097279in}}%
\pgfpathlineto{\pgfqpoint{1.825904in}{1.093857in}}%
\pgfpathlineto{\pgfqpoint{1.814357in}{1.083302in}}%
\pgfpathlineto{\pgfqpoint{1.811274in}{1.080246in}}%
\pgfpathlineto{\pgfqpoint{1.799173in}{1.066635in}}%
\pgfpathlineto{\pgfqpoint{1.798700in}{1.066000in}}%
\pgfpathlineto{\pgfqpoint{1.789627in}{1.053024in}}%
\pgfpathlineto{\pgfqpoint{1.783044in}{1.041042in}}%
\pgfpathlineto{\pgfqpoint{1.782188in}{1.039413in}}%
\pgfpathlineto{\pgfqpoint{1.777310in}{1.025802in}}%
\pgfpathlineto{\pgfqpoint{1.774817in}{1.012191in}}%
\pgfpathlineto{\pgfqpoint{1.774817in}{0.998579in}}%
\pgfpathlineto{\pgfqpoint{1.777310in}{0.984968in}}%
\pgfpathlineto{\pgfqpoint{1.782188in}{0.971357in}}%
\pgfpathlineto{\pgfqpoint{1.783044in}{0.969728in}}%
\pgfpathlineto{\pgfqpoint{1.789627in}{0.957746in}}%
\pgfpathlineto{\pgfqpoint{1.798700in}{0.944770in}}%
\pgfpathlineto{\pgfqpoint{1.799173in}{0.944135in}}%
\pgfpathlineto{\pgfqpoint{1.811274in}{0.930524in}}%
\pgfpathlineto{\pgfqpoint{1.814357in}{0.927468in}}%
\pgfpathlineto{\pgfqpoint{1.825904in}{0.916913in}}%
\pgfpathlineto{\pgfqpoint{1.830014in}{0.913491in}}%
\pgfpathlineto{\pgfqpoint{1.843657in}{0.903302in}}%
\pgfpathlineto{\pgfqpoint{1.845670in}{0.901899in}}%
\pgfpathlineto{\pgfqpoint{1.861327in}{0.892527in}}%
\pgfpathlineto{\pgfqpoint{1.867145in}{0.889691in}}%
\pgfpathlineto{\pgfqpoint{1.876983in}{0.885145in}}%
\pgfpathlineto{\pgfqpoint{1.892640in}{0.879785in}}%
\pgfpathlineto{\pgfqpoint{1.908296in}{0.876476in}}%
\pgfpathlineto{\pgfqpoint{1.913856in}{0.876079in}}%
\pgfpathlineto{\pgfqpoint{1.923953in}{0.875392in}}%
\pgfpathclose%
\pgfpathmoveto{\pgfqpoint{0.389609in}{1.550299in}}%
\pgfpathlineto{\pgfqpoint{0.405266in}{1.553521in}}%
\pgfpathlineto{\pgfqpoint{0.414570in}{1.556635in}}%
\pgfpathlineto{\pgfqpoint{0.420923in}{1.558865in}}%
\pgfpathlineto{\pgfqpoint{0.436579in}{1.566291in}}%
\pgfpathlineto{\pgfqpoint{0.443364in}{1.570246in}}%
\pgfpathlineto{\pgfqpoint{0.452236in}{1.575743in}}%
\pgfpathlineto{\pgfqpoint{0.463541in}{1.583857in}}%
\pgfpathlineto{\pgfqpoint{0.467892in}{1.587247in}}%
\pgfpathlineto{\pgfqpoint{0.479649in}{1.597468in}}%
\pgfpathlineto{\pgfqpoint{0.483549in}{1.601251in}}%
\pgfpathlineto{\pgfqpoint{0.492883in}{1.611079in}}%
\pgfpathlineto{\pgfqpoint{0.499205in}{1.618793in}}%
\pgfpathlineto{\pgfqpoint{0.503755in}{1.624691in}}%
\pgfpathlineto{\pgfqpoint{0.512297in}{1.638302in}}%
\pgfpathlineto{\pgfqpoint{0.514862in}{1.643824in}}%
\pgfpathlineto{\pgfqpoint{0.518444in}{1.651913in}}%
\pgfpathlineto{\pgfqpoint{0.522150in}{1.665524in}}%
\pgfpathlineto{\pgfqpoint{0.523403in}{1.679135in}}%
\pgfpathlineto{\pgfqpoint{0.514862in}{1.679135in}}%
\pgfpathlineto{\pgfqpoint{0.499205in}{1.679135in}}%
\pgfpathlineto{\pgfqpoint{0.483549in}{1.679135in}}%
\pgfpathlineto{\pgfqpoint{0.467892in}{1.679135in}}%
\pgfpathlineto{\pgfqpoint{0.452236in}{1.679135in}}%
\pgfpathlineto{\pgfqpoint{0.436579in}{1.679135in}}%
\pgfpathlineto{\pgfqpoint{0.420923in}{1.679135in}}%
\pgfpathlineto{\pgfqpoint{0.405266in}{1.679135in}}%
\pgfpathlineto{\pgfqpoint{0.389609in}{1.679135in}}%
\pgfpathlineto{\pgfqpoint{0.373953in}{1.679135in}}%
\pgfpathlineto{\pgfqpoint{0.373953in}{1.665524in}}%
\pgfpathlineto{\pgfqpoint{0.373953in}{1.651913in}}%
\pgfpathlineto{\pgfqpoint{0.373953in}{1.638302in}}%
\pgfpathlineto{\pgfqpoint{0.373953in}{1.624691in}}%
\pgfpathlineto{\pgfqpoint{0.373953in}{1.611079in}}%
\pgfpathlineto{\pgfqpoint{0.373953in}{1.597468in}}%
\pgfpathlineto{\pgfqpoint{0.373953in}{1.583857in}}%
\pgfpathlineto{\pgfqpoint{0.373953in}{1.570246in}}%
\pgfpathlineto{\pgfqpoint{0.373953in}{1.556635in}}%
\pgfpathlineto{\pgfqpoint{0.373953in}{1.549210in}}%
\pgfpathlineto{\pgfqpoint{0.389609in}{1.550299in}}%
\pgfpathclose%
\pgfpathmoveto{\pgfqpoint{1.109812in}{1.555891in}}%
\pgfpathlineto{\pgfqpoint{1.125468in}{1.551650in}}%
\pgfpathlineto{\pgfqpoint{1.141125in}{1.549483in}}%
\pgfpathlineto{\pgfqpoint{1.156781in}{1.549483in}}%
\pgfpathlineto{\pgfqpoint{1.172438in}{1.551650in}}%
\pgfpathlineto{\pgfqpoint{1.188094in}{1.555891in}}%
\pgfpathlineto{\pgfqpoint{1.189968in}{1.556635in}}%
\pgfpathlineto{\pgfqpoint{1.203751in}{1.562358in}}%
\pgfpathlineto{\pgfqpoint{1.218677in}{1.570246in}}%
\pgfpathlineto{\pgfqpoint{1.219407in}{1.570657in}}%
\pgfpathlineto{\pgfqpoint{1.235064in}{1.581177in}}%
\pgfpathlineto{\pgfqpoint{1.238579in}{1.583857in}}%
\pgfpathlineto{\pgfqpoint{1.250721in}{1.593895in}}%
\pgfpathlineto{\pgfqpoint{1.254656in}{1.597468in}}%
\pgfpathlineto{\pgfqpoint{1.266377in}{1.609330in}}%
\pgfpathlineto{\pgfqpoint{1.267990in}{1.611079in}}%
\pgfpathlineto{\pgfqpoint{1.278771in}{1.624691in}}%
\pgfpathlineto{\pgfqpoint{1.282034in}{1.629749in}}%
\pgfpathlineto{\pgfqpoint{1.287262in}{1.638302in}}%
\pgfpathlineto{\pgfqpoint{1.293427in}{1.651913in}}%
\pgfpathlineto{\pgfqpoint{1.297234in}{1.665524in}}%
\pgfpathlineto{\pgfqpoint{1.297690in}{1.670357in}}%
\pgfpathlineto{\pgfqpoint{1.298481in}{1.679135in}}%
\pgfpathlineto{\pgfqpoint{1.297690in}{1.679135in}}%
\pgfpathlineto{\pgfqpoint{1.282034in}{1.679135in}}%
\pgfpathlineto{\pgfqpoint{1.266377in}{1.679135in}}%
\pgfpathlineto{\pgfqpoint{1.250721in}{1.679135in}}%
\pgfpathlineto{\pgfqpoint{1.235064in}{1.679135in}}%
\pgfpathlineto{\pgfqpoint{1.219407in}{1.679135in}}%
\pgfpathlineto{\pgfqpoint{1.203751in}{1.679135in}}%
\pgfpathlineto{\pgfqpoint{1.188094in}{1.679135in}}%
\pgfpathlineto{\pgfqpoint{1.172438in}{1.679135in}}%
\pgfpathlineto{\pgfqpoint{1.156781in}{1.679135in}}%
\pgfpathlineto{\pgfqpoint{1.141125in}{1.679135in}}%
\pgfpathlineto{\pgfqpoint{1.125468in}{1.679135in}}%
\pgfpathlineto{\pgfqpoint{1.109812in}{1.679135in}}%
\pgfpathlineto{\pgfqpoint{1.094155in}{1.679135in}}%
\pgfpathlineto{\pgfqpoint{1.078498in}{1.679135in}}%
\pgfpathlineto{\pgfqpoint{1.062842in}{1.679135in}}%
\pgfpathlineto{\pgfqpoint{1.047185in}{1.679135in}}%
\pgfpathlineto{\pgfqpoint{1.031529in}{1.679135in}}%
\pgfpathlineto{\pgfqpoint{1.015872in}{1.679135in}}%
\pgfpathlineto{\pgfqpoint{1.000216in}{1.679135in}}%
\pgfpathlineto{\pgfqpoint{0.999425in}{1.679135in}}%
\pgfpathlineto{\pgfqpoint{1.000216in}{1.670357in}}%
\pgfpathlineto{\pgfqpoint{1.000672in}{1.665524in}}%
\pgfpathlineto{\pgfqpoint{1.004478in}{1.651913in}}%
\pgfpathlineto{\pgfqpoint{1.010644in}{1.638302in}}%
\pgfpathlineto{\pgfqpoint{1.015872in}{1.629749in}}%
\pgfpathlineto{\pgfqpoint{1.019134in}{1.624691in}}%
\pgfpathlineto{\pgfqpoint{1.029915in}{1.611079in}}%
\pgfpathlineto{\pgfqpoint{1.031529in}{1.609330in}}%
\pgfpathlineto{\pgfqpoint{1.043250in}{1.597468in}}%
\pgfpathlineto{\pgfqpoint{1.047185in}{1.593895in}}%
\pgfpathlineto{\pgfqpoint{1.059327in}{1.583857in}}%
\pgfpathlineto{\pgfqpoint{1.062842in}{1.581177in}}%
\pgfpathlineto{\pgfqpoint{1.078498in}{1.570657in}}%
\pgfpathlineto{\pgfqpoint{1.079228in}{1.570246in}}%
\pgfpathlineto{\pgfqpoint{1.094155in}{1.562358in}}%
\pgfpathlineto{\pgfqpoint{1.107937in}{1.556635in}}%
\pgfpathlineto{\pgfqpoint{1.109812in}{1.555891in}}%
\pgfpathclose%
\pgfpathmoveto{\pgfqpoint{1.892640in}{1.553521in}}%
\pgfpathlineto{\pgfqpoint{1.908296in}{1.550299in}}%
\pgfpathlineto{\pgfqpoint{1.923953in}{1.549210in}}%
\pgfpathlineto{\pgfqpoint{1.923953in}{1.556635in}}%
\pgfpathlineto{\pgfqpoint{1.923953in}{1.570246in}}%
\pgfpathlineto{\pgfqpoint{1.923953in}{1.583857in}}%
\pgfpathlineto{\pgfqpoint{1.923953in}{1.597468in}}%
\pgfpathlineto{\pgfqpoint{1.923953in}{1.611079in}}%
\pgfpathlineto{\pgfqpoint{1.923953in}{1.624691in}}%
\pgfpathlineto{\pgfqpoint{1.923953in}{1.638302in}}%
\pgfpathlineto{\pgfqpoint{1.923953in}{1.651913in}}%
\pgfpathlineto{\pgfqpoint{1.923953in}{1.665524in}}%
\pgfpathlineto{\pgfqpoint{1.923953in}{1.679135in}}%
\pgfpathlineto{\pgfqpoint{1.908296in}{1.679135in}}%
\pgfpathlineto{\pgfqpoint{1.892640in}{1.679135in}}%
\pgfpathlineto{\pgfqpoint{1.876983in}{1.679135in}}%
\pgfpathlineto{\pgfqpoint{1.861327in}{1.679135in}}%
\pgfpathlineto{\pgfqpoint{1.845670in}{1.679135in}}%
\pgfpathlineto{\pgfqpoint{1.830014in}{1.679135in}}%
\pgfpathlineto{\pgfqpoint{1.814357in}{1.679135in}}%
\pgfpathlineto{\pgfqpoint{1.798700in}{1.679135in}}%
\pgfpathlineto{\pgfqpoint{1.783044in}{1.679135in}}%
\pgfpathlineto{\pgfqpoint{1.774503in}{1.679135in}}%
\pgfpathlineto{\pgfqpoint{1.775756in}{1.665524in}}%
\pgfpathlineto{\pgfqpoint{1.779461in}{1.651913in}}%
\pgfpathlineto{\pgfqpoint{1.783044in}{1.643824in}}%
\pgfpathlineto{\pgfqpoint{1.785609in}{1.638302in}}%
\pgfpathlineto{\pgfqpoint{1.794150in}{1.624691in}}%
\pgfpathlineto{\pgfqpoint{1.798700in}{1.618793in}}%
\pgfpathlineto{\pgfqpoint{1.805023in}{1.611079in}}%
\pgfpathlineto{\pgfqpoint{1.814357in}{1.601251in}}%
\pgfpathlineto{\pgfqpoint{1.818257in}{1.597468in}}%
\pgfpathlineto{\pgfqpoint{1.830014in}{1.587247in}}%
\pgfpathlineto{\pgfqpoint{1.834365in}{1.583857in}}%
\pgfpathlineto{\pgfqpoint{1.845670in}{1.575743in}}%
\pgfpathlineto{\pgfqpoint{1.854542in}{1.570246in}}%
\pgfpathlineto{\pgfqpoint{1.861327in}{1.566291in}}%
\pgfpathlineto{\pgfqpoint{1.876983in}{1.558865in}}%
\pgfpathlineto{\pgfqpoint{1.883336in}{1.556635in}}%
\pgfpathlineto{\pgfqpoint{1.892640in}{1.553521in}}%
\pgfpathclose%
\pgfusepath{fill}%
\end{pgfscope}%
\begin{pgfscope}%
\pgfsetbuttcap%
\pgfsetroundjoin%
\definecolor{currentfill}{rgb}{0.000000,0.000000,0.000000}%
\pgfsetfillcolor{currentfill}%
\pgfsetlinewidth{0.803000pt}%
\definecolor{currentstroke}{rgb}{0.000000,0.000000,0.000000}%
\pgfsetstrokecolor{currentstroke}%
\pgfsetdash{}{0pt}%
\pgfsys@defobject{currentmarker}{\pgfqpoint{0.000000in}{-0.048611in}}{\pgfqpoint{0.000000in}{0.000000in}}{%
\pgfpathmoveto{\pgfqpoint{0.000000in}{0.000000in}}%
\pgfpathlineto{\pgfqpoint{0.000000in}{-0.048611in}}%
\pgfusepath{stroke,fill}%
}%
\begin{pgfscope}%
\pgfsys@transformshift{0.373953in}{0.331635in}%
\pgfsys@useobject{currentmarker}{}%
\end{pgfscope}%
\end{pgfscope}%
\begin{pgfscope}%
\definecolor{textcolor}{rgb}{0.000000,0.000000,0.000000}%
\pgfsetstrokecolor{textcolor}%
\pgfsetfillcolor{textcolor}%
\pgftext[x=0.373953in,y=0.234413in,,top]{\color{textcolor}{\sffamily\fontsize{10.000000}{12.000000}\selectfont\catcode`\^=\active\def^{\ifmmode\sp\else\^{}\fi}\catcode`\%=\active\def%{\%}0}}%
\end{pgfscope}%
\begin{pgfscope}%
\pgfsetbuttcap%
\pgfsetroundjoin%
\definecolor{currentfill}{rgb}{0.000000,0.000000,0.000000}%
\pgfsetfillcolor{currentfill}%
\pgfsetlinewidth{0.803000pt}%
\definecolor{currentstroke}{rgb}{0.000000,0.000000,0.000000}%
\pgfsetstrokecolor{currentstroke}%
\pgfsetdash{}{0pt}%
\pgfsys@defobject{currentmarker}{\pgfqpoint{0.000000in}{-0.048611in}}{\pgfqpoint{0.000000in}{0.000000in}}{%
\pgfpathmoveto{\pgfqpoint{0.000000in}{0.000000in}}%
\pgfpathlineto{\pgfqpoint{0.000000in}{-0.048611in}}%
\pgfusepath{stroke,fill}%
}%
\begin{pgfscope}%
\pgfsys@transformshift{1.019786in}{0.331635in}%
\pgfsys@useobject{currentmarker}{}%
\end{pgfscope}%
\end{pgfscope}%
\begin{pgfscope}%
\definecolor{textcolor}{rgb}{0.000000,0.000000,0.000000}%
\pgfsetstrokecolor{textcolor}%
\pgfsetfillcolor{textcolor}%
\pgftext[x=1.019786in,y=0.234413in,,top]{\color{textcolor}{\sffamily\fontsize{10.000000}{12.000000}\selectfont\catcode`\^=\active\def^{\ifmmode\sp\else\^{}\fi}\catcode`\%=\active\def%{\%}5}}%
\end{pgfscope}%
\begin{pgfscope}%
\pgfsetbuttcap%
\pgfsetroundjoin%
\definecolor{currentfill}{rgb}{0.000000,0.000000,0.000000}%
\pgfsetfillcolor{currentfill}%
\pgfsetlinewidth{0.803000pt}%
\definecolor{currentstroke}{rgb}{0.000000,0.000000,0.000000}%
\pgfsetstrokecolor{currentstroke}%
\pgfsetdash{}{0pt}%
\pgfsys@defobject{currentmarker}{\pgfqpoint{0.000000in}{-0.048611in}}{\pgfqpoint{0.000000in}{0.000000in}}{%
\pgfpathmoveto{\pgfqpoint{0.000000in}{0.000000in}}%
\pgfpathlineto{\pgfqpoint{0.000000in}{-0.048611in}}%
\pgfusepath{stroke,fill}%
}%
\begin{pgfscope}%
\pgfsys@transformshift{1.665620in}{0.331635in}%
\pgfsys@useobject{currentmarker}{}%
\end{pgfscope}%
\end{pgfscope}%
\begin{pgfscope}%
\definecolor{textcolor}{rgb}{0.000000,0.000000,0.000000}%
\pgfsetstrokecolor{textcolor}%
\pgfsetfillcolor{textcolor}%
\pgftext[x=1.665620in,y=0.234413in,,top]{\color{textcolor}{\sffamily\fontsize{10.000000}{12.000000}\selectfont\catcode`\^=\active\def^{\ifmmode\sp\else\^{}\fi}\catcode`\%=\active\def%{\%}10}}%
\end{pgfscope}%
\begin{pgfscope}%
\pgfsetbuttcap%
\pgfsetroundjoin%
\definecolor{currentfill}{rgb}{0.000000,0.000000,0.000000}%
\pgfsetfillcolor{currentfill}%
\pgfsetlinewidth{0.803000pt}%
\definecolor{currentstroke}{rgb}{0.000000,0.000000,0.000000}%
\pgfsetstrokecolor{currentstroke}%
\pgfsetdash{}{0pt}%
\pgfsys@defobject{currentmarker}{\pgfqpoint{-0.048611in}{0.000000in}}{\pgfqpoint{-0.000000in}{0.000000in}}{%
\pgfpathmoveto{\pgfqpoint{-0.000000in}{0.000000in}}%
\pgfpathlineto{\pgfqpoint{-0.048611in}{0.000000in}}%
\pgfusepath{stroke,fill}%
}%
\begin{pgfscope}%
\pgfsys@transformshift{0.373953in}{0.331635in}%
\pgfsys@useobject{currentmarker}{}%
\end{pgfscope}%
\end{pgfscope}%
\begin{pgfscope}%
\definecolor{textcolor}{rgb}{0.000000,0.000000,0.000000}%
\pgfsetstrokecolor{textcolor}%
\pgfsetfillcolor{textcolor}%
\pgftext[x=0.188365in, y=0.278873in, left, base]{\color{textcolor}{\sffamily\fontsize{10.000000}{12.000000}\selectfont\catcode`\^=\active\def^{\ifmmode\sp\else\^{}\fi}\catcode`\%=\active\def%{\%}0}}%
\end{pgfscope}%
\begin{pgfscope}%
\pgfsetbuttcap%
\pgfsetroundjoin%
\definecolor{currentfill}{rgb}{0.000000,0.000000,0.000000}%
\pgfsetfillcolor{currentfill}%
\pgfsetlinewidth{0.803000pt}%
\definecolor{currentstroke}{rgb}{0.000000,0.000000,0.000000}%
\pgfsetstrokecolor{currentstroke}%
\pgfsetdash{}{0pt}%
\pgfsys@defobject{currentmarker}{\pgfqpoint{-0.048611in}{0.000000in}}{\pgfqpoint{-0.000000in}{0.000000in}}{%
\pgfpathmoveto{\pgfqpoint{-0.000000in}{0.000000in}}%
\pgfpathlineto{\pgfqpoint{-0.048611in}{0.000000in}}%
\pgfusepath{stroke,fill}%
}%
\begin{pgfscope}%
\pgfsys@transformshift{0.373953in}{0.893093in}%
\pgfsys@useobject{currentmarker}{}%
\end{pgfscope}%
\end{pgfscope}%
\begin{pgfscope}%
\definecolor{textcolor}{rgb}{0.000000,0.000000,0.000000}%
\pgfsetstrokecolor{textcolor}%
\pgfsetfillcolor{textcolor}%
\pgftext[x=0.188365in, y=0.840332in, left, base]{\color{textcolor}{\sffamily\fontsize{10.000000}{12.000000}\selectfont\catcode`\^=\active\def^{\ifmmode\sp\else\^{}\fi}\catcode`\%=\active\def%{\%}5}}%
\end{pgfscope}%
\begin{pgfscope}%
\pgfsetbuttcap%
\pgfsetroundjoin%
\definecolor{currentfill}{rgb}{0.000000,0.000000,0.000000}%
\pgfsetfillcolor{currentfill}%
\pgfsetlinewidth{0.803000pt}%
\definecolor{currentstroke}{rgb}{0.000000,0.000000,0.000000}%
\pgfsetstrokecolor{currentstroke}%
\pgfsetdash{}{0pt}%
\pgfsys@defobject{currentmarker}{\pgfqpoint{-0.048611in}{0.000000in}}{\pgfqpoint{-0.000000in}{0.000000in}}{%
\pgfpathmoveto{\pgfqpoint{-0.000000in}{0.000000in}}%
\pgfpathlineto{\pgfqpoint{-0.048611in}{0.000000in}}%
\pgfusepath{stroke,fill}%
}%
\begin{pgfscope}%
\pgfsys@transformshift{0.373953in}{1.454552in}%
\pgfsys@useobject{currentmarker}{}%
\end{pgfscope}%
\end{pgfscope}%
\begin{pgfscope}%
\definecolor{textcolor}{rgb}{0.000000,0.000000,0.000000}%
\pgfsetstrokecolor{textcolor}%
\pgfsetfillcolor{textcolor}%
\pgftext[x=0.100000in, y=1.401790in, left, base]{\color{textcolor}{\sffamily\fontsize{10.000000}{12.000000}\selectfont\catcode`\^=\active\def^{\ifmmode\sp\else\^{}\fi}\catcode`\%=\active\def%{\%}10}}%
\end{pgfscope}%
\begin{pgfscope}%
\pgfsetrectcap%
\pgfsetmiterjoin%
\pgfsetlinewidth{0.803000pt}%
\definecolor{currentstroke}{rgb}{0.000000,0.000000,0.000000}%
\pgfsetstrokecolor{currentstroke}%
\pgfsetdash{}{0pt}%
\pgfpathmoveto{\pgfqpoint{0.373953in}{0.331635in}}%
\pgfpathlineto{\pgfqpoint{0.373953in}{1.679135in}}%
\pgfusepath{stroke}%
\end{pgfscope}%
\begin{pgfscope}%
\pgfsetrectcap%
\pgfsetmiterjoin%
\pgfsetlinewidth{0.803000pt}%
\definecolor{currentstroke}{rgb}{0.000000,0.000000,0.000000}%
\pgfsetstrokecolor{currentstroke}%
\pgfsetdash{}{0pt}%
\pgfpathmoveto{\pgfqpoint{1.923953in}{0.331635in}}%
\pgfpathlineto{\pgfqpoint{1.923953in}{1.679135in}}%
\pgfusepath{stroke}%
\end{pgfscope}%
\begin{pgfscope}%
\pgfsetrectcap%
\pgfsetmiterjoin%
\pgfsetlinewidth{0.803000pt}%
\definecolor{currentstroke}{rgb}{0.000000,0.000000,0.000000}%
\pgfsetstrokecolor{currentstroke}%
\pgfsetdash{}{0pt}%
\pgfpathmoveto{\pgfqpoint{0.373953in}{0.331635in}}%
\pgfpathlineto{\pgfqpoint{1.923953in}{0.331635in}}%
\pgfusepath{stroke}%
\end{pgfscope}%
\begin{pgfscope}%
\pgfsetrectcap%
\pgfsetmiterjoin%
\pgfsetlinewidth{0.803000pt}%
\definecolor{currentstroke}{rgb}{0.000000,0.000000,0.000000}%
\pgfsetstrokecolor{currentstroke}%
\pgfsetdash{}{0pt}%
\pgfpathmoveto{\pgfqpoint{0.373953in}{1.679135in}}%
\pgfpathlineto{\pgfqpoint{1.923953in}{1.679135in}}%
\pgfusepath{stroke}%
\end{pgfscope}%
\end{pgfpicture}%
\makeatother%
\endgroup%

        \caption{$c=2$}
        \label{fig:5-experiments-periodic-gaussian-well-2}
    \end{subfigure}
    \begin{subfigure}[b]{0.32\columnwidth}
        %% Creator: Matplotlib, PGF backend
%%
%% To include the figure in your LaTeX document, write
%%   \input{<filename>.pgf}
%%
%% Make sure the required packages are loaded in your preamble
%%   \usepackage{pgf}
%%
%% Also ensure that all the required font packages are loaded; for instance,
%% the lmodern package is sometimes necessary when using math font.
%%   \usepackage{lmodern}
%%
%% Figures using additional raster images can only be included by \input if
%% they are in the same directory as the main LaTeX file. For loading figures
%% from other directories you can use the `import` package
%%   \usepackage{import}
%%
%% and then include the figures with
%%   \import{<path to file>}{<filename>.pgf}
%%
%% Matplotlib used the following preamble
%%   \def\mathdefault#1{#1}
%%   \everymath=\expandafter{\the\everymath\displaystyle}
%%   
%%   \usepackage{fontspec}
%%   \setmainfont{DejaVuSans.ttf}[Path=\detokenize{C:/Users/fabio/AppData/Local/Programs/Python/Python311/Lib/site-packages/matplotlib/mpl-data/fonts/ttf/}]
%%   \setsansfont{DejaVuSans.ttf}[Path=\detokenize{C:/Users/fabio/AppData/Local/Programs/Python/Python311/Lib/site-packages/matplotlib/mpl-data/fonts/ttf/}]
%%   \setmonofont{DejaVuSansMono.ttf}[Path=\detokenize{C:/Users/fabio/AppData/Local/Programs/Python/Python311/Lib/site-packages/matplotlib/mpl-data/fonts/ttf/}]
%%   \makeatletter\@ifpackageloaded{underscore}{}{\usepackage[strings]{underscore}}\makeatother
%%
\begingroup%
\makeatletter%
\begin{pgfpicture}%
\pgfpathrectangle{\pgfpointorigin}{\pgfqpoint{2.092011in}{1.869331in}}%
\pgfusepath{use as bounding box, clip}%
\begin{pgfscope}%
\pgfsetbuttcap%
\pgfsetmiterjoin%
\definecolor{currentfill}{rgb}{1.000000,1.000000,1.000000}%
\pgfsetfillcolor{currentfill}%
\pgfsetlinewidth{0.000000pt}%
\definecolor{currentstroke}{rgb}{1.000000,1.000000,1.000000}%
\pgfsetstrokecolor{currentstroke}%
\pgfsetdash{}{0pt}%
\pgfpathmoveto{\pgfqpoint{0.000000in}{0.000000in}}%
\pgfpathlineto{\pgfqpoint{2.092011in}{0.000000in}}%
\pgfpathlineto{\pgfqpoint{2.092011in}{1.869331in}}%
\pgfpathlineto{\pgfqpoint{0.000000in}{1.869331in}}%
\pgfpathlineto{\pgfqpoint{0.000000in}{0.000000in}}%
\pgfpathclose%
\pgfusepath{fill}%
\end{pgfscope}%
\begin{pgfscope}%
\pgfsetbuttcap%
\pgfsetmiterjoin%
\definecolor{currentfill}{rgb}{1.000000,1.000000,1.000000}%
\pgfsetfillcolor{currentfill}%
\pgfsetlinewidth{0.000000pt}%
\definecolor{currentstroke}{rgb}{0.000000,0.000000,0.000000}%
\pgfsetstrokecolor{currentstroke}%
\pgfsetstrokeopacity{0.000000}%
\pgfsetdash{}{0pt}%
\pgfpathmoveto{\pgfqpoint{0.360415in}{0.358518in}}%
\pgfpathlineto{\pgfqpoint{1.910415in}{0.358518in}}%
\pgfpathlineto{\pgfqpoint{1.910415in}{1.706018in}}%
\pgfpathlineto{\pgfqpoint{0.360415in}{1.706018in}}%
\pgfpathlineto{\pgfqpoint{0.360415in}{0.358518in}}%
\pgfpathclose%
\pgfusepath{fill}%
\end{pgfscope}%
\begin{pgfscope}%
\pgfpathrectangle{\pgfqpoint{0.360415in}{0.358518in}}{\pgfqpoint{1.550000in}{1.347500in}}%
\pgfusepath{clip}%
\pgfsetbuttcap%
\pgfsetroundjoin%
\definecolor{currentfill}{rgb}{0.993545,0.862859,0.619299}%
\pgfsetfillcolor{currentfill}%
\pgfsetlinewidth{0.000000pt}%
\definecolor{currentstroke}{rgb}{0.000000,0.000000,0.000000}%
\pgfsetstrokecolor{currentstroke}%
\pgfsetdash{}{0pt}%
\pgfpathmoveto{\pgfqpoint{0.501324in}{0.466207in}}%
\pgfpathlineto{\pgfqpoint{0.516981in}{0.463796in}}%
\pgfpathlineto{\pgfqpoint{0.532608in}{0.467407in}}%
\pgfpathlineto{\pgfqpoint{0.532637in}{0.467418in}}%
\pgfpathlineto{\pgfqpoint{0.546418in}{0.481018in}}%
\pgfpathlineto{\pgfqpoint{0.548294in}{0.487484in}}%
\pgfpathlineto{\pgfqpoint{0.549657in}{0.494629in}}%
\pgfpathlineto{\pgfqpoint{0.548294in}{0.499408in}}%
\pgfpathlineto{\pgfqpoint{0.544450in}{0.508240in}}%
\pgfpathlineto{\pgfqpoint{0.532637in}{0.518509in}}%
\pgfpathlineto{\pgfqpoint{0.522478in}{0.521851in}}%
\pgfpathlineto{\pgfqpoint{0.516981in}{0.523036in}}%
\pgfpathlineto{\pgfqpoint{0.508762in}{0.521851in}}%
\pgfpathlineto{\pgfqpoint{0.501324in}{0.520220in}}%
\pgfpathlineto{\pgfqpoint{0.485680in}{0.508240in}}%
\pgfpathlineto{\pgfqpoint{0.485668in}{0.508214in}}%
\pgfpathlineto{\pgfqpoint{0.481514in}{0.494629in}}%
\pgfpathlineto{\pgfqpoint{0.484288in}{0.481018in}}%
\pgfpathlineto{\pgfqpoint{0.485668in}{0.479044in}}%
\pgfpathlineto{\pgfqpoint{0.499054in}{0.467407in}}%
\pgfpathlineto{\pgfqpoint{0.501324in}{0.466207in}}%
\pgfpathclose%
\pgfpathmoveto{\pgfqpoint{0.814455in}{0.465242in}}%
\pgfpathlineto{\pgfqpoint{0.830112in}{0.464037in}}%
\pgfpathlineto{\pgfqpoint{0.841094in}{0.467407in}}%
\pgfpathlineto{\pgfqpoint{0.845769in}{0.469741in}}%
\pgfpathlineto{\pgfqpoint{0.856001in}{0.481018in}}%
\pgfpathlineto{\pgfqpoint{0.859539in}{0.494629in}}%
\pgfpathlineto{\pgfqpoint{0.854231in}{0.508240in}}%
\pgfpathlineto{\pgfqpoint{0.845769in}{0.516455in}}%
\pgfpathlineto{\pgfqpoint{0.833461in}{0.521851in}}%
\pgfpathlineto{\pgfqpoint{0.830112in}{0.522810in}}%
\pgfpathlineto{\pgfqpoint{0.816853in}{0.521851in}}%
\pgfpathlineto{\pgfqpoint{0.814455in}{0.521588in}}%
\pgfpathlineto{\pgfqpoint{0.798799in}{0.511316in}}%
\pgfpathlineto{\pgfqpoint{0.796162in}{0.508240in}}%
\pgfpathlineto{\pgfqpoint{0.791695in}{0.494629in}}%
\pgfpathlineto{\pgfqpoint{0.794672in}{0.481018in}}%
\pgfpathlineto{\pgfqpoint{0.798799in}{0.475554in}}%
\pgfpathlineto{\pgfqpoint{0.809705in}{0.467407in}}%
\pgfpathlineto{\pgfqpoint{0.814455in}{0.465242in}}%
\pgfpathclose%
\pgfpathmoveto{\pgfqpoint{1.127587in}{0.464519in}}%
\pgfpathlineto{\pgfqpoint{1.143243in}{0.464519in}}%
\pgfpathlineto{\pgfqpoint{1.150808in}{0.467407in}}%
\pgfpathlineto{\pgfqpoint{1.158900in}{0.472453in}}%
\pgfpathlineto{\pgfqpoint{1.165952in}{0.481018in}}%
\pgfpathlineto{\pgfqpoint{1.169179in}{0.494629in}}%
\pgfpathlineto{\pgfqpoint{1.164337in}{0.508240in}}%
\pgfpathlineto{\pgfqpoint{1.158900in}{0.514057in}}%
\pgfpathlineto{\pgfqpoint{1.144663in}{0.521851in}}%
\pgfpathlineto{\pgfqpoint{1.143243in}{0.522357in}}%
\pgfpathlineto{\pgfqpoint{1.127587in}{0.522357in}}%
\pgfpathlineto{\pgfqpoint{1.126167in}{0.521851in}}%
\pgfpathlineto{\pgfqpoint{1.111930in}{0.514057in}}%
\pgfpathlineto{\pgfqpoint{1.106493in}{0.508240in}}%
\pgfpathlineto{\pgfqpoint{1.101651in}{0.494629in}}%
\pgfpathlineto{\pgfqpoint{1.104878in}{0.481018in}}%
\pgfpathlineto{\pgfqpoint{1.111930in}{0.472453in}}%
\pgfpathlineto{\pgfqpoint{1.120022in}{0.467407in}}%
\pgfpathlineto{\pgfqpoint{1.127587in}{0.464519in}}%
\pgfpathclose%
\pgfpathmoveto{\pgfqpoint{1.440718in}{0.464037in}}%
\pgfpathlineto{\pgfqpoint{1.456375in}{0.465242in}}%
\pgfpathlineto{\pgfqpoint{1.461125in}{0.467407in}}%
\pgfpathlineto{\pgfqpoint{1.472031in}{0.475554in}}%
\pgfpathlineto{\pgfqpoint{1.476158in}{0.481018in}}%
\pgfpathlineto{\pgfqpoint{1.479135in}{0.494629in}}%
\pgfpathlineto{\pgfqpoint{1.474668in}{0.508240in}}%
\pgfpathlineto{\pgfqpoint{1.472031in}{0.511316in}}%
\pgfpathlineto{\pgfqpoint{1.456375in}{0.521588in}}%
\pgfpathlineto{\pgfqpoint{1.453977in}{0.521851in}}%
\pgfpathlineto{\pgfqpoint{1.440718in}{0.522810in}}%
\pgfpathlineto{\pgfqpoint{1.437369in}{0.521851in}}%
\pgfpathlineto{\pgfqpoint{1.425061in}{0.516455in}}%
\pgfpathlineto{\pgfqpoint{1.416599in}{0.508240in}}%
\pgfpathlineto{\pgfqpoint{1.411291in}{0.494629in}}%
\pgfpathlineto{\pgfqpoint{1.414829in}{0.481018in}}%
\pgfpathlineto{\pgfqpoint{1.425061in}{0.469741in}}%
\pgfpathlineto{\pgfqpoint{1.429736in}{0.467407in}}%
\pgfpathlineto{\pgfqpoint{1.440718in}{0.464037in}}%
\pgfpathclose%
\pgfpathmoveto{\pgfqpoint{1.753849in}{0.463796in}}%
\pgfpathlineto{\pgfqpoint{1.769506in}{0.466207in}}%
\pgfpathlineto{\pgfqpoint{1.771776in}{0.467407in}}%
\pgfpathlineto{\pgfqpoint{1.785162in}{0.479044in}}%
\pgfpathlineto{\pgfqpoint{1.786542in}{0.481018in}}%
\pgfpathlineto{\pgfqpoint{1.789316in}{0.494629in}}%
\pgfpathlineto{\pgfqpoint{1.785162in}{0.508214in}}%
\pgfpathlineto{\pgfqpoint{1.785150in}{0.508240in}}%
\pgfpathlineto{\pgfqpoint{1.769506in}{0.520220in}}%
\pgfpathlineto{\pgfqpoint{1.762068in}{0.521851in}}%
\pgfpathlineto{\pgfqpoint{1.753849in}{0.523036in}}%
\pgfpathlineto{\pgfqpoint{1.748352in}{0.521851in}}%
\pgfpathlineto{\pgfqpoint{1.738193in}{0.518509in}}%
\pgfpathlineto{\pgfqpoint{1.726380in}{0.508240in}}%
\pgfpathlineto{\pgfqpoint{1.722536in}{0.499408in}}%
\pgfpathlineto{\pgfqpoint{1.721173in}{0.494629in}}%
\pgfpathlineto{\pgfqpoint{1.722536in}{0.487484in}}%
\pgfpathlineto{\pgfqpoint{1.724412in}{0.481018in}}%
\pgfpathlineto{\pgfqpoint{1.738193in}{0.467418in}}%
\pgfpathlineto{\pgfqpoint{1.738222in}{0.467407in}}%
\pgfpathlineto{\pgfqpoint{1.753849in}{0.463796in}}%
\pgfpathclose%
\pgfpathmoveto{\pgfqpoint{0.501324in}{0.736041in}}%
\pgfpathlineto{\pgfqpoint{0.516981in}{0.733453in}}%
\pgfpathlineto{\pgfqpoint{0.532637in}{0.737336in}}%
\pgfpathlineto{\pgfqpoint{0.536175in}{0.739629in}}%
\pgfpathlineto{\pgfqpoint{0.547992in}{0.753240in}}%
\pgfpathlineto{\pgfqpoint{0.548294in}{0.755324in}}%
\pgfpathlineto{\pgfqpoint{0.549397in}{0.766851in}}%
\pgfpathlineto{\pgfqpoint{0.548294in}{0.769762in}}%
\pgfpathlineto{\pgfqpoint{0.542087in}{0.780462in}}%
\pgfpathlineto{\pgfqpoint{0.532637in}{0.787819in}}%
\pgfpathlineto{\pgfqpoint{0.516981in}{0.792434in}}%
\pgfpathlineto{\pgfqpoint{0.501324in}{0.789358in}}%
\pgfpathlineto{\pgfqpoint{0.488353in}{0.780462in}}%
\pgfpathlineto{\pgfqpoint{0.485668in}{0.776399in}}%
\pgfpathlineto{\pgfqpoint{0.481791in}{0.766851in}}%
\pgfpathlineto{\pgfqpoint{0.483178in}{0.753240in}}%
\pgfpathlineto{\pgfqpoint{0.485668in}{0.749110in}}%
\pgfpathlineto{\pgfqpoint{0.495039in}{0.739629in}}%
\pgfpathlineto{\pgfqpoint{0.501324in}{0.736041in}}%
\pgfpathclose%
\pgfpathmoveto{\pgfqpoint{0.814455in}{0.735006in}}%
\pgfpathlineto{\pgfqpoint{0.830112in}{0.733712in}}%
\pgfpathlineto{\pgfqpoint{0.845769in}{0.738891in}}%
\pgfpathlineto{\pgfqpoint{0.846788in}{0.739629in}}%
\pgfpathlineto{\pgfqpoint{0.857417in}{0.753240in}}%
\pgfpathlineto{\pgfqpoint{0.859185in}{0.766851in}}%
\pgfpathlineto{\pgfqpoint{0.852106in}{0.780462in}}%
\pgfpathlineto{\pgfqpoint{0.845769in}{0.785971in}}%
\pgfpathlineto{\pgfqpoint{0.830112in}{0.792126in}}%
\pgfpathlineto{\pgfqpoint{0.814455in}{0.790589in}}%
\pgfpathlineto{\pgfqpoint{0.798799in}{0.781349in}}%
\pgfpathlineto{\pgfqpoint{0.797950in}{0.780462in}}%
\pgfpathlineto{\pgfqpoint{0.791993in}{0.766851in}}%
\pgfpathlineto{\pgfqpoint{0.793481in}{0.753240in}}%
\pgfpathlineto{\pgfqpoint{0.798799in}{0.745075in}}%
\pgfpathlineto{\pgfqpoint{0.805063in}{0.739629in}}%
\pgfpathlineto{\pgfqpoint{0.814455in}{0.735006in}}%
\pgfpathclose%
\pgfpathmoveto{\pgfqpoint{1.127587in}{0.734229in}}%
\pgfpathlineto{\pgfqpoint{1.143243in}{0.734229in}}%
\pgfpathlineto{\pgfqpoint{1.156336in}{0.739629in}}%
\pgfpathlineto{\pgfqpoint{1.158900in}{0.741489in}}%
\pgfpathlineto{\pgfqpoint{1.167243in}{0.753240in}}%
\pgfpathlineto{\pgfqpoint{1.168856in}{0.766851in}}%
\pgfpathlineto{\pgfqpoint{1.162399in}{0.780462in}}%
\pgfpathlineto{\pgfqpoint{1.158900in}{0.783815in}}%
\pgfpathlineto{\pgfqpoint{1.143243in}{0.791511in}}%
\pgfpathlineto{\pgfqpoint{1.127587in}{0.791511in}}%
\pgfpathlineto{\pgfqpoint{1.111930in}{0.783815in}}%
\pgfpathlineto{\pgfqpoint{1.108431in}{0.780462in}}%
\pgfpathlineto{\pgfqpoint{1.101974in}{0.766851in}}%
\pgfpathlineto{\pgfqpoint{1.103587in}{0.753240in}}%
\pgfpathlineto{\pgfqpoint{1.111930in}{0.741489in}}%
\pgfpathlineto{\pgfqpoint{1.114494in}{0.739629in}}%
\pgfpathlineto{\pgfqpoint{1.127587in}{0.734229in}}%
\pgfpathclose%
\pgfpathmoveto{\pgfqpoint{1.425061in}{0.738891in}}%
\pgfpathlineto{\pgfqpoint{1.440718in}{0.733712in}}%
\pgfpathlineto{\pgfqpoint{1.456375in}{0.735006in}}%
\pgfpathlineto{\pgfqpoint{1.465767in}{0.739629in}}%
\pgfpathlineto{\pgfqpoint{1.472031in}{0.745075in}}%
\pgfpathlineto{\pgfqpoint{1.477349in}{0.753240in}}%
\pgfpathlineto{\pgfqpoint{1.478837in}{0.766851in}}%
\pgfpathlineto{\pgfqpoint{1.472880in}{0.780462in}}%
\pgfpathlineto{\pgfqpoint{1.472031in}{0.781349in}}%
\pgfpathlineto{\pgfqpoint{1.456375in}{0.790589in}}%
\pgfpathlineto{\pgfqpoint{1.440718in}{0.792126in}}%
\pgfpathlineto{\pgfqpoint{1.425061in}{0.785971in}}%
\pgfpathlineto{\pgfqpoint{1.418724in}{0.780462in}}%
\pgfpathlineto{\pgfqpoint{1.411645in}{0.766851in}}%
\pgfpathlineto{\pgfqpoint{1.413413in}{0.753240in}}%
\pgfpathlineto{\pgfqpoint{1.424042in}{0.739629in}}%
\pgfpathlineto{\pgfqpoint{1.425061in}{0.738891in}}%
\pgfpathclose%
\pgfpathmoveto{\pgfqpoint{1.738193in}{0.737336in}}%
\pgfpathlineto{\pgfqpoint{1.753849in}{0.733453in}}%
\pgfpathlineto{\pgfqpoint{1.769506in}{0.736041in}}%
\pgfpathlineto{\pgfqpoint{1.775791in}{0.739629in}}%
\pgfpathlineto{\pgfqpoint{1.785162in}{0.749110in}}%
\pgfpathlineto{\pgfqpoint{1.787652in}{0.753240in}}%
\pgfpathlineto{\pgfqpoint{1.789039in}{0.766851in}}%
\pgfpathlineto{\pgfqpoint{1.785162in}{0.776399in}}%
\pgfpathlineto{\pgfqpoint{1.782477in}{0.780462in}}%
\pgfpathlineto{\pgfqpoint{1.769506in}{0.789358in}}%
\pgfpathlineto{\pgfqpoint{1.753849in}{0.792434in}}%
\pgfpathlineto{\pgfqpoint{1.738193in}{0.787819in}}%
\pgfpathlineto{\pgfqpoint{1.728743in}{0.780462in}}%
\pgfpathlineto{\pgfqpoint{1.722536in}{0.769762in}}%
\pgfpathlineto{\pgfqpoint{1.721433in}{0.766851in}}%
\pgfpathlineto{\pgfqpoint{1.722536in}{0.755324in}}%
\pgfpathlineto{\pgfqpoint{1.722838in}{0.753240in}}%
\pgfpathlineto{\pgfqpoint{1.734655in}{0.739629in}}%
\pgfpathlineto{\pgfqpoint{1.738193in}{0.737336in}}%
\pgfpathclose%
\pgfpathmoveto{\pgfqpoint{0.501324in}{1.005720in}}%
\pgfpathlineto{\pgfqpoint{0.516981in}{1.002915in}}%
\pgfpathlineto{\pgfqpoint{0.532637in}{1.007124in}}%
\pgfpathlineto{\pgfqpoint{0.539329in}{1.011851in}}%
\pgfpathlineto{\pgfqpoint{0.548294in}{1.024228in}}%
\pgfpathlineto{\pgfqpoint{0.548875in}{1.025462in}}%
\pgfpathlineto{\pgfqpoint{0.548875in}{1.039073in}}%
\pgfpathlineto{\pgfqpoint{0.548294in}{1.040308in}}%
\pgfpathlineto{\pgfqpoint{0.539329in}{1.052684in}}%
\pgfpathlineto{\pgfqpoint{0.532637in}{1.057411in}}%
\pgfpathlineto{\pgfqpoint{0.516981in}{1.061620in}}%
\pgfpathlineto{\pgfqpoint{0.501324in}{1.058815in}}%
\pgfpathlineto{\pgfqpoint{0.491472in}{1.052684in}}%
\pgfpathlineto{\pgfqpoint{0.485668in}{1.045649in}}%
\pgfpathlineto{\pgfqpoint{0.482346in}{1.039073in}}%
\pgfpathlineto{\pgfqpoint{0.482346in}{1.025462in}}%
\pgfpathlineto{\pgfqpoint{0.485668in}{1.018886in}}%
\pgfpathlineto{\pgfqpoint{0.491472in}{1.011851in}}%
\pgfpathlineto{\pgfqpoint{0.501324in}{1.005720in}}%
\pgfpathclose%
\pgfpathmoveto{\pgfqpoint{0.814455in}{1.004598in}}%
\pgfpathlineto{\pgfqpoint{0.830112in}{1.003196in}}%
\pgfpathlineto{\pgfqpoint{0.845769in}{1.008809in}}%
\pgfpathlineto{\pgfqpoint{0.849625in}{1.011851in}}%
\pgfpathlineto{\pgfqpoint{0.858478in}{1.025462in}}%
\pgfpathlineto{\pgfqpoint{0.858478in}{1.039073in}}%
\pgfpathlineto{\pgfqpoint{0.849625in}{1.052684in}}%
\pgfpathlineto{\pgfqpoint{0.845769in}{1.055726in}}%
\pgfpathlineto{\pgfqpoint{0.830112in}{1.061340in}}%
\pgfpathlineto{\pgfqpoint{0.814455in}{1.059937in}}%
\pgfpathlineto{\pgfqpoint{0.800939in}{1.052684in}}%
\pgfpathlineto{\pgfqpoint{0.798799in}{1.050455in}}%
\pgfpathlineto{\pgfqpoint{0.792588in}{1.039073in}}%
\pgfpathlineto{\pgfqpoint{0.792588in}{1.025462in}}%
\pgfpathlineto{\pgfqpoint{0.798799in}{1.014080in}}%
\pgfpathlineto{\pgfqpoint{0.800939in}{1.011851in}}%
\pgfpathlineto{\pgfqpoint{0.814455in}{1.004598in}}%
\pgfpathclose%
\pgfpathmoveto{\pgfqpoint{1.111930in}{1.010776in}}%
\pgfpathlineto{\pgfqpoint{1.127587in}{1.003756in}}%
\pgfpathlineto{\pgfqpoint{1.143243in}{1.003756in}}%
\pgfpathlineto{\pgfqpoint{1.158900in}{1.010776in}}%
\pgfpathlineto{\pgfqpoint{1.160137in}{1.011851in}}%
\pgfpathlineto{\pgfqpoint{1.168211in}{1.025462in}}%
\pgfpathlineto{\pgfqpoint{1.168211in}{1.039073in}}%
\pgfpathlineto{\pgfqpoint{1.160137in}{1.052684in}}%
\pgfpathlineto{\pgfqpoint{1.158900in}{1.053759in}}%
\pgfpathlineto{\pgfqpoint{1.143243in}{1.060779in}}%
\pgfpathlineto{\pgfqpoint{1.127587in}{1.060779in}}%
\pgfpathlineto{\pgfqpoint{1.111930in}{1.053759in}}%
\pgfpathlineto{\pgfqpoint{1.110693in}{1.052684in}}%
\pgfpathlineto{\pgfqpoint{1.102619in}{1.039073in}}%
\pgfpathlineto{\pgfqpoint{1.102619in}{1.025462in}}%
\pgfpathlineto{\pgfqpoint{1.110693in}{1.011851in}}%
\pgfpathlineto{\pgfqpoint{1.111930in}{1.010776in}}%
\pgfpathclose%
\pgfpathmoveto{\pgfqpoint{1.425061in}{1.008809in}}%
\pgfpathlineto{\pgfqpoint{1.440718in}{1.003196in}}%
\pgfpathlineto{\pgfqpoint{1.456375in}{1.004598in}}%
\pgfpathlineto{\pgfqpoint{1.469891in}{1.011851in}}%
\pgfpathlineto{\pgfqpoint{1.472031in}{1.014080in}}%
\pgfpathlineto{\pgfqpoint{1.478242in}{1.025462in}}%
\pgfpathlineto{\pgfqpoint{1.478242in}{1.039073in}}%
\pgfpathlineto{\pgfqpoint{1.472031in}{1.050455in}}%
\pgfpathlineto{\pgfqpoint{1.469891in}{1.052684in}}%
\pgfpathlineto{\pgfqpoint{1.456375in}{1.059937in}}%
\pgfpathlineto{\pgfqpoint{1.440718in}{1.061340in}}%
\pgfpathlineto{\pgfqpoint{1.425061in}{1.055726in}}%
\pgfpathlineto{\pgfqpoint{1.421205in}{1.052684in}}%
\pgfpathlineto{\pgfqpoint{1.412352in}{1.039073in}}%
\pgfpathlineto{\pgfqpoint{1.412352in}{1.025462in}}%
\pgfpathlineto{\pgfqpoint{1.421205in}{1.011851in}}%
\pgfpathlineto{\pgfqpoint{1.425061in}{1.008809in}}%
\pgfpathclose%
\pgfpathmoveto{\pgfqpoint{1.738193in}{1.007124in}}%
\pgfpathlineto{\pgfqpoint{1.753849in}{1.002915in}}%
\pgfpathlineto{\pgfqpoint{1.769506in}{1.005720in}}%
\pgfpathlineto{\pgfqpoint{1.779358in}{1.011851in}}%
\pgfpathlineto{\pgfqpoint{1.785162in}{1.018886in}}%
\pgfpathlineto{\pgfqpoint{1.788484in}{1.025462in}}%
\pgfpathlineto{\pgfqpoint{1.788484in}{1.039073in}}%
\pgfpathlineto{\pgfqpoint{1.785162in}{1.045649in}}%
\pgfpathlineto{\pgfqpoint{1.779358in}{1.052684in}}%
\pgfpathlineto{\pgfqpoint{1.769506in}{1.058815in}}%
\pgfpathlineto{\pgfqpoint{1.753849in}{1.061620in}}%
\pgfpathlineto{\pgfqpoint{1.738193in}{1.057411in}}%
\pgfpathlineto{\pgfqpoint{1.731501in}{1.052684in}}%
\pgfpathlineto{\pgfqpoint{1.722536in}{1.040308in}}%
\pgfpathlineto{\pgfqpoint{1.721955in}{1.039073in}}%
\pgfpathlineto{\pgfqpoint{1.721955in}{1.025462in}}%
\pgfpathlineto{\pgfqpoint{1.722536in}{1.024228in}}%
\pgfpathlineto{\pgfqpoint{1.731501in}{1.011851in}}%
\pgfpathlineto{\pgfqpoint{1.738193in}{1.007124in}}%
\pgfpathclose%
\pgfpathmoveto{\pgfqpoint{0.501324in}{1.275177in}}%
\pgfpathlineto{\pgfqpoint{0.516981in}{1.272102in}}%
\pgfpathlineto{\pgfqpoint{0.532637in}{1.276716in}}%
\pgfpathlineto{\pgfqpoint{0.542087in}{1.284073in}}%
\pgfpathlineto{\pgfqpoint{0.548294in}{1.294773in}}%
\pgfpathlineto{\pgfqpoint{0.549397in}{1.297684in}}%
\pgfpathlineto{\pgfqpoint{0.548294in}{1.309211in}}%
\pgfpathlineto{\pgfqpoint{0.547992in}{1.311295in}}%
\pgfpathlineto{\pgfqpoint{0.536175in}{1.324907in}}%
\pgfpathlineto{\pgfqpoint{0.532637in}{1.327199in}}%
\pgfpathlineto{\pgfqpoint{0.516981in}{1.331082in}}%
\pgfpathlineto{\pgfqpoint{0.501324in}{1.328494in}}%
\pgfpathlineto{\pgfqpoint{0.495039in}{1.324907in}}%
\pgfpathlineto{\pgfqpoint{0.485668in}{1.315425in}}%
\pgfpathlineto{\pgfqpoint{0.483178in}{1.311295in}}%
\pgfpathlineto{\pgfqpoint{0.481791in}{1.297684in}}%
\pgfpathlineto{\pgfqpoint{0.485668in}{1.288137in}}%
\pgfpathlineto{\pgfqpoint{0.488353in}{1.284073in}}%
\pgfpathlineto{\pgfqpoint{0.501324in}{1.275177in}}%
\pgfpathclose%
\pgfpathmoveto{\pgfqpoint{0.798799in}{1.283187in}}%
\pgfpathlineto{\pgfqpoint{0.814455in}{1.273947in}}%
\pgfpathlineto{\pgfqpoint{0.830112in}{1.272409in}}%
\pgfpathlineto{\pgfqpoint{0.845769in}{1.278564in}}%
\pgfpathlineto{\pgfqpoint{0.852106in}{1.284073in}}%
\pgfpathlineto{\pgfqpoint{0.859185in}{1.297684in}}%
\pgfpathlineto{\pgfqpoint{0.857417in}{1.311295in}}%
\pgfpathlineto{\pgfqpoint{0.846788in}{1.324907in}}%
\pgfpathlineto{\pgfqpoint{0.845769in}{1.325644in}}%
\pgfpathlineto{\pgfqpoint{0.830112in}{1.330823in}}%
\pgfpathlineto{\pgfqpoint{0.814455in}{1.329530in}}%
\pgfpathlineto{\pgfqpoint{0.805063in}{1.324907in}}%
\pgfpathlineto{\pgfqpoint{0.798799in}{1.319460in}}%
\pgfpathlineto{\pgfqpoint{0.793481in}{1.311295in}}%
\pgfpathlineto{\pgfqpoint{0.791993in}{1.297684in}}%
\pgfpathlineto{\pgfqpoint{0.797950in}{1.284073in}}%
\pgfpathlineto{\pgfqpoint{0.798799in}{1.283187in}}%
\pgfpathclose%
\pgfpathmoveto{\pgfqpoint{1.111930in}{1.280721in}}%
\pgfpathlineto{\pgfqpoint{1.127587in}{1.273024in}}%
\pgfpathlineto{\pgfqpoint{1.143243in}{1.273024in}}%
\pgfpathlineto{\pgfqpoint{1.158900in}{1.280721in}}%
\pgfpathlineto{\pgfqpoint{1.162399in}{1.284073in}}%
\pgfpathlineto{\pgfqpoint{1.168856in}{1.297684in}}%
\pgfpathlineto{\pgfqpoint{1.167243in}{1.311295in}}%
\pgfpathlineto{\pgfqpoint{1.158900in}{1.323046in}}%
\pgfpathlineto{\pgfqpoint{1.156336in}{1.324907in}}%
\pgfpathlineto{\pgfqpoint{1.143243in}{1.330306in}}%
\pgfpathlineto{\pgfqpoint{1.127587in}{1.330306in}}%
\pgfpathlineto{\pgfqpoint{1.114494in}{1.324907in}}%
\pgfpathlineto{\pgfqpoint{1.111930in}{1.323046in}}%
\pgfpathlineto{\pgfqpoint{1.103587in}{1.311295in}}%
\pgfpathlineto{\pgfqpoint{1.101974in}{1.297684in}}%
\pgfpathlineto{\pgfqpoint{1.108431in}{1.284073in}}%
\pgfpathlineto{\pgfqpoint{1.111930in}{1.280721in}}%
\pgfpathclose%
\pgfpathmoveto{\pgfqpoint{1.425061in}{1.278564in}}%
\pgfpathlineto{\pgfqpoint{1.440718in}{1.272409in}}%
\pgfpathlineto{\pgfqpoint{1.456375in}{1.273947in}}%
\pgfpathlineto{\pgfqpoint{1.472031in}{1.283187in}}%
\pgfpathlineto{\pgfqpoint{1.472880in}{1.284073in}}%
\pgfpathlineto{\pgfqpoint{1.478837in}{1.297684in}}%
\pgfpathlineto{\pgfqpoint{1.477349in}{1.311295in}}%
\pgfpathlineto{\pgfqpoint{1.472031in}{1.319460in}}%
\pgfpathlineto{\pgfqpoint{1.465767in}{1.324907in}}%
\pgfpathlineto{\pgfqpoint{1.456375in}{1.329530in}}%
\pgfpathlineto{\pgfqpoint{1.440718in}{1.330823in}}%
\pgfpathlineto{\pgfqpoint{1.425061in}{1.325644in}}%
\pgfpathlineto{\pgfqpoint{1.424042in}{1.324907in}}%
\pgfpathlineto{\pgfqpoint{1.413413in}{1.311295in}}%
\pgfpathlineto{\pgfqpoint{1.411645in}{1.297684in}}%
\pgfpathlineto{\pgfqpoint{1.418724in}{1.284073in}}%
\pgfpathlineto{\pgfqpoint{1.425061in}{1.278564in}}%
\pgfpathclose%
\pgfpathmoveto{\pgfqpoint{1.738193in}{1.276716in}}%
\pgfpathlineto{\pgfqpoint{1.753849in}{1.272102in}}%
\pgfpathlineto{\pgfqpoint{1.769506in}{1.275177in}}%
\pgfpathlineto{\pgfqpoint{1.782477in}{1.284073in}}%
\pgfpathlineto{\pgfqpoint{1.785162in}{1.288137in}}%
\pgfpathlineto{\pgfqpoint{1.789039in}{1.297684in}}%
\pgfpathlineto{\pgfqpoint{1.787652in}{1.311295in}}%
\pgfpathlineto{\pgfqpoint{1.785162in}{1.315425in}}%
\pgfpathlineto{\pgfqpoint{1.775791in}{1.324907in}}%
\pgfpathlineto{\pgfqpoint{1.769506in}{1.328494in}}%
\pgfpathlineto{\pgfqpoint{1.753849in}{1.331082in}}%
\pgfpathlineto{\pgfqpoint{1.738193in}{1.327199in}}%
\pgfpathlineto{\pgfqpoint{1.734655in}{1.324907in}}%
\pgfpathlineto{\pgfqpoint{1.722838in}{1.311295in}}%
\pgfpathlineto{\pgfqpoint{1.722536in}{1.309211in}}%
\pgfpathlineto{\pgfqpoint{1.721433in}{1.297684in}}%
\pgfpathlineto{\pgfqpoint{1.722536in}{1.294773in}}%
\pgfpathlineto{\pgfqpoint{1.728743in}{1.284073in}}%
\pgfpathlineto{\pgfqpoint{1.738193in}{1.276716in}}%
\pgfpathclose%
\pgfpathmoveto{\pgfqpoint{0.516981in}{1.541499in}}%
\pgfpathlineto{\pgfqpoint{0.522478in}{1.542684in}}%
\pgfpathlineto{\pgfqpoint{0.532637in}{1.546026in}}%
\pgfpathlineto{\pgfqpoint{0.544450in}{1.556295in}}%
\pgfpathlineto{\pgfqpoint{0.548294in}{1.565127in}}%
\pgfpathlineto{\pgfqpoint{0.549657in}{1.569907in}}%
\pgfpathlineto{\pgfqpoint{0.548294in}{1.577051in}}%
\pgfpathlineto{\pgfqpoint{0.546418in}{1.583518in}}%
\pgfpathlineto{\pgfqpoint{0.532637in}{1.597118in}}%
\pgfpathlineto{\pgfqpoint{0.532608in}{1.597129in}}%
\pgfpathlineto{\pgfqpoint{0.516981in}{1.600740in}}%
\pgfpathlineto{\pgfqpoint{0.501324in}{1.598328in}}%
\pgfpathlineto{\pgfqpoint{0.499054in}{1.597129in}}%
\pgfpathlineto{\pgfqpoint{0.485668in}{1.585491in}}%
\pgfpathlineto{\pgfqpoint{0.484288in}{1.583518in}}%
\pgfpathlineto{\pgfqpoint{0.481514in}{1.569907in}}%
\pgfpathlineto{\pgfqpoint{0.485668in}{1.556321in}}%
\pgfpathlineto{\pgfqpoint{0.485680in}{1.556295in}}%
\pgfpathlineto{\pgfqpoint{0.501324in}{1.544315in}}%
\pgfpathlineto{\pgfqpoint{0.508762in}{1.542684in}}%
\pgfpathlineto{\pgfqpoint{0.516981in}{1.541499in}}%
\pgfpathclose%
\pgfpathmoveto{\pgfqpoint{0.830112in}{1.541726in}}%
\pgfpathlineto{\pgfqpoint{0.833461in}{1.542684in}}%
\pgfpathlineto{\pgfqpoint{0.845769in}{1.548080in}}%
\pgfpathlineto{\pgfqpoint{0.854231in}{1.556295in}}%
\pgfpathlineto{\pgfqpoint{0.859539in}{1.569907in}}%
\pgfpathlineto{\pgfqpoint{0.856001in}{1.583518in}}%
\pgfpathlineto{\pgfqpoint{0.845769in}{1.594794in}}%
\pgfpathlineto{\pgfqpoint{0.841094in}{1.597129in}}%
\pgfpathlineto{\pgfqpoint{0.830112in}{1.600499in}}%
\pgfpathlineto{\pgfqpoint{0.814455in}{1.599293in}}%
\pgfpathlineto{\pgfqpoint{0.809705in}{1.597129in}}%
\pgfpathlineto{\pgfqpoint{0.798799in}{1.588981in}}%
\pgfpathlineto{\pgfqpoint{0.794672in}{1.583518in}}%
\pgfpathlineto{\pgfqpoint{0.791695in}{1.569907in}}%
\pgfpathlineto{\pgfqpoint{0.796162in}{1.556295in}}%
\pgfpathlineto{\pgfqpoint{0.798799in}{1.553220in}}%
\pgfpathlineto{\pgfqpoint{0.814455in}{1.542947in}}%
\pgfpathlineto{\pgfqpoint{0.816853in}{1.542684in}}%
\pgfpathlineto{\pgfqpoint{0.830112in}{1.541726in}}%
\pgfpathclose%
\pgfpathmoveto{\pgfqpoint{1.127587in}{1.542179in}}%
\pgfpathlineto{\pgfqpoint{1.143243in}{1.542179in}}%
\pgfpathlineto{\pgfqpoint{1.144663in}{1.542684in}}%
\pgfpathlineto{\pgfqpoint{1.158900in}{1.550478in}}%
\pgfpathlineto{\pgfqpoint{1.164337in}{1.556295in}}%
\pgfpathlineto{\pgfqpoint{1.169179in}{1.569907in}}%
\pgfpathlineto{\pgfqpoint{1.165952in}{1.583518in}}%
\pgfpathlineto{\pgfqpoint{1.158900in}{1.592082in}}%
\pgfpathlineto{\pgfqpoint{1.150808in}{1.597129in}}%
\pgfpathlineto{\pgfqpoint{1.143243in}{1.600016in}}%
\pgfpathlineto{\pgfqpoint{1.127587in}{1.600016in}}%
\pgfpathlineto{\pgfqpoint{1.120022in}{1.597129in}}%
\pgfpathlineto{\pgfqpoint{1.111930in}{1.592082in}}%
\pgfpathlineto{\pgfqpoint{1.104878in}{1.583518in}}%
\pgfpathlineto{\pgfqpoint{1.101651in}{1.569907in}}%
\pgfpathlineto{\pgfqpoint{1.106493in}{1.556295in}}%
\pgfpathlineto{\pgfqpoint{1.111930in}{1.550478in}}%
\pgfpathlineto{\pgfqpoint{1.126167in}{1.542684in}}%
\pgfpathlineto{\pgfqpoint{1.127587in}{1.542179in}}%
\pgfpathclose%
\pgfpathmoveto{\pgfqpoint{1.440718in}{1.541726in}}%
\pgfpathlineto{\pgfqpoint{1.453977in}{1.542684in}}%
\pgfpathlineto{\pgfqpoint{1.456375in}{1.542947in}}%
\pgfpathlineto{\pgfqpoint{1.472031in}{1.553220in}}%
\pgfpathlineto{\pgfqpoint{1.474668in}{1.556295in}}%
\pgfpathlineto{\pgfqpoint{1.479135in}{1.569907in}}%
\pgfpathlineto{\pgfqpoint{1.476158in}{1.583518in}}%
\pgfpathlineto{\pgfqpoint{1.472031in}{1.588981in}}%
\pgfpathlineto{\pgfqpoint{1.461125in}{1.597129in}}%
\pgfpathlineto{\pgfqpoint{1.456375in}{1.599293in}}%
\pgfpathlineto{\pgfqpoint{1.440718in}{1.600499in}}%
\pgfpathlineto{\pgfqpoint{1.429736in}{1.597129in}}%
\pgfpathlineto{\pgfqpoint{1.425061in}{1.594794in}}%
\pgfpathlineto{\pgfqpoint{1.414829in}{1.583518in}}%
\pgfpathlineto{\pgfqpoint{1.411291in}{1.569907in}}%
\pgfpathlineto{\pgfqpoint{1.416599in}{1.556295in}}%
\pgfpathlineto{\pgfqpoint{1.425061in}{1.548080in}}%
\pgfpathlineto{\pgfqpoint{1.437369in}{1.542684in}}%
\pgfpathlineto{\pgfqpoint{1.440718in}{1.541726in}}%
\pgfpathclose%
\pgfpathmoveto{\pgfqpoint{1.753849in}{1.541499in}}%
\pgfpathlineto{\pgfqpoint{1.762068in}{1.542684in}}%
\pgfpathlineto{\pgfqpoint{1.769506in}{1.544315in}}%
\pgfpathlineto{\pgfqpoint{1.785150in}{1.556295in}}%
\pgfpathlineto{\pgfqpoint{1.785162in}{1.556321in}}%
\pgfpathlineto{\pgfqpoint{1.789316in}{1.569907in}}%
\pgfpathlineto{\pgfqpoint{1.786542in}{1.583518in}}%
\pgfpathlineto{\pgfqpoint{1.785162in}{1.585491in}}%
\pgfpathlineto{\pgfqpoint{1.771776in}{1.597129in}}%
\pgfpathlineto{\pgfqpoint{1.769506in}{1.598328in}}%
\pgfpathlineto{\pgfqpoint{1.753849in}{1.600740in}}%
\pgfpathlineto{\pgfqpoint{1.738222in}{1.597129in}}%
\pgfpathlineto{\pgfqpoint{1.738193in}{1.597118in}}%
\pgfpathlineto{\pgfqpoint{1.724412in}{1.583518in}}%
\pgfpathlineto{\pgfqpoint{1.722536in}{1.577051in}}%
\pgfpathlineto{\pgfqpoint{1.721173in}{1.569907in}}%
\pgfpathlineto{\pgfqpoint{1.722536in}{1.565127in}}%
\pgfpathlineto{\pgfqpoint{1.726380in}{1.556295in}}%
\pgfpathlineto{\pgfqpoint{1.738193in}{1.546026in}}%
\pgfpathlineto{\pgfqpoint{1.748352in}{1.542684in}}%
\pgfpathlineto{\pgfqpoint{1.753849in}{1.541499in}}%
\pgfpathclose%
\pgfusepath{fill}%
\end{pgfscope}%
\begin{pgfscope}%
\pgfpathrectangle{\pgfqpoint{0.360415in}{0.358518in}}{\pgfqpoint{1.550000in}{1.347500in}}%
\pgfusepath{clip}%
\pgfsetbuttcap%
\pgfsetroundjoin%
\definecolor{currentfill}{rgb}{0.993326,0.602275,0.414390}%
\pgfsetfillcolor{currentfill}%
\pgfsetlinewidth{0.000000pt}%
\definecolor{currentstroke}{rgb}{0.000000,0.000000,0.000000}%
\pgfsetstrokecolor{currentstroke}%
\pgfsetdash{}{0pt}%
\pgfpathmoveto{\pgfqpoint{0.501324in}{0.438008in}}%
\pgfpathlineto{\pgfqpoint{0.516981in}{0.436483in}}%
\pgfpathlineto{\pgfqpoint{0.532637in}{0.438771in}}%
\pgfpathlineto{\pgfqpoint{0.536336in}{0.440184in}}%
\pgfpathlineto{\pgfqpoint{0.548294in}{0.445429in}}%
\pgfpathlineto{\pgfqpoint{0.560239in}{0.453795in}}%
\pgfpathlineto{\pgfqpoint{0.563950in}{0.457264in}}%
\pgfpathlineto{\pgfqpoint{0.572364in}{0.467407in}}%
\pgfpathlineto{\pgfqpoint{0.579015in}{0.481018in}}%
\pgfpathlineto{\pgfqpoint{0.579607in}{0.485244in}}%
\pgfpathlineto{\pgfqpoint{0.580803in}{0.494629in}}%
\pgfpathlineto{\pgfqpoint{0.579607in}{0.500907in}}%
\pgfpathlineto{\pgfqpoint{0.578066in}{0.508240in}}%
\pgfpathlineto{\pgfqpoint{0.570461in}{0.521851in}}%
\pgfpathlineto{\pgfqpoint{0.563950in}{0.529173in}}%
\pgfpathlineto{\pgfqpoint{0.556717in}{0.535462in}}%
\pgfpathlineto{\pgfqpoint{0.548294in}{0.541122in}}%
\pgfpathlineto{\pgfqpoint{0.532637in}{0.547733in}}%
\pgfpathlineto{\pgfqpoint{0.524202in}{0.549073in}}%
\pgfpathlineto{\pgfqpoint{0.516981in}{0.550113in}}%
\pgfpathlineto{\pgfqpoint{0.506186in}{0.549073in}}%
\pgfpathlineto{\pgfqpoint{0.501324in}{0.548559in}}%
\pgfpathlineto{\pgfqpoint{0.485668in}{0.542776in}}%
\pgfpathlineto{\pgfqpoint{0.474001in}{0.535462in}}%
\pgfpathlineto{\pgfqpoint{0.470011in}{0.532236in}}%
\pgfpathlineto{\pgfqpoint{0.460387in}{0.521851in}}%
\pgfpathlineto{\pgfqpoint{0.454354in}{0.511455in}}%
\pgfpathlineto{\pgfqpoint{0.452729in}{0.508240in}}%
\pgfpathlineto{\pgfqpoint{0.450097in}{0.494629in}}%
\pgfpathlineto{\pgfqpoint{0.451851in}{0.481018in}}%
\pgfpathlineto{\pgfqpoint{0.454354in}{0.475396in}}%
\pgfpathlineto{\pgfqpoint{0.458424in}{0.467407in}}%
\pgfpathlineto{\pgfqpoint{0.470011in}{0.454005in}}%
\pgfpathlineto{\pgfqpoint{0.470252in}{0.453795in}}%
\pgfpathlineto{\pgfqpoint{0.485668in}{0.443722in}}%
\pgfpathlineto{\pgfqpoint{0.494857in}{0.440184in}}%
\pgfpathlineto{\pgfqpoint{0.501324in}{0.438008in}}%
\pgfpathclose%
\pgfpathmoveto{\pgfqpoint{0.499054in}{0.467407in}}%
\pgfpathlineto{\pgfqpoint{0.485668in}{0.479044in}}%
\pgfpathlineto{\pgfqpoint{0.484288in}{0.481018in}}%
\pgfpathlineto{\pgfqpoint{0.481514in}{0.494629in}}%
\pgfpathlineto{\pgfqpoint{0.485668in}{0.508214in}}%
\pgfpathlineto{\pgfqpoint{0.485680in}{0.508240in}}%
\pgfpathlineto{\pgfqpoint{0.501324in}{0.520220in}}%
\pgfpathlineto{\pgfqpoint{0.508762in}{0.521851in}}%
\pgfpathlineto{\pgfqpoint{0.516981in}{0.523036in}}%
\pgfpathlineto{\pgfqpoint{0.522478in}{0.521851in}}%
\pgfpathlineto{\pgfqpoint{0.532637in}{0.518509in}}%
\pgfpathlineto{\pgfqpoint{0.544450in}{0.508240in}}%
\pgfpathlineto{\pgfqpoint{0.548294in}{0.499408in}}%
\pgfpathlineto{\pgfqpoint{0.549657in}{0.494629in}}%
\pgfpathlineto{\pgfqpoint{0.548294in}{0.487484in}}%
\pgfpathlineto{\pgfqpoint{0.546418in}{0.481018in}}%
\pgfpathlineto{\pgfqpoint{0.532637in}{0.467418in}}%
\pgfpathlineto{\pgfqpoint{0.532608in}{0.467407in}}%
\pgfpathlineto{\pgfqpoint{0.516981in}{0.463796in}}%
\pgfpathlineto{\pgfqpoint{0.501324in}{0.466207in}}%
\pgfpathlineto{\pgfqpoint{0.499054in}{0.467407in}}%
\pgfpathclose%
\pgfpathmoveto{\pgfqpoint{0.814455in}{0.437398in}}%
\pgfpathlineto{\pgfqpoint{0.830112in}{0.436635in}}%
\pgfpathlineto{\pgfqpoint{0.845769in}{0.439688in}}%
\pgfpathlineto{\pgfqpoint{0.846933in}{0.440184in}}%
\pgfpathlineto{\pgfqpoint{0.861425in}{0.447306in}}%
\pgfpathlineto{\pgfqpoint{0.870128in}{0.453795in}}%
\pgfpathlineto{\pgfqpoint{0.877082in}{0.460735in}}%
\pgfpathlineto{\pgfqpoint{0.882414in}{0.467407in}}%
\pgfpathlineto{\pgfqpoint{0.888893in}{0.481018in}}%
\pgfpathlineto{\pgfqpoint{0.890741in}{0.494629in}}%
\pgfpathlineto{\pgfqpoint{0.887968in}{0.508240in}}%
\pgfpathlineto{\pgfqpoint{0.880561in}{0.521851in}}%
\pgfpathlineto{\pgfqpoint{0.877082in}{0.525912in}}%
\pgfpathlineto{\pgfqpoint{0.866793in}{0.535462in}}%
\pgfpathlineto{\pgfqpoint{0.861425in}{0.539302in}}%
\pgfpathlineto{\pgfqpoint{0.845769in}{0.546743in}}%
\pgfpathlineto{\pgfqpoint{0.834760in}{0.549073in}}%
\pgfpathlineto{\pgfqpoint{0.830112in}{0.549962in}}%
\pgfpathlineto{\pgfqpoint{0.814455in}{0.549206in}}%
\pgfpathlineto{\pgfqpoint{0.813987in}{0.549073in}}%
\pgfpathlineto{\pgfqpoint{0.798799in}{0.544264in}}%
\pgfpathlineto{\pgfqpoint{0.783631in}{0.535462in}}%
\pgfpathlineto{\pgfqpoint{0.783142in}{0.535097in}}%
\pgfpathlineto{\pgfqpoint{0.770289in}{0.521851in}}%
\pgfpathlineto{\pgfqpoint{0.767486in}{0.517250in}}%
\pgfpathlineto{\pgfqpoint{0.762815in}{0.508240in}}%
\pgfpathlineto{\pgfqpoint{0.760149in}{0.494629in}}%
\pgfpathlineto{\pgfqpoint{0.761926in}{0.481018in}}%
\pgfpathlineto{\pgfqpoint{0.767486in}{0.468841in}}%
\pgfpathlineto{\pgfqpoint{0.768253in}{0.467407in}}%
\pgfpathlineto{\pgfqpoint{0.780465in}{0.453795in}}%
\pgfpathlineto{\pgfqpoint{0.783142in}{0.451572in}}%
\pgfpathlineto{\pgfqpoint{0.798799in}{0.442187in}}%
\pgfpathlineto{\pgfqpoint{0.804853in}{0.440184in}}%
\pgfpathlineto{\pgfqpoint{0.814455in}{0.437398in}}%
\pgfpathclose%
\pgfpathmoveto{\pgfqpoint{0.809705in}{0.467407in}}%
\pgfpathlineto{\pgfqpoint{0.798799in}{0.475554in}}%
\pgfpathlineto{\pgfqpoint{0.794672in}{0.481018in}}%
\pgfpathlineto{\pgfqpoint{0.791695in}{0.494629in}}%
\pgfpathlineto{\pgfqpoint{0.796162in}{0.508240in}}%
\pgfpathlineto{\pgfqpoint{0.798799in}{0.511316in}}%
\pgfpathlineto{\pgfqpoint{0.814455in}{0.521588in}}%
\pgfpathlineto{\pgfqpoint{0.816853in}{0.521851in}}%
\pgfpathlineto{\pgfqpoint{0.830112in}{0.522810in}}%
\pgfpathlineto{\pgfqpoint{0.833461in}{0.521851in}}%
\pgfpathlineto{\pgfqpoint{0.845769in}{0.516455in}}%
\pgfpathlineto{\pgfqpoint{0.854231in}{0.508240in}}%
\pgfpathlineto{\pgfqpoint{0.859539in}{0.494629in}}%
\pgfpathlineto{\pgfqpoint{0.856001in}{0.481018in}}%
\pgfpathlineto{\pgfqpoint{0.845769in}{0.469741in}}%
\pgfpathlineto{\pgfqpoint{0.841094in}{0.467407in}}%
\pgfpathlineto{\pgfqpoint{0.830112in}{0.464037in}}%
\pgfpathlineto{\pgfqpoint{0.814455in}{0.465242in}}%
\pgfpathlineto{\pgfqpoint{0.809705in}{0.467407in}}%
\pgfpathclose%
\pgfpathmoveto{\pgfqpoint{1.127587in}{0.436940in}}%
\pgfpathlineto{\pgfqpoint{1.143243in}{0.436940in}}%
\pgfpathlineto{\pgfqpoint{1.156586in}{0.440184in}}%
\pgfpathlineto{\pgfqpoint{1.158900in}{0.440823in}}%
\pgfpathlineto{\pgfqpoint{1.174556in}{0.449354in}}%
\pgfpathlineto{\pgfqpoint{1.180186in}{0.453795in}}%
\pgfpathlineto{\pgfqpoint{1.190213in}{0.464415in}}%
\pgfpathlineto{\pgfqpoint{1.192526in}{0.467407in}}%
\pgfpathlineto{\pgfqpoint{1.198866in}{0.481018in}}%
\pgfpathlineto{\pgfqpoint{1.200674in}{0.494629in}}%
\pgfpathlineto{\pgfqpoint{1.197962in}{0.508240in}}%
\pgfpathlineto{\pgfqpoint{1.190713in}{0.521851in}}%
\pgfpathlineto{\pgfqpoint{1.190213in}{0.522454in}}%
\pgfpathlineto{\pgfqpoint{1.177007in}{0.535462in}}%
\pgfpathlineto{\pgfqpoint{1.174556in}{0.537317in}}%
\pgfpathlineto{\pgfqpoint{1.158900in}{0.545586in}}%
\pgfpathlineto{\pgfqpoint{1.145709in}{0.549073in}}%
\pgfpathlineto{\pgfqpoint{1.143243in}{0.549660in}}%
\pgfpathlineto{\pgfqpoint{1.127587in}{0.549660in}}%
\pgfpathlineto{\pgfqpoint{1.125121in}{0.549073in}}%
\pgfpathlineto{\pgfqpoint{1.111930in}{0.545586in}}%
\pgfpathlineto{\pgfqpoint{1.096274in}{0.537317in}}%
\pgfpathlineto{\pgfqpoint{1.093823in}{0.535462in}}%
\pgfpathlineto{\pgfqpoint{1.080617in}{0.522454in}}%
\pgfpathlineto{\pgfqpoint{1.080117in}{0.521851in}}%
\pgfpathlineto{\pgfqpoint{1.072868in}{0.508240in}}%
\pgfpathlineto{\pgfqpoint{1.070156in}{0.494629in}}%
\pgfpathlineto{\pgfqpoint{1.071964in}{0.481018in}}%
\pgfpathlineto{\pgfqpoint{1.078304in}{0.467407in}}%
\pgfpathlineto{\pgfqpoint{1.080617in}{0.464415in}}%
\pgfpathlineto{\pgfqpoint{1.090644in}{0.453795in}}%
\pgfpathlineto{\pgfqpoint{1.096274in}{0.449354in}}%
\pgfpathlineto{\pgfqpoint{1.111930in}{0.440823in}}%
\pgfpathlineto{\pgfqpoint{1.114244in}{0.440184in}}%
\pgfpathlineto{\pgfqpoint{1.127587in}{0.436940in}}%
\pgfpathclose%
\pgfpathmoveto{\pgfqpoint{1.120022in}{0.467407in}}%
\pgfpathlineto{\pgfqpoint{1.111930in}{0.472453in}}%
\pgfpathlineto{\pgfqpoint{1.104878in}{0.481018in}}%
\pgfpathlineto{\pgfqpoint{1.101651in}{0.494629in}}%
\pgfpathlineto{\pgfqpoint{1.106493in}{0.508240in}}%
\pgfpathlineto{\pgfqpoint{1.111930in}{0.514057in}}%
\pgfpathlineto{\pgfqpoint{1.126167in}{0.521851in}}%
\pgfpathlineto{\pgfqpoint{1.127587in}{0.522357in}}%
\pgfpathlineto{\pgfqpoint{1.143243in}{0.522357in}}%
\pgfpathlineto{\pgfqpoint{1.144663in}{0.521851in}}%
\pgfpathlineto{\pgfqpoint{1.158900in}{0.514057in}}%
\pgfpathlineto{\pgfqpoint{1.164337in}{0.508240in}}%
\pgfpathlineto{\pgfqpoint{1.169179in}{0.494629in}}%
\pgfpathlineto{\pgfqpoint{1.165952in}{0.481018in}}%
\pgfpathlineto{\pgfqpoint{1.158900in}{0.472453in}}%
\pgfpathlineto{\pgfqpoint{1.150808in}{0.467407in}}%
\pgfpathlineto{\pgfqpoint{1.143243in}{0.464519in}}%
\pgfpathlineto{\pgfqpoint{1.127587in}{0.464519in}}%
\pgfpathlineto{\pgfqpoint{1.120022in}{0.467407in}}%
\pgfpathclose%
\pgfpathmoveto{\pgfqpoint{1.425061in}{0.439688in}}%
\pgfpathlineto{\pgfqpoint{1.440718in}{0.436635in}}%
\pgfpathlineto{\pgfqpoint{1.456375in}{0.437398in}}%
\pgfpathlineto{\pgfqpoint{1.465977in}{0.440184in}}%
\pgfpathlineto{\pgfqpoint{1.472031in}{0.442187in}}%
\pgfpathlineto{\pgfqpoint{1.487688in}{0.451572in}}%
\pgfpathlineto{\pgfqpoint{1.490365in}{0.453795in}}%
\pgfpathlineto{\pgfqpoint{1.502577in}{0.467407in}}%
\pgfpathlineto{\pgfqpoint{1.503344in}{0.468841in}}%
\pgfpathlineto{\pgfqpoint{1.508904in}{0.481018in}}%
\pgfpathlineto{\pgfqpoint{1.510681in}{0.494629in}}%
\pgfpathlineto{\pgfqpoint{1.508015in}{0.508240in}}%
\pgfpathlineto{\pgfqpoint{1.503344in}{0.517250in}}%
\pgfpathlineto{\pgfqpoint{1.500541in}{0.521851in}}%
\pgfpathlineto{\pgfqpoint{1.487688in}{0.535097in}}%
\pgfpathlineto{\pgfqpoint{1.487199in}{0.535462in}}%
\pgfpathlineto{\pgfqpoint{1.472031in}{0.544264in}}%
\pgfpathlineto{\pgfqpoint{1.456843in}{0.549073in}}%
\pgfpathlineto{\pgfqpoint{1.456375in}{0.549206in}}%
\pgfpathlineto{\pgfqpoint{1.440718in}{0.549962in}}%
\pgfpathlineto{\pgfqpoint{1.436070in}{0.549073in}}%
\pgfpathlineto{\pgfqpoint{1.425061in}{0.546743in}}%
\pgfpathlineto{\pgfqpoint{1.409405in}{0.539302in}}%
\pgfpathlineto{\pgfqpoint{1.404037in}{0.535462in}}%
\pgfpathlineto{\pgfqpoint{1.393748in}{0.525912in}}%
\pgfpathlineto{\pgfqpoint{1.390269in}{0.521851in}}%
\pgfpathlineto{\pgfqpoint{1.382862in}{0.508240in}}%
\pgfpathlineto{\pgfqpoint{1.380089in}{0.494629in}}%
\pgfpathlineto{\pgfqpoint{1.381937in}{0.481018in}}%
\pgfpathlineto{\pgfqpoint{1.388416in}{0.467407in}}%
\pgfpathlineto{\pgfqpoint{1.393748in}{0.460735in}}%
\pgfpathlineto{\pgfqpoint{1.400702in}{0.453795in}}%
\pgfpathlineto{\pgfqpoint{1.409405in}{0.447306in}}%
\pgfpathlineto{\pgfqpoint{1.423897in}{0.440184in}}%
\pgfpathlineto{\pgfqpoint{1.425061in}{0.439688in}}%
\pgfpathclose%
\pgfpathmoveto{\pgfqpoint{1.429736in}{0.467407in}}%
\pgfpathlineto{\pgfqpoint{1.425061in}{0.469741in}}%
\pgfpathlineto{\pgfqpoint{1.414829in}{0.481018in}}%
\pgfpathlineto{\pgfqpoint{1.411291in}{0.494629in}}%
\pgfpathlineto{\pgfqpoint{1.416599in}{0.508240in}}%
\pgfpathlineto{\pgfqpoint{1.425061in}{0.516455in}}%
\pgfpathlineto{\pgfqpoint{1.437369in}{0.521851in}}%
\pgfpathlineto{\pgfqpoint{1.440718in}{0.522810in}}%
\pgfpathlineto{\pgfqpoint{1.453977in}{0.521851in}}%
\pgfpathlineto{\pgfqpoint{1.456375in}{0.521588in}}%
\pgfpathlineto{\pgfqpoint{1.472031in}{0.511316in}}%
\pgfpathlineto{\pgfqpoint{1.474668in}{0.508240in}}%
\pgfpathlineto{\pgfqpoint{1.479135in}{0.494629in}}%
\pgfpathlineto{\pgfqpoint{1.476158in}{0.481018in}}%
\pgfpathlineto{\pgfqpoint{1.472031in}{0.475554in}}%
\pgfpathlineto{\pgfqpoint{1.461125in}{0.467407in}}%
\pgfpathlineto{\pgfqpoint{1.456375in}{0.465242in}}%
\pgfpathlineto{\pgfqpoint{1.440718in}{0.464037in}}%
\pgfpathlineto{\pgfqpoint{1.429736in}{0.467407in}}%
\pgfpathclose%
\pgfpathmoveto{\pgfqpoint{1.738193in}{0.438771in}}%
\pgfpathlineto{\pgfqpoint{1.753849in}{0.436483in}}%
\pgfpathlineto{\pgfqpoint{1.769506in}{0.438008in}}%
\pgfpathlineto{\pgfqpoint{1.775973in}{0.440184in}}%
\pgfpathlineto{\pgfqpoint{1.785162in}{0.443722in}}%
\pgfpathlineto{\pgfqpoint{1.800578in}{0.453795in}}%
\pgfpathlineto{\pgfqpoint{1.800819in}{0.454005in}}%
\pgfpathlineto{\pgfqpoint{1.812406in}{0.467407in}}%
\pgfpathlineto{\pgfqpoint{1.816476in}{0.475396in}}%
\pgfpathlineto{\pgfqpoint{1.818979in}{0.481018in}}%
\pgfpathlineto{\pgfqpoint{1.820733in}{0.494629in}}%
\pgfpathlineto{\pgfqpoint{1.818101in}{0.508240in}}%
\pgfpathlineto{\pgfqpoint{1.816476in}{0.511455in}}%
\pgfpathlineto{\pgfqpoint{1.810443in}{0.521851in}}%
\pgfpathlineto{\pgfqpoint{1.800819in}{0.532236in}}%
\pgfpathlineto{\pgfqpoint{1.796829in}{0.535462in}}%
\pgfpathlineto{\pgfqpoint{1.785162in}{0.542776in}}%
\pgfpathlineto{\pgfqpoint{1.769506in}{0.548559in}}%
\pgfpathlineto{\pgfqpoint{1.764644in}{0.549073in}}%
\pgfpathlineto{\pgfqpoint{1.753849in}{0.550113in}}%
\pgfpathlineto{\pgfqpoint{1.746628in}{0.549073in}}%
\pgfpathlineto{\pgfqpoint{1.738193in}{0.547733in}}%
\pgfpathlineto{\pgfqpoint{1.722536in}{0.541122in}}%
\pgfpathlineto{\pgfqpoint{1.714113in}{0.535462in}}%
\pgfpathlineto{\pgfqpoint{1.706880in}{0.529173in}}%
\pgfpathlineto{\pgfqpoint{1.700369in}{0.521851in}}%
\pgfpathlineto{\pgfqpoint{1.692764in}{0.508240in}}%
\pgfpathlineto{\pgfqpoint{1.691223in}{0.500907in}}%
\pgfpathlineto{\pgfqpoint{1.690027in}{0.494629in}}%
\pgfpathlineto{\pgfqpoint{1.691223in}{0.485244in}}%
\pgfpathlineto{\pgfqpoint{1.691815in}{0.481018in}}%
\pgfpathlineto{\pgfqpoint{1.698466in}{0.467407in}}%
\pgfpathlineto{\pgfqpoint{1.706880in}{0.457264in}}%
\pgfpathlineto{\pgfqpoint{1.710591in}{0.453795in}}%
\pgfpathlineto{\pgfqpoint{1.722536in}{0.445429in}}%
\pgfpathlineto{\pgfqpoint{1.734494in}{0.440184in}}%
\pgfpathlineto{\pgfqpoint{1.738193in}{0.438771in}}%
\pgfpathclose%
\pgfpathmoveto{\pgfqpoint{1.738222in}{0.467407in}}%
\pgfpathlineto{\pgfqpoint{1.738193in}{0.467418in}}%
\pgfpathlineto{\pgfqpoint{1.724412in}{0.481018in}}%
\pgfpathlineto{\pgfqpoint{1.722536in}{0.487484in}}%
\pgfpathlineto{\pgfqpoint{1.721173in}{0.494629in}}%
\pgfpathlineto{\pgfqpoint{1.722536in}{0.499408in}}%
\pgfpathlineto{\pgfqpoint{1.726380in}{0.508240in}}%
\pgfpathlineto{\pgfqpoint{1.738193in}{0.518509in}}%
\pgfpathlineto{\pgfqpoint{1.748352in}{0.521851in}}%
\pgfpathlineto{\pgfqpoint{1.753849in}{0.523036in}}%
\pgfpathlineto{\pgfqpoint{1.762068in}{0.521851in}}%
\pgfpathlineto{\pgfqpoint{1.769506in}{0.520220in}}%
\pgfpathlineto{\pgfqpoint{1.785150in}{0.508240in}}%
\pgfpathlineto{\pgfqpoint{1.785162in}{0.508214in}}%
\pgfpathlineto{\pgfqpoint{1.789316in}{0.494629in}}%
\pgfpathlineto{\pgfqpoint{1.786542in}{0.481018in}}%
\pgfpathlineto{\pgfqpoint{1.785162in}{0.479044in}}%
\pgfpathlineto{\pgfqpoint{1.771776in}{0.467407in}}%
\pgfpathlineto{\pgfqpoint{1.769506in}{0.466207in}}%
\pgfpathlineto{\pgfqpoint{1.753849in}{0.463796in}}%
\pgfpathlineto{\pgfqpoint{1.738222in}{0.467407in}}%
\pgfpathclose%
\pgfpathmoveto{\pgfqpoint{0.501324in}{0.707573in}}%
\pgfpathlineto{\pgfqpoint{0.516981in}{0.706028in}}%
\pgfpathlineto{\pgfqpoint{0.532637in}{0.708346in}}%
\pgfpathlineto{\pgfqpoint{0.543002in}{0.712407in}}%
\pgfpathlineto{\pgfqpoint{0.548294in}{0.714843in}}%
\pgfpathlineto{\pgfqpoint{0.563531in}{0.726018in}}%
\pgfpathlineto{\pgfqpoint{0.563950in}{0.726443in}}%
\pgfpathlineto{\pgfqpoint{0.574075in}{0.739629in}}%
\pgfpathlineto{\pgfqpoint{0.579607in}{0.752832in}}%
\pgfpathlineto{\pgfqpoint{0.579760in}{0.753240in}}%
\pgfpathlineto{\pgfqpoint{0.580629in}{0.766851in}}%
\pgfpathlineto{\pgfqpoint{0.579607in}{0.770892in}}%
\pgfpathlineto{\pgfqpoint{0.576926in}{0.780462in}}%
\pgfpathlineto{\pgfqpoint{0.568368in}{0.794073in}}%
\pgfpathlineto{\pgfqpoint{0.563950in}{0.798740in}}%
\pgfpathlineto{\pgfqpoint{0.552965in}{0.807684in}}%
\pgfpathlineto{\pgfqpoint{0.548294in}{0.810709in}}%
\pgfpathlineto{\pgfqpoint{0.532637in}{0.817149in}}%
\pgfpathlineto{\pgfqpoint{0.516981in}{0.819559in}}%
\pgfpathlineto{\pgfqpoint{0.501324in}{0.817953in}}%
\pgfpathlineto{\pgfqpoint{0.485668in}{0.812320in}}%
\pgfpathlineto{\pgfqpoint{0.477994in}{0.807684in}}%
\pgfpathlineto{\pgfqpoint{0.470011in}{0.801639in}}%
\pgfpathlineto{\pgfqpoint{0.462547in}{0.794073in}}%
\pgfpathlineto{\pgfqpoint{0.454354in}{0.781474in}}%
\pgfpathlineto{\pgfqpoint{0.453783in}{0.780462in}}%
\pgfpathlineto{\pgfqpoint{0.450272in}{0.766851in}}%
\pgfpathlineto{\pgfqpoint{0.451149in}{0.753240in}}%
\pgfpathlineto{\pgfqpoint{0.454354in}{0.744892in}}%
\pgfpathlineto{\pgfqpoint{0.456658in}{0.739629in}}%
\pgfpathlineto{\pgfqpoint{0.467454in}{0.726018in}}%
\pgfpathlineto{\pgfqpoint{0.470011in}{0.723690in}}%
\pgfpathlineto{\pgfqpoint{0.485668in}{0.713074in}}%
\pgfpathlineto{\pgfqpoint{0.487318in}{0.712407in}}%
\pgfpathlineto{\pgfqpoint{0.501324in}{0.707573in}}%
\pgfpathclose%
\pgfpathmoveto{\pgfqpoint{0.495039in}{0.739629in}}%
\pgfpathlineto{\pgfqpoint{0.485668in}{0.749110in}}%
\pgfpathlineto{\pgfqpoint{0.483178in}{0.753240in}}%
\pgfpathlineto{\pgfqpoint{0.481791in}{0.766851in}}%
\pgfpathlineto{\pgfqpoint{0.485668in}{0.776399in}}%
\pgfpathlineto{\pgfqpoint{0.488353in}{0.780462in}}%
\pgfpathlineto{\pgfqpoint{0.501324in}{0.789358in}}%
\pgfpathlineto{\pgfqpoint{0.516981in}{0.792434in}}%
\pgfpathlineto{\pgfqpoint{0.532637in}{0.787819in}}%
\pgfpathlineto{\pgfqpoint{0.542087in}{0.780462in}}%
\pgfpathlineto{\pgfqpoint{0.548294in}{0.769762in}}%
\pgfpathlineto{\pgfqpoint{0.549397in}{0.766851in}}%
\pgfpathlineto{\pgfqpoint{0.548294in}{0.755324in}}%
\pgfpathlineto{\pgfqpoint{0.547992in}{0.753240in}}%
\pgfpathlineto{\pgfqpoint{0.536175in}{0.739629in}}%
\pgfpathlineto{\pgfqpoint{0.532637in}{0.737336in}}%
\pgfpathlineto{\pgfqpoint{0.516981in}{0.733453in}}%
\pgfpathlineto{\pgfqpoint{0.501324in}{0.736041in}}%
\pgfpathlineto{\pgfqpoint{0.495039in}{0.739629in}}%
\pgfpathclose%
\pgfpathmoveto{\pgfqpoint{0.798799in}{0.711596in}}%
\pgfpathlineto{\pgfqpoint{0.814455in}{0.706955in}}%
\pgfpathlineto{\pgfqpoint{0.830112in}{0.706182in}}%
\pgfpathlineto{\pgfqpoint{0.845769in}{0.709274in}}%
\pgfpathlineto{\pgfqpoint{0.852929in}{0.712407in}}%
\pgfpathlineto{\pgfqpoint{0.861425in}{0.716790in}}%
\pgfpathlineto{\pgfqpoint{0.873244in}{0.726018in}}%
\pgfpathlineto{\pgfqpoint{0.877082in}{0.730168in}}%
\pgfpathlineto{\pgfqpoint{0.884081in}{0.739629in}}%
\pgfpathlineto{\pgfqpoint{0.889632in}{0.753240in}}%
\pgfpathlineto{\pgfqpoint{0.890556in}{0.766851in}}%
\pgfpathlineto{\pgfqpoint{0.886858in}{0.780462in}}%
\pgfpathlineto{\pgfqpoint{0.878522in}{0.794073in}}%
\pgfpathlineto{\pgfqpoint{0.877082in}{0.795652in}}%
\pgfpathlineto{\pgfqpoint{0.863241in}{0.807684in}}%
\pgfpathlineto{\pgfqpoint{0.861425in}{0.808936in}}%
\pgfpathlineto{\pgfqpoint{0.845769in}{0.816184in}}%
\pgfpathlineto{\pgfqpoint{0.830112in}{0.819398in}}%
\pgfpathlineto{\pgfqpoint{0.814455in}{0.818595in}}%
\pgfpathlineto{\pgfqpoint{0.798799in}{0.813770in}}%
\pgfpathlineto{\pgfqpoint{0.787917in}{0.807684in}}%
\pgfpathlineto{\pgfqpoint{0.783142in}{0.804348in}}%
\pgfpathlineto{\pgfqpoint{0.772528in}{0.794073in}}%
\pgfpathlineto{\pgfqpoint{0.767486in}{0.786687in}}%
\pgfpathlineto{\pgfqpoint{0.763883in}{0.780462in}}%
\pgfpathlineto{\pgfqpoint{0.760326in}{0.766851in}}%
\pgfpathlineto{\pgfqpoint{0.761215in}{0.753240in}}%
\pgfpathlineto{\pgfqpoint{0.766554in}{0.739629in}}%
\pgfpathlineto{\pgfqpoint{0.767486in}{0.738289in}}%
\pgfpathlineto{\pgfqpoint{0.777617in}{0.726018in}}%
\pgfpathlineto{\pgfqpoint{0.783142in}{0.721214in}}%
\pgfpathlineto{\pgfqpoint{0.797258in}{0.712407in}}%
\pgfpathlineto{\pgfqpoint{0.798799in}{0.711596in}}%
\pgfpathclose%
\pgfpathmoveto{\pgfqpoint{0.805063in}{0.739629in}}%
\pgfpathlineto{\pgfqpoint{0.798799in}{0.745075in}}%
\pgfpathlineto{\pgfqpoint{0.793481in}{0.753240in}}%
\pgfpathlineto{\pgfqpoint{0.791993in}{0.766851in}}%
\pgfpathlineto{\pgfqpoint{0.797950in}{0.780462in}}%
\pgfpathlineto{\pgfqpoint{0.798799in}{0.781349in}}%
\pgfpathlineto{\pgfqpoint{0.814455in}{0.790589in}}%
\pgfpathlineto{\pgfqpoint{0.830112in}{0.792126in}}%
\pgfpathlineto{\pgfqpoint{0.845769in}{0.785971in}}%
\pgfpathlineto{\pgfqpoint{0.852106in}{0.780462in}}%
\pgfpathlineto{\pgfqpoint{0.859185in}{0.766851in}}%
\pgfpathlineto{\pgfqpoint{0.857417in}{0.753240in}}%
\pgfpathlineto{\pgfqpoint{0.846788in}{0.739629in}}%
\pgfpathlineto{\pgfqpoint{0.845769in}{0.738891in}}%
\pgfpathlineto{\pgfqpoint{0.830112in}{0.733712in}}%
\pgfpathlineto{\pgfqpoint{0.814455in}{0.735006in}}%
\pgfpathlineto{\pgfqpoint{0.805063in}{0.739629in}}%
\pgfpathclose%
\pgfpathmoveto{\pgfqpoint{1.111930in}{0.710358in}}%
\pgfpathlineto{\pgfqpoint{1.127587in}{0.706491in}}%
\pgfpathlineto{\pgfqpoint{1.143243in}{0.706491in}}%
\pgfpathlineto{\pgfqpoint{1.158900in}{0.710358in}}%
\pgfpathlineto{\pgfqpoint{1.163150in}{0.712407in}}%
\pgfpathlineto{\pgfqpoint{1.174556in}{0.718914in}}%
\pgfpathlineto{\pgfqpoint{1.183155in}{0.726018in}}%
\pgfpathlineto{\pgfqpoint{1.190213in}{0.734118in}}%
\pgfpathlineto{\pgfqpoint{1.194158in}{0.739629in}}%
\pgfpathlineto{\pgfqpoint{1.199590in}{0.753240in}}%
\pgfpathlineto{\pgfqpoint{1.200494in}{0.766851in}}%
\pgfpathlineto{\pgfqpoint{1.196875in}{0.780462in}}%
\pgfpathlineto{\pgfqpoint{1.190213in}{0.791643in}}%
\pgfpathlineto{\pgfqpoint{1.188462in}{0.794073in}}%
\pgfpathlineto{\pgfqpoint{1.174556in}{0.806865in}}%
\pgfpathlineto{\pgfqpoint{1.173274in}{0.807684in}}%
\pgfpathlineto{\pgfqpoint{1.158900in}{0.815057in}}%
\pgfpathlineto{\pgfqpoint{1.143243in}{0.819077in}}%
\pgfpathlineto{\pgfqpoint{1.127587in}{0.819077in}}%
\pgfpathlineto{\pgfqpoint{1.111930in}{0.815057in}}%
\pgfpathlineto{\pgfqpoint{1.097556in}{0.807684in}}%
\pgfpathlineto{\pgfqpoint{1.096274in}{0.806865in}}%
\pgfpathlineto{\pgfqpoint{1.082368in}{0.794073in}}%
\pgfpathlineto{\pgfqpoint{1.080617in}{0.791643in}}%
\pgfpathlineto{\pgfqpoint{1.073955in}{0.780462in}}%
\pgfpathlineto{\pgfqpoint{1.070336in}{0.766851in}}%
\pgfpathlineto{\pgfqpoint{1.071240in}{0.753240in}}%
\pgfpathlineto{\pgfqpoint{1.076672in}{0.739629in}}%
\pgfpathlineto{\pgfqpoint{1.080617in}{0.734118in}}%
\pgfpathlineto{\pgfqpoint{1.087675in}{0.726018in}}%
\pgfpathlineto{\pgfqpoint{1.096274in}{0.718914in}}%
\pgfpathlineto{\pgfqpoint{1.107680in}{0.712407in}}%
\pgfpathlineto{\pgfqpoint{1.111930in}{0.710358in}}%
\pgfpathclose%
\pgfpathmoveto{\pgfqpoint{1.114494in}{0.739629in}}%
\pgfpathlineto{\pgfqpoint{1.111930in}{0.741489in}}%
\pgfpathlineto{\pgfqpoint{1.103587in}{0.753240in}}%
\pgfpathlineto{\pgfqpoint{1.101974in}{0.766851in}}%
\pgfpathlineto{\pgfqpoint{1.108431in}{0.780462in}}%
\pgfpathlineto{\pgfqpoint{1.111930in}{0.783815in}}%
\pgfpathlineto{\pgfqpoint{1.127587in}{0.791511in}}%
\pgfpathlineto{\pgfqpoint{1.143243in}{0.791511in}}%
\pgfpathlineto{\pgfqpoint{1.158900in}{0.783815in}}%
\pgfpathlineto{\pgfqpoint{1.162399in}{0.780462in}}%
\pgfpathlineto{\pgfqpoint{1.168856in}{0.766851in}}%
\pgfpathlineto{\pgfqpoint{1.167243in}{0.753240in}}%
\pgfpathlineto{\pgfqpoint{1.158900in}{0.741489in}}%
\pgfpathlineto{\pgfqpoint{1.156336in}{0.739629in}}%
\pgfpathlineto{\pgfqpoint{1.143243in}{0.734229in}}%
\pgfpathlineto{\pgfqpoint{1.127587in}{0.734229in}}%
\pgfpathlineto{\pgfqpoint{1.114494in}{0.739629in}}%
\pgfpathclose%
\pgfpathmoveto{\pgfqpoint{1.425061in}{0.709274in}}%
\pgfpathlineto{\pgfqpoint{1.440718in}{0.706182in}}%
\pgfpathlineto{\pgfqpoint{1.456375in}{0.706955in}}%
\pgfpathlineto{\pgfqpoint{1.472031in}{0.711596in}}%
\pgfpathlineto{\pgfqpoint{1.473572in}{0.712407in}}%
\pgfpathlineto{\pgfqpoint{1.487688in}{0.721214in}}%
\pgfpathlineto{\pgfqpoint{1.493213in}{0.726018in}}%
\pgfpathlineto{\pgfqpoint{1.503344in}{0.738289in}}%
\pgfpathlineto{\pgfqpoint{1.504276in}{0.739629in}}%
\pgfpathlineto{\pgfqpoint{1.509615in}{0.753240in}}%
\pgfpathlineto{\pgfqpoint{1.510504in}{0.766851in}}%
\pgfpathlineto{\pgfqpoint{1.506947in}{0.780462in}}%
\pgfpathlineto{\pgfqpoint{1.503344in}{0.786687in}}%
\pgfpathlineto{\pgfqpoint{1.498302in}{0.794073in}}%
\pgfpathlineto{\pgfqpoint{1.487688in}{0.804348in}}%
\pgfpathlineto{\pgfqpoint{1.482913in}{0.807684in}}%
\pgfpathlineto{\pgfqpoint{1.472031in}{0.813770in}}%
\pgfpathlineto{\pgfqpoint{1.456375in}{0.818595in}}%
\pgfpathlineto{\pgfqpoint{1.440718in}{0.819398in}}%
\pgfpathlineto{\pgfqpoint{1.425061in}{0.816184in}}%
\pgfpathlineto{\pgfqpoint{1.409405in}{0.808936in}}%
\pgfpathlineto{\pgfqpoint{1.407589in}{0.807684in}}%
\pgfpathlineto{\pgfqpoint{1.393748in}{0.795652in}}%
\pgfpathlineto{\pgfqpoint{1.392308in}{0.794073in}}%
\pgfpathlineto{\pgfqpoint{1.383972in}{0.780462in}}%
\pgfpathlineto{\pgfqpoint{1.380274in}{0.766851in}}%
\pgfpathlineto{\pgfqpoint{1.381198in}{0.753240in}}%
\pgfpathlineto{\pgfqpoint{1.386749in}{0.739629in}}%
\pgfpathlineto{\pgfqpoint{1.393748in}{0.730168in}}%
\pgfpathlineto{\pgfqpoint{1.397586in}{0.726018in}}%
\pgfpathlineto{\pgfqpoint{1.409405in}{0.716790in}}%
\pgfpathlineto{\pgfqpoint{1.417901in}{0.712407in}}%
\pgfpathlineto{\pgfqpoint{1.425061in}{0.709274in}}%
\pgfpathclose%
\pgfpathmoveto{\pgfqpoint{1.424042in}{0.739629in}}%
\pgfpathlineto{\pgfqpoint{1.413413in}{0.753240in}}%
\pgfpathlineto{\pgfqpoint{1.411645in}{0.766851in}}%
\pgfpathlineto{\pgfqpoint{1.418724in}{0.780462in}}%
\pgfpathlineto{\pgfqpoint{1.425061in}{0.785971in}}%
\pgfpathlineto{\pgfqpoint{1.440718in}{0.792126in}}%
\pgfpathlineto{\pgfqpoint{1.456375in}{0.790589in}}%
\pgfpathlineto{\pgfqpoint{1.472031in}{0.781349in}}%
\pgfpathlineto{\pgfqpoint{1.472880in}{0.780462in}}%
\pgfpathlineto{\pgfqpoint{1.478837in}{0.766851in}}%
\pgfpathlineto{\pgfqpoint{1.477349in}{0.753240in}}%
\pgfpathlineto{\pgfqpoint{1.472031in}{0.745075in}}%
\pgfpathlineto{\pgfqpoint{1.465767in}{0.739629in}}%
\pgfpathlineto{\pgfqpoint{1.456375in}{0.735006in}}%
\pgfpathlineto{\pgfqpoint{1.440718in}{0.733712in}}%
\pgfpathlineto{\pgfqpoint{1.425061in}{0.738891in}}%
\pgfpathlineto{\pgfqpoint{1.424042in}{0.739629in}}%
\pgfpathclose%
\pgfpathmoveto{\pgfqpoint{1.738193in}{0.708346in}}%
\pgfpathlineto{\pgfqpoint{1.753849in}{0.706028in}}%
\pgfpathlineto{\pgfqpoint{1.769506in}{0.707573in}}%
\pgfpathlineto{\pgfqpoint{1.783512in}{0.712407in}}%
\pgfpathlineto{\pgfqpoint{1.785162in}{0.713074in}}%
\pgfpathlineto{\pgfqpoint{1.800819in}{0.723690in}}%
\pgfpathlineto{\pgfqpoint{1.803376in}{0.726018in}}%
\pgfpathlineto{\pgfqpoint{1.814172in}{0.739629in}}%
\pgfpathlineto{\pgfqpoint{1.816476in}{0.744892in}}%
\pgfpathlineto{\pgfqpoint{1.819681in}{0.753240in}}%
\pgfpathlineto{\pgfqpoint{1.820558in}{0.766851in}}%
\pgfpathlineto{\pgfqpoint{1.817047in}{0.780462in}}%
\pgfpathlineto{\pgfqpoint{1.816476in}{0.781474in}}%
\pgfpathlineto{\pgfqpoint{1.808283in}{0.794073in}}%
\pgfpathlineto{\pgfqpoint{1.800819in}{0.801639in}}%
\pgfpathlineto{\pgfqpoint{1.792836in}{0.807684in}}%
\pgfpathlineto{\pgfqpoint{1.785162in}{0.812320in}}%
\pgfpathlineto{\pgfqpoint{1.769506in}{0.817953in}}%
\pgfpathlineto{\pgfqpoint{1.753849in}{0.819559in}}%
\pgfpathlineto{\pgfqpoint{1.738193in}{0.817149in}}%
\pgfpathlineto{\pgfqpoint{1.722536in}{0.810709in}}%
\pgfpathlineto{\pgfqpoint{1.717865in}{0.807684in}}%
\pgfpathlineto{\pgfqpoint{1.706880in}{0.798740in}}%
\pgfpathlineto{\pgfqpoint{1.702462in}{0.794073in}}%
\pgfpathlineto{\pgfqpoint{1.693904in}{0.780462in}}%
\pgfpathlineto{\pgfqpoint{1.691223in}{0.770892in}}%
\pgfpathlineto{\pgfqpoint{1.690201in}{0.766851in}}%
\pgfpathlineto{\pgfqpoint{1.691070in}{0.753240in}}%
\pgfpathlineto{\pgfqpoint{1.691223in}{0.752832in}}%
\pgfpathlineto{\pgfqpoint{1.696755in}{0.739629in}}%
\pgfpathlineto{\pgfqpoint{1.706880in}{0.726443in}}%
\pgfpathlineto{\pgfqpoint{1.707299in}{0.726018in}}%
\pgfpathlineto{\pgfqpoint{1.722536in}{0.714843in}}%
\pgfpathlineto{\pgfqpoint{1.727828in}{0.712407in}}%
\pgfpathlineto{\pgfqpoint{1.738193in}{0.708346in}}%
\pgfpathclose%
\pgfpathmoveto{\pgfqpoint{1.734655in}{0.739629in}}%
\pgfpathlineto{\pgfqpoint{1.722838in}{0.753240in}}%
\pgfpathlineto{\pgfqpoint{1.722536in}{0.755324in}}%
\pgfpathlineto{\pgfqpoint{1.721433in}{0.766851in}}%
\pgfpathlineto{\pgfqpoint{1.722536in}{0.769762in}}%
\pgfpathlineto{\pgfqpoint{1.728743in}{0.780462in}}%
\pgfpathlineto{\pgfqpoint{1.738193in}{0.787819in}}%
\pgfpathlineto{\pgfqpoint{1.753849in}{0.792434in}}%
\pgfpathlineto{\pgfqpoint{1.769506in}{0.789358in}}%
\pgfpathlineto{\pgfqpoint{1.782477in}{0.780462in}}%
\pgfpathlineto{\pgfqpoint{1.785162in}{0.776399in}}%
\pgfpathlineto{\pgfqpoint{1.789039in}{0.766851in}}%
\pgfpathlineto{\pgfqpoint{1.787652in}{0.753240in}}%
\pgfpathlineto{\pgfqpoint{1.785162in}{0.749110in}}%
\pgfpathlineto{\pgfqpoint{1.775791in}{0.739629in}}%
\pgfpathlineto{\pgfqpoint{1.769506in}{0.736041in}}%
\pgfpathlineto{\pgfqpoint{1.753849in}{0.733453in}}%
\pgfpathlineto{\pgfqpoint{1.738193in}{0.737336in}}%
\pgfpathlineto{\pgfqpoint{1.734655in}{0.739629in}}%
\pgfpathclose%
\pgfpathmoveto{\pgfqpoint{0.485668in}{0.982618in}}%
\pgfpathlineto{\pgfqpoint{0.501324in}{0.977106in}}%
\pgfpathlineto{\pgfqpoint{0.516981in}{0.975534in}}%
\pgfpathlineto{\pgfqpoint{0.532637in}{0.977892in}}%
\pgfpathlineto{\pgfqpoint{0.548294in}{0.984194in}}%
\pgfpathlineto{\pgfqpoint{0.548987in}{0.984629in}}%
\pgfpathlineto{\pgfqpoint{0.563950in}{0.996109in}}%
\pgfpathlineto{\pgfqpoint{0.566084in}{0.998240in}}%
\pgfpathlineto{\pgfqpoint{0.575596in}{1.011851in}}%
\pgfpathlineto{\pgfqpoint{0.579607in}{1.023319in}}%
\pgfpathlineto{\pgfqpoint{0.580282in}{1.025462in}}%
\pgfpathlineto{\pgfqpoint{0.580282in}{1.039073in}}%
\pgfpathlineto{\pgfqpoint{0.579607in}{1.041217in}}%
\pgfpathlineto{\pgfqpoint{0.575596in}{1.052684in}}%
\pgfpathlineto{\pgfqpoint{0.566084in}{1.066295in}}%
\pgfpathlineto{\pgfqpoint{0.563950in}{1.068426in}}%
\pgfpathlineto{\pgfqpoint{0.548987in}{1.079907in}}%
\pgfpathlineto{\pgfqpoint{0.548294in}{1.080341in}}%
\pgfpathlineto{\pgfqpoint{0.532637in}{1.086643in}}%
\pgfpathlineto{\pgfqpoint{0.516981in}{1.089001in}}%
\pgfpathlineto{\pgfqpoint{0.501324in}{1.087429in}}%
\pgfpathlineto{\pgfqpoint{0.485668in}{1.081918in}}%
\pgfpathlineto{\pgfqpoint{0.482227in}{1.079907in}}%
\pgfpathlineto{\pgfqpoint{0.470011in}{1.071189in}}%
\pgfpathlineto{\pgfqpoint{0.464902in}{1.066295in}}%
\pgfpathlineto{\pgfqpoint{0.455089in}{1.052684in}}%
\pgfpathlineto{\pgfqpoint{0.454354in}{1.050673in}}%
\pgfpathlineto{\pgfqpoint{0.450623in}{1.039073in}}%
\pgfpathlineto{\pgfqpoint{0.450623in}{1.025462in}}%
\pgfpathlineto{\pgfqpoint{0.454354in}{1.013862in}}%
\pgfpathlineto{\pgfqpoint{0.455089in}{1.011851in}}%
\pgfpathlineto{\pgfqpoint{0.464902in}{0.998240in}}%
\pgfpathlineto{\pgfqpoint{0.470011in}{0.993346in}}%
\pgfpathlineto{\pgfqpoint{0.482227in}{0.984629in}}%
\pgfpathlineto{\pgfqpoint{0.485668in}{0.982618in}}%
\pgfpathclose%
\pgfpathmoveto{\pgfqpoint{0.491472in}{1.011851in}}%
\pgfpathlineto{\pgfqpoint{0.485668in}{1.018886in}}%
\pgfpathlineto{\pgfqpoint{0.482346in}{1.025462in}}%
\pgfpathlineto{\pgfqpoint{0.482346in}{1.039073in}}%
\pgfpathlineto{\pgfqpoint{0.485668in}{1.045649in}}%
\pgfpathlineto{\pgfqpoint{0.491472in}{1.052684in}}%
\pgfpathlineto{\pgfqpoint{0.501324in}{1.058815in}}%
\pgfpathlineto{\pgfqpoint{0.516981in}{1.061620in}}%
\pgfpathlineto{\pgfqpoint{0.532637in}{1.057411in}}%
\pgfpathlineto{\pgfqpoint{0.539329in}{1.052684in}}%
\pgfpathlineto{\pgfqpoint{0.548294in}{1.040308in}}%
\pgfpathlineto{\pgfqpoint{0.548875in}{1.039073in}}%
\pgfpathlineto{\pgfqpoint{0.548875in}{1.025462in}}%
\pgfpathlineto{\pgfqpoint{0.548294in}{1.024228in}}%
\pgfpathlineto{\pgfqpoint{0.539329in}{1.011851in}}%
\pgfpathlineto{\pgfqpoint{0.532637in}{1.007124in}}%
\pgfpathlineto{\pgfqpoint{0.516981in}{1.002915in}}%
\pgfpathlineto{\pgfqpoint{0.501324in}{1.005720in}}%
\pgfpathlineto{\pgfqpoint{0.491472in}{1.011851in}}%
\pgfpathclose%
\pgfpathmoveto{\pgfqpoint{0.798799in}{0.981199in}}%
\pgfpathlineto{\pgfqpoint{0.814455in}{0.976477in}}%
\pgfpathlineto{\pgfqpoint{0.830112in}{0.975691in}}%
\pgfpathlineto{\pgfqpoint{0.845769in}{0.978837in}}%
\pgfpathlineto{\pgfqpoint{0.858630in}{0.984629in}}%
\pgfpathlineto{\pgfqpoint{0.861425in}{0.986151in}}%
\pgfpathlineto{\pgfqpoint{0.876139in}{0.998240in}}%
\pgfpathlineto{\pgfqpoint{0.877082in}{0.999355in}}%
\pgfpathlineto{\pgfqpoint{0.885563in}{1.011851in}}%
\pgfpathlineto{\pgfqpoint{0.890187in}{1.025462in}}%
\pgfpathlineto{\pgfqpoint{0.890187in}{1.039073in}}%
\pgfpathlineto{\pgfqpoint{0.885563in}{1.052684in}}%
\pgfpathlineto{\pgfqpoint{0.877082in}{1.065181in}}%
\pgfpathlineto{\pgfqpoint{0.876139in}{1.066295in}}%
\pgfpathlineto{\pgfqpoint{0.861425in}{1.078385in}}%
\pgfpathlineto{\pgfqpoint{0.858630in}{1.079907in}}%
\pgfpathlineto{\pgfqpoint{0.845769in}{1.085699in}}%
\pgfpathlineto{\pgfqpoint{0.830112in}{1.088844in}}%
\pgfpathlineto{\pgfqpoint{0.814455in}{1.088058in}}%
\pgfpathlineto{\pgfqpoint{0.798799in}{1.083336in}}%
\pgfpathlineto{\pgfqpoint{0.792460in}{1.079907in}}%
\pgfpathlineto{\pgfqpoint{0.783142in}{1.073771in}}%
\pgfpathlineto{\pgfqpoint{0.774971in}{1.066295in}}%
\pgfpathlineto{\pgfqpoint{0.767486in}{1.056379in}}%
\pgfpathlineto{\pgfqpoint{0.765129in}{1.052684in}}%
\pgfpathlineto{\pgfqpoint{0.760682in}{1.039073in}}%
\pgfpathlineto{\pgfqpoint{0.760682in}{1.025462in}}%
\pgfpathlineto{\pgfqpoint{0.765129in}{1.011851in}}%
\pgfpathlineto{\pgfqpoint{0.767486in}{1.008156in}}%
\pgfpathlineto{\pgfqpoint{0.774971in}{0.998240in}}%
\pgfpathlineto{\pgfqpoint{0.783142in}{0.990764in}}%
\pgfpathlineto{\pgfqpoint{0.792460in}{0.984629in}}%
\pgfpathlineto{\pgfqpoint{0.798799in}{0.981199in}}%
\pgfpathclose%
\pgfpathmoveto{\pgfqpoint{0.800939in}{1.011851in}}%
\pgfpathlineto{\pgfqpoint{0.798799in}{1.014080in}}%
\pgfpathlineto{\pgfqpoint{0.792588in}{1.025462in}}%
\pgfpathlineto{\pgfqpoint{0.792588in}{1.039073in}}%
\pgfpathlineto{\pgfqpoint{0.798799in}{1.050455in}}%
\pgfpathlineto{\pgfqpoint{0.800939in}{1.052684in}}%
\pgfpathlineto{\pgfqpoint{0.814455in}{1.059937in}}%
\pgfpathlineto{\pgfqpoint{0.830112in}{1.061340in}}%
\pgfpathlineto{\pgfqpoint{0.845769in}{1.055726in}}%
\pgfpathlineto{\pgfqpoint{0.849625in}{1.052684in}}%
\pgfpathlineto{\pgfqpoint{0.858478in}{1.039073in}}%
\pgfpathlineto{\pgfqpoint{0.858478in}{1.025462in}}%
\pgfpathlineto{\pgfqpoint{0.849625in}{1.011851in}}%
\pgfpathlineto{\pgfqpoint{0.845769in}{1.008809in}}%
\pgfpathlineto{\pgfqpoint{0.830112in}{1.003196in}}%
\pgfpathlineto{\pgfqpoint{0.814455in}{1.004598in}}%
\pgfpathlineto{\pgfqpoint{0.800939in}{1.011851in}}%
\pgfpathclose%
\pgfpathmoveto{\pgfqpoint{1.111930in}{0.979939in}}%
\pgfpathlineto{\pgfqpoint{1.127587in}{0.976005in}}%
\pgfpathlineto{\pgfqpoint{1.143243in}{0.976005in}}%
\pgfpathlineto{\pgfqpoint{1.158900in}{0.979939in}}%
\pgfpathlineto{\pgfqpoint{1.168349in}{0.984629in}}%
\pgfpathlineto{\pgfqpoint{1.174556in}{0.988365in}}%
\pgfpathlineto{\pgfqpoint{1.185915in}{0.998240in}}%
\pgfpathlineto{\pgfqpoint{1.190213in}{1.003636in}}%
\pgfpathlineto{\pgfqpoint{1.195608in}{1.011851in}}%
\pgfpathlineto{\pgfqpoint{1.200132in}{1.025462in}}%
\pgfpathlineto{\pgfqpoint{1.200132in}{1.039073in}}%
\pgfpathlineto{\pgfqpoint{1.195608in}{1.052684in}}%
\pgfpathlineto{\pgfqpoint{1.190213in}{1.060899in}}%
\pgfpathlineto{\pgfqpoint{1.185915in}{1.066295in}}%
\pgfpathlineto{\pgfqpoint{1.174556in}{1.076170in}}%
\pgfpathlineto{\pgfqpoint{1.168349in}{1.079907in}}%
\pgfpathlineto{\pgfqpoint{1.158900in}{1.084596in}}%
\pgfpathlineto{\pgfqpoint{1.143243in}{1.088530in}}%
\pgfpathlineto{\pgfqpoint{1.127587in}{1.088530in}}%
\pgfpathlineto{\pgfqpoint{1.111930in}{1.084596in}}%
\pgfpathlineto{\pgfqpoint{1.102481in}{1.079907in}}%
\pgfpathlineto{\pgfqpoint{1.096274in}{1.076170in}}%
\pgfpathlineto{\pgfqpoint{1.084915in}{1.066295in}}%
\pgfpathlineto{\pgfqpoint{1.080617in}{1.060899in}}%
\pgfpathlineto{\pgfqpoint{1.075222in}{1.052684in}}%
\pgfpathlineto{\pgfqpoint{1.070698in}{1.039073in}}%
\pgfpathlineto{\pgfqpoint{1.070698in}{1.025462in}}%
\pgfpathlineto{\pgfqpoint{1.075222in}{1.011851in}}%
\pgfpathlineto{\pgfqpoint{1.080617in}{1.003636in}}%
\pgfpathlineto{\pgfqpoint{1.084915in}{0.998240in}}%
\pgfpathlineto{\pgfqpoint{1.096274in}{0.988365in}}%
\pgfpathlineto{\pgfqpoint{1.102481in}{0.984629in}}%
\pgfpathlineto{\pgfqpoint{1.111930in}{0.979939in}}%
\pgfpathclose%
\pgfpathmoveto{\pgfqpoint{1.110693in}{1.011851in}}%
\pgfpathlineto{\pgfqpoint{1.102619in}{1.025462in}}%
\pgfpathlineto{\pgfqpoint{1.102619in}{1.039073in}}%
\pgfpathlineto{\pgfqpoint{1.110693in}{1.052684in}}%
\pgfpathlineto{\pgfqpoint{1.111930in}{1.053759in}}%
\pgfpathlineto{\pgfqpoint{1.127587in}{1.060779in}}%
\pgfpathlineto{\pgfqpoint{1.143243in}{1.060779in}}%
\pgfpathlineto{\pgfqpoint{1.158900in}{1.053759in}}%
\pgfpathlineto{\pgfqpoint{1.160137in}{1.052684in}}%
\pgfpathlineto{\pgfqpoint{1.168211in}{1.039073in}}%
\pgfpathlineto{\pgfqpoint{1.168211in}{1.025462in}}%
\pgfpathlineto{\pgfqpoint{1.160137in}{1.011851in}}%
\pgfpathlineto{\pgfqpoint{1.158900in}{1.010776in}}%
\pgfpathlineto{\pgfqpoint{1.143243in}{1.003756in}}%
\pgfpathlineto{\pgfqpoint{1.127587in}{1.003756in}}%
\pgfpathlineto{\pgfqpoint{1.111930in}{1.010776in}}%
\pgfpathlineto{\pgfqpoint{1.110693in}{1.011851in}}%
\pgfpathclose%
\pgfpathmoveto{\pgfqpoint{1.425061in}{0.978837in}}%
\pgfpathlineto{\pgfqpoint{1.440718in}{0.975691in}}%
\pgfpathlineto{\pgfqpoint{1.456375in}{0.976477in}}%
\pgfpathlineto{\pgfqpoint{1.472031in}{0.981199in}}%
\pgfpathlineto{\pgfqpoint{1.478370in}{0.984629in}}%
\pgfpathlineto{\pgfqpoint{1.487688in}{0.990764in}}%
\pgfpathlineto{\pgfqpoint{1.495859in}{0.998240in}}%
\pgfpathlineto{\pgfqpoint{1.503344in}{1.008156in}}%
\pgfpathlineto{\pgfqpoint{1.505701in}{1.011851in}}%
\pgfpathlineto{\pgfqpoint{1.510148in}{1.025462in}}%
\pgfpathlineto{\pgfqpoint{1.510148in}{1.039073in}}%
\pgfpathlineto{\pgfqpoint{1.505701in}{1.052684in}}%
\pgfpathlineto{\pgfqpoint{1.503344in}{1.056379in}}%
\pgfpathlineto{\pgfqpoint{1.495859in}{1.066295in}}%
\pgfpathlineto{\pgfqpoint{1.487688in}{1.073771in}}%
\pgfpathlineto{\pgfqpoint{1.478370in}{1.079907in}}%
\pgfpathlineto{\pgfqpoint{1.472031in}{1.083336in}}%
\pgfpathlineto{\pgfqpoint{1.456375in}{1.088058in}}%
\pgfpathlineto{\pgfqpoint{1.440718in}{1.088844in}}%
\pgfpathlineto{\pgfqpoint{1.425061in}{1.085699in}}%
\pgfpathlineto{\pgfqpoint{1.412200in}{1.079907in}}%
\pgfpathlineto{\pgfqpoint{1.409405in}{1.078385in}}%
\pgfpathlineto{\pgfqpoint{1.394691in}{1.066295in}}%
\pgfpathlineto{\pgfqpoint{1.393748in}{1.065181in}}%
\pgfpathlineto{\pgfqpoint{1.385267in}{1.052684in}}%
\pgfpathlineto{\pgfqpoint{1.380643in}{1.039073in}}%
\pgfpathlineto{\pgfqpoint{1.380643in}{1.025462in}}%
\pgfpathlineto{\pgfqpoint{1.385267in}{1.011851in}}%
\pgfpathlineto{\pgfqpoint{1.393748in}{0.999355in}}%
\pgfpathlineto{\pgfqpoint{1.394691in}{0.998240in}}%
\pgfpathlineto{\pgfqpoint{1.409405in}{0.986151in}}%
\pgfpathlineto{\pgfqpoint{1.412200in}{0.984629in}}%
\pgfpathlineto{\pgfqpoint{1.425061in}{0.978837in}}%
\pgfpathclose%
\pgfpathmoveto{\pgfqpoint{1.421205in}{1.011851in}}%
\pgfpathlineto{\pgfqpoint{1.412352in}{1.025462in}}%
\pgfpathlineto{\pgfqpoint{1.412352in}{1.039073in}}%
\pgfpathlineto{\pgfqpoint{1.421205in}{1.052684in}}%
\pgfpathlineto{\pgfqpoint{1.425061in}{1.055726in}}%
\pgfpathlineto{\pgfqpoint{1.440718in}{1.061340in}}%
\pgfpathlineto{\pgfqpoint{1.456375in}{1.059937in}}%
\pgfpathlineto{\pgfqpoint{1.469891in}{1.052684in}}%
\pgfpathlineto{\pgfqpoint{1.472031in}{1.050455in}}%
\pgfpathlineto{\pgfqpoint{1.478242in}{1.039073in}}%
\pgfpathlineto{\pgfqpoint{1.478242in}{1.025462in}}%
\pgfpathlineto{\pgfqpoint{1.472031in}{1.014080in}}%
\pgfpathlineto{\pgfqpoint{1.469891in}{1.011851in}}%
\pgfpathlineto{\pgfqpoint{1.456375in}{1.004598in}}%
\pgfpathlineto{\pgfqpoint{1.440718in}{1.003196in}}%
\pgfpathlineto{\pgfqpoint{1.425061in}{1.008809in}}%
\pgfpathlineto{\pgfqpoint{1.421205in}{1.011851in}}%
\pgfpathclose%
\pgfpathmoveto{\pgfqpoint{1.722536in}{0.984194in}}%
\pgfpathlineto{\pgfqpoint{1.738193in}{0.977892in}}%
\pgfpathlineto{\pgfqpoint{1.753849in}{0.975534in}}%
\pgfpathlineto{\pgfqpoint{1.769506in}{0.977106in}}%
\pgfpathlineto{\pgfqpoint{1.785162in}{0.982618in}}%
\pgfpathlineto{\pgfqpoint{1.788603in}{0.984629in}}%
\pgfpathlineto{\pgfqpoint{1.800819in}{0.993346in}}%
\pgfpathlineto{\pgfqpoint{1.805928in}{0.998240in}}%
\pgfpathlineto{\pgfqpoint{1.815741in}{1.011851in}}%
\pgfpathlineto{\pgfqpoint{1.816476in}{1.013862in}}%
\pgfpathlineto{\pgfqpoint{1.820207in}{1.025462in}}%
\pgfpathlineto{\pgfqpoint{1.820207in}{1.039073in}}%
\pgfpathlineto{\pgfqpoint{1.816476in}{1.050673in}}%
\pgfpathlineto{\pgfqpoint{1.815741in}{1.052684in}}%
\pgfpathlineto{\pgfqpoint{1.805928in}{1.066295in}}%
\pgfpathlineto{\pgfqpoint{1.800819in}{1.071189in}}%
\pgfpathlineto{\pgfqpoint{1.788603in}{1.079907in}}%
\pgfpathlineto{\pgfqpoint{1.785162in}{1.081918in}}%
\pgfpathlineto{\pgfqpoint{1.769506in}{1.087429in}}%
\pgfpathlineto{\pgfqpoint{1.753849in}{1.089001in}}%
\pgfpathlineto{\pgfqpoint{1.738193in}{1.086643in}}%
\pgfpathlineto{\pgfqpoint{1.722536in}{1.080341in}}%
\pgfpathlineto{\pgfqpoint{1.721843in}{1.079907in}}%
\pgfpathlineto{\pgfqpoint{1.706880in}{1.068426in}}%
\pgfpathlineto{\pgfqpoint{1.704746in}{1.066295in}}%
\pgfpathlineto{\pgfqpoint{1.695234in}{1.052684in}}%
\pgfpathlineto{\pgfqpoint{1.691223in}{1.041217in}}%
\pgfpathlineto{\pgfqpoint{1.690548in}{1.039073in}}%
\pgfpathlineto{\pgfqpoint{1.690548in}{1.025462in}}%
\pgfpathlineto{\pgfqpoint{1.691223in}{1.023319in}}%
\pgfpathlineto{\pgfqpoint{1.695234in}{1.011851in}}%
\pgfpathlineto{\pgfqpoint{1.704746in}{0.998240in}}%
\pgfpathlineto{\pgfqpoint{1.706880in}{0.996109in}}%
\pgfpathlineto{\pgfqpoint{1.721843in}{0.984629in}}%
\pgfpathlineto{\pgfqpoint{1.722536in}{0.984194in}}%
\pgfpathclose%
\pgfpathmoveto{\pgfqpoint{1.731501in}{1.011851in}}%
\pgfpathlineto{\pgfqpoint{1.722536in}{1.024228in}}%
\pgfpathlineto{\pgfqpoint{1.721955in}{1.025462in}}%
\pgfpathlineto{\pgfqpoint{1.721955in}{1.039073in}}%
\pgfpathlineto{\pgfqpoint{1.722536in}{1.040308in}}%
\pgfpathlineto{\pgfqpoint{1.731501in}{1.052684in}}%
\pgfpathlineto{\pgfqpoint{1.738193in}{1.057411in}}%
\pgfpathlineto{\pgfqpoint{1.753849in}{1.061620in}}%
\pgfpathlineto{\pgfqpoint{1.769506in}{1.058815in}}%
\pgfpathlineto{\pgfqpoint{1.779358in}{1.052684in}}%
\pgfpathlineto{\pgfqpoint{1.785162in}{1.045649in}}%
\pgfpathlineto{\pgfqpoint{1.788484in}{1.039073in}}%
\pgfpathlineto{\pgfqpoint{1.788484in}{1.025462in}}%
\pgfpathlineto{\pgfqpoint{1.785162in}{1.018886in}}%
\pgfpathlineto{\pgfqpoint{1.779358in}{1.011851in}}%
\pgfpathlineto{\pgfqpoint{1.769506in}{1.005720in}}%
\pgfpathlineto{\pgfqpoint{1.753849in}{1.002915in}}%
\pgfpathlineto{\pgfqpoint{1.738193in}{1.007124in}}%
\pgfpathlineto{\pgfqpoint{1.731501in}{1.011851in}}%
\pgfpathclose%
\pgfpathmoveto{\pgfqpoint{0.485668in}{1.252215in}}%
\pgfpathlineto{\pgfqpoint{0.501324in}{1.246583in}}%
\pgfpathlineto{\pgfqpoint{0.516981in}{1.244976in}}%
\pgfpathlineto{\pgfqpoint{0.532637in}{1.247386in}}%
\pgfpathlineto{\pgfqpoint{0.548294in}{1.253826in}}%
\pgfpathlineto{\pgfqpoint{0.552965in}{1.256851in}}%
\pgfpathlineto{\pgfqpoint{0.563950in}{1.265795in}}%
\pgfpathlineto{\pgfqpoint{0.568368in}{1.270462in}}%
\pgfpathlineto{\pgfqpoint{0.576926in}{1.284073in}}%
\pgfpathlineto{\pgfqpoint{0.579607in}{1.293644in}}%
\pgfpathlineto{\pgfqpoint{0.580629in}{1.297684in}}%
\pgfpathlineto{\pgfqpoint{0.579760in}{1.311295in}}%
\pgfpathlineto{\pgfqpoint{0.579607in}{1.311703in}}%
\pgfpathlineto{\pgfqpoint{0.574075in}{1.324907in}}%
\pgfpathlineto{\pgfqpoint{0.563950in}{1.338093in}}%
\pgfpathlineto{\pgfqpoint{0.563531in}{1.338518in}}%
\pgfpathlineto{\pgfqpoint{0.548294in}{1.349692in}}%
\pgfpathlineto{\pgfqpoint{0.543002in}{1.352129in}}%
\pgfpathlineto{\pgfqpoint{0.532637in}{1.356189in}}%
\pgfpathlineto{\pgfqpoint{0.516981in}{1.358507in}}%
\pgfpathlineto{\pgfqpoint{0.501324in}{1.356962in}}%
\pgfpathlineto{\pgfqpoint{0.487318in}{1.352129in}}%
\pgfpathlineto{\pgfqpoint{0.485668in}{1.351462in}}%
\pgfpathlineto{\pgfqpoint{0.470011in}{1.340845in}}%
\pgfpathlineto{\pgfqpoint{0.467454in}{1.338518in}}%
\pgfpathlineto{\pgfqpoint{0.456658in}{1.324907in}}%
\pgfpathlineto{\pgfqpoint{0.454354in}{1.319643in}}%
\pgfpathlineto{\pgfqpoint{0.451149in}{1.311295in}}%
\pgfpathlineto{\pgfqpoint{0.450272in}{1.297684in}}%
\pgfpathlineto{\pgfqpoint{0.453783in}{1.284073in}}%
\pgfpathlineto{\pgfqpoint{0.454354in}{1.283061in}}%
\pgfpathlineto{\pgfqpoint{0.462547in}{1.270462in}}%
\pgfpathlineto{\pgfqpoint{0.470011in}{1.262896in}}%
\pgfpathlineto{\pgfqpoint{0.477994in}{1.256851in}}%
\pgfpathlineto{\pgfqpoint{0.485668in}{1.252215in}}%
\pgfpathclose%
\pgfpathmoveto{\pgfqpoint{0.488353in}{1.284073in}}%
\pgfpathlineto{\pgfqpoint{0.485668in}{1.288137in}}%
\pgfpathlineto{\pgfqpoint{0.481791in}{1.297684in}}%
\pgfpathlineto{\pgfqpoint{0.483178in}{1.311295in}}%
\pgfpathlineto{\pgfqpoint{0.485668in}{1.315425in}}%
\pgfpathlineto{\pgfqpoint{0.495039in}{1.324907in}}%
\pgfpathlineto{\pgfqpoint{0.501324in}{1.328494in}}%
\pgfpathlineto{\pgfqpoint{0.516981in}{1.331082in}}%
\pgfpathlineto{\pgfqpoint{0.532637in}{1.327199in}}%
\pgfpathlineto{\pgfqpoint{0.536175in}{1.324907in}}%
\pgfpathlineto{\pgfqpoint{0.547992in}{1.311295in}}%
\pgfpathlineto{\pgfqpoint{0.548294in}{1.309211in}}%
\pgfpathlineto{\pgfqpoint{0.549397in}{1.297684in}}%
\pgfpathlineto{\pgfqpoint{0.548294in}{1.294773in}}%
\pgfpathlineto{\pgfqpoint{0.542087in}{1.284073in}}%
\pgfpathlineto{\pgfqpoint{0.532637in}{1.276716in}}%
\pgfpathlineto{\pgfqpoint{0.516981in}{1.272102in}}%
\pgfpathlineto{\pgfqpoint{0.501324in}{1.275177in}}%
\pgfpathlineto{\pgfqpoint{0.488353in}{1.284073in}}%
\pgfpathclose%
\pgfpathmoveto{\pgfqpoint{0.798799in}{1.250766in}}%
\pgfpathlineto{\pgfqpoint{0.814455in}{1.245940in}}%
\pgfpathlineto{\pgfqpoint{0.830112in}{1.245137in}}%
\pgfpathlineto{\pgfqpoint{0.845769in}{1.248351in}}%
\pgfpathlineto{\pgfqpoint{0.861425in}{1.255599in}}%
\pgfpathlineto{\pgfqpoint{0.863241in}{1.256851in}}%
\pgfpathlineto{\pgfqpoint{0.877082in}{1.268883in}}%
\pgfpathlineto{\pgfqpoint{0.878522in}{1.270462in}}%
\pgfpathlineto{\pgfqpoint{0.886858in}{1.284073in}}%
\pgfpathlineto{\pgfqpoint{0.890556in}{1.297684in}}%
\pgfpathlineto{\pgfqpoint{0.889632in}{1.311295in}}%
\pgfpathlineto{\pgfqpoint{0.884081in}{1.324907in}}%
\pgfpathlineto{\pgfqpoint{0.877082in}{1.334367in}}%
\pgfpathlineto{\pgfqpoint{0.873244in}{1.338518in}}%
\pgfpathlineto{\pgfqpoint{0.861425in}{1.347745in}}%
\pgfpathlineto{\pgfqpoint{0.852929in}{1.352129in}}%
\pgfpathlineto{\pgfqpoint{0.845769in}{1.355261in}}%
\pgfpathlineto{\pgfqpoint{0.830112in}{1.358353in}}%
\pgfpathlineto{\pgfqpoint{0.814455in}{1.357580in}}%
\pgfpathlineto{\pgfqpoint{0.798799in}{1.352939in}}%
\pgfpathlineto{\pgfqpoint{0.797258in}{1.352129in}}%
\pgfpathlineto{\pgfqpoint{0.783142in}{1.343321in}}%
\pgfpathlineto{\pgfqpoint{0.777617in}{1.338518in}}%
\pgfpathlineto{\pgfqpoint{0.767486in}{1.326246in}}%
\pgfpathlineto{\pgfqpoint{0.766554in}{1.324907in}}%
\pgfpathlineto{\pgfqpoint{0.761215in}{1.311295in}}%
\pgfpathlineto{\pgfqpoint{0.760326in}{1.297684in}}%
\pgfpathlineto{\pgfqpoint{0.763883in}{1.284073in}}%
\pgfpathlineto{\pgfqpoint{0.767486in}{1.277849in}}%
\pgfpathlineto{\pgfqpoint{0.772528in}{1.270462in}}%
\pgfpathlineto{\pgfqpoint{0.783142in}{1.260187in}}%
\pgfpathlineto{\pgfqpoint{0.787917in}{1.256851in}}%
\pgfpathlineto{\pgfqpoint{0.798799in}{1.250766in}}%
\pgfpathclose%
\pgfpathmoveto{\pgfqpoint{0.797950in}{1.284073in}}%
\pgfpathlineto{\pgfqpoint{0.791993in}{1.297684in}}%
\pgfpathlineto{\pgfqpoint{0.793481in}{1.311295in}}%
\pgfpathlineto{\pgfqpoint{0.798799in}{1.319460in}}%
\pgfpathlineto{\pgfqpoint{0.805063in}{1.324907in}}%
\pgfpathlineto{\pgfqpoint{0.814455in}{1.329530in}}%
\pgfpathlineto{\pgfqpoint{0.830112in}{1.330823in}}%
\pgfpathlineto{\pgfqpoint{0.845769in}{1.325644in}}%
\pgfpathlineto{\pgfqpoint{0.846788in}{1.324907in}}%
\pgfpathlineto{\pgfqpoint{0.857417in}{1.311295in}}%
\pgfpathlineto{\pgfqpoint{0.859185in}{1.297684in}}%
\pgfpathlineto{\pgfqpoint{0.852106in}{1.284073in}}%
\pgfpathlineto{\pgfqpoint{0.845769in}{1.278564in}}%
\pgfpathlineto{\pgfqpoint{0.830112in}{1.272409in}}%
\pgfpathlineto{\pgfqpoint{0.814455in}{1.273947in}}%
\pgfpathlineto{\pgfqpoint{0.798799in}{1.283187in}}%
\pgfpathlineto{\pgfqpoint{0.797950in}{1.284073in}}%
\pgfpathclose%
\pgfpathmoveto{\pgfqpoint{1.111930in}{1.249478in}}%
\pgfpathlineto{\pgfqpoint{1.127587in}{1.245458in}}%
\pgfpathlineto{\pgfqpoint{1.143243in}{1.245458in}}%
\pgfpathlineto{\pgfqpoint{1.158900in}{1.249478in}}%
\pgfpathlineto{\pgfqpoint{1.173274in}{1.256851in}}%
\pgfpathlineto{\pgfqpoint{1.174556in}{1.257670in}}%
\pgfpathlineto{\pgfqpoint{1.188462in}{1.270462in}}%
\pgfpathlineto{\pgfqpoint{1.190213in}{1.272892in}}%
\pgfpathlineto{\pgfqpoint{1.196875in}{1.284073in}}%
\pgfpathlineto{\pgfqpoint{1.200494in}{1.297684in}}%
\pgfpathlineto{\pgfqpoint{1.199590in}{1.311295in}}%
\pgfpathlineto{\pgfqpoint{1.194158in}{1.324907in}}%
\pgfpathlineto{\pgfqpoint{1.190213in}{1.330417in}}%
\pgfpathlineto{\pgfqpoint{1.183155in}{1.338518in}}%
\pgfpathlineto{\pgfqpoint{1.174556in}{1.345621in}}%
\pgfpathlineto{\pgfqpoint{1.163150in}{1.352129in}}%
\pgfpathlineto{\pgfqpoint{1.158900in}{1.354178in}}%
\pgfpathlineto{\pgfqpoint{1.143243in}{1.358044in}}%
\pgfpathlineto{\pgfqpoint{1.127587in}{1.358044in}}%
\pgfpathlineto{\pgfqpoint{1.111930in}{1.354178in}}%
\pgfpathlineto{\pgfqpoint{1.107680in}{1.352129in}}%
\pgfpathlineto{\pgfqpoint{1.096274in}{1.345621in}}%
\pgfpathlineto{\pgfqpoint{1.087675in}{1.338518in}}%
\pgfpathlineto{\pgfqpoint{1.080617in}{1.330417in}}%
\pgfpathlineto{\pgfqpoint{1.076672in}{1.324907in}}%
\pgfpathlineto{\pgfqpoint{1.071240in}{1.311295in}}%
\pgfpathlineto{\pgfqpoint{1.070336in}{1.297684in}}%
\pgfpathlineto{\pgfqpoint{1.073955in}{1.284073in}}%
\pgfpathlineto{\pgfqpoint{1.080617in}{1.272892in}}%
\pgfpathlineto{\pgfqpoint{1.082368in}{1.270462in}}%
\pgfpathlineto{\pgfqpoint{1.096274in}{1.257670in}}%
\pgfpathlineto{\pgfqpoint{1.097556in}{1.256851in}}%
\pgfpathlineto{\pgfqpoint{1.111930in}{1.249478in}}%
\pgfpathclose%
\pgfpathmoveto{\pgfqpoint{1.108431in}{1.284073in}}%
\pgfpathlineto{\pgfqpoint{1.101974in}{1.297684in}}%
\pgfpathlineto{\pgfqpoint{1.103587in}{1.311295in}}%
\pgfpathlineto{\pgfqpoint{1.111930in}{1.323046in}}%
\pgfpathlineto{\pgfqpoint{1.114494in}{1.324907in}}%
\pgfpathlineto{\pgfqpoint{1.127587in}{1.330306in}}%
\pgfpathlineto{\pgfqpoint{1.143243in}{1.330306in}}%
\pgfpathlineto{\pgfqpoint{1.156336in}{1.324907in}}%
\pgfpathlineto{\pgfqpoint{1.158900in}{1.323046in}}%
\pgfpathlineto{\pgfqpoint{1.167243in}{1.311295in}}%
\pgfpathlineto{\pgfqpoint{1.168856in}{1.297684in}}%
\pgfpathlineto{\pgfqpoint{1.162399in}{1.284073in}}%
\pgfpathlineto{\pgfqpoint{1.158900in}{1.280721in}}%
\pgfpathlineto{\pgfqpoint{1.143243in}{1.273024in}}%
\pgfpathlineto{\pgfqpoint{1.127587in}{1.273024in}}%
\pgfpathlineto{\pgfqpoint{1.111930in}{1.280721in}}%
\pgfpathlineto{\pgfqpoint{1.108431in}{1.284073in}}%
\pgfpathclose%
\pgfpathmoveto{\pgfqpoint{1.409405in}{1.255599in}}%
\pgfpathlineto{\pgfqpoint{1.425061in}{1.248351in}}%
\pgfpathlineto{\pgfqpoint{1.440718in}{1.245137in}}%
\pgfpathlineto{\pgfqpoint{1.456375in}{1.245940in}}%
\pgfpathlineto{\pgfqpoint{1.472031in}{1.250766in}}%
\pgfpathlineto{\pgfqpoint{1.482913in}{1.256851in}}%
\pgfpathlineto{\pgfqpoint{1.487688in}{1.260187in}}%
\pgfpathlineto{\pgfqpoint{1.498302in}{1.270462in}}%
\pgfpathlineto{\pgfqpoint{1.503344in}{1.277849in}}%
\pgfpathlineto{\pgfqpoint{1.506947in}{1.284073in}}%
\pgfpathlineto{\pgfqpoint{1.510504in}{1.297684in}}%
\pgfpathlineto{\pgfqpoint{1.509615in}{1.311295in}}%
\pgfpathlineto{\pgfqpoint{1.504276in}{1.324907in}}%
\pgfpathlineto{\pgfqpoint{1.503344in}{1.326246in}}%
\pgfpathlineto{\pgfqpoint{1.493213in}{1.338518in}}%
\pgfpathlineto{\pgfqpoint{1.487688in}{1.343321in}}%
\pgfpathlineto{\pgfqpoint{1.473572in}{1.352129in}}%
\pgfpathlineto{\pgfqpoint{1.472031in}{1.352939in}}%
\pgfpathlineto{\pgfqpoint{1.456375in}{1.357580in}}%
\pgfpathlineto{\pgfqpoint{1.440718in}{1.358353in}}%
\pgfpathlineto{\pgfqpoint{1.425061in}{1.355261in}}%
\pgfpathlineto{\pgfqpoint{1.417901in}{1.352129in}}%
\pgfpathlineto{\pgfqpoint{1.409405in}{1.347745in}}%
\pgfpathlineto{\pgfqpoint{1.397586in}{1.338518in}}%
\pgfpathlineto{\pgfqpoint{1.393748in}{1.334367in}}%
\pgfpathlineto{\pgfqpoint{1.386749in}{1.324907in}}%
\pgfpathlineto{\pgfqpoint{1.381198in}{1.311295in}}%
\pgfpathlineto{\pgfqpoint{1.380274in}{1.297684in}}%
\pgfpathlineto{\pgfqpoint{1.383972in}{1.284073in}}%
\pgfpathlineto{\pgfqpoint{1.392308in}{1.270462in}}%
\pgfpathlineto{\pgfqpoint{1.393748in}{1.268883in}}%
\pgfpathlineto{\pgfqpoint{1.407589in}{1.256851in}}%
\pgfpathlineto{\pgfqpoint{1.409405in}{1.255599in}}%
\pgfpathclose%
\pgfpathmoveto{\pgfqpoint{1.418724in}{1.284073in}}%
\pgfpathlineto{\pgfqpoint{1.411645in}{1.297684in}}%
\pgfpathlineto{\pgfqpoint{1.413413in}{1.311295in}}%
\pgfpathlineto{\pgfqpoint{1.424042in}{1.324907in}}%
\pgfpathlineto{\pgfqpoint{1.425061in}{1.325644in}}%
\pgfpathlineto{\pgfqpoint{1.440718in}{1.330823in}}%
\pgfpathlineto{\pgfqpoint{1.456375in}{1.329530in}}%
\pgfpathlineto{\pgfqpoint{1.465767in}{1.324907in}}%
\pgfpathlineto{\pgfqpoint{1.472031in}{1.319460in}}%
\pgfpathlineto{\pgfqpoint{1.477349in}{1.311295in}}%
\pgfpathlineto{\pgfqpoint{1.478837in}{1.297684in}}%
\pgfpathlineto{\pgfqpoint{1.472880in}{1.284073in}}%
\pgfpathlineto{\pgfqpoint{1.472031in}{1.283187in}}%
\pgfpathlineto{\pgfqpoint{1.456375in}{1.273947in}}%
\pgfpathlineto{\pgfqpoint{1.440718in}{1.272409in}}%
\pgfpathlineto{\pgfqpoint{1.425061in}{1.278564in}}%
\pgfpathlineto{\pgfqpoint{1.418724in}{1.284073in}}%
\pgfpathclose%
\pgfpathmoveto{\pgfqpoint{1.722536in}{1.253826in}}%
\pgfpathlineto{\pgfqpoint{1.738193in}{1.247386in}}%
\pgfpathlineto{\pgfqpoint{1.753849in}{1.244976in}}%
\pgfpathlineto{\pgfqpoint{1.769506in}{1.246583in}}%
\pgfpathlineto{\pgfqpoint{1.785162in}{1.252215in}}%
\pgfpathlineto{\pgfqpoint{1.792836in}{1.256851in}}%
\pgfpathlineto{\pgfqpoint{1.800819in}{1.262896in}}%
\pgfpathlineto{\pgfqpoint{1.808283in}{1.270462in}}%
\pgfpathlineto{\pgfqpoint{1.816476in}{1.283061in}}%
\pgfpathlineto{\pgfqpoint{1.817047in}{1.284073in}}%
\pgfpathlineto{\pgfqpoint{1.820558in}{1.297684in}}%
\pgfpathlineto{\pgfqpoint{1.819681in}{1.311295in}}%
\pgfpathlineto{\pgfqpoint{1.816476in}{1.319643in}}%
\pgfpathlineto{\pgfqpoint{1.814172in}{1.324907in}}%
\pgfpathlineto{\pgfqpoint{1.803376in}{1.338518in}}%
\pgfpathlineto{\pgfqpoint{1.800819in}{1.340845in}}%
\pgfpathlineto{\pgfqpoint{1.785162in}{1.351462in}}%
\pgfpathlineto{\pgfqpoint{1.783512in}{1.352129in}}%
\pgfpathlineto{\pgfqpoint{1.769506in}{1.356962in}}%
\pgfpathlineto{\pgfqpoint{1.753849in}{1.358507in}}%
\pgfpathlineto{\pgfqpoint{1.738193in}{1.356189in}}%
\pgfpathlineto{\pgfqpoint{1.727828in}{1.352129in}}%
\pgfpathlineto{\pgfqpoint{1.722536in}{1.349692in}}%
\pgfpathlineto{\pgfqpoint{1.707299in}{1.338518in}}%
\pgfpathlineto{\pgfqpoint{1.706880in}{1.338093in}}%
\pgfpathlineto{\pgfqpoint{1.696755in}{1.324907in}}%
\pgfpathlineto{\pgfqpoint{1.691223in}{1.311703in}}%
\pgfpathlineto{\pgfqpoint{1.691070in}{1.311295in}}%
\pgfpathlineto{\pgfqpoint{1.690201in}{1.297684in}}%
\pgfpathlineto{\pgfqpoint{1.691223in}{1.293644in}}%
\pgfpathlineto{\pgfqpoint{1.693904in}{1.284073in}}%
\pgfpathlineto{\pgfqpoint{1.702462in}{1.270462in}}%
\pgfpathlineto{\pgfqpoint{1.706880in}{1.265795in}}%
\pgfpathlineto{\pgfqpoint{1.717865in}{1.256851in}}%
\pgfpathlineto{\pgfqpoint{1.722536in}{1.253826in}}%
\pgfpathclose%
\pgfpathmoveto{\pgfqpoint{1.728743in}{1.284073in}}%
\pgfpathlineto{\pgfqpoint{1.722536in}{1.294773in}}%
\pgfpathlineto{\pgfqpoint{1.721433in}{1.297684in}}%
\pgfpathlineto{\pgfqpoint{1.722536in}{1.309211in}}%
\pgfpathlineto{\pgfqpoint{1.722838in}{1.311295in}}%
\pgfpathlineto{\pgfqpoint{1.734655in}{1.324907in}}%
\pgfpathlineto{\pgfqpoint{1.738193in}{1.327199in}}%
\pgfpathlineto{\pgfqpoint{1.753849in}{1.331082in}}%
\pgfpathlineto{\pgfqpoint{1.769506in}{1.328494in}}%
\pgfpathlineto{\pgfqpoint{1.775791in}{1.324907in}}%
\pgfpathlineto{\pgfqpoint{1.785162in}{1.315425in}}%
\pgfpathlineto{\pgfqpoint{1.787652in}{1.311295in}}%
\pgfpathlineto{\pgfqpoint{1.789039in}{1.297684in}}%
\pgfpathlineto{\pgfqpoint{1.785162in}{1.288137in}}%
\pgfpathlineto{\pgfqpoint{1.782477in}{1.284073in}}%
\pgfpathlineto{\pgfqpoint{1.769506in}{1.275177in}}%
\pgfpathlineto{\pgfqpoint{1.753849in}{1.272102in}}%
\pgfpathlineto{\pgfqpoint{1.738193in}{1.276716in}}%
\pgfpathlineto{\pgfqpoint{1.728743in}{1.284073in}}%
\pgfpathclose%
\pgfpathmoveto{\pgfqpoint{0.516981in}{1.514422in}}%
\pgfpathlineto{\pgfqpoint{0.524202in}{1.515462in}}%
\pgfpathlineto{\pgfqpoint{0.532637in}{1.516802in}}%
\pgfpathlineto{\pgfqpoint{0.548294in}{1.523413in}}%
\pgfpathlineto{\pgfqpoint{0.556717in}{1.529073in}}%
\pgfpathlineto{\pgfqpoint{0.563950in}{1.535362in}}%
\pgfpathlineto{\pgfqpoint{0.570461in}{1.542684in}}%
\pgfpathlineto{\pgfqpoint{0.578066in}{1.556295in}}%
\pgfpathlineto{\pgfqpoint{0.579607in}{1.563629in}}%
\pgfpathlineto{\pgfqpoint{0.580803in}{1.569907in}}%
\pgfpathlineto{\pgfqpoint{0.579607in}{1.579291in}}%
\pgfpathlineto{\pgfqpoint{0.579015in}{1.583518in}}%
\pgfpathlineto{\pgfqpoint{0.572364in}{1.597129in}}%
\pgfpathlineto{\pgfqpoint{0.563950in}{1.607271in}}%
\pgfpathlineto{\pgfqpoint{0.560239in}{1.610740in}}%
\pgfpathlineto{\pgfqpoint{0.548294in}{1.619106in}}%
\pgfpathlineto{\pgfqpoint{0.536336in}{1.624351in}}%
\pgfpathlineto{\pgfqpoint{0.532637in}{1.625764in}}%
\pgfpathlineto{\pgfqpoint{0.516981in}{1.628052in}}%
\pgfpathlineto{\pgfqpoint{0.501324in}{1.626527in}}%
\pgfpathlineto{\pgfqpoint{0.494857in}{1.624351in}}%
\pgfpathlineto{\pgfqpoint{0.485668in}{1.620813in}}%
\pgfpathlineto{\pgfqpoint{0.470252in}{1.610740in}}%
\pgfpathlineto{\pgfqpoint{0.470011in}{1.610530in}}%
\pgfpathlineto{\pgfqpoint{0.458424in}{1.597129in}}%
\pgfpathlineto{\pgfqpoint{0.454354in}{1.589139in}}%
\pgfpathlineto{\pgfqpoint{0.451851in}{1.583518in}}%
\pgfpathlineto{\pgfqpoint{0.450097in}{1.569907in}}%
\pgfpathlineto{\pgfqpoint{0.452729in}{1.556295in}}%
\pgfpathlineto{\pgfqpoint{0.454354in}{1.553080in}}%
\pgfpathlineto{\pgfqpoint{0.460387in}{1.542684in}}%
\pgfpathlineto{\pgfqpoint{0.470011in}{1.532299in}}%
\pgfpathlineto{\pgfqpoint{0.474001in}{1.529073in}}%
\pgfpathlineto{\pgfqpoint{0.485668in}{1.521759in}}%
\pgfpathlineto{\pgfqpoint{0.501324in}{1.515977in}}%
\pgfpathlineto{\pgfqpoint{0.506186in}{1.515462in}}%
\pgfpathlineto{\pgfqpoint{0.516981in}{1.514422in}}%
\pgfpathclose%
\pgfpathmoveto{\pgfqpoint{0.508762in}{1.542684in}}%
\pgfpathlineto{\pgfqpoint{0.501324in}{1.544315in}}%
\pgfpathlineto{\pgfqpoint{0.485680in}{1.556295in}}%
\pgfpathlineto{\pgfqpoint{0.485668in}{1.556321in}}%
\pgfpathlineto{\pgfqpoint{0.481514in}{1.569907in}}%
\pgfpathlineto{\pgfqpoint{0.484288in}{1.583518in}}%
\pgfpathlineto{\pgfqpoint{0.485668in}{1.585491in}}%
\pgfpathlineto{\pgfqpoint{0.499054in}{1.597129in}}%
\pgfpathlineto{\pgfqpoint{0.501324in}{1.598328in}}%
\pgfpathlineto{\pgfqpoint{0.516981in}{1.600740in}}%
\pgfpathlineto{\pgfqpoint{0.532608in}{1.597129in}}%
\pgfpathlineto{\pgfqpoint{0.532637in}{1.597118in}}%
\pgfpathlineto{\pgfqpoint{0.546418in}{1.583518in}}%
\pgfpathlineto{\pgfqpoint{0.548294in}{1.577051in}}%
\pgfpathlineto{\pgfqpoint{0.549657in}{1.569907in}}%
\pgfpathlineto{\pgfqpoint{0.548294in}{1.565127in}}%
\pgfpathlineto{\pgfqpoint{0.544450in}{1.556295in}}%
\pgfpathlineto{\pgfqpoint{0.532637in}{1.546026in}}%
\pgfpathlineto{\pgfqpoint{0.522478in}{1.542684in}}%
\pgfpathlineto{\pgfqpoint{0.516981in}{1.541499in}}%
\pgfpathlineto{\pgfqpoint{0.508762in}{1.542684in}}%
\pgfpathclose%
\pgfpathmoveto{\pgfqpoint{0.814455in}{1.515329in}}%
\pgfpathlineto{\pgfqpoint{0.830112in}{1.514573in}}%
\pgfpathlineto{\pgfqpoint{0.834760in}{1.515462in}}%
\pgfpathlineto{\pgfqpoint{0.845769in}{1.517793in}}%
\pgfpathlineto{\pgfqpoint{0.861425in}{1.525233in}}%
\pgfpathlineto{\pgfqpoint{0.866793in}{1.529073in}}%
\pgfpathlineto{\pgfqpoint{0.877082in}{1.538623in}}%
\pgfpathlineto{\pgfqpoint{0.880561in}{1.542684in}}%
\pgfpathlineto{\pgfqpoint{0.887968in}{1.556295in}}%
\pgfpathlineto{\pgfqpoint{0.890741in}{1.569907in}}%
\pgfpathlineto{\pgfqpoint{0.888893in}{1.583518in}}%
\pgfpathlineto{\pgfqpoint{0.882414in}{1.597129in}}%
\pgfpathlineto{\pgfqpoint{0.877082in}{1.603800in}}%
\pgfpathlineto{\pgfqpoint{0.870128in}{1.610740in}}%
\pgfpathlineto{\pgfqpoint{0.861425in}{1.617229in}}%
\pgfpathlineto{\pgfqpoint{0.846933in}{1.624351in}}%
\pgfpathlineto{\pgfqpoint{0.845769in}{1.624848in}}%
\pgfpathlineto{\pgfqpoint{0.830112in}{1.627900in}}%
\pgfpathlineto{\pgfqpoint{0.814455in}{1.627137in}}%
\pgfpathlineto{\pgfqpoint{0.804853in}{1.624351in}}%
\pgfpathlineto{\pgfqpoint{0.798799in}{1.622348in}}%
\pgfpathlineto{\pgfqpoint{0.783142in}{1.612963in}}%
\pgfpathlineto{\pgfqpoint{0.780465in}{1.610740in}}%
\pgfpathlineto{\pgfqpoint{0.768253in}{1.597129in}}%
\pgfpathlineto{\pgfqpoint{0.767486in}{1.595694in}}%
\pgfpathlineto{\pgfqpoint{0.761926in}{1.583518in}}%
\pgfpathlineto{\pgfqpoint{0.760149in}{1.569907in}}%
\pgfpathlineto{\pgfqpoint{0.762815in}{1.556295in}}%
\pgfpathlineto{\pgfqpoint{0.767486in}{1.547285in}}%
\pgfpathlineto{\pgfqpoint{0.770289in}{1.542684in}}%
\pgfpathlineto{\pgfqpoint{0.783142in}{1.529438in}}%
\pgfpathlineto{\pgfqpoint{0.783631in}{1.529073in}}%
\pgfpathlineto{\pgfqpoint{0.798799in}{1.520271in}}%
\pgfpathlineto{\pgfqpoint{0.813987in}{1.515462in}}%
\pgfpathlineto{\pgfqpoint{0.814455in}{1.515329in}}%
\pgfpathclose%
\pgfpathmoveto{\pgfqpoint{0.816853in}{1.542684in}}%
\pgfpathlineto{\pgfqpoint{0.814455in}{1.542947in}}%
\pgfpathlineto{\pgfqpoint{0.798799in}{1.553220in}}%
\pgfpathlineto{\pgfqpoint{0.796162in}{1.556295in}}%
\pgfpathlineto{\pgfqpoint{0.791695in}{1.569907in}}%
\pgfpathlineto{\pgfqpoint{0.794672in}{1.583518in}}%
\pgfpathlineto{\pgfqpoint{0.798799in}{1.588981in}}%
\pgfpathlineto{\pgfqpoint{0.809705in}{1.597129in}}%
\pgfpathlineto{\pgfqpoint{0.814455in}{1.599293in}}%
\pgfpathlineto{\pgfqpoint{0.830112in}{1.600499in}}%
\pgfpathlineto{\pgfqpoint{0.841094in}{1.597129in}}%
\pgfpathlineto{\pgfqpoint{0.845769in}{1.594794in}}%
\pgfpathlineto{\pgfqpoint{0.856001in}{1.583518in}}%
\pgfpathlineto{\pgfqpoint{0.859539in}{1.569907in}}%
\pgfpathlineto{\pgfqpoint{0.854231in}{1.556295in}}%
\pgfpathlineto{\pgfqpoint{0.845769in}{1.548080in}}%
\pgfpathlineto{\pgfqpoint{0.833461in}{1.542684in}}%
\pgfpathlineto{\pgfqpoint{0.830112in}{1.541726in}}%
\pgfpathlineto{\pgfqpoint{0.816853in}{1.542684in}}%
\pgfpathclose%
\pgfpathmoveto{\pgfqpoint{1.127587in}{1.514875in}}%
\pgfpathlineto{\pgfqpoint{1.143243in}{1.514875in}}%
\pgfpathlineto{\pgfqpoint{1.145709in}{1.515462in}}%
\pgfpathlineto{\pgfqpoint{1.158900in}{1.518949in}}%
\pgfpathlineto{\pgfqpoint{1.174556in}{1.527218in}}%
\pgfpathlineto{\pgfqpoint{1.177007in}{1.529073in}}%
\pgfpathlineto{\pgfqpoint{1.190213in}{1.542082in}}%
\pgfpathlineto{\pgfqpoint{1.190713in}{1.542684in}}%
\pgfpathlineto{\pgfqpoint{1.197962in}{1.556295in}}%
\pgfpathlineto{\pgfqpoint{1.200674in}{1.569907in}}%
\pgfpathlineto{\pgfqpoint{1.198866in}{1.583518in}}%
\pgfpathlineto{\pgfqpoint{1.192526in}{1.597129in}}%
\pgfpathlineto{\pgfqpoint{1.190213in}{1.600120in}}%
\pgfpathlineto{\pgfqpoint{1.180186in}{1.610740in}}%
\pgfpathlineto{\pgfqpoint{1.174556in}{1.615181in}}%
\pgfpathlineto{\pgfqpoint{1.158900in}{1.623712in}}%
\pgfpathlineto{\pgfqpoint{1.156586in}{1.624351in}}%
\pgfpathlineto{\pgfqpoint{1.143243in}{1.627595in}}%
\pgfpathlineto{\pgfqpoint{1.127587in}{1.627595in}}%
\pgfpathlineto{\pgfqpoint{1.114244in}{1.624351in}}%
\pgfpathlineto{\pgfqpoint{1.111930in}{1.623712in}}%
\pgfpathlineto{\pgfqpoint{1.096274in}{1.615181in}}%
\pgfpathlineto{\pgfqpoint{1.090644in}{1.610740in}}%
\pgfpathlineto{\pgfqpoint{1.080617in}{1.600120in}}%
\pgfpathlineto{\pgfqpoint{1.078304in}{1.597129in}}%
\pgfpathlineto{\pgfqpoint{1.071964in}{1.583518in}}%
\pgfpathlineto{\pgfqpoint{1.070156in}{1.569907in}}%
\pgfpathlineto{\pgfqpoint{1.072868in}{1.556295in}}%
\pgfpathlineto{\pgfqpoint{1.080117in}{1.542684in}}%
\pgfpathlineto{\pgfqpoint{1.080617in}{1.542082in}}%
\pgfpathlineto{\pgfqpoint{1.093823in}{1.529073in}}%
\pgfpathlineto{\pgfqpoint{1.096274in}{1.527218in}}%
\pgfpathlineto{\pgfqpoint{1.111930in}{1.518949in}}%
\pgfpathlineto{\pgfqpoint{1.125121in}{1.515462in}}%
\pgfpathlineto{\pgfqpoint{1.127587in}{1.514875in}}%
\pgfpathclose%
\pgfpathmoveto{\pgfqpoint{1.126167in}{1.542684in}}%
\pgfpathlineto{\pgfqpoint{1.111930in}{1.550478in}}%
\pgfpathlineto{\pgfqpoint{1.106493in}{1.556295in}}%
\pgfpathlineto{\pgfqpoint{1.101651in}{1.569907in}}%
\pgfpathlineto{\pgfqpoint{1.104878in}{1.583518in}}%
\pgfpathlineto{\pgfqpoint{1.111930in}{1.592082in}}%
\pgfpathlineto{\pgfqpoint{1.120022in}{1.597129in}}%
\pgfpathlineto{\pgfqpoint{1.127587in}{1.600016in}}%
\pgfpathlineto{\pgfqpoint{1.143243in}{1.600016in}}%
\pgfpathlineto{\pgfqpoint{1.150808in}{1.597129in}}%
\pgfpathlineto{\pgfqpoint{1.158900in}{1.592082in}}%
\pgfpathlineto{\pgfqpoint{1.165952in}{1.583518in}}%
\pgfpathlineto{\pgfqpoint{1.169179in}{1.569907in}}%
\pgfpathlineto{\pgfqpoint{1.164337in}{1.556295in}}%
\pgfpathlineto{\pgfqpoint{1.158900in}{1.550478in}}%
\pgfpathlineto{\pgfqpoint{1.144663in}{1.542684in}}%
\pgfpathlineto{\pgfqpoint{1.143243in}{1.542179in}}%
\pgfpathlineto{\pgfqpoint{1.127587in}{1.542179in}}%
\pgfpathlineto{\pgfqpoint{1.126167in}{1.542684in}}%
\pgfpathclose%
\pgfpathmoveto{\pgfqpoint{1.440718in}{1.514573in}}%
\pgfpathlineto{\pgfqpoint{1.456375in}{1.515329in}}%
\pgfpathlineto{\pgfqpoint{1.456843in}{1.515462in}}%
\pgfpathlineto{\pgfqpoint{1.472031in}{1.520271in}}%
\pgfpathlineto{\pgfqpoint{1.487199in}{1.529073in}}%
\pgfpathlineto{\pgfqpoint{1.487688in}{1.529438in}}%
\pgfpathlineto{\pgfqpoint{1.500541in}{1.542684in}}%
\pgfpathlineto{\pgfqpoint{1.503344in}{1.547285in}}%
\pgfpathlineto{\pgfqpoint{1.508015in}{1.556295in}}%
\pgfpathlineto{\pgfqpoint{1.510681in}{1.569907in}}%
\pgfpathlineto{\pgfqpoint{1.508904in}{1.583518in}}%
\pgfpathlineto{\pgfqpoint{1.503344in}{1.595694in}}%
\pgfpathlineto{\pgfqpoint{1.502577in}{1.597129in}}%
\pgfpathlineto{\pgfqpoint{1.490365in}{1.610740in}}%
\pgfpathlineto{\pgfqpoint{1.487688in}{1.612963in}}%
\pgfpathlineto{\pgfqpoint{1.472031in}{1.622348in}}%
\pgfpathlineto{\pgfqpoint{1.465977in}{1.624351in}}%
\pgfpathlineto{\pgfqpoint{1.456375in}{1.627137in}}%
\pgfpathlineto{\pgfqpoint{1.440718in}{1.627900in}}%
\pgfpathlineto{\pgfqpoint{1.425061in}{1.624848in}}%
\pgfpathlineto{\pgfqpoint{1.423897in}{1.624351in}}%
\pgfpathlineto{\pgfqpoint{1.409405in}{1.617229in}}%
\pgfpathlineto{\pgfqpoint{1.400702in}{1.610740in}}%
\pgfpathlineto{\pgfqpoint{1.393748in}{1.603800in}}%
\pgfpathlineto{\pgfqpoint{1.388416in}{1.597129in}}%
\pgfpathlineto{\pgfqpoint{1.381937in}{1.583518in}}%
\pgfpathlineto{\pgfqpoint{1.380089in}{1.569907in}}%
\pgfpathlineto{\pgfqpoint{1.382862in}{1.556295in}}%
\pgfpathlineto{\pgfqpoint{1.390269in}{1.542684in}}%
\pgfpathlineto{\pgfqpoint{1.393748in}{1.538623in}}%
\pgfpathlineto{\pgfqpoint{1.404037in}{1.529073in}}%
\pgfpathlineto{\pgfqpoint{1.409405in}{1.525233in}}%
\pgfpathlineto{\pgfqpoint{1.425061in}{1.517793in}}%
\pgfpathlineto{\pgfqpoint{1.436070in}{1.515462in}}%
\pgfpathlineto{\pgfqpoint{1.440718in}{1.514573in}}%
\pgfpathclose%
\pgfpathmoveto{\pgfqpoint{1.437369in}{1.542684in}}%
\pgfpathlineto{\pgfqpoint{1.425061in}{1.548080in}}%
\pgfpathlineto{\pgfqpoint{1.416599in}{1.556295in}}%
\pgfpathlineto{\pgfqpoint{1.411291in}{1.569907in}}%
\pgfpathlineto{\pgfqpoint{1.414829in}{1.583518in}}%
\pgfpathlineto{\pgfqpoint{1.425061in}{1.594794in}}%
\pgfpathlineto{\pgfqpoint{1.429736in}{1.597129in}}%
\pgfpathlineto{\pgfqpoint{1.440718in}{1.600499in}}%
\pgfpathlineto{\pgfqpoint{1.456375in}{1.599293in}}%
\pgfpathlineto{\pgfqpoint{1.461125in}{1.597129in}}%
\pgfpathlineto{\pgfqpoint{1.472031in}{1.588981in}}%
\pgfpathlineto{\pgfqpoint{1.476158in}{1.583518in}}%
\pgfpathlineto{\pgfqpoint{1.479135in}{1.569907in}}%
\pgfpathlineto{\pgfqpoint{1.474668in}{1.556295in}}%
\pgfpathlineto{\pgfqpoint{1.472031in}{1.553220in}}%
\pgfpathlineto{\pgfqpoint{1.456375in}{1.542947in}}%
\pgfpathlineto{\pgfqpoint{1.453977in}{1.542684in}}%
\pgfpathlineto{\pgfqpoint{1.440718in}{1.541726in}}%
\pgfpathlineto{\pgfqpoint{1.437369in}{1.542684in}}%
\pgfpathclose%
\pgfpathmoveto{\pgfqpoint{1.753849in}{1.514422in}}%
\pgfpathlineto{\pgfqpoint{1.764644in}{1.515462in}}%
\pgfpathlineto{\pgfqpoint{1.769506in}{1.515977in}}%
\pgfpathlineto{\pgfqpoint{1.785162in}{1.521759in}}%
\pgfpathlineto{\pgfqpoint{1.796829in}{1.529073in}}%
\pgfpathlineto{\pgfqpoint{1.800819in}{1.532299in}}%
\pgfpathlineto{\pgfqpoint{1.810443in}{1.542684in}}%
\pgfpathlineto{\pgfqpoint{1.816476in}{1.553080in}}%
\pgfpathlineto{\pgfqpoint{1.818101in}{1.556295in}}%
\pgfpathlineto{\pgfqpoint{1.820733in}{1.569907in}}%
\pgfpathlineto{\pgfqpoint{1.818979in}{1.583518in}}%
\pgfpathlineto{\pgfqpoint{1.816476in}{1.589139in}}%
\pgfpathlineto{\pgfqpoint{1.812406in}{1.597129in}}%
\pgfpathlineto{\pgfqpoint{1.800819in}{1.610530in}}%
\pgfpathlineto{\pgfqpoint{1.800578in}{1.610740in}}%
\pgfpathlineto{\pgfqpoint{1.785162in}{1.620813in}}%
\pgfpathlineto{\pgfqpoint{1.775973in}{1.624351in}}%
\pgfpathlineto{\pgfqpoint{1.769506in}{1.626527in}}%
\pgfpathlineto{\pgfqpoint{1.753849in}{1.628052in}}%
\pgfpathlineto{\pgfqpoint{1.738193in}{1.625764in}}%
\pgfpathlineto{\pgfqpoint{1.734494in}{1.624351in}}%
\pgfpathlineto{\pgfqpoint{1.722536in}{1.619106in}}%
\pgfpathlineto{\pgfqpoint{1.710591in}{1.610740in}}%
\pgfpathlineto{\pgfqpoint{1.706880in}{1.607271in}}%
\pgfpathlineto{\pgfqpoint{1.698466in}{1.597129in}}%
\pgfpathlineto{\pgfqpoint{1.691815in}{1.583518in}}%
\pgfpathlineto{\pgfqpoint{1.691223in}{1.579291in}}%
\pgfpathlineto{\pgfqpoint{1.690027in}{1.569907in}}%
\pgfpathlineto{\pgfqpoint{1.691223in}{1.563629in}}%
\pgfpathlineto{\pgfqpoint{1.692764in}{1.556295in}}%
\pgfpathlineto{\pgfqpoint{1.700369in}{1.542684in}}%
\pgfpathlineto{\pgfqpoint{1.706880in}{1.535362in}}%
\pgfpathlineto{\pgfqpoint{1.714113in}{1.529073in}}%
\pgfpathlineto{\pgfqpoint{1.722536in}{1.523413in}}%
\pgfpathlineto{\pgfqpoint{1.738193in}{1.516802in}}%
\pgfpathlineto{\pgfqpoint{1.746628in}{1.515462in}}%
\pgfpathlineto{\pgfqpoint{1.753849in}{1.514422in}}%
\pgfpathclose%
\pgfpathmoveto{\pgfqpoint{1.748352in}{1.542684in}}%
\pgfpathlineto{\pgfqpoint{1.738193in}{1.546026in}}%
\pgfpathlineto{\pgfqpoint{1.726380in}{1.556295in}}%
\pgfpathlineto{\pgfqpoint{1.722536in}{1.565127in}}%
\pgfpathlineto{\pgfqpoint{1.721173in}{1.569907in}}%
\pgfpathlineto{\pgfqpoint{1.722536in}{1.577051in}}%
\pgfpathlineto{\pgfqpoint{1.724412in}{1.583518in}}%
\pgfpathlineto{\pgfqpoint{1.738193in}{1.597118in}}%
\pgfpathlineto{\pgfqpoint{1.738222in}{1.597129in}}%
\pgfpathlineto{\pgfqpoint{1.753849in}{1.600740in}}%
\pgfpathlineto{\pgfqpoint{1.769506in}{1.598328in}}%
\pgfpathlineto{\pgfqpoint{1.771776in}{1.597129in}}%
\pgfpathlineto{\pgfqpoint{1.785162in}{1.585491in}}%
\pgfpathlineto{\pgfqpoint{1.786542in}{1.583518in}}%
\pgfpathlineto{\pgfqpoint{1.789316in}{1.569907in}}%
\pgfpathlineto{\pgfqpoint{1.785162in}{1.556321in}}%
\pgfpathlineto{\pgfqpoint{1.785150in}{1.556295in}}%
\pgfpathlineto{\pgfqpoint{1.769506in}{1.544315in}}%
\pgfpathlineto{\pgfqpoint{1.762068in}{1.542684in}}%
\pgfpathlineto{\pgfqpoint{1.753849in}{1.541499in}}%
\pgfpathlineto{\pgfqpoint{1.748352in}{1.542684in}}%
\pgfpathclose%
\pgfusepath{fill}%
\end{pgfscope}%
\begin{pgfscope}%
\pgfpathrectangle{\pgfqpoint{0.360415in}{0.358518in}}{\pgfqpoint{1.550000in}{1.347500in}}%
\pgfusepath{clip}%
\pgfsetbuttcap%
\pgfsetroundjoin%
\definecolor{currentfill}{rgb}{0.921884,0.341098,0.377376}%
\pgfsetfillcolor{currentfill}%
\pgfsetlinewidth{0.000000pt}%
\definecolor{currentstroke}{rgb}{0.000000,0.000000,0.000000}%
\pgfsetstrokecolor{currentstroke}%
\pgfsetdash{}{0pt}%
\pgfpathmoveto{\pgfqpoint{0.485668in}{0.419683in}}%
\pgfpathlineto{\pgfqpoint{0.501324in}{0.414945in}}%
\pgfpathlineto{\pgfqpoint{0.516981in}{0.413594in}}%
\pgfpathlineto{\pgfqpoint{0.532637in}{0.415621in}}%
\pgfpathlineto{\pgfqpoint{0.548294in}{0.421039in}}%
\pgfpathlineto{\pgfqpoint{0.558300in}{0.426573in}}%
\pgfpathlineto{\pgfqpoint{0.563950in}{0.429854in}}%
\pgfpathlineto{\pgfqpoint{0.577405in}{0.440184in}}%
\pgfpathlineto{\pgfqpoint{0.579607in}{0.442184in}}%
\pgfpathlineto{\pgfqpoint{0.590683in}{0.453795in}}%
\pgfpathlineto{\pgfqpoint{0.595263in}{0.460354in}}%
\pgfpathlineto{\pgfqpoint{0.600066in}{0.467407in}}%
\pgfpathlineto{\pgfqpoint{0.605579in}{0.481018in}}%
\pgfpathlineto{\pgfqpoint{0.607151in}{0.494629in}}%
\pgfpathlineto{\pgfqpoint{0.604792in}{0.508240in}}%
\pgfpathlineto{\pgfqpoint{0.598489in}{0.521851in}}%
\pgfpathlineto{\pgfqpoint{0.595263in}{0.526270in}}%
\pgfpathlineto{\pgfqpoint{0.588361in}{0.535462in}}%
\pgfpathlineto{\pgfqpoint{0.579607in}{0.544267in}}%
\pgfpathlineto{\pgfqpoint{0.574079in}{0.549073in}}%
\pgfpathlineto{\pgfqpoint{0.563950in}{0.556684in}}%
\pgfpathlineto{\pgfqpoint{0.553377in}{0.562684in}}%
\pgfpathlineto{\pgfqpoint{0.548294in}{0.565488in}}%
\pgfpathlineto{\pgfqpoint{0.532637in}{0.570968in}}%
\pgfpathlineto{\pgfqpoint{0.516981in}{0.573019in}}%
\pgfpathlineto{\pgfqpoint{0.501324in}{0.571652in}}%
\pgfpathlineto{\pgfqpoint{0.485668in}{0.566859in}}%
\pgfpathlineto{\pgfqpoint{0.477556in}{0.562684in}}%
\pgfpathlineto{\pgfqpoint{0.470011in}{0.558702in}}%
\pgfpathlineto{\pgfqpoint{0.456655in}{0.549073in}}%
\pgfpathlineto{\pgfqpoint{0.454354in}{0.547159in}}%
\pgfpathlineto{\pgfqpoint{0.442472in}{0.535462in}}%
\pgfpathlineto{\pgfqpoint{0.438698in}{0.530550in}}%
\pgfpathlineto{\pgfqpoint{0.432331in}{0.521851in}}%
\pgfpathlineto{\pgfqpoint{0.426100in}{0.508240in}}%
\pgfpathlineto{\pgfqpoint{0.423768in}{0.494629in}}%
\pgfpathlineto{\pgfqpoint{0.425323in}{0.481018in}}%
\pgfpathlineto{\pgfqpoint{0.430772in}{0.467407in}}%
\pgfpathlineto{\pgfqpoint{0.438698in}{0.455799in}}%
\pgfpathlineto{\pgfqpoint{0.440130in}{0.453795in}}%
\pgfpathlineto{\pgfqpoint{0.453340in}{0.440184in}}%
\pgfpathlineto{\pgfqpoint{0.454354in}{0.439302in}}%
\pgfpathlineto{\pgfqpoint{0.470011in}{0.427818in}}%
\pgfpathlineto{\pgfqpoint{0.472316in}{0.426573in}}%
\pgfpathlineto{\pgfqpoint{0.485668in}{0.419683in}}%
\pgfpathclose%
\pgfpathmoveto{\pgfqpoint{0.494857in}{0.440184in}}%
\pgfpathlineto{\pgfqpoint{0.485668in}{0.443722in}}%
\pgfpathlineto{\pgfqpoint{0.470252in}{0.453795in}}%
\pgfpathlineto{\pgfqpoint{0.470011in}{0.454005in}}%
\pgfpathlineto{\pgfqpoint{0.458424in}{0.467407in}}%
\pgfpathlineto{\pgfqpoint{0.454354in}{0.475396in}}%
\pgfpathlineto{\pgfqpoint{0.451851in}{0.481018in}}%
\pgfpathlineto{\pgfqpoint{0.450097in}{0.494629in}}%
\pgfpathlineto{\pgfqpoint{0.452729in}{0.508240in}}%
\pgfpathlineto{\pgfqpoint{0.454354in}{0.511455in}}%
\pgfpathlineto{\pgfqpoint{0.460387in}{0.521851in}}%
\pgfpathlineto{\pgfqpoint{0.470011in}{0.532236in}}%
\pgfpathlineto{\pgfqpoint{0.474001in}{0.535462in}}%
\pgfpathlineto{\pgfqpoint{0.485668in}{0.542776in}}%
\pgfpathlineto{\pgfqpoint{0.501324in}{0.548559in}}%
\pgfpathlineto{\pgfqpoint{0.506186in}{0.549073in}}%
\pgfpathlineto{\pgfqpoint{0.516981in}{0.550113in}}%
\pgfpathlineto{\pgfqpoint{0.524202in}{0.549073in}}%
\pgfpathlineto{\pgfqpoint{0.532637in}{0.547733in}}%
\pgfpathlineto{\pgfqpoint{0.548294in}{0.541122in}}%
\pgfpathlineto{\pgfqpoint{0.556717in}{0.535462in}}%
\pgfpathlineto{\pgfqpoint{0.563950in}{0.529173in}}%
\pgfpathlineto{\pgfqpoint{0.570461in}{0.521851in}}%
\pgfpathlineto{\pgfqpoint{0.578066in}{0.508240in}}%
\pgfpathlineto{\pgfqpoint{0.579607in}{0.500907in}}%
\pgfpathlineto{\pgfqpoint{0.580803in}{0.494629in}}%
\pgfpathlineto{\pgfqpoint{0.579607in}{0.485244in}}%
\pgfpathlineto{\pgfqpoint{0.579015in}{0.481018in}}%
\pgfpathlineto{\pgfqpoint{0.572364in}{0.467407in}}%
\pgfpathlineto{\pgfqpoint{0.563950in}{0.457264in}}%
\pgfpathlineto{\pgfqpoint{0.560239in}{0.453795in}}%
\pgfpathlineto{\pgfqpoint{0.548294in}{0.445429in}}%
\pgfpathlineto{\pgfqpoint{0.536336in}{0.440184in}}%
\pgfpathlineto{\pgfqpoint{0.532637in}{0.438771in}}%
\pgfpathlineto{\pgfqpoint{0.516981in}{0.436483in}}%
\pgfpathlineto{\pgfqpoint{0.501324in}{0.438008in}}%
\pgfpathlineto{\pgfqpoint{0.494857in}{0.440184in}}%
\pgfpathclose%
\pgfpathmoveto{\pgfqpoint{0.783142in}{0.425918in}}%
\pgfpathlineto{\pgfqpoint{0.798799in}{0.418464in}}%
\pgfpathlineto{\pgfqpoint{0.814455in}{0.414405in}}%
\pgfpathlineto{\pgfqpoint{0.830112in}{0.413729in}}%
\pgfpathlineto{\pgfqpoint{0.845769in}{0.416433in}}%
\pgfpathlineto{\pgfqpoint{0.861425in}{0.422530in}}%
\pgfpathlineto{\pgfqpoint{0.868292in}{0.426573in}}%
\pgfpathlineto{\pgfqpoint{0.877082in}{0.432023in}}%
\pgfpathlineto{\pgfqpoint{0.887325in}{0.440184in}}%
\pgfpathlineto{\pgfqpoint{0.892738in}{0.445299in}}%
\pgfpathlineto{\pgfqpoint{0.900700in}{0.453795in}}%
\pgfpathlineto{\pgfqpoint{0.908395in}{0.465026in}}%
\pgfpathlineto{\pgfqpoint{0.910018in}{0.467407in}}%
\pgfpathlineto{\pgfqpoint{0.915619in}{0.481018in}}%
\pgfpathlineto{\pgfqpoint{0.917216in}{0.494629in}}%
\pgfpathlineto{\pgfqpoint{0.914820in}{0.508240in}}%
\pgfpathlineto{\pgfqpoint{0.908416in}{0.521851in}}%
\pgfpathlineto{\pgfqpoint{0.908395in}{0.521880in}}%
\pgfpathlineto{\pgfqpoint{0.898390in}{0.535462in}}%
\pgfpathlineto{\pgfqpoint{0.892738in}{0.541248in}}%
\pgfpathlineto{\pgfqpoint{0.884085in}{0.549073in}}%
\pgfpathlineto{\pgfqpoint{0.877082in}{0.554534in}}%
\pgfpathlineto{\pgfqpoint{0.863631in}{0.562684in}}%
\pgfpathlineto{\pgfqpoint{0.861425in}{0.563980in}}%
\pgfpathlineto{\pgfqpoint{0.845769in}{0.570147in}}%
\pgfpathlineto{\pgfqpoint{0.830112in}{0.572882in}}%
\pgfpathlineto{\pgfqpoint{0.814455in}{0.572199in}}%
\pgfpathlineto{\pgfqpoint{0.798799in}{0.568093in}}%
\pgfpathlineto{\pgfqpoint{0.787447in}{0.562684in}}%
\pgfpathlineto{\pgfqpoint{0.783142in}{0.560588in}}%
\pgfpathlineto{\pgfqpoint{0.767486in}{0.549849in}}%
\pgfpathlineto{\pgfqpoint{0.766551in}{0.549073in}}%
\pgfpathlineto{\pgfqpoint{0.752424in}{0.535462in}}%
\pgfpathlineto{\pgfqpoint{0.751829in}{0.534708in}}%
\pgfpathlineto{\pgfqpoint{0.742357in}{0.521851in}}%
\pgfpathlineto{\pgfqpoint{0.736173in}{0.508241in}}%
\pgfpathlineto{\pgfqpoint{0.736172in}{0.508240in}}%
\pgfpathlineto{\pgfqpoint{0.733666in}{0.494629in}}%
\pgfpathlineto{\pgfqpoint{0.735337in}{0.481018in}}%
\pgfpathlineto{\pgfqpoint{0.736173in}{0.479032in}}%
\pgfpathlineto{\pgfqpoint{0.740810in}{0.467407in}}%
\pgfpathlineto{\pgfqpoint{0.750093in}{0.453795in}}%
\pgfpathlineto{\pgfqpoint{0.751829in}{0.451897in}}%
\pgfpathlineto{\pgfqpoint{0.763434in}{0.440184in}}%
\pgfpathlineto{\pgfqpoint{0.767486in}{0.436750in}}%
\pgfpathlineto{\pgfqpoint{0.782143in}{0.426573in}}%
\pgfpathlineto{\pgfqpoint{0.783142in}{0.425918in}}%
\pgfpathclose%
\pgfpathmoveto{\pgfqpoint{0.804853in}{0.440184in}}%
\pgfpathlineto{\pgfqpoint{0.798799in}{0.442187in}}%
\pgfpathlineto{\pgfqpoint{0.783142in}{0.451572in}}%
\pgfpathlineto{\pgfqpoint{0.780465in}{0.453795in}}%
\pgfpathlineto{\pgfqpoint{0.768253in}{0.467407in}}%
\pgfpathlineto{\pgfqpoint{0.767486in}{0.468841in}}%
\pgfpathlineto{\pgfqpoint{0.761926in}{0.481018in}}%
\pgfpathlineto{\pgfqpoint{0.760149in}{0.494629in}}%
\pgfpathlineto{\pgfqpoint{0.762815in}{0.508240in}}%
\pgfpathlineto{\pgfqpoint{0.767486in}{0.517250in}}%
\pgfpathlineto{\pgfqpoint{0.770289in}{0.521851in}}%
\pgfpathlineto{\pgfqpoint{0.783142in}{0.535097in}}%
\pgfpathlineto{\pgfqpoint{0.783631in}{0.535462in}}%
\pgfpathlineto{\pgfqpoint{0.798799in}{0.544264in}}%
\pgfpathlineto{\pgfqpoint{0.813987in}{0.549073in}}%
\pgfpathlineto{\pgfqpoint{0.814455in}{0.549206in}}%
\pgfpathlineto{\pgfqpoint{0.830112in}{0.549962in}}%
\pgfpathlineto{\pgfqpoint{0.834760in}{0.549073in}}%
\pgfpathlineto{\pgfqpoint{0.845769in}{0.546743in}}%
\pgfpathlineto{\pgfqpoint{0.861425in}{0.539302in}}%
\pgfpathlineto{\pgfqpoint{0.866793in}{0.535462in}}%
\pgfpathlineto{\pgfqpoint{0.877082in}{0.525912in}}%
\pgfpathlineto{\pgfqpoint{0.880561in}{0.521851in}}%
\pgfpathlineto{\pgfqpoint{0.887968in}{0.508240in}}%
\pgfpathlineto{\pgfqpoint{0.890741in}{0.494629in}}%
\pgfpathlineto{\pgfqpoint{0.888893in}{0.481018in}}%
\pgfpathlineto{\pgfqpoint{0.882414in}{0.467407in}}%
\pgfpathlineto{\pgfqpoint{0.877082in}{0.460735in}}%
\pgfpathlineto{\pgfqpoint{0.870128in}{0.453795in}}%
\pgfpathlineto{\pgfqpoint{0.861425in}{0.447306in}}%
\pgfpathlineto{\pgfqpoint{0.846933in}{0.440184in}}%
\pgfpathlineto{\pgfqpoint{0.845769in}{0.439688in}}%
\pgfpathlineto{\pgfqpoint{0.830112in}{0.436635in}}%
\pgfpathlineto{\pgfqpoint{0.814455in}{0.437398in}}%
\pgfpathlineto{\pgfqpoint{0.804853in}{0.440184in}}%
\pgfpathclose%
\pgfpathmoveto{\pgfqpoint{1.096274in}{0.424156in}}%
\pgfpathlineto{\pgfqpoint{1.111930in}{0.417381in}}%
\pgfpathlineto{\pgfqpoint{1.127587in}{0.413999in}}%
\pgfpathlineto{\pgfqpoint{1.143243in}{0.413999in}}%
\pgfpathlineto{\pgfqpoint{1.158900in}{0.417381in}}%
\pgfpathlineto{\pgfqpoint{1.174556in}{0.424156in}}%
\pgfpathlineto{\pgfqpoint{1.178435in}{0.426573in}}%
\pgfpathlineto{\pgfqpoint{1.190213in}{0.434322in}}%
\pgfpathlineto{\pgfqpoint{1.197332in}{0.440184in}}%
\pgfpathlineto{\pgfqpoint{1.205870in}{0.448539in}}%
\pgfpathlineto{\pgfqpoint{1.210727in}{0.453795in}}%
\pgfpathlineto{\pgfqpoint{1.219979in}{0.467407in}}%
\pgfpathlineto{\pgfqpoint{1.221526in}{0.471242in}}%
\pgfpathlineto{\pgfqpoint{1.225600in}{0.481018in}}%
\pgfpathlineto{\pgfqpoint{1.227230in}{0.494629in}}%
\pgfpathlineto{\pgfqpoint{1.224785in}{0.508240in}}%
\pgfpathlineto{\pgfqpoint{1.221526in}{0.515128in}}%
\pgfpathlineto{\pgfqpoint{1.218437in}{0.521851in}}%
\pgfpathlineto{\pgfqpoint{1.208417in}{0.535462in}}%
\pgfpathlineto{\pgfqpoint{1.205870in}{0.538108in}}%
\pgfpathlineto{\pgfqpoint{1.194161in}{0.549073in}}%
\pgfpathlineto{\pgfqpoint{1.190213in}{0.552255in}}%
\pgfpathlineto{\pgfqpoint{1.174556in}{0.562340in}}%
\pgfpathlineto{\pgfqpoint{1.173784in}{0.562684in}}%
\pgfpathlineto{\pgfqpoint{1.158900in}{0.569189in}}%
\pgfpathlineto{\pgfqpoint{1.143243in}{0.572609in}}%
\pgfpathlineto{\pgfqpoint{1.127587in}{0.572609in}}%
\pgfpathlineto{\pgfqpoint{1.111930in}{0.569189in}}%
\pgfpathlineto{\pgfqpoint{1.097046in}{0.562684in}}%
\pgfpathlineto{\pgfqpoint{1.096274in}{0.562340in}}%
\pgfpathlineto{\pgfqpoint{1.080617in}{0.552255in}}%
\pgfpathlineto{\pgfqpoint{1.076669in}{0.549073in}}%
\pgfpathlineto{\pgfqpoint{1.064960in}{0.538108in}}%
\pgfpathlineto{\pgfqpoint{1.062413in}{0.535462in}}%
\pgfpathlineto{\pgfqpoint{1.052393in}{0.521851in}}%
\pgfpathlineto{\pgfqpoint{1.049304in}{0.515128in}}%
\pgfpathlineto{\pgfqpoint{1.046045in}{0.508240in}}%
\pgfpathlineto{\pgfqpoint{1.043600in}{0.494629in}}%
\pgfpathlineto{\pgfqpoint{1.045230in}{0.481018in}}%
\pgfpathlineto{\pgfqpoint{1.049304in}{0.471242in}}%
\pgfpathlineto{\pgfqpoint{1.050851in}{0.467407in}}%
\pgfpathlineto{\pgfqpoint{1.060103in}{0.453795in}}%
\pgfpathlineto{\pgfqpoint{1.064960in}{0.448539in}}%
\pgfpathlineto{\pgfqpoint{1.073498in}{0.440184in}}%
\pgfpathlineto{\pgfqpoint{1.080617in}{0.434322in}}%
\pgfpathlineto{\pgfqpoint{1.092395in}{0.426573in}}%
\pgfpathlineto{\pgfqpoint{1.096274in}{0.424156in}}%
\pgfpathclose%
\pgfpathmoveto{\pgfqpoint{1.114244in}{0.440184in}}%
\pgfpathlineto{\pgfqpoint{1.111930in}{0.440823in}}%
\pgfpathlineto{\pgfqpoint{1.096274in}{0.449354in}}%
\pgfpathlineto{\pgfqpoint{1.090644in}{0.453795in}}%
\pgfpathlineto{\pgfqpoint{1.080617in}{0.464415in}}%
\pgfpathlineto{\pgfqpoint{1.078304in}{0.467407in}}%
\pgfpathlineto{\pgfqpoint{1.071964in}{0.481018in}}%
\pgfpathlineto{\pgfqpoint{1.070156in}{0.494629in}}%
\pgfpathlineto{\pgfqpoint{1.072868in}{0.508240in}}%
\pgfpathlineto{\pgfqpoint{1.080117in}{0.521851in}}%
\pgfpathlineto{\pgfqpoint{1.080617in}{0.522454in}}%
\pgfpathlineto{\pgfqpoint{1.093823in}{0.535462in}}%
\pgfpathlineto{\pgfqpoint{1.096274in}{0.537317in}}%
\pgfpathlineto{\pgfqpoint{1.111930in}{0.545586in}}%
\pgfpathlineto{\pgfqpoint{1.125121in}{0.549073in}}%
\pgfpathlineto{\pgfqpoint{1.127587in}{0.549660in}}%
\pgfpathlineto{\pgfqpoint{1.143243in}{0.549660in}}%
\pgfpathlineto{\pgfqpoint{1.145709in}{0.549073in}}%
\pgfpathlineto{\pgfqpoint{1.158900in}{0.545586in}}%
\pgfpathlineto{\pgfqpoint{1.174556in}{0.537317in}}%
\pgfpathlineto{\pgfqpoint{1.177007in}{0.535462in}}%
\pgfpathlineto{\pgfqpoint{1.190213in}{0.522454in}}%
\pgfpathlineto{\pgfqpoint{1.190713in}{0.521851in}}%
\pgfpathlineto{\pgfqpoint{1.197962in}{0.508240in}}%
\pgfpathlineto{\pgfqpoint{1.200674in}{0.494629in}}%
\pgfpathlineto{\pgfqpoint{1.198866in}{0.481018in}}%
\pgfpathlineto{\pgfqpoint{1.192526in}{0.467407in}}%
\pgfpathlineto{\pgfqpoint{1.190213in}{0.464415in}}%
\pgfpathlineto{\pgfqpoint{1.180186in}{0.453795in}}%
\pgfpathlineto{\pgfqpoint{1.174556in}{0.449354in}}%
\pgfpathlineto{\pgfqpoint{1.158900in}{0.440823in}}%
\pgfpathlineto{\pgfqpoint{1.156586in}{0.440184in}}%
\pgfpathlineto{\pgfqpoint{1.143243in}{0.436940in}}%
\pgfpathlineto{\pgfqpoint{1.127587in}{0.436940in}}%
\pgfpathlineto{\pgfqpoint{1.114244in}{0.440184in}}%
\pgfpathclose%
\pgfpathmoveto{\pgfqpoint{1.409405in}{0.422530in}}%
\pgfpathlineto{\pgfqpoint{1.425061in}{0.416433in}}%
\pgfpathlineto{\pgfqpoint{1.440718in}{0.413729in}}%
\pgfpathlineto{\pgfqpoint{1.456375in}{0.414405in}}%
\pgfpathlineto{\pgfqpoint{1.472031in}{0.418464in}}%
\pgfpathlineto{\pgfqpoint{1.487688in}{0.425918in}}%
\pgfpathlineto{\pgfqpoint{1.488687in}{0.426573in}}%
\pgfpathlineto{\pgfqpoint{1.503344in}{0.436750in}}%
\pgfpathlineto{\pgfqpoint{1.507396in}{0.440184in}}%
\pgfpathlineto{\pgfqpoint{1.519001in}{0.451897in}}%
\pgfpathlineto{\pgfqpoint{1.520737in}{0.453795in}}%
\pgfpathlineto{\pgfqpoint{1.530020in}{0.467407in}}%
\pgfpathlineto{\pgfqpoint{1.534657in}{0.479032in}}%
\pgfpathlineto{\pgfqpoint{1.535493in}{0.481018in}}%
\pgfpathlineto{\pgfqpoint{1.537164in}{0.494629in}}%
\pgfpathlineto{\pgfqpoint{1.534658in}{0.508240in}}%
\pgfpathlineto{\pgfqpoint{1.534657in}{0.508241in}}%
\pgfpathlineto{\pgfqpoint{1.528473in}{0.521851in}}%
\pgfpathlineto{\pgfqpoint{1.519001in}{0.534708in}}%
\pgfpathlineto{\pgfqpoint{1.518406in}{0.535462in}}%
\pgfpathlineto{\pgfqpoint{1.504279in}{0.549073in}}%
\pgfpathlineto{\pgfqpoint{1.503344in}{0.549849in}}%
\pgfpathlineto{\pgfqpoint{1.487688in}{0.560588in}}%
\pgfpathlineto{\pgfqpoint{1.483383in}{0.562684in}}%
\pgfpathlineto{\pgfqpoint{1.472031in}{0.568093in}}%
\pgfpathlineto{\pgfqpoint{1.456375in}{0.572199in}}%
\pgfpathlineto{\pgfqpoint{1.440718in}{0.572882in}}%
\pgfpathlineto{\pgfqpoint{1.425061in}{0.570147in}}%
\pgfpathlineto{\pgfqpoint{1.409405in}{0.563980in}}%
\pgfpathlineto{\pgfqpoint{1.407199in}{0.562684in}}%
\pgfpathlineto{\pgfqpoint{1.393748in}{0.554534in}}%
\pgfpathlineto{\pgfqpoint{1.386745in}{0.549073in}}%
\pgfpathlineto{\pgfqpoint{1.378092in}{0.541248in}}%
\pgfpathlineto{\pgfqpoint{1.372440in}{0.535462in}}%
\pgfpathlineto{\pgfqpoint{1.362435in}{0.521880in}}%
\pgfpathlineto{\pgfqpoint{1.362414in}{0.521851in}}%
\pgfpathlineto{\pgfqpoint{1.356010in}{0.508240in}}%
\pgfpathlineto{\pgfqpoint{1.353614in}{0.494629in}}%
\pgfpathlineto{\pgfqpoint{1.355211in}{0.481018in}}%
\pgfpathlineto{\pgfqpoint{1.360812in}{0.467407in}}%
\pgfpathlineto{\pgfqpoint{1.362435in}{0.465026in}}%
\pgfpathlineto{\pgfqpoint{1.370130in}{0.453795in}}%
\pgfpathlineto{\pgfqpoint{1.378092in}{0.445299in}}%
\pgfpathlineto{\pgfqpoint{1.383505in}{0.440184in}}%
\pgfpathlineto{\pgfqpoint{1.393748in}{0.432023in}}%
\pgfpathlineto{\pgfqpoint{1.402538in}{0.426573in}}%
\pgfpathlineto{\pgfqpoint{1.409405in}{0.422530in}}%
\pgfpathclose%
\pgfpathmoveto{\pgfqpoint{1.423897in}{0.440184in}}%
\pgfpathlineto{\pgfqpoint{1.409405in}{0.447306in}}%
\pgfpathlineto{\pgfqpoint{1.400702in}{0.453795in}}%
\pgfpathlineto{\pgfqpoint{1.393748in}{0.460735in}}%
\pgfpathlineto{\pgfqpoint{1.388416in}{0.467407in}}%
\pgfpathlineto{\pgfqpoint{1.381937in}{0.481018in}}%
\pgfpathlineto{\pgfqpoint{1.380089in}{0.494629in}}%
\pgfpathlineto{\pgfqpoint{1.382862in}{0.508240in}}%
\pgfpathlineto{\pgfqpoint{1.390269in}{0.521851in}}%
\pgfpathlineto{\pgfqpoint{1.393748in}{0.525912in}}%
\pgfpathlineto{\pgfqpoint{1.404037in}{0.535462in}}%
\pgfpathlineto{\pgfqpoint{1.409405in}{0.539302in}}%
\pgfpathlineto{\pgfqpoint{1.425061in}{0.546743in}}%
\pgfpathlineto{\pgfqpoint{1.436070in}{0.549073in}}%
\pgfpathlineto{\pgfqpoint{1.440718in}{0.549962in}}%
\pgfpathlineto{\pgfqpoint{1.456375in}{0.549206in}}%
\pgfpathlineto{\pgfqpoint{1.456843in}{0.549073in}}%
\pgfpathlineto{\pgfqpoint{1.472031in}{0.544264in}}%
\pgfpathlineto{\pgfqpoint{1.487199in}{0.535462in}}%
\pgfpathlineto{\pgfqpoint{1.487688in}{0.535097in}}%
\pgfpathlineto{\pgfqpoint{1.500541in}{0.521851in}}%
\pgfpathlineto{\pgfqpoint{1.503344in}{0.517250in}}%
\pgfpathlineto{\pgfqpoint{1.508015in}{0.508240in}}%
\pgfpathlineto{\pgfqpoint{1.510681in}{0.494629in}}%
\pgfpathlineto{\pgfqpoint{1.508904in}{0.481018in}}%
\pgfpathlineto{\pgfqpoint{1.503344in}{0.468841in}}%
\pgfpathlineto{\pgfqpoint{1.502577in}{0.467407in}}%
\pgfpathlineto{\pgfqpoint{1.490365in}{0.453795in}}%
\pgfpathlineto{\pgfqpoint{1.487688in}{0.451572in}}%
\pgfpathlineto{\pgfqpoint{1.472031in}{0.442187in}}%
\pgfpathlineto{\pgfqpoint{1.465977in}{0.440184in}}%
\pgfpathlineto{\pgfqpoint{1.456375in}{0.437398in}}%
\pgfpathlineto{\pgfqpoint{1.440718in}{0.436635in}}%
\pgfpathlineto{\pgfqpoint{1.425061in}{0.439688in}}%
\pgfpathlineto{\pgfqpoint{1.423897in}{0.440184in}}%
\pgfpathclose%
\pgfpathmoveto{\pgfqpoint{1.722536in}{0.421039in}}%
\pgfpathlineto{\pgfqpoint{1.738193in}{0.415621in}}%
\pgfpathlineto{\pgfqpoint{1.753849in}{0.413594in}}%
\pgfpathlineto{\pgfqpoint{1.769506in}{0.414945in}}%
\pgfpathlineto{\pgfqpoint{1.785162in}{0.419683in}}%
\pgfpathlineto{\pgfqpoint{1.798514in}{0.426573in}}%
\pgfpathlineto{\pgfqpoint{1.800819in}{0.427818in}}%
\pgfpathlineto{\pgfqpoint{1.816476in}{0.439302in}}%
\pgfpathlineto{\pgfqpoint{1.817490in}{0.440184in}}%
\pgfpathlineto{\pgfqpoint{1.830700in}{0.453795in}}%
\pgfpathlineto{\pgfqpoint{1.832132in}{0.455799in}}%
\pgfpathlineto{\pgfqpoint{1.840058in}{0.467407in}}%
\pgfpathlineto{\pgfqpoint{1.845507in}{0.481018in}}%
\pgfpathlineto{\pgfqpoint{1.847062in}{0.494629in}}%
\pgfpathlineto{\pgfqpoint{1.844730in}{0.508240in}}%
\pgfpathlineto{\pgfqpoint{1.838499in}{0.521851in}}%
\pgfpathlineto{\pgfqpoint{1.832132in}{0.530550in}}%
\pgfpathlineto{\pgfqpoint{1.828358in}{0.535462in}}%
\pgfpathlineto{\pgfqpoint{1.816476in}{0.547159in}}%
\pgfpathlineto{\pgfqpoint{1.814175in}{0.549073in}}%
\pgfpathlineto{\pgfqpoint{1.800819in}{0.558702in}}%
\pgfpathlineto{\pgfqpoint{1.793274in}{0.562684in}}%
\pgfpathlineto{\pgfqpoint{1.785162in}{0.566859in}}%
\pgfpathlineto{\pgfqpoint{1.769506in}{0.571652in}}%
\pgfpathlineto{\pgfqpoint{1.753849in}{0.573019in}}%
\pgfpathlineto{\pgfqpoint{1.738193in}{0.570968in}}%
\pgfpathlineto{\pgfqpoint{1.722536in}{0.565488in}}%
\pgfpathlineto{\pgfqpoint{1.717453in}{0.562684in}}%
\pgfpathlineto{\pgfqpoint{1.706880in}{0.556684in}}%
\pgfpathlineto{\pgfqpoint{1.696751in}{0.549073in}}%
\pgfpathlineto{\pgfqpoint{1.691223in}{0.544267in}}%
\pgfpathlineto{\pgfqpoint{1.682469in}{0.535462in}}%
\pgfpathlineto{\pgfqpoint{1.675567in}{0.526270in}}%
\pgfpathlineto{\pgfqpoint{1.672341in}{0.521851in}}%
\pgfpathlineto{\pgfqpoint{1.666038in}{0.508240in}}%
\pgfpathlineto{\pgfqpoint{1.663679in}{0.494629in}}%
\pgfpathlineto{\pgfqpoint{1.665251in}{0.481018in}}%
\pgfpathlineto{\pgfqpoint{1.670764in}{0.467407in}}%
\pgfpathlineto{\pgfqpoint{1.675567in}{0.460354in}}%
\pgfpathlineto{\pgfqpoint{1.680147in}{0.453795in}}%
\pgfpathlineto{\pgfqpoint{1.691223in}{0.442184in}}%
\pgfpathlineto{\pgfqpoint{1.693425in}{0.440184in}}%
\pgfpathlineto{\pgfqpoint{1.706880in}{0.429854in}}%
\pgfpathlineto{\pgfqpoint{1.712530in}{0.426573in}}%
\pgfpathlineto{\pgfqpoint{1.722536in}{0.421039in}}%
\pgfpathclose%
\pgfpathmoveto{\pgfqpoint{1.734494in}{0.440184in}}%
\pgfpathlineto{\pgfqpoint{1.722536in}{0.445429in}}%
\pgfpathlineto{\pgfqpoint{1.710591in}{0.453795in}}%
\pgfpathlineto{\pgfqpoint{1.706880in}{0.457264in}}%
\pgfpathlineto{\pgfqpoint{1.698466in}{0.467407in}}%
\pgfpathlineto{\pgfqpoint{1.691815in}{0.481018in}}%
\pgfpathlineto{\pgfqpoint{1.691223in}{0.485244in}}%
\pgfpathlineto{\pgfqpoint{1.690027in}{0.494629in}}%
\pgfpathlineto{\pgfqpoint{1.691223in}{0.500907in}}%
\pgfpathlineto{\pgfqpoint{1.692764in}{0.508240in}}%
\pgfpathlineto{\pgfqpoint{1.700369in}{0.521851in}}%
\pgfpathlineto{\pgfqpoint{1.706880in}{0.529173in}}%
\pgfpathlineto{\pgfqpoint{1.714113in}{0.535462in}}%
\pgfpathlineto{\pgfqpoint{1.722536in}{0.541122in}}%
\pgfpathlineto{\pgfqpoint{1.738193in}{0.547733in}}%
\pgfpathlineto{\pgfqpoint{1.746628in}{0.549073in}}%
\pgfpathlineto{\pgfqpoint{1.753849in}{0.550113in}}%
\pgfpathlineto{\pgfqpoint{1.764644in}{0.549073in}}%
\pgfpathlineto{\pgfqpoint{1.769506in}{0.548559in}}%
\pgfpathlineto{\pgfqpoint{1.785162in}{0.542776in}}%
\pgfpathlineto{\pgfqpoint{1.796829in}{0.535462in}}%
\pgfpathlineto{\pgfqpoint{1.800819in}{0.532236in}}%
\pgfpathlineto{\pgfqpoint{1.810443in}{0.521851in}}%
\pgfpathlineto{\pgfqpoint{1.816476in}{0.511455in}}%
\pgfpathlineto{\pgfqpoint{1.818101in}{0.508240in}}%
\pgfpathlineto{\pgfqpoint{1.820733in}{0.494629in}}%
\pgfpathlineto{\pgfqpoint{1.818979in}{0.481018in}}%
\pgfpathlineto{\pgfqpoint{1.816476in}{0.475396in}}%
\pgfpathlineto{\pgfqpoint{1.812406in}{0.467407in}}%
\pgfpathlineto{\pgfqpoint{1.800819in}{0.454005in}}%
\pgfpathlineto{\pgfqpoint{1.800578in}{0.453795in}}%
\pgfpathlineto{\pgfqpoint{1.785162in}{0.443722in}}%
\pgfpathlineto{\pgfqpoint{1.775973in}{0.440184in}}%
\pgfpathlineto{\pgfqpoint{1.769506in}{0.438008in}}%
\pgfpathlineto{\pgfqpoint{1.753849in}{0.436483in}}%
\pgfpathlineto{\pgfqpoint{1.738193in}{0.438771in}}%
\pgfpathlineto{\pgfqpoint{1.734494in}{0.440184in}}%
\pgfpathclose%
\pgfpathmoveto{\pgfqpoint{0.501324in}{0.684457in}}%
\pgfpathlineto{\pgfqpoint{0.516981in}{0.683005in}}%
\pgfpathlineto{\pgfqpoint{0.532637in}{0.685184in}}%
\pgfpathlineto{\pgfqpoint{0.532638in}{0.685184in}}%
\pgfpathlineto{\pgfqpoint{0.548294in}{0.690561in}}%
\pgfpathlineto{\pgfqpoint{0.563083in}{0.698795in}}%
\pgfpathlineto{\pgfqpoint{0.563950in}{0.699313in}}%
\pgfpathlineto{\pgfqpoint{0.579607in}{0.711594in}}%
\pgfpathlineto{\pgfqpoint{0.580499in}{0.712407in}}%
\pgfpathlineto{\pgfqpoint{0.592852in}{0.726018in}}%
\pgfpathlineto{\pgfqpoint{0.595263in}{0.729760in}}%
\pgfpathlineto{\pgfqpoint{0.601485in}{0.739629in}}%
\pgfpathlineto{\pgfqpoint{0.606208in}{0.753240in}}%
\pgfpathlineto{\pgfqpoint{0.606994in}{0.766851in}}%
\pgfpathlineto{\pgfqpoint{0.603848in}{0.780462in}}%
\pgfpathlineto{\pgfqpoint{0.596754in}{0.794073in}}%
\pgfpathlineto{\pgfqpoint{0.595263in}{0.795991in}}%
\pgfpathlineto{\pgfqpoint{0.585889in}{0.807684in}}%
\pgfpathlineto{\pgfqpoint{0.579607in}{0.813773in}}%
\pgfpathlineto{\pgfqpoint{0.570606in}{0.821295in}}%
\pgfpathlineto{\pgfqpoint{0.563950in}{0.826208in}}%
\pgfpathlineto{\pgfqpoint{0.548328in}{0.834907in}}%
\pgfpathlineto{\pgfqpoint{0.548294in}{0.834925in}}%
\pgfpathlineto{\pgfqpoint{0.532637in}{0.840492in}}%
\pgfpathlineto{\pgfqpoint{0.516981in}{0.842576in}}%
\pgfpathlineto{\pgfqpoint{0.501324in}{0.841187in}}%
\pgfpathlineto{\pgfqpoint{0.485668in}{0.836318in}}%
\pgfpathlineto{\pgfqpoint{0.482929in}{0.834907in}}%
\pgfpathlineto{\pgfqpoint{0.470011in}{0.828217in}}%
\pgfpathlineto{\pgfqpoint{0.460237in}{0.821295in}}%
\pgfpathlineto{\pgfqpoint{0.454354in}{0.816589in}}%
\pgfpathlineto{\pgfqpoint{0.444967in}{0.807684in}}%
\pgfpathlineto{\pgfqpoint{0.438698in}{0.800043in}}%
\pgfpathlineto{\pgfqpoint{0.434047in}{0.794073in}}%
\pgfpathlineto{\pgfqpoint{0.427034in}{0.780462in}}%
\pgfpathlineto{\pgfqpoint{0.423924in}{0.766851in}}%
\pgfpathlineto{\pgfqpoint{0.424701in}{0.753240in}}%
\pgfpathlineto{\pgfqpoint{0.429370in}{0.739629in}}%
\pgfpathlineto{\pgfqpoint{0.437944in}{0.726018in}}%
\pgfpathlineto{\pgfqpoint{0.438698in}{0.725149in}}%
\pgfpathlineto{\pgfqpoint{0.450404in}{0.712407in}}%
\pgfpathlineto{\pgfqpoint{0.454354in}{0.708884in}}%
\pgfpathlineto{\pgfqpoint{0.467827in}{0.698795in}}%
\pgfpathlineto{\pgfqpoint{0.470011in}{0.697286in}}%
\pgfpathlineto{\pgfqpoint{0.485668in}{0.689216in}}%
\pgfpathlineto{\pgfqpoint{0.499040in}{0.685184in}}%
\pgfpathlineto{\pgfqpoint{0.501324in}{0.684457in}}%
\pgfpathclose%
\pgfpathmoveto{\pgfqpoint{0.487318in}{0.712407in}}%
\pgfpathlineto{\pgfqpoint{0.485668in}{0.713074in}}%
\pgfpathlineto{\pgfqpoint{0.470011in}{0.723690in}}%
\pgfpathlineto{\pgfqpoint{0.467454in}{0.726018in}}%
\pgfpathlineto{\pgfqpoint{0.456658in}{0.739629in}}%
\pgfpathlineto{\pgfqpoint{0.454354in}{0.744892in}}%
\pgfpathlineto{\pgfqpoint{0.451149in}{0.753240in}}%
\pgfpathlineto{\pgfqpoint{0.450272in}{0.766851in}}%
\pgfpathlineto{\pgfqpoint{0.453783in}{0.780462in}}%
\pgfpathlineto{\pgfqpoint{0.454354in}{0.781474in}}%
\pgfpathlineto{\pgfqpoint{0.462547in}{0.794073in}}%
\pgfpathlineto{\pgfqpoint{0.470011in}{0.801639in}}%
\pgfpathlineto{\pgfqpoint{0.477994in}{0.807684in}}%
\pgfpathlineto{\pgfqpoint{0.485668in}{0.812320in}}%
\pgfpathlineto{\pgfqpoint{0.501324in}{0.817953in}}%
\pgfpathlineto{\pgfqpoint{0.516981in}{0.819559in}}%
\pgfpathlineto{\pgfqpoint{0.532637in}{0.817149in}}%
\pgfpathlineto{\pgfqpoint{0.548294in}{0.810709in}}%
\pgfpathlineto{\pgfqpoint{0.552965in}{0.807684in}}%
\pgfpathlineto{\pgfqpoint{0.563950in}{0.798740in}}%
\pgfpathlineto{\pgfqpoint{0.568368in}{0.794073in}}%
\pgfpathlineto{\pgfqpoint{0.576926in}{0.780462in}}%
\pgfpathlineto{\pgfqpoint{0.579607in}{0.770892in}}%
\pgfpathlineto{\pgfqpoint{0.580629in}{0.766851in}}%
\pgfpathlineto{\pgfqpoint{0.579760in}{0.753240in}}%
\pgfpathlineto{\pgfqpoint{0.579607in}{0.752832in}}%
\pgfpathlineto{\pgfqpoint{0.574075in}{0.739629in}}%
\pgfpathlineto{\pgfqpoint{0.563950in}{0.726443in}}%
\pgfpathlineto{\pgfqpoint{0.563531in}{0.726018in}}%
\pgfpathlineto{\pgfqpoint{0.548294in}{0.714843in}}%
\pgfpathlineto{\pgfqpoint{0.543002in}{0.712407in}}%
\pgfpathlineto{\pgfqpoint{0.532637in}{0.708346in}}%
\pgfpathlineto{\pgfqpoint{0.516981in}{0.706028in}}%
\pgfpathlineto{\pgfqpoint{0.501324in}{0.707573in}}%
\pgfpathlineto{\pgfqpoint{0.487318in}{0.712407in}}%
\pgfpathclose%
\pgfpathmoveto{\pgfqpoint{0.814455in}{0.683876in}}%
\pgfpathlineto{\pgfqpoint{0.830112in}{0.683151in}}%
\pgfpathlineto{\pgfqpoint{0.841118in}{0.685184in}}%
\pgfpathlineto{\pgfqpoint{0.845769in}{0.685990in}}%
\pgfpathlineto{\pgfqpoint{0.861425in}{0.692041in}}%
\pgfpathlineto{\pgfqpoint{0.872820in}{0.698795in}}%
\pgfpathlineto{\pgfqpoint{0.877082in}{0.701510in}}%
\pgfpathlineto{\pgfqpoint{0.890418in}{0.712407in}}%
\pgfpathlineto{\pgfqpoint{0.892738in}{0.714708in}}%
\pgfpathlineto{\pgfqpoint{0.902859in}{0.726018in}}%
\pgfpathlineto{\pgfqpoint{0.908395in}{0.734773in}}%
\pgfpathlineto{\pgfqpoint{0.911460in}{0.739629in}}%
\pgfpathlineto{\pgfqpoint{0.916258in}{0.753240in}}%
\pgfpathlineto{\pgfqpoint{0.917057in}{0.766851in}}%
\pgfpathlineto{\pgfqpoint{0.913860in}{0.780462in}}%
\pgfpathlineto{\pgfqpoint{0.908395in}{0.790865in}}%
\pgfpathlineto{\pgfqpoint{0.906718in}{0.794073in}}%
\pgfpathlineto{\pgfqpoint{0.895929in}{0.807684in}}%
\pgfpathlineto{\pgfqpoint{0.892738in}{0.810832in}}%
\pgfpathlineto{\pgfqpoint{0.880702in}{0.821295in}}%
\pgfpathlineto{\pgfqpoint{0.877082in}{0.824069in}}%
\pgfpathlineto{\pgfqpoint{0.861425in}{0.833449in}}%
\pgfpathlineto{\pgfqpoint{0.857734in}{0.834907in}}%
\pgfpathlineto{\pgfqpoint{0.845769in}{0.839658in}}%
\pgfpathlineto{\pgfqpoint{0.830112in}{0.842437in}}%
\pgfpathlineto{\pgfqpoint{0.814455in}{0.841743in}}%
\pgfpathlineto{\pgfqpoint{0.798799in}{0.837571in}}%
\pgfpathlineto{\pgfqpoint{0.793214in}{0.834907in}}%
\pgfpathlineto{\pgfqpoint{0.783142in}{0.830094in}}%
\pgfpathlineto{\pgfqpoint{0.770133in}{0.821295in}}%
\pgfpathlineto{\pgfqpoint{0.767486in}{0.819278in}}%
\pgfpathlineto{\pgfqpoint{0.754951in}{0.807684in}}%
\pgfpathlineto{\pgfqpoint{0.751829in}{0.803980in}}%
\pgfpathlineto{\pgfqpoint{0.744060in}{0.794073in}}%
\pgfpathlineto{\pgfqpoint{0.737099in}{0.780462in}}%
\pgfpathlineto{\pgfqpoint{0.736173in}{0.776419in}}%
\pgfpathlineto{\pgfqpoint{0.733833in}{0.766851in}}%
\pgfpathlineto{\pgfqpoint{0.734668in}{0.753240in}}%
\pgfpathlineto{\pgfqpoint{0.736173in}{0.749096in}}%
\pgfpathlineto{\pgfqpoint{0.739418in}{0.739629in}}%
\pgfpathlineto{\pgfqpoint{0.747928in}{0.726018in}}%
\pgfpathlineto{\pgfqpoint{0.751829in}{0.721551in}}%
\pgfpathlineto{\pgfqpoint{0.760459in}{0.712407in}}%
\pgfpathlineto{\pgfqpoint{0.767486in}{0.706298in}}%
\pgfpathlineto{\pgfqpoint{0.778004in}{0.698795in}}%
\pgfpathlineto{\pgfqpoint{0.783142in}{0.695404in}}%
\pgfpathlineto{\pgfqpoint{0.798799in}{0.688006in}}%
\pgfpathlineto{\pgfqpoint{0.809689in}{0.685184in}}%
\pgfpathlineto{\pgfqpoint{0.814455in}{0.683876in}}%
\pgfpathclose%
\pgfpathmoveto{\pgfqpoint{0.797258in}{0.712407in}}%
\pgfpathlineto{\pgfqpoint{0.783142in}{0.721214in}}%
\pgfpathlineto{\pgfqpoint{0.777617in}{0.726018in}}%
\pgfpathlineto{\pgfqpoint{0.767486in}{0.738289in}}%
\pgfpathlineto{\pgfqpoint{0.766554in}{0.739629in}}%
\pgfpathlineto{\pgfqpoint{0.761215in}{0.753240in}}%
\pgfpathlineto{\pgfqpoint{0.760326in}{0.766851in}}%
\pgfpathlineto{\pgfqpoint{0.763883in}{0.780462in}}%
\pgfpathlineto{\pgfqpoint{0.767486in}{0.786687in}}%
\pgfpathlineto{\pgfqpoint{0.772528in}{0.794073in}}%
\pgfpathlineto{\pgfqpoint{0.783142in}{0.804348in}}%
\pgfpathlineto{\pgfqpoint{0.787917in}{0.807684in}}%
\pgfpathlineto{\pgfqpoint{0.798799in}{0.813770in}}%
\pgfpathlineto{\pgfqpoint{0.814455in}{0.818595in}}%
\pgfpathlineto{\pgfqpoint{0.830112in}{0.819398in}}%
\pgfpathlineto{\pgfqpoint{0.845769in}{0.816184in}}%
\pgfpathlineto{\pgfqpoint{0.861425in}{0.808936in}}%
\pgfpathlineto{\pgfqpoint{0.863241in}{0.807684in}}%
\pgfpathlineto{\pgfqpoint{0.877082in}{0.795652in}}%
\pgfpathlineto{\pgfqpoint{0.878522in}{0.794073in}}%
\pgfpathlineto{\pgfqpoint{0.886858in}{0.780462in}}%
\pgfpathlineto{\pgfqpoint{0.890556in}{0.766851in}}%
\pgfpathlineto{\pgfqpoint{0.889632in}{0.753240in}}%
\pgfpathlineto{\pgfqpoint{0.884081in}{0.739629in}}%
\pgfpathlineto{\pgfqpoint{0.877082in}{0.730168in}}%
\pgfpathlineto{\pgfqpoint{0.873244in}{0.726018in}}%
\pgfpathlineto{\pgfqpoint{0.861425in}{0.716790in}}%
\pgfpathlineto{\pgfqpoint{0.852929in}{0.712407in}}%
\pgfpathlineto{\pgfqpoint{0.845769in}{0.709274in}}%
\pgfpathlineto{\pgfqpoint{0.830112in}{0.706182in}}%
\pgfpathlineto{\pgfqpoint{0.814455in}{0.706955in}}%
\pgfpathlineto{\pgfqpoint{0.798799in}{0.711596in}}%
\pgfpathlineto{\pgfqpoint{0.797258in}{0.712407in}}%
\pgfpathclose%
\pgfpathmoveto{\pgfqpoint{1.127587in}{0.683441in}}%
\pgfpathlineto{\pgfqpoint{1.143243in}{0.683441in}}%
\pgfpathlineto{\pgfqpoint{1.150827in}{0.685184in}}%
\pgfpathlineto{\pgfqpoint{1.158900in}{0.686930in}}%
\pgfpathlineto{\pgfqpoint{1.174556in}{0.693656in}}%
\pgfpathlineto{\pgfqpoint{1.182751in}{0.698795in}}%
\pgfpathlineto{\pgfqpoint{1.190213in}{0.703839in}}%
\pgfpathlineto{\pgfqpoint{1.200358in}{0.712407in}}%
\pgfpathlineto{\pgfqpoint{1.205870in}{0.718068in}}%
\pgfpathlineto{\pgfqpoint{1.212884in}{0.726018in}}%
\pgfpathlineto{\pgfqpoint{1.221366in}{0.739629in}}%
\pgfpathlineto{\pgfqpoint{1.221526in}{0.740090in}}%
\pgfpathlineto{\pgfqpoint{1.226252in}{0.753240in}}%
\pgfpathlineto{\pgfqpoint{1.227067in}{0.766851in}}%
\pgfpathlineto{\pgfqpoint{1.223806in}{0.780462in}}%
\pgfpathlineto{\pgfqpoint{1.221526in}{0.784777in}}%
\pgfpathlineto{\pgfqpoint{1.216740in}{0.794073in}}%
\pgfpathlineto{\pgfqpoint{1.205958in}{0.807684in}}%
\pgfpathlineto{\pgfqpoint{1.205870in}{0.807773in}}%
\pgfpathlineto{\pgfqpoint{1.190851in}{0.821295in}}%
\pgfpathlineto{\pgfqpoint{1.190213in}{0.821801in}}%
\pgfpathlineto{\pgfqpoint{1.174556in}{0.831838in}}%
\pgfpathlineto{\pgfqpoint{1.167533in}{0.834907in}}%
\pgfpathlineto{\pgfqpoint{1.158900in}{0.838684in}}%
\pgfpathlineto{\pgfqpoint{1.143243in}{0.842159in}}%
\pgfpathlineto{\pgfqpoint{1.127587in}{0.842159in}}%
\pgfpathlineto{\pgfqpoint{1.111930in}{0.838684in}}%
\pgfpathlineto{\pgfqpoint{1.103297in}{0.834907in}}%
\pgfpathlineto{\pgfqpoint{1.096274in}{0.831838in}}%
\pgfpathlineto{\pgfqpoint{1.080617in}{0.821801in}}%
\pgfpathlineto{\pgfqpoint{1.079979in}{0.821295in}}%
\pgfpathlineto{\pgfqpoint{1.064960in}{0.807773in}}%
\pgfpathlineto{\pgfqpoint{1.064872in}{0.807684in}}%
\pgfpathlineto{\pgfqpoint{1.054090in}{0.794073in}}%
\pgfpathlineto{\pgfqpoint{1.049304in}{0.784777in}}%
\pgfpathlineto{\pgfqpoint{1.047024in}{0.780462in}}%
\pgfpathlineto{\pgfqpoint{1.043763in}{0.766851in}}%
\pgfpathlineto{\pgfqpoint{1.044578in}{0.753240in}}%
\pgfpathlineto{\pgfqpoint{1.049304in}{0.740090in}}%
\pgfpathlineto{\pgfqpoint{1.049464in}{0.739629in}}%
\pgfpathlineto{\pgfqpoint{1.057946in}{0.726018in}}%
\pgfpathlineto{\pgfqpoint{1.064960in}{0.718068in}}%
\pgfpathlineto{\pgfqpoint{1.070472in}{0.712407in}}%
\pgfpathlineto{\pgfqpoint{1.080617in}{0.703839in}}%
\pgfpathlineto{\pgfqpoint{1.088079in}{0.698795in}}%
\pgfpathlineto{\pgfqpoint{1.096274in}{0.693656in}}%
\pgfpathlineto{\pgfqpoint{1.111930in}{0.686930in}}%
\pgfpathlineto{\pgfqpoint{1.120003in}{0.685184in}}%
\pgfpathlineto{\pgfqpoint{1.127587in}{0.683441in}}%
\pgfpathclose%
\pgfpathmoveto{\pgfqpoint{1.107680in}{0.712407in}}%
\pgfpathlineto{\pgfqpoint{1.096274in}{0.718914in}}%
\pgfpathlineto{\pgfqpoint{1.087675in}{0.726018in}}%
\pgfpathlineto{\pgfqpoint{1.080617in}{0.734118in}}%
\pgfpathlineto{\pgfqpoint{1.076672in}{0.739629in}}%
\pgfpathlineto{\pgfqpoint{1.071240in}{0.753240in}}%
\pgfpathlineto{\pgfqpoint{1.070336in}{0.766851in}}%
\pgfpathlineto{\pgfqpoint{1.073955in}{0.780462in}}%
\pgfpathlineto{\pgfqpoint{1.080617in}{0.791643in}}%
\pgfpathlineto{\pgfqpoint{1.082368in}{0.794073in}}%
\pgfpathlineto{\pgfqpoint{1.096274in}{0.806865in}}%
\pgfpathlineto{\pgfqpoint{1.097556in}{0.807684in}}%
\pgfpathlineto{\pgfqpoint{1.111930in}{0.815057in}}%
\pgfpathlineto{\pgfqpoint{1.127587in}{0.819077in}}%
\pgfpathlineto{\pgfqpoint{1.143243in}{0.819077in}}%
\pgfpathlineto{\pgfqpoint{1.158900in}{0.815057in}}%
\pgfpathlineto{\pgfqpoint{1.173274in}{0.807684in}}%
\pgfpathlineto{\pgfqpoint{1.174556in}{0.806865in}}%
\pgfpathlineto{\pgfqpoint{1.188462in}{0.794073in}}%
\pgfpathlineto{\pgfqpoint{1.190213in}{0.791643in}}%
\pgfpathlineto{\pgfqpoint{1.196875in}{0.780462in}}%
\pgfpathlineto{\pgfqpoint{1.200494in}{0.766851in}}%
\pgfpathlineto{\pgfqpoint{1.199590in}{0.753240in}}%
\pgfpathlineto{\pgfqpoint{1.194158in}{0.739629in}}%
\pgfpathlineto{\pgfqpoint{1.190213in}{0.734118in}}%
\pgfpathlineto{\pgfqpoint{1.183155in}{0.726018in}}%
\pgfpathlineto{\pgfqpoint{1.174556in}{0.718914in}}%
\pgfpathlineto{\pgfqpoint{1.163150in}{0.712407in}}%
\pgfpathlineto{\pgfqpoint{1.158900in}{0.710358in}}%
\pgfpathlineto{\pgfqpoint{1.143243in}{0.706491in}}%
\pgfpathlineto{\pgfqpoint{1.127587in}{0.706491in}}%
\pgfpathlineto{\pgfqpoint{1.111930in}{0.710358in}}%
\pgfpathlineto{\pgfqpoint{1.107680in}{0.712407in}}%
\pgfpathclose%
\pgfpathmoveto{\pgfqpoint{1.440718in}{0.683151in}}%
\pgfpathlineto{\pgfqpoint{1.456375in}{0.683876in}}%
\pgfpathlineto{\pgfqpoint{1.461141in}{0.685184in}}%
\pgfpathlineto{\pgfqpoint{1.472031in}{0.688006in}}%
\pgfpathlineto{\pgfqpoint{1.487688in}{0.695404in}}%
\pgfpathlineto{\pgfqpoint{1.492826in}{0.698795in}}%
\pgfpathlineto{\pgfqpoint{1.503344in}{0.706298in}}%
\pgfpathlineto{\pgfqpoint{1.510371in}{0.712407in}}%
\pgfpathlineto{\pgfqpoint{1.519001in}{0.721551in}}%
\pgfpathlineto{\pgfqpoint{1.522902in}{0.726018in}}%
\pgfpathlineto{\pgfqpoint{1.531412in}{0.739629in}}%
\pgfpathlineto{\pgfqpoint{1.534657in}{0.749096in}}%
\pgfpathlineto{\pgfqpoint{1.536162in}{0.753240in}}%
\pgfpathlineto{\pgfqpoint{1.536997in}{0.766851in}}%
\pgfpathlineto{\pgfqpoint{1.534657in}{0.776419in}}%
\pgfpathlineto{\pgfqpoint{1.533731in}{0.780462in}}%
\pgfpathlineto{\pgfqpoint{1.526770in}{0.794073in}}%
\pgfpathlineto{\pgfqpoint{1.519001in}{0.803980in}}%
\pgfpathlineto{\pgfqpoint{1.515879in}{0.807684in}}%
\pgfpathlineto{\pgfqpoint{1.503344in}{0.819278in}}%
\pgfpathlineto{\pgfqpoint{1.500697in}{0.821295in}}%
\pgfpathlineto{\pgfqpoint{1.487688in}{0.830094in}}%
\pgfpathlineto{\pgfqpoint{1.477616in}{0.834907in}}%
\pgfpathlineto{\pgfqpoint{1.472031in}{0.837571in}}%
\pgfpathlineto{\pgfqpoint{1.456375in}{0.841743in}}%
\pgfpathlineto{\pgfqpoint{1.440718in}{0.842437in}}%
\pgfpathlineto{\pgfqpoint{1.425061in}{0.839658in}}%
\pgfpathlineto{\pgfqpoint{1.413096in}{0.834907in}}%
\pgfpathlineto{\pgfqpoint{1.409405in}{0.833449in}}%
\pgfpathlineto{\pgfqpoint{1.393748in}{0.824069in}}%
\pgfpathlineto{\pgfqpoint{1.390128in}{0.821295in}}%
\pgfpathlineto{\pgfqpoint{1.378092in}{0.810832in}}%
\pgfpathlineto{\pgfqpoint{1.374901in}{0.807684in}}%
\pgfpathlineto{\pgfqpoint{1.364112in}{0.794073in}}%
\pgfpathlineto{\pgfqpoint{1.362435in}{0.790865in}}%
\pgfpathlineto{\pgfqpoint{1.356970in}{0.780462in}}%
\pgfpathlineto{\pgfqpoint{1.353773in}{0.766851in}}%
\pgfpathlineto{\pgfqpoint{1.354572in}{0.753240in}}%
\pgfpathlineto{\pgfqpoint{1.359370in}{0.739629in}}%
\pgfpathlineto{\pgfqpoint{1.362435in}{0.734773in}}%
\pgfpathlineto{\pgfqpoint{1.367971in}{0.726018in}}%
\pgfpathlineto{\pgfqpoint{1.378092in}{0.714708in}}%
\pgfpathlineto{\pgfqpoint{1.380412in}{0.712407in}}%
\pgfpathlineto{\pgfqpoint{1.393748in}{0.701510in}}%
\pgfpathlineto{\pgfqpoint{1.398010in}{0.698795in}}%
\pgfpathlineto{\pgfqpoint{1.409405in}{0.692041in}}%
\pgfpathlineto{\pgfqpoint{1.425061in}{0.685990in}}%
\pgfpathlineto{\pgfqpoint{1.429712in}{0.685184in}}%
\pgfpathlineto{\pgfqpoint{1.440718in}{0.683151in}}%
\pgfpathclose%
\pgfpathmoveto{\pgfqpoint{1.417901in}{0.712407in}}%
\pgfpathlineto{\pgfqpoint{1.409405in}{0.716790in}}%
\pgfpathlineto{\pgfqpoint{1.397586in}{0.726018in}}%
\pgfpathlineto{\pgfqpoint{1.393748in}{0.730168in}}%
\pgfpathlineto{\pgfqpoint{1.386749in}{0.739629in}}%
\pgfpathlineto{\pgfqpoint{1.381198in}{0.753240in}}%
\pgfpathlineto{\pgfqpoint{1.380274in}{0.766851in}}%
\pgfpathlineto{\pgfqpoint{1.383972in}{0.780462in}}%
\pgfpathlineto{\pgfqpoint{1.392308in}{0.794073in}}%
\pgfpathlineto{\pgfqpoint{1.393748in}{0.795652in}}%
\pgfpathlineto{\pgfqpoint{1.407589in}{0.807684in}}%
\pgfpathlineto{\pgfqpoint{1.409405in}{0.808936in}}%
\pgfpathlineto{\pgfqpoint{1.425061in}{0.816184in}}%
\pgfpathlineto{\pgfqpoint{1.440718in}{0.819398in}}%
\pgfpathlineto{\pgfqpoint{1.456375in}{0.818595in}}%
\pgfpathlineto{\pgfqpoint{1.472031in}{0.813770in}}%
\pgfpathlineto{\pgfqpoint{1.482913in}{0.807684in}}%
\pgfpathlineto{\pgfqpoint{1.487688in}{0.804348in}}%
\pgfpathlineto{\pgfqpoint{1.498302in}{0.794073in}}%
\pgfpathlineto{\pgfqpoint{1.503344in}{0.786687in}}%
\pgfpathlineto{\pgfqpoint{1.506947in}{0.780462in}}%
\pgfpathlineto{\pgfqpoint{1.510504in}{0.766851in}}%
\pgfpathlineto{\pgfqpoint{1.509615in}{0.753240in}}%
\pgfpathlineto{\pgfqpoint{1.504276in}{0.739629in}}%
\pgfpathlineto{\pgfqpoint{1.503344in}{0.738289in}}%
\pgfpathlineto{\pgfqpoint{1.493213in}{0.726018in}}%
\pgfpathlineto{\pgfqpoint{1.487688in}{0.721214in}}%
\pgfpathlineto{\pgfqpoint{1.473572in}{0.712407in}}%
\pgfpathlineto{\pgfqpoint{1.472031in}{0.711596in}}%
\pgfpathlineto{\pgfqpoint{1.456375in}{0.706955in}}%
\pgfpathlineto{\pgfqpoint{1.440718in}{0.706182in}}%
\pgfpathlineto{\pgfqpoint{1.425061in}{0.709274in}}%
\pgfpathlineto{\pgfqpoint{1.417901in}{0.712407in}}%
\pgfpathclose%
\pgfpathmoveto{\pgfqpoint{1.738193in}{0.685184in}}%
\pgfpathlineto{\pgfqpoint{1.753849in}{0.683005in}}%
\pgfpathlineto{\pgfqpoint{1.769506in}{0.684457in}}%
\pgfpathlineto{\pgfqpoint{1.771790in}{0.685184in}}%
\pgfpathlineto{\pgfqpoint{1.785162in}{0.689216in}}%
\pgfpathlineto{\pgfqpoint{1.800819in}{0.697286in}}%
\pgfpathlineto{\pgfqpoint{1.803003in}{0.698795in}}%
\pgfpathlineto{\pgfqpoint{1.816476in}{0.708884in}}%
\pgfpathlineto{\pgfqpoint{1.820426in}{0.712407in}}%
\pgfpathlineto{\pgfqpoint{1.832132in}{0.725149in}}%
\pgfpathlineto{\pgfqpoint{1.832886in}{0.726018in}}%
\pgfpathlineto{\pgfqpoint{1.841460in}{0.739629in}}%
\pgfpathlineto{\pgfqpoint{1.846129in}{0.753240in}}%
\pgfpathlineto{\pgfqpoint{1.846906in}{0.766851in}}%
\pgfpathlineto{\pgfqpoint{1.843796in}{0.780462in}}%
\pgfpathlineto{\pgfqpoint{1.836783in}{0.794073in}}%
\pgfpathlineto{\pgfqpoint{1.832132in}{0.800043in}}%
\pgfpathlineto{\pgfqpoint{1.825863in}{0.807684in}}%
\pgfpathlineto{\pgfqpoint{1.816476in}{0.816589in}}%
\pgfpathlineto{\pgfqpoint{1.810593in}{0.821295in}}%
\pgfpathlineto{\pgfqpoint{1.800819in}{0.828217in}}%
\pgfpathlineto{\pgfqpoint{1.787901in}{0.834907in}}%
\pgfpathlineto{\pgfqpoint{1.785162in}{0.836318in}}%
\pgfpathlineto{\pgfqpoint{1.769506in}{0.841187in}}%
\pgfpathlineto{\pgfqpoint{1.753849in}{0.842576in}}%
\pgfpathlineto{\pgfqpoint{1.738193in}{0.840492in}}%
\pgfpathlineto{\pgfqpoint{1.722536in}{0.834925in}}%
\pgfpathlineto{\pgfqpoint{1.722502in}{0.834907in}}%
\pgfpathlineto{\pgfqpoint{1.706880in}{0.826208in}}%
\pgfpathlineto{\pgfqpoint{1.700224in}{0.821295in}}%
\pgfpathlineto{\pgfqpoint{1.691223in}{0.813773in}}%
\pgfpathlineto{\pgfqpoint{1.684941in}{0.807684in}}%
\pgfpathlineto{\pgfqpoint{1.675567in}{0.795991in}}%
\pgfpathlineto{\pgfqpoint{1.674076in}{0.794073in}}%
\pgfpathlineto{\pgfqpoint{1.666982in}{0.780462in}}%
\pgfpathlineto{\pgfqpoint{1.663836in}{0.766851in}}%
\pgfpathlineto{\pgfqpoint{1.664622in}{0.753240in}}%
\pgfpathlineto{\pgfqpoint{1.669345in}{0.739629in}}%
\pgfpathlineto{\pgfqpoint{1.675567in}{0.729760in}}%
\pgfpathlineto{\pgfqpoint{1.677978in}{0.726018in}}%
\pgfpathlineto{\pgfqpoint{1.690331in}{0.712407in}}%
\pgfpathlineto{\pgfqpoint{1.691223in}{0.711594in}}%
\pgfpathlineto{\pgfqpoint{1.706880in}{0.699313in}}%
\pgfpathlineto{\pgfqpoint{1.707747in}{0.698795in}}%
\pgfpathlineto{\pgfqpoint{1.722536in}{0.690561in}}%
\pgfpathlineto{\pgfqpoint{1.738192in}{0.685184in}}%
\pgfpathlineto{\pgfqpoint{1.738193in}{0.685184in}}%
\pgfpathclose%
\pgfpathmoveto{\pgfqpoint{1.727828in}{0.712407in}}%
\pgfpathlineto{\pgfqpoint{1.722536in}{0.714843in}}%
\pgfpathlineto{\pgfqpoint{1.707299in}{0.726018in}}%
\pgfpathlineto{\pgfqpoint{1.706880in}{0.726443in}}%
\pgfpathlineto{\pgfqpoint{1.696755in}{0.739629in}}%
\pgfpathlineto{\pgfqpoint{1.691223in}{0.752832in}}%
\pgfpathlineto{\pgfqpoint{1.691070in}{0.753240in}}%
\pgfpathlineto{\pgfqpoint{1.690201in}{0.766851in}}%
\pgfpathlineto{\pgfqpoint{1.691223in}{0.770892in}}%
\pgfpathlineto{\pgfqpoint{1.693904in}{0.780462in}}%
\pgfpathlineto{\pgfqpoint{1.702462in}{0.794073in}}%
\pgfpathlineto{\pgfqpoint{1.706880in}{0.798740in}}%
\pgfpathlineto{\pgfqpoint{1.717865in}{0.807684in}}%
\pgfpathlineto{\pgfqpoint{1.722536in}{0.810709in}}%
\pgfpathlineto{\pgfqpoint{1.738193in}{0.817149in}}%
\pgfpathlineto{\pgfqpoint{1.753849in}{0.819559in}}%
\pgfpathlineto{\pgfqpoint{1.769506in}{0.817953in}}%
\pgfpathlineto{\pgfqpoint{1.785162in}{0.812320in}}%
\pgfpathlineto{\pgfqpoint{1.792836in}{0.807684in}}%
\pgfpathlineto{\pgfqpoint{1.800819in}{0.801639in}}%
\pgfpathlineto{\pgfqpoint{1.808283in}{0.794073in}}%
\pgfpathlineto{\pgfqpoint{1.816476in}{0.781474in}}%
\pgfpathlineto{\pgfqpoint{1.817047in}{0.780462in}}%
\pgfpathlineto{\pgfqpoint{1.820558in}{0.766851in}}%
\pgfpathlineto{\pgfqpoint{1.819681in}{0.753240in}}%
\pgfpathlineto{\pgfqpoint{1.816476in}{0.744892in}}%
\pgfpathlineto{\pgfqpoint{1.814172in}{0.739629in}}%
\pgfpathlineto{\pgfqpoint{1.803376in}{0.726018in}}%
\pgfpathlineto{\pgfqpoint{1.800819in}{0.723690in}}%
\pgfpathlineto{\pgfqpoint{1.785162in}{0.713074in}}%
\pgfpathlineto{\pgfqpoint{1.783512in}{0.712407in}}%
\pgfpathlineto{\pgfqpoint{1.769506in}{0.707573in}}%
\pgfpathlineto{\pgfqpoint{1.753849in}{0.706028in}}%
\pgfpathlineto{\pgfqpoint{1.738193in}{0.708346in}}%
\pgfpathlineto{\pgfqpoint{1.727828in}{0.712407in}}%
\pgfpathclose%
\pgfpathmoveto{\pgfqpoint{0.501324in}{0.953864in}}%
\pgfpathlineto{\pgfqpoint{0.516981in}{0.952448in}}%
\pgfpathlineto{\pgfqpoint{0.532637in}{0.954573in}}%
\pgfpathlineto{\pgfqpoint{0.540560in}{0.957407in}}%
\pgfpathlineto{\pgfqpoint{0.548294in}{0.960092in}}%
\pgfpathlineto{\pgfqpoint{0.563950in}{0.968803in}}%
\pgfpathlineto{\pgfqpoint{0.566994in}{0.971018in}}%
\pgfpathlineto{\pgfqpoint{0.579607in}{0.981196in}}%
\pgfpathlineto{\pgfqpoint{0.583267in}{0.984629in}}%
\pgfpathlineto{\pgfqpoint{0.594868in}{0.998240in}}%
\pgfpathlineto{\pgfqpoint{0.595263in}{0.998912in}}%
\pgfpathlineto{\pgfqpoint{0.602745in}{1.011851in}}%
\pgfpathlineto{\pgfqpoint{0.606680in}{1.025462in}}%
\pgfpathlineto{\pgfqpoint{0.606680in}{1.039073in}}%
\pgfpathlineto{\pgfqpoint{0.602745in}{1.052684in}}%
\pgfpathlineto{\pgfqpoint{0.595263in}{1.065624in}}%
\pgfpathlineto{\pgfqpoint{0.594868in}{1.066295in}}%
\pgfpathlineto{\pgfqpoint{0.583267in}{1.079907in}}%
\pgfpathlineto{\pgfqpoint{0.579607in}{1.083339in}}%
\pgfpathlineto{\pgfqpoint{0.566994in}{1.093518in}}%
\pgfpathlineto{\pgfqpoint{0.563950in}{1.095733in}}%
\pgfpathlineto{\pgfqpoint{0.548294in}{1.104443in}}%
\pgfpathlineto{\pgfqpoint{0.540560in}{1.107129in}}%
\pgfpathlineto{\pgfqpoint{0.532637in}{1.109962in}}%
\pgfpathlineto{\pgfqpoint{0.516981in}{1.112088in}}%
\pgfpathlineto{\pgfqpoint{0.501324in}{1.110671in}}%
\pgfpathlineto{\pgfqpoint{0.490080in}{1.107129in}}%
\pgfpathlineto{\pgfqpoint{0.485668in}{1.105784in}}%
\pgfpathlineto{\pgfqpoint{0.470011in}{1.097740in}}%
\pgfpathlineto{\pgfqpoint{0.463964in}{1.093518in}}%
\pgfpathlineto{\pgfqpoint{0.454354in}{1.086096in}}%
\pgfpathlineto{\pgfqpoint{0.447611in}{1.079907in}}%
\pgfpathlineto{\pgfqpoint{0.438698in}{1.069668in}}%
\pgfpathlineto{\pgfqpoint{0.435918in}{1.066295in}}%
\pgfpathlineto{\pgfqpoint{0.428124in}{1.052684in}}%
\pgfpathlineto{\pgfqpoint{0.424234in}{1.039073in}}%
\pgfpathlineto{\pgfqpoint{0.424234in}{1.025462in}}%
\pgfpathlineto{\pgfqpoint{0.428124in}{1.011851in}}%
\pgfpathlineto{\pgfqpoint{0.435918in}{0.998240in}}%
\pgfpathlineto{\pgfqpoint{0.438698in}{0.994868in}}%
\pgfpathlineto{\pgfqpoint{0.447611in}{0.984629in}}%
\pgfpathlineto{\pgfqpoint{0.454354in}{0.978440in}}%
\pgfpathlineto{\pgfqpoint{0.463964in}{0.971018in}}%
\pgfpathlineto{\pgfqpoint{0.470011in}{0.966795in}}%
\pgfpathlineto{\pgfqpoint{0.485668in}{0.958752in}}%
\pgfpathlineto{\pgfqpoint{0.490080in}{0.957407in}}%
\pgfpathlineto{\pgfqpoint{0.501324in}{0.953864in}}%
\pgfpathclose%
\pgfpathmoveto{\pgfqpoint{0.482227in}{0.984629in}}%
\pgfpathlineto{\pgfqpoint{0.470011in}{0.993346in}}%
\pgfpathlineto{\pgfqpoint{0.464902in}{0.998240in}}%
\pgfpathlineto{\pgfqpoint{0.455089in}{1.011851in}}%
\pgfpathlineto{\pgfqpoint{0.454354in}{1.013862in}}%
\pgfpathlineto{\pgfqpoint{0.450623in}{1.025462in}}%
\pgfpathlineto{\pgfqpoint{0.450623in}{1.039073in}}%
\pgfpathlineto{\pgfqpoint{0.454354in}{1.050673in}}%
\pgfpathlineto{\pgfqpoint{0.455089in}{1.052684in}}%
\pgfpathlineto{\pgfqpoint{0.464902in}{1.066295in}}%
\pgfpathlineto{\pgfqpoint{0.470011in}{1.071189in}}%
\pgfpathlineto{\pgfqpoint{0.482227in}{1.079907in}}%
\pgfpathlineto{\pgfqpoint{0.485668in}{1.081918in}}%
\pgfpathlineto{\pgfqpoint{0.501324in}{1.087429in}}%
\pgfpathlineto{\pgfqpoint{0.516981in}{1.089001in}}%
\pgfpathlineto{\pgfqpoint{0.532637in}{1.086643in}}%
\pgfpathlineto{\pgfqpoint{0.548294in}{1.080341in}}%
\pgfpathlineto{\pgfqpoint{0.548987in}{1.079907in}}%
\pgfpathlineto{\pgfqpoint{0.563950in}{1.068426in}}%
\pgfpathlineto{\pgfqpoint{0.566084in}{1.066295in}}%
\pgfpathlineto{\pgfqpoint{0.575596in}{1.052684in}}%
\pgfpathlineto{\pgfqpoint{0.579607in}{1.041217in}}%
\pgfpathlineto{\pgfqpoint{0.580282in}{1.039073in}}%
\pgfpathlineto{\pgfqpoint{0.580282in}{1.025462in}}%
\pgfpathlineto{\pgfqpoint{0.579607in}{1.023319in}}%
\pgfpathlineto{\pgfqpoint{0.575596in}{1.011851in}}%
\pgfpathlineto{\pgfqpoint{0.566084in}{0.998240in}}%
\pgfpathlineto{\pgfqpoint{0.563950in}{0.996109in}}%
\pgfpathlineto{\pgfqpoint{0.548987in}{0.984629in}}%
\pgfpathlineto{\pgfqpoint{0.548294in}{0.984194in}}%
\pgfpathlineto{\pgfqpoint{0.532637in}{0.977892in}}%
\pgfpathlineto{\pgfqpoint{0.516981in}{0.975534in}}%
\pgfpathlineto{\pgfqpoint{0.501324in}{0.977106in}}%
\pgfpathlineto{\pgfqpoint{0.485668in}{0.982618in}}%
\pgfpathlineto{\pgfqpoint{0.482227in}{0.984629in}}%
\pgfpathclose%
\pgfpathmoveto{\pgfqpoint{0.814455in}{0.953298in}}%
\pgfpathlineto{\pgfqpoint{0.830112in}{0.952589in}}%
\pgfpathlineto{\pgfqpoint{0.845769in}{0.955424in}}%
\pgfpathlineto{\pgfqpoint{0.850732in}{0.957407in}}%
\pgfpathlineto{\pgfqpoint{0.861425in}{0.961567in}}%
\pgfpathlineto{\pgfqpoint{0.877082in}{0.970941in}}%
\pgfpathlineto{\pgfqpoint{0.877183in}{0.971018in}}%
\pgfpathlineto{\pgfqpoint{0.892738in}{0.984074in}}%
\pgfpathlineto{\pgfqpoint{0.893319in}{0.984629in}}%
\pgfpathlineto{\pgfqpoint{0.904865in}{0.998240in}}%
\pgfpathlineto{\pgfqpoint{0.908395in}{1.004346in}}%
\pgfpathlineto{\pgfqpoint{0.912740in}{1.011851in}}%
\pgfpathlineto{\pgfqpoint{0.916737in}{1.025462in}}%
\pgfpathlineto{\pgfqpoint{0.916737in}{1.039073in}}%
\pgfpathlineto{\pgfqpoint{0.912740in}{1.052684in}}%
\pgfpathlineto{\pgfqpoint{0.908395in}{1.060189in}}%
\pgfpathlineto{\pgfqpoint{0.904865in}{1.066295in}}%
\pgfpathlineto{\pgfqpoint{0.893319in}{1.079907in}}%
\pgfpathlineto{\pgfqpoint{0.892738in}{1.080461in}}%
\pgfpathlineto{\pgfqpoint{0.877183in}{1.093518in}}%
\pgfpathlineto{\pgfqpoint{0.877082in}{1.093594in}}%
\pgfpathlineto{\pgfqpoint{0.861425in}{1.102968in}}%
\pgfpathlineto{\pgfqpoint{0.850732in}{1.107129in}}%
\pgfpathlineto{\pgfqpoint{0.845769in}{1.109111in}}%
\pgfpathlineto{\pgfqpoint{0.830112in}{1.111946in}}%
\pgfpathlineto{\pgfqpoint{0.814455in}{1.111238in}}%
\pgfpathlineto{\pgfqpoint{0.799329in}{1.107129in}}%
\pgfpathlineto{\pgfqpoint{0.798799in}{1.106990in}}%
\pgfpathlineto{\pgfqpoint{0.783142in}{1.099616in}}%
\pgfpathlineto{\pgfqpoint{0.773998in}{1.093518in}}%
\pgfpathlineto{\pgfqpoint{0.767486in}{1.088727in}}%
\pgfpathlineto{\pgfqpoint{0.757631in}{1.079907in}}%
\pgfpathlineto{\pgfqpoint{0.751829in}{1.073420in}}%
\pgfpathlineto{\pgfqpoint{0.745917in}{1.066295in}}%
\pgfpathlineto{\pgfqpoint{0.738181in}{1.052684in}}%
\pgfpathlineto{\pgfqpoint{0.736173in}{1.045666in}}%
\pgfpathlineto{\pgfqpoint{0.734167in}{1.039073in}}%
\pgfpathlineto{\pgfqpoint{0.734167in}{1.025462in}}%
\pgfpathlineto{\pgfqpoint{0.736173in}{1.018869in}}%
\pgfpathlineto{\pgfqpoint{0.738181in}{1.011851in}}%
\pgfpathlineto{\pgfqpoint{0.745917in}{0.998240in}}%
\pgfpathlineto{\pgfqpoint{0.751829in}{0.991116in}}%
\pgfpathlineto{\pgfqpoint{0.757631in}{0.984629in}}%
\pgfpathlineto{\pgfqpoint{0.767486in}{0.975809in}}%
\pgfpathlineto{\pgfqpoint{0.773998in}{0.971018in}}%
\pgfpathlineto{\pgfqpoint{0.783142in}{0.964919in}}%
\pgfpathlineto{\pgfqpoint{0.798799in}{0.957545in}}%
\pgfpathlineto{\pgfqpoint{0.799329in}{0.957407in}}%
\pgfpathlineto{\pgfqpoint{0.814455in}{0.953298in}}%
\pgfpathclose%
\pgfpathmoveto{\pgfqpoint{0.792460in}{0.984629in}}%
\pgfpathlineto{\pgfqpoint{0.783142in}{0.990764in}}%
\pgfpathlineto{\pgfqpoint{0.774971in}{0.998240in}}%
\pgfpathlineto{\pgfqpoint{0.767486in}{1.008156in}}%
\pgfpathlineto{\pgfqpoint{0.765129in}{1.011851in}}%
\pgfpathlineto{\pgfqpoint{0.760682in}{1.025462in}}%
\pgfpathlineto{\pgfqpoint{0.760682in}{1.039073in}}%
\pgfpathlineto{\pgfqpoint{0.765129in}{1.052684in}}%
\pgfpathlineto{\pgfqpoint{0.767486in}{1.056379in}}%
\pgfpathlineto{\pgfqpoint{0.774971in}{1.066295in}}%
\pgfpathlineto{\pgfqpoint{0.783142in}{1.073771in}}%
\pgfpathlineto{\pgfqpoint{0.792460in}{1.079907in}}%
\pgfpathlineto{\pgfqpoint{0.798799in}{1.083336in}}%
\pgfpathlineto{\pgfqpoint{0.814455in}{1.088058in}}%
\pgfpathlineto{\pgfqpoint{0.830112in}{1.088844in}}%
\pgfpathlineto{\pgfqpoint{0.845769in}{1.085699in}}%
\pgfpathlineto{\pgfqpoint{0.858630in}{1.079907in}}%
\pgfpathlineto{\pgfqpoint{0.861425in}{1.078385in}}%
\pgfpathlineto{\pgfqpoint{0.876139in}{1.066295in}}%
\pgfpathlineto{\pgfqpoint{0.877082in}{1.065181in}}%
\pgfpathlineto{\pgfqpoint{0.885563in}{1.052684in}}%
\pgfpathlineto{\pgfqpoint{0.890187in}{1.039073in}}%
\pgfpathlineto{\pgfqpoint{0.890187in}{1.025462in}}%
\pgfpathlineto{\pgfqpoint{0.885563in}{1.011851in}}%
\pgfpathlineto{\pgfqpoint{0.877082in}{0.999355in}}%
\pgfpathlineto{\pgfqpoint{0.876139in}{0.998240in}}%
\pgfpathlineto{\pgfqpoint{0.861425in}{0.986151in}}%
\pgfpathlineto{\pgfqpoint{0.858630in}{0.984629in}}%
\pgfpathlineto{\pgfqpoint{0.845769in}{0.978837in}}%
\pgfpathlineto{\pgfqpoint{0.830112in}{0.975691in}}%
\pgfpathlineto{\pgfqpoint{0.814455in}{0.976477in}}%
\pgfpathlineto{\pgfqpoint{0.798799in}{0.981199in}}%
\pgfpathlineto{\pgfqpoint{0.792460in}{0.984629in}}%
\pgfpathclose%
\pgfpathmoveto{\pgfqpoint{1.111930in}{0.956418in}}%
\pgfpathlineto{\pgfqpoint{1.127587in}{0.952873in}}%
\pgfpathlineto{\pgfqpoint{1.143243in}{0.952873in}}%
\pgfpathlineto{\pgfqpoint{1.158900in}{0.956418in}}%
\pgfpathlineto{\pgfqpoint{1.161146in}{0.957407in}}%
\pgfpathlineto{\pgfqpoint{1.174556in}{0.963177in}}%
\pgfpathlineto{\pgfqpoint{1.186929in}{0.971018in}}%
\pgfpathlineto{\pgfqpoint{1.190213in}{0.973307in}}%
\pgfpathlineto{\pgfqpoint{1.203236in}{0.984629in}}%
\pgfpathlineto{\pgfqpoint{1.205870in}{0.987483in}}%
\pgfpathlineto{\pgfqpoint{1.214889in}{0.998240in}}%
\pgfpathlineto{\pgfqpoint{1.221526in}{1.009898in}}%
\pgfpathlineto{\pgfqpoint{1.222663in}{1.011851in}}%
\pgfpathlineto{\pgfqpoint{1.226741in}{1.025462in}}%
\pgfpathlineto{\pgfqpoint{1.226741in}{1.039073in}}%
\pgfpathlineto{\pgfqpoint{1.222663in}{1.052684in}}%
\pgfpathlineto{\pgfqpoint{1.221526in}{1.054637in}}%
\pgfpathlineto{\pgfqpoint{1.214889in}{1.066295in}}%
\pgfpathlineto{\pgfqpoint{1.205870in}{1.077052in}}%
\pgfpathlineto{\pgfqpoint{1.203236in}{1.079907in}}%
\pgfpathlineto{\pgfqpoint{1.190213in}{1.091228in}}%
\pgfpathlineto{\pgfqpoint{1.186929in}{1.093518in}}%
\pgfpathlineto{\pgfqpoint{1.174556in}{1.101359in}}%
\pgfpathlineto{\pgfqpoint{1.161146in}{1.107129in}}%
\pgfpathlineto{\pgfqpoint{1.158900in}{1.108117in}}%
\pgfpathlineto{\pgfqpoint{1.143243in}{1.111663in}}%
\pgfpathlineto{\pgfqpoint{1.127587in}{1.111663in}}%
\pgfpathlineto{\pgfqpoint{1.111930in}{1.108117in}}%
\pgfpathlineto{\pgfqpoint{1.109684in}{1.107129in}}%
\pgfpathlineto{\pgfqpoint{1.096274in}{1.101359in}}%
\pgfpathlineto{\pgfqpoint{1.083901in}{1.093518in}}%
\pgfpathlineto{\pgfqpoint{1.080617in}{1.091228in}}%
\pgfpathlineto{\pgfqpoint{1.067594in}{1.079907in}}%
\pgfpathlineto{\pgfqpoint{1.064960in}{1.077052in}}%
\pgfpathlineto{\pgfqpoint{1.055941in}{1.066295in}}%
\pgfpathlineto{\pgfqpoint{1.049304in}{1.054637in}}%
\pgfpathlineto{\pgfqpoint{1.048167in}{1.052684in}}%
\pgfpathlineto{\pgfqpoint{1.044089in}{1.039073in}}%
\pgfpathlineto{\pgfqpoint{1.044089in}{1.025462in}}%
\pgfpathlineto{\pgfqpoint{1.048167in}{1.011851in}}%
\pgfpathlineto{\pgfqpoint{1.049304in}{1.009898in}}%
\pgfpathlineto{\pgfqpoint{1.055941in}{0.998240in}}%
\pgfpathlineto{\pgfqpoint{1.064960in}{0.987483in}}%
\pgfpathlineto{\pgfqpoint{1.067594in}{0.984629in}}%
\pgfpathlineto{\pgfqpoint{1.080617in}{0.973307in}}%
\pgfpathlineto{\pgfqpoint{1.083901in}{0.971018in}}%
\pgfpathlineto{\pgfqpoint{1.096274in}{0.963177in}}%
\pgfpathlineto{\pgfqpoint{1.109684in}{0.957407in}}%
\pgfpathlineto{\pgfqpoint{1.111930in}{0.956418in}}%
\pgfpathclose%
\pgfpathmoveto{\pgfqpoint{1.102481in}{0.984629in}}%
\pgfpathlineto{\pgfqpoint{1.096274in}{0.988365in}}%
\pgfpathlineto{\pgfqpoint{1.084915in}{0.998240in}}%
\pgfpathlineto{\pgfqpoint{1.080617in}{1.003636in}}%
\pgfpathlineto{\pgfqpoint{1.075222in}{1.011851in}}%
\pgfpathlineto{\pgfqpoint{1.070698in}{1.025462in}}%
\pgfpathlineto{\pgfqpoint{1.070698in}{1.039073in}}%
\pgfpathlineto{\pgfqpoint{1.075222in}{1.052684in}}%
\pgfpathlineto{\pgfqpoint{1.080617in}{1.060899in}}%
\pgfpathlineto{\pgfqpoint{1.084915in}{1.066295in}}%
\pgfpathlineto{\pgfqpoint{1.096274in}{1.076170in}}%
\pgfpathlineto{\pgfqpoint{1.102481in}{1.079907in}}%
\pgfpathlineto{\pgfqpoint{1.111930in}{1.084596in}}%
\pgfpathlineto{\pgfqpoint{1.127587in}{1.088530in}}%
\pgfpathlineto{\pgfqpoint{1.143243in}{1.088530in}}%
\pgfpathlineto{\pgfqpoint{1.158900in}{1.084596in}}%
\pgfpathlineto{\pgfqpoint{1.168349in}{1.079907in}}%
\pgfpathlineto{\pgfqpoint{1.174556in}{1.076170in}}%
\pgfpathlineto{\pgfqpoint{1.185915in}{1.066295in}}%
\pgfpathlineto{\pgfqpoint{1.190213in}{1.060899in}}%
\pgfpathlineto{\pgfqpoint{1.195608in}{1.052684in}}%
\pgfpathlineto{\pgfqpoint{1.200132in}{1.039073in}}%
\pgfpathlineto{\pgfqpoint{1.200132in}{1.025462in}}%
\pgfpathlineto{\pgfqpoint{1.195608in}{1.011851in}}%
\pgfpathlineto{\pgfqpoint{1.190213in}{1.003636in}}%
\pgfpathlineto{\pgfqpoint{1.185915in}{0.998240in}}%
\pgfpathlineto{\pgfqpoint{1.174556in}{0.988365in}}%
\pgfpathlineto{\pgfqpoint{1.168349in}{0.984629in}}%
\pgfpathlineto{\pgfqpoint{1.158900in}{0.979939in}}%
\pgfpathlineto{\pgfqpoint{1.143243in}{0.976005in}}%
\pgfpathlineto{\pgfqpoint{1.127587in}{0.976005in}}%
\pgfpathlineto{\pgfqpoint{1.111930in}{0.979939in}}%
\pgfpathlineto{\pgfqpoint{1.102481in}{0.984629in}}%
\pgfpathclose%
\pgfpathmoveto{\pgfqpoint{1.425061in}{0.955424in}}%
\pgfpathlineto{\pgfqpoint{1.440718in}{0.952589in}}%
\pgfpathlineto{\pgfqpoint{1.456375in}{0.953298in}}%
\pgfpathlineto{\pgfqpoint{1.471501in}{0.957407in}}%
\pgfpathlineto{\pgfqpoint{1.472031in}{0.957545in}}%
\pgfpathlineto{\pgfqpoint{1.487688in}{0.964919in}}%
\pgfpathlineto{\pgfqpoint{1.496832in}{0.971018in}}%
\pgfpathlineto{\pgfqpoint{1.503344in}{0.975809in}}%
\pgfpathlineto{\pgfqpoint{1.513199in}{0.984629in}}%
\pgfpathlineto{\pgfqpoint{1.519001in}{0.991116in}}%
\pgfpathlineto{\pgfqpoint{1.524913in}{0.998240in}}%
\pgfpathlineto{\pgfqpoint{1.532649in}{1.011851in}}%
\pgfpathlineto{\pgfqpoint{1.534657in}{1.018869in}}%
\pgfpathlineto{\pgfqpoint{1.536663in}{1.025462in}}%
\pgfpathlineto{\pgfqpoint{1.536663in}{1.039073in}}%
\pgfpathlineto{\pgfqpoint{1.534657in}{1.045666in}}%
\pgfpathlineto{\pgfqpoint{1.532649in}{1.052684in}}%
\pgfpathlineto{\pgfqpoint{1.524913in}{1.066295in}}%
\pgfpathlineto{\pgfqpoint{1.519001in}{1.073420in}}%
\pgfpathlineto{\pgfqpoint{1.513199in}{1.079907in}}%
\pgfpathlineto{\pgfqpoint{1.503344in}{1.088727in}}%
\pgfpathlineto{\pgfqpoint{1.496832in}{1.093518in}}%
\pgfpathlineto{\pgfqpoint{1.487688in}{1.099616in}}%
\pgfpathlineto{\pgfqpoint{1.472031in}{1.106990in}}%
\pgfpathlineto{\pgfqpoint{1.471501in}{1.107129in}}%
\pgfpathlineto{\pgfqpoint{1.456375in}{1.111238in}}%
\pgfpathlineto{\pgfqpoint{1.440718in}{1.111946in}}%
\pgfpathlineto{\pgfqpoint{1.425061in}{1.109111in}}%
\pgfpathlineto{\pgfqpoint{1.420098in}{1.107129in}}%
\pgfpathlineto{\pgfqpoint{1.409405in}{1.102968in}}%
\pgfpathlineto{\pgfqpoint{1.393748in}{1.093594in}}%
\pgfpathlineto{\pgfqpoint{1.393647in}{1.093518in}}%
\pgfpathlineto{\pgfqpoint{1.378092in}{1.080461in}}%
\pgfpathlineto{\pgfqpoint{1.377511in}{1.079907in}}%
\pgfpathlineto{\pgfqpoint{1.365965in}{1.066295in}}%
\pgfpathlineto{\pgfqpoint{1.362435in}{1.060189in}}%
\pgfpathlineto{\pgfqpoint{1.358090in}{1.052684in}}%
\pgfpathlineto{\pgfqpoint{1.354093in}{1.039073in}}%
\pgfpathlineto{\pgfqpoint{1.354093in}{1.025462in}}%
\pgfpathlineto{\pgfqpoint{1.358090in}{1.011851in}}%
\pgfpathlineto{\pgfqpoint{1.362435in}{1.004346in}}%
\pgfpathlineto{\pgfqpoint{1.365965in}{0.998240in}}%
\pgfpathlineto{\pgfqpoint{1.377511in}{0.984629in}}%
\pgfpathlineto{\pgfqpoint{1.378092in}{0.984074in}}%
\pgfpathlineto{\pgfqpoint{1.393647in}{0.971018in}}%
\pgfpathlineto{\pgfqpoint{1.393748in}{0.970941in}}%
\pgfpathlineto{\pgfqpoint{1.409405in}{0.961567in}}%
\pgfpathlineto{\pgfqpoint{1.420098in}{0.957407in}}%
\pgfpathlineto{\pgfqpoint{1.425061in}{0.955424in}}%
\pgfpathclose%
\pgfpathmoveto{\pgfqpoint{1.412200in}{0.984629in}}%
\pgfpathlineto{\pgfqpoint{1.409405in}{0.986151in}}%
\pgfpathlineto{\pgfqpoint{1.394691in}{0.998240in}}%
\pgfpathlineto{\pgfqpoint{1.393748in}{0.999355in}}%
\pgfpathlineto{\pgfqpoint{1.385267in}{1.011851in}}%
\pgfpathlineto{\pgfqpoint{1.380643in}{1.025462in}}%
\pgfpathlineto{\pgfqpoint{1.380643in}{1.039073in}}%
\pgfpathlineto{\pgfqpoint{1.385267in}{1.052684in}}%
\pgfpathlineto{\pgfqpoint{1.393748in}{1.065181in}}%
\pgfpathlineto{\pgfqpoint{1.394691in}{1.066295in}}%
\pgfpathlineto{\pgfqpoint{1.409405in}{1.078385in}}%
\pgfpathlineto{\pgfqpoint{1.412200in}{1.079907in}}%
\pgfpathlineto{\pgfqpoint{1.425061in}{1.085699in}}%
\pgfpathlineto{\pgfqpoint{1.440718in}{1.088844in}}%
\pgfpathlineto{\pgfqpoint{1.456375in}{1.088058in}}%
\pgfpathlineto{\pgfqpoint{1.472031in}{1.083336in}}%
\pgfpathlineto{\pgfqpoint{1.478370in}{1.079907in}}%
\pgfpathlineto{\pgfqpoint{1.487688in}{1.073771in}}%
\pgfpathlineto{\pgfqpoint{1.495859in}{1.066295in}}%
\pgfpathlineto{\pgfqpoint{1.503344in}{1.056379in}}%
\pgfpathlineto{\pgfqpoint{1.505701in}{1.052684in}}%
\pgfpathlineto{\pgfqpoint{1.510148in}{1.039073in}}%
\pgfpathlineto{\pgfqpoint{1.510148in}{1.025462in}}%
\pgfpathlineto{\pgfqpoint{1.505701in}{1.011851in}}%
\pgfpathlineto{\pgfqpoint{1.503344in}{1.008156in}}%
\pgfpathlineto{\pgfqpoint{1.495859in}{0.998240in}}%
\pgfpathlineto{\pgfqpoint{1.487688in}{0.990764in}}%
\pgfpathlineto{\pgfqpoint{1.478370in}{0.984629in}}%
\pgfpathlineto{\pgfqpoint{1.472031in}{0.981199in}}%
\pgfpathlineto{\pgfqpoint{1.456375in}{0.976477in}}%
\pgfpathlineto{\pgfqpoint{1.440718in}{0.975691in}}%
\pgfpathlineto{\pgfqpoint{1.425061in}{0.978837in}}%
\pgfpathlineto{\pgfqpoint{1.412200in}{0.984629in}}%
\pgfpathclose%
\pgfpathmoveto{\pgfqpoint{1.738193in}{0.954573in}}%
\pgfpathlineto{\pgfqpoint{1.753849in}{0.952448in}}%
\pgfpathlineto{\pgfqpoint{1.769506in}{0.953864in}}%
\pgfpathlineto{\pgfqpoint{1.780750in}{0.957407in}}%
\pgfpathlineto{\pgfqpoint{1.785162in}{0.958752in}}%
\pgfpathlineto{\pgfqpoint{1.800819in}{0.966795in}}%
\pgfpathlineto{\pgfqpoint{1.806866in}{0.971018in}}%
\pgfpathlineto{\pgfqpoint{1.816476in}{0.978440in}}%
\pgfpathlineto{\pgfqpoint{1.823219in}{0.984629in}}%
\pgfpathlineto{\pgfqpoint{1.832132in}{0.994868in}}%
\pgfpathlineto{\pgfqpoint{1.834912in}{0.998240in}}%
\pgfpathlineto{\pgfqpoint{1.842706in}{1.011851in}}%
\pgfpathlineto{\pgfqpoint{1.846596in}{1.025462in}}%
\pgfpathlineto{\pgfqpoint{1.846596in}{1.039073in}}%
\pgfpathlineto{\pgfqpoint{1.842706in}{1.052684in}}%
\pgfpathlineto{\pgfqpoint{1.834912in}{1.066295in}}%
\pgfpathlineto{\pgfqpoint{1.832132in}{1.069668in}}%
\pgfpathlineto{\pgfqpoint{1.823219in}{1.079907in}}%
\pgfpathlineto{\pgfqpoint{1.816476in}{1.086096in}}%
\pgfpathlineto{\pgfqpoint{1.806866in}{1.093518in}}%
\pgfpathlineto{\pgfqpoint{1.800819in}{1.097740in}}%
\pgfpathlineto{\pgfqpoint{1.785162in}{1.105784in}}%
\pgfpathlineto{\pgfqpoint{1.780750in}{1.107129in}}%
\pgfpathlineto{\pgfqpoint{1.769506in}{1.110671in}}%
\pgfpathlineto{\pgfqpoint{1.753849in}{1.112088in}}%
\pgfpathlineto{\pgfqpoint{1.738193in}{1.109962in}}%
\pgfpathlineto{\pgfqpoint{1.730270in}{1.107129in}}%
\pgfpathlineto{\pgfqpoint{1.722536in}{1.104443in}}%
\pgfpathlineto{\pgfqpoint{1.706880in}{1.095733in}}%
\pgfpathlineto{\pgfqpoint{1.703836in}{1.093518in}}%
\pgfpathlineto{\pgfqpoint{1.691223in}{1.083339in}}%
\pgfpathlineto{\pgfqpoint{1.687563in}{1.079907in}}%
\pgfpathlineto{\pgfqpoint{1.675962in}{1.066295in}}%
\pgfpathlineto{\pgfqpoint{1.675567in}{1.065624in}}%
\pgfpathlineto{\pgfqpoint{1.668085in}{1.052684in}}%
\pgfpathlineto{\pgfqpoint{1.664150in}{1.039073in}}%
\pgfpathlineto{\pgfqpoint{1.664150in}{1.025462in}}%
\pgfpathlineto{\pgfqpoint{1.668085in}{1.011851in}}%
\pgfpathlineto{\pgfqpoint{1.675567in}{0.998912in}}%
\pgfpathlineto{\pgfqpoint{1.675962in}{0.998240in}}%
\pgfpathlineto{\pgfqpoint{1.687563in}{0.984629in}}%
\pgfpathlineto{\pgfqpoint{1.691223in}{0.981196in}}%
\pgfpathlineto{\pgfqpoint{1.703836in}{0.971018in}}%
\pgfpathlineto{\pgfqpoint{1.706880in}{0.968803in}}%
\pgfpathlineto{\pgfqpoint{1.722536in}{0.960092in}}%
\pgfpathlineto{\pgfqpoint{1.730270in}{0.957407in}}%
\pgfpathlineto{\pgfqpoint{1.738193in}{0.954573in}}%
\pgfpathclose%
\pgfpathmoveto{\pgfqpoint{1.721843in}{0.984629in}}%
\pgfpathlineto{\pgfqpoint{1.706880in}{0.996109in}}%
\pgfpathlineto{\pgfqpoint{1.704746in}{0.998240in}}%
\pgfpathlineto{\pgfqpoint{1.695234in}{1.011851in}}%
\pgfpathlineto{\pgfqpoint{1.691223in}{1.023319in}}%
\pgfpathlineto{\pgfqpoint{1.690548in}{1.025462in}}%
\pgfpathlineto{\pgfqpoint{1.690548in}{1.039073in}}%
\pgfpathlineto{\pgfqpoint{1.691223in}{1.041217in}}%
\pgfpathlineto{\pgfqpoint{1.695234in}{1.052684in}}%
\pgfpathlineto{\pgfqpoint{1.704746in}{1.066295in}}%
\pgfpathlineto{\pgfqpoint{1.706880in}{1.068426in}}%
\pgfpathlineto{\pgfqpoint{1.721843in}{1.079907in}}%
\pgfpathlineto{\pgfqpoint{1.722536in}{1.080341in}}%
\pgfpathlineto{\pgfqpoint{1.738193in}{1.086643in}}%
\pgfpathlineto{\pgfqpoint{1.753849in}{1.089001in}}%
\pgfpathlineto{\pgfqpoint{1.769506in}{1.087429in}}%
\pgfpathlineto{\pgfqpoint{1.785162in}{1.081918in}}%
\pgfpathlineto{\pgfqpoint{1.788603in}{1.079907in}}%
\pgfpathlineto{\pgfqpoint{1.800819in}{1.071189in}}%
\pgfpathlineto{\pgfqpoint{1.805928in}{1.066295in}}%
\pgfpathlineto{\pgfqpoint{1.815741in}{1.052684in}}%
\pgfpathlineto{\pgfqpoint{1.816476in}{1.050673in}}%
\pgfpathlineto{\pgfqpoint{1.820207in}{1.039073in}}%
\pgfpathlineto{\pgfqpoint{1.820207in}{1.025462in}}%
\pgfpathlineto{\pgfqpoint{1.816476in}{1.013862in}}%
\pgfpathlineto{\pgfqpoint{1.815741in}{1.011851in}}%
\pgfpathlineto{\pgfqpoint{1.805928in}{0.998240in}}%
\pgfpathlineto{\pgfqpoint{1.800819in}{0.993346in}}%
\pgfpathlineto{\pgfqpoint{1.788603in}{0.984629in}}%
\pgfpathlineto{\pgfqpoint{1.785162in}{0.982618in}}%
\pgfpathlineto{\pgfqpoint{1.769506in}{0.977106in}}%
\pgfpathlineto{\pgfqpoint{1.753849in}{0.975534in}}%
\pgfpathlineto{\pgfqpoint{1.738193in}{0.977892in}}%
\pgfpathlineto{\pgfqpoint{1.722536in}{0.984194in}}%
\pgfpathlineto{\pgfqpoint{1.721843in}{0.984629in}}%
\pgfpathclose%
\pgfpathmoveto{\pgfqpoint{0.485668in}{1.228217in}}%
\pgfpathlineto{\pgfqpoint{0.501324in}{1.223348in}}%
\pgfpathlineto{\pgfqpoint{0.516981in}{1.221960in}}%
\pgfpathlineto{\pgfqpoint{0.532637in}{1.224043in}}%
\pgfpathlineto{\pgfqpoint{0.548294in}{1.229610in}}%
\pgfpathlineto{\pgfqpoint{0.548328in}{1.229629in}}%
\pgfpathlineto{\pgfqpoint{0.563950in}{1.238327in}}%
\pgfpathlineto{\pgfqpoint{0.570606in}{1.243240in}}%
\pgfpathlineto{\pgfqpoint{0.579607in}{1.250763in}}%
\pgfpathlineto{\pgfqpoint{0.585889in}{1.256851in}}%
\pgfpathlineto{\pgfqpoint{0.595263in}{1.268544in}}%
\pgfpathlineto{\pgfqpoint{0.596754in}{1.270462in}}%
\pgfpathlineto{\pgfqpoint{0.603848in}{1.284073in}}%
\pgfpathlineto{\pgfqpoint{0.606994in}{1.297684in}}%
\pgfpathlineto{\pgfqpoint{0.606208in}{1.311295in}}%
\pgfpathlineto{\pgfqpoint{0.601485in}{1.324907in}}%
\pgfpathlineto{\pgfqpoint{0.595263in}{1.334776in}}%
\pgfpathlineto{\pgfqpoint{0.592852in}{1.338518in}}%
\pgfpathlineto{\pgfqpoint{0.580499in}{1.352129in}}%
\pgfpathlineto{\pgfqpoint{0.579607in}{1.352942in}}%
\pgfpathlineto{\pgfqpoint{0.563950in}{1.365222in}}%
\pgfpathlineto{\pgfqpoint{0.563083in}{1.365740in}}%
\pgfpathlineto{\pgfqpoint{0.548294in}{1.373974in}}%
\pgfpathlineto{\pgfqpoint{0.532638in}{1.379351in}}%
\pgfpathlineto{\pgfqpoint{0.532637in}{1.379351in}}%
\pgfpathlineto{\pgfqpoint{0.516981in}{1.381530in}}%
\pgfpathlineto{\pgfqpoint{0.501324in}{1.380078in}}%
\pgfpathlineto{\pgfqpoint{0.499040in}{1.379351in}}%
\pgfpathlineto{\pgfqpoint{0.485668in}{1.375319in}}%
\pgfpathlineto{\pgfqpoint{0.470011in}{1.367249in}}%
\pgfpathlineto{\pgfqpoint{0.467827in}{1.365740in}}%
\pgfpathlineto{\pgfqpoint{0.454354in}{1.355651in}}%
\pgfpathlineto{\pgfqpoint{0.450404in}{1.352129in}}%
\pgfpathlineto{\pgfqpoint{0.438698in}{1.339386in}}%
\pgfpathlineto{\pgfqpoint{0.437944in}{1.338518in}}%
\pgfpathlineto{\pgfqpoint{0.429370in}{1.324907in}}%
\pgfpathlineto{\pgfqpoint{0.424701in}{1.311295in}}%
\pgfpathlineto{\pgfqpoint{0.423924in}{1.297684in}}%
\pgfpathlineto{\pgfqpoint{0.427034in}{1.284073in}}%
\pgfpathlineto{\pgfqpoint{0.434047in}{1.270462in}}%
\pgfpathlineto{\pgfqpoint{0.438698in}{1.264492in}}%
\pgfpathlineto{\pgfqpoint{0.444967in}{1.256851in}}%
\pgfpathlineto{\pgfqpoint{0.454354in}{1.247946in}}%
\pgfpathlineto{\pgfqpoint{0.460237in}{1.243240in}}%
\pgfpathlineto{\pgfqpoint{0.470011in}{1.236318in}}%
\pgfpathlineto{\pgfqpoint{0.482929in}{1.229629in}}%
\pgfpathlineto{\pgfqpoint{0.485668in}{1.228217in}}%
\pgfpathclose%
\pgfpathmoveto{\pgfqpoint{0.477994in}{1.256851in}}%
\pgfpathlineto{\pgfqpoint{0.470011in}{1.262896in}}%
\pgfpathlineto{\pgfqpoint{0.462547in}{1.270462in}}%
\pgfpathlineto{\pgfqpoint{0.454354in}{1.283061in}}%
\pgfpathlineto{\pgfqpoint{0.453783in}{1.284073in}}%
\pgfpathlineto{\pgfqpoint{0.450272in}{1.297684in}}%
\pgfpathlineto{\pgfqpoint{0.451149in}{1.311295in}}%
\pgfpathlineto{\pgfqpoint{0.454354in}{1.319643in}}%
\pgfpathlineto{\pgfqpoint{0.456658in}{1.324907in}}%
\pgfpathlineto{\pgfqpoint{0.467454in}{1.338518in}}%
\pgfpathlineto{\pgfqpoint{0.470011in}{1.340845in}}%
\pgfpathlineto{\pgfqpoint{0.485668in}{1.351462in}}%
\pgfpathlineto{\pgfqpoint{0.487318in}{1.352129in}}%
\pgfpathlineto{\pgfqpoint{0.501324in}{1.356962in}}%
\pgfpathlineto{\pgfqpoint{0.516981in}{1.358507in}}%
\pgfpathlineto{\pgfqpoint{0.532637in}{1.356189in}}%
\pgfpathlineto{\pgfqpoint{0.543002in}{1.352129in}}%
\pgfpathlineto{\pgfqpoint{0.548294in}{1.349692in}}%
\pgfpathlineto{\pgfqpoint{0.563531in}{1.338518in}}%
\pgfpathlineto{\pgfqpoint{0.563950in}{1.338093in}}%
\pgfpathlineto{\pgfqpoint{0.574075in}{1.324907in}}%
\pgfpathlineto{\pgfqpoint{0.579607in}{1.311703in}}%
\pgfpathlineto{\pgfqpoint{0.579760in}{1.311295in}}%
\pgfpathlineto{\pgfqpoint{0.580629in}{1.297684in}}%
\pgfpathlineto{\pgfqpoint{0.579607in}{1.293644in}}%
\pgfpathlineto{\pgfqpoint{0.576926in}{1.284073in}}%
\pgfpathlineto{\pgfqpoint{0.568368in}{1.270462in}}%
\pgfpathlineto{\pgfqpoint{0.563950in}{1.265795in}}%
\pgfpathlineto{\pgfqpoint{0.552965in}{1.256851in}}%
\pgfpathlineto{\pgfqpoint{0.548294in}{1.253826in}}%
\pgfpathlineto{\pgfqpoint{0.532637in}{1.247386in}}%
\pgfpathlineto{\pgfqpoint{0.516981in}{1.244976in}}%
\pgfpathlineto{\pgfqpoint{0.501324in}{1.246583in}}%
\pgfpathlineto{\pgfqpoint{0.485668in}{1.252215in}}%
\pgfpathlineto{\pgfqpoint{0.477994in}{1.256851in}}%
\pgfpathclose%
\pgfpathmoveto{\pgfqpoint{0.798799in}{1.226964in}}%
\pgfpathlineto{\pgfqpoint{0.814455in}{1.222793in}}%
\pgfpathlineto{\pgfqpoint{0.830112in}{1.222099in}}%
\pgfpathlineto{\pgfqpoint{0.845769in}{1.224877in}}%
\pgfpathlineto{\pgfqpoint{0.857734in}{1.229629in}}%
\pgfpathlineto{\pgfqpoint{0.861425in}{1.231087in}}%
\pgfpathlineto{\pgfqpoint{0.877082in}{1.240466in}}%
\pgfpathlineto{\pgfqpoint{0.880702in}{1.243240in}}%
\pgfpathlineto{\pgfqpoint{0.892738in}{1.253703in}}%
\pgfpathlineto{\pgfqpoint{0.895929in}{1.256851in}}%
\pgfpathlineto{\pgfqpoint{0.906718in}{1.270462in}}%
\pgfpathlineto{\pgfqpoint{0.908395in}{1.273671in}}%
\pgfpathlineto{\pgfqpoint{0.913860in}{1.284073in}}%
\pgfpathlineto{\pgfqpoint{0.917057in}{1.297684in}}%
\pgfpathlineto{\pgfqpoint{0.916258in}{1.311295in}}%
\pgfpathlineto{\pgfqpoint{0.911460in}{1.324907in}}%
\pgfpathlineto{\pgfqpoint{0.908395in}{1.329762in}}%
\pgfpathlineto{\pgfqpoint{0.902859in}{1.338518in}}%
\pgfpathlineto{\pgfqpoint{0.892738in}{1.349827in}}%
\pgfpathlineto{\pgfqpoint{0.890418in}{1.352129in}}%
\pgfpathlineto{\pgfqpoint{0.877082in}{1.363026in}}%
\pgfpathlineto{\pgfqpoint{0.872820in}{1.365740in}}%
\pgfpathlineto{\pgfqpoint{0.861425in}{1.372494in}}%
\pgfpathlineto{\pgfqpoint{0.845769in}{1.378546in}}%
\pgfpathlineto{\pgfqpoint{0.841118in}{1.379351in}}%
\pgfpathlineto{\pgfqpoint{0.830112in}{1.381385in}}%
\pgfpathlineto{\pgfqpoint{0.814455in}{1.380659in}}%
\pgfpathlineto{\pgfqpoint{0.809689in}{1.379351in}}%
\pgfpathlineto{\pgfqpoint{0.798799in}{1.376530in}}%
\pgfpathlineto{\pgfqpoint{0.783142in}{1.369131in}}%
\pgfpathlineto{\pgfqpoint{0.778004in}{1.365740in}}%
\pgfpathlineto{\pgfqpoint{0.767486in}{1.358237in}}%
\pgfpathlineto{\pgfqpoint{0.760459in}{1.352129in}}%
\pgfpathlineto{\pgfqpoint{0.751829in}{1.342984in}}%
\pgfpathlineto{\pgfqpoint{0.747928in}{1.338518in}}%
\pgfpathlineto{\pgfqpoint{0.739418in}{1.324907in}}%
\pgfpathlineto{\pgfqpoint{0.736173in}{1.315439in}}%
\pgfpathlineto{\pgfqpoint{0.734668in}{1.311295in}}%
\pgfpathlineto{\pgfqpoint{0.733833in}{1.297684in}}%
\pgfpathlineto{\pgfqpoint{0.736173in}{1.288116in}}%
\pgfpathlineto{\pgfqpoint{0.737099in}{1.284073in}}%
\pgfpathlineto{\pgfqpoint{0.744060in}{1.270462in}}%
\pgfpathlineto{\pgfqpoint{0.751829in}{1.260556in}}%
\pgfpathlineto{\pgfqpoint{0.754951in}{1.256851in}}%
\pgfpathlineto{\pgfqpoint{0.767486in}{1.245257in}}%
\pgfpathlineto{\pgfqpoint{0.770133in}{1.243240in}}%
\pgfpathlineto{\pgfqpoint{0.783142in}{1.234441in}}%
\pgfpathlineto{\pgfqpoint{0.793214in}{1.229629in}}%
\pgfpathlineto{\pgfqpoint{0.798799in}{1.226964in}}%
\pgfpathclose%
\pgfpathmoveto{\pgfqpoint{0.787917in}{1.256851in}}%
\pgfpathlineto{\pgfqpoint{0.783142in}{1.260187in}}%
\pgfpathlineto{\pgfqpoint{0.772528in}{1.270462in}}%
\pgfpathlineto{\pgfqpoint{0.767486in}{1.277849in}}%
\pgfpathlineto{\pgfqpoint{0.763883in}{1.284073in}}%
\pgfpathlineto{\pgfqpoint{0.760326in}{1.297684in}}%
\pgfpathlineto{\pgfqpoint{0.761215in}{1.311295in}}%
\pgfpathlineto{\pgfqpoint{0.766554in}{1.324907in}}%
\pgfpathlineto{\pgfqpoint{0.767486in}{1.326246in}}%
\pgfpathlineto{\pgfqpoint{0.777617in}{1.338518in}}%
\pgfpathlineto{\pgfqpoint{0.783142in}{1.343321in}}%
\pgfpathlineto{\pgfqpoint{0.797258in}{1.352129in}}%
\pgfpathlineto{\pgfqpoint{0.798799in}{1.352939in}}%
\pgfpathlineto{\pgfqpoint{0.814455in}{1.357580in}}%
\pgfpathlineto{\pgfqpoint{0.830112in}{1.358353in}}%
\pgfpathlineto{\pgfqpoint{0.845769in}{1.355261in}}%
\pgfpathlineto{\pgfqpoint{0.852929in}{1.352129in}}%
\pgfpathlineto{\pgfqpoint{0.861425in}{1.347745in}}%
\pgfpathlineto{\pgfqpoint{0.873244in}{1.338518in}}%
\pgfpathlineto{\pgfqpoint{0.877082in}{1.334367in}}%
\pgfpathlineto{\pgfqpoint{0.884081in}{1.324907in}}%
\pgfpathlineto{\pgfqpoint{0.889632in}{1.311295in}}%
\pgfpathlineto{\pgfqpoint{0.890556in}{1.297684in}}%
\pgfpathlineto{\pgfqpoint{0.886858in}{1.284073in}}%
\pgfpathlineto{\pgfqpoint{0.878522in}{1.270462in}}%
\pgfpathlineto{\pgfqpoint{0.877082in}{1.268883in}}%
\pgfpathlineto{\pgfqpoint{0.863241in}{1.256851in}}%
\pgfpathlineto{\pgfqpoint{0.861425in}{1.255599in}}%
\pgfpathlineto{\pgfqpoint{0.845769in}{1.248351in}}%
\pgfpathlineto{\pgfqpoint{0.830112in}{1.245137in}}%
\pgfpathlineto{\pgfqpoint{0.814455in}{1.245940in}}%
\pgfpathlineto{\pgfqpoint{0.798799in}{1.250766in}}%
\pgfpathlineto{\pgfqpoint{0.787917in}{1.256851in}}%
\pgfpathclose%
\pgfpathmoveto{\pgfqpoint{1.111930in}{1.225851in}}%
\pgfpathlineto{\pgfqpoint{1.127587in}{1.222376in}}%
\pgfpathlineto{\pgfqpoint{1.143243in}{1.222376in}}%
\pgfpathlineto{\pgfqpoint{1.158900in}{1.225851in}}%
\pgfpathlineto{\pgfqpoint{1.167533in}{1.229629in}}%
\pgfpathlineto{\pgfqpoint{1.174556in}{1.232697in}}%
\pgfpathlineto{\pgfqpoint{1.190213in}{1.242735in}}%
\pgfpathlineto{\pgfqpoint{1.190851in}{1.243240in}}%
\pgfpathlineto{\pgfqpoint{1.205870in}{1.256763in}}%
\pgfpathlineto{\pgfqpoint{1.205958in}{1.256851in}}%
\pgfpathlineto{\pgfqpoint{1.216740in}{1.270462in}}%
\pgfpathlineto{\pgfqpoint{1.221526in}{1.279758in}}%
\pgfpathlineto{\pgfqpoint{1.223806in}{1.284073in}}%
\pgfpathlineto{\pgfqpoint{1.227067in}{1.297684in}}%
\pgfpathlineto{\pgfqpoint{1.226252in}{1.311295in}}%
\pgfpathlineto{\pgfqpoint{1.221526in}{1.324446in}}%
\pgfpathlineto{\pgfqpoint{1.221366in}{1.324907in}}%
\pgfpathlineto{\pgfqpoint{1.212884in}{1.338518in}}%
\pgfpathlineto{\pgfqpoint{1.205870in}{1.346467in}}%
\pgfpathlineto{\pgfqpoint{1.200358in}{1.352129in}}%
\pgfpathlineto{\pgfqpoint{1.190213in}{1.360696in}}%
\pgfpathlineto{\pgfqpoint{1.182751in}{1.365740in}}%
\pgfpathlineto{\pgfqpoint{1.174556in}{1.370880in}}%
\pgfpathlineto{\pgfqpoint{1.158900in}{1.377605in}}%
\pgfpathlineto{\pgfqpoint{1.150827in}{1.379351in}}%
\pgfpathlineto{\pgfqpoint{1.143243in}{1.381094in}}%
\pgfpathlineto{\pgfqpoint{1.127587in}{1.381094in}}%
\pgfpathlineto{\pgfqpoint{1.120003in}{1.379351in}}%
\pgfpathlineto{\pgfqpoint{1.111930in}{1.377605in}}%
\pgfpathlineto{\pgfqpoint{1.096274in}{1.370880in}}%
\pgfpathlineto{\pgfqpoint{1.088079in}{1.365740in}}%
\pgfpathlineto{\pgfqpoint{1.080617in}{1.360696in}}%
\pgfpathlineto{\pgfqpoint{1.070472in}{1.352129in}}%
\pgfpathlineto{\pgfqpoint{1.064960in}{1.346467in}}%
\pgfpathlineto{\pgfqpoint{1.057946in}{1.338518in}}%
\pgfpathlineto{\pgfqpoint{1.049464in}{1.324907in}}%
\pgfpathlineto{\pgfqpoint{1.049304in}{1.324446in}}%
\pgfpathlineto{\pgfqpoint{1.044578in}{1.311295in}}%
\pgfpathlineto{\pgfqpoint{1.043763in}{1.297684in}}%
\pgfpathlineto{\pgfqpoint{1.047024in}{1.284073in}}%
\pgfpathlineto{\pgfqpoint{1.049304in}{1.279758in}}%
\pgfpathlineto{\pgfqpoint{1.054090in}{1.270462in}}%
\pgfpathlineto{\pgfqpoint{1.064872in}{1.256851in}}%
\pgfpathlineto{\pgfqpoint{1.064960in}{1.256763in}}%
\pgfpathlineto{\pgfqpoint{1.079979in}{1.243240in}}%
\pgfpathlineto{\pgfqpoint{1.080617in}{1.242735in}}%
\pgfpathlineto{\pgfqpoint{1.096274in}{1.232697in}}%
\pgfpathlineto{\pgfqpoint{1.103297in}{1.229629in}}%
\pgfpathlineto{\pgfqpoint{1.111930in}{1.225851in}}%
\pgfpathclose%
\pgfpathmoveto{\pgfqpoint{1.097556in}{1.256851in}}%
\pgfpathlineto{\pgfqpoint{1.096274in}{1.257670in}}%
\pgfpathlineto{\pgfqpoint{1.082368in}{1.270462in}}%
\pgfpathlineto{\pgfqpoint{1.080617in}{1.272892in}}%
\pgfpathlineto{\pgfqpoint{1.073955in}{1.284073in}}%
\pgfpathlineto{\pgfqpoint{1.070336in}{1.297684in}}%
\pgfpathlineto{\pgfqpoint{1.071240in}{1.311295in}}%
\pgfpathlineto{\pgfqpoint{1.076672in}{1.324907in}}%
\pgfpathlineto{\pgfqpoint{1.080617in}{1.330417in}}%
\pgfpathlineto{\pgfqpoint{1.087675in}{1.338518in}}%
\pgfpathlineto{\pgfqpoint{1.096274in}{1.345621in}}%
\pgfpathlineto{\pgfqpoint{1.107680in}{1.352129in}}%
\pgfpathlineto{\pgfqpoint{1.111930in}{1.354178in}}%
\pgfpathlineto{\pgfqpoint{1.127587in}{1.358044in}}%
\pgfpathlineto{\pgfqpoint{1.143243in}{1.358044in}}%
\pgfpathlineto{\pgfqpoint{1.158900in}{1.354178in}}%
\pgfpathlineto{\pgfqpoint{1.163150in}{1.352129in}}%
\pgfpathlineto{\pgfqpoint{1.174556in}{1.345621in}}%
\pgfpathlineto{\pgfqpoint{1.183155in}{1.338518in}}%
\pgfpathlineto{\pgfqpoint{1.190213in}{1.330417in}}%
\pgfpathlineto{\pgfqpoint{1.194158in}{1.324907in}}%
\pgfpathlineto{\pgfqpoint{1.199590in}{1.311295in}}%
\pgfpathlineto{\pgfqpoint{1.200494in}{1.297684in}}%
\pgfpathlineto{\pgfqpoint{1.196875in}{1.284073in}}%
\pgfpathlineto{\pgfqpoint{1.190213in}{1.272892in}}%
\pgfpathlineto{\pgfqpoint{1.188462in}{1.270462in}}%
\pgfpathlineto{\pgfqpoint{1.174556in}{1.257670in}}%
\pgfpathlineto{\pgfqpoint{1.173274in}{1.256851in}}%
\pgfpathlineto{\pgfqpoint{1.158900in}{1.249478in}}%
\pgfpathlineto{\pgfqpoint{1.143243in}{1.245458in}}%
\pgfpathlineto{\pgfqpoint{1.127587in}{1.245458in}}%
\pgfpathlineto{\pgfqpoint{1.111930in}{1.249478in}}%
\pgfpathlineto{\pgfqpoint{1.097556in}{1.256851in}}%
\pgfpathclose%
\pgfpathmoveto{\pgfqpoint{1.425061in}{1.224877in}}%
\pgfpathlineto{\pgfqpoint{1.440718in}{1.222099in}}%
\pgfpathlineto{\pgfqpoint{1.456375in}{1.222793in}}%
\pgfpathlineto{\pgfqpoint{1.472031in}{1.226964in}}%
\pgfpathlineto{\pgfqpoint{1.477616in}{1.229629in}}%
\pgfpathlineto{\pgfqpoint{1.487688in}{1.234441in}}%
\pgfpathlineto{\pgfqpoint{1.500697in}{1.243240in}}%
\pgfpathlineto{\pgfqpoint{1.503344in}{1.245257in}}%
\pgfpathlineto{\pgfqpoint{1.515879in}{1.256851in}}%
\pgfpathlineto{\pgfqpoint{1.519001in}{1.260556in}}%
\pgfpathlineto{\pgfqpoint{1.526770in}{1.270462in}}%
\pgfpathlineto{\pgfqpoint{1.533731in}{1.284073in}}%
\pgfpathlineto{\pgfqpoint{1.534657in}{1.288116in}}%
\pgfpathlineto{\pgfqpoint{1.536997in}{1.297684in}}%
\pgfpathlineto{\pgfqpoint{1.536162in}{1.311295in}}%
\pgfpathlineto{\pgfqpoint{1.534657in}{1.315439in}}%
\pgfpathlineto{\pgfqpoint{1.531412in}{1.324907in}}%
\pgfpathlineto{\pgfqpoint{1.522902in}{1.338518in}}%
\pgfpathlineto{\pgfqpoint{1.519001in}{1.342984in}}%
\pgfpathlineto{\pgfqpoint{1.510371in}{1.352129in}}%
\pgfpathlineto{\pgfqpoint{1.503344in}{1.358237in}}%
\pgfpathlineto{\pgfqpoint{1.492826in}{1.365740in}}%
\pgfpathlineto{\pgfqpoint{1.487688in}{1.369131in}}%
\pgfpathlineto{\pgfqpoint{1.472031in}{1.376530in}}%
\pgfpathlineto{\pgfqpoint{1.461141in}{1.379351in}}%
\pgfpathlineto{\pgfqpoint{1.456375in}{1.380659in}}%
\pgfpathlineto{\pgfqpoint{1.440718in}{1.381385in}}%
\pgfpathlineto{\pgfqpoint{1.429712in}{1.379351in}}%
\pgfpathlineto{\pgfqpoint{1.425061in}{1.378546in}}%
\pgfpathlineto{\pgfqpoint{1.409405in}{1.372494in}}%
\pgfpathlineto{\pgfqpoint{1.398010in}{1.365740in}}%
\pgfpathlineto{\pgfqpoint{1.393748in}{1.363026in}}%
\pgfpathlineto{\pgfqpoint{1.380412in}{1.352129in}}%
\pgfpathlineto{\pgfqpoint{1.378092in}{1.349827in}}%
\pgfpathlineto{\pgfqpoint{1.367971in}{1.338518in}}%
\pgfpathlineto{\pgfqpoint{1.362435in}{1.329762in}}%
\pgfpathlineto{\pgfqpoint{1.359370in}{1.324907in}}%
\pgfpathlineto{\pgfqpoint{1.354572in}{1.311295in}}%
\pgfpathlineto{\pgfqpoint{1.353773in}{1.297684in}}%
\pgfpathlineto{\pgfqpoint{1.356970in}{1.284073in}}%
\pgfpathlineto{\pgfqpoint{1.362435in}{1.273671in}}%
\pgfpathlineto{\pgfqpoint{1.364112in}{1.270462in}}%
\pgfpathlineto{\pgfqpoint{1.374901in}{1.256851in}}%
\pgfpathlineto{\pgfqpoint{1.378092in}{1.253703in}}%
\pgfpathlineto{\pgfqpoint{1.390128in}{1.243240in}}%
\pgfpathlineto{\pgfqpoint{1.393748in}{1.240466in}}%
\pgfpathlineto{\pgfqpoint{1.409405in}{1.231087in}}%
\pgfpathlineto{\pgfqpoint{1.413096in}{1.229629in}}%
\pgfpathlineto{\pgfqpoint{1.425061in}{1.224877in}}%
\pgfpathclose%
\pgfpathmoveto{\pgfqpoint{1.407589in}{1.256851in}}%
\pgfpathlineto{\pgfqpoint{1.393748in}{1.268883in}}%
\pgfpathlineto{\pgfqpoint{1.392308in}{1.270462in}}%
\pgfpathlineto{\pgfqpoint{1.383972in}{1.284073in}}%
\pgfpathlineto{\pgfqpoint{1.380274in}{1.297684in}}%
\pgfpathlineto{\pgfqpoint{1.381198in}{1.311295in}}%
\pgfpathlineto{\pgfqpoint{1.386749in}{1.324907in}}%
\pgfpathlineto{\pgfqpoint{1.393748in}{1.334367in}}%
\pgfpathlineto{\pgfqpoint{1.397586in}{1.338518in}}%
\pgfpathlineto{\pgfqpoint{1.409405in}{1.347745in}}%
\pgfpathlineto{\pgfqpoint{1.417901in}{1.352129in}}%
\pgfpathlineto{\pgfqpoint{1.425061in}{1.355261in}}%
\pgfpathlineto{\pgfqpoint{1.440718in}{1.358353in}}%
\pgfpathlineto{\pgfqpoint{1.456375in}{1.357580in}}%
\pgfpathlineto{\pgfqpoint{1.472031in}{1.352939in}}%
\pgfpathlineto{\pgfqpoint{1.473572in}{1.352129in}}%
\pgfpathlineto{\pgfqpoint{1.487688in}{1.343321in}}%
\pgfpathlineto{\pgfqpoint{1.493213in}{1.338518in}}%
\pgfpathlineto{\pgfqpoint{1.503344in}{1.326246in}}%
\pgfpathlineto{\pgfqpoint{1.504276in}{1.324907in}}%
\pgfpathlineto{\pgfqpoint{1.509615in}{1.311295in}}%
\pgfpathlineto{\pgfqpoint{1.510504in}{1.297684in}}%
\pgfpathlineto{\pgfqpoint{1.506947in}{1.284073in}}%
\pgfpathlineto{\pgfqpoint{1.503344in}{1.277849in}}%
\pgfpathlineto{\pgfqpoint{1.498302in}{1.270462in}}%
\pgfpathlineto{\pgfqpoint{1.487688in}{1.260187in}}%
\pgfpathlineto{\pgfqpoint{1.482913in}{1.256851in}}%
\pgfpathlineto{\pgfqpoint{1.472031in}{1.250766in}}%
\pgfpathlineto{\pgfqpoint{1.456375in}{1.245940in}}%
\pgfpathlineto{\pgfqpoint{1.440718in}{1.245137in}}%
\pgfpathlineto{\pgfqpoint{1.425061in}{1.248351in}}%
\pgfpathlineto{\pgfqpoint{1.409405in}{1.255599in}}%
\pgfpathlineto{\pgfqpoint{1.407589in}{1.256851in}}%
\pgfpathclose%
\pgfpathmoveto{\pgfqpoint{1.722536in}{1.229610in}}%
\pgfpathlineto{\pgfqpoint{1.738193in}{1.224043in}}%
\pgfpathlineto{\pgfqpoint{1.753849in}{1.221960in}}%
\pgfpathlineto{\pgfqpoint{1.769506in}{1.223348in}}%
\pgfpathlineto{\pgfqpoint{1.785162in}{1.228217in}}%
\pgfpathlineto{\pgfqpoint{1.787901in}{1.229629in}}%
\pgfpathlineto{\pgfqpoint{1.800819in}{1.236318in}}%
\pgfpathlineto{\pgfqpoint{1.810593in}{1.243240in}}%
\pgfpathlineto{\pgfqpoint{1.816476in}{1.247946in}}%
\pgfpathlineto{\pgfqpoint{1.825863in}{1.256851in}}%
\pgfpathlineto{\pgfqpoint{1.832132in}{1.264492in}}%
\pgfpathlineto{\pgfqpoint{1.836783in}{1.270462in}}%
\pgfpathlineto{\pgfqpoint{1.843796in}{1.284073in}}%
\pgfpathlineto{\pgfqpoint{1.846906in}{1.297684in}}%
\pgfpathlineto{\pgfqpoint{1.846129in}{1.311295in}}%
\pgfpathlineto{\pgfqpoint{1.841460in}{1.324907in}}%
\pgfpathlineto{\pgfqpoint{1.832886in}{1.338518in}}%
\pgfpathlineto{\pgfqpoint{1.832132in}{1.339386in}}%
\pgfpathlineto{\pgfqpoint{1.820426in}{1.352129in}}%
\pgfpathlineto{\pgfqpoint{1.816476in}{1.355651in}}%
\pgfpathlineto{\pgfqpoint{1.803003in}{1.365740in}}%
\pgfpathlineto{\pgfqpoint{1.800819in}{1.367249in}}%
\pgfpathlineto{\pgfqpoint{1.785162in}{1.375319in}}%
\pgfpathlineto{\pgfqpoint{1.771790in}{1.379351in}}%
\pgfpathlineto{\pgfqpoint{1.769506in}{1.380078in}}%
\pgfpathlineto{\pgfqpoint{1.753849in}{1.381530in}}%
\pgfpathlineto{\pgfqpoint{1.738193in}{1.379351in}}%
\pgfpathlineto{\pgfqpoint{1.738192in}{1.379351in}}%
\pgfpathlineto{\pgfqpoint{1.722536in}{1.373974in}}%
\pgfpathlineto{\pgfqpoint{1.707747in}{1.365740in}}%
\pgfpathlineto{\pgfqpoint{1.706880in}{1.365222in}}%
\pgfpathlineto{\pgfqpoint{1.691223in}{1.352942in}}%
\pgfpathlineto{\pgfqpoint{1.690331in}{1.352129in}}%
\pgfpathlineto{\pgfqpoint{1.677978in}{1.338518in}}%
\pgfpathlineto{\pgfqpoint{1.675567in}{1.334776in}}%
\pgfpathlineto{\pgfqpoint{1.669345in}{1.324907in}}%
\pgfpathlineto{\pgfqpoint{1.664622in}{1.311295in}}%
\pgfpathlineto{\pgfqpoint{1.663836in}{1.297684in}}%
\pgfpathlineto{\pgfqpoint{1.666982in}{1.284073in}}%
\pgfpathlineto{\pgfqpoint{1.674076in}{1.270462in}}%
\pgfpathlineto{\pgfqpoint{1.675567in}{1.268544in}}%
\pgfpathlineto{\pgfqpoint{1.684941in}{1.256851in}}%
\pgfpathlineto{\pgfqpoint{1.691223in}{1.250763in}}%
\pgfpathlineto{\pgfqpoint{1.700224in}{1.243240in}}%
\pgfpathlineto{\pgfqpoint{1.706880in}{1.238327in}}%
\pgfpathlineto{\pgfqpoint{1.722502in}{1.229629in}}%
\pgfpathlineto{\pgfqpoint{1.722536in}{1.229610in}}%
\pgfpathclose%
\pgfpathmoveto{\pgfqpoint{1.717865in}{1.256851in}}%
\pgfpathlineto{\pgfqpoint{1.706880in}{1.265795in}}%
\pgfpathlineto{\pgfqpoint{1.702462in}{1.270462in}}%
\pgfpathlineto{\pgfqpoint{1.693904in}{1.284073in}}%
\pgfpathlineto{\pgfqpoint{1.691223in}{1.293644in}}%
\pgfpathlineto{\pgfqpoint{1.690201in}{1.297684in}}%
\pgfpathlineto{\pgfqpoint{1.691070in}{1.311295in}}%
\pgfpathlineto{\pgfqpoint{1.691223in}{1.311703in}}%
\pgfpathlineto{\pgfqpoint{1.696755in}{1.324907in}}%
\pgfpathlineto{\pgfqpoint{1.706880in}{1.338093in}}%
\pgfpathlineto{\pgfqpoint{1.707299in}{1.338518in}}%
\pgfpathlineto{\pgfqpoint{1.722536in}{1.349692in}}%
\pgfpathlineto{\pgfqpoint{1.727828in}{1.352129in}}%
\pgfpathlineto{\pgfqpoint{1.738193in}{1.356189in}}%
\pgfpathlineto{\pgfqpoint{1.753849in}{1.358507in}}%
\pgfpathlineto{\pgfqpoint{1.769506in}{1.356962in}}%
\pgfpathlineto{\pgfqpoint{1.783512in}{1.352129in}}%
\pgfpathlineto{\pgfqpoint{1.785162in}{1.351462in}}%
\pgfpathlineto{\pgfqpoint{1.800819in}{1.340845in}}%
\pgfpathlineto{\pgfqpoint{1.803376in}{1.338518in}}%
\pgfpathlineto{\pgfqpoint{1.814172in}{1.324907in}}%
\pgfpathlineto{\pgfqpoint{1.816476in}{1.319643in}}%
\pgfpathlineto{\pgfqpoint{1.819681in}{1.311295in}}%
\pgfpathlineto{\pgfqpoint{1.820558in}{1.297684in}}%
\pgfpathlineto{\pgfqpoint{1.817047in}{1.284073in}}%
\pgfpathlineto{\pgfqpoint{1.816476in}{1.283061in}}%
\pgfpathlineto{\pgfqpoint{1.808283in}{1.270462in}}%
\pgfpathlineto{\pgfqpoint{1.800819in}{1.262896in}}%
\pgfpathlineto{\pgfqpoint{1.792836in}{1.256851in}}%
\pgfpathlineto{\pgfqpoint{1.785162in}{1.252215in}}%
\pgfpathlineto{\pgfqpoint{1.769506in}{1.246583in}}%
\pgfpathlineto{\pgfqpoint{1.753849in}{1.244976in}}%
\pgfpathlineto{\pgfqpoint{1.738193in}{1.247386in}}%
\pgfpathlineto{\pgfqpoint{1.722536in}{1.253826in}}%
\pgfpathlineto{\pgfqpoint{1.717865in}{1.256851in}}%
\pgfpathclose%
\pgfpathmoveto{\pgfqpoint{0.485668in}{1.497676in}}%
\pgfpathlineto{\pgfqpoint{0.501324in}{1.492883in}}%
\pgfpathlineto{\pgfqpoint{0.516981in}{1.491516in}}%
\pgfpathlineto{\pgfqpoint{0.532637in}{1.493567in}}%
\pgfpathlineto{\pgfqpoint{0.548294in}{1.499047in}}%
\pgfpathlineto{\pgfqpoint{0.553377in}{1.501851in}}%
\pgfpathlineto{\pgfqpoint{0.563950in}{1.507851in}}%
\pgfpathlineto{\pgfqpoint{0.574079in}{1.515462in}}%
\pgfpathlineto{\pgfqpoint{0.579607in}{1.520268in}}%
\pgfpathlineto{\pgfqpoint{0.588361in}{1.529073in}}%
\pgfpathlineto{\pgfqpoint{0.595263in}{1.538266in}}%
\pgfpathlineto{\pgfqpoint{0.598489in}{1.542684in}}%
\pgfpathlineto{\pgfqpoint{0.604792in}{1.556295in}}%
\pgfpathlineto{\pgfqpoint{0.607151in}{1.569907in}}%
\pgfpathlineto{\pgfqpoint{0.605579in}{1.583518in}}%
\pgfpathlineto{\pgfqpoint{0.600066in}{1.597129in}}%
\pgfpathlineto{\pgfqpoint{0.595263in}{1.604181in}}%
\pgfpathlineto{\pgfqpoint{0.590683in}{1.610740in}}%
\pgfpathlineto{\pgfqpoint{0.579607in}{1.622351in}}%
\pgfpathlineto{\pgfqpoint{0.577405in}{1.624351in}}%
\pgfpathlineto{\pgfqpoint{0.563950in}{1.634681in}}%
\pgfpathlineto{\pgfqpoint{0.558300in}{1.637962in}}%
\pgfpathlineto{\pgfqpoint{0.548294in}{1.643497in}}%
\pgfpathlineto{\pgfqpoint{0.532637in}{1.648914in}}%
\pgfpathlineto{\pgfqpoint{0.516981in}{1.650941in}}%
\pgfpathlineto{\pgfqpoint{0.501324in}{1.649590in}}%
\pgfpathlineto{\pgfqpoint{0.485668in}{1.644852in}}%
\pgfpathlineto{\pgfqpoint{0.472316in}{1.637962in}}%
\pgfpathlineto{\pgfqpoint{0.470011in}{1.636717in}}%
\pgfpathlineto{\pgfqpoint{0.454354in}{1.625233in}}%
\pgfpathlineto{\pgfqpoint{0.453340in}{1.624351in}}%
\pgfpathlineto{\pgfqpoint{0.440130in}{1.610740in}}%
\pgfpathlineto{\pgfqpoint{0.438698in}{1.608736in}}%
\pgfpathlineto{\pgfqpoint{0.430772in}{1.597129in}}%
\pgfpathlineto{\pgfqpoint{0.425323in}{1.583518in}}%
\pgfpathlineto{\pgfqpoint{0.423768in}{1.569907in}}%
\pgfpathlineto{\pgfqpoint{0.426100in}{1.556295in}}%
\pgfpathlineto{\pgfqpoint{0.432331in}{1.542684in}}%
\pgfpathlineto{\pgfqpoint{0.438698in}{1.533986in}}%
\pgfpathlineto{\pgfqpoint{0.442472in}{1.529073in}}%
\pgfpathlineto{\pgfqpoint{0.454354in}{1.517376in}}%
\pgfpathlineto{\pgfqpoint{0.456655in}{1.515462in}}%
\pgfpathlineto{\pgfqpoint{0.470011in}{1.505833in}}%
\pgfpathlineto{\pgfqpoint{0.477556in}{1.501851in}}%
\pgfpathlineto{\pgfqpoint{0.485668in}{1.497676in}}%
\pgfpathclose%
\pgfpathmoveto{\pgfqpoint{0.506186in}{1.515462in}}%
\pgfpathlineto{\pgfqpoint{0.501324in}{1.515977in}}%
\pgfpathlineto{\pgfqpoint{0.485668in}{1.521759in}}%
\pgfpathlineto{\pgfqpoint{0.474001in}{1.529073in}}%
\pgfpathlineto{\pgfqpoint{0.470011in}{1.532299in}}%
\pgfpathlineto{\pgfqpoint{0.460387in}{1.542684in}}%
\pgfpathlineto{\pgfqpoint{0.454354in}{1.553080in}}%
\pgfpathlineto{\pgfqpoint{0.452729in}{1.556295in}}%
\pgfpathlineto{\pgfqpoint{0.450097in}{1.569907in}}%
\pgfpathlineto{\pgfqpoint{0.451851in}{1.583518in}}%
\pgfpathlineto{\pgfqpoint{0.454354in}{1.589139in}}%
\pgfpathlineto{\pgfqpoint{0.458424in}{1.597129in}}%
\pgfpathlineto{\pgfqpoint{0.470011in}{1.610530in}}%
\pgfpathlineto{\pgfqpoint{0.470252in}{1.610740in}}%
\pgfpathlineto{\pgfqpoint{0.485668in}{1.620813in}}%
\pgfpathlineto{\pgfqpoint{0.494857in}{1.624351in}}%
\pgfpathlineto{\pgfqpoint{0.501324in}{1.626527in}}%
\pgfpathlineto{\pgfqpoint{0.516981in}{1.628052in}}%
\pgfpathlineto{\pgfqpoint{0.532637in}{1.625764in}}%
\pgfpathlineto{\pgfqpoint{0.536336in}{1.624351in}}%
\pgfpathlineto{\pgfqpoint{0.548294in}{1.619106in}}%
\pgfpathlineto{\pgfqpoint{0.560239in}{1.610740in}}%
\pgfpathlineto{\pgfqpoint{0.563950in}{1.607271in}}%
\pgfpathlineto{\pgfqpoint{0.572364in}{1.597129in}}%
\pgfpathlineto{\pgfqpoint{0.579015in}{1.583518in}}%
\pgfpathlineto{\pgfqpoint{0.579607in}{1.579291in}}%
\pgfpathlineto{\pgfqpoint{0.580803in}{1.569907in}}%
\pgfpathlineto{\pgfqpoint{0.579607in}{1.563629in}}%
\pgfpathlineto{\pgfqpoint{0.578066in}{1.556295in}}%
\pgfpathlineto{\pgfqpoint{0.570461in}{1.542684in}}%
\pgfpathlineto{\pgfqpoint{0.563950in}{1.535362in}}%
\pgfpathlineto{\pgfqpoint{0.556717in}{1.529073in}}%
\pgfpathlineto{\pgfqpoint{0.548294in}{1.523413in}}%
\pgfpathlineto{\pgfqpoint{0.532637in}{1.516802in}}%
\pgfpathlineto{\pgfqpoint{0.524202in}{1.515462in}}%
\pgfpathlineto{\pgfqpoint{0.516981in}{1.514422in}}%
\pgfpathlineto{\pgfqpoint{0.506186in}{1.515462in}}%
\pgfpathclose%
\pgfpathmoveto{\pgfqpoint{0.798799in}{1.496443in}}%
\pgfpathlineto{\pgfqpoint{0.814455in}{1.492336in}}%
\pgfpathlineto{\pgfqpoint{0.830112in}{1.491653in}}%
\pgfpathlineto{\pgfqpoint{0.845769in}{1.494388in}}%
\pgfpathlineto{\pgfqpoint{0.861425in}{1.500555in}}%
\pgfpathlineto{\pgfqpoint{0.863631in}{1.501851in}}%
\pgfpathlineto{\pgfqpoint{0.877082in}{1.510001in}}%
\pgfpathlineto{\pgfqpoint{0.884085in}{1.515462in}}%
\pgfpathlineto{\pgfqpoint{0.892738in}{1.523287in}}%
\pgfpathlineto{\pgfqpoint{0.898390in}{1.529073in}}%
\pgfpathlineto{\pgfqpoint{0.908395in}{1.542655in}}%
\pgfpathlineto{\pgfqpoint{0.908416in}{1.542684in}}%
\pgfpathlineto{\pgfqpoint{0.914820in}{1.556295in}}%
\pgfpathlineto{\pgfqpoint{0.917216in}{1.569907in}}%
\pgfpathlineto{\pgfqpoint{0.915619in}{1.583518in}}%
\pgfpathlineto{\pgfqpoint{0.910018in}{1.597129in}}%
\pgfpathlineto{\pgfqpoint{0.908395in}{1.599510in}}%
\pgfpathlineto{\pgfqpoint{0.900700in}{1.610740in}}%
\pgfpathlineto{\pgfqpoint{0.892738in}{1.619237in}}%
\pgfpathlineto{\pgfqpoint{0.887325in}{1.624351in}}%
\pgfpathlineto{\pgfqpoint{0.877082in}{1.632512in}}%
\pgfpathlineto{\pgfqpoint{0.868292in}{1.637962in}}%
\pgfpathlineto{\pgfqpoint{0.861425in}{1.642006in}}%
\pgfpathlineto{\pgfqpoint{0.845769in}{1.648102in}}%
\pgfpathlineto{\pgfqpoint{0.830112in}{1.650806in}}%
\pgfpathlineto{\pgfqpoint{0.814455in}{1.650131in}}%
\pgfpathlineto{\pgfqpoint{0.798799in}{1.646071in}}%
\pgfpathlineto{\pgfqpoint{0.783142in}{1.638617in}}%
\pgfpathlineto{\pgfqpoint{0.782143in}{1.637962in}}%
\pgfpathlineto{\pgfqpoint{0.767486in}{1.627786in}}%
\pgfpathlineto{\pgfqpoint{0.763434in}{1.624351in}}%
\pgfpathlineto{\pgfqpoint{0.751829in}{1.612638in}}%
\pgfpathlineto{\pgfqpoint{0.750093in}{1.610740in}}%
\pgfpathlineto{\pgfqpoint{0.740810in}{1.597129in}}%
\pgfpathlineto{\pgfqpoint{0.736173in}{1.585503in}}%
\pgfpathlineto{\pgfqpoint{0.735337in}{1.583518in}}%
\pgfpathlineto{\pgfqpoint{0.733666in}{1.569907in}}%
\pgfpathlineto{\pgfqpoint{0.736172in}{1.556295in}}%
\pgfpathlineto{\pgfqpoint{0.736173in}{1.556295in}}%
\pgfpathlineto{\pgfqpoint{0.742357in}{1.542684in}}%
\pgfpathlineto{\pgfqpoint{0.751829in}{1.529827in}}%
\pgfpathlineto{\pgfqpoint{0.752424in}{1.529073in}}%
\pgfpathlineto{\pgfqpoint{0.766551in}{1.515462in}}%
\pgfpathlineto{\pgfqpoint{0.767486in}{1.514686in}}%
\pgfpathlineto{\pgfqpoint{0.783142in}{1.503947in}}%
\pgfpathlineto{\pgfqpoint{0.787447in}{1.501851in}}%
\pgfpathlineto{\pgfqpoint{0.798799in}{1.496443in}}%
\pgfpathclose%
\pgfpathmoveto{\pgfqpoint{0.813987in}{1.515462in}}%
\pgfpathlineto{\pgfqpoint{0.798799in}{1.520271in}}%
\pgfpathlineto{\pgfqpoint{0.783631in}{1.529073in}}%
\pgfpathlineto{\pgfqpoint{0.783142in}{1.529438in}}%
\pgfpathlineto{\pgfqpoint{0.770289in}{1.542684in}}%
\pgfpathlineto{\pgfqpoint{0.767486in}{1.547285in}}%
\pgfpathlineto{\pgfqpoint{0.762815in}{1.556295in}}%
\pgfpathlineto{\pgfqpoint{0.760149in}{1.569907in}}%
\pgfpathlineto{\pgfqpoint{0.761926in}{1.583518in}}%
\pgfpathlineto{\pgfqpoint{0.767486in}{1.595694in}}%
\pgfpathlineto{\pgfqpoint{0.768253in}{1.597129in}}%
\pgfpathlineto{\pgfqpoint{0.780465in}{1.610740in}}%
\pgfpathlineto{\pgfqpoint{0.783142in}{1.612963in}}%
\pgfpathlineto{\pgfqpoint{0.798799in}{1.622348in}}%
\pgfpathlineto{\pgfqpoint{0.804853in}{1.624351in}}%
\pgfpathlineto{\pgfqpoint{0.814455in}{1.627137in}}%
\pgfpathlineto{\pgfqpoint{0.830112in}{1.627900in}}%
\pgfpathlineto{\pgfqpoint{0.845769in}{1.624848in}}%
\pgfpathlineto{\pgfqpoint{0.846933in}{1.624351in}}%
\pgfpathlineto{\pgfqpoint{0.861425in}{1.617229in}}%
\pgfpathlineto{\pgfqpoint{0.870128in}{1.610740in}}%
\pgfpathlineto{\pgfqpoint{0.877082in}{1.603800in}}%
\pgfpathlineto{\pgfqpoint{0.882414in}{1.597129in}}%
\pgfpathlineto{\pgfqpoint{0.888893in}{1.583518in}}%
\pgfpathlineto{\pgfqpoint{0.890741in}{1.569907in}}%
\pgfpathlineto{\pgfqpoint{0.887968in}{1.556295in}}%
\pgfpathlineto{\pgfqpoint{0.880561in}{1.542684in}}%
\pgfpathlineto{\pgfqpoint{0.877082in}{1.538623in}}%
\pgfpathlineto{\pgfqpoint{0.866793in}{1.529073in}}%
\pgfpathlineto{\pgfqpoint{0.861425in}{1.525233in}}%
\pgfpathlineto{\pgfqpoint{0.845769in}{1.517793in}}%
\pgfpathlineto{\pgfqpoint{0.834760in}{1.515462in}}%
\pgfpathlineto{\pgfqpoint{0.830112in}{1.514573in}}%
\pgfpathlineto{\pgfqpoint{0.814455in}{1.515329in}}%
\pgfpathlineto{\pgfqpoint{0.813987in}{1.515462in}}%
\pgfpathclose%
\pgfpathmoveto{\pgfqpoint{1.111930in}{1.495347in}}%
\pgfpathlineto{\pgfqpoint{1.127587in}{1.491926in}}%
\pgfpathlineto{\pgfqpoint{1.143243in}{1.491926in}}%
\pgfpathlineto{\pgfqpoint{1.158900in}{1.495347in}}%
\pgfpathlineto{\pgfqpoint{1.173784in}{1.501851in}}%
\pgfpathlineto{\pgfqpoint{1.174556in}{1.502195in}}%
\pgfpathlineto{\pgfqpoint{1.190213in}{1.512280in}}%
\pgfpathlineto{\pgfqpoint{1.194161in}{1.515462in}}%
\pgfpathlineto{\pgfqpoint{1.205870in}{1.526428in}}%
\pgfpathlineto{\pgfqpoint{1.208417in}{1.529073in}}%
\pgfpathlineto{\pgfqpoint{1.218437in}{1.542684in}}%
\pgfpathlineto{\pgfqpoint{1.221526in}{1.549408in}}%
\pgfpathlineto{\pgfqpoint{1.224785in}{1.556295in}}%
\pgfpathlineto{\pgfqpoint{1.227230in}{1.569907in}}%
\pgfpathlineto{\pgfqpoint{1.225600in}{1.583518in}}%
\pgfpathlineto{\pgfqpoint{1.221526in}{1.593293in}}%
\pgfpathlineto{\pgfqpoint{1.219979in}{1.597129in}}%
\pgfpathlineto{\pgfqpoint{1.210727in}{1.610740in}}%
\pgfpathlineto{\pgfqpoint{1.205870in}{1.615997in}}%
\pgfpathlineto{\pgfqpoint{1.197332in}{1.624351in}}%
\pgfpathlineto{\pgfqpoint{1.190213in}{1.630213in}}%
\pgfpathlineto{\pgfqpoint{1.178435in}{1.637962in}}%
\pgfpathlineto{\pgfqpoint{1.174556in}{1.640379in}}%
\pgfpathlineto{\pgfqpoint{1.158900in}{1.647155in}}%
\pgfpathlineto{\pgfqpoint{1.143243in}{1.650536in}}%
\pgfpathlineto{\pgfqpoint{1.127587in}{1.650536in}}%
\pgfpathlineto{\pgfqpoint{1.111930in}{1.647155in}}%
\pgfpathlineto{\pgfqpoint{1.096274in}{1.640379in}}%
\pgfpathlineto{\pgfqpoint{1.092395in}{1.637962in}}%
\pgfpathlineto{\pgfqpoint{1.080617in}{1.630213in}}%
\pgfpathlineto{\pgfqpoint{1.073498in}{1.624351in}}%
\pgfpathlineto{\pgfqpoint{1.064960in}{1.615997in}}%
\pgfpathlineto{\pgfqpoint{1.060103in}{1.610740in}}%
\pgfpathlineto{\pgfqpoint{1.050851in}{1.597129in}}%
\pgfpathlineto{\pgfqpoint{1.049304in}{1.593293in}}%
\pgfpathlineto{\pgfqpoint{1.045230in}{1.583518in}}%
\pgfpathlineto{\pgfqpoint{1.043600in}{1.569907in}}%
\pgfpathlineto{\pgfqpoint{1.046045in}{1.556295in}}%
\pgfpathlineto{\pgfqpoint{1.049304in}{1.549408in}}%
\pgfpathlineto{\pgfqpoint{1.052393in}{1.542684in}}%
\pgfpathlineto{\pgfqpoint{1.062413in}{1.529073in}}%
\pgfpathlineto{\pgfqpoint{1.064960in}{1.526428in}}%
\pgfpathlineto{\pgfqpoint{1.076669in}{1.515462in}}%
\pgfpathlineto{\pgfqpoint{1.080617in}{1.512280in}}%
\pgfpathlineto{\pgfqpoint{1.096274in}{1.502195in}}%
\pgfpathlineto{\pgfqpoint{1.097046in}{1.501851in}}%
\pgfpathlineto{\pgfqpoint{1.111930in}{1.495347in}}%
\pgfpathclose%
\pgfpathmoveto{\pgfqpoint{1.125121in}{1.515462in}}%
\pgfpathlineto{\pgfqpoint{1.111930in}{1.518949in}}%
\pgfpathlineto{\pgfqpoint{1.096274in}{1.527218in}}%
\pgfpathlineto{\pgfqpoint{1.093823in}{1.529073in}}%
\pgfpathlineto{\pgfqpoint{1.080617in}{1.542082in}}%
\pgfpathlineto{\pgfqpoint{1.080117in}{1.542684in}}%
\pgfpathlineto{\pgfqpoint{1.072868in}{1.556295in}}%
\pgfpathlineto{\pgfqpoint{1.070156in}{1.569907in}}%
\pgfpathlineto{\pgfqpoint{1.071964in}{1.583518in}}%
\pgfpathlineto{\pgfqpoint{1.078304in}{1.597129in}}%
\pgfpathlineto{\pgfqpoint{1.080617in}{1.600120in}}%
\pgfpathlineto{\pgfqpoint{1.090644in}{1.610740in}}%
\pgfpathlineto{\pgfqpoint{1.096274in}{1.615181in}}%
\pgfpathlineto{\pgfqpoint{1.111930in}{1.623712in}}%
\pgfpathlineto{\pgfqpoint{1.114244in}{1.624351in}}%
\pgfpathlineto{\pgfqpoint{1.127587in}{1.627595in}}%
\pgfpathlineto{\pgfqpoint{1.143243in}{1.627595in}}%
\pgfpathlineto{\pgfqpoint{1.156586in}{1.624351in}}%
\pgfpathlineto{\pgfqpoint{1.158900in}{1.623712in}}%
\pgfpathlineto{\pgfqpoint{1.174556in}{1.615181in}}%
\pgfpathlineto{\pgfqpoint{1.180186in}{1.610740in}}%
\pgfpathlineto{\pgfqpoint{1.190213in}{1.600120in}}%
\pgfpathlineto{\pgfqpoint{1.192526in}{1.597129in}}%
\pgfpathlineto{\pgfqpoint{1.198866in}{1.583518in}}%
\pgfpathlineto{\pgfqpoint{1.200674in}{1.569907in}}%
\pgfpathlineto{\pgfqpoint{1.197962in}{1.556295in}}%
\pgfpathlineto{\pgfqpoint{1.190713in}{1.542684in}}%
\pgfpathlineto{\pgfqpoint{1.190213in}{1.542082in}}%
\pgfpathlineto{\pgfqpoint{1.177007in}{1.529073in}}%
\pgfpathlineto{\pgfqpoint{1.174556in}{1.527218in}}%
\pgfpathlineto{\pgfqpoint{1.158900in}{1.518949in}}%
\pgfpathlineto{\pgfqpoint{1.145709in}{1.515462in}}%
\pgfpathlineto{\pgfqpoint{1.143243in}{1.514875in}}%
\pgfpathlineto{\pgfqpoint{1.127587in}{1.514875in}}%
\pgfpathlineto{\pgfqpoint{1.125121in}{1.515462in}}%
\pgfpathclose%
\pgfpathmoveto{\pgfqpoint{1.409405in}{1.500555in}}%
\pgfpathlineto{\pgfqpoint{1.425061in}{1.494388in}}%
\pgfpathlineto{\pgfqpoint{1.440718in}{1.491653in}}%
\pgfpathlineto{\pgfqpoint{1.456375in}{1.492336in}}%
\pgfpathlineto{\pgfqpoint{1.472031in}{1.496443in}}%
\pgfpathlineto{\pgfqpoint{1.483383in}{1.501851in}}%
\pgfpathlineto{\pgfqpoint{1.487688in}{1.503947in}}%
\pgfpathlineto{\pgfqpoint{1.503344in}{1.514686in}}%
\pgfpathlineto{\pgfqpoint{1.504279in}{1.515462in}}%
\pgfpathlineto{\pgfqpoint{1.518406in}{1.529073in}}%
\pgfpathlineto{\pgfqpoint{1.519001in}{1.529827in}}%
\pgfpathlineto{\pgfqpoint{1.528473in}{1.542684in}}%
\pgfpathlineto{\pgfqpoint{1.534657in}{1.556295in}}%
\pgfpathlineto{\pgfqpoint{1.534658in}{1.556295in}}%
\pgfpathlineto{\pgfqpoint{1.537164in}{1.569907in}}%
\pgfpathlineto{\pgfqpoint{1.535493in}{1.583518in}}%
\pgfpathlineto{\pgfqpoint{1.534657in}{1.585503in}}%
\pgfpathlineto{\pgfqpoint{1.530020in}{1.597129in}}%
\pgfpathlineto{\pgfqpoint{1.520737in}{1.610740in}}%
\pgfpathlineto{\pgfqpoint{1.519001in}{1.612638in}}%
\pgfpathlineto{\pgfqpoint{1.507396in}{1.624351in}}%
\pgfpathlineto{\pgfqpoint{1.503344in}{1.627786in}}%
\pgfpathlineto{\pgfqpoint{1.488687in}{1.637962in}}%
\pgfpathlineto{\pgfqpoint{1.487688in}{1.638617in}}%
\pgfpathlineto{\pgfqpoint{1.472031in}{1.646071in}}%
\pgfpathlineto{\pgfqpoint{1.456375in}{1.650131in}}%
\pgfpathlineto{\pgfqpoint{1.440718in}{1.650806in}}%
\pgfpathlineto{\pgfqpoint{1.425061in}{1.648102in}}%
\pgfpathlineto{\pgfqpoint{1.409405in}{1.642006in}}%
\pgfpathlineto{\pgfqpoint{1.402538in}{1.637962in}}%
\pgfpathlineto{\pgfqpoint{1.393748in}{1.632512in}}%
\pgfpathlineto{\pgfqpoint{1.383505in}{1.624351in}}%
\pgfpathlineto{\pgfqpoint{1.378092in}{1.619237in}}%
\pgfpathlineto{\pgfqpoint{1.370130in}{1.610740in}}%
\pgfpathlineto{\pgfqpoint{1.362435in}{1.599510in}}%
\pgfpathlineto{\pgfqpoint{1.360812in}{1.597129in}}%
\pgfpathlineto{\pgfqpoint{1.355211in}{1.583518in}}%
\pgfpathlineto{\pgfqpoint{1.353614in}{1.569907in}}%
\pgfpathlineto{\pgfqpoint{1.356010in}{1.556295in}}%
\pgfpathlineto{\pgfqpoint{1.362414in}{1.542684in}}%
\pgfpathlineto{\pgfqpoint{1.362435in}{1.542655in}}%
\pgfpathlineto{\pgfqpoint{1.372440in}{1.529073in}}%
\pgfpathlineto{\pgfqpoint{1.378092in}{1.523287in}}%
\pgfpathlineto{\pgfqpoint{1.386745in}{1.515462in}}%
\pgfpathlineto{\pgfqpoint{1.393748in}{1.510001in}}%
\pgfpathlineto{\pgfqpoint{1.407199in}{1.501851in}}%
\pgfpathlineto{\pgfqpoint{1.409405in}{1.500555in}}%
\pgfpathclose%
\pgfpathmoveto{\pgfqpoint{1.436070in}{1.515462in}}%
\pgfpathlineto{\pgfqpoint{1.425061in}{1.517793in}}%
\pgfpathlineto{\pgfqpoint{1.409405in}{1.525233in}}%
\pgfpathlineto{\pgfqpoint{1.404037in}{1.529073in}}%
\pgfpathlineto{\pgfqpoint{1.393748in}{1.538623in}}%
\pgfpathlineto{\pgfqpoint{1.390269in}{1.542684in}}%
\pgfpathlineto{\pgfqpoint{1.382862in}{1.556295in}}%
\pgfpathlineto{\pgfqpoint{1.380089in}{1.569907in}}%
\pgfpathlineto{\pgfqpoint{1.381937in}{1.583518in}}%
\pgfpathlineto{\pgfqpoint{1.388416in}{1.597129in}}%
\pgfpathlineto{\pgfqpoint{1.393748in}{1.603800in}}%
\pgfpathlineto{\pgfqpoint{1.400702in}{1.610740in}}%
\pgfpathlineto{\pgfqpoint{1.409405in}{1.617229in}}%
\pgfpathlineto{\pgfqpoint{1.423897in}{1.624351in}}%
\pgfpathlineto{\pgfqpoint{1.425061in}{1.624848in}}%
\pgfpathlineto{\pgfqpoint{1.440718in}{1.627900in}}%
\pgfpathlineto{\pgfqpoint{1.456375in}{1.627137in}}%
\pgfpathlineto{\pgfqpoint{1.465977in}{1.624351in}}%
\pgfpathlineto{\pgfqpoint{1.472031in}{1.622348in}}%
\pgfpathlineto{\pgfqpoint{1.487688in}{1.612963in}}%
\pgfpathlineto{\pgfqpoint{1.490365in}{1.610740in}}%
\pgfpathlineto{\pgfqpoint{1.502577in}{1.597129in}}%
\pgfpathlineto{\pgfqpoint{1.503344in}{1.595694in}}%
\pgfpathlineto{\pgfqpoint{1.508904in}{1.583518in}}%
\pgfpathlineto{\pgfqpoint{1.510681in}{1.569907in}}%
\pgfpathlineto{\pgfqpoint{1.508015in}{1.556295in}}%
\pgfpathlineto{\pgfqpoint{1.503344in}{1.547285in}}%
\pgfpathlineto{\pgfqpoint{1.500541in}{1.542684in}}%
\pgfpathlineto{\pgfqpoint{1.487688in}{1.529438in}}%
\pgfpathlineto{\pgfqpoint{1.487199in}{1.529073in}}%
\pgfpathlineto{\pgfqpoint{1.472031in}{1.520271in}}%
\pgfpathlineto{\pgfqpoint{1.456843in}{1.515462in}}%
\pgfpathlineto{\pgfqpoint{1.456375in}{1.515329in}}%
\pgfpathlineto{\pgfqpoint{1.440718in}{1.514573in}}%
\pgfpathlineto{\pgfqpoint{1.436070in}{1.515462in}}%
\pgfpathclose%
\pgfpathmoveto{\pgfqpoint{1.722536in}{1.499047in}}%
\pgfpathlineto{\pgfqpoint{1.738193in}{1.493567in}}%
\pgfpathlineto{\pgfqpoint{1.753849in}{1.491516in}}%
\pgfpathlineto{\pgfqpoint{1.769506in}{1.492883in}}%
\pgfpathlineto{\pgfqpoint{1.785162in}{1.497676in}}%
\pgfpathlineto{\pgfqpoint{1.793274in}{1.501851in}}%
\pgfpathlineto{\pgfqpoint{1.800819in}{1.505833in}}%
\pgfpathlineto{\pgfqpoint{1.814175in}{1.515462in}}%
\pgfpathlineto{\pgfqpoint{1.816476in}{1.517376in}}%
\pgfpathlineto{\pgfqpoint{1.828358in}{1.529073in}}%
\pgfpathlineto{\pgfqpoint{1.832132in}{1.533986in}}%
\pgfpathlineto{\pgfqpoint{1.838499in}{1.542684in}}%
\pgfpathlineto{\pgfqpoint{1.844730in}{1.556295in}}%
\pgfpathlineto{\pgfqpoint{1.847062in}{1.569907in}}%
\pgfpathlineto{\pgfqpoint{1.845507in}{1.583518in}}%
\pgfpathlineto{\pgfqpoint{1.840058in}{1.597129in}}%
\pgfpathlineto{\pgfqpoint{1.832132in}{1.608736in}}%
\pgfpathlineto{\pgfqpoint{1.830700in}{1.610740in}}%
\pgfpathlineto{\pgfqpoint{1.817490in}{1.624351in}}%
\pgfpathlineto{\pgfqpoint{1.816476in}{1.625233in}}%
\pgfpathlineto{\pgfqpoint{1.800819in}{1.636717in}}%
\pgfpathlineto{\pgfqpoint{1.798514in}{1.637962in}}%
\pgfpathlineto{\pgfqpoint{1.785162in}{1.644852in}}%
\pgfpathlineto{\pgfqpoint{1.769506in}{1.649590in}}%
\pgfpathlineto{\pgfqpoint{1.753849in}{1.650941in}}%
\pgfpathlineto{\pgfqpoint{1.738193in}{1.648914in}}%
\pgfpathlineto{\pgfqpoint{1.722536in}{1.643497in}}%
\pgfpathlineto{\pgfqpoint{1.712530in}{1.637962in}}%
\pgfpathlineto{\pgfqpoint{1.706880in}{1.634681in}}%
\pgfpathlineto{\pgfqpoint{1.693425in}{1.624351in}}%
\pgfpathlineto{\pgfqpoint{1.691223in}{1.622351in}}%
\pgfpathlineto{\pgfqpoint{1.680147in}{1.610740in}}%
\pgfpathlineto{\pgfqpoint{1.675567in}{1.604181in}}%
\pgfpathlineto{\pgfqpoint{1.670764in}{1.597129in}}%
\pgfpathlineto{\pgfqpoint{1.665251in}{1.583518in}}%
\pgfpathlineto{\pgfqpoint{1.663679in}{1.569907in}}%
\pgfpathlineto{\pgfqpoint{1.666038in}{1.556295in}}%
\pgfpathlineto{\pgfqpoint{1.672341in}{1.542684in}}%
\pgfpathlineto{\pgfqpoint{1.675567in}{1.538266in}}%
\pgfpathlineto{\pgfqpoint{1.682469in}{1.529073in}}%
\pgfpathlineto{\pgfqpoint{1.691223in}{1.520268in}}%
\pgfpathlineto{\pgfqpoint{1.696751in}{1.515462in}}%
\pgfpathlineto{\pgfqpoint{1.706880in}{1.507851in}}%
\pgfpathlineto{\pgfqpoint{1.717453in}{1.501851in}}%
\pgfpathlineto{\pgfqpoint{1.722536in}{1.499047in}}%
\pgfpathclose%
\pgfpathmoveto{\pgfqpoint{1.746628in}{1.515462in}}%
\pgfpathlineto{\pgfqpoint{1.738193in}{1.516802in}}%
\pgfpathlineto{\pgfqpoint{1.722536in}{1.523413in}}%
\pgfpathlineto{\pgfqpoint{1.714113in}{1.529073in}}%
\pgfpathlineto{\pgfqpoint{1.706880in}{1.535362in}}%
\pgfpathlineto{\pgfqpoint{1.700369in}{1.542684in}}%
\pgfpathlineto{\pgfqpoint{1.692764in}{1.556295in}}%
\pgfpathlineto{\pgfqpoint{1.691223in}{1.563629in}}%
\pgfpathlineto{\pgfqpoint{1.690027in}{1.569907in}}%
\pgfpathlineto{\pgfqpoint{1.691223in}{1.579291in}}%
\pgfpathlineto{\pgfqpoint{1.691815in}{1.583518in}}%
\pgfpathlineto{\pgfqpoint{1.698466in}{1.597129in}}%
\pgfpathlineto{\pgfqpoint{1.706880in}{1.607271in}}%
\pgfpathlineto{\pgfqpoint{1.710591in}{1.610740in}}%
\pgfpathlineto{\pgfqpoint{1.722536in}{1.619106in}}%
\pgfpathlineto{\pgfqpoint{1.734494in}{1.624351in}}%
\pgfpathlineto{\pgfqpoint{1.738193in}{1.625764in}}%
\pgfpathlineto{\pgfqpoint{1.753849in}{1.628052in}}%
\pgfpathlineto{\pgfqpoint{1.769506in}{1.626527in}}%
\pgfpathlineto{\pgfqpoint{1.775973in}{1.624351in}}%
\pgfpathlineto{\pgfqpoint{1.785162in}{1.620813in}}%
\pgfpathlineto{\pgfqpoint{1.800578in}{1.610740in}}%
\pgfpathlineto{\pgfqpoint{1.800819in}{1.610530in}}%
\pgfpathlineto{\pgfqpoint{1.812406in}{1.597129in}}%
\pgfpathlineto{\pgfqpoint{1.816476in}{1.589139in}}%
\pgfpathlineto{\pgfqpoint{1.818979in}{1.583518in}}%
\pgfpathlineto{\pgfqpoint{1.820733in}{1.569907in}}%
\pgfpathlineto{\pgfqpoint{1.818101in}{1.556295in}}%
\pgfpathlineto{\pgfqpoint{1.816476in}{1.553080in}}%
\pgfpathlineto{\pgfqpoint{1.810443in}{1.542684in}}%
\pgfpathlineto{\pgfqpoint{1.800819in}{1.532299in}}%
\pgfpathlineto{\pgfqpoint{1.796829in}{1.529073in}}%
\pgfpathlineto{\pgfqpoint{1.785162in}{1.521759in}}%
\pgfpathlineto{\pgfqpoint{1.769506in}{1.515977in}}%
\pgfpathlineto{\pgfqpoint{1.764644in}{1.515462in}}%
\pgfpathlineto{\pgfqpoint{1.753849in}{1.514422in}}%
\pgfpathlineto{\pgfqpoint{1.746628in}{1.515462in}}%
\pgfpathclose%
\pgfusepath{fill}%
\end{pgfscope}%
\begin{pgfscope}%
\pgfpathrectangle{\pgfqpoint{0.360415in}{0.358518in}}{\pgfqpoint{1.550000in}{1.347500in}}%
\pgfusepath{clip}%
\pgfsetbuttcap%
\pgfsetroundjoin%
\definecolor{currentfill}{rgb}{0.709962,0.212797,0.477201}%
\pgfsetfillcolor{currentfill}%
\pgfsetlinewidth{0.000000pt}%
\definecolor{currentstroke}{rgb}{0.000000,0.000000,0.000000}%
\pgfsetstrokecolor{currentstroke}%
\pgfsetdash{}{0pt}%
\pgfpathmoveto{\pgfqpoint{0.516981in}{0.385408in}}%
\pgfpathlineto{\pgfqpoint{0.518346in}{0.385740in}}%
\pgfpathlineto{\pgfqpoint{0.532637in}{0.387998in}}%
\pgfpathlineto{\pgfqpoint{0.548294in}{0.394601in}}%
\pgfpathlineto{\pgfqpoint{0.555398in}{0.399351in}}%
\pgfpathlineto{\pgfqpoint{0.563950in}{0.404103in}}%
\pgfpathlineto{\pgfqpoint{0.575973in}{0.412962in}}%
\pgfpathlineto{\pgfqpoint{0.579607in}{0.415528in}}%
\pgfpathlineto{\pgfqpoint{0.592947in}{0.426573in}}%
\pgfpathlineto{\pgfqpoint{0.595263in}{0.428634in}}%
\pgfpathlineto{\pgfqpoint{0.607615in}{0.440184in}}%
\pgfpathlineto{\pgfqpoint{0.610920in}{0.443891in}}%
\pgfpathlineto{\pgfqpoint{0.620371in}{0.453795in}}%
\pgfpathlineto{\pgfqpoint{0.626577in}{0.462679in}}%
\pgfpathlineto{\pgfqpoint{0.630695in}{0.467407in}}%
\pgfpathlineto{\pgfqpoint{0.637797in}{0.481018in}}%
\pgfpathlineto{\pgfqpoint{0.639823in}{0.494629in}}%
\pgfpathlineto{\pgfqpoint{0.636784in}{0.508240in}}%
\pgfpathlineto{\pgfqpoint{0.628663in}{0.521851in}}%
\pgfpathlineto{\pgfqpoint{0.626577in}{0.524086in}}%
\pgfpathlineto{\pgfqpoint{0.618032in}{0.535462in}}%
\pgfpathlineto{\pgfqpoint{0.610920in}{0.542613in}}%
\pgfpathlineto{\pgfqpoint{0.604900in}{0.549073in}}%
\pgfpathlineto{\pgfqpoint{0.595263in}{0.557894in}}%
\pgfpathlineto{\pgfqpoint{0.589753in}{0.562684in}}%
\pgfpathlineto{\pgfqpoint{0.579607in}{0.571062in}}%
\pgfpathlineto{\pgfqpoint{0.572176in}{0.576295in}}%
\pgfpathlineto{\pgfqpoint{0.563950in}{0.582478in}}%
\pgfpathlineto{\pgfqpoint{0.550864in}{0.589907in}}%
\pgfpathlineto{\pgfqpoint{0.548294in}{0.591720in}}%
\pgfpathlineto{\pgfqpoint{0.532637in}{0.598780in}}%
\pgfpathlineto{\pgfqpoint{0.516981in}{0.601422in}}%
\pgfpathlineto{\pgfqpoint{0.501324in}{0.599661in}}%
\pgfpathlineto{\pgfqpoint{0.485668in}{0.593487in}}%
\pgfpathlineto{\pgfqpoint{0.480229in}{0.589907in}}%
\pgfpathlineto{\pgfqpoint{0.470011in}{0.584512in}}%
\pgfpathlineto{\pgfqpoint{0.458618in}{0.576295in}}%
\pgfpathlineto{\pgfqpoint{0.454354in}{0.573422in}}%
\pgfpathlineto{\pgfqpoint{0.441068in}{0.562684in}}%
\pgfpathlineto{\pgfqpoint{0.438698in}{0.560671in}}%
\pgfpathlineto{\pgfqpoint{0.425993in}{0.549073in}}%
\pgfpathlineto{\pgfqpoint{0.423041in}{0.545914in}}%
\pgfpathlineto{\pgfqpoint{0.412851in}{0.535462in}}%
\pgfpathlineto{\pgfqpoint{0.407385in}{0.528027in}}%
\pgfpathlineto{\pgfqpoint{0.401921in}{0.521851in}}%
\pgfpathlineto{\pgfqpoint{0.394325in}{0.508240in}}%
\pgfpathlineto{\pgfqpoint{0.391728in}{0.495815in}}%
\pgfpathlineto{\pgfqpoint{0.391346in}{0.494629in}}%
\pgfpathlineto{\pgfqpoint{0.391728in}{0.492855in}}%
\pgfpathlineto{\pgfqpoint{0.393377in}{0.481018in}}%
\pgfpathlineto{\pgfqpoint{0.400021in}{0.467407in}}%
\pgfpathlineto{\pgfqpoint{0.407385in}{0.458485in}}%
\pgfpathlineto{\pgfqpoint{0.410591in}{0.453795in}}%
\pgfpathlineto{\pgfqpoint{0.423041in}{0.440485in}}%
\pgfpathlineto{\pgfqpoint{0.423310in}{0.440184in}}%
\pgfpathlineto{\pgfqpoint{0.437849in}{0.426573in}}%
\pgfpathlineto{\pgfqpoint{0.438698in}{0.425835in}}%
\pgfpathlineto{\pgfqpoint{0.454354in}{0.413196in}}%
\pgfpathlineto{\pgfqpoint{0.454700in}{0.412962in}}%
\pgfpathlineto{\pgfqpoint{0.470011in}{0.402138in}}%
\pgfpathlineto{\pgfqpoint{0.475405in}{0.399351in}}%
\pgfpathlineto{\pgfqpoint{0.485668in}{0.392949in}}%
\pgfpathlineto{\pgfqpoint{0.501324in}{0.387174in}}%
\pgfpathlineto{\pgfqpoint{0.514940in}{0.385740in}}%
\pgfpathlineto{\pgfqpoint{0.516981in}{0.385408in}}%
\pgfpathclose%
\pgfpathmoveto{\pgfqpoint{0.472316in}{0.426573in}}%
\pgfpathlineto{\pgfqpoint{0.470011in}{0.427818in}}%
\pgfpathlineto{\pgfqpoint{0.454354in}{0.439302in}}%
\pgfpathlineto{\pgfqpoint{0.453340in}{0.440184in}}%
\pgfpathlineto{\pgfqpoint{0.440130in}{0.453795in}}%
\pgfpathlineto{\pgfqpoint{0.438698in}{0.455799in}}%
\pgfpathlineto{\pgfqpoint{0.430772in}{0.467407in}}%
\pgfpathlineto{\pgfqpoint{0.425323in}{0.481018in}}%
\pgfpathlineto{\pgfqpoint{0.423768in}{0.494629in}}%
\pgfpathlineto{\pgfqpoint{0.426100in}{0.508240in}}%
\pgfpathlineto{\pgfqpoint{0.432331in}{0.521851in}}%
\pgfpathlineto{\pgfqpoint{0.438698in}{0.530550in}}%
\pgfpathlineto{\pgfqpoint{0.442472in}{0.535462in}}%
\pgfpathlineto{\pgfqpoint{0.454354in}{0.547159in}}%
\pgfpathlineto{\pgfqpoint{0.456655in}{0.549073in}}%
\pgfpathlineto{\pgfqpoint{0.470011in}{0.558702in}}%
\pgfpathlineto{\pgfqpoint{0.477556in}{0.562684in}}%
\pgfpathlineto{\pgfqpoint{0.485668in}{0.566859in}}%
\pgfpathlineto{\pgfqpoint{0.501324in}{0.571652in}}%
\pgfpathlineto{\pgfqpoint{0.516981in}{0.573019in}}%
\pgfpathlineto{\pgfqpoint{0.532637in}{0.570968in}}%
\pgfpathlineto{\pgfqpoint{0.548294in}{0.565488in}}%
\pgfpathlineto{\pgfqpoint{0.553377in}{0.562684in}}%
\pgfpathlineto{\pgfqpoint{0.563950in}{0.556684in}}%
\pgfpathlineto{\pgfqpoint{0.574079in}{0.549073in}}%
\pgfpathlineto{\pgfqpoint{0.579607in}{0.544267in}}%
\pgfpathlineto{\pgfqpoint{0.588361in}{0.535462in}}%
\pgfpathlineto{\pgfqpoint{0.595263in}{0.526270in}}%
\pgfpathlineto{\pgfqpoint{0.598489in}{0.521851in}}%
\pgfpathlineto{\pgfqpoint{0.604792in}{0.508240in}}%
\pgfpathlineto{\pgfqpoint{0.607151in}{0.494629in}}%
\pgfpathlineto{\pgfqpoint{0.605579in}{0.481018in}}%
\pgfpathlineto{\pgfqpoint{0.600066in}{0.467407in}}%
\pgfpathlineto{\pgfqpoint{0.595263in}{0.460354in}}%
\pgfpathlineto{\pgfqpoint{0.590683in}{0.453795in}}%
\pgfpathlineto{\pgfqpoint{0.579607in}{0.442184in}}%
\pgfpathlineto{\pgfqpoint{0.577405in}{0.440184in}}%
\pgfpathlineto{\pgfqpoint{0.563950in}{0.429854in}}%
\pgfpathlineto{\pgfqpoint{0.558300in}{0.426573in}}%
\pgfpathlineto{\pgfqpoint{0.548294in}{0.421039in}}%
\pgfpathlineto{\pgfqpoint{0.532637in}{0.415621in}}%
\pgfpathlineto{\pgfqpoint{0.516981in}{0.413594in}}%
\pgfpathlineto{\pgfqpoint{0.501324in}{0.414945in}}%
\pgfpathlineto{\pgfqpoint{0.485668in}{0.419683in}}%
\pgfpathlineto{\pgfqpoint{0.472316in}{0.426573in}}%
\pgfpathclose%
\pgfpathmoveto{\pgfqpoint{0.830112in}{0.385664in}}%
\pgfpathlineto{\pgfqpoint{0.830346in}{0.385740in}}%
\pgfpathlineto{\pgfqpoint{0.845769in}{0.388987in}}%
\pgfpathlineto{\pgfqpoint{0.861425in}{0.396419in}}%
\pgfpathlineto{\pgfqpoint{0.865545in}{0.399351in}}%
\pgfpathlineto{\pgfqpoint{0.877082in}{0.406196in}}%
\pgfpathlineto{\pgfqpoint{0.885930in}{0.412962in}}%
\pgfpathlineto{\pgfqpoint{0.892738in}{0.417963in}}%
\pgfpathlineto{\pgfqpoint{0.902954in}{0.426573in}}%
\pgfpathlineto{\pgfqpoint{0.908395in}{0.431506in}}%
\pgfpathlineto{\pgfqpoint{0.917687in}{0.440184in}}%
\pgfpathlineto{\pgfqpoint{0.924051in}{0.447310in}}%
\pgfpathlineto{\pgfqpoint{0.930399in}{0.453795in}}%
\pgfpathlineto{\pgfqpoint{0.939708in}{0.466764in}}%
\pgfpathlineto{\pgfqpoint{0.940304in}{0.467407in}}%
\pgfpathlineto{\pgfqpoint{0.947967in}{0.481018in}}%
\pgfpathlineto{\pgfqpoint{0.950153in}{0.494629in}}%
\pgfpathlineto{\pgfqpoint{0.946874in}{0.508240in}}%
\pgfpathlineto{\pgfqpoint{0.939708in}{0.519430in}}%
\pgfpathlineto{\pgfqpoint{0.938525in}{0.521851in}}%
\pgfpathlineto{\pgfqpoint{0.927965in}{0.535462in}}%
\pgfpathlineto{\pgfqpoint{0.924051in}{0.539299in}}%
\pgfpathlineto{\pgfqpoint{0.914930in}{0.549073in}}%
\pgfpathlineto{\pgfqpoint{0.908395in}{0.555046in}}%
\pgfpathlineto{\pgfqpoint{0.899775in}{0.562684in}}%
\pgfpathlineto{\pgfqpoint{0.892738in}{0.568599in}}%
\pgfpathlineto{\pgfqpoint{0.882232in}{0.576295in}}%
\pgfpathlineto{\pgfqpoint{0.877082in}{0.580312in}}%
\pgfpathlineto{\pgfqpoint{0.861425in}{0.589807in}}%
\pgfpathlineto{\pgfqpoint{0.861178in}{0.589907in}}%
\pgfpathlineto{\pgfqpoint{0.845769in}{0.597722in}}%
\pgfpathlineto{\pgfqpoint{0.830112in}{0.601246in}}%
\pgfpathlineto{\pgfqpoint{0.814455in}{0.600366in}}%
\pgfpathlineto{\pgfqpoint{0.798799in}{0.595076in}}%
\pgfpathlineto{\pgfqpoint{0.790316in}{0.589907in}}%
\pgfpathlineto{\pgfqpoint{0.783142in}{0.586412in}}%
\pgfpathlineto{\pgfqpoint{0.768454in}{0.576295in}}%
\pgfpathlineto{\pgfqpoint{0.767486in}{0.575674in}}%
\pgfpathlineto{\pgfqpoint{0.751829in}{0.563380in}}%
\pgfpathlineto{\pgfqpoint{0.751021in}{0.562684in}}%
\pgfpathlineto{\pgfqpoint{0.736173in}{0.549177in}}%
\pgfpathlineto{\pgfqpoint{0.736057in}{0.549073in}}%
\pgfpathlineto{\pgfqpoint{0.722990in}{0.535462in}}%
\pgfpathlineto{\pgfqpoint{0.720516in}{0.532044in}}%
\pgfpathlineto{\pgfqpoint{0.711900in}{0.521851in}}%
\pgfpathlineto{\pgfqpoint{0.704859in}{0.508476in}}%
\pgfpathlineto{\pgfqpoint{0.704681in}{0.508240in}}%
\pgfpathlineto{\pgfqpoint{0.700734in}{0.494629in}}%
\pgfpathlineto{\pgfqpoint{0.703365in}{0.481018in}}%
\pgfpathlineto{\pgfqpoint{0.704859in}{0.478765in}}%
\pgfpathlineto{\pgfqpoint{0.710108in}{0.467407in}}%
\pgfpathlineto{\pgfqpoint{0.720516in}{0.454209in}}%
\pgfpathlineto{\pgfqpoint{0.720795in}{0.453795in}}%
\pgfpathlineto{\pgfqpoint{0.733174in}{0.440184in}}%
\pgfpathlineto{\pgfqpoint{0.736173in}{0.437428in}}%
\pgfpathlineto{\pgfqpoint{0.747834in}{0.426573in}}%
\pgfpathlineto{\pgfqpoint{0.751829in}{0.423122in}}%
\pgfpathlineto{\pgfqpoint{0.764776in}{0.412962in}}%
\pgfpathlineto{\pgfqpoint{0.767486in}{0.410757in}}%
\pgfpathlineto{\pgfqpoint{0.783142in}{0.400303in}}%
\pgfpathlineto{\pgfqpoint{0.785138in}{0.399351in}}%
\pgfpathlineto{\pgfqpoint{0.798799in}{0.391463in}}%
\pgfpathlineto{\pgfqpoint{0.814455in}{0.386515in}}%
\pgfpathlineto{\pgfqpoint{0.829184in}{0.385740in}}%
\pgfpathlineto{\pgfqpoint{0.830112in}{0.385664in}}%
\pgfpathclose%
\pgfpathmoveto{\pgfqpoint{0.782143in}{0.426573in}}%
\pgfpathlineto{\pgfqpoint{0.767486in}{0.436750in}}%
\pgfpathlineto{\pgfqpoint{0.763434in}{0.440184in}}%
\pgfpathlineto{\pgfqpoint{0.751829in}{0.451897in}}%
\pgfpathlineto{\pgfqpoint{0.750093in}{0.453795in}}%
\pgfpathlineto{\pgfqpoint{0.740810in}{0.467407in}}%
\pgfpathlineto{\pgfqpoint{0.736173in}{0.479032in}}%
\pgfpathlineto{\pgfqpoint{0.735337in}{0.481018in}}%
\pgfpathlineto{\pgfqpoint{0.733666in}{0.494629in}}%
\pgfpathlineto{\pgfqpoint{0.736172in}{0.508240in}}%
\pgfpathlineto{\pgfqpoint{0.736173in}{0.508241in}}%
\pgfpathlineto{\pgfqpoint{0.742357in}{0.521851in}}%
\pgfpathlineto{\pgfqpoint{0.751829in}{0.534708in}}%
\pgfpathlineto{\pgfqpoint{0.752424in}{0.535462in}}%
\pgfpathlineto{\pgfqpoint{0.766551in}{0.549073in}}%
\pgfpathlineto{\pgfqpoint{0.767486in}{0.549849in}}%
\pgfpathlineto{\pgfqpoint{0.783142in}{0.560588in}}%
\pgfpathlineto{\pgfqpoint{0.787447in}{0.562684in}}%
\pgfpathlineto{\pgfqpoint{0.798799in}{0.568093in}}%
\pgfpathlineto{\pgfqpoint{0.814455in}{0.572199in}}%
\pgfpathlineto{\pgfqpoint{0.830112in}{0.572882in}}%
\pgfpathlineto{\pgfqpoint{0.845769in}{0.570147in}}%
\pgfpathlineto{\pgfqpoint{0.861425in}{0.563980in}}%
\pgfpathlineto{\pgfqpoint{0.863631in}{0.562684in}}%
\pgfpathlineto{\pgfqpoint{0.877082in}{0.554534in}}%
\pgfpathlineto{\pgfqpoint{0.884085in}{0.549073in}}%
\pgfpathlineto{\pgfqpoint{0.892738in}{0.541248in}}%
\pgfpathlineto{\pgfqpoint{0.898390in}{0.535462in}}%
\pgfpathlineto{\pgfqpoint{0.908395in}{0.521880in}}%
\pgfpathlineto{\pgfqpoint{0.908416in}{0.521851in}}%
\pgfpathlineto{\pgfqpoint{0.914820in}{0.508240in}}%
\pgfpathlineto{\pgfqpoint{0.917216in}{0.494629in}}%
\pgfpathlineto{\pgfqpoint{0.915619in}{0.481018in}}%
\pgfpathlineto{\pgfqpoint{0.910018in}{0.467407in}}%
\pgfpathlineto{\pgfqpoint{0.908395in}{0.465026in}}%
\pgfpathlineto{\pgfqpoint{0.900700in}{0.453795in}}%
\pgfpathlineto{\pgfqpoint{0.892738in}{0.445299in}}%
\pgfpathlineto{\pgfqpoint{0.887325in}{0.440184in}}%
\pgfpathlineto{\pgfqpoint{0.877082in}{0.432023in}}%
\pgfpathlineto{\pgfqpoint{0.868292in}{0.426573in}}%
\pgfpathlineto{\pgfqpoint{0.861425in}{0.422530in}}%
\pgfpathlineto{\pgfqpoint{0.845769in}{0.416433in}}%
\pgfpathlineto{\pgfqpoint{0.830112in}{0.413729in}}%
\pgfpathlineto{\pgfqpoint{0.814455in}{0.414405in}}%
\pgfpathlineto{\pgfqpoint{0.798799in}{0.418464in}}%
\pgfpathlineto{\pgfqpoint{0.783142in}{0.425918in}}%
\pgfpathlineto{\pgfqpoint{0.782143in}{0.426573in}}%
\pgfpathclose%
\pgfpathmoveto{\pgfqpoint{1.440718in}{0.385664in}}%
\pgfpathlineto{\pgfqpoint{1.441646in}{0.385740in}}%
\pgfpathlineto{\pgfqpoint{1.456375in}{0.386515in}}%
\pgfpathlineto{\pgfqpoint{1.472031in}{0.391463in}}%
\pgfpathlineto{\pgfqpoint{1.485692in}{0.399351in}}%
\pgfpathlineto{\pgfqpoint{1.487688in}{0.400303in}}%
\pgfpathlineto{\pgfqpoint{1.503344in}{0.410757in}}%
\pgfpathlineto{\pgfqpoint{1.506054in}{0.412962in}}%
\pgfpathlineto{\pgfqpoint{1.519001in}{0.423122in}}%
\pgfpathlineto{\pgfqpoint{1.522996in}{0.426573in}}%
\pgfpathlineto{\pgfqpoint{1.534657in}{0.437428in}}%
\pgfpathlineto{\pgfqpoint{1.537656in}{0.440184in}}%
\pgfpathlineto{\pgfqpoint{1.550035in}{0.453795in}}%
\pgfpathlineto{\pgfqpoint{1.550314in}{0.454209in}}%
\pgfpathlineto{\pgfqpoint{1.560722in}{0.467407in}}%
\pgfpathlineto{\pgfqpoint{1.565971in}{0.478765in}}%
\pgfpathlineto{\pgfqpoint{1.567465in}{0.481018in}}%
\pgfpathlineto{\pgfqpoint{1.570096in}{0.494629in}}%
\pgfpathlineto{\pgfqpoint{1.566149in}{0.508240in}}%
\pgfpathlineto{\pgfqpoint{1.565971in}{0.508476in}}%
\pgfpathlineto{\pgfqpoint{1.558930in}{0.521851in}}%
\pgfpathlineto{\pgfqpoint{1.550314in}{0.532044in}}%
\pgfpathlineto{\pgfqpoint{1.547840in}{0.535462in}}%
\pgfpathlineto{\pgfqpoint{1.534773in}{0.549073in}}%
\pgfpathlineto{\pgfqpoint{1.534657in}{0.549177in}}%
\pgfpathlineto{\pgfqpoint{1.519809in}{0.562684in}}%
\pgfpathlineto{\pgfqpoint{1.519001in}{0.563380in}}%
\pgfpathlineto{\pgfqpoint{1.503344in}{0.575674in}}%
\pgfpathlineto{\pgfqpoint{1.502376in}{0.576295in}}%
\pgfpathlineto{\pgfqpoint{1.487688in}{0.586412in}}%
\pgfpathlineto{\pgfqpoint{1.480514in}{0.589907in}}%
\pgfpathlineto{\pgfqpoint{1.472031in}{0.595076in}}%
\pgfpathlineto{\pgfqpoint{1.456375in}{0.600366in}}%
\pgfpathlineto{\pgfqpoint{1.440718in}{0.601246in}}%
\pgfpathlineto{\pgfqpoint{1.425061in}{0.597722in}}%
\pgfpathlineto{\pgfqpoint{1.409652in}{0.589907in}}%
\pgfpathlineto{\pgfqpoint{1.409405in}{0.589807in}}%
\pgfpathlineto{\pgfqpoint{1.393748in}{0.580312in}}%
\pgfpathlineto{\pgfqpoint{1.388598in}{0.576295in}}%
\pgfpathlineto{\pgfqpoint{1.378092in}{0.568599in}}%
\pgfpathlineto{\pgfqpoint{1.371055in}{0.562684in}}%
\pgfpathlineto{\pgfqpoint{1.362435in}{0.555046in}}%
\pgfpathlineto{\pgfqpoint{1.355900in}{0.549073in}}%
\pgfpathlineto{\pgfqpoint{1.346779in}{0.539299in}}%
\pgfpathlineto{\pgfqpoint{1.342865in}{0.535462in}}%
\pgfpathlineto{\pgfqpoint{1.332305in}{0.521851in}}%
\pgfpathlineto{\pgfqpoint{1.331122in}{0.519430in}}%
\pgfpathlineto{\pgfqpoint{1.323956in}{0.508240in}}%
\pgfpathlineto{\pgfqpoint{1.320677in}{0.494629in}}%
\pgfpathlineto{\pgfqpoint{1.322863in}{0.481018in}}%
\pgfpathlineto{\pgfqpoint{1.330526in}{0.467407in}}%
\pgfpathlineto{\pgfqpoint{1.331122in}{0.466764in}}%
\pgfpathlineto{\pgfqpoint{1.340431in}{0.453795in}}%
\pgfpathlineto{\pgfqpoint{1.346779in}{0.447310in}}%
\pgfpathlineto{\pgfqpoint{1.353143in}{0.440184in}}%
\pgfpathlineto{\pgfqpoint{1.362435in}{0.431506in}}%
\pgfpathlineto{\pgfqpoint{1.367876in}{0.426573in}}%
\pgfpathlineto{\pgfqpoint{1.378092in}{0.417963in}}%
\pgfpathlineto{\pgfqpoint{1.384900in}{0.412962in}}%
\pgfpathlineto{\pgfqpoint{1.393748in}{0.406196in}}%
\pgfpathlineto{\pgfqpoint{1.405285in}{0.399351in}}%
\pgfpathlineto{\pgfqpoint{1.409405in}{0.396419in}}%
\pgfpathlineto{\pgfqpoint{1.425061in}{0.388987in}}%
\pgfpathlineto{\pgfqpoint{1.440484in}{0.385740in}}%
\pgfpathlineto{\pgfqpoint{1.440718in}{0.385664in}}%
\pgfpathclose%
\pgfpathmoveto{\pgfqpoint{1.402538in}{0.426573in}}%
\pgfpathlineto{\pgfqpoint{1.393748in}{0.432023in}}%
\pgfpathlineto{\pgfqpoint{1.383505in}{0.440184in}}%
\pgfpathlineto{\pgfqpoint{1.378092in}{0.445299in}}%
\pgfpathlineto{\pgfqpoint{1.370130in}{0.453795in}}%
\pgfpathlineto{\pgfqpoint{1.362435in}{0.465026in}}%
\pgfpathlineto{\pgfqpoint{1.360812in}{0.467407in}}%
\pgfpathlineto{\pgfqpoint{1.355211in}{0.481018in}}%
\pgfpathlineto{\pgfqpoint{1.353614in}{0.494629in}}%
\pgfpathlineto{\pgfqpoint{1.356010in}{0.508240in}}%
\pgfpathlineto{\pgfqpoint{1.362414in}{0.521851in}}%
\pgfpathlineto{\pgfqpoint{1.362435in}{0.521880in}}%
\pgfpathlineto{\pgfqpoint{1.372440in}{0.535462in}}%
\pgfpathlineto{\pgfqpoint{1.378092in}{0.541248in}}%
\pgfpathlineto{\pgfqpoint{1.386745in}{0.549073in}}%
\pgfpathlineto{\pgfqpoint{1.393748in}{0.554534in}}%
\pgfpathlineto{\pgfqpoint{1.407199in}{0.562684in}}%
\pgfpathlineto{\pgfqpoint{1.409405in}{0.563980in}}%
\pgfpathlineto{\pgfqpoint{1.425061in}{0.570147in}}%
\pgfpathlineto{\pgfqpoint{1.440718in}{0.572882in}}%
\pgfpathlineto{\pgfqpoint{1.456375in}{0.572199in}}%
\pgfpathlineto{\pgfqpoint{1.472031in}{0.568093in}}%
\pgfpathlineto{\pgfqpoint{1.483383in}{0.562684in}}%
\pgfpathlineto{\pgfqpoint{1.487688in}{0.560588in}}%
\pgfpathlineto{\pgfqpoint{1.503344in}{0.549849in}}%
\pgfpathlineto{\pgfqpoint{1.504279in}{0.549073in}}%
\pgfpathlineto{\pgfqpoint{1.518406in}{0.535462in}}%
\pgfpathlineto{\pgfqpoint{1.519001in}{0.534708in}}%
\pgfpathlineto{\pgfqpoint{1.528473in}{0.521851in}}%
\pgfpathlineto{\pgfqpoint{1.534657in}{0.508241in}}%
\pgfpathlineto{\pgfqpoint{1.534658in}{0.508240in}}%
\pgfpathlineto{\pgfqpoint{1.537164in}{0.494629in}}%
\pgfpathlineto{\pgfqpoint{1.535493in}{0.481018in}}%
\pgfpathlineto{\pgfqpoint{1.534657in}{0.479032in}}%
\pgfpathlineto{\pgfqpoint{1.530020in}{0.467407in}}%
\pgfpathlineto{\pgfqpoint{1.520737in}{0.453795in}}%
\pgfpathlineto{\pgfqpoint{1.519001in}{0.451897in}}%
\pgfpathlineto{\pgfqpoint{1.507396in}{0.440184in}}%
\pgfpathlineto{\pgfqpoint{1.503344in}{0.436750in}}%
\pgfpathlineto{\pgfqpoint{1.488687in}{0.426573in}}%
\pgfpathlineto{\pgfqpoint{1.487688in}{0.425918in}}%
\pgfpathlineto{\pgfqpoint{1.472031in}{0.418464in}}%
\pgfpathlineto{\pgfqpoint{1.456375in}{0.414405in}}%
\pgfpathlineto{\pgfqpoint{1.440718in}{0.413729in}}%
\pgfpathlineto{\pgfqpoint{1.425061in}{0.416433in}}%
\pgfpathlineto{\pgfqpoint{1.409405in}{0.422530in}}%
\pgfpathlineto{\pgfqpoint{1.402538in}{0.426573in}}%
\pgfpathclose%
\pgfpathmoveto{\pgfqpoint{1.753849in}{0.385408in}}%
\pgfpathlineto{\pgfqpoint{1.755890in}{0.385740in}}%
\pgfpathlineto{\pgfqpoint{1.769506in}{0.387174in}}%
\pgfpathlineto{\pgfqpoint{1.785162in}{0.392949in}}%
\pgfpathlineto{\pgfqpoint{1.795425in}{0.399351in}}%
\pgfpathlineto{\pgfqpoint{1.800819in}{0.402138in}}%
\pgfpathlineto{\pgfqpoint{1.816130in}{0.412962in}}%
\pgfpathlineto{\pgfqpoint{1.816476in}{0.413196in}}%
\pgfpathlineto{\pgfqpoint{1.832132in}{0.425835in}}%
\pgfpathlineto{\pgfqpoint{1.832981in}{0.426573in}}%
\pgfpathlineto{\pgfqpoint{1.847520in}{0.440184in}}%
\pgfpathlineto{\pgfqpoint{1.847789in}{0.440485in}}%
\pgfpathlineto{\pgfqpoint{1.860239in}{0.453795in}}%
\pgfpathlineto{\pgfqpoint{1.863445in}{0.458485in}}%
\pgfpathlineto{\pgfqpoint{1.870809in}{0.467407in}}%
\pgfpathlineto{\pgfqpoint{1.877453in}{0.481018in}}%
\pgfpathlineto{\pgfqpoint{1.879102in}{0.492855in}}%
\pgfpathlineto{\pgfqpoint{1.879484in}{0.494629in}}%
\pgfpathlineto{\pgfqpoint{1.879102in}{0.495815in}}%
\pgfpathlineto{\pgfqpoint{1.876505in}{0.508240in}}%
\pgfpathlineto{\pgfqpoint{1.868909in}{0.521851in}}%
\pgfpathlineto{\pgfqpoint{1.863445in}{0.528027in}}%
\pgfpathlineto{\pgfqpoint{1.857979in}{0.535462in}}%
\pgfpathlineto{\pgfqpoint{1.847789in}{0.545914in}}%
\pgfpathlineto{\pgfqpoint{1.844837in}{0.549073in}}%
\pgfpathlineto{\pgfqpoint{1.832132in}{0.560671in}}%
\pgfpathlineto{\pgfqpoint{1.829762in}{0.562684in}}%
\pgfpathlineto{\pgfqpoint{1.816476in}{0.573422in}}%
\pgfpathlineto{\pgfqpoint{1.812212in}{0.576295in}}%
\pgfpathlineto{\pgfqpoint{1.800819in}{0.584512in}}%
\pgfpathlineto{\pgfqpoint{1.790601in}{0.589907in}}%
\pgfpathlineto{\pgfqpoint{1.785162in}{0.593487in}}%
\pgfpathlineto{\pgfqpoint{1.769506in}{0.599661in}}%
\pgfpathlineto{\pgfqpoint{1.753849in}{0.601422in}}%
\pgfpathlineto{\pgfqpoint{1.738193in}{0.598780in}}%
\pgfpathlineto{\pgfqpoint{1.722536in}{0.591720in}}%
\pgfpathlineto{\pgfqpoint{1.719966in}{0.589907in}}%
\pgfpathlineto{\pgfqpoint{1.706880in}{0.582478in}}%
\pgfpathlineto{\pgfqpoint{1.698654in}{0.576295in}}%
\pgfpathlineto{\pgfqpoint{1.691223in}{0.571062in}}%
\pgfpathlineto{\pgfqpoint{1.681077in}{0.562684in}}%
\pgfpathlineto{\pgfqpoint{1.675567in}{0.557894in}}%
\pgfpathlineto{\pgfqpoint{1.665930in}{0.549073in}}%
\pgfpathlineto{\pgfqpoint{1.659910in}{0.542613in}}%
\pgfpathlineto{\pgfqpoint{1.652798in}{0.535462in}}%
\pgfpathlineto{\pgfqpoint{1.644253in}{0.524086in}}%
\pgfpathlineto{\pgfqpoint{1.642167in}{0.521851in}}%
\pgfpathlineto{\pgfqpoint{1.634046in}{0.508240in}}%
\pgfpathlineto{\pgfqpoint{1.631007in}{0.494629in}}%
\pgfpathlineto{\pgfqpoint{1.633033in}{0.481018in}}%
\pgfpathlineto{\pgfqpoint{1.640135in}{0.467407in}}%
\pgfpathlineto{\pgfqpoint{1.644253in}{0.462679in}}%
\pgfpathlineto{\pgfqpoint{1.650459in}{0.453795in}}%
\pgfpathlineto{\pgfqpoint{1.659910in}{0.443891in}}%
\pgfpathlineto{\pgfqpoint{1.663215in}{0.440184in}}%
\pgfpathlineto{\pgfqpoint{1.675567in}{0.428634in}}%
\pgfpathlineto{\pgfqpoint{1.677883in}{0.426573in}}%
\pgfpathlineto{\pgfqpoint{1.691223in}{0.415528in}}%
\pgfpathlineto{\pgfqpoint{1.694857in}{0.412962in}}%
\pgfpathlineto{\pgfqpoint{1.706880in}{0.404103in}}%
\pgfpathlineto{\pgfqpoint{1.715432in}{0.399351in}}%
\pgfpathlineto{\pgfqpoint{1.722536in}{0.394601in}}%
\pgfpathlineto{\pgfqpoint{1.738193in}{0.387998in}}%
\pgfpathlineto{\pgfqpoint{1.752484in}{0.385740in}}%
\pgfpathlineto{\pgfqpoint{1.753849in}{0.385408in}}%
\pgfpathclose%
\pgfpathmoveto{\pgfqpoint{1.712530in}{0.426573in}}%
\pgfpathlineto{\pgfqpoint{1.706880in}{0.429854in}}%
\pgfpathlineto{\pgfqpoint{1.693425in}{0.440184in}}%
\pgfpathlineto{\pgfqpoint{1.691223in}{0.442184in}}%
\pgfpathlineto{\pgfqpoint{1.680147in}{0.453795in}}%
\pgfpathlineto{\pgfqpoint{1.675567in}{0.460354in}}%
\pgfpathlineto{\pgfqpoint{1.670764in}{0.467407in}}%
\pgfpathlineto{\pgfqpoint{1.665251in}{0.481018in}}%
\pgfpathlineto{\pgfqpoint{1.663679in}{0.494629in}}%
\pgfpathlineto{\pgfqpoint{1.666038in}{0.508240in}}%
\pgfpathlineto{\pgfqpoint{1.672341in}{0.521851in}}%
\pgfpathlineto{\pgfqpoint{1.675567in}{0.526270in}}%
\pgfpathlineto{\pgfqpoint{1.682469in}{0.535462in}}%
\pgfpathlineto{\pgfqpoint{1.691223in}{0.544267in}}%
\pgfpathlineto{\pgfqpoint{1.696751in}{0.549073in}}%
\pgfpathlineto{\pgfqpoint{1.706880in}{0.556684in}}%
\pgfpathlineto{\pgfqpoint{1.717453in}{0.562684in}}%
\pgfpathlineto{\pgfqpoint{1.722536in}{0.565488in}}%
\pgfpathlineto{\pgfqpoint{1.738193in}{0.570968in}}%
\pgfpathlineto{\pgfqpoint{1.753849in}{0.573019in}}%
\pgfpathlineto{\pgfqpoint{1.769506in}{0.571652in}}%
\pgfpathlineto{\pgfqpoint{1.785162in}{0.566859in}}%
\pgfpathlineto{\pgfqpoint{1.793274in}{0.562684in}}%
\pgfpathlineto{\pgfqpoint{1.800819in}{0.558702in}}%
\pgfpathlineto{\pgfqpoint{1.814175in}{0.549073in}}%
\pgfpathlineto{\pgfqpoint{1.816476in}{0.547159in}}%
\pgfpathlineto{\pgfqpoint{1.828358in}{0.535462in}}%
\pgfpathlineto{\pgfqpoint{1.832132in}{0.530550in}}%
\pgfpathlineto{\pgfqpoint{1.838499in}{0.521851in}}%
\pgfpathlineto{\pgfqpoint{1.844730in}{0.508240in}}%
\pgfpathlineto{\pgfqpoint{1.847062in}{0.494629in}}%
\pgfpathlineto{\pgfqpoint{1.845507in}{0.481018in}}%
\pgfpathlineto{\pgfqpoint{1.840058in}{0.467407in}}%
\pgfpathlineto{\pgfqpoint{1.832132in}{0.455799in}}%
\pgfpathlineto{\pgfqpoint{1.830700in}{0.453795in}}%
\pgfpathlineto{\pgfqpoint{1.817490in}{0.440184in}}%
\pgfpathlineto{\pgfqpoint{1.816476in}{0.439302in}}%
\pgfpathlineto{\pgfqpoint{1.800819in}{0.427818in}}%
\pgfpathlineto{\pgfqpoint{1.798514in}{0.426573in}}%
\pgfpathlineto{\pgfqpoint{1.785162in}{0.419683in}}%
\pgfpathlineto{\pgfqpoint{1.769506in}{0.414945in}}%
\pgfpathlineto{\pgfqpoint{1.753849in}{0.413594in}}%
\pgfpathlineto{\pgfqpoint{1.738193in}{0.415621in}}%
\pgfpathlineto{\pgfqpoint{1.722536in}{0.421039in}}%
\pgfpathlineto{\pgfqpoint{1.712530in}{0.426573in}}%
\pgfpathclose%
\pgfpathmoveto{\pgfqpoint{1.096274in}{0.398402in}}%
\pgfpathlineto{\pgfqpoint{1.111930in}{0.390142in}}%
\pgfpathlineto{\pgfqpoint{1.127587in}{0.386021in}}%
\pgfpathlineto{\pgfqpoint{1.143243in}{0.386021in}}%
\pgfpathlineto{\pgfqpoint{1.158900in}{0.390142in}}%
\pgfpathlineto{\pgfqpoint{1.174556in}{0.398402in}}%
\pgfpathlineto{\pgfqpoint{1.175817in}{0.399351in}}%
\pgfpathlineto{\pgfqpoint{1.190213in}{0.408415in}}%
\pgfpathlineto{\pgfqpoint{1.195967in}{0.412962in}}%
\pgfpathlineto{\pgfqpoint{1.205870in}{0.420497in}}%
\pgfpathlineto{\pgfqpoint{1.212979in}{0.426573in}}%
\pgfpathlineto{\pgfqpoint{1.221526in}{0.434441in}}%
\pgfpathlineto{\pgfqpoint{1.227710in}{0.440184in}}%
\pgfpathlineto{\pgfqpoint{1.237183in}{0.450726in}}%
\pgfpathlineto{\pgfqpoint{1.240283in}{0.453795in}}%
\pgfpathlineto{\pgfqpoint{1.250488in}{0.467407in}}%
\pgfpathlineto{\pgfqpoint{1.252839in}{0.472706in}}%
\pgfpathlineto{\pgfqpoint{1.257891in}{0.481018in}}%
\pgfpathlineto{\pgfqpoint{1.260274in}{0.494629in}}%
\pgfpathlineto{\pgfqpoint{1.256699in}{0.508240in}}%
\pgfpathlineto{\pgfqpoint{1.252839in}{0.513834in}}%
\pgfpathlineto{\pgfqpoint{1.248787in}{0.521851in}}%
\pgfpathlineto{\pgfqpoint{1.237736in}{0.535462in}}%
\pgfpathlineto{\pgfqpoint{1.237183in}{0.535987in}}%
\pgfpathlineto{\pgfqpoint{1.224897in}{0.549073in}}%
\pgfpathlineto{\pgfqpoint{1.221526in}{0.552137in}}%
\pgfpathlineto{\pgfqpoint{1.209802in}{0.562684in}}%
\pgfpathlineto{\pgfqpoint{1.205870in}{0.566036in}}%
\pgfpathlineto{\pgfqpoint{1.192348in}{0.576295in}}%
\pgfpathlineto{\pgfqpoint{1.190213in}{0.578016in}}%
\pgfpathlineto{\pgfqpoint{1.174556in}{0.588177in}}%
\pgfpathlineto{\pgfqpoint{1.170673in}{0.589907in}}%
\pgfpathlineto{\pgfqpoint{1.158900in}{0.596488in}}%
\pgfpathlineto{\pgfqpoint{1.143243in}{0.600894in}}%
\pgfpathlineto{\pgfqpoint{1.127587in}{0.600894in}}%
\pgfpathlineto{\pgfqpoint{1.111930in}{0.596488in}}%
\pgfpathlineto{\pgfqpoint{1.100157in}{0.589907in}}%
\pgfpathlineto{\pgfqpoint{1.096274in}{0.588177in}}%
\pgfpathlineto{\pgfqpoint{1.080617in}{0.578016in}}%
\pgfpathlineto{\pgfqpoint{1.078482in}{0.576295in}}%
\pgfpathlineto{\pgfqpoint{1.064960in}{0.566036in}}%
\pgfpathlineto{\pgfqpoint{1.061028in}{0.562684in}}%
\pgfpathlineto{\pgfqpoint{1.049304in}{0.552137in}}%
\pgfpathlineto{\pgfqpoint{1.045933in}{0.549073in}}%
\pgfpathlineto{\pgfqpoint{1.033647in}{0.535987in}}%
\pgfpathlineto{\pgfqpoint{1.033094in}{0.535462in}}%
\pgfpathlineto{\pgfqpoint{1.022043in}{0.521851in}}%
\pgfpathlineto{\pgfqpoint{1.017991in}{0.513834in}}%
\pgfpathlineto{\pgfqpoint{1.014131in}{0.508240in}}%
\pgfpathlineto{\pgfqpoint{1.010556in}{0.494629in}}%
\pgfpathlineto{\pgfqpoint{1.012939in}{0.481018in}}%
\pgfpathlineto{\pgfqpoint{1.017991in}{0.472706in}}%
\pgfpathlineto{\pgfqpoint{1.020342in}{0.467407in}}%
\pgfpathlineto{\pgfqpoint{1.030547in}{0.453795in}}%
\pgfpathlineto{\pgfqpoint{1.033647in}{0.450726in}}%
\pgfpathlineto{\pgfqpoint{1.043120in}{0.440184in}}%
\pgfpathlineto{\pgfqpoint{1.049304in}{0.434441in}}%
\pgfpathlineto{\pgfqpoint{1.057851in}{0.426573in}}%
\pgfpathlineto{\pgfqpoint{1.064960in}{0.420497in}}%
\pgfpathlineto{\pgfqpoint{1.074863in}{0.412962in}}%
\pgfpathlineto{\pgfqpoint{1.080617in}{0.408415in}}%
\pgfpathlineto{\pgfqpoint{1.095013in}{0.399351in}}%
\pgfpathlineto{\pgfqpoint{1.096274in}{0.398402in}}%
\pgfpathclose%
\pgfpathmoveto{\pgfqpoint{1.092395in}{0.426573in}}%
\pgfpathlineto{\pgfqpoint{1.080617in}{0.434322in}}%
\pgfpathlineto{\pgfqpoint{1.073498in}{0.440184in}}%
\pgfpathlineto{\pgfqpoint{1.064960in}{0.448539in}}%
\pgfpathlineto{\pgfqpoint{1.060103in}{0.453795in}}%
\pgfpathlineto{\pgfqpoint{1.050851in}{0.467407in}}%
\pgfpathlineto{\pgfqpoint{1.049304in}{0.471242in}}%
\pgfpathlineto{\pgfqpoint{1.045230in}{0.481018in}}%
\pgfpathlineto{\pgfqpoint{1.043600in}{0.494629in}}%
\pgfpathlineto{\pgfqpoint{1.046045in}{0.508240in}}%
\pgfpathlineto{\pgfqpoint{1.049304in}{0.515128in}}%
\pgfpathlineto{\pgfqpoint{1.052393in}{0.521851in}}%
\pgfpathlineto{\pgfqpoint{1.062413in}{0.535462in}}%
\pgfpathlineto{\pgfqpoint{1.064960in}{0.538108in}}%
\pgfpathlineto{\pgfqpoint{1.076669in}{0.549073in}}%
\pgfpathlineto{\pgfqpoint{1.080617in}{0.552255in}}%
\pgfpathlineto{\pgfqpoint{1.096274in}{0.562340in}}%
\pgfpathlineto{\pgfqpoint{1.097046in}{0.562684in}}%
\pgfpathlineto{\pgfqpoint{1.111930in}{0.569189in}}%
\pgfpathlineto{\pgfqpoint{1.127587in}{0.572609in}}%
\pgfpathlineto{\pgfqpoint{1.143243in}{0.572609in}}%
\pgfpathlineto{\pgfqpoint{1.158900in}{0.569189in}}%
\pgfpathlineto{\pgfqpoint{1.173784in}{0.562684in}}%
\pgfpathlineto{\pgfqpoint{1.174556in}{0.562340in}}%
\pgfpathlineto{\pgfqpoint{1.190213in}{0.552255in}}%
\pgfpathlineto{\pgfqpoint{1.194161in}{0.549073in}}%
\pgfpathlineto{\pgfqpoint{1.205870in}{0.538108in}}%
\pgfpathlineto{\pgfqpoint{1.208417in}{0.535462in}}%
\pgfpathlineto{\pgfqpoint{1.218437in}{0.521851in}}%
\pgfpathlineto{\pgfqpoint{1.221526in}{0.515128in}}%
\pgfpathlineto{\pgfqpoint{1.224785in}{0.508240in}}%
\pgfpathlineto{\pgfqpoint{1.227230in}{0.494629in}}%
\pgfpathlineto{\pgfqpoint{1.225600in}{0.481018in}}%
\pgfpathlineto{\pgfqpoint{1.221526in}{0.471242in}}%
\pgfpathlineto{\pgfqpoint{1.219979in}{0.467407in}}%
\pgfpathlineto{\pgfqpoint{1.210727in}{0.453795in}}%
\pgfpathlineto{\pgfqpoint{1.205870in}{0.448539in}}%
\pgfpathlineto{\pgfqpoint{1.197332in}{0.440184in}}%
\pgfpathlineto{\pgfqpoint{1.190213in}{0.434322in}}%
\pgfpathlineto{\pgfqpoint{1.178435in}{0.426573in}}%
\pgfpathlineto{\pgfqpoint{1.174556in}{0.424156in}}%
\pgfpathlineto{\pgfqpoint{1.158900in}{0.417381in}}%
\pgfpathlineto{\pgfqpoint{1.143243in}{0.413999in}}%
\pgfpathlineto{\pgfqpoint{1.127587in}{0.413999in}}%
\pgfpathlineto{\pgfqpoint{1.111930in}{0.417381in}}%
\pgfpathlineto{\pgfqpoint{1.096274in}{0.424156in}}%
\pgfpathlineto{\pgfqpoint{1.092395in}{0.426573in}}%
\pgfpathclose%
\pgfpathmoveto{\pgfqpoint{0.501324in}{0.656663in}}%
\pgfpathlineto{\pgfqpoint{0.516981in}{0.654376in}}%
\pgfpathlineto{\pgfqpoint{0.532637in}{0.657807in}}%
\pgfpathlineto{\pgfqpoint{0.532909in}{0.657962in}}%
\pgfpathlineto{\pgfqpoint{0.548294in}{0.664083in}}%
\pgfpathlineto{\pgfqpoint{0.560018in}{0.671573in}}%
\pgfpathlineto{\pgfqpoint{0.563950in}{0.673724in}}%
\pgfpathlineto{\pgfqpoint{0.579607in}{0.685084in}}%
\pgfpathlineto{\pgfqpoint{0.579726in}{0.685184in}}%
\pgfpathlineto{\pgfqpoint{0.595263in}{0.698093in}}%
\pgfpathlineto{\pgfqpoint{0.596064in}{0.698795in}}%
\pgfpathlineto{\pgfqpoint{0.610205in}{0.712407in}}%
\pgfpathlineto{\pgfqpoint{0.610920in}{0.713248in}}%
\pgfpathlineto{\pgfqpoint{0.622557in}{0.726018in}}%
\pgfpathlineto{\pgfqpoint{0.626577in}{0.732254in}}%
\pgfpathlineto{\pgfqpoint{0.632523in}{0.739629in}}%
\pgfpathlineto{\pgfqpoint{0.638608in}{0.753240in}}%
\pgfpathlineto{\pgfqpoint{0.639620in}{0.766851in}}%
\pgfpathlineto{\pgfqpoint{0.635567in}{0.780462in}}%
\pgfpathlineto{\pgfqpoint{0.626577in}{0.793858in}}%
\pgfpathlineto{\pgfqpoint{0.626462in}{0.794073in}}%
\pgfpathlineto{\pgfqpoint{0.615541in}{0.807684in}}%
\pgfpathlineto{\pgfqpoint{0.610920in}{0.812161in}}%
\pgfpathlineto{\pgfqpoint{0.602067in}{0.821295in}}%
\pgfpathlineto{\pgfqpoint{0.595263in}{0.827413in}}%
\pgfpathlineto{\pgfqpoint{0.586478in}{0.834907in}}%
\pgfpathlineto{\pgfqpoint{0.579607in}{0.840588in}}%
\pgfpathlineto{\pgfqpoint{0.568363in}{0.848518in}}%
\pgfpathlineto{\pgfqpoint{0.563950in}{0.851920in}}%
\pgfpathlineto{\pgfqpoint{0.548294in}{0.861100in}}%
\pgfpathlineto{\pgfqpoint{0.545510in}{0.862129in}}%
\pgfpathlineto{\pgfqpoint{0.532637in}{0.868359in}}%
\pgfpathlineto{\pgfqpoint{0.516981in}{0.871209in}}%
\pgfpathlineto{\pgfqpoint{0.501324in}{0.869309in}}%
\pgfpathlineto{\pgfqpoint{0.485668in}{0.862647in}}%
\pgfpathlineto{\pgfqpoint{0.484928in}{0.862129in}}%
\pgfpathlineto{\pgfqpoint{0.470011in}{0.854036in}}%
\pgfpathlineto{\pgfqpoint{0.462551in}{0.848518in}}%
\pgfpathlineto{\pgfqpoint{0.454354in}{0.842985in}}%
\pgfpathlineto{\pgfqpoint{0.444372in}{0.834907in}}%
\pgfpathlineto{\pgfqpoint{0.438698in}{0.830177in}}%
\pgfpathlineto{\pgfqpoint{0.428794in}{0.821295in}}%
\pgfpathlineto{\pgfqpoint{0.423041in}{0.815377in}}%
\pgfpathlineto{\pgfqpoint{0.415258in}{0.807684in}}%
\pgfpathlineto{\pgfqpoint{0.407385in}{0.797654in}}%
\pgfpathlineto{\pgfqpoint{0.404012in}{0.794073in}}%
\pgfpathlineto{\pgfqpoint{0.395464in}{0.780462in}}%
\pgfpathlineto{\pgfqpoint{0.391728in}{0.767055in}}%
\pgfpathlineto{\pgfqpoint{0.391641in}{0.766851in}}%
\pgfpathlineto{\pgfqpoint{0.391728in}{0.766044in}}%
\pgfpathlineto{\pgfqpoint{0.392619in}{0.753240in}}%
\pgfpathlineto{\pgfqpoint{0.398311in}{0.739629in}}%
\pgfpathlineto{\pgfqpoint{0.407385in}{0.727753in}}%
\pgfpathlineto{\pgfqpoint{0.408479in}{0.726018in}}%
\pgfpathlineto{\pgfqpoint{0.420505in}{0.712407in}}%
\pgfpathlineto{\pgfqpoint{0.423041in}{0.710051in}}%
\pgfpathlineto{\pgfqpoint{0.434728in}{0.698795in}}%
\pgfpathlineto{\pgfqpoint{0.438698in}{0.695322in}}%
\pgfpathlineto{\pgfqpoint{0.451183in}{0.685184in}}%
\pgfpathlineto{\pgfqpoint{0.454354in}{0.682578in}}%
\pgfpathlineto{\pgfqpoint{0.470011in}{0.671815in}}%
\pgfpathlineto{\pgfqpoint{0.470487in}{0.671573in}}%
\pgfpathlineto{\pgfqpoint{0.485668in}{0.662525in}}%
\pgfpathlineto{\pgfqpoint{0.498733in}{0.657962in}}%
\pgfpathlineto{\pgfqpoint{0.501324in}{0.656663in}}%
\pgfpathclose%
\pgfpathmoveto{\pgfqpoint{0.499040in}{0.685184in}}%
\pgfpathlineto{\pgfqpoint{0.485668in}{0.689216in}}%
\pgfpathlineto{\pgfqpoint{0.470011in}{0.697286in}}%
\pgfpathlineto{\pgfqpoint{0.467827in}{0.698795in}}%
\pgfpathlineto{\pgfqpoint{0.454354in}{0.708884in}}%
\pgfpathlineto{\pgfqpoint{0.450404in}{0.712407in}}%
\pgfpathlineto{\pgfqpoint{0.438698in}{0.725149in}}%
\pgfpathlineto{\pgfqpoint{0.437944in}{0.726018in}}%
\pgfpathlineto{\pgfqpoint{0.429370in}{0.739629in}}%
\pgfpathlineto{\pgfqpoint{0.424701in}{0.753240in}}%
\pgfpathlineto{\pgfqpoint{0.423924in}{0.766851in}}%
\pgfpathlineto{\pgfqpoint{0.427034in}{0.780462in}}%
\pgfpathlineto{\pgfqpoint{0.434047in}{0.794073in}}%
\pgfpathlineto{\pgfqpoint{0.438698in}{0.800043in}}%
\pgfpathlineto{\pgfqpoint{0.444967in}{0.807684in}}%
\pgfpathlineto{\pgfqpoint{0.454354in}{0.816589in}}%
\pgfpathlineto{\pgfqpoint{0.460237in}{0.821295in}}%
\pgfpathlineto{\pgfqpoint{0.470011in}{0.828217in}}%
\pgfpathlineto{\pgfqpoint{0.482929in}{0.834907in}}%
\pgfpathlineto{\pgfqpoint{0.485668in}{0.836318in}}%
\pgfpathlineto{\pgfqpoint{0.501324in}{0.841187in}}%
\pgfpathlineto{\pgfqpoint{0.516981in}{0.842576in}}%
\pgfpathlineto{\pgfqpoint{0.532637in}{0.840492in}}%
\pgfpathlineto{\pgfqpoint{0.548294in}{0.834925in}}%
\pgfpathlineto{\pgfqpoint{0.548328in}{0.834907in}}%
\pgfpathlineto{\pgfqpoint{0.563950in}{0.826208in}}%
\pgfpathlineto{\pgfqpoint{0.570606in}{0.821295in}}%
\pgfpathlineto{\pgfqpoint{0.579607in}{0.813773in}}%
\pgfpathlineto{\pgfqpoint{0.585889in}{0.807684in}}%
\pgfpathlineto{\pgfqpoint{0.595263in}{0.795991in}}%
\pgfpathlineto{\pgfqpoint{0.596754in}{0.794073in}}%
\pgfpathlineto{\pgfqpoint{0.603848in}{0.780462in}}%
\pgfpathlineto{\pgfqpoint{0.606994in}{0.766851in}}%
\pgfpathlineto{\pgfqpoint{0.606208in}{0.753240in}}%
\pgfpathlineto{\pgfqpoint{0.601485in}{0.739629in}}%
\pgfpathlineto{\pgfqpoint{0.595263in}{0.729760in}}%
\pgfpathlineto{\pgfqpoint{0.592852in}{0.726018in}}%
\pgfpathlineto{\pgfqpoint{0.580499in}{0.712407in}}%
\pgfpathlineto{\pgfqpoint{0.579607in}{0.711594in}}%
\pgfpathlineto{\pgfqpoint{0.563950in}{0.699313in}}%
\pgfpathlineto{\pgfqpoint{0.563083in}{0.698795in}}%
\pgfpathlineto{\pgfqpoint{0.548294in}{0.690561in}}%
\pgfpathlineto{\pgfqpoint{0.532638in}{0.685184in}}%
\pgfpathlineto{\pgfqpoint{0.532637in}{0.685184in}}%
\pgfpathlineto{\pgfqpoint{0.516981in}{0.683005in}}%
\pgfpathlineto{\pgfqpoint{0.501324in}{0.684457in}}%
\pgfpathlineto{\pgfqpoint{0.499040in}{0.685184in}}%
\pgfpathclose%
\pgfpathmoveto{\pgfqpoint{0.814455in}{0.655748in}}%
\pgfpathlineto{\pgfqpoint{0.830112in}{0.654604in}}%
\pgfpathlineto{\pgfqpoint{0.841643in}{0.657962in}}%
\pgfpathlineto{\pgfqpoint{0.845769in}{0.658790in}}%
\pgfpathlineto{\pgfqpoint{0.861425in}{0.665796in}}%
\pgfpathlineto{\pgfqpoint{0.869919in}{0.671573in}}%
\pgfpathlineto{\pgfqpoint{0.877082in}{0.675756in}}%
\pgfpathlineto{\pgfqpoint{0.889596in}{0.685184in}}%
\pgfpathlineto{\pgfqpoint{0.892738in}{0.687509in}}%
\pgfpathlineto{\pgfqpoint{0.906043in}{0.698795in}}%
\pgfpathlineto{\pgfqpoint{0.908395in}{0.700986in}}%
\pgfpathlineto{\pgfqpoint{0.920319in}{0.712407in}}%
\pgfpathlineto{\pgfqpoint{0.924051in}{0.716794in}}%
\pgfpathlineto{\pgfqpoint{0.932673in}{0.726018in}}%
\pgfpathlineto{\pgfqpoint{0.939708in}{0.736639in}}%
\pgfpathlineto{\pgfqpoint{0.942277in}{0.739629in}}%
\pgfpathlineto{\pgfqpoint{0.948842in}{0.753240in}}%
\pgfpathlineto{\pgfqpoint{0.949934in}{0.766851in}}%
\pgfpathlineto{\pgfqpoint{0.945561in}{0.780462in}}%
\pgfpathlineto{\pgfqpoint{0.939708in}{0.788648in}}%
\pgfpathlineto{\pgfqpoint{0.936737in}{0.794073in}}%
\pgfpathlineto{\pgfqpoint{0.925373in}{0.807684in}}%
\pgfpathlineto{\pgfqpoint{0.924051in}{0.808933in}}%
\pgfpathlineto{\pgfqpoint{0.912051in}{0.821295in}}%
\pgfpathlineto{\pgfqpoint{0.908395in}{0.824579in}}%
\pgfpathlineto{\pgfqpoint{0.896515in}{0.834907in}}%
\pgfpathlineto{\pgfqpoint{0.892738in}{0.838085in}}%
\pgfpathlineto{\pgfqpoint{0.878518in}{0.848518in}}%
\pgfpathlineto{\pgfqpoint{0.877082in}{0.849667in}}%
\pgfpathlineto{\pgfqpoint{0.861425in}{0.859546in}}%
\pgfpathlineto{\pgfqpoint{0.855184in}{0.862129in}}%
\pgfpathlineto{\pgfqpoint{0.845769in}{0.867217in}}%
\pgfpathlineto{\pgfqpoint{0.830112in}{0.871019in}}%
\pgfpathlineto{\pgfqpoint{0.814455in}{0.870069in}}%
\pgfpathlineto{\pgfqpoint{0.798799in}{0.864362in}}%
\pgfpathlineto{\pgfqpoint{0.795360in}{0.862129in}}%
\pgfpathlineto{\pgfqpoint{0.783142in}{0.856013in}}%
\pgfpathlineto{\pgfqpoint{0.772533in}{0.848518in}}%
\pgfpathlineto{\pgfqpoint{0.767486in}{0.845273in}}%
\pgfpathlineto{\pgfqpoint{0.754349in}{0.834907in}}%
\pgfpathlineto{\pgfqpoint{0.751829in}{0.832862in}}%
\pgfpathlineto{\pgfqpoint{0.738846in}{0.821295in}}%
\pgfpathlineto{\pgfqpoint{0.736173in}{0.818564in}}%
\pgfpathlineto{\pgfqpoint{0.725327in}{0.807684in}}%
\pgfpathlineto{\pgfqpoint{0.720516in}{0.801457in}}%
\pgfpathlineto{\pgfqpoint{0.713871in}{0.794073in}}%
\pgfpathlineto{\pgfqpoint{0.705812in}{0.780462in}}%
\pgfpathlineto{\pgfqpoint{0.704859in}{0.776875in}}%
\pgfpathlineto{\pgfqpoint{0.700997in}{0.766851in}}%
\pgfpathlineto{\pgfqpoint{0.702312in}{0.753240in}}%
\pgfpathlineto{\pgfqpoint{0.704859in}{0.748788in}}%
\pgfpathlineto{\pgfqpoint{0.708496in}{0.739629in}}%
\pgfpathlineto{\pgfqpoint{0.718350in}{0.726018in}}%
\pgfpathlineto{\pgfqpoint{0.720516in}{0.723856in}}%
\pgfpathlineto{\pgfqpoint{0.730422in}{0.712407in}}%
\pgfpathlineto{\pgfqpoint{0.736173in}{0.706985in}}%
\pgfpathlineto{\pgfqpoint{0.744737in}{0.698795in}}%
\pgfpathlineto{\pgfqpoint{0.751829in}{0.692630in}}%
\pgfpathlineto{\pgfqpoint{0.761249in}{0.685184in}}%
\pgfpathlineto{\pgfqpoint{0.767486in}{0.680185in}}%
\pgfpathlineto{\pgfqpoint{0.780656in}{0.671573in}}%
\pgfpathlineto{\pgfqpoint{0.783142in}{0.669690in}}%
\pgfpathlineto{\pgfqpoint{0.798799in}{0.661124in}}%
\pgfpathlineto{\pgfqpoint{0.809335in}{0.657962in}}%
\pgfpathlineto{\pgfqpoint{0.814455in}{0.655748in}}%
\pgfpathclose%
\pgfpathmoveto{\pgfqpoint{0.809689in}{0.685184in}}%
\pgfpathlineto{\pgfqpoint{0.798799in}{0.688006in}}%
\pgfpathlineto{\pgfqpoint{0.783142in}{0.695404in}}%
\pgfpathlineto{\pgfqpoint{0.778004in}{0.698795in}}%
\pgfpathlineto{\pgfqpoint{0.767486in}{0.706298in}}%
\pgfpathlineto{\pgfqpoint{0.760459in}{0.712407in}}%
\pgfpathlineto{\pgfqpoint{0.751829in}{0.721551in}}%
\pgfpathlineto{\pgfqpoint{0.747928in}{0.726018in}}%
\pgfpathlineto{\pgfqpoint{0.739418in}{0.739629in}}%
\pgfpathlineto{\pgfqpoint{0.736173in}{0.749096in}}%
\pgfpathlineto{\pgfqpoint{0.734668in}{0.753240in}}%
\pgfpathlineto{\pgfqpoint{0.733833in}{0.766851in}}%
\pgfpathlineto{\pgfqpoint{0.736173in}{0.776419in}}%
\pgfpathlineto{\pgfqpoint{0.737099in}{0.780462in}}%
\pgfpathlineto{\pgfqpoint{0.744060in}{0.794073in}}%
\pgfpathlineto{\pgfqpoint{0.751829in}{0.803980in}}%
\pgfpathlineto{\pgfqpoint{0.754951in}{0.807684in}}%
\pgfpathlineto{\pgfqpoint{0.767486in}{0.819278in}}%
\pgfpathlineto{\pgfqpoint{0.770133in}{0.821295in}}%
\pgfpathlineto{\pgfqpoint{0.783142in}{0.830094in}}%
\pgfpathlineto{\pgfqpoint{0.793214in}{0.834907in}}%
\pgfpathlineto{\pgfqpoint{0.798799in}{0.837571in}}%
\pgfpathlineto{\pgfqpoint{0.814455in}{0.841743in}}%
\pgfpathlineto{\pgfqpoint{0.830112in}{0.842437in}}%
\pgfpathlineto{\pgfqpoint{0.845769in}{0.839658in}}%
\pgfpathlineto{\pgfqpoint{0.857734in}{0.834907in}}%
\pgfpathlineto{\pgfqpoint{0.861425in}{0.833449in}}%
\pgfpathlineto{\pgfqpoint{0.877082in}{0.824069in}}%
\pgfpathlineto{\pgfqpoint{0.880702in}{0.821295in}}%
\pgfpathlineto{\pgfqpoint{0.892738in}{0.810832in}}%
\pgfpathlineto{\pgfqpoint{0.895929in}{0.807684in}}%
\pgfpathlineto{\pgfqpoint{0.906718in}{0.794073in}}%
\pgfpathlineto{\pgfqpoint{0.908395in}{0.790865in}}%
\pgfpathlineto{\pgfqpoint{0.913860in}{0.780462in}}%
\pgfpathlineto{\pgfqpoint{0.917057in}{0.766851in}}%
\pgfpathlineto{\pgfqpoint{0.916258in}{0.753240in}}%
\pgfpathlineto{\pgfqpoint{0.911460in}{0.739629in}}%
\pgfpathlineto{\pgfqpoint{0.908395in}{0.734773in}}%
\pgfpathlineto{\pgfqpoint{0.902859in}{0.726018in}}%
\pgfpathlineto{\pgfqpoint{0.892738in}{0.714708in}}%
\pgfpathlineto{\pgfqpoint{0.890418in}{0.712407in}}%
\pgfpathlineto{\pgfqpoint{0.877082in}{0.701510in}}%
\pgfpathlineto{\pgfqpoint{0.872820in}{0.698795in}}%
\pgfpathlineto{\pgfqpoint{0.861425in}{0.692041in}}%
\pgfpathlineto{\pgfqpoint{0.845769in}{0.685990in}}%
\pgfpathlineto{\pgfqpoint{0.841118in}{0.685184in}}%
\pgfpathlineto{\pgfqpoint{0.830112in}{0.683151in}}%
\pgfpathlineto{\pgfqpoint{0.814455in}{0.683876in}}%
\pgfpathlineto{\pgfqpoint{0.809689in}{0.685184in}}%
\pgfpathclose%
\pgfpathmoveto{\pgfqpoint{1.127587in}{0.655062in}}%
\pgfpathlineto{\pgfqpoint{1.143243in}{0.655062in}}%
\pgfpathlineto{\pgfqpoint{1.151249in}{0.657962in}}%
\pgfpathlineto{\pgfqpoint{1.158900in}{0.659879in}}%
\pgfpathlineto{\pgfqpoint{1.174556in}{0.667666in}}%
\pgfpathlineto{\pgfqpoint{1.179986in}{0.671573in}}%
\pgfpathlineto{\pgfqpoint{1.190213in}{0.677910in}}%
\pgfpathlineto{\pgfqpoint{1.199555in}{0.685184in}}%
\pgfpathlineto{\pgfqpoint{1.205870in}{0.690023in}}%
\pgfpathlineto{\pgfqpoint{1.216065in}{0.698795in}}%
\pgfpathlineto{\pgfqpoint{1.221526in}{0.703959in}}%
\pgfpathlineto{\pgfqpoint{1.230395in}{0.712407in}}%
\pgfpathlineto{\pgfqpoint{1.237183in}{0.720337in}}%
\pgfpathlineto{\pgfqpoint{1.242663in}{0.726018in}}%
\pgfpathlineto{\pgfqpoint{1.252018in}{0.739629in}}%
\pgfpathlineto{\pgfqpoint{1.252839in}{0.741782in}}%
\pgfpathlineto{\pgfqpoint{1.258844in}{0.753240in}}%
\pgfpathlineto{\pgfqpoint{1.260036in}{0.766851in}}%
\pgfpathlineto{\pgfqpoint{1.255267in}{0.780462in}}%
\pgfpathlineto{\pgfqpoint{1.252839in}{0.783614in}}%
\pgfpathlineto{\pgfqpoint{1.246915in}{0.794073in}}%
\pgfpathlineto{\pgfqpoint{1.237183in}{0.805308in}}%
\pgfpathlineto{\pgfqpoint{1.235367in}{0.807684in}}%
\pgfpathlineto{\pgfqpoint{1.221960in}{0.821295in}}%
\pgfpathlineto{\pgfqpoint{1.221526in}{0.821683in}}%
\pgfpathlineto{\pgfqpoint{1.206544in}{0.834907in}}%
\pgfpathlineto{\pgfqpoint{1.205870in}{0.835482in}}%
\pgfpathlineto{\pgfqpoint{1.190213in}{0.847448in}}%
\pgfpathlineto{\pgfqpoint{1.188458in}{0.848518in}}%
\pgfpathlineto{\pgfqpoint{1.174556in}{0.857850in}}%
\pgfpathlineto{\pgfqpoint{1.165207in}{0.862129in}}%
\pgfpathlineto{\pgfqpoint{1.158900in}{0.865885in}}%
\pgfpathlineto{\pgfqpoint{1.143243in}{0.870639in}}%
\pgfpathlineto{\pgfqpoint{1.127587in}{0.870639in}}%
\pgfpathlineto{\pgfqpoint{1.111930in}{0.865885in}}%
\pgfpathlineto{\pgfqpoint{1.105623in}{0.862129in}}%
\pgfpathlineto{\pgfqpoint{1.096274in}{0.857850in}}%
\pgfpathlineto{\pgfqpoint{1.082372in}{0.848518in}}%
\pgfpathlineto{\pgfqpoint{1.080617in}{0.847448in}}%
\pgfpathlineto{\pgfqpoint{1.064960in}{0.835482in}}%
\pgfpathlineto{\pgfqpoint{1.064286in}{0.834907in}}%
\pgfpathlineto{\pgfqpoint{1.049304in}{0.821683in}}%
\pgfpathlineto{\pgfqpoint{1.048870in}{0.821295in}}%
\pgfpathlineto{\pgfqpoint{1.035463in}{0.807684in}}%
\pgfpathlineto{\pgfqpoint{1.033647in}{0.805308in}}%
\pgfpathlineto{\pgfqpoint{1.023915in}{0.794073in}}%
\pgfpathlineto{\pgfqpoint{1.017991in}{0.783614in}}%
\pgfpathlineto{\pgfqpoint{1.015563in}{0.780462in}}%
\pgfpathlineto{\pgfqpoint{1.010794in}{0.766851in}}%
\pgfpathlineto{\pgfqpoint{1.011986in}{0.753240in}}%
\pgfpathlineto{\pgfqpoint{1.017991in}{0.741782in}}%
\pgfpathlineto{\pgfqpoint{1.018812in}{0.739629in}}%
\pgfpathlineto{\pgfqpoint{1.028167in}{0.726018in}}%
\pgfpathlineto{\pgfqpoint{1.033647in}{0.720337in}}%
\pgfpathlineto{\pgfqpoint{1.040435in}{0.712407in}}%
\pgfpathlineto{\pgfqpoint{1.049304in}{0.703959in}}%
\pgfpathlineto{\pgfqpoint{1.054765in}{0.698795in}}%
\pgfpathlineto{\pgfqpoint{1.064960in}{0.690023in}}%
\pgfpathlineto{\pgfqpoint{1.071275in}{0.685184in}}%
\pgfpathlineto{\pgfqpoint{1.080617in}{0.677910in}}%
\pgfpathlineto{\pgfqpoint{1.090844in}{0.671573in}}%
\pgfpathlineto{\pgfqpoint{1.096274in}{0.667666in}}%
\pgfpathlineto{\pgfqpoint{1.111930in}{0.659879in}}%
\pgfpathlineto{\pgfqpoint{1.119581in}{0.657962in}}%
\pgfpathlineto{\pgfqpoint{1.127587in}{0.655062in}}%
\pgfpathclose%
\pgfpathmoveto{\pgfqpoint{1.120003in}{0.685184in}}%
\pgfpathlineto{\pgfqpoint{1.111930in}{0.686930in}}%
\pgfpathlineto{\pgfqpoint{1.096274in}{0.693656in}}%
\pgfpathlineto{\pgfqpoint{1.088079in}{0.698795in}}%
\pgfpathlineto{\pgfqpoint{1.080617in}{0.703839in}}%
\pgfpathlineto{\pgfqpoint{1.070472in}{0.712407in}}%
\pgfpathlineto{\pgfqpoint{1.064960in}{0.718068in}}%
\pgfpathlineto{\pgfqpoint{1.057946in}{0.726018in}}%
\pgfpathlineto{\pgfqpoint{1.049464in}{0.739629in}}%
\pgfpathlineto{\pgfqpoint{1.049304in}{0.740090in}}%
\pgfpathlineto{\pgfqpoint{1.044578in}{0.753240in}}%
\pgfpathlineto{\pgfqpoint{1.043763in}{0.766851in}}%
\pgfpathlineto{\pgfqpoint{1.047024in}{0.780462in}}%
\pgfpathlineto{\pgfqpoint{1.049304in}{0.784777in}}%
\pgfpathlineto{\pgfqpoint{1.054090in}{0.794073in}}%
\pgfpathlineto{\pgfqpoint{1.064872in}{0.807684in}}%
\pgfpathlineto{\pgfqpoint{1.064960in}{0.807773in}}%
\pgfpathlineto{\pgfqpoint{1.079979in}{0.821295in}}%
\pgfpathlineto{\pgfqpoint{1.080617in}{0.821801in}}%
\pgfpathlineto{\pgfqpoint{1.096274in}{0.831838in}}%
\pgfpathlineto{\pgfqpoint{1.103297in}{0.834907in}}%
\pgfpathlineto{\pgfqpoint{1.111930in}{0.838684in}}%
\pgfpathlineto{\pgfqpoint{1.127587in}{0.842159in}}%
\pgfpathlineto{\pgfqpoint{1.143243in}{0.842159in}}%
\pgfpathlineto{\pgfqpoint{1.158900in}{0.838684in}}%
\pgfpathlineto{\pgfqpoint{1.167533in}{0.834907in}}%
\pgfpathlineto{\pgfqpoint{1.174556in}{0.831838in}}%
\pgfpathlineto{\pgfqpoint{1.190213in}{0.821801in}}%
\pgfpathlineto{\pgfqpoint{1.190851in}{0.821295in}}%
\pgfpathlineto{\pgfqpoint{1.205870in}{0.807773in}}%
\pgfpathlineto{\pgfqpoint{1.205958in}{0.807684in}}%
\pgfpathlineto{\pgfqpoint{1.216740in}{0.794073in}}%
\pgfpathlineto{\pgfqpoint{1.221526in}{0.784777in}}%
\pgfpathlineto{\pgfqpoint{1.223806in}{0.780462in}}%
\pgfpathlineto{\pgfqpoint{1.227067in}{0.766851in}}%
\pgfpathlineto{\pgfqpoint{1.226252in}{0.753240in}}%
\pgfpathlineto{\pgfqpoint{1.221526in}{0.740090in}}%
\pgfpathlineto{\pgfqpoint{1.221366in}{0.739629in}}%
\pgfpathlineto{\pgfqpoint{1.212884in}{0.726018in}}%
\pgfpathlineto{\pgfqpoint{1.205870in}{0.718068in}}%
\pgfpathlineto{\pgfqpoint{1.200358in}{0.712407in}}%
\pgfpathlineto{\pgfqpoint{1.190213in}{0.703839in}}%
\pgfpathlineto{\pgfqpoint{1.182751in}{0.698795in}}%
\pgfpathlineto{\pgfqpoint{1.174556in}{0.693656in}}%
\pgfpathlineto{\pgfqpoint{1.158900in}{0.686930in}}%
\pgfpathlineto{\pgfqpoint{1.150827in}{0.685184in}}%
\pgfpathlineto{\pgfqpoint{1.143243in}{0.683441in}}%
\pgfpathlineto{\pgfqpoint{1.127587in}{0.683441in}}%
\pgfpathlineto{\pgfqpoint{1.120003in}{0.685184in}}%
\pgfpathclose%
\pgfpathmoveto{\pgfqpoint{1.440718in}{0.654604in}}%
\pgfpathlineto{\pgfqpoint{1.456375in}{0.655748in}}%
\pgfpathlineto{\pgfqpoint{1.461495in}{0.657962in}}%
\pgfpathlineto{\pgfqpoint{1.472031in}{0.661124in}}%
\pgfpathlineto{\pgfqpoint{1.487688in}{0.669690in}}%
\pgfpathlineto{\pgfqpoint{1.490174in}{0.671573in}}%
\pgfpathlineto{\pgfqpoint{1.503344in}{0.680185in}}%
\pgfpathlineto{\pgfqpoint{1.509581in}{0.685184in}}%
\pgfpathlineto{\pgfqpoint{1.519001in}{0.692630in}}%
\pgfpathlineto{\pgfqpoint{1.526093in}{0.698795in}}%
\pgfpathlineto{\pgfqpoint{1.534657in}{0.706985in}}%
\pgfpathlineto{\pgfqpoint{1.540408in}{0.712407in}}%
\pgfpathlineto{\pgfqpoint{1.550314in}{0.723856in}}%
\pgfpathlineto{\pgfqpoint{1.552480in}{0.726018in}}%
\pgfpathlineto{\pgfqpoint{1.562334in}{0.739629in}}%
\pgfpathlineto{\pgfqpoint{1.565971in}{0.748788in}}%
\pgfpathlineto{\pgfqpoint{1.568518in}{0.753240in}}%
\pgfpathlineto{\pgfqpoint{1.569833in}{0.766851in}}%
\pgfpathlineto{\pgfqpoint{1.565971in}{0.776875in}}%
\pgfpathlineto{\pgfqpoint{1.565018in}{0.780462in}}%
\pgfpathlineto{\pgfqpoint{1.556959in}{0.794073in}}%
\pgfpathlineto{\pgfqpoint{1.550314in}{0.801457in}}%
\pgfpathlineto{\pgfqpoint{1.545503in}{0.807684in}}%
\pgfpathlineto{\pgfqpoint{1.534657in}{0.818564in}}%
\pgfpathlineto{\pgfqpoint{1.531984in}{0.821295in}}%
\pgfpathlineto{\pgfqpoint{1.519001in}{0.832862in}}%
\pgfpathlineto{\pgfqpoint{1.516481in}{0.834907in}}%
\pgfpathlineto{\pgfqpoint{1.503344in}{0.845273in}}%
\pgfpathlineto{\pgfqpoint{1.498297in}{0.848518in}}%
\pgfpathlineto{\pgfqpoint{1.487688in}{0.856013in}}%
\pgfpathlineto{\pgfqpoint{1.475470in}{0.862129in}}%
\pgfpathlineto{\pgfqpoint{1.472031in}{0.864362in}}%
\pgfpathlineto{\pgfqpoint{1.456375in}{0.870069in}}%
\pgfpathlineto{\pgfqpoint{1.440718in}{0.871019in}}%
\pgfpathlineto{\pgfqpoint{1.425061in}{0.867217in}}%
\pgfpathlineto{\pgfqpoint{1.415646in}{0.862129in}}%
\pgfpathlineto{\pgfqpoint{1.409405in}{0.859546in}}%
\pgfpathlineto{\pgfqpoint{1.393748in}{0.849667in}}%
\pgfpathlineto{\pgfqpoint{1.392312in}{0.848518in}}%
\pgfpathlineto{\pgfqpoint{1.378092in}{0.838085in}}%
\pgfpathlineto{\pgfqpoint{1.374315in}{0.834907in}}%
\pgfpathlineto{\pgfqpoint{1.362435in}{0.824579in}}%
\pgfpathlineto{\pgfqpoint{1.358779in}{0.821295in}}%
\pgfpathlineto{\pgfqpoint{1.346779in}{0.808933in}}%
\pgfpathlineto{\pgfqpoint{1.345457in}{0.807684in}}%
\pgfpathlineto{\pgfqpoint{1.334093in}{0.794073in}}%
\pgfpathlineto{\pgfqpoint{1.331122in}{0.788648in}}%
\pgfpathlineto{\pgfqpoint{1.325269in}{0.780462in}}%
\pgfpathlineto{\pgfqpoint{1.320896in}{0.766851in}}%
\pgfpathlineto{\pgfqpoint{1.321988in}{0.753240in}}%
\pgfpathlineto{\pgfqpoint{1.328553in}{0.739629in}}%
\pgfpathlineto{\pgfqpoint{1.331122in}{0.736639in}}%
\pgfpathlineto{\pgfqpoint{1.338157in}{0.726018in}}%
\pgfpathlineto{\pgfqpoint{1.346779in}{0.716794in}}%
\pgfpathlineto{\pgfqpoint{1.350511in}{0.712407in}}%
\pgfpathlineto{\pgfqpoint{1.362435in}{0.700986in}}%
\pgfpathlineto{\pgfqpoint{1.364787in}{0.698795in}}%
\pgfpathlineto{\pgfqpoint{1.378092in}{0.687509in}}%
\pgfpathlineto{\pgfqpoint{1.381234in}{0.685184in}}%
\pgfpathlineto{\pgfqpoint{1.393748in}{0.675756in}}%
\pgfpathlineto{\pgfqpoint{1.400911in}{0.671573in}}%
\pgfpathlineto{\pgfqpoint{1.409405in}{0.665796in}}%
\pgfpathlineto{\pgfqpoint{1.425061in}{0.658790in}}%
\pgfpathlineto{\pgfqpoint{1.429187in}{0.657962in}}%
\pgfpathlineto{\pgfqpoint{1.440718in}{0.654604in}}%
\pgfpathclose%
\pgfpathmoveto{\pgfqpoint{1.429712in}{0.685184in}}%
\pgfpathlineto{\pgfqpoint{1.425061in}{0.685990in}}%
\pgfpathlineto{\pgfqpoint{1.409405in}{0.692041in}}%
\pgfpathlineto{\pgfqpoint{1.398010in}{0.698795in}}%
\pgfpathlineto{\pgfqpoint{1.393748in}{0.701510in}}%
\pgfpathlineto{\pgfqpoint{1.380412in}{0.712407in}}%
\pgfpathlineto{\pgfqpoint{1.378092in}{0.714708in}}%
\pgfpathlineto{\pgfqpoint{1.367971in}{0.726018in}}%
\pgfpathlineto{\pgfqpoint{1.362435in}{0.734773in}}%
\pgfpathlineto{\pgfqpoint{1.359370in}{0.739629in}}%
\pgfpathlineto{\pgfqpoint{1.354572in}{0.753240in}}%
\pgfpathlineto{\pgfqpoint{1.353773in}{0.766851in}}%
\pgfpathlineto{\pgfqpoint{1.356970in}{0.780462in}}%
\pgfpathlineto{\pgfqpoint{1.362435in}{0.790865in}}%
\pgfpathlineto{\pgfqpoint{1.364112in}{0.794073in}}%
\pgfpathlineto{\pgfqpoint{1.374901in}{0.807684in}}%
\pgfpathlineto{\pgfqpoint{1.378092in}{0.810832in}}%
\pgfpathlineto{\pgfqpoint{1.390128in}{0.821295in}}%
\pgfpathlineto{\pgfqpoint{1.393748in}{0.824069in}}%
\pgfpathlineto{\pgfqpoint{1.409405in}{0.833449in}}%
\pgfpathlineto{\pgfqpoint{1.413096in}{0.834907in}}%
\pgfpathlineto{\pgfqpoint{1.425061in}{0.839658in}}%
\pgfpathlineto{\pgfqpoint{1.440718in}{0.842437in}}%
\pgfpathlineto{\pgfqpoint{1.456375in}{0.841743in}}%
\pgfpathlineto{\pgfqpoint{1.472031in}{0.837571in}}%
\pgfpathlineto{\pgfqpoint{1.477616in}{0.834907in}}%
\pgfpathlineto{\pgfqpoint{1.487688in}{0.830094in}}%
\pgfpathlineto{\pgfqpoint{1.500697in}{0.821295in}}%
\pgfpathlineto{\pgfqpoint{1.503344in}{0.819278in}}%
\pgfpathlineto{\pgfqpoint{1.515879in}{0.807684in}}%
\pgfpathlineto{\pgfqpoint{1.519001in}{0.803980in}}%
\pgfpathlineto{\pgfqpoint{1.526770in}{0.794073in}}%
\pgfpathlineto{\pgfqpoint{1.533731in}{0.780462in}}%
\pgfpathlineto{\pgfqpoint{1.534657in}{0.776419in}}%
\pgfpathlineto{\pgfqpoint{1.536997in}{0.766851in}}%
\pgfpathlineto{\pgfqpoint{1.536162in}{0.753240in}}%
\pgfpathlineto{\pgfqpoint{1.534657in}{0.749096in}}%
\pgfpathlineto{\pgfqpoint{1.531412in}{0.739629in}}%
\pgfpathlineto{\pgfqpoint{1.522902in}{0.726018in}}%
\pgfpathlineto{\pgfqpoint{1.519001in}{0.721551in}}%
\pgfpathlineto{\pgfqpoint{1.510371in}{0.712407in}}%
\pgfpathlineto{\pgfqpoint{1.503344in}{0.706298in}}%
\pgfpathlineto{\pgfqpoint{1.492826in}{0.698795in}}%
\pgfpathlineto{\pgfqpoint{1.487688in}{0.695404in}}%
\pgfpathlineto{\pgfqpoint{1.472031in}{0.688006in}}%
\pgfpathlineto{\pgfqpoint{1.461141in}{0.685184in}}%
\pgfpathlineto{\pgfqpoint{1.456375in}{0.683876in}}%
\pgfpathlineto{\pgfqpoint{1.440718in}{0.683151in}}%
\pgfpathlineto{\pgfqpoint{1.429712in}{0.685184in}}%
\pgfpathclose%
\pgfpathmoveto{\pgfqpoint{1.738193in}{0.657807in}}%
\pgfpathlineto{\pgfqpoint{1.753849in}{0.654376in}}%
\pgfpathlineto{\pgfqpoint{1.769506in}{0.656663in}}%
\pgfpathlineto{\pgfqpoint{1.772097in}{0.657962in}}%
\pgfpathlineto{\pgfqpoint{1.785162in}{0.662525in}}%
\pgfpathlineto{\pgfqpoint{1.800343in}{0.671573in}}%
\pgfpathlineto{\pgfqpoint{1.800819in}{0.671815in}}%
\pgfpathlineto{\pgfqpoint{1.816476in}{0.682578in}}%
\pgfpathlineto{\pgfqpoint{1.819647in}{0.685184in}}%
\pgfpathlineto{\pgfqpoint{1.832132in}{0.695322in}}%
\pgfpathlineto{\pgfqpoint{1.836102in}{0.698795in}}%
\pgfpathlineto{\pgfqpoint{1.847789in}{0.710051in}}%
\pgfpathlineto{\pgfqpoint{1.850325in}{0.712407in}}%
\pgfpathlineto{\pgfqpoint{1.862351in}{0.726018in}}%
\pgfpathlineto{\pgfqpoint{1.863445in}{0.727753in}}%
\pgfpathlineto{\pgfqpoint{1.872519in}{0.739629in}}%
\pgfpathlineto{\pgfqpoint{1.878211in}{0.753240in}}%
\pgfpathlineto{\pgfqpoint{1.879102in}{0.766044in}}%
\pgfpathlineto{\pgfqpoint{1.879189in}{0.766851in}}%
\pgfpathlineto{\pgfqpoint{1.879102in}{0.767055in}}%
\pgfpathlineto{\pgfqpoint{1.875366in}{0.780462in}}%
\pgfpathlineto{\pgfqpoint{1.866818in}{0.794073in}}%
\pgfpathlineto{\pgfqpoint{1.863445in}{0.797654in}}%
\pgfpathlineto{\pgfqpoint{1.855572in}{0.807684in}}%
\pgfpathlineto{\pgfqpoint{1.847789in}{0.815377in}}%
\pgfpathlineto{\pgfqpoint{1.842036in}{0.821295in}}%
\pgfpathlineto{\pgfqpoint{1.832132in}{0.830177in}}%
\pgfpathlineto{\pgfqpoint{1.826458in}{0.834907in}}%
\pgfpathlineto{\pgfqpoint{1.816476in}{0.842985in}}%
\pgfpathlineto{\pgfqpoint{1.808279in}{0.848518in}}%
\pgfpathlineto{\pgfqpoint{1.800819in}{0.854036in}}%
\pgfpathlineto{\pgfqpoint{1.785902in}{0.862129in}}%
\pgfpathlineto{\pgfqpoint{1.785162in}{0.862647in}}%
\pgfpathlineto{\pgfqpoint{1.769506in}{0.869309in}}%
\pgfpathlineto{\pgfqpoint{1.753849in}{0.871209in}}%
\pgfpathlineto{\pgfqpoint{1.738193in}{0.868359in}}%
\pgfpathlineto{\pgfqpoint{1.725320in}{0.862129in}}%
\pgfpathlineto{\pgfqpoint{1.722536in}{0.861100in}}%
\pgfpathlineto{\pgfqpoint{1.706880in}{0.851920in}}%
\pgfpathlineto{\pgfqpoint{1.702467in}{0.848518in}}%
\pgfpathlineto{\pgfqpoint{1.691223in}{0.840588in}}%
\pgfpathlineto{\pgfqpoint{1.684352in}{0.834907in}}%
\pgfpathlineto{\pgfqpoint{1.675567in}{0.827413in}}%
\pgfpathlineto{\pgfqpoint{1.668763in}{0.821295in}}%
\pgfpathlineto{\pgfqpoint{1.659910in}{0.812161in}}%
\pgfpathlineto{\pgfqpoint{1.655289in}{0.807684in}}%
\pgfpathlineto{\pgfqpoint{1.644368in}{0.794073in}}%
\pgfpathlineto{\pgfqpoint{1.644253in}{0.793858in}}%
\pgfpathlineto{\pgfqpoint{1.635263in}{0.780462in}}%
\pgfpathlineto{\pgfqpoint{1.631210in}{0.766851in}}%
\pgfpathlineto{\pgfqpoint{1.632222in}{0.753240in}}%
\pgfpathlineto{\pgfqpoint{1.638307in}{0.739629in}}%
\pgfpathlineto{\pgfqpoint{1.644253in}{0.732254in}}%
\pgfpathlineto{\pgfqpoint{1.648273in}{0.726018in}}%
\pgfpathlineto{\pgfqpoint{1.659910in}{0.713248in}}%
\pgfpathlineto{\pgfqpoint{1.660625in}{0.712407in}}%
\pgfpathlineto{\pgfqpoint{1.674766in}{0.698795in}}%
\pgfpathlineto{\pgfqpoint{1.675567in}{0.698093in}}%
\pgfpathlineto{\pgfqpoint{1.691104in}{0.685184in}}%
\pgfpathlineto{\pgfqpoint{1.691223in}{0.685084in}}%
\pgfpathlineto{\pgfqpoint{1.706880in}{0.673724in}}%
\pgfpathlineto{\pgfqpoint{1.710812in}{0.671573in}}%
\pgfpathlineto{\pgfqpoint{1.722536in}{0.664083in}}%
\pgfpathlineto{\pgfqpoint{1.737921in}{0.657962in}}%
\pgfpathlineto{\pgfqpoint{1.738193in}{0.657807in}}%
\pgfpathclose%
\pgfpathmoveto{\pgfqpoint{1.738192in}{0.685184in}}%
\pgfpathlineto{\pgfqpoint{1.722536in}{0.690561in}}%
\pgfpathlineto{\pgfqpoint{1.707747in}{0.698795in}}%
\pgfpathlineto{\pgfqpoint{1.706880in}{0.699313in}}%
\pgfpathlineto{\pgfqpoint{1.691223in}{0.711594in}}%
\pgfpathlineto{\pgfqpoint{1.690331in}{0.712407in}}%
\pgfpathlineto{\pgfqpoint{1.677978in}{0.726018in}}%
\pgfpathlineto{\pgfqpoint{1.675567in}{0.729760in}}%
\pgfpathlineto{\pgfqpoint{1.669345in}{0.739629in}}%
\pgfpathlineto{\pgfqpoint{1.664622in}{0.753240in}}%
\pgfpathlineto{\pgfqpoint{1.663836in}{0.766851in}}%
\pgfpathlineto{\pgfqpoint{1.666982in}{0.780462in}}%
\pgfpathlineto{\pgfqpoint{1.674076in}{0.794073in}}%
\pgfpathlineto{\pgfqpoint{1.675567in}{0.795991in}}%
\pgfpathlineto{\pgfqpoint{1.684941in}{0.807684in}}%
\pgfpathlineto{\pgfqpoint{1.691223in}{0.813773in}}%
\pgfpathlineto{\pgfqpoint{1.700224in}{0.821295in}}%
\pgfpathlineto{\pgfqpoint{1.706880in}{0.826208in}}%
\pgfpathlineto{\pgfqpoint{1.722502in}{0.834907in}}%
\pgfpathlineto{\pgfqpoint{1.722536in}{0.834925in}}%
\pgfpathlineto{\pgfqpoint{1.738193in}{0.840492in}}%
\pgfpathlineto{\pgfqpoint{1.753849in}{0.842576in}}%
\pgfpathlineto{\pgfqpoint{1.769506in}{0.841187in}}%
\pgfpathlineto{\pgfqpoint{1.785162in}{0.836318in}}%
\pgfpathlineto{\pgfqpoint{1.787901in}{0.834907in}}%
\pgfpathlineto{\pgfqpoint{1.800819in}{0.828217in}}%
\pgfpathlineto{\pgfqpoint{1.810593in}{0.821295in}}%
\pgfpathlineto{\pgfqpoint{1.816476in}{0.816589in}}%
\pgfpathlineto{\pgfqpoint{1.825863in}{0.807684in}}%
\pgfpathlineto{\pgfqpoint{1.832132in}{0.800043in}}%
\pgfpathlineto{\pgfqpoint{1.836783in}{0.794073in}}%
\pgfpathlineto{\pgfqpoint{1.843796in}{0.780462in}}%
\pgfpathlineto{\pgfqpoint{1.846906in}{0.766851in}}%
\pgfpathlineto{\pgfqpoint{1.846129in}{0.753240in}}%
\pgfpathlineto{\pgfqpoint{1.841460in}{0.739629in}}%
\pgfpathlineto{\pgfqpoint{1.832886in}{0.726018in}}%
\pgfpathlineto{\pgfqpoint{1.832132in}{0.725149in}}%
\pgfpathlineto{\pgfqpoint{1.820426in}{0.712407in}}%
\pgfpathlineto{\pgfqpoint{1.816476in}{0.708884in}}%
\pgfpathlineto{\pgfqpoint{1.803003in}{0.698795in}}%
\pgfpathlineto{\pgfqpoint{1.800819in}{0.697286in}}%
\pgfpathlineto{\pgfqpoint{1.785162in}{0.689216in}}%
\pgfpathlineto{\pgfqpoint{1.771790in}{0.685184in}}%
\pgfpathlineto{\pgfqpoint{1.769506in}{0.684457in}}%
\pgfpathlineto{\pgfqpoint{1.753849in}{0.683005in}}%
\pgfpathlineto{\pgfqpoint{1.738193in}{0.685184in}}%
\pgfpathlineto{\pgfqpoint{1.738192in}{0.685184in}}%
\pgfpathclose%
\pgfpathmoveto{\pgfqpoint{0.501324in}{0.925793in}}%
\pgfpathlineto{\pgfqpoint{0.516981in}{0.923721in}}%
\pgfpathlineto{\pgfqpoint{0.532637in}{0.926829in}}%
\pgfpathlineto{\pgfqpoint{0.539072in}{0.930184in}}%
\pgfpathlineto{\pgfqpoint{0.548294in}{0.933707in}}%
\pgfpathlineto{\pgfqpoint{0.563950in}{0.943315in}}%
\pgfpathlineto{\pgfqpoint{0.564555in}{0.943795in}}%
\pgfpathlineto{\pgfqpoint{0.579607in}{0.954476in}}%
\pgfpathlineto{\pgfqpoint{0.583131in}{0.957407in}}%
\pgfpathlineto{\pgfqpoint{0.595263in}{0.967599in}}%
\pgfpathlineto{\pgfqpoint{0.599119in}{0.971018in}}%
\pgfpathlineto{\pgfqpoint{0.610920in}{0.982773in}}%
\pgfpathlineto{\pgfqpoint{0.612899in}{0.984629in}}%
\pgfpathlineto{\pgfqpoint{0.624587in}{0.998240in}}%
\pgfpathlineto{\pgfqpoint{0.626577in}{1.001616in}}%
\pgfpathlineto{\pgfqpoint{0.634147in}{1.011851in}}%
\pgfpathlineto{\pgfqpoint{0.639216in}{1.025462in}}%
\pgfpathlineto{\pgfqpoint{0.639216in}{1.039073in}}%
\pgfpathlineto{\pgfqpoint{0.634147in}{1.052684in}}%
\pgfpathlineto{\pgfqpoint{0.626577in}{1.062920in}}%
\pgfpathlineto{\pgfqpoint{0.624587in}{1.066295in}}%
\pgfpathlineto{\pgfqpoint{0.612899in}{1.079907in}}%
\pgfpathlineto{\pgfqpoint{0.610920in}{1.081762in}}%
\pgfpathlineto{\pgfqpoint{0.599119in}{1.093518in}}%
\pgfpathlineto{\pgfqpoint{0.595263in}{1.096936in}}%
\pgfpathlineto{\pgfqpoint{0.583131in}{1.107129in}}%
\pgfpathlineto{\pgfqpoint{0.579607in}{1.110059in}}%
\pgfpathlineto{\pgfqpoint{0.564555in}{1.120740in}}%
\pgfpathlineto{\pgfqpoint{0.563950in}{1.121221in}}%
\pgfpathlineto{\pgfqpoint{0.548294in}{1.130828in}}%
\pgfpathlineto{\pgfqpoint{0.539072in}{1.134351in}}%
\pgfpathlineto{\pgfqpoint{0.532637in}{1.137706in}}%
\pgfpathlineto{\pgfqpoint{0.516981in}{1.140814in}}%
\pgfpathlineto{\pgfqpoint{0.501324in}{1.138743in}}%
\pgfpathlineto{\pgfqpoint{0.491763in}{1.134351in}}%
\pgfpathlineto{\pgfqpoint{0.485668in}{1.132307in}}%
\pgfpathlineto{\pgfqpoint{0.470011in}{1.123435in}}%
\pgfpathlineto{\pgfqpoint{0.466480in}{1.120740in}}%
\pgfpathlineto{\pgfqpoint{0.454354in}{1.112505in}}%
\pgfpathlineto{\pgfqpoint{0.447748in}{1.107129in}}%
\pgfpathlineto{\pgfqpoint{0.438698in}{1.099698in}}%
\pgfpathlineto{\pgfqpoint{0.431708in}{1.093518in}}%
\pgfpathlineto{\pgfqpoint{0.423041in}{1.084909in}}%
\pgfpathlineto{\pgfqpoint{0.417811in}{1.079907in}}%
\pgfpathlineto{\pgfqpoint{0.407385in}{1.067391in}}%
\pgfpathlineto{\pgfqpoint{0.406293in}{1.066295in}}%
\pgfpathlineto{\pgfqpoint{0.396792in}{1.052684in}}%
\pgfpathlineto{\pgfqpoint{0.392051in}{1.039073in}}%
\pgfpathlineto{\pgfqpoint{0.392051in}{1.025462in}}%
\pgfpathlineto{\pgfqpoint{0.396792in}{1.011851in}}%
\pgfpathlineto{\pgfqpoint{0.406293in}{0.998240in}}%
\pgfpathlineto{\pgfqpoint{0.407385in}{0.997144in}}%
\pgfpathlineto{\pgfqpoint{0.417811in}{0.984629in}}%
\pgfpathlineto{\pgfqpoint{0.423041in}{0.979626in}}%
\pgfpathlineto{\pgfqpoint{0.431708in}{0.971018in}}%
\pgfpathlineto{\pgfqpoint{0.438698in}{0.964837in}}%
\pgfpathlineto{\pgfqpoint{0.447748in}{0.957407in}}%
\pgfpathlineto{\pgfqpoint{0.454354in}{0.952030in}}%
\pgfpathlineto{\pgfqpoint{0.466480in}{0.943795in}}%
\pgfpathlineto{\pgfqpoint{0.470011in}{0.941100in}}%
\pgfpathlineto{\pgfqpoint{0.485668in}{0.932228in}}%
\pgfpathlineto{\pgfqpoint{0.491763in}{0.930184in}}%
\pgfpathlineto{\pgfqpoint{0.501324in}{0.925793in}}%
\pgfpathclose%
\pgfpathmoveto{\pgfqpoint{0.490080in}{0.957407in}}%
\pgfpathlineto{\pgfqpoint{0.485668in}{0.958752in}}%
\pgfpathlineto{\pgfqpoint{0.470011in}{0.966795in}}%
\pgfpathlineto{\pgfqpoint{0.463964in}{0.971018in}}%
\pgfpathlineto{\pgfqpoint{0.454354in}{0.978440in}}%
\pgfpathlineto{\pgfqpoint{0.447611in}{0.984629in}}%
\pgfpathlineto{\pgfqpoint{0.438698in}{0.994868in}}%
\pgfpathlineto{\pgfqpoint{0.435918in}{0.998240in}}%
\pgfpathlineto{\pgfqpoint{0.428124in}{1.011851in}}%
\pgfpathlineto{\pgfqpoint{0.424234in}{1.025462in}}%
\pgfpathlineto{\pgfqpoint{0.424234in}{1.039073in}}%
\pgfpathlineto{\pgfqpoint{0.428124in}{1.052684in}}%
\pgfpathlineto{\pgfqpoint{0.435918in}{1.066295in}}%
\pgfpathlineto{\pgfqpoint{0.438698in}{1.069668in}}%
\pgfpathlineto{\pgfqpoint{0.447611in}{1.079907in}}%
\pgfpathlineto{\pgfqpoint{0.454354in}{1.086096in}}%
\pgfpathlineto{\pgfqpoint{0.463964in}{1.093518in}}%
\pgfpathlineto{\pgfqpoint{0.470011in}{1.097740in}}%
\pgfpathlineto{\pgfqpoint{0.485668in}{1.105784in}}%
\pgfpathlineto{\pgfqpoint{0.490080in}{1.107129in}}%
\pgfpathlineto{\pgfqpoint{0.501324in}{1.110671in}}%
\pgfpathlineto{\pgfqpoint{0.516981in}{1.112088in}}%
\pgfpathlineto{\pgfqpoint{0.532637in}{1.109962in}}%
\pgfpathlineto{\pgfqpoint{0.540560in}{1.107129in}}%
\pgfpathlineto{\pgfqpoint{0.548294in}{1.104443in}}%
\pgfpathlineto{\pgfqpoint{0.563950in}{1.095733in}}%
\pgfpathlineto{\pgfqpoint{0.566994in}{1.093518in}}%
\pgfpathlineto{\pgfqpoint{0.579607in}{1.083339in}}%
\pgfpathlineto{\pgfqpoint{0.583267in}{1.079907in}}%
\pgfpathlineto{\pgfqpoint{0.594868in}{1.066295in}}%
\pgfpathlineto{\pgfqpoint{0.595263in}{1.065624in}}%
\pgfpathlineto{\pgfqpoint{0.602745in}{1.052684in}}%
\pgfpathlineto{\pgfqpoint{0.606680in}{1.039073in}}%
\pgfpathlineto{\pgfqpoint{0.606680in}{1.025462in}}%
\pgfpathlineto{\pgfqpoint{0.602745in}{1.011851in}}%
\pgfpathlineto{\pgfqpoint{0.595263in}{0.998912in}}%
\pgfpathlineto{\pgfqpoint{0.594868in}{0.998240in}}%
\pgfpathlineto{\pgfqpoint{0.583267in}{0.984629in}}%
\pgfpathlineto{\pgfqpoint{0.579607in}{0.981196in}}%
\pgfpathlineto{\pgfqpoint{0.566994in}{0.971018in}}%
\pgfpathlineto{\pgfqpoint{0.563950in}{0.968803in}}%
\pgfpathlineto{\pgfqpoint{0.548294in}{0.960092in}}%
\pgfpathlineto{\pgfqpoint{0.540560in}{0.957407in}}%
\pgfpathlineto{\pgfqpoint{0.532637in}{0.954573in}}%
\pgfpathlineto{\pgfqpoint{0.516981in}{0.952448in}}%
\pgfpathlineto{\pgfqpoint{0.501324in}{0.953864in}}%
\pgfpathlineto{\pgfqpoint{0.490080in}{0.957407in}}%
\pgfpathclose%
\pgfpathmoveto{\pgfqpoint{0.814455in}{0.924964in}}%
\pgfpathlineto{\pgfqpoint{0.830112in}{0.923928in}}%
\pgfpathlineto{\pgfqpoint{0.845769in}{0.928073in}}%
\pgfpathlineto{\pgfqpoint{0.849394in}{0.930184in}}%
\pgfpathlineto{\pgfqpoint{0.861425in}{0.935334in}}%
\pgfpathlineto{\pgfqpoint{0.874349in}{0.943795in}}%
\pgfpathlineto{\pgfqpoint{0.877082in}{0.945374in}}%
\pgfpathlineto{\pgfqpoint{0.892738in}{0.957029in}}%
\pgfpathlineto{\pgfqpoint{0.893184in}{0.957407in}}%
\pgfpathlineto{\pgfqpoint{0.908395in}{0.970431in}}%
\pgfpathlineto{\pgfqpoint{0.909057in}{0.971018in}}%
\pgfpathlineto{\pgfqpoint{0.922821in}{0.984629in}}%
\pgfpathlineto{\pgfqpoint{0.924051in}{0.986155in}}%
\pgfpathlineto{\pgfqpoint{0.934786in}{0.998240in}}%
\pgfpathlineto{\pgfqpoint{0.939708in}{1.006368in}}%
\pgfpathlineto{\pgfqpoint{0.944029in}{1.011851in}}%
\pgfpathlineto{\pgfqpoint{0.949497in}{1.025462in}}%
\pgfpathlineto{\pgfqpoint{0.949497in}{1.039073in}}%
\pgfpathlineto{\pgfqpoint{0.944029in}{1.052684in}}%
\pgfpathlineto{\pgfqpoint{0.939708in}{1.058167in}}%
\pgfpathlineto{\pgfqpoint{0.934786in}{1.066295in}}%
\pgfpathlineto{\pgfqpoint{0.924051in}{1.078381in}}%
\pgfpathlineto{\pgfqpoint{0.922821in}{1.079907in}}%
\pgfpathlineto{\pgfqpoint{0.909057in}{1.093518in}}%
\pgfpathlineto{\pgfqpoint{0.908395in}{1.094104in}}%
\pgfpathlineto{\pgfqpoint{0.893184in}{1.107129in}}%
\pgfpathlineto{\pgfqpoint{0.892738in}{1.107506in}}%
\pgfpathlineto{\pgfqpoint{0.877082in}{1.119161in}}%
\pgfpathlineto{\pgfqpoint{0.874349in}{1.120740in}}%
\pgfpathlineto{\pgfqpoint{0.861425in}{1.129201in}}%
\pgfpathlineto{\pgfqpoint{0.849394in}{1.134351in}}%
\pgfpathlineto{\pgfqpoint{0.845769in}{1.136462in}}%
\pgfpathlineto{\pgfqpoint{0.830112in}{1.140607in}}%
\pgfpathlineto{\pgfqpoint{0.814455in}{1.139572in}}%
\pgfpathlineto{\pgfqpoint{0.801275in}{1.134351in}}%
\pgfpathlineto{\pgfqpoint{0.798799in}{1.133637in}}%
\pgfpathlineto{\pgfqpoint{0.783142in}{1.125504in}}%
\pgfpathlineto{\pgfqpoint{0.776607in}{1.120740in}}%
\pgfpathlineto{\pgfqpoint{0.767486in}{1.114839in}}%
\pgfpathlineto{\pgfqpoint{0.757769in}{1.107129in}}%
\pgfpathlineto{\pgfqpoint{0.751829in}{1.102381in}}%
\pgfpathlineto{\pgfqpoint{0.741739in}{1.093518in}}%
\pgfpathlineto{\pgfqpoint{0.736173in}{1.088028in}}%
\pgfpathlineto{\pgfqpoint{0.727806in}{1.079907in}}%
\pgfpathlineto{\pgfqpoint{0.720516in}{1.071016in}}%
\pgfpathlineto{\pgfqpoint{0.716021in}{1.066295in}}%
\pgfpathlineto{\pgfqpoint{0.707064in}{1.052684in}}%
\pgfpathlineto{\pgfqpoint{0.704859in}{1.046033in}}%
\pgfpathlineto{\pgfqpoint{0.701523in}{1.039073in}}%
\pgfpathlineto{\pgfqpoint{0.701523in}{1.025462in}}%
\pgfpathlineto{\pgfqpoint{0.704859in}{1.018502in}}%
\pgfpathlineto{\pgfqpoint{0.707064in}{1.011851in}}%
\pgfpathlineto{\pgfqpoint{0.716021in}{0.998240in}}%
\pgfpathlineto{\pgfqpoint{0.720516in}{0.993519in}}%
\pgfpathlineto{\pgfqpoint{0.727806in}{0.984629in}}%
\pgfpathlineto{\pgfqpoint{0.736173in}{0.976507in}}%
\pgfpathlineto{\pgfqpoint{0.741739in}{0.971018in}}%
\pgfpathlineto{\pgfqpoint{0.751829in}{0.962154in}}%
\pgfpathlineto{\pgfqpoint{0.757769in}{0.957407in}}%
\pgfpathlineto{\pgfqpoint{0.767486in}{0.949696in}}%
\pgfpathlineto{\pgfqpoint{0.776607in}{0.943795in}}%
\pgfpathlineto{\pgfqpoint{0.783142in}{0.939031in}}%
\pgfpathlineto{\pgfqpoint{0.798799in}{0.930898in}}%
\pgfpathlineto{\pgfqpoint{0.801275in}{0.930184in}}%
\pgfpathlineto{\pgfqpoint{0.814455in}{0.924964in}}%
\pgfpathclose%
\pgfpathmoveto{\pgfqpoint{0.799329in}{0.957407in}}%
\pgfpathlineto{\pgfqpoint{0.798799in}{0.957545in}}%
\pgfpathlineto{\pgfqpoint{0.783142in}{0.964919in}}%
\pgfpathlineto{\pgfqpoint{0.773998in}{0.971018in}}%
\pgfpathlineto{\pgfqpoint{0.767486in}{0.975809in}}%
\pgfpathlineto{\pgfqpoint{0.757631in}{0.984629in}}%
\pgfpathlineto{\pgfqpoint{0.751829in}{0.991116in}}%
\pgfpathlineto{\pgfqpoint{0.745917in}{0.998240in}}%
\pgfpathlineto{\pgfqpoint{0.738181in}{1.011851in}}%
\pgfpathlineto{\pgfqpoint{0.736173in}{1.018869in}}%
\pgfpathlineto{\pgfqpoint{0.734167in}{1.025462in}}%
\pgfpathlineto{\pgfqpoint{0.734167in}{1.039073in}}%
\pgfpathlineto{\pgfqpoint{0.736173in}{1.045666in}}%
\pgfpathlineto{\pgfqpoint{0.738181in}{1.052684in}}%
\pgfpathlineto{\pgfqpoint{0.745917in}{1.066295in}}%
\pgfpathlineto{\pgfqpoint{0.751829in}{1.073420in}}%
\pgfpathlineto{\pgfqpoint{0.757631in}{1.079907in}}%
\pgfpathlineto{\pgfqpoint{0.767486in}{1.088727in}}%
\pgfpathlineto{\pgfqpoint{0.773998in}{1.093518in}}%
\pgfpathlineto{\pgfqpoint{0.783142in}{1.099616in}}%
\pgfpathlineto{\pgfqpoint{0.798799in}{1.106990in}}%
\pgfpathlineto{\pgfqpoint{0.799329in}{1.107129in}}%
\pgfpathlineto{\pgfqpoint{0.814455in}{1.111238in}}%
\pgfpathlineto{\pgfqpoint{0.830112in}{1.111946in}}%
\pgfpathlineto{\pgfqpoint{0.845769in}{1.109111in}}%
\pgfpathlineto{\pgfqpoint{0.850732in}{1.107129in}}%
\pgfpathlineto{\pgfqpoint{0.861425in}{1.102968in}}%
\pgfpathlineto{\pgfqpoint{0.877082in}{1.093594in}}%
\pgfpathlineto{\pgfqpoint{0.877183in}{1.093518in}}%
\pgfpathlineto{\pgfqpoint{0.892738in}{1.080461in}}%
\pgfpathlineto{\pgfqpoint{0.893319in}{1.079907in}}%
\pgfpathlineto{\pgfqpoint{0.904865in}{1.066295in}}%
\pgfpathlineto{\pgfqpoint{0.908395in}{1.060189in}}%
\pgfpathlineto{\pgfqpoint{0.912740in}{1.052684in}}%
\pgfpathlineto{\pgfqpoint{0.916737in}{1.039073in}}%
\pgfpathlineto{\pgfqpoint{0.916737in}{1.025462in}}%
\pgfpathlineto{\pgfqpoint{0.912740in}{1.011851in}}%
\pgfpathlineto{\pgfqpoint{0.908395in}{1.004346in}}%
\pgfpathlineto{\pgfqpoint{0.904865in}{0.998240in}}%
\pgfpathlineto{\pgfqpoint{0.893319in}{0.984629in}}%
\pgfpathlineto{\pgfqpoint{0.892738in}{0.984074in}}%
\pgfpathlineto{\pgfqpoint{0.877183in}{0.971018in}}%
\pgfpathlineto{\pgfqpoint{0.877082in}{0.970941in}}%
\pgfpathlineto{\pgfqpoint{0.861425in}{0.961567in}}%
\pgfpathlineto{\pgfqpoint{0.850732in}{0.957407in}}%
\pgfpathlineto{\pgfqpoint{0.845769in}{0.955424in}}%
\pgfpathlineto{\pgfqpoint{0.830112in}{0.952589in}}%
\pgfpathlineto{\pgfqpoint{0.814455in}{0.953298in}}%
\pgfpathlineto{\pgfqpoint{0.799329in}{0.957407in}}%
\pgfpathclose%
\pgfpathmoveto{\pgfqpoint{1.111930in}{0.929526in}}%
\pgfpathlineto{\pgfqpoint{1.127587in}{0.924342in}}%
\pgfpathlineto{\pgfqpoint{1.143243in}{0.924342in}}%
\pgfpathlineto{\pgfqpoint{1.158900in}{0.929526in}}%
\pgfpathlineto{\pgfqpoint{1.159926in}{0.930184in}}%
\pgfpathlineto{\pgfqpoint{1.174556in}{0.937109in}}%
\pgfpathlineto{\pgfqpoint{1.184208in}{0.943795in}}%
\pgfpathlineto{\pgfqpoint{1.190213in}{0.947476in}}%
\pgfpathlineto{\pgfqpoint{1.203095in}{0.957407in}}%
\pgfpathlineto{\pgfqpoint{1.205870in}{0.959556in}}%
\pgfpathlineto{\pgfqpoint{1.219053in}{0.971018in}}%
\pgfpathlineto{\pgfqpoint{1.221526in}{0.973430in}}%
\pgfpathlineto{\pgfqpoint{1.232949in}{0.984629in}}%
\pgfpathlineto{\pgfqpoint{1.237183in}{0.989849in}}%
\pgfpathlineto{\pgfqpoint{1.244874in}{0.998240in}}%
\pgfpathlineto{\pgfqpoint{1.252839in}{1.010959in}}%
\pgfpathlineto{\pgfqpoint{1.253596in}{1.011851in}}%
\pgfpathlineto{\pgfqpoint{1.259559in}{1.025462in}}%
\pgfpathlineto{\pgfqpoint{1.259559in}{1.039073in}}%
\pgfpathlineto{\pgfqpoint{1.253596in}{1.052684in}}%
\pgfpathlineto{\pgfqpoint{1.252839in}{1.053576in}}%
\pgfpathlineto{\pgfqpoint{1.244874in}{1.066295in}}%
\pgfpathlineto{\pgfqpoint{1.237183in}{1.074686in}}%
\pgfpathlineto{\pgfqpoint{1.232949in}{1.079907in}}%
\pgfpathlineto{\pgfqpoint{1.221526in}{1.091106in}}%
\pgfpathlineto{\pgfqpoint{1.219053in}{1.093518in}}%
\pgfpathlineto{\pgfqpoint{1.205870in}{1.104979in}}%
\pgfpathlineto{\pgfqpoint{1.203095in}{1.107129in}}%
\pgfpathlineto{\pgfqpoint{1.190213in}{1.117059in}}%
\pgfpathlineto{\pgfqpoint{1.184208in}{1.120740in}}%
\pgfpathlineto{\pgfqpoint{1.174556in}{1.127426in}}%
\pgfpathlineto{\pgfqpoint{1.159926in}{1.134351in}}%
\pgfpathlineto{\pgfqpoint{1.158900in}{1.135009in}}%
\pgfpathlineto{\pgfqpoint{1.143243in}{1.140193in}}%
\pgfpathlineto{\pgfqpoint{1.127587in}{1.140193in}}%
\pgfpathlineto{\pgfqpoint{1.111930in}{1.135009in}}%
\pgfpathlineto{\pgfqpoint{1.110904in}{1.134351in}}%
\pgfpathlineto{\pgfqpoint{1.096274in}{1.127426in}}%
\pgfpathlineto{\pgfqpoint{1.086622in}{1.120740in}}%
\pgfpathlineto{\pgfqpoint{1.080617in}{1.117059in}}%
\pgfpathlineto{\pgfqpoint{1.067735in}{1.107129in}}%
\pgfpathlineto{\pgfqpoint{1.064960in}{1.104979in}}%
\pgfpathlineto{\pgfqpoint{1.051777in}{1.093518in}}%
\pgfpathlineto{\pgfqpoint{1.049304in}{1.091106in}}%
\pgfpathlineto{\pgfqpoint{1.037881in}{1.079907in}}%
\pgfpathlineto{\pgfqpoint{1.033647in}{1.074686in}}%
\pgfpathlineto{\pgfqpoint{1.025956in}{1.066295in}}%
\pgfpathlineto{\pgfqpoint{1.017991in}{1.053576in}}%
\pgfpathlineto{\pgfqpoint{1.017234in}{1.052684in}}%
\pgfpathlineto{\pgfqpoint{1.011271in}{1.039073in}}%
\pgfpathlineto{\pgfqpoint{1.011271in}{1.025462in}}%
\pgfpathlineto{\pgfqpoint{1.017234in}{1.011851in}}%
\pgfpathlineto{\pgfqpoint{1.017991in}{1.010959in}}%
\pgfpathlineto{\pgfqpoint{1.025956in}{0.998240in}}%
\pgfpathlineto{\pgfqpoint{1.033647in}{0.989849in}}%
\pgfpathlineto{\pgfqpoint{1.037881in}{0.984629in}}%
\pgfpathlineto{\pgfqpoint{1.049304in}{0.973430in}}%
\pgfpathlineto{\pgfqpoint{1.051777in}{0.971018in}}%
\pgfpathlineto{\pgfqpoint{1.064960in}{0.959556in}}%
\pgfpathlineto{\pgfqpoint{1.067735in}{0.957407in}}%
\pgfpathlineto{\pgfqpoint{1.080617in}{0.947476in}}%
\pgfpathlineto{\pgfqpoint{1.086622in}{0.943795in}}%
\pgfpathlineto{\pgfqpoint{1.096274in}{0.937109in}}%
\pgfpathlineto{\pgfqpoint{1.110904in}{0.930184in}}%
\pgfpathlineto{\pgfqpoint{1.111930in}{0.929526in}}%
\pgfpathclose%
\pgfpathmoveto{\pgfqpoint{1.109684in}{0.957407in}}%
\pgfpathlineto{\pgfqpoint{1.096274in}{0.963177in}}%
\pgfpathlineto{\pgfqpoint{1.083901in}{0.971018in}}%
\pgfpathlineto{\pgfqpoint{1.080617in}{0.973307in}}%
\pgfpathlineto{\pgfqpoint{1.067594in}{0.984629in}}%
\pgfpathlineto{\pgfqpoint{1.064960in}{0.987483in}}%
\pgfpathlineto{\pgfqpoint{1.055941in}{0.998240in}}%
\pgfpathlineto{\pgfqpoint{1.049304in}{1.009898in}}%
\pgfpathlineto{\pgfqpoint{1.048167in}{1.011851in}}%
\pgfpathlineto{\pgfqpoint{1.044089in}{1.025462in}}%
\pgfpathlineto{\pgfqpoint{1.044089in}{1.039073in}}%
\pgfpathlineto{\pgfqpoint{1.048167in}{1.052684in}}%
\pgfpathlineto{\pgfqpoint{1.049304in}{1.054637in}}%
\pgfpathlineto{\pgfqpoint{1.055941in}{1.066295in}}%
\pgfpathlineto{\pgfqpoint{1.064960in}{1.077052in}}%
\pgfpathlineto{\pgfqpoint{1.067594in}{1.079907in}}%
\pgfpathlineto{\pgfqpoint{1.080617in}{1.091228in}}%
\pgfpathlineto{\pgfqpoint{1.083901in}{1.093518in}}%
\pgfpathlineto{\pgfqpoint{1.096274in}{1.101359in}}%
\pgfpathlineto{\pgfqpoint{1.109684in}{1.107129in}}%
\pgfpathlineto{\pgfqpoint{1.111930in}{1.108117in}}%
\pgfpathlineto{\pgfqpoint{1.127587in}{1.111663in}}%
\pgfpathlineto{\pgfqpoint{1.143243in}{1.111663in}}%
\pgfpathlineto{\pgfqpoint{1.158900in}{1.108117in}}%
\pgfpathlineto{\pgfqpoint{1.161146in}{1.107129in}}%
\pgfpathlineto{\pgfqpoint{1.174556in}{1.101359in}}%
\pgfpathlineto{\pgfqpoint{1.186929in}{1.093518in}}%
\pgfpathlineto{\pgfqpoint{1.190213in}{1.091228in}}%
\pgfpathlineto{\pgfqpoint{1.203236in}{1.079907in}}%
\pgfpathlineto{\pgfqpoint{1.205870in}{1.077052in}}%
\pgfpathlineto{\pgfqpoint{1.214889in}{1.066295in}}%
\pgfpathlineto{\pgfqpoint{1.221526in}{1.054637in}}%
\pgfpathlineto{\pgfqpoint{1.222663in}{1.052684in}}%
\pgfpathlineto{\pgfqpoint{1.226741in}{1.039073in}}%
\pgfpathlineto{\pgfqpoint{1.226741in}{1.025462in}}%
\pgfpathlineto{\pgfqpoint{1.222663in}{1.011851in}}%
\pgfpathlineto{\pgfqpoint{1.221526in}{1.009898in}}%
\pgfpathlineto{\pgfqpoint{1.214889in}{0.998240in}}%
\pgfpathlineto{\pgfqpoint{1.205870in}{0.987483in}}%
\pgfpathlineto{\pgfqpoint{1.203236in}{0.984629in}}%
\pgfpathlineto{\pgfqpoint{1.190213in}{0.973307in}}%
\pgfpathlineto{\pgfqpoint{1.186929in}{0.971018in}}%
\pgfpathlineto{\pgfqpoint{1.174556in}{0.963177in}}%
\pgfpathlineto{\pgfqpoint{1.161146in}{0.957407in}}%
\pgfpathlineto{\pgfqpoint{1.158900in}{0.956418in}}%
\pgfpathlineto{\pgfqpoint{1.143243in}{0.952873in}}%
\pgfpathlineto{\pgfqpoint{1.127587in}{0.952873in}}%
\pgfpathlineto{\pgfqpoint{1.111930in}{0.956418in}}%
\pgfpathlineto{\pgfqpoint{1.109684in}{0.957407in}}%
\pgfpathclose%
\pgfpathmoveto{\pgfqpoint{1.425061in}{0.928073in}}%
\pgfpathlineto{\pgfqpoint{1.440718in}{0.923928in}}%
\pgfpathlineto{\pgfqpoint{1.456375in}{0.924964in}}%
\pgfpathlineto{\pgfqpoint{1.469555in}{0.930184in}}%
\pgfpathlineto{\pgfqpoint{1.472031in}{0.930898in}}%
\pgfpathlineto{\pgfqpoint{1.487688in}{0.939031in}}%
\pgfpathlineto{\pgfqpoint{1.494223in}{0.943795in}}%
\pgfpathlineto{\pgfqpoint{1.503344in}{0.949696in}}%
\pgfpathlineto{\pgfqpoint{1.513061in}{0.957407in}}%
\pgfpathlineto{\pgfqpoint{1.519001in}{0.962154in}}%
\pgfpathlineto{\pgfqpoint{1.529091in}{0.971018in}}%
\pgfpathlineto{\pgfqpoint{1.534657in}{0.976507in}}%
\pgfpathlineto{\pgfqpoint{1.543024in}{0.984629in}}%
\pgfpathlineto{\pgfqpoint{1.550314in}{0.993519in}}%
\pgfpathlineto{\pgfqpoint{1.554809in}{0.998240in}}%
\pgfpathlineto{\pgfqpoint{1.563766in}{1.011851in}}%
\pgfpathlineto{\pgfqpoint{1.565971in}{1.018502in}}%
\pgfpathlineto{\pgfqpoint{1.569307in}{1.025462in}}%
\pgfpathlineto{\pgfqpoint{1.569307in}{1.039073in}}%
\pgfpathlineto{\pgfqpoint{1.565971in}{1.046033in}}%
\pgfpathlineto{\pgfqpoint{1.563766in}{1.052684in}}%
\pgfpathlineto{\pgfqpoint{1.554809in}{1.066295in}}%
\pgfpathlineto{\pgfqpoint{1.550314in}{1.071016in}}%
\pgfpathlineto{\pgfqpoint{1.543024in}{1.079907in}}%
\pgfpathlineto{\pgfqpoint{1.534657in}{1.088028in}}%
\pgfpathlineto{\pgfqpoint{1.529091in}{1.093518in}}%
\pgfpathlineto{\pgfqpoint{1.519001in}{1.102381in}}%
\pgfpathlineto{\pgfqpoint{1.513061in}{1.107129in}}%
\pgfpathlineto{\pgfqpoint{1.503344in}{1.114839in}}%
\pgfpathlineto{\pgfqpoint{1.494223in}{1.120740in}}%
\pgfpathlineto{\pgfqpoint{1.487688in}{1.125504in}}%
\pgfpathlineto{\pgfqpoint{1.472031in}{1.133637in}}%
\pgfpathlineto{\pgfqpoint{1.469555in}{1.134351in}}%
\pgfpathlineto{\pgfqpoint{1.456375in}{1.139572in}}%
\pgfpathlineto{\pgfqpoint{1.440718in}{1.140607in}}%
\pgfpathlineto{\pgfqpoint{1.425061in}{1.136462in}}%
\pgfpathlineto{\pgfqpoint{1.421436in}{1.134351in}}%
\pgfpathlineto{\pgfqpoint{1.409405in}{1.129201in}}%
\pgfpathlineto{\pgfqpoint{1.396481in}{1.120740in}}%
\pgfpathlineto{\pgfqpoint{1.393748in}{1.119161in}}%
\pgfpathlineto{\pgfqpoint{1.378092in}{1.107506in}}%
\pgfpathlineto{\pgfqpoint{1.377646in}{1.107129in}}%
\pgfpathlineto{\pgfqpoint{1.362435in}{1.094104in}}%
\pgfpathlineto{\pgfqpoint{1.361773in}{1.093518in}}%
\pgfpathlineto{\pgfqpoint{1.348009in}{1.079907in}}%
\pgfpathlineto{\pgfqpoint{1.346779in}{1.078381in}}%
\pgfpathlineto{\pgfqpoint{1.336044in}{1.066295in}}%
\pgfpathlineto{\pgfqpoint{1.331122in}{1.058167in}}%
\pgfpathlineto{\pgfqpoint{1.326801in}{1.052684in}}%
\pgfpathlineto{\pgfqpoint{1.321333in}{1.039073in}}%
\pgfpathlineto{\pgfqpoint{1.321333in}{1.025462in}}%
\pgfpathlineto{\pgfqpoint{1.326801in}{1.011851in}}%
\pgfpathlineto{\pgfqpoint{1.331122in}{1.006368in}}%
\pgfpathlineto{\pgfqpoint{1.336044in}{0.998240in}}%
\pgfpathlineto{\pgfqpoint{1.346779in}{0.986155in}}%
\pgfpathlineto{\pgfqpoint{1.348009in}{0.984629in}}%
\pgfpathlineto{\pgfqpoint{1.361773in}{0.971018in}}%
\pgfpathlineto{\pgfqpoint{1.362435in}{0.970431in}}%
\pgfpathlineto{\pgfqpoint{1.377646in}{0.957407in}}%
\pgfpathlineto{\pgfqpoint{1.378092in}{0.957029in}}%
\pgfpathlineto{\pgfqpoint{1.393748in}{0.945374in}}%
\pgfpathlineto{\pgfqpoint{1.396481in}{0.943795in}}%
\pgfpathlineto{\pgfqpoint{1.409405in}{0.935334in}}%
\pgfpathlineto{\pgfqpoint{1.421436in}{0.930184in}}%
\pgfpathlineto{\pgfqpoint{1.425061in}{0.928073in}}%
\pgfpathclose%
\pgfpathmoveto{\pgfqpoint{1.420098in}{0.957407in}}%
\pgfpathlineto{\pgfqpoint{1.409405in}{0.961567in}}%
\pgfpathlineto{\pgfqpoint{1.393748in}{0.970941in}}%
\pgfpathlineto{\pgfqpoint{1.393647in}{0.971018in}}%
\pgfpathlineto{\pgfqpoint{1.378092in}{0.984074in}}%
\pgfpathlineto{\pgfqpoint{1.377511in}{0.984629in}}%
\pgfpathlineto{\pgfqpoint{1.365965in}{0.998240in}}%
\pgfpathlineto{\pgfqpoint{1.362435in}{1.004346in}}%
\pgfpathlineto{\pgfqpoint{1.358090in}{1.011851in}}%
\pgfpathlineto{\pgfqpoint{1.354093in}{1.025462in}}%
\pgfpathlineto{\pgfqpoint{1.354093in}{1.039073in}}%
\pgfpathlineto{\pgfqpoint{1.358090in}{1.052684in}}%
\pgfpathlineto{\pgfqpoint{1.362435in}{1.060189in}}%
\pgfpathlineto{\pgfqpoint{1.365965in}{1.066295in}}%
\pgfpathlineto{\pgfqpoint{1.377511in}{1.079907in}}%
\pgfpathlineto{\pgfqpoint{1.378092in}{1.080461in}}%
\pgfpathlineto{\pgfqpoint{1.393647in}{1.093518in}}%
\pgfpathlineto{\pgfqpoint{1.393748in}{1.093594in}}%
\pgfpathlineto{\pgfqpoint{1.409405in}{1.102968in}}%
\pgfpathlineto{\pgfqpoint{1.420098in}{1.107129in}}%
\pgfpathlineto{\pgfqpoint{1.425061in}{1.109111in}}%
\pgfpathlineto{\pgfqpoint{1.440718in}{1.111946in}}%
\pgfpathlineto{\pgfqpoint{1.456375in}{1.111238in}}%
\pgfpathlineto{\pgfqpoint{1.471501in}{1.107129in}}%
\pgfpathlineto{\pgfqpoint{1.472031in}{1.106990in}}%
\pgfpathlineto{\pgfqpoint{1.487688in}{1.099616in}}%
\pgfpathlineto{\pgfqpoint{1.496832in}{1.093518in}}%
\pgfpathlineto{\pgfqpoint{1.503344in}{1.088727in}}%
\pgfpathlineto{\pgfqpoint{1.513199in}{1.079907in}}%
\pgfpathlineto{\pgfqpoint{1.519001in}{1.073420in}}%
\pgfpathlineto{\pgfqpoint{1.524913in}{1.066295in}}%
\pgfpathlineto{\pgfqpoint{1.532649in}{1.052684in}}%
\pgfpathlineto{\pgfqpoint{1.534657in}{1.045666in}}%
\pgfpathlineto{\pgfqpoint{1.536663in}{1.039073in}}%
\pgfpathlineto{\pgfqpoint{1.536663in}{1.025462in}}%
\pgfpathlineto{\pgfqpoint{1.534657in}{1.018869in}}%
\pgfpathlineto{\pgfqpoint{1.532649in}{1.011851in}}%
\pgfpathlineto{\pgfqpoint{1.524913in}{0.998240in}}%
\pgfpathlineto{\pgfqpoint{1.519001in}{0.991116in}}%
\pgfpathlineto{\pgfqpoint{1.513199in}{0.984629in}}%
\pgfpathlineto{\pgfqpoint{1.503344in}{0.975809in}}%
\pgfpathlineto{\pgfqpoint{1.496832in}{0.971018in}}%
\pgfpathlineto{\pgfqpoint{1.487688in}{0.964919in}}%
\pgfpathlineto{\pgfqpoint{1.472031in}{0.957545in}}%
\pgfpathlineto{\pgfqpoint{1.471501in}{0.957407in}}%
\pgfpathlineto{\pgfqpoint{1.456375in}{0.953298in}}%
\pgfpathlineto{\pgfqpoint{1.440718in}{0.952589in}}%
\pgfpathlineto{\pgfqpoint{1.425061in}{0.955424in}}%
\pgfpathlineto{\pgfqpoint{1.420098in}{0.957407in}}%
\pgfpathclose%
\pgfpathmoveto{\pgfqpoint{1.738193in}{0.926829in}}%
\pgfpathlineto{\pgfqpoint{1.753849in}{0.923721in}}%
\pgfpathlineto{\pgfqpoint{1.769506in}{0.925793in}}%
\pgfpathlineto{\pgfqpoint{1.779067in}{0.930184in}}%
\pgfpathlineto{\pgfqpoint{1.785162in}{0.932228in}}%
\pgfpathlineto{\pgfqpoint{1.800819in}{0.941100in}}%
\pgfpathlineto{\pgfqpoint{1.804350in}{0.943795in}}%
\pgfpathlineto{\pgfqpoint{1.816476in}{0.952030in}}%
\pgfpathlineto{\pgfqpoint{1.823082in}{0.957407in}}%
\pgfpathlineto{\pgfqpoint{1.832132in}{0.964837in}}%
\pgfpathlineto{\pgfqpoint{1.839122in}{0.971018in}}%
\pgfpathlineto{\pgfqpoint{1.847789in}{0.979626in}}%
\pgfpathlineto{\pgfqpoint{1.853019in}{0.984629in}}%
\pgfpathlineto{\pgfqpoint{1.863445in}{0.997144in}}%
\pgfpathlineto{\pgfqpoint{1.864537in}{0.998240in}}%
\pgfpathlineto{\pgfqpoint{1.874038in}{1.011851in}}%
\pgfpathlineto{\pgfqpoint{1.878779in}{1.025462in}}%
\pgfpathlineto{\pgfqpoint{1.878779in}{1.039073in}}%
\pgfpathlineto{\pgfqpoint{1.874038in}{1.052684in}}%
\pgfpathlineto{\pgfqpoint{1.864537in}{1.066295in}}%
\pgfpathlineto{\pgfqpoint{1.863445in}{1.067391in}}%
\pgfpathlineto{\pgfqpoint{1.853019in}{1.079907in}}%
\pgfpathlineto{\pgfqpoint{1.847789in}{1.084909in}}%
\pgfpathlineto{\pgfqpoint{1.839122in}{1.093518in}}%
\pgfpathlineto{\pgfqpoint{1.832132in}{1.099698in}}%
\pgfpathlineto{\pgfqpoint{1.823082in}{1.107129in}}%
\pgfpathlineto{\pgfqpoint{1.816476in}{1.112505in}}%
\pgfpathlineto{\pgfqpoint{1.804350in}{1.120740in}}%
\pgfpathlineto{\pgfqpoint{1.800819in}{1.123435in}}%
\pgfpathlineto{\pgfqpoint{1.785162in}{1.132307in}}%
\pgfpathlineto{\pgfqpoint{1.779067in}{1.134351in}}%
\pgfpathlineto{\pgfqpoint{1.769506in}{1.138743in}}%
\pgfpathlineto{\pgfqpoint{1.753849in}{1.140814in}}%
\pgfpathlineto{\pgfqpoint{1.738193in}{1.137706in}}%
\pgfpathlineto{\pgfqpoint{1.731758in}{1.134351in}}%
\pgfpathlineto{\pgfqpoint{1.722536in}{1.130828in}}%
\pgfpathlineto{\pgfqpoint{1.706880in}{1.121221in}}%
\pgfpathlineto{\pgfqpoint{1.706275in}{1.120740in}}%
\pgfpathlineto{\pgfqpoint{1.691223in}{1.110059in}}%
\pgfpathlineto{\pgfqpoint{1.687699in}{1.107129in}}%
\pgfpathlineto{\pgfqpoint{1.675567in}{1.096936in}}%
\pgfpathlineto{\pgfqpoint{1.671711in}{1.093518in}}%
\pgfpathlineto{\pgfqpoint{1.659910in}{1.081762in}}%
\pgfpathlineto{\pgfqpoint{1.657931in}{1.079907in}}%
\pgfpathlineto{\pgfqpoint{1.646243in}{1.066295in}}%
\pgfpathlineto{\pgfqpoint{1.644253in}{1.062920in}}%
\pgfpathlineto{\pgfqpoint{1.636683in}{1.052684in}}%
\pgfpathlineto{\pgfqpoint{1.631614in}{1.039073in}}%
\pgfpathlineto{\pgfqpoint{1.631614in}{1.025462in}}%
\pgfpathlineto{\pgfqpoint{1.636683in}{1.011851in}}%
\pgfpathlineto{\pgfqpoint{1.644253in}{1.001616in}}%
\pgfpathlineto{\pgfqpoint{1.646243in}{0.998240in}}%
\pgfpathlineto{\pgfqpoint{1.657931in}{0.984629in}}%
\pgfpathlineto{\pgfqpoint{1.659910in}{0.982773in}}%
\pgfpathlineto{\pgfqpoint{1.671711in}{0.971018in}}%
\pgfpathlineto{\pgfqpoint{1.675567in}{0.967599in}}%
\pgfpathlineto{\pgfqpoint{1.687699in}{0.957407in}}%
\pgfpathlineto{\pgfqpoint{1.691223in}{0.954476in}}%
\pgfpathlineto{\pgfqpoint{1.706275in}{0.943795in}}%
\pgfpathlineto{\pgfqpoint{1.706880in}{0.943315in}}%
\pgfpathlineto{\pgfqpoint{1.722536in}{0.933707in}}%
\pgfpathlineto{\pgfqpoint{1.731758in}{0.930184in}}%
\pgfpathlineto{\pgfqpoint{1.738193in}{0.926829in}}%
\pgfpathclose%
\pgfpathmoveto{\pgfqpoint{1.730270in}{0.957407in}}%
\pgfpathlineto{\pgfqpoint{1.722536in}{0.960092in}}%
\pgfpathlineto{\pgfqpoint{1.706880in}{0.968803in}}%
\pgfpathlineto{\pgfqpoint{1.703836in}{0.971018in}}%
\pgfpathlineto{\pgfqpoint{1.691223in}{0.981196in}}%
\pgfpathlineto{\pgfqpoint{1.687563in}{0.984629in}}%
\pgfpathlineto{\pgfqpoint{1.675962in}{0.998240in}}%
\pgfpathlineto{\pgfqpoint{1.675567in}{0.998912in}}%
\pgfpathlineto{\pgfqpoint{1.668085in}{1.011851in}}%
\pgfpathlineto{\pgfqpoint{1.664150in}{1.025462in}}%
\pgfpathlineto{\pgfqpoint{1.664150in}{1.039073in}}%
\pgfpathlineto{\pgfqpoint{1.668085in}{1.052684in}}%
\pgfpathlineto{\pgfqpoint{1.675567in}{1.065624in}}%
\pgfpathlineto{\pgfqpoint{1.675962in}{1.066295in}}%
\pgfpathlineto{\pgfqpoint{1.687563in}{1.079907in}}%
\pgfpathlineto{\pgfqpoint{1.691223in}{1.083339in}}%
\pgfpathlineto{\pgfqpoint{1.703836in}{1.093518in}}%
\pgfpathlineto{\pgfqpoint{1.706880in}{1.095733in}}%
\pgfpathlineto{\pgfqpoint{1.722536in}{1.104443in}}%
\pgfpathlineto{\pgfqpoint{1.730270in}{1.107129in}}%
\pgfpathlineto{\pgfqpoint{1.738193in}{1.109962in}}%
\pgfpathlineto{\pgfqpoint{1.753849in}{1.112088in}}%
\pgfpathlineto{\pgfqpoint{1.769506in}{1.110671in}}%
\pgfpathlineto{\pgfqpoint{1.780750in}{1.107129in}}%
\pgfpathlineto{\pgfqpoint{1.785162in}{1.105784in}}%
\pgfpathlineto{\pgfqpoint{1.800819in}{1.097740in}}%
\pgfpathlineto{\pgfqpoint{1.806866in}{1.093518in}}%
\pgfpathlineto{\pgfqpoint{1.816476in}{1.086096in}}%
\pgfpathlineto{\pgfqpoint{1.823219in}{1.079907in}}%
\pgfpathlineto{\pgfqpoint{1.832132in}{1.069668in}}%
\pgfpathlineto{\pgfqpoint{1.834912in}{1.066295in}}%
\pgfpathlineto{\pgfqpoint{1.842706in}{1.052684in}}%
\pgfpathlineto{\pgfqpoint{1.846596in}{1.039073in}}%
\pgfpathlineto{\pgfqpoint{1.846596in}{1.025462in}}%
\pgfpathlineto{\pgfqpoint{1.842706in}{1.011851in}}%
\pgfpathlineto{\pgfqpoint{1.834912in}{0.998240in}}%
\pgfpathlineto{\pgfqpoint{1.832132in}{0.994868in}}%
\pgfpathlineto{\pgfqpoint{1.823219in}{0.984629in}}%
\pgfpathlineto{\pgfqpoint{1.816476in}{0.978440in}}%
\pgfpathlineto{\pgfqpoint{1.806866in}{0.971018in}}%
\pgfpathlineto{\pgfqpoint{1.800819in}{0.966795in}}%
\pgfpathlineto{\pgfqpoint{1.785162in}{0.958752in}}%
\pgfpathlineto{\pgfqpoint{1.780750in}{0.957407in}}%
\pgfpathlineto{\pgfqpoint{1.769506in}{0.953864in}}%
\pgfpathlineto{\pgfqpoint{1.753849in}{0.952448in}}%
\pgfpathlineto{\pgfqpoint{1.738193in}{0.954573in}}%
\pgfpathlineto{\pgfqpoint{1.730270in}{0.957407in}}%
\pgfpathclose%
\pgfpathmoveto{\pgfqpoint{0.485668in}{1.201888in}}%
\pgfpathlineto{\pgfqpoint{0.501324in}{1.195226in}}%
\pgfpathlineto{\pgfqpoint{0.516981in}{1.193326in}}%
\pgfpathlineto{\pgfqpoint{0.532637in}{1.196177in}}%
\pgfpathlineto{\pgfqpoint{0.545510in}{1.202407in}}%
\pgfpathlineto{\pgfqpoint{0.548294in}{1.203435in}}%
\pgfpathlineto{\pgfqpoint{0.563950in}{1.212615in}}%
\pgfpathlineto{\pgfqpoint{0.568363in}{1.216018in}}%
\pgfpathlineto{\pgfqpoint{0.579607in}{1.223948in}}%
\pgfpathlineto{\pgfqpoint{0.586478in}{1.229629in}}%
\pgfpathlineto{\pgfqpoint{0.595263in}{1.237122in}}%
\pgfpathlineto{\pgfqpoint{0.602067in}{1.243240in}}%
\pgfpathlineto{\pgfqpoint{0.610920in}{1.252374in}}%
\pgfpathlineto{\pgfqpoint{0.615541in}{1.256851in}}%
\pgfpathlineto{\pgfqpoint{0.626462in}{1.270462in}}%
\pgfpathlineto{\pgfqpoint{0.626577in}{1.270677in}}%
\pgfpathlineto{\pgfqpoint{0.635567in}{1.284073in}}%
\pgfpathlineto{\pgfqpoint{0.639620in}{1.297684in}}%
\pgfpathlineto{\pgfqpoint{0.638608in}{1.311295in}}%
\pgfpathlineto{\pgfqpoint{0.632523in}{1.324907in}}%
\pgfpathlineto{\pgfqpoint{0.626577in}{1.332281in}}%
\pgfpathlineto{\pgfqpoint{0.622557in}{1.338518in}}%
\pgfpathlineto{\pgfqpoint{0.610920in}{1.351287in}}%
\pgfpathlineto{\pgfqpoint{0.610205in}{1.352129in}}%
\pgfpathlineto{\pgfqpoint{0.596064in}{1.365740in}}%
\pgfpathlineto{\pgfqpoint{0.595263in}{1.366442in}}%
\pgfpathlineto{\pgfqpoint{0.579726in}{1.379351in}}%
\pgfpathlineto{\pgfqpoint{0.579607in}{1.379451in}}%
\pgfpathlineto{\pgfqpoint{0.563950in}{1.390812in}}%
\pgfpathlineto{\pgfqpoint{0.560018in}{1.392962in}}%
\pgfpathlineto{\pgfqpoint{0.548294in}{1.400453in}}%
\pgfpathlineto{\pgfqpoint{0.532909in}{1.406573in}}%
\pgfpathlineto{\pgfqpoint{0.532637in}{1.406728in}}%
\pgfpathlineto{\pgfqpoint{0.516981in}{1.410160in}}%
\pgfpathlineto{\pgfqpoint{0.501324in}{1.407872in}}%
\pgfpathlineto{\pgfqpoint{0.498733in}{1.406573in}}%
\pgfpathlineto{\pgfqpoint{0.485668in}{1.402010in}}%
\pgfpathlineto{\pgfqpoint{0.470487in}{1.392962in}}%
\pgfpathlineto{\pgfqpoint{0.470011in}{1.392720in}}%
\pgfpathlineto{\pgfqpoint{0.454354in}{1.381958in}}%
\pgfpathlineto{\pgfqpoint{0.451183in}{1.379351in}}%
\pgfpathlineto{\pgfqpoint{0.438698in}{1.369213in}}%
\pgfpathlineto{\pgfqpoint{0.434728in}{1.365740in}}%
\pgfpathlineto{\pgfqpoint{0.423041in}{1.354485in}}%
\pgfpathlineto{\pgfqpoint{0.420505in}{1.352129in}}%
\pgfpathlineto{\pgfqpoint{0.408479in}{1.338518in}}%
\pgfpathlineto{\pgfqpoint{0.407385in}{1.336783in}}%
\pgfpathlineto{\pgfqpoint{0.398311in}{1.324907in}}%
\pgfpathlineto{\pgfqpoint{0.392619in}{1.311295in}}%
\pgfpathlineto{\pgfqpoint{0.391728in}{1.298491in}}%
\pgfpathlineto{\pgfqpoint{0.391641in}{1.297684in}}%
\pgfpathlineto{\pgfqpoint{0.391728in}{1.297481in}}%
\pgfpathlineto{\pgfqpoint{0.395464in}{1.284073in}}%
\pgfpathlineto{\pgfqpoint{0.404012in}{1.270462in}}%
\pgfpathlineto{\pgfqpoint{0.407385in}{1.266881in}}%
\pgfpathlineto{\pgfqpoint{0.415258in}{1.256851in}}%
\pgfpathlineto{\pgfqpoint{0.423041in}{1.249158in}}%
\pgfpathlineto{\pgfqpoint{0.428794in}{1.243240in}}%
\pgfpathlineto{\pgfqpoint{0.438698in}{1.234359in}}%
\pgfpathlineto{\pgfqpoint{0.444372in}{1.229629in}}%
\pgfpathlineto{\pgfqpoint{0.454354in}{1.221551in}}%
\pgfpathlineto{\pgfqpoint{0.462551in}{1.216018in}}%
\pgfpathlineto{\pgfqpoint{0.470011in}{1.210499in}}%
\pgfpathlineto{\pgfqpoint{0.484928in}{1.202407in}}%
\pgfpathlineto{\pgfqpoint{0.485668in}{1.201888in}}%
\pgfpathclose%
\pgfpathmoveto{\pgfqpoint{0.482929in}{1.229629in}}%
\pgfpathlineto{\pgfqpoint{0.470011in}{1.236318in}}%
\pgfpathlineto{\pgfqpoint{0.460237in}{1.243240in}}%
\pgfpathlineto{\pgfqpoint{0.454354in}{1.247946in}}%
\pgfpathlineto{\pgfqpoint{0.444967in}{1.256851in}}%
\pgfpathlineto{\pgfqpoint{0.438698in}{1.264492in}}%
\pgfpathlineto{\pgfqpoint{0.434047in}{1.270462in}}%
\pgfpathlineto{\pgfqpoint{0.427034in}{1.284073in}}%
\pgfpathlineto{\pgfqpoint{0.423924in}{1.297684in}}%
\pgfpathlineto{\pgfqpoint{0.424701in}{1.311295in}}%
\pgfpathlineto{\pgfqpoint{0.429370in}{1.324907in}}%
\pgfpathlineto{\pgfqpoint{0.437944in}{1.338518in}}%
\pgfpathlineto{\pgfqpoint{0.438698in}{1.339386in}}%
\pgfpathlineto{\pgfqpoint{0.450404in}{1.352129in}}%
\pgfpathlineto{\pgfqpoint{0.454354in}{1.355651in}}%
\pgfpathlineto{\pgfqpoint{0.467827in}{1.365740in}}%
\pgfpathlineto{\pgfqpoint{0.470011in}{1.367249in}}%
\pgfpathlineto{\pgfqpoint{0.485668in}{1.375319in}}%
\pgfpathlineto{\pgfqpoint{0.499040in}{1.379351in}}%
\pgfpathlineto{\pgfqpoint{0.501324in}{1.380078in}}%
\pgfpathlineto{\pgfqpoint{0.516981in}{1.381530in}}%
\pgfpathlineto{\pgfqpoint{0.532637in}{1.379351in}}%
\pgfpathlineto{\pgfqpoint{0.532638in}{1.379351in}}%
\pgfpathlineto{\pgfqpoint{0.548294in}{1.373974in}}%
\pgfpathlineto{\pgfqpoint{0.563083in}{1.365740in}}%
\pgfpathlineto{\pgfqpoint{0.563950in}{1.365222in}}%
\pgfpathlineto{\pgfqpoint{0.579607in}{1.352942in}}%
\pgfpathlineto{\pgfqpoint{0.580499in}{1.352129in}}%
\pgfpathlineto{\pgfqpoint{0.592852in}{1.338518in}}%
\pgfpathlineto{\pgfqpoint{0.595263in}{1.334776in}}%
\pgfpathlineto{\pgfqpoint{0.601485in}{1.324907in}}%
\pgfpathlineto{\pgfqpoint{0.606208in}{1.311295in}}%
\pgfpathlineto{\pgfqpoint{0.606994in}{1.297684in}}%
\pgfpathlineto{\pgfqpoint{0.603848in}{1.284073in}}%
\pgfpathlineto{\pgfqpoint{0.596754in}{1.270462in}}%
\pgfpathlineto{\pgfqpoint{0.595263in}{1.268544in}}%
\pgfpathlineto{\pgfqpoint{0.585889in}{1.256851in}}%
\pgfpathlineto{\pgfqpoint{0.579607in}{1.250763in}}%
\pgfpathlineto{\pgfqpoint{0.570606in}{1.243240in}}%
\pgfpathlineto{\pgfqpoint{0.563950in}{1.238327in}}%
\pgfpathlineto{\pgfqpoint{0.548328in}{1.229629in}}%
\pgfpathlineto{\pgfqpoint{0.548294in}{1.229610in}}%
\pgfpathlineto{\pgfqpoint{0.532637in}{1.224043in}}%
\pgfpathlineto{\pgfqpoint{0.516981in}{1.221960in}}%
\pgfpathlineto{\pgfqpoint{0.501324in}{1.223348in}}%
\pgfpathlineto{\pgfqpoint{0.485668in}{1.228217in}}%
\pgfpathlineto{\pgfqpoint{0.482929in}{1.229629in}}%
\pgfpathclose%
\pgfpathmoveto{\pgfqpoint{0.798799in}{1.200173in}}%
\pgfpathlineto{\pgfqpoint{0.814455in}{1.194466in}}%
\pgfpathlineto{\pgfqpoint{0.830112in}{1.193516in}}%
\pgfpathlineto{\pgfqpoint{0.845769in}{1.197318in}}%
\pgfpathlineto{\pgfqpoint{0.855184in}{1.202407in}}%
\pgfpathlineto{\pgfqpoint{0.861425in}{1.204990in}}%
\pgfpathlineto{\pgfqpoint{0.877082in}{1.214869in}}%
\pgfpathlineto{\pgfqpoint{0.878518in}{1.216018in}}%
\pgfpathlineto{\pgfqpoint{0.892738in}{1.226450in}}%
\pgfpathlineto{\pgfqpoint{0.896515in}{1.229629in}}%
\pgfpathlineto{\pgfqpoint{0.908395in}{1.239957in}}%
\pgfpathlineto{\pgfqpoint{0.912051in}{1.243240in}}%
\pgfpathlineto{\pgfqpoint{0.924051in}{1.255603in}}%
\pgfpathlineto{\pgfqpoint{0.925373in}{1.256851in}}%
\pgfpathlineto{\pgfqpoint{0.936737in}{1.270462in}}%
\pgfpathlineto{\pgfqpoint{0.939708in}{1.275888in}}%
\pgfpathlineto{\pgfqpoint{0.945561in}{1.284073in}}%
\pgfpathlineto{\pgfqpoint{0.949934in}{1.297684in}}%
\pgfpathlineto{\pgfqpoint{0.948842in}{1.311295in}}%
\pgfpathlineto{\pgfqpoint{0.942277in}{1.324907in}}%
\pgfpathlineto{\pgfqpoint{0.939708in}{1.327896in}}%
\pgfpathlineto{\pgfqpoint{0.932673in}{1.338518in}}%
\pgfpathlineto{\pgfqpoint{0.924051in}{1.347741in}}%
\pgfpathlineto{\pgfqpoint{0.920319in}{1.352129in}}%
\pgfpathlineto{\pgfqpoint{0.908395in}{1.363549in}}%
\pgfpathlineto{\pgfqpoint{0.906043in}{1.365740in}}%
\pgfpathlineto{\pgfqpoint{0.892738in}{1.377027in}}%
\pgfpathlineto{\pgfqpoint{0.889596in}{1.379351in}}%
\pgfpathlineto{\pgfqpoint{0.877082in}{1.388779in}}%
\pgfpathlineto{\pgfqpoint{0.869919in}{1.392962in}}%
\pgfpathlineto{\pgfqpoint{0.861425in}{1.398739in}}%
\pgfpathlineto{\pgfqpoint{0.845769in}{1.405745in}}%
\pgfpathlineto{\pgfqpoint{0.841643in}{1.406573in}}%
\pgfpathlineto{\pgfqpoint{0.830112in}{1.409931in}}%
\pgfpathlineto{\pgfqpoint{0.814455in}{1.408788in}}%
\pgfpathlineto{\pgfqpoint{0.809335in}{1.406573in}}%
\pgfpathlineto{\pgfqpoint{0.798799in}{1.403412in}}%
\pgfpathlineto{\pgfqpoint{0.783142in}{1.394845in}}%
\pgfpathlineto{\pgfqpoint{0.780656in}{1.392962in}}%
\pgfpathlineto{\pgfqpoint{0.767486in}{1.384350in}}%
\pgfpathlineto{\pgfqpoint{0.761249in}{1.379351in}}%
\pgfpathlineto{\pgfqpoint{0.751829in}{1.371906in}}%
\pgfpathlineto{\pgfqpoint{0.744737in}{1.365740in}}%
\pgfpathlineto{\pgfqpoint{0.736173in}{1.357550in}}%
\pgfpathlineto{\pgfqpoint{0.730422in}{1.352129in}}%
\pgfpathlineto{\pgfqpoint{0.720516in}{1.340679in}}%
\pgfpathlineto{\pgfqpoint{0.718350in}{1.338518in}}%
\pgfpathlineto{\pgfqpoint{0.708496in}{1.324907in}}%
\pgfpathlineto{\pgfqpoint{0.704859in}{1.315747in}}%
\pgfpathlineto{\pgfqpoint{0.702312in}{1.311295in}}%
\pgfpathlineto{\pgfqpoint{0.700997in}{1.297684in}}%
\pgfpathlineto{\pgfqpoint{0.704859in}{1.287660in}}%
\pgfpathlineto{\pgfqpoint{0.705812in}{1.284073in}}%
\pgfpathlineto{\pgfqpoint{0.713871in}{1.270462in}}%
\pgfpathlineto{\pgfqpoint{0.720516in}{1.263078in}}%
\pgfpathlineto{\pgfqpoint{0.725327in}{1.256851in}}%
\pgfpathlineto{\pgfqpoint{0.736173in}{1.245971in}}%
\pgfpathlineto{\pgfqpoint{0.738846in}{1.243240in}}%
\pgfpathlineto{\pgfqpoint{0.751829in}{1.231674in}}%
\pgfpathlineto{\pgfqpoint{0.754349in}{1.229629in}}%
\pgfpathlineto{\pgfqpoint{0.767486in}{1.219263in}}%
\pgfpathlineto{\pgfqpoint{0.772533in}{1.216018in}}%
\pgfpathlineto{\pgfqpoint{0.783142in}{1.208522in}}%
\pgfpathlineto{\pgfqpoint{0.795360in}{1.202407in}}%
\pgfpathlineto{\pgfqpoint{0.798799in}{1.200173in}}%
\pgfpathclose%
\pgfpathmoveto{\pgfqpoint{0.793214in}{1.229629in}}%
\pgfpathlineto{\pgfqpoint{0.783142in}{1.234441in}}%
\pgfpathlineto{\pgfqpoint{0.770133in}{1.243240in}}%
\pgfpathlineto{\pgfqpoint{0.767486in}{1.245257in}}%
\pgfpathlineto{\pgfqpoint{0.754951in}{1.256851in}}%
\pgfpathlineto{\pgfqpoint{0.751829in}{1.260556in}}%
\pgfpathlineto{\pgfqpoint{0.744060in}{1.270462in}}%
\pgfpathlineto{\pgfqpoint{0.737099in}{1.284073in}}%
\pgfpathlineto{\pgfqpoint{0.736173in}{1.288116in}}%
\pgfpathlineto{\pgfqpoint{0.733833in}{1.297684in}}%
\pgfpathlineto{\pgfqpoint{0.734668in}{1.311295in}}%
\pgfpathlineto{\pgfqpoint{0.736173in}{1.315439in}}%
\pgfpathlineto{\pgfqpoint{0.739418in}{1.324907in}}%
\pgfpathlineto{\pgfqpoint{0.747928in}{1.338518in}}%
\pgfpathlineto{\pgfqpoint{0.751829in}{1.342984in}}%
\pgfpathlineto{\pgfqpoint{0.760459in}{1.352129in}}%
\pgfpathlineto{\pgfqpoint{0.767486in}{1.358237in}}%
\pgfpathlineto{\pgfqpoint{0.778004in}{1.365740in}}%
\pgfpathlineto{\pgfqpoint{0.783142in}{1.369131in}}%
\pgfpathlineto{\pgfqpoint{0.798799in}{1.376530in}}%
\pgfpathlineto{\pgfqpoint{0.809689in}{1.379351in}}%
\pgfpathlineto{\pgfqpoint{0.814455in}{1.380659in}}%
\pgfpathlineto{\pgfqpoint{0.830112in}{1.381385in}}%
\pgfpathlineto{\pgfqpoint{0.841118in}{1.379351in}}%
\pgfpathlineto{\pgfqpoint{0.845769in}{1.378546in}}%
\pgfpathlineto{\pgfqpoint{0.861425in}{1.372494in}}%
\pgfpathlineto{\pgfqpoint{0.872820in}{1.365740in}}%
\pgfpathlineto{\pgfqpoint{0.877082in}{1.363026in}}%
\pgfpathlineto{\pgfqpoint{0.890418in}{1.352129in}}%
\pgfpathlineto{\pgfqpoint{0.892738in}{1.349827in}}%
\pgfpathlineto{\pgfqpoint{0.902859in}{1.338518in}}%
\pgfpathlineto{\pgfqpoint{0.908395in}{1.329762in}}%
\pgfpathlineto{\pgfqpoint{0.911460in}{1.324907in}}%
\pgfpathlineto{\pgfqpoint{0.916258in}{1.311295in}}%
\pgfpathlineto{\pgfqpoint{0.917057in}{1.297684in}}%
\pgfpathlineto{\pgfqpoint{0.913860in}{1.284073in}}%
\pgfpathlineto{\pgfqpoint{0.908395in}{1.273671in}}%
\pgfpathlineto{\pgfqpoint{0.906718in}{1.270462in}}%
\pgfpathlineto{\pgfqpoint{0.895929in}{1.256851in}}%
\pgfpathlineto{\pgfqpoint{0.892738in}{1.253703in}}%
\pgfpathlineto{\pgfqpoint{0.880702in}{1.243240in}}%
\pgfpathlineto{\pgfqpoint{0.877082in}{1.240466in}}%
\pgfpathlineto{\pgfqpoint{0.861425in}{1.231087in}}%
\pgfpathlineto{\pgfqpoint{0.857734in}{1.229629in}}%
\pgfpathlineto{\pgfqpoint{0.845769in}{1.224877in}}%
\pgfpathlineto{\pgfqpoint{0.830112in}{1.222099in}}%
\pgfpathlineto{\pgfqpoint{0.814455in}{1.222793in}}%
\pgfpathlineto{\pgfqpoint{0.798799in}{1.226964in}}%
\pgfpathlineto{\pgfqpoint{0.793214in}{1.229629in}}%
\pgfpathclose%
\pgfpathmoveto{\pgfqpoint{1.111930in}{1.198650in}}%
\pgfpathlineto{\pgfqpoint{1.127587in}{1.193896in}}%
\pgfpathlineto{\pgfqpoint{1.143243in}{1.193896in}}%
\pgfpathlineto{\pgfqpoint{1.158900in}{1.198650in}}%
\pgfpathlineto{\pgfqpoint{1.165207in}{1.202407in}}%
\pgfpathlineto{\pgfqpoint{1.174556in}{1.206686in}}%
\pgfpathlineto{\pgfqpoint{1.188458in}{1.216018in}}%
\pgfpathlineto{\pgfqpoint{1.190213in}{1.217087in}}%
\pgfpathlineto{\pgfqpoint{1.205870in}{1.229053in}}%
\pgfpathlineto{\pgfqpoint{1.206544in}{1.229629in}}%
\pgfpathlineto{\pgfqpoint{1.221526in}{1.242852in}}%
\pgfpathlineto{\pgfqpoint{1.221960in}{1.243240in}}%
\pgfpathlineto{\pgfqpoint{1.235367in}{1.256851in}}%
\pgfpathlineto{\pgfqpoint{1.237183in}{1.259227in}}%
\pgfpathlineto{\pgfqpoint{1.246915in}{1.270462in}}%
\pgfpathlineto{\pgfqpoint{1.252839in}{1.280922in}}%
\pgfpathlineto{\pgfqpoint{1.255267in}{1.284073in}}%
\pgfpathlineto{\pgfqpoint{1.260036in}{1.297684in}}%
\pgfpathlineto{\pgfqpoint{1.258844in}{1.311295in}}%
\pgfpathlineto{\pgfqpoint{1.252839in}{1.322754in}}%
\pgfpathlineto{\pgfqpoint{1.252018in}{1.324907in}}%
\pgfpathlineto{\pgfqpoint{1.242663in}{1.338518in}}%
\pgfpathlineto{\pgfqpoint{1.237183in}{1.344199in}}%
\pgfpathlineto{\pgfqpoint{1.230395in}{1.352129in}}%
\pgfpathlineto{\pgfqpoint{1.221526in}{1.360576in}}%
\pgfpathlineto{\pgfqpoint{1.216065in}{1.365740in}}%
\pgfpathlineto{\pgfqpoint{1.205870in}{1.374512in}}%
\pgfpathlineto{\pgfqpoint{1.199555in}{1.379351in}}%
\pgfpathlineto{\pgfqpoint{1.190213in}{1.386625in}}%
\pgfpathlineto{\pgfqpoint{1.179986in}{1.392962in}}%
\pgfpathlineto{\pgfqpoint{1.174556in}{1.396870in}}%
\pgfpathlineto{\pgfqpoint{1.158900in}{1.404657in}}%
\pgfpathlineto{\pgfqpoint{1.151249in}{1.406573in}}%
\pgfpathlineto{\pgfqpoint{1.143243in}{1.409474in}}%
\pgfpathlineto{\pgfqpoint{1.127587in}{1.409474in}}%
\pgfpathlineto{\pgfqpoint{1.119581in}{1.406573in}}%
\pgfpathlineto{\pgfqpoint{1.111930in}{1.404657in}}%
\pgfpathlineto{\pgfqpoint{1.096274in}{1.396870in}}%
\pgfpathlineto{\pgfqpoint{1.090844in}{1.392962in}}%
\pgfpathlineto{\pgfqpoint{1.080617in}{1.386625in}}%
\pgfpathlineto{\pgfqpoint{1.071275in}{1.379351in}}%
\pgfpathlineto{\pgfqpoint{1.064960in}{1.374512in}}%
\pgfpathlineto{\pgfqpoint{1.054765in}{1.365740in}}%
\pgfpathlineto{\pgfqpoint{1.049304in}{1.360576in}}%
\pgfpathlineto{\pgfqpoint{1.040435in}{1.352129in}}%
\pgfpathlineto{\pgfqpoint{1.033647in}{1.344199in}}%
\pgfpathlineto{\pgfqpoint{1.028167in}{1.338518in}}%
\pgfpathlineto{\pgfqpoint{1.018812in}{1.324907in}}%
\pgfpathlineto{\pgfqpoint{1.017991in}{1.322754in}}%
\pgfpathlineto{\pgfqpoint{1.011986in}{1.311295in}}%
\pgfpathlineto{\pgfqpoint{1.010794in}{1.297684in}}%
\pgfpathlineto{\pgfqpoint{1.015563in}{1.284073in}}%
\pgfpathlineto{\pgfqpoint{1.017991in}{1.280922in}}%
\pgfpathlineto{\pgfqpoint{1.023915in}{1.270462in}}%
\pgfpathlineto{\pgfqpoint{1.033647in}{1.259227in}}%
\pgfpathlineto{\pgfqpoint{1.035463in}{1.256851in}}%
\pgfpathlineto{\pgfqpoint{1.048870in}{1.243240in}}%
\pgfpathlineto{\pgfqpoint{1.049304in}{1.242852in}}%
\pgfpathlineto{\pgfqpoint{1.064286in}{1.229629in}}%
\pgfpathlineto{\pgfqpoint{1.064960in}{1.229053in}}%
\pgfpathlineto{\pgfqpoint{1.080617in}{1.217087in}}%
\pgfpathlineto{\pgfqpoint{1.082372in}{1.216018in}}%
\pgfpathlineto{\pgfqpoint{1.096274in}{1.206686in}}%
\pgfpathlineto{\pgfqpoint{1.105623in}{1.202407in}}%
\pgfpathlineto{\pgfqpoint{1.111930in}{1.198650in}}%
\pgfpathclose%
\pgfpathmoveto{\pgfqpoint{1.103297in}{1.229629in}}%
\pgfpathlineto{\pgfqpoint{1.096274in}{1.232697in}}%
\pgfpathlineto{\pgfqpoint{1.080617in}{1.242735in}}%
\pgfpathlineto{\pgfqpoint{1.079979in}{1.243240in}}%
\pgfpathlineto{\pgfqpoint{1.064960in}{1.256763in}}%
\pgfpathlineto{\pgfqpoint{1.064872in}{1.256851in}}%
\pgfpathlineto{\pgfqpoint{1.054090in}{1.270462in}}%
\pgfpathlineto{\pgfqpoint{1.049304in}{1.279758in}}%
\pgfpathlineto{\pgfqpoint{1.047024in}{1.284073in}}%
\pgfpathlineto{\pgfqpoint{1.043763in}{1.297684in}}%
\pgfpathlineto{\pgfqpoint{1.044578in}{1.311295in}}%
\pgfpathlineto{\pgfqpoint{1.049304in}{1.324446in}}%
\pgfpathlineto{\pgfqpoint{1.049464in}{1.324907in}}%
\pgfpathlineto{\pgfqpoint{1.057946in}{1.338518in}}%
\pgfpathlineto{\pgfqpoint{1.064960in}{1.346467in}}%
\pgfpathlineto{\pgfqpoint{1.070472in}{1.352129in}}%
\pgfpathlineto{\pgfqpoint{1.080617in}{1.360696in}}%
\pgfpathlineto{\pgfqpoint{1.088079in}{1.365740in}}%
\pgfpathlineto{\pgfqpoint{1.096274in}{1.370880in}}%
\pgfpathlineto{\pgfqpoint{1.111930in}{1.377605in}}%
\pgfpathlineto{\pgfqpoint{1.120003in}{1.379351in}}%
\pgfpathlineto{\pgfqpoint{1.127587in}{1.381094in}}%
\pgfpathlineto{\pgfqpoint{1.143243in}{1.381094in}}%
\pgfpathlineto{\pgfqpoint{1.150827in}{1.379351in}}%
\pgfpathlineto{\pgfqpoint{1.158900in}{1.377605in}}%
\pgfpathlineto{\pgfqpoint{1.174556in}{1.370880in}}%
\pgfpathlineto{\pgfqpoint{1.182751in}{1.365740in}}%
\pgfpathlineto{\pgfqpoint{1.190213in}{1.360696in}}%
\pgfpathlineto{\pgfqpoint{1.200358in}{1.352129in}}%
\pgfpathlineto{\pgfqpoint{1.205870in}{1.346467in}}%
\pgfpathlineto{\pgfqpoint{1.212884in}{1.338518in}}%
\pgfpathlineto{\pgfqpoint{1.221366in}{1.324907in}}%
\pgfpathlineto{\pgfqpoint{1.221526in}{1.324446in}}%
\pgfpathlineto{\pgfqpoint{1.226252in}{1.311295in}}%
\pgfpathlineto{\pgfqpoint{1.227067in}{1.297684in}}%
\pgfpathlineto{\pgfqpoint{1.223806in}{1.284073in}}%
\pgfpathlineto{\pgfqpoint{1.221526in}{1.279758in}}%
\pgfpathlineto{\pgfqpoint{1.216740in}{1.270462in}}%
\pgfpathlineto{\pgfqpoint{1.205958in}{1.256851in}}%
\pgfpathlineto{\pgfqpoint{1.205870in}{1.256763in}}%
\pgfpathlineto{\pgfqpoint{1.190851in}{1.243240in}}%
\pgfpathlineto{\pgfqpoint{1.190213in}{1.242735in}}%
\pgfpathlineto{\pgfqpoint{1.174556in}{1.232697in}}%
\pgfpathlineto{\pgfqpoint{1.167533in}{1.229629in}}%
\pgfpathlineto{\pgfqpoint{1.158900in}{1.225851in}}%
\pgfpathlineto{\pgfqpoint{1.143243in}{1.222376in}}%
\pgfpathlineto{\pgfqpoint{1.127587in}{1.222376in}}%
\pgfpathlineto{\pgfqpoint{1.111930in}{1.225851in}}%
\pgfpathlineto{\pgfqpoint{1.103297in}{1.229629in}}%
\pgfpathclose%
\pgfpathmoveto{\pgfqpoint{1.425061in}{1.197318in}}%
\pgfpathlineto{\pgfqpoint{1.440718in}{1.193516in}}%
\pgfpathlineto{\pgfqpoint{1.456375in}{1.194466in}}%
\pgfpathlineto{\pgfqpoint{1.472031in}{1.200173in}}%
\pgfpathlineto{\pgfqpoint{1.475470in}{1.202407in}}%
\pgfpathlineto{\pgfqpoint{1.487688in}{1.208522in}}%
\pgfpathlineto{\pgfqpoint{1.498297in}{1.216018in}}%
\pgfpathlineto{\pgfqpoint{1.503344in}{1.219263in}}%
\pgfpathlineto{\pgfqpoint{1.516481in}{1.229629in}}%
\pgfpathlineto{\pgfqpoint{1.519001in}{1.231674in}}%
\pgfpathlineto{\pgfqpoint{1.531984in}{1.243240in}}%
\pgfpathlineto{\pgfqpoint{1.534657in}{1.245971in}}%
\pgfpathlineto{\pgfqpoint{1.545503in}{1.256851in}}%
\pgfpathlineto{\pgfqpoint{1.550314in}{1.263078in}}%
\pgfpathlineto{\pgfqpoint{1.556959in}{1.270462in}}%
\pgfpathlineto{\pgfqpoint{1.565018in}{1.284073in}}%
\pgfpathlineto{\pgfqpoint{1.565971in}{1.287660in}}%
\pgfpathlineto{\pgfqpoint{1.569833in}{1.297684in}}%
\pgfpathlineto{\pgfqpoint{1.568518in}{1.311295in}}%
\pgfpathlineto{\pgfqpoint{1.565971in}{1.315747in}}%
\pgfpathlineto{\pgfqpoint{1.562334in}{1.324907in}}%
\pgfpathlineto{\pgfqpoint{1.552480in}{1.338518in}}%
\pgfpathlineto{\pgfqpoint{1.550314in}{1.340679in}}%
\pgfpathlineto{\pgfqpoint{1.540408in}{1.352129in}}%
\pgfpathlineto{\pgfqpoint{1.534657in}{1.357550in}}%
\pgfpathlineto{\pgfqpoint{1.526093in}{1.365740in}}%
\pgfpathlineto{\pgfqpoint{1.519001in}{1.371906in}}%
\pgfpathlineto{\pgfqpoint{1.509581in}{1.379351in}}%
\pgfpathlineto{\pgfqpoint{1.503344in}{1.384350in}}%
\pgfpathlineto{\pgfqpoint{1.490174in}{1.392962in}}%
\pgfpathlineto{\pgfqpoint{1.487688in}{1.394845in}}%
\pgfpathlineto{\pgfqpoint{1.472031in}{1.403412in}}%
\pgfpathlineto{\pgfqpoint{1.461495in}{1.406573in}}%
\pgfpathlineto{\pgfqpoint{1.456375in}{1.408788in}}%
\pgfpathlineto{\pgfqpoint{1.440718in}{1.409931in}}%
\pgfpathlineto{\pgfqpoint{1.429187in}{1.406573in}}%
\pgfpathlineto{\pgfqpoint{1.425061in}{1.405745in}}%
\pgfpathlineto{\pgfqpoint{1.409405in}{1.398739in}}%
\pgfpathlineto{\pgfqpoint{1.400911in}{1.392962in}}%
\pgfpathlineto{\pgfqpoint{1.393748in}{1.388779in}}%
\pgfpathlineto{\pgfqpoint{1.381234in}{1.379351in}}%
\pgfpathlineto{\pgfqpoint{1.378092in}{1.377027in}}%
\pgfpathlineto{\pgfqpoint{1.364787in}{1.365740in}}%
\pgfpathlineto{\pgfqpoint{1.362435in}{1.363549in}}%
\pgfpathlineto{\pgfqpoint{1.350511in}{1.352129in}}%
\pgfpathlineto{\pgfqpoint{1.346779in}{1.347741in}}%
\pgfpathlineto{\pgfqpoint{1.338157in}{1.338518in}}%
\pgfpathlineto{\pgfqpoint{1.331122in}{1.327896in}}%
\pgfpathlineto{\pgfqpoint{1.328553in}{1.324907in}}%
\pgfpathlineto{\pgfqpoint{1.321988in}{1.311295in}}%
\pgfpathlineto{\pgfqpoint{1.320896in}{1.297684in}}%
\pgfpathlineto{\pgfqpoint{1.325269in}{1.284073in}}%
\pgfpathlineto{\pgfqpoint{1.331122in}{1.275888in}}%
\pgfpathlineto{\pgfqpoint{1.334093in}{1.270462in}}%
\pgfpathlineto{\pgfqpoint{1.345457in}{1.256851in}}%
\pgfpathlineto{\pgfqpoint{1.346779in}{1.255603in}}%
\pgfpathlineto{\pgfqpoint{1.358779in}{1.243240in}}%
\pgfpathlineto{\pgfqpoint{1.362435in}{1.239957in}}%
\pgfpathlineto{\pgfqpoint{1.374315in}{1.229629in}}%
\pgfpathlineto{\pgfqpoint{1.378092in}{1.226450in}}%
\pgfpathlineto{\pgfqpoint{1.392312in}{1.216018in}}%
\pgfpathlineto{\pgfqpoint{1.393748in}{1.214869in}}%
\pgfpathlineto{\pgfqpoint{1.409405in}{1.204990in}}%
\pgfpathlineto{\pgfqpoint{1.415646in}{1.202407in}}%
\pgfpathlineto{\pgfqpoint{1.425061in}{1.197318in}}%
\pgfpathclose%
\pgfpathmoveto{\pgfqpoint{1.413096in}{1.229629in}}%
\pgfpathlineto{\pgfqpoint{1.409405in}{1.231087in}}%
\pgfpathlineto{\pgfqpoint{1.393748in}{1.240466in}}%
\pgfpathlineto{\pgfqpoint{1.390128in}{1.243240in}}%
\pgfpathlineto{\pgfqpoint{1.378092in}{1.253703in}}%
\pgfpathlineto{\pgfqpoint{1.374901in}{1.256851in}}%
\pgfpathlineto{\pgfqpoint{1.364112in}{1.270462in}}%
\pgfpathlineto{\pgfqpoint{1.362435in}{1.273671in}}%
\pgfpathlineto{\pgfqpoint{1.356970in}{1.284073in}}%
\pgfpathlineto{\pgfqpoint{1.353773in}{1.297684in}}%
\pgfpathlineto{\pgfqpoint{1.354572in}{1.311295in}}%
\pgfpathlineto{\pgfqpoint{1.359370in}{1.324907in}}%
\pgfpathlineto{\pgfqpoint{1.362435in}{1.329762in}}%
\pgfpathlineto{\pgfqpoint{1.367971in}{1.338518in}}%
\pgfpathlineto{\pgfqpoint{1.378092in}{1.349827in}}%
\pgfpathlineto{\pgfqpoint{1.380412in}{1.352129in}}%
\pgfpathlineto{\pgfqpoint{1.393748in}{1.363026in}}%
\pgfpathlineto{\pgfqpoint{1.398010in}{1.365740in}}%
\pgfpathlineto{\pgfqpoint{1.409405in}{1.372494in}}%
\pgfpathlineto{\pgfqpoint{1.425061in}{1.378546in}}%
\pgfpathlineto{\pgfqpoint{1.429712in}{1.379351in}}%
\pgfpathlineto{\pgfqpoint{1.440718in}{1.381385in}}%
\pgfpathlineto{\pgfqpoint{1.456375in}{1.380659in}}%
\pgfpathlineto{\pgfqpoint{1.461141in}{1.379351in}}%
\pgfpathlineto{\pgfqpoint{1.472031in}{1.376530in}}%
\pgfpathlineto{\pgfqpoint{1.487688in}{1.369131in}}%
\pgfpathlineto{\pgfqpoint{1.492826in}{1.365740in}}%
\pgfpathlineto{\pgfqpoint{1.503344in}{1.358237in}}%
\pgfpathlineto{\pgfqpoint{1.510371in}{1.352129in}}%
\pgfpathlineto{\pgfqpoint{1.519001in}{1.342984in}}%
\pgfpathlineto{\pgfqpoint{1.522902in}{1.338518in}}%
\pgfpathlineto{\pgfqpoint{1.531412in}{1.324907in}}%
\pgfpathlineto{\pgfqpoint{1.534657in}{1.315439in}}%
\pgfpathlineto{\pgfqpoint{1.536162in}{1.311295in}}%
\pgfpathlineto{\pgfqpoint{1.536997in}{1.297684in}}%
\pgfpathlineto{\pgfqpoint{1.534657in}{1.288116in}}%
\pgfpathlineto{\pgfqpoint{1.533731in}{1.284073in}}%
\pgfpathlineto{\pgfqpoint{1.526770in}{1.270462in}}%
\pgfpathlineto{\pgfqpoint{1.519001in}{1.260556in}}%
\pgfpathlineto{\pgfqpoint{1.515879in}{1.256851in}}%
\pgfpathlineto{\pgfqpoint{1.503344in}{1.245257in}}%
\pgfpathlineto{\pgfqpoint{1.500697in}{1.243240in}}%
\pgfpathlineto{\pgfqpoint{1.487688in}{1.234441in}}%
\pgfpathlineto{\pgfqpoint{1.477616in}{1.229629in}}%
\pgfpathlineto{\pgfqpoint{1.472031in}{1.226964in}}%
\pgfpathlineto{\pgfqpoint{1.456375in}{1.222793in}}%
\pgfpathlineto{\pgfqpoint{1.440718in}{1.222099in}}%
\pgfpathlineto{\pgfqpoint{1.425061in}{1.224877in}}%
\pgfpathlineto{\pgfqpoint{1.413096in}{1.229629in}}%
\pgfpathclose%
\pgfpathmoveto{\pgfqpoint{1.738193in}{1.196177in}}%
\pgfpathlineto{\pgfqpoint{1.753849in}{1.193326in}}%
\pgfpathlineto{\pgfqpoint{1.769506in}{1.195226in}}%
\pgfpathlineto{\pgfqpoint{1.785162in}{1.201888in}}%
\pgfpathlineto{\pgfqpoint{1.785902in}{1.202407in}}%
\pgfpathlineto{\pgfqpoint{1.800819in}{1.210499in}}%
\pgfpathlineto{\pgfqpoint{1.808279in}{1.216018in}}%
\pgfpathlineto{\pgfqpoint{1.816476in}{1.221551in}}%
\pgfpathlineto{\pgfqpoint{1.826458in}{1.229629in}}%
\pgfpathlineto{\pgfqpoint{1.832132in}{1.234359in}}%
\pgfpathlineto{\pgfqpoint{1.842036in}{1.243240in}}%
\pgfpathlineto{\pgfqpoint{1.847789in}{1.249158in}}%
\pgfpathlineto{\pgfqpoint{1.855572in}{1.256851in}}%
\pgfpathlineto{\pgfqpoint{1.863445in}{1.266881in}}%
\pgfpathlineto{\pgfqpoint{1.866818in}{1.270462in}}%
\pgfpathlineto{\pgfqpoint{1.875366in}{1.284073in}}%
\pgfpathlineto{\pgfqpoint{1.879102in}{1.297481in}}%
\pgfpathlineto{\pgfqpoint{1.879189in}{1.297684in}}%
\pgfpathlineto{\pgfqpoint{1.879102in}{1.298491in}}%
\pgfpathlineto{\pgfqpoint{1.878211in}{1.311295in}}%
\pgfpathlineto{\pgfqpoint{1.872519in}{1.324907in}}%
\pgfpathlineto{\pgfqpoint{1.863445in}{1.336783in}}%
\pgfpathlineto{\pgfqpoint{1.862351in}{1.338518in}}%
\pgfpathlineto{\pgfqpoint{1.850325in}{1.352129in}}%
\pgfpathlineto{\pgfqpoint{1.847789in}{1.354485in}}%
\pgfpathlineto{\pgfqpoint{1.836102in}{1.365740in}}%
\pgfpathlineto{\pgfqpoint{1.832132in}{1.369213in}}%
\pgfpathlineto{\pgfqpoint{1.819647in}{1.379351in}}%
\pgfpathlineto{\pgfqpoint{1.816476in}{1.381958in}}%
\pgfpathlineto{\pgfqpoint{1.800819in}{1.392720in}}%
\pgfpathlineto{\pgfqpoint{1.800343in}{1.392962in}}%
\pgfpathlineto{\pgfqpoint{1.785162in}{1.402010in}}%
\pgfpathlineto{\pgfqpoint{1.772097in}{1.406573in}}%
\pgfpathlineto{\pgfqpoint{1.769506in}{1.407872in}}%
\pgfpathlineto{\pgfqpoint{1.753849in}{1.410160in}}%
\pgfpathlineto{\pgfqpoint{1.738193in}{1.406728in}}%
\pgfpathlineto{\pgfqpoint{1.737921in}{1.406573in}}%
\pgfpathlineto{\pgfqpoint{1.722536in}{1.400453in}}%
\pgfpathlineto{\pgfqpoint{1.710812in}{1.392962in}}%
\pgfpathlineto{\pgfqpoint{1.706880in}{1.390812in}}%
\pgfpathlineto{\pgfqpoint{1.691223in}{1.379451in}}%
\pgfpathlineto{\pgfqpoint{1.691104in}{1.379351in}}%
\pgfpathlineto{\pgfqpoint{1.675567in}{1.366442in}}%
\pgfpathlineto{\pgfqpoint{1.674766in}{1.365740in}}%
\pgfpathlineto{\pgfqpoint{1.660625in}{1.352129in}}%
\pgfpathlineto{\pgfqpoint{1.659910in}{1.351287in}}%
\pgfpathlineto{\pgfqpoint{1.648273in}{1.338518in}}%
\pgfpathlineto{\pgfqpoint{1.644253in}{1.332281in}}%
\pgfpathlineto{\pgfqpoint{1.638307in}{1.324907in}}%
\pgfpathlineto{\pgfqpoint{1.632222in}{1.311295in}}%
\pgfpathlineto{\pgfqpoint{1.631210in}{1.297684in}}%
\pgfpathlineto{\pgfqpoint{1.635263in}{1.284073in}}%
\pgfpathlineto{\pgfqpoint{1.644253in}{1.270677in}}%
\pgfpathlineto{\pgfqpoint{1.644368in}{1.270462in}}%
\pgfpathlineto{\pgfqpoint{1.655289in}{1.256851in}}%
\pgfpathlineto{\pgfqpoint{1.659910in}{1.252374in}}%
\pgfpathlineto{\pgfqpoint{1.668763in}{1.243240in}}%
\pgfpathlineto{\pgfqpoint{1.675567in}{1.237122in}}%
\pgfpathlineto{\pgfqpoint{1.684352in}{1.229629in}}%
\pgfpathlineto{\pgfqpoint{1.691223in}{1.223948in}}%
\pgfpathlineto{\pgfqpoint{1.702467in}{1.216018in}}%
\pgfpathlineto{\pgfqpoint{1.706880in}{1.212615in}}%
\pgfpathlineto{\pgfqpoint{1.722536in}{1.203435in}}%
\pgfpathlineto{\pgfqpoint{1.725320in}{1.202407in}}%
\pgfpathlineto{\pgfqpoint{1.738193in}{1.196177in}}%
\pgfpathclose%
\pgfpathmoveto{\pgfqpoint{1.722502in}{1.229629in}}%
\pgfpathlineto{\pgfqpoint{1.706880in}{1.238327in}}%
\pgfpathlineto{\pgfqpoint{1.700224in}{1.243240in}}%
\pgfpathlineto{\pgfqpoint{1.691223in}{1.250763in}}%
\pgfpathlineto{\pgfqpoint{1.684941in}{1.256851in}}%
\pgfpathlineto{\pgfqpoint{1.675567in}{1.268544in}}%
\pgfpathlineto{\pgfqpoint{1.674076in}{1.270462in}}%
\pgfpathlineto{\pgfqpoint{1.666982in}{1.284073in}}%
\pgfpathlineto{\pgfqpoint{1.663836in}{1.297684in}}%
\pgfpathlineto{\pgfqpoint{1.664622in}{1.311295in}}%
\pgfpathlineto{\pgfqpoint{1.669345in}{1.324907in}}%
\pgfpathlineto{\pgfqpoint{1.675567in}{1.334776in}}%
\pgfpathlineto{\pgfqpoint{1.677978in}{1.338518in}}%
\pgfpathlineto{\pgfqpoint{1.690331in}{1.352129in}}%
\pgfpathlineto{\pgfqpoint{1.691223in}{1.352942in}}%
\pgfpathlineto{\pgfqpoint{1.706880in}{1.365222in}}%
\pgfpathlineto{\pgfqpoint{1.707747in}{1.365740in}}%
\pgfpathlineto{\pgfqpoint{1.722536in}{1.373974in}}%
\pgfpathlineto{\pgfqpoint{1.738192in}{1.379351in}}%
\pgfpathlineto{\pgfqpoint{1.738193in}{1.379351in}}%
\pgfpathlineto{\pgfqpoint{1.753849in}{1.381530in}}%
\pgfpathlineto{\pgfqpoint{1.769506in}{1.380078in}}%
\pgfpathlineto{\pgfqpoint{1.771790in}{1.379351in}}%
\pgfpathlineto{\pgfqpoint{1.785162in}{1.375319in}}%
\pgfpathlineto{\pgfqpoint{1.800819in}{1.367249in}}%
\pgfpathlineto{\pgfqpoint{1.803003in}{1.365740in}}%
\pgfpathlineto{\pgfqpoint{1.816476in}{1.355651in}}%
\pgfpathlineto{\pgfqpoint{1.820426in}{1.352129in}}%
\pgfpathlineto{\pgfqpoint{1.832132in}{1.339386in}}%
\pgfpathlineto{\pgfqpoint{1.832886in}{1.338518in}}%
\pgfpathlineto{\pgfqpoint{1.841460in}{1.324907in}}%
\pgfpathlineto{\pgfqpoint{1.846129in}{1.311295in}}%
\pgfpathlineto{\pgfqpoint{1.846906in}{1.297684in}}%
\pgfpathlineto{\pgfqpoint{1.843796in}{1.284073in}}%
\pgfpathlineto{\pgfqpoint{1.836783in}{1.270462in}}%
\pgfpathlineto{\pgfqpoint{1.832132in}{1.264492in}}%
\pgfpathlineto{\pgfqpoint{1.825863in}{1.256851in}}%
\pgfpathlineto{\pgfqpoint{1.816476in}{1.247946in}}%
\pgfpathlineto{\pgfqpoint{1.810593in}{1.243240in}}%
\pgfpathlineto{\pgfqpoint{1.800819in}{1.236318in}}%
\pgfpathlineto{\pgfqpoint{1.787901in}{1.229629in}}%
\pgfpathlineto{\pgfqpoint{1.785162in}{1.228217in}}%
\pgfpathlineto{\pgfqpoint{1.769506in}{1.223348in}}%
\pgfpathlineto{\pgfqpoint{1.753849in}{1.221960in}}%
\pgfpathlineto{\pgfqpoint{1.738193in}{1.224043in}}%
\pgfpathlineto{\pgfqpoint{1.722536in}{1.229610in}}%
\pgfpathlineto{\pgfqpoint{1.722502in}{1.229629in}}%
\pgfpathclose%
\pgfpathmoveto{\pgfqpoint{0.485668in}{1.471049in}}%
\pgfpathlineto{\pgfqpoint{0.501324in}{1.464874in}}%
\pgfpathlineto{\pgfqpoint{0.516981in}{1.463113in}}%
\pgfpathlineto{\pgfqpoint{0.532637in}{1.465755in}}%
\pgfpathlineto{\pgfqpoint{0.548294in}{1.472815in}}%
\pgfpathlineto{\pgfqpoint{0.550864in}{1.474629in}}%
\pgfpathlineto{\pgfqpoint{0.563950in}{1.482057in}}%
\pgfpathlineto{\pgfqpoint{0.572176in}{1.488240in}}%
\pgfpathlineto{\pgfqpoint{0.579607in}{1.493473in}}%
\pgfpathlineto{\pgfqpoint{0.589753in}{1.501851in}}%
\pgfpathlineto{\pgfqpoint{0.595263in}{1.506641in}}%
\pgfpathlineto{\pgfqpoint{0.604900in}{1.515462in}}%
\pgfpathlineto{\pgfqpoint{0.610920in}{1.521922in}}%
\pgfpathlineto{\pgfqpoint{0.618032in}{1.529073in}}%
\pgfpathlineto{\pgfqpoint{0.626577in}{1.540450in}}%
\pgfpathlineto{\pgfqpoint{0.628663in}{1.542684in}}%
\pgfpathlineto{\pgfqpoint{0.636784in}{1.556295in}}%
\pgfpathlineto{\pgfqpoint{0.639823in}{1.569907in}}%
\pgfpathlineto{\pgfqpoint{0.637797in}{1.583518in}}%
\pgfpathlineto{\pgfqpoint{0.630695in}{1.597129in}}%
\pgfpathlineto{\pgfqpoint{0.626577in}{1.601857in}}%
\pgfpathlineto{\pgfqpoint{0.620371in}{1.610740in}}%
\pgfpathlineto{\pgfqpoint{0.610920in}{1.620645in}}%
\pgfpathlineto{\pgfqpoint{0.607615in}{1.624351in}}%
\pgfpathlineto{\pgfqpoint{0.595263in}{1.635902in}}%
\pgfpathlineto{\pgfqpoint{0.592947in}{1.637962in}}%
\pgfpathlineto{\pgfqpoint{0.579607in}{1.649007in}}%
\pgfpathlineto{\pgfqpoint{0.575973in}{1.651573in}}%
\pgfpathlineto{\pgfqpoint{0.563950in}{1.660432in}}%
\pgfpathlineto{\pgfqpoint{0.555398in}{1.665184in}}%
\pgfpathlineto{\pgfqpoint{0.548294in}{1.669934in}}%
\pgfpathlineto{\pgfqpoint{0.532637in}{1.676537in}}%
\pgfpathlineto{\pgfqpoint{0.518346in}{1.678795in}}%
\pgfpathlineto{\pgfqpoint{0.516981in}{1.679127in}}%
\pgfpathlineto{\pgfqpoint{0.514940in}{1.678795in}}%
\pgfpathlineto{\pgfqpoint{0.501324in}{1.677362in}}%
\pgfpathlineto{\pgfqpoint{0.485668in}{1.671586in}}%
\pgfpathlineto{\pgfqpoint{0.475405in}{1.665184in}}%
\pgfpathlineto{\pgfqpoint{0.470011in}{1.662397in}}%
\pgfpathlineto{\pgfqpoint{0.454700in}{1.651573in}}%
\pgfpathlineto{\pgfqpoint{0.454354in}{1.651339in}}%
\pgfpathlineto{\pgfqpoint{0.438698in}{1.638700in}}%
\pgfpathlineto{\pgfqpoint{0.437849in}{1.637962in}}%
\pgfpathlineto{\pgfqpoint{0.423310in}{1.624351in}}%
\pgfpathlineto{\pgfqpoint{0.423041in}{1.624050in}}%
\pgfpathlineto{\pgfqpoint{0.410591in}{1.610740in}}%
\pgfpathlineto{\pgfqpoint{0.407385in}{1.606051in}}%
\pgfpathlineto{\pgfqpoint{0.400021in}{1.597129in}}%
\pgfpathlineto{\pgfqpoint{0.393377in}{1.583518in}}%
\pgfpathlineto{\pgfqpoint{0.391728in}{1.571680in}}%
\pgfpathlineto{\pgfqpoint{0.391346in}{1.569907in}}%
\pgfpathlineto{\pgfqpoint{0.391728in}{1.568720in}}%
\pgfpathlineto{\pgfqpoint{0.394325in}{1.556295in}}%
\pgfpathlineto{\pgfqpoint{0.401921in}{1.542684in}}%
\pgfpathlineto{\pgfqpoint{0.407385in}{1.536509in}}%
\pgfpathlineto{\pgfqpoint{0.412851in}{1.529073in}}%
\pgfpathlineto{\pgfqpoint{0.423041in}{1.518621in}}%
\pgfpathlineto{\pgfqpoint{0.425993in}{1.515462in}}%
\pgfpathlineto{\pgfqpoint{0.438698in}{1.503865in}}%
\pgfpathlineto{\pgfqpoint{0.441068in}{1.501851in}}%
\pgfpathlineto{\pgfqpoint{0.454354in}{1.491113in}}%
\pgfpathlineto{\pgfqpoint{0.458618in}{1.488240in}}%
\pgfpathlineto{\pgfqpoint{0.470011in}{1.480023in}}%
\pgfpathlineto{\pgfqpoint{0.480229in}{1.474629in}}%
\pgfpathlineto{\pgfqpoint{0.485668in}{1.471049in}}%
\pgfpathclose%
\pgfpathmoveto{\pgfqpoint{0.477556in}{1.501851in}}%
\pgfpathlineto{\pgfqpoint{0.470011in}{1.505833in}}%
\pgfpathlineto{\pgfqpoint{0.456655in}{1.515462in}}%
\pgfpathlineto{\pgfqpoint{0.454354in}{1.517376in}}%
\pgfpathlineto{\pgfqpoint{0.442472in}{1.529073in}}%
\pgfpathlineto{\pgfqpoint{0.438698in}{1.533986in}}%
\pgfpathlineto{\pgfqpoint{0.432331in}{1.542684in}}%
\pgfpathlineto{\pgfqpoint{0.426100in}{1.556295in}}%
\pgfpathlineto{\pgfqpoint{0.423768in}{1.569907in}}%
\pgfpathlineto{\pgfqpoint{0.425323in}{1.583518in}}%
\pgfpathlineto{\pgfqpoint{0.430772in}{1.597129in}}%
\pgfpathlineto{\pgfqpoint{0.438698in}{1.608736in}}%
\pgfpathlineto{\pgfqpoint{0.440130in}{1.610740in}}%
\pgfpathlineto{\pgfqpoint{0.453340in}{1.624351in}}%
\pgfpathlineto{\pgfqpoint{0.454354in}{1.625233in}}%
\pgfpathlineto{\pgfqpoint{0.470011in}{1.636717in}}%
\pgfpathlineto{\pgfqpoint{0.472316in}{1.637962in}}%
\pgfpathlineto{\pgfqpoint{0.485668in}{1.644852in}}%
\pgfpathlineto{\pgfqpoint{0.501324in}{1.649590in}}%
\pgfpathlineto{\pgfqpoint{0.516981in}{1.650941in}}%
\pgfpathlineto{\pgfqpoint{0.532637in}{1.648914in}}%
\pgfpathlineto{\pgfqpoint{0.548294in}{1.643497in}}%
\pgfpathlineto{\pgfqpoint{0.558300in}{1.637962in}}%
\pgfpathlineto{\pgfqpoint{0.563950in}{1.634681in}}%
\pgfpathlineto{\pgfqpoint{0.577405in}{1.624351in}}%
\pgfpathlineto{\pgfqpoint{0.579607in}{1.622351in}}%
\pgfpathlineto{\pgfqpoint{0.590683in}{1.610740in}}%
\pgfpathlineto{\pgfqpoint{0.595263in}{1.604181in}}%
\pgfpathlineto{\pgfqpoint{0.600066in}{1.597129in}}%
\pgfpathlineto{\pgfqpoint{0.605579in}{1.583518in}}%
\pgfpathlineto{\pgfqpoint{0.607151in}{1.569907in}}%
\pgfpathlineto{\pgfqpoint{0.604792in}{1.556295in}}%
\pgfpathlineto{\pgfqpoint{0.598489in}{1.542684in}}%
\pgfpathlineto{\pgfqpoint{0.595263in}{1.538266in}}%
\pgfpathlineto{\pgfqpoint{0.588361in}{1.529073in}}%
\pgfpathlineto{\pgfqpoint{0.579607in}{1.520268in}}%
\pgfpathlineto{\pgfqpoint{0.574079in}{1.515462in}}%
\pgfpathlineto{\pgfqpoint{0.563950in}{1.507851in}}%
\pgfpathlineto{\pgfqpoint{0.553377in}{1.501851in}}%
\pgfpathlineto{\pgfqpoint{0.548294in}{1.499047in}}%
\pgfpathlineto{\pgfqpoint{0.532637in}{1.493567in}}%
\pgfpathlineto{\pgfqpoint{0.516981in}{1.491516in}}%
\pgfpathlineto{\pgfqpoint{0.501324in}{1.492883in}}%
\pgfpathlineto{\pgfqpoint{0.485668in}{1.497676in}}%
\pgfpathlineto{\pgfqpoint{0.477556in}{1.501851in}}%
\pgfpathclose%
\pgfpathmoveto{\pgfqpoint{0.798799in}{1.469460in}}%
\pgfpathlineto{\pgfqpoint{0.814455in}{1.464169in}}%
\pgfpathlineto{\pgfqpoint{0.830112in}{1.463289in}}%
\pgfpathlineto{\pgfqpoint{0.845769in}{1.466813in}}%
\pgfpathlineto{\pgfqpoint{0.861178in}{1.474629in}}%
\pgfpathlineto{\pgfqpoint{0.861425in}{1.474728in}}%
\pgfpathlineto{\pgfqpoint{0.877082in}{1.484223in}}%
\pgfpathlineto{\pgfqpoint{0.882232in}{1.488240in}}%
\pgfpathlineto{\pgfqpoint{0.892738in}{1.495936in}}%
\pgfpathlineto{\pgfqpoint{0.899775in}{1.501851in}}%
\pgfpathlineto{\pgfqpoint{0.908395in}{1.509489in}}%
\pgfpathlineto{\pgfqpoint{0.914930in}{1.515462in}}%
\pgfpathlineto{\pgfqpoint{0.924051in}{1.525237in}}%
\pgfpathlineto{\pgfqpoint{0.927965in}{1.529073in}}%
\pgfpathlineto{\pgfqpoint{0.938525in}{1.542684in}}%
\pgfpathlineto{\pgfqpoint{0.939708in}{1.545105in}}%
\pgfpathlineto{\pgfqpoint{0.946874in}{1.556295in}}%
\pgfpathlineto{\pgfqpoint{0.950153in}{1.569907in}}%
\pgfpathlineto{\pgfqpoint{0.947967in}{1.583518in}}%
\pgfpathlineto{\pgfqpoint{0.940304in}{1.597129in}}%
\pgfpathlineto{\pgfqpoint{0.939708in}{1.597772in}}%
\pgfpathlineto{\pgfqpoint{0.930399in}{1.610740in}}%
\pgfpathlineto{\pgfqpoint{0.924051in}{1.617225in}}%
\pgfpathlineto{\pgfqpoint{0.917687in}{1.624351in}}%
\pgfpathlineto{\pgfqpoint{0.908395in}{1.633029in}}%
\pgfpathlineto{\pgfqpoint{0.902954in}{1.637962in}}%
\pgfpathlineto{\pgfqpoint{0.892738in}{1.646572in}}%
\pgfpathlineto{\pgfqpoint{0.885930in}{1.651573in}}%
\pgfpathlineto{\pgfqpoint{0.877082in}{1.658339in}}%
\pgfpathlineto{\pgfqpoint{0.865545in}{1.665184in}}%
\pgfpathlineto{\pgfqpoint{0.861425in}{1.668117in}}%
\pgfpathlineto{\pgfqpoint{0.845769in}{1.675548in}}%
\pgfpathlineto{\pgfqpoint{0.830346in}{1.678795in}}%
\pgfpathlineto{\pgfqpoint{0.830112in}{1.678871in}}%
\pgfpathlineto{\pgfqpoint{0.829184in}{1.678795in}}%
\pgfpathlineto{\pgfqpoint{0.814455in}{1.678021in}}%
\pgfpathlineto{\pgfqpoint{0.798799in}{1.673072in}}%
\pgfpathlineto{\pgfqpoint{0.785138in}{1.665184in}}%
\pgfpathlineto{\pgfqpoint{0.783142in}{1.664233in}}%
\pgfpathlineto{\pgfqpoint{0.767486in}{1.653778in}}%
\pgfpathlineto{\pgfqpoint{0.764776in}{1.651573in}}%
\pgfpathlineto{\pgfqpoint{0.751829in}{1.641413in}}%
\pgfpathlineto{\pgfqpoint{0.747834in}{1.637962in}}%
\pgfpathlineto{\pgfqpoint{0.736173in}{1.627108in}}%
\pgfpathlineto{\pgfqpoint{0.733174in}{1.624351in}}%
\pgfpathlineto{\pgfqpoint{0.720795in}{1.610740in}}%
\pgfpathlineto{\pgfqpoint{0.720516in}{1.610326in}}%
\pgfpathlineto{\pgfqpoint{0.710108in}{1.597129in}}%
\pgfpathlineto{\pgfqpoint{0.704859in}{1.585770in}}%
\pgfpathlineto{\pgfqpoint{0.703365in}{1.583518in}}%
\pgfpathlineto{\pgfqpoint{0.700734in}{1.569907in}}%
\pgfpathlineto{\pgfqpoint{0.704681in}{1.556295in}}%
\pgfpathlineto{\pgfqpoint{0.704859in}{1.556059in}}%
\pgfpathlineto{\pgfqpoint{0.711900in}{1.542684in}}%
\pgfpathlineto{\pgfqpoint{0.720516in}{1.532491in}}%
\pgfpathlineto{\pgfqpoint{0.722990in}{1.529073in}}%
\pgfpathlineto{\pgfqpoint{0.736057in}{1.515462in}}%
\pgfpathlineto{\pgfqpoint{0.736173in}{1.515358in}}%
\pgfpathlineto{\pgfqpoint{0.751021in}{1.501851in}}%
\pgfpathlineto{\pgfqpoint{0.751829in}{1.501155in}}%
\pgfpathlineto{\pgfqpoint{0.767486in}{1.488861in}}%
\pgfpathlineto{\pgfqpoint{0.768454in}{1.488240in}}%
\pgfpathlineto{\pgfqpoint{0.783142in}{1.478124in}}%
\pgfpathlineto{\pgfqpoint{0.790316in}{1.474629in}}%
\pgfpathlineto{\pgfqpoint{0.798799in}{1.469460in}}%
\pgfpathclose%
\pgfpathmoveto{\pgfqpoint{0.787447in}{1.501851in}}%
\pgfpathlineto{\pgfqpoint{0.783142in}{1.503947in}}%
\pgfpathlineto{\pgfqpoint{0.767486in}{1.514686in}}%
\pgfpathlineto{\pgfqpoint{0.766551in}{1.515462in}}%
\pgfpathlineto{\pgfqpoint{0.752424in}{1.529073in}}%
\pgfpathlineto{\pgfqpoint{0.751829in}{1.529827in}}%
\pgfpathlineto{\pgfqpoint{0.742357in}{1.542684in}}%
\pgfpathlineto{\pgfqpoint{0.736173in}{1.556295in}}%
\pgfpathlineto{\pgfqpoint{0.736172in}{1.556295in}}%
\pgfpathlineto{\pgfqpoint{0.733666in}{1.569907in}}%
\pgfpathlineto{\pgfqpoint{0.735337in}{1.583518in}}%
\pgfpathlineto{\pgfqpoint{0.736173in}{1.585503in}}%
\pgfpathlineto{\pgfqpoint{0.740810in}{1.597129in}}%
\pgfpathlineto{\pgfqpoint{0.750093in}{1.610740in}}%
\pgfpathlineto{\pgfqpoint{0.751829in}{1.612638in}}%
\pgfpathlineto{\pgfqpoint{0.763434in}{1.624351in}}%
\pgfpathlineto{\pgfqpoint{0.767486in}{1.627786in}}%
\pgfpathlineto{\pgfqpoint{0.782143in}{1.637962in}}%
\pgfpathlineto{\pgfqpoint{0.783142in}{1.638617in}}%
\pgfpathlineto{\pgfqpoint{0.798799in}{1.646071in}}%
\pgfpathlineto{\pgfqpoint{0.814455in}{1.650131in}}%
\pgfpathlineto{\pgfqpoint{0.830112in}{1.650806in}}%
\pgfpathlineto{\pgfqpoint{0.845769in}{1.648102in}}%
\pgfpathlineto{\pgfqpoint{0.861425in}{1.642006in}}%
\pgfpathlineto{\pgfqpoint{0.868292in}{1.637962in}}%
\pgfpathlineto{\pgfqpoint{0.877082in}{1.632512in}}%
\pgfpathlineto{\pgfqpoint{0.887325in}{1.624351in}}%
\pgfpathlineto{\pgfqpoint{0.892738in}{1.619237in}}%
\pgfpathlineto{\pgfqpoint{0.900700in}{1.610740in}}%
\pgfpathlineto{\pgfqpoint{0.908395in}{1.599510in}}%
\pgfpathlineto{\pgfqpoint{0.910018in}{1.597129in}}%
\pgfpathlineto{\pgfqpoint{0.915619in}{1.583518in}}%
\pgfpathlineto{\pgfqpoint{0.917216in}{1.569907in}}%
\pgfpathlineto{\pgfqpoint{0.914820in}{1.556295in}}%
\pgfpathlineto{\pgfqpoint{0.908416in}{1.542684in}}%
\pgfpathlineto{\pgfqpoint{0.908395in}{1.542655in}}%
\pgfpathlineto{\pgfqpoint{0.898390in}{1.529073in}}%
\pgfpathlineto{\pgfqpoint{0.892738in}{1.523287in}}%
\pgfpathlineto{\pgfqpoint{0.884085in}{1.515462in}}%
\pgfpathlineto{\pgfqpoint{0.877082in}{1.510001in}}%
\pgfpathlineto{\pgfqpoint{0.863631in}{1.501851in}}%
\pgfpathlineto{\pgfqpoint{0.861425in}{1.500555in}}%
\pgfpathlineto{\pgfqpoint{0.845769in}{1.494388in}}%
\pgfpathlineto{\pgfqpoint{0.830112in}{1.491653in}}%
\pgfpathlineto{\pgfqpoint{0.814455in}{1.492336in}}%
\pgfpathlineto{\pgfqpoint{0.798799in}{1.496443in}}%
\pgfpathlineto{\pgfqpoint{0.787447in}{1.501851in}}%
\pgfpathclose%
\pgfpathmoveto{\pgfqpoint{1.111930in}{1.468048in}}%
\pgfpathlineto{\pgfqpoint{1.127587in}{1.463641in}}%
\pgfpathlineto{\pgfqpoint{1.143243in}{1.463641in}}%
\pgfpathlineto{\pgfqpoint{1.158900in}{1.468048in}}%
\pgfpathlineto{\pgfqpoint{1.170673in}{1.474629in}}%
\pgfpathlineto{\pgfqpoint{1.174556in}{1.476358in}}%
\pgfpathlineto{\pgfqpoint{1.190213in}{1.486519in}}%
\pgfpathlineto{\pgfqpoint{1.192348in}{1.488240in}}%
\pgfpathlineto{\pgfqpoint{1.205870in}{1.498499in}}%
\pgfpathlineto{\pgfqpoint{1.209802in}{1.501851in}}%
\pgfpathlineto{\pgfqpoint{1.221526in}{1.512398in}}%
\pgfpathlineto{\pgfqpoint{1.224897in}{1.515462in}}%
\pgfpathlineto{\pgfqpoint{1.237183in}{1.528548in}}%
\pgfpathlineto{\pgfqpoint{1.237736in}{1.529073in}}%
\pgfpathlineto{\pgfqpoint{1.248787in}{1.542684in}}%
\pgfpathlineto{\pgfqpoint{1.252839in}{1.550701in}}%
\pgfpathlineto{\pgfqpoint{1.256699in}{1.556295in}}%
\pgfpathlineto{\pgfqpoint{1.260274in}{1.569907in}}%
\pgfpathlineto{\pgfqpoint{1.257891in}{1.583518in}}%
\pgfpathlineto{\pgfqpoint{1.252839in}{1.591830in}}%
\pgfpathlineto{\pgfqpoint{1.250488in}{1.597129in}}%
\pgfpathlineto{\pgfqpoint{1.240283in}{1.610740in}}%
\pgfpathlineto{\pgfqpoint{1.237183in}{1.613809in}}%
\pgfpathlineto{\pgfqpoint{1.227710in}{1.624351in}}%
\pgfpathlineto{\pgfqpoint{1.221526in}{1.630094in}}%
\pgfpathlineto{\pgfqpoint{1.212979in}{1.637962in}}%
\pgfpathlineto{\pgfqpoint{1.205870in}{1.644039in}}%
\pgfpathlineto{\pgfqpoint{1.195967in}{1.651573in}}%
\pgfpathlineto{\pgfqpoint{1.190213in}{1.656120in}}%
\pgfpathlineto{\pgfqpoint{1.175817in}{1.665184in}}%
\pgfpathlineto{\pgfqpoint{1.174556in}{1.666134in}}%
\pgfpathlineto{\pgfqpoint{1.158900in}{1.674393in}}%
\pgfpathlineto{\pgfqpoint{1.143243in}{1.678515in}}%
\pgfpathlineto{\pgfqpoint{1.127587in}{1.678515in}}%
\pgfpathlineto{\pgfqpoint{1.111930in}{1.674393in}}%
\pgfpathlineto{\pgfqpoint{1.096274in}{1.666134in}}%
\pgfpathlineto{\pgfqpoint{1.095013in}{1.665184in}}%
\pgfpathlineto{\pgfqpoint{1.080617in}{1.656120in}}%
\pgfpathlineto{\pgfqpoint{1.074863in}{1.651573in}}%
\pgfpathlineto{\pgfqpoint{1.064960in}{1.644039in}}%
\pgfpathlineto{\pgfqpoint{1.057851in}{1.637962in}}%
\pgfpathlineto{\pgfqpoint{1.049304in}{1.630094in}}%
\pgfpathlineto{\pgfqpoint{1.043120in}{1.624351in}}%
\pgfpathlineto{\pgfqpoint{1.033647in}{1.613809in}}%
\pgfpathlineto{\pgfqpoint{1.030547in}{1.610740in}}%
\pgfpathlineto{\pgfqpoint{1.020342in}{1.597129in}}%
\pgfpathlineto{\pgfqpoint{1.017991in}{1.591830in}}%
\pgfpathlineto{\pgfqpoint{1.012939in}{1.583518in}}%
\pgfpathlineto{\pgfqpoint{1.010556in}{1.569907in}}%
\pgfpathlineto{\pgfqpoint{1.014131in}{1.556295in}}%
\pgfpathlineto{\pgfqpoint{1.017991in}{1.550701in}}%
\pgfpathlineto{\pgfqpoint{1.022043in}{1.542684in}}%
\pgfpathlineto{\pgfqpoint{1.033094in}{1.529073in}}%
\pgfpathlineto{\pgfqpoint{1.033647in}{1.528548in}}%
\pgfpathlineto{\pgfqpoint{1.045933in}{1.515462in}}%
\pgfpathlineto{\pgfqpoint{1.049304in}{1.512398in}}%
\pgfpathlineto{\pgfqpoint{1.061028in}{1.501851in}}%
\pgfpathlineto{\pgfqpoint{1.064960in}{1.498499in}}%
\pgfpathlineto{\pgfqpoint{1.078482in}{1.488240in}}%
\pgfpathlineto{\pgfqpoint{1.080617in}{1.486519in}}%
\pgfpathlineto{\pgfqpoint{1.096274in}{1.476358in}}%
\pgfpathlineto{\pgfqpoint{1.100157in}{1.474629in}}%
\pgfpathlineto{\pgfqpoint{1.111930in}{1.468048in}}%
\pgfpathclose%
\pgfpathmoveto{\pgfqpoint{1.097046in}{1.501851in}}%
\pgfpathlineto{\pgfqpoint{1.096274in}{1.502195in}}%
\pgfpathlineto{\pgfqpoint{1.080617in}{1.512280in}}%
\pgfpathlineto{\pgfqpoint{1.076669in}{1.515462in}}%
\pgfpathlineto{\pgfqpoint{1.064960in}{1.526428in}}%
\pgfpathlineto{\pgfqpoint{1.062413in}{1.529073in}}%
\pgfpathlineto{\pgfqpoint{1.052393in}{1.542684in}}%
\pgfpathlineto{\pgfqpoint{1.049304in}{1.549408in}}%
\pgfpathlineto{\pgfqpoint{1.046045in}{1.556295in}}%
\pgfpathlineto{\pgfqpoint{1.043600in}{1.569907in}}%
\pgfpathlineto{\pgfqpoint{1.045230in}{1.583518in}}%
\pgfpathlineto{\pgfqpoint{1.049304in}{1.593293in}}%
\pgfpathlineto{\pgfqpoint{1.050851in}{1.597129in}}%
\pgfpathlineto{\pgfqpoint{1.060103in}{1.610740in}}%
\pgfpathlineto{\pgfqpoint{1.064960in}{1.615997in}}%
\pgfpathlineto{\pgfqpoint{1.073498in}{1.624351in}}%
\pgfpathlineto{\pgfqpoint{1.080617in}{1.630213in}}%
\pgfpathlineto{\pgfqpoint{1.092395in}{1.637962in}}%
\pgfpathlineto{\pgfqpoint{1.096274in}{1.640379in}}%
\pgfpathlineto{\pgfqpoint{1.111930in}{1.647155in}}%
\pgfpathlineto{\pgfqpoint{1.127587in}{1.650536in}}%
\pgfpathlineto{\pgfqpoint{1.143243in}{1.650536in}}%
\pgfpathlineto{\pgfqpoint{1.158900in}{1.647155in}}%
\pgfpathlineto{\pgfqpoint{1.174556in}{1.640379in}}%
\pgfpathlineto{\pgfqpoint{1.178435in}{1.637962in}}%
\pgfpathlineto{\pgfqpoint{1.190213in}{1.630213in}}%
\pgfpathlineto{\pgfqpoint{1.197332in}{1.624351in}}%
\pgfpathlineto{\pgfqpoint{1.205870in}{1.615997in}}%
\pgfpathlineto{\pgfqpoint{1.210727in}{1.610740in}}%
\pgfpathlineto{\pgfqpoint{1.219979in}{1.597129in}}%
\pgfpathlineto{\pgfqpoint{1.221526in}{1.593293in}}%
\pgfpathlineto{\pgfqpoint{1.225600in}{1.583518in}}%
\pgfpathlineto{\pgfqpoint{1.227230in}{1.569907in}}%
\pgfpathlineto{\pgfqpoint{1.224785in}{1.556295in}}%
\pgfpathlineto{\pgfqpoint{1.221526in}{1.549408in}}%
\pgfpathlineto{\pgfqpoint{1.218437in}{1.542684in}}%
\pgfpathlineto{\pgfqpoint{1.208417in}{1.529073in}}%
\pgfpathlineto{\pgfqpoint{1.205870in}{1.526428in}}%
\pgfpathlineto{\pgfqpoint{1.194161in}{1.515462in}}%
\pgfpathlineto{\pgfqpoint{1.190213in}{1.512280in}}%
\pgfpathlineto{\pgfqpoint{1.174556in}{1.502195in}}%
\pgfpathlineto{\pgfqpoint{1.173784in}{1.501851in}}%
\pgfpathlineto{\pgfqpoint{1.158900in}{1.495347in}}%
\pgfpathlineto{\pgfqpoint{1.143243in}{1.491926in}}%
\pgfpathlineto{\pgfqpoint{1.127587in}{1.491926in}}%
\pgfpathlineto{\pgfqpoint{1.111930in}{1.495347in}}%
\pgfpathlineto{\pgfqpoint{1.097046in}{1.501851in}}%
\pgfpathclose%
\pgfpathmoveto{\pgfqpoint{1.425061in}{1.466813in}}%
\pgfpathlineto{\pgfqpoint{1.440718in}{1.463289in}}%
\pgfpathlineto{\pgfqpoint{1.456375in}{1.464169in}}%
\pgfpathlineto{\pgfqpoint{1.472031in}{1.469460in}}%
\pgfpathlineto{\pgfqpoint{1.480514in}{1.474629in}}%
\pgfpathlineto{\pgfqpoint{1.487688in}{1.478124in}}%
\pgfpathlineto{\pgfqpoint{1.502376in}{1.488240in}}%
\pgfpathlineto{\pgfqpoint{1.503344in}{1.488861in}}%
\pgfpathlineto{\pgfqpoint{1.519001in}{1.501155in}}%
\pgfpathlineto{\pgfqpoint{1.519809in}{1.501851in}}%
\pgfpathlineto{\pgfqpoint{1.534657in}{1.515358in}}%
\pgfpathlineto{\pgfqpoint{1.534773in}{1.515462in}}%
\pgfpathlineto{\pgfqpoint{1.547840in}{1.529073in}}%
\pgfpathlineto{\pgfqpoint{1.550314in}{1.532491in}}%
\pgfpathlineto{\pgfqpoint{1.558930in}{1.542684in}}%
\pgfpathlineto{\pgfqpoint{1.565971in}{1.556059in}}%
\pgfpathlineto{\pgfqpoint{1.566149in}{1.556295in}}%
\pgfpathlineto{\pgfqpoint{1.570096in}{1.569907in}}%
\pgfpathlineto{\pgfqpoint{1.567465in}{1.583518in}}%
\pgfpathlineto{\pgfqpoint{1.565971in}{1.585770in}}%
\pgfpathlineto{\pgfqpoint{1.560722in}{1.597129in}}%
\pgfpathlineto{\pgfqpoint{1.550314in}{1.610326in}}%
\pgfpathlineto{\pgfqpoint{1.550035in}{1.610740in}}%
\pgfpathlineto{\pgfqpoint{1.537656in}{1.624351in}}%
\pgfpathlineto{\pgfqpoint{1.534657in}{1.627108in}}%
\pgfpathlineto{\pgfqpoint{1.522996in}{1.637962in}}%
\pgfpathlineto{\pgfqpoint{1.519001in}{1.641413in}}%
\pgfpathlineto{\pgfqpoint{1.506054in}{1.651573in}}%
\pgfpathlineto{\pgfqpoint{1.503344in}{1.653778in}}%
\pgfpathlineto{\pgfqpoint{1.487688in}{1.664233in}}%
\pgfpathlineto{\pgfqpoint{1.485692in}{1.665184in}}%
\pgfpathlineto{\pgfqpoint{1.472031in}{1.673072in}}%
\pgfpathlineto{\pgfqpoint{1.456375in}{1.678021in}}%
\pgfpathlineto{\pgfqpoint{1.441646in}{1.678795in}}%
\pgfpathlineto{\pgfqpoint{1.440718in}{1.678871in}}%
\pgfpathlineto{\pgfqpoint{1.440484in}{1.678795in}}%
\pgfpathlineto{\pgfqpoint{1.425061in}{1.675548in}}%
\pgfpathlineto{\pgfqpoint{1.409405in}{1.668117in}}%
\pgfpathlineto{\pgfqpoint{1.405285in}{1.665184in}}%
\pgfpathlineto{\pgfqpoint{1.393748in}{1.658339in}}%
\pgfpathlineto{\pgfqpoint{1.384900in}{1.651573in}}%
\pgfpathlineto{\pgfqpoint{1.378092in}{1.646572in}}%
\pgfpathlineto{\pgfqpoint{1.367876in}{1.637962in}}%
\pgfpathlineto{\pgfqpoint{1.362435in}{1.633029in}}%
\pgfpathlineto{\pgfqpoint{1.353143in}{1.624351in}}%
\pgfpathlineto{\pgfqpoint{1.346779in}{1.617225in}}%
\pgfpathlineto{\pgfqpoint{1.340431in}{1.610740in}}%
\pgfpathlineto{\pgfqpoint{1.331122in}{1.597772in}}%
\pgfpathlineto{\pgfqpoint{1.330526in}{1.597129in}}%
\pgfpathlineto{\pgfqpoint{1.322863in}{1.583518in}}%
\pgfpathlineto{\pgfqpoint{1.320677in}{1.569907in}}%
\pgfpathlineto{\pgfqpoint{1.323956in}{1.556295in}}%
\pgfpathlineto{\pgfqpoint{1.331122in}{1.545105in}}%
\pgfpathlineto{\pgfqpoint{1.332305in}{1.542684in}}%
\pgfpathlineto{\pgfqpoint{1.342865in}{1.529073in}}%
\pgfpathlineto{\pgfqpoint{1.346779in}{1.525237in}}%
\pgfpathlineto{\pgfqpoint{1.355900in}{1.515462in}}%
\pgfpathlineto{\pgfqpoint{1.362435in}{1.509489in}}%
\pgfpathlineto{\pgfqpoint{1.371055in}{1.501851in}}%
\pgfpathlineto{\pgfqpoint{1.378092in}{1.495936in}}%
\pgfpathlineto{\pgfqpoint{1.388598in}{1.488240in}}%
\pgfpathlineto{\pgfqpoint{1.393748in}{1.484223in}}%
\pgfpathlineto{\pgfqpoint{1.409405in}{1.474728in}}%
\pgfpathlineto{\pgfqpoint{1.409652in}{1.474629in}}%
\pgfpathlineto{\pgfqpoint{1.425061in}{1.466813in}}%
\pgfpathclose%
\pgfpathmoveto{\pgfqpoint{1.407199in}{1.501851in}}%
\pgfpathlineto{\pgfqpoint{1.393748in}{1.510001in}}%
\pgfpathlineto{\pgfqpoint{1.386745in}{1.515462in}}%
\pgfpathlineto{\pgfqpoint{1.378092in}{1.523287in}}%
\pgfpathlineto{\pgfqpoint{1.372440in}{1.529073in}}%
\pgfpathlineto{\pgfqpoint{1.362435in}{1.542655in}}%
\pgfpathlineto{\pgfqpoint{1.362414in}{1.542684in}}%
\pgfpathlineto{\pgfqpoint{1.356010in}{1.556295in}}%
\pgfpathlineto{\pgfqpoint{1.353614in}{1.569907in}}%
\pgfpathlineto{\pgfqpoint{1.355211in}{1.583518in}}%
\pgfpathlineto{\pgfqpoint{1.360812in}{1.597129in}}%
\pgfpathlineto{\pgfqpoint{1.362435in}{1.599510in}}%
\pgfpathlineto{\pgfqpoint{1.370130in}{1.610740in}}%
\pgfpathlineto{\pgfqpoint{1.378092in}{1.619237in}}%
\pgfpathlineto{\pgfqpoint{1.383505in}{1.624351in}}%
\pgfpathlineto{\pgfqpoint{1.393748in}{1.632512in}}%
\pgfpathlineto{\pgfqpoint{1.402538in}{1.637962in}}%
\pgfpathlineto{\pgfqpoint{1.409405in}{1.642006in}}%
\pgfpathlineto{\pgfqpoint{1.425061in}{1.648102in}}%
\pgfpathlineto{\pgfqpoint{1.440718in}{1.650806in}}%
\pgfpathlineto{\pgfqpoint{1.456375in}{1.650131in}}%
\pgfpathlineto{\pgfqpoint{1.472031in}{1.646071in}}%
\pgfpathlineto{\pgfqpoint{1.487688in}{1.638617in}}%
\pgfpathlineto{\pgfqpoint{1.488687in}{1.637962in}}%
\pgfpathlineto{\pgfqpoint{1.503344in}{1.627786in}}%
\pgfpathlineto{\pgfqpoint{1.507396in}{1.624351in}}%
\pgfpathlineto{\pgfqpoint{1.519001in}{1.612638in}}%
\pgfpathlineto{\pgfqpoint{1.520737in}{1.610740in}}%
\pgfpathlineto{\pgfqpoint{1.530020in}{1.597129in}}%
\pgfpathlineto{\pgfqpoint{1.534657in}{1.585503in}}%
\pgfpathlineto{\pgfqpoint{1.535493in}{1.583518in}}%
\pgfpathlineto{\pgfqpoint{1.537164in}{1.569907in}}%
\pgfpathlineto{\pgfqpoint{1.534658in}{1.556295in}}%
\pgfpathlineto{\pgfqpoint{1.534657in}{1.556295in}}%
\pgfpathlineto{\pgfqpoint{1.528473in}{1.542684in}}%
\pgfpathlineto{\pgfqpoint{1.519001in}{1.529827in}}%
\pgfpathlineto{\pgfqpoint{1.518406in}{1.529073in}}%
\pgfpathlineto{\pgfqpoint{1.504279in}{1.515462in}}%
\pgfpathlineto{\pgfqpoint{1.503344in}{1.514686in}}%
\pgfpathlineto{\pgfqpoint{1.487688in}{1.503947in}}%
\pgfpathlineto{\pgfqpoint{1.483383in}{1.501851in}}%
\pgfpathlineto{\pgfqpoint{1.472031in}{1.496443in}}%
\pgfpathlineto{\pgfqpoint{1.456375in}{1.492336in}}%
\pgfpathlineto{\pgfqpoint{1.440718in}{1.491653in}}%
\pgfpathlineto{\pgfqpoint{1.425061in}{1.494388in}}%
\pgfpathlineto{\pgfqpoint{1.409405in}{1.500555in}}%
\pgfpathlineto{\pgfqpoint{1.407199in}{1.501851in}}%
\pgfpathclose%
\pgfpathmoveto{\pgfqpoint{1.722536in}{1.472815in}}%
\pgfpathlineto{\pgfqpoint{1.738193in}{1.465755in}}%
\pgfpathlineto{\pgfqpoint{1.753849in}{1.463113in}}%
\pgfpathlineto{\pgfqpoint{1.769506in}{1.464874in}}%
\pgfpathlineto{\pgfqpoint{1.785162in}{1.471049in}}%
\pgfpathlineto{\pgfqpoint{1.790601in}{1.474629in}}%
\pgfpathlineto{\pgfqpoint{1.800819in}{1.480023in}}%
\pgfpathlineto{\pgfqpoint{1.812212in}{1.488240in}}%
\pgfpathlineto{\pgfqpoint{1.816476in}{1.491113in}}%
\pgfpathlineto{\pgfqpoint{1.829762in}{1.501851in}}%
\pgfpathlineto{\pgfqpoint{1.832132in}{1.503865in}}%
\pgfpathlineto{\pgfqpoint{1.844837in}{1.515462in}}%
\pgfpathlineto{\pgfqpoint{1.847789in}{1.518621in}}%
\pgfpathlineto{\pgfqpoint{1.857979in}{1.529073in}}%
\pgfpathlineto{\pgfqpoint{1.863445in}{1.536509in}}%
\pgfpathlineto{\pgfqpoint{1.868909in}{1.542684in}}%
\pgfpathlineto{\pgfqpoint{1.876505in}{1.556295in}}%
\pgfpathlineto{\pgfqpoint{1.879102in}{1.568720in}}%
\pgfpathlineto{\pgfqpoint{1.879484in}{1.569907in}}%
\pgfpathlineto{\pgfqpoint{1.879102in}{1.571680in}}%
\pgfpathlineto{\pgfqpoint{1.877453in}{1.583518in}}%
\pgfpathlineto{\pgfqpoint{1.870809in}{1.597129in}}%
\pgfpathlineto{\pgfqpoint{1.863445in}{1.606051in}}%
\pgfpathlineto{\pgfqpoint{1.860239in}{1.610740in}}%
\pgfpathlineto{\pgfqpoint{1.847789in}{1.624050in}}%
\pgfpathlineto{\pgfqpoint{1.847520in}{1.624351in}}%
\pgfpathlineto{\pgfqpoint{1.832981in}{1.637962in}}%
\pgfpathlineto{\pgfqpoint{1.832132in}{1.638700in}}%
\pgfpathlineto{\pgfqpoint{1.816476in}{1.651339in}}%
\pgfpathlineto{\pgfqpoint{1.816130in}{1.651573in}}%
\pgfpathlineto{\pgfqpoint{1.800819in}{1.662397in}}%
\pgfpathlineto{\pgfqpoint{1.795425in}{1.665184in}}%
\pgfpathlineto{\pgfqpoint{1.785162in}{1.671586in}}%
\pgfpathlineto{\pgfqpoint{1.769506in}{1.677362in}}%
\pgfpathlineto{\pgfqpoint{1.755890in}{1.678795in}}%
\pgfpathlineto{\pgfqpoint{1.753849in}{1.679127in}}%
\pgfpathlineto{\pgfqpoint{1.752484in}{1.678795in}}%
\pgfpathlineto{\pgfqpoint{1.738193in}{1.676537in}}%
\pgfpathlineto{\pgfqpoint{1.722536in}{1.669934in}}%
\pgfpathlineto{\pgfqpoint{1.715432in}{1.665184in}}%
\pgfpathlineto{\pgfqpoint{1.706880in}{1.660432in}}%
\pgfpathlineto{\pgfqpoint{1.694857in}{1.651573in}}%
\pgfpathlineto{\pgfqpoint{1.691223in}{1.649007in}}%
\pgfpathlineto{\pgfqpoint{1.677883in}{1.637962in}}%
\pgfpathlineto{\pgfqpoint{1.675567in}{1.635902in}}%
\pgfpathlineto{\pgfqpoint{1.663215in}{1.624351in}}%
\pgfpathlineto{\pgfqpoint{1.659910in}{1.620645in}}%
\pgfpathlineto{\pgfqpoint{1.650459in}{1.610740in}}%
\pgfpathlineto{\pgfqpoint{1.644253in}{1.601857in}}%
\pgfpathlineto{\pgfqpoint{1.640135in}{1.597129in}}%
\pgfpathlineto{\pgfqpoint{1.633033in}{1.583518in}}%
\pgfpathlineto{\pgfqpoint{1.631007in}{1.569907in}}%
\pgfpathlineto{\pgfqpoint{1.634046in}{1.556295in}}%
\pgfpathlineto{\pgfqpoint{1.642167in}{1.542684in}}%
\pgfpathlineto{\pgfqpoint{1.644253in}{1.540450in}}%
\pgfpathlineto{\pgfqpoint{1.652798in}{1.529073in}}%
\pgfpathlineto{\pgfqpoint{1.659910in}{1.521922in}}%
\pgfpathlineto{\pgfqpoint{1.665930in}{1.515462in}}%
\pgfpathlineto{\pgfqpoint{1.675567in}{1.506641in}}%
\pgfpathlineto{\pgfqpoint{1.681077in}{1.501851in}}%
\pgfpathlineto{\pgfqpoint{1.691223in}{1.493473in}}%
\pgfpathlineto{\pgfqpoint{1.698654in}{1.488240in}}%
\pgfpathlineto{\pgfqpoint{1.706880in}{1.482057in}}%
\pgfpathlineto{\pgfqpoint{1.719966in}{1.474629in}}%
\pgfpathlineto{\pgfqpoint{1.722536in}{1.472815in}}%
\pgfpathclose%
\pgfpathmoveto{\pgfqpoint{1.717453in}{1.501851in}}%
\pgfpathlineto{\pgfqpoint{1.706880in}{1.507851in}}%
\pgfpathlineto{\pgfqpoint{1.696751in}{1.515462in}}%
\pgfpathlineto{\pgfqpoint{1.691223in}{1.520268in}}%
\pgfpathlineto{\pgfqpoint{1.682469in}{1.529073in}}%
\pgfpathlineto{\pgfqpoint{1.675567in}{1.538266in}}%
\pgfpathlineto{\pgfqpoint{1.672341in}{1.542684in}}%
\pgfpathlineto{\pgfqpoint{1.666038in}{1.556295in}}%
\pgfpathlineto{\pgfqpoint{1.663679in}{1.569907in}}%
\pgfpathlineto{\pgfqpoint{1.665251in}{1.583518in}}%
\pgfpathlineto{\pgfqpoint{1.670764in}{1.597129in}}%
\pgfpathlineto{\pgfqpoint{1.675567in}{1.604181in}}%
\pgfpathlineto{\pgfqpoint{1.680147in}{1.610740in}}%
\pgfpathlineto{\pgfqpoint{1.691223in}{1.622351in}}%
\pgfpathlineto{\pgfqpoint{1.693425in}{1.624351in}}%
\pgfpathlineto{\pgfqpoint{1.706880in}{1.634681in}}%
\pgfpathlineto{\pgfqpoint{1.712530in}{1.637962in}}%
\pgfpathlineto{\pgfqpoint{1.722536in}{1.643497in}}%
\pgfpathlineto{\pgfqpoint{1.738193in}{1.648914in}}%
\pgfpathlineto{\pgfqpoint{1.753849in}{1.650941in}}%
\pgfpathlineto{\pgfqpoint{1.769506in}{1.649590in}}%
\pgfpathlineto{\pgfqpoint{1.785162in}{1.644852in}}%
\pgfpathlineto{\pgfqpoint{1.798514in}{1.637962in}}%
\pgfpathlineto{\pgfqpoint{1.800819in}{1.636717in}}%
\pgfpathlineto{\pgfqpoint{1.816476in}{1.625233in}}%
\pgfpathlineto{\pgfqpoint{1.817490in}{1.624351in}}%
\pgfpathlineto{\pgfqpoint{1.830700in}{1.610740in}}%
\pgfpathlineto{\pgfqpoint{1.832132in}{1.608736in}}%
\pgfpathlineto{\pgfqpoint{1.840058in}{1.597129in}}%
\pgfpathlineto{\pgfqpoint{1.845507in}{1.583518in}}%
\pgfpathlineto{\pgfqpoint{1.847062in}{1.569907in}}%
\pgfpathlineto{\pgfqpoint{1.844730in}{1.556295in}}%
\pgfpathlineto{\pgfqpoint{1.838499in}{1.542684in}}%
\pgfpathlineto{\pgfqpoint{1.832132in}{1.533986in}}%
\pgfpathlineto{\pgfqpoint{1.828358in}{1.529073in}}%
\pgfpathlineto{\pgfqpoint{1.816476in}{1.517376in}}%
\pgfpathlineto{\pgfqpoint{1.814175in}{1.515462in}}%
\pgfpathlineto{\pgfqpoint{1.800819in}{1.505833in}}%
\pgfpathlineto{\pgfqpoint{1.793274in}{1.501851in}}%
\pgfpathlineto{\pgfqpoint{1.785162in}{1.497676in}}%
\pgfpathlineto{\pgfqpoint{1.769506in}{1.492883in}}%
\pgfpathlineto{\pgfqpoint{1.753849in}{1.491516in}}%
\pgfpathlineto{\pgfqpoint{1.738193in}{1.493567in}}%
\pgfpathlineto{\pgfqpoint{1.722536in}{1.499047in}}%
\pgfpathlineto{\pgfqpoint{1.717453in}{1.501851in}}%
\pgfpathclose%
\pgfusepath{fill}%
\end{pgfscope}%
\begin{pgfscope}%
\pgfpathrectangle{\pgfqpoint{0.360415in}{0.358518in}}{\pgfqpoint{1.550000in}{1.347500in}}%
\pgfusepath{clip}%
\pgfsetbuttcap%
\pgfsetroundjoin%
\definecolor{currentfill}{rgb}{0.481929,0.136891,0.507989}%
\pgfsetfillcolor{currentfill}%
\pgfsetlinewidth{0.000000pt}%
\definecolor{currentstroke}{rgb}{0.000000,0.000000,0.000000}%
\pgfsetstrokecolor{currentstroke}%
\pgfsetdash{}{0pt}%
\pgfpathmoveto{\pgfqpoint{0.470011in}{0.358518in}}%
\pgfpathlineto{\pgfqpoint{0.485668in}{0.358518in}}%
\pgfpathlineto{\pgfqpoint{0.501324in}{0.358518in}}%
\pgfpathlineto{\pgfqpoint{0.516981in}{0.358518in}}%
\pgfpathlineto{\pgfqpoint{0.532637in}{0.358518in}}%
\pgfpathlineto{\pgfqpoint{0.548294in}{0.358518in}}%
\pgfpathlineto{\pgfqpoint{0.563950in}{0.358518in}}%
\pgfpathlineto{\pgfqpoint{0.568251in}{0.358518in}}%
\pgfpathlineto{\pgfqpoint{0.572036in}{0.372129in}}%
\pgfpathlineto{\pgfqpoint{0.579607in}{0.381922in}}%
\pgfpathlineto{\pgfqpoint{0.582207in}{0.385740in}}%
\pgfpathlineto{\pgfqpoint{0.595263in}{0.398928in}}%
\pgfpathlineto{\pgfqpoint{0.595665in}{0.399351in}}%
\pgfpathlineto{\pgfqpoint{0.610803in}{0.412962in}}%
\pgfpathlineto{\pgfqpoint{0.610920in}{0.413064in}}%
\pgfpathlineto{\pgfqpoint{0.626577in}{0.425984in}}%
\pgfpathlineto{\pgfqpoint{0.627441in}{0.426573in}}%
\pgfpathlineto{\pgfqpoint{0.642233in}{0.437270in}}%
\pgfpathlineto{\pgfqpoint{0.648506in}{0.440184in}}%
\pgfpathlineto{\pgfqpoint{0.657890in}{0.445221in}}%
\pgfpathlineto{\pgfqpoint{0.673546in}{0.447272in}}%
\pgfpathlineto{\pgfqpoint{0.689203in}{0.442569in}}%
\pgfpathlineto{\pgfqpoint{0.692709in}{0.440184in}}%
\pgfpathlineto{\pgfqpoint{0.704859in}{0.433096in}}%
\pgfpathlineto{\pgfqpoint{0.712977in}{0.426573in}}%
\pgfpathlineto{\pgfqpoint{0.720516in}{0.420899in}}%
\pgfpathlineto{\pgfqpoint{0.729787in}{0.412962in}}%
\pgfpathlineto{\pgfqpoint{0.736173in}{0.407328in}}%
\pgfpathlineto{\pgfqpoint{0.745128in}{0.399351in}}%
\pgfpathlineto{\pgfqpoint{0.751829in}{0.392356in}}%
\pgfpathlineto{\pgfqpoint{0.758714in}{0.385740in}}%
\pgfpathlineto{\pgfqpoint{0.767486in}{0.373446in}}%
\pgfpathlineto{\pgfqpoint{0.768603in}{0.372129in}}%
\pgfpathlineto{\pgfqpoint{0.772653in}{0.358518in}}%
\pgfpathlineto{\pgfqpoint{0.783142in}{0.358518in}}%
\pgfpathlineto{\pgfqpoint{0.798799in}{0.358518in}}%
\pgfpathlineto{\pgfqpoint{0.814455in}{0.358518in}}%
\pgfpathlineto{\pgfqpoint{0.830112in}{0.358518in}}%
\pgfpathlineto{\pgfqpoint{0.845769in}{0.358518in}}%
\pgfpathlineto{\pgfqpoint{0.861425in}{0.358518in}}%
\pgfpathlineto{\pgfqpoint{0.877082in}{0.358518in}}%
\pgfpathlineto{\pgfqpoint{0.878408in}{0.358518in}}%
\pgfpathlineto{\pgfqpoint{0.882095in}{0.372129in}}%
\pgfpathlineto{\pgfqpoint{0.892232in}{0.385740in}}%
\pgfpathlineto{\pgfqpoint{0.892738in}{0.386195in}}%
\pgfpathlineto{\pgfqpoint{0.905652in}{0.399351in}}%
\pgfpathlineto{\pgfqpoint{0.908395in}{0.401731in}}%
\pgfpathlineto{\pgfqpoint{0.920926in}{0.412962in}}%
\pgfpathlineto{\pgfqpoint{0.924051in}{0.415683in}}%
\pgfpathlineto{\pgfqpoint{0.937548in}{0.426573in}}%
\pgfpathlineto{\pgfqpoint{0.939708in}{0.428443in}}%
\pgfpathlineto{\pgfqpoint{0.955364in}{0.439133in}}%
\pgfpathlineto{\pgfqpoint{0.957995in}{0.440184in}}%
\pgfpathlineto{\pgfqpoint{0.971021in}{0.446171in}}%
\pgfpathlineto{\pgfqpoint{0.986678in}{0.446858in}}%
\pgfpathlineto{\pgfqpoint{1.002334in}{0.440896in}}%
\pgfpathlineto{\pgfqpoint{1.003288in}{0.440184in}}%
\pgfpathlineto{\pgfqpoint{1.017991in}{0.430820in}}%
\pgfpathlineto{\pgfqpoint{1.023066in}{0.426573in}}%
\pgfpathlineto{\pgfqpoint{1.033647in}{0.418299in}}%
\pgfpathlineto{\pgfqpoint{1.039815in}{0.412962in}}%
\pgfpathlineto{\pgfqpoint{1.049304in}{0.404505in}}%
\pgfpathlineto{\pgfqpoint{1.055155in}{0.399351in}}%
\pgfpathlineto{\pgfqpoint{1.064960in}{0.389220in}}%
\pgfpathlineto{\pgfqpoint{1.068696in}{0.385740in}}%
\pgfpathlineto{\pgfqpoint{1.078616in}{0.372129in}}%
\pgfpathlineto{\pgfqpoint{1.080617in}{0.364532in}}%
\pgfpathlineto{\pgfqpoint{1.082498in}{0.358518in}}%
\pgfpathlineto{\pgfqpoint{1.096274in}{0.358518in}}%
\pgfpathlineto{\pgfqpoint{1.111930in}{0.358518in}}%
\pgfpathlineto{\pgfqpoint{1.127587in}{0.358518in}}%
\pgfpathlineto{\pgfqpoint{1.143243in}{0.358518in}}%
\pgfpathlineto{\pgfqpoint{1.158900in}{0.358518in}}%
\pgfpathlineto{\pgfqpoint{1.174556in}{0.358518in}}%
\pgfpathlineto{\pgfqpoint{1.188332in}{0.358518in}}%
\pgfpathlineto{\pgfqpoint{1.190213in}{0.364532in}}%
\pgfpathlineto{\pgfqpoint{1.192214in}{0.372129in}}%
\pgfpathlineto{\pgfqpoint{1.202134in}{0.385740in}}%
\pgfpathlineto{\pgfqpoint{1.205870in}{0.389220in}}%
\pgfpathlineto{\pgfqpoint{1.215675in}{0.399351in}}%
\pgfpathlineto{\pgfqpoint{1.221526in}{0.404505in}}%
\pgfpathlineto{\pgfqpoint{1.231015in}{0.412962in}}%
\pgfpathlineto{\pgfqpoint{1.237183in}{0.418299in}}%
\pgfpathlineto{\pgfqpoint{1.247764in}{0.426573in}}%
\pgfpathlineto{\pgfqpoint{1.252839in}{0.430820in}}%
\pgfpathlineto{\pgfqpoint{1.267542in}{0.440184in}}%
\pgfpathlineto{\pgfqpoint{1.268496in}{0.440896in}}%
\pgfpathlineto{\pgfqpoint{1.284152in}{0.446858in}}%
\pgfpathlineto{\pgfqpoint{1.299809in}{0.446171in}}%
\pgfpathlineto{\pgfqpoint{1.312835in}{0.440184in}}%
\pgfpathlineto{\pgfqpoint{1.315466in}{0.439133in}}%
\pgfpathlineto{\pgfqpoint{1.331122in}{0.428443in}}%
\pgfpathlineto{\pgfqpoint{1.333282in}{0.426573in}}%
\pgfpathlineto{\pgfqpoint{1.346779in}{0.415683in}}%
\pgfpathlineto{\pgfqpoint{1.349904in}{0.412962in}}%
\pgfpathlineto{\pgfqpoint{1.362435in}{0.401731in}}%
\pgfpathlineto{\pgfqpoint{1.365178in}{0.399351in}}%
\pgfpathlineto{\pgfqpoint{1.378092in}{0.386195in}}%
\pgfpathlineto{\pgfqpoint{1.378598in}{0.385740in}}%
\pgfpathlineto{\pgfqpoint{1.388735in}{0.372129in}}%
\pgfpathlineto{\pgfqpoint{1.392422in}{0.358518in}}%
\pgfpathlineto{\pgfqpoint{1.393748in}{0.358518in}}%
\pgfpathlineto{\pgfqpoint{1.409405in}{0.358518in}}%
\pgfpathlineto{\pgfqpoint{1.425061in}{0.358518in}}%
\pgfpathlineto{\pgfqpoint{1.440718in}{0.358518in}}%
\pgfpathlineto{\pgfqpoint{1.456375in}{0.358518in}}%
\pgfpathlineto{\pgfqpoint{1.472031in}{0.358518in}}%
\pgfpathlineto{\pgfqpoint{1.487688in}{0.358518in}}%
\pgfpathlineto{\pgfqpoint{1.498177in}{0.358518in}}%
\pgfpathlineto{\pgfqpoint{1.502227in}{0.372129in}}%
\pgfpathlineto{\pgfqpoint{1.503344in}{0.373446in}}%
\pgfpathlineto{\pgfqpoint{1.512116in}{0.385740in}}%
\pgfpathlineto{\pgfqpoint{1.519001in}{0.392356in}}%
\pgfpathlineto{\pgfqpoint{1.525702in}{0.399351in}}%
\pgfpathlineto{\pgfqpoint{1.534657in}{0.407328in}}%
\pgfpathlineto{\pgfqpoint{1.541043in}{0.412962in}}%
\pgfpathlineto{\pgfqpoint{1.550314in}{0.420899in}}%
\pgfpathlineto{\pgfqpoint{1.557853in}{0.426573in}}%
\pgfpathlineto{\pgfqpoint{1.565971in}{0.433096in}}%
\pgfpathlineto{\pgfqpoint{1.578121in}{0.440184in}}%
\pgfpathlineto{\pgfqpoint{1.581627in}{0.442569in}}%
\pgfpathlineto{\pgfqpoint{1.597284in}{0.447272in}}%
\pgfpathlineto{\pgfqpoint{1.612940in}{0.445221in}}%
\pgfpathlineto{\pgfqpoint{1.622324in}{0.440184in}}%
\pgfpathlineto{\pgfqpoint{1.628597in}{0.437270in}}%
\pgfpathlineto{\pgfqpoint{1.643389in}{0.426573in}}%
\pgfpathlineto{\pgfqpoint{1.644253in}{0.425984in}}%
\pgfpathlineto{\pgfqpoint{1.659910in}{0.413064in}}%
\pgfpathlineto{\pgfqpoint{1.660027in}{0.412962in}}%
\pgfpathlineto{\pgfqpoint{1.675165in}{0.399351in}}%
\pgfpathlineto{\pgfqpoint{1.675567in}{0.398928in}}%
\pgfpathlineto{\pgfqpoint{1.688623in}{0.385740in}}%
\pgfpathlineto{\pgfqpoint{1.691223in}{0.381922in}}%
\pgfpathlineto{\pgfqpoint{1.698794in}{0.372129in}}%
\pgfpathlineto{\pgfqpoint{1.702579in}{0.358518in}}%
\pgfpathlineto{\pgfqpoint{1.706880in}{0.358518in}}%
\pgfpathlineto{\pgfqpoint{1.722536in}{0.358518in}}%
\pgfpathlineto{\pgfqpoint{1.738193in}{0.358518in}}%
\pgfpathlineto{\pgfqpoint{1.753849in}{0.358518in}}%
\pgfpathlineto{\pgfqpoint{1.769506in}{0.358518in}}%
\pgfpathlineto{\pgfqpoint{1.785162in}{0.358518in}}%
\pgfpathlineto{\pgfqpoint{1.800819in}{0.358518in}}%
\pgfpathlineto{\pgfqpoint{1.808163in}{0.358518in}}%
\pgfpathlineto{\pgfqpoint{1.812068in}{0.372129in}}%
\pgfpathlineto{\pgfqpoint{1.816476in}{0.377585in}}%
\pgfpathlineto{\pgfqpoint{1.822149in}{0.385740in}}%
\pgfpathlineto{\pgfqpoint{1.832132in}{0.395595in}}%
\pgfpathlineto{\pgfqpoint{1.835707in}{0.399351in}}%
\pgfpathlineto{\pgfqpoint{1.847789in}{0.410189in}}%
\pgfpathlineto{\pgfqpoint{1.850978in}{0.412962in}}%
\pgfpathlineto{\pgfqpoint{1.863445in}{0.423466in}}%
\pgfpathlineto{\pgfqpoint{1.867766in}{0.426573in}}%
\pgfpathlineto{\pgfqpoint{1.879102in}{0.435252in}}%
\pgfpathlineto{\pgfqpoint{1.888482in}{0.440184in}}%
\pgfpathlineto{\pgfqpoint{1.894758in}{0.444016in}}%
\pgfpathlineto{\pgfqpoint{1.910415in}{0.447411in}}%
\pgfpathlineto{\pgfqpoint{1.910415in}{0.453795in}}%
\pgfpathlineto{\pgfqpoint{1.910415in}{0.467407in}}%
\pgfpathlineto{\pgfqpoint{1.910415in}{0.481018in}}%
\pgfpathlineto{\pgfqpoint{1.910415in}{0.494629in}}%
\pgfpathlineto{\pgfqpoint{1.910415in}{0.508240in}}%
\pgfpathlineto{\pgfqpoint{1.910415in}{0.521851in}}%
\pgfpathlineto{\pgfqpoint{1.910415in}{0.535462in}}%
\pgfpathlineto{\pgfqpoint{1.910415in}{0.539201in}}%
\pgfpathlineto{\pgfqpoint{1.894758in}{0.542492in}}%
\pgfpathlineto{\pgfqpoint{1.883494in}{0.549073in}}%
\pgfpathlineto{\pgfqpoint{1.879102in}{0.551333in}}%
\pgfpathlineto{\pgfqpoint{1.863931in}{0.562684in}}%
\pgfpathlineto{\pgfqpoint{1.863445in}{0.563033in}}%
\pgfpathlineto{\pgfqpoint{1.847789in}{0.576193in}}%
\pgfpathlineto{\pgfqpoint{1.847671in}{0.576295in}}%
\pgfpathlineto{\pgfqpoint{1.832810in}{0.589907in}}%
\pgfpathlineto{\pgfqpoint{1.832132in}{0.590658in}}%
\pgfpathlineto{\pgfqpoint{1.819828in}{0.603518in}}%
\pgfpathlineto{\pgfqpoint{1.816476in}{0.608971in}}%
\pgfpathlineto{\pgfqpoint{1.810682in}{0.617129in}}%
\pgfpathlineto{\pgfqpoint{1.808322in}{0.630740in}}%
\pgfpathlineto{\pgfqpoint{1.813732in}{0.644351in}}%
\pgfpathlineto{\pgfqpoint{1.816476in}{0.647399in}}%
\pgfpathlineto{\pgfqpoint{1.824629in}{0.657962in}}%
\pgfpathlineto{\pgfqpoint{1.832132in}{0.665019in}}%
\pgfpathlineto{\pgfqpoint{1.838659in}{0.671573in}}%
\pgfpathlineto{\pgfqpoint{1.847789in}{0.679633in}}%
\pgfpathlineto{\pgfqpoint{1.854269in}{0.685184in}}%
\pgfpathlineto{\pgfqpoint{1.863445in}{0.692970in}}%
\pgfpathlineto{\pgfqpoint{1.871492in}{0.698795in}}%
\pgfpathlineto{\pgfqpoint{1.879102in}{0.704781in}}%
\pgfpathlineto{\pgfqpoint{1.893243in}{0.712407in}}%
\pgfpathlineto{\pgfqpoint{1.894758in}{0.713378in}}%
\pgfpathlineto{\pgfqpoint{1.910415in}{0.716899in}}%
\pgfpathlineto{\pgfqpoint{1.910415in}{0.726018in}}%
\pgfpathlineto{\pgfqpoint{1.910415in}{0.739629in}}%
\pgfpathlineto{\pgfqpoint{1.910415in}{0.753240in}}%
\pgfpathlineto{\pgfqpoint{1.910415in}{0.766851in}}%
\pgfpathlineto{\pgfqpoint{1.910415in}{0.780462in}}%
\pgfpathlineto{\pgfqpoint{1.910415in}{0.794073in}}%
\pgfpathlineto{\pgfqpoint{1.910415in}{0.807684in}}%
\pgfpathlineto{\pgfqpoint{1.910415in}{0.808837in}}%
\pgfpathlineto{\pgfqpoint{1.894758in}{0.812043in}}%
\pgfpathlineto{\pgfqpoint{1.879102in}{0.820855in}}%
\pgfpathlineto{\pgfqpoint{1.878578in}{0.821295in}}%
\pgfpathlineto{\pgfqpoint{1.863445in}{0.832522in}}%
\pgfpathlineto{\pgfqpoint{1.860708in}{0.834907in}}%
\pgfpathlineto{\pgfqpoint{1.847789in}{0.845800in}}%
\pgfpathlineto{\pgfqpoint{1.844659in}{0.848518in}}%
\pgfpathlineto{\pgfqpoint{1.832132in}{0.860251in}}%
\pgfpathlineto{\pgfqpoint{1.829982in}{0.862129in}}%
\pgfpathlineto{\pgfqpoint{1.817685in}{0.875740in}}%
\pgfpathlineto{\pgfqpoint{1.816476in}{0.878027in}}%
\pgfpathlineto{\pgfqpoint{1.809589in}{0.889351in}}%
\pgfpathlineto{\pgfqpoint{1.808799in}{0.902962in}}%
\pgfpathlineto{\pgfqpoint{1.815657in}{0.916573in}}%
\pgfpathlineto{\pgfqpoint{1.816476in}{0.917402in}}%
\pgfpathlineto{\pgfqpoint{1.827247in}{0.930184in}}%
\pgfpathlineto{\pgfqpoint{1.832132in}{0.934596in}}%
\pgfpathlineto{\pgfqpoint{1.841649in}{0.943795in}}%
\pgfpathlineto{\pgfqpoint{1.847789in}{0.949158in}}%
\pgfpathlineto{\pgfqpoint{1.857517in}{0.957407in}}%
\pgfpathlineto{\pgfqpoint{1.863445in}{0.962493in}}%
\pgfpathlineto{\pgfqpoint{1.875099in}{0.971018in}}%
\pgfpathlineto{\pgfqpoint{1.879102in}{0.974265in}}%
\pgfpathlineto{\pgfqpoint{1.894758in}{0.982889in}}%
\pgfpathlineto{\pgfqpoint{1.903497in}{0.984629in}}%
\pgfpathlineto{\pgfqpoint{1.910415in}{0.986264in}}%
\pgfpathlineto{\pgfqpoint{1.910415in}{0.998240in}}%
\pgfpathlineto{\pgfqpoint{1.910415in}{1.011851in}}%
\pgfpathlineto{\pgfqpoint{1.910415in}{1.025462in}}%
\pgfpathlineto{\pgfqpoint{1.910415in}{1.039073in}}%
\pgfpathlineto{\pgfqpoint{1.910415in}{1.052684in}}%
\pgfpathlineto{\pgfqpoint{1.910415in}{1.066295in}}%
\pgfpathlineto{\pgfqpoint{1.910415in}{1.078271in}}%
\pgfpathlineto{\pgfqpoint{1.903497in}{1.079907in}}%
\pgfpathlineto{\pgfqpoint{1.894758in}{1.081646in}}%
\pgfpathlineto{\pgfqpoint{1.879102in}{1.090270in}}%
\pgfpathlineto{\pgfqpoint{1.875099in}{1.093518in}}%
\pgfpathlineto{\pgfqpoint{1.863445in}{1.102042in}}%
\pgfpathlineto{\pgfqpoint{1.857517in}{1.107129in}}%
\pgfpathlineto{\pgfqpoint{1.847789in}{1.115378in}}%
\pgfpathlineto{\pgfqpoint{1.841649in}{1.120740in}}%
\pgfpathlineto{\pgfqpoint{1.832132in}{1.129939in}}%
\pgfpathlineto{\pgfqpoint{1.827247in}{1.134351in}}%
\pgfpathlineto{\pgfqpoint{1.816476in}{1.147133in}}%
\pgfpathlineto{\pgfqpoint{1.815657in}{1.147962in}}%
\pgfpathlineto{\pgfqpoint{1.808799in}{1.161573in}}%
\pgfpathlineto{\pgfqpoint{1.809589in}{1.175184in}}%
\pgfpathlineto{\pgfqpoint{1.816476in}{1.186508in}}%
\pgfpathlineto{\pgfqpoint{1.817685in}{1.188795in}}%
\pgfpathlineto{\pgfqpoint{1.829982in}{1.202407in}}%
\pgfpathlineto{\pgfqpoint{1.832132in}{1.204285in}}%
\pgfpathlineto{\pgfqpoint{1.844659in}{1.216018in}}%
\pgfpathlineto{\pgfqpoint{1.847789in}{1.218735in}}%
\pgfpathlineto{\pgfqpoint{1.860708in}{1.229629in}}%
\pgfpathlineto{\pgfqpoint{1.863445in}{1.232013in}}%
\pgfpathlineto{\pgfqpoint{1.878578in}{1.243240in}}%
\pgfpathlineto{\pgfqpoint{1.879102in}{1.243680in}}%
\pgfpathlineto{\pgfqpoint{1.894758in}{1.252492in}}%
\pgfpathlineto{\pgfqpoint{1.910415in}{1.255698in}}%
\pgfpathlineto{\pgfqpoint{1.910415in}{1.256851in}}%
\pgfpathlineto{\pgfqpoint{1.910415in}{1.270462in}}%
\pgfpathlineto{\pgfqpoint{1.910415in}{1.284073in}}%
\pgfpathlineto{\pgfqpoint{1.910415in}{1.297684in}}%
\pgfpathlineto{\pgfqpoint{1.910415in}{1.311295in}}%
\pgfpathlineto{\pgfqpoint{1.910415in}{1.324907in}}%
\pgfpathlineto{\pgfqpoint{1.910415in}{1.338518in}}%
\pgfpathlineto{\pgfqpoint{1.910415in}{1.347636in}}%
\pgfpathlineto{\pgfqpoint{1.894758in}{1.351157in}}%
\pgfpathlineto{\pgfqpoint{1.893243in}{1.352129in}}%
\pgfpathlineto{\pgfqpoint{1.879102in}{1.359754in}}%
\pgfpathlineto{\pgfqpoint{1.871492in}{1.365740in}}%
\pgfpathlineto{\pgfqpoint{1.863445in}{1.371565in}}%
\pgfpathlineto{\pgfqpoint{1.854269in}{1.379351in}}%
\pgfpathlineto{\pgfqpoint{1.847789in}{1.384902in}}%
\pgfpathlineto{\pgfqpoint{1.838659in}{1.392962in}}%
\pgfpathlineto{\pgfqpoint{1.832132in}{1.399516in}}%
\pgfpathlineto{\pgfqpoint{1.824629in}{1.406573in}}%
\pgfpathlineto{\pgfqpoint{1.816476in}{1.417136in}}%
\pgfpathlineto{\pgfqpoint{1.813732in}{1.420184in}}%
\pgfpathlineto{\pgfqpoint{1.808322in}{1.433795in}}%
\pgfpathlineto{\pgfqpoint{1.810682in}{1.447407in}}%
\pgfpathlineto{\pgfqpoint{1.816476in}{1.455564in}}%
\pgfpathlineto{\pgfqpoint{1.819828in}{1.461018in}}%
\pgfpathlineto{\pgfqpoint{1.832132in}{1.473877in}}%
\pgfpathlineto{\pgfqpoint{1.832810in}{1.474629in}}%
\pgfpathlineto{\pgfqpoint{1.847671in}{1.488240in}}%
\pgfpathlineto{\pgfqpoint{1.847789in}{1.488342in}}%
\pgfpathlineto{\pgfqpoint{1.863445in}{1.501502in}}%
\pgfpathlineto{\pgfqpoint{1.863931in}{1.501851in}}%
\pgfpathlineto{\pgfqpoint{1.879102in}{1.513202in}}%
\pgfpathlineto{\pgfqpoint{1.883494in}{1.515462in}}%
\pgfpathlineto{\pgfqpoint{1.894758in}{1.522044in}}%
\pgfpathlineto{\pgfqpoint{1.910415in}{1.525335in}}%
\pgfpathlineto{\pgfqpoint{1.910415in}{1.529073in}}%
\pgfpathlineto{\pgfqpoint{1.910415in}{1.542684in}}%
\pgfpathlineto{\pgfqpoint{1.910415in}{1.556295in}}%
\pgfpathlineto{\pgfqpoint{1.910415in}{1.569907in}}%
\pgfpathlineto{\pgfqpoint{1.910415in}{1.583518in}}%
\pgfpathlineto{\pgfqpoint{1.910415in}{1.597129in}}%
\pgfpathlineto{\pgfqpoint{1.910415in}{1.610740in}}%
\pgfpathlineto{\pgfqpoint{1.910415in}{1.617124in}}%
\pgfpathlineto{\pgfqpoint{1.894758in}{1.620519in}}%
\pgfpathlineto{\pgfqpoint{1.888482in}{1.624351in}}%
\pgfpathlineto{\pgfqpoint{1.879102in}{1.629283in}}%
\pgfpathlineto{\pgfqpoint{1.867766in}{1.637962in}}%
\pgfpathlineto{\pgfqpoint{1.863445in}{1.641070in}}%
\pgfpathlineto{\pgfqpoint{1.850978in}{1.651573in}}%
\pgfpathlineto{\pgfqpoint{1.847789in}{1.654346in}}%
\pgfpathlineto{\pgfqpoint{1.835707in}{1.665184in}}%
\pgfpathlineto{\pgfqpoint{1.832132in}{1.668941in}}%
\pgfpathlineto{\pgfqpoint{1.822149in}{1.678795in}}%
\pgfpathlineto{\pgfqpoint{1.816476in}{1.686950in}}%
\pgfpathlineto{\pgfqpoint{1.812068in}{1.692407in}}%
\pgfpathlineto{\pgfqpoint{1.808163in}{1.706018in}}%
\pgfpathlineto{\pgfqpoint{1.800819in}{1.706018in}}%
\pgfpathlineto{\pgfqpoint{1.785162in}{1.706018in}}%
\pgfpathlineto{\pgfqpoint{1.769506in}{1.706018in}}%
\pgfpathlineto{\pgfqpoint{1.753849in}{1.706018in}}%
\pgfpathlineto{\pgfqpoint{1.738193in}{1.706018in}}%
\pgfpathlineto{\pgfqpoint{1.722536in}{1.706018in}}%
\pgfpathlineto{\pgfqpoint{1.706880in}{1.706018in}}%
\pgfpathlineto{\pgfqpoint{1.702579in}{1.706018in}}%
\pgfpathlineto{\pgfqpoint{1.698794in}{1.692407in}}%
\pgfpathlineto{\pgfqpoint{1.691223in}{1.682613in}}%
\pgfpathlineto{\pgfqpoint{1.688623in}{1.678795in}}%
\pgfpathlineto{\pgfqpoint{1.675567in}{1.665607in}}%
\pgfpathlineto{\pgfqpoint{1.675165in}{1.665184in}}%
\pgfpathlineto{\pgfqpoint{1.660027in}{1.651573in}}%
\pgfpathlineto{\pgfqpoint{1.659910in}{1.651471in}}%
\pgfpathlineto{\pgfqpoint{1.644253in}{1.638551in}}%
\pgfpathlineto{\pgfqpoint{1.643389in}{1.637962in}}%
\pgfpathlineto{\pgfqpoint{1.628597in}{1.627265in}}%
\pgfpathlineto{\pgfqpoint{1.622324in}{1.624351in}}%
\pgfpathlineto{\pgfqpoint{1.612940in}{1.619315in}}%
\pgfpathlineto{\pgfqpoint{1.597284in}{1.617263in}}%
\pgfpathlineto{\pgfqpoint{1.581627in}{1.621966in}}%
\pgfpathlineto{\pgfqpoint{1.578121in}{1.624351in}}%
\pgfpathlineto{\pgfqpoint{1.565971in}{1.631439in}}%
\pgfpathlineto{\pgfqpoint{1.557853in}{1.637962in}}%
\pgfpathlineto{\pgfqpoint{1.550314in}{1.643637in}}%
\pgfpathlineto{\pgfqpoint{1.541043in}{1.651573in}}%
\pgfpathlineto{\pgfqpoint{1.534657in}{1.657207in}}%
\pgfpathlineto{\pgfqpoint{1.525702in}{1.665184in}}%
\pgfpathlineto{\pgfqpoint{1.519001in}{1.672180in}}%
\pgfpathlineto{\pgfqpoint{1.512116in}{1.678795in}}%
\pgfpathlineto{\pgfqpoint{1.503344in}{1.691089in}}%
\pgfpathlineto{\pgfqpoint{1.502227in}{1.692407in}}%
\pgfpathlineto{\pgfqpoint{1.498177in}{1.706018in}}%
\pgfpathlineto{\pgfqpoint{1.487688in}{1.706018in}}%
\pgfpathlineto{\pgfqpoint{1.472031in}{1.706018in}}%
\pgfpathlineto{\pgfqpoint{1.456375in}{1.706018in}}%
\pgfpathlineto{\pgfqpoint{1.440718in}{1.706018in}}%
\pgfpathlineto{\pgfqpoint{1.425061in}{1.706018in}}%
\pgfpathlineto{\pgfqpoint{1.409405in}{1.706018in}}%
\pgfpathlineto{\pgfqpoint{1.393748in}{1.706018in}}%
\pgfpathlineto{\pgfqpoint{1.392422in}{1.706018in}}%
\pgfpathlineto{\pgfqpoint{1.388735in}{1.692407in}}%
\pgfpathlineto{\pgfqpoint{1.378598in}{1.678795in}}%
\pgfpathlineto{\pgfqpoint{1.378092in}{1.678340in}}%
\pgfpathlineto{\pgfqpoint{1.365178in}{1.665184in}}%
\pgfpathlineto{\pgfqpoint{1.362435in}{1.662804in}}%
\pgfpathlineto{\pgfqpoint{1.349904in}{1.651573in}}%
\pgfpathlineto{\pgfqpoint{1.346779in}{1.648852in}}%
\pgfpathlineto{\pgfqpoint{1.333282in}{1.637962in}}%
\pgfpathlineto{\pgfqpoint{1.331122in}{1.636092in}}%
\pgfpathlineto{\pgfqpoint{1.315466in}{1.625403in}}%
\pgfpathlineto{\pgfqpoint{1.312835in}{1.624351in}}%
\pgfpathlineto{\pgfqpoint{1.299809in}{1.618364in}}%
\pgfpathlineto{\pgfqpoint{1.284152in}{1.617678in}}%
\pgfpathlineto{\pgfqpoint{1.268496in}{1.623640in}}%
\pgfpathlineto{\pgfqpoint{1.267542in}{1.624351in}}%
\pgfpathlineto{\pgfqpoint{1.252839in}{1.633715in}}%
\pgfpathlineto{\pgfqpoint{1.247764in}{1.637962in}}%
\pgfpathlineto{\pgfqpoint{1.237183in}{1.646236in}}%
\pgfpathlineto{\pgfqpoint{1.231015in}{1.651573in}}%
\pgfpathlineto{\pgfqpoint{1.221526in}{1.660030in}}%
\pgfpathlineto{\pgfqpoint{1.215675in}{1.665184in}}%
\pgfpathlineto{\pgfqpoint{1.205870in}{1.675315in}}%
\pgfpathlineto{\pgfqpoint{1.202134in}{1.678795in}}%
\pgfpathlineto{\pgfqpoint{1.192214in}{1.692407in}}%
\pgfpathlineto{\pgfqpoint{1.190213in}{1.700003in}}%
\pgfpathlineto{\pgfqpoint{1.188332in}{1.706018in}}%
\pgfpathlineto{\pgfqpoint{1.174556in}{1.706018in}}%
\pgfpathlineto{\pgfqpoint{1.158900in}{1.706018in}}%
\pgfpathlineto{\pgfqpoint{1.143243in}{1.706018in}}%
\pgfpathlineto{\pgfqpoint{1.127587in}{1.706018in}}%
\pgfpathlineto{\pgfqpoint{1.111930in}{1.706018in}}%
\pgfpathlineto{\pgfqpoint{1.096274in}{1.706018in}}%
\pgfpathlineto{\pgfqpoint{1.082498in}{1.706018in}}%
\pgfpathlineto{\pgfqpoint{1.080617in}{1.700003in}}%
\pgfpathlineto{\pgfqpoint{1.078616in}{1.692407in}}%
\pgfpathlineto{\pgfqpoint{1.068696in}{1.678795in}}%
\pgfpathlineto{\pgfqpoint{1.064960in}{1.675315in}}%
\pgfpathlineto{\pgfqpoint{1.055155in}{1.665184in}}%
\pgfpathlineto{\pgfqpoint{1.049304in}{1.660030in}}%
\pgfpathlineto{\pgfqpoint{1.039815in}{1.651573in}}%
\pgfpathlineto{\pgfqpoint{1.033647in}{1.646236in}}%
\pgfpathlineto{\pgfqpoint{1.023066in}{1.637962in}}%
\pgfpathlineto{\pgfqpoint{1.017991in}{1.633715in}}%
\pgfpathlineto{\pgfqpoint{1.003288in}{1.624351in}}%
\pgfpathlineto{\pgfqpoint{1.002334in}{1.623640in}}%
\pgfpathlineto{\pgfqpoint{0.986678in}{1.617678in}}%
\pgfpathlineto{\pgfqpoint{0.971021in}{1.618364in}}%
\pgfpathlineto{\pgfqpoint{0.957995in}{1.624351in}}%
\pgfpathlineto{\pgfqpoint{0.955364in}{1.625403in}}%
\pgfpathlineto{\pgfqpoint{0.939708in}{1.636092in}}%
\pgfpathlineto{\pgfqpoint{0.937548in}{1.637962in}}%
\pgfpathlineto{\pgfqpoint{0.924051in}{1.648852in}}%
\pgfpathlineto{\pgfqpoint{0.920926in}{1.651573in}}%
\pgfpathlineto{\pgfqpoint{0.908395in}{1.662804in}}%
\pgfpathlineto{\pgfqpoint{0.905652in}{1.665184in}}%
\pgfpathlineto{\pgfqpoint{0.892738in}{1.678340in}}%
\pgfpathlineto{\pgfqpoint{0.892232in}{1.678795in}}%
\pgfpathlineto{\pgfqpoint{0.882095in}{1.692407in}}%
\pgfpathlineto{\pgfqpoint{0.878408in}{1.706018in}}%
\pgfpathlineto{\pgfqpoint{0.877082in}{1.706018in}}%
\pgfpathlineto{\pgfqpoint{0.861425in}{1.706018in}}%
\pgfpathlineto{\pgfqpoint{0.845769in}{1.706018in}}%
\pgfpathlineto{\pgfqpoint{0.830112in}{1.706018in}}%
\pgfpathlineto{\pgfqpoint{0.814455in}{1.706018in}}%
\pgfpathlineto{\pgfqpoint{0.798799in}{1.706018in}}%
\pgfpathlineto{\pgfqpoint{0.783142in}{1.706018in}}%
\pgfpathlineto{\pgfqpoint{0.772653in}{1.706018in}}%
\pgfpathlineto{\pgfqpoint{0.768603in}{1.692407in}}%
\pgfpathlineto{\pgfqpoint{0.767486in}{1.691089in}}%
\pgfpathlineto{\pgfqpoint{0.758714in}{1.678795in}}%
\pgfpathlineto{\pgfqpoint{0.751829in}{1.672180in}}%
\pgfpathlineto{\pgfqpoint{0.745128in}{1.665184in}}%
\pgfpathlineto{\pgfqpoint{0.736173in}{1.657207in}}%
\pgfpathlineto{\pgfqpoint{0.729787in}{1.651573in}}%
\pgfpathlineto{\pgfqpoint{0.720516in}{1.643637in}}%
\pgfpathlineto{\pgfqpoint{0.712977in}{1.637962in}}%
\pgfpathlineto{\pgfqpoint{0.704859in}{1.631439in}}%
\pgfpathlineto{\pgfqpoint{0.692709in}{1.624351in}}%
\pgfpathlineto{\pgfqpoint{0.689203in}{1.621966in}}%
\pgfpathlineto{\pgfqpoint{0.673546in}{1.617263in}}%
\pgfpathlineto{\pgfqpoint{0.657890in}{1.619315in}}%
\pgfpathlineto{\pgfqpoint{0.648506in}{1.624351in}}%
\pgfpathlineto{\pgfqpoint{0.642233in}{1.627265in}}%
\pgfpathlineto{\pgfqpoint{0.627441in}{1.637962in}}%
\pgfpathlineto{\pgfqpoint{0.626577in}{1.638551in}}%
\pgfpathlineto{\pgfqpoint{0.610920in}{1.651471in}}%
\pgfpathlineto{\pgfqpoint{0.610803in}{1.651573in}}%
\pgfpathlineto{\pgfqpoint{0.595665in}{1.665184in}}%
\pgfpathlineto{\pgfqpoint{0.595263in}{1.665607in}}%
\pgfpathlineto{\pgfqpoint{0.582207in}{1.678795in}}%
\pgfpathlineto{\pgfqpoint{0.579607in}{1.682613in}}%
\pgfpathlineto{\pgfqpoint{0.572036in}{1.692407in}}%
\pgfpathlineto{\pgfqpoint{0.568251in}{1.706018in}}%
\pgfpathlineto{\pgfqpoint{0.563950in}{1.706018in}}%
\pgfpathlineto{\pgfqpoint{0.548294in}{1.706018in}}%
\pgfpathlineto{\pgfqpoint{0.532637in}{1.706018in}}%
\pgfpathlineto{\pgfqpoint{0.516981in}{1.706018in}}%
\pgfpathlineto{\pgfqpoint{0.501324in}{1.706018in}}%
\pgfpathlineto{\pgfqpoint{0.485668in}{1.706018in}}%
\pgfpathlineto{\pgfqpoint{0.470011in}{1.706018in}}%
\pgfpathlineto{\pgfqpoint{0.462667in}{1.706018in}}%
\pgfpathlineto{\pgfqpoint{0.458762in}{1.692407in}}%
\pgfpathlineto{\pgfqpoint{0.454354in}{1.686950in}}%
\pgfpathlineto{\pgfqpoint{0.448681in}{1.678795in}}%
\pgfpathlineto{\pgfqpoint{0.438698in}{1.668941in}}%
\pgfpathlineto{\pgfqpoint{0.435123in}{1.665184in}}%
\pgfpathlineto{\pgfqpoint{0.423041in}{1.654346in}}%
\pgfpathlineto{\pgfqpoint{0.419852in}{1.651573in}}%
\pgfpathlineto{\pgfqpoint{0.407385in}{1.641070in}}%
\pgfpathlineto{\pgfqpoint{0.403064in}{1.637962in}}%
\pgfpathlineto{\pgfqpoint{0.391728in}{1.629283in}}%
\pgfpathlineto{\pgfqpoint{0.382348in}{1.624351in}}%
\pgfpathlineto{\pgfqpoint{0.376072in}{1.620519in}}%
\pgfpathlineto{\pgfqpoint{0.360415in}{1.617124in}}%
\pgfpathlineto{\pgfqpoint{0.360415in}{1.610740in}}%
\pgfpathlineto{\pgfqpoint{0.360415in}{1.597129in}}%
\pgfpathlineto{\pgfqpoint{0.360415in}{1.583518in}}%
\pgfpathlineto{\pgfqpoint{0.360415in}{1.569907in}}%
\pgfpathlineto{\pgfqpoint{0.360415in}{1.556295in}}%
\pgfpathlineto{\pgfqpoint{0.360415in}{1.542684in}}%
\pgfpathlineto{\pgfqpoint{0.360415in}{1.529073in}}%
\pgfpathlineto{\pgfqpoint{0.360415in}{1.525335in}}%
\pgfpathlineto{\pgfqpoint{0.376072in}{1.522044in}}%
\pgfpathlineto{\pgfqpoint{0.387336in}{1.515462in}}%
\pgfpathlineto{\pgfqpoint{0.391728in}{1.513202in}}%
\pgfpathlineto{\pgfqpoint{0.406899in}{1.501851in}}%
\pgfpathlineto{\pgfqpoint{0.407385in}{1.501502in}}%
\pgfpathlineto{\pgfqpoint{0.423041in}{1.488342in}}%
\pgfpathlineto{\pgfqpoint{0.423159in}{1.488240in}}%
\pgfpathlineto{\pgfqpoint{0.438020in}{1.474629in}}%
\pgfpathlineto{\pgfqpoint{0.438698in}{1.473877in}}%
\pgfpathlineto{\pgfqpoint{0.451002in}{1.461018in}}%
\pgfpathlineto{\pgfqpoint{0.454354in}{1.455564in}}%
\pgfpathlineto{\pgfqpoint{0.460148in}{1.447407in}}%
\pgfpathlineto{\pgfqpoint{0.462508in}{1.433795in}}%
\pgfpathlineto{\pgfqpoint{0.457098in}{1.420184in}}%
\pgfpathlineto{\pgfqpoint{0.454354in}{1.417136in}}%
\pgfpathlineto{\pgfqpoint{0.446201in}{1.406573in}}%
\pgfpathlineto{\pgfqpoint{0.438698in}{1.399516in}}%
\pgfpathlineto{\pgfqpoint{0.432171in}{1.392962in}}%
\pgfpathlineto{\pgfqpoint{0.423041in}{1.384902in}}%
\pgfpathlineto{\pgfqpoint{0.416561in}{1.379351in}}%
\pgfpathlineto{\pgfqpoint{0.407385in}{1.371565in}}%
\pgfpathlineto{\pgfqpoint{0.399338in}{1.365740in}}%
\pgfpathlineto{\pgfqpoint{0.391728in}{1.359754in}}%
\pgfpathlineto{\pgfqpoint{0.377587in}{1.352129in}}%
\pgfpathlineto{\pgfqpoint{0.376072in}{1.351157in}}%
\pgfpathlineto{\pgfqpoint{0.360415in}{1.347636in}}%
\pgfpathlineto{\pgfqpoint{0.360415in}{1.338518in}}%
\pgfpathlineto{\pgfqpoint{0.360415in}{1.324907in}}%
\pgfpathlineto{\pgfqpoint{0.360415in}{1.311295in}}%
\pgfpathlineto{\pgfqpoint{0.360415in}{1.297684in}}%
\pgfpathlineto{\pgfqpoint{0.360415in}{1.284073in}}%
\pgfpathlineto{\pgfqpoint{0.360415in}{1.270462in}}%
\pgfpathlineto{\pgfqpoint{0.360415in}{1.256851in}}%
\pgfpathlineto{\pgfqpoint{0.360415in}{1.255698in}}%
\pgfpathlineto{\pgfqpoint{0.376072in}{1.252492in}}%
\pgfpathlineto{\pgfqpoint{0.391728in}{1.243680in}}%
\pgfpathlineto{\pgfqpoint{0.392252in}{1.243240in}}%
\pgfpathlineto{\pgfqpoint{0.407385in}{1.232013in}}%
\pgfpathlineto{\pgfqpoint{0.410122in}{1.229629in}}%
\pgfpathlineto{\pgfqpoint{0.423041in}{1.218735in}}%
\pgfpathlineto{\pgfqpoint{0.426171in}{1.216018in}}%
\pgfpathlineto{\pgfqpoint{0.438698in}{1.204285in}}%
\pgfpathlineto{\pgfqpoint{0.440848in}{1.202407in}}%
\pgfpathlineto{\pgfqpoint{0.453145in}{1.188795in}}%
\pgfpathlineto{\pgfqpoint{0.454354in}{1.186508in}}%
\pgfpathlineto{\pgfqpoint{0.461241in}{1.175184in}}%
\pgfpathlineto{\pgfqpoint{0.462031in}{1.161573in}}%
\pgfpathlineto{\pgfqpoint{0.455173in}{1.147962in}}%
\pgfpathlineto{\pgfqpoint{0.454354in}{1.147133in}}%
\pgfpathlineto{\pgfqpoint{0.443583in}{1.134351in}}%
\pgfpathlineto{\pgfqpoint{0.438698in}{1.129939in}}%
\pgfpathlineto{\pgfqpoint{0.429181in}{1.120740in}}%
\pgfpathlineto{\pgfqpoint{0.423041in}{1.115378in}}%
\pgfpathlineto{\pgfqpoint{0.413313in}{1.107129in}}%
\pgfpathlineto{\pgfqpoint{0.407385in}{1.102042in}}%
\pgfpathlineto{\pgfqpoint{0.395731in}{1.093518in}}%
\pgfpathlineto{\pgfqpoint{0.391728in}{1.090270in}}%
\pgfpathlineto{\pgfqpoint{0.376072in}{1.081646in}}%
\pgfpathlineto{\pgfqpoint{0.367333in}{1.079907in}}%
\pgfpathlineto{\pgfqpoint{0.360415in}{1.078271in}}%
\pgfpathlineto{\pgfqpoint{0.360415in}{1.066295in}}%
\pgfpathlineto{\pgfqpoint{0.360415in}{1.052684in}}%
\pgfpathlineto{\pgfqpoint{0.360415in}{1.039073in}}%
\pgfpathlineto{\pgfqpoint{0.360415in}{1.025462in}}%
\pgfpathlineto{\pgfqpoint{0.360415in}{1.011851in}}%
\pgfpathlineto{\pgfqpoint{0.360415in}{0.998240in}}%
\pgfpathlineto{\pgfqpoint{0.360415in}{0.986264in}}%
\pgfpathlineto{\pgfqpoint{0.367333in}{0.984629in}}%
\pgfpathlineto{\pgfqpoint{0.376072in}{0.982889in}}%
\pgfpathlineto{\pgfqpoint{0.391728in}{0.974265in}}%
\pgfpathlineto{\pgfqpoint{0.395731in}{0.971018in}}%
\pgfpathlineto{\pgfqpoint{0.407385in}{0.962493in}}%
\pgfpathlineto{\pgfqpoint{0.413313in}{0.957407in}}%
\pgfpathlineto{\pgfqpoint{0.423041in}{0.949158in}}%
\pgfpathlineto{\pgfqpoint{0.429181in}{0.943795in}}%
\pgfpathlineto{\pgfqpoint{0.438698in}{0.934596in}}%
\pgfpathlineto{\pgfqpoint{0.443583in}{0.930184in}}%
\pgfpathlineto{\pgfqpoint{0.454354in}{0.917402in}}%
\pgfpathlineto{\pgfqpoint{0.455173in}{0.916573in}}%
\pgfpathlineto{\pgfqpoint{0.462031in}{0.902962in}}%
\pgfpathlineto{\pgfqpoint{0.461241in}{0.889351in}}%
\pgfpathlineto{\pgfqpoint{0.454354in}{0.878027in}}%
\pgfpathlineto{\pgfqpoint{0.453145in}{0.875740in}}%
\pgfpathlineto{\pgfqpoint{0.440848in}{0.862129in}}%
\pgfpathlineto{\pgfqpoint{0.438698in}{0.860251in}}%
\pgfpathlineto{\pgfqpoint{0.426171in}{0.848518in}}%
\pgfpathlineto{\pgfqpoint{0.423041in}{0.845800in}}%
\pgfpathlineto{\pgfqpoint{0.410122in}{0.834907in}}%
\pgfpathlineto{\pgfqpoint{0.407385in}{0.832522in}}%
\pgfpathlineto{\pgfqpoint{0.392252in}{0.821295in}}%
\pgfpathlineto{\pgfqpoint{0.391728in}{0.820855in}}%
\pgfpathlineto{\pgfqpoint{0.376072in}{0.812043in}}%
\pgfpathlineto{\pgfqpoint{0.360415in}{0.808837in}}%
\pgfpathlineto{\pgfqpoint{0.360415in}{0.807684in}}%
\pgfpathlineto{\pgfqpoint{0.360415in}{0.794073in}}%
\pgfpathlineto{\pgfqpoint{0.360415in}{0.780462in}}%
\pgfpathlineto{\pgfqpoint{0.360415in}{0.766851in}}%
\pgfpathlineto{\pgfqpoint{0.360415in}{0.753240in}}%
\pgfpathlineto{\pgfqpoint{0.360415in}{0.739629in}}%
\pgfpathlineto{\pgfqpoint{0.360415in}{0.726018in}}%
\pgfpathlineto{\pgfqpoint{0.360415in}{0.716899in}}%
\pgfpathlineto{\pgfqpoint{0.376072in}{0.713378in}}%
\pgfpathlineto{\pgfqpoint{0.377587in}{0.712407in}}%
\pgfpathlineto{\pgfqpoint{0.391728in}{0.704781in}}%
\pgfpathlineto{\pgfqpoint{0.399338in}{0.698795in}}%
\pgfpathlineto{\pgfqpoint{0.407385in}{0.692970in}}%
\pgfpathlineto{\pgfqpoint{0.416561in}{0.685184in}}%
\pgfpathlineto{\pgfqpoint{0.423041in}{0.679633in}}%
\pgfpathlineto{\pgfqpoint{0.432171in}{0.671573in}}%
\pgfpathlineto{\pgfqpoint{0.438698in}{0.665019in}}%
\pgfpathlineto{\pgfqpoint{0.446201in}{0.657962in}}%
\pgfpathlineto{\pgfqpoint{0.454354in}{0.647399in}}%
\pgfpathlineto{\pgfqpoint{0.457098in}{0.644351in}}%
\pgfpathlineto{\pgfqpoint{0.462508in}{0.630740in}}%
\pgfpathlineto{\pgfqpoint{0.460148in}{0.617129in}}%
\pgfpathlineto{\pgfqpoint{0.454354in}{0.608971in}}%
\pgfpathlineto{\pgfqpoint{0.451002in}{0.603518in}}%
\pgfpathlineto{\pgfqpoint{0.438698in}{0.590658in}}%
\pgfpathlineto{\pgfqpoint{0.438020in}{0.589907in}}%
\pgfpathlineto{\pgfqpoint{0.423159in}{0.576295in}}%
\pgfpathlineto{\pgfqpoint{0.423041in}{0.576193in}}%
\pgfpathlineto{\pgfqpoint{0.407385in}{0.563033in}}%
\pgfpathlineto{\pgfqpoint{0.406899in}{0.562684in}}%
\pgfpathlineto{\pgfqpoint{0.391728in}{0.551333in}}%
\pgfpathlineto{\pgfqpoint{0.387336in}{0.549073in}}%
\pgfpathlineto{\pgfqpoint{0.376072in}{0.542492in}}%
\pgfpathlineto{\pgfqpoint{0.360415in}{0.539201in}}%
\pgfpathlineto{\pgfqpoint{0.360415in}{0.535462in}}%
\pgfpathlineto{\pgfqpoint{0.360415in}{0.521851in}}%
\pgfpathlineto{\pgfqpoint{0.360415in}{0.508240in}}%
\pgfpathlineto{\pgfqpoint{0.360415in}{0.494629in}}%
\pgfpathlineto{\pgfqpoint{0.360415in}{0.481018in}}%
\pgfpathlineto{\pgfqpoint{0.360415in}{0.467407in}}%
\pgfpathlineto{\pgfqpoint{0.360415in}{0.453795in}}%
\pgfpathlineto{\pgfqpoint{0.360415in}{0.447411in}}%
\pgfpathlineto{\pgfqpoint{0.376072in}{0.444016in}}%
\pgfpathlineto{\pgfqpoint{0.382348in}{0.440184in}}%
\pgfpathlineto{\pgfqpoint{0.391728in}{0.435252in}}%
\pgfpathlineto{\pgfqpoint{0.403064in}{0.426573in}}%
\pgfpathlineto{\pgfqpoint{0.407385in}{0.423466in}}%
\pgfpathlineto{\pgfqpoint{0.419852in}{0.412962in}}%
\pgfpathlineto{\pgfqpoint{0.423041in}{0.410189in}}%
\pgfpathlineto{\pgfqpoint{0.435123in}{0.399351in}}%
\pgfpathlineto{\pgfqpoint{0.438698in}{0.395595in}}%
\pgfpathlineto{\pgfqpoint{0.448681in}{0.385740in}}%
\pgfpathlineto{\pgfqpoint{0.454354in}{0.377585in}}%
\pgfpathlineto{\pgfqpoint{0.458762in}{0.372129in}}%
\pgfpathlineto{\pgfqpoint{0.462667in}{0.358518in}}%
\pgfpathlineto{\pgfqpoint{0.470011in}{0.358518in}}%
\pgfpathclose%
\pgfpathmoveto{\pgfqpoint{0.514940in}{0.385740in}}%
\pgfpathlineto{\pgfqpoint{0.501324in}{0.387174in}}%
\pgfpathlineto{\pgfqpoint{0.485668in}{0.392949in}}%
\pgfpathlineto{\pgfqpoint{0.475405in}{0.399351in}}%
\pgfpathlineto{\pgfqpoint{0.470011in}{0.402138in}}%
\pgfpathlineto{\pgfqpoint{0.454700in}{0.412962in}}%
\pgfpathlineto{\pgfqpoint{0.454354in}{0.413196in}}%
\pgfpathlineto{\pgfqpoint{0.438698in}{0.425835in}}%
\pgfpathlineto{\pgfqpoint{0.437849in}{0.426573in}}%
\pgfpathlineto{\pgfqpoint{0.423310in}{0.440184in}}%
\pgfpathlineto{\pgfqpoint{0.423041in}{0.440485in}}%
\pgfpathlineto{\pgfqpoint{0.410591in}{0.453795in}}%
\pgfpathlineto{\pgfqpoint{0.407385in}{0.458485in}}%
\pgfpathlineto{\pgfqpoint{0.400021in}{0.467407in}}%
\pgfpathlineto{\pgfqpoint{0.393377in}{0.481018in}}%
\pgfpathlineto{\pgfqpoint{0.391728in}{0.492855in}}%
\pgfpathlineto{\pgfqpoint{0.391346in}{0.494629in}}%
\pgfpathlineto{\pgfqpoint{0.391728in}{0.495815in}}%
\pgfpathlineto{\pgfqpoint{0.394325in}{0.508240in}}%
\pgfpathlineto{\pgfqpoint{0.401921in}{0.521851in}}%
\pgfpathlineto{\pgfqpoint{0.407385in}{0.528027in}}%
\pgfpathlineto{\pgfqpoint{0.412851in}{0.535462in}}%
\pgfpathlineto{\pgfqpoint{0.423041in}{0.545914in}}%
\pgfpathlineto{\pgfqpoint{0.425993in}{0.549073in}}%
\pgfpathlineto{\pgfqpoint{0.438698in}{0.560671in}}%
\pgfpathlineto{\pgfqpoint{0.441068in}{0.562684in}}%
\pgfpathlineto{\pgfqpoint{0.454354in}{0.573422in}}%
\pgfpathlineto{\pgfqpoint{0.458618in}{0.576295in}}%
\pgfpathlineto{\pgfqpoint{0.470011in}{0.584512in}}%
\pgfpathlineto{\pgfqpoint{0.480229in}{0.589907in}}%
\pgfpathlineto{\pgfqpoint{0.485668in}{0.593487in}}%
\pgfpathlineto{\pgfqpoint{0.501324in}{0.599661in}}%
\pgfpathlineto{\pgfqpoint{0.516981in}{0.601422in}}%
\pgfpathlineto{\pgfqpoint{0.532637in}{0.598780in}}%
\pgfpathlineto{\pgfqpoint{0.548294in}{0.591720in}}%
\pgfpathlineto{\pgfqpoint{0.550864in}{0.589907in}}%
\pgfpathlineto{\pgfqpoint{0.563950in}{0.582478in}}%
\pgfpathlineto{\pgfqpoint{0.572176in}{0.576295in}}%
\pgfpathlineto{\pgfqpoint{0.579607in}{0.571062in}}%
\pgfpathlineto{\pgfqpoint{0.589753in}{0.562684in}}%
\pgfpathlineto{\pgfqpoint{0.595263in}{0.557894in}}%
\pgfpathlineto{\pgfqpoint{0.604900in}{0.549073in}}%
\pgfpathlineto{\pgfqpoint{0.610920in}{0.542613in}}%
\pgfpathlineto{\pgfqpoint{0.618032in}{0.535462in}}%
\pgfpathlineto{\pgfqpoint{0.626577in}{0.524086in}}%
\pgfpathlineto{\pgfqpoint{0.628663in}{0.521851in}}%
\pgfpathlineto{\pgfqpoint{0.636784in}{0.508240in}}%
\pgfpathlineto{\pgfqpoint{0.639823in}{0.494629in}}%
\pgfpathlineto{\pgfqpoint{0.637797in}{0.481018in}}%
\pgfpathlineto{\pgfqpoint{0.630695in}{0.467407in}}%
\pgfpathlineto{\pgfqpoint{0.626577in}{0.462679in}}%
\pgfpathlineto{\pgfqpoint{0.620371in}{0.453795in}}%
\pgfpathlineto{\pgfqpoint{0.610920in}{0.443891in}}%
\pgfpathlineto{\pgfqpoint{0.607615in}{0.440184in}}%
\pgfpathlineto{\pgfqpoint{0.595263in}{0.428634in}}%
\pgfpathlineto{\pgfqpoint{0.592947in}{0.426573in}}%
\pgfpathlineto{\pgfqpoint{0.579607in}{0.415528in}}%
\pgfpathlineto{\pgfqpoint{0.575973in}{0.412962in}}%
\pgfpathlineto{\pgfqpoint{0.563950in}{0.404103in}}%
\pgfpathlineto{\pgfqpoint{0.555398in}{0.399351in}}%
\pgfpathlineto{\pgfqpoint{0.548294in}{0.394601in}}%
\pgfpathlineto{\pgfqpoint{0.532637in}{0.387998in}}%
\pgfpathlineto{\pgfqpoint{0.518346in}{0.385740in}}%
\pgfpathlineto{\pgfqpoint{0.516981in}{0.385408in}}%
\pgfpathlineto{\pgfqpoint{0.514940in}{0.385740in}}%
\pgfpathclose%
\pgfpathmoveto{\pgfqpoint{0.829184in}{0.385740in}}%
\pgfpathlineto{\pgfqpoint{0.814455in}{0.386515in}}%
\pgfpathlineto{\pgfqpoint{0.798799in}{0.391463in}}%
\pgfpathlineto{\pgfqpoint{0.785138in}{0.399351in}}%
\pgfpathlineto{\pgfqpoint{0.783142in}{0.400303in}}%
\pgfpathlineto{\pgfqpoint{0.767486in}{0.410757in}}%
\pgfpathlineto{\pgfqpoint{0.764776in}{0.412962in}}%
\pgfpathlineto{\pgfqpoint{0.751829in}{0.423122in}}%
\pgfpathlineto{\pgfqpoint{0.747834in}{0.426573in}}%
\pgfpathlineto{\pgfqpoint{0.736173in}{0.437428in}}%
\pgfpathlineto{\pgfqpoint{0.733174in}{0.440184in}}%
\pgfpathlineto{\pgfqpoint{0.720795in}{0.453795in}}%
\pgfpathlineto{\pgfqpoint{0.720516in}{0.454209in}}%
\pgfpathlineto{\pgfqpoint{0.710108in}{0.467407in}}%
\pgfpathlineto{\pgfqpoint{0.704859in}{0.478765in}}%
\pgfpathlineto{\pgfqpoint{0.703365in}{0.481018in}}%
\pgfpathlineto{\pgfqpoint{0.700734in}{0.494629in}}%
\pgfpathlineto{\pgfqpoint{0.704681in}{0.508240in}}%
\pgfpathlineto{\pgfqpoint{0.704859in}{0.508476in}}%
\pgfpathlineto{\pgfqpoint{0.711900in}{0.521851in}}%
\pgfpathlineto{\pgfqpoint{0.720516in}{0.532044in}}%
\pgfpathlineto{\pgfqpoint{0.722990in}{0.535462in}}%
\pgfpathlineto{\pgfqpoint{0.736057in}{0.549073in}}%
\pgfpathlineto{\pgfqpoint{0.736173in}{0.549177in}}%
\pgfpathlineto{\pgfqpoint{0.751021in}{0.562684in}}%
\pgfpathlineto{\pgfqpoint{0.751829in}{0.563380in}}%
\pgfpathlineto{\pgfqpoint{0.767486in}{0.575674in}}%
\pgfpathlineto{\pgfqpoint{0.768454in}{0.576295in}}%
\pgfpathlineto{\pgfqpoint{0.783142in}{0.586412in}}%
\pgfpathlineto{\pgfqpoint{0.790316in}{0.589907in}}%
\pgfpathlineto{\pgfqpoint{0.798799in}{0.595076in}}%
\pgfpathlineto{\pgfqpoint{0.814455in}{0.600366in}}%
\pgfpathlineto{\pgfqpoint{0.830112in}{0.601246in}}%
\pgfpathlineto{\pgfqpoint{0.845769in}{0.597722in}}%
\pgfpathlineto{\pgfqpoint{0.861178in}{0.589907in}}%
\pgfpathlineto{\pgfqpoint{0.861425in}{0.589807in}}%
\pgfpathlineto{\pgfqpoint{0.877082in}{0.580312in}}%
\pgfpathlineto{\pgfqpoint{0.882232in}{0.576295in}}%
\pgfpathlineto{\pgfqpoint{0.892738in}{0.568599in}}%
\pgfpathlineto{\pgfqpoint{0.899775in}{0.562684in}}%
\pgfpathlineto{\pgfqpoint{0.908395in}{0.555046in}}%
\pgfpathlineto{\pgfqpoint{0.914930in}{0.549073in}}%
\pgfpathlineto{\pgfqpoint{0.924051in}{0.539299in}}%
\pgfpathlineto{\pgfqpoint{0.927965in}{0.535462in}}%
\pgfpathlineto{\pgfqpoint{0.938525in}{0.521851in}}%
\pgfpathlineto{\pgfqpoint{0.939708in}{0.519430in}}%
\pgfpathlineto{\pgfqpoint{0.946874in}{0.508240in}}%
\pgfpathlineto{\pgfqpoint{0.950153in}{0.494629in}}%
\pgfpathlineto{\pgfqpoint{0.947967in}{0.481018in}}%
\pgfpathlineto{\pgfqpoint{0.940304in}{0.467407in}}%
\pgfpathlineto{\pgfqpoint{0.939708in}{0.466764in}}%
\pgfpathlineto{\pgfqpoint{0.930399in}{0.453795in}}%
\pgfpathlineto{\pgfqpoint{0.924051in}{0.447310in}}%
\pgfpathlineto{\pgfqpoint{0.917687in}{0.440184in}}%
\pgfpathlineto{\pgfqpoint{0.908395in}{0.431506in}}%
\pgfpathlineto{\pgfqpoint{0.902954in}{0.426573in}}%
\pgfpathlineto{\pgfqpoint{0.892738in}{0.417963in}}%
\pgfpathlineto{\pgfqpoint{0.885930in}{0.412962in}}%
\pgfpathlineto{\pgfqpoint{0.877082in}{0.406196in}}%
\pgfpathlineto{\pgfqpoint{0.865545in}{0.399351in}}%
\pgfpathlineto{\pgfqpoint{0.861425in}{0.396419in}}%
\pgfpathlineto{\pgfqpoint{0.845769in}{0.388987in}}%
\pgfpathlineto{\pgfqpoint{0.830346in}{0.385740in}}%
\pgfpathlineto{\pgfqpoint{0.830112in}{0.385664in}}%
\pgfpathlineto{\pgfqpoint{0.829184in}{0.385740in}}%
\pgfpathclose%
\pgfpathmoveto{\pgfqpoint{1.440484in}{0.385740in}}%
\pgfpathlineto{\pgfqpoint{1.425061in}{0.388987in}}%
\pgfpathlineto{\pgfqpoint{1.409405in}{0.396419in}}%
\pgfpathlineto{\pgfqpoint{1.405285in}{0.399351in}}%
\pgfpathlineto{\pgfqpoint{1.393748in}{0.406196in}}%
\pgfpathlineto{\pgfqpoint{1.384900in}{0.412962in}}%
\pgfpathlineto{\pgfqpoint{1.378092in}{0.417963in}}%
\pgfpathlineto{\pgfqpoint{1.367876in}{0.426573in}}%
\pgfpathlineto{\pgfqpoint{1.362435in}{0.431506in}}%
\pgfpathlineto{\pgfqpoint{1.353143in}{0.440184in}}%
\pgfpathlineto{\pgfqpoint{1.346779in}{0.447310in}}%
\pgfpathlineto{\pgfqpoint{1.340431in}{0.453795in}}%
\pgfpathlineto{\pgfqpoint{1.331122in}{0.466764in}}%
\pgfpathlineto{\pgfqpoint{1.330526in}{0.467407in}}%
\pgfpathlineto{\pgfqpoint{1.322863in}{0.481018in}}%
\pgfpathlineto{\pgfqpoint{1.320677in}{0.494629in}}%
\pgfpathlineto{\pgfqpoint{1.323956in}{0.508240in}}%
\pgfpathlineto{\pgfqpoint{1.331122in}{0.519430in}}%
\pgfpathlineto{\pgfqpoint{1.332305in}{0.521851in}}%
\pgfpathlineto{\pgfqpoint{1.342865in}{0.535462in}}%
\pgfpathlineto{\pgfqpoint{1.346779in}{0.539299in}}%
\pgfpathlineto{\pgfqpoint{1.355900in}{0.549073in}}%
\pgfpathlineto{\pgfqpoint{1.362435in}{0.555046in}}%
\pgfpathlineto{\pgfqpoint{1.371055in}{0.562684in}}%
\pgfpathlineto{\pgfqpoint{1.378092in}{0.568599in}}%
\pgfpathlineto{\pgfqpoint{1.388598in}{0.576295in}}%
\pgfpathlineto{\pgfqpoint{1.393748in}{0.580312in}}%
\pgfpathlineto{\pgfqpoint{1.409405in}{0.589807in}}%
\pgfpathlineto{\pgfqpoint{1.409652in}{0.589907in}}%
\pgfpathlineto{\pgfqpoint{1.425061in}{0.597722in}}%
\pgfpathlineto{\pgfqpoint{1.440718in}{0.601246in}}%
\pgfpathlineto{\pgfqpoint{1.456375in}{0.600366in}}%
\pgfpathlineto{\pgfqpoint{1.472031in}{0.595076in}}%
\pgfpathlineto{\pgfqpoint{1.480514in}{0.589907in}}%
\pgfpathlineto{\pgfqpoint{1.487688in}{0.586412in}}%
\pgfpathlineto{\pgfqpoint{1.502376in}{0.576295in}}%
\pgfpathlineto{\pgfqpoint{1.503344in}{0.575674in}}%
\pgfpathlineto{\pgfqpoint{1.519001in}{0.563380in}}%
\pgfpathlineto{\pgfqpoint{1.519809in}{0.562684in}}%
\pgfpathlineto{\pgfqpoint{1.534657in}{0.549177in}}%
\pgfpathlineto{\pgfqpoint{1.534773in}{0.549073in}}%
\pgfpathlineto{\pgfqpoint{1.547840in}{0.535462in}}%
\pgfpathlineto{\pgfqpoint{1.550314in}{0.532044in}}%
\pgfpathlineto{\pgfqpoint{1.558930in}{0.521851in}}%
\pgfpathlineto{\pgfqpoint{1.565971in}{0.508476in}}%
\pgfpathlineto{\pgfqpoint{1.566149in}{0.508240in}}%
\pgfpathlineto{\pgfqpoint{1.570096in}{0.494629in}}%
\pgfpathlineto{\pgfqpoint{1.567465in}{0.481018in}}%
\pgfpathlineto{\pgfqpoint{1.565971in}{0.478765in}}%
\pgfpathlineto{\pgfqpoint{1.560722in}{0.467407in}}%
\pgfpathlineto{\pgfqpoint{1.550314in}{0.454209in}}%
\pgfpathlineto{\pgfqpoint{1.550035in}{0.453795in}}%
\pgfpathlineto{\pgfqpoint{1.537656in}{0.440184in}}%
\pgfpathlineto{\pgfqpoint{1.534657in}{0.437428in}}%
\pgfpathlineto{\pgfqpoint{1.522996in}{0.426573in}}%
\pgfpathlineto{\pgfqpoint{1.519001in}{0.423122in}}%
\pgfpathlineto{\pgfqpoint{1.506054in}{0.412962in}}%
\pgfpathlineto{\pgfqpoint{1.503344in}{0.410757in}}%
\pgfpathlineto{\pgfqpoint{1.487688in}{0.400303in}}%
\pgfpathlineto{\pgfqpoint{1.485692in}{0.399351in}}%
\pgfpathlineto{\pgfqpoint{1.472031in}{0.391463in}}%
\pgfpathlineto{\pgfqpoint{1.456375in}{0.386515in}}%
\pgfpathlineto{\pgfqpoint{1.441646in}{0.385740in}}%
\pgfpathlineto{\pgfqpoint{1.440718in}{0.385664in}}%
\pgfpathlineto{\pgfqpoint{1.440484in}{0.385740in}}%
\pgfpathclose%
\pgfpathmoveto{\pgfqpoint{1.752484in}{0.385740in}}%
\pgfpathlineto{\pgfqpoint{1.738193in}{0.387998in}}%
\pgfpathlineto{\pgfqpoint{1.722536in}{0.394601in}}%
\pgfpathlineto{\pgfqpoint{1.715432in}{0.399351in}}%
\pgfpathlineto{\pgfqpoint{1.706880in}{0.404103in}}%
\pgfpathlineto{\pgfqpoint{1.694857in}{0.412962in}}%
\pgfpathlineto{\pgfqpoint{1.691223in}{0.415528in}}%
\pgfpathlineto{\pgfqpoint{1.677883in}{0.426573in}}%
\pgfpathlineto{\pgfqpoint{1.675567in}{0.428634in}}%
\pgfpathlineto{\pgfqpoint{1.663215in}{0.440184in}}%
\pgfpathlineto{\pgfqpoint{1.659910in}{0.443891in}}%
\pgfpathlineto{\pgfqpoint{1.650459in}{0.453795in}}%
\pgfpathlineto{\pgfqpoint{1.644253in}{0.462679in}}%
\pgfpathlineto{\pgfqpoint{1.640135in}{0.467407in}}%
\pgfpathlineto{\pgfqpoint{1.633033in}{0.481018in}}%
\pgfpathlineto{\pgfqpoint{1.631007in}{0.494629in}}%
\pgfpathlineto{\pgfqpoint{1.634046in}{0.508240in}}%
\pgfpathlineto{\pgfqpoint{1.642167in}{0.521851in}}%
\pgfpathlineto{\pgfqpoint{1.644253in}{0.524086in}}%
\pgfpathlineto{\pgfqpoint{1.652798in}{0.535462in}}%
\pgfpathlineto{\pgfqpoint{1.659910in}{0.542613in}}%
\pgfpathlineto{\pgfqpoint{1.665930in}{0.549073in}}%
\pgfpathlineto{\pgfqpoint{1.675567in}{0.557894in}}%
\pgfpathlineto{\pgfqpoint{1.681077in}{0.562684in}}%
\pgfpathlineto{\pgfqpoint{1.691223in}{0.571062in}}%
\pgfpathlineto{\pgfqpoint{1.698654in}{0.576295in}}%
\pgfpathlineto{\pgfqpoint{1.706880in}{0.582478in}}%
\pgfpathlineto{\pgfqpoint{1.719966in}{0.589907in}}%
\pgfpathlineto{\pgfqpoint{1.722536in}{0.591720in}}%
\pgfpathlineto{\pgfqpoint{1.738193in}{0.598780in}}%
\pgfpathlineto{\pgfqpoint{1.753849in}{0.601422in}}%
\pgfpathlineto{\pgfqpoint{1.769506in}{0.599661in}}%
\pgfpathlineto{\pgfqpoint{1.785162in}{0.593487in}}%
\pgfpathlineto{\pgfqpoint{1.790601in}{0.589907in}}%
\pgfpathlineto{\pgfqpoint{1.800819in}{0.584512in}}%
\pgfpathlineto{\pgfqpoint{1.812212in}{0.576295in}}%
\pgfpathlineto{\pgfqpoint{1.816476in}{0.573422in}}%
\pgfpathlineto{\pgfqpoint{1.829762in}{0.562684in}}%
\pgfpathlineto{\pgfqpoint{1.832132in}{0.560671in}}%
\pgfpathlineto{\pgfqpoint{1.844837in}{0.549073in}}%
\pgfpathlineto{\pgfqpoint{1.847789in}{0.545914in}}%
\pgfpathlineto{\pgfqpoint{1.857979in}{0.535462in}}%
\pgfpathlineto{\pgfqpoint{1.863445in}{0.528027in}}%
\pgfpathlineto{\pgfqpoint{1.868909in}{0.521851in}}%
\pgfpathlineto{\pgfqpoint{1.876505in}{0.508240in}}%
\pgfpathlineto{\pgfqpoint{1.879102in}{0.495815in}}%
\pgfpathlineto{\pgfqpoint{1.879484in}{0.494629in}}%
\pgfpathlineto{\pgfqpoint{1.879102in}{0.492855in}}%
\pgfpathlineto{\pgfqpoint{1.877453in}{0.481018in}}%
\pgfpathlineto{\pgfqpoint{1.870809in}{0.467407in}}%
\pgfpathlineto{\pgfqpoint{1.863445in}{0.458485in}}%
\pgfpathlineto{\pgfqpoint{1.860239in}{0.453795in}}%
\pgfpathlineto{\pgfqpoint{1.847789in}{0.440485in}}%
\pgfpathlineto{\pgfqpoint{1.847520in}{0.440184in}}%
\pgfpathlineto{\pgfqpoint{1.832981in}{0.426573in}}%
\pgfpathlineto{\pgfqpoint{1.832132in}{0.425835in}}%
\pgfpathlineto{\pgfqpoint{1.816476in}{0.413196in}}%
\pgfpathlineto{\pgfqpoint{1.816130in}{0.412962in}}%
\pgfpathlineto{\pgfqpoint{1.800819in}{0.402138in}}%
\pgfpathlineto{\pgfqpoint{1.795425in}{0.399351in}}%
\pgfpathlineto{\pgfqpoint{1.785162in}{0.392949in}}%
\pgfpathlineto{\pgfqpoint{1.769506in}{0.387174in}}%
\pgfpathlineto{\pgfqpoint{1.755890in}{0.385740in}}%
\pgfpathlineto{\pgfqpoint{1.753849in}{0.385408in}}%
\pgfpathlineto{\pgfqpoint{1.752484in}{0.385740in}}%
\pgfpathclose%
\pgfpathmoveto{\pgfqpoint{1.095013in}{0.399351in}}%
\pgfpathlineto{\pgfqpoint{1.080617in}{0.408415in}}%
\pgfpathlineto{\pgfqpoint{1.074863in}{0.412962in}}%
\pgfpathlineto{\pgfqpoint{1.064960in}{0.420497in}}%
\pgfpathlineto{\pgfqpoint{1.057851in}{0.426573in}}%
\pgfpathlineto{\pgfqpoint{1.049304in}{0.434441in}}%
\pgfpathlineto{\pgfqpoint{1.043120in}{0.440184in}}%
\pgfpathlineto{\pgfqpoint{1.033647in}{0.450726in}}%
\pgfpathlineto{\pgfqpoint{1.030547in}{0.453795in}}%
\pgfpathlineto{\pgfqpoint{1.020342in}{0.467407in}}%
\pgfpathlineto{\pgfqpoint{1.017991in}{0.472706in}}%
\pgfpathlineto{\pgfqpoint{1.012939in}{0.481018in}}%
\pgfpathlineto{\pgfqpoint{1.010556in}{0.494629in}}%
\pgfpathlineto{\pgfqpoint{1.014131in}{0.508240in}}%
\pgfpathlineto{\pgfqpoint{1.017991in}{0.513834in}}%
\pgfpathlineto{\pgfqpoint{1.022043in}{0.521851in}}%
\pgfpathlineto{\pgfqpoint{1.033094in}{0.535462in}}%
\pgfpathlineto{\pgfqpoint{1.033647in}{0.535987in}}%
\pgfpathlineto{\pgfqpoint{1.045933in}{0.549073in}}%
\pgfpathlineto{\pgfqpoint{1.049304in}{0.552137in}}%
\pgfpathlineto{\pgfqpoint{1.061028in}{0.562684in}}%
\pgfpathlineto{\pgfqpoint{1.064960in}{0.566036in}}%
\pgfpathlineto{\pgfqpoint{1.078482in}{0.576295in}}%
\pgfpathlineto{\pgfqpoint{1.080617in}{0.578016in}}%
\pgfpathlineto{\pgfqpoint{1.096274in}{0.588177in}}%
\pgfpathlineto{\pgfqpoint{1.100157in}{0.589907in}}%
\pgfpathlineto{\pgfqpoint{1.111930in}{0.596488in}}%
\pgfpathlineto{\pgfqpoint{1.127587in}{0.600894in}}%
\pgfpathlineto{\pgfqpoint{1.143243in}{0.600894in}}%
\pgfpathlineto{\pgfqpoint{1.158900in}{0.596488in}}%
\pgfpathlineto{\pgfqpoint{1.170673in}{0.589907in}}%
\pgfpathlineto{\pgfqpoint{1.174556in}{0.588177in}}%
\pgfpathlineto{\pgfqpoint{1.190213in}{0.578016in}}%
\pgfpathlineto{\pgfqpoint{1.192348in}{0.576295in}}%
\pgfpathlineto{\pgfqpoint{1.205870in}{0.566036in}}%
\pgfpathlineto{\pgfqpoint{1.209802in}{0.562684in}}%
\pgfpathlineto{\pgfqpoint{1.221526in}{0.552137in}}%
\pgfpathlineto{\pgfqpoint{1.224897in}{0.549073in}}%
\pgfpathlineto{\pgfqpoint{1.237183in}{0.535987in}}%
\pgfpathlineto{\pgfqpoint{1.237736in}{0.535462in}}%
\pgfpathlineto{\pgfqpoint{1.248787in}{0.521851in}}%
\pgfpathlineto{\pgfqpoint{1.252839in}{0.513834in}}%
\pgfpathlineto{\pgfqpoint{1.256699in}{0.508240in}}%
\pgfpathlineto{\pgfqpoint{1.260274in}{0.494629in}}%
\pgfpathlineto{\pgfqpoint{1.257891in}{0.481018in}}%
\pgfpathlineto{\pgfqpoint{1.252839in}{0.472706in}}%
\pgfpathlineto{\pgfqpoint{1.250488in}{0.467407in}}%
\pgfpathlineto{\pgfqpoint{1.240283in}{0.453795in}}%
\pgfpathlineto{\pgfqpoint{1.237183in}{0.450726in}}%
\pgfpathlineto{\pgfqpoint{1.227710in}{0.440184in}}%
\pgfpathlineto{\pgfqpoint{1.221526in}{0.434441in}}%
\pgfpathlineto{\pgfqpoint{1.212979in}{0.426573in}}%
\pgfpathlineto{\pgfqpoint{1.205870in}{0.420497in}}%
\pgfpathlineto{\pgfqpoint{1.195967in}{0.412962in}}%
\pgfpathlineto{\pgfqpoint{1.190213in}{0.408415in}}%
\pgfpathlineto{\pgfqpoint{1.175817in}{0.399351in}}%
\pgfpathlineto{\pgfqpoint{1.174556in}{0.398402in}}%
\pgfpathlineto{\pgfqpoint{1.158900in}{0.390142in}}%
\pgfpathlineto{\pgfqpoint{1.143243in}{0.386021in}}%
\pgfpathlineto{\pgfqpoint{1.127587in}{0.386021in}}%
\pgfpathlineto{\pgfqpoint{1.111930in}{0.390142in}}%
\pgfpathlineto{\pgfqpoint{1.096274in}{0.398402in}}%
\pgfpathlineto{\pgfqpoint{1.095013in}{0.399351in}}%
\pgfpathclose%
\pgfpathmoveto{\pgfqpoint{0.642804in}{0.549073in}}%
\pgfpathlineto{\pgfqpoint{0.642233in}{0.549333in}}%
\pgfpathlineto{\pgfqpoint{0.626577in}{0.560522in}}%
\pgfpathlineto{\pgfqpoint{0.624089in}{0.562684in}}%
\pgfpathlineto{\pgfqpoint{0.610920in}{0.573555in}}%
\pgfpathlineto{\pgfqpoint{0.607768in}{0.576295in}}%
\pgfpathlineto{\pgfqpoint{0.595263in}{0.587744in}}%
\pgfpathlineto{\pgfqpoint{0.592777in}{0.589907in}}%
\pgfpathlineto{\pgfqpoint{0.579906in}{0.603518in}}%
\pgfpathlineto{\pgfqpoint{0.579607in}{0.604014in}}%
\pgfpathlineto{\pgfqpoint{0.570693in}{0.617129in}}%
\pgfpathlineto{\pgfqpoint{0.568405in}{0.630740in}}%
\pgfpathlineto{\pgfqpoint{0.573649in}{0.644351in}}%
\pgfpathlineto{\pgfqpoint{0.579607in}{0.651267in}}%
\pgfpathlineto{\pgfqpoint{0.584665in}{0.657962in}}%
\pgfpathlineto{\pgfqpoint{0.595263in}{0.668162in}}%
\pgfpathlineto{\pgfqpoint{0.598651in}{0.671573in}}%
\pgfpathlineto{\pgfqpoint{0.610920in}{0.682436in}}%
\pgfpathlineto{\pgfqpoint{0.614193in}{0.685184in}}%
\pgfpathlineto{\pgfqpoint{0.626577in}{0.695470in}}%
\pgfpathlineto{\pgfqpoint{0.631425in}{0.698795in}}%
\pgfpathlineto{\pgfqpoint{0.642233in}{0.706825in}}%
\pgfpathlineto{\pgfqpoint{0.653949in}{0.712407in}}%
\pgfpathlineto{\pgfqpoint{0.657890in}{0.714627in}}%
\pgfpathlineto{\pgfqpoint{0.673546in}{0.716755in}}%
\pgfpathlineto{\pgfqpoint{0.687476in}{0.712407in}}%
\pgfpathlineto{\pgfqpoint{0.689203in}{0.711943in}}%
\pgfpathlineto{\pgfqpoint{0.704859in}{0.702597in}}%
\pgfpathlineto{\pgfqpoint{0.709464in}{0.698795in}}%
\pgfpathlineto{\pgfqpoint{0.720516in}{0.690422in}}%
\pgfpathlineto{\pgfqpoint{0.726592in}{0.685184in}}%
\pgfpathlineto{\pgfqpoint{0.736173in}{0.676855in}}%
\pgfpathlineto{\pgfqpoint{0.742198in}{0.671573in}}%
\pgfpathlineto{\pgfqpoint{0.751829in}{0.661966in}}%
\pgfpathlineto{\pgfqpoint{0.756202in}{0.657962in}}%
\pgfpathlineto{\pgfqpoint{0.766953in}{0.644351in}}%
\pgfpathlineto{\pgfqpoint{0.767486in}{0.642849in}}%
\pgfpathlineto{\pgfqpoint{0.772488in}{0.630740in}}%
\pgfpathlineto{\pgfqpoint{0.770040in}{0.617129in}}%
\pgfpathlineto{\pgfqpoint{0.767486in}{0.613702in}}%
\pgfpathlineto{\pgfqpoint{0.761066in}{0.603518in}}%
\pgfpathlineto{\pgfqpoint{0.751829in}{0.594121in}}%
\pgfpathlineto{\pgfqpoint{0.748004in}{0.589907in}}%
\pgfpathlineto{\pgfqpoint{0.736173in}{0.579141in}}%
\pgfpathlineto{\pgfqpoint{0.733011in}{0.576295in}}%
\pgfpathlineto{\pgfqpoint{0.720516in}{0.565630in}}%
\pgfpathlineto{\pgfqpoint{0.716592in}{0.562684in}}%
\pgfpathlineto{\pgfqpoint{0.704859in}{0.553470in}}%
\pgfpathlineto{\pgfqpoint{0.697158in}{0.549073in}}%
\pgfpathlineto{\pgfqpoint{0.689203in}{0.543894in}}%
\pgfpathlineto{\pgfqpoint{0.673546in}{0.539335in}}%
\pgfpathlineto{\pgfqpoint{0.657890in}{0.541324in}}%
\pgfpathlineto{\pgfqpoint{0.642804in}{0.549073in}}%
\pgfpathclose%
\pgfpathmoveto{\pgfqpoint{0.953123in}{0.549073in}}%
\pgfpathlineto{\pgfqpoint{0.939708in}{0.558083in}}%
\pgfpathlineto{\pgfqpoint{0.934268in}{0.562684in}}%
\pgfpathlineto{\pgfqpoint{0.924051in}{0.570906in}}%
\pgfpathlineto{\pgfqpoint{0.917843in}{0.576295in}}%
\pgfpathlineto{\pgfqpoint{0.908395in}{0.584934in}}%
\pgfpathlineto{\pgfqpoint{0.902784in}{0.589907in}}%
\pgfpathlineto{\pgfqpoint{0.892738in}{0.600707in}}%
\pgfpathlineto{\pgfqpoint{0.889787in}{0.603518in}}%
\pgfpathlineto{\pgfqpoint{0.880787in}{0.617129in}}%
\pgfpathlineto{\pgfqpoint{0.878559in}{0.630740in}}%
\pgfpathlineto{\pgfqpoint{0.883667in}{0.644351in}}%
\pgfpathlineto{\pgfqpoint{0.892738in}{0.655304in}}%
\pgfpathlineto{\pgfqpoint{0.894711in}{0.657962in}}%
\pgfpathlineto{\pgfqpoint{0.908395in}{0.671385in}}%
\pgfpathlineto{\pgfqpoint{0.908582in}{0.671573in}}%
\pgfpathlineto{\pgfqpoint{0.923982in}{0.685184in}}%
\pgfpathlineto{\pgfqpoint{0.924051in}{0.685245in}}%
\pgfpathlineto{\pgfqpoint{0.939708in}{0.697904in}}%
\pgfpathlineto{\pgfqpoint{0.941092in}{0.698795in}}%
\pgfpathlineto{\pgfqpoint{0.955364in}{0.708712in}}%
\pgfpathlineto{\pgfqpoint{0.964376in}{0.712407in}}%
\pgfpathlineto{\pgfqpoint{0.971021in}{0.715613in}}%
\pgfpathlineto{\pgfqpoint{0.986678in}{0.716325in}}%
\pgfpathlineto{\pgfqpoint{0.996509in}{0.712407in}}%
\pgfpathlineto{\pgfqpoint{1.002334in}{0.710423in}}%
\pgfpathlineto{\pgfqpoint{1.017991in}{0.700291in}}%
\pgfpathlineto{\pgfqpoint{1.019731in}{0.698795in}}%
\pgfpathlineto{\pgfqpoint{1.033647in}{0.687842in}}%
\pgfpathlineto{\pgfqpoint{1.036697in}{0.685184in}}%
\pgfpathlineto{\pgfqpoint{1.049304in}{0.674114in}}%
\pgfpathlineto{\pgfqpoint{1.052234in}{0.671573in}}%
\pgfpathlineto{\pgfqpoint{1.064960in}{0.659009in}}%
\pgfpathlineto{\pgfqpoint{1.066140in}{0.657962in}}%
\pgfpathlineto{\pgfqpoint{1.077078in}{0.644351in}}%
\pgfpathlineto{\pgfqpoint{1.080617in}{0.634659in}}%
\pgfpathlineto{\pgfqpoint{1.082326in}{0.630740in}}%
\pgfpathlineto{\pgfqpoint{1.080617in}{0.621654in}}%
\pgfpathlineto{\pgfqpoint{1.079896in}{0.617129in}}%
\pgfpathlineto{\pgfqpoint{1.071089in}{0.603518in}}%
\pgfpathlineto{\pgfqpoint{1.064960in}{0.597473in}}%
\pgfpathlineto{\pgfqpoint{1.058021in}{0.589907in}}%
\pgfpathlineto{\pgfqpoint{1.049304in}{0.582062in}}%
\pgfpathlineto{\pgfqpoint{1.042961in}{0.576295in}}%
\pgfpathlineto{\pgfqpoint{1.033647in}{0.568259in}}%
\pgfpathlineto{\pgfqpoint{1.026499in}{0.562684in}}%
\pgfpathlineto{\pgfqpoint{1.017991in}{0.555726in}}%
\pgfpathlineto{\pgfqpoint{1.007317in}{0.549073in}}%
\pgfpathlineto{\pgfqpoint{1.002334in}{0.545516in}}%
\pgfpathlineto{\pgfqpoint{0.986678in}{0.539737in}}%
\pgfpathlineto{\pgfqpoint{0.971021in}{0.540402in}}%
\pgfpathlineto{\pgfqpoint{0.955364in}{0.547343in}}%
\pgfpathlineto{\pgfqpoint{0.953123in}{0.549073in}}%
\pgfpathclose%
\pgfpathmoveto{\pgfqpoint{1.263513in}{0.549073in}}%
\pgfpathlineto{\pgfqpoint{1.252839in}{0.555726in}}%
\pgfpathlineto{\pgfqpoint{1.244331in}{0.562684in}}%
\pgfpathlineto{\pgfqpoint{1.237183in}{0.568259in}}%
\pgfpathlineto{\pgfqpoint{1.227869in}{0.576295in}}%
\pgfpathlineto{\pgfqpoint{1.221526in}{0.582062in}}%
\pgfpathlineto{\pgfqpoint{1.212809in}{0.589907in}}%
\pgfpathlineto{\pgfqpoint{1.205870in}{0.597473in}}%
\pgfpathlineto{\pgfqpoint{1.199741in}{0.603518in}}%
\pgfpathlineto{\pgfqpoint{1.190934in}{0.617129in}}%
\pgfpathlineto{\pgfqpoint{1.190213in}{0.621654in}}%
\pgfpathlineto{\pgfqpoint{1.188504in}{0.630740in}}%
\pgfpathlineto{\pgfqpoint{1.190213in}{0.634659in}}%
\pgfpathlineto{\pgfqpoint{1.193752in}{0.644351in}}%
\pgfpathlineto{\pgfqpoint{1.204690in}{0.657962in}}%
\pgfpathlineto{\pgfqpoint{1.205870in}{0.659009in}}%
\pgfpathlineto{\pgfqpoint{1.218596in}{0.671573in}}%
\pgfpathlineto{\pgfqpoint{1.221526in}{0.674114in}}%
\pgfpathlineto{\pgfqpoint{1.234133in}{0.685184in}}%
\pgfpathlineto{\pgfqpoint{1.237183in}{0.687842in}}%
\pgfpathlineto{\pgfqpoint{1.251099in}{0.698795in}}%
\pgfpathlineto{\pgfqpoint{1.252839in}{0.700291in}}%
\pgfpathlineto{\pgfqpoint{1.268496in}{0.710423in}}%
\pgfpathlineto{\pgfqpoint{1.274321in}{0.712407in}}%
\pgfpathlineto{\pgfqpoint{1.284152in}{0.716325in}}%
\pgfpathlineto{\pgfqpoint{1.299809in}{0.715613in}}%
\pgfpathlineto{\pgfqpoint{1.306454in}{0.712407in}}%
\pgfpathlineto{\pgfqpoint{1.315466in}{0.708712in}}%
\pgfpathlineto{\pgfqpoint{1.329738in}{0.698795in}}%
\pgfpathlineto{\pgfqpoint{1.331122in}{0.697904in}}%
\pgfpathlineto{\pgfqpoint{1.346779in}{0.685245in}}%
\pgfpathlineto{\pgfqpoint{1.346848in}{0.685184in}}%
\pgfpathlineto{\pgfqpoint{1.362248in}{0.671573in}}%
\pgfpathlineto{\pgfqpoint{1.362435in}{0.671385in}}%
\pgfpathlineto{\pgfqpoint{1.376119in}{0.657962in}}%
\pgfpathlineto{\pgfqpoint{1.378092in}{0.655304in}}%
\pgfpathlineto{\pgfqpoint{1.387163in}{0.644351in}}%
\pgfpathlineto{\pgfqpoint{1.392271in}{0.630740in}}%
\pgfpathlineto{\pgfqpoint{1.390043in}{0.617129in}}%
\pgfpathlineto{\pgfqpoint{1.381043in}{0.603518in}}%
\pgfpathlineto{\pgfqpoint{1.378092in}{0.600707in}}%
\pgfpathlineto{\pgfqpoint{1.368046in}{0.589907in}}%
\pgfpathlineto{\pgfqpoint{1.362435in}{0.584934in}}%
\pgfpathlineto{\pgfqpoint{1.352987in}{0.576295in}}%
\pgfpathlineto{\pgfqpoint{1.346779in}{0.570906in}}%
\pgfpathlineto{\pgfqpoint{1.336562in}{0.562684in}}%
\pgfpathlineto{\pgfqpoint{1.331122in}{0.558083in}}%
\pgfpathlineto{\pgfqpoint{1.317707in}{0.549073in}}%
\pgfpathlineto{\pgfqpoint{1.315466in}{0.547343in}}%
\pgfpathlineto{\pgfqpoint{1.299809in}{0.540402in}}%
\pgfpathlineto{\pgfqpoint{1.284152in}{0.539737in}}%
\pgfpathlineto{\pgfqpoint{1.268496in}{0.545516in}}%
\pgfpathlineto{\pgfqpoint{1.263513in}{0.549073in}}%
\pgfpathclose%
\pgfpathmoveto{\pgfqpoint{1.573672in}{0.549073in}}%
\pgfpathlineto{\pgfqpoint{1.565971in}{0.553470in}}%
\pgfpathlineto{\pgfqpoint{1.554238in}{0.562684in}}%
\pgfpathlineto{\pgfqpoint{1.550314in}{0.565630in}}%
\pgfpathlineto{\pgfqpoint{1.537819in}{0.576295in}}%
\pgfpathlineto{\pgfqpoint{1.534657in}{0.579141in}}%
\pgfpathlineto{\pgfqpoint{1.522826in}{0.589907in}}%
\pgfpathlineto{\pgfqpoint{1.519001in}{0.594121in}}%
\pgfpathlineto{\pgfqpoint{1.509764in}{0.603518in}}%
\pgfpathlineto{\pgfqpoint{1.503344in}{0.613702in}}%
\pgfpathlineto{\pgfqpoint{1.500790in}{0.617129in}}%
\pgfpathlineto{\pgfqpoint{1.498342in}{0.630740in}}%
\pgfpathlineto{\pgfqpoint{1.503344in}{0.642849in}}%
\pgfpathlineto{\pgfqpoint{1.503877in}{0.644351in}}%
\pgfpathlineto{\pgfqpoint{1.514628in}{0.657962in}}%
\pgfpathlineto{\pgfqpoint{1.519001in}{0.661966in}}%
\pgfpathlineto{\pgfqpoint{1.528632in}{0.671573in}}%
\pgfpathlineto{\pgfqpoint{1.534657in}{0.676855in}}%
\pgfpathlineto{\pgfqpoint{1.544238in}{0.685184in}}%
\pgfpathlineto{\pgfqpoint{1.550314in}{0.690422in}}%
\pgfpathlineto{\pgfqpoint{1.561366in}{0.698795in}}%
\pgfpathlineto{\pgfqpoint{1.565971in}{0.702597in}}%
\pgfpathlineto{\pgfqpoint{1.581627in}{0.711943in}}%
\pgfpathlineto{\pgfqpoint{1.583354in}{0.712407in}}%
\pgfpathlineto{\pgfqpoint{1.597284in}{0.716755in}}%
\pgfpathlineto{\pgfqpoint{1.612940in}{0.714627in}}%
\pgfpathlineto{\pgfqpoint{1.616881in}{0.712407in}}%
\pgfpathlineto{\pgfqpoint{1.628597in}{0.706825in}}%
\pgfpathlineto{\pgfqpoint{1.639405in}{0.698795in}}%
\pgfpathlineto{\pgfqpoint{1.644253in}{0.695470in}}%
\pgfpathlineto{\pgfqpoint{1.656637in}{0.685184in}}%
\pgfpathlineto{\pgfqpoint{1.659910in}{0.682436in}}%
\pgfpathlineto{\pgfqpoint{1.672179in}{0.671573in}}%
\pgfpathlineto{\pgfqpoint{1.675567in}{0.668162in}}%
\pgfpathlineto{\pgfqpoint{1.686165in}{0.657962in}}%
\pgfpathlineto{\pgfqpoint{1.691223in}{0.651267in}}%
\pgfpathlineto{\pgfqpoint{1.697181in}{0.644351in}}%
\pgfpathlineto{\pgfqpoint{1.702425in}{0.630740in}}%
\pgfpathlineto{\pgfqpoint{1.700137in}{0.617129in}}%
\pgfpathlineto{\pgfqpoint{1.691223in}{0.604014in}}%
\pgfpathlineto{\pgfqpoint{1.690924in}{0.603518in}}%
\pgfpathlineto{\pgfqpoint{1.678053in}{0.589907in}}%
\pgfpathlineto{\pgfqpoint{1.675567in}{0.587744in}}%
\pgfpathlineto{\pgfqpoint{1.663062in}{0.576295in}}%
\pgfpathlineto{\pgfqpoint{1.659910in}{0.573555in}}%
\pgfpathlineto{\pgfqpoint{1.646741in}{0.562684in}}%
\pgfpathlineto{\pgfqpoint{1.644253in}{0.560522in}}%
\pgfpathlineto{\pgfqpoint{1.628597in}{0.549333in}}%
\pgfpathlineto{\pgfqpoint{1.628026in}{0.549073in}}%
\pgfpathlineto{\pgfqpoint{1.612940in}{0.541324in}}%
\pgfpathlineto{\pgfqpoint{1.597284in}{0.539335in}}%
\pgfpathlineto{\pgfqpoint{1.581627in}{0.543894in}}%
\pgfpathlineto{\pgfqpoint{1.573672in}{0.549073in}}%
\pgfpathclose%
\pgfpathmoveto{\pgfqpoint{0.498733in}{0.657962in}}%
\pgfpathlineto{\pgfqpoint{0.485668in}{0.662525in}}%
\pgfpathlineto{\pgfqpoint{0.470487in}{0.671573in}}%
\pgfpathlineto{\pgfqpoint{0.470011in}{0.671815in}}%
\pgfpathlineto{\pgfqpoint{0.454354in}{0.682578in}}%
\pgfpathlineto{\pgfqpoint{0.451183in}{0.685184in}}%
\pgfpathlineto{\pgfqpoint{0.438698in}{0.695322in}}%
\pgfpathlineto{\pgfqpoint{0.434728in}{0.698795in}}%
\pgfpathlineto{\pgfqpoint{0.423041in}{0.710051in}}%
\pgfpathlineto{\pgfqpoint{0.420505in}{0.712407in}}%
\pgfpathlineto{\pgfqpoint{0.408479in}{0.726018in}}%
\pgfpathlineto{\pgfqpoint{0.407385in}{0.727753in}}%
\pgfpathlineto{\pgfqpoint{0.398311in}{0.739629in}}%
\pgfpathlineto{\pgfqpoint{0.392619in}{0.753240in}}%
\pgfpathlineto{\pgfqpoint{0.391728in}{0.766044in}}%
\pgfpathlineto{\pgfqpoint{0.391641in}{0.766851in}}%
\pgfpathlineto{\pgfqpoint{0.391728in}{0.767055in}}%
\pgfpathlineto{\pgfqpoint{0.395464in}{0.780462in}}%
\pgfpathlineto{\pgfqpoint{0.404012in}{0.794073in}}%
\pgfpathlineto{\pgfqpoint{0.407385in}{0.797654in}}%
\pgfpathlineto{\pgfqpoint{0.415258in}{0.807684in}}%
\pgfpathlineto{\pgfqpoint{0.423041in}{0.815377in}}%
\pgfpathlineto{\pgfqpoint{0.428794in}{0.821295in}}%
\pgfpathlineto{\pgfqpoint{0.438698in}{0.830177in}}%
\pgfpathlineto{\pgfqpoint{0.444372in}{0.834907in}}%
\pgfpathlineto{\pgfqpoint{0.454354in}{0.842985in}}%
\pgfpathlineto{\pgfqpoint{0.462551in}{0.848518in}}%
\pgfpathlineto{\pgfqpoint{0.470011in}{0.854036in}}%
\pgfpathlineto{\pgfqpoint{0.484928in}{0.862129in}}%
\pgfpathlineto{\pgfqpoint{0.485668in}{0.862647in}}%
\pgfpathlineto{\pgfqpoint{0.501324in}{0.869309in}}%
\pgfpathlineto{\pgfqpoint{0.516981in}{0.871209in}}%
\pgfpathlineto{\pgfqpoint{0.532637in}{0.868359in}}%
\pgfpathlineto{\pgfqpoint{0.545510in}{0.862129in}}%
\pgfpathlineto{\pgfqpoint{0.548294in}{0.861100in}}%
\pgfpathlineto{\pgfqpoint{0.563950in}{0.851920in}}%
\pgfpathlineto{\pgfqpoint{0.568363in}{0.848518in}}%
\pgfpathlineto{\pgfqpoint{0.579607in}{0.840588in}}%
\pgfpathlineto{\pgfqpoint{0.586478in}{0.834907in}}%
\pgfpathlineto{\pgfqpoint{0.595263in}{0.827413in}}%
\pgfpathlineto{\pgfqpoint{0.602067in}{0.821295in}}%
\pgfpathlineto{\pgfqpoint{0.610920in}{0.812161in}}%
\pgfpathlineto{\pgfqpoint{0.615541in}{0.807684in}}%
\pgfpathlineto{\pgfqpoint{0.626462in}{0.794073in}}%
\pgfpathlineto{\pgfqpoint{0.626577in}{0.793858in}}%
\pgfpathlineto{\pgfqpoint{0.635567in}{0.780462in}}%
\pgfpathlineto{\pgfqpoint{0.639620in}{0.766851in}}%
\pgfpathlineto{\pgfqpoint{0.638608in}{0.753240in}}%
\pgfpathlineto{\pgfqpoint{0.632523in}{0.739629in}}%
\pgfpathlineto{\pgfqpoint{0.626577in}{0.732254in}}%
\pgfpathlineto{\pgfqpoint{0.622557in}{0.726018in}}%
\pgfpathlineto{\pgfqpoint{0.610920in}{0.713248in}}%
\pgfpathlineto{\pgfqpoint{0.610205in}{0.712407in}}%
\pgfpathlineto{\pgfqpoint{0.596064in}{0.698795in}}%
\pgfpathlineto{\pgfqpoint{0.595263in}{0.698093in}}%
\pgfpathlineto{\pgfqpoint{0.579726in}{0.685184in}}%
\pgfpathlineto{\pgfqpoint{0.579607in}{0.685084in}}%
\pgfpathlineto{\pgfqpoint{0.563950in}{0.673724in}}%
\pgfpathlineto{\pgfqpoint{0.560018in}{0.671573in}}%
\pgfpathlineto{\pgfqpoint{0.548294in}{0.664083in}}%
\pgfpathlineto{\pgfqpoint{0.532909in}{0.657962in}}%
\pgfpathlineto{\pgfqpoint{0.532637in}{0.657807in}}%
\pgfpathlineto{\pgfqpoint{0.516981in}{0.654376in}}%
\pgfpathlineto{\pgfqpoint{0.501324in}{0.656663in}}%
\pgfpathlineto{\pgfqpoint{0.498733in}{0.657962in}}%
\pgfpathclose%
\pgfpathmoveto{\pgfqpoint{0.809335in}{0.657962in}}%
\pgfpathlineto{\pgfqpoint{0.798799in}{0.661124in}}%
\pgfpathlineto{\pgfqpoint{0.783142in}{0.669690in}}%
\pgfpathlineto{\pgfqpoint{0.780656in}{0.671573in}}%
\pgfpathlineto{\pgfqpoint{0.767486in}{0.680185in}}%
\pgfpathlineto{\pgfqpoint{0.761249in}{0.685184in}}%
\pgfpathlineto{\pgfqpoint{0.751829in}{0.692630in}}%
\pgfpathlineto{\pgfqpoint{0.744737in}{0.698795in}}%
\pgfpathlineto{\pgfqpoint{0.736173in}{0.706985in}}%
\pgfpathlineto{\pgfqpoint{0.730422in}{0.712407in}}%
\pgfpathlineto{\pgfqpoint{0.720516in}{0.723856in}}%
\pgfpathlineto{\pgfqpoint{0.718350in}{0.726018in}}%
\pgfpathlineto{\pgfqpoint{0.708496in}{0.739629in}}%
\pgfpathlineto{\pgfqpoint{0.704859in}{0.748788in}}%
\pgfpathlineto{\pgfqpoint{0.702312in}{0.753240in}}%
\pgfpathlineto{\pgfqpoint{0.700997in}{0.766851in}}%
\pgfpathlineto{\pgfqpoint{0.704859in}{0.776875in}}%
\pgfpathlineto{\pgfqpoint{0.705812in}{0.780462in}}%
\pgfpathlineto{\pgfqpoint{0.713871in}{0.794073in}}%
\pgfpathlineto{\pgfqpoint{0.720516in}{0.801457in}}%
\pgfpathlineto{\pgfqpoint{0.725327in}{0.807684in}}%
\pgfpathlineto{\pgfqpoint{0.736173in}{0.818564in}}%
\pgfpathlineto{\pgfqpoint{0.738846in}{0.821295in}}%
\pgfpathlineto{\pgfqpoint{0.751829in}{0.832862in}}%
\pgfpathlineto{\pgfqpoint{0.754349in}{0.834907in}}%
\pgfpathlineto{\pgfqpoint{0.767486in}{0.845273in}}%
\pgfpathlineto{\pgfqpoint{0.772533in}{0.848518in}}%
\pgfpathlineto{\pgfqpoint{0.783142in}{0.856013in}}%
\pgfpathlineto{\pgfqpoint{0.795360in}{0.862129in}}%
\pgfpathlineto{\pgfqpoint{0.798799in}{0.864362in}}%
\pgfpathlineto{\pgfqpoint{0.814455in}{0.870069in}}%
\pgfpathlineto{\pgfqpoint{0.830112in}{0.871019in}}%
\pgfpathlineto{\pgfqpoint{0.845769in}{0.867217in}}%
\pgfpathlineto{\pgfqpoint{0.855184in}{0.862129in}}%
\pgfpathlineto{\pgfqpoint{0.861425in}{0.859546in}}%
\pgfpathlineto{\pgfqpoint{0.877082in}{0.849667in}}%
\pgfpathlineto{\pgfqpoint{0.878518in}{0.848518in}}%
\pgfpathlineto{\pgfqpoint{0.892738in}{0.838085in}}%
\pgfpathlineto{\pgfqpoint{0.896515in}{0.834907in}}%
\pgfpathlineto{\pgfqpoint{0.908395in}{0.824579in}}%
\pgfpathlineto{\pgfqpoint{0.912051in}{0.821295in}}%
\pgfpathlineto{\pgfqpoint{0.924051in}{0.808933in}}%
\pgfpathlineto{\pgfqpoint{0.925373in}{0.807684in}}%
\pgfpathlineto{\pgfqpoint{0.936737in}{0.794073in}}%
\pgfpathlineto{\pgfqpoint{0.939708in}{0.788648in}}%
\pgfpathlineto{\pgfqpoint{0.945561in}{0.780462in}}%
\pgfpathlineto{\pgfqpoint{0.949934in}{0.766851in}}%
\pgfpathlineto{\pgfqpoint{0.948842in}{0.753240in}}%
\pgfpathlineto{\pgfqpoint{0.942277in}{0.739629in}}%
\pgfpathlineto{\pgfqpoint{0.939708in}{0.736639in}}%
\pgfpathlineto{\pgfqpoint{0.932673in}{0.726018in}}%
\pgfpathlineto{\pgfqpoint{0.924051in}{0.716794in}}%
\pgfpathlineto{\pgfqpoint{0.920319in}{0.712407in}}%
\pgfpathlineto{\pgfqpoint{0.908395in}{0.700986in}}%
\pgfpathlineto{\pgfqpoint{0.906043in}{0.698795in}}%
\pgfpathlineto{\pgfqpoint{0.892738in}{0.687509in}}%
\pgfpathlineto{\pgfqpoint{0.889596in}{0.685184in}}%
\pgfpathlineto{\pgfqpoint{0.877082in}{0.675756in}}%
\pgfpathlineto{\pgfqpoint{0.869919in}{0.671573in}}%
\pgfpathlineto{\pgfqpoint{0.861425in}{0.665796in}}%
\pgfpathlineto{\pgfqpoint{0.845769in}{0.658790in}}%
\pgfpathlineto{\pgfqpoint{0.841643in}{0.657962in}}%
\pgfpathlineto{\pgfqpoint{0.830112in}{0.654604in}}%
\pgfpathlineto{\pgfqpoint{0.814455in}{0.655748in}}%
\pgfpathlineto{\pgfqpoint{0.809335in}{0.657962in}}%
\pgfpathclose%
\pgfpathmoveto{\pgfqpoint{1.119581in}{0.657962in}}%
\pgfpathlineto{\pgfqpoint{1.111930in}{0.659879in}}%
\pgfpathlineto{\pgfqpoint{1.096274in}{0.667666in}}%
\pgfpathlineto{\pgfqpoint{1.090844in}{0.671573in}}%
\pgfpathlineto{\pgfqpoint{1.080617in}{0.677910in}}%
\pgfpathlineto{\pgfqpoint{1.071275in}{0.685184in}}%
\pgfpathlineto{\pgfqpoint{1.064960in}{0.690023in}}%
\pgfpathlineto{\pgfqpoint{1.054765in}{0.698795in}}%
\pgfpathlineto{\pgfqpoint{1.049304in}{0.703959in}}%
\pgfpathlineto{\pgfqpoint{1.040435in}{0.712407in}}%
\pgfpathlineto{\pgfqpoint{1.033647in}{0.720337in}}%
\pgfpathlineto{\pgfqpoint{1.028167in}{0.726018in}}%
\pgfpathlineto{\pgfqpoint{1.018812in}{0.739629in}}%
\pgfpathlineto{\pgfqpoint{1.017991in}{0.741782in}}%
\pgfpathlineto{\pgfqpoint{1.011986in}{0.753240in}}%
\pgfpathlineto{\pgfqpoint{1.010794in}{0.766851in}}%
\pgfpathlineto{\pgfqpoint{1.015563in}{0.780462in}}%
\pgfpathlineto{\pgfqpoint{1.017991in}{0.783614in}}%
\pgfpathlineto{\pgfqpoint{1.023915in}{0.794073in}}%
\pgfpathlineto{\pgfqpoint{1.033647in}{0.805308in}}%
\pgfpathlineto{\pgfqpoint{1.035463in}{0.807684in}}%
\pgfpathlineto{\pgfqpoint{1.048870in}{0.821295in}}%
\pgfpathlineto{\pgfqpoint{1.049304in}{0.821683in}}%
\pgfpathlineto{\pgfqpoint{1.064286in}{0.834907in}}%
\pgfpathlineto{\pgfqpoint{1.064960in}{0.835482in}}%
\pgfpathlineto{\pgfqpoint{1.080617in}{0.847448in}}%
\pgfpathlineto{\pgfqpoint{1.082372in}{0.848518in}}%
\pgfpathlineto{\pgfqpoint{1.096274in}{0.857850in}}%
\pgfpathlineto{\pgfqpoint{1.105623in}{0.862129in}}%
\pgfpathlineto{\pgfqpoint{1.111930in}{0.865885in}}%
\pgfpathlineto{\pgfqpoint{1.127587in}{0.870639in}}%
\pgfpathlineto{\pgfqpoint{1.143243in}{0.870639in}}%
\pgfpathlineto{\pgfqpoint{1.158900in}{0.865885in}}%
\pgfpathlineto{\pgfqpoint{1.165207in}{0.862129in}}%
\pgfpathlineto{\pgfqpoint{1.174556in}{0.857850in}}%
\pgfpathlineto{\pgfqpoint{1.188458in}{0.848518in}}%
\pgfpathlineto{\pgfqpoint{1.190213in}{0.847448in}}%
\pgfpathlineto{\pgfqpoint{1.205870in}{0.835482in}}%
\pgfpathlineto{\pgfqpoint{1.206544in}{0.834907in}}%
\pgfpathlineto{\pgfqpoint{1.221526in}{0.821683in}}%
\pgfpathlineto{\pgfqpoint{1.221960in}{0.821295in}}%
\pgfpathlineto{\pgfqpoint{1.235367in}{0.807684in}}%
\pgfpathlineto{\pgfqpoint{1.237183in}{0.805308in}}%
\pgfpathlineto{\pgfqpoint{1.246915in}{0.794073in}}%
\pgfpathlineto{\pgfqpoint{1.252839in}{0.783614in}}%
\pgfpathlineto{\pgfqpoint{1.255267in}{0.780462in}}%
\pgfpathlineto{\pgfqpoint{1.260036in}{0.766851in}}%
\pgfpathlineto{\pgfqpoint{1.258844in}{0.753240in}}%
\pgfpathlineto{\pgfqpoint{1.252839in}{0.741782in}}%
\pgfpathlineto{\pgfqpoint{1.252018in}{0.739629in}}%
\pgfpathlineto{\pgfqpoint{1.242663in}{0.726018in}}%
\pgfpathlineto{\pgfqpoint{1.237183in}{0.720337in}}%
\pgfpathlineto{\pgfqpoint{1.230395in}{0.712407in}}%
\pgfpathlineto{\pgfqpoint{1.221526in}{0.703959in}}%
\pgfpathlineto{\pgfqpoint{1.216065in}{0.698795in}}%
\pgfpathlineto{\pgfqpoint{1.205870in}{0.690023in}}%
\pgfpathlineto{\pgfqpoint{1.199555in}{0.685184in}}%
\pgfpathlineto{\pgfqpoint{1.190213in}{0.677910in}}%
\pgfpathlineto{\pgfqpoint{1.179986in}{0.671573in}}%
\pgfpathlineto{\pgfqpoint{1.174556in}{0.667666in}}%
\pgfpathlineto{\pgfqpoint{1.158900in}{0.659879in}}%
\pgfpathlineto{\pgfqpoint{1.151249in}{0.657962in}}%
\pgfpathlineto{\pgfqpoint{1.143243in}{0.655062in}}%
\pgfpathlineto{\pgfqpoint{1.127587in}{0.655062in}}%
\pgfpathlineto{\pgfqpoint{1.119581in}{0.657962in}}%
\pgfpathclose%
\pgfpathmoveto{\pgfqpoint{1.429187in}{0.657962in}}%
\pgfpathlineto{\pgfqpoint{1.425061in}{0.658790in}}%
\pgfpathlineto{\pgfqpoint{1.409405in}{0.665796in}}%
\pgfpathlineto{\pgfqpoint{1.400911in}{0.671573in}}%
\pgfpathlineto{\pgfqpoint{1.393748in}{0.675756in}}%
\pgfpathlineto{\pgfqpoint{1.381234in}{0.685184in}}%
\pgfpathlineto{\pgfqpoint{1.378092in}{0.687509in}}%
\pgfpathlineto{\pgfqpoint{1.364787in}{0.698795in}}%
\pgfpathlineto{\pgfqpoint{1.362435in}{0.700986in}}%
\pgfpathlineto{\pgfqpoint{1.350511in}{0.712407in}}%
\pgfpathlineto{\pgfqpoint{1.346779in}{0.716794in}}%
\pgfpathlineto{\pgfqpoint{1.338157in}{0.726018in}}%
\pgfpathlineto{\pgfqpoint{1.331122in}{0.736639in}}%
\pgfpathlineto{\pgfqpoint{1.328553in}{0.739629in}}%
\pgfpathlineto{\pgfqpoint{1.321988in}{0.753240in}}%
\pgfpathlineto{\pgfqpoint{1.320896in}{0.766851in}}%
\pgfpathlineto{\pgfqpoint{1.325269in}{0.780462in}}%
\pgfpathlineto{\pgfqpoint{1.331122in}{0.788648in}}%
\pgfpathlineto{\pgfqpoint{1.334093in}{0.794073in}}%
\pgfpathlineto{\pgfqpoint{1.345457in}{0.807684in}}%
\pgfpathlineto{\pgfqpoint{1.346779in}{0.808933in}}%
\pgfpathlineto{\pgfqpoint{1.358779in}{0.821295in}}%
\pgfpathlineto{\pgfqpoint{1.362435in}{0.824579in}}%
\pgfpathlineto{\pgfqpoint{1.374315in}{0.834907in}}%
\pgfpathlineto{\pgfqpoint{1.378092in}{0.838085in}}%
\pgfpathlineto{\pgfqpoint{1.392312in}{0.848518in}}%
\pgfpathlineto{\pgfqpoint{1.393748in}{0.849667in}}%
\pgfpathlineto{\pgfqpoint{1.409405in}{0.859546in}}%
\pgfpathlineto{\pgfqpoint{1.415646in}{0.862129in}}%
\pgfpathlineto{\pgfqpoint{1.425061in}{0.867217in}}%
\pgfpathlineto{\pgfqpoint{1.440718in}{0.871019in}}%
\pgfpathlineto{\pgfqpoint{1.456375in}{0.870069in}}%
\pgfpathlineto{\pgfqpoint{1.472031in}{0.864362in}}%
\pgfpathlineto{\pgfqpoint{1.475470in}{0.862129in}}%
\pgfpathlineto{\pgfqpoint{1.487688in}{0.856013in}}%
\pgfpathlineto{\pgfqpoint{1.498297in}{0.848518in}}%
\pgfpathlineto{\pgfqpoint{1.503344in}{0.845273in}}%
\pgfpathlineto{\pgfqpoint{1.516481in}{0.834907in}}%
\pgfpathlineto{\pgfqpoint{1.519001in}{0.832862in}}%
\pgfpathlineto{\pgfqpoint{1.531984in}{0.821295in}}%
\pgfpathlineto{\pgfqpoint{1.534657in}{0.818564in}}%
\pgfpathlineto{\pgfqpoint{1.545503in}{0.807684in}}%
\pgfpathlineto{\pgfqpoint{1.550314in}{0.801457in}}%
\pgfpathlineto{\pgfqpoint{1.556959in}{0.794073in}}%
\pgfpathlineto{\pgfqpoint{1.565018in}{0.780462in}}%
\pgfpathlineto{\pgfqpoint{1.565971in}{0.776875in}}%
\pgfpathlineto{\pgfqpoint{1.569833in}{0.766851in}}%
\pgfpathlineto{\pgfqpoint{1.568518in}{0.753240in}}%
\pgfpathlineto{\pgfqpoint{1.565971in}{0.748788in}}%
\pgfpathlineto{\pgfqpoint{1.562334in}{0.739629in}}%
\pgfpathlineto{\pgfqpoint{1.552480in}{0.726018in}}%
\pgfpathlineto{\pgfqpoint{1.550314in}{0.723856in}}%
\pgfpathlineto{\pgfqpoint{1.540408in}{0.712407in}}%
\pgfpathlineto{\pgfqpoint{1.534657in}{0.706985in}}%
\pgfpathlineto{\pgfqpoint{1.526093in}{0.698795in}}%
\pgfpathlineto{\pgfqpoint{1.519001in}{0.692630in}}%
\pgfpathlineto{\pgfqpoint{1.509581in}{0.685184in}}%
\pgfpathlineto{\pgfqpoint{1.503344in}{0.680185in}}%
\pgfpathlineto{\pgfqpoint{1.490174in}{0.671573in}}%
\pgfpathlineto{\pgfqpoint{1.487688in}{0.669690in}}%
\pgfpathlineto{\pgfqpoint{1.472031in}{0.661124in}}%
\pgfpathlineto{\pgfqpoint{1.461495in}{0.657962in}}%
\pgfpathlineto{\pgfqpoint{1.456375in}{0.655748in}}%
\pgfpathlineto{\pgfqpoint{1.440718in}{0.654604in}}%
\pgfpathlineto{\pgfqpoint{1.429187in}{0.657962in}}%
\pgfpathclose%
\pgfpathmoveto{\pgfqpoint{1.737921in}{0.657962in}}%
\pgfpathlineto{\pgfqpoint{1.722536in}{0.664083in}}%
\pgfpathlineto{\pgfqpoint{1.710812in}{0.671573in}}%
\pgfpathlineto{\pgfqpoint{1.706880in}{0.673724in}}%
\pgfpathlineto{\pgfqpoint{1.691223in}{0.685084in}}%
\pgfpathlineto{\pgfqpoint{1.691104in}{0.685184in}}%
\pgfpathlineto{\pgfqpoint{1.675567in}{0.698093in}}%
\pgfpathlineto{\pgfqpoint{1.674766in}{0.698795in}}%
\pgfpathlineto{\pgfqpoint{1.660625in}{0.712407in}}%
\pgfpathlineto{\pgfqpoint{1.659910in}{0.713248in}}%
\pgfpathlineto{\pgfqpoint{1.648273in}{0.726018in}}%
\pgfpathlineto{\pgfqpoint{1.644253in}{0.732254in}}%
\pgfpathlineto{\pgfqpoint{1.638307in}{0.739629in}}%
\pgfpathlineto{\pgfqpoint{1.632222in}{0.753240in}}%
\pgfpathlineto{\pgfqpoint{1.631210in}{0.766851in}}%
\pgfpathlineto{\pgfqpoint{1.635263in}{0.780462in}}%
\pgfpathlineto{\pgfqpoint{1.644253in}{0.793858in}}%
\pgfpathlineto{\pgfqpoint{1.644368in}{0.794073in}}%
\pgfpathlineto{\pgfqpoint{1.655289in}{0.807684in}}%
\pgfpathlineto{\pgfqpoint{1.659910in}{0.812161in}}%
\pgfpathlineto{\pgfqpoint{1.668763in}{0.821295in}}%
\pgfpathlineto{\pgfqpoint{1.675567in}{0.827413in}}%
\pgfpathlineto{\pgfqpoint{1.684352in}{0.834907in}}%
\pgfpathlineto{\pgfqpoint{1.691223in}{0.840588in}}%
\pgfpathlineto{\pgfqpoint{1.702467in}{0.848518in}}%
\pgfpathlineto{\pgfqpoint{1.706880in}{0.851920in}}%
\pgfpathlineto{\pgfqpoint{1.722536in}{0.861100in}}%
\pgfpathlineto{\pgfqpoint{1.725320in}{0.862129in}}%
\pgfpathlineto{\pgfqpoint{1.738193in}{0.868359in}}%
\pgfpathlineto{\pgfqpoint{1.753849in}{0.871209in}}%
\pgfpathlineto{\pgfqpoint{1.769506in}{0.869309in}}%
\pgfpathlineto{\pgfqpoint{1.785162in}{0.862647in}}%
\pgfpathlineto{\pgfqpoint{1.785902in}{0.862129in}}%
\pgfpathlineto{\pgfqpoint{1.800819in}{0.854036in}}%
\pgfpathlineto{\pgfqpoint{1.808279in}{0.848518in}}%
\pgfpathlineto{\pgfqpoint{1.816476in}{0.842985in}}%
\pgfpathlineto{\pgfqpoint{1.826458in}{0.834907in}}%
\pgfpathlineto{\pgfqpoint{1.832132in}{0.830177in}}%
\pgfpathlineto{\pgfqpoint{1.842036in}{0.821295in}}%
\pgfpathlineto{\pgfqpoint{1.847789in}{0.815377in}}%
\pgfpathlineto{\pgfqpoint{1.855572in}{0.807684in}}%
\pgfpathlineto{\pgfqpoint{1.863445in}{0.797654in}}%
\pgfpathlineto{\pgfqpoint{1.866818in}{0.794073in}}%
\pgfpathlineto{\pgfqpoint{1.875366in}{0.780462in}}%
\pgfpathlineto{\pgfqpoint{1.879102in}{0.767055in}}%
\pgfpathlineto{\pgfqpoint{1.879189in}{0.766851in}}%
\pgfpathlineto{\pgfqpoint{1.879102in}{0.766044in}}%
\pgfpathlineto{\pgfqpoint{1.878211in}{0.753240in}}%
\pgfpathlineto{\pgfqpoint{1.872519in}{0.739629in}}%
\pgfpathlineto{\pgfqpoint{1.863445in}{0.727753in}}%
\pgfpathlineto{\pgfqpoint{1.862351in}{0.726018in}}%
\pgfpathlineto{\pgfqpoint{1.850325in}{0.712407in}}%
\pgfpathlineto{\pgfqpoint{1.847789in}{0.710051in}}%
\pgfpathlineto{\pgfqpoint{1.836102in}{0.698795in}}%
\pgfpathlineto{\pgfqpoint{1.832132in}{0.695322in}}%
\pgfpathlineto{\pgfqpoint{1.819647in}{0.685184in}}%
\pgfpathlineto{\pgfqpoint{1.816476in}{0.682578in}}%
\pgfpathlineto{\pgfqpoint{1.800819in}{0.671815in}}%
\pgfpathlineto{\pgfqpoint{1.800343in}{0.671573in}}%
\pgfpathlineto{\pgfqpoint{1.785162in}{0.662525in}}%
\pgfpathlineto{\pgfqpoint{1.772097in}{0.657962in}}%
\pgfpathlineto{\pgfqpoint{1.769506in}{0.656663in}}%
\pgfpathlineto{\pgfqpoint{1.753849in}{0.654376in}}%
\pgfpathlineto{\pgfqpoint{1.738193in}{0.657807in}}%
\pgfpathlineto{\pgfqpoint{1.737921in}{0.657962in}}%
\pgfpathclose%
\pgfpathmoveto{\pgfqpoint{0.639000in}{0.821295in}}%
\pgfpathlineto{\pgfqpoint{0.626577in}{0.830029in}}%
\pgfpathlineto{\pgfqpoint{0.620856in}{0.834907in}}%
\pgfpathlineto{\pgfqpoint{0.610920in}{0.843120in}}%
\pgfpathlineto{\pgfqpoint{0.604720in}{0.848518in}}%
\pgfpathlineto{\pgfqpoint{0.595263in}{0.857399in}}%
\pgfpathlineto{\pgfqpoint{0.589971in}{0.862129in}}%
\pgfpathlineto{\pgfqpoint{0.579607in}{0.873792in}}%
\pgfpathlineto{\pgfqpoint{0.577617in}{0.875740in}}%
\pgfpathlineto{\pgfqpoint{0.569633in}{0.889351in}}%
\pgfpathlineto{\pgfqpoint{0.568868in}{0.902962in}}%
\pgfpathlineto{\pgfqpoint{0.575516in}{0.916573in}}%
\pgfpathlineto{\pgfqpoint{0.579607in}{0.920905in}}%
\pgfpathlineto{\pgfqpoint{0.587260in}{0.930184in}}%
\pgfpathlineto{\pgfqpoint{0.595263in}{0.937581in}}%
\pgfpathlineto{\pgfqpoint{0.601676in}{0.943795in}}%
\pgfpathlineto{\pgfqpoint{0.610920in}{0.951892in}}%
\pgfpathlineto{\pgfqpoint{0.617554in}{0.957407in}}%
\pgfpathlineto{\pgfqpoint{0.626577in}{0.964985in}}%
\pgfpathlineto{\pgfqpoint{0.635281in}{0.971018in}}%
\pgfpathlineto{\pgfqpoint{0.642233in}{0.976345in}}%
\pgfpathlineto{\pgfqpoint{0.657890in}{0.984002in}}%
\pgfpathlineto{\pgfqpoint{0.663095in}{0.984629in}}%
\pgfpathlineto{\pgfqpoint{0.673546in}{0.986114in}}%
\pgfpathlineto{\pgfqpoint{0.678055in}{0.984629in}}%
\pgfpathlineto{\pgfqpoint{0.689203in}{0.981552in}}%
\pgfpathlineto{\pgfqpoint{0.704859in}{0.972043in}}%
\pgfpathlineto{\pgfqpoint{0.706064in}{0.971018in}}%
\pgfpathlineto{\pgfqpoint{0.720516in}{0.959954in}}%
\pgfpathlineto{\pgfqpoint{0.723438in}{0.957407in}}%
\pgfpathlineto{\pgfqpoint{0.736173in}{0.946447in}}%
\pgfpathlineto{\pgfqpoint{0.739230in}{0.943795in}}%
\pgfpathlineto{\pgfqpoint{0.751829in}{0.931697in}}%
\pgfpathlineto{\pgfqpoint{0.753550in}{0.930184in}}%
\pgfpathlineto{\pgfqpoint{0.765205in}{0.916573in}}%
\pgfpathlineto{\pgfqpoint{0.767486in}{0.911509in}}%
\pgfpathlineto{\pgfqpoint{0.771993in}{0.902962in}}%
\pgfpathlineto{\pgfqpoint{0.771174in}{0.889351in}}%
\pgfpathlineto{\pgfqpoint{0.767486in}{0.883574in}}%
\pgfpathlineto{\pgfqpoint{0.763236in}{0.875740in}}%
\pgfpathlineto{\pgfqpoint{0.751829in}{0.863332in}}%
\pgfpathlineto{\pgfqpoint{0.750804in}{0.862129in}}%
\pgfpathlineto{\pgfqpoint{0.736243in}{0.848518in}}%
\pgfpathlineto{\pgfqpoint{0.736173in}{0.848457in}}%
\pgfpathlineto{\pgfqpoint{0.720516in}{0.835069in}}%
\pgfpathlineto{\pgfqpoint{0.720300in}{0.834907in}}%
\pgfpathlineto{\pgfqpoint{0.704859in}{0.823010in}}%
\pgfpathlineto{\pgfqpoint{0.701802in}{0.821295in}}%
\pgfpathlineto{\pgfqpoint{0.689203in}{0.813409in}}%
\pgfpathlineto{\pgfqpoint{0.673546in}{0.808968in}}%
\pgfpathlineto{\pgfqpoint{0.657890in}{0.810906in}}%
\pgfpathlineto{\pgfqpoint{0.642233in}{0.818730in}}%
\pgfpathlineto{\pgfqpoint{0.639000in}{0.821295in}}%
\pgfpathclose%
\pgfpathmoveto{\pgfqpoint{0.949265in}{0.821295in}}%
\pgfpathlineto{\pgfqpoint{0.939708in}{0.827601in}}%
\pgfpathlineto{\pgfqpoint{0.930904in}{0.834907in}}%
\pgfpathlineto{\pgfqpoint{0.924051in}{0.840429in}}%
\pgfpathlineto{\pgfqpoint{0.914747in}{0.848518in}}%
\pgfpathlineto{\pgfqpoint{0.908395in}{0.854475in}}%
\pgfpathlineto{\pgfqpoint{0.899991in}{0.862129in}}%
\pgfpathlineto{\pgfqpoint{0.892738in}{0.870438in}}%
\pgfpathlineto{\pgfqpoint{0.887531in}{0.875740in}}%
\pgfpathlineto{\pgfqpoint{0.879755in}{0.889351in}}%
\pgfpathlineto{\pgfqpoint{0.879009in}{0.902962in}}%
\pgfpathlineto{\pgfqpoint{0.885484in}{0.916573in}}%
\pgfpathlineto{\pgfqpoint{0.892738in}{0.924562in}}%
\pgfpathlineto{\pgfqpoint{0.897293in}{0.930184in}}%
\pgfpathlineto{\pgfqpoint{0.908395in}{0.940641in}}%
\pgfpathlineto{\pgfqpoint{0.911654in}{0.943795in}}%
\pgfpathlineto{\pgfqpoint{0.924051in}{0.954638in}}%
\pgfpathlineto{\pgfqpoint{0.927467in}{0.957407in}}%
\pgfpathlineto{\pgfqpoint{0.939708in}{0.967411in}}%
\pgfpathlineto{\pgfqpoint{0.945252in}{0.971018in}}%
\pgfpathlineto{\pgfqpoint{0.955364in}{0.978265in}}%
\pgfpathlineto{\pgfqpoint{0.970443in}{0.984629in}}%
\pgfpathlineto{\pgfqpoint{0.971021in}{0.984923in}}%
\pgfpathlineto{\pgfqpoint{0.986678in}{0.985665in}}%
\pgfpathlineto{\pgfqpoint{0.989143in}{0.984629in}}%
\pgfpathlineto{\pgfqpoint{1.002334in}{0.980006in}}%
\pgfpathlineto{\pgfqpoint{1.015900in}{0.971018in}}%
\pgfpathlineto{\pgfqpoint{1.017991in}{0.969755in}}%
\pgfpathlineto{\pgfqpoint{1.033615in}{0.957407in}}%
\pgfpathlineto{\pgfqpoint{1.033647in}{0.957381in}}%
\pgfpathlineto{\pgfqpoint{1.049275in}{0.943795in}}%
\pgfpathlineto{\pgfqpoint{1.049304in}{0.943767in}}%
\pgfpathlineto{\pgfqpoint{1.063508in}{0.930184in}}%
\pgfpathlineto{\pgfqpoint{1.064960in}{0.928366in}}%
\pgfpathlineto{\pgfqpoint{1.075299in}{0.916573in}}%
\pgfpathlineto{\pgfqpoint{1.080617in}{0.905105in}}%
\pgfpathlineto{\pgfqpoint{1.081810in}{0.902962in}}%
\pgfpathlineto{\pgfqpoint{1.080956in}{0.889351in}}%
\pgfpathlineto{\pgfqpoint{1.080617in}{0.888848in}}%
\pgfpathlineto{\pgfqpoint{1.073296in}{0.875740in}}%
\pgfpathlineto{\pgfqpoint{1.064960in}{0.866949in}}%
\pgfpathlineto{\pgfqpoint{1.060812in}{0.862129in}}%
\pgfpathlineto{\pgfqpoint{1.049304in}{0.851487in}}%
\pgfpathlineto{\pgfqpoint{1.046119in}{0.848518in}}%
\pgfpathlineto{\pgfqpoint{1.033647in}{0.837740in}}%
\pgfpathlineto{\pgfqpoint{1.030019in}{0.834907in}}%
\pgfpathlineto{\pgfqpoint{1.017991in}{0.825255in}}%
\pgfpathlineto{\pgfqpoint{1.011524in}{0.821295in}}%
\pgfpathlineto{\pgfqpoint{1.002334in}{0.814989in}}%
\pgfpathlineto{\pgfqpoint{0.986678in}{0.809360in}}%
\pgfpathlineto{\pgfqpoint{0.971021in}{0.810008in}}%
\pgfpathlineto{\pgfqpoint{0.955364in}{0.816768in}}%
\pgfpathlineto{\pgfqpoint{0.949265in}{0.821295in}}%
\pgfpathclose%
\pgfpathmoveto{\pgfqpoint{1.259306in}{0.821295in}}%
\pgfpathlineto{\pgfqpoint{1.252839in}{0.825255in}}%
\pgfpathlineto{\pgfqpoint{1.240811in}{0.834907in}}%
\pgfpathlineto{\pgfqpoint{1.237183in}{0.837740in}}%
\pgfpathlineto{\pgfqpoint{1.224711in}{0.848518in}}%
\pgfpathlineto{\pgfqpoint{1.221526in}{0.851487in}}%
\pgfpathlineto{\pgfqpoint{1.210018in}{0.862129in}}%
\pgfpathlineto{\pgfqpoint{1.205870in}{0.866949in}}%
\pgfpathlineto{\pgfqpoint{1.197534in}{0.875740in}}%
\pgfpathlineto{\pgfqpoint{1.190213in}{0.888848in}}%
\pgfpathlineto{\pgfqpoint{1.189874in}{0.889351in}}%
\pgfpathlineto{\pgfqpoint{1.189020in}{0.902962in}}%
\pgfpathlineto{\pgfqpoint{1.190213in}{0.905105in}}%
\pgfpathlineto{\pgfqpoint{1.195531in}{0.916573in}}%
\pgfpathlineto{\pgfqpoint{1.205870in}{0.928366in}}%
\pgfpathlineto{\pgfqpoint{1.207322in}{0.930184in}}%
\pgfpathlineto{\pgfqpoint{1.221526in}{0.943767in}}%
\pgfpathlineto{\pgfqpoint{1.221555in}{0.943795in}}%
\pgfpathlineto{\pgfqpoint{1.237183in}{0.957381in}}%
\pgfpathlineto{\pgfqpoint{1.237215in}{0.957407in}}%
\pgfpathlineto{\pgfqpoint{1.252839in}{0.969755in}}%
\pgfpathlineto{\pgfqpoint{1.254930in}{0.971018in}}%
\pgfpathlineto{\pgfqpoint{1.268496in}{0.980006in}}%
\pgfpathlineto{\pgfqpoint{1.281687in}{0.984629in}}%
\pgfpathlineto{\pgfqpoint{1.284152in}{0.985665in}}%
\pgfpathlineto{\pgfqpoint{1.299809in}{0.984923in}}%
\pgfpathlineto{\pgfqpoint{1.300387in}{0.984629in}}%
\pgfpathlineto{\pgfqpoint{1.315466in}{0.978265in}}%
\pgfpathlineto{\pgfqpoint{1.325578in}{0.971018in}}%
\pgfpathlineto{\pgfqpoint{1.331122in}{0.967411in}}%
\pgfpathlineto{\pgfqpoint{1.343363in}{0.957407in}}%
\pgfpathlineto{\pgfqpoint{1.346779in}{0.954638in}}%
\pgfpathlineto{\pgfqpoint{1.359176in}{0.943795in}}%
\pgfpathlineto{\pgfqpoint{1.362435in}{0.940641in}}%
\pgfpathlineto{\pgfqpoint{1.373537in}{0.930184in}}%
\pgfpathlineto{\pgfqpoint{1.378092in}{0.924562in}}%
\pgfpathlineto{\pgfqpoint{1.385346in}{0.916573in}}%
\pgfpathlineto{\pgfqpoint{1.391821in}{0.902962in}}%
\pgfpathlineto{\pgfqpoint{1.391075in}{0.889351in}}%
\pgfpathlineto{\pgfqpoint{1.383299in}{0.875740in}}%
\pgfpathlineto{\pgfqpoint{1.378092in}{0.870438in}}%
\pgfpathlineto{\pgfqpoint{1.370839in}{0.862129in}}%
\pgfpathlineto{\pgfqpoint{1.362435in}{0.854475in}}%
\pgfpathlineto{\pgfqpoint{1.356083in}{0.848518in}}%
\pgfpathlineto{\pgfqpoint{1.346779in}{0.840429in}}%
\pgfpathlineto{\pgfqpoint{1.339926in}{0.834907in}}%
\pgfpathlineto{\pgfqpoint{1.331122in}{0.827601in}}%
\pgfpathlineto{\pgfqpoint{1.321565in}{0.821295in}}%
\pgfpathlineto{\pgfqpoint{1.315466in}{0.816768in}}%
\pgfpathlineto{\pgfqpoint{1.299809in}{0.810008in}}%
\pgfpathlineto{\pgfqpoint{1.284152in}{0.809360in}}%
\pgfpathlineto{\pgfqpoint{1.268496in}{0.814989in}}%
\pgfpathlineto{\pgfqpoint{1.259306in}{0.821295in}}%
\pgfpathclose%
\pgfpathmoveto{\pgfqpoint{1.569028in}{0.821295in}}%
\pgfpathlineto{\pgfqpoint{1.565971in}{0.823010in}}%
\pgfpathlineto{\pgfqpoint{1.550530in}{0.834907in}}%
\pgfpathlineto{\pgfqpoint{1.550314in}{0.835069in}}%
\pgfpathlineto{\pgfqpoint{1.534657in}{0.848457in}}%
\pgfpathlineto{\pgfqpoint{1.534587in}{0.848518in}}%
\pgfpathlineto{\pgfqpoint{1.520026in}{0.862129in}}%
\pgfpathlineto{\pgfqpoint{1.519001in}{0.863332in}}%
\pgfpathlineto{\pgfqpoint{1.507594in}{0.875740in}}%
\pgfpathlineto{\pgfqpoint{1.503344in}{0.883574in}}%
\pgfpathlineto{\pgfqpoint{1.499656in}{0.889351in}}%
\pgfpathlineto{\pgfqpoint{1.498837in}{0.902962in}}%
\pgfpathlineto{\pgfqpoint{1.503344in}{0.911509in}}%
\pgfpathlineto{\pgfqpoint{1.505625in}{0.916573in}}%
\pgfpathlineto{\pgfqpoint{1.517280in}{0.930184in}}%
\pgfpathlineto{\pgfqpoint{1.519001in}{0.931697in}}%
\pgfpathlineto{\pgfqpoint{1.531600in}{0.943795in}}%
\pgfpathlineto{\pgfqpoint{1.534657in}{0.946447in}}%
\pgfpathlineto{\pgfqpoint{1.547392in}{0.957407in}}%
\pgfpathlineto{\pgfqpoint{1.550314in}{0.959954in}}%
\pgfpathlineto{\pgfqpoint{1.564766in}{0.971018in}}%
\pgfpathlineto{\pgfqpoint{1.565971in}{0.972043in}}%
\pgfpathlineto{\pgfqpoint{1.581627in}{0.981552in}}%
\pgfpathlineto{\pgfqpoint{1.592775in}{0.984629in}}%
\pgfpathlineto{\pgfqpoint{1.597284in}{0.986114in}}%
\pgfpathlineto{\pgfqpoint{1.607735in}{0.984629in}}%
\pgfpathlineto{\pgfqpoint{1.612940in}{0.984002in}}%
\pgfpathlineto{\pgfqpoint{1.628597in}{0.976345in}}%
\pgfpathlineto{\pgfqpoint{1.635549in}{0.971018in}}%
\pgfpathlineto{\pgfqpoint{1.644253in}{0.964985in}}%
\pgfpathlineto{\pgfqpoint{1.653276in}{0.957407in}}%
\pgfpathlineto{\pgfqpoint{1.659910in}{0.951892in}}%
\pgfpathlineto{\pgfqpoint{1.669154in}{0.943795in}}%
\pgfpathlineto{\pgfqpoint{1.675567in}{0.937581in}}%
\pgfpathlineto{\pgfqpoint{1.683570in}{0.930184in}}%
\pgfpathlineto{\pgfqpoint{1.691223in}{0.920905in}}%
\pgfpathlineto{\pgfqpoint{1.695314in}{0.916573in}}%
\pgfpathlineto{\pgfqpoint{1.701962in}{0.902962in}}%
\pgfpathlineto{\pgfqpoint{1.701197in}{0.889351in}}%
\pgfpathlineto{\pgfqpoint{1.693213in}{0.875740in}}%
\pgfpathlineto{\pgfqpoint{1.691223in}{0.873792in}}%
\pgfpathlineto{\pgfqpoint{1.680859in}{0.862129in}}%
\pgfpathlineto{\pgfqpoint{1.675567in}{0.857399in}}%
\pgfpathlineto{\pgfqpoint{1.666110in}{0.848518in}}%
\pgfpathlineto{\pgfqpoint{1.659910in}{0.843120in}}%
\pgfpathlineto{\pgfqpoint{1.649974in}{0.834907in}}%
\pgfpathlineto{\pgfqpoint{1.644253in}{0.830029in}}%
\pgfpathlineto{\pgfqpoint{1.631830in}{0.821295in}}%
\pgfpathlineto{\pgfqpoint{1.628597in}{0.818730in}}%
\pgfpathlineto{\pgfqpoint{1.612940in}{0.810906in}}%
\pgfpathlineto{\pgfqpoint{1.597284in}{0.808968in}}%
\pgfpathlineto{\pgfqpoint{1.581627in}{0.813409in}}%
\pgfpathlineto{\pgfqpoint{1.569028in}{0.821295in}}%
\pgfpathclose%
\pgfpathmoveto{\pgfqpoint{0.491763in}{0.930184in}}%
\pgfpathlineto{\pgfqpoint{0.485668in}{0.932228in}}%
\pgfpathlineto{\pgfqpoint{0.470011in}{0.941100in}}%
\pgfpathlineto{\pgfqpoint{0.466480in}{0.943795in}}%
\pgfpathlineto{\pgfqpoint{0.454354in}{0.952030in}}%
\pgfpathlineto{\pgfqpoint{0.447748in}{0.957407in}}%
\pgfpathlineto{\pgfqpoint{0.438698in}{0.964837in}}%
\pgfpathlineto{\pgfqpoint{0.431708in}{0.971018in}}%
\pgfpathlineto{\pgfqpoint{0.423041in}{0.979626in}}%
\pgfpathlineto{\pgfqpoint{0.417811in}{0.984629in}}%
\pgfpathlineto{\pgfqpoint{0.407385in}{0.997144in}}%
\pgfpathlineto{\pgfqpoint{0.406293in}{0.998240in}}%
\pgfpathlineto{\pgfqpoint{0.396792in}{1.011851in}}%
\pgfpathlineto{\pgfqpoint{0.392051in}{1.025462in}}%
\pgfpathlineto{\pgfqpoint{0.392051in}{1.039073in}}%
\pgfpathlineto{\pgfqpoint{0.396792in}{1.052684in}}%
\pgfpathlineto{\pgfqpoint{0.406293in}{1.066295in}}%
\pgfpathlineto{\pgfqpoint{0.407385in}{1.067391in}}%
\pgfpathlineto{\pgfqpoint{0.417811in}{1.079907in}}%
\pgfpathlineto{\pgfqpoint{0.423041in}{1.084909in}}%
\pgfpathlineto{\pgfqpoint{0.431708in}{1.093518in}}%
\pgfpathlineto{\pgfqpoint{0.438698in}{1.099698in}}%
\pgfpathlineto{\pgfqpoint{0.447748in}{1.107129in}}%
\pgfpathlineto{\pgfqpoint{0.454354in}{1.112505in}}%
\pgfpathlineto{\pgfqpoint{0.466480in}{1.120740in}}%
\pgfpathlineto{\pgfqpoint{0.470011in}{1.123435in}}%
\pgfpathlineto{\pgfqpoint{0.485668in}{1.132307in}}%
\pgfpathlineto{\pgfqpoint{0.491763in}{1.134351in}}%
\pgfpathlineto{\pgfqpoint{0.501324in}{1.138743in}}%
\pgfpathlineto{\pgfqpoint{0.516981in}{1.140814in}}%
\pgfpathlineto{\pgfqpoint{0.532637in}{1.137706in}}%
\pgfpathlineto{\pgfqpoint{0.539072in}{1.134351in}}%
\pgfpathlineto{\pgfqpoint{0.548294in}{1.130828in}}%
\pgfpathlineto{\pgfqpoint{0.563950in}{1.121221in}}%
\pgfpathlineto{\pgfqpoint{0.564555in}{1.120740in}}%
\pgfpathlineto{\pgfqpoint{0.579607in}{1.110059in}}%
\pgfpathlineto{\pgfqpoint{0.583131in}{1.107129in}}%
\pgfpathlineto{\pgfqpoint{0.595263in}{1.096936in}}%
\pgfpathlineto{\pgfqpoint{0.599119in}{1.093518in}}%
\pgfpathlineto{\pgfqpoint{0.610920in}{1.081762in}}%
\pgfpathlineto{\pgfqpoint{0.612899in}{1.079907in}}%
\pgfpathlineto{\pgfqpoint{0.624587in}{1.066295in}}%
\pgfpathlineto{\pgfqpoint{0.626577in}{1.062920in}}%
\pgfpathlineto{\pgfqpoint{0.634147in}{1.052684in}}%
\pgfpathlineto{\pgfqpoint{0.639216in}{1.039073in}}%
\pgfpathlineto{\pgfqpoint{0.639216in}{1.025462in}}%
\pgfpathlineto{\pgfqpoint{0.634147in}{1.011851in}}%
\pgfpathlineto{\pgfqpoint{0.626577in}{1.001616in}}%
\pgfpathlineto{\pgfqpoint{0.624587in}{0.998240in}}%
\pgfpathlineto{\pgfqpoint{0.612899in}{0.984629in}}%
\pgfpathlineto{\pgfqpoint{0.610920in}{0.982773in}}%
\pgfpathlineto{\pgfqpoint{0.599119in}{0.971018in}}%
\pgfpathlineto{\pgfqpoint{0.595263in}{0.967599in}}%
\pgfpathlineto{\pgfqpoint{0.583131in}{0.957407in}}%
\pgfpathlineto{\pgfqpoint{0.579607in}{0.954476in}}%
\pgfpathlineto{\pgfqpoint{0.564555in}{0.943795in}}%
\pgfpathlineto{\pgfqpoint{0.563950in}{0.943315in}}%
\pgfpathlineto{\pgfqpoint{0.548294in}{0.933707in}}%
\pgfpathlineto{\pgfqpoint{0.539072in}{0.930184in}}%
\pgfpathlineto{\pgfqpoint{0.532637in}{0.926829in}}%
\pgfpathlineto{\pgfqpoint{0.516981in}{0.923721in}}%
\pgfpathlineto{\pgfqpoint{0.501324in}{0.925793in}}%
\pgfpathlineto{\pgfqpoint{0.491763in}{0.930184in}}%
\pgfpathclose%
\pgfpathmoveto{\pgfqpoint{0.801275in}{0.930184in}}%
\pgfpathlineto{\pgfqpoint{0.798799in}{0.930898in}}%
\pgfpathlineto{\pgfqpoint{0.783142in}{0.939031in}}%
\pgfpathlineto{\pgfqpoint{0.776607in}{0.943795in}}%
\pgfpathlineto{\pgfqpoint{0.767486in}{0.949696in}}%
\pgfpathlineto{\pgfqpoint{0.757769in}{0.957407in}}%
\pgfpathlineto{\pgfqpoint{0.751829in}{0.962154in}}%
\pgfpathlineto{\pgfqpoint{0.741739in}{0.971018in}}%
\pgfpathlineto{\pgfqpoint{0.736173in}{0.976507in}}%
\pgfpathlineto{\pgfqpoint{0.727806in}{0.984629in}}%
\pgfpathlineto{\pgfqpoint{0.720516in}{0.993519in}}%
\pgfpathlineto{\pgfqpoint{0.716021in}{0.998240in}}%
\pgfpathlineto{\pgfqpoint{0.707064in}{1.011851in}}%
\pgfpathlineto{\pgfqpoint{0.704859in}{1.018502in}}%
\pgfpathlineto{\pgfqpoint{0.701523in}{1.025462in}}%
\pgfpathlineto{\pgfqpoint{0.701523in}{1.039073in}}%
\pgfpathlineto{\pgfqpoint{0.704859in}{1.046033in}}%
\pgfpathlineto{\pgfqpoint{0.707064in}{1.052684in}}%
\pgfpathlineto{\pgfqpoint{0.716021in}{1.066295in}}%
\pgfpathlineto{\pgfqpoint{0.720516in}{1.071016in}}%
\pgfpathlineto{\pgfqpoint{0.727806in}{1.079907in}}%
\pgfpathlineto{\pgfqpoint{0.736173in}{1.088028in}}%
\pgfpathlineto{\pgfqpoint{0.741739in}{1.093518in}}%
\pgfpathlineto{\pgfqpoint{0.751829in}{1.102381in}}%
\pgfpathlineto{\pgfqpoint{0.757769in}{1.107129in}}%
\pgfpathlineto{\pgfqpoint{0.767486in}{1.114839in}}%
\pgfpathlineto{\pgfqpoint{0.776607in}{1.120740in}}%
\pgfpathlineto{\pgfqpoint{0.783142in}{1.125504in}}%
\pgfpathlineto{\pgfqpoint{0.798799in}{1.133637in}}%
\pgfpathlineto{\pgfqpoint{0.801275in}{1.134351in}}%
\pgfpathlineto{\pgfqpoint{0.814455in}{1.139572in}}%
\pgfpathlineto{\pgfqpoint{0.830112in}{1.140607in}}%
\pgfpathlineto{\pgfqpoint{0.845769in}{1.136462in}}%
\pgfpathlineto{\pgfqpoint{0.849394in}{1.134351in}}%
\pgfpathlineto{\pgfqpoint{0.861425in}{1.129201in}}%
\pgfpathlineto{\pgfqpoint{0.874349in}{1.120740in}}%
\pgfpathlineto{\pgfqpoint{0.877082in}{1.119161in}}%
\pgfpathlineto{\pgfqpoint{0.892738in}{1.107506in}}%
\pgfpathlineto{\pgfqpoint{0.893184in}{1.107129in}}%
\pgfpathlineto{\pgfqpoint{0.908395in}{1.094104in}}%
\pgfpathlineto{\pgfqpoint{0.909057in}{1.093518in}}%
\pgfpathlineto{\pgfqpoint{0.922821in}{1.079907in}}%
\pgfpathlineto{\pgfqpoint{0.924051in}{1.078381in}}%
\pgfpathlineto{\pgfqpoint{0.934786in}{1.066295in}}%
\pgfpathlineto{\pgfqpoint{0.939708in}{1.058167in}}%
\pgfpathlineto{\pgfqpoint{0.944029in}{1.052684in}}%
\pgfpathlineto{\pgfqpoint{0.949497in}{1.039073in}}%
\pgfpathlineto{\pgfqpoint{0.949497in}{1.025462in}}%
\pgfpathlineto{\pgfqpoint{0.944029in}{1.011851in}}%
\pgfpathlineto{\pgfqpoint{0.939708in}{1.006368in}}%
\pgfpathlineto{\pgfqpoint{0.934786in}{0.998240in}}%
\pgfpathlineto{\pgfqpoint{0.924051in}{0.986155in}}%
\pgfpathlineto{\pgfqpoint{0.922821in}{0.984629in}}%
\pgfpathlineto{\pgfqpoint{0.909057in}{0.971018in}}%
\pgfpathlineto{\pgfqpoint{0.908395in}{0.970431in}}%
\pgfpathlineto{\pgfqpoint{0.893184in}{0.957407in}}%
\pgfpathlineto{\pgfqpoint{0.892738in}{0.957029in}}%
\pgfpathlineto{\pgfqpoint{0.877082in}{0.945374in}}%
\pgfpathlineto{\pgfqpoint{0.874349in}{0.943795in}}%
\pgfpathlineto{\pgfqpoint{0.861425in}{0.935334in}}%
\pgfpathlineto{\pgfqpoint{0.849394in}{0.930184in}}%
\pgfpathlineto{\pgfqpoint{0.845769in}{0.928073in}}%
\pgfpathlineto{\pgfqpoint{0.830112in}{0.923928in}}%
\pgfpathlineto{\pgfqpoint{0.814455in}{0.924964in}}%
\pgfpathlineto{\pgfqpoint{0.801275in}{0.930184in}}%
\pgfpathclose%
\pgfpathmoveto{\pgfqpoint{1.110904in}{0.930184in}}%
\pgfpathlineto{\pgfqpoint{1.096274in}{0.937109in}}%
\pgfpathlineto{\pgfqpoint{1.086622in}{0.943795in}}%
\pgfpathlineto{\pgfqpoint{1.080617in}{0.947476in}}%
\pgfpathlineto{\pgfqpoint{1.067735in}{0.957407in}}%
\pgfpathlineto{\pgfqpoint{1.064960in}{0.959556in}}%
\pgfpathlineto{\pgfqpoint{1.051777in}{0.971018in}}%
\pgfpathlineto{\pgfqpoint{1.049304in}{0.973430in}}%
\pgfpathlineto{\pgfqpoint{1.037881in}{0.984629in}}%
\pgfpathlineto{\pgfqpoint{1.033647in}{0.989849in}}%
\pgfpathlineto{\pgfqpoint{1.025956in}{0.998240in}}%
\pgfpathlineto{\pgfqpoint{1.017991in}{1.010959in}}%
\pgfpathlineto{\pgfqpoint{1.017234in}{1.011851in}}%
\pgfpathlineto{\pgfqpoint{1.011271in}{1.025462in}}%
\pgfpathlineto{\pgfqpoint{1.011271in}{1.039073in}}%
\pgfpathlineto{\pgfqpoint{1.017234in}{1.052684in}}%
\pgfpathlineto{\pgfqpoint{1.017991in}{1.053576in}}%
\pgfpathlineto{\pgfqpoint{1.025956in}{1.066295in}}%
\pgfpathlineto{\pgfqpoint{1.033647in}{1.074686in}}%
\pgfpathlineto{\pgfqpoint{1.037881in}{1.079907in}}%
\pgfpathlineto{\pgfqpoint{1.049304in}{1.091106in}}%
\pgfpathlineto{\pgfqpoint{1.051777in}{1.093518in}}%
\pgfpathlineto{\pgfqpoint{1.064960in}{1.104979in}}%
\pgfpathlineto{\pgfqpoint{1.067735in}{1.107129in}}%
\pgfpathlineto{\pgfqpoint{1.080617in}{1.117059in}}%
\pgfpathlineto{\pgfqpoint{1.086622in}{1.120740in}}%
\pgfpathlineto{\pgfqpoint{1.096274in}{1.127426in}}%
\pgfpathlineto{\pgfqpoint{1.110904in}{1.134351in}}%
\pgfpathlineto{\pgfqpoint{1.111930in}{1.135009in}}%
\pgfpathlineto{\pgfqpoint{1.127587in}{1.140193in}}%
\pgfpathlineto{\pgfqpoint{1.143243in}{1.140193in}}%
\pgfpathlineto{\pgfqpoint{1.158900in}{1.135009in}}%
\pgfpathlineto{\pgfqpoint{1.159926in}{1.134351in}}%
\pgfpathlineto{\pgfqpoint{1.174556in}{1.127426in}}%
\pgfpathlineto{\pgfqpoint{1.184208in}{1.120740in}}%
\pgfpathlineto{\pgfqpoint{1.190213in}{1.117059in}}%
\pgfpathlineto{\pgfqpoint{1.203095in}{1.107129in}}%
\pgfpathlineto{\pgfqpoint{1.205870in}{1.104979in}}%
\pgfpathlineto{\pgfqpoint{1.219053in}{1.093518in}}%
\pgfpathlineto{\pgfqpoint{1.221526in}{1.091106in}}%
\pgfpathlineto{\pgfqpoint{1.232949in}{1.079907in}}%
\pgfpathlineto{\pgfqpoint{1.237183in}{1.074686in}}%
\pgfpathlineto{\pgfqpoint{1.244874in}{1.066295in}}%
\pgfpathlineto{\pgfqpoint{1.252839in}{1.053576in}}%
\pgfpathlineto{\pgfqpoint{1.253596in}{1.052684in}}%
\pgfpathlineto{\pgfqpoint{1.259559in}{1.039073in}}%
\pgfpathlineto{\pgfqpoint{1.259559in}{1.025462in}}%
\pgfpathlineto{\pgfqpoint{1.253596in}{1.011851in}}%
\pgfpathlineto{\pgfqpoint{1.252839in}{1.010959in}}%
\pgfpathlineto{\pgfqpoint{1.244874in}{0.998240in}}%
\pgfpathlineto{\pgfqpoint{1.237183in}{0.989849in}}%
\pgfpathlineto{\pgfqpoint{1.232949in}{0.984629in}}%
\pgfpathlineto{\pgfqpoint{1.221526in}{0.973430in}}%
\pgfpathlineto{\pgfqpoint{1.219053in}{0.971018in}}%
\pgfpathlineto{\pgfqpoint{1.205870in}{0.959556in}}%
\pgfpathlineto{\pgfqpoint{1.203095in}{0.957407in}}%
\pgfpathlineto{\pgfqpoint{1.190213in}{0.947476in}}%
\pgfpathlineto{\pgfqpoint{1.184208in}{0.943795in}}%
\pgfpathlineto{\pgfqpoint{1.174556in}{0.937109in}}%
\pgfpathlineto{\pgfqpoint{1.159926in}{0.930184in}}%
\pgfpathlineto{\pgfqpoint{1.158900in}{0.929526in}}%
\pgfpathlineto{\pgfqpoint{1.143243in}{0.924342in}}%
\pgfpathlineto{\pgfqpoint{1.127587in}{0.924342in}}%
\pgfpathlineto{\pgfqpoint{1.111930in}{0.929526in}}%
\pgfpathlineto{\pgfqpoint{1.110904in}{0.930184in}}%
\pgfpathclose%
\pgfpathmoveto{\pgfqpoint{1.421436in}{0.930184in}}%
\pgfpathlineto{\pgfqpoint{1.409405in}{0.935334in}}%
\pgfpathlineto{\pgfqpoint{1.396481in}{0.943795in}}%
\pgfpathlineto{\pgfqpoint{1.393748in}{0.945374in}}%
\pgfpathlineto{\pgfqpoint{1.378092in}{0.957029in}}%
\pgfpathlineto{\pgfqpoint{1.377646in}{0.957407in}}%
\pgfpathlineto{\pgfqpoint{1.362435in}{0.970431in}}%
\pgfpathlineto{\pgfqpoint{1.361773in}{0.971018in}}%
\pgfpathlineto{\pgfqpoint{1.348009in}{0.984629in}}%
\pgfpathlineto{\pgfqpoint{1.346779in}{0.986155in}}%
\pgfpathlineto{\pgfqpoint{1.336044in}{0.998240in}}%
\pgfpathlineto{\pgfqpoint{1.331122in}{1.006368in}}%
\pgfpathlineto{\pgfqpoint{1.326801in}{1.011851in}}%
\pgfpathlineto{\pgfqpoint{1.321333in}{1.025462in}}%
\pgfpathlineto{\pgfqpoint{1.321333in}{1.039073in}}%
\pgfpathlineto{\pgfqpoint{1.326801in}{1.052684in}}%
\pgfpathlineto{\pgfqpoint{1.331122in}{1.058167in}}%
\pgfpathlineto{\pgfqpoint{1.336044in}{1.066295in}}%
\pgfpathlineto{\pgfqpoint{1.346779in}{1.078381in}}%
\pgfpathlineto{\pgfqpoint{1.348009in}{1.079907in}}%
\pgfpathlineto{\pgfqpoint{1.361773in}{1.093518in}}%
\pgfpathlineto{\pgfqpoint{1.362435in}{1.094104in}}%
\pgfpathlineto{\pgfqpoint{1.377646in}{1.107129in}}%
\pgfpathlineto{\pgfqpoint{1.378092in}{1.107506in}}%
\pgfpathlineto{\pgfqpoint{1.393748in}{1.119161in}}%
\pgfpathlineto{\pgfqpoint{1.396481in}{1.120740in}}%
\pgfpathlineto{\pgfqpoint{1.409405in}{1.129201in}}%
\pgfpathlineto{\pgfqpoint{1.421436in}{1.134351in}}%
\pgfpathlineto{\pgfqpoint{1.425061in}{1.136462in}}%
\pgfpathlineto{\pgfqpoint{1.440718in}{1.140607in}}%
\pgfpathlineto{\pgfqpoint{1.456375in}{1.139572in}}%
\pgfpathlineto{\pgfqpoint{1.469555in}{1.134351in}}%
\pgfpathlineto{\pgfqpoint{1.472031in}{1.133637in}}%
\pgfpathlineto{\pgfqpoint{1.487688in}{1.125504in}}%
\pgfpathlineto{\pgfqpoint{1.494223in}{1.120740in}}%
\pgfpathlineto{\pgfqpoint{1.503344in}{1.114839in}}%
\pgfpathlineto{\pgfqpoint{1.513061in}{1.107129in}}%
\pgfpathlineto{\pgfqpoint{1.519001in}{1.102381in}}%
\pgfpathlineto{\pgfqpoint{1.529091in}{1.093518in}}%
\pgfpathlineto{\pgfqpoint{1.534657in}{1.088028in}}%
\pgfpathlineto{\pgfqpoint{1.543024in}{1.079907in}}%
\pgfpathlineto{\pgfqpoint{1.550314in}{1.071016in}}%
\pgfpathlineto{\pgfqpoint{1.554809in}{1.066295in}}%
\pgfpathlineto{\pgfqpoint{1.563766in}{1.052684in}}%
\pgfpathlineto{\pgfqpoint{1.565971in}{1.046033in}}%
\pgfpathlineto{\pgfqpoint{1.569307in}{1.039073in}}%
\pgfpathlineto{\pgfqpoint{1.569307in}{1.025462in}}%
\pgfpathlineto{\pgfqpoint{1.565971in}{1.018502in}}%
\pgfpathlineto{\pgfqpoint{1.563766in}{1.011851in}}%
\pgfpathlineto{\pgfqpoint{1.554809in}{0.998240in}}%
\pgfpathlineto{\pgfqpoint{1.550314in}{0.993519in}}%
\pgfpathlineto{\pgfqpoint{1.543024in}{0.984629in}}%
\pgfpathlineto{\pgfqpoint{1.534657in}{0.976507in}}%
\pgfpathlineto{\pgfqpoint{1.529091in}{0.971018in}}%
\pgfpathlineto{\pgfqpoint{1.519001in}{0.962154in}}%
\pgfpathlineto{\pgfqpoint{1.513061in}{0.957407in}}%
\pgfpathlineto{\pgfqpoint{1.503344in}{0.949696in}}%
\pgfpathlineto{\pgfqpoint{1.494223in}{0.943795in}}%
\pgfpathlineto{\pgfqpoint{1.487688in}{0.939031in}}%
\pgfpathlineto{\pgfqpoint{1.472031in}{0.930898in}}%
\pgfpathlineto{\pgfqpoint{1.469555in}{0.930184in}}%
\pgfpathlineto{\pgfqpoint{1.456375in}{0.924964in}}%
\pgfpathlineto{\pgfqpoint{1.440718in}{0.923928in}}%
\pgfpathlineto{\pgfqpoint{1.425061in}{0.928073in}}%
\pgfpathlineto{\pgfqpoint{1.421436in}{0.930184in}}%
\pgfpathclose%
\pgfpathmoveto{\pgfqpoint{1.731758in}{0.930184in}}%
\pgfpathlineto{\pgfqpoint{1.722536in}{0.933707in}}%
\pgfpathlineto{\pgfqpoint{1.706880in}{0.943315in}}%
\pgfpathlineto{\pgfqpoint{1.706275in}{0.943795in}}%
\pgfpathlineto{\pgfqpoint{1.691223in}{0.954476in}}%
\pgfpathlineto{\pgfqpoint{1.687699in}{0.957407in}}%
\pgfpathlineto{\pgfqpoint{1.675567in}{0.967599in}}%
\pgfpathlineto{\pgfqpoint{1.671711in}{0.971018in}}%
\pgfpathlineto{\pgfqpoint{1.659910in}{0.982773in}}%
\pgfpathlineto{\pgfqpoint{1.657931in}{0.984629in}}%
\pgfpathlineto{\pgfqpoint{1.646243in}{0.998240in}}%
\pgfpathlineto{\pgfqpoint{1.644253in}{1.001616in}}%
\pgfpathlineto{\pgfqpoint{1.636683in}{1.011851in}}%
\pgfpathlineto{\pgfqpoint{1.631614in}{1.025462in}}%
\pgfpathlineto{\pgfqpoint{1.631614in}{1.039073in}}%
\pgfpathlineto{\pgfqpoint{1.636683in}{1.052684in}}%
\pgfpathlineto{\pgfqpoint{1.644253in}{1.062920in}}%
\pgfpathlineto{\pgfqpoint{1.646243in}{1.066295in}}%
\pgfpathlineto{\pgfqpoint{1.657931in}{1.079907in}}%
\pgfpathlineto{\pgfqpoint{1.659910in}{1.081762in}}%
\pgfpathlineto{\pgfqpoint{1.671711in}{1.093518in}}%
\pgfpathlineto{\pgfqpoint{1.675567in}{1.096936in}}%
\pgfpathlineto{\pgfqpoint{1.687699in}{1.107129in}}%
\pgfpathlineto{\pgfqpoint{1.691223in}{1.110059in}}%
\pgfpathlineto{\pgfqpoint{1.706275in}{1.120740in}}%
\pgfpathlineto{\pgfqpoint{1.706880in}{1.121221in}}%
\pgfpathlineto{\pgfqpoint{1.722536in}{1.130828in}}%
\pgfpathlineto{\pgfqpoint{1.731758in}{1.134351in}}%
\pgfpathlineto{\pgfqpoint{1.738193in}{1.137706in}}%
\pgfpathlineto{\pgfqpoint{1.753849in}{1.140814in}}%
\pgfpathlineto{\pgfqpoint{1.769506in}{1.138743in}}%
\pgfpathlineto{\pgfqpoint{1.779067in}{1.134351in}}%
\pgfpathlineto{\pgfqpoint{1.785162in}{1.132307in}}%
\pgfpathlineto{\pgfqpoint{1.800819in}{1.123435in}}%
\pgfpathlineto{\pgfqpoint{1.804350in}{1.120740in}}%
\pgfpathlineto{\pgfqpoint{1.816476in}{1.112505in}}%
\pgfpathlineto{\pgfqpoint{1.823082in}{1.107129in}}%
\pgfpathlineto{\pgfqpoint{1.832132in}{1.099698in}}%
\pgfpathlineto{\pgfqpoint{1.839122in}{1.093518in}}%
\pgfpathlineto{\pgfqpoint{1.847789in}{1.084909in}}%
\pgfpathlineto{\pgfqpoint{1.853019in}{1.079907in}}%
\pgfpathlineto{\pgfqpoint{1.863445in}{1.067391in}}%
\pgfpathlineto{\pgfqpoint{1.864537in}{1.066295in}}%
\pgfpathlineto{\pgfqpoint{1.874038in}{1.052684in}}%
\pgfpathlineto{\pgfqpoint{1.878779in}{1.039073in}}%
\pgfpathlineto{\pgfqpoint{1.878779in}{1.025462in}}%
\pgfpathlineto{\pgfqpoint{1.874038in}{1.011851in}}%
\pgfpathlineto{\pgfqpoint{1.864537in}{0.998240in}}%
\pgfpathlineto{\pgfqpoint{1.863445in}{0.997144in}}%
\pgfpathlineto{\pgfqpoint{1.853019in}{0.984629in}}%
\pgfpathlineto{\pgfqpoint{1.847789in}{0.979626in}}%
\pgfpathlineto{\pgfqpoint{1.839122in}{0.971018in}}%
\pgfpathlineto{\pgfqpoint{1.832132in}{0.964837in}}%
\pgfpathlineto{\pgfqpoint{1.823082in}{0.957407in}}%
\pgfpathlineto{\pgfqpoint{1.816476in}{0.952030in}}%
\pgfpathlineto{\pgfqpoint{1.804350in}{0.943795in}}%
\pgfpathlineto{\pgfqpoint{1.800819in}{0.941100in}}%
\pgfpathlineto{\pgfqpoint{1.785162in}{0.932228in}}%
\pgfpathlineto{\pgfqpoint{1.779067in}{0.930184in}}%
\pgfpathlineto{\pgfqpoint{1.769506in}{0.925793in}}%
\pgfpathlineto{\pgfqpoint{1.753849in}{0.923721in}}%
\pgfpathlineto{\pgfqpoint{1.738193in}{0.926829in}}%
\pgfpathlineto{\pgfqpoint{1.731758in}{0.930184in}}%
\pgfpathclose%
\pgfpathmoveto{\pgfqpoint{0.663095in}{1.079907in}}%
\pgfpathlineto{\pgfqpoint{0.657890in}{1.080533in}}%
\pgfpathlineto{\pgfqpoint{0.642233in}{1.088190in}}%
\pgfpathlineto{\pgfqpoint{0.635281in}{1.093518in}}%
\pgfpathlineto{\pgfqpoint{0.626577in}{1.099551in}}%
\pgfpathlineto{\pgfqpoint{0.617554in}{1.107129in}}%
\pgfpathlineto{\pgfqpoint{0.610920in}{1.112643in}}%
\pgfpathlineto{\pgfqpoint{0.601676in}{1.120740in}}%
\pgfpathlineto{\pgfqpoint{0.595263in}{1.126955in}}%
\pgfpathlineto{\pgfqpoint{0.587260in}{1.134351in}}%
\pgfpathlineto{\pgfqpoint{0.579607in}{1.143630in}}%
\pgfpathlineto{\pgfqpoint{0.575516in}{1.147962in}}%
\pgfpathlineto{\pgfqpoint{0.568868in}{1.161573in}}%
\pgfpathlineto{\pgfqpoint{0.569633in}{1.175184in}}%
\pgfpathlineto{\pgfqpoint{0.577617in}{1.188795in}}%
\pgfpathlineto{\pgfqpoint{0.579607in}{1.190744in}}%
\pgfpathlineto{\pgfqpoint{0.589971in}{1.202407in}}%
\pgfpathlineto{\pgfqpoint{0.595263in}{1.207136in}}%
\pgfpathlineto{\pgfqpoint{0.604720in}{1.216018in}}%
\pgfpathlineto{\pgfqpoint{0.610920in}{1.221415in}}%
\pgfpathlineto{\pgfqpoint{0.620856in}{1.229629in}}%
\pgfpathlineto{\pgfqpoint{0.626577in}{1.234506in}}%
\pgfpathlineto{\pgfqpoint{0.639000in}{1.243240in}}%
\pgfpathlineto{\pgfqpoint{0.642233in}{1.245805in}}%
\pgfpathlineto{\pgfqpoint{0.657890in}{1.253630in}}%
\pgfpathlineto{\pgfqpoint{0.673546in}{1.255567in}}%
\pgfpathlineto{\pgfqpoint{0.689203in}{1.251126in}}%
\pgfpathlineto{\pgfqpoint{0.701802in}{1.243240in}}%
\pgfpathlineto{\pgfqpoint{0.704859in}{1.241525in}}%
\pgfpathlineto{\pgfqpoint{0.720300in}{1.229629in}}%
\pgfpathlineto{\pgfqpoint{0.720516in}{1.229466in}}%
\pgfpathlineto{\pgfqpoint{0.736173in}{1.216078in}}%
\pgfpathlineto{\pgfqpoint{0.736243in}{1.216018in}}%
\pgfpathlineto{\pgfqpoint{0.750804in}{1.202407in}}%
\pgfpathlineto{\pgfqpoint{0.751829in}{1.201203in}}%
\pgfpathlineto{\pgfqpoint{0.763236in}{1.188795in}}%
\pgfpathlineto{\pgfqpoint{0.767486in}{1.180961in}}%
\pgfpathlineto{\pgfqpoint{0.771174in}{1.175184in}}%
\pgfpathlineto{\pgfqpoint{0.771993in}{1.161573in}}%
\pgfpathlineto{\pgfqpoint{0.767486in}{1.153027in}}%
\pgfpathlineto{\pgfqpoint{0.765205in}{1.147962in}}%
\pgfpathlineto{\pgfqpoint{0.753550in}{1.134351in}}%
\pgfpathlineto{\pgfqpoint{0.751829in}{1.132838in}}%
\pgfpathlineto{\pgfqpoint{0.739230in}{1.120740in}}%
\pgfpathlineto{\pgfqpoint{0.736173in}{1.118088in}}%
\pgfpathlineto{\pgfqpoint{0.723438in}{1.107129in}}%
\pgfpathlineto{\pgfqpoint{0.720516in}{1.104581in}}%
\pgfpathlineto{\pgfqpoint{0.706064in}{1.093518in}}%
\pgfpathlineto{\pgfqpoint{0.704859in}{1.092492in}}%
\pgfpathlineto{\pgfqpoint{0.689203in}{1.082983in}}%
\pgfpathlineto{\pgfqpoint{0.678055in}{1.079907in}}%
\pgfpathlineto{\pgfqpoint{0.673546in}{1.078421in}}%
\pgfpathlineto{\pgfqpoint{0.663095in}{1.079907in}}%
\pgfpathclose%
\pgfpathmoveto{\pgfqpoint{0.970443in}{1.079907in}}%
\pgfpathlineto{\pgfqpoint{0.955364in}{1.086271in}}%
\pgfpathlineto{\pgfqpoint{0.945252in}{1.093518in}}%
\pgfpathlineto{\pgfqpoint{0.939708in}{1.097124in}}%
\pgfpathlineto{\pgfqpoint{0.927467in}{1.107129in}}%
\pgfpathlineto{\pgfqpoint{0.924051in}{1.109897in}}%
\pgfpathlineto{\pgfqpoint{0.911654in}{1.120740in}}%
\pgfpathlineto{\pgfqpoint{0.908395in}{1.123894in}}%
\pgfpathlineto{\pgfqpoint{0.897293in}{1.134351in}}%
\pgfpathlineto{\pgfqpoint{0.892738in}{1.139973in}}%
\pgfpathlineto{\pgfqpoint{0.885484in}{1.147962in}}%
\pgfpathlineto{\pgfqpoint{0.879009in}{1.161573in}}%
\pgfpathlineto{\pgfqpoint{0.879755in}{1.175184in}}%
\pgfpathlineto{\pgfqpoint{0.887531in}{1.188795in}}%
\pgfpathlineto{\pgfqpoint{0.892738in}{1.194098in}}%
\pgfpathlineto{\pgfqpoint{0.899991in}{1.202407in}}%
\pgfpathlineto{\pgfqpoint{0.908395in}{1.210060in}}%
\pgfpathlineto{\pgfqpoint{0.914747in}{1.216018in}}%
\pgfpathlineto{\pgfqpoint{0.924051in}{1.224107in}}%
\pgfpathlineto{\pgfqpoint{0.930904in}{1.229629in}}%
\pgfpathlineto{\pgfqpoint{0.939708in}{1.236934in}}%
\pgfpathlineto{\pgfqpoint{0.949265in}{1.243240in}}%
\pgfpathlineto{\pgfqpoint{0.955364in}{1.247767in}}%
\pgfpathlineto{\pgfqpoint{0.971021in}{1.254527in}}%
\pgfpathlineto{\pgfqpoint{0.986678in}{1.255175in}}%
\pgfpathlineto{\pgfqpoint{1.002334in}{1.249546in}}%
\pgfpathlineto{\pgfqpoint{1.011524in}{1.243240in}}%
\pgfpathlineto{\pgfqpoint{1.017991in}{1.239280in}}%
\pgfpathlineto{\pgfqpoint{1.030019in}{1.229629in}}%
\pgfpathlineto{\pgfqpoint{1.033647in}{1.226795in}}%
\pgfpathlineto{\pgfqpoint{1.046119in}{1.216018in}}%
\pgfpathlineto{\pgfqpoint{1.049304in}{1.213048in}}%
\pgfpathlineto{\pgfqpoint{1.060812in}{1.202407in}}%
\pgfpathlineto{\pgfqpoint{1.064960in}{1.197587in}}%
\pgfpathlineto{\pgfqpoint{1.073296in}{1.188795in}}%
\pgfpathlineto{\pgfqpoint{1.080617in}{1.175687in}}%
\pgfpathlineto{\pgfqpoint{1.080956in}{1.175184in}}%
\pgfpathlineto{\pgfqpoint{1.081810in}{1.161573in}}%
\pgfpathlineto{\pgfqpoint{1.080617in}{1.159430in}}%
\pgfpathlineto{\pgfqpoint{1.075299in}{1.147962in}}%
\pgfpathlineto{\pgfqpoint{1.064960in}{1.136169in}}%
\pgfpathlineto{\pgfqpoint{1.063508in}{1.134351in}}%
\pgfpathlineto{\pgfqpoint{1.049304in}{1.120768in}}%
\pgfpathlineto{\pgfqpoint{1.049275in}{1.120740in}}%
\pgfpathlineto{\pgfqpoint{1.033647in}{1.107154in}}%
\pgfpathlineto{\pgfqpoint{1.033615in}{1.107129in}}%
\pgfpathlineto{\pgfqpoint{1.017991in}{1.094780in}}%
\pgfpathlineto{\pgfqpoint{1.015900in}{1.093518in}}%
\pgfpathlineto{\pgfqpoint{1.002334in}{1.084530in}}%
\pgfpathlineto{\pgfqpoint{0.989143in}{1.079907in}}%
\pgfpathlineto{\pgfqpoint{0.986678in}{1.078870in}}%
\pgfpathlineto{\pgfqpoint{0.971021in}{1.079612in}}%
\pgfpathlineto{\pgfqpoint{0.970443in}{1.079907in}}%
\pgfpathclose%
\pgfpathmoveto{\pgfqpoint{1.281687in}{1.079907in}}%
\pgfpathlineto{\pgfqpoint{1.268496in}{1.084530in}}%
\pgfpathlineto{\pgfqpoint{1.254930in}{1.093518in}}%
\pgfpathlineto{\pgfqpoint{1.252839in}{1.094780in}}%
\pgfpathlineto{\pgfqpoint{1.237215in}{1.107129in}}%
\pgfpathlineto{\pgfqpoint{1.237183in}{1.107154in}}%
\pgfpathlineto{\pgfqpoint{1.221555in}{1.120740in}}%
\pgfpathlineto{\pgfqpoint{1.221526in}{1.120768in}}%
\pgfpathlineto{\pgfqpoint{1.207322in}{1.134351in}}%
\pgfpathlineto{\pgfqpoint{1.205870in}{1.136169in}}%
\pgfpathlineto{\pgfqpoint{1.195531in}{1.147962in}}%
\pgfpathlineto{\pgfqpoint{1.190213in}{1.159430in}}%
\pgfpathlineto{\pgfqpoint{1.189020in}{1.161573in}}%
\pgfpathlineto{\pgfqpoint{1.189874in}{1.175184in}}%
\pgfpathlineto{\pgfqpoint{1.190213in}{1.175687in}}%
\pgfpathlineto{\pgfqpoint{1.197534in}{1.188795in}}%
\pgfpathlineto{\pgfqpoint{1.205870in}{1.197587in}}%
\pgfpathlineto{\pgfqpoint{1.210018in}{1.202407in}}%
\pgfpathlineto{\pgfqpoint{1.221526in}{1.213048in}}%
\pgfpathlineto{\pgfqpoint{1.224711in}{1.216018in}}%
\pgfpathlineto{\pgfqpoint{1.237183in}{1.226795in}}%
\pgfpathlineto{\pgfqpoint{1.240811in}{1.229629in}}%
\pgfpathlineto{\pgfqpoint{1.252839in}{1.239280in}}%
\pgfpathlineto{\pgfqpoint{1.259306in}{1.243240in}}%
\pgfpathlineto{\pgfqpoint{1.268496in}{1.249546in}}%
\pgfpathlineto{\pgfqpoint{1.284152in}{1.255175in}}%
\pgfpathlineto{\pgfqpoint{1.299809in}{1.254527in}}%
\pgfpathlineto{\pgfqpoint{1.315466in}{1.247767in}}%
\pgfpathlineto{\pgfqpoint{1.321565in}{1.243240in}}%
\pgfpathlineto{\pgfqpoint{1.331122in}{1.236934in}}%
\pgfpathlineto{\pgfqpoint{1.339926in}{1.229629in}}%
\pgfpathlineto{\pgfqpoint{1.346779in}{1.224107in}}%
\pgfpathlineto{\pgfqpoint{1.356083in}{1.216018in}}%
\pgfpathlineto{\pgfqpoint{1.362435in}{1.210060in}}%
\pgfpathlineto{\pgfqpoint{1.370839in}{1.202407in}}%
\pgfpathlineto{\pgfqpoint{1.378092in}{1.194098in}}%
\pgfpathlineto{\pgfqpoint{1.383299in}{1.188795in}}%
\pgfpathlineto{\pgfqpoint{1.391075in}{1.175184in}}%
\pgfpathlineto{\pgfqpoint{1.391821in}{1.161573in}}%
\pgfpathlineto{\pgfqpoint{1.385346in}{1.147962in}}%
\pgfpathlineto{\pgfqpoint{1.378092in}{1.139973in}}%
\pgfpathlineto{\pgfqpoint{1.373537in}{1.134351in}}%
\pgfpathlineto{\pgfqpoint{1.362435in}{1.123894in}}%
\pgfpathlineto{\pgfqpoint{1.359176in}{1.120740in}}%
\pgfpathlineto{\pgfqpoint{1.346779in}{1.109897in}}%
\pgfpathlineto{\pgfqpoint{1.343363in}{1.107129in}}%
\pgfpathlineto{\pgfqpoint{1.331122in}{1.097124in}}%
\pgfpathlineto{\pgfqpoint{1.325578in}{1.093518in}}%
\pgfpathlineto{\pgfqpoint{1.315466in}{1.086271in}}%
\pgfpathlineto{\pgfqpoint{1.300387in}{1.079907in}}%
\pgfpathlineto{\pgfqpoint{1.299809in}{1.079612in}}%
\pgfpathlineto{\pgfqpoint{1.284152in}{1.078870in}}%
\pgfpathlineto{\pgfqpoint{1.281687in}{1.079907in}}%
\pgfpathclose%
\pgfpathmoveto{\pgfqpoint{1.592775in}{1.079907in}}%
\pgfpathlineto{\pgfqpoint{1.581627in}{1.082983in}}%
\pgfpathlineto{\pgfqpoint{1.565971in}{1.092492in}}%
\pgfpathlineto{\pgfqpoint{1.564766in}{1.093518in}}%
\pgfpathlineto{\pgfqpoint{1.550314in}{1.104581in}}%
\pgfpathlineto{\pgfqpoint{1.547392in}{1.107129in}}%
\pgfpathlineto{\pgfqpoint{1.534657in}{1.118088in}}%
\pgfpathlineto{\pgfqpoint{1.531600in}{1.120740in}}%
\pgfpathlineto{\pgfqpoint{1.519001in}{1.132838in}}%
\pgfpathlineto{\pgfqpoint{1.517280in}{1.134351in}}%
\pgfpathlineto{\pgfqpoint{1.505625in}{1.147962in}}%
\pgfpathlineto{\pgfqpoint{1.503344in}{1.153027in}}%
\pgfpathlineto{\pgfqpoint{1.498837in}{1.161573in}}%
\pgfpathlineto{\pgfqpoint{1.499656in}{1.175184in}}%
\pgfpathlineto{\pgfqpoint{1.503344in}{1.180961in}}%
\pgfpathlineto{\pgfqpoint{1.507594in}{1.188795in}}%
\pgfpathlineto{\pgfqpoint{1.519001in}{1.201203in}}%
\pgfpathlineto{\pgfqpoint{1.520026in}{1.202407in}}%
\pgfpathlineto{\pgfqpoint{1.534587in}{1.216018in}}%
\pgfpathlineto{\pgfqpoint{1.534657in}{1.216078in}}%
\pgfpathlineto{\pgfqpoint{1.550314in}{1.229466in}}%
\pgfpathlineto{\pgfqpoint{1.550530in}{1.229629in}}%
\pgfpathlineto{\pgfqpoint{1.565971in}{1.241525in}}%
\pgfpathlineto{\pgfqpoint{1.569028in}{1.243240in}}%
\pgfpathlineto{\pgfqpoint{1.581627in}{1.251126in}}%
\pgfpathlineto{\pgfqpoint{1.597284in}{1.255567in}}%
\pgfpathlineto{\pgfqpoint{1.612940in}{1.253630in}}%
\pgfpathlineto{\pgfqpoint{1.628597in}{1.245805in}}%
\pgfpathlineto{\pgfqpoint{1.631830in}{1.243240in}}%
\pgfpathlineto{\pgfqpoint{1.644253in}{1.234506in}}%
\pgfpathlineto{\pgfqpoint{1.649974in}{1.229629in}}%
\pgfpathlineto{\pgfqpoint{1.659910in}{1.221415in}}%
\pgfpathlineto{\pgfqpoint{1.666110in}{1.216018in}}%
\pgfpathlineto{\pgfqpoint{1.675567in}{1.207136in}}%
\pgfpathlineto{\pgfqpoint{1.680859in}{1.202407in}}%
\pgfpathlineto{\pgfqpoint{1.691223in}{1.190744in}}%
\pgfpathlineto{\pgfqpoint{1.693213in}{1.188795in}}%
\pgfpathlineto{\pgfqpoint{1.701197in}{1.175184in}}%
\pgfpathlineto{\pgfqpoint{1.701962in}{1.161573in}}%
\pgfpathlineto{\pgfqpoint{1.695314in}{1.147962in}}%
\pgfpathlineto{\pgfqpoint{1.691223in}{1.143630in}}%
\pgfpathlineto{\pgfqpoint{1.683570in}{1.134351in}}%
\pgfpathlineto{\pgfqpoint{1.675567in}{1.126955in}}%
\pgfpathlineto{\pgfqpoint{1.669154in}{1.120740in}}%
\pgfpathlineto{\pgfqpoint{1.659910in}{1.112643in}}%
\pgfpathlineto{\pgfqpoint{1.653276in}{1.107129in}}%
\pgfpathlineto{\pgfqpoint{1.644253in}{1.099551in}}%
\pgfpathlineto{\pgfqpoint{1.635549in}{1.093518in}}%
\pgfpathlineto{\pgfqpoint{1.628597in}{1.088190in}}%
\pgfpathlineto{\pgfqpoint{1.612940in}{1.080533in}}%
\pgfpathlineto{\pgfqpoint{1.607735in}{1.079907in}}%
\pgfpathlineto{\pgfqpoint{1.597284in}{1.078421in}}%
\pgfpathlineto{\pgfqpoint{1.592775in}{1.079907in}}%
\pgfpathclose%
\pgfpathmoveto{\pgfqpoint{0.484928in}{1.202407in}}%
\pgfpathlineto{\pgfqpoint{0.470011in}{1.210499in}}%
\pgfpathlineto{\pgfqpoint{0.462551in}{1.216018in}}%
\pgfpathlineto{\pgfqpoint{0.454354in}{1.221551in}}%
\pgfpathlineto{\pgfqpoint{0.444372in}{1.229629in}}%
\pgfpathlineto{\pgfqpoint{0.438698in}{1.234359in}}%
\pgfpathlineto{\pgfqpoint{0.428794in}{1.243240in}}%
\pgfpathlineto{\pgfqpoint{0.423041in}{1.249158in}}%
\pgfpathlineto{\pgfqpoint{0.415258in}{1.256851in}}%
\pgfpathlineto{\pgfqpoint{0.407385in}{1.266881in}}%
\pgfpathlineto{\pgfqpoint{0.404012in}{1.270462in}}%
\pgfpathlineto{\pgfqpoint{0.395464in}{1.284073in}}%
\pgfpathlineto{\pgfqpoint{0.391728in}{1.297481in}}%
\pgfpathlineto{\pgfqpoint{0.391641in}{1.297684in}}%
\pgfpathlineto{\pgfqpoint{0.391728in}{1.298491in}}%
\pgfpathlineto{\pgfqpoint{0.392619in}{1.311295in}}%
\pgfpathlineto{\pgfqpoint{0.398311in}{1.324907in}}%
\pgfpathlineto{\pgfqpoint{0.407385in}{1.336783in}}%
\pgfpathlineto{\pgfqpoint{0.408479in}{1.338518in}}%
\pgfpathlineto{\pgfqpoint{0.420505in}{1.352129in}}%
\pgfpathlineto{\pgfqpoint{0.423041in}{1.354485in}}%
\pgfpathlineto{\pgfqpoint{0.434728in}{1.365740in}}%
\pgfpathlineto{\pgfqpoint{0.438698in}{1.369213in}}%
\pgfpathlineto{\pgfqpoint{0.451183in}{1.379351in}}%
\pgfpathlineto{\pgfqpoint{0.454354in}{1.381958in}}%
\pgfpathlineto{\pgfqpoint{0.470011in}{1.392720in}}%
\pgfpathlineto{\pgfqpoint{0.470487in}{1.392962in}}%
\pgfpathlineto{\pgfqpoint{0.485668in}{1.402010in}}%
\pgfpathlineto{\pgfqpoint{0.498733in}{1.406573in}}%
\pgfpathlineto{\pgfqpoint{0.501324in}{1.407872in}}%
\pgfpathlineto{\pgfqpoint{0.516981in}{1.410160in}}%
\pgfpathlineto{\pgfqpoint{0.532637in}{1.406728in}}%
\pgfpathlineto{\pgfqpoint{0.532909in}{1.406573in}}%
\pgfpathlineto{\pgfqpoint{0.548294in}{1.400453in}}%
\pgfpathlineto{\pgfqpoint{0.560018in}{1.392962in}}%
\pgfpathlineto{\pgfqpoint{0.563950in}{1.390812in}}%
\pgfpathlineto{\pgfqpoint{0.579607in}{1.379451in}}%
\pgfpathlineto{\pgfqpoint{0.579726in}{1.379351in}}%
\pgfpathlineto{\pgfqpoint{0.595263in}{1.366442in}}%
\pgfpathlineto{\pgfqpoint{0.596064in}{1.365740in}}%
\pgfpathlineto{\pgfqpoint{0.610205in}{1.352129in}}%
\pgfpathlineto{\pgfqpoint{0.610920in}{1.351287in}}%
\pgfpathlineto{\pgfqpoint{0.622557in}{1.338518in}}%
\pgfpathlineto{\pgfqpoint{0.626577in}{1.332281in}}%
\pgfpathlineto{\pgfqpoint{0.632523in}{1.324907in}}%
\pgfpathlineto{\pgfqpoint{0.638608in}{1.311295in}}%
\pgfpathlineto{\pgfqpoint{0.639620in}{1.297684in}}%
\pgfpathlineto{\pgfqpoint{0.635567in}{1.284073in}}%
\pgfpathlineto{\pgfqpoint{0.626577in}{1.270677in}}%
\pgfpathlineto{\pgfqpoint{0.626462in}{1.270462in}}%
\pgfpathlineto{\pgfqpoint{0.615541in}{1.256851in}}%
\pgfpathlineto{\pgfqpoint{0.610920in}{1.252374in}}%
\pgfpathlineto{\pgfqpoint{0.602067in}{1.243240in}}%
\pgfpathlineto{\pgfqpoint{0.595263in}{1.237122in}}%
\pgfpathlineto{\pgfqpoint{0.586478in}{1.229629in}}%
\pgfpathlineto{\pgfqpoint{0.579607in}{1.223948in}}%
\pgfpathlineto{\pgfqpoint{0.568363in}{1.216018in}}%
\pgfpathlineto{\pgfqpoint{0.563950in}{1.212615in}}%
\pgfpathlineto{\pgfqpoint{0.548294in}{1.203435in}}%
\pgfpathlineto{\pgfqpoint{0.545510in}{1.202407in}}%
\pgfpathlineto{\pgfqpoint{0.532637in}{1.196177in}}%
\pgfpathlineto{\pgfqpoint{0.516981in}{1.193326in}}%
\pgfpathlineto{\pgfqpoint{0.501324in}{1.195226in}}%
\pgfpathlineto{\pgfqpoint{0.485668in}{1.201888in}}%
\pgfpathlineto{\pgfqpoint{0.484928in}{1.202407in}}%
\pgfpathclose%
\pgfpathmoveto{\pgfqpoint{0.795360in}{1.202407in}}%
\pgfpathlineto{\pgfqpoint{0.783142in}{1.208522in}}%
\pgfpathlineto{\pgfqpoint{0.772533in}{1.216018in}}%
\pgfpathlineto{\pgfqpoint{0.767486in}{1.219263in}}%
\pgfpathlineto{\pgfqpoint{0.754349in}{1.229629in}}%
\pgfpathlineto{\pgfqpoint{0.751829in}{1.231674in}}%
\pgfpathlineto{\pgfqpoint{0.738846in}{1.243240in}}%
\pgfpathlineto{\pgfqpoint{0.736173in}{1.245971in}}%
\pgfpathlineto{\pgfqpoint{0.725327in}{1.256851in}}%
\pgfpathlineto{\pgfqpoint{0.720516in}{1.263078in}}%
\pgfpathlineto{\pgfqpoint{0.713871in}{1.270462in}}%
\pgfpathlineto{\pgfqpoint{0.705812in}{1.284073in}}%
\pgfpathlineto{\pgfqpoint{0.704859in}{1.287660in}}%
\pgfpathlineto{\pgfqpoint{0.700997in}{1.297684in}}%
\pgfpathlineto{\pgfqpoint{0.702312in}{1.311295in}}%
\pgfpathlineto{\pgfqpoint{0.704859in}{1.315747in}}%
\pgfpathlineto{\pgfqpoint{0.708496in}{1.324907in}}%
\pgfpathlineto{\pgfqpoint{0.718350in}{1.338518in}}%
\pgfpathlineto{\pgfqpoint{0.720516in}{1.340679in}}%
\pgfpathlineto{\pgfqpoint{0.730422in}{1.352129in}}%
\pgfpathlineto{\pgfqpoint{0.736173in}{1.357550in}}%
\pgfpathlineto{\pgfqpoint{0.744737in}{1.365740in}}%
\pgfpathlineto{\pgfqpoint{0.751829in}{1.371906in}}%
\pgfpathlineto{\pgfqpoint{0.761249in}{1.379351in}}%
\pgfpathlineto{\pgfqpoint{0.767486in}{1.384350in}}%
\pgfpathlineto{\pgfqpoint{0.780656in}{1.392962in}}%
\pgfpathlineto{\pgfqpoint{0.783142in}{1.394845in}}%
\pgfpathlineto{\pgfqpoint{0.798799in}{1.403412in}}%
\pgfpathlineto{\pgfqpoint{0.809335in}{1.406573in}}%
\pgfpathlineto{\pgfqpoint{0.814455in}{1.408788in}}%
\pgfpathlineto{\pgfqpoint{0.830112in}{1.409931in}}%
\pgfpathlineto{\pgfqpoint{0.841643in}{1.406573in}}%
\pgfpathlineto{\pgfqpoint{0.845769in}{1.405745in}}%
\pgfpathlineto{\pgfqpoint{0.861425in}{1.398739in}}%
\pgfpathlineto{\pgfqpoint{0.869919in}{1.392962in}}%
\pgfpathlineto{\pgfqpoint{0.877082in}{1.388779in}}%
\pgfpathlineto{\pgfqpoint{0.889596in}{1.379351in}}%
\pgfpathlineto{\pgfqpoint{0.892738in}{1.377027in}}%
\pgfpathlineto{\pgfqpoint{0.906043in}{1.365740in}}%
\pgfpathlineto{\pgfqpoint{0.908395in}{1.363549in}}%
\pgfpathlineto{\pgfqpoint{0.920319in}{1.352129in}}%
\pgfpathlineto{\pgfqpoint{0.924051in}{1.347741in}}%
\pgfpathlineto{\pgfqpoint{0.932673in}{1.338518in}}%
\pgfpathlineto{\pgfqpoint{0.939708in}{1.327896in}}%
\pgfpathlineto{\pgfqpoint{0.942277in}{1.324907in}}%
\pgfpathlineto{\pgfqpoint{0.948842in}{1.311295in}}%
\pgfpathlineto{\pgfqpoint{0.949934in}{1.297684in}}%
\pgfpathlineto{\pgfqpoint{0.945561in}{1.284073in}}%
\pgfpathlineto{\pgfqpoint{0.939708in}{1.275888in}}%
\pgfpathlineto{\pgfqpoint{0.936737in}{1.270462in}}%
\pgfpathlineto{\pgfqpoint{0.925373in}{1.256851in}}%
\pgfpathlineto{\pgfqpoint{0.924051in}{1.255603in}}%
\pgfpathlineto{\pgfqpoint{0.912051in}{1.243240in}}%
\pgfpathlineto{\pgfqpoint{0.908395in}{1.239957in}}%
\pgfpathlineto{\pgfqpoint{0.896515in}{1.229629in}}%
\pgfpathlineto{\pgfqpoint{0.892738in}{1.226450in}}%
\pgfpathlineto{\pgfqpoint{0.878518in}{1.216018in}}%
\pgfpathlineto{\pgfqpoint{0.877082in}{1.214869in}}%
\pgfpathlineto{\pgfqpoint{0.861425in}{1.204990in}}%
\pgfpathlineto{\pgfqpoint{0.855184in}{1.202407in}}%
\pgfpathlineto{\pgfqpoint{0.845769in}{1.197318in}}%
\pgfpathlineto{\pgfqpoint{0.830112in}{1.193516in}}%
\pgfpathlineto{\pgfqpoint{0.814455in}{1.194466in}}%
\pgfpathlineto{\pgfqpoint{0.798799in}{1.200173in}}%
\pgfpathlineto{\pgfqpoint{0.795360in}{1.202407in}}%
\pgfpathclose%
\pgfpathmoveto{\pgfqpoint{1.105623in}{1.202407in}}%
\pgfpathlineto{\pgfqpoint{1.096274in}{1.206686in}}%
\pgfpathlineto{\pgfqpoint{1.082372in}{1.216018in}}%
\pgfpathlineto{\pgfqpoint{1.080617in}{1.217087in}}%
\pgfpathlineto{\pgfqpoint{1.064960in}{1.229053in}}%
\pgfpathlineto{\pgfqpoint{1.064286in}{1.229629in}}%
\pgfpathlineto{\pgfqpoint{1.049304in}{1.242852in}}%
\pgfpathlineto{\pgfqpoint{1.048870in}{1.243240in}}%
\pgfpathlineto{\pgfqpoint{1.035463in}{1.256851in}}%
\pgfpathlineto{\pgfqpoint{1.033647in}{1.259227in}}%
\pgfpathlineto{\pgfqpoint{1.023915in}{1.270462in}}%
\pgfpathlineto{\pgfqpoint{1.017991in}{1.280922in}}%
\pgfpathlineto{\pgfqpoint{1.015563in}{1.284073in}}%
\pgfpathlineto{\pgfqpoint{1.010794in}{1.297684in}}%
\pgfpathlineto{\pgfqpoint{1.011986in}{1.311295in}}%
\pgfpathlineto{\pgfqpoint{1.017991in}{1.322754in}}%
\pgfpathlineto{\pgfqpoint{1.018812in}{1.324907in}}%
\pgfpathlineto{\pgfqpoint{1.028167in}{1.338518in}}%
\pgfpathlineto{\pgfqpoint{1.033647in}{1.344199in}}%
\pgfpathlineto{\pgfqpoint{1.040435in}{1.352129in}}%
\pgfpathlineto{\pgfqpoint{1.049304in}{1.360576in}}%
\pgfpathlineto{\pgfqpoint{1.054765in}{1.365740in}}%
\pgfpathlineto{\pgfqpoint{1.064960in}{1.374512in}}%
\pgfpathlineto{\pgfqpoint{1.071275in}{1.379351in}}%
\pgfpathlineto{\pgfqpoint{1.080617in}{1.386625in}}%
\pgfpathlineto{\pgfqpoint{1.090844in}{1.392962in}}%
\pgfpathlineto{\pgfqpoint{1.096274in}{1.396870in}}%
\pgfpathlineto{\pgfqpoint{1.111930in}{1.404657in}}%
\pgfpathlineto{\pgfqpoint{1.119581in}{1.406573in}}%
\pgfpathlineto{\pgfqpoint{1.127587in}{1.409474in}}%
\pgfpathlineto{\pgfqpoint{1.143243in}{1.409474in}}%
\pgfpathlineto{\pgfqpoint{1.151249in}{1.406573in}}%
\pgfpathlineto{\pgfqpoint{1.158900in}{1.404657in}}%
\pgfpathlineto{\pgfqpoint{1.174556in}{1.396870in}}%
\pgfpathlineto{\pgfqpoint{1.179986in}{1.392962in}}%
\pgfpathlineto{\pgfqpoint{1.190213in}{1.386625in}}%
\pgfpathlineto{\pgfqpoint{1.199555in}{1.379351in}}%
\pgfpathlineto{\pgfqpoint{1.205870in}{1.374512in}}%
\pgfpathlineto{\pgfqpoint{1.216065in}{1.365740in}}%
\pgfpathlineto{\pgfqpoint{1.221526in}{1.360576in}}%
\pgfpathlineto{\pgfqpoint{1.230395in}{1.352129in}}%
\pgfpathlineto{\pgfqpoint{1.237183in}{1.344199in}}%
\pgfpathlineto{\pgfqpoint{1.242663in}{1.338518in}}%
\pgfpathlineto{\pgfqpoint{1.252018in}{1.324907in}}%
\pgfpathlineto{\pgfqpoint{1.252839in}{1.322754in}}%
\pgfpathlineto{\pgfqpoint{1.258844in}{1.311295in}}%
\pgfpathlineto{\pgfqpoint{1.260036in}{1.297684in}}%
\pgfpathlineto{\pgfqpoint{1.255267in}{1.284073in}}%
\pgfpathlineto{\pgfqpoint{1.252839in}{1.280922in}}%
\pgfpathlineto{\pgfqpoint{1.246915in}{1.270462in}}%
\pgfpathlineto{\pgfqpoint{1.237183in}{1.259227in}}%
\pgfpathlineto{\pgfqpoint{1.235367in}{1.256851in}}%
\pgfpathlineto{\pgfqpoint{1.221960in}{1.243240in}}%
\pgfpathlineto{\pgfqpoint{1.221526in}{1.242852in}}%
\pgfpathlineto{\pgfqpoint{1.206544in}{1.229629in}}%
\pgfpathlineto{\pgfqpoint{1.205870in}{1.229053in}}%
\pgfpathlineto{\pgfqpoint{1.190213in}{1.217087in}}%
\pgfpathlineto{\pgfqpoint{1.188458in}{1.216018in}}%
\pgfpathlineto{\pgfqpoint{1.174556in}{1.206686in}}%
\pgfpathlineto{\pgfqpoint{1.165207in}{1.202407in}}%
\pgfpathlineto{\pgfqpoint{1.158900in}{1.198650in}}%
\pgfpathlineto{\pgfqpoint{1.143243in}{1.193896in}}%
\pgfpathlineto{\pgfqpoint{1.127587in}{1.193896in}}%
\pgfpathlineto{\pgfqpoint{1.111930in}{1.198650in}}%
\pgfpathlineto{\pgfqpoint{1.105623in}{1.202407in}}%
\pgfpathclose%
\pgfpathmoveto{\pgfqpoint{1.415646in}{1.202407in}}%
\pgfpathlineto{\pgfqpoint{1.409405in}{1.204990in}}%
\pgfpathlineto{\pgfqpoint{1.393748in}{1.214869in}}%
\pgfpathlineto{\pgfqpoint{1.392312in}{1.216018in}}%
\pgfpathlineto{\pgfqpoint{1.378092in}{1.226450in}}%
\pgfpathlineto{\pgfqpoint{1.374315in}{1.229629in}}%
\pgfpathlineto{\pgfqpoint{1.362435in}{1.239957in}}%
\pgfpathlineto{\pgfqpoint{1.358779in}{1.243240in}}%
\pgfpathlineto{\pgfqpoint{1.346779in}{1.255603in}}%
\pgfpathlineto{\pgfqpoint{1.345457in}{1.256851in}}%
\pgfpathlineto{\pgfqpoint{1.334093in}{1.270462in}}%
\pgfpathlineto{\pgfqpoint{1.331122in}{1.275888in}}%
\pgfpathlineto{\pgfqpoint{1.325269in}{1.284073in}}%
\pgfpathlineto{\pgfqpoint{1.320896in}{1.297684in}}%
\pgfpathlineto{\pgfqpoint{1.321988in}{1.311295in}}%
\pgfpathlineto{\pgfqpoint{1.328553in}{1.324907in}}%
\pgfpathlineto{\pgfqpoint{1.331122in}{1.327896in}}%
\pgfpathlineto{\pgfqpoint{1.338157in}{1.338518in}}%
\pgfpathlineto{\pgfqpoint{1.346779in}{1.347741in}}%
\pgfpathlineto{\pgfqpoint{1.350511in}{1.352129in}}%
\pgfpathlineto{\pgfqpoint{1.362435in}{1.363549in}}%
\pgfpathlineto{\pgfqpoint{1.364787in}{1.365740in}}%
\pgfpathlineto{\pgfqpoint{1.378092in}{1.377027in}}%
\pgfpathlineto{\pgfqpoint{1.381234in}{1.379351in}}%
\pgfpathlineto{\pgfqpoint{1.393748in}{1.388779in}}%
\pgfpathlineto{\pgfqpoint{1.400911in}{1.392962in}}%
\pgfpathlineto{\pgfqpoint{1.409405in}{1.398739in}}%
\pgfpathlineto{\pgfqpoint{1.425061in}{1.405745in}}%
\pgfpathlineto{\pgfqpoint{1.429187in}{1.406573in}}%
\pgfpathlineto{\pgfqpoint{1.440718in}{1.409931in}}%
\pgfpathlineto{\pgfqpoint{1.456375in}{1.408788in}}%
\pgfpathlineto{\pgfqpoint{1.461495in}{1.406573in}}%
\pgfpathlineto{\pgfqpoint{1.472031in}{1.403412in}}%
\pgfpathlineto{\pgfqpoint{1.487688in}{1.394845in}}%
\pgfpathlineto{\pgfqpoint{1.490174in}{1.392962in}}%
\pgfpathlineto{\pgfqpoint{1.503344in}{1.384350in}}%
\pgfpathlineto{\pgfqpoint{1.509581in}{1.379351in}}%
\pgfpathlineto{\pgfqpoint{1.519001in}{1.371906in}}%
\pgfpathlineto{\pgfqpoint{1.526093in}{1.365740in}}%
\pgfpathlineto{\pgfqpoint{1.534657in}{1.357550in}}%
\pgfpathlineto{\pgfqpoint{1.540408in}{1.352129in}}%
\pgfpathlineto{\pgfqpoint{1.550314in}{1.340679in}}%
\pgfpathlineto{\pgfqpoint{1.552480in}{1.338518in}}%
\pgfpathlineto{\pgfqpoint{1.562334in}{1.324907in}}%
\pgfpathlineto{\pgfqpoint{1.565971in}{1.315747in}}%
\pgfpathlineto{\pgfqpoint{1.568518in}{1.311295in}}%
\pgfpathlineto{\pgfqpoint{1.569833in}{1.297684in}}%
\pgfpathlineto{\pgfqpoint{1.565971in}{1.287660in}}%
\pgfpathlineto{\pgfqpoint{1.565018in}{1.284073in}}%
\pgfpathlineto{\pgfqpoint{1.556959in}{1.270462in}}%
\pgfpathlineto{\pgfqpoint{1.550314in}{1.263078in}}%
\pgfpathlineto{\pgfqpoint{1.545503in}{1.256851in}}%
\pgfpathlineto{\pgfqpoint{1.534657in}{1.245971in}}%
\pgfpathlineto{\pgfqpoint{1.531984in}{1.243240in}}%
\pgfpathlineto{\pgfqpoint{1.519001in}{1.231674in}}%
\pgfpathlineto{\pgfqpoint{1.516481in}{1.229629in}}%
\pgfpathlineto{\pgfqpoint{1.503344in}{1.219263in}}%
\pgfpathlineto{\pgfqpoint{1.498297in}{1.216018in}}%
\pgfpathlineto{\pgfqpoint{1.487688in}{1.208522in}}%
\pgfpathlineto{\pgfqpoint{1.475470in}{1.202407in}}%
\pgfpathlineto{\pgfqpoint{1.472031in}{1.200173in}}%
\pgfpathlineto{\pgfqpoint{1.456375in}{1.194466in}}%
\pgfpathlineto{\pgfqpoint{1.440718in}{1.193516in}}%
\pgfpathlineto{\pgfqpoint{1.425061in}{1.197318in}}%
\pgfpathlineto{\pgfqpoint{1.415646in}{1.202407in}}%
\pgfpathclose%
\pgfpathmoveto{\pgfqpoint{1.725320in}{1.202407in}}%
\pgfpathlineto{\pgfqpoint{1.722536in}{1.203435in}}%
\pgfpathlineto{\pgfqpoint{1.706880in}{1.212615in}}%
\pgfpathlineto{\pgfqpoint{1.702467in}{1.216018in}}%
\pgfpathlineto{\pgfqpoint{1.691223in}{1.223948in}}%
\pgfpathlineto{\pgfqpoint{1.684352in}{1.229629in}}%
\pgfpathlineto{\pgfqpoint{1.675567in}{1.237122in}}%
\pgfpathlineto{\pgfqpoint{1.668763in}{1.243240in}}%
\pgfpathlineto{\pgfqpoint{1.659910in}{1.252374in}}%
\pgfpathlineto{\pgfqpoint{1.655289in}{1.256851in}}%
\pgfpathlineto{\pgfqpoint{1.644368in}{1.270462in}}%
\pgfpathlineto{\pgfqpoint{1.644253in}{1.270677in}}%
\pgfpathlineto{\pgfqpoint{1.635263in}{1.284073in}}%
\pgfpathlineto{\pgfqpoint{1.631210in}{1.297684in}}%
\pgfpathlineto{\pgfqpoint{1.632222in}{1.311295in}}%
\pgfpathlineto{\pgfqpoint{1.638307in}{1.324907in}}%
\pgfpathlineto{\pgfqpoint{1.644253in}{1.332281in}}%
\pgfpathlineto{\pgfqpoint{1.648273in}{1.338518in}}%
\pgfpathlineto{\pgfqpoint{1.659910in}{1.351287in}}%
\pgfpathlineto{\pgfqpoint{1.660625in}{1.352129in}}%
\pgfpathlineto{\pgfqpoint{1.674766in}{1.365740in}}%
\pgfpathlineto{\pgfqpoint{1.675567in}{1.366442in}}%
\pgfpathlineto{\pgfqpoint{1.691104in}{1.379351in}}%
\pgfpathlineto{\pgfqpoint{1.691223in}{1.379451in}}%
\pgfpathlineto{\pgfqpoint{1.706880in}{1.390812in}}%
\pgfpathlineto{\pgfqpoint{1.710812in}{1.392962in}}%
\pgfpathlineto{\pgfqpoint{1.722536in}{1.400453in}}%
\pgfpathlineto{\pgfqpoint{1.737921in}{1.406573in}}%
\pgfpathlineto{\pgfqpoint{1.738193in}{1.406728in}}%
\pgfpathlineto{\pgfqpoint{1.753849in}{1.410160in}}%
\pgfpathlineto{\pgfqpoint{1.769506in}{1.407872in}}%
\pgfpathlineto{\pgfqpoint{1.772097in}{1.406573in}}%
\pgfpathlineto{\pgfqpoint{1.785162in}{1.402010in}}%
\pgfpathlineto{\pgfqpoint{1.800343in}{1.392962in}}%
\pgfpathlineto{\pgfqpoint{1.800819in}{1.392720in}}%
\pgfpathlineto{\pgfqpoint{1.816476in}{1.381958in}}%
\pgfpathlineto{\pgfqpoint{1.819647in}{1.379351in}}%
\pgfpathlineto{\pgfqpoint{1.832132in}{1.369213in}}%
\pgfpathlineto{\pgfqpoint{1.836102in}{1.365740in}}%
\pgfpathlineto{\pgfqpoint{1.847789in}{1.354485in}}%
\pgfpathlineto{\pgfqpoint{1.850325in}{1.352129in}}%
\pgfpathlineto{\pgfqpoint{1.862351in}{1.338518in}}%
\pgfpathlineto{\pgfqpoint{1.863445in}{1.336783in}}%
\pgfpathlineto{\pgfqpoint{1.872519in}{1.324907in}}%
\pgfpathlineto{\pgfqpoint{1.878211in}{1.311295in}}%
\pgfpathlineto{\pgfqpoint{1.879102in}{1.298491in}}%
\pgfpathlineto{\pgfqpoint{1.879189in}{1.297684in}}%
\pgfpathlineto{\pgfqpoint{1.879102in}{1.297481in}}%
\pgfpathlineto{\pgfqpoint{1.875366in}{1.284073in}}%
\pgfpathlineto{\pgfqpoint{1.866818in}{1.270462in}}%
\pgfpathlineto{\pgfqpoint{1.863445in}{1.266881in}}%
\pgfpathlineto{\pgfqpoint{1.855572in}{1.256851in}}%
\pgfpathlineto{\pgfqpoint{1.847789in}{1.249158in}}%
\pgfpathlineto{\pgfqpoint{1.842036in}{1.243240in}}%
\pgfpathlineto{\pgfqpoint{1.832132in}{1.234359in}}%
\pgfpathlineto{\pgfqpoint{1.826458in}{1.229629in}}%
\pgfpathlineto{\pgfqpoint{1.816476in}{1.221551in}}%
\pgfpathlineto{\pgfqpoint{1.808279in}{1.216018in}}%
\pgfpathlineto{\pgfqpoint{1.800819in}{1.210499in}}%
\pgfpathlineto{\pgfqpoint{1.785902in}{1.202407in}}%
\pgfpathlineto{\pgfqpoint{1.785162in}{1.201888in}}%
\pgfpathlineto{\pgfqpoint{1.769506in}{1.195226in}}%
\pgfpathlineto{\pgfqpoint{1.753849in}{1.193326in}}%
\pgfpathlineto{\pgfqpoint{1.738193in}{1.196177in}}%
\pgfpathlineto{\pgfqpoint{1.725320in}{1.202407in}}%
\pgfpathclose%
\pgfpathmoveto{\pgfqpoint{0.653949in}{1.352129in}}%
\pgfpathlineto{\pgfqpoint{0.642233in}{1.357710in}}%
\pgfpathlineto{\pgfqpoint{0.631425in}{1.365740in}}%
\pgfpathlineto{\pgfqpoint{0.626577in}{1.369065in}}%
\pgfpathlineto{\pgfqpoint{0.614193in}{1.379351in}}%
\pgfpathlineto{\pgfqpoint{0.610920in}{1.382099in}}%
\pgfpathlineto{\pgfqpoint{0.598651in}{1.392962in}}%
\pgfpathlineto{\pgfqpoint{0.595263in}{1.396373in}}%
\pgfpathlineto{\pgfqpoint{0.584665in}{1.406573in}}%
\pgfpathlineto{\pgfqpoint{0.579607in}{1.413269in}}%
\pgfpathlineto{\pgfqpoint{0.573649in}{1.420184in}}%
\pgfpathlineto{\pgfqpoint{0.568405in}{1.433795in}}%
\pgfpathlineto{\pgfqpoint{0.570693in}{1.447407in}}%
\pgfpathlineto{\pgfqpoint{0.579607in}{1.460521in}}%
\pgfpathlineto{\pgfqpoint{0.579906in}{1.461018in}}%
\pgfpathlineto{\pgfqpoint{0.592777in}{1.474629in}}%
\pgfpathlineto{\pgfqpoint{0.595263in}{1.476791in}}%
\pgfpathlineto{\pgfqpoint{0.607768in}{1.488240in}}%
\pgfpathlineto{\pgfqpoint{0.610920in}{1.490980in}}%
\pgfpathlineto{\pgfqpoint{0.624089in}{1.501851in}}%
\pgfpathlineto{\pgfqpoint{0.626577in}{1.504013in}}%
\pgfpathlineto{\pgfqpoint{0.642233in}{1.515202in}}%
\pgfpathlineto{\pgfqpoint{0.642804in}{1.515462in}}%
\pgfpathlineto{\pgfqpoint{0.657890in}{1.523211in}}%
\pgfpathlineto{\pgfqpoint{0.673546in}{1.525200in}}%
\pgfpathlineto{\pgfqpoint{0.689203in}{1.520641in}}%
\pgfpathlineto{\pgfqpoint{0.697158in}{1.515462in}}%
\pgfpathlineto{\pgfqpoint{0.704859in}{1.511065in}}%
\pgfpathlineto{\pgfqpoint{0.716592in}{1.501851in}}%
\pgfpathlineto{\pgfqpoint{0.720516in}{1.498906in}}%
\pgfpathlineto{\pgfqpoint{0.733011in}{1.488240in}}%
\pgfpathlineto{\pgfqpoint{0.736173in}{1.485395in}}%
\pgfpathlineto{\pgfqpoint{0.748004in}{1.474629in}}%
\pgfpathlineto{\pgfqpoint{0.751829in}{1.470414in}}%
\pgfpathlineto{\pgfqpoint{0.761066in}{1.461018in}}%
\pgfpathlineto{\pgfqpoint{0.767486in}{1.450833in}}%
\pgfpathlineto{\pgfqpoint{0.770040in}{1.447407in}}%
\pgfpathlineto{\pgfqpoint{0.772488in}{1.433795in}}%
\pgfpathlineto{\pgfqpoint{0.767486in}{1.421686in}}%
\pgfpathlineto{\pgfqpoint{0.766953in}{1.420184in}}%
\pgfpathlineto{\pgfqpoint{0.756202in}{1.406573in}}%
\pgfpathlineto{\pgfqpoint{0.751829in}{1.402570in}}%
\pgfpathlineto{\pgfqpoint{0.742198in}{1.392962in}}%
\pgfpathlineto{\pgfqpoint{0.736173in}{1.387680in}}%
\pgfpathlineto{\pgfqpoint{0.726592in}{1.379351in}}%
\pgfpathlineto{\pgfqpoint{0.720516in}{1.374113in}}%
\pgfpathlineto{\pgfqpoint{0.709464in}{1.365740in}}%
\pgfpathlineto{\pgfqpoint{0.704859in}{1.361938in}}%
\pgfpathlineto{\pgfqpoint{0.689203in}{1.352592in}}%
\pgfpathlineto{\pgfqpoint{0.687476in}{1.352129in}}%
\pgfpathlineto{\pgfqpoint{0.673546in}{1.347780in}}%
\pgfpathlineto{\pgfqpoint{0.657890in}{1.349908in}}%
\pgfpathlineto{\pgfqpoint{0.653949in}{1.352129in}}%
\pgfpathclose%
\pgfpathmoveto{\pgfqpoint{0.964376in}{1.352129in}}%
\pgfpathlineto{\pgfqpoint{0.955364in}{1.355823in}}%
\pgfpathlineto{\pgfqpoint{0.941092in}{1.365740in}}%
\pgfpathlineto{\pgfqpoint{0.939708in}{1.366631in}}%
\pgfpathlineto{\pgfqpoint{0.924051in}{1.379290in}}%
\pgfpathlineto{\pgfqpoint{0.923982in}{1.379351in}}%
\pgfpathlineto{\pgfqpoint{0.908582in}{1.392962in}}%
\pgfpathlineto{\pgfqpoint{0.908395in}{1.393150in}}%
\pgfpathlineto{\pgfqpoint{0.894711in}{1.406573in}}%
\pgfpathlineto{\pgfqpoint{0.892738in}{1.409231in}}%
\pgfpathlineto{\pgfqpoint{0.883667in}{1.420184in}}%
\pgfpathlineto{\pgfqpoint{0.878559in}{1.433795in}}%
\pgfpathlineto{\pgfqpoint{0.880787in}{1.447407in}}%
\pgfpathlineto{\pgfqpoint{0.889787in}{1.461018in}}%
\pgfpathlineto{\pgfqpoint{0.892738in}{1.463828in}}%
\pgfpathlineto{\pgfqpoint{0.902784in}{1.474629in}}%
\pgfpathlineto{\pgfqpoint{0.908395in}{1.479602in}}%
\pgfpathlineto{\pgfqpoint{0.917843in}{1.488240in}}%
\pgfpathlineto{\pgfqpoint{0.924051in}{1.493630in}}%
\pgfpathlineto{\pgfqpoint{0.934268in}{1.501851in}}%
\pgfpathlineto{\pgfqpoint{0.939708in}{1.506452in}}%
\pgfpathlineto{\pgfqpoint{0.953123in}{1.515462in}}%
\pgfpathlineto{\pgfqpoint{0.955364in}{1.517192in}}%
\pgfpathlineto{\pgfqpoint{0.971021in}{1.524133in}}%
\pgfpathlineto{\pgfqpoint{0.986678in}{1.524798in}}%
\pgfpathlineto{\pgfqpoint{1.002334in}{1.519019in}}%
\pgfpathlineto{\pgfqpoint{1.007317in}{1.515462in}}%
\pgfpathlineto{\pgfqpoint{1.017991in}{1.508809in}}%
\pgfpathlineto{\pgfqpoint{1.026499in}{1.501851in}}%
\pgfpathlineto{\pgfqpoint{1.033647in}{1.496276in}}%
\pgfpathlineto{\pgfqpoint{1.042961in}{1.488240in}}%
\pgfpathlineto{\pgfqpoint{1.049304in}{1.482473in}}%
\pgfpathlineto{\pgfqpoint{1.058021in}{1.474629in}}%
\pgfpathlineto{\pgfqpoint{1.064960in}{1.467062in}}%
\pgfpathlineto{\pgfqpoint{1.071089in}{1.461018in}}%
\pgfpathlineto{\pgfqpoint{1.079896in}{1.447407in}}%
\pgfpathlineto{\pgfqpoint{1.080617in}{1.442881in}}%
\pgfpathlineto{\pgfqpoint{1.082326in}{1.433795in}}%
\pgfpathlineto{\pgfqpoint{1.080617in}{1.429876in}}%
\pgfpathlineto{\pgfqpoint{1.077078in}{1.420184in}}%
\pgfpathlineto{\pgfqpoint{1.066140in}{1.406573in}}%
\pgfpathlineto{\pgfqpoint{1.064960in}{1.405526in}}%
\pgfpathlineto{\pgfqpoint{1.052234in}{1.392962in}}%
\pgfpathlineto{\pgfqpoint{1.049304in}{1.390422in}}%
\pgfpathlineto{\pgfqpoint{1.036697in}{1.379351in}}%
\pgfpathlineto{\pgfqpoint{1.033647in}{1.376693in}}%
\pgfpathlineto{\pgfqpoint{1.019731in}{1.365740in}}%
\pgfpathlineto{\pgfqpoint{1.017991in}{1.364244in}}%
\pgfpathlineto{\pgfqpoint{1.002334in}{1.354112in}}%
\pgfpathlineto{\pgfqpoint{0.996509in}{1.352129in}}%
\pgfpathlineto{\pgfqpoint{0.986678in}{1.348210in}}%
\pgfpathlineto{\pgfqpoint{0.971021in}{1.348922in}}%
\pgfpathlineto{\pgfqpoint{0.964376in}{1.352129in}}%
\pgfpathclose%
\pgfpathmoveto{\pgfqpoint{1.274321in}{1.352129in}}%
\pgfpathlineto{\pgfqpoint{1.268496in}{1.354112in}}%
\pgfpathlineto{\pgfqpoint{1.252839in}{1.364244in}}%
\pgfpathlineto{\pgfqpoint{1.251099in}{1.365740in}}%
\pgfpathlineto{\pgfqpoint{1.237183in}{1.376693in}}%
\pgfpathlineto{\pgfqpoint{1.234133in}{1.379351in}}%
\pgfpathlineto{\pgfqpoint{1.221526in}{1.390422in}}%
\pgfpathlineto{\pgfqpoint{1.218596in}{1.392962in}}%
\pgfpathlineto{\pgfqpoint{1.205870in}{1.405526in}}%
\pgfpathlineto{\pgfqpoint{1.204690in}{1.406573in}}%
\pgfpathlineto{\pgfqpoint{1.193752in}{1.420184in}}%
\pgfpathlineto{\pgfqpoint{1.190213in}{1.429876in}}%
\pgfpathlineto{\pgfqpoint{1.188504in}{1.433795in}}%
\pgfpathlineto{\pgfqpoint{1.190213in}{1.442881in}}%
\pgfpathlineto{\pgfqpoint{1.190934in}{1.447407in}}%
\pgfpathlineto{\pgfqpoint{1.199741in}{1.461018in}}%
\pgfpathlineto{\pgfqpoint{1.205870in}{1.467062in}}%
\pgfpathlineto{\pgfqpoint{1.212809in}{1.474629in}}%
\pgfpathlineto{\pgfqpoint{1.221526in}{1.482473in}}%
\pgfpathlineto{\pgfqpoint{1.227869in}{1.488240in}}%
\pgfpathlineto{\pgfqpoint{1.237183in}{1.496276in}}%
\pgfpathlineto{\pgfqpoint{1.244331in}{1.501851in}}%
\pgfpathlineto{\pgfqpoint{1.252839in}{1.508809in}}%
\pgfpathlineto{\pgfqpoint{1.263513in}{1.515462in}}%
\pgfpathlineto{\pgfqpoint{1.268496in}{1.519019in}}%
\pgfpathlineto{\pgfqpoint{1.284152in}{1.524798in}}%
\pgfpathlineto{\pgfqpoint{1.299809in}{1.524133in}}%
\pgfpathlineto{\pgfqpoint{1.315466in}{1.517192in}}%
\pgfpathlineto{\pgfqpoint{1.317707in}{1.515462in}}%
\pgfpathlineto{\pgfqpoint{1.331122in}{1.506452in}}%
\pgfpathlineto{\pgfqpoint{1.336562in}{1.501851in}}%
\pgfpathlineto{\pgfqpoint{1.346779in}{1.493630in}}%
\pgfpathlineto{\pgfqpoint{1.352987in}{1.488240in}}%
\pgfpathlineto{\pgfqpoint{1.362435in}{1.479602in}}%
\pgfpathlineto{\pgfqpoint{1.368046in}{1.474629in}}%
\pgfpathlineto{\pgfqpoint{1.378092in}{1.463828in}}%
\pgfpathlineto{\pgfqpoint{1.381043in}{1.461018in}}%
\pgfpathlineto{\pgfqpoint{1.390043in}{1.447407in}}%
\pgfpathlineto{\pgfqpoint{1.392271in}{1.433795in}}%
\pgfpathlineto{\pgfqpoint{1.387163in}{1.420184in}}%
\pgfpathlineto{\pgfqpoint{1.378092in}{1.409231in}}%
\pgfpathlineto{\pgfqpoint{1.376119in}{1.406573in}}%
\pgfpathlineto{\pgfqpoint{1.362435in}{1.393150in}}%
\pgfpathlineto{\pgfqpoint{1.362248in}{1.392962in}}%
\pgfpathlineto{\pgfqpoint{1.346848in}{1.379351in}}%
\pgfpathlineto{\pgfqpoint{1.346779in}{1.379290in}}%
\pgfpathlineto{\pgfqpoint{1.331122in}{1.366631in}}%
\pgfpathlineto{\pgfqpoint{1.329738in}{1.365740in}}%
\pgfpathlineto{\pgfqpoint{1.315466in}{1.355823in}}%
\pgfpathlineto{\pgfqpoint{1.306454in}{1.352129in}}%
\pgfpathlineto{\pgfqpoint{1.299809in}{1.348922in}}%
\pgfpathlineto{\pgfqpoint{1.284152in}{1.348210in}}%
\pgfpathlineto{\pgfqpoint{1.274321in}{1.352129in}}%
\pgfpathclose%
\pgfpathmoveto{\pgfqpoint{1.583354in}{1.352129in}}%
\pgfpathlineto{\pgfqpoint{1.581627in}{1.352592in}}%
\pgfpathlineto{\pgfqpoint{1.565971in}{1.361938in}}%
\pgfpathlineto{\pgfqpoint{1.561366in}{1.365740in}}%
\pgfpathlineto{\pgfqpoint{1.550314in}{1.374113in}}%
\pgfpathlineto{\pgfqpoint{1.544238in}{1.379351in}}%
\pgfpathlineto{\pgfqpoint{1.534657in}{1.387680in}}%
\pgfpathlineto{\pgfqpoint{1.528632in}{1.392962in}}%
\pgfpathlineto{\pgfqpoint{1.519001in}{1.402570in}}%
\pgfpathlineto{\pgfqpoint{1.514628in}{1.406573in}}%
\pgfpathlineto{\pgfqpoint{1.503877in}{1.420184in}}%
\pgfpathlineto{\pgfqpoint{1.503344in}{1.421686in}}%
\pgfpathlineto{\pgfqpoint{1.498342in}{1.433795in}}%
\pgfpathlineto{\pgfqpoint{1.500790in}{1.447407in}}%
\pgfpathlineto{\pgfqpoint{1.503344in}{1.450833in}}%
\pgfpathlineto{\pgfqpoint{1.509764in}{1.461018in}}%
\pgfpathlineto{\pgfqpoint{1.519001in}{1.470414in}}%
\pgfpathlineto{\pgfqpoint{1.522826in}{1.474629in}}%
\pgfpathlineto{\pgfqpoint{1.534657in}{1.485395in}}%
\pgfpathlineto{\pgfqpoint{1.537819in}{1.488240in}}%
\pgfpathlineto{\pgfqpoint{1.550314in}{1.498906in}}%
\pgfpathlineto{\pgfqpoint{1.554238in}{1.501851in}}%
\pgfpathlineto{\pgfqpoint{1.565971in}{1.511065in}}%
\pgfpathlineto{\pgfqpoint{1.573672in}{1.515462in}}%
\pgfpathlineto{\pgfqpoint{1.581627in}{1.520641in}}%
\pgfpathlineto{\pgfqpoint{1.597284in}{1.525200in}}%
\pgfpathlineto{\pgfqpoint{1.612940in}{1.523211in}}%
\pgfpathlineto{\pgfqpoint{1.628026in}{1.515462in}}%
\pgfpathlineto{\pgfqpoint{1.628597in}{1.515202in}}%
\pgfpathlineto{\pgfqpoint{1.644253in}{1.504013in}}%
\pgfpathlineto{\pgfqpoint{1.646741in}{1.501851in}}%
\pgfpathlineto{\pgfqpoint{1.659910in}{1.490980in}}%
\pgfpathlineto{\pgfqpoint{1.663062in}{1.488240in}}%
\pgfpathlineto{\pgfqpoint{1.675567in}{1.476791in}}%
\pgfpathlineto{\pgfqpoint{1.678053in}{1.474629in}}%
\pgfpathlineto{\pgfqpoint{1.690924in}{1.461018in}}%
\pgfpathlineto{\pgfqpoint{1.691223in}{1.460521in}}%
\pgfpathlineto{\pgfqpoint{1.700137in}{1.447407in}}%
\pgfpathlineto{\pgfqpoint{1.702425in}{1.433795in}}%
\pgfpathlineto{\pgfqpoint{1.697181in}{1.420184in}}%
\pgfpathlineto{\pgfqpoint{1.691223in}{1.413269in}}%
\pgfpathlineto{\pgfqpoint{1.686165in}{1.406573in}}%
\pgfpathlineto{\pgfqpoint{1.675567in}{1.396373in}}%
\pgfpathlineto{\pgfqpoint{1.672179in}{1.392962in}}%
\pgfpathlineto{\pgfqpoint{1.659910in}{1.382099in}}%
\pgfpathlineto{\pgfqpoint{1.656637in}{1.379351in}}%
\pgfpathlineto{\pgfqpoint{1.644253in}{1.369065in}}%
\pgfpathlineto{\pgfqpoint{1.639405in}{1.365740in}}%
\pgfpathlineto{\pgfqpoint{1.628597in}{1.357710in}}%
\pgfpathlineto{\pgfqpoint{1.616881in}{1.352129in}}%
\pgfpathlineto{\pgfqpoint{1.612940in}{1.349908in}}%
\pgfpathlineto{\pgfqpoint{1.597284in}{1.347780in}}%
\pgfpathlineto{\pgfqpoint{1.583354in}{1.352129in}}%
\pgfpathclose%
\pgfpathmoveto{\pgfqpoint{0.480229in}{1.474629in}}%
\pgfpathlineto{\pgfqpoint{0.470011in}{1.480023in}}%
\pgfpathlineto{\pgfqpoint{0.458618in}{1.488240in}}%
\pgfpathlineto{\pgfqpoint{0.454354in}{1.491113in}}%
\pgfpathlineto{\pgfqpoint{0.441068in}{1.501851in}}%
\pgfpathlineto{\pgfqpoint{0.438698in}{1.503865in}}%
\pgfpathlineto{\pgfqpoint{0.425993in}{1.515462in}}%
\pgfpathlineto{\pgfqpoint{0.423041in}{1.518621in}}%
\pgfpathlineto{\pgfqpoint{0.412851in}{1.529073in}}%
\pgfpathlineto{\pgfqpoint{0.407385in}{1.536509in}}%
\pgfpathlineto{\pgfqpoint{0.401921in}{1.542684in}}%
\pgfpathlineto{\pgfqpoint{0.394325in}{1.556295in}}%
\pgfpathlineto{\pgfqpoint{0.391728in}{1.568720in}}%
\pgfpathlineto{\pgfqpoint{0.391346in}{1.569907in}}%
\pgfpathlineto{\pgfqpoint{0.391728in}{1.571680in}}%
\pgfpathlineto{\pgfqpoint{0.393377in}{1.583518in}}%
\pgfpathlineto{\pgfqpoint{0.400021in}{1.597129in}}%
\pgfpathlineto{\pgfqpoint{0.407385in}{1.606051in}}%
\pgfpathlineto{\pgfqpoint{0.410591in}{1.610740in}}%
\pgfpathlineto{\pgfqpoint{0.423041in}{1.624050in}}%
\pgfpathlineto{\pgfqpoint{0.423310in}{1.624351in}}%
\pgfpathlineto{\pgfqpoint{0.437849in}{1.637962in}}%
\pgfpathlineto{\pgfqpoint{0.438698in}{1.638700in}}%
\pgfpathlineto{\pgfqpoint{0.454354in}{1.651339in}}%
\pgfpathlineto{\pgfqpoint{0.454700in}{1.651573in}}%
\pgfpathlineto{\pgfqpoint{0.470011in}{1.662397in}}%
\pgfpathlineto{\pgfqpoint{0.475405in}{1.665184in}}%
\pgfpathlineto{\pgfqpoint{0.485668in}{1.671586in}}%
\pgfpathlineto{\pgfqpoint{0.501324in}{1.677362in}}%
\pgfpathlineto{\pgfqpoint{0.514940in}{1.678795in}}%
\pgfpathlineto{\pgfqpoint{0.516981in}{1.679127in}}%
\pgfpathlineto{\pgfqpoint{0.518346in}{1.678795in}}%
\pgfpathlineto{\pgfqpoint{0.532637in}{1.676537in}}%
\pgfpathlineto{\pgfqpoint{0.548294in}{1.669934in}}%
\pgfpathlineto{\pgfqpoint{0.555398in}{1.665184in}}%
\pgfpathlineto{\pgfqpoint{0.563950in}{1.660432in}}%
\pgfpathlineto{\pgfqpoint{0.575973in}{1.651573in}}%
\pgfpathlineto{\pgfqpoint{0.579607in}{1.649007in}}%
\pgfpathlineto{\pgfqpoint{0.592947in}{1.637962in}}%
\pgfpathlineto{\pgfqpoint{0.595263in}{1.635902in}}%
\pgfpathlineto{\pgfqpoint{0.607615in}{1.624351in}}%
\pgfpathlineto{\pgfqpoint{0.610920in}{1.620645in}}%
\pgfpathlineto{\pgfqpoint{0.620371in}{1.610740in}}%
\pgfpathlineto{\pgfqpoint{0.626577in}{1.601857in}}%
\pgfpathlineto{\pgfqpoint{0.630695in}{1.597129in}}%
\pgfpathlineto{\pgfqpoint{0.637797in}{1.583518in}}%
\pgfpathlineto{\pgfqpoint{0.639823in}{1.569907in}}%
\pgfpathlineto{\pgfqpoint{0.636784in}{1.556295in}}%
\pgfpathlineto{\pgfqpoint{0.628663in}{1.542684in}}%
\pgfpathlineto{\pgfqpoint{0.626577in}{1.540450in}}%
\pgfpathlineto{\pgfqpoint{0.618032in}{1.529073in}}%
\pgfpathlineto{\pgfqpoint{0.610920in}{1.521922in}}%
\pgfpathlineto{\pgfqpoint{0.604900in}{1.515462in}}%
\pgfpathlineto{\pgfqpoint{0.595263in}{1.506641in}}%
\pgfpathlineto{\pgfqpoint{0.589753in}{1.501851in}}%
\pgfpathlineto{\pgfqpoint{0.579607in}{1.493473in}}%
\pgfpathlineto{\pgfqpoint{0.572176in}{1.488240in}}%
\pgfpathlineto{\pgfqpoint{0.563950in}{1.482057in}}%
\pgfpathlineto{\pgfqpoint{0.550864in}{1.474629in}}%
\pgfpathlineto{\pgfqpoint{0.548294in}{1.472815in}}%
\pgfpathlineto{\pgfqpoint{0.532637in}{1.465755in}}%
\pgfpathlineto{\pgfqpoint{0.516981in}{1.463113in}}%
\pgfpathlineto{\pgfqpoint{0.501324in}{1.464874in}}%
\pgfpathlineto{\pgfqpoint{0.485668in}{1.471049in}}%
\pgfpathlineto{\pgfqpoint{0.480229in}{1.474629in}}%
\pgfpathclose%
\pgfpathmoveto{\pgfqpoint{0.790316in}{1.474629in}}%
\pgfpathlineto{\pgfqpoint{0.783142in}{1.478124in}}%
\pgfpathlineto{\pgfqpoint{0.768454in}{1.488240in}}%
\pgfpathlineto{\pgfqpoint{0.767486in}{1.488861in}}%
\pgfpathlineto{\pgfqpoint{0.751829in}{1.501155in}}%
\pgfpathlineto{\pgfqpoint{0.751021in}{1.501851in}}%
\pgfpathlineto{\pgfqpoint{0.736173in}{1.515358in}}%
\pgfpathlineto{\pgfqpoint{0.736057in}{1.515462in}}%
\pgfpathlineto{\pgfqpoint{0.722990in}{1.529073in}}%
\pgfpathlineto{\pgfqpoint{0.720516in}{1.532491in}}%
\pgfpathlineto{\pgfqpoint{0.711900in}{1.542684in}}%
\pgfpathlineto{\pgfqpoint{0.704859in}{1.556059in}}%
\pgfpathlineto{\pgfqpoint{0.704681in}{1.556295in}}%
\pgfpathlineto{\pgfqpoint{0.700734in}{1.569907in}}%
\pgfpathlineto{\pgfqpoint{0.703365in}{1.583518in}}%
\pgfpathlineto{\pgfqpoint{0.704859in}{1.585770in}}%
\pgfpathlineto{\pgfqpoint{0.710108in}{1.597129in}}%
\pgfpathlineto{\pgfqpoint{0.720516in}{1.610326in}}%
\pgfpathlineto{\pgfqpoint{0.720795in}{1.610740in}}%
\pgfpathlineto{\pgfqpoint{0.733174in}{1.624351in}}%
\pgfpathlineto{\pgfqpoint{0.736173in}{1.627108in}}%
\pgfpathlineto{\pgfqpoint{0.747834in}{1.637962in}}%
\pgfpathlineto{\pgfqpoint{0.751829in}{1.641413in}}%
\pgfpathlineto{\pgfqpoint{0.764776in}{1.651573in}}%
\pgfpathlineto{\pgfqpoint{0.767486in}{1.653778in}}%
\pgfpathlineto{\pgfqpoint{0.783142in}{1.664233in}}%
\pgfpathlineto{\pgfqpoint{0.785138in}{1.665184in}}%
\pgfpathlineto{\pgfqpoint{0.798799in}{1.673072in}}%
\pgfpathlineto{\pgfqpoint{0.814455in}{1.678021in}}%
\pgfpathlineto{\pgfqpoint{0.829184in}{1.678795in}}%
\pgfpathlineto{\pgfqpoint{0.830112in}{1.678871in}}%
\pgfpathlineto{\pgfqpoint{0.830346in}{1.678795in}}%
\pgfpathlineto{\pgfqpoint{0.845769in}{1.675548in}}%
\pgfpathlineto{\pgfqpoint{0.861425in}{1.668117in}}%
\pgfpathlineto{\pgfqpoint{0.865545in}{1.665184in}}%
\pgfpathlineto{\pgfqpoint{0.877082in}{1.658339in}}%
\pgfpathlineto{\pgfqpoint{0.885930in}{1.651573in}}%
\pgfpathlineto{\pgfqpoint{0.892738in}{1.646572in}}%
\pgfpathlineto{\pgfqpoint{0.902954in}{1.637962in}}%
\pgfpathlineto{\pgfqpoint{0.908395in}{1.633029in}}%
\pgfpathlineto{\pgfqpoint{0.917687in}{1.624351in}}%
\pgfpathlineto{\pgfqpoint{0.924051in}{1.617225in}}%
\pgfpathlineto{\pgfqpoint{0.930399in}{1.610740in}}%
\pgfpathlineto{\pgfqpoint{0.939708in}{1.597772in}}%
\pgfpathlineto{\pgfqpoint{0.940304in}{1.597129in}}%
\pgfpathlineto{\pgfqpoint{0.947967in}{1.583518in}}%
\pgfpathlineto{\pgfqpoint{0.950153in}{1.569907in}}%
\pgfpathlineto{\pgfqpoint{0.946874in}{1.556295in}}%
\pgfpathlineto{\pgfqpoint{0.939708in}{1.545105in}}%
\pgfpathlineto{\pgfqpoint{0.938525in}{1.542684in}}%
\pgfpathlineto{\pgfqpoint{0.927965in}{1.529073in}}%
\pgfpathlineto{\pgfqpoint{0.924051in}{1.525237in}}%
\pgfpathlineto{\pgfqpoint{0.914930in}{1.515462in}}%
\pgfpathlineto{\pgfqpoint{0.908395in}{1.509489in}}%
\pgfpathlineto{\pgfqpoint{0.899775in}{1.501851in}}%
\pgfpathlineto{\pgfqpoint{0.892738in}{1.495936in}}%
\pgfpathlineto{\pgfqpoint{0.882232in}{1.488240in}}%
\pgfpathlineto{\pgfqpoint{0.877082in}{1.484223in}}%
\pgfpathlineto{\pgfqpoint{0.861425in}{1.474728in}}%
\pgfpathlineto{\pgfqpoint{0.861178in}{1.474629in}}%
\pgfpathlineto{\pgfqpoint{0.845769in}{1.466813in}}%
\pgfpathlineto{\pgfqpoint{0.830112in}{1.463289in}}%
\pgfpathlineto{\pgfqpoint{0.814455in}{1.464169in}}%
\pgfpathlineto{\pgfqpoint{0.798799in}{1.469460in}}%
\pgfpathlineto{\pgfqpoint{0.790316in}{1.474629in}}%
\pgfpathclose%
\pgfpathmoveto{\pgfqpoint{1.100157in}{1.474629in}}%
\pgfpathlineto{\pgfqpoint{1.096274in}{1.476358in}}%
\pgfpathlineto{\pgfqpoint{1.080617in}{1.486519in}}%
\pgfpathlineto{\pgfqpoint{1.078482in}{1.488240in}}%
\pgfpathlineto{\pgfqpoint{1.064960in}{1.498499in}}%
\pgfpathlineto{\pgfqpoint{1.061028in}{1.501851in}}%
\pgfpathlineto{\pgfqpoint{1.049304in}{1.512398in}}%
\pgfpathlineto{\pgfqpoint{1.045933in}{1.515462in}}%
\pgfpathlineto{\pgfqpoint{1.033647in}{1.528548in}}%
\pgfpathlineto{\pgfqpoint{1.033094in}{1.529073in}}%
\pgfpathlineto{\pgfqpoint{1.022043in}{1.542684in}}%
\pgfpathlineto{\pgfqpoint{1.017991in}{1.550701in}}%
\pgfpathlineto{\pgfqpoint{1.014131in}{1.556295in}}%
\pgfpathlineto{\pgfqpoint{1.010556in}{1.569907in}}%
\pgfpathlineto{\pgfqpoint{1.012939in}{1.583518in}}%
\pgfpathlineto{\pgfqpoint{1.017991in}{1.591830in}}%
\pgfpathlineto{\pgfqpoint{1.020342in}{1.597129in}}%
\pgfpathlineto{\pgfqpoint{1.030547in}{1.610740in}}%
\pgfpathlineto{\pgfqpoint{1.033647in}{1.613809in}}%
\pgfpathlineto{\pgfqpoint{1.043120in}{1.624351in}}%
\pgfpathlineto{\pgfqpoint{1.049304in}{1.630094in}}%
\pgfpathlineto{\pgfqpoint{1.057851in}{1.637962in}}%
\pgfpathlineto{\pgfqpoint{1.064960in}{1.644039in}}%
\pgfpathlineto{\pgfqpoint{1.074863in}{1.651573in}}%
\pgfpathlineto{\pgfqpoint{1.080617in}{1.656120in}}%
\pgfpathlineto{\pgfqpoint{1.095013in}{1.665184in}}%
\pgfpathlineto{\pgfqpoint{1.096274in}{1.666134in}}%
\pgfpathlineto{\pgfqpoint{1.111930in}{1.674393in}}%
\pgfpathlineto{\pgfqpoint{1.127587in}{1.678515in}}%
\pgfpathlineto{\pgfqpoint{1.143243in}{1.678515in}}%
\pgfpathlineto{\pgfqpoint{1.158900in}{1.674393in}}%
\pgfpathlineto{\pgfqpoint{1.174556in}{1.666134in}}%
\pgfpathlineto{\pgfqpoint{1.175817in}{1.665184in}}%
\pgfpathlineto{\pgfqpoint{1.190213in}{1.656120in}}%
\pgfpathlineto{\pgfqpoint{1.195967in}{1.651573in}}%
\pgfpathlineto{\pgfqpoint{1.205870in}{1.644039in}}%
\pgfpathlineto{\pgfqpoint{1.212979in}{1.637962in}}%
\pgfpathlineto{\pgfqpoint{1.221526in}{1.630094in}}%
\pgfpathlineto{\pgfqpoint{1.227710in}{1.624351in}}%
\pgfpathlineto{\pgfqpoint{1.237183in}{1.613809in}}%
\pgfpathlineto{\pgfqpoint{1.240283in}{1.610740in}}%
\pgfpathlineto{\pgfqpoint{1.250488in}{1.597129in}}%
\pgfpathlineto{\pgfqpoint{1.252839in}{1.591830in}}%
\pgfpathlineto{\pgfqpoint{1.257891in}{1.583518in}}%
\pgfpathlineto{\pgfqpoint{1.260274in}{1.569907in}}%
\pgfpathlineto{\pgfqpoint{1.256699in}{1.556295in}}%
\pgfpathlineto{\pgfqpoint{1.252839in}{1.550701in}}%
\pgfpathlineto{\pgfqpoint{1.248787in}{1.542684in}}%
\pgfpathlineto{\pgfqpoint{1.237736in}{1.529073in}}%
\pgfpathlineto{\pgfqpoint{1.237183in}{1.528548in}}%
\pgfpathlineto{\pgfqpoint{1.224897in}{1.515462in}}%
\pgfpathlineto{\pgfqpoint{1.221526in}{1.512398in}}%
\pgfpathlineto{\pgfqpoint{1.209802in}{1.501851in}}%
\pgfpathlineto{\pgfqpoint{1.205870in}{1.498499in}}%
\pgfpathlineto{\pgfqpoint{1.192348in}{1.488240in}}%
\pgfpathlineto{\pgfqpoint{1.190213in}{1.486519in}}%
\pgfpathlineto{\pgfqpoint{1.174556in}{1.476358in}}%
\pgfpathlineto{\pgfqpoint{1.170673in}{1.474629in}}%
\pgfpathlineto{\pgfqpoint{1.158900in}{1.468048in}}%
\pgfpathlineto{\pgfqpoint{1.143243in}{1.463641in}}%
\pgfpathlineto{\pgfqpoint{1.127587in}{1.463641in}}%
\pgfpathlineto{\pgfqpoint{1.111930in}{1.468048in}}%
\pgfpathlineto{\pgfqpoint{1.100157in}{1.474629in}}%
\pgfpathclose%
\pgfpathmoveto{\pgfqpoint{1.409652in}{1.474629in}}%
\pgfpathlineto{\pgfqpoint{1.409405in}{1.474728in}}%
\pgfpathlineto{\pgfqpoint{1.393748in}{1.484223in}}%
\pgfpathlineto{\pgfqpoint{1.388598in}{1.488240in}}%
\pgfpathlineto{\pgfqpoint{1.378092in}{1.495936in}}%
\pgfpathlineto{\pgfqpoint{1.371055in}{1.501851in}}%
\pgfpathlineto{\pgfqpoint{1.362435in}{1.509489in}}%
\pgfpathlineto{\pgfqpoint{1.355900in}{1.515462in}}%
\pgfpathlineto{\pgfqpoint{1.346779in}{1.525237in}}%
\pgfpathlineto{\pgfqpoint{1.342865in}{1.529073in}}%
\pgfpathlineto{\pgfqpoint{1.332305in}{1.542684in}}%
\pgfpathlineto{\pgfqpoint{1.331122in}{1.545105in}}%
\pgfpathlineto{\pgfqpoint{1.323956in}{1.556295in}}%
\pgfpathlineto{\pgfqpoint{1.320677in}{1.569907in}}%
\pgfpathlineto{\pgfqpoint{1.322863in}{1.583518in}}%
\pgfpathlineto{\pgfqpoint{1.330526in}{1.597129in}}%
\pgfpathlineto{\pgfqpoint{1.331122in}{1.597772in}}%
\pgfpathlineto{\pgfqpoint{1.340431in}{1.610740in}}%
\pgfpathlineto{\pgfqpoint{1.346779in}{1.617225in}}%
\pgfpathlineto{\pgfqpoint{1.353143in}{1.624351in}}%
\pgfpathlineto{\pgfqpoint{1.362435in}{1.633029in}}%
\pgfpathlineto{\pgfqpoint{1.367876in}{1.637962in}}%
\pgfpathlineto{\pgfqpoint{1.378092in}{1.646572in}}%
\pgfpathlineto{\pgfqpoint{1.384900in}{1.651573in}}%
\pgfpathlineto{\pgfqpoint{1.393748in}{1.658339in}}%
\pgfpathlineto{\pgfqpoint{1.405285in}{1.665184in}}%
\pgfpathlineto{\pgfqpoint{1.409405in}{1.668117in}}%
\pgfpathlineto{\pgfqpoint{1.425061in}{1.675548in}}%
\pgfpathlineto{\pgfqpoint{1.440484in}{1.678795in}}%
\pgfpathlineto{\pgfqpoint{1.440718in}{1.678871in}}%
\pgfpathlineto{\pgfqpoint{1.441646in}{1.678795in}}%
\pgfpathlineto{\pgfqpoint{1.456375in}{1.678021in}}%
\pgfpathlineto{\pgfqpoint{1.472031in}{1.673072in}}%
\pgfpathlineto{\pgfqpoint{1.485692in}{1.665184in}}%
\pgfpathlineto{\pgfqpoint{1.487688in}{1.664233in}}%
\pgfpathlineto{\pgfqpoint{1.503344in}{1.653778in}}%
\pgfpathlineto{\pgfqpoint{1.506054in}{1.651573in}}%
\pgfpathlineto{\pgfqpoint{1.519001in}{1.641413in}}%
\pgfpathlineto{\pgfqpoint{1.522996in}{1.637962in}}%
\pgfpathlineto{\pgfqpoint{1.534657in}{1.627108in}}%
\pgfpathlineto{\pgfqpoint{1.537656in}{1.624351in}}%
\pgfpathlineto{\pgfqpoint{1.550035in}{1.610740in}}%
\pgfpathlineto{\pgfqpoint{1.550314in}{1.610326in}}%
\pgfpathlineto{\pgfqpoint{1.560722in}{1.597129in}}%
\pgfpathlineto{\pgfqpoint{1.565971in}{1.585770in}}%
\pgfpathlineto{\pgfqpoint{1.567465in}{1.583518in}}%
\pgfpathlineto{\pgfqpoint{1.570096in}{1.569907in}}%
\pgfpathlineto{\pgfqpoint{1.566149in}{1.556295in}}%
\pgfpathlineto{\pgfqpoint{1.565971in}{1.556059in}}%
\pgfpathlineto{\pgfqpoint{1.558930in}{1.542684in}}%
\pgfpathlineto{\pgfqpoint{1.550314in}{1.532491in}}%
\pgfpathlineto{\pgfqpoint{1.547840in}{1.529073in}}%
\pgfpathlineto{\pgfqpoint{1.534773in}{1.515462in}}%
\pgfpathlineto{\pgfqpoint{1.534657in}{1.515358in}}%
\pgfpathlineto{\pgfqpoint{1.519809in}{1.501851in}}%
\pgfpathlineto{\pgfqpoint{1.519001in}{1.501155in}}%
\pgfpathlineto{\pgfqpoint{1.503344in}{1.488861in}}%
\pgfpathlineto{\pgfqpoint{1.502376in}{1.488240in}}%
\pgfpathlineto{\pgfqpoint{1.487688in}{1.478124in}}%
\pgfpathlineto{\pgfqpoint{1.480514in}{1.474629in}}%
\pgfpathlineto{\pgfqpoint{1.472031in}{1.469460in}}%
\pgfpathlineto{\pgfqpoint{1.456375in}{1.464169in}}%
\pgfpathlineto{\pgfqpoint{1.440718in}{1.463289in}}%
\pgfpathlineto{\pgfqpoint{1.425061in}{1.466813in}}%
\pgfpathlineto{\pgfqpoint{1.409652in}{1.474629in}}%
\pgfpathclose%
\pgfpathmoveto{\pgfqpoint{1.719966in}{1.474629in}}%
\pgfpathlineto{\pgfqpoint{1.706880in}{1.482057in}}%
\pgfpathlineto{\pgfqpoint{1.698654in}{1.488240in}}%
\pgfpathlineto{\pgfqpoint{1.691223in}{1.493473in}}%
\pgfpathlineto{\pgfqpoint{1.681077in}{1.501851in}}%
\pgfpathlineto{\pgfqpoint{1.675567in}{1.506641in}}%
\pgfpathlineto{\pgfqpoint{1.665930in}{1.515462in}}%
\pgfpathlineto{\pgfqpoint{1.659910in}{1.521922in}}%
\pgfpathlineto{\pgfqpoint{1.652798in}{1.529073in}}%
\pgfpathlineto{\pgfqpoint{1.644253in}{1.540450in}}%
\pgfpathlineto{\pgfqpoint{1.642167in}{1.542684in}}%
\pgfpathlineto{\pgfqpoint{1.634046in}{1.556295in}}%
\pgfpathlineto{\pgfqpoint{1.631007in}{1.569907in}}%
\pgfpathlineto{\pgfqpoint{1.633033in}{1.583518in}}%
\pgfpathlineto{\pgfqpoint{1.640135in}{1.597129in}}%
\pgfpathlineto{\pgfqpoint{1.644253in}{1.601857in}}%
\pgfpathlineto{\pgfqpoint{1.650459in}{1.610740in}}%
\pgfpathlineto{\pgfqpoint{1.659910in}{1.620645in}}%
\pgfpathlineto{\pgfqpoint{1.663215in}{1.624351in}}%
\pgfpathlineto{\pgfqpoint{1.675567in}{1.635902in}}%
\pgfpathlineto{\pgfqpoint{1.677883in}{1.637962in}}%
\pgfpathlineto{\pgfqpoint{1.691223in}{1.649007in}}%
\pgfpathlineto{\pgfqpoint{1.694857in}{1.651573in}}%
\pgfpathlineto{\pgfqpoint{1.706880in}{1.660432in}}%
\pgfpathlineto{\pgfqpoint{1.715432in}{1.665184in}}%
\pgfpathlineto{\pgfqpoint{1.722536in}{1.669934in}}%
\pgfpathlineto{\pgfqpoint{1.738193in}{1.676537in}}%
\pgfpathlineto{\pgfqpoint{1.752484in}{1.678795in}}%
\pgfpathlineto{\pgfqpoint{1.753849in}{1.679127in}}%
\pgfpathlineto{\pgfqpoint{1.755890in}{1.678795in}}%
\pgfpathlineto{\pgfqpoint{1.769506in}{1.677362in}}%
\pgfpathlineto{\pgfqpoint{1.785162in}{1.671586in}}%
\pgfpathlineto{\pgfqpoint{1.795425in}{1.665184in}}%
\pgfpathlineto{\pgfqpoint{1.800819in}{1.662397in}}%
\pgfpathlineto{\pgfqpoint{1.816130in}{1.651573in}}%
\pgfpathlineto{\pgfqpoint{1.816476in}{1.651339in}}%
\pgfpathlineto{\pgfqpoint{1.832132in}{1.638700in}}%
\pgfpathlineto{\pgfqpoint{1.832981in}{1.637962in}}%
\pgfpathlineto{\pgfqpoint{1.847520in}{1.624351in}}%
\pgfpathlineto{\pgfqpoint{1.847789in}{1.624050in}}%
\pgfpathlineto{\pgfqpoint{1.860239in}{1.610740in}}%
\pgfpathlineto{\pgfqpoint{1.863445in}{1.606051in}}%
\pgfpathlineto{\pgfqpoint{1.870809in}{1.597129in}}%
\pgfpathlineto{\pgfqpoint{1.877453in}{1.583518in}}%
\pgfpathlineto{\pgfqpoint{1.879102in}{1.571680in}}%
\pgfpathlineto{\pgfqpoint{1.879484in}{1.569907in}}%
\pgfpathlineto{\pgfqpoint{1.879102in}{1.568720in}}%
\pgfpathlineto{\pgfqpoint{1.876505in}{1.556295in}}%
\pgfpathlineto{\pgfqpoint{1.868909in}{1.542684in}}%
\pgfpathlineto{\pgfqpoint{1.863445in}{1.536509in}}%
\pgfpathlineto{\pgfqpoint{1.857979in}{1.529073in}}%
\pgfpathlineto{\pgfqpoint{1.847789in}{1.518621in}}%
\pgfpathlineto{\pgfqpoint{1.844837in}{1.515462in}}%
\pgfpathlineto{\pgfqpoint{1.832132in}{1.503865in}}%
\pgfpathlineto{\pgfqpoint{1.829762in}{1.501851in}}%
\pgfpathlineto{\pgfqpoint{1.816476in}{1.491113in}}%
\pgfpathlineto{\pgfqpoint{1.812212in}{1.488240in}}%
\pgfpathlineto{\pgfqpoint{1.800819in}{1.480023in}}%
\pgfpathlineto{\pgfqpoint{1.790601in}{1.474629in}}%
\pgfpathlineto{\pgfqpoint{1.785162in}{1.471049in}}%
\pgfpathlineto{\pgfqpoint{1.769506in}{1.464874in}}%
\pgfpathlineto{\pgfqpoint{1.753849in}{1.463113in}}%
\pgfpathlineto{\pgfqpoint{1.738193in}{1.465755in}}%
\pgfpathlineto{\pgfqpoint{1.722536in}{1.472815in}}%
\pgfpathlineto{\pgfqpoint{1.719966in}{1.474629in}}%
\pgfpathclose%
\pgfusepath{fill}%
\end{pgfscope}%
\begin{pgfscope}%
\pgfpathrectangle{\pgfqpoint{0.360415in}{0.358518in}}{\pgfqpoint{1.550000in}{1.347500in}}%
\pgfusepath{clip}%
\pgfsetbuttcap%
\pgfsetroundjoin%
\definecolor{currentfill}{rgb}{0.252220,0.059415,0.453248}%
\pgfsetfillcolor{currentfill}%
\pgfsetlinewidth{0.000000pt}%
\definecolor{currentstroke}{rgb}{0.000000,0.000000,0.000000}%
\pgfsetstrokecolor{currentstroke}%
\pgfsetdash{}{0pt}%
\pgfpathmoveto{\pgfqpoint{0.423041in}{0.358518in}}%
\pgfpathlineto{\pgfqpoint{0.438698in}{0.358518in}}%
\pgfpathlineto{\pgfqpoint{0.454354in}{0.358518in}}%
\pgfpathlineto{\pgfqpoint{0.462667in}{0.358518in}}%
\pgfpathlineto{\pgfqpoint{0.458762in}{0.372129in}}%
\pgfpathlineto{\pgfqpoint{0.454354in}{0.377585in}}%
\pgfpathlineto{\pgfqpoint{0.448681in}{0.385740in}}%
\pgfpathlineto{\pgfqpoint{0.438698in}{0.395595in}}%
\pgfpathlineto{\pgfqpoint{0.435123in}{0.399351in}}%
\pgfpathlineto{\pgfqpoint{0.423041in}{0.410189in}}%
\pgfpathlineto{\pgfqpoint{0.419852in}{0.412962in}}%
\pgfpathlineto{\pgfqpoint{0.407385in}{0.423466in}}%
\pgfpathlineto{\pgfqpoint{0.403064in}{0.426573in}}%
\pgfpathlineto{\pgfqpoint{0.391728in}{0.435252in}}%
\pgfpathlineto{\pgfqpoint{0.382348in}{0.440184in}}%
\pgfpathlineto{\pgfqpoint{0.376072in}{0.444016in}}%
\pgfpathlineto{\pgfqpoint{0.360415in}{0.447411in}}%
\pgfpathlineto{\pgfqpoint{0.360415in}{0.440184in}}%
\pgfpathlineto{\pgfqpoint{0.360415in}{0.426573in}}%
\pgfpathlineto{\pgfqpoint{0.360415in}{0.412962in}}%
\pgfpathlineto{\pgfqpoint{0.360415in}{0.410372in}}%
\pgfpathlineto{\pgfqpoint{0.376072in}{0.407574in}}%
\pgfpathlineto{\pgfqpoint{0.391728in}{0.399885in}}%
\pgfpathlineto{\pgfqpoint{0.392457in}{0.399351in}}%
\pgfpathlineto{\pgfqpoint{0.407385in}{0.386373in}}%
\pgfpathlineto{\pgfqpoint{0.407999in}{0.385740in}}%
\pgfpathlineto{\pgfqpoint{0.416844in}{0.372129in}}%
\pgfpathlineto{\pgfqpoint{0.420062in}{0.358518in}}%
\pgfpathlineto{\pgfqpoint{0.423041in}{0.358518in}}%
\pgfpathclose%
\pgfpathmoveto{\pgfqpoint{0.579607in}{0.358518in}}%
\pgfpathlineto{\pgfqpoint{0.595263in}{0.358518in}}%
\pgfpathlineto{\pgfqpoint{0.610610in}{0.358518in}}%
\pgfpathlineto{\pgfqpoint{0.610920in}{0.359932in}}%
\pgfpathlineto{\pgfqpoint{0.613900in}{0.372129in}}%
\pgfpathlineto{\pgfqpoint{0.623054in}{0.385740in}}%
\pgfpathlineto{\pgfqpoint{0.626577in}{0.389297in}}%
\pgfpathlineto{\pgfqpoint{0.638782in}{0.399351in}}%
\pgfpathlineto{\pgfqpoint{0.642233in}{0.401739in}}%
\pgfpathlineto{\pgfqpoint{0.657890in}{0.408567in}}%
\pgfpathlineto{\pgfqpoint{0.673546in}{0.410258in}}%
\pgfpathlineto{\pgfqpoint{0.689203in}{0.406383in}}%
\pgfpathlineto{\pgfqpoint{0.702086in}{0.399351in}}%
\pgfpathlineto{\pgfqpoint{0.704859in}{0.397528in}}%
\pgfpathlineto{\pgfqpoint{0.717779in}{0.385740in}}%
\pgfpathlineto{\pgfqpoint{0.720516in}{0.382086in}}%
\pgfpathlineto{\pgfqpoint{0.726867in}{0.372129in}}%
\pgfpathlineto{\pgfqpoint{0.729992in}{0.358518in}}%
\pgfpathlineto{\pgfqpoint{0.736173in}{0.358518in}}%
\pgfpathlineto{\pgfqpoint{0.751829in}{0.358518in}}%
\pgfpathlineto{\pgfqpoint{0.767486in}{0.358518in}}%
\pgfpathlineto{\pgfqpoint{0.772653in}{0.358518in}}%
\pgfpathlineto{\pgfqpoint{0.768603in}{0.372129in}}%
\pgfpathlineto{\pgfqpoint{0.767486in}{0.373446in}}%
\pgfpathlineto{\pgfqpoint{0.758714in}{0.385740in}}%
\pgfpathlineto{\pgfqpoint{0.751829in}{0.392356in}}%
\pgfpathlineto{\pgfqpoint{0.745128in}{0.399351in}}%
\pgfpathlineto{\pgfqpoint{0.736173in}{0.407328in}}%
\pgfpathlineto{\pgfqpoint{0.729787in}{0.412962in}}%
\pgfpathlineto{\pgfqpoint{0.720516in}{0.420899in}}%
\pgfpathlineto{\pgfqpoint{0.712977in}{0.426573in}}%
\pgfpathlineto{\pgfqpoint{0.704859in}{0.433096in}}%
\pgfpathlineto{\pgfqpoint{0.692709in}{0.440184in}}%
\pgfpathlineto{\pgfqpoint{0.689203in}{0.442569in}}%
\pgfpathlineto{\pgfqpoint{0.673546in}{0.447272in}}%
\pgfpathlineto{\pgfqpoint{0.657890in}{0.445221in}}%
\pgfpathlineto{\pgfqpoint{0.648506in}{0.440184in}}%
\pgfpathlineto{\pgfqpoint{0.642233in}{0.437270in}}%
\pgfpathlineto{\pgfqpoint{0.627441in}{0.426573in}}%
\pgfpathlineto{\pgfqpoint{0.626577in}{0.425984in}}%
\pgfpathlineto{\pgfqpoint{0.610920in}{0.413064in}}%
\pgfpathlineto{\pgfqpoint{0.610803in}{0.412962in}}%
\pgfpathlineto{\pgfqpoint{0.595665in}{0.399351in}}%
\pgfpathlineto{\pgfqpoint{0.595263in}{0.398928in}}%
\pgfpathlineto{\pgfqpoint{0.582207in}{0.385740in}}%
\pgfpathlineto{\pgfqpoint{0.579607in}{0.381922in}}%
\pgfpathlineto{\pgfqpoint{0.572036in}{0.372129in}}%
\pgfpathlineto{\pgfqpoint{0.568251in}{0.358518in}}%
\pgfpathlineto{\pgfqpoint{0.579607in}{0.358518in}}%
\pgfpathclose%
\pgfpathmoveto{\pgfqpoint{0.892738in}{0.358518in}}%
\pgfpathlineto{\pgfqpoint{0.908395in}{0.358518in}}%
\pgfpathlineto{\pgfqpoint{0.920730in}{0.358518in}}%
\pgfpathlineto{\pgfqpoint{0.923719in}{0.372129in}}%
\pgfpathlineto{\pgfqpoint{0.924051in}{0.372659in}}%
\pgfpathlineto{\pgfqpoint{0.933191in}{0.385740in}}%
\pgfpathlineto{\pgfqpoint{0.939708in}{0.392144in}}%
\pgfpathlineto{\pgfqpoint{0.949030in}{0.399351in}}%
\pgfpathlineto{\pgfqpoint{0.955364in}{0.403451in}}%
\pgfpathlineto{\pgfqpoint{0.971021in}{0.409350in}}%
\pgfpathlineto{\pgfqpoint{0.986678in}{0.409916in}}%
\pgfpathlineto{\pgfqpoint{1.002334in}{0.405004in}}%
\pgfpathlineto{\pgfqpoint{1.011781in}{0.399351in}}%
\pgfpathlineto{\pgfqpoint{1.017991in}{0.394895in}}%
\pgfpathlineto{\pgfqpoint{1.027625in}{0.385740in}}%
\pgfpathlineto{\pgfqpoint{1.033647in}{0.377388in}}%
\pgfpathlineto{\pgfqpoint{1.036966in}{0.372129in}}%
\pgfpathlineto{\pgfqpoint{1.040015in}{0.358518in}}%
\pgfpathlineto{\pgfqpoint{1.049304in}{0.358518in}}%
\pgfpathlineto{\pgfqpoint{1.064960in}{0.358518in}}%
\pgfpathlineto{\pgfqpoint{1.080617in}{0.358518in}}%
\pgfpathlineto{\pgfqpoint{1.082498in}{0.358518in}}%
\pgfpathlineto{\pgfqpoint{1.080617in}{0.364532in}}%
\pgfpathlineto{\pgfqpoint{1.078616in}{0.372129in}}%
\pgfpathlineto{\pgfqpoint{1.068696in}{0.385740in}}%
\pgfpathlineto{\pgfqpoint{1.064960in}{0.389220in}}%
\pgfpathlineto{\pgfqpoint{1.055155in}{0.399351in}}%
\pgfpathlineto{\pgfqpoint{1.049304in}{0.404505in}}%
\pgfpathlineto{\pgfqpoint{1.039815in}{0.412962in}}%
\pgfpathlineto{\pgfqpoint{1.033647in}{0.418299in}}%
\pgfpathlineto{\pgfqpoint{1.023066in}{0.426573in}}%
\pgfpathlineto{\pgfqpoint{1.017991in}{0.430820in}}%
\pgfpathlineto{\pgfqpoint{1.003288in}{0.440184in}}%
\pgfpathlineto{\pgfqpoint{1.002334in}{0.440896in}}%
\pgfpathlineto{\pgfqpoint{0.986678in}{0.446858in}}%
\pgfpathlineto{\pgfqpoint{0.971021in}{0.446171in}}%
\pgfpathlineto{\pgfqpoint{0.957995in}{0.440184in}}%
\pgfpathlineto{\pgfqpoint{0.955364in}{0.439133in}}%
\pgfpathlineto{\pgfqpoint{0.939708in}{0.428443in}}%
\pgfpathlineto{\pgfqpoint{0.937548in}{0.426573in}}%
\pgfpathlineto{\pgfqpoint{0.924051in}{0.415683in}}%
\pgfpathlineto{\pgfqpoint{0.920926in}{0.412962in}}%
\pgfpathlineto{\pgfqpoint{0.908395in}{0.401731in}}%
\pgfpathlineto{\pgfqpoint{0.905652in}{0.399351in}}%
\pgfpathlineto{\pgfqpoint{0.892738in}{0.386195in}}%
\pgfpathlineto{\pgfqpoint{0.892232in}{0.385740in}}%
\pgfpathlineto{\pgfqpoint{0.882095in}{0.372129in}}%
\pgfpathlineto{\pgfqpoint{0.878408in}{0.358518in}}%
\pgfpathlineto{\pgfqpoint{0.892738in}{0.358518in}}%
\pgfpathclose%
\pgfpathmoveto{\pgfqpoint{1.190213in}{0.358518in}}%
\pgfpathlineto{\pgfqpoint{1.205870in}{0.358518in}}%
\pgfpathlineto{\pgfqpoint{1.221526in}{0.358518in}}%
\pgfpathlineto{\pgfqpoint{1.230815in}{0.358518in}}%
\pgfpathlineto{\pgfqpoint{1.233864in}{0.372129in}}%
\pgfpathlineto{\pgfqpoint{1.237183in}{0.377388in}}%
\pgfpathlineto{\pgfqpoint{1.243205in}{0.385740in}}%
\pgfpathlineto{\pgfqpoint{1.252839in}{0.394895in}}%
\pgfpathlineto{\pgfqpoint{1.259049in}{0.399351in}}%
\pgfpathlineto{\pgfqpoint{1.268496in}{0.405004in}}%
\pgfpathlineto{\pgfqpoint{1.284152in}{0.409916in}}%
\pgfpathlineto{\pgfqpoint{1.299809in}{0.409350in}}%
\pgfpathlineto{\pgfqpoint{1.315466in}{0.403451in}}%
\pgfpathlineto{\pgfqpoint{1.321800in}{0.399351in}}%
\pgfpathlineto{\pgfqpoint{1.331122in}{0.392144in}}%
\pgfpathlineto{\pgfqpoint{1.337639in}{0.385740in}}%
\pgfpathlineto{\pgfqpoint{1.346779in}{0.372659in}}%
\pgfpathlineto{\pgfqpoint{1.347111in}{0.372129in}}%
\pgfpathlineto{\pgfqpoint{1.350100in}{0.358518in}}%
\pgfpathlineto{\pgfqpoint{1.362435in}{0.358518in}}%
\pgfpathlineto{\pgfqpoint{1.378092in}{0.358518in}}%
\pgfpathlineto{\pgfqpoint{1.392422in}{0.358518in}}%
\pgfpathlineto{\pgfqpoint{1.388735in}{0.372129in}}%
\pgfpathlineto{\pgfqpoint{1.378598in}{0.385740in}}%
\pgfpathlineto{\pgfqpoint{1.378092in}{0.386195in}}%
\pgfpathlineto{\pgfqpoint{1.365178in}{0.399351in}}%
\pgfpathlineto{\pgfqpoint{1.362435in}{0.401731in}}%
\pgfpathlineto{\pgfqpoint{1.349904in}{0.412962in}}%
\pgfpathlineto{\pgfqpoint{1.346779in}{0.415683in}}%
\pgfpathlineto{\pgfqpoint{1.333282in}{0.426573in}}%
\pgfpathlineto{\pgfqpoint{1.331122in}{0.428443in}}%
\pgfpathlineto{\pgfqpoint{1.315466in}{0.439133in}}%
\pgfpathlineto{\pgfqpoint{1.312835in}{0.440184in}}%
\pgfpathlineto{\pgfqpoint{1.299809in}{0.446171in}}%
\pgfpathlineto{\pgfqpoint{1.284152in}{0.446858in}}%
\pgfpathlineto{\pgfqpoint{1.268496in}{0.440896in}}%
\pgfpathlineto{\pgfqpoint{1.267542in}{0.440184in}}%
\pgfpathlineto{\pgfqpoint{1.252839in}{0.430820in}}%
\pgfpathlineto{\pgfqpoint{1.247764in}{0.426573in}}%
\pgfpathlineto{\pgfqpoint{1.237183in}{0.418299in}}%
\pgfpathlineto{\pgfqpoint{1.231015in}{0.412962in}}%
\pgfpathlineto{\pgfqpoint{1.221526in}{0.404505in}}%
\pgfpathlineto{\pgfqpoint{1.215675in}{0.399351in}}%
\pgfpathlineto{\pgfqpoint{1.205870in}{0.389220in}}%
\pgfpathlineto{\pgfqpoint{1.202134in}{0.385740in}}%
\pgfpathlineto{\pgfqpoint{1.192214in}{0.372129in}}%
\pgfpathlineto{\pgfqpoint{1.190213in}{0.364532in}}%
\pgfpathlineto{\pgfqpoint{1.188332in}{0.358518in}}%
\pgfpathlineto{\pgfqpoint{1.190213in}{0.358518in}}%
\pgfpathclose%
\pgfpathmoveto{\pgfqpoint{1.503344in}{0.358518in}}%
\pgfpathlineto{\pgfqpoint{1.519001in}{0.358518in}}%
\pgfpathlineto{\pgfqpoint{1.534657in}{0.358518in}}%
\pgfpathlineto{\pgfqpoint{1.540838in}{0.358518in}}%
\pgfpathlineto{\pgfqpoint{1.543963in}{0.372129in}}%
\pgfpathlineto{\pgfqpoint{1.550314in}{0.382086in}}%
\pgfpathlineto{\pgfqpoint{1.553051in}{0.385740in}}%
\pgfpathlineto{\pgfqpoint{1.565971in}{0.397528in}}%
\pgfpathlineto{\pgfqpoint{1.568744in}{0.399351in}}%
\pgfpathlineto{\pgfqpoint{1.581627in}{0.406383in}}%
\pgfpathlineto{\pgfqpoint{1.597284in}{0.410258in}}%
\pgfpathlineto{\pgfqpoint{1.612940in}{0.408567in}}%
\pgfpathlineto{\pgfqpoint{1.628597in}{0.401739in}}%
\pgfpathlineto{\pgfqpoint{1.632048in}{0.399351in}}%
\pgfpathlineto{\pgfqpoint{1.644253in}{0.389297in}}%
\pgfpathlineto{\pgfqpoint{1.647776in}{0.385740in}}%
\pgfpathlineto{\pgfqpoint{1.656930in}{0.372129in}}%
\pgfpathlineto{\pgfqpoint{1.659910in}{0.359932in}}%
\pgfpathlineto{\pgfqpoint{1.660220in}{0.358518in}}%
\pgfpathlineto{\pgfqpoint{1.675567in}{0.358518in}}%
\pgfpathlineto{\pgfqpoint{1.691223in}{0.358518in}}%
\pgfpathlineto{\pgfqpoint{1.702579in}{0.358518in}}%
\pgfpathlineto{\pgfqpoint{1.698794in}{0.372129in}}%
\pgfpathlineto{\pgfqpoint{1.691223in}{0.381922in}}%
\pgfpathlineto{\pgfqpoint{1.688623in}{0.385740in}}%
\pgfpathlineto{\pgfqpoint{1.675567in}{0.398928in}}%
\pgfpathlineto{\pgfqpoint{1.675165in}{0.399351in}}%
\pgfpathlineto{\pgfqpoint{1.660027in}{0.412962in}}%
\pgfpathlineto{\pgfqpoint{1.659910in}{0.413064in}}%
\pgfpathlineto{\pgfqpoint{1.644253in}{0.425984in}}%
\pgfpathlineto{\pgfqpoint{1.643389in}{0.426573in}}%
\pgfpathlineto{\pgfqpoint{1.628597in}{0.437270in}}%
\pgfpathlineto{\pgfqpoint{1.622324in}{0.440184in}}%
\pgfpathlineto{\pgfqpoint{1.612940in}{0.445221in}}%
\pgfpathlineto{\pgfqpoint{1.597284in}{0.447272in}}%
\pgfpathlineto{\pgfqpoint{1.581627in}{0.442569in}}%
\pgfpathlineto{\pgfqpoint{1.578121in}{0.440184in}}%
\pgfpathlineto{\pgfqpoint{1.565971in}{0.433096in}}%
\pgfpathlineto{\pgfqpoint{1.557853in}{0.426573in}}%
\pgfpathlineto{\pgfqpoint{1.550314in}{0.420899in}}%
\pgfpathlineto{\pgfqpoint{1.541043in}{0.412962in}}%
\pgfpathlineto{\pgfqpoint{1.534657in}{0.407328in}}%
\pgfpathlineto{\pgfqpoint{1.525702in}{0.399351in}}%
\pgfpathlineto{\pgfqpoint{1.519001in}{0.392356in}}%
\pgfpathlineto{\pgfqpoint{1.512116in}{0.385740in}}%
\pgfpathlineto{\pgfqpoint{1.503344in}{0.373446in}}%
\pgfpathlineto{\pgfqpoint{1.502227in}{0.372129in}}%
\pgfpathlineto{\pgfqpoint{1.498177in}{0.358518in}}%
\pgfpathlineto{\pgfqpoint{1.503344in}{0.358518in}}%
\pgfpathclose%
\pgfpathmoveto{\pgfqpoint{1.816476in}{0.358518in}}%
\pgfpathlineto{\pgfqpoint{1.832132in}{0.358518in}}%
\pgfpathlineto{\pgfqpoint{1.847789in}{0.358518in}}%
\pgfpathlineto{\pgfqpoint{1.850768in}{0.358518in}}%
\pgfpathlineto{\pgfqpoint{1.853986in}{0.372129in}}%
\pgfpathlineto{\pgfqpoint{1.862831in}{0.385740in}}%
\pgfpathlineto{\pgfqpoint{1.863445in}{0.386373in}}%
\pgfpathlineto{\pgfqpoint{1.878373in}{0.399351in}}%
\pgfpathlineto{\pgfqpoint{1.879102in}{0.399885in}}%
\pgfpathlineto{\pgfqpoint{1.894758in}{0.407574in}}%
\pgfpathlineto{\pgfqpoint{1.910415in}{0.410372in}}%
\pgfpathlineto{\pgfqpoint{1.910415in}{0.412962in}}%
\pgfpathlineto{\pgfqpoint{1.910415in}{0.426573in}}%
\pgfpathlineto{\pgfqpoint{1.910415in}{0.440184in}}%
\pgfpathlineto{\pgfqpoint{1.910415in}{0.447411in}}%
\pgfpathlineto{\pgfqpoint{1.894758in}{0.444016in}}%
\pgfpathlineto{\pgfqpoint{1.888482in}{0.440184in}}%
\pgfpathlineto{\pgfqpoint{1.879102in}{0.435252in}}%
\pgfpathlineto{\pgfqpoint{1.867766in}{0.426573in}}%
\pgfpathlineto{\pgfqpoint{1.863445in}{0.423466in}}%
\pgfpathlineto{\pgfqpoint{1.850978in}{0.412962in}}%
\pgfpathlineto{\pgfqpoint{1.847789in}{0.410189in}}%
\pgfpathlineto{\pgfqpoint{1.835707in}{0.399351in}}%
\pgfpathlineto{\pgfqpoint{1.832132in}{0.395595in}}%
\pgfpathlineto{\pgfqpoint{1.822149in}{0.385740in}}%
\pgfpathlineto{\pgfqpoint{1.816476in}{0.377585in}}%
\pgfpathlineto{\pgfqpoint{1.812068in}{0.372129in}}%
\pgfpathlineto{\pgfqpoint{1.808163in}{0.358518in}}%
\pgfpathlineto{\pgfqpoint{1.816476in}{0.358518in}}%
\pgfpathclose%
\pgfpathmoveto{\pgfqpoint{0.376072in}{0.542492in}}%
\pgfpathlineto{\pgfqpoint{0.387336in}{0.549073in}}%
\pgfpathlineto{\pgfqpoint{0.391728in}{0.551333in}}%
\pgfpathlineto{\pgfqpoint{0.406899in}{0.562684in}}%
\pgfpathlineto{\pgfqpoint{0.407385in}{0.563033in}}%
\pgfpathlineto{\pgfqpoint{0.423041in}{0.576193in}}%
\pgfpathlineto{\pgfqpoint{0.423159in}{0.576295in}}%
\pgfpathlineto{\pgfqpoint{0.438020in}{0.589907in}}%
\pgfpathlineto{\pgfqpoint{0.438698in}{0.590658in}}%
\pgfpathlineto{\pgfqpoint{0.451002in}{0.603518in}}%
\pgfpathlineto{\pgfqpoint{0.454354in}{0.608971in}}%
\pgfpathlineto{\pgfqpoint{0.460148in}{0.617129in}}%
\pgfpathlineto{\pgfqpoint{0.462508in}{0.630740in}}%
\pgfpathlineto{\pgfqpoint{0.457098in}{0.644351in}}%
\pgfpathlineto{\pgfqpoint{0.454354in}{0.647399in}}%
\pgfpathlineto{\pgfqpoint{0.446201in}{0.657962in}}%
\pgfpathlineto{\pgfqpoint{0.438698in}{0.665019in}}%
\pgfpathlineto{\pgfqpoint{0.432171in}{0.671573in}}%
\pgfpathlineto{\pgfqpoint{0.423041in}{0.679633in}}%
\pgfpathlineto{\pgfqpoint{0.416561in}{0.685184in}}%
\pgfpathlineto{\pgfqpoint{0.407385in}{0.692970in}}%
\pgfpathlineto{\pgfqpoint{0.399338in}{0.698795in}}%
\pgfpathlineto{\pgfqpoint{0.391728in}{0.704781in}}%
\pgfpathlineto{\pgfqpoint{0.377587in}{0.712407in}}%
\pgfpathlineto{\pgfqpoint{0.376072in}{0.713378in}}%
\pgfpathlineto{\pgfqpoint{0.360415in}{0.716899in}}%
\pgfpathlineto{\pgfqpoint{0.360415in}{0.712407in}}%
\pgfpathlineto{\pgfqpoint{0.360415in}{0.698795in}}%
\pgfpathlineto{\pgfqpoint{0.360415in}{0.685184in}}%
\pgfpathlineto{\pgfqpoint{0.360415in}{0.679811in}}%
\pgfpathlineto{\pgfqpoint{0.376072in}{0.677094in}}%
\pgfpathlineto{\pgfqpoint{0.387525in}{0.671573in}}%
\pgfpathlineto{\pgfqpoint{0.391728in}{0.669194in}}%
\pgfpathlineto{\pgfqpoint{0.405288in}{0.657962in}}%
\pgfpathlineto{\pgfqpoint{0.407385in}{0.655551in}}%
\pgfpathlineto{\pgfqpoint{0.415473in}{0.644351in}}%
\pgfpathlineto{\pgfqpoint{0.419930in}{0.630740in}}%
\pgfpathlineto{\pgfqpoint{0.417986in}{0.617129in}}%
\pgfpathlineto{\pgfqpoint{0.410132in}{0.603518in}}%
\pgfpathlineto{\pgfqpoint{0.407385in}{0.600517in}}%
\pgfpathlineto{\pgfqpoint{0.395819in}{0.589907in}}%
\pgfpathlineto{\pgfqpoint{0.391728in}{0.586844in}}%
\pgfpathlineto{\pgfqpoint{0.376072in}{0.578886in}}%
\pgfpathlineto{\pgfqpoint{0.362042in}{0.576295in}}%
\pgfpathlineto{\pgfqpoint{0.360415in}{0.576026in}}%
\pgfpathlineto{\pgfqpoint{0.360415in}{0.562684in}}%
\pgfpathlineto{\pgfqpoint{0.360415in}{0.549073in}}%
\pgfpathlineto{\pgfqpoint{0.360415in}{0.539201in}}%
\pgfpathlineto{\pgfqpoint{0.376072in}{0.542492in}}%
\pgfpathclose%
\pgfpathmoveto{\pgfqpoint{0.657890in}{0.541324in}}%
\pgfpathlineto{\pgfqpoint{0.673546in}{0.539335in}}%
\pgfpathlineto{\pgfqpoint{0.689203in}{0.543894in}}%
\pgfpathlineto{\pgfqpoint{0.697158in}{0.549073in}}%
\pgfpathlineto{\pgfqpoint{0.704859in}{0.553470in}}%
\pgfpathlineto{\pgfqpoint{0.716592in}{0.562684in}}%
\pgfpathlineto{\pgfqpoint{0.720516in}{0.565630in}}%
\pgfpathlineto{\pgfqpoint{0.733011in}{0.576295in}}%
\pgfpathlineto{\pgfqpoint{0.736173in}{0.579141in}}%
\pgfpathlineto{\pgfqpoint{0.748004in}{0.589907in}}%
\pgfpathlineto{\pgfqpoint{0.751829in}{0.594121in}}%
\pgfpathlineto{\pgfqpoint{0.761066in}{0.603518in}}%
\pgfpathlineto{\pgfqpoint{0.767486in}{0.613702in}}%
\pgfpathlineto{\pgfqpoint{0.770040in}{0.617129in}}%
\pgfpathlineto{\pgfqpoint{0.772488in}{0.630740in}}%
\pgfpathlineto{\pgfqpoint{0.767486in}{0.642849in}}%
\pgfpathlineto{\pgfqpoint{0.766953in}{0.644351in}}%
\pgfpathlineto{\pgfqpoint{0.756202in}{0.657962in}}%
\pgfpathlineto{\pgfqpoint{0.751829in}{0.661966in}}%
\pgfpathlineto{\pgfqpoint{0.742198in}{0.671573in}}%
\pgfpathlineto{\pgfqpoint{0.736173in}{0.676855in}}%
\pgfpathlineto{\pgfqpoint{0.726592in}{0.685184in}}%
\pgfpathlineto{\pgfqpoint{0.720516in}{0.690422in}}%
\pgfpathlineto{\pgfqpoint{0.709464in}{0.698795in}}%
\pgfpathlineto{\pgfqpoint{0.704859in}{0.702597in}}%
\pgfpathlineto{\pgfqpoint{0.689203in}{0.711943in}}%
\pgfpathlineto{\pgfqpoint{0.687476in}{0.712407in}}%
\pgfpathlineto{\pgfqpoint{0.673546in}{0.716755in}}%
\pgfpathlineto{\pgfqpoint{0.657890in}{0.714627in}}%
\pgfpathlineto{\pgfqpoint{0.653949in}{0.712407in}}%
\pgfpathlineto{\pgfqpoint{0.642233in}{0.706825in}}%
\pgfpathlineto{\pgfqpoint{0.631425in}{0.698795in}}%
\pgfpathlineto{\pgfqpoint{0.626577in}{0.695470in}}%
\pgfpathlineto{\pgfqpoint{0.614193in}{0.685184in}}%
\pgfpathlineto{\pgfqpoint{0.610920in}{0.682436in}}%
\pgfpathlineto{\pgfqpoint{0.598651in}{0.671573in}}%
\pgfpathlineto{\pgfqpoint{0.595263in}{0.668162in}}%
\pgfpathlineto{\pgfqpoint{0.584665in}{0.657962in}}%
\pgfpathlineto{\pgfqpoint{0.579607in}{0.651267in}}%
\pgfpathlineto{\pgfqpoint{0.573649in}{0.644351in}}%
\pgfpathlineto{\pgfqpoint{0.568405in}{0.630740in}}%
\pgfpathlineto{\pgfqpoint{0.570693in}{0.617129in}}%
\pgfpathlineto{\pgfqpoint{0.579607in}{0.604014in}}%
\pgfpathlineto{\pgfqpoint{0.579906in}{0.603518in}}%
\pgfpathlineto{\pgfqpoint{0.592777in}{0.589907in}}%
\pgfpathlineto{\pgfqpoint{0.595263in}{0.587744in}}%
\pgfpathlineto{\pgfqpoint{0.607768in}{0.576295in}}%
\pgfpathlineto{\pgfqpoint{0.610920in}{0.573555in}}%
\pgfpathlineto{\pgfqpoint{0.624089in}{0.562684in}}%
\pgfpathlineto{\pgfqpoint{0.626577in}{0.560522in}}%
\pgfpathlineto{\pgfqpoint{0.642233in}{0.549333in}}%
\pgfpathlineto{\pgfqpoint{0.642804in}{0.549073in}}%
\pgfpathlineto{\pgfqpoint{0.657890in}{0.541324in}}%
\pgfpathclose%
\pgfpathmoveto{\pgfqpoint{0.671891in}{0.576295in}}%
\pgfpathlineto{\pgfqpoint{0.657890in}{0.577859in}}%
\pgfpathlineto{\pgfqpoint{0.642233in}{0.584925in}}%
\pgfpathlineto{\pgfqpoint{0.635186in}{0.589907in}}%
\pgfpathlineto{\pgfqpoint{0.626577in}{0.597392in}}%
\pgfpathlineto{\pgfqpoint{0.620846in}{0.603518in}}%
\pgfpathlineto{\pgfqpoint{0.612718in}{0.617129in}}%
\pgfpathlineto{\pgfqpoint{0.610920in}{0.629301in}}%
\pgfpathlineto{\pgfqpoint{0.610731in}{0.630740in}}%
\pgfpathlineto{\pgfqpoint{0.610920in}{0.631361in}}%
\pgfpathlineto{\pgfqpoint{0.615319in}{0.644351in}}%
\pgfpathlineto{\pgfqpoint{0.625413in}{0.657962in}}%
\pgfpathlineto{\pgfqpoint{0.626577in}{0.659082in}}%
\pgfpathlineto{\pgfqpoint{0.642233in}{0.671395in}}%
\pgfpathlineto{\pgfqpoint{0.642589in}{0.671573in}}%
\pgfpathlineto{\pgfqpoint{0.657890in}{0.678058in}}%
\pgfpathlineto{\pgfqpoint{0.673546in}{0.679700in}}%
\pgfpathlineto{\pgfqpoint{0.689203in}{0.675937in}}%
\pgfpathlineto{\pgfqpoint{0.697326in}{0.671573in}}%
\pgfpathlineto{\pgfqpoint{0.704859in}{0.666842in}}%
\pgfpathlineto{\pgfqpoint{0.715074in}{0.657962in}}%
\pgfpathlineto{\pgfqpoint{0.720516in}{0.651413in}}%
\pgfpathlineto{\pgfqpoint{0.725535in}{0.644351in}}%
\pgfpathlineto{\pgfqpoint{0.729864in}{0.630740in}}%
\pgfpathlineto{\pgfqpoint{0.727976in}{0.617129in}}%
\pgfpathlineto{\pgfqpoint{0.720516in}{0.603827in}}%
\pgfpathlineto{\pgfqpoint{0.720312in}{0.603518in}}%
\pgfpathlineto{\pgfqpoint{0.706147in}{0.589907in}}%
\pgfpathlineto{\pgfqpoint{0.704859in}{0.588895in}}%
\pgfpathlineto{\pgfqpoint{0.689203in}{0.580119in}}%
\pgfpathlineto{\pgfqpoint{0.674260in}{0.576295in}}%
\pgfpathlineto{\pgfqpoint{0.673546in}{0.576131in}}%
\pgfpathlineto{\pgfqpoint{0.671891in}{0.576295in}}%
\pgfpathclose%
\pgfpathmoveto{\pgfqpoint{0.955364in}{0.547343in}}%
\pgfpathlineto{\pgfqpoint{0.971021in}{0.540402in}}%
\pgfpathlineto{\pgfqpoint{0.986678in}{0.539737in}}%
\pgfpathlineto{\pgfqpoint{1.002334in}{0.545516in}}%
\pgfpathlineto{\pgfqpoint{1.007317in}{0.549073in}}%
\pgfpathlineto{\pgfqpoint{1.017991in}{0.555726in}}%
\pgfpathlineto{\pgfqpoint{1.026499in}{0.562684in}}%
\pgfpathlineto{\pgfqpoint{1.033647in}{0.568259in}}%
\pgfpathlineto{\pgfqpoint{1.042961in}{0.576295in}}%
\pgfpathlineto{\pgfqpoint{1.049304in}{0.582062in}}%
\pgfpathlineto{\pgfqpoint{1.058021in}{0.589907in}}%
\pgfpathlineto{\pgfqpoint{1.064960in}{0.597473in}}%
\pgfpathlineto{\pgfqpoint{1.071089in}{0.603518in}}%
\pgfpathlineto{\pgfqpoint{1.079896in}{0.617129in}}%
\pgfpathlineto{\pgfqpoint{1.080617in}{0.621654in}}%
\pgfpathlineto{\pgfqpoint{1.082326in}{0.630740in}}%
\pgfpathlineto{\pgfqpoint{1.080617in}{0.634659in}}%
\pgfpathlineto{\pgfqpoint{1.077078in}{0.644351in}}%
\pgfpathlineto{\pgfqpoint{1.066140in}{0.657962in}}%
\pgfpathlineto{\pgfqpoint{1.064960in}{0.659009in}}%
\pgfpathlineto{\pgfqpoint{1.052234in}{0.671573in}}%
\pgfpathlineto{\pgfqpoint{1.049304in}{0.674114in}}%
\pgfpathlineto{\pgfqpoint{1.036697in}{0.685184in}}%
\pgfpathlineto{\pgfqpoint{1.033647in}{0.687842in}}%
\pgfpathlineto{\pgfqpoint{1.019731in}{0.698795in}}%
\pgfpathlineto{\pgfqpoint{1.017991in}{0.700291in}}%
\pgfpathlineto{\pgfqpoint{1.002334in}{0.710423in}}%
\pgfpathlineto{\pgfqpoint{0.996509in}{0.712407in}}%
\pgfpathlineto{\pgfqpoint{0.986678in}{0.716325in}}%
\pgfpathlineto{\pgfqpoint{0.971021in}{0.715613in}}%
\pgfpathlineto{\pgfqpoint{0.964376in}{0.712407in}}%
\pgfpathlineto{\pgfqpoint{0.955364in}{0.708712in}}%
\pgfpathlineto{\pgfqpoint{0.941092in}{0.698795in}}%
\pgfpathlineto{\pgfqpoint{0.939708in}{0.697904in}}%
\pgfpathlineto{\pgfqpoint{0.924051in}{0.685245in}}%
\pgfpathlineto{\pgfqpoint{0.923982in}{0.685184in}}%
\pgfpathlineto{\pgfqpoint{0.908582in}{0.671573in}}%
\pgfpathlineto{\pgfqpoint{0.908395in}{0.671385in}}%
\pgfpathlineto{\pgfqpoint{0.894711in}{0.657962in}}%
\pgfpathlineto{\pgfqpoint{0.892738in}{0.655304in}}%
\pgfpathlineto{\pgfqpoint{0.883667in}{0.644351in}}%
\pgfpathlineto{\pgfqpoint{0.878559in}{0.630740in}}%
\pgfpathlineto{\pgfqpoint{0.880787in}{0.617129in}}%
\pgfpathlineto{\pgfqpoint{0.889787in}{0.603518in}}%
\pgfpathlineto{\pgfqpoint{0.892738in}{0.600707in}}%
\pgfpathlineto{\pgfqpoint{0.902784in}{0.589907in}}%
\pgfpathlineto{\pgfqpoint{0.908395in}{0.584934in}}%
\pgfpathlineto{\pgfqpoint{0.917843in}{0.576295in}}%
\pgfpathlineto{\pgfqpoint{0.924051in}{0.570906in}}%
\pgfpathlineto{\pgfqpoint{0.934268in}{0.562684in}}%
\pgfpathlineto{\pgfqpoint{0.939708in}{0.558083in}}%
\pgfpathlineto{\pgfqpoint{0.953123in}{0.549073in}}%
\pgfpathlineto{\pgfqpoint{0.955364in}{0.547343in}}%
\pgfpathclose%
\pgfpathmoveto{\pgfqpoint{0.945151in}{0.589907in}}%
\pgfpathlineto{\pgfqpoint{0.939708in}{0.594347in}}%
\pgfpathlineto{\pgfqpoint{0.930893in}{0.603518in}}%
\pgfpathlineto{\pgfqpoint{0.924051in}{0.614603in}}%
\pgfpathlineto{\pgfqpoint{0.922658in}{0.617129in}}%
\pgfpathlineto{\pgfqpoint{0.920852in}{0.630740in}}%
\pgfpathlineto{\pgfqpoint{0.924051in}{0.641211in}}%
\pgfpathlineto{\pgfqpoint{0.925142in}{0.644351in}}%
\pgfpathlineto{\pgfqpoint{0.935645in}{0.657962in}}%
\pgfpathlineto{\pgfqpoint{0.939708in}{0.661766in}}%
\pgfpathlineto{\pgfqpoint{0.952983in}{0.671573in}}%
\pgfpathlineto{\pgfqpoint{0.955364in}{0.673090in}}%
\pgfpathlineto{\pgfqpoint{0.971021in}{0.678819in}}%
\pgfpathlineto{\pgfqpoint{0.986678in}{0.679368in}}%
\pgfpathlineto{\pgfqpoint{1.002334in}{0.674598in}}%
\pgfpathlineto{\pgfqpoint{1.007470in}{0.671573in}}%
\pgfpathlineto{\pgfqpoint{1.017991in}{0.664359in}}%
\pgfpathlineto{\pgfqpoint{1.025057in}{0.657962in}}%
\pgfpathlineto{\pgfqpoint{1.033647in}{0.647223in}}%
\pgfpathlineto{\pgfqpoint{1.035667in}{0.644351in}}%
\pgfpathlineto{\pgfqpoint{1.039890in}{0.630740in}}%
\pgfpathlineto{\pgfqpoint{1.038048in}{0.617129in}}%
\pgfpathlineto{\pgfqpoint{1.033647in}{0.609197in}}%
\pgfpathlineto{\pgfqpoint{1.030030in}{0.603518in}}%
\pgfpathlineto{\pgfqpoint{1.017991in}{0.591406in}}%
\pgfpathlineto{\pgfqpoint{1.016010in}{0.589907in}}%
\pgfpathlineto{\pgfqpoint{1.002334in}{0.581547in}}%
\pgfpathlineto{\pgfqpoint{0.986678in}{0.576463in}}%
\pgfpathlineto{\pgfqpoint{0.971021in}{0.577048in}}%
\pgfpathlineto{\pgfqpoint{0.955364in}{0.583154in}}%
\pgfpathlineto{\pgfqpoint{0.945151in}{0.589907in}}%
\pgfpathclose%
\pgfpathmoveto{\pgfqpoint{1.268496in}{0.545516in}}%
\pgfpathlineto{\pgfqpoint{1.284152in}{0.539737in}}%
\pgfpathlineto{\pgfqpoint{1.299809in}{0.540402in}}%
\pgfpathlineto{\pgfqpoint{1.315466in}{0.547343in}}%
\pgfpathlineto{\pgfqpoint{1.317707in}{0.549073in}}%
\pgfpathlineto{\pgfqpoint{1.331122in}{0.558083in}}%
\pgfpathlineto{\pgfqpoint{1.336562in}{0.562684in}}%
\pgfpathlineto{\pgfqpoint{1.346779in}{0.570906in}}%
\pgfpathlineto{\pgfqpoint{1.352987in}{0.576295in}}%
\pgfpathlineto{\pgfqpoint{1.362435in}{0.584934in}}%
\pgfpathlineto{\pgfqpoint{1.368046in}{0.589907in}}%
\pgfpathlineto{\pgfqpoint{1.378092in}{0.600707in}}%
\pgfpathlineto{\pgfqpoint{1.381043in}{0.603518in}}%
\pgfpathlineto{\pgfqpoint{1.390043in}{0.617129in}}%
\pgfpathlineto{\pgfqpoint{1.392271in}{0.630740in}}%
\pgfpathlineto{\pgfqpoint{1.387163in}{0.644351in}}%
\pgfpathlineto{\pgfqpoint{1.378092in}{0.655304in}}%
\pgfpathlineto{\pgfqpoint{1.376119in}{0.657962in}}%
\pgfpathlineto{\pgfqpoint{1.362435in}{0.671385in}}%
\pgfpathlineto{\pgfqpoint{1.362248in}{0.671573in}}%
\pgfpathlineto{\pgfqpoint{1.346848in}{0.685184in}}%
\pgfpathlineto{\pgfqpoint{1.346779in}{0.685245in}}%
\pgfpathlineto{\pgfqpoint{1.331122in}{0.697904in}}%
\pgfpathlineto{\pgfqpoint{1.329738in}{0.698795in}}%
\pgfpathlineto{\pgfqpoint{1.315466in}{0.708712in}}%
\pgfpathlineto{\pgfqpoint{1.306454in}{0.712407in}}%
\pgfpathlineto{\pgfqpoint{1.299809in}{0.715613in}}%
\pgfpathlineto{\pgfqpoint{1.284152in}{0.716325in}}%
\pgfpathlineto{\pgfqpoint{1.274321in}{0.712407in}}%
\pgfpathlineto{\pgfqpoint{1.268496in}{0.710423in}}%
\pgfpathlineto{\pgfqpoint{1.252839in}{0.700291in}}%
\pgfpathlineto{\pgfqpoint{1.251099in}{0.698795in}}%
\pgfpathlineto{\pgfqpoint{1.237183in}{0.687842in}}%
\pgfpathlineto{\pgfqpoint{1.234133in}{0.685184in}}%
\pgfpathlineto{\pgfqpoint{1.221526in}{0.674114in}}%
\pgfpathlineto{\pgfqpoint{1.218596in}{0.671573in}}%
\pgfpathlineto{\pgfqpoint{1.205870in}{0.659009in}}%
\pgfpathlineto{\pgfqpoint{1.204690in}{0.657962in}}%
\pgfpathlineto{\pgfqpoint{1.193752in}{0.644351in}}%
\pgfpathlineto{\pgfqpoint{1.190213in}{0.634659in}}%
\pgfpathlineto{\pgfqpoint{1.188504in}{0.630740in}}%
\pgfpathlineto{\pgfqpoint{1.190213in}{0.621654in}}%
\pgfpathlineto{\pgfqpoint{1.190934in}{0.617129in}}%
\pgfpathlineto{\pgfqpoint{1.199741in}{0.603518in}}%
\pgfpathlineto{\pgfqpoint{1.205870in}{0.597473in}}%
\pgfpathlineto{\pgfqpoint{1.212809in}{0.589907in}}%
\pgfpathlineto{\pgfqpoint{1.221526in}{0.582062in}}%
\pgfpathlineto{\pgfqpoint{1.227869in}{0.576295in}}%
\pgfpathlineto{\pgfqpoint{1.237183in}{0.568259in}}%
\pgfpathlineto{\pgfqpoint{1.244331in}{0.562684in}}%
\pgfpathlineto{\pgfqpoint{1.252839in}{0.555726in}}%
\pgfpathlineto{\pgfqpoint{1.263513in}{0.549073in}}%
\pgfpathlineto{\pgfqpoint{1.268496in}{0.545516in}}%
\pgfpathclose%
\pgfpathmoveto{\pgfqpoint{1.254820in}{0.589907in}}%
\pgfpathlineto{\pgfqpoint{1.252839in}{0.591406in}}%
\pgfpathlineto{\pgfqpoint{1.240800in}{0.603518in}}%
\pgfpathlineto{\pgfqpoint{1.237183in}{0.609197in}}%
\pgfpathlineto{\pgfqpoint{1.232782in}{0.617129in}}%
\pgfpathlineto{\pgfqpoint{1.230940in}{0.630740in}}%
\pgfpathlineto{\pgfqpoint{1.235163in}{0.644351in}}%
\pgfpathlineto{\pgfqpoint{1.237183in}{0.647223in}}%
\pgfpathlineto{\pgfqpoint{1.245773in}{0.657962in}}%
\pgfpathlineto{\pgfqpoint{1.252839in}{0.664359in}}%
\pgfpathlineto{\pgfqpoint{1.263360in}{0.671573in}}%
\pgfpathlineto{\pgfqpoint{1.268496in}{0.674598in}}%
\pgfpathlineto{\pgfqpoint{1.284152in}{0.679368in}}%
\pgfpathlineto{\pgfqpoint{1.299809in}{0.678819in}}%
\pgfpathlineto{\pgfqpoint{1.315466in}{0.673090in}}%
\pgfpathlineto{\pgfqpoint{1.317847in}{0.671573in}}%
\pgfpathlineto{\pgfqpoint{1.331122in}{0.661766in}}%
\pgfpathlineto{\pgfqpoint{1.335185in}{0.657962in}}%
\pgfpathlineto{\pgfqpoint{1.345688in}{0.644351in}}%
\pgfpathlineto{\pgfqpoint{1.346779in}{0.641211in}}%
\pgfpathlineto{\pgfqpoint{1.349978in}{0.630740in}}%
\pgfpathlineto{\pgfqpoint{1.348172in}{0.617129in}}%
\pgfpathlineto{\pgfqpoint{1.346779in}{0.614603in}}%
\pgfpathlineto{\pgfqpoint{1.339937in}{0.603518in}}%
\pgfpathlineto{\pgfqpoint{1.331122in}{0.594347in}}%
\pgfpathlineto{\pgfqpoint{1.325679in}{0.589907in}}%
\pgfpathlineto{\pgfqpoint{1.315466in}{0.583154in}}%
\pgfpathlineto{\pgfqpoint{1.299809in}{0.577048in}}%
\pgfpathlineto{\pgfqpoint{1.284152in}{0.576463in}}%
\pgfpathlineto{\pgfqpoint{1.268496in}{0.581547in}}%
\pgfpathlineto{\pgfqpoint{1.254820in}{0.589907in}}%
\pgfpathclose%
\pgfpathmoveto{\pgfqpoint{1.581627in}{0.543894in}}%
\pgfpathlineto{\pgfqpoint{1.597284in}{0.539335in}}%
\pgfpathlineto{\pgfqpoint{1.612940in}{0.541324in}}%
\pgfpathlineto{\pgfqpoint{1.628026in}{0.549073in}}%
\pgfpathlineto{\pgfqpoint{1.628597in}{0.549333in}}%
\pgfpathlineto{\pgfqpoint{1.644253in}{0.560522in}}%
\pgfpathlineto{\pgfqpoint{1.646741in}{0.562684in}}%
\pgfpathlineto{\pgfqpoint{1.659910in}{0.573555in}}%
\pgfpathlineto{\pgfqpoint{1.663062in}{0.576295in}}%
\pgfpathlineto{\pgfqpoint{1.675567in}{0.587744in}}%
\pgfpathlineto{\pgfqpoint{1.678053in}{0.589907in}}%
\pgfpathlineto{\pgfqpoint{1.690924in}{0.603518in}}%
\pgfpathlineto{\pgfqpoint{1.691223in}{0.604014in}}%
\pgfpathlineto{\pgfqpoint{1.700137in}{0.617129in}}%
\pgfpathlineto{\pgfqpoint{1.702425in}{0.630740in}}%
\pgfpathlineto{\pgfqpoint{1.697181in}{0.644351in}}%
\pgfpathlineto{\pgfqpoint{1.691223in}{0.651267in}}%
\pgfpathlineto{\pgfqpoint{1.686165in}{0.657962in}}%
\pgfpathlineto{\pgfqpoint{1.675567in}{0.668162in}}%
\pgfpathlineto{\pgfqpoint{1.672179in}{0.671573in}}%
\pgfpathlineto{\pgfqpoint{1.659910in}{0.682436in}}%
\pgfpathlineto{\pgfqpoint{1.656637in}{0.685184in}}%
\pgfpathlineto{\pgfqpoint{1.644253in}{0.695470in}}%
\pgfpathlineto{\pgfqpoint{1.639405in}{0.698795in}}%
\pgfpathlineto{\pgfqpoint{1.628597in}{0.706825in}}%
\pgfpathlineto{\pgfqpoint{1.616881in}{0.712407in}}%
\pgfpathlineto{\pgfqpoint{1.612940in}{0.714627in}}%
\pgfpathlineto{\pgfqpoint{1.597284in}{0.716755in}}%
\pgfpathlineto{\pgfqpoint{1.583354in}{0.712407in}}%
\pgfpathlineto{\pgfqpoint{1.581627in}{0.711943in}}%
\pgfpathlineto{\pgfqpoint{1.565971in}{0.702597in}}%
\pgfpathlineto{\pgfqpoint{1.561366in}{0.698795in}}%
\pgfpathlineto{\pgfqpoint{1.550314in}{0.690422in}}%
\pgfpathlineto{\pgfqpoint{1.544238in}{0.685184in}}%
\pgfpathlineto{\pgfqpoint{1.534657in}{0.676855in}}%
\pgfpathlineto{\pgfqpoint{1.528632in}{0.671573in}}%
\pgfpathlineto{\pgfqpoint{1.519001in}{0.661966in}}%
\pgfpathlineto{\pgfqpoint{1.514628in}{0.657962in}}%
\pgfpathlineto{\pgfqpoint{1.503877in}{0.644351in}}%
\pgfpathlineto{\pgfqpoint{1.503344in}{0.642849in}}%
\pgfpathlineto{\pgfqpoint{1.498342in}{0.630740in}}%
\pgfpathlineto{\pgfqpoint{1.500790in}{0.617129in}}%
\pgfpathlineto{\pgfqpoint{1.503344in}{0.613702in}}%
\pgfpathlineto{\pgfqpoint{1.509764in}{0.603518in}}%
\pgfpathlineto{\pgfqpoint{1.519001in}{0.594121in}}%
\pgfpathlineto{\pgfqpoint{1.522826in}{0.589907in}}%
\pgfpathlineto{\pgfqpoint{1.534657in}{0.579141in}}%
\pgfpathlineto{\pgfqpoint{1.537819in}{0.576295in}}%
\pgfpathlineto{\pgfqpoint{1.550314in}{0.565630in}}%
\pgfpathlineto{\pgfqpoint{1.554238in}{0.562684in}}%
\pgfpathlineto{\pgfqpoint{1.565971in}{0.553470in}}%
\pgfpathlineto{\pgfqpoint{1.573672in}{0.549073in}}%
\pgfpathlineto{\pgfqpoint{1.581627in}{0.543894in}}%
\pgfpathclose%
\pgfpathmoveto{\pgfqpoint{1.596570in}{0.576295in}}%
\pgfpathlineto{\pgfqpoint{1.581627in}{0.580119in}}%
\pgfpathlineto{\pgfqpoint{1.565971in}{0.588895in}}%
\pgfpathlineto{\pgfqpoint{1.564683in}{0.589907in}}%
\pgfpathlineto{\pgfqpoint{1.550518in}{0.603518in}}%
\pgfpathlineto{\pgfqpoint{1.550314in}{0.603827in}}%
\pgfpathlineto{\pgfqpoint{1.542854in}{0.617129in}}%
\pgfpathlineto{\pgfqpoint{1.540966in}{0.630740in}}%
\pgfpathlineto{\pgfqpoint{1.545295in}{0.644351in}}%
\pgfpathlineto{\pgfqpoint{1.550314in}{0.651413in}}%
\pgfpathlineto{\pgfqpoint{1.555756in}{0.657962in}}%
\pgfpathlineto{\pgfqpoint{1.565971in}{0.666842in}}%
\pgfpathlineto{\pgfqpoint{1.573504in}{0.671573in}}%
\pgfpathlineto{\pgfqpoint{1.581627in}{0.675937in}}%
\pgfpathlineto{\pgfqpoint{1.597284in}{0.679700in}}%
\pgfpathlineto{\pgfqpoint{1.612940in}{0.678058in}}%
\pgfpathlineto{\pgfqpoint{1.628241in}{0.671573in}}%
\pgfpathlineto{\pgfqpoint{1.628597in}{0.671395in}}%
\pgfpathlineto{\pgfqpoint{1.644253in}{0.659082in}}%
\pgfpathlineto{\pgfqpoint{1.645417in}{0.657962in}}%
\pgfpathlineto{\pgfqpoint{1.655511in}{0.644351in}}%
\pgfpathlineto{\pgfqpoint{1.659910in}{0.631361in}}%
\pgfpathlineto{\pgfqpoint{1.660099in}{0.630740in}}%
\pgfpathlineto{\pgfqpoint{1.659910in}{0.629301in}}%
\pgfpathlineto{\pgfqpoint{1.658112in}{0.617129in}}%
\pgfpathlineto{\pgfqpoint{1.649984in}{0.603518in}}%
\pgfpathlineto{\pgfqpoint{1.644253in}{0.597392in}}%
\pgfpathlineto{\pgfqpoint{1.635644in}{0.589907in}}%
\pgfpathlineto{\pgfqpoint{1.628597in}{0.584925in}}%
\pgfpathlineto{\pgfqpoint{1.612940in}{0.577859in}}%
\pgfpathlineto{\pgfqpoint{1.598939in}{0.576295in}}%
\pgfpathlineto{\pgfqpoint{1.597284in}{0.576131in}}%
\pgfpathlineto{\pgfqpoint{1.596570in}{0.576295in}}%
\pgfpathclose%
\pgfpathmoveto{\pgfqpoint{1.894758in}{0.542492in}}%
\pgfpathlineto{\pgfqpoint{1.910415in}{0.539201in}}%
\pgfpathlineto{\pgfqpoint{1.910415in}{0.549073in}}%
\pgfpathlineto{\pgfqpoint{1.910415in}{0.562684in}}%
\pgfpathlineto{\pgfqpoint{1.910415in}{0.576026in}}%
\pgfpathlineto{\pgfqpoint{1.908788in}{0.576295in}}%
\pgfpathlineto{\pgfqpoint{1.894758in}{0.578886in}}%
\pgfpathlineto{\pgfqpoint{1.879102in}{0.586844in}}%
\pgfpathlineto{\pgfqpoint{1.875011in}{0.589907in}}%
\pgfpathlineto{\pgfqpoint{1.863445in}{0.600517in}}%
\pgfpathlineto{\pgfqpoint{1.860698in}{0.603518in}}%
\pgfpathlineto{\pgfqpoint{1.852844in}{0.617129in}}%
\pgfpathlineto{\pgfqpoint{1.850900in}{0.630740in}}%
\pgfpathlineto{\pgfqpoint{1.855357in}{0.644351in}}%
\pgfpathlineto{\pgfqpoint{1.863445in}{0.655551in}}%
\pgfpathlineto{\pgfqpoint{1.865542in}{0.657962in}}%
\pgfpathlineto{\pgfqpoint{1.879102in}{0.669194in}}%
\pgfpathlineto{\pgfqpoint{1.883305in}{0.671573in}}%
\pgfpathlineto{\pgfqpoint{1.894758in}{0.677094in}}%
\pgfpathlineto{\pgfqpoint{1.910415in}{0.679811in}}%
\pgfpathlineto{\pgfqpoint{1.910415in}{0.685184in}}%
\pgfpathlineto{\pgfqpoint{1.910415in}{0.698795in}}%
\pgfpathlineto{\pgfqpoint{1.910415in}{0.712407in}}%
\pgfpathlineto{\pgfqpoint{1.910415in}{0.716899in}}%
\pgfpathlineto{\pgfqpoint{1.894758in}{0.713378in}}%
\pgfpathlineto{\pgfqpoint{1.893243in}{0.712407in}}%
\pgfpathlineto{\pgfqpoint{1.879102in}{0.704781in}}%
\pgfpathlineto{\pgfqpoint{1.871492in}{0.698795in}}%
\pgfpathlineto{\pgfqpoint{1.863445in}{0.692970in}}%
\pgfpathlineto{\pgfqpoint{1.854269in}{0.685184in}}%
\pgfpathlineto{\pgfqpoint{1.847789in}{0.679633in}}%
\pgfpathlineto{\pgfqpoint{1.838659in}{0.671573in}}%
\pgfpathlineto{\pgfqpoint{1.832132in}{0.665019in}}%
\pgfpathlineto{\pgfqpoint{1.824629in}{0.657962in}}%
\pgfpathlineto{\pgfqpoint{1.816476in}{0.647399in}}%
\pgfpathlineto{\pgfqpoint{1.813732in}{0.644351in}}%
\pgfpathlineto{\pgfqpoint{1.808322in}{0.630740in}}%
\pgfpathlineto{\pgfqpoint{1.810682in}{0.617129in}}%
\pgfpathlineto{\pgfqpoint{1.816476in}{0.608971in}}%
\pgfpathlineto{\pgfqpoint{1.819828in}{0.603518in}}%
\pgfpathlineto{\pgfqpoint{1.832132in}{0.590658in}}%
\pgfpathlineto{\pgfqpoint{1.832810in}{0.589907in}}%
\pgfpathlineto{\pgfqpoint{1.847671in}{0.576295in}}%
\pgfpathlineto{\pgfqpoint{1.847789in}{0.576193in}}%
\pgfpathlineto{\pgfqpoint{1.863445in}{0.563033in}}%
\pgfpathlineto{\pgfqpoint{1.863931in}{0.562684in}}%
\pgfpathlineto{\pgfqpoint{1.879102in}{0.551333in}}%
\pgfpathlineto{\pgfqpoint{1.883494in}{0.549073in}}%
\pgfpathlineto{\pgfqpoint{1.894758in}{0.542492in}}%
\pgfpathclose%
\pgfpathmoveto{\pgfqpoint{0.376072in}{0.812043in}}%
\pgfpathlineto{\pgfqpoint{0.391728in}{0.820855in}}%
\pgfpathlineto{\pgfqpoint{0.392252in}{0.821295in}}%
\pgfpathlineto{\pgfqpoint{0.407385in}{0.832522in}}%
\pgfpathlineto{\pgfqpoint{0.410122in}{0.834907in}}%
\pgfpathlineto{\pgfqpoint{0.423041in}{0.845800in}}%
\pgfpathlineto{\pgfqpoint{0.426171in}{0.848518in}}%
\pgfpathlineto{\pgfqpoint{0.438698in}{0.860251in}}%
\pgfpathlineto{\pgfqpoint{0.440848in}{0.862129in}}%
\pgfpathlineto{\pgfqpoint{0.453145in}{0.875740in}}%
\pgfpathlineto{\pgfqpoint{0.454354in}{0.878027in}}%
\pgfpathlineto{\pgfqpoint{0.461241in}{0.889351in}}%
\pgfpathlineto{\pgfqpoint{0.462031in}{0.902962in}}%
\pgfpathlineto{\pgfqpoint{0.455173in}{0.916573in}}%
\pgfpathlineto{\pgfqpoint{0.454354in}{0.917402in}}%
\pgfpathlineto{\pgfqpoint{0.443583in}{0.930184in}}%
\pgfpathlineto{\pgfqpoint{0.438698in}{0.934596in}}%
\pgfpathlineto{\pgfqpoint{0.429181in}{0.943795in}}%
\pgfpathlineto{\pgfqpoint{0.423041in}{0.949158in}}%
\pgfpathlineto{\pgfqpoint{0.413313in}{0.957407in}}%
\pgfpathlineto{\pgfqpoint{0.407385in}{0.962493in}}%
\pgfpathlineto{\pgfqpoint{0.395731in}{0.971018in}}%
\pgfpathlineto{\pgfqpoint{0.391728in}{0.974265in}}%
\pgfpathlineto{\pgfqpoint{0.376072in}{0.982889in}}%
\pgfpathlineto{\pgfqpoint{0.367333in}{0.984629in}}%
\pgfpathlineto{\pgfqpoint{0.360415in}{0.986264in}}%
\pgfpathlineto{\pgfqpoint{0.360415in}{0.984629in}}%
\pgfpathlineto{\pgfqpoint{0.360415in}{0.971018in}}%
\pgfpathlineto{\pgfqpoint{0.360415in}{0.957407in}}%
\pgfpathlineto{\pgfqpoint{0.360415in}{0.949331in}}%
\pgfpathlineto{\pgfqpoint{0.376072in}{0.946680in}}%
\pgfpathlineto{\pgfqpoint{0.382121in}{0.943795in}}%
\pgfpathlineto{\pgfqpoint{0.391728in}{0.938560in}}%
\pgfpathlineto{\pgfqpoint{0.402259in}{0.930184in}}%
\pgfpathlineto{\pgfqpoint{0.407385in}{0.924786in}}%
\pgfpathlineto{\pgfqpoint{0.413887in}{0.916573in}}%
\pgfpathlineto{\pgfqpoint{0.419537in}{0.902962in}}%
\pgfpathlineto{\pgfqpoint{0.418887in}{0.889351in}}%
\pgfpathlineto{\pgfqpoint{0.412101in}{0.875740in}}%
\pgfpathlineto{\pgfqpoint{0.407385in}{0.870233in}}%
\pgfpathlineto{\pgfqpoint{0.399095in}{0.862129in}}%
\pgfpathlineto{\pgfqpoint{0.391728in}{0.856463in}}%
\pgfpathlineto{\pgfqpoint{0.376682in}{0.848518in}}%
\pgfpathlineto{\pgfqpoint{0.376072in}{0.848228in}}%
\pgfpathlineto{\pgfqpoint{0.360415in}{0.845630in}}%
\pgfpathlineto{\pgfqpoint{0.360415in}{0.834907in}}%
\pgfpathlineto{\pgfqpoint{0.360415in}{0.821295in}}%
\pgfpathlineto{\pgfqpoint{0.360415in}{0.808837in}}%
\pgfpathlineto{\pgfqpoint{0.376072in}{0.812043in}}%
\pgfpathclose%
\pgfpathmoveto{\pgfqpoint{0.642233in}{0.818730in}}%
\pgfpathlineto{\pgfqpoint{0.657890in}{0.810906in}}%
\pgfpathlineto{\pgfqpoint{0.673546in}{0.808968in}}%
\pgfpathlineto{\pgfqpoint{0.689203in}{0.813409in}}%
\pgfpathlineto{\pgfqpoint{0.701802in}{0.821295in}}%
\pgfpathlineto{\pgfqpoint{0.704859in}{0.823010in}}%
\pgfpathlineto{\pgfqpoint{0.720300in}{0.834907in}}%
\pgfpathlineto{\pgfqpoint{0.720516in}{0.835069in}}%
\pgfpathlineto{\pgfqpoint{0.736173in}{0.848457in}}%
\pgfpathlineto{\pgfqpoint{0.736243in}{0.848518in}}%
\pgfpathlineto{\pgfqpoint{0.750804in}{0.862129in}}%
\pgfpathlineto{\pgfqpoint{0.751829in}{0.863332in}}%
\pgfpathlineto{\pgfqpoint{0.763236in}{0.875740in}}%
\pgfpathlineto{\pgfqpoint{0.767486in}{0.883574in}}%
\pgfpathlineto{\pgfqpoint{0.771174in}{0.889351in}}%
\pgfpathlineto{\pgfqpoint{0.771993in}{0.902962in}}%
\pgfpathlineto{\pgfqpoint{0.767486in}{0.911509in}}%
\pgfpathlineto{\pgfqpoint{0.765205in}{0.916573in}}%
\pgfpathlineto{\pgfqpoint{0.753550in}{0.930184in}}%
\pgfpathlineto{\pgfqpoint{0.751829in}{0.931697in}}%
\pgfpathlineto{\pgfqpoint{0.739230in}{0.943795in}}%
\pgfpathlineto{\pgfqpoint{0.736173in}{0.946447in}}%
\pgfpathlineto{\pgfqpoint{0.723438in}{0.957407in}}%
\pgfpathlineto{\pgfqpoint{0.720516in}{0.959954in}}%
\pgfpathlineto{\pgfqpoint{0.706064in}{0.971018in}}%
\pgfpathlineto{\pgfqpoint{0.704859in}{0.972043in}}%
\pgfpathlineto{\pgfqpoint{0.689203in}{0.981552in}}%
\pgfpathlineto{\pgfqpoint{0.678055in}{0.984629in}}%
\pgfpathlineto{\pgfqpoint{0.673546in}{0.986114in}}%
\pgfpathlineto{\pgfqpoint{0.663095in}{0.984629in}}%
\pgfpathlineto{\pgfqpoint{0.657890in}{0.984002in}}%
\pgfpathlineto{\pgfqpoint{0.642233in}{0.976345in}}%
\pgfpathlineto{\pgfqpoint{0.635281in}{0.971018in}}%
\pgfpathlineto{\pgfqpoint{0.626577in}{0.964985in}}%
\pgfpathlineto{\pgfqpoint{0.617554in}{0.957407in}}%
\pgfpathlineto{\pgfqpoint{0.610920in}{0.951892in}}%
\pgfpathlineto{\pgfqpoint{0.601676in}{0.943795in}}%
\pgfpathlineto{\pgfqpoint{0.595263in}{0.937581in}}%
\pgfpathlineto{\pgfqpoint{0.587260in}{0.930184in}}%
\pgfpathlineto{\pgfqpoint{0.579607in}{0.920905in}}%
\pgfpathlineto{\pgfqpoint{0.575516in}{0.916573in}}%
\pgfpathlineto{\pgfqpoint{0.568868in}{0.902962in}}%
\pgfpathlineto{\pgfqpoint{0.569633in}{0.889351in}}%
\pgfpathlineto{\pgfqpoint{0.577617in}{0.875740in}}%
\pgfpathlineto{\pgfqpoint{0.579607in}{0.873792in}}%
\pgfpathlineto{\pgfqpoint{0.589971in}{0.862129in}}%
\pgfpathlineto{\pgfqpoint{0.595263in}{0.857399in}}%
\pgfpathlineto{\pgfqpoint{0.604720in}{0.848518in}}%
\pgfpathlineto{\pgfqpoint{0.610920in}{0.843120in}}%
\pgfpathlineto{\pgfqpoint{0.620856in}{0.834907in}}%
\pgfpathlineto{\pgfqpoint{0.626577in}{0.830029in}}%
\pgfpathlineto{\pgfqpoint{0.639000in}{0.821295in}}%
\pgfpathlineto{\pgfqpoint{0.642233in}{0.818730in}}%
\pgfpathclose%
\pgfpathmoveto{\pgfqpoint{0.654984in}{0.848518in}}%
\pgfpathlineto{\pgfqpoint{0.642233in}{0.854466in}}%
\pgfpathlineto{\pgfqpoint{0.631685in}{0.862129in}}%
\pgfpathlineto{\pgfqpoint{0.626577in}{0.866860in}}%
\pgfpathlineto{\pgfqpoint{0.618809in}{0.875740in}}%
\pgfpathlineto{\pgfqpoint{0.611786in}{0.889351in}}%
\pgfpathlineto{\pgfqpoint{0.611113in}{0.902962in}}%
\pgfpathlineto{\pgfqpoint{0.616960in}{0.916573in}}%
\pgfpathlineto{\pgfqpoint{0.626577in}{0.928463in}}%
\pgfpathlineto{\pgfqpoint{0.628302in}{0.930184in}}%
\pgfpathlineto{\pgfqpoint{0.642233in}{0.940650in}}%
\pgfpathlineto{\pgfqpoint{0.648766in}{0.943795in}}%
\pgfpathlineto{\pgfqpoint{0.657890in}{0.947621in}}%
\pgfpathlineto{\pgfqpoint{0.673546in}{0.949223in}}%
\pgfpathlineto{\pgfqpoint{0.689203in}{0.945551in}}%
\pgfpathlineto{\pgfqpoint{0.692506in}{0.943795in}}%
\pgfpathlineto{\pgfqpoint{0.704859in}{0.936327in}}%
\pgfpathlineto{\pgfqpoint{0.712218in}{0.930184in}}%
\pgfpathlineto{\pgfqpoint{0.720516in}{0.921038in}}%
\pgfpathlineto{\pgfqpoint{0.723995in}{0.916573in}}%
\pgfpathlineto{\pgfqpoint{0.729482in}{0.902962in}}%
\pgfpathlineto{\pgfqpoint{0.728851in}{0.889351in}}%
\pgfpathlineto{\pgfqpoint{0.722261in}{0.875740in}}%
\pgfpathlineto{\pgfqpoint{0.720516in}{0.873670in}}%
\pgfpathlineto{\pgfqpoint{0.709235in}{0.862129in}}%
\pgfpathlineto{\pgfqpoint{0.704859in}{0.858597in}}%
\pgfpathlineto{\pgfqpoint{0.689203in}{0.849466in}}%
\pgfpathlineto{\pgfqpoint{0.685591in}{0.848518in}}%
\pgfpathlineto{\pgfqpoint{0.673546in}{0.845737in}}%
\pgfpathlineto{\pgfqpoint{0.657890in}{0.847307in}}%
\pgfpathlineto{\pgfqpoint{0.654984in}{0.848518in}}%
\pgfpathclose%
\pgfpathmoveto{\pgfqpoint{0.955364in}{0.816768in}}%
\pgfpathlineto{\pgfqpoint{0.971021in}{0.810008in}}%
\pgfpathlineto{\pgfqpoint{0.986678in}{0.809360in}}%
\pgfpathlineto{\pgfqpoint{1.002334in}{0.814989in}}%
\pgfpathlineto{\pgfqpoint{1.011524in}{0.821295in}}%
\pgfpathlineto{\pgfqpoint{1.017991in}{0.825255in}}%
\pgfpathlineto{\pgfqpoint{1.030019in}{0.834907in}}%
\pgfpathlineto{\pgfqpoint{1.033647in}{0.837740in}}%
\pgfpathlineto{\pgfqpoint{1.046119in}{0.848518in}}%
\pgfpathlineto{\pgfqpoint{1.049304in}{0.851487in}}%
\pgfpathlineto{\pgfqpoint{1.060812in}{0.862129in}}%
\pgfpathlineto{\pgfqpoint{1.064960in}{0.866949in}}%
\pgfpathlineto{\pgfqpoint{1.073296in}{0.875740in}}%
\pgfpathlineto{\pgfqpoint{1.080617in}{0.888848in}}%
\pgfpathlineto{\pgfqpoint{1.080956in}{0.889351in}}%
\pgfpathlineto{\pgfqpoint{1.081810in}{0.902962in}}%
\pgfpathlineto{\pgfqpoint{1.080617in}{0.905105in}}%
\pgfpathlineto{\pgfqpoint{1.075299in}{0.916573in}}%
\pgfpathlineto{\pgfqpoint{1.064960in}{0.928366in}}%
\pgfpathlineto{\pgfqpoint{1.063508in}{0.930184in}}%
\pgfpathlineto{\pgfqpoint{1.049304in}{0.943767in}}%
\pgfpathlineto{\pgfqpoint{1.049275in}{0.943795in}}%
\pgfpathlineto{\pgfqpoint{1.033647in}{0.957381in}}%
\pgfpathlineto{\pgfqpoint{1.033615in}{0.957407in}}%
\pgfpathlineto{\pgfqpoint{1.017991in}{0.969755in}}%
\pgfpathlineto{\pgfqpoint{1.015900in}{0.971018in}}%
\pgfpathlineto{\pgfqpoint{1.002334in}{0.980006in}}%
\pgfpathlineto{\pgfqpoint{0.989143in}{0.984629in}}%
\pgfpathlineto{\pgfqpoint{0.986678in}{0.985665in}}%
\pgfpathlineto{\pgfqpoint{0.971021in}{0.984923in}}%
\pgfpathlineto{\pgfqpoint{0.970443in}{0.984629in}}%
\pgfpathlineto{\pgfqpoint{0.955364in}{0.978265in}}%
\pgfpathlineto{\pgfqpoint{0.945252in}{0.971018in}}%
\pgfpathlineto{\pgfqpoint{0.939708in}{0.967411in}}%
\pgfpathlineto{\pgfqpoint{0.927467in}{0.957407in}}%
\pgfpathlineto{\pgfqpoint{0.924051in}{0.954638in}}%
\pgfpathlineto{\pgfqpoint{0.911654in}{0.943795in}}%
\pgfpathlineto{\pgfqpoint{0.908395in}{0.940641in}}%
\pgfpathlineto{\pgfqpoint{0.897293in}{0.930184in}}%
\pgfpathlineto{\pgfqpoint{0.892738in}{0.924562in}}%
\pgfpathlineto{\pgfqpoint{0.885484in}{0.916573in}}%
\pgfpathlineto{\pgfqpoint{0.879009in}{0.902962in}}%
\pgfpathlineto{\pgfqpoint{0.879755in}{0.889351in}}%
\pgfpathlineto{\pgfqpoint{0.887531in}{0.875740in}}%
\pgfpathlineto{\pgfqpoint{0.892738in}{0.870438in}}%
\pgfpathlineto{\pgfqpoint{0.899991in}{0.862129in}}%
\pgfpathlineto{\pgfqpoint{0.908395in}{0.854475in}}%
\pgfpathlineto{\pgfqpoint{0.914747in}{0.848518in}}%
\pgfpathlineto{\pgfqpoint{0.924051in}{0.840429in}}%
\pgfpathlineto{\pgfqpoint{0.930904in}{0.834907in}}%
\pgfpathlineto{\pgfqpoint{0.939708in}{0.827601in}}%
\pgfpathlineto{\pgfqpoint{0.949265in}{0.821295in}}%
\pgfpathlineto{\pgfqpoint{0.955364in}{0.816768in}}%
\pgfpathclose%
\pgfpathmoveto{\pgfqpoint{0.965590in}{0.848518in}}%
\pgfpathlineto{\pgfqpoint{0.955364in}{0.852623in}}%
\pgfpathlineto{\pgfqpoint{0.941373in}{0.862129in}}%
\pgfpathlineto{\pgfqpoint{0.939708in}{0.863576in}}%
\pgfpathlineto{\pgfqpoint{0.928773in}{0.875740in}}%
\pgfpathlineto{\pgfqpoint{0.924051in}{0.884629in}}%
\pgfpathlineto{\pgfqpoint{0.921822in}{0.889351in}}%
\pgfpathlineto{\pgfqpoint{0.921218in}{0.902962in}}%
\pgfpathlineto{\pgfqpoint{0.924051in}{0.910228in}}%
\pgfpathlineto{\pgfqpoint{0.926850in}{0.916573in}}%
\pgfpathlineto{\pgfqpoint{0.938236in}{0.930184in}}%
\pgfpathlineto{\pgfqpoint{0.939708in}{0.931508in}}%
\pgfpathlineto{\pgfqpoint{0.955364in}{0.942579in}}%
\pgfpathlineto{\pgfqpoint{0.958300in}{0.943795in}}%
\pgfpathlineto{\pgfqpoint{0.971021in}{0.948363in}}%
\pgfpathlineto{\pgfqpoint{0.986678in}{0.948899in}}%
\pgfpathlineto{\pgfqpoint{1.002334in}{0.944244in}}%
\pgfpathlineto{\pgfqpoint{1.003104in}{0.943795in}}%
\pgfpathlineto{\pgfqpoint{1.017991in}{0.933970in}}%
\pgfpathlineto{\pgfqpoint{1.022345in}{0.930184in}}%
\pgfpathlineto{\pgfqpoint{1.033647in}{0.917243in}}%
\pgfpathlineto{\pgfqpoint{1.034163in}{0.916573in}}%
\pgfpathlineto{\pgfqpoint{1.039518in}{0.902962in}}%
\pgfpathlineto{\pgfqpoint{1.038901in}{0.889351in}}%
\pgfpathlineto{\pgfqpoint{1.033647in}{0.878292in}}%
\pgfpathlineto{\pgfqpoint{1.032248in}{0.875740in}}%
\pgfpathlineto{\pgfqpoint{1.019513in}{0.862129in}}%
\pgfpathlineto{\pgfqpoint{1.017991in}{0.860849in}}%
\pgfpathlineto{\pgfqpoint{1.002334in}{0.850951in}}%
\pgfpathlineto{\pgfqpoint{0.995035in}{0.848518in}}%
\pgfpathlineto{\pgfqpoint{0.986678in}{0.846054in}}%
\pgfpathlineto{\pgfqpoint{0.971021in}{0.846579in}}%
\pgfpathlineto{\pgfqpoint{0.965590in}{0.848518in}}%
\pgfpathclose%
\pgfpathmoveto{\pgfqpoint{1.268496in}{0.814989in}}%
\pgfpathlineto{\pgfqpoint{1.284152in}{0.809360in}}%
\pgfpathlineto{\pgfqpoint{1.299809in}{0.810008in}}%
\pgfpathlineto{\pgfqpoint{1.315466in}{0.816768in}}%
\pgfpathlineto{\pgfqpoint{1.321565in}{0.821295in}}%
\pgfpathlineto{\pgfqpoint{1.331122in}{0.827601in}}%
\pgfpathlineto{\pgfqpoint{1.339926in}{0.834907in}}%
\pgfpathlineto{\pgfqpoint{1.346779in}{0.840429in}}%
\pgfpathlineto{\pgfqpoint{1.356083in}{0.848518in}}%
\pgfpathlineto{\pgfqpoint{1.362435in}{0.854475in}}%
\pgfpathlineto{\pgfqpoint{1.370839in}{0.862129in}}%
\pgfpathlineto{\pgfqpoint{1.378092in}{0.870438in}}%
\pgfpathlineto{\pgfqpoint{1.383299in}{0.875740in}}%
\pgfpathlineto{\pgfqpoint{1.391075in}{0.889351in}}%
\pgfpathlineto{\pgfqpoint{1.391821in}{0.902962in}}%
\pgfpathlineto{\pgfqpoint{1.385346in}{0.916573in}}%
\pgfpathlineto{\pgfqpoint{1.378092in}{0.924562in}}%
\pgfpathlineto{\pgfqpoint{1.373537in}{0.930184in}}%
\pgfpathlineto{\pgfqpoint{1.362435in}{0.940641in}}%
\pgfpathlineto{\pgfqpoint{1.359176in}{0.943795in}}%
\pgfpathlineto{\pgfqpoint{1.346779in}{0.954638in}}%
\pgfpathlineto{\pgfqpoint{1.343363in}{0.957407in}}%
\pgfpathlineto{\pgfqpoint{1.331122in}{0.967411in}}%
\pgfpathlineto{\pgfqpoint{1.325578in}{0.971018in}}%
\pgfpathlineto{\pgfqpoint{1.315466in}{0.978265in}}%
\pgfpathlineto{\pgfqpoint{1.300387in}{0.984629in}}%
\pgfpathlineto{\pgfqpoint{1.299809in}{0.984923in}}%
\pgfpathlineto{\pgfqpoint{1.284152in}{0.985665in}}%
\pgfpathlineto{\pgfqpoint{1.281687in}{0.984629in}}%
\pgfpathlineto{\pgfqpoint{1.268496in}{0.980006in}}%
\pgfpathlineto{\pgfqpoint{1.254930in}{0.971018in}}%
\pgfpathlineto{\pgfqpoint{1.252839in}{0.969755in}}%
\pgfpathlineto{\pgfqpoint{1.237215in}{0.957407in}}%
\pgfpathlineto{\pgfqpoint{1.237183in}{0.957381in}}%
\pgfpathlineto{\pgfqpoint{1.221555in}{0.943795in}}%
\pgfpathlineto{\pgfqpoint{1.221526in}{0.943767in}}%
\pgfpathlineto{\pgfqpoint{1.207322in}{0.930184in}}%
\pgfpathlineto{\pgfqpoint{1.205870in}{0.928366in}}%
\pgfpathlineto{\pgfqpoint{1.195531in}{0.916573in}}%
\pgfpathlineto{\pgfqpoint{1.190213in}{0.905105in}}%
\pgfpathlineto{\pgfqpoint{1.189020in}{0.902962in}}%
\pgfpathlineto{\pgfqpoint{1.189874in}{0.889351in}}%
\pgfpathlineto{\pgfqpoint{1.190213in}{0.888848in}}%
\pgfpathlineto{\pgfqpoint{1.197534in}{0.875740in}}%
\pgfpathlineto{\pgfqpoint{1.205870in}{0.866949in}}%
\pgfpathlineto{\pgfqpoint{1.210018in}{0.862129in}}%
\pgfpathlineto{\pgfqpoint{1.221526in}{0.851487in}}%
\pgfpathlineto{\pgfqpoint{1.224711in}{0.848518in}}%
\pgfpathlineto{\pgfqpoint{1.237183in}{0.837740in}}%
\pgfpathlineto{\pgfqpoint{1.240811in}{0.834907in}}%
\pgfpathlineto{\pgfqpoint{1.252839in}{0.825255in}}%
\pgfpathlineto{\pgfqpoint{1.259306in}{0.821295in}}%
\pgfpathlineto{\pgfqpoint{1.268496in}{0.814989in}}%
\pgfpathclose%
\pgfpathmoveto{\pgfqpoint{1.275795in}{0.848518in}}%
\pgfpathlineto{\pgfqpoint{1.268496in}{0.850951in}}%
\pgfpathlineto{\pgfqpoint{1.252839in}{0.860849in}}%
\pgfpathlineto{\pgfqpoint{1.251317in}{0.862129in}}%
\pgfpathlineto{\pgfqpoint{1.238582in}{0.875740in}}%
\pgfpathlineto{\pgfqpoint{1.237183in}{0.878292in}}%
\pgfpathlineto{\pgfqpoint{1.231929in}{0.889351in}}%
\pgfpathlineto{\pgfqpoint{1.231312in}{0.902962in}}%
\pgfpathlineto{\pgfqpoint{1.236667in}{0.916573in}}%
\pgfpathlineto{\pgfqpoint{1.237183in}{0.917243in}}%
\pgfpathlineto{\pgfqpoint{1.248485in}{0.930184in}}%
\pgfpathlineto{\pgfqpoint{1.252839in}{0.933970in}}%
\pgfpathlineto{\pgfqpoint{1.267726in}{0.943795in}}%
\pgfpathlineto{\pgfqpoint{1.268496in}{0.944244in}}%
\pgfpathlineto{\pgfqpoint{1.284152in}{0.948899in}}%
\pgfpathlineto{\pgfqpoint{1.299809in}{0.948363in}}%
\pgfpathlineto{\pgfqpoint{1.312530in}{0.943795in}}%
\pgfpathlineto{\pgfqpoint{1.315466in}{0.942579in}}%
\pgfpathlineto{\pgfqpoint{1.331122in}{0.931508in}}%
\pgfpathlineto{\pgfqpoint{1.332594in}{0.930184in}}%
\pgfpathlineto{\pgfqpoint{1.343980in}{0.916573in}}%
\pgfpathlineto{\pgfqpoint{1.346779in}{0.910228in}}%
\pgfpathlineto{\pgfqpoint{1.349612in}{0.902962in}}%
\pgfpathlineto{\pgfqpoint{1.349008in}{0.889351in}}%
\pgfpathlineto{\pgfqpoint{1.346779in}{0.884629in}}%
\pgfpathlineto{\pgfqpoint{1.342057in}{0.875740in}}%
\pgfpathlineto{\pgfqpoint{1.331122in}{0.863576in}}%
\pgfpathlineto{\pgfqpoint{1.329457in}{0.862129in}}%
\pgfpathlineto{\pgfqpoint{1.315466in}{0.852623in}}%
\pgfpathlineto{\pgfqpoint{1.305240in}{0.848518in}}%
\pgfpathlineto{\pgfqpoint{1.299809in}{0.846579in}}%
\pgfpathlineto{\pgfqpoint{1.284152in}{0.846054in}}%
\pgfpathlineto{\pgfqpoint{1.275795in}{0.848518in}}%
\pgfpathclose%
\pgfpathmoveto{\pgfqpoint{1.581627in}{0.813409in}}%
\pgfpathlineto{\pgfqpoint{1.597284in}{0.808968in}}%
\pgfpathlineto{\pgfqpoint{1.612940in}{0.810906in}}%
\pgfpathlineto{\pgfqpoint{1.628597in}{0.818730in}}%
\pgfpathlineto{\pgfqpoint{1.631830in}{0.821295in}}%
\pgfpathlineto{\pgfqpoint{1.644253in}{0.830029in}}%
\pgfpathlineto{\pgfqpoint{1.649974in}{0.834907in}}%
\pgfpathlineto{\pgfqpoint{1.659910in}{0.843120in}}%
\pgfpathlineto{\pgfqpoint{1.666110in}{0.848518in}}%
\pgfpathlineto{\pgfqpoint{1.675567in}{0.857399in}}%
\pgfpathlineto{\pgfqpoint{1.680859in}{0.862129in}}%
\pgfpathlineto{\pgfqpoint{1.691223in}{0.873792in}}%
\pgfpathlineto{\pgfqpoint{1.693213in}{0.875740in}}%
\pgfpathlineto{\pgfqpoint{1.701197in}{0.889351in}}%
\pgfpathlineto{\pgfqpoint{1.701962in}{0.902962in}}%
\pgfpathlineto{\pgfqpoint{1.695314in}{0.916573in}}%
\pgfpathlineto{\pgfqpoint{1.691223in}{0.920905in}}%
\pgfpathlineto{\pgfqpoint{1.683570in}{0.930184in}}%
\pgfpathlineto{\pgfqpoint{1.675567in}{0.937581in}}%
\pgfpathlineto{\pgfqpoint{1.669154in}{0.943795in}}%
\pgfpathlineto{\pgfqpoint{1.659910in}{0.951892in}}%
\pgfpathlineto{\pgfqpoint{1.653276in}{0.957407in}}%
\pgfpathlineto{\pgfqpoint{1.644253in}{0.964985in}}%
\pgfpathlineto{\pgfqpoint{1.635549in}{0.971018in}}%
\pgfpathlineto{\pgfqpoint{1.628597in}{0.976345in}}%
\pgfpathlineto{\pgfqpoint{1.612940in}{0.984002in}}%
\pgfpathlineto{\pgfqpoint{1.607735in}{0.984629in}}%
\pgfpathlineto{\pgfqpoint{1.597284in}{0.986114in}}%
\pgfpathlineto{\pgfqpoint{1.592775in}{0.984629in}}%
\pgfpathlineto{\pgfqpoint{1.581627in}{0.981552in}}%
\pgfpathlineto{\pgfqpoint{1.565971in}{0.972043in}}%
\pgfpathlineto{\pgfqpoint{1.564766in}{0.971018in}}%
\pgfpathlineto{\pgfqpoint{1.550314in}{0.959954in}}%
\pgfpathlineto{\pgfqpoint{1.547392in}{0.957407in}}%
\pgfpathlineto{\pgfqpoint{1.534657in}{0.946447in}}%
\pgfpathlineto{\pgfqpoint{1.531600in}{0.943795in}}%
\pgfpathlineto{\pgfqpoint{1.519001in}{0.931697in}}%
\pgfpathlineto{\pgfqpoint{1.517280in}{0.930184in}}%
\pgfpathlineto{\pgfqpoint{1.505625in}{0.916573in}}%
\pgfpathlineto{\pgfqpoint{1.503344in}{0.911509in}}%
\pgfpathlineto{\pgfqpoint{1.498837in}{0.902962in}}%
\pgfpathlineto{\pgfqpoint{1.499656in}{0.889351in}}%
\pgfpathlineto{\pgfqpoint{1.503344in}{0.883574in}}%
\pgfpathlineto{\pgfqpoint{1.507594in}{0.875740in}}%
\pgfpathlineto{\pgfqpoint{1.519001in}{0.863332in}}%
\pgfpathlineto{\pgfqpoint{1.520026in}{0.862129in}}%
\pgfpathlineto{\pgfqpoint{1.534587in}{0.848518in}}%
\pgfpathlineto{\pgfqpoint{1.534657in}{0.848457in}}%
\pgfpathlineto{\pgfqpoint{1.550314in}{0.835069in}}%
\pgfpathlineto{\pgfqpoint{1.550530in}{0.834907in}}%
\pgfpathlineto{\pgfqpoint{1.565971in}{0.823010in}}%
\pgfpathlineto{\pgfqpoint{1.569028in}{0.821295in}}%
\pgfpathlineto{\pgfqpoint{1.581627in}{0.813409in}}%
\pgfpathclose%
\pgfpathmoveto{\pgfqpoint{1.585239in}{0.848518in}}%
\pgfpathlineto{\pgfqpoint{1.581627in}{0.849466in}}%
\pgfpathlineto{\pgfqpoint{1.565971in}{0.858597in}}%
\pgfpathlineto{\pgfqpoint{1.561595in}{0.862129in}}%
\pgfpathlineto{\pgfqpoint{1.550314in}{0.873670in}}%
\pgfpathlineto{\pgfqpoint{1.548569in}{0.875740in}}%
\pgfpathlineto{\pgfqpoint{1.541979in}{0.889351in}}%
\pgfpathlineto{\pgfqpoint{1.541348in}{0.902962in}}%
\pgfpathlineto{\pgfqpoint{1.546835in}{0.916573in}}%
\pgfpathlineto{\pgfqpoint{1.550314in}{0.921038in}}%
\pgfpathlineto{\pgfqpoint{1.558612in}{0.930184in}}%
\pgfpathlineto{\pgfqpoint{1.565971in}{0.936327in}}%
\pgfpathlineto{\pgfqpoint{1.578324in}{0.943795in}}%
\pgfpathlineto{\pgfqpoint{1.581627in}{0.945551in}}%
\pgfpathlineto{\pgfqpoint{1.597284in}{0.949223in}}%
\pgfpathlineto{\pgfqpoint{1.612940in}{0.947621in}}%
\pgfpathlineto{\pgfqpoint{1.622064in}{0.943795in}}%
\pgfpathlineto{\pgfqpoint{1.628597in}{0.940650in}}%
\pgfpathlineto{\pgfqpoint{1.642528in}{0.930184in}}%
\pgfpathlineto{\pgfqpoint{1.644253in}{0.928463in}}%
\pgfpathlineto{\pgfqpoint{1.653870in}{0.916573in}}%
\pgfpathlineto{\pgfqpoint{1.659717in}{0.902962in}}%
\pgfpathlineto{\pgfqpoint{1.659044in}{0.889351in}}%
\pgfpathlineto{\pgfqpoint{1.652021in}{0.875740in}}%
\pgfpathlineto{\pgfqpoint{1.644253in}{0.866860in}}%
\pgfpathlineto{\pgfqpoint{1.639145in}{0.862129in}}%
\pgfpathlineto{\pgfqpoint{1.628597in}{0.854466in}}%
\pgfpathlineto{\pgfqpoint{1.615846in}{0.848518in}}%
\pgfpathlineto{\pgfqpoint{1.612940in}{0.847307in}}%
\pgfpathlineto{\pgfqpoint{1.597284in}{0.845737in}}%
\pgfpathlineto{\pgfqpoint{1.585239in}{0.848518in}}%
\pgfpathclose%
\pgfpathmoveto{\pgfqpoint{1.879102in}{0.820855in}}%
\pgfpathlineto{\pgfqpoint{1.894758in}{0.812043in}}%
\pgfpathlineto{\pgfqpoint{1.910415in}{0.808837in}}%
\pgfpathlineto{\pgfqpoint{1.910415in}{0.821295in}}%
\pgfpathlineto{\pgfqpoint{1.910415in}{0.834907in}}%
\pgfpathlineto{\pgfqpoint{1.910415in}{0.845630in}}%
\pgfpathlineto{\pgfqpoint{1.894758in}{0.848228in}}%
\pgfpathlineto{\pgfqpoint{1.894148in}{0.848518in}}%
\pgfpathlineto{\pgfqpoint{1.879102in}{0.856463in}}%
\pgfpathlineto{\pgfqpoint{1.871735in}{0.862129in}}%
\pgfpathlineto{\pgfqpoint{1.863445in}{0.870233in}}%
\pgfpathlineto{\pgfqpoint{1.858729in}{0.875740in}}%
\pgfpathlineto{\pgfqpoint{1.851943in}{0.889351in}}%
\pgfpathlineto{\pgfqpoint{1.851293in}{0.902962in}}%
\pgfpathlineto{\pgfqpoint{1.856943in}{0.916573in}}%
\pgfpathlineto{\pgfqpoint{1.863445in}{0.924786in}}%
\pgfpathlineto{\pgfqpoint{1.868571in}{0.930184in}}%
\pgfpathlineto{\pgfqpoint{1.879102in}{0.938560in}}%
\pgfpathlineto{\pgfqpoint{1.888709in}{0.943795in}}%
\pgfpathlineto{\pgfqpoint{1.894758in}{0.946680in}}%
\pgfpathlineto{\pgfqpoint{1.910415in}{0.949331in}}%
\pgfpathlineto{\pgfqpoint{1.910415in}{0.957407in}}%
\pgfpathlineto{\pgfqpoint{1.910415in}{0.971018in}}%
\pgfpathlineto{\pgfqpoint{1.910415in}{0.984629in}}%
\pgfpathlineto{\pgfqpoint{1.910415in}{0.986264in}}%
\pgfpathlineto{\pgfqpoint{1.903497in}{0.984629in}}%
\pgfpathlineto{\pgfqpoint{1.894758in}{0.982889in}}%
\pgfpathlineto{\pgfqpoint{1.879102in}{0.974265in}}%
\pgfpathlineto{\pgfqpoint{1.875099in}{0.971018in}}%
\pgfpathlineto{\pgfqpoint{1.863445in}{0.962493in}}%
\pgfpathlineto{\pgfqpoint{1.857517in}{0.957407in}}%
\pgfpathlineto{\pgfqpoint{1.847789in}{0.949158in}}%
\pgfpathlineto{\pgfqpoint{1.841649in}{0.943795in}}%
\pgfpathlineto{\pgfqpoint{1.832132in}{0.934596in}}%
\pgfpathlineto{\pgfqpoint{1.827247in}{0.930184in}}%
\pgfpathlineto{\pgfqpoint{1.816476in}{0.917402in}}%
\pgfpathlineto{\pgfqpoint{1.815657in}{0.916573in}}%
\pgfpathlineto{\pgfqpoint{1.808799in}{0.902962in}}%
\pgfpathlineto{\pgfqpoint{1.809589in}{0.889351in}}%
\pgfpathlineto{\pgfqpoint{1.816476in}{0.878027in}}%
\pgfpathlineto{\pgfqpoint{1.817685in}{0.875740in}}%
\pgfpathlineto{\pgfqpoint{1.829982in}{0.862129in}}%
\pgfpathlineto{\pgfqpoint{1.832132in}{0.860251in}}%
\pgfpathlineto{\pgfqpoint{1.844659in}{0.848518in}}%
\pgfpathlineto{\pgfqpoint{1.847789in}{0.845800in}}%
\pgfpathlineto{\pgfqpoint{1.860708in}{0.834907in}}%
\pgfpathlineto{\pgfqpoint{1.863445in}{0.832522in}}%
\pgfpathlineto{\pgfqpoint{1.878578in}{0.821295in}}%
\pgfpathlineto{\pgfqpoint{1.879102in}{0.820855in}}%
\pgfpathclose%
\pgfpathmoveto{\pgfqpoint{0.367333in}{1.079907in}}%
\pgfpathlineto{\pgfqpoint{0.376072in}{1.081646in}}%
\pgfpathlineto{\pgfqpoint{0.391728in}{1.090270in}}%
\pgfpathlineto{\pgfqpoint{0.395731in}{1.093518in}}%
\pgfpathlineto{\pgfqpoint{0.407385in}{1.102042in}}%
\pgfpathlineto{\pgfqpoint{0.413313in}{1.107129in}}%
\pgfpathlineto{\pgfqpoint{0.423041in}{1.115378in}}%
\pgfpathlineto{\pgfqpoint{0.429181in}{1.120740in}}%
\pgfpathlineto{\pgfqpoint{0.438698in}{1.129939in}}%
\pgfpathlineto{\pgfqpoint{0.443583in}{1.134351in}}%
\pgfpathlineto{\pgfqpoint{0.454354in}{1.147133in}}%
\pgfpathlineto{\pgfqpoint{0.455173in}{1.147962in}}%
\pgfpathlineto{\pgfqpoint{0.462031in}{1.161573in}}%
\pgfpathlineto{\pgfqpoint{0.461241in}{1.175184in}}%
\pgfpathlineto{\pgfqpoint{0.454354in}{1.186508in}}%
\pgfpathlineto{\pgfqpoint{0.453145in}{1.188795in}}%
\pgfpathlineto{\pgfqpoint{0.440848in}{1.202407in}}%
\pgfpathlineto{\pgfqpoint{0.438698in}{1.204285in}}%
\pgfpathlineto{\pgfqpoint{0.426171in}{1.216018in}}%
\pgfpathlineto{\pgfqpoint{0.423041in}{1.218735in}}%
\pgfpathlineto{\pgfqpoint{0.410122in}{1.229629in}}%
\pgfpathlineto{\pgfqpoint{0.407385in}{1.232013in}}%
\pgfpathlineto{\pgfqpoint{0.392252in}{1.243240in}}%
\pgfpathlineto{\pgfqpoint{0.391728in}{1.243680in}}%
\pgfpathlineto{\pgfqpoint{0.376072in}{1.252492in}}%
\pgfpathlineto{\pgfqpoint{0.360415in}{1.255698in}}%
\pgfpathlineto{\pgfqpoint{0.360415in}{1.243240in}}%
\pgfpathlineto{\pgfqpoint{0.360415in}{1.229629in}}%
\pgfpathlineto{\pgfqpoint{0.360415in}{1.218905in}}%
\pgfpathlineto{\pgfqpoint{0.376072in}{1.216307in}}%
\pgfpathlineto{\pgfqpoint{0.376682in}{1.216018in}}%
\pgfpathlineto{\pgfqpoint{0.391728in}{1.208072in}}%
\pgfpathlineto{\pgfqpoint{0.399095in}{1.202407in}}%
\pgfpathlineto{\pgfqpoint{0.407385in}{1.194303in}}%
\pgfpathlineto{\pgfqpoint{0.412101in}{1.188795in}}%
\pgfpathlineto{\pgfqpoint{0.418887in}{1.175184in}}%
\pgfpathlineto{\pgfqpoint{0.419537in}{1.161573in}}%
\pgfpathlineto{\pgfqpoint{0.413887in}{1.147962in}}%
\pgfpathlineto{\pgfqpoint{0.407385in}{1.139750in}}%
\pgfpathlineto{\pgfqpoint{0.402259in}{1.134351in}}%
\pgfpathlineto{\pgfqpoint{0.391728in}{1.125975in}}%
\pgfpathlineto{\pgfqpoint{0.382121in}{1.120740in}}%
\pgfpathlineto{\pgfqpoint{0.376072in}{1.117855in}}%
\pgfpathlineto{\pgfqpoint{0.360415in}{1.115204in}}%
\pgfpathlineto{\pgfqpoint{0.360415in}{1.107129in}}%
\pgfpathlineto{\pgfqpoint{0.360415in}{1.093518in}}%
\pgfpathlineto{\pgfqpoint{0.360415in}{1.079907in}}%
\pgfpathlineto{\pgfqpoint{0.360415in}{1.078271in}}%
\pgfpathlineto{\pgfqpoint{0.367333in}{1.079907in}}%
\pgfpathclose%
\pgfpathmoveto{\pgfqpoint{0.673546in}{1.078421in}}%
\pgfpathlineto{\pgfqpoint{0.678055in}{1.079907in}}%
\pgfpathlineto{\pgfqpoint{0.689203in}{1.082983in}}%
\pgfpathlineto{\pgfqpoint{0.704859in}{1.092492in}}%
\pgfpathlineto{\pgfqpoint{0.706064in}{1.093518in}}%
\pgfpathlineto{\pgfqpoint{0.720516in}{1.104581in}}%
\pgfpathlineto{\pgfqpoint{0.723438in}{1.107129in}}%
\pgfpathlineto{\pgfqpoint{0.736173in}{1.118088in}}%
\pgfpathlineto{\pgfqpoint{0.739230in}{1.120740in}}%
\pgfpathlineto{\pgfqpoint{0.751829in}{1.132838in}}%
\pgfpathlineto{\pgfqpoint{0.753550in}{1.134351in}}%
\pgfpathlineto{\pgfqpoint{0.765205in}{1.147962in}}%
\pgfpathlineto{\pgfqpoint{0.767486in}{1.153027in}}%
\pgfpathlineto{\pgfqpoint{0.771993in}{1.161573in}}%
\pgfpathlineto{\pgfqpoint{0.771174in}{1.175184in}}%
\pgfpathlineto{\pgfqpoint{0.767486in}{1.180961in}}%
\pgfpathlineto{\pgfqpoint{0.763236in}{1.188795in}}%
\pgfpathlineto{\pgfqpoint{0.751829in}{1.201203in}}%
\pgfpathlineto{\pgfqpoint{0.750804in}{1.202407in}}%
\pgfpathlineto{\pgfqpoint{0.736243in}{1.216018in}}%
\pgfpathlineto{\pgfqpoint{0.736173in}{1.216078in}}%
\pgfpathlineto{\pgfqpoint{0.720516in}{1.229466in}}%
\pgfpathlineto{\pgfqpoint{0.720300in}{1.229629in}}%
\pgfpathlineto{\pgfqpoint{0.704859in}{1.241525in}}%
\pgfpathlineto{\pgfqpoint{0.701802in}{1.243240in}}%
\pgfpathlineto{\pgfqpoint{0.689203in}{1.251126in}}%
\pgfpathlineto{\pgfqpoint{0.673546in}{1.255567in}}%
\pgfpathlineto{\pgfqpoint{0.657890in}{1.253630in}}%
\pgfpathlineto{\pgfqpoint{0.642233in}{1.245805in}}%
\pgfpathlineto{\pgfqpoint{0.639000in}{1.243240in}}%
\pgfpathlineto{\pgfqpoint{0.626577in}{1.234506in}}%
\pgfpathlineto{\pgfqpoint{0.620856in}{1.229629in}}%
\pgfpathlineto{\pgfqpoint{0.610920in}{1.221415in}}%
\pgfpathlineto{\pgfqpoint{0.604720in}{1.216018in}}%
\pgfpathlineto{\pgfqpoint{0.595263in}{1.207136in}}%
\pgfpathlineto{\pgfqpoint{0.589971in}{1.202407in}}%
\pgfpathlineto{\pgfqpoint{0.579607in}{1.190744in}}%
\pgfpathlineto{\pgfqpoint{0.577617in}{1.188795in}}%
\pgfpathlineto{\pgfqpoint{0.569633in}{1.175184in}}%
\pgfpathlineto{\pgfqpoint{0.568868in}{1.161573in}}%
\pgfpathlineto{\pgfqpoint{0.575516in}{1.147962in}}%
\pgfpathlineto{\pgfqpoint{0.579607in}{1.143630in}}%
\pgfpathlineto{\pgfqpoint{0.587260in}{1.134351in}}%
\pgfpathlineto{\pgfqpoint{0.595263in}{1.126955in}}%
\pgfpathlineto{\pgfqpoint{0.601676in}{1.120740in}}%
\pgfpathlineto{\pgfqpoint{0.610920in}{1.112643in}}%
\pgfpathlineto{\pgfqpoint{0.617554in}{1.107129in}}%
\pgfpathlineto{\pgfqpoint{0.626577in}{1.099551in}}%
\pgfpathlineto{\pgfqpoint{0.635281in}{1.093518in}}%
\pgfpathlineto{\pgfqpoint{0.642233in}{1.088190in}}%
\pgfpathlineto{\pgfqpoint{0.657890in}{1.080533in}}%
\pgfpathlineto{\pgfqpoint{0.663095in}{1.079907in}}%
\pgfpathlineto{\pgfqpoint{0.673546in}{1.078421in}}%
\pgfpathclose%
\pgfpathmoveto{\pgfqpoint{0.648766in}{1.120740in}}%
\pgfpathlineto{\pgfqpoint{0.642233in}{1.123885in}}%
\pgfpathlineto{\pgfqpoint{0.628302in}{1.134351in}}%
\pgfpathlineto{\pgfqpoint{0.626577in}{1.136073in}}%
\pgfpathlineto{\pgfqpoint{0.616960in}{1.147962in}}%
\pgfpathlineto{\pgfqpoint{0.611113in}{1.161573in}}%
\pgfpathlineto{\pgfqpoint{0.611786in}{1.175184in}}%
\pgfpathlineto{\pgfqpoint{0.618809in}{1.188795in}}%
\pgfpathlineto{\pgfqpoint{0.626577in}{1.197675in}}%
\pgfpathlineto{\pgfqpoint{0.631685in}{1.202407in}}%
\pgfpathlineto{\pgfqpoint{0.642233in}{1.210069in}}%
\pgfpathlineto{\pgfqpoint{0.654984in}{1.216018in}}%
\pgfpathlineto{\pgfqpoint{0.657890in}{1.217229in}}%
\pgfpathlineto{\pgfqpoint{0.673546in}{1.218799in}}%
\pgfpathlineto{\pgfqpoint{0.685591in}{1.216018in}}%
\pgfpathlineto{\pgfqpoint{0.689203in}{1.215069in}}%
\pgfpathlineto{\pgfqpoint{0.704859in}{1.205939in}}%
\pgfpathlineto{\pgfqpoint{0.709235in}{1.202407in}}%
\pgfpathlineto{\pgfqpoint{0.720516in}{1.190865in}}%
\pgfpathlineto{\pgfqpoint{0.722261in}{1.188795in}}%
\pgfpathlineto{\pgfqpoint{0.728851in}{1.175184in}}%
\pgfpathlineto{\pgfqpoint{0.729482in}{1.161573in}}%
\pgfpathlineto{\pgfqpoint{0.723995in}{1.147962in}}%
\pgfpathlineto{\pgfqpoint{0.720516in}{1.143497in}}%
\pgfpathlineto{\pgfqpoint{0.712218in}{1.134351in}}%
\pgfpathlineto{\pgfqpoint{0.704859in}{1.128208in}}%
\pgfpathlineto{\pgfqpoint{0.692506in}{1.120740in}}%
\pgfpathlineto{\pgfqpoint{0.689203in}{1.118984in}}%
\pgfpathlineto{\pgfqpoint{0.673546in}{1.115313in}}%
\pgfpathlineto{\pgfqpoint{0.657890in}{1.116915in}}%
\pgfpathlineto{\pgfqpoint{0.648766in}{1.120740in}}%
\pgfpathclose%
\pgfpathmoveto{\pgfqpoint{0.971021in}{1.079612in}}%
\pgfpathlineto{\pgfqpoint{0.986678in}{1.078870in}}%
\pgfpathlineto{\pgfqpoint{0.989143in}{1.079907in}}%
\pgfpathlineto{\pgfqpoint{1.002334in}{1.084530in}}%
\pgfpathlineto{\pgfqpoint{1.015900in}{1.093518in}}%
\pgfpathlineto{\pgfqpoint{1.017991in}{1.094780in}}%
\pgfpathlineto{\pgfqpoint{1.033615in}{1.107129in}}%
\pgfpathlineto{\pgfqpoint{1.033647in}{1.107154in}}%
\pgfpathlineto{\pgfqpoint{1.049275in}{1.120740in}}%
\pgfpathlineto{\pgfqpoint{1.049304in}{1.120768in}}%
\pgfpathlineto{\pgfqpoint{1.063508in}{1.134351in}}%
\pgfpathlineto{\pgfqpoint{1.064960in}{1.136169in}}%
\pgfpathlineto{\pgfqpoint{1.075299in}{1.147962in}}%
\pgfpathlineto{\pgfqpoint{1.080617in}{1.159430in}}%
\pgfpathlineto{\pgfqpoint{1.081810in}{1.161573in}}%
\pgfpathlineto{\pgfqpoint{1.080956in}{1.175184in}}%
\pgfpathlineto{\pgfqpoint{1.080617in}{1.175687in}}%
\pgfpathlineto{\pgfqpoint{1.073296in}{1.188795in}}%
\pgfpathlineto{\pgfqpoint{1.064960in}{1.197587in}}%
\pgfpathlineto{\pgfqpoint{1.060812in}{1.202407in}}%
\pgfpathlineto{\pgfqpoint{1.049304in}{1.213048in}}%
\pgfpathlineto{\pgfqpoint{1.046119in}{1.216018in}}%
\pgfpathlineto{\pgfqpoint{1.033647in}{1.226795in}}%
\pgfpathlineto{\pgfqpoint{1.030019in}{1.229629in}}%
\pgfpathlineto{\pgfqpoint{1.017991in}{1.239280in}}%
\pgfpathlineto{\pgfqpoint{1.011524in}{1.243240in}}%
\pgfpathlineto{\pgfqpoint{1.002334in}{1.249546in}}%
\pgfpathlineto{\pgfqpoint{0.986678in}{1.255175in}}%
\pgfpathlineto{\pgfqpoint{0.971021in}{1.254527in}}%
\pgfpathlineto{\pgfqpoint{0.955364in}{1.247767in}}%
\pgfpathlineto{\pgfqpoint{0.949265in}{1.243240in}}%
\pgfpathlineto{\pgfqpoint{0.939708in}{1.236934in}}%
\pgfpathlineto{\pgfqpoint{0.930904in}{1.229629in}}%
\pgfpathlineto{\pgfqpoint{0.924051in}{1.224107in}}%
\pgfpathlineto{\pgfqpoint{0.914747in}{1.216018in}}%
\pgfpathlineto{\pgfqpoint{0.908395in}{1.210060in}}%
\pgfpathlineto{\pgfqpoint{0.899991in}{1.202407in}}%
\pgfpathlineto{\pgfqpoint{0.892738in}{1.194098in}}%
\pgfpathlineto{\pgfqpoint{0.887531in}{1.188795in}}%
\pgfpathlineto{\pgfqpoint{0.879755in}{1.175184in}}%
\pgfpathlineto{\pgfqpoint{0.879009in}{1.161573in}}%
\pgfpathlineto{\pgfqpoint{0.885484in}{1.147962in}}%
\pgfpathlineto{\pgfqpoint{0.892738in}{1.139973in}}%
\pgfpathlineto{\pgfqpoint{0.897293in}{1.134351in}}%
\pgfpathlineto{\pgfqpoint{0.908395in}{1.123894in}}%
\pgfpathlineto{\pgfqpoint{0.911654in}{1.120740in}}%
\pgfpathlineto{\pgfqpoint{0.924051in}{1.109897in}}%
\pgfpathlineto{\pgfqpoint{0.927467in}{1.107129in}}%
\pgfpathlineto{\pgfqpoint{0.939708in}{1.097124in}}%
\pgfpathlineto{\pgfqpoint{0.945252in}{1.093518in}}%
\pgfpathlineto{\pgfqpoint{0.955364in}{1.086271in}}%
\pgfpathlineto{\pgfqpoint{0.970443in}{1.079907in}}%
\pgfpathlineto{\pgfqpoint{0.971021in}{1.079612in}}%
\pgfpathclose%
\pgfpathmoveto{\pgfqpoint{0.958300in}{1.120740in}}%
\pgfpathlineto{\pgfqpoint{0.955364in}{1.121956in}}%
\pgfpathlineto{\pgfqpoint{0.939708in}{1.133028in}}%
\pgfpathlineto{\pgfqpoint{0.938236in}{1.134351in}}%
\pgfpathlineto{\pgfqpoint{0.926850in}{1.147962in}}%
\pgfpathlineto{\pgfqpoint{0.924051in}{1.154308in}}%
\pgfpathlineto{\pgfqpoint{0.921218in}{1.161573in}}%
\pgfpathlineto{\pgfqpoint{0.921822in}{1.175184in}}%
\pgfpathlineto{\pgfqpoint{0.924051in}{1.179906in}}%
\pgfpathlineto{\pgfqpoint{0.928773in}{1.188795in}}%
\pgfpathlineto{\pgfqpoint{0.939708in}{1.200959in}}%
\pgfpathlineto{\pgfqpoint{0.941373in}{1.202407in}}%
\pgfpathlineto{\pgfqpoint{0.955364in}{1.211912in}}%
\pgfpathlineto{\pgfqpoint{0.965590in}{1.216018in}}%
\pgfpathlineto{\pgfqpoint{0.971021in}{1.217956in}}%
\pgfpathlineto{\pgfqpoint{0.986678in}{1.218481in}}%
\pgfpathlineto{\pgfqpoint{0.995035in}{1.216018in}}%
\pgfpathlineto{\pgfqpoint{1.002334in}{1.213584in}}%
\pgfpathlineto{\pgfqpoint{1.017991in}{1.203686in}}%
\pgfpathlineto{\pgfqpoint{1.019513in}{1.202407in}}%
\pgfpathlineto{\pgfqpoint{1.032248in}{1.188795in}}%
\pgfpathlineto{\pgfqpoint{1.033647in}{1.186244in}}%
\pgfpathlineto{\pgfqpoint{1.038901in}{1.175184in}}%
\pgfpathlineto{\pgfqpoint{1.039518in}{1.161573in}}%
\pgfpathlineto{\pgfqpoint{1.034163in}{1.147962in}}%
\pgfpathlineto{\pgfqpoint{1.033647in}{1.147293in}}%
\pgfpathlineto{\pgfqpoint{1.022345in}{1.134351in}}%
\pgfpathlineto{\pgfqpoint{1.017991in}{1.130565in}}%
\pgfpathlineto{\pgfqpoint{1.003104in}{1.120740in}}%
\pgfpathlineto{\pgfqpoint{1.002334in}{1.120291in}}%
\pgfpathlineto{\pgfqpoint{0.986678in}{1.115637in}}%
\pgfpathlineto{\pgfqpoint{0.971021in}{1.116172in}}%
\pgfpathlineto{\pgfqpoint{0.958300in}{1.120740in}}%
\pgfpathclose%
\pgfpathmoveto{\pgfqpoint{1.284152in}{1.078870in}}%
\pgfpathlineto{\pgfqpoint{1.299809in}{1.079612in}}%
\pgfpathlineto{\pgfqpoint{1.300387in}{1.079907in}}%
\pgfpathlineto{\pgfqpoint{1.315466in}{1.086271in}}%
\pgfpathlineto{\pgfqpoint{1.325578in}{1.093518in}}%
\pgfpathlineto{\pgfqpoint{1.331122in}{1.097124in}}%
\pgfpathlineto{\pgfqpoint{1.343363in}{1.107129in}}%
\pgfpathlineto{\pgfqpoint{1.346779in}{1.109897in}}%
\pgfpathlineto{\pgfqpoint{1.359176in}{1.120740in}}%
\pgfpathlineto{\pgfqpoint{1.362435in}{1.123894in}}%
\pgfpathlineto{\pgfqpoint{1.373537in}{1.134351in}}%
\pgfpathlineto{\pgfqpoint{1.378092in}{1.139973in}}%
\pgfpathlineto{\pgfqpoint{1.385346in}{1.147962in}}%
\pgfpathlineto{\pgfqpoint{1.391821in}{1.161573in}}%
\pgfpathlineto{\pgfqpoint{1.391075in}{1.175184in}}%
\pgfpathlineto{\pgfqpoint{1.383299in}{1.188795in}}%
\pgfpathlineto{\pgfqpoint{1.378092in}{1.194098in}}%
\pgfpathlineto{\pgfqpoint{1.370839in}{1.202407in}}%
\pgfpathlineto{\pgfqpoint{1.362435in}{1.210060in}}%
\pgfpathlineto{\pgfqpoint{1.356083in}{1.216018in}}%
\pgfpathlineto{\pgfqpoint{1.346779in}{1.224107in}}%
\pgfpathlineto{\pgfqpoint{1.339926in}{1.229629in}}%
\pgfpathlineto{\pgfqpoint{1.331122in}{1.236934in}}%
\pgfpathlineto{\pgfqpoint{1.321565in}{1.243240in}}%
\pgfpathlineto{\pgfqpoint{1.315466in}{1.247767in}}%
\pgfpathlineto{\pgfqpoint{1.299809in}{1.254527in}}%
\pgfpathlineto{\pgfqpoint{1.284152in}{1.255175in}}%
\pgfpathlineto{\pgfqpoint{1.268496in}{1.249546in}}%
\pgfpathlineto{\pgfqpoint{1.259306in}{1.243240in}}%
\pgfpathlineto{\pgfqpoint{1.252839in}{1.239280in}}%
\pgfpathlineto{\pgfqpoint{1.240811in}{1.229629in}}%
\pgfpathlineto{\pgfqpoint{1.237183in}{1.226795in}}%
\pgfpathlineto{\pgfqpoint{1.224711in}{1.216018in}}%
\pgfpathlineto{\pgfqpoint{1.221526in}{1.213048in}}%
\pgfpathlineto{\pgfqpoint{1.210018in}{1.202407in}}%
\pgfpathlineto{\pgfqpoint{1.205870in}{1.197587in}}%
\pgfpathlineto{\pgfqpoint{1.197534in}{1.188795in}}%
\pgfpathlineto{\pgfqpoint{1.190213in}{1.175687in}}%
\pgfpathlineto{\pgfqpoint{1.189874in}{1.175184in}}%
\pgfpathlineto{\pgfqpoint{1.189020in}{1.161573in}}%
\pgfpathlineto{\pgfqpoint{1.190213in}{1.159430in}}%
\pgfpathlineto{\pgfqpoint{1.195531in}{1.147962in}}%
\pgfpathlineto{\pgfqpoint{1.205870in}{1.136169in}}%
\pgfpathlineto{\pgfqpoint{1.207322in}{1.134351in}}%
\pgfpathlineto{\pgfqpoint{1.221526in}{1.120768in}}%
\pgfpathlineto{\pgfqpoint{1.221555in}{1.120740in}}%
\pgfpathlineto{\pgfqpoint{1.237183in}{1.107154in}}%
\pgfpathlineto{\pgfqpoint{1.237215in}{1.107129in}}%
\pgfpathlineto{\pgfqpoint{1.252839in}{1.094780in}}%
\pgfpathlineto{\pgfqpoint{1.254930in}{1.093518in}}%
\pgfpathlineto{\pgfqpoint{1.268496in}{1.084530in}}%
\pgfpathlineto{\pgfqpoint{1.281687in}{1.079907in}}%
\pgfpathlineto{\pgfqpoint{1.284152in}{1.078870in}}%
\pgfpathclose%
\pgfpathmoveto{\pgfqpoint{1.267726in}{1.120740in}}%
\pgfpathlineto{\pgfqpoint{1.252839in}{1.130565in}}%
\pgfpathlineto{\pgfqpoint{1.248485in}{1.134351in}}%
\pgfpathlineto{\pgfqpoint{1.237183in}{1.147293in}}%
\pgfpathlineto{\pgfqpoint{1.236667in}{1.147962in}}%
\pgfpathlineto{\pgfqpoint{1.231312in}{1.161573in}}%
\pgfpathlineto{\pgfqpoint{1.231929in}{1.175184in}}%
\pgfpathlineto{\pgfqpoint{1.237183in}{1.186244in}}%
\pgfpathlineto{\pgfqpoint{1.238582in}{1.188795in}}%
\pgfpathlineto{\pgfqpoint{1.251317in}{1.202407in}}%
\pgfpathlineto{\pgfqpoint{1.252839in}{1.203686in}}%
\pgfpathlineto{\pgfqpoint{1.268496in}{1.213584in}}%
\pgfpathlineto{\pgfqpoint{1.275795in}{1.216018in}}%
\pgfpathlineto{\pgfqpoint{1.284152in}{1.218481in}}%
\pgfpathlineto{\pgfqpoint{1.299809in}{1.217956in}}%
\pgfpathlineto{\pgfqpoint{1.305240in}{1.216018in}}%
\pgfpathlineto{\pgfqpoint{1.315466in}{1.211912in}}%
\pgfpathlineto{\pgfqpoint{1.329457in}{1.202407in}}%
\pgfpathlineto{\pgfqpoint{1.331122in}{1.200959in}}%
\pgfpathlineto{\pgfqpoint{1.342057in}{1.188795in}}%
\pgfpathlineto{\pgfqpoint{1.346779in}{1.179906in}}%
\pgfpathlineto{\pgfqpoint{1.349008in}{1.175184in}}%
\pgfpathlineto{\pgfqpoint{1.349612in}{1.161573in}}%
\pgfpathlineto{\pgfqpoint{1.346779in}{1.154308in}}%
\pgfpathlineto{\pgfqpoint{1.343980in}{1.147962in}}%
\pgfpathlineto{\pgfqpoint{1.332594in}{1.134351in}}%
\pgfpathlineto{\pgfqpoint{1.331122in}{1.133028in}}%
\pgfpathlineto{\pgfqpoint{1.315466in}{1.121956in}}%
\pgfpathlineto{\pgfqpoint{1.312530in}{1.120740in}}%
\pgfpathlineto{\pgfqpoint{1.299809in}{1.116172in}}%
\pgfpathlineto{\pgfqpoint{1.284152in}{1.115637in}}%
\pgfpathlineto{\pgfqpoint{1.268496in}{1.120291in}}%
\pgfpathlineto{\pgfqpoint{1.267726in}{1.120740in}}%
\pgfpathclose%
\pgfpathmoveto{\pgfqpoint{1.597284in}{1.078421in}}%
\pgfpathlineto{\pgfqpoint{1.607735in}{1.079907in}}%
\pgfpathlineto{\pgfqpoint{1.612940in}{1.080533in}}%
\pgfpathlineto{\pgfqpoint{1.628597in}{1.088190in}}%
\pgfpathlineto{\pgfqpoint{1.635549in}{1.093518in}}%
\pgfpathlineto{\pgfqpoint{1.644253in}{1.099551in}}%
\pgfpathlineto{\pgfqpoint{1.653276in}{1.107129in}}%
\pgfpathlineto{\pgfqpoint{1.659910in}{1.112643in}}%
\pgfpathlineto{\pgfqpoint{1.669154in}{1.120740in}}%
\pgfpathlineto{\pgfqpoint{1.675567in}{1.126955in}}%
\pgfpathlineto{\pgfqpoint{1.683570in}{1.134351in}}%
\pgfpathlineto{\pgfqpoint{1.691223in}{1.143630in}}%
\pgfpathlineto{\pgfqpoint{1.695314in}{1.147962in}}%
\pgfpathlineto{\pgfqpoint{1.701962in}{1.161573in}}%
\pgfpathlineto{\pgfqpoint{1.701197in}{1.175184in}}%
\pgfpathlineto{\pgfqpoint{1.693213in}{1.188795in}}%
\pgfpathlineto{\pgfqpoint{1.691223in}{1.190744in}}%
\pgfpathlineto{\pgfqpoint{1.680859in}{1.202407in}}%
\pgfpathlineto{\pgfqpoint{1.675567in}{1.207136in}}%
\pgfpathlineto{\pgfqpoint{1.666110in}{1.216018in}}%
\pgfpathlineto{\pgfqpoint{1.659910in}{1.221415in}}%
\pgfpathlineto{\pgfqpoint{1.649974in}{1.229629in}}%
\pgfpathlineto{\pgfqpoint{1.644253in}{1.234506in}}%
\pgfpathlineto{\pgfqpoint{1.631830in}{1.243240in}}%
\pgfpathlineto{\pgfqpoint{1.628597in}{1.245805in}}%
\pgfpathlineto{\pgfqpoint{1.612940in}{1.253630in}}%
\pgfpathlineto{\pgfqpoint{1.597284in}{1.255567in}}%
\pgfpathlineto{\pgfqpoint{1.581627in}{1.251126in}}%
\pgfpathlineto{\pgfqpoint{1.569028in}{1.243240in}}%
\pgfpathlineto{\pgfqpoint{1.565971in}{1.241525in}}%
\pgfpathlineto{\pgfqpoint{1.550530in}{1.229629in}}%
\pgfpathlineto{\pgfqpoint{1.550314in}{1.229466in}}%
\pgfpathlineto{\pgfqpoint{1.534657in}{1.216078in}}%
\pgfpathlineto{\pgfqpoint{1.534587in}{1.216018in}}%
\pgfpathlineto{\pgfqpoint{1.520026in}{1.202407in}}%
\pgfpathlineto{\pgfqpoint{1.519001in}{1.201203in}}%
\pgfpathlineto{\pgfqpoint{1.507594in}{1.188795in}}%
\pgfpathlineto{\pgfqpoint{1.503344in}{1.180961in}}%
\pgfpathlineto{\pgfqpoint{1.499656in}{1.175184in}}%
\pgfpathlineto{\pgfqpoint{1.498837in}{1.161573in}}%
\pgfpathlineto{\pgfqpoint{1.503344in}{1.153027in}}%
\pgfpathlineto{\pgfqpoint{1.505625in}{1.147962in}}%
\pgfpathlineto{\pgfqpoint{1.517280in}{1.134351in}}%
\pgfpathlineto{\pgfqpoint{1.519001in}{1.132838in}}%
\pgfpathlineto{\pgfqpoint{1.531600in}{1.120740in}}%
\pgfpathlineto{\pgfqpoint{1.534657in}{1.118088in}}%
\pgfpathlineto{\pgfqpoint{1.547392in}{1.107129in}}%
\pgfpathlineto{\pgfqpoint{1.550314in}{1.104581in}}%
\pgfpathlineto{\pgfqpoint{1.564766in}{1.093518in}}%
\pgfpathlineto{\pgfqpoint{1.565971in}{1.092492in}}%
\pgfpathlineto{\pgfqpoint{1.581627in}{1.082983in}}%
\pgfpathlineto{\pgfqpoint{1.592775in}{1.079907in}}%
\pgfpathlineto{\pgfqpoint{1.597284in}{1.078421in}}%
\pgfpathclose%
\pgfpathmoveto{\pgfqpoint{1.578324in}{1.120740in}}%
\pgfpathlineto{\pgfqpoint{1.565971in}{1.128208in}}%
\pgfpathlineto{\pgfqpoint{1.558612in}{1.134351in}}%
\pgfpathlineto{\pgfqpoint{1.550314in}{1.143497in}}%
\pgfpathlineto{\pgfqpoint{1.546835in}{1.147962in}}%
\pgfpathlineto{\pgfqpoint{1.541348in}{1.161573in}}%
\pgfpathlineto{\pgfqpoint{1.541979in}{1.175184in}}%
\pgfpathlineto{\pgfqpoint{1.548569in}{1.188795in}}%
\pgfpathlineto{\pgfqpoint{1.550314in}{1.190865in}}%
\pgfpathlineto{\pgfqpoint{1.561595in}{1.202407in}}%
\pgfpathlineto{\pgfqpoint{1.565971in}{1.205939in}}%
\pgfpathlineto{\pgfqpoint{1.581627in}{1.215069in}}%
\pgfpathlineto{\pgfqpoint{1.585239in}{1.216018in}}%
\pgfpathlineto{\pgfqpoint{1.597284in}{1.218799in}}%
\pgfpathlineto{\pgfqpoint{1.612940in}{1.217229in}}%
\pgfpathlineto{\pgfqpoint{1.615846in}{1.216018in}}%
\pgfpathlineto{\pgfqpoint{1.628597in}{1.210069in}}%
\pgfpathlineto{\pgfqpoint{1.639145in}{1.202407in}}%
\pgfpathlineto{\pgfqpoint{1.644253in}{1.197675in}}%
\pgfpathlineto{\pgfqpoint{1.652021in}{1.188795in}}%
\pgfpathlineto{\pgfqpoint{1.659044in}{1.175184in}}%
\pgfpathlineto{\pgfqpoint{1.659717in}{1.161573in}}%
\pgfpathlineto{\pgfqpoint{1.653870in}{1.147962in}}%
\pgfpathlineto{\pgfqpoint{1.644253in}{1.136073in}}%
\pgfpathlineto{\pgfqpoint{1.642528in}{1.134351in}}%
\pgfpathlineto{\pgfqpoint{1.628597in}{1.123885in}}%
\pgfpathlineto{\pgfqpoint{1.622064in}{1.120740in}}%
\pgfpathlineto{\pgfqpoint{1.612940in}{1.116915in}}%
\pgfpathlineto{\pgfqpoint{1.597284in}{1.115313in}}%
\pgfpathlineto{\pgfqpoint{1.581627in}{1.118984in}}%
\pgfpathlineto{\pgfqpoint{1.578324in}{1.120740in}}%
\pgfpathclose%
\pgfpathmoveto{\pgfqpoint{1.910415in}{1.078271in}}%
\pgfpathlineto{\pgfqpoint{1.910415in}{1.079907in}}%
\pgfpathlineto{\pgfqpoint{1.910415in}{1.093518in}}%
\pgfpathlineto{\pgfqpoint{1.910415in}{1.107129in}}%
\pgfpathlineto{\pgfqpoint{1.910415in}{1.115204in}}%
\pgfpathlineto{\pgfqpoint{1.894758in}{1.117855in}}%
\pgfpathlineto{\pgfqpoint{1.888709in}{1.120740in}}%
\pgfpathlineto{\pgfqpoint{1.879102in}{1.125975in}}%
\pgfpathlineto{\pgfqpoint{1.868571in}{1.134351in}}%
\pgfpathlineto{\pgfqpoint{1.863445in}{1.139750in}}%
\pgfpathlineto{\pgfqpoint{1.856943in}{1.147962in}}%
\pgfpathlineto{\pgfqpoint{1.851293in}{1.161573in}}%
\pgfpathlineto{\pgfqpoint{1.851943in}{1.175184in}}%
\pgfpathlineto{\pgfqpoint{1.858729in}{1.188795in}}%
\pgfpathlineto{\pgfqpoint{1.863445in}{1.194303in}}%
\pgfpathlineto{\pgfqpoint{1.871735in}{1.202407in}}%
\pgfpathlineto{\pgfqpoint{1.879102in}{1.208072in}}%
\pgfpathlineto{\pgfqpoint{1.894148in}{1.216018in}}%
\pgfpathlineto{\pgfqpoint{1.894758in}{1.216307in}}%
\pgfpathlineto{\pgfqpoint{1.910415in}{1.218905in}}%
\pgfpathlineto{\pgfqpoint{1.910415in}{1.229629in}}%
\pgfpathlineto{\pgfqpoint{1.910415in}{1.243240in}}%
\pgfpathlineto{\pgfqpoint{1.910415in}{1.255698in}}%
\pgfpathlineto{\pgfqpoint{1.894758in}{1.252492in}}%
\pgfpathlineto{\pgfqpoint{1.879102in}{1.243680in}}%
\pgfpathlineto{\pgfqpoint{1.878578in}{1.243240in}}%
\pgfpathlineto{\pgfqpoint{1.863445in}{1.232013in}}%
\pgfpathlineto{\pgfqpoint{1.860708in}{1.229629in}}%
\pgfpathlineto{\pgfqpoint{1.847789in}{1.218735in}}%
\pgfpathlineto{\pgfqpoint{1.844659in}{1.216018in}}%
\pgfpathlineto{\pgfqpoint{1.832132in}{1.204285in}}%
\pgfpathlineto{\pgfqpoint{1.829982in}{1.202407in}}%
\pgfpathlineto{\pgfqpoint{1.817685in}{1.188795in}}%
\pgfpathlineto{\pgfqpoint{1.816476in}{1.186508in}}%
\pgfpathlineto{\pgfqpoint{1.809589in}{1.175184in}}%
\pgfpathlineto{\pgfqpoint{1.808799in}{1.161573in}}%
\pgfpathlineto{\pgfqpoint{1.815657in}{1.147962in}}%
\pgfpathlineto{\pgfqpoint{1.816476in}{1.147133in}}%
\pgfpathlineto{\pgfqpoint{1.827247in}{1.134351in}}%
\pgfpathlineto{\pgfqpoint{1.832132in}{1.129939in}}%
\pgfpathlineto{\pgfqpoint{1.841649in}{1.120740in}}%
\pgfpathlineto{\pgfqpoint{1.847789in}{1.115378in}}%
\pgfpathlineto{\pgfqpoint{1.857517in}{1.107129in}}%
\pgfpathlineto{\pgfqpoint{1.863445in}{1.102042in}}%
\pgfpathlineto{\pgfqpoint{1.875099in}{1.093518in}}%
\pgfpathlineto{\pgfqpoint{1.879102in}{1.090270in}}%
\pgfpathlineto{\pgfqpoint{1.894758in}{1.081646in}}%
\pgfpathlineto{\pgfqpoint{1.903497in}{1.079907in}}%
\pgfpathlineto{\pgfqpoint{1.910415in}{1.078271in}}%
\pgfpathclose%
\pgfpathmoveto{\pgfqpoint{0.376072in}{1.351157in}}%
\pgfpathlineto{\pgfqpoint{0.377587in}{1.352129in}}%
\pgfpathlineto{\pgfqpoint{0.391728in}{1.359754in}}%
\pgfpathlineto{\pgfqpoint{0.399338in}{1.365740in}}%
\pgfpathlineto{\pgfqpoint{0.407385in}{1.371565in}}%
\pgfpathlineto{\pgfqpoint{0.416561in}{1.379351in}}%
\pgfpathlineto{\pgfqpoint{0.423041in}{1.384902in}}%
\pgfpathlineto{\pgfqpoint{0.432171in}{1.392962in}}%
\pgfpathlineto{\pgfqpoint{0.438698in}{1.399516in}}%
\pgfpathlineto{\pgfqpoint{0.446201in}{1.406573in}}%
\pgfpathlineto{\pgfqpoint{0.454354in}{1.417136in}}%
\pgfpathlineto{\pgfqpoint{0.457098in}{1.420184in}}%
\pgfpathlineto{\pgfqpoint{0.462508in}{1.433795in}}%
\pgfpathlineto{\pgfqpoint{0.460148in}{1.447407in}}%
\pgfpathlineto{\pgfqpoint{0.454354in}{1.455564in}}%
\pgfpathlineto{\pgfqpoint{0.451002in}{1.461018in}}%
\pgfpathlineto{\pgfqpoint{0.438698in}{1.473877in}}%
\pgfpathlineto{\pgfqpoint{0.438020in}{1.474629in}}%
\pgfpathlineto{\pgfqpoint{0.423159in}{1.488240in}}%
\pgfpathlineto{\pgfqpoint{0.423041in}{1.488342in}}%
\pgfpathlineto{\pgfqpoint{0.407385in}{1.501502in}}%
\pgfpathlineto{\pgfqpoint{0.406899in}{1.501851in}}%
\pgfpathlineto{\pgfqpoint{0.391728in}{1.513202in}}%
\pgfpathlineto{\pgfqpoint{0.387336in}{1.515462in}}%
\pgfpathlineto{\pgfqpoint{0.376072in}{1.522044in}}%
\pgfpathlineto{\pgfqpoint{0.360415in}{1.525335in}}%
\pgfpathlineto{\pgfqpoint{0.360415in}{1.515462in}}%
\pgfpathlineto{\pgfqpoint{0.360415in}{1.501851in}}%
\pgfpathlineto{\pgfqpoint{0.360415in}{1.488509in}}%
\pgfpathlineto{\pgfqpoint{0.362042in}{1.488240in}}%
\pgfpathlineto{\pgfqpoint{0.376072in}{1.485649in}}%
\pgfpathlineto{\pgfqpoint{0.391728in}{1.477691in}}%
\pgfpathlineto{\pgfqpoint{0.395819in}{1.474629in}}%
\pgfpathlineto{\pgfqpoint{0.407385in}{1.464018in}}%
\pgfpathlineto{\pgfqpoint{0.410132in}{1.461018in}}%
\pgfpathlineto{\pgfqpoint{0.417986in}{1.447407in}}%
\pgfpathlineto{\pgfqpoint{0.419930in}{1.433795in}}%
\pgfpathlineto{\pgfqpoint{0.415473in}{1.420184in}}%
\pgfpathlineto{\pgfqpoint{0.407385in}{1.408984in}}%
\pgfpathlineto{\pgfqpoint{0.405288in}{1.406573in}}%
\pgfpathlineto{\pgfqpoint{0.391728in}{1.395341in}}%
\pgfpathlineto{\pgfqpoint{0.387525in}{1.392962in}}%
\pgfpathlineto{\pgfqpoint{0.376072in}{1.387441in}}%
\pgfpathlineto{\pgfqpoint{0.360415in}{1.384724in}}%
\pgfpathlineto{\pgfqpoint{0.360415in}{1.379351in}}%
\pgfpathlineto{\pgfqpoint{0.360415in}{1.365740in}}%
\pgfpathlineto{\pgfqpoint{0.360415in}{1.352129in}}%
\pgfpathlineto{\pgfqpoint{0.360415in}{1.347636in}}%
\pgfpathlineto{\pgfqpoint{0.376072in}{1.351157in}}%
\pgfpathclose%
\pgfpathmoveto{\pgfqpoint{0.657890in}{1.349908in}}%
\pgfpathlineto{\pgfqpoint{0.673546in}{1.347780in}}%
\pgfpathlineto{\pgfqpoint{0.687476in}{1.352129in}}%
\pgfpathlineto{\pgfqpoint{0.689203in}{1.352592in}}%
\pgfpathlineto{\pgfqpoint{0.704859in}{1.361938in}}%
\pgfpathlineto{\pgfqpoint{0.709464in}{1.365740in}}%
\pgfpathlineto{\pgfqpoint{0.720516in}{1.374113in}}%
\pgfpathlineto{\pgfqpoint{0.726592in}{1.379351in}}%
\pgfpathlineto{\pgfqpoint{0.736173in}{1.387680in}}%
\pgfpathlineto{\pgfqpoint{0.742198in}{1.392962in}}%
\pgfpathlineto{\pgfqpoint{0.751829in}{1.402570in}}%
\pgfpathlineto{\pgfqpoint{0.756202in}{1.406573in}}%
\pgfpathlineto{\pgfqpoint{0.766953in}{1.420184in}}%
\pgfpathlineto{\pgfqpoint{0.767486in}{1.421686in}}%
\pgfpathlineto{\pgfqpoint{0.772488in}{1.433795in}}%
\pgfpathlineto{\pgfqpoint{0.770040in}{1.447407in}}%
\pgfpathlineto{\pgfqpoint{0.767486in}{1.450833in}}%
\pgfpathlineto{\pgfqpoint{0.761066in}{1.461018in}}%
\pgfpathlineto{\pgfqpoint{0.751829in}{1.470414in}}%
\pgfpathlineto{\pgfqpoint{0.748004in}{1.474629in}}%
\pgfpathlineto{\pgfqpoint{0.736173in}{1.485395in}}%
\pgfpathlineto{\pgfqpoint{0.733011in}{1.488240in}}%
\pgfpathlineto{\pgfqpoint{0.720516in}{1.498906in}}%
\pgfpathlineto{\pgfqpoint{0.716592in}{1.501851in}}%
\pgfpathlineto{\pgfqpoint{0.704859in}{1.511065in}}%
\pgfpathlineto{\pgfqpoint{0.697158in}{1.515462in}}%
\pgfpathlineto{\pgfqpoint{0.689203in}{1.520641in}}%
\pgfpathlineto{\pgfqpoint{0.673546in}{1.525200in}}%
\pgfpathlineto{\pgfqpoint{0.657890in}{1.523211in}}%
\pgfpathlineto{\pgfqpoint{0.642804in}{1.515462in}}%
\pgfpathlineto{\pgfqpoint{0.642233in}{1.515202in}}%
\pgfpathlineto{\pgfqpoint{0.626577in}{1.504013in}}%
\pgfpathlineto{\pgfqpoint{0.624089in}{1.501851in}}%
\pgfpathlineto{\pgfqpoint{0.610920in}{1.490980in}}%
\pgfpathlineto{\pgfqpoint{0.607768in}{1.488240in}}%
\pgfpathlineto{\pgfqpoint{0.595263in}{1.476791in}}%
\pgfpathlineto{\pgfqpoint{0.592777in}{1.474629in}}%
\pgfpathlineto{\pgfqpoint{0.579906in}{1.461018in}}%
\pgfpathlineto{\pgfqpoint{0.579607in}{1.460521in}}%
\pgfpathlineto{\pgfqpoint{0.570693in}{1.447407in}}%
\pgfpathlineto{\pgfqpoint{0.568405in}{1.433795in}}%
\pgfpathlineto{\pgfqpoint{0.573649in}{1.420184in}}%
\pgfpathlineto{\pgfqpoint{0.579607in}{1.413269in}}%
\pgfpathlineto{\pgfqpoint{0.584665in}{1.406573in}}%
\pgfpathlineto{\pgfqpoint{0.595263in}{1.396373in}}%
\pgfpathlineto{\pgfqpoint{0.598651in}{1.392962in}}%
\pgfpathlineto{\pgfqpoint{0.610920in}{1.382099in}}%
\pgfpathlineto{\pgfqpoint{0.614193in}{1.379351in}}%
\pgfpathlineto{\pgfqpoint{0.626577in}{1.369065in}}%
\pgfpathlineto{\pgfqpoint{0.631425in}{1.365740in}}%
\pgfpathlineto{\pgfqpoint{0.642233in}{1.357710in}}%
\pgfpathlineto{\pgfqpoint{0.653949in}{1.352129in}}%
\pgfpathlineto{\pgfqpoint{0.657890in}{1.349908in}}%
\pgfpathclose%
\pgfpathmoveto{\pgfqpoint{0.642589in}{1.392962in}}%
\pgfpathlineto{\pgfqpoint{0.642233in}{1.393140in}}%
\pgfpathlineto{\pgfqpoint{0.626577in}{1.405454in}}%
\pgfpathlineto{\pgfqpoint{0.625413in}{1.406573in}}%
\pgfpathlineto{\pgfqpoint{0.615319in}{1.420184in}}%
\pgfpathlineto{\pgfqpoint{0.610920in}{1.433175in}}%
\pgfpathlineto{\pgfqpoint{0.610731in}{1.433795in}}%
\pgfpathlineto{\pgfqpoint{0.610920in}{1.435234in}}%
\pgfpathlineto{\pgfqpoint{0.612718in}{1.447407in}}%
\pgfpathlineto{\pgfqpoint{0.620846in}{1.461018in}}%
\pgfpathlineto{\pgfqpoint{0.626577in}{1.467144in}}%
\pgfpathlineto{\pgfqpoint{0.635186in}{1.474629in}}%
\pgfpathlineto{\pgfqpoint{0.642233in}{1.479610in}}%
\pgfpathlineto{\pgfqpoint{0.657890in}{1.486677in}}%
\pgfpathlineto{\pgfqpoint{0.671891in}{1.488240in}}%
\pgfpathlineto{\pgfqpoint{0.673546in}{1.488404in}}%
\pgfpathlineto{\pgfqpoint{0.674260in}{1.488240in}}%
\pgfpathlineto{\pgfqpoint{0.689203in}{1.484416in}}%
\pgfpathlineto{\pgfqpoint{0.704859in}{1.475640in}}%
\pgfpathlineto{\pgfqpoint{0.706147in}{1.474629in}}%
\pgfpathlineto{\pgfqpoint{0.720312in}{1.461018in}}%
\pgfpathlineto{\pgfqpoint{0.720516in}{1.460709in}}%
\pgfpathlineto{\pgfqpoint{0.727976in}{1.447407in}}%
\pgfpathlineto{\pgfqpoint{0.729864in}{1.433795in}}%
\pgfpathlineto{\pgfqpoint{0.725535in}{1.420184in}}%
\pgfpathlineto{\pgfqpoint{0.720516in}{1.413122in}}%
\pgfpathlineto{\pgfqpoint{0.715074in}{1.406573in}}%
\pgfpathlineto{\pgfqpoint{0.704859in}{1.397693in}}%
\pgfpathlineto{\pgfqpoint{0.697326in}{1.392962in}}%
\pgfpathlineto{\pgfqpoint{0.689203in}{1.388598in}}%
\pgfpathlineto{\pgfqpoint{0.673546in}{1.384835in}}%
\pgfpathlineto{\pgfqpoint{0.657890in}{1.386477in}}%
\pgfpathlineto{\pgfqpoint{0.642589in}{1.392962in}}%
\pgfpathclose%
\pgfpathmoveto{\pgfqpoint{0.971021in}{1.348922in}}%
\pgfpathlineto{\pgfqpoint{0.986678in}{1.348210in}}%
\pgfpathlineto{\pgfqpoint{0.996509in}{1.352129in}}%
\pgfpathlineto{\pgfqpoint{1.002334in}{1.354112in}}%
\pgfpathlineto{\pgfqpoint{1.017991in}{1.364244in}}%
\pgfpathlineto{\pgfqpoint{1.019731in}{1.365740in}}%
\pgfpathlineto{\pgfqpoint{1.033647in}{1.376693in}}%
\pgfpathlineto{\pgfqpoint{1.036697in}{1.379351in}}%
\pgfpathlineto{\pgfqpoint{1.049304in}{1.390422in}}%
\pgfpathlineto{\pgfqpoint{1.052234in}{1.392962in}}%
\pgfpathlineto{\pgfqpoint{1.064960in}{1.405526in}}%
\pgfpathlineto{\pgfqpoint{1.066140in}{1.406573in}}%
\pgfpathlineto{\pgfqpoint{1.077078in}{1.420184in}}%
\pgfpathlineto{\pgfqpoint{1.080617in}{1.429876in}}%
\pgfpathlineto{\pgfqpoint{1.082326in}{1.433795in}}%
\pgfpathlineto{\pgfqpoint{1.080617in}{1.442881in}}%
\pgfpathlineto{\pgfqpoint{1.079896in}{1.447407in}}%
\pgfpathlineto{\pgfqpoint{1.071089in}{1.461018in}}%
\pgfpathlineto{\pgfqpoint{1.064960in}{1.467062in}}%
\pgfpathlineto{\pgfqpoint{1.058021in}{1.474629in}}%
\pgfpathlineto{\pgfqpoint{1.049304in}{1.482473in}}%
\pgfpathlineto{\pgfqpoint{1.042961in}{1.488240in}}%
\pgfpathlineto{\pgfqpoint{1.033647in}{1.496276in}}%
\pgfpathlineto{\pgfqpoint{1.026499in}{1.501851in}}%
\pgfpathlineto{\pgfqpoint{1.017991in}{1.508809in}}%
\pgfpathlineto{\pgfqpoint{1.007317in}{1.515462in}}%
\pgfpathlineto{\pgfqpoint{1.002334in}{1.519019in}}%
\pgfpathlineto{\pgfqpoint{0.986678in}{1.524798in}}%
\pgfpathlineto{\pgfqpoint{0.971021in}{1.524133in}}%
\pgfpathlineto{\pgfqpoint{0.955364in}{1.517192in}}%
\pgfpathlineto{\pgfqpoint{0.953123in}{1.515462in}}%
\pgfpathlineto{\pgfqpoint{0.939708in}{1.506452in}}%
\pgfpathlineto{\pgfqpoint{0.934268in}{1.501851in}}%
\pgfpathlineto{\pgfqpoint{0.924051in}{1.493630in}}%
\pgfpathlineto{\pgfqpoint{0.917843in}{1.488240in}}%
\pgfpathlineto{\pgfqpoint{0.908395in}{1.479602in}}%
\pgfpathlineto{\pgfqpoint{0.902784in}{1.474629in}}%
\pgfpathlineto{\pgfqpoint{0.892738in}{1.463828in}}%
\pgfpathlineto{\pgfqpoint{0.889787in}{1.461018in}}%
\pgfpathlineto{\pgfqpoint{0.880787in}{1.447407in}}%
\pgfpathlineto{\pgfqpoint{0.878559in}{1.433795in}}%
\pgfpathlineto{\pgfqpoint{0.883667in}{1.420184in}}%
\pgfpathlineto{\pgfqpoint{0.892738in}{1.409231in}}%
\pgfpathlineto{\pgfqpoint{0.894711in}{1.406573in}}%
\pgfpathlineto{\pgfqpoint{0.908395in}{1.393150in}}%
\pgfpathlineto{\pgfqpoint{0.908582in}{1.392962in}}%
\pgfpathlineto{\pgfqpoint{0.923982in}{1.379351in}}%
\pgfpathlineto{\pgfqpoint{0.924051in}{1.379290in}}%
\pgfpathlineto{\pgfqpoint{0.939708in}{1.366631in}}%
\pgfpathlineto{\pgfqpoint{0.941092in}{1.365740in}}%
\pgfpathlineto{\pgfqpoint{0.955364in}{1.355823in}}%
\pgfpathlineto{\pgfqpoint{0.964376in}{1.352129in}}%
\pgfpathlineto{\pgfqpoint{0.971021in}{1.348922in}}%
\pgfpathclose%
\pgfpathmoveto{\pgfqpoint{0.952983in}{1.392962in}}%
\pgfpathlineto{\pgfqpoint{0.939708in}{1.402769in}}%
\pgfpathlineto{\pgfqpoint{0.935645in}{1.406573in}}%
\pgfpathlineto{\pgfqpoint{0.925142in}{1.420184in}}%
\pgfpathlineto{\pgfqpoint{0.924051in}{1.423324in}}%
\pgfpathlineto{\pgfqpoint{0.920852in}{1.433795in}}%
\pgfpathlineto{\pgfqpoint{0.922658in}{1.447407in}}%
\pgfpathlineto{\pgfqpoint{0.924051in}{1.449933in}}%
\pgfpathlineto{\pgfqpoint{0.930893in}{1.461018in}}%
\pgfpathlineto{\pgfqpoint{0.939708in}{1.470188in}}%
\pgfpathlineto{\pgfqpoint{0.945151in}{1.474629in}}%
\pgfpathlineto{\pgfqpoint{0.955364in}{1.481382in}}%
\pgfpathlineto{\pgfqpoint{0.971021in}{1.487487in}}%
\pgfpathlineto{\pgfqpoint{0.986678in}{1.488072in}}%
\pgfpathlineto{\pgfqpoint{1.002334in}{1.482989in}}%
\pgfpathlineto{\pgfqpoint{1.016010in}{1.474629in}}%
\pgfpathlineto{\pgfqpoint{1.017991in}{1.473129in}}%
\pgfpathlineto{\pgfqpoint{1.030030in}{1.461018in}}%
\pgfpathlineto{\pgfqpoint{1.033647in}{1.455338in}}%
\pgfpathlineto{\pgfqpoint{1.038048in}{1.447407in}}%
\pgfpathlineto{\pgfqpoint{1.039890in}{1.433795in}}%
\pgfpathlineto{\pgfqpoint{1.035667in}{1.420184in}}%
\pgfpathlineto{\pgfqpoint{1.033647in}{1.417313in}}%
\pgfpathlineto{\pgfqpoint{1.025057in}{1.406573in}}%
\pgfpathlineto{\pgfqpoint{1.017991in}{1.400176in}}%
\pgfpathlineto{\pgfqpoint{1.007470in}{1.392962in}}%
\pgfpathlineto{\pgfqpoint{1.002334in}{1.389938in}}%
\pgfpathlineto{\pgfqpoint{0.986678in}{1.385167in}}%
\pgfpathlineto{\pgfqpoint{0.971021in}{1.385716in}}%
\pgfpathlineto{\pgfqpoint{0.955364in}{1.391445in}}%
\pgfpathlineto{\pgfqpoint{0.952983in}{1.392962in}}%
\pgfpathclose%
\pgfpathmoveto{\pgfqpoint{1.284152in}{1.348210in}}%
\pgfpathlineto{\pgfqpoint{1.299809in}{1.348922in}}%
\pgfpathlineto{\pgfqpoint{1.306454in}{1.352129in}}%
\pgfpathlineto{\pgfqpoint{1.315466in}{1.355823in}}%
\pgfpathlineto{\pgfqpoint{1.329738in}{1.365740in}}%
\pgfpathlineto{\pgfqpoint{1.331122in}{1.366631in}}%
\pgfpathlineto{\pgfqpoint{1.346779in}{1.379290in}}%
\pgfpathlineto{\pgfqpoint{1.346848in}{1.379351in}}%
\pgfpathlineto{\pgfqpoint{1.362248in}{1.392962in}}%
\pgfpathlineto{\pgfqpoint{1.362435in}{1.393150in}}%
\pgfpathlineto{\pgfqpoint{1.376119in}{1.406573in}}%
\pgfpathlineto{\pgfqpoint{1.378092in}{1.409231in}}%
\pgfpathlineto{\pgfqpoint{1.387163in}{1.420184in}}%
\pgfpathlineto{\pgfqpoint{1.392271in}{1.433795in}}%
\pgfpathlineto{\pgfqpoint{1.390043in}{1.447407in}}%
\pgfpathlineto{\pgfqpoint{1.381043in}{1.461018in}}%
\pgfpathlineto{\pgfqpoint{1.378092in}{1.463828in}}%
\pgfpathlineto{\pgfqpoint{1.368046in}{1.474629in}}%
\pgfpathlineto{\pgfqpoint{1.362435in}{1.479602in}}%
\pgfpathlineto{\pgfqpoint{1.352987in}{1.488240in}}%
\pgfpathlineto{\pgfqpoint{1.346779in}{1.493630in}}%
\pgfpathlineto{\pgfqpoint{1.336562in}{1.501851in}}%
\pgfpathlineto{\pgfqpoint{1.331122in}{1.506452in}}%
\pgfpathlineto{\pgfqpoint{1.317707in}{1.515462in}}%
\pgfpathlineto{\pgfqpoint{1.315466in}{1.517192in}}%
\pgfpathlineto{\pgfqpoint{1.299809in}{1.524133in}}%
\pgfpathlineto{\pgfqpoint{1.284152in}{1.524798in}}%
\pgfpathlineto{\pgfqpoint{1.268496in}{1.519019in}}%
\pgfpathlineto{\pgfqpoint{1.263513in}{1.515462in}}%
\pgfpathlineto{\pgfqpoint{1.252839in}{1.508809in}}%
\pgfpathlineto{\pgfqpoint{1.244331in}{1.501851in}}%
\pgfpathlineto{\pgfqpoint{1.237183in}{1.496276in}}%
\pgfpathlineto{\pgfqpoint{1.227869in}{1.488240in}}%
\pgfpathlineto{\pgfqpoint{1.221526in}{1.482473in}}%
\pgfpathlineto{\pgfqpoint{1.212809in}{1.474629in}}%
\pgfpathlineto{\pgfqpoint{1.205870in}{1.467062in}}%
\pgfpathlineto{\pgfqpoint{1.199741in}{1.461018in}}%
\pgfpathlineto{\pgfqpoint{1.190934in}{1.447407in}}%
\pgfpathlineto{\pgfqpoint{1.190213in}{1.442881in}}%
\pgfpathlineto{\pgfqpoint{1.188504in}{1.433795in}}%
\pgfpathlineto{\pgfqpoint{1.190213in}{1.429876in}}%
\pgfpathlineto{\pgfqpoint{1.193752in}{1.420184in}}%
\pgfpathlineto{\pgfqpoint{1.204690in}{1.406573in}}%
\pgfpathlineto{\pgfqpoint{1.205870in}{1.405526in}}%
\pgfpathlineto{\pgfqpoint{1.218596in}{1.392962in}}%
\pgfpathlineto{\pgfqpoint{1.221526in}{1.390422in}}%
\pgfpathlineto{\pgfqpoint{1.234133in}{1.379351in}}%
\pgfpathlineto{\pgfqpoint{1.237183in}{1.376693in}}%
\pgfpathlineto{\pgfqpoint{1.251099in}{1.365740in}}%
\pgfpathlineto{\pgfqpoint{1.252839in}{1.364244in}}%
\pgfpathlineto{\pgfqpoint{1.268496in}{1.354112in}}%
\pgfpathlineto{\pgfqpoint{1.274321in}{1.352129in}}%
\pgfpathlineto{\pgfqpoint{1.284152in}{1.348210in}}%
\pgfpathclose%
\pgfpathmoveto{\pgfqpoint{1.263360in}{1.392962in}}%
\pgfpathlineto{\pgfqpoint{1.252839in}{1.400176in}}%
\pgfpathlineto{\pgfqpoint{1.245773in}{1.406573in}}%
\pgfpathlineto{\pgfqpoint{1.237183in}{1.417313in}}%
\pgfpathlineto{\pgfqpoint{1.235163in}{1.420184in}}%
\pgfpathlineto{\pgfqpoint{1.230940in}{1.433795in}}%
\pgfpathlineto{\pgfqpoint{1.232782in}{1.447407in}}%
\pgfpathlineto{\pgfqpoint{1.237183in}{1.455338in}}%
\pgfpathlineto{\pgfqpoint{1.240800in}{1.461018in}}%
\pgfpathlineto{\pgfqpoint{1.252839in}{1.473129in}}%
\pgfpathlineto{\pgfqpoint{1.254820in}{1.474629in}}%
\pgfpathlineto{\pgfqpoint{1.268496in}{1.482989in}}%
\pgfpathlineto{\pgfqpoint{1.284152in}{1.488072in}}%
\pgfpathlineto{\pgfqpoint{1.299809in}{1.487487in}}%
\pgfpathlineto{\pgfqpoint{1.315466in}{1.481382in}}%
\pgfpathlineto{\pgfqpoint{1.325679in}{1.474629in}}%
\pgfpathlineto{\pgfqpoint{1.331122in}{1.470188in}}%
\pgfpathlineto{\pgfqpoint{1.339937in}{1.461018in}}%
\pgfpathlineto{\pgfqpoint{1.346779in}{1.449933in}}%
\pgfpathlineto{\pgfqpoint{1.348172in}{1.447407in}}%
\pgfpathlineto{\pgfqpoint{1.349978in}{1.433795in}}%
\pgfpathlineto{\pgfqpoint{1.346779in}{1.423324in}}%
\pgfpathlineto{\pgfqpoint{1.345688in}{1.420184in}}%
\pgfpathlineto{\pgfqpoint{1.335185in}{1.406573in}}%
\pgfpathlineto{\pgfqpoint{1.331122in}{1.402769in}}%
\pgfpathlineto{\pgfqpoint{1.317847in}{1.392962in}}%
\pgfpathlineto{\pgfqpoint{1.315466in}{1.391445in}}%
\pgfpathlineto{\pgfqpoint{1.299809in}{1.385716in}}%
\pgfpathlineto{\pgfqpoint{1.284152in}{1.385167in}}%
\pgfpathlineto{\pgfqpoint{1.268496in}{1.389938in}}%
\pgfpathlineto{\pgfqpoint{1.263360in}{1.392962in}}%
\pgfpathclose%
\pgfpathmoveto{\pgfqpoint{1.597284in}{1.347780in}}%
\pgfpathlineto{\pgfqpoint{1.612940in}{1.349908in}}%
\pgfpathlineto{\pgfqpoint{1.616881in}{1.352129in}}%
\pgfpathlineto{\pgfqpoint{1.628597in}{1.357710in}}%
\pgfpathlineto{\pgfqpoint{1.639405in}{1.365740in}}%
\pgfpathlineto{\pgfqpoint{1.644253in}{1.369065in}}%
\pgfpathlineto{\pgfqpoint{1.656637in}{1.379351in}}%
\pgfpathlineto{\pgfqpoint{1.659910in}{1.382099in}}%
\pgfpathlineto{\pgfqpoint{1.672179in}{1.392962in}}%
\pgfpathlineto{\pgfqpoint{1.675567in}{1.396373in}}%
\pgfpathlineto{\pgfqpoint{1.686165in}{1.406573in}}%
\pgfpathlineto{\pgfqpoint{1.691223in}{1.413269in}}%
\pgfpathlineto{\pgfqpoint{1.697181in}{1.420184in}}%
\pgfpathlineto{\pgfqpoint{1.702425in}{1.433795in}}%
\pgfpathlineto{\pgfqpoint{1.700137in}{1.447407in}}%
\pgfpathlineto{\pgfqpoint{1.691223in}{1.460521in}}%
\pgfpathlineto{\pgfqpoint{1.690924in}{1.461018in}}%
\pgfpathlineto{\pgfqpoint{1.678053in}{1.474629in}}%
\pgfpathlineto{\pgfqpoint{1.675567in}{1.476791in}}%
\pgfpathlineto{\pgfqpoint{1.663062in}{1.488240in}}%
\pgfpathlineto{\pgfqpoint{1.659910in}{1.490980in}}%
\pgfpathlineto{\pgfqpoint{1.646741in}{1.501851in}}%
\pgfpathlineto{\pgfqpoint{1.644253in}{1.504013in}}%
\pgfpathlineto{\pgfqpoint{1.628597in}{1.515202in}}%
\pgfpathlineto{\pgfqpoint{1.628026in}{1.515462in}}%
\pgfpathlineto{\pgfqpoint{1.612940in}{1.523211in}}%
\pgfpathlineto{\pgfqpoint{1.597284in}{1.525200in}}%
\pgfpathlineto{\pgfqpoint{1.581627in}{1.520641in}}%
\pgfpathlineto{\pgfqpoint{1.573672in}{1.515462in}}%
\pgfpathlineto{\pgfqpoint{1.565971in}{1.511065in}}%
\pgfpathlineto{\pgfqpoint{1.554238in}{1.501851in}}%
\pgfpathlineto{\pgfqpoint{1.550314in}{1.498906in}}%
\pgfpathlineto{\pgfqpoint{1.537819in}{1.488240in}}%
\pgfpathlineto{\pgfqpoint{1.534657in}{1.485395in}}%
\pgfpathlineto{\pgfqpoint{1.522826in}{1.474629in}}%
\pgfpathlineto{\pgfqpoint{1.519001in}{1.470414in}}%
\pgfpathlineto{\pgfqpoint{1.509764in}{1.461018in}}%
\pgfpathlineto{\pgfqpoint{1.503344in}{1.450833in}}%
\pgfpathlineto{\pgfqpoint{1.500790in}{1.447407in}}%
\pgfpathlineto{\pgfqpoint{1.498342in}{1.433795in}}%
\pgfpathlineto{\pgfqpoint{1.503344in}{1.421686in}}%
\pgfpathlineto{\pgfqpoint{1.503877in}{1.420184in}}%
\pgfpathlineto{\pgfqpoint{1.514628in}{1.406573in}}%
\pgfpathlineto{\pgfqpoint{1.519001in}{1.402570in}}%
\pgfpathlineto{\pgfqpoint{1.528632in}{1.392962in}}%
\pgfpathlineto{\pgfqpoint{1.534657in}{1.387680in}}%
\pgfpathlineto{\pgfqpoint{1.544238in}{1.379351in}}%
\pgfpathlineto{\pgfqpoint{1.550314in}{1.374113in}}%
\pgfpathlineto{\pgfqpoint{1.561366in}{1.365740in}}%
\pgfpathlineto{\pgfqpoint{1.565971in}{1.361938in}}%
\pgfpathlineto{\pgfqpoint{1.581627in}{1.352592in}}%
\pgfpathlineto{\pgfqpoint{1.583354in}{1.352129in}}%
\pgfpathlineto{\pgfqpoint{1.597284in}{1.347780in}}%
\pgfpathclose%
\pgfpathmoveto{\pgfqpoint{1.573504in}{1.392962in}}%
\pgfpathlineto{\pgfqpoint{1.565971in}{1.397693in}}%
\pgfpathlineto{\pgfqpoint{1.555756in}{1.406573in}}%
\pgfpathlineto{\pgfqpoint{1.550314in}{1.413122in}}%
\pgfpathlineto{\pgfqpoint{1.545295in}{1.420184in}}%
\pgfpathlineto{\pgfqpoint{1.540966in}{1.433795in}}%
\pgfpathlineto{\pgfqpoint{1.542854in}{1.447407in}}%
\pgfpathlineto{\pgfqpoint{1.550314in}{1.460709in}}%
\pgfpathlineto{\pgfqpoint{1.550518in}{1.461018in}}%
\pgfpathlineto{\pgfqpoint{1.564683in}{1.474629in}}%
\pgfpathlineto{\pgfqpoint{1.565971in}{1.475640in}}%
\pgfpathlineto{\pgfqpoint{1.581627in}{1.484416in}}%
\pgfpathlineto{\pgfqpoint{1.596570in}{1.488240in}}%
\pgfpathlineto{\pgfqpoint{1.597284in}{1.488404in}}%
\pgfpathlineto{\pgfqpoint{1.598939in}{1.488240in}}%
\pgfpathlineto{\pgfqpoint{1.612940in}{1.486677in}}%
\pgfpathlineto{\pgfqpoint{1.628597in}{1.479610in}}%
\pgfpathlineto{\pgfqpoint{1.635644in}{1.474629in}}%
\pgfpathlineto{\pgfqpoint{1.644253in}{1.467144in}}%
\pgfpathlineto{\pgfqpoint{1.649984in}{1.461018in}}%
\pgfpathlineto{\pgfqpoint{1.658112in}{1.447407in}}%
\pgfpathlineto{\pgfqpoint{1.659910in}{1.435234in}}%
\pgfpathlineto{\pgfqpoint{1.660099in}{1.433795in}}%
\pgfpathlineto{\pgfqpoint{1.659910in}{1.433175in}}%
\pgfpathlineto{\pgfqpoint{1.655511in}{1.420184in}}%
\pgfpathlineto{\pgfqpoint{1.645417in}{1.406573in}}%
\pgfpathlineto{\pgfqpoint{1.644253in}{1.405454in}}%
\pgfpathlineto{\pgfqpoint{1.628597in}{1.393140in}}%
\pgfpathlineto{\pgfqpoint{1.628241in}{1.392962in}}%
\pgfpathlineto{\pgfqpoint{1.612940in}{1.386477in}}%
\pgfpathlineto{\pgfqpoint{1.597284in}{1.384835in}}%
\pgfpathlineto{\pgfqpoint{1.581627in}{1.388598in}}%
\pgfpathlineto{\pgfqpoint{1.573504in}{1.392962in}}%
\pgfpathclose%
\pgfpathmoveto{\pgfqpoint{1.894758in}{1.351157in}}%
\pgfpathlineto{\pgfqpoint{1.910415in}{1.347636in}}%
\pgfpathlineto{\pgfqpoint{1.910415in}{1.352129in}}%
\pgfpathlineto{\pgfqpoint{1.910415in}{1.365740in}}%
\pgfpathlineto{\pgfqpoint{1.910415in}{1.379351in}}%
\pgfpathlineto{\pgfqpoint{1.910415in}{1.384724in}}%
\pgfpathlineto{\pgfqpoint{1.894758in}{1.387441in}}%
\pgfpathlineto{\pgfqpoint{1.883305in}{1.392962in}}%
\pgfpathlineto{\pgfqpoint{1.879102in}{1.395341in}}%
\pgfpathlineto{\pgfqpoint{1.865542in}{1.406573in}}%
\pgfpathlineto{\pgfqpoint{1.863445in}{1.408984in}}%
\pgfpathlineto{\pgfqpoint{1.855357in}{1.420184in}}%
\pgfpathlineto{\pgfqpoint{1.850900in}{1.433795in}}%
\pgfpathlineto{\pgfqpoint{1.852844in}{1.447407in}}%
\pgfpathlineto{\pgfqpoint{1.860698in}{1.461018in}}%
\pgfpathlineto{\pgfqpoint{1.863445in}{1.464018in}}%
\pgfpathlineto{\pgfqpoint{1.875011in}{1.474629in}}%
\pgfpathlineto{\pgfqpoint{1.879102in}{1.477691in}}%
\pgfpathlineto{\pgfqpoint{1.894758in}{1.485649in}}%
\pgfpathlineto{\pgfqpoint{1.908788in}{1.488240in}}%
\pgfpathlineto{\pgfqpoint{1.910415in}{1.488509in}}%
\pgfpathlineto{\pgfqpoint{1.910415in}{1.501851in}}%
\pgfpathlineto{\pgfqpoint{1.910415in}{1.515462in}}%
\pgfpathlineto{\pgfqpoint{1.910415in}{1.525335in}}%
\pgfpathlineto{\pgfqpoint{1.894758in}{1.522044in}}%
\pgfpathlineto{\pgfqpoint{1.883494in}{1.515462in}}%
\pgfpathlineto{\pgfqpoint{1.879102in}{1.513202in}}%
\pgfpathlineto{\pgfqpoint{1.863931in}{1.501851in}}%
\pgfpathlineto{\pgfqpoint{1.863445in}{1.501502in}}%
\pgfpathlineto{\pgfqpoint{1.847789in}{1.488342in}}%
\pgfpathlineto{\pgfqpoint{1.847671in}{1.488240in}}%
\pgfpathlineto{\pgfqpoint{1.832810in}{1.474629in}}%
\pgfpathlineto{\pgfqpoint{1.832132in}{1.473877in}}%
\pgfpathlineto{\pgfqpoint{1.819828in}{1.461018in}}%
\pgfpathlineto{\pgfqpoint{1.816476in}{1.455564in}}%
\pgfpathlineto{\pgfqpoint{1.810682in}{1.447407in}}%
\pgfpathlineto{\pgfqpoint{1.808322in}{1.433795in}}%
\pgfpathlineto{\pgfqpoint{1.813732in}{1.420184in}}%
\pgfpathlineto{\pgfqpoint{1.816476in}{1.417136in}}%
\pgfpathlineto{\pgfqpoint{1.824629in}{1.406573in}}%
\pgfpathlineto{\pgfqpoint{1.832132in}{1.399516in}}%
\pgfpathlineto{\pgfqpoint{1.838659in}{1.392962in}}%
\pgfpathlineto{\pgfqpoint{1.847789in}{1.384902in}}%
\pgfpathlineto{\pgfqpoint{1.854269in}{1.379351in}}%
\pgfpathlineto{\pgfqpoint{1.863445in}{1.371565in}}%
\pgfpathlineto{\pgfqpoint{1.871492in}{1.365740in}}%
\pgfpathlineto{\pgfqpoint{1.879102in}{1.359754in}}%
\pgfpathlineto{\pgfqpoint{1.893243in}{1.352129in}}%
\pgfpathlineto{\pgfqpoint{1.894758in}{1.351157in}}%
\pgfpathclose%
\pgfpathmoveto{\pgfqpoint{0.376072in}{1.620519in}}%
\pgfpathlineto{\pgfqpoint{0.382348in}{1.624351in}}%
\pgfpathlineto{\pgfqpoint{0.391728in}{1.629283in}}%
\pgfpathlineto{\pgfqpoint{0.403064in}{1.637962in}}%
\pgfpathlineto{\pgfqpoint{0.407385in}{1.641070in}}%
\pgfpathlineto{\pgfqpoint{0.419852in}{1.651573in}}%
\pgfpathlineto{\pgfqpoint{0.423041in}{1.654346in}}%
\pgfpathlineto{\pgfqpoint{0.435123in}{1.665184in}}%
\pgfpathlineto{\pgfqpoint{0.438698in}{1.668941in}}%
\pgfpathlineto{\pgfqpoint{0.448681in}{1.678795in}}%
\pgfpathlineto{\pgfqpoint{0.454354in}{1.686950in}}%
\pgfpathlineto{\pgfqpoint{0.458762in}{1.692407in}}%
\pgfpathlineto{\pgfqpoint{0.462667in}{1.706018in}}%
\pgfpathlineto{\pgfqpoint{0.454354in}{1.706018in}}%
\pgfpathlineto{\pgfqpoint{0.438698in}{1.706018in}}%
\pgfpathlineto{\pgfqpoint{0.423041in}{1.706018in}}%
\pgfpathlineto{\pgfqpoint{0.420062in}{1.706018in}}%
\pgfpathlineto{\pgfqpoint{0.416844in}{1.692407in}}%
\pgfpathlineto{\pgfqpoint{0.407999in}{1.678795in}}%
\pgfpathlineto{\pgfqpoint{0.407385in}{1.678162in}}%
\pgfpathlineto{\pgfqpoint{0.392457in}{1.665184in}}%
\pgfpathlineto{\pgfqpoint{0.391728in}{1.664651in}}%
\pgfpathlineto{\pgfqpoint{0.376072in}{1.656961in}}%
\pgfpathlineto{\pgfqpoint{0.360415in}{1.654163in}}%
\pgfpathlineto{\pgfqpoint{0.360415in}{1.651573in}}%
\pgfpathlineto{\pgfqpoint{0.360415in}{1.637962in}}%
\pgfpathlineto{\pgfqpoint{0.360415in}{1.624351in}}%
\pgfpathlineto{\pgfqpoint{0.360415in}{1.617124in}}%
\pgfpathlineto{\pgfqpoint{0.376072in}{1.620519in}}%
\pgfpathclose%
\pgfpathmoveto{\pgfqpoint{0.657890in}{1.619315in}}%
\pgfpathlineto{\pgfqpoint{0.673546in}{1.617263in}}%
\pgfpathlineto{\pgfqpoint{0.689203in}{1.621966in}}%
\pgfpathlineto{\pgfqpoint{0.692709in}{1.624351in}}%
\pgfpathlineto{\pgfqpoint{0.704859in}{1.631439in}}%
\pgfpathlineto{\pgfqpoint{0.712977in}{1.637962in}}%
\pgfpathlineto{\pgfqpoint{0.720516in}{1.643637in}}%
\pgfpathlineto{\pgfqpoint{0.729787in}{1.651573in}}%
\pgfpathlineto{\pgfqpoint{0.736173in}{1.657207in}}%
\pgfpathlineto{\pgfqpoint{0.745128in}{1.665184in}}%
\pgfpathlineto{\pgfqpoint{0.751829in}{1.672180in}}%
\pgfpathlineto{\pgfqpoint{0.758714in}{1.678795in}}%
\pgfpathlineto{\pgfqpoint{0.767486in}{1.691089in}}%
\pgfpathlineto{\pgfqpoint{0.768603in}{1.692407in}}%
\pgfpathlineto{\pgfqpoint{0.772653in}{1.706018in}}%
\pgfpathlineto{\pgfqpoint{0.767486in}{1.706018in}}%
\pgfpathlineto{\pgfqpoint{0.751829in}{1.706018in}}%
\pgfpathlineto{\pgfqpoint{0.736173in}{1.706018in}}%
\pgfpathlineto{\pgfqpoint{0.729992in}{1.706018in}}%
\pgfpathlineto{\pgfqpoint{0.726867in}{1.692407in}}%
\pgfpathlineto{\pgfqpoint{0.720516in}{1.682449in}}%
\pgfpathlineto{\pgfqpoint{0.717779in}{1.678795in}}%
\pgfpathlineto{\pgfqpoint{0.704859in}{1.667007in}}%
\pgfpathlineto{\pgfqpoint{0.702086in}{1.665184in}}%
\pgfpathlineto{\pgfqpoint{0.689203in}{1.658153in}}%
\pgfpathlineto{\pgfqpoint{0.673546in}{1.654278in}}%
\pgfpathlineto{\pgfqpoint{0.657890in}{1.655968in}}%
\pgfpathlineto{\pgfqpoint{0.642233in}{1.662796in}}%
\pgfpathlineto{\pgfqpoint{0.638782in}{1.665184in}}%
\pgfpathlineto{\pgfqpoint{0.626577in}{1.675239in}}%
\pgfpathlineto{\pgfqpoint{0.623054in}{1.678795in}}%
\pgfpathlineto{\pgfqpoint{0.613900in}{1.692407in}}%
\pgfpathlineto{\pgfqpoint{0.610920in}{1.704603in}}%
\pgfpathlineto{\pgfqpoint{0.610610in}{1.706018in}}%
\pgfpathlineto{\pgfqpoint{0.595263in}{1.706018in}}%
\pgfpathlineto{\pgfqpoint{0.579607in}{1.706018in}}%
\pgfpathlineto{\pgfqpoint{0.568251in}{1.706018in}}%
\pgfpathlineto{\pgfqpoint{0.572036in}{1.692407in}}%
\pgfpathlineto{\pgfqpoint{0.579607in}{1.682613in}}%
\pgfpathlineto{\pgfqpoint{0.582207in}{1.678795in}}%
\pgfpathlineto{\pgfqpoint{0.595263in}{1.665607in}}%
\pgfpathlineto{\pgfqpoint{0.595665in}{1.665184in}}%
\pgfpathlineto{\pgfqpoint{0.610803in}{1.651573in}}%
\pgfpathlineto{\pgfqpoint{0.610920in}{1.651471in}}%
\pgfpathlineto{\pgfqpoint{0.626577in}{1.638551in}}%
\pgfpathlineto{\pgfqpoint{0.627441in}{1.637962in}}%
\pgfpathlineto{\pgfqpoint{0.642233in}{1.627265in}}%
\pgfpathlineto{\pgfqpoint{0.648506in}{1.624351in}}%
\pgfpathlineto{\pgfqpoint{0.657890in}{1.619315in}}%
\pgfpathclose%
\pgfpathmoveto{\pgfqpoint{0.971021in}{1.618364in}}%
\pgfpathlineto{\pgfqpoint{0.986678in}{1.617678in}}%
\pgfpathlineto{\pgfqpoint{1.002334in}{1.623640in}}%
\pgfpathlineto{\pgfqpoint{1.003288in}{1.624351in}}%
\pgfpathlineto{\pgfqpoint{1.017991in}{1.633715in}}%
\pgfpathlineto{\pgfqpoint{1.023066in}{1.637962in}}%
\pgfpathlineto{\pgfqpoint{1.033647in}{1.646236in}}%
\pgfpathlineto{\pgfqpoint{1.039815in}{1.651573in}}%
\pgfpathlineto{\pgfqpoint{1.049304in}{1.660030in}}%
\pgfpathlineto{\pgfqpoint{1.055155in}{1.665184in}}%
\pgfpathlineto{\pgfqpoint{1.064960in}{1.675315in}}%
\pgfpathlineto{\pgfqpoint{1.068696in}{1.678795in}}%
\pgfpathlineto{\pgfqpoint{1.078616in}{1.692407in}}%
\pgfpathlineto{\pgfqpoint{1.080617in}{1.700003in}}%
\pgfpathlineto{\pgfqpoint{1.082498in}{1.706018in}}%
\pgfpathlineto{\pgfqpoint{1.080617in}{1.706018in}}%
\pgfpathlineto{\pgfqpoint{1.064960in}{1.706018in}}%
\pgfpathlineto{\pgfqpoint{1.049304in}{1.706018in}}%
\pgfpathlineto{\pgfqpoint{1.040015in}{1.706018in}}%
\pgfpathlineto{\pgfqpoint{1.036966in}{1.692407in}}%
\pgfpathlineto{\pgfqpoint{1.033647in}{1.687147in}}%
\pgfpathlineto{\pgfqpoint{1.027625in}{1.678795in}}%
\pgfpathlineto{\pgfqpoint{1.017991in}{1.669640in}}%
\pgfpathlineto{\pgfqpoint{1.011781in}{1.665184in}}%
\pgfpathlineto{\pgfqpoint{1.002334in}{1.659532in}}%
\pgfpathlineto{\pgfqpoint{0.986678in}{1.654619in}}%
\pgfpathlineto{\pgfqpoint{0.971021in}{1.655185in}}%
\pgfpathlineto{\pgfqpoint{0.955364in}{1.661084in}}%
\pgfpathlineto{\pgfqpoint{0.949030in}{1.665184in}}%
\pgfpathlineto{\pgfqpoint{0.939708in}{1.672391in}}%
\pgfpathlineto{\pgfqpoint{0.933191in}{1.678795in}}%
\pgfpathlineto{\pgfqpoint{0.924051in}{1.691876in}}%
\pgfpathlineto{\pgfqpoint{0.923719in}{1.692407in}}%
\pgfpathlineto{\pgfqpoint{0.920730in}{1.706018in}}%
\pgfpathlineto{\pgfqpoint{0.908395in}{1.706018in}}%
\pgfpathlineto{\pgfqpoint{0.892738in}{1.706018in}}%
\pgfpathlineto{\pgfqpoint{0.878408in}{1.706018in}}%
\pgfpathlineto{\pgfqpoint{0.882095in}{1.692407in}}%
\pgfpathlineto{\pgfqpoint{0.892232in}{1.678795in}}%
\pgfpathlineto{\pgfqpoint{0.892738in}{1.678340in}}%
\pgfpathlineto{\pgfqpoint{0.905652in}{1.665184in}}%
\pgfpathlineto{\pgfqpoint{0.908395in}{1.662804in}}%
\pgfpathlineto{\pgfqpoint{0.920926in}{1.651573in}}%
\pgfpathlineto{\pgfqpoint{0.924051in}{1.648852in}}%
\pgfpathlineto{\pgfqpoint{0.937548in}{1.637962in}}%
\pgfpathlineto{\pgfqpoint{0.939708in}{1.636092in}}%
\pgfpathlineto{\pgfqpoint{0.955364in}{1.625403in}}%
\pgfpathlineto{\pgfqpoint{0.957995in}{1.624351in}}%
\pgfpathlineto{\pgfqpoint{0.971021in}{1.618364in}}%
\pgfpathclose%
\pgfpathmoveto{\pgfqpoint{1.268496in}{1.623640in}}%
\pgfpathlineto{\pgfqpoint{1.284152in}{1.617678in}}%
\pgfpathlineto{\pgfqpoint{1.299809in}{1.618364in}}%
\pgfpathlineto{\pgfqpoint{1.312835in}{1.624351in}}%
\pgfpathlineto{\pgfqpoint{1.315466in}{1.625403in}}%
\pgfpathlineto{\pgfqpoint{1.331122in}{1.636092in}}%
\pgfpathlineto{\pgfqpoint{1.333282in}{1.637962in}}%
\pgfpathlineto{\pgfqpoint{1.346779in}{1.648852in}}%
\pgfpathlineto{\pgfqpoint{1.349904in}{1.651573in}}%
\pgfpathlineto{\pgfqpoint{1.362435in}{1.662804in}}%
\pgfpathlineto{\pgfqpoint{1.365178in}{1.665184in}}%
\pgfpathlineto{\pgfqpoint{1.378092in}{1.678340in}}%
\pgfpathlineto{\pgfqpoint{1.378598in}{1.678795in}}%
\pgfpathlineto{\pgfqpoint{1.388735in}{1.692407in}}%
\pgfpathlineto{\pgfqpoint{1.392422in}{1.706018in}}%
\pgfpathlineto{\pgfqpoint{1.378092in}{1.706018in}}%
\pgfpathlineto{\pgfqpoint{1.362435in}{1.706018in}}%
\pgfpathlineto{\pgfqpoint{1.350100in}{1.706018in}}%
\pgfpathlineto{\pgfqpoint{1.347111in}{1.692407in}}%
\pgfpathlineto{\pgfqpoint{1.346779in}{1.691876in}}%
\pgfpathlineto{\pgfqpoint{1.337639in}{1.678795in}}%
\pgfpathlineto{\pgfqpoint{1.331122in}{1.672391in}}%
\pgfpathlineto{\pgfqpoint{1.321800in}{1.665184in}}%
\pgfpathlineto{\pgfqpoint{1.315466in}{1.661084in}}%
\pgfpathlineto{\pgfqpoint{1.299809in}{1.655185in}}%
\pgfpathlineto{\pgfqpoint{1.284152in}{1.654619in}}%
\pgfpathlineto{\pgfqpoint{1.268496in}{1.659532in}}%
\pgfpathlineto{\pgfqpoint{1.259049in}{1.665184in}}%
\pgfpathlineto{\pgfqpoint{1.252839in}{1.669640in}}%
\pgfpathlineto{\pgfqpoint{1.243205in}{1.678795in}}%
\pgfpathlineto{\pgfqpoint{1.237183in}{1.687147in}}%
\pgfpathlineto{\pgfqpoint{1.233864in}{1.692407in}}%
\pgfpathlineto{\pgfqpoint{1.230815in}{1.706018in}}%
\pgfpathlineto{\pgfqpoint{1.221526in}{1.706018in}}%
\pgfpathlineto{\pgfqpoint{1.205870in}{1.706018in}}%
\pgfpathlineto{\pgfqpoint{1.190213in}{1.706018in}}%
\pgfpathlineto{\pgfqpoint{1.188332in}{1.706018in}}%
\pgfpathlineto{\pgfqpoint{1.190213in}{1.700003in}}%
\pgfpathlineto{\pgfqpoint{1.192214in}{1.692407in}}%
\pgfpathlineto{\pgfqpoint{1.202134in}{1.678795in}}%
\pgfpathlineto{\pgfqpoint{1.205870in}{1.675315in}}%
\pgfpathlineto{\pgfqpoint{1.215675in}{1.665184in}}%
\pgfpathlineto{\pgfqpoint{1.221526in}{1.660030in}}%
\pgfpathlineto{\pgfqpoint{1.231015in}{1.651573in}}%
\pgfpathlineto{\pgfqpoint{1.237183in}{1.646236in}}%
\pgfpathlineto{\pgfqpoint{1.247764in}{1.637962in}}%
\pgfpathlineto{\pgfqpoint{1.252839in}{1.633715in}}%
\pgfpathlineto{\pgfqpoint{1.267542in}{1.624351in}}%
\pgfpathlineto{\pgfqpoint{1.268496in}{1.623640in}}%
\pgfpathclose%
\pgfpathmoveto{\pgfqpoint{1.581627in}{1.621966in}}%
\pgfpathlineto{\pgfqpoint{1.597284in}{1.617263in}}%
\pgfpathlineto{\pgfqpoint{1.612940in}{1.619315in}}%
\pgfpathlineto{\pgfqpoint{1.622324in}{1.624351in}}%
\pgfpathlineto{\pgfqpoint{1.628597in}{1.627265in}}%
\pgfpathlineto{\pgfqpoint{1.643389in}{1.637962in}}%
\pgfpathlineto{\pgfqpoint{1.644253in}{1.638551in}}%
\pgfpathlineto{\pgfqpoint{1.659910in}{1.651471in}}%
\pgfpathlineto{\pgfqpoint{1.660027in}{1.651573in}}%
\pgfpathlineto{\pgfqpoint{1.675165in}{1.665184in}}%
\pgfpathlineto{\pgfqpoint{1.675567in}{1.665607in}}%
\pgfpathlineto{\pgfqpoint{1.688623in}{1.678795in}}%
\pgfpathlineto{\pgfqpoint{1.691223in}{1.682613in}}%
\pgfpathlineto{\pgfqpoint{1.698794in}{1.692407in}}%
\pgfpathlineto{\pgfqpoint{1.702579in}{1.706018in}}%
\pgfpathlineto{\pgfqpoint{1.691223in}{1.706018in}}%
\pgfpathlineto{\pgfqpoint{1.675567in}{1.706018in}}%
\pgfpathlineto{\pgfqpoint{1.660220in}{1.706018in}}%
\pgfpathlineto{\pgfqpoint{1.659910in}{1.704603in}}%
\pgfpathlineto{\pgfqpoint{1.656930in}{1.692407in}}%
\pgfpathlineto{\pgfqpoint{1.647776in}{1.678795in}}%
\pgfpathlineto{\pgfqpoint{1.644253in}{1.675239in}}%
\pgfpathlineto{\pgfqpoint{1.632048in}{1.665184in}}%
\pgfpathlineto{\pgfqpoint{1.628597in}{1.662796in}}%
\pgfpathlineto{\pgfqpoint{1.612940in}{1.655968in}}%
\pgfpathlineto{\pgfqpoint{1.597284in}{1.654278in}}%
\pgfpathlineto{\pgfqpoint{1.581627in}{1.658153in}}%
\pgfpathlineto{\pgfqpoint{1.568744in}{1.665184in}}%
\pgfpathlineto{\pgfqpoint{1.565971in}{1.667007in}}%
\pgfpathlineto{\pgfqpoint{1.553051in}{1.678795in}}%
\pgfpathlineto{\pgfqpoint{1.550314in}{1.682449in}}%
\pgfpathlineto{\pgfqpoint{1.543963in}{1.692407in}}%
\pgfpathlineto{\pgfqpoint{1.540838in}{1.706018in}}%
\pgfpathlineto{\pgfqpoint{1.534657in}{1.706018in}}%
\pgfpathlineto{\pgfqpoint{1.519001in}{1.706018in}}%
\pgfpathlineto{\pgfqpoint{1.503344in}{1.706018in}}%
\pgfpathlineto{\pgfqpoint{1.498177in}{1.706018in}}%
\pgfpathlineto{\pgfqpoint{1.502227in}{1.692407in}}%
\pgfpathlineto{\pgfqpoint{1.503344in}{1.691089in}}%
\pgfpathlineto{\pgfqpoint{1.512116in}{1.678795in}}%
\pgfpathlineto{\pgfqpoint{1.519001in}{1.672180in}}%
\pgfpathlineto{\pgfqpoint{1.525702in}{1.665184in}}%
\pgfpathlineto{\pgfqpoint{1.534657in}{1.657207in}}%
\pgfpathlineto{\pgfqpoint{1.541043in}{1.651573in}}%
\pgfpathlineto{\pgfqpoint{1.550314in}{1.643637in}}%
\pgfpathlineto{\pgfqpoint{1.557853in}{1.637962in}}%
\pgfpathlineto{\pgfqpoint{1.565971in}{1.631439in}}%
\pgfpathlineto{\pgfqpoint{1.578121in}{1.624351in}}%
\pgfpathlineto{\pgfqpoint{1.581627in}{1.621966in}}%
\pgfpathclose%
\pgfpathmoveto{\pgfqpoint{1.894758in}{1.620519in}}%
\pgfpathlineto{\pgfqpoint{1.910415in}{1.617124in}}%
\pgfpathlineto{\pgfqpoint{1.910415in}{1.624351in}}%
\pgfpathlineto{\pgfqpoint{1.910415in}{1.637962in}}%
\pgfpathlineto{\pgfqpoint{1.910415in}{1.651573in}}%
\pgfpathlineto{\pgfqpoint{1.910415in}{1.654163in}}%
\pgfpathlineto{\pgfqpoint{1.894758in}{1.656961in}}%
\pgfpathlineto{\pgfqpoint{1.879102in}{1.664651in}}%
\pgfpathlineto{\pgfqpoint{1.878373in}{1.665184in}}%
\pgfpathlineto{\pgfqpoint{1.863445in}{1.678162in}}%
\pgfpathlineto{\pgfqpoint{1.862831in}{1.678795in}}%
\pgfpathlineto{\pgfqpoint{1.853986in}{1.692407in}}%
\pgfpathlineto{\pgfqpoint{1.850768in}{1.706018in}}%
\pgfpathlineto{\pgfqpoint{1.847789in}{1.706018in}}%
\pgfpathlineto{\pgfqpoint{1.832132in}{1.706018in}}%
\pgfpathlineto{\pgfqpoint{1.816476in}{1.706018in}}%
\pgfpathlineto{\pgfqpoint{1.808163in}{1.706018in}}%
\pgfpathlineto{\pgfqpoint{1.812068in}{1.692407in}}%
\pgfpathlineto{\pgfqpoint{1.816476in}{1.686950in}}%
\pgfpathlineto{\pgfqpoint{1.822149in}{1.678795in}}%
\pgfpathlineto{\pgfqpoint{1.832132in}{1.668941in}}%
\pgfpathlineto{\pgfqpoint{1.835707in}{1.665184in}}%
\pgfpathlineto{\pgfqpoint{1.847789in}{1.654346in}}%
\pgfpathlineto{\pgfqpoint{1.850978in}{1.651573in}}%
\pgfpathlineto{\pgfqpoint{1.863445in}{1.641070in}}%
\pgfpathlineto{\pgfqpoint{1.867766in}{1.637962in}}%
\pgfpathlineto{\pgfqpoint{1.879102in}{1.629283in}}%
\pgfpathlineto{\pgfqpoint{1.888482in}{1.624351in}}%
\pgfpathlineto{\pgfqpoint{1.894758in}{1.620519in}}%
\pgfpathclose%
\pgfusepath{fill}%
\end{pgfscope}%
\begin{pgfscope}%
\pgfpathrectangle{\pgfqpoint{0.360415in}{0.358518in}}{\pgfqpoint{1.550000in}{1.347500in}}%
\pgfusepath{clip}%
\pgfsetbuttcap%
\pgfsetroundjoin%
\definecolor{currentfill}{rgb}{0.048062,0.036607,0.150327}%
\pgfsetfillcolor{currentfill}%
\pgfsetlinewidth{0.000000pt}%
\definecolor{currentstroke}{rgb}{0.000000,0.000000,0.000000}%
\pgfsetstrokecolor{currentstroke}%
\pgfsetdash{}{0pt}%
\pgfpathmoveto{\pgfqpoint{0.376072in}{0.358518in}}%
\pgfpathlineto{\pgfqpoint{0.391728in}{0.358518in}}%
\pgfpathlineto{\pgfqpoint{0.407385in}{0.358518in}}%
\pgfpathlineto{\pgfqpoint{0.420062in}{0.358518in}}%
\pgfpathlineto{\pgfqpoint{0.416844in}{0.372129in}}%
\pgfpathlineto{\pgfqpoint{0.407999in}{0.385740in}}%
\pgfpathlineto{\pgfqpoint{0.407385in}{0.386373in}}%
\pgfpathlineto{\pgfqpoint{0.392457in}{0.399351in}}%
\pgfpathlineto{\pgfqpoint{0.391728in}{0.399885in}}%
\pgfpathlineto{\pgfqpoint{0.376072in}{0.407574in}}%
\pgfpathlineto{\pgfqpoint{0.360415in}{0.410372in}}%
\pgfpathlineto{\pgfqpoint{0.360415in}{0.399351in}}%
\pgfpathlineto{\pgfqpoint{0.360415in}{0.385740in}}%
\pgfpathlineto{\pgfqpoint{0.360415in}{0.372129in}}%
\pgfpathlineto{\pgfqpoint{0.360415in}{0.358518in}}%
\pgfpathlineto{\pgfqpoint{0.376072in}{0.358518in}}%
\pgfpathclose%
\pgfpathmoveto{\pgfqpoint{0.610920in}{0.358518in}}%
\pgfpathlineto{\pgfqpoint{0.626577in}{0.358518in}}%
\pgfpathlineto{\pgfqpoint{0.642233in}{0.358518in}}%
\pgfpathlineto{\pgfqpoint{0.657890in}{0.358518in}}%
\pgfpathlineto{\pgfqpoint{0.673546in}{0.358518in}}%
\pgfpathlineto{\pgfqpoint{0.689203in}{0.358518in}}%
\pgfpathlineto{\pgfqpoint{0.704859in}{0.358518in}}%
\pgfpathlineto{\pgfqpoint{0.720516in}{0.358518in}}%
\pgfpathlineto{\pgfqpoint{0.729992in}{0.358518in}}%
\pgfpathlineto{\pgfqpoint{0.726867in}{0.372129in}}%
\pgfpathlineto{\pgfqpoint{0.720516in}{0.382086in}}%
\pgfpathlineto{\pgfqpoint{0.717779in}{0.385740in}}%
\pgfpathlineto{\pgfqpoint{0.704859in}{0.397528in}}%
\pgfpathlineto{\pgfqpoint{0.702086in}{0.399351in}}%
\pgfpathlineto{\pgfqpoint{0.689203in}{0.406383in}}%
\pgfpathlineto{\pgfqpoint{0.673546in}{0.410258in}}%
\pgfpathlineto{\pgfqpoint{0.657890in}{0.408567in}}%
\pgfpathlineto{\pgfqpoint{0.642233in}{0.401739in}}%
\pgfpathlineto{\pgfqpoint{0.638782in}{0.399351in}}%
\pgfpathlineto{\pgfqpoint{0.626577in}{0.389297in}}%
\pgfpathlineto{\pgfqpoint{0.623054in}{0.385740in}}%
\pgfpathlineto{\pgfqpoint{0.613900in}{0.372129in}}%
\pgfpathlineto{\pgfqpoint{0.610920in}{0.359932in}}%
\pgfpathlineto{\pgfqpoint{0.610610in}{0.358518in}}%
\pgfpathlineto{\pgfqpoint{0.610920in}{0.358518in}}%
\pgfpathclose%
\pgfpathmoveto{\pgfqpoint{0.924051in}{0.358518in}}%
\pgfpathlineto{\pgfqpoint{0.939708in}{0.358518in}}%
\pgfpathlineto{\pgfqpoint{0.955364in}{0.358518in}}%
\pgfpathlineto{\pgfqpoint{0.971021in}{0.358518in}}%
\pgfpathlineto{\pgfqpoint{0.986678in}{0.358518in}}%
\pgfpathlineto{\pgfqpoint{1.002334in}{0.358518in}}%
\pgfpathlineto{\pgfqpoint{1.017991in}{0.358518in}}%
\pgfpathlineto{\pgfqpoint{1.033647in}{0.358518in}}%
\pgfpathlineto{\pgfqpoint{1.040015in}{0.358518in}}%
\pgfpathlineto{\pgfqpoint{1.036966in}{0.372129in}}%
\pgfpathlineto{\pgfqpoint{1.033647in}{0.377388in}}%
\pgfpathlineto{\pgfqpoint{1.027625in}{0.385740in}}%
\pgfpathlineto{\pgfqpoint{1.017991in}{0.394895in}}%
\pgfpathlineto{\pgfqpoint{1.011781in}{0.399351in}}%
\pgfpathlineto{\pgfqpoint{1.002334in}{0.405004in}}%
\pgfpathlineto{\pgfqpoint{0.986678in}{0.409916in}}%
\pgfpathlineto{\pgfqpoint{0.971021in}{0.409350in}}%
\pgfpathlineto{\pgfqpoint{0.955364in}{0.403451in}}%
\pgfpathlineto{\pgfqpoint{0.949030in}{0.399351in}}%
\pgfpathlineto{\pgfqpoint{0.939708in}{0.392144in}}%
\pgfpathlineto{\pgfqpoint{0.933191in}{0.385740in}}%
\pgfpathlineto{\pgfqpoint{0.924051in}{0.372659in}}%
\pgfpathlineto{\pgfqpoint{0.923719in}{0.372129in}}%
\pgfpathlineto{\pgfqpoint{0.920730in}{0.358518in}}%
\pgfpathlineto{\pgfqpoint{0.924051in}{0.358518in}}%
\pgfpathclose%
\pgfpathmoveto{\pgfqpoint{1.237183in}{0.358518in}}%
\pgfpathlineto{\pgfqpoint{1.252839in}{0.358518in}}%
\pgfpathlineto{\pgfqpoint{1.268496in}{0.358518in}}%
\pgfpathlineto{\pgfqpoint{1.284152in}{0.358518in}}%
\pgfpathlineto{\pgfqpoint{1.299809in}{0.358518in}}%
\pgfpathlineto{\pgfqpoint{1.315466in}{0.358518in}}%
\pgfpathlineto{\pgfqpoint{1.331122in}{0.358518in}}%
\pgfpathlineto{\pgfqpoint{1.346779in}{0.358518in}}%
\pgfpathlineto{\pgfqpoint{1.350100in}{0.358518in}}%
\pgfpathlineto{\pgfqpoint{1.347111in}{0.372129in}}%
\pgfpathlineto{\pgfqpoint{1.346779in}{0.372659in}}%
\pgfpathlineto{\pgfqpoint{1.337639in}{0.385740in}}%
\pgfpathlineto{\pgfqpoint{1.331122in}{0.392144in}}%
\pgfpathlineto{\pgfqpoint{1.321800in}{0.399351in}}%
\pgfpathlineto{\pgfqpoint{1.315466in}{0.403451in}}%
\pgfpathlineto{\pgfqpoint{1.299809in}{0.409350in}}%
\pgfpathlineto{\pgfqpoint{1.284152in}{0.409916in}}%
\pgfpathlineto{\pgfqpoint{1.268496in}{0.405004in}}%
\pgfpathlineto{\pgfqpoint{1.259049in}{0.399351in}}%
\pgfpathlineto{\pgfqpoint{1.252839in}{0.394895in}}%
\pgfpathlineto{\pgfqpoint{1.243205in}{0.385740in}}%
\pgfpathlineto{\pgfqpoint{1.237183in}{0.377388in}}%
\pgfpathlineto{\pgfqpoint{1.233864in}{0.372129in}}%
\pgfpathlineto{\pgfqpoint{1.230815in}{0.358518in}}%
\pgfpathlineto{\pgfqpoint{1.237183in}{0.358518in}}%
\pgfpathclose%
\pgfpathmoveto{\pgfqpoint{1.550314in}{0.358518in}}%
\pgfpathlineto{\pgfqpoint{1.565971in}{0.358518in}}%
\pgfpathlineto{\pgfqpoint{1.581627in}{0.358518in}}%
\pgfpathlineto{\pgfqpoint{1.597284in}{0.358518in}}%
\pgfpathlineto{\pgfqpoint{1.612940in}{0.358518in}}%
\pgfpathlineto{\pgfqpoint{1.628597in}{0.358518in}}%
\pgfpathlineto{\pgfqpoint{1.644253in}{0.358518in}}%
\pgfpathlineto{\pgfqpoint{1.659910in}{0.358518in}}%
\pgfpathlineto{\pgfqpoint{1.660220in}{0.358518in}}%
\pgfpathlineto{\pgfqpoint{1.659910in}{0.359932in}}%
\pgfpathlineto{\pgfqpoint{1.656930in}{0.372129in}}%
\pgfpathlineto{\pgfqpoint{1.647776in}{0.385740in}}%
\pgfpathlineto{\pgfqpoint{1.644253in}{0.389297in}}%
\pgfpathlineto{\pgfqpoint{1.632048in}{0.399351in}}%
\pgfpathlineto{\pgfqpoint{1.628597in}{0.401739in}}%
\pgfpathlineto{\pgfqpoint{1.612940in}{0.408567in}}%
\pgfpathlineto{\pgfqpoint{1.597284in}{0.410258in}}%
\pgfpathlineto{\pgfqpoint{1.581627in}{0.406383in}}%
\pgfpathlineto{\pgfqpoint{1.568744in}{0.399351in}}%
\pgfpathlineto{\pgfqpoint{1.565971in}{0.397528in}}%
\pgfpathlineto{\pgfqpoint{1.553051in}{0.385740in}}%
\pgfpathlineto{\pgfqpoint{1.550314in}{0.382086in}}%
\pgfpathlineto{\pgfqpoint{1.543963in}{0.372129in}}%
\pgfpathlineto{\pgfqpoint{1.540838in}{0.358518in}}%
\pgfpathlineto{\pgfqpoint{1.550314in}{0.358518in}}%
\pgfpathclose%
\pgfpathmoveto{\pgfqpoint{1.863445in}{0.358518in}}%
\pgfpathlineto{\pgfqpoint{1.879102in}{0.358518in}}%
\pgfpathlineto{\pgfqpoint{1.894758in}{0.358518in}}%
\pgfpathlineto{\pgfqpoint{1.910415in}{0.358518in}}%
\pgfpathlineto{\pgfqpoint{1.910415in}{0.372129in}}%
\pgfpathlineto{\pgfqpoint{1.910415in}{0.385740in}}%
\pgfpathlineto{\pgfqpoint{1.910415in}{0.399351in}}%
\pgfpathlineto{\pgfqpoint{1.910415in}{0.410372in}}%
\pgfpathlineto{\pgfqpoint{1.894758in}{0.407574in}}%
\pgfpathlineto{\pgfqpoint{1.879102in}{0.399885in}}%
\pgfpathlineto{\pgfqpoint{1.878373in}{0.399351in}}%
\pgfpathlineto{\pgfqpoint{1.863445in}{0.386373in}}%
\pgfpathlineto{\pgfqpoint{1.862831in}{0.385740in}}%
\pgfpathlineto{\pgfqpoint{1.853986in}{0.372129in}}%
\pgfpathlineto{\pgfqpoint{1.850768in}{0.358518in}}%
\pgfpathlineto{\pgfqpoint{1.863445in}{0.358518in}}%
\pgfpathclose%
\pgfpathmoveto{\pgfqpoint{0.362042in}{0.576295in}}%
\pgfpathlineto{\pgfqpoint{0.376072in}{0.578886in}}%
\pgfpathlineto{\pgfqpoint{0.391728in}{0.586844in}}%
\pgfpathlineto{\pgfqpoint{0.395819in}{0.589907in}}%
\pgfpathlineto{\pgfqpoint{0.407385in}{0.600517in}}%
\pgfpathlineto{\pgfqpoint{0.410132in}{0.603518in}}%
\pgfpathlineto{\pgfqpoint{0.417986in}{0.617129in}}%
\pgfpathlineto{\pgfqpoint{0.419930in}{0.630740in}}%
\pgfpathlineto{\pgfqpoint{0.415473in}{0.644351in}}%
\pgfpathlineto{\pgfqpoint{0.407385in}{0.655551in}}%
\pgfpathlineto{\pgfqpoint{0.405288in}{0.657962in}}%
\pgfpathlineto{\pgfqpoint{0.391728in}{0.669194in}}%
\pgfpathlineto{\pgfqpoint{0.387525in}{0.671573in}}%
\pgfpathlineto{\pgfqpoint{0.376072in}{0.677094in}}%
\pgfpathlineto{\pgfqpoint{0.360415in}{0.679811in}}%
\pgfpathlineto{\pgfqpoint{0.360415in}{0.671573in}}%
\pgfpathlineto{\pgfqpoint{0.360415in}{0.657962in}}%
\pgfpathlineto{\pgfqpoint{0.360415in}{0.644351in}}%
\pgfpathlineto{\pgfqpoint{0.360415in}{0.630740in}}%
\pgfpathlineto{\pgfqpoint{0.360415in}{0.617129in}}%
\pgfpathlineto{\pgfqpoint{0.360415in}{0.603518in}}%
\pgfpathlineto{\pgfqpoint{0.360415in}{0.589907in}}%
\pgfpathlineto{\pgfqpoint{0.360415in}{0.576295in}}%
\pgfpathlineto{\pgfqpoint{0.360415in}{0.576026in}}%
\pgfpathlineto{\pgfqpoint{0.362042in}{0.576295in}}%
\pgfpathclose%
\pgfpathmoveto{\pgfqpoint{0.673546in}{0.576131in}}%
\pgfpathlineto{\pgfqpoint{0.674260in}{0.576295in}}%
\pgfpathlineto{\pgfqpoint{0.689203in}{0.580119in}}%
\pgfpathlineto{\pgfqpoint{0.704859in}{0.588895in}}%
\pgfpathlineto{\pgfqpoint{0.706147in}{0.589907in}}%
\pgfpathlineto{\pgfqpoint{0.720312in}{0.603518in}}%
\pgfpathlineto{\pgfqpoint{0.720516in}{0.603827in}}%
\pgfpathlineto{\pgfqpoint{0.727976in}{0.617129in}}%
\pgfpathlineto{\pgfqpoint{0.729864in}{0.630740in}}%
\pgfpathlineto{\pgfqpoint{0.725535in}{0.644351in}}%
\pgfpathlineto{\pgfqpoint{0.720516in}{0.651413in}}%
\pgfpathlineto{\pgfqpoint{0.715074in}{0.657962in}}%
\pgfpathlineto{\pgfqpoint{0.704859in}{0.666842in}}%
\pgfpathlineto{\pgfqpoint{0.697326in}{0.671573in}}%
\pgfpathlineto{\pgfqpoint{0.689203in}{0.675937in}}%
\pgfpathlineto{\pgfqpoint{0.673546in}{0.679700in}}%
\pgfpathlineto{\pgfqpoint{0.657890in}{0.678058in}}%
\pgfpathlineto{\pgfqpoint{0.642589in}{0.671573in}}%
\pgfpathlineto{\pgfqpoint{0.642233in}{0.671395in}}%
\pgfpathlineto{\pgfqpoint{0.626577in}{0.659082in}}%
\pgfpathlineto{\pgfqpoint{0.625413in}{0.657962in}}%
\pgfpathlineto{\pgfqpoint{0.615319in}{0.644351in}}%
\pgfpathlineto{\pgfqpoint{0.610920in}{0.631361in}}%
\pgfpathlineto{\pgfqpoint{0.610731in}{0.630740in}}%
\pgfpathlineto{\pgfqpoint{0.610920in}{0.629301in}}%
\pgfpathlineto{\pgfqpoint{0.612718in}{0.617129in}}%
\pgfpathlineto{\pgfqpoint{0.620846in}{0.603518in}}%
\pgfpathlineto{\pgfqpoint{0.626577in}{0.597392in}}%
\pgfpathlineto{\pgfqpoint{0.635186in}{0.589907in}}%
\pgfpathlineto{\pgfqpoint{0.642233in}{0.584925in}}%
\pgfpathlineto{\pgfqpoint{0.657890in}{0.577859in}}%
\pgfpathlineto{\pgfqpoint{0.671891in}{0.576295in}}%
\pgfpathlineto{\pgfqpoint{0.673546in}{0.576131in}}%
\pgfpathclose%
\pgfpathmoveto{\pgfqpoint{1.597284in}{0.576131in}}%
\pgfpathlineto{\pgfqpoint{1.598939in}{0.576295in}}%
\pgfpathlineto{\pgfqpoint{1.612940in}{0.577859in}}%
\pgfpathlineto{\pgfqpoint{1.628597in}{0.584925in}}%
\pgfpathlineto{\pgfqpoint{1.635644in}{0.589907in}}%
\pgfpathlineto{\pgfqpoint{1.644253in}{0.597392in}}%
\pgfpathlineto{\pgfqpoint{1.649984in}{0.603518in}}%
\pgfpathlineto{\pgfqpoint{1.658112in}{0.617129in}}%
\pgfpathlineto{\pgfqpoint{1.659910in}{0.629301in}}%
\pgfpathlineto{\pgfqpoint{1.660099in}{0.630740in}}%
\pgfpathlineto{\pgfqpoint{1.659910in}{0.631361in}}%
\pgfpathlineto{\pgfqpoint{1.655511in}{0.644351in}}%
\pgfpathlineto{\pgfqpoint{1.645417in}{0.657962in}}%
\pgfpathlineto{\pgfqpoint{1.644253in}{0.659082in}}%
\pgfpathlineto{\pgfqpoint{1.628597in}{0.671395in}}%
\pgfpathlineto{\pgfqpoint{1.628241in}{0.671573in}}%
\pgfpathlineto{\pgfqpoint{1.612940in}{0.678058in}}%
\pgfpathlineto{\pgfqpoint{1.597284in}{0.679700in}}%
\pgfpathlineto{\pgfqpoint{1.581627in}{0.675937in}}%
\pgfpathlineto{\pgfqpoint{1.573504in}{0.671573in}}%
\pgfpathlineto{\pgfqpoint{1.565971in}{0.666842in}}%
\pgfpathlineto{\pgfqpoint{1.555756in}{0.657962in}}%
\pgfpathlineto{\pgfqpoint{1.550314in}{0.651413in}}%
\pgfpathlineto{\pgfqpoint{1.545295in}{0.644351in}}%
\pgfpathlineto{\pgfqpoint{1.540966in}{0.630740in}}%
\pgfpathlineto{\pgfqpoint{1.542854in}{0.617129in}}%
\pgfpathlineto{\pgfqpoint{1.550314in}{0.603827in}}%
\pgfpathlineto{\pgfqpoint{1.550518in}{0.603518in}}%
\pgfpathlineto{\pgfqpoint{1.564683in}{0.589907in}}%
\pgfpathlineto{\pgfqpoint{1.565971in}{0.588895in}}%
\pgfpathlineto{\pgfqpoint{1.581627in}{0.580119in}}%
\pgfpathlineto{\pgfqpoint{1.596570in}{0.576295in}}%
\pgfpathlineto{\pgfqpoint{1.597284in}{0.576131in}}%
\pgfpathclose%
\pgfpathmoveto{\pgfqpoint{1.910415in}{0.576026in}}%
\pgfpathlineto{\pgfqpoint{1.910415in}{0.576295in}}%
\pgfpathlineto{\pgfqpoint{1.910415in}{0.589907in}}%
\pgfpathlineto{\pgfqpoint{1.910415in}{0.603518in}}%
\pgfpathlineto{\pgfqpoint{1.910415in}{0.617129in}}%
\pgfpathlineto{\pgfqpoint{1.910415in}{0.630740in}}%
\pgfpathlineto{\pgfqpoint{1.910415in}{0.644351in}}%
\pgfpathlineto{\pgfqpoint{1.910415in}{0.657962in}}%
\pgfpathlineto{\pgfqpoint{1.910415in}{0.671573in}}%
\pgfpathlineto{\pgfqpoint{1.910415in}{0.679811in}}%
\pgfpathlineto{\pgfqpoint{1.894758in}{0.677094in}}%
\pgfpathlineto{\pgfqpoint{1.883305in}{0.671573in}}%
\pgfpathlineto{\pgfqpoint{1.879102in}{0.669194in}}%
\pgfpathlineto{\pgfqpoint{1.865542in}{0.657962in}}%
\pgfpathlineto{\pgfqpoint{1.863445in}{0.655551in}}%
\pgfpathlineto{\pgfqpoint{1.855357in}{0.644351in}}%
\pgfpathlineto{\pgfqpoint{1.850900in}{0.630740in}}%
\pgfpathlineto{\pgfqpoint{1.852844in}{0.617129in}}%
\pgfpathlineto{\pgfqpoint{1.860698in}{0.603518in}}%
\pgfpathlineto{\pgfqpoint{1.863445in}{0.600517in}}%
\pgfpathlineto{\pgfqpoint{1.875011in}{0.589907in}}%
\pgfpathlineto{\pgfqpoint{1.879102in}{0.586844in}}%
\pgfpathlineto{\pgfqpoint{1.894758in}{0.578886in}}%
\pgfpathlineto{\pgfqpoint{1.908788in}{0.576295in}}%
\pgfpathlineto{\pgfqpoint{1.910415in}{0.576026in}}%
\pgfpathclose%
\pgfpathmoveto{\pgfqpoint{0.955364in}{0.583154in}}%
\pgfpathlineto{\pgfqpoint{0.971021in}{0.577048in}}%
\pgfpathlineto{\pgfqpoint{0.986678in}{0.576463in}}%
\pgfpathlineto{\pgfqpoint{1.002334in}{0.581547in}}%
\pgfpathlineto{\pgfqpoint{1.016010in}{0.589907in}}%
\pgfpathlineto{\pgfqpoint{1.017991in}{0.591406in}}%
\pgfpathlineto{\pgfqpoint{1.030030in}{0.603518in}}%
\pgfpathlineto{\pgfqpoint{1.033647in}{0.609197in}}%
\pgfpathlineto{\pgfqpoint{1.038048in}{0.617129in}}%
\pgfpathlineto{\pgfqpoint{1.039890in}{0.630740in}}%
\pgfpathlineto{\pgfqpoint{1.035667in}{0.644351in}}%
\pgfpathlineto{\pgfqpoint{1.033647in}{0.647223in}}%
\pgfpathlineto{\pgfqpoint{1.025057in}{0.657962in}}%
\pgfpathlineto{\pgfqpoint{1.017991in}{0.664359in}}%
\pgfpathlineto{\pgfqpoint{1.007470in}{0.671573in}}%
\pgfpathlineto{\pgfqpoint{1.002334in}{0.674598in}}%
\pgfpathlineto{\pgfqpoint{0.986678in}{0.679368in}}%
\pgfpathlineto{\pgfqpoint{0.971021in}{0.678819in}}%
\pgfpathlineto{\pgfqpoint{0.955364in}{0.673090in}}%
\pgfpathlineto{\pgfqpoint{0.952983in}{0.671573in}}%
\pgfpathlineto{\pgfqpoint{0.939708in}{0.661766in}}%
\pgfpathlineto{\pgfqpoint{0.935645in}{0.657962in}}%
\pgfpathlineto{\pgfqpoint{0.925142in}{0.644351in}}%
\pgfpathlineto{\pgfqpoint{0.924051in}{0.641211in}}%
\pgfpathlineto{\pgfqpoint{0.920852in}{0.630740in}}%
\pgfpathlineto{\pgfqpoint{0.922658in}{0.617129in}}%
\pgfpathlineto{\pgfqpoint{0.924051in}{0.614603in}}%
\pgfpathlineto{\pgfqpoint{0.930893in}{0.603518in}}%
\pgfpathlineto{\pgfqpoint{0.939708in}{0.594347in}}%
\pgfpathlineto{\pgfqpoint{0.945151in}{0.589907in}}%
\pgfpathlineto{\pgfqpoint{0.955364in}{0.583154in}}%
\pgfpathclose%
\pgfpathmoveto{\pgfqpoint{1.268496in}{0.581547in}}%
\pgfpathlineto{\pgfqpoint{1.284152in}{0.576463in}}%
\pgfpathlineto{\pgfqpoint{1.299809in}{0.577048in}}%
\pgfpathlineto{\pgfqpoint{1.315466in}{0.583154in}}%
\pgfpathlineto{\pgfqpoint{1.325679in}{0.589907in}}%
\pgfpathlineto{\pgfqpoint{1.331122in}{0.594347in}}%
\pgfpathlineto{\pgfqpoint{1.339937in}{0.603518in}}%
\pgfpathlineto{\pgfqpoint{1.346779in}{0.614603in}}%
\pgfpathlineto{\pgfqpoint{1.348172in}{0.617129in}}%
\pgfpathlineto{\pgfqpoint{1.349978in}{0.630740in}}%
\pgfpathlineto{\pgfqpoint{1.346779in}{0.641211in}}%
\pgfpathlineto{\pgfqpoint{1.345688in}{0.644351in}}%
\pgfpathlineto{\pgfqpoint{1.335185in}{0.657962in}}%
\pgfpathlineto{\pgfqpoint{1.331122in}{0.661766in}}%
\pgfpathlineto{\pgfqpoint{1.317847in}{0.671573in}}%
\pgfpathlineto{\pgfqpoint{1.315466in}{0.673090in}}%
\pgfpathlineto{\pgfqpoint{1.299809in}{0.678819in}}%
\pgfpathlineto{\pgfqpoint{1.284152in}{0.679368in}}%
\pgfpathlineto{\pgfqpoint{1.268496in}{0.674598in}}%
\pgfpathlineto{\pgfqpoint{1.263360in}{0.671573in}}%
\pgfpathlineto{\pgfqpoint{1.252839in}{0.664359in}}%
\pgfpathlineto{\pgfqpoint{1.245773in}{0.657962in}}%
\pgfpathlineto{\pgfqpoint{1.237183in}{0.647223in}}%
\pgfpathlineto{\pgfqpoint{1.235163in}{0.644351in}}%
\pgfpathlineto{\pgfqpoint{1.230940in}{0.630740in}}%
\pgfpathlineto{\pgfqpoint{1.232782in}{0.617129in}}%
\pgfpathlineto{\pgfqpoint{1.237183in}{0.609197in}}%
\pgfpathlineto{\pgfqpoint{1.240800in}{0.603518in}}%
\pgfpathlineto{\pgfqpoint{1.252839in}{0.591406in}}%
\pgfpathlineto{\pgfqpoint{1.254820in}{0.589907in}}%
\pgfpathlineto{\pgfqpoint{1.268496in}{0.581547in}}%
\pgfpathclose%
\pgfpathmoveto{\pgfqpoint{0.376072in}{0.848228in}}%
\pgfpathlineto{\pgfqpoint{0.376682in}{0.848518in}}%
\pgfpathlineto{\pgfqpoint{0.391728in}{0.856463in}}%
\pgfpathlineto{\pgfqpoint{0.399095in}{0.862129in}}%
\pgfpathlineto{\pgfqpoint{0.407385in}{0.870233in}}%
\pgfpathlineto{\pgfqpoint{0.412101in}{0.875740in}}%
\pgfpathlineto{\pgfqpoint{0.418887in}{0.889351in}}%
\pgfpathlineto{\pgfqpoint{0.419537in}{0.902962in}}%
\pgfpathlineto{\pgfqpoint{0.413887in}{0.916573in}}%
\pgfpathlineto{\pgfqpoint{0.407385in}{0.924786in}}%
\pgfpathlineto{\pgfqpoint{0.402259in}{0.930184in}}%
\pgfpathlineto{\pgfqpoint{0.391728in}{0.938560in}}%
\pgfpathlineto{\pgfqpoint{0.382121in}{0.943795in}}%
\pgfpathlineto{\pgfqpoint{0.376072in}{0.946680in}}%
\pgfpathlineto{\pgfqpoint{0.360415in}{0.949331in}}%
\pgfpathlineto{\pgfqpoint{0.360415in}{0.943795in}}%
\pgfpathlineto{\pgfqpoint{0.360415in}{0.930184in}}%
\pgfpathlineto{\pgfqpoint{0.360415in}{0.916573in}}%
\pgfpathlineto{\pgfqpoint{0.360415in}{0.902962in}}%
\pgfpathlineto{\pgfqpoint{0.360415in}{0.889351in}}%
\pgfpathlineto{\pgfqpoint{0.360415in}{0.875740in}}%
\pgfpathlineto{\pgfqpoint{0.360415in}{0.862129in}}%
\pgfpathlineto{\pgfqpoint{0.360415in}{0.848518in}}%
\pgfpathlineto{\pgfqpoint{0.360415in}{0.845630in}}%
\pgfpathlineto{\pgfqpoint{0.376072in}{0.848228in}}%
\pgfpathclose%
\pgfpathmoveto{\pgfqpoint{0.657890in}{0.847307in}}%
\pgfpathlineto{\pgfqpoint{0.673546in}{0.845737in}}%
\pgfpathlineto{\pgfqpoint{0.685591in}{0.848518in}}%
\pgfpathlineto{\pgfqpoint{0.689203in}{0.849466in}}%
\pgfpathlineto{\pgfqpoint{0.704859in}{0.858597in}}%
\pgfpathlineto{\pgfqpoint{0.709235in}{0.862129in}}%
\pgfpathlineto{\pgfqpoint{0.720516in}{0.873670in}}%
\pgfpathlineto{\pgfqpoint{0.722261in}{0.875740in}}%
\pgfpathlineto{\pgfqpoint{0.728851in}{0.889351in}}%
\pgfpathlineto{\pgfqpoint{0.729482in}{0.902962in}}%
\pgfpathlineto{\pgfqpoint{0.723995in}{0.916573in}}%
\pgfpathlineto{\pgfqpoint{0.720516in}{0.921038in}}%
\pgfpathlineto{\pgfqpoint{0.712218in}{0.930184in}}%
\pgfpathlineto{\pgfqpoint{0.704859in}{0.936327in}}%
\pgfpathlineto{\pgfqpoint{0.692506in}{0.943795in}}%
\pgfpathlineto{\pgfqpoint{0.689203in}{0.945551in}}%
\pgfpathlineto{\pgfqpoint{0.673546in}{0.949223in}}%
\pgfpathlineto{\pgfqpoint{0.657890in}{0.947621in}}%
\pgfpathlineto{\pgfqpoint{0.648766in}{0.943795in}}%
\pgfpathlineto{\pgfqpoint{0.642233in}{0.940650in}}%
\pgfpathlineto{\pgfqpoint{0.628302in}{0.930184in}}%
\pgfpathlineto{\pgfqpoint{0.626577in}{0.928463in}}%
\pgfpathlineto{\pgfqpoint{0.616960in}{0.916573in}}%
\pgfpathlineto{\pgfqpoint{0.611113in}{0.902962in}}%
\pgfpathlineto{\pgfqpoint{0.611786in}{0.889351in}}%
\pgfpathlineto{\pgfqpoint{0.618809in}{0.875740in}}%
\pgfpathlineto{\pgfqpoint{0.626577in}{0.866860in}}%
\pgfpathlineto{\pgfqpoint{0.631685in}{0.862129in}}%
\pgfpathlineto{\pgfqpoint{0.642233in}{0.854466in}}%
\pgfpathlineto{\pgfqpoint{0.654984in}{0.848518in}}%
\pgfpathlineto{\pgfqpoint{0.657890in}{0.847307in}}%
\pgfpathclose%
\pgfpathmoveto{\pgfqpoint{0.971021in}{0.846579in}}%
\pgfpathlineto{\pgfqpoint{0.986678in}{0.846054in}}%
\pgfpathlineto{\pgfqpoint{0.995035in}{0.848518in}}%
\pgfpathlineto{\pgfqpoint{1.002334in}{0.850951in}}%
\pgfpathlineto{\pgfqpoint{1.017991in}{0.860849in}}%
\pgfpathlineto{\pgfqpoint{1.019513in}{0.862129in}}%
\pgfpathlineto{\pgfqpoint{1.032248in}{0.875740in}}%
\pgfpathlineto{\pgfqpoint{1.033647in}{0.878292in}}%
\pgfpathlineto{\pgfqpoint{1.038901in}{0.889351in}}%
\pgfpathlineto{\pgfqpoint{1.039518in}{0.902962in}}%
\pgfpathlineto{\pgfqpoint{1.034163in}{0.916573in}}%
\pgfpathlineto{\pgfqpoint{1.033647in}{0.917243in}}%
\pgfpathlineto{\pgfqpoint{1.022345in}{0.930184in}}%
\pgfpathlineto{\pgfqpoint{1.017991in}{0.933970in}}%
\pgfpathlineto{\pgfqpoint{1.003104in}{0.943795in}}%
\pgfpathlineto{\pgfqpoint{1.002334in}{0.944244in}}%
\pgfpathlineto{\pgfqpoint{0.986678in}{0.948899in}}%
\pgfpathlineto{\pgfqpoint{0.971021in}{0.948363in}}%
\pgfpathlineto{\pgfqpoint{0.958300in}{0.943795in}}%
\pgfpathlineto{\pgfqpoint{0.955364in}{0.942579in}}%
\pgfpathlineto{\pgfqpoint{0.939708in}{0.931508in}}%
\pgfpathlineto{\pgfqpoint{0.938236in}{0.930184in}}%
\pgfpathlineto{\pgfqpoint{0.926850in}{0.916573in}}%
\pgfpathlineto{\pgfqpoint{0.924051in}{0.910228in}}%
\pgfpathlineto{\pgfqpoint{0.921218in}{0.902962in}}%
\pgfpathlineto{\pgfqpoint{0.921822in}{0.889351in}}%
\pgfpathlineto{\pgfqpoint{0.924051in}{0.884629in}}%
\pgfpathlineto{\pgfqpoint{0.928773in}{0.875740in}}%
\pgfpathlineto{\pgfqpoint{0.939708in}{0.863576in}}%
\pgfpathlineto{\pgfqpoint{0.941373in}{0.862129in}}%
\pgfpathlineto{\pgfqpoint{0.955364in}{0.852623in}}%
\pgfpathlineto{\pgfqpoint{0.965590in}{0.848518in}}%
\pgfpathlineto{\pgfqpoint{0.971021in}{0.846579in}}%
\pgfpathclose%
\pgfpathmoveto{\pgfqpoint{1.284152in}{0.846054in}}%
\pgfpathlineto{\pgfqpoint{1.299809in}{0.846579in}}%
\pgfpathlineto{\pgfqpoint{1.305240in}{0.848518in}}%
\pgfpathlineto{\pgfqpoint{1.315466in}{0.852623in}}%
\pgfpathlineto{\pgfqpoint{1.329457in}{0.862129in}}%
\pgfpathlineto{\pgfqpoint{1.331122in}{0.863576in}}%
\pgfpathlineto{\pgfqpoint{1.342057in}{0.875740in}}%
\pgfpathlineto{\pgfqpoint{1.346779in}{0.884629in}}%
\pgfpathlineto{\pgfqpoint{1.349008in}{0.889351in}}%
\pgfpathlineto{\pgfqpoint{1.349612in}{0.902962in}}%
\pgfpathlineto{\pgfqpoint{1.346779in}{0.910228in}}%
\pgfpathlineto{\pgfqpoint{1.343980in}{0.916573in}}%
\pgfpathlineto{\pgfqpoint{1.332594in}{0.930184in}}%
\pgfpathlineto{\pgfqpoint{1.331122in}{0.931508in}}%
\pgfpathlineto{\pgfqpoint{1.315466in}{0.942579in}}%
\pgfpathlineto{\pgfqpoint{1.312530in}{0.943795in}}%
\pgfpathlineto{\pgfqpoint{1.299809in}{0.948363in}}%
\pgfpathlineto{\pgfqpoint{1.284152in}{0.948899in}}%
\pgfpathlineto{\pgfqpoint{1.268496in}{0.944244in}}%
\pgfpathlineto{\pgfqpoint{1.267726in}{0.943795in}}%
\pgfpathlineto{\pgfqpoint{1.252839in}{0.933970in}}%
\pgfpathlineto{\pgfqpoint{1.248485in}{0.930184in}}%
\pgfpathlineto{\pgfqpoint{1.237183in}{0.917243in}}%
\pgfpathlineto{\pgfqpoint{1.236667in}{0.916573in}}%
\pgfpathlineto{\pgfqpoint{1.231312in}{0.902962in}}%
\pgfpathlineto{\pgfqpoint{1.231929in}{0.889351in}}%
\pgfpathlineto{\pgfqpoint{1.237183in}{0.878292in}}%
\pgfpathlineto{\pgfqpoint{1.238582in}{0.875740in}}%
\pgfpathlineto{\pgfqpoint{1.251317in}{0.862129in}}%
\pgfpathlineto{\pgfqpoint{1.252839in}{0.860849in}}%
\pgfpathlineto{\pgfqpoint{1.268496in}{0.850951in}}%
\pgfpathlineto{\pgfqpoint{1.275795in}{0.848518in}}%
\pgfpathlineto{\pgfqpoint{1.284152in}{0.846054in}}%
\pgfpathclose%
\pgfpathmoveto{\pgfqpoint{1.597284in}{0.845737in}}%
\pgfpathlineto{\pgfqpoint{1.612940in}{0.847307in}}%
\pgfpathlineto{\pgfqpoint{1.615846in}{0.848518in}}%
\pgfpathlineto{\pgfqpoint{1.628597in}{0.854466in}}%
\pgfpathlineto{\pgfqpoint{1.639145in}{0.862129in}}%
\pgfpathlineto{\pgfqpoint{1.644253in}{0.866860in}}%
\pgfpathlineto{\pgfqpoint{1.652021in}{0.875740in}}%
\pgfpathlineto{\pgfqpoint{1.659044in}{0.889351in}}%
\pgfpathlineto{\pgfqpoint{1.659717in}{0.902962in}}%
\pgfpathlineto{\pgfqpoint{1.653870in}{0.916573in}}%
\pgfpathlineto{\pgfqpoint{1.644253in}{0.928463in}}%
\pgfpathlineto{\pgfqpoint{1.642528in}{0.930184in}}%
\pgfpathlineto{\pgfqpoint{1.628597in}{0.940650in}}%
\pgfpathlineto{\pgfqpoint{1.622064in}{0.943795in}}%
\pgfpathlineto{\pgfqpoint{1.612940in}{0.947621in}}%
\pgfpathlineto{\pgfqpoint{1.597284in}{0.949223in}}%
\pgfpathlineto{\pgfqpoint{1.581627in}{0.945551in}}%
\pgfpathlineto{\pgfqpoint{1.578324in}{0.943795in}}%
\pgfpathlineto{\pgfqpoint{1.565971in}{0.936327in}}%
\pgfpathlineto{\pgfqpoint{1.558612in}{0.930184in}}%
\pgfpathlineto{\pgfqpoint{1.550314in}{0.921038in}}%
\pgfpathlineto{\pgfqpoint{1.546835in}{0.916573in}}%
\pgfpathlineto{\pgfqpoint{1.541348in}{0.902962in}}%
\pgfpathlineto{\pgfqpoint{1.541979in}{0.889351in}}%
\pgfpathlineto{\pgfqpoint{1.548569in}{0.875740in}}%
\pgfpathlineto{\pgfqpoint{1.550314in}{0.873670in}}%
\pgfpathlineto{\pgfqpoint{1.561595in}{0.862129in}}%
\pgfpathlineto{\pgfqpoint{1.565971in}{0.858597in}}%
\pgfpathlineto{\pgfqpoint{1.581627in}{0.849466in}}%
\pgfpathlineto{\pgfqpoint{1.585239in}{0.848518in}}%
\pgfpathlineto{\pgfqpoint{1.597284in}{0.845737in}}%
\pgfpathclose%
\pgfpathmoveto{\pgfqpoint{1.894758in}{0.848228in}}%
\pgfpathlineto{\pgfqpoint{1.910415in}{0.845630in}}%
\pgfpathlineto{\pgfqpoint{1.910415in}{0.848518in}}%
\pgfpathlineto{\pgfqpoint{1.910415in}{0.862129in}}%
\pgfpathlineto{\pgfqpoint{1.910415in}{0.875740in}}%
\pgfpathlineto{\pgfqpoint{1.910415in}{0.889351in}}%
\pgfpathlineto{\pgfqpoint{1.910415in}{0.902962in}}%
\pgfpathlineto{\pgfqpoint{1.910415in}{0.916573in}}%
\pgfpathlineto{\pgfqpoint{1.910415in}{0.930184in}}%
\pgfpathlineto{\pgfqpoint{1.910415in}{0.943795in}}%
\pgfpathlineto{\pgfqpoint{1.910415in}{0.949331in}}%
\pgfpathlineto{\pgfqpoint{1.894758in}{0.946680in}}%
\pgfpathlineto{\pgfqpoint{1.888709in}{0.943795in}}%
\pgfpathlineto{\pgfqpoint{1.879102in}{0.938560in}}%
\pgfpathlineto{\pgfqpoint{1.868571in}{0.930184in}}%
\pgfpathlineto{\pgfqpoint{1.863445in}{0.924786in}}%
\pgfpathlineto{\pgfqpoint{1.856943in}{0.916573in}}%
\pgfpathlineto{\pgfqpoint{1.851293in}{0.902962in}}%
\pgfpathlineto{\pgfqpoint{1.851943in}{0.889351in}}%
\pgfpathlineto{\pgfqpoint{1.858729in}{0.875740in}}%
\pgfpathlineto{\pgfqpoint{1.863445in}{0.870233in}}%
\pgfpathlineto{\pgfqpoint{1.871735in}{0.862129in}}%
\pgfpathlineto{\pgfqpoint{1.879102in}{0.856463in}}%
\pgfpathlineto{\pgfqpoint{1.894148in}{0.848518in}}%
\pgfpathlineto{\pgfqpoint{1.894758in}{0.848228in}}%
\pgfpathclose%
\pgfpathmoveto{\pgfqpoint{0.376072in}{1.117855in}}%
\pgfpathlineto{\pgfqpoint{0.382121in}{1.120740in}}%
\pgfpathlineto{\pgfqpoint{0.391728in}{1.125975in}}%
\pgfpathlineto{\pgfqpoint{0.402259in}{1.134351in}}%
\pgfpathlineto{\pgfqpoint{0.407385in}{1.139750in}}%
\pgfpathlineto{\pgfqpoint{0.413887in}{1.147962in}}%
\pgfpathlineto{\pgfqpoint{0.419537in}{1.161573in}}%
\pgfpathlineto{\pgfqpoint{0.418887in}{1.175184in}}%
\pgfpathlineto{\pgfqpoint{0.412101in}{1.188795in}}%
\pgfpathlineto{\pgfqpoint{0.407385in}{1.194303in}}%
\pgfpathlineto{\pgfqpoint{0.399095in}{1.202407in}}%
\pgfpathlineto{\pgfqpoint{0.391728in}{1.208072in}}%
\pgfpathlineto{\pgfqpoint{0.376682in}{1.216018in}}%
\pgfpathlineto{\pgfqpoint{0.376072in}{1.216307in}}%
\pgfpathlineto{\pgfqpoint{0.360415in}{1.218905in}}%
\pgfpathlineto{\pgfqpoint{0.360415in}{1.216018in}}%
\pgfpathlineto{\pgfqpoint{0.360415in}{1.202407in}}%
\pgfpathlineto{\pgfqpoint{0.360415in}{1.188795in}}%
\pgfpathlineto{\pgfqpoint{0.360415in}{1.175184in}}%
\pgfpathlineto{\pgfqpoint{0.360415in}{1.161573in}}%
\pgfpathlineto{\pgfqpoint{0.360415in}{1.147962in}}%
\pgfpathlineto{\pgfqpoint{0.360415in}{1.134351in}}%
\pgfpathlineto{\pgfqpoint{0.360415in}{1.120740in}}%
\pgfpathlineto{\pgfqpoint{0.360415in}{1.115204in}}%
\pgfpathlineto{\pgfqpoint{0.376072in}{1.117855in}}%
\pgfpathclose%
\pgfpathmoveto{\pgfqpoint{0.657890in}{1.116915in}}%
\pgfpathlineto{\pgfqpoint{0.673546in}{1.115313in}}%
\pgfpathlineto{\pgfqpoint{0.689203in}{1.118984in}}%
\pgfpathlineto{\pgfqpoint{0.692506in}{1.120740in}}%
\pgfpathlineto{\pgfqpoint{0.704859in}{1.128208in}}%
\pgfpathlineto{\pgfqpoint{0.712218in}{1.134351in}}%
\pgfpathlineto{\pgfqpoint{0.720516in}{1.143497in}}%
\pgfpathlineto{\pgfqpoint{0.723995in}{1.147962in}}%
\pgfpathlineto{\pgfqpoint{0.729482in}{1.161573in}}%
\pgfpathlineto{\pgfqpoint{0.728851in}{1.175184in}}%
\pgfpathlineto{\pgfqpoint{0.722261in}{1.188795in}}%
\pgfpathlineto{\pgfqpoint{0.720516in}{1.190865in}}%
\pgfpathlineto{\pgfqpoint{0.709235in}{1.202407in}}%
\pgfpathlineto{\pgfqpoint{0.704859in}{1.205939in}}%
\pgfpathlineto{\pgfqpoint{0.689203in}{1.215069in}}%
\pgfpathlineto{\pgfqpoint{0.685591in}{1.216018in}}%
\pgfpathlineto{\pgfqpoint{0.673546in}{1.218799in}}%
\pgfpathlineto{\pgfqpoint{0.657890in}{1.217229in}}%
\pgfpathlineto{\pgfqpoint{0.654984in}{1.216018in}}%
\pgfpathlineto{\pgfqpoint{0.642233in}{1.210069in}}%
\pgfpathlineto{\pgfqpoint{0.631685in}{1.202407in}}%
\pgfpathlineto{\pgfqpoint{0.626577in}{1.197675in}}%
\pgfpathlineto{\pgfqpoint{0.618809in}{1.188795in}}%
\pgfpathlineto{\pgfqpoint{0.611786in}{1.175184in}}%
\pgfpathlineto{\pgfqpoint{0.611113in}{1.161573in}}%
\pgfpathlineto{\pgfqpoint{0.616960in}{1.147962in}}%
\pgfpathlineto{\pgfqpoint{0.626577in}{1.136073in}}%
\pgfpathlineto{\pgfqpoint{0.628302in}{1.134351in}}%
\pgfpathlineto{\pgfqpoint{0.642233in}{1.123885in}}%
\pgfpathlineto{\pgfqpoint{0.648766in}{1.120740in}}%
\pgfpathlineto{\pgfqpoint{0.657890in}{1.116915in}}%
\pgfpathclose%
\pgfpathmoveto{\pgfqpoint{0.971021in}{1.116172in}}%
\pgfpathlineto{\pgfqpoint{0.986678in}{1.115637in}}%
\pgfpathlineto{\pgfqpoint{1.002334in}{1.120291in}}%
\pgfpathlineto{\pgfqpoint{1.003104in}{1.120740in}}%
\pgfpathlineto{\pgfqpoint{1.017991in}{1.130565in}}%
\pgfpathlineto{\pgfqpoint{1.022345in}{1.134351in}}%
\pgfpathlineto{\pgfqpoint{1.033647in}{1.147293in}}%
\pgfpathlineto{\pgfqpoint{1.034163in}{1.147962in}}%
\pgfpathlineto{\pgfqpoint{1.039518in}{1.161573in}}%
\pgfpathlineto{\pgfqpoint{1.038901in}{1.175184in}}%
\pgfpathlineto{\pgfqpoint{1.033647in}{1.186244in}}%
\pgfpathlineto{\pgfqpoint{1.032248in}{1.188795in}}%
\pgfpathlineto{\pgfqpoint{1.019513in}{1.202407in}}%
\pgfpathlineto{\pgfqpoint{1.017991in}{1.203686in}}%
\pgfpathlineto{\pgfqpoint{1.002334in}{1.213584in}}%
\pgfpathlineto{\pgfqpoint{0.995035in}{1.216018in}}%
\pgfpathlineto{\pgfqpoint{0.986678in}{1.218481in}}%
\pgfpathlineto{\pgfqpoint{0.971021in}{1.217956in}}%
\pgfpathlineto{\pgfqpoint{0.965590in}{1.216018in}}%
\pgfpathlineto{\pgfqpoint{0.955364in}{1.211912in}}%
\pgfpathlineto{\pgfqpoint{0.941373in}{1.202407in}}%
\pgfpathlineto{\pgfqpoint{0.939708in}{1.200959in}}%
\pgfpathlineto{\pgfqpoint{0.928773in}{1.188795in}}%
\pgfpathlineto{\pgfqpoint{0.924051in}{1.179906in}}%
\pgfpathlineto{\pgfqpoint{0.921822in}{1.175184in}}%
\pgfpathlineto{\pgfqpoint{0.921218in}{1.161573in}}%
\pgfpathlineto{\pgfqpoint{0.924051in}{1.154308in}}%
\pgfpathlineto{\pgfqpoint{0.926850in}{1.147962in}}%
\pgfpathlineto{\pgfqpoint{0.938236in}{1.134351in}}%
\pgfpathlineto{\pgfqpoint{0.939708in}{1.133028in}}%
\pgfpathlineto{\pgfqpoint{0.955364in}{1.121956in}}%
\pgfpathlineto{\pgfqpoint{0.958300in}{1.120740in}}%
\pgfpathlineto{\pgfqpoint{0.971021in}{1.116172in}}%
\pgfpathclose%
\pgfpathmoveto{\pgfqpoint{1.268496in}{1.120291in}}%
\pgfpathlineto{\pgfqpoint{1.284152in}{1.115637in}}%
\pgfpathlineto{\pgfqpoint{1.299809in}{1.116172in}}%
\pgfpathlineto{\pgfqpoint{1.312530in}{1.120740in}}%
\pgfpathlineto{\pgfqpoint{1.315466in}{1.121956in}}%
\pgfpathlineto{\pgfqpoint{1.331122in}{1.133028in}}%
\pgfpathlineto{\pgfqpoint{1.332594in}{1.134351in}}%
\pgfpathlineto{\pgfqpoint{1.343980in}{1.147962in}}%
\pgfpathlineto{\pgfqpoint{1.346779in}{1.154308in}}%
\pgfpathlineto{\pgfqpoint{1.349612in}{1.161573in}}%
\pgfpathlineto{\pgfqpoint{1.349008in}{1.175184in}}%
\pgfpathlineto{\pgfqpoint{1.346779in}{1.179906in}}%
\pgfpathlineto{\pgfqpoint{1.342057in}{1.188795in}}%
\pgfpathlineto{\pgfqpoint{1.331122in}{1.200959in}}%
\pgfpathlineto{\pgfqpoint{1.329457in}{1.202407in}}%
\pgfpathlineto{\pgfqpoint{1.315466in}{1.211912in}}%
\pgfpathlineto{\pgfqpoint{1.305240in}{1.216018in}}%
\pgfpathlineto{\pgfqpoint{1.299809in}{1.217956in}}%
\pgfpathlineto{\pgfqpoint{1.284152in}{1.218481in}}%
\pgfpathlineto{\pgfqpoint{1.275795in}{1.216018in}}%
\pgfpathlineto{\pgfqpoint{1.268496in}{1.213584in}}%
\pgfpathlineto{\pgfqpoint{1.252839in}{1.203686in}}%
\pgfpathlineto{\pgfqpoint{1.251317in}{1.202407in}}%
\pgfpathlineto{\pgfqpoint{1.238582in}{1.188795in}}%
\pgfpathlineto{\pgfqpoint{1.237183in}{1.186244in}}%
\pgfpathlineto{\pgfqpoint{1.231929in}{1.175184in}}%
\pgfpathlineto{\pgfqpoint{1.231312in}{1.161573in}}%
\pgfpathlineto{\pgfqpoint{1.236667in}{1.147962in}}%
\pgfpathlineto{\pgfqpoint{1.237183in}{1.147293in}}%
\pgfpathlineto{\pgfqpoint{1.248485in}{1.134351in}}%
\pgfpathlineto{\pgfqpoint{1.252839in}{1.130565in}}%
\pgfpathlineto{\pgfqpoint{1.267726in}{1.120740in}}%
\pgfpathlineto{\pgfqpoint{1.268496in}{1.120291in}}%
\pgfpathclose%
\pgfpathmoveto{\pgfqpoint{1.581627in}{1.118984in}}%
\pgfpathlineto{\pgfqpoint{1.597284in}{1.115313in}}%
\pgfpathlineto{\pgfqpoint{1.612940in}{1.116915in}}%
\pgfpathlineto{\pgfqpoint{1.622064in}{1.120740in}}%
\pgfpathlineto{\pgfqpoint{1.628597in}{1.123885in}}%
\pgfpathlineto{\pgfqpoint{1.642528in}{1.134351in}}%
\pgfpathlineto{\pgfqpoint{1.644253in}{1.136073in}}%
\pgfpathlineto{\pgfqpoint{1.653870in}{1.147962in}}%
\pgfpathlineto{\pgfqpoint{1.659717in}{1.161573in}}%
\pgfpathlineto{\pgfqpoint{1.659044in}{1.175184in}}%
\pgfpathlineto{\pgfqpoint{1.652021in}{1.188795in}}%
\pgfpathlineto{\pgfqpoint{1.644253in}{1.197675in}}%
\pgfpathlineto{\pgfqpoint{1.639145in}{1.202407in}}%
\pgfpathlineto{\pgfqpoint{1.628597in}{1.210069in}}%
\pgfpathlineto{\pgfqpoint{1.615846in}{1.216018in}}%
\pgfpathlineto{\pgfqpoint{1.612940in}{1.217229in}}%
\pgfpathlineto{\pgfqpoint{1.597284in}{1.218799in}}%
\pgfpathlineto{\pgfqpoint{1.585239in}{1.216018in}}%
\pgfpathlineto{\pgfqpoint{1.581627in}{1.215069in}}%
\pgfpathlineto{\pgfqpoint{1.565971in}{1.205939in}}%
\pgfpathlineto{\pgfqpoint{1.561595in}{1.202407in}}%
\pgfpathlineto{\pgfqpoint{1.550314in}{1.190865in}}%
\pgfpathlineto{\pgfqpoint{1.548569in}{1.188795in}}%
\pgfpathlineto{\pgfqpoint{1.541979in}{1.175184in}}%
\pgfpathlineto{\pgfqpoint{1.541348in}{1.161573in}}%
\pgfpathlineto{\pgfqpoint{1.546835in}{1.147962in}}%
\pgfpathlineto{\pgfqpoint{1.550314in}{1.143497in}}%
\pgfpathlineto{\pgfqpoint{1.558612in}{1.134351in}}%
\pgfpathlineto{\pgfqpoint{1.565971in}{1.128208in}}%
\pgfpathlineto{\pgfqpoint{1.578324in}{1.120740in}}%
\pgfpathlineto{\pgfqpoint{1.581627in}{1.118984in}}%
\pgfpathclose%
\pgfpathmoveto{\pgfqpoint{1.894758in}{1.117855in}}%
\pgfpathlineto{\pgfqpoint{1.910415in}{1.115204in}}%
\pgfpathlineto{\pgfqpoint{1.910415in}{1.120740in}}%
\pgfpathlineto{\pgfqpoint{1.910415in}{1.134351in}}%
\pgfpathlineto{\pgfqpoint{1.910415in}{1.147962in}}%
\pgfpathlineto{\pgfqpoint{1.910415in}{1.161573in}}%
\pgfpathlineto{\pgfqpoint{1.910415in}{1.175184in}}%
\pgfpathlineto{\pgfqpoint{1.910415in}{1.188795in}}%
\pgfpathlineto{\pgfqpoint{1.910415in}{1.202407in}}%
\pgfpathlineto{\pgfqpoint{1.910415in}{1.216018in}}%
\pgfpathlineto{\pgfqpoint{1.910415in}{1.218905in}}%
\pgfpathlineto{\pgfqpoint{1.894758in}{1.216307in}}%
\pgfpathlineto{\pgfqpoint{1.894148in}{1.216018in}}%
\pgfpathlineto{\pgfqpoint{1.879102in}{1.208072in}}%
\pgfpathlineto{\pgfqpoint{1.871735in}{1.202407in}}%
\pgfpathlineto{\pgfqpoint{1.863445in}{1.194303in}}%
\pgfpathlineto{\pgfqpoint{1.858729in}{1.188795in}}%
\pgfpathlineto{\pgfqpoint{1.851943in}{1.175184in}}%
\pgfpathlineto{\pgfqpoint{1.851293in}{1.161573in}}%
\pgfpathlineto{\pgfqpoint{1.856943in}{1.147962in}}%
\pgfpathlineto{\pgfqpoint{1.863445in}{1.139750in}}%
\pgfpathlineto{\pgfqpoint{1.868571in}{1.134351in}}%
\pgfpathlineto{\pgfqpoint{1.879102in}{1.125975in}}%
\pgfpathlineto{\pgfqpoint{1.888709in}{1.120740in}}%
\pgfpathlineto{\pgfqpoint{1.894758in}{1.117855in}}%
\pgfpathclose%
\pgfpathmoveto{\pgfqpoint{0.376072in}{1.387441in}}%
\pgfpathlineto{\pgfqpoint{0.387525in}{1.392962in}}%
\pgfpathlineto{\pgfqpoint{0.391728in}{1.395341in}}%
\pgfpathlineto{\pgfqpoint{0.405288in}{1.406573in}}%
\pgfpathlineto{\pgfqpoint{0.407385in}{1.408984in}}%
\pgfpathlineto{\pgfqpoint{0.415473in}{1.420184in}}%
\pgfpathlineto{\pgfqpoint{0.419930in}{1.433795in}}%
\pgfpathlineto{\pgfqpoint{0.417986in}{1.447407in}}%
\pgfpathlineto{\pgfqpoint{0.410132in}{1.461018in}}%
\pgfpathlineto{\pgfqpoint{0.407385in}{1.464018in}}%
\pgfpathlineto{\pgfqpoint{0.395819in}{1.474629in}}%
\pgfpathlineto{\pgfqpoint{0.391728in}{1.477691in}}%
\pgfpathlineto{\pgfqpoint{0.376072in}{1.485649in}}%
\pgfpathlineto{\pgfqpoint{0.362042in}{1.488240in}}%
\pgfpathlineto{\pgfqpoint{0.360415in}{1.488509in}}%
\pgfpathlineto{\pgfqpoint{0.360415in}{1.488240in}}%
\pgfpathlineto{\pgfqpoint{0.360415in}{1.474629in}}%
\pgfpathlineto{\pgfqpoint{0.360415in}{1.461018in}}%
\pgfpathlineto{\pgfqpoint{0.360415in}{1.447407in}}%
\pgfpathlineto{\pgfqpoint{0.360415in}{1.433795in}}%
\pgfpathlineto{\pgfqpoint{0.360415in}{1.420184in}}%
\pgfpathlineto{\pgfqpoint{0.360415in}{1.406573in}}%
\pgfpathlineto{\pgfqpoint{0.360415in}{1.392962in}}%
\pgfpathlineto{\pgfqpoint{0.360415in}{1.384724in}}%
\pgfpathlineto{\pgfqpoint{0.376072in}{1.387441in}}%
\pgfpathclose%
\pgfpathmoveto{\pgfqpoint{0.657890in}{1.386477in}}%
\pgfpathlineto{\pgfqpoint{0.673546in}{1.384835in}}%
\pgfpathlineto{\pgfqpoint{0.689203in}{1.388598in}}%
\pgfpathlineto{\pgfqpoint{0.697326in}{1.392962in}}%
\pgfpathlineto{\pgfqpoint{0.704859in}{1.397693in}}%
\pgfpathlineto{\pgfqpoint{0.715074in}{1.406573in}}%
\pgfpathlineto{\pgfqpoint{0.720516in}{1.413122in}}%
\pgfpathlineto{\pgfqpoint{0.725535in}{1.420184in}}%
\pgfpathlineto{\pgfqpoint{0.729864in}{1.433795in}}%
\pgfpathlineto{\pgfqpoint{0.727976in}{1.447407in}}%
\pgfpathlineto{\pgfqpoint{0.720516in}{1.460709in}}%
\pgfpathlineto{\pgfqpoint{0.720312in}{1.461018in}}%
\pgfpathlineto{\pgfqpoint{0.706147in}{1.474629in}}%
\pgfpathlineto{\pgfqpoint{0.704859in}{1.475640in}}%
\pgfpathlineto{\pgfqpoint{0.689203in}{1.484416in}}%
\pgfpathlineto{\pgfqpoint{0.674260in}{1.488240in}}%
\pgfpathlineto{\pgfqpoint{0.673546in}{1.488404in}}%
\pgfpathlineto{\pgfqpoint{0.671891in}{1.488240in}}%
\pgfpathlineto{\pgfqpoint{0.657890in}{1.486677in}}%
\pgfpathlineto{\pgfqpoint{0.642233in}{1.479610in}}%
\pgfpathlineto{\pgfqpoint{0.635186in}{1.474629in}}%
\pgfpathlineto{\pgfqpoint{0.626577in}{1.467144in}}%
\pgfpathlineto{\pgfqpoint{0.620846in}{1.461018in}}%
\pgfpathlineto{\pgfqpoint{0.612718in}{1.447407in}}%
\pgfpathlineto{\pgfqpoint{0.610920in}{1.435234in}}%
\pgfpathlineto{\pgfqpoint{0.610731in}{1.433795in}}%
\pgfpathlineto{\pgfqpoint{0.610920in}{1.433175in}}%
\pgfpathlineto{\pgfqpoint{0.615319in}{1.420184in}}%
\pgfpathlineto{\pgfqpoint{0.625413in}{1.406573in}}%
\pgfpathlineto{\pgfqpoint{0.626577in}{1.405454in}}%
\pgfpathlineto{\pgfqpoint{0.642233in}{1.393140in}}%
\pgfpathlineto{\pgfqpoint{0.642589in}{1.392962in}}%
\pgfpathlineto{\pgfqpoint{0.657890in}{1.386477in}}%
\pgfpathclose%
\pgfpathmoveto{\pgfqpoint{0.955364in}{1.391445in}}%
\pgfpathlineto{\pgfqpoint{0.971021in}{1.385716in}}%
\pgfpathlineto{\pgfqpoint{0.986678in}{1.385167in}}%
\pgfpathlineto{\pgfqpoint{1.002334in}{1.389938in}}%
\pgfpathlineto{\pgfqpoint{1.007470in}{1.392962in}}%
\pgfpathlineto{\pgfqpoint{1.017991in}{1.400176in}}%
\pgfpathlineto{\pgfqpoint{1.025057in}{1.406573in}}%
\pgfpathlineto{\pgfqpoint{1.033647in}{1.417313in}}%
\pgfpathlineto{\pgfqpoint{1.035667in}{1.420184in}}%
\pgfpathlineto{\pgfqpoint{1.039890in}{1.433795in}}%
\pgfpathlineto{\pgfqpoint{1.038048in}{1.447407in}}%
\pgfpathlineto{\pgfqpoint{1.033647in}{1.455338in}}%
\pgfpathlineto{\pgfqpoint{1.030030in}{1.461018in}}%
\pgfpathlineto{\pgfqpoint{1.017991in}{1.473129in}}%
\pgfpathlineto{\pgfqpoint{1.016010in}{1.474629in}}%
\pgfpathlineto{\pgfqpoint{1.002334in}{1.482989in}}%
\pgfpathlineto{\pgfqpoint{0.986678in}{1.488072in}}%
\pgfpathlineto{\pgfqpoint{0.971021in}{1.487487in}}%
\pgfpathlineto{\pgfqpoint{0.955364in}{1.481382in}}%
\pgfpathlineto{\pgfqpoint{0.945151in}{1.474629in}}%
\pgfpathlineto{\pgfqpoint{0.939708in}{1.470188in}}%
\pgfpathlineto{\pgfqpoint{0.930893in}{1.461018in}}%
\pgfpathlineto{\pgfqpoint{0.924051in}{1.449933in}}%
\pgfpathlineto{\pgfqpoint{0.922658in}{1.447407in}}%
\pgfpathlineto{\pgfqpoint{0.920852in}{1.433795in}}%
\pgfpathlineto{\pgfqpoint{0.924051in}{1.423324in}}%
\pgfpathlineto{\pgfqpoint{0.925142in}{1.420184in}}%
\pgfpathlineto{\pgfqpoint{0.935645in}{1.406573in}}%
\pgfpathlineto{\pgfqpoint{0.939708in}{1.402769in}}%
\pgfpathlineto{\pgfqpoint{0.952983in}{1.392962in}}%
\pgfpathlineto{\pgfqpoint{0.955364in}{1.391445in}}%
\pgfpathclose%
\pgfpathmoveto{\pgfqpoint{1.268496in}{1.389938in}}%
\pgfpathlineto{\pgfqpoint{1.284152in}{1.385167in}}%
\pgfpathlineto{\pgfqpoint{1.299809in}{1.385716in}}%
\pgfpathlineto{\pgfqpoint{1.315466in}{1.391445in}}%
\pgfpathlineto{\pgfqpoint{1.317847in}{1.392962in}}%
\pgfpathlineto{\pgfqpoint{1.331122in}{1.402769in}}%
\pgfpathlineto{\pgfqpoint{1.335185in}{1.406573in}}%
\pgfpathlineto{\pgfqpoint{1.345688in}{1.420184in}}%
\pgfpathlineto{\pgfqpoint{1.346779in}{1.423324in}}%
\pgfpathlineto{\pgfqpoint{1.349978in}{1.433795in}}%
\pgfpathlineto{\pgfqpoint{1.348172in}{1.447407in}}%
\pgfpathlineto{\pgfqpoint{1.346779in}{1.449933in}}%
\pgfpathlineto{\pgfqpoint{1.339937in}{1.461018in}}%
\pgfpathlineto{\pgfqpoint{1.331122in}{1.470188in}}%
\pgfpathlineto{\pgfqpoint{1.325679in}{1.474629in}}%
\pgfpathlineto{\pgfqpoint{1.315466in}{1.481382in}}%
\pgfpathlineto{\pgfqpoint{1.299809in}{1.487487in}}%
\pgfpathlineto{\pgfqpoint{1.284152in}{1.488072in}}%
\pgfpathlineto{\pgfqpoint{1.268496in}{1.482989in}}%
\pgfpathlineto{\pgfqpoint{1.254820in}{1.474629in}}%
\pgfpathlineto{\pgfqpoint{1.252839in}{1.473129in}}%
\pgfpathlineto{\pgfqpoint{1.240800in}{1.461018in}}%
\pgfpathlineto{\pgfqpoint{1.237183in}{1.455338in}}%
\pgfpathlineto{\pgfqpoint{1.232782in}{1.447407in}}%
\pgfpathlineto{\pgfqpoint{1.230940in}{1.433795in}}%
\pgfpathlineto{\pgfqpoint{1.235163in}{1.420184in}}%
\pgfpathlineto{\pgfqpoint{1.237183in}{1.417313in}}%
\pgfpathlineto{\pgfqpoint{1.245773in}{1.406573in}}%
\pgfpathlineto{\pgfqpoint{1.252839in}{1.400176in}}%
\pgfpathlineto{\pgfqpoint{1.263360in}{1.392962in}}%
\pgfpathlineto{\pgfqpoint{1.268496in}{1.389938in}}%
\pgfpathclose%
\pgfpathmoveto{\pgfqpoint{1.581627in}{1.388598in}}%
\pgfpathlineto{\pgfqpoint{1.597284in}{1.384835in}}%
\pgfpathlineto{\pgfqpoint{1.612940in}{1.386477in}}%
\pgfpathlineto{\pgfqpoint{1.628241in}{1.392962in}}%
\pgfpathlineto{\pgfqpoint{1.628597in}{1.393140in}}%
\pgfpathlineto{\pgfqpoint{1.644253in}{1.405454in}}%
\pgfpathlineto{\pgfqpoint{1.645417in}{1.406573in}}%
\pgfpathlineto{\pgfqpoint{1.655511in}{1.420184in}}%
\pgfpathlineto{\pgfqpoint{1.659910in}{1.433175in}}%
\pgfpathlineto{\pgfqpoint{1.660099in}{1.433795in}}%
\pgfpathlineto{\pgfqpoint{1.659910in}{1.435234in}}%
\pgfpathlineto{\pgfqpoint{1.658112in}{1.447407in}}%
\pgfpathlineto{\pgfqpoint{1.649984in}{1.461018in}}%
\pgfpathlineto{\pgfqpoint{1.644253in}{1.467144in}}%
\pgfpathlineto{\pgfqpoint{1.635644in}{1.474629in}}%
\pgfpathlineto{\pgfqpoint{1.628597in}{1.479610in}}%
\pgfpathlineto{\pgfqpoint{1.612940in}{1.486677in}}%
\pgfpathlineto{\pgfqpoint{1.598939in}{1.488240in}}%
\pgfpathlineto{\pgfqpoint{1.597284in}{1.488404in}}%
\pgfpathlineto{\pgfqpoint{1.596570in}{1.488240in}}%
\pgfpathlineto{\pgfqpoint{1.581627in}{1.484416in}}%
\pgfpathlineto{\pgfqpoint{1.565971in}{1.475640in}}%
\pgfpathlineto{\pgfqpoint{1.564683in}{1.474629in}}%
\pgfpathlineto{\pgfqpoint{1.550518in}{1.461018in}}%
\pgfpathlineto{\pgfqpoint{1.550314in}{1.460709in}}%
\pgfpathlineto{\pgfqpoint{1.542854in}{1.447407in}}%
\pgfpathlineto{\pgfqpoint{1.540966in}{1.433795in}}%
\pgfpathlineto{\pgfqpoint{1.545295in}{1.420184in}}%
\pgfpathlineto{\pgfqpoint{1.550314in}{1.413122in}}%
\pgfpathlineto{\pgfqpoint{1.555756in}{1.406573in}}%
\pgfpathlineto{\pgfqpoint{1.565971in}{1.397693in}}%
\pgfpathlineto{\pgfqpoint{1.573504in}{1.392962in}}%
\pgfpathlineto{\pgfqpoint{1.581627in}{1.388598in}}%
\pgfpathclose%
\pgfpathmoveto{\pgfqpoint{1.894758in}{1.387441in}}%
\pgfpathlineto{\pgfqpoint{1.910415in}{1.384724in}}%
\pgfpathlineto{\pgfqpoint{1.910415in}{1.392962in}}%
\pgfpathlineto{\pgfqpoint{1.910415in}{1.406573in}}%
\pgfpathlineto{\pgfqpoint{1.910415in}{1.420184in}}%
\pgfpathlineto{\pgfqpoint{1.910415in}{1.433795in}}%
\pgfpathlineto{\pgfqpoint{1.910415in}{1.447407in}}%
\pgfpathlineto{\pgfqpoint{1.910415in}{1.461018in}}%
\pgfpathlineto{\pgfqpoint{1.910415in}{1.474629in}}%
\pgfpathlineto{\pgfqpoint{1.910415in}{1.488240in}}%
\pgfpathlineto{\pgfqpoint{1.910415in}{1.488509in}}%
\pgfpathlineto{\pgfqpoint{1.908788in}{1.488240in}}%
\pgfpathlineto{\pgfqpoint{1.894758in}{1.485649in}}%
\pgfpathlineto{\pgfqpoint{1.879102in}{1.477691in}}%
\pgfpathlineto{\pgfqpoint{1.875011in}{1.474629in}}%
\pgfpathlineto{\pgfqpoint{1.863445in}{1.464018in}}%
\pgfpathlineto{\pgfqpoint{1.860698in}{1.461018in}}%
\pgfpathlineto{\pgfqpoint{1.852844in}{1.447407in}}%
\pgfpathlineto{\pgfqpoint{1.850900in}{1.433795in}}%
\pgfpathlineto{\pgfqpoint{1.855357in}{1.420184in}}%
\pgfpathlineto{\pgfqpoint{1.863445in}{1.408984in}}%
\pgfpathlineto{\pgfqpoint{1.865542in}{1.406573in}}%
\pgfpathlineto{\pgfqpoint{1.879102in}{1.395341in}}%
\pgfpathlineto{\pgfqpoint{1.883305in}{1.392962in}}%
\pgfpathlineto{\pgfqpoint{1.894758in}{1.387441in}}%
\pgfpathclose%
\pgfpathmoveto{\pgfqpoint{0.376072in}{1.656961in}}%
\pgfpathlineto{\pgfqpoint{0.391728in}{1.664651in}}%
\pgfpathlineto{\pgfqpoint{0.392457in}{1.665184in}}%
\pgfpathlineto{\pgfqpoint{0.407385in}{1.678162in}}%
\pgfpathlineto{\pgfqpoint{0.407999in}{1.678795in}}%
\pgfpathlineto{\pgfqpoint{0.416844in}{1.692407in}}%
\pgfpathlineto{\pgfqpoint{0.420062in}{1.706018in}}%
\pgfpathlineto{\pgfqpoint{0.407385in}{1.706018in}}%
\pgfpathlineto{\pgfqpoint{0.391728in}{1.706018in}}%
\pgfpathlineto{\pgfqpoint{0.376072in}{1.706018in}}%
\pgfpathlineto{\pgfqpoint{0.360415in}{1.706018in}}%
\pgfpathlineto{\pgfqpoint{0.360415in}{1.692407in}}%
\pgfpathlineto{\pgfqpoint{0.360415in}{1.678795in}}%
\pgfpathlineto{\pgfqpoint{0.360415in}{1.665184in}}%
\pgfpathlineto{\pgfqpoint{0.360415in}{1.654163in}}%
\pgfpathlineto{\pgfqpoint{0.376072in}{1.656961in}}%
\pgfpathclose%
\pgfpathmoveto{\pgfqpoint{0.642233in}{1.662796in}}%
\pgfpathlineto{\pgfqpoint{0.657890in}{1.655968in}}%
\pgfpathlineto{\pgfqpoint{0.673546in}{1.654278in}}%
\pgfpathlineto{\pgfqpoint{0.689203in}{1.658153in}}%
\pgfpathlineto{\pgfqpoint{0.702086in}{1.665184in}}%
\pgfpathlineto{\pgfqpoint{0.704859in}{1.667007in}}%
\pgfpathlineto{\pgfqpoint{0.717779in}{1.678795in}}%
\pgfpathlineto{\pgfqpoint{0.720516in}{1.682449in}}%
\pgfpathlineto{\pgfqpoint{0.726867in}{1.692407in}}%
\pgfpathlineto{\pgfqpoint{0.729992in}{1.706018in}}%
\pgfpathlineto{\pgfqpoint{0.720516in}{1.706018in}}%
\pgfpathlineto{\pgfqpoint{0.704859in}{1.706018in}}%
\pgfpathlineto{\pgfqpoint{0.689203in}{1.706018in}}%
\pgfpathlineto{\pgfqpoint{0.673546in}{1.706018in}}%
\pgfpathlineto{\pgfqpoint{0.657890in}{1.706018in}}%
\pgfpathlineto{\pgfqpoint{0.642233in}{1.706018in}}%
\pgfpathlineto{\pgfqpoint{0.626577in}{1.706018in}}%
\pgfpathlineto{\pgfqpoint{0.610920in}{1.706018in}}%
\pgfpathlineto{\pgfqpoint{0.610610in}{1.706018in}}%
\pgfpathlineto{\pgfqpoint{0.610920in}{1.704603in}}%
\pgfpathlineto{\pgfqpoint{0.613900in}{1.692407in}}%
\pgfpathlineto{\pgfqpoint{0.623054in}{1.678795in}}%
\pgfpathlineto{\pgfqpoint{0.626577in}{1.675239in}}%
\pgfpathlineto{\pgfqpoint{0.638782in}{1.665184in}}%
\pgfpathlineto{\pgfqpoint{0.642233in}{1.662796in}}%
\pgfpathclose%
\pgfpathmoveto{\pgfqpoint{0.955364in}{1.661084in}}%
\pgfpathlineto{\pgfqpoint{0.971021in}{1.655185in}}%
\pgfpathlineto{\pgfqpoint{0.986678in}{1.654619in}}%
\pgfpathlineto{\pgfqpoint{1.002334in}{1.659532in}}%
\pgfpathlineto{\pgfqpoint{1.011781in}{1.665184in}}%
\pgfpathlineto{\pgfqpoint{1.017991in}{1.669640in}}%
\pgfpathlineto{\pgfqpoint{1.027625in}{1.678795in}}%
\pgfpathlineto{\pgfqpoint{1.033647in}{1.687147in}}%
\pgfpathlineto{\pgfqpoint{1.036966in}{1.692407in}}%
\pgfpathlineto{\pgfqpoint{1.040015in}{1.706018in}}%
\pgfpathlineto{\pgfqpoint{1.033647in}{1.706018in}}%
\pgfpathlineto{\pgfqpoint{1.017991in}{1.706018in}}%
\pgfpathlineto{\pgfqpoint{1.002334in}{1.706018in}}%
\pgfpathlineto{\pgfqpoint{0.986678in}{1.706018in}}%
\pgfpathlineto{\pgfqpoint{0.971021in}{1.706018in}}%
\pgfpathlineto{\pgfqpoint{0.955364in}{1.706018in}}%
\pgfpathlineto{\pgfqpoint{0.939708in}{1.706018in}}%
\pgfpathlineto{\pgfqpoint{0.924051in}{1.706018in}}%
\pgfpathlineto{\pgfqpoint{0.920730in}{1.706018in}}%
\pgfpathlineto{\pgfqpoint{0.923719in}{1.692407in}}%
\pgfpathlineto{\pgfqpoint{0.924051in}{1.691876in}}%
\pgfpathlineto{\pgfqpoint{0.933191in}{1.678795in}}%
\pgfpathlineto{\pgfqpoint{0.939708in}{1.672391in}}%
\pgfpathlineto{\pgfqpoint{0.949030in}{1.665184in}}%
\pgfpathlineto{\pgfqpoint{0.955364in}{1.661084in}}%
\pgfpathclose%
\pgfpathmoveto{\pgfqpoint{1.268496in}{1.659532in}}%
\pgfpathlineto{\pgfqpoint{1.284152in}{1.654619in}}%
\pgfpathlineto{\pgfqpoint{1.299809in}{1.655185in}}%
\pgfpathlineto{\pgfqpoint{1.315466in}{1.661084in}}%
\pgfpathlineto{\pgfqpoint{1.321800in}{1.665184in}}%
\pgfpathlineto{\pgfqpoint{1.331122in}{1.672391in}}%
\pgfpathlineto{\pgfqpoint{1.337639in}{1.678795in}}%
\pgfpathlineto{\pgfqpoint{1.346779in}{1.691876in}}%
\pgfpathlineto{\pgfqpoint{1.347111in}{1.692407in}}%
\pgfpathlineto{\pgfqpoint{1.350100in}{1.706018in}}%
\pgfpathlineto{\pgfqpoint{1.346779in}{1.706018in}}%
\pgfpathlineto{\pgfqpoint{1.331122in}{1.706018in}}%
\pgfpathlineto{\pgfqpoint{1.315466in}{1.706018in}}%
\pgfpathlineto{\pgfqpoint{1.299809in}{1.706018in}}%
\pgfpathlineto{\pgfqpoint{1.284152in}{1.706018in}}%
\pgfpathlineto{\pgfqpoint{1.268496in}{1.706018in}}%
\pgfpathlineto{\pgfqpoint{1.252839in}{1.706018in}}%
\pgfpathlineto{\pgfqpoint{1.237183in}{1.706018in}}%
\pgfpathlineto{\pgfqpoint{1.230815in}{1.706018in}}%
\pgfpathlineto{\pgfqpoint{1.233864in}{1.692407in}}%
\pgfpathlineto{\pgfqpoint{1.237183in}{1.687147in}}%
\pgfpathlineto{\pgfqpoint{1.243205in}{1.678795in}}%
\pgfpathlineto{\pgfqpoint{1.252839in}{1.669640in}}%
\pgfpathlineto{\pgfqpoint{1.259049in}{1.665184in}}%
\pgfpathlineto{\pgfqpoint{1.268496in}{1.659532in}}%
\pgfpathclose%
\pgfpathmoveto{\pgfqpoint{1.581627in}{1.658153in}}%
\pgfpathlineto{\pgfqpoint{1.597284in}{1.654278in}}%
\pgfpathlineto{\pgfqpoint{1.612940in}{1.655968in}}%
\pgfpathlineto{\pgfqpoint{1.628597in}{1.662796in}}%
\pgfpathlineto{\pgfqpoint{1.632048in}{1.665184in}}%
\pgfpathlineto{\pgfqpoint{1.644253in}{1.675239in}}%
\pgfpathlineto{\pgfqpoint{1.647776in}{1.678795in}}%
\pgfpathlineto{\pgfqpoint{1.656930in}{1.692407in}}%
\pgfpathlineto{\pgfqpoint{1.659910in}{1.704603in}}%
\pgfpathlineto{\pgfqpoint{1.660220in}{1.706018in}}%
\pgfpathlineto{\pgfqpoint{1.659910in}{1.706018in}}%
\pgfpathlineto{\pgfqpoint{1.644253in}{1.706018in}}%
\pgfpathlineto{\pgfqpoint{1.628597in}{1.706018in}}%
\pgfpathlineto{\pgfqpoint{1.612940in}{1.706018in}}%
\pgfpathlineto{\pgfqpoint{1.597284in}{1.706018in}}%
\pgfpathlineto{\pgfqpoint{1.581627in}{1.706018in}}%
\pgfpathlineto{\pgfqpoint{1.565971in}{1.706018in}}%
\pgfpathlineto{\pgfqpoint{1.550314in}{1.706018in}}%
\pgfpathlineto{\pgfqpoint{1.540838in}{1.706018in}}%
\pgfpathlineto{\pgfqpoint{1.543963in}{1.692407in}}%
\pgfpathlineto{\pgfqpoint{1.550314in}{1.682449in}}%
\pgfpathlineto{\pgfqpoint{1.553051in}{1.678795in}}%
\pgfpathlineto{\pgfqpoint{1.565971in}{1.667007in}}%
\pgfpathlineto{\pgfqpoint{1.568744in}{1.665184in}}%
\pgfpathlineto{\pgfqpoint{1.581627in}{1.658153in}}%
\pgfpathclose%
\pgfpathmoveto{\pgfqpoint{1.879102in}{1.664651in}}%
\pgfpathlineto{\pgfqpoint{1.894758in}{1.656961in}}%
\pgfpathlineto{\pgfqpoint{1.910415in}{1.654163in}}%
\pgfpathlineto{\pgfqpoint{1.910415in}{1.665184in}}%
\pgfpathlineto{\pgfqpoint{1.910415in}{1.678795in}}%
\pgfpathlineto{\pgfqpoint{1.910415in}{1.692407in}}%
\pgfpathlineto{\pgfqpoint{1.910415in}{1.706018in}}%
\pgfpathlineto{\pgfqpoint{1.894758in}{1.706018in}}%
\pgfpathlineto{\pgfqpoint{1.879102in}{1.706018in}}%
\pgfpathlineto{\pgfqpoint{1.863445in}{1.706018in}}%
\pgfpathlineto{\pgfqpoint{1.850768in}{1.706018in}}%
\pgfpathlineto{\pgfqpoint{1.853986in}{1.692407in}}%
\pgfpathlineto{\pgfqpoint{1.862831in}{1.678795in}}%
\pgfpathlineto{\pgfqpoint{1.863445in}{1.678162in}}%
\pgfpathlineto{\pgfqpoint{1.878373in}{1.665184in}}%
\pgfpathlineto{\pgfqpoint{1.879102in}{1.664651in}}%
\pgfpathclose%
\pgfusepath{fill}%
\end{pgfscope}%
\begin{pgfscope}%
\pgfsetbuttcap%
\pgfsetroundjoin%
\definecolor{currentfill}{rgb}{0.000000,0.000000,0.000000}%
\pgfsetfillcolor{currentfill}%
\pgfsetlinewidth{0.803000pt}%
\definecolor{currentstroke}{rgb}{0.000000,0.000000,0.000000}%
\pgfsetstrokecolor{currentstroke}%
\pgfsetdash{}{0pt}%
\pgfsys@defobject{currentmarker}{\pgfqpoint{0.000000in}{-0.048611in}}{\pgfqpoint{0.000000in}{0.000000in}}{%
\pgfpathmoveto{\pgfqpoint{0.000000in}{0.000000in}}%
\pgfpathlineto{\pgfqpoint{0.000000in}{-0.048611in}}%
\pgfusepath{stroke,fill}%
}%
\begin{pgfscope}%
\pgfsys@transformshift{0.360415in}{0.358518in}%
\pgfsys@useobject{currentmarker}{}%
\end{pgfscope}%
\end{pgfscope}%
\begin{pgfscope}%
\definecolor{textcolor}{rgb}{0.000000,0.000000,0.000000}%
\pgfsetstrokecolor{textcolor}%
\pgfsetfillcolor{textcolor}%
\pgftext[x=0.360415in,y=0.261295in,,top]{\color{textcolor}{\rmfamily\fontsize{12.000000}{14.400000}\selectfont\catcode`\^=\active\def^{\ifmmode\sp\else\^{}\fi}\catcode`\%=\active\def%{\%}$\mathdefault{0}$}}%
\end{pgfscope}%
\begin{pgfscope}%
\pgfsetbuttcap%
\pgfsetroundjoin%
\definecolor{currentfill}{rgb}{0.000000,0.000000,0.000000}%
\pgfsetfillcolor{currentfill}%
\pgfsetlinewidth{0.803000pt}%
\definecolor{currentstroke}{rgb}{0.000000,0.000000,0.000000}%
\pgfsetstrokecolor{currentstroke}%
\pgfsetdash{}{0pt}%
\pgfsys@defobject{currentmarker}{\pgfqpoint{0.000000in}{-0.048611in}}{\pgfqpoint{0.000000in}{0.000000in}}{%
\pgfpathmoveto{\pgfqpoint{0.000000in}{0.000000in}}%
\pgfpathlineto{\pgfqpoint{0.000000in}{-0.048611in}}%
\pgfusepath{stroke,fill}%
}%
\begin{pgfscope}%
\pgfsys@transformshift{0.877082in}{0.358518in}%
\pgfsys@useobject{currentmarker}{}%
\end{pgfscope}%
\end{pgfscope}%
\begin{pgfscope}%
\definecolor{textcolor}{rgb}{0.000000,0.000000,0.000000}%
\pgfsetstrokecolor{textcolor}%
\pgfsetfillcolor{textcolor}%
\pgftext[x=0.877082in,y=0.261295in,,top]{\color{textcolor}{\rmfamily\fontsize{12.000000}{14.400000}\selectfont\catcode`\^=\active\def^{\ifmmode\sp\else\^{}\fi}\catcode`\%=\active\def%{\%}$\mathdefault{10}$}}%
\end{pgfscope}%
\begin{pgfscope}%
\pgfsetbuttcap%
\pgfsetroundjoin%
\definecolor{currentfill}{rgb}{0.000000,0.000000,0.000000}%
\pgfsetfillcolor{currentfill}%
\pgfsetlinewidth{0.803000pt}%
\definecolor{currentstroke}{rgb}{0.000000,0.000000,0.000000}%
\pgfsetstrokecolor{currentstroke}%
\pgfsetdash{}{0pt}%
\pgfsys@defobject{currentmarker}{\pgfqpoint{0.000000in}{-0.048611in}}{\pgfqpoint{0.000000in}{0.000000in}}{%
\pgfpathmoveto{\pgfqpoint{0.000000in}{0.000000in}}%
\pgfpathlineto{\pgfqpoint{0.000000in}{-0.048611in}}%
\pgfusepath{stroke,fill}%
}%
\begin{pgfscope}%
\pgfsys@transformshift{1.393748in}{0.358518in}%
\pgfsys@useobject{currentmarker}{}%
\end{pgfscope}%
\end{pgfscope}%
\begin{pgfscope}%
\definecolor{textcolor}{rgb}{0.000000,0.000000,0.000000}%
\pgfsetstrokecolor{textcolor}%
\pgfsetfillcolor{textcolor}%
\pgftext[x=1.393748in,y=0.261295in,,top]{\color{textcolor}{\rmfamily\fontsize{12.000000}{14.400000}\selectfont\catcode`\^=\active\def^{\ifmmode\sp\else\^{}\fi}\catcode`\%=\active\def%{\%}$\mathdefault{20}$}}%
\end{pgfscope}%
\begin{pgfscope}%
\pgfsetbuttcap%
\pgfsetroundjoin%
\definecolor{currentfill}{rgb}{0.000000,0.000000,0.000000}%
\pgfsetfillcolor{currentfill}%
\pgfsetlinewidth{0.803000pt}%
\definecolor{currentstroke}{rgb}{0.000000,0.000000,0.000000}%
\pgfsetstrokecolor{currentstroke}%
\pgfsetdash{}{0pt}%
\pgfsys@defobject{currentmarker}{\pgfqpoint{0.000000in}{-0.048611in}}{\pgfqpoint{0.000000in}{0.000000in}}{%
\pgfpathmoveto{\pgfqpoint{0.000000in}{0.000000in}}%
\pgfpathlineto{\pgfqpoint{0.000000in}{-0.048611in}}%
\pgfusepath{stroke,fill}%
}%
\begin{pgfscope}%
\pgfsys@transformshift{1.910415in}{0.358518in}%
\pgfsys@useobject{currentmarker}{}%
\end{pgfscope}%
\end{pgfscope}%
\begin{pgfscope}%
\definecolor{textcolor}{rgb}{0.000000,0.000000,0.000000}%
\pgfsetstrokecolor{textcolor}%
\pgfsetfillcolor{textcolor}%
\pgftext[x=1.910415in,y=0.261295in,,top]{\color{textcolor}{\rmfamily\fontsize{12.000000}{14.400000}\selectfont\catcode`\^=\active\def^{\ifmmode\sp\else\^{}\fi}\catcode`\%=\active\def%{\%}$\mathdefault{30}$}}%
\end{pgfscope}%
\begin{pgfscope}%
\pgfsetbuttcap%
\pgfsetroundjoin%
\definecolor{currentfill}{rgb}{0.000000,0.000000,0.000000}%
\pgfsetfillcolor{currentfill}%
\pgfsetlinewidth{0.803000pt}%
\definecolor{currentstroke}{rgb}{0.000000,0.000000,0.000000}%
\pgfsetstrokecolor{currentstroke}%
\pgfsetdash{}{0pt}%
\pgfsys@defobject{currentmarker}{\pgfqpoint{-0.048611in}{0.000000in}}{\pgfqpoint{-0.000000in}{0.000000in}}{%
\pgfpathmoveto{\pgfqpoint{-0.000000in}{0.000000in}}%
\pgfpathlineto{\pgfqpoint{-0.048611in}{0.000000in}}%
\pgfusepath{stroke,fill}%
}%
\begin{pgfscope}%
\pgfsys@transformshift{0.360415in}{0.358518in}%
\pgfsys@useobject{currentmarker}{}%
\end{pgfscope}%
\end{pgfscope}%
\begin{pgfscope}%
\definecolor{textcolor}{rgb}{0.000000,0.000000,0.000000}%
\pgfsetstrokecolor{textcolor}%
\pgfsetfillcolor{textcolor}%
\pgftext[x=0.181596in, y=0.295204in, left, base]{\color{textcolor}{\rmfamily\fontsize{12.000000}{14.400000}\selectfont\catcode`\^=\active\def^{\ifmmode\sp\else\^{}\fi}\catcode`\%=\active\def%{\%}$\mathdefault{0}$}}%
\end{pgfscope}%
\begin{pgfscope}%
\pgfsetbuttcap%
\pgfsetroundjoin%
\definecolor{currentfill}{rgb}{0.000000,0.000000,0.000000}%
\pgfsetfillcolor{currentfill}%
\pgfsetlinewidth{0.803000pt}%
\definecolor{currentstroke}{rgb}{0.000000,0.000000,0.000000}%
\pgfsetstrokecolor{currentstroke}%
\pgfsetdash{}{0pt}%
\pgfsys@defobject{currentmarker}{\pgfqpoint{-0.048611in}{0.000000in}}{\pgfqpoint{-0.000000in}{0.000000in}}{%
\pgfpathmoveto{\pgfqpoint{-0.000000in}{0.000000in}}%
\pgfpathlineto{\pgfqpoint{-0.048611in}{0.000000in}}%
\pgfusepath{stroke,fill}%
}%
\begin{pgfscope}%
\pgfsys@transformshift{0.360415in}{0.807684in}%
\pgfsys@useobject{currentmarker}{}%
\end{pgfscope}%
\end{pgfscope}%
\begin{pgfscope}%
\definecolor{textcolor}{rgb}{0.000000,0.000000,0.000000}%
\pgfsetstrokecolor{textcolor}%
\pgfsetfillcolor{textcolor}%
\pgftext[x=0.100000in, y=0.744370in, left, base]{\color{textcolor}{\rmfamily\fontsize{12.000000}{14.400000}\selectfont\catcode`\^=\active\def^{\ifmmode\sp\else\^{}\fi}\catcode`\%=\active\def%{\%}$\mathdefault{10}$}}%
\end{pgfscope}%
\begin{pgfscope}%
\pgfsetbuttcap%
\pgfsetroundjoin%
\definecolor{currentfill}{rgb}{0.000000,0.000000,0.000000}%
\pgfsetfillcolor{currentfill}%
\pgfsetlinewidth{0.803000pt}%
\definecolor{currentstroke}{rgb}{0.000000,0.000000,0.000000}%
\pgfsetstrokecolor{currentstroke}%
\pgfsetdash{}{0pt}%
\pgfsys@defobject{currentmarker}{\pgfqpoint{-0.048611in}{0.000000in}}{\pgfqpoint{-0.000000in}{0.000000in}}{%
\pgfpathmoveto{\pgfqpoint{-0.000000in}{0.000000in}}%
\pgfpathlineto{\pgfqpoint{-0.048611in}{0.000000in}}%
\pgfusepath{stroke,fill}%
}%
\begin{pgfscope}%
\pgfsys@transformshift{0.360415in}{1.256851in}%
\pgfsys@useobject{currentmarker}{}%
\end{pgfscope}%
\end{pgfscope}%
\begin{pgfscope}%
\definecolor{textcolor}{rgb}{0.000000,0.000000,0.000000}%
\pgfsetstrokecolor{textcolor}%
\pgfsetfillcolor{textcolor}%
\pgftext[x=0.100000in, y=1.193537in, left, base]{\color{textcolor}{\rmfamily\fontsize{12.000000}{14.400000}\selectfont\catcode`\^=\active\def^{\ifmmode\sp\else\^{}\fi}\catcode`\%=\active\def%{\%}$\mathdefault{20}$}}%
\end{pgfscope}%
\begin{pgfscope}%
\pgfsetbuttcap%
\pgfsetroundjoin%
\definecolor{currentfill}{rgb}{0.000000,0.000000,0.000000}%
\pgfsetfillcolor{currentfill}%
\pgfsetlinewidth{0.803000pt}%
\definecolor{currentstroke}{rgb}{0.000000,0.000000,0.000000}%
\pgfsetstrokecolor{currentstroke}%
\pgfsetdash{}{0pt}%
\pgfsys@defobject{currentmarker}{\pgfqpoint{-0.048611in}{0.000000in}}{\pgfqpoint{-0.000000in}{0.000000in}}{%
\pgfpathmoveto{\pgfqpoint{-0.000000in}{0.000000in}}%
\pgfpathlineto{\pgfqpoint{-0.048611in}{0.000000in}}%
\pgfusepath{stroke,fill}%
}%
\begin{pgfscope}%
\pgfsys@transformshift{0.360415in}{1.706018in}%
\pgfsys@useobject{currentmarker}{}%
\end{pgfscope}%
\end{pgfscope}%
\begin{pgfscope}%
\definecolor{textcolor}{rgb}{0.000000,0.000000,0.000000}%
\pgfsetstrokecolor{textcolor}%
\pgfsetfillcolor{textcolor}%
\pgftext[x=0.100000in, y=1.642704in, left, base]{\color{textcolor}{\rmfamily\fontsize{12.000000}{14.400000}\selectfont\catcode`\^=\active\def^{\ifmmode\sp\else\^{}\fi}\catcode`\%=\active\def%{\%}$\mathdefault{30}$}}%
\end{pgfscope}%
\begin{pgfscope}%
\pgfsetrectcap%
\pgfsetmiterjoin%
\pgfsetlinewidth{0.803000pt}%
\definecolor{currentstroke}{rgb}{0.000000,0.000000,0.000000}%
\pgfsetstrokecolor{currentstroke}%
\pgfsetdash{}{0pt}%
\pgfpathmoveto{\pgfqpoint{0.360415in}{0.358518in}}%
\pgfpathlineto{\pgfqpoint{0.360415in}{1.706018in}}%
\pgfusepath{stroke}%
\end{pgfscope}%
\begin{pgfscope}%
\pgfsetrectcap%
\pgfsetmiterjoin%
\pgfsetlinewidth{0.803000pt}%
\definecolor{currentstroke}{rgb}{0.000000,0.000000,0.000000}%
\pgfsetstrokecolor{currentstroke}%
\pgfsetdash{}{0pt}%
\pgfpathmoveto{\pgfqpoint{1.910415in}{0.358518in}}%
\pgfpathlineto{\pgfqpoint{1.910415in}{1.706018in}}%
\pgfusepath{stroke}%
\end{pgfscope}%
\begin{pgfscope}%
\pgfsetrectcap%
\pgfsetmiterjoin%
\pgfsetlinewidth{0.803000pt}%
\definecolor{currentstroke}{rgb}{0.000000,0.000000,0.000000}%
\pgfsetstrokecolor{currentstroke}%
\pgfsetdash{}{0pt}%
\pgfpathmoveto{\pgfqpoint{0.360415in}{0.358518in}}%
\pgfpathlineto{\pgfqpoint{1.910415in}{0.358518in}}%
\pgfusepath{stroke}%
\end{pgfscope}%
\begin{pgfscope}%
\pgfsetrectcap%
\pgfsetmiterjoin%
\pgfsetlinewidth{0.803000pt}%
\definecolor{currentstroke}{rgb}{0.000000,0.000000,0.000000}%
\pgfsetstrokecolor{currentstroke}%
\pgfsetdash{}{0pt}%
\pgfpathmoveto{\pgfqpoint{0.360415in}{1.706018in}}%
\pgfpathlineto{\pgfqpoint{1.910415in}{1.706018in}}%
\pgfusepath{stroke}%
\end{pgfscope}%
\end{pgfpicture}%
\makeatother%
\endgroup%

        \caption{$c=5$}
        \label{fig:5-experiments-periodic-gaussian-well-5}
    \end{subfigure}
    \caption{Two dimensional periodic potential $V$ for different sizes $c$ of the computational domain.}
    \label{fig:5-experiments-periodic-gaussian-well}
\end{figure}

For Gaussian \gls{smoothing-kernel} \refequ{equ:1-introduction-def-gaussian-kernel}
with \gls{smoothing-parameter} $=0.05$ we plot for two choices of \gls{chebyshev-degree}
the convergence of the error with \gls{sketch-size} in \reffig{fig:5-experiments-electronic-structure-convergence-nv}
and equally for two choices of \gls{sketch-size} $+$ \gls{num-hutchinson-queries} the convergence of the
error with \gls{chebyshev-degree} in \reffig{fig:5-experiments-electronic-structure-convergence-nv}.
In our experiments, we always use \gls{sketch-size} $=$ \gls{num-hutchinson-queries} for
the \gls{NCPP} method.\\

\begin{figure}[ht]
    \begin{subfigure}[b]{0.49\columnwidth}
        %% Creator: Matplotlib, PGF backend
%%
%% To include the figure in your LaTeX document, write
%%   \input{<filename>.pgf}
%%
%% Make sure the required packages are loaded in your preamble
%%   \usepackage{pgf}
%%
%% Also ensure that all the required font packages are loaded; for instance,
%% the lmodern package is sometimes necessary when using math font.
%%   \usepackage{lmodern}
%%
%% Figures using additional raster images can only be included by \input if
%% they are in the same directory as the main LaTeX file. For loading figures
%% from other directories you can use the `import` package
%%   \usepackage{import}
%%
%% and then include the figures with
%%   \import{<path to file>}{<filename>.pgf}
%%
%% Matplotlib used the following preamble
%%   \def\mathdefault#1{#1}
%%   \everymath=\expandafter{\the\everymath\displaystyle}
%%   \IfFileExists{scrextend.sty}{
%%     \usepackage[fontsize=12.000000pt]{scrextend}
%%   }{
%%     \renewcommand{\normalsize}{\fontsize{12.000000}{14.400000}\selectfont}
%%     \normalsize
%%   }
%%   
%%   \ifdefined\pdftexversion\else  % non-pdftex case.
%%     \usepackage{fontspec}
%%     \setmainfont{DejaVuSans.ttf}[Path=\detokenize{/opt/hostedtoolcache/Python/3.12.9/x64/lib/python3.12/site-packages/matplotlib/mpl-data/fonts/ttf/}]
%%     \setsansfont{DejaVuSans.ttf}[Path=\detokenize{/opt/hostedtoolcache/Python/3.12.9/x64/lib/python3.12/site-packages/matplotlib/mpl-data/fonts/ttf/}]
%%     \setmonofont{DejaVuSansMono.ttf}[Path=\detokenize{/opt/hostedtoolcache/Python/3.12.9/x64/lib/python3.12/site-packages/matplotlib/mpl-data/fonts/ttf/}]
%%   \fi
%%   \makeatletter\@ifpackageloaded{underscore}{}{\usepackage[strings]{underscore}}\makeatother
%%
\begingroup%
\makeatletter%
\begin{pgfpicture}%
\pgfpathrectangle{\pgfpointorigin}{\pgfqpoint{2.759413in}{2.574073in}}%
\pgfusepath{use as bounding box, clip}%
\begin{pgfscope}%
\pgfsetbuttcap%
\pgfsetmiterjoin%
\definecolor{currentfill}{rgb}{1.000000,1.000000,1.000000}%
\pgfsetfillcolor{currentfill}%
\pgfsetlinewidth{0.000000pt}%
\definecolor{currentstroke}{rgb}{1.000000,1.000000,1.000000}%
\pgfsetstrokecolor{currentstroke}%
\pgfsetdash{}{0pt}%
\pgfpathmoveto{\pgfqpoint{0.000000in}{0.000000in}}%
\pgfpathlineto{\pgfqpoint{2.759413in}{0.000000in}}%
\pgfpathlineto{\pgfqpoint{2.759413in}{2.574073in}}%
\pgfpathlineto{\pgfqpoint{0.000000in}{2.574073in}}%
\pgfpathlineto{\pgfqpoint{0.000000in}{0.000000in}}%
\pgfpathclose%
\pgfusepath{fill}%
\end{pgfscope}%
\begin{pgfscope}%
\pgfsetbuttcap%
\pgfsetmiterjoin%
\definecolor{currentfill}{rgb}{1.000000,1.000000,1.000000}%
\pgfsetfillcolor{currentfill}%
\pgfsetlinewidth{0.000000pt}%
\definecolor{currentstroke}{rgb}{0.000000,0.000000,0.000000}%
\pgfsetstrokecolor{currentstroke}%
\pgfsetstrokeopacity{0.000000}%
\pgfsetdash{}{0pt}%
\pgfpathmoveto{\pgfqpoint{0.721913in}{0.549073in}}%
\pgfpathlineto{\pgfqpoint{2.659413in}{0.549073in}}%
\pgfpathlineto{\pgfqpoint{2.659413in}{2.474073in}}%
\pgfpathlineto{\pgfqpoint{0.721913in}{2.474073in}}%
\pgfpathlineto{\pgfqpoint{0.721913in}{0.549073in}}%
\pgfpathclose%
\pgfusepath{fill}%
\end{pgfscope}%
\begin{pgfscope}%
\pgfsetbuttcap%
\pgfsetroundjoin%
\definecolor{currentfill}{rgb}{0.000000,0.000000,0.000000}%
\pgfsetfillcolor{currentfill}%
\pgfsetlinewidth{0.803000pt}%
\definecolor{currentstroke}{rgb}{0.000000,0.000000,0.000000}%
\pgfsetstrokecolor{currentstroke}%
\pgfsetdash{}{0pt}%
\pgfsys@defobject{currentmarker}{\pgfqpoint{0.000000in}{-0.048611in}}{\pgfqpoint{0.000000in}{0.000000in}}{%
\pgfpathmoveto{\pgfqpoint{0.000000in}{0.000000in}}%
\pgfpathlineto{\pgfqpoint{0.000000in}{-0.048611in}}%
\pgfusepath{stroke,fill}%
}%
\begin{pgfscope}%
\pgfsys@transformshift{0.910580in}{0.549073in}%
\pgfsys@useobject{currentmarker}{}%
\end{pgfscope}%
\end{pgfscope}%
\begin{pgfscope}%
\definecolor{textcolor}{rgb}{0.000000,0.000000,0.000000}%
\pgfsetstrokecolor{textcolor}%
\pgfsetfillcolor{textcolor}%
\pgftext[x=0.910580in,y=0.451851in,,top]{\color{textcolor}{\rmfamily\fontsize{12.000000}{14.400000}\selectfont\catcode`\^=\active\def^{\ifmmode\sp\else\^{}\fi}\catcode`\%=\active\def%{\%}$\mathdefault{10^{1}}$}}%
\end{pgfscope}%
\begin{pgfscope}%
\pgfsetbuttcap%
\pgfsetroundjoin%
\definecolor{currentfill}{rgb}{0.000000,0.000000,0.000000}%
\pgfsetfillcolor{currentfill}%
\pgfsetlinewidth{0.803000pt}%
\definecolor{currentstroke}{rgb}{0.000000,0.000000,0.000000}%
\pgfsetstrokecolor{currentstroke}%
\pgfsetdash{}{0pt}%
\pgfsys@defobject{currentmarker}{\pgfqpoint{0.000000in}{-0.048611in}}{\pgfqpoint{0.000000in}{0.000000in}}{%
\pgfpathmoveto{\pgfqpoint{0.000000in}{0.000000in}}%
\pgfpathlineto{\pgfqpoint{0.000000in}{-0.048611in}}%
\pgfusepath{stroke,fill}%
}%
\begin{pgfscope}%
\pgfsys@transformshift{1.948634in}{0.549073in}%
\pgfsys@useobject{currentmarker}{}%
\end{pgfscope}%
\end{pgfscope}%
\begin{pgfscope}%
\definecolor{textcolor}{rgb}{0.000000,0.000000,0.000000}%
\pgfsetstrokecolor{textcolor}%
\pgfsetfillcolor{textcolor}%
\pgftext[x=1.948634in,y=0.451851in,,top]{\color{textcolor}{\rmfamily\fontsize{12.000000}{14.400000}\selectfont\catcode`\^=\active\def^{\ifmmode\sp\else\^{}\fi}\catcode`\%=\active\def%{\%}$\mathdefault{10^{2}}$}}%
\end{pgfscope}%
\begin{pgfscope}%
\pgfsetbuttcap%
\pgfsetroundjoin%
\definecolor{currentfill}{rgb}{0.000000,0.000000,0.000000}%
\pgfsetfillcolor{currentfill}%
\pgfsetlinewidth{0.602250pt}%
\definecolor{currentstroke}{rgb}{0.000000,0.000000,0.000000}%
\pgfsetstrokecolor{currentstroke}%
\pgfsetdash{}{0pt}%
\pgfsys@defobject{currentmarker}{\pgfqpoint{0.000000in}{-0.027778in}}{\pgfqpoint{0.000000in}{0.000000in}}{%
\pgfpathmoveto{\pgfqpoint{0.000000in}{0.000000in}}%
\pgfpathlineto{\pgfqpoint{0.000000in}{-0.027778in}}%
\pgfusepath{stroke,fill}%
}%
\begin{pgfscope}%
\pgfsys@transformshift{0.749783in}{0.549073in}%
\pgfsys@useobject{currentmarker}{}%
\end{pgfscope}%
\end{pgfscope}%
\begin{pgfscope}%
\pgfsetbuttcap%
\pgfsetroundjoin%
\definecolor{currentfill}{rgb}{0.000000,0.000000,0.000000}%
\pgfsetfillcolor{currentfill}%
\pgfsetlinewidth{0.602250pt}%
\definecolor{currentstroke}{rgb}{0.000000,0.000000,0.000000}%
\pgfsetstrokecolor{currentstroke}%
\pgfsetdash{}{0pt}%
\pgfsys@defobject{currentmarker}{\pgfqpoint{0.000000in}{-0.027778in}}{\pgfqpoint{0.000000in}{0.000000in}}{%
\pgfpathmoveto{\pgfqpoint{0.000000in}{0.000000in}}%
\pgfpathlineto{\pgfqpoint{0.000000in}{-0.027778in}}%
\pgfusepath{stroke,fill}%
}%
\begin{pgfscope}%
\pgfsys@transformshift{0.809982in}{0.549073in}%
\pgfsys@useobject{currentmarker}{}%
\end{pgfscope}%
\end{pgfscope}%
\begin{pgfscope}%
\pgfsetbuttcap%
\pgfsetroundjoin%
\definecolor{currentfill}{rgb}{0.000000,0.000000,0.000000}%
\pgfsetfillcolor{currentfill}%
\pgfsetlinewidth{0.602250pt}%
\definecolor{currentstroke}{rgb}{0.000000,0.000000,0.000000}%
\pgfsetstrokecolor{currentstroke}%
\pgfsetdash{}{0pt}%
\pgfsys@defobject{currentmarker}{\pgfqpoint{0.000000in}{-0.027778in}}{\pgfqpoint{0.000000in}{0.000000in}}{%
\pgfpathmoveto{\pgfqpoint{0.000000in}{0.000000in}}%
\pgfpathlineto{\pgfqpoint{0.000000in}{-0.027778in}}%
\pgfusepath{stroke,fill}%
}%
\begin{pgfscope}%
\pgfsys@transformshift{0.863081in}{0.549073in}%
\pgfsys@useobject{currentmarker}{}%
\end{pgfscope}%
\end{pgfscope}%
\begin{pgfscope}%
\pgfsetbuttcap%
\pgfsetroundjoin%
\definecolor{currentfill}{rgb}{0.000000,0.000000,0.000000}%
\pgfsetfillcolor{currentfill}%
\pgfsetlinewidth{0.602250pt}%
\definecolor{currentstroke}{rgb}{0.000000,0.000000,0.000000}%
\pgfsetstrokecolor{currentstroke}%
\pgfsetdash{}{0pt}%
\pgfsys@defobject{currentmarker}{\pgfqpoint{0.000000in}{-0.027778in}}{\pgfqpoint{0.000000in}{0.000000in}}{%
\pgfpathmoveto{\pgfqpoint{0.000000in}{0.000000in}}%
\pgfpathlineto{\pgfqpoint{0.000000in}{-0.027778in}}%
\pgfusepath{stroke,fill}%
}%
\begin{pgfscope}%
\pgfsys@transformshift{1.223065in}{0.549073in}%
\pgfsys@useobject{currentmarker}{}%
\end{pgfscope}%
\end{pgfscope}%
\begin{pgfscope}%
\pgfsetbuttcap%
\pgfsetroundjoin%
\definecolor{currentfill}{rgb}{0.000000,0.000000,0.000000}%
\pgfsetfillcolor{currentfill}%
\pgfsetlinewidth{0.602250pt}%
\definecolor{currentstroke}{rgb}{0.000000,0.000000,0.000000}%
\pgfsetstrokecolor{currentstroke}%
\pgfsetdash{}{0pt}%
\pgfsys@defobject{currentmarker}{\pgfqpoint{0.000000in}{-0.027778in}}{\pgfqpoint{0.000000in}{0.000000in}}{%
\pgfpathmoveto{\pgfqpoint{0.000000in}{0.000000in}}%
\pgfpathlineto{\pgfqpoint{0.000000in}{-0.027778in}}%
\pgfusepath{stroke,fill}%
}%
\begin{pgfscope}%
\pgfsys@transformshift{1.405857in}{0.549073in}%
\pgfsys@useobject{currentmarker}{}%
\end{pgfscope}%
\end{pgfscope}%
\begin{pgfscope}%
\pgfsetbuttcap%
\pgfsetroundjoin%
\definecolor{currentfill}{rgb}{0.000000,0.000000,0.000000}%
\pgfsetfillcolor{currentfill}%
\pgfsetlinewidth{0.602250pt}%
\definecolor{currentstroke}{rgb}{0.000000,0.000000,0.000000}%
\pgfsetstrokecolor{currentstroke}%
\pgfsetdash{}{0pt}%
\pgfsys@defobject{currentmarker}{\pgfqpoint{0.000000in}{-0.027778in}}{\pgfqpoint{0.000000in}{0.000000in}}{%
\pgfpathmoveto{\pgfqpoint{0.000000in}{0.000000in}}%
\pgfpathlineto{\pgfqpoint{0.000000in}{-0.027778in}}%
\pgfusepath{stroke,fill}%
}%
\begin{pgfscope}%
\pgfsys@transformshift{1.535551in}{0.549073in}%
\pgfsys@useobject{currentmarker}{}%
\end{pgfscope}%
\end{pgfscope}%
\begin{pgfscope}%
\pgfsetbuttcap%
\pgfsetroundjoin%
\definecolor{currentfill}{rgb}{0.000000,0.000000,0.000000}%
\pgfsetfillcolor{currentfill}%
\pgfsetlinewidth{0.602250pt}%
\definecolor{currentstroke}{rgb}{0.000000,0.000000,0.000000}%
\pgfsetstrokecolor{currentstroke}%
\pgfsetdash{}{0pt}%
\pgfsys@defobject{currentmarker}{\pgfqpoint{0.000000in}{-0.027778in}}{\pgfqpoint{0.000000in}{0.000000in}}{%
\pgfpathmoveto{\pgfqpoint{0.000000in}{0.000000in}}%
\pgfpathlineto{\pgfqpoint{0.000000in}{-0.027778in}}%
\pgfusepath{stroke,fill}%
}%
\begin{pgfscope}%
\pgfsys@transformshift{1.636148in}{0.549073in}%
\pgfsys@useobject{currentmarker}{}%
\end{pgfscope}%
\end{pgfscope}%
\begin{pgfscope}%
\pgfsetbuttcap%
\pgfsetroundjoin%
\definecolor{currentfill}{rgb}{0.000000,0.000000,0.000000}%
\pgfsetfillcolor{currentfill}%
\pgfsetlinewidth{0.602250pt}%
\definecolor{currentstroke}{rgb}{0.000000,0.000000,0.000000}%
\pgfsetstrokecolor{currentstroke}%
\pgfsetdash{}{0pt}%
\pgfsys@defobject{currentmarker}{\pgfqpoint{0.000000in}{-0.027778in}}{\pgfqpoint{0.000000in}{0.000000in}}{%
\pgfpathmoveto{\pgfqpoint{0.000000in}{0.000000in}}%
\pgfpathlineto{\pgfqpoint{0.000000in}{-0.027778in}}%
\pgfusepath{stroke,fill}%
}%
\begin{pgfscope}%
\pgfsys@transformshift{1.718343in}{0.549073in}%
\pgfsys@useobject{currentmarker}{}%
\end{pgfscope}%
\end{pgfscope}%
\begin{pgfscope}%
\pgfsetbuttcap%
\pgfsetroundjoin%
\definecolor{currentfill}{rgb}{0.000000,0.000000,0.000000}%
\pgfsetfillcolor{currentfill}%
\pgfsetlinewidth{0.602250pt}%
\definecolor{currentstroke}{rgb}{0.000000,0.000000,0.000000}%
\pgfsetstrokecolor{currentstroke}%
\pgfsetdash{}{0pt}%
\pgfsys@defobject{currentmarker}{\pgfqpoint{0.000000in}{-0.027778in}}{\pgfqpoint{0.000000in}{0.000000in}}{%
\pgfpathmoveto{\pgfqpoint{0.000000in}{0.000000in}}%
\pgfpathlineto{\pgfqpoint{0.000000in}{-0.027778in}}%
\pgfusepath{stroke,fill}%
}%
\begin{pgfscope}%
\pgfsys@transformshift{1.787837in}{0.549073in}%
\pgfsys@useobject{currentmarker}{}%
\end{pgfscope}%
\end{pgfscope}%
\begin{pgfscope}%
\pgfsetbuttcap%
\pgfsetroundjoin%
\definecolor{currentfill}{rgb}{0.000000,0.000000,0.000000}%
\pgfsetfillcolor{currentfill}%
\pgfsetlinewidth{0.602250pt}%
\definecolor{currentstroke}{rgb}{0.000000,0.000000,0.000000}%
\pgfsetstrokecolor{currentstroke}%
\pgfsetdash{}{0pt}%
\pgfsys@defobject{currentmarker}{\pgfqpoint{0.000000in}{-0.027778in}}{\pgfqpoint{0.000000in}{0.000000in}}{%
\pgfpathmoveto{\pgfqpoint{0.000000in}{0.000000in}}%
\pgfpathlineto{\pgfqpoint{0.000000in}{-0.027778in}}%
\pgfusepath{stroke,fill}%
}%
\begin{pgfscope}%
\pgfsys@transformshift{1.848036in}{0.549073in}%
\pgfsys@useobject{currentmarker}{}%
\end{pgfscope}%
\end{pgfscope}%
\begin{pgfscope}%
\pgfsetbuttcap%
\pgfsetroundjoin%
\definecolor{currentfill}{rgb}{0.000000,0.000000,0.000000}%
\pgfsetfillcolor{currentfill}%
\pgfsetlinewidth{0.602250pt}%
\definecolor{currentstroke}{rgb}{0.000000,0.000000,0.000000}%
\pgfsetstrokecolor{currentstroke}%
\pgfsetdash{}{0pt}%
\pgfsys@defobject{currentmarker}{\pgfqpoint{0.000000in}{-0.027778in}}{\pgfqpoint{0.000000in}{0.000000in}}{%
\pgfpathmoveto{\pgfqpoint{0.000000in}{0.000000in}}%
\pgfpathlineto{\pgfqpoint{0.000000in}{-0.027778in}}%
\pgfusepath{stroke,fill}%
}%
\begin{pgfscope}%
\pgfsys@transformshift{1.901135in}{0.549073in}%
\pgfsys@useobject{currentmarker}{}%
\end{pgfscope}%
\end{pgfscope}%
\begin{pgfscope}%
\pgfsetbuttcap%
\pgfsetroundjoin%
\definecolor{currentfill}{rgb}{0.000000,0.000000,0.000000}%
\pgfsetfillcolor{currentfill}%
\pgfsetlinewidth{0.602250pt}%
\definecolor{currentstroke}{rgb}{0.000000,0.000000,0.000000}%
\pgfsetstrokecolor{currentstroke}%
\pgfsetdash{}{0pt}%
\pgfsys@defobject{currentmarker}{\pgfqpoint{0.000000in}{-0.027778in}}{\pgfqpoint{0.000000in}{0.000000in}}{%
\pgfpathmoveto{\pgfqpoint{0.000000in}{0.000000in}}%
\pgfpathlineto{\pgfqpoint{0.000000in}{-0.027778in}}%
\pgfusepath{stroke,fill}%
}%
\begin{pgfscope}%
\pgfsys@transformshift{2.261120in}{0.549073in}%
\pgfsys@useobject{currentmarker}{}%
\end{pgfscope}%
\end{pgfscope}%
\begin{pgfscope}%
\pgfsetbuttcap%
\pgfsetroundjoin%
\definecolor{currentfill}{rgb}{0.000000,0.000000,0.000000}%
\pgfsetfillcolor{currentfill}%
\pgfsetlinewidth{0.602250pt}%
\definecolor{currentstroke}{rgb}{0.000000,0.000000,0.000000}%
\pgfsetstrokecolor{currentstroke}%
\pgfsetdash{}{0pt}%
\pgfsys@defobject{currentmarker}{\pgfqpoint{0.000000in}{-0.027778in}}{\pgfqpoint{0.000000in}{0.000000in}}{%
\pgfpathmoveto{\pgfqpoint{0.000000in}{0.000000in}}%
\pgfpathlineto{\pgfqpoint{0.000000in}{-0.027778in}}%
\pgfusepath{stroke,fill}%
}%
\begin{pgfscope}%
\pgfsys@transformshift{2.443912in}{0.549073in}%
\pgfsys@useobject{currentmarker}{}%
\end{pgfscope}%
\end{pgfscope}%
\begin{pgfscope}%
\pgfsetbuttcap%
\pgfsetroundjoin%
\definecolor{currentfill}{rgb}{0.000000,0.000000,0.000000}%
\pgfsetfillcolor{currentfill}%
\pgfsetlinewidth{0.602250pt}%
\definecolor{currentstroke}{rgb}{0.000000,0.000000,0.000000}%
\pgfsetstrokecolor{currentstroke}%
\pgfsetdash{}{0pt}%
\pgfsys@defobject{currentmarker}{\pgfqpoint{0.000000in}{-0.027778in}}{\pgfqpoint{0.000000in}{0.000000in}}{%
\pgfpathmoveto{\pgfqpoint{0.000000in}{0.000000in}}%
\pgfpathlineto{\pgfqpoint{0.000000in}{-0.027778in}}%
\pgfusepath{stroke,fill}%
}%
\begin{pgfscope}%
\pgfsys@transformshift{2.573605in}{0.549073in}%
\pgfsys@useobject{currentmarker}{}%
\end{pgfscope}%
\end{pgfscope}%
\begin{pgfscope}%
\definecolor{textcolor}{rgb}{0.000000,0.000000,0.000000}%
\pgfsetstrokecolor{textcolor}%
\pgfsetfillcolor{textcolor}%
\pgftext[x=1.690663in,y=0.248148in,,top]{\color{textcolor}{\rmfamily\fontsize{12.000000}{14.400000}\selectfont\catcode`\^=\active\def^{\ifmmode\sp\else\^{}\fi}\catcode`\%=\active\def%{\%}$n_{\Omega} + n_{\Psi}$}}%
\end{pgfscope}%
\begin{pgfscope}%
\pgfsetbuttcap%
\pgfsetroundjoin%
\definecolor{currentfill}{rgb}{0.000000,0.000000,0.000000}%
\pgfsetfillcolor{currentfill}%
\pgfsetlinewidth{0.803000pt}%
\definecolor{currentstroke}{rgb}{0.000000,0.000000,0.000000}%
\pgfsetstrokecolor{currentstroke}%
\pgfsetdash{}{0pt}%
\pgfsys@defobject{currentmarker}{\pgfqpoint{-0.048611in}{0.000000in}}{\pgfqpoint{-0.000000in}{0.000000in}}{%
\pgfpathmoveto{\pgfqpoint{-0.000000in}{0.000000in}}%
\pgfpathlineto{\pgfqpoint{-0.048611in}{0.000000in}}%
\pgfusepath{stroke,fill}%
}%
\begin{pgfscope}%
\pgfsys@transformshift{0.721913in}{1.510193in}%
\pgfsys@useobject{currentmarker}{}%
\end{pgfscope}%
\end{pgfscope}%
\begin{pgfscope}%
\definecolor{textcolor}{rgb}{0.000000,0.000000,0.000000}%
\pgfsetstrokecolor{textcolor}%
\pgfsetfillcolor{textcolor}%
\pgftext[x=0.303703in, y=1.452323in, left, base]{\color{textcolor}{\rmfamily\fontsize{12.000000}{14.400000}\selectfont\catcode`\^=\active\def^{\ifmmode\sp\else\^{}\fi}\catcode`\%=\active\def%{\%}$\mathdefault{10^{-1}}$}}%
\end{pgfscope}%
\begin{pgfscope}%
\pgfsetbuttcap%
\pgfsetroundjoin%
\definecolor{currentfill}{rgb}{0.000000,0.000000,0.000000}%
\pgfsetfillcolor{currentfill}%
\pgfsetlinewidth{0.602250pt}%
\definecolor{currentstroke}{rgb}{0.000000,0.000000,0.000000}%
\pgfsetstrokecolor{currentstroke}%
\pgfsetdash{}{0pt}%
\pgfsys@defobject{currentmarker}{\pgfqpoint{-0.027778in}{0.000000in}}{\pgfqpoint{-0.000000in}{0.000000in}}{%
\pgfpathmoveto{\pgfqpoint{-0.000000in}{0.000000in}}%
\pgfpathlineto{\pgfqpoint{-0.027778in}{0.000000in}}%
\pgfusepath{stroke,fill}%
}%
\begin{pgfscope}%
\pgfsys@transformshift{0.721913in}{0.657928in}%
\pgfsys@useobject{currentmarker}{}%
\end{pgfscope}%
\end{pgfscope}%
\begin{pgfscope}%
\pgfsetbuttcap%
\pgfsetroundjoin%
\definecolor{currentfill}{rgb}{0.000000,0.000000,0.000000}%
\pgfsetfillcolor{currentfill}%
\pgfsetlinewidth{0.602250pt}%
\definecolor{currentstroke}{rgb}{0.000000,0.000000,0.000000}%
\pgfsetstrokecolor{currentstroke}%
\pgfsetdash{}{0pt}%
\pgfsys@defobject{currentmarker}{\pgfqpoint{-0.027778in}{0.000000in}}{\pgfqpoint{-0.000000in}{0.000000in}}{%
\pgfpathmoveto{\pgfqpoint{-0.000000in}{0.000000in}}%
\pgfpathlineto{\pgfqpoint{-0.027778in}{0.000000in}}%
\pgfusepath{stroke,fill}%
}%
\begin{pgfscope}%
\pgfsys@transformshift{0.721913in}{0.872639in}%
\pgfsys@useobject{currentmarker}{}%
\end{pgfscope}%
\end{pgfscope}%
\begin{pgfscope}%
\pgfsetbuttcap%
\pgfsetroundjoin%
\definecolor{currentfill}{rgb}{0.000000,0.000000,0.000000}%
\pgfsetfillcolor{currentfill}%
\pgfsetlinewidth{0.602250pt}%
\definecolor{currentstroke}{rgb}{0.000000,0.000000,0.000000}%
\pgfsetstrokecolor{currentstroke}%
\pgfsetdash{}{0pt}%
\pgfsys@defobject{currentmarker}{\pgfqpoint{-0.027778in}{0.000000in}}{\pgfqpoint{-0.000000in}{0.000000in}}{%
\pgfpathmoveto{\pgfqpoint{-0.000000in}{0.000000in}}%
\pgfpathlineto{\pgfqpoint{-0.027778in}{0.000000in}}%
\pgfusepath{stroke,fill}%
}%
\begin{pgfscope}%
\pgfsys@transformshift{0.721913in}{1.024979in}%
\pgfsys@useobject{currentmarker}{}%
\end{pgfscope}%
\end{pgfscope}%
\begin{pgfscope}%
\pgfsetbuttcap%
\pgfsetroundjoin%
\definecolor{currentfill}{rgb}{0.000000,0.000000,0.000000}%
\pgfsetfillcolor{currentfill}%
\pgfsetlinewidth{0.602250pt}%
\definecolor{currentstroke}{rgb}{0.000000,0.000000,0.000000}%
\pgfsetstrokecolor{currentstroke}%
\pgfsetdash{}{0pt}%
\pgfsys@defobject{currentmarker}{\pgfqpoint{-0.027778in}{0.000000in}}{\pgfqpoint{-0.000000in}{0.000000in}}{%
\pgfpathmoveto{\pgfqpoint{-0.000000in}{0.000000in}}%
\pgfpathlineto{\pgfqpoint{-0.027778in}{0.000000in}}%
\pgfusepath{stroke,fill}%
}%
\begin{pgfscope}%
\pgfsys@transformshift{0.721913in}{1.143143in}%
\pgfsys@useobject{currentmarker}{}%
\end{pgfscope}%
\end{pgfscope}%
\begin{pgfscope}%
\pgfsetbuttcap%
\pgfsetroundjoin%
\definecolor{currentfill}{rgb}{0.000000,0.000000,0.000000}%
\pgfsetfillcolor{currentfill}%
\pgfsetlinewidth{0.602250pt}%
\definecolor{currentstroke}{rgb}{0.000000,0.000000,0.000000}%
\pgfsetstrokecolor{currentstroke}%
\pgfsetdash{}{0pt}%
\pgfsys@defobject{currentmarker}{\pgfqpoint{-0.027778in}{0.000000in}}{\pgfqpoint{-0.000000in}{0.000000in}}{%
\pgfpathmoveto{\pgfqpoint{-0.000000in}{0.000000in}}%
\pgfpathlineto{\pgfqpoint{-0.027778in}{0.000000in}}%
\pgfusepath{stroke,fill}%
}%
\begin{pgfscope}%
\pgfsys@transformshift{0.721913in}{1.239690in}%
\pgfsys@useobject{currentmarker}{}%
\end{pgfscope}%
\end{pgfscope}%
\begin{pgfscope}%
\pgfsetbuttcap%
\pgfsetroundjoin%
\definecolor{currentfill}{rgb}{0.000000,0.000000,0.000000}%
\pgfsetfillcolor{currentfill}%
\pgfsetlinewidth{0.602250pt}%
\definecolor{currentstroke}{rgb}{0.000000,0.000000,0.000000}%
\pgfsetstrokecolor{currentstroke}%
\pgfsetdash{}{0pt}%
\pgfsys@defobject{currentmarker}{\pgfqpoint{-0.027778in}{0.000000in}}{\pgfqpoint{-0.000000in}{0.000000in}}{%
\pgfpathmoveto{\pgfqpoint{-0.000000in}{0.000000in}}%
\pgfpathlineto{\pgfqpoint{-0.027778in}{0.000000in}}%
\pgfusepath{stroke,fill}%
}%
\begin{pgfscope}%
\pgfsys@transformshift{0.721913in}{1.321319in}%
\pgfsys@useobject{currentmarker}{}%
\end{pgfscope}%
\end{pgfscope}%
\begin{pgfscope}%
\pgfsetbuttcap%
\pgfsetroundjoin%
\definecolor{currentfill}{rgb}{0.000000,0.000000,0.000000}%
\pgfsetfillcolor{currentfill}%
\pgfsetlinewidth{0.602250pt}%
\definecolor{currentstroke}{rgb}{0.000000,0.000000,0.000000}%
\pgfsetstrokecolor{currentstroke}%
\pgfsetdash{}{0pt}%
\pgfsys@defobject{currentmarker}{\pgfqpoint{-0.027778in}{0.000000in}}{\pgfqpoint{-0.000000in}{0.000000in}}{%
\pgfpathmoveto{\pgfqpoint{-0.000000in}{0.000000in}}%
\pgfpathlineto{\pgfqpoint{-0.027778in}{0.000000in}}%
\pgfusepath{stroke,fill}%
}%
\begin{pgfscope}%
\pgfsys@transformshift{0.721913in}{1.392029in}%
\pgfsys@useobject{currentmarker}{}%
\end{pgfscope}%
\end{pgfscope}%
\begin{pgfscope}%
\pgfsetbuttcap%
\pgfsetroundjoin%
\definecolor{currentfill}{rgb}{0.000000,0.000000,0.000000}%
\pgfsetfillcolor{currentfill}%
\pgfsetlinewidth{0.602250pt}%
\definecolor{currentstroke}{rgb}{0.000000,0.000000,0.000000}%
\pgfsetstrokecolor{currentstroke}%
\pgfsetdash{}{0pt}%
\pgfsys@defobject{currentmarker}{\pgfqpoint{-0.027778in}{0.000000in}}{\pgfqpoint{-0.000000in}{0.000000in}}{%
\pgfpathmoveto{\pgfqpoint{-0.000000in}{0.000000in}}%
\pgfpathlineto{\pgfqpoint{-0.027778in}{0.000000in}}%
\pgfusepath{stroke,fill}%
}%
\begin{pgfscope}%
\pgfsys@transformshift{0.721913in}{1.454401in}%
\pgfsys@useobject{currentmarker}{}%
\end{pgfscope}%
\end{pgfscope}%
\begin{pgfscope}%
\pgfsetbuttcap%
\pgfsetroundjoin%
\definecolor{currentfill}{rgb}{0.000000,0.000000,0.000000}%
\pgfsetfillcolor{currentfill}%
\pgfsetlinewidth{0.602250pt}%
\definecolor{currentstroke}{rgb}{0.000000,0.000000,0.000000}%
\pgfsetstrokecolor{currentstroke}%
\pgfsetdash{}{0pt}%
\pgfsys@defobject{currentmarker}{\pgfqpoint{-0.027778in}{0.000000in}}{\pgfqpoint{-0.000000in}{0.000000in}}{%
\pgfpathmoveto{\pgfqpoint{-0.000000in}{0.000000in}}%
\pgfpathlineto{\pgfqpoint{-0.027778in}{0.000000in}}%
\pgfusepath{stroke,fill}%
}%
\begin{pgfscope}%
\pgfsys@transformshift{0.721913in}{1.877244in}%
\pgfsys@useobject{currentmarker}{}%
\end{pgfscope}%
\end{pgfscope}%
\begin{pgfscope}%
\pgfsetbuttcap%
\pgfsetroundjoin%
\definecolor{currentfill}{rgb}{0.000000,0.000000,0.000000}%
\pgfsetfillcolor{currentfill}%
\pgfsetlinewidth{0.602250pt}%
\definecolor{currentstroke}{rgb}{0.000000,0.000000,0.000000}%
\pgfsetstrokecolor{currentstroke}%
\pgfsetdash{}{0pt}%
\pgfsys@defobject{currentmarker}{\pgfqpoint{-0.027778in}{0.000000in}}{\pgfqpoint{-0.000000in}{0.000000in}}{%
\pgfpathmoveto{\pgfqpoint{-0.000000in}{0.000000in}}%
\pgfpathlineto{\pgfqpoint{-0.027778in}{0.000000in}}%
\pgfusepath{stroke,fill}%
}%
\begin{pgfscope}%
\pgfsys@transformshift{0.721913in}{2.091955in}%
\pgfsys@useobject{currentmarker}{}%
\end{pgfscope}%
\end{pgfscope}%
\begin{pgfscope}%
\pgfsetbuttcap%
\pgfsetroundjoin%
\definecolor{currentfill}{rgb}{0.000000,0.000000,0.000000}%
\pgfsetfillcolor{currentfill}%
\pgfsetlinewidth{0.602250pt}%
\definecolor{currentstroke}{rgb}{0.000000,0.000000,0.000000}%
\pgfsetstrokecolor{currentstroke}%
\pgfsetdash{}{0pt}%
\pgfsys@defobject{currentmarker}{\pgfqpoint{-0.027778in}{0.000000in}}{\pgfqpoint{-0.000000in}{0.000000in}}{%
\pgfpathmoveto{\pgfqpoint{-0.000000in}{0.000000in}}%
\pgfpathlineto{\pgfqpoint{-0.027778in}{0.000000in}}%
\pgfusepath{stroke,fill}%
}%
\begin{pgfscope}%
\pgfsys@transformshift{0.721913in}{2.244295in}%
\pgfsys@useobject{currentmarker}{}%
\end{pgfscope}%
\end{pgfscope}%
\begin{pgfscope}%
\pgfsetbuttcap%
\pgfsetroundjoin%
\definecolor{currentfill}{rgb}{0.000000,0.000000,0.000000}%
\pgfsetfillcolor{currentfill}%
\pgfsetlinewidth{0.602250pt}%
\definecolor{currentstroke}{rgb}{0.000000,0.000000,0.000000}%
\pgfsetstrokecolor{currentstroke}%
\pgfsetdash{}{0pt}%
\pgfsys@defobject{currentmarker}{\pgfqpoint{-0.027778in}{0.000000in}}{\pgfqpoint{-0.000000in}{0.000000in}}{%
\pgfpathmoveto{\pgfqpoint{-0.000000in}{0.000000in}}%
\pgfpathlineto{\pgfqpoint{-0.027778in}{0.000000in}}%
\pgfusepath{stroke,fill}%
}%
\begin{pgfscope}%
\pgfsys@transformshift{0.721913in}{2.362458in}%
\pgfsys@useobject{currentmarker}{}%
\end{pgfscope}%
\end{pgfscope}%
\begin{pgfscope}%
\pgfsetbuttcap%
\pgfsetroundjoin%
\definecolor{currentfill}{rgb}{0.000000,0.000000,0.000000}%
\pgfsetfillcolor{currentfill}%
\pgfsetlinewidth{0.602250pt}%
\definecolor{currentstroke}{rgb}{0.000000,0.000000,0.000000}%
\pgfsetstrokecolor{currentstroke}%
\pgfsetdash{}{0pt}%
\pgfsys@defobject{currentmarker}{\pgfqpoint{-0.027778in}{0.000000in}}{\pgfqpoint{-0.000000in}{0.000000in}}{%
\pgfpathmoveto{\pgfqpoint{-0.000000in}{0.000000in}}%
\pgfpathlineto{\pgfqpoint{-0.027778in}{0.000000in}}%
\pgfusepath{stroke,fill}%
}%
\begin{pgfscope}%
\pgfsys@transformshift{0.721913in}{2.459005in}%
\pgfsys@useobject{currentmarker}{}%
\end{pgfscope}%
\end{pgfscope}%
\begin{pgfscope}%
\definecolor{textcolor}{rgb}{0.000000,0.000000,0.000000}%
\pgfsetstrokecolor{textcolor}%
\pgfsetfillcolor{textcolor}%
\pgftext[x=0.248148in,y=1.511573in,,bottom,rotate=90.000000]{\color{textcolor}{\rmfamily\fontsize{12.000000}{14.400000}\selectfont\catcode`\^=\active\def^{\ifmmode\sp\else\^{}\fi}\catcode`\%=\active\def%{\%}$L^1$ relative error}}%
\end{pgfscope}%
\begin{pgfscope}%
\pgfpathrectangle{\pgfqpoint{0.721913in}{0.549073in}}{\pgfqpoint{1.937500in}{1.925000in}}%
\pgfusepath{clip}%
\pgfsetrectcap%
\pgfsetroundjoin%
\pgfsetlinewidth{1.003750pt}%
\definecolor{currentstroke}{rgb}{0.537255,0.647059,0.760784}%
\pgfsetstrokecolor{currentstroke}%
\pgfsetdash{}{0pt}%
\pgfpathmoveto{\pgfqpoint{0.809982in}{1.521477in}}%
\pgfpathlineto{\pgfqpoint{1.122467in}{1.387337in}}%
\pgfpathlineto{\pgfqpoint{1.420640in}{1.308277in}}%
\pgfpathlineto{\pgfqpoint{1.710766in}{1.191597in}}%
\pgfpathlineto{\pgfqpoint{1.999725in}{1.028173in}}%
\pgfpathlineto{\pgfqpoint{2.285257in}{0.918221in}}%
\pgfpathlineto{\pgfqpoint{2.571345in}{0.847886in}}%
\pgfusepath{stroke}%
\end{pgfscope}%
\begin{pgfscope}%
\pgfpathrectangle{\pgfqpoint{0.721913in}{0.549073in}}{\pgfqpoint{1.937500in}{1.925000in}}%
\pgfusepath{clip}%
\pgfsetbuttcap%
\pgfsetroundjoin%
\definecolor{currentfill}{rgb}{0.537255,0.647059,0.760784}%
\pgfsetfillcolor{currentfill}%
\pgfsetlinewidth{1.003750pt}%
\definecolor{currentstroke}{rgb}{0.537255,0.647059,0.760784}%
\pgfsetstrokecolor{currentstroke}%
\pgfsetdash{}{0pt}%
\pgfsys@defobject{currentmarker}{\pgfqpoint{-0.020833in}{-0.020833in}}{\pgfqpoint{0.020833in}{0.020833in}}{%
\pgfpathmoveto{\pgfqpoint{0.000000in}{-0.020833in}}%
\pgfpathcurveto{\pgfqpoint{0.005525in}{-0.020833in}}{\pgfqpoint{0.010825in}{-0.018638in}}{\pgfqpoint{0.014731in}{-0.014731in}}%
\pgfpathcurveto{\pgfqpoint{0.018638in}{-0.010825in}}{\pgfqpoint{0.020833in}{-0.005525in}}{\pgfqpoint{0.020833in}{0.000000in}}%
\pgfpathcurveto{\pgfqpoint{0.020833in}{0.005525in}}{\pgfqpoint{0.018638in}{0.010825in}}{\pgfqpoint{0.014731in}{0.014731in}}%
\pgfpathcurveto{\pgfqpoint{0.010825in}{0.018638in}}{\pgfqpoint{0.005525in}{0.020833in}}{\pgfqpoint{0.000000in}{0.020833in}}%
\pgfpathcurveto{\pgfqpoint{-0.005525in}{0.020833in}}{\pgfqpoint{-0.010825in}{0.018638in}}{\pgfqpoint{-0.014731in}{0.014731in}}%
\pgfpathcurveto{\pgfqpoint{-0.018638in}{0.010825in}}{\pgfqpoint{-0.020833in}{0.005525in}}{\pgfqpoint{-0.020833in}{0.000000in}}%
\pgfpathcurveto{\pgfqpoint{-0.020833in}{-0.005525in}}{\pgfqpoint{-0.018638in}{-0.010825in}}{\pgfqpoint{-0.014731in}{-0.014731in}}%
\pgfpathcurveto{\pgfqpoint{-0.010825in}{-0.018638in}}{\pgfqpoint{-0.005525in}{-0.020833in}}{\pgfqpoint{0.000000in}{-0.020833in}}%
\pgfpathlineto{\pgfqpoint{0.000000in}{-0.020833in}}%
\pgfpathclose%
\pgfusepath{stroke,fill}%
}%
\begin{pgfscope}%
\pgfsys@transformshift{0.809982in}{1.521477in}%
\pgfsys@useobject{currentmarker}{}%
\end{pgfscope}%
\begin{pgfscope}%
\pgfsys@transformshift{1.122467in}{1.387337in}%
\pgfsys@useobject{currentmarker}{}%
\end{pgfscope}%
\begin{pgfscope}%
\pgfsys@transformshift{1.420640in}{1.308277in}%
\pgfsys@useobject{currentmarker}{}%
\end{pgfscope}%
\begin{pgfscope}%
\pgfsys@transformshift{1.710766in}{1.191597in}%
\pgfsys@useobject{currentmarker}{}%
\end{pgfscope}%
\begin{pgfscope}%
\pgfsys@transformshift{1.999725in}{1.028173in}%
\pgfsys@useobject{currentmarker}{}%
\end{pgfscope}%
\begin{pgfscope}%
\pgfsys@transformshift{2.285257in}{0.918221in}%
\pgfsys@useobject{currentmarker}{}%
\end{pgfscope}%
\begin{pgfscope}%
\pgfsys@transformshift{2.571345in}{0.847886in}%
\pgfsys@useobject{currentmarker}{}%
\end{pgfscope}%
\end{pgfscope}%
\begin{pgfscope}%
\pgfpathrectangle{\pgfqpoint{0.721913in}{0.549073in}}{\pgfqpoint{1.937500in}{1.925000in}}%
\pgfusepath{clip}%
\pgfsetrectcap%
\pgfsetroundjoin%
\pgfsetlinewidth{1.003750pt}%
\definecolor{currentstroke}{rgb}{0.184314,0.270588,0.360784}%
\pgfsetstrokecolor{currentstroke}%
\pgfsetdash{}{0pt}%
\pgfpathmoveto{\pgfqpoint{0.809982in}{2.386573in}}%
\pgfpathlineto{\pgfqpoint{1.122467in}{2.136759in}}%
\pgfpathlineto{\pgfqpoint{1.420640in}{1.797616in}}%
\pgfpathlineto{\pgfqpoint{1.710766in}{1.442523in}}%
\pgfpathlineto{\pgfqpoint{1.999725in}{1.412894in}}%
\pgfpathlineto{\pgfqpoint{2.285257in}{1.376396in}}%
\pgfpathlineto{\pgfqpoint{2.571345in}{1.466256in}}%
\pgfusepath{stroke}%
\end{pgfscope}%
\begin{pgfscope}%
\pgfpathrectangle{\pgfqpoint{0.721913in}{0.549073in}}{\pgfqpoint{1.937500in}{1.925000in}}%
\pgfusepath{clip}%
\pgfsetbuttcap%
\pgfsetroundjoin%
\definecolor{currentfill}{rgb}{0.184314,0.270588,0.360784}%
\pgfsetfillcolor{currentfill}%
\pgfsetlinewidth{1.003750pt}%
\definecolor{currentstroke}{rgb}{0.184314,0.270588,0.360784}%
\pgfsetstrokecolor{currentstroke}%
\pgfsetdash{}{0pt}%
\pgfsys@defobject{currentmarker}{\pgfqpoint{-0.020833in}{-0.020833in}}{\pgfqpoint{0.020833in}{0.020833in}}{%
\pgfpathmoveto{\pgfqpoint{0.000000in}{-0.020833in}}%
\pgfpathcurveto{\pgfqpoint{0.005525in}{-0.020833in}}{\pgfqpoint{0.010825in}{-0.018638in}}{\pgfqpoint{0.014731in}{-0.014731in}}%
\pgfpathcurveto{\pgfqpoint{0.018638in}{-0.010825in}}{\pgfqpoint{0.020833in}{-0.005525in}}{\pgfqpoint{0.020833in}{0.000000in}}%
\pgfpathcurveto{\pgfqpoint{0.020833in}{0.005525in}}{\pgfqpoint{0.018638in}{0.010825in}}{\pgfqpoint{0.014731in}{0.014731in}}%
\pgfpathcurveto{\pgfqpoint{0.010825in}{0.018638in}}{\pgfqpoint{0.005525in}{0.020833in}}{\pgfqpoint{0.000000in}{0.020833in}}%
\pgfpathcurveto{\pgfqpoint{-0.005525in}{0.020833in}}{\pgfqpoint{-0.010825in}{0.018638in}}{\pgfqpoint{-0.014731in}{0.014731in}}%
\pgfpathcurveto{\pgfqpoint{-0.018638in}{0.010825in}}{\pgfqpoint{-0.020833in}{0.005525in}}{\pgfqpoint{-0.020833in}{0.000000in}}%
\pgfpathcurveto{\pgfqpoint{-0.020833in}{-0.005525in}}{\pgfqpoint{-0.018638in}{-0.010825in}}{\pgfqpoint{-0.014731in}{-0.014731in}}%
\pgfpathcurveto{\pgfqpoint{-0.010825in}{-0.018638in}}{\pgfqpoint{-0.005525in}{-0.020833in}}{\pgfqpoint{0.000000in}{-0.020833in}}%
\pgfpathlineto{\pgfqpoint{0.000000in}{-0.020833in}}%
\pgfpathclose%
\pgfusepath{stroke,fill}%
}%
\begin{pgfscope}%
\pgfsys@transformshift{0.809982in}{2.386573in}%
\pgfsys@useobject{currentmarker}{}%
\end{pgfscope}%
\begin{pgfscope}%
\pgfsys@transformshift{1.122467in}{2.136759in}%
\pgfsys@useobject{currentmarker}{}%
\end{pgfscope}%
\begin{pgfscope}%
\pgfsys@transformshift{1.420640in}{1.797616in}%
\pgfsys@useobject{currentmarker}{}%
\end{pgfscope}%
\begin{pgfscope}%
\pgfsys@transformshift{1.710766in}{1.442523in}%
\pgfsys@useobject{currentmarker}{}%
\end{pgfscope}%
\begin{pgfscope}%
\pgfsys@transformshift{1.999725in}{1.412894in}%
\pgfsys@useobject{currentmarker}{}%
\end{pgfscope}%
\begin{pgfscope}%
\pgfsys@transformshift{2.285257in}{1.376396in}%
\pgfsys@useobject{currentmarker}{}%
\end{pgfscope}%
\begin{pgfscope}%
\pgfsys@transformshift{2.571345in}{1.466256in}%
\pgfsys@useobject{currentmarker}{}%
\end{pgfscope}%
\end{pgfscope}%
\begin{pgfscope}%
\pgfpathrectangle{\pgfqpoint{0.721913in}{0.549073in}}{\pgfqpoint{1.937500in}{1.925000in}}%
\pgfusepath{clip}%
\pgfsetrectcap%
\pgfsetroundjoin%
\pgfsetlinewidth{1.003750pt}%
\definecolor{currentstroke}{rgb}{0.976471,0.505882,0.145098}%
\pgfsetstrokecolor{currentstroke}%
\pgfsetdash{}{0pt}%
\pgfpathmoveto{\pgfqpoint{0.809982in}{1.602731in}}%
\pgfpathlineto{\pgfqpoint{1.122467in}{1.219095in}}%
\pgfpathlineto{\pgfqpoint{1.420640in}{1.025863in}}%
\pgfpathlineto{\pgfqpoint{1.710766in}{0.933165in}}%
\pgfpathlineto{\pgfqpoint{1.999725in}{0.636573in}}%
\pgfpathlineto{\pgfqpoint{2.285257in}{0.706162in}}%
\pgfpathlineto{\pgfqpoint{2.571345in}{0.735778in}}%
\pgfusepath{stroke}%
\end{pgfscope}%
\begin{pgfscope}%
\pgfpathrectangle{\pgfqpoint{0.721913in}{0.549073in}}{\pgfqpoint{1.937500in}{1.925000in}}%
\pgfusepath{clip}%
\pgfsetbuttcap%
\pgfsetroundjoin%
\definecolor{currentfill}{rgb}{0.976471,0.505882,0.145098}%
\pgfsetfillcolor{currentfill}%
\pgfsetlinewidth{1.003750pt}%
\definecolor{currentstroke}{rgb}{0.976471,0.505882,0.145098}%
\pgfsetstrokecolor{currentstroke}%
\pgfsetdash{}{0pt}%
\pgfsys@defobject{currentmarker}{\pgfqpoint{-0.020833in}{-0.020833in}}{\pgfqpoint{0.020833in}{0.020833in}}{%
\pgfpathmoveto{\pgfqpoint{0.000000in}{-0.020833in}}%
\pgfpathcurveto{\pgfqpoint{0.005525in}{-0.020833in}}{\pgfqpoint{0.010825in}{-0.018638in}}{\pgfqpoint{0.014731in}{-0.014731in}}%
\pgfpathcurveto{\pgfqpoint{0.018638in}{-0.010825in}}{\pgfqpoint{0.020833in}{-0.005525in}}{\pgfqpoint{0.020833in}{0.000000in}}%
\pgfpathcurveto{\pgfqpoint{0.020833in}{0.005525in}}{\pgfqpoint{0.018638in}{0.010825in}}{\pgfqpoint{0.014731in}{0.014731in}}%
\pgfpathcurveto{\pgfqpoint{0.010825in}{0.018638in}}{\pgfqpoint{0.005525in}{0.020833in}}{\pgfqpoint{0.000000in}{0.020833in}}%
\pgfpathcurveto{\pgfqpoint{-0.005525in}{0.020833in}}{\pgfqpoint{-0.010825in}{0.018638in}}{\pgfqpoint{-0.014731in}{0.014731in}}%
\pgfpathcurveto{\pgfqpoint{-0.018638in}{0.010825in}}{\pgfqpoint{-0.020833in}{0.005525in}}{\pgfqpoint{-0.020833in}{0.000000in}}%
\pgfpathcurveto{\pgfqpoint{-0.020833in}{-0.005525in}}{\pgfqpoint{-0.018638in}{-0.010825in}}{\pgfqpoint{-0.014731in}{-0.014731in}}%
\pgfpathcurveto{\pgfqpoint{-0.010825in}{-0.018638in}}{\pgfqpoint{-0.005525in}{-0.020833in}}{\pgfqpoint{0.000000in}{-0.020833in}}%
\pgfpathlineto{\pgfqpoint{0.000000in}{-0.020833in}}%
\pgfpathclose%
\pgfusepath{stroke,fill}%
}%
\begin{pgfscope}%
\pgfsys@transformshift{0.809982in}{1.602731in}%
\pgfsys@useobject{currentmarker}{}%
\end{pgfscope}%
\begin{pgfscope}%
\pgfsys@transformshift{1.122467in}{1.219095in}%
\pgfsys@useobject{currentmarker}{}%
\end{pgfscope}%
\begin{pgfscope}%
\pgfsys@transformshift{1.420640in}{1.025863in}%
\pgfsys@useobject{currentmarker}{}%
\end{pgfscope}%
\begin{pgfscope}%
\pgfsys@transformshift{1.710766in}{0.933165in}%
\pgfsys@useobject{currentmarker}{}%
\end{pgfscope}%
\begin{pgfscope}%
\pgfsys@transformshift{1.999725in}{0.636573in}%
\pgfsys@useobject{currentmarker}{}%
\end{pgfscope}%
\begin{pgfscope}%
\pgfsys@transformshift{2.285257in}{0.706162in}%
\pgfsys@useobject{currentmarker}{}%
\end{pgfscope}%
\begin{pgfscope}%
\pgfsys@transformshift{2.571345in}{0.735778in}%
\pgfsys@useobject{currentmarker}{}%
\end{pgfscope}%
\end{pgfscope}%
\begin{pgfscope}%
\pgfsetrectcap%
\pgfsetmiterjoin%
\pgfsetlinewidth{0.803000pt}%
\definecolor{currentstroke}{rgb}{0.000000,0.000000,0.000000}%
\pgfsetstrokecolor{currentstroke}%
\pgfsetdash{}{0pt}%
\pgfpathmoveto{\pgfqpoint{0.721913in}{0.549073in}}%
\pgfpathlineto{\pgfqpoint{0.721913in}{2.474073in}}%
\pgfusepath{stroke}%
\end{pgfscope}%
\begin{pgfscope}%
\pgfsetrectcap%
\pgfsetmiterjoin%
\pgfsetlinewidth{0.803000pt}%
\definecolor{currentstroke}{rgb}{0.000000,0.000000,0.000000}%
\pgfsetstrokecolor{currentstroke}%
\pgfsetdash{}{0pt}%
\pgfpathmoveto{\pgfqpoint{2.659413in}{0.549073in}}%
\pgfpathlineto{\pgfqpoint{2.659413in}{2.474073in}}%
\pgfusepath{stroke}%
\end{pgfscope}%
\begin{pgfscope}%
\pgfsetrectcap%
\pgfsetmiterjoin%
\pgfsetlinewidth{0.803000pt}%
\definecolor{currentstroke}{rgb}{0.000000,0.000000,0.000000}%
\pgfsetstrokecolor{currentstroke}%
\pgfsetdash{}{0pt}%
\pgfpathmoveto{\pgfqpoint{0.721913in}{0.549073in}}%
\pgfpathlineto{\pgfqpoint{2.659413in}{0.549073in}}%
\pgfusepath{stroke}%
\end{pgfscope}%
\begin{pgfscope}%
\pgfsetrectcap%
\pgfsetmiterjoin%
\pgfsetlinewidth{0.803000pt}%
\definecolor{currentstroke}{rgb}{0.000000,0.000000,0.000000}%
\pgfsetstrokecolor{currentstroke}%
\pgfsetdash{}{0pt}%
\pgfpathmoveto{\pgfqpoint{0.721913in}{2.474073in}}%
\pgfpathlineto{\pgfqpoint{2.659413in}{2.474073in}}%
\pgfusepath{stroke}%
\end{pgfscope}%
\begin{pgfscope}%
\pgfsetbuttcap%
\pgfsetmiterjoin%
\definecolor{currentfill}{rgb}{1.000000,1.000000,1.000000}%
\pgfsetfillcolor{currentfill}%
\pgfsetfillopacity{0.800000}%
\pgfsetlinewidth{1.003750pt}%
\definecolor{currentstroke}{rgb}{0.800000,0.800000,0.800000}%
\pgfsetstrokecolor{currentstroke}%
\pgfsetstrokeopacity{0.800000}%
\pgfsetdash{}{0pt}%
\pgfpathmoveto{\pgfqpoint{1.515360in}{1.643518in}}%
\pgfpathlineto{\pgfqpoint{2.542747in}{1.643518in}}%
\pgfpathquadraticcurveto{\pgfqpoint{2.576080in}{1.643518in}}{\pgfqpoint{2.576080in}{1.676852in}}%
\pgfpathlineto{\pgfqpoint{2.576080in}{2.357406in}}%
\pgfpathquadraticcurveto{\pgfqpoint{2.576080in}{2.390739in}}{\pgfqpoint{2.542747in}{2.390739in}}%
\pgfpathlineto{\pgfqpoint{1.515360in}{2.390739in}}%
\pgfpathquadraticcurveto{\pgfqpoint{1.482027in}{2.390739in}}{\pgfqpoint{1.482027in}{2.357406in}}%
\pgfpathlineto{\pgfqpoint{1.482027in}{1.676852in}}%
\pgfpathquadraticcurveto{\pgfqpoint{1.482027in}{1.643518in}}{\pgfqpoint{1.515360in}{1.643518in}}%
\pgfpathlineto{\pgfqpoint{1.515360in}{1.643518in}}%
\pgfpathclose%
\pgfusepath{stroke,fill}%
\end{pgfscope}%
\begin{pgfscope}%
\pgfsetrectcap%
\pgfsetroundjoin%
\pgfsetlinewidth{1.003750pt}%
\definecolor{currentstroke}{rgb}{0.537255,0.647059,0.760784}%
\pgfsetstrokecolor{currentstroke}%
\pgfsetdash{}{0pt}%
\pgfpathmoveto{\pgfqpoint{1.548693in}{2.265739in}}%
\pgfpathlineto{\pgfqpoint{1.715360in}{2.265739in}}%
\pgfpathlineto{\pgfqpoint{1.882027in}{2.265739in}}%
\pgfusepath{stroke}%
\end{pgfscope}%
\begin{pgfscope}%
\pgfsetbuttcap%
\pgfsetroundjoin%
\definecolor{currentfill}{rgb}{0.537255,0.647059,0.760784}%
\pgfsetfillcolor{currentfill}%
\pgfsetlinewidth{1.003750pt}%
\definecolor{currentstroke}{rgb}{0.537255,0.647059,0.760784}%
\pgfsetstrokecolor{currentstroke}%
\pgfsetdash{}{0pt}%
\pgfsys@defobject{currentmarker}{\pgfqpoint{-0.020833in}{-0.020833in}}{\pgfqpoint{0.020833in}{0.020833in}}{%
\pgfpathmoveto{\pgfqpoint{0.000000in}{-0.020833in}}%
\pgfpathcurveto{\pgfqpoint{0.005525in}{-0.020833in}}{\pgfqpoint{0.010825in}{-0.018638in}}{\pgfqpoint{0.014731in}{-0.014731in}}%
\pgfpathcurveto{\pgfqpoint{0.018638in}{-0.010825in}}{\pgfqpoint{0.020833in}{-0.005525in}}{\pgfqpoint{0.020833in}{0.000000in}}%
\pgfpathcurveto{\pgfqpoint{0.020833in}{0.005525in}}{\pgfqpoint{0.018638in}{0.010825in}}{\pgfqpoint{0.014731in}{0.014731in}}%
\pgfpathcurveto{\pgfqpoint{0.010825in}{0.018638in}}{\pgfqpoint{0.005525in}{0.020833in}}{\pgfqpoint{0.000000in}{0.020833in}}%
\pgfpathcurveto{\pgfqpoint{-0.005525in}{0.020833in}}{\pgfqpoint{-0.010825in}{0.018638in}}{\pgfqpoint{-0.014731in}{0.014731in}}%
\pgfpathcurveto{\pgfqpoint{-0.018638in}{0.010825in}}{\pgfqpoint{-0.020833in}{0.005525in}}{\pgfqpoint{-0.020833in}{0.000000in}}%
\pgfpathcurveto{\pgfqpoint{-0.020833in}{-0.005525in}}{\pgfqpoint{-0.018638in}{-0.010825in}}{\pgfqpoint{-0.014731in}{-0.014731in}}%
\pgfpathcurveto{\pgfqpoint{-0.010825in}{-0.018638in}}{\pgfqpoint{-0.005525in}{-0.020833in}}{\pgfqpoint{0.000000in}{-0.020833in}}%
\pgfpathlineto{\pgfqpoint{0.000000in}{-0.020833in}}%
\pgfpathclose%
\pgfusepath{stroke,fill}%
}%
\begin{pgfscope}%
\pgfsys@transformshift{1.715360in}{2.265739in}%
\pgfsys@useobject{currentmarker}{}%
\end{pgfscope}%
\end{pgfscope}%
\begin{pgfscope}%
\definecolor{textcolor}{rgb}{0.000000,0.000000,0.000000}%
\pgfsetstrokecolor{textcolor}%
\pgfsetfillcolor{textcolor}%
\pgftext[x=2.015360in,y=2.207406in,left,base]{\color{textcolor}{\rmfamily\fontsize{12.000000}{14.400000}\selectfont\catcode`\^=\active\def^{\ifmmode\sp\else\^{}\fi}\catcode`\%=\active\def%{\%}DGC}}%
\end{pgfscope}%
\begin{pgfscope}%
\pgfsetrectcap%
\pgfsetroundjoin%
\pgfsetlinewidth{1.003750pt}%
\definecolor{currentstroke}{rgb}{0.184314,0.270588,0.360784}%
\pgfsetstrokecolor{currentstroke}%
\pgfsetdash{}{0pt}%
\pgfpathmoveto{\pgfqpoint{1.548693in}{2.033332in}}%
\pgfpathlineto{\pgfqpoint{1.715360in}{2.033332in}}%
\pgfpathlineto{\pgfqpoint{1.882027in}{2.033332in}}%
\pgfusepath{stroke}%
\end{pgfscope}%
\begin{pgfscope}%
\pgfsetbuttcap%
\pgfsetroundjoin%
\definecolor{currentfill}{rgb}{0.184314,0.270588,0.360784}%
\pgfsetfillcolor{currentfill}%
\pgfsetlinewidth{1.003750pt}%
\definecolor{currentstroke}{rgb}{0.184314,0.270588,0.360784}%
\pgfsetstrokecolor{currentstroke}%
\pgfsetdash{}{0pt}%
\pgfsys@defobject{currentmarker}{\pgfqpoint{-0.020833in}{-0.020833in}}{\pgfqpoint{0.020833in}{0.020833in}}{%
\pgfpathmoveto{\pgfqpoint{0.000000in}{-0.020833in}}%
\pgfpathcurveto{\pgfqpoint{0.005525in}{-0.020833in}}{\pgfqpoint{0.010825in}{-0.018638in}}{\pgfqpoint{0.014731in}{-0.014731in}}%
\pgfpathcurveto{\pgfqpoint{0.018638in}{-0.010825in}}{\pgfqpoint{0.020833in}{-0.005525in}}{\pgfqpoint{0.020833in}{0.000000in}}%
\pgfpathcurveto{\pgfqpoint{0.020833in}{0.005525in}}{\pgfqpoint{0.018638in}{0.010825in}}{\pgfqpoint{0.014731in}{0.014731in}}%
\pgfpathcurveto{\pgfqpoint{0.010825in}{0.018638in}}{\pgfqpoint{0.005525in}{0.020833in}}{\pgfqpoint{0.000000in}{0.020833in}}%
\pgfpathcurveto{\pgfqpoint{-0.005525in}{0.020833in}}{\pgfqpoint{-0.010825in}{0.018638in}}{\pgfqpoint{-0.014731in}{0.014731in}}%
\pgfpathcurveto{\pgfqpoint{-0.018638in}{0.010825in}}{\pgfqpoint{-0.020833in}{0.005525in}}{\pgfqpoint{-0.020833in}{0.000000in}}%
\pgfpathcurveto{\pgfqpoint{-0.020833in}{-0.005525in}}{\pgfqpoint{-0.018638in}{-0.010825in}}{\pgfqpoint{-0.014731in}{-0.014731in}}%
\pgfpathcurveto{\pgfqpoint{-0.010825in}{-0.018638in}}{\pgfqpoint{-0.005525in}{-0.020833in}}{\pgfqpoint{0.000000in}{-0.020833in}}%
\pgfpathlineto{\pgfqpoint{0.000000in}{-0.020833in}}%
\pgfpathclose%
\pgfusepath{stroke,fill}%
}%
\begin{pgfscope}%
\pgfsys@transformshift{1.715360in}{2.033332in}%
\pgfsys@useobject{currentmarker}{}%
\end{pgfscope}%
\end{pgfscope}%
\begin{pgfscope}%
\definecolor{textcolor}{rgb}{0.000000,0.000000,0.000000}%
\pgfsetstrokecolor{textcolor}%
\pgfsetfillcolor{textcolor}%
\pgftext[x=2.015360in,y=1.974999in,left,base]{\color{textcolor}{\rmfamily\fontsize{12.000000}{14.400000}\selectfont\catcode`\^=\active\def^{\ifmmode\sp\else\^{}\fi}\catcode`\%=\active\def%{\%}NC}}%
\end{pgfscope}%
\begin{pgfscope}%
\pgfsetrectcap%
\pgfsetroundjoin%
\pgfsetlinewidth{1.003750pt}%
\definecolor{currentstroke}{rgb}{0.976471,0.505882,0.145098}%
\pgfsetstrokecolor{currentstroke}%
\pgfsetdash{}{0pt}%
\pgfpathmoveto{\pgfqpoint{1.548693in}{1.800925in}}%
\pgfpathlineto{\pgfqpoint{1.715360in}{1.800925in}}%
\pgfpathlineto{\pgfqpoint{1.882027in}{1.800925in}}%
\pgfusepath{stroke}%
\end{pgfscope}%
\begin{pgfscope}%
\pgfsetbuttcap%
\pgfsetroundjoin%
\definecolor{currentfill}{rgb}{0.976471,0.505882,0.145098}%
\pgfsetfillcolor{currentfill}%
\pgfsetlinewidth{1.003750pt}%
\definecolor{currentstroke}{rgb}{0.976471,0.505882,0.145098}%
\pgfsetstrokecolor{currentstroke}%
\pgfsetdash{}{0pt}%
\pgfsys@defobject{currentmarker}{\pgfqpoint{-0.020833in}{-0.020833in}}{\pgfqpoint{0.020833in}{0.020833in}}{%
\pgfpathmoveto{\pgfqpoint{0.000000in}{-0.020833in}}%
\pgfpathcurveto{\pgfqpoint{0.005525in}{-0.020833in}}{\pgfqpoint{0.010825in}{-0.018638in}}{\pgfqpoint{0.014731in}{-0.014731in}}%
\pgfpathcurveto{\pgfqpoint{0.018638in}{-0.010825in}}{\pgfqpoint{0.020833in}{-0.005525in}}{\pgfqpoint{0.020833in}{0.000000in}}%
\pgfpathcurveto{\pgfqpoint{0.020833in}{0.005525in}}{\pgfqpoint{0.018638in}{0.010825in}}{\pgfqpoint{0.014731in}{0.014731in}}%
\pgfpathcurveto{\pgfqpoint{0.010825in}{0.018638in}}{\pgfqpoint{0.005525in}{0.020833in}}{\pgfqpoint{0.000000in}{0.020833in}}%
\pgfpathcurveto{\pgfqpoint{-0.005525in}{0.020833in}}{\pgfqpoint{-0.010825in}{0.018638in}}{\pgfqpoint{-0.014731in}{0.014731in}}%
\pgfpathcurveto{\pgfqpoint{-0.018638in}{0.010825in}}{\pgfqpoint{-0.020833in}{0.005525in}}{\pgfqpoint{-0.020833in}{0.000000in}}%
\pgfpathcurveto{\pgfqpoint{-0.020833in}{-0.005525in}}{\pgfqpoint{-0.018638in}{-0.010825in}}{\pgfqpoint{-0.014731in}{-0.014731in}}%
\pgfpathcurveto{\pgfqpoint{-0.010825in}{-0.018638in}}{\pgfqpoint{-0.005525in}{-0.020833in}}{\pgfqpoint{0.000000in}{-0.020833in}}%
\pgfpathlineto{\pgfqpoint{0.000000in}{-0.020833in}}%
\pgfpathclose%
\pgfusepath{stroke,fill}%
}%
\begin{pgfscope}%
\pgfsys@transformshift{1.715360in}{1.800925in}%
\pgfsys@useobject{currentmarker}{}%
\end{pgfscope}%
\end{pgfscope}%
\begin{pgfscope}%
\definecolor{textcolor}{rgb}{0.000000,0.000000,0.000000}%
\pgfsetstrokecolor{textcolor}%
\pgfsetfillcolor{textcolor}%
\pgftext[x=2.015360in,y=1.742592in,left,base]{\color{textcolor}{\rmfamily\fontsize{12.000000}{14.400000}\selectfont\catcode`\^=\active\def^{\ifmmode\sp\else\^{}\fi}\catcode`\%=\active\def%{\%}NC++}}%
\end{pgfscope}%
\end{pgfpicture}%
\makeatother%
\endgroup%

        \caption{\gls{chebyshev-degree} $=800$}
        \label{fig:5-experiments-electronic-structure-convergence-nv-m800}
    \end{subfigure}
    \begin{subfigure}[b]{0.49\columnwidth}
        %% Creator: Matplotlib, PGF backend
%%
%% To include the figure in your LaTeX document, write
%%   \input{<filename>.pgf}
%%
%% Make sure the required packages are loaded in your preamble
%%   \usepackage{pgf}
%%
%% Also ensure that all the required font packages are loaded; for instance,
%% the lmodern package is sometimes necessary when using math font.
%%   \usepackage{lmodern}
%%
%% Figures using additional raster images can only be included by \input if
%% they are in the same directory as the main LaTeX file. For loading figures
%% from other directories you can use the `import` package
%%   \usepackage{import}
%%
%% and then include the figures with
%%   \import{<path to file>}{<filename>.pgf}
%%
%% Matplotlib used the following preamble
%%   \def\mathdefault#1{#1}
%%   \everymath=\expandafter{\the\everymath\displaystyle}
%%   
%%   \usepackage{fontspec}
%%   \setmainfont{DejaVuSerif.ttf}[Path=\detokenize{C:/Users/fabio/Documents/Work/MasterThesis/Rand-SD/.venv/Lib/site-packages/matplotlib/mpl-data/fonts/ttf/}]
%%   \setsansfont{DejaVuSans.ttf}[Path=\detokenize{C:/Users/fabio/Documents/Work/MasterThesis/Rand-SD/.venv/Lib/site-packages/matplotlib/mpl-data/fonts/ttf/}]
%%   \setmonofont{DejaVuSansMono.ttf}[Path=\detokenize{C:/Users/fabio/Documents/Work/MasterThesis/Rand-SD/.venv/Lib/site-packages/matplotlib/mpl-data/fonts/ttf/}]
%%   \makeatletter\@ifpackageloaded{underscore}{}{\usepackage[strings]{underscore}}\makeatother
%%
\begingroup%
\makeatletter%
\begin{pgfpicture}%
\pgfpathrectangle{\pgfpointorigin}{\pgfqpoint{2.712693in}{2.546603in}}%
\pgfusepath{use as bounding box, clip}%
\begin{pgfscope}%
\pgfsetbuttcap%
\pgfsetmiterjoin%
\definecolor{currentfill}{rgb}{1.000000,1.000000,1.000000}%
\pgfsetfillcolor{currentfill}%
\pgfsetlinewidth{0.000000pt}%
\definecolor{currentstroke}{rgb}{1.000000,1.000000,1.000000}%
\pgfsetstrokecolor{currentstroke}%
\pgfsetdash{}{0pt}%
\pgfpathmoveto{\pgfqpoint{0.000000in}{0.000000in}}%
\pgfpathlineto{\pgfqpoint{2.712693in}{0.000000in}}%
\pgfpathlineto{\pgfqpoint{2.712693in}{2.546603in}}%
\pgfpathlineto{\pgfqpoint{0.000000in}{2.546603in}}%
\pgfpathlineto{\pgfqpoint{0.000000in}{0.000000in}}%
\pgfpathclose%
\pgfusepath{fill}%
\end{pgfscope}%
\begin{pgfscope}%
\pgfsetbuttcap%
\pgfsetmiterjoin%
\definecolor{currentfill}{rgb}{1.000000,1.000000,1.000000}%
\pgfsetfillcolor{currentfill}%
\pgfsetlinewidth{0.000000pt}%
\definecolor{currentstroke}{rgb}{0.000000,0.000000,0.000000}%
\pgfsetstrokecolor{currentstroke}%
\pgfsetstrokeopacity{0.000000}%
\pgfsetdash{}{0pt}%
\pgfpathmoveto{\pgfqpoint{0.675193in}{0.521603in}}%
\pgfpathlineto{\pgfqpoint{2.612693in}{0.521603in}}%
\pgfpathlineto{\pgfqpoint{2.612693in}{2.446603in}}%
\pgfpathlineto{\pgfqpoint{0.675193in}{2.446603in}}%
\pgfpathlineto{\pgfqpoint{0.675193in}{0.521603in}}%
\pgfpathclose%
\pgfusepath{fill}%
\end{pgfscope}%
\begin{pgfscope}%
\pgfsetbuttcap%
\pgfsetroundjoin%
\definecolor{currentfill}{rgb}{0.000000,0.000000,0.000000}%
\pgfsetfillcolor{currentfill}%
\pgfsetlinewidth{0.803000pt}%
\definecolor{currentstroke}{rgb}{0.000000,0.000000,0.000000}%
\pgfsetstrokecolor{currentstroke}%
\pgfsetdash{}{0pt}%
\pgfsys@defobject{currentmarker}{\pgfqpoint{0.000000in}{-0.048611in}}{\pgfqpoint{0.000000in}{0.000000in}}{%
\pgfpathmoveto{\pgfqpoint{0.000000in}{0.000000in}}%
\pgfpathlineto{\pgfqpoint{0.000000in}{-0.048611in}}%
\pgfusepath{stroke,fill}%
}%
\begin{pgfscope}%
\pgfsys@transformshift{1.724845in}{0.521603in}%
\pgfsys@useobject{currentmarker}{}%
\end{pgfscope}%
\end{pgfscope}%
\begin{pgfscope}%
\definecolor{textcolor}{rgb}{0.000000,0.000000,0.000000}%
\pgfsetstrokecolor{textcolor}%
\pgfsetfillcolor{textcolor}%
\pgftext[x=1.724845in,y=0.424381in,,top]{\color{textcolor}{\sffamily\fontsize{10.000000}{12.000000}\selectfont\catcode`\^=\active\def^{\ifmmode\sp\else\^{}\fi}\catcode`\%=\active\def%{\%}$\mathdefault{10^{2}}$}}%
\end{pgfscope}%
\begin{pgfscope}%
\pgfsetbuttcap%
\pgfsetroundjoin%
\definecolor{currentfill}{rgb}{0.000000,0.000000,0.000000}%
\pgfsetfillcolor{currentfill}%
\pgfsetlinewidth{0.602250pt}%
\definecolor{currentstroke}{rgb}{0.000000,0.000000,0.000000}%
\pgfsetstrokecolor{currentstroke}%
\pgfsetdash{}{0pt}%
\pgfsys@defobject{currentmarker}{\pgfqpoint{0.000000in}{-0.027778in}}{\pgfqpoint{0.000000in}{0.000000in}}{%
\pgfpathmoveto{\pgfqpoint{0.000000in}{0.000000in}}%
\pgfpathlineto{\pgfqpoint{0.000000in}{-0.027778in}}%
\pgfusepath{stroke,fill}%
}%
\begin{pgfscope}%
\pgfsys@transformshift{0.792961in}{0.521603in}%
\pgfsys@useobject{currentmarker}{}%
\end{pgfscope}%
\end{pgfscope}%
\begin{pgfscope}%
\pgfsetbuttcap%
\pgfsetroundjoin%
\definecolor{currentfill}{rgb}{0.000000,0.000000,0.000000}%
\pgfsetfillcolor{currentfill}%
\pgfsetlinewidth{0.602250pt}%
\definecolor{currentstroke}{rgb}{0.000000,0.000000,0.000000}%
\pgfsetstrokecolor{currentstroke}%
\pgfsetdash{}{0pt}%
\pgfsys@defobject{currentmarker}{\pgfqpoint{0.000000in}{-0.027778in}}{\pgfqpoint{0.000000in}{0.000000in}}{%
\pgfpathmoveto{\pgfqpoint{0.000000in}{0.000000in}}%
\pgfpathlineto{\pgfqpoint{0.000000in}{-0.027778in}}%
\pgfusepath{stroke,fill}%
}%
\begin{pgfscope}%
\pgfsys@transformshift{1.027730in}{0.521603in}%
\pgfsys@useobject{currentmarker}{}%
\end{pgfscope}%
\end{pgfscope}%
\begin{pgfscope}%
\pgfsetbuttcap%
\pgfsetroundjoin%
\definecolor{currentfill}{rgb}{0.000000,0.000000,0.000000}%
\pgfsetfillcolor{currentfill}%
\pgfsetlinewidth{0.602250pt}%
\definecolor{currentstroke}{rgb}{0.000000,0.000000,0.000000}%
\pgfsetstrokecolor{currentstroke}%
\pgfsetdash{}{0pt}%
\pgfsys@defobject{currentmarker}{\pgfqpoint{0.000000in}{-0.027778in}}{\pgfqpoint{0.000000in}{0.000000in}}{%
\pgfpathmoveto{\pgfqpoint{0.000000in}{0.000000in}}%
\pgfpathlineto{\pgfqpoint{0.000000in}{-0.027778in}}%
\pgfusepath{stroke,fill}%
}%
\begin{pgfscope}%
\pgfsys@transformshift{1.194302in}{0.521603in}%
\pgfsys@useobject{currentmarker}{}%
\end{pgfscope}%
\end{pgfscope}%
\begin{pgfscope}%
\pgfsetbuttcap%
\pgfsetroundjoin%
\definecolor{currentfill}{rgb}{0.000000,0.000000,0.000000}%
\pgfsetfillcolor{currentfill}%
\pgfsetlinewidth{0.602250pt}%
\definecolor{currentstroke}{rgb}{0.000000,0.000000,0.000000}%
\pgfsetstrokecolor{currentstroke}%
\pgfsetdash{}{0pt}%
\pgfsys@defobject{currentmarker}{\pgfqpoint{0.000000in}{-0.027778in}}{\pgfqpoint{0.000000in}{0.000000in}}{%
\pgfpathmoveto{\pgfqpoint{0.000000in}{0.000000in}}%
\pgfpathlineto{\pgfqpoint{0.000000in}{-0.027778in}}%
\pgfusepath{stroke,fill}%
}%
\begin{pgfscope}%
\pgfsys@transformshift{1.323505in}{0.521603in}%
\pgfsys@useobject{currentmarker}{}%
\end{pgfscope}%
\end{pgfscope}%
\begin{pgfscope}%
\pgfsetbuttcap%
\pgfsetroundjoin%
\definecolor{currentfill}{rgb}{0.000000,0.000000,0.000000}%
\pgfsetfillcolor{currentfill}%
\pgfsetlinewidth{0.602250pt}%
\definecolor{currentstroke}{rgb}{0.000000,0.000000,0.000000}%
\pgfsetstrokecolor{currentstroke}%
\pgfsetdash{}{0pt}%
\pgfsys@defobject{currentmarker}{\pgfqpoint{0.000000in}{-0.027778in}}{\pgfqpoint{0.000000in}{0.000000in}}{%
\pgfpathmoveto{\pgfqpoint{0.000000in}{0.000000in}}%
\pgfpathlineto{\pgfqpoint{0.000000in}{-0.027778in}}%
\pgfusepath{stroke,fill}%
}%
\begin{pgfscope}%
\pgfsys@transformshift{1.429071in}{0.521603in}%
\pgfsys@useobject{currentmarker}{}%
\end{pgfscope}%
\end{pgfscope}%
\begin{pgfscope}%
\pgfsetbuttcap%
\pgfsetroundjoin%
\definecolor{currentfill}{rgb}{0.000000,0.000000,0.000000}%
\pgfsetfillcolor{currentfill}%
\pgfsetlinewidth{0.602250pt}%
\definecolor{currentstroke}{rgb}{0.000000,0.000000,0.000000}%
\pgfsetstrokecolor{currentstroke}%
\pgfsetdash{}{0pt}%
\pgfsys@defobject{currentmarker}{\pgfqpoint{0.000000in}{-0.027778in}}{\pgfqpoint{0.000000in}{0.000000in}}{%
\pgfpathmoveto{\pgfqpoint{0.000000in}{0.000000in}}%
\pgfpathlineto{\pgfqpoint{0.000000in}{-0.027778in}}%
\pgfusepath{stroke,fill}%
}%
\begin{pgfscope}%
\pgfsys@transformshift{1.518326in}{0.521603in}%
\pgfsys@useobject{currentmarker}{}%
\end{pgfscope}%
\end{pgfscope}%
\begin{pgfscope}%
\pgfsetbuttcap%
\pgfsetroundjoin%
\definecolor{currentfill}{rgb}{0.000000,0.000000,0.000000}%
\pgfsetfillcolor{currentfill}%
\pgfsetlinewidth{0.602250pt}%
\definecolor{currentstroke}{rgb}{0.000000,0.000000,0.000000}%
\pgfsetstrokecolor{currentstroke}%
\pgfsetdash{}{0pt}%
\pgfsys@defobject{currentmarker}{\pgfqpoint{0.000000in}{-0.027778in}}{\pgfqpoint{0.000000in}{0.000000in}}{%
\pgfpathmoveto{\pgfqpoint{0.000000in}{0.000000in}}%
\pgfpathlineto{\pgfqpoint{0.000000in}{-0.027778in}}%
\pgfusepath{stroke,fill}%
}%
\begin{pgfscope}%
\pgfsys@transformshift{1.595642in}{0.521603in}%
\pgfsys@useobject{currentmarker}{}%
\end{pgfscope}%
\end{pgfscope}%
\begin{pgfscope}%
\pgfsetbuttcap%
\pgfsetroundjoin%
\definecolor{currentfill}{rgb}{0.000000,0.000000,0.000000}%
\pgfsetfillcolor{currentfill}%
\pgfsetlinewidth{0.602250pt}%
\definecolor{currentstroke}{rgb}{0.000000,0.000000,0.000000}%
\pgfsetstrokecolor{currentstroke}%
\pgfsetdash{}{0pt}%
\pgfsys@defobject{currentmarker}{\pgfqpoint{0.000000in}{-0.027778in}}{\pgfqpoint{0.000000in}{0.000000in}}{%
\pgfpathmoveto{\pgfqpoint{0.000000in}{0.000000in}}%
\pgfpathlineto{\pgfqpoint{0.000000in}{-0.027778in}}%
\pgfusepath{stroke,fill}%
}%
\begin{pgfscope}%
\pgfsys@transformshift{1.663840in}{0.521603in}%
\pgfsys@useobject{currentmarker}{}%
\end{pgfscope}%
\end{pgfscope}%
\begin{pgfscope}%
\pgfsetbuttcap%
\pgfsetroundjoin%
\definecolor{currentfill}{rgb}{0.000000,0.000000,0.000000}%
\pgfsetfillcolor{currentfill}%
\pgfsetlinewidth{0.602250pt}%
\definecolor{currentstroke}{rgb}{0.000000,0.000000,0.000000}%
\pgfsetstrokecolor{currentstroke}%
\pgfsetdash{}{0pt}%
\pgfsys@defobject{currentmarker}{\pgfqpoint{0.000000in}{-0.027778in}}{\pgfqpoint{0.000000in}{0.000000in}}{%
\pgfpathmoveto{\pgfqpoint{0.000000in}{0.000000in}}%
\pgfpathlineto{\pgfqpoint{0.000000in}{-0.027778in}}%
\pgfusepath{stroke,fill}%
}%
\begin{pgfscope}%
\pgfsys@transformshift{2.126186in}{0.521603in}%
\pgfsys@useobject{currentmarker}{}%
\end{pgfscope}%
\end{pgfscope}%
\begin{pgfscope}%
\pgfsetbuttcap%
\pgfsetroundjoin%
\definecolor{currentfill}{rgb}{0.000000,0.000000,0.000000}%
\pgfsetfillcolor{currentfill}%
\pgfsetlinewidth{0.602250pt}%
\definecolor{currentstroke}{rgb}{0.000000,0.000000,0.000000}%
\pgfsetstrokecolor{currentstroke}%
\pgfsetdash{}{0pt}%
\pgfsys@defobject{currentmarker}{\pgfqpoint{0.000000in}{-0.027778in}}{\pgfqpoint{0.000000in}{0.000000in}}{%
\pgfpathmoveto{\pgfqpoint{0.000000in}{0.000000in}}%
\pgfpathlineto{\pgfqpoint{0.000000in}{-0.027778in}}%
\pgfusepath{stroke,fill}%
}%
\begin{pgfscope}%
\pgfsys@transformshift{2.360956in}{0.521603in}%
\pgfsys@useobject{currentmarker}{}%
\end{pgfscope}%
\end{pgfscope}%
\begin{pgfscope}%
\pgfsetbuttcap%
\pgfsetroundjoin%
\definecolor{currentfill}{rgb}{0.000000,0.000000,0.000000}%
\pgfsetfillcolor{currentfill}%
\pgfsetlinewidth{0.602250pt}%
\definecolor{currentstroke}{rgb}{0.000000,0.000000,0.000000}%
\pgfsetstrokecolor{currentstroke}%
\pgfsetdash{}{0pt}%
\pgfsys@defobject{currentmarker}{\pgfqpoint{0.000000in}{-0.027778in}}{\pgfqpoint{0.000000in}{0.000000in}}{%
\pgfpathmoveto{\pgfqpoint{0.000000in}{0.000000in}}%
\pgfpathlineto{\pgfqpoint{0.000000in}{-0.027778in}}%
\pgfusepath{stroke,fill}%
}%
\begin{pgfscope}%
\pgfsys@transformshift{2.527527in}{0.521603in}%
\pgfsys@useobject{currentmarker}{}%
\end{pgfscope}%
\end{pgfscope}%
\begin{pgfscope}%
\definecolor{textcolor}{rgb}{0.000000,0.000000,0.000000}%
\pgfsetstrokecolor{textcolor}%
\pgfsetfillcolor{textcolor}%
\pgftext[x=1.643943in,y=0.234413in,,top]{\color{textcolor}{\sffamily\fontsize{10.000000}{12.000000}\selectfont\catcode`\^=\active\def^{\ifmmode\sp\else\^{}\fi}\catcode`\%=\active\def%{\%}$n_v$}}%
\end{pgfscope}%
\begin{pgfscope}%
\pgfsetbuttcap%
\pgfsetroundjoin%
\definecolor{currentfill}{rgb}{0.000000,0.000000,0.000000}%
\pgfsetfillcolor{currentfill}%
\pgfsetlinewidth{0.803000pt}%
\definecolor{currentstroke}{rgb}{0.000000,0.000000,0.000000}%
\pgfsetstrokecolor{currentstroke}%
\pgfsetdash{}{0pt}%
\pgfsys@defobject{currentmarker}{\pgfqpoint{-0.048611in}{0.000000in}}{\pgfqpoint{-0.000000in}{0.000000in}}{%
\pgfpathmoveto{\pgfqpoint{-0.000000in}{0.000000in}}%
\pgfpathlineto{\pgfqpoint{-0.048611in}{0.000000in}}%
\pgfusepath{stroke,fill}%
}%
\begin{pgfscope}%
\pgfsys@transformshift{0.675193in}{0.896835in}%
\pgfsys@useobject{currentmarker}{}%
\end{pgfscope}%
\end{pgfscope}%
\begin{pgfscope}%
\definecolor{textcolor}{rgb}{0.000000,0.000000,0.000000}%
\pgfsetstrokecolor{textcolor}%
\pgfsetfillcolor{textcolor}%
\pgftext[x=0.289968in, y=0.844073in, left, base]{\color{textcolor}{\sffamily\fontsize{10.000000}{12.000000}\selectfont\catcode`\^=\active\def^{\ifmmode\sp\else\^{}\fi}\catcode`\%=\active\def%{\%}$\mathdefault{10^{-6}}$}}%
\end{pgfscope}%
\begin{pgfscope}%
\pgfsetbuttcap%
\pgfsetroundjoin%
\definecolor{currentfill}{rgb}{0.000000,0.000000,0.000000}%
\pgfsetfillcolor{currentfill}%
\pgfsetlinewidth{0.803000pt}%
\definecolor{currentstroke}{rgb}{0.000000,0.000000,0.000000}%
\pgfsetstrokecolor{currentstroke}%
\pgfsetdash{}{0pt}%
\pgfsys@defobject{currentmarker}{\pgfqpoint{-0.048611in}{0.000000in}}{\pgfqpoint{-0.000000in}{0.000000in}}{%
\pgfpathmoveto{\pgfqpoint{-0.000000in}{0.000000in}}%
\pgfpathlineto{\pgfqpoint{-0.048611in}{0.000000in}}%
\pgfusepath{stroke,fill}%
}%
\begin{pgfscope}%
\pgfsys@transformshift{0.675193in}{1.444787in}%
\pgfsys@useobject{currentmarker}{}%
\end{pgfscope}%
\end{pgfscope}%
\begin{pgfscope}%
\definecolor{textcolor}{rgb}{0.000000,0.000000,0.000000}%
\pgfsetstrokecolor{textcolor}%
\pgfsetfillcolor{textcolor}%
\pgftext[x=0.289968in, y=1.392026in, left, base]{\color{textcolor}{\sffamily\fontsize{10.000000}{12.000000}\selectfont\catcode`\^=\active\def^{\ifmmode\sp\else\^{}\fi}\catcode`\%=\active\def%{\%}$\mathdefault{10^{-4}}$}}%
\end{pgfscope}%
\begin{pgfscope}%
\pgfsetbuttcap%
\pgfsetroundjoin%
\definecolor{currentfill}{rgb}{0.000000,0.000000,0.000000}%
\pgfsetfillcolor{currentfill}%
\pgfsetlinewidth{0.803000pt}%
\definecolor{currentstroke}{rgb}{0.000000,0.000000,0.000000}%
\pgfsetstrokecolor{currentstroke}%
\pgfsetdash{}{0pt}%
\pgfsys@defobject{currentmarker}{\pgfqpoint{-0.048611in}{0.000000in}}{\pgfqpoint{-0.000000in}{0.000000in}}{%
\pgfpathmoveto{\pgfqpoint{-0.000000in}{0.000000in}}%
\pgfpathlineto{\pgfqpoint{-0.048611in}{0.000000in}}%
\pgfusepath{stroke,fill}%
}%
\begin{pgfscope}%
\pgfsys@transformshift{0.675193in}{1.992739in}%
\pgfsys@useobject{currentmarker}{}%
\end{pgfscope}%
\end{pgfscope}%
\begin{pgfscope}%
\definecolor{textcolor}{rgb}{0.000000,0.000000,0.000000}%
\pgfsetstrokecolor{textcolor}%
\pgfsetfillcolor{textcolor}%
\pgftext[x=0.289968in, y=1.939978in, left, base]{\color{textcolor}{\sffamily\fontsize{10.000000}{12.000000}\selectfont\catcode`\^=\active\def^{\ifmmode\sp\else\^{}\fi}\catcode`\%=\active\def%{\%}$\mathdefault{10^{-2}}$}}%
\end{pgfscope}%
\begin{pgfscope}%
\definecolor{textcolor}{rgb}{0.000000,0.000000,0.000000}%
\pgfsetstrokecolor{textcolor}%
\pgfsetfillcolor{textcolor}%
\pgftext[x=0.234413in,y=1.484103in,,bottom,rotate=90.000000]{\color{textcolor}{\sffamily\fontsize{10.000000}{12.000000}\selectfont\catcode`\^=\active\def^{\ifmmode\sp\else\^{}\fi}\catcode`\%=\active\def%{\%}$L^1$ error}}%
\end{pgfscope}%
\begin{pgfscope}%
\pgfpathrectangle{\pgfqpoint{0.675193in}{0.521603in}}{\pgfqpoint{1.937500in}{1.925000in}}%
\pgfusepath{clip}%
\pgfsetrectcap%
\pgfsetroundjoin%
\pgfsetlinewidth{1.003750pt}%
\definecolor{currentstroke}{rgb}{0.001462,0.000466,0.013866}%
\pgfsetstrokecolor{currentstroke}%
\pgfsetdash{}{0pt}%
\pgfpathmoveto{\pgfqpoint{0.763261in}{2.203543in}}%
\pgfpathlineto{\pgfqpoint{1.133297in}{2.196426in}}%
\pgfpathlineto{\pgfqpoint{1.484257in}{2.153709in}}%
\pgfpathlineto{\pgfqpoint{1.830412in}{2.096945in}}%
\pgfpathlineto{\pgfqpoint{2.176084in}{2.077846in}}%
\pgfpathlineto{\pgfqpoint{2.524625in}{2.006425in}}%
\pgfusepath{stroke}%
\end{pgfscope}%
\begin{pgfscope}%
\pgfpathrectangle{\pgfqpoint{0.675193in}{0.521603in}}{\pgfqpoint{1.937500in}{1.925000in}}%
\pgfusepath{clip}%
\pgfsetbuttcap%
\pgfsetroundjoin%
\definecolor{currentfill}{rgb}{0.001462,0.000466,0.013866}%
\pgfsetfillcolor{currentfill}%
\pgfsetlinewidth{1.003750pt}%
\definecolor{currentstroke}{rgb}{0.001462,0.000466,0.013866}%
\pgfsetstrokecolor{currentstroke}%
\pgfsetdash{}{0pt}%
\pgfsys@defobject{currentmarker}{\pgfqpoint{-0.020833in}{-0.020833in}}{\pgfqpoint{0.020833in}{0.020833in}}{%
\pgfpathmoveto{\pgfqpoint{0.000000in}{-0.020833in}}%
\pgfpathcurveto{\pgfqpoint{0.005525in}{-0.020833in}}{\pgfqpoint{0.010825in}{-0.018638in}}{\pgfqpoint{0.014731in}{-0.014731in}}%
\pgfpathcurveto{\pgfqpoint{0.018638in}{-0.010825in}}{\pgfqpoint{0.020833in}{-0.005525in}}{\pgfqpoint{0.020833in}{0.000000in}}%
\pgfpathcurveto{\pgfqpoint{0.020833in}{0.005525in}}{\pgfqpoint{0.018638in}{0.010825in}}{\pgfqpoint{0.014731in}{0.014731in}}%
\pgfpathcurveto{\pgfqpoint{0.010825in}{0.018638in}}{\pgfqpoint{0.005525in}{0.020833in}}{\pgfqpoint{0.000000in}{0.020833in}}%
\pgfpathcurveto{\pgfqpoint{-0.005525in}{0.020833in}}{\pgfqpoint{-0.010825in}{0.018638in}}{\pgfqpoint{-0.014731in}{0.014731in}}%
\pgfpathcurveto{\pgfqpoint{-0.018638in}{0.010825in}}{\pgfqpoint{-0.020833in}{0.005525in}}{\pgfqpoint{-0.020833in}{0.000000in}}%
\pgfpathcurveto{\pgfqpoint{-0.020833in}{-0.005525in}}{\pgfqpoint{-0.018638in}{-0.010825in}}{\pgfqpoint{-0.014731in}{-0.014731in}}%
\pgfpathcurveto{\pgfqpoint{-0.010825in}{-0.018638in}}{\pgfqpoint{-0.005525in}{-0.020833in}}{\pgfqpoint{0.000000in}{-0.020833in}}%
\pgfpathlineto{\pgfqpoint{0.000000in}{-0.020833in}}%
\pgfpathclose%
\pgfusepath{stroke,fill}%
}%
\begin{pgfscope}%
\pgfsys@transformshift{0.763261in}{2.203543in}%
\pgfsys@useobject{currentmarker}{}%
\end{pgfscope}%
\begin{pgfscope}%
\pgfsys@transformshift{1.133297in}{2.196426in}%
\pgfsys@useobject{currentmarker}{}%
\end{pgfscope}%
\begin{pgfscope}%
\pgfsys@transformshift{1.484257in}{2.153709in}%
\pgfsys@useobject{currentmarker}{}%
\end{pgfscope}%
\begin{pgfscope}%
\pgfsys@transformshift{1.830412in}{2.096945in}%
\pgfsys@useobject{currentmarker}{}%
\end{pgfscope}%
\begin{pgfscope}%
\pgfsys@transformshift{2.176084in}{2.077846in}%
\pgfsys@useobject{currentmarker}{}%
\end{pgfscope}%
\begin{pgfscope}%
\pgfsys@transformshift{2.524625in}{2.006425in}%
\pgfsys@useobject{currentmarker}{}%
\end{pgfscope}%
\end{pgfscope}%
\begin{pgfscope}%
\pgfpathrectangle{\pgfqpoint{0.675193in}{0.521603in}}{\pgfqpoint{1.937500in}{1.925000in}}%
\pgfusepath{clip}%
\pgfsetrectcap%
\pgfsetroundjoin%
\pgfsetlinewidth{1.003750pt}%
\definecolor{currentstroke}{rgb}{0.445163,0.122724,0.506901}%
\pgfsetstrokecolor{currentstroke}%
\pgfsetdash{}{0pt}%
\pgfpathmoveto{\pgfqpoint{0.763261in}{2.359103in}}%
\pgfpathlineto{\pgfqpoint{1.133297in}{2.230187in}}%
\pgfpathlineto{\pgfqpoint{1.484257in}{1.742286in}}%
\pgfpathlineto{\pgfqpoint{1.830412in}{0.629376in}}%
\pgfpathlineto{\pgfqpoint{2.176084in}{0.635047in}}%
\pgfpathlineto{\pgfqpoint{2.524625in}{0.647899in}}%
\pgfusepath{stroke}%
\end{pgfscope}%
\begin{pgfscope}%
\pgfpathrectangle{\pgfqpoint{0.675193in}{0.521603in}}{\pgfqpoint{1.937500in}{1.925000in}}%
\pgfusepath{clip}%
\pgfsetbuttcap%
\pgfsetroundjoin%
\definecolor{currentfill}{rgb}{0.445163,0.122724,0.506901}%
\pgfsetfillcolor{currentfill}%
\pgfsetlinewidth{1.003750pt}%
\definecolor{currentstroke}{rgb}{0.445163,0.122724,0.506901}%
\pgfsetstrokecolor{currentstroke}%
\pgfsetdash{}{0pt}%
\pgfsys@defobject{currentmarker}{\pgfqpoint{-0.020833in}{-0.020833in}}{\pgfqpoint{0.020833in}{0.020833in}}{%
\pgfpathmoveto{\pgfqpoint{0.000000in}{-0.020833in}}%
\pgfpathcurveto{\pgfqpoint{0.005525in}{-0.020833in}}{\pgfqpoint{0.010825in}{-0.018638in}}{\pgfqpoint{0.014731in}{-0.014731in}}%
\pgfpathcurveto{\pgfqpoint{0.018638in}{-0.010825in}}{\pgfqpoint{0.020833in}{-0.005525in}}{\pgfqpoint{0.020833in}{0.000000in}}%
\pgfpathcurveto{\pgfqpoint{0.020833in}{0.005525in}}{\pgfqpoint{0.018638in}{0.010825in}}{\pgfqpoint{0.014731in}{0.014731in}}%
\pgfpathcurveto{\pgfqpoint{0.010825in}{0.018638in}}{\pgfqpoint{0.005525in}{0.020833in}}{\pgfqpoint{0.000000in}{0.020833in}}%
\pgfpathcurveto{\pgfqpoint{-0.005525in}{0.020833in}}{\pgfqpoint{-0.010825in}{0.018638in}}{\pgfqpoint{-0.014731in}{0.014731in}}%
\pgfpathcurveto{\pgfqpoint{-0.018638in}{0.010825in}}{\pgfqpoint{-0.020833in}{0.005525in}}{\pgfqpoint{-0.020833in}{0.000000in}}%
\pgfpathcurveto{\pgfqpoint{-0.020833in}{-0.005525in}}{\pgfqpoint{-0.018638in}{-0.010825in}}{\pgfqpoint{-0.014731in}{-0.014731in}}%
\pgfpathcurveto{\pgfqpoint{-0.010825in}{-0.018638in}}{\pgfqpoint{-0.005525in}{-0.020833in}}{\pgfqpoint{0.000000in}{-0.020833in}}%
\pgfpathlineto{\pgfqpoint{0.000000in}{-0.020833in}}%
\pgfpathclose%
\pgfusepath{stroke,fill}%
}%
\begin{pgfscope}%
\pgfsys@transformshift{0.763261in}{2.359103in}%
\pgfsys@useobject{currentmarker}{}%
\end{pgfscope}%
\begin{pgfscope}%
\pgfsys@transformshift{1.133297in}{2.230187in}%
\pgfsys@useobject{currentmarker}{}%
\end{pgfscope}%
\begin{pgfscope}%
\pgfsys@transformshift{1.484257in}{1.742286in}%
\pgfsys@useobject{currentmarker}{}%
\end{pgfscope}%
\begin{pgfscope}%
\pgfsys@transformshift{1.830412in}{0.629376in}%
\pgfsys@useobject{currentmarker}{}%
\end{pgfscope}%
\begin{pgfscope}%
\pgfsys@transformshift{2.176084in}{0.635047in}%
\pgfsys@useobject{currentmarker}{}%
\end{pgfscope}%
\begin{pgfscope}%
\pgfsys@transformshift{2.524625in}{0.647899in}%
\pgfsys@useobject{currentmarker}{}%
\end{pgfscope}%
\end{pgfscope}%
\begin{pgfscope}%
\pgfpathrectangle{\pgfqpoint{0.675193in}{0.521603in}}{\pgfqpoint{1.937500in}{1.925000in}}%
\pgfusepath{clip}%
\pgfsetrectcap%
\pgfsetroundjoin%
\pgfsetlinewidth{1.003750pt}%
\definecolor{currentstroke}{rgb}{0.944006,0.377643,0.365136}%
\pgfsetstrokecolor{currentstroke}%
\pgfsetdash{}{0pt}%
\pgfpathmoveto{\pgfqpoint{0.763261in}{2.196342in}}%
\pgfpathlineto{\pgfqpoint{1.133297in}{1.947321in}}%
\pgfpathlineto{\pgfqpoint{1.484257in}{1.866944in}}%
\pgfpathlineto{\pgfqpoint{1.830412in}{1.493754in}}%
\pgfpathlineto{\pgfqpoint{2.176084in}{0.609798in}}%
\pgfpathlineto{\pgfqpoint{2.524625in}{0.609103in}}%
\pgfusepath{stroke}%
\end{pgfscope}%
\begin{pgfscope}%
\pgfpathrectangle{\pgfqpoint{0.675193in}{0.521603in}}{\pgfqpoint{1.937500in}{1.925000in}}%
\pgfusepath{clip}%
\pgfsetbuttcap%
\pgfsetroundjoin%
\definecolor{currentfill}{rgb}{0.944006,0.377643,0.365136}%
\pgfsetfillcolor{currentfill}%
\pgfsetlinewidth{1.003750pt}%
\definecolor{currentstroke}{rgb}{0.944006,0.377643,0.365136}%
\pgfsetstrokecolor{currentstroke}%
\pgfsetdash{}{0pt}%
\pgfsys@defobject{currentmarker}{\pgfqpoint{-0.020833in}{-0.020833in}}{\pgfqpoint{0.020833in}{0.020833in}}{%
\pgfpathmoveto{\pgfqpoint{0.000000in}{-0.020833in}}%
\pgfpathcurveto{\pgfqpoint{0.005525in}{-0.020833in}}{\pgfqpoint{0.010825in}{-0.018638in}}{\pgfqpoint{0.014731in}{-0.014731in}}%
\pgfpathcurveto{\pgfqpoint{0.018638in}{-0.010825in}}{\pgfqpoint{0.020833in}{-0.005525in}}{\pgfqpoint{0.020833in}{0.000000in}}%
\pgfpathcurveto{\pgfqpoint{0.020833in}{0.005525in}}{\pgfqpoint{0.018638in}{0.010825in}}{\pgfqpoint{0.014731in}{0.014731in}}%
\pgfpathcurveto{\pgfqpoint{0.010825in}{0.018638in}}{\pgfqpoint{0.005525in}{0.020833in}}{\pgfqpoint{0.000000in}{0.020833in}}%
\pgfpathcurveto{\pgfqpoint{-0.005525in}{0.020833in}}{\pgfqpoint{-0.010825in}{0.018638in}}{\pgfqpoint{-0.014731in}{0.014731in}}%
\pgfpathcurveto{\pgfqpoint{-0.018638in}{0.010825in}}{\pgfqpoint{-0.020833in}{0.005525in}}{\pgfqpoint{-0.020833in}{0.000000in}}%
\pgfpathcurveto{\pgfqpoint{-0.020833in}{-0.005525in}}{\pgfqpoint{-0.018638in}{-0.010825in}}{\pgfqpoint{-0.014731in}{-0.014731in}}%
\pgfpathcurveto{\pgfqpoint{-0.010825in}{-0.018638in}}{\pgfqpoint{-0.005525in}{-0.020833in}}{\pgfqpoint{0.000000in}{-0.020833in}}%
\pgfpathlineto{\pgfqpoint{0.000000in}{-0.020833in}}%
\pgfpathclose%
\pgfusepath{stroke,fill}%
}%
\begin{pgfscope}%
\pgfsys@transformshift{0.763261in}{2.196342in}%
\pgfsys@useobject{currentmarker}{}%
\end{pgfscope}%
\begin{pgfscope}%
\pgfsys@transformshift{1.133297in}{1.947321in}%
\pgfsys@useobject{currentmarker}{}%
\end{pgfscope}%
\begin{pgfscope}%
\pgfsys@transformshift{1.484257in}{1.866944in}%
\pgfsys@useobject{currentmarker}{}%
\end{pgfscope}%
\begin{pgfscope}%
\pgfsys@transformshift{1.830412in}{1.493754in}%
\pgfsys@useobject{currentmarker}{}%
\end{pgfscope}%
\begin{pgfscope}%
\pgfsys@transformshift{2.176084in}{0.609798in}%
\pgfsys@useobject{currentmarker}{}%
\end{pgfscope}%
\begin{pgfscope}%
\pgfsys@transformshift{2.524625in}{0.609103in}%
\pgfsys@useobject{currentmarker}{}%
\end{pgfscope}%
\end{pgfscope}%
\begin{pgfscope}%
\pgfsetrectcap%
\pgfsetmiterjoin%
\pgfsetlinewidth{0.803000pt}%
\definecolor{currentstroke}{rgb}{0.000000,0.000000,0.000000}%
\pgfsetstrokecolor{currentstroke}%
\pgfsetdash{}{0pt}%
\pgfpathmoveto{\pgfqpoint{0.675193in}{0.521603in}}%
\pgfpathlineto{\pgfqpoint{0.675193in}{2.446603in}}%
\pgfusepath{stroke}%
\end{pgfscope}%
\begin{pgfscope}%
\pgfsetrectcap%
\pgfsetmiterjoin%
\pgfsetlinewidth{0.803000pt}%
\definecolor{currentstroke}{rgb}{0.000000,0.000000,0.000000}%
\pgfsetstrokecolor{currentstroke}%
\pgfsetdash{}{0pt}%
\pgfpathmoveto{\pgfqpoint{2.612693in}{0.521603in}}%
\pgfpathlineto{\pgfqpoint{2.612693in}{2.446603in}}%
\pgfusepath{stroke}%
\end{pgfscope}%
\begin{pgfscope}%
\pgfsetrectcap%
\pgfsetmiterjoin%
\pgfsetlinewidth{0.803000pt}%
\definecolor{currentstroke}{rgb}{0.000000,0.000000,0.000000}%
\pgfsetstrokecolor{currentstroke}%
\pgfsetdash{}{0pt}%
\pgfpathmoveto{\pgfqpoint{0.675193in}{0.521603in}}%
\pgfpathlineto{\pgfqpoint{2.612693in}{0.521603in}}%
\pgfusepath{stroke}%
\end{pgfscope}%
\begin{pgfscope}%
\pgfsetrectcap%
\pgfsetmiterjoin%
\pgfsetlinewidth{0.803000pt}%
\definecolor{currentstroke}{rgb}{0.000000,0.000000,0.000000}%
\pgfsetstrokecolor{currentstroke}%
\pgfsetdash{}{0pt}%
\pgfpathmoveto{\pgfqpoint{0.675193in}{2.446603in}}%
\pgfpathlineto{\pgfqpoint{2.612693in}{2.446603in}}%
\pgfusepath{stroke}%
\end{pgfscope}%
\begin{pgfscope}%
\pgfsetbuttcap%
\pgfsetmiterjoin%
\definecolor{currentfill}{rgb}{1.000000,1.000000,1.000000}%
\pgfsetfillcolor{currentfill}%
\pgfsetfillopacity{0.800000}%
\pgfsetlinewidth{1.003750pt}%
\definecolor{currentstroke}{rgb}{0.800000,0.800000,0.800000}%
\pgfsetstrokecolor{currentstroke}%
\pgfsetstrokeopacity{0.800000}%
\pgfsetdash{}{0pt}%
\pgfpathmoveto{\pgfqpoint{0.772415in}{0.591048in}}%
\pgfpathlineto{\pgfqpoint{1.650481in}{0.591048in}}%
\pgfpathquadraticcurveto{\pgfqpoint{1.678258in}{0.591048in}}{\pgfqpoint{1.678258in}{0.618826in}}%
\pgfpathlineto{\pgfqpoint{1.678258in}{1.216508in}}%
\pgfpathquadraticcurveto{\pgfqpoint{1.678258in}{1.244286in}}{\pgfqpoint{1.650481in}{1.244286in}}%
\pgfpathlineto{\pgfqpoint{0.772415in}{1.244286in}}%
\pgfpathquadraticcurveto{\pgfqpoint{0.744638in}{1.244286in}}{\pgfqpoint{0.744638in}{1.216508in}}%
\pgfpathlineto{\pgfqpoint{0.744638in}{0.618826in}}%
\pgfpathquadraticcurveto{\pgfqpoint{0.744638in}{0.591048in}}{\pgfqpoint{0.772415in}{0.591048in}}%
\pgfpathlineto{\pgfqpoint{0.772415in}{0.591048in}}%
\pgfpathclose%
\pgfusepath{stroke,fill}%
\end{pgfscope}%
\begin{pgfscope}%
\pgfsetrectcap%
\pgfsetroundjoin%
\pgfsetlinewidth{1.003750pt}%
\definecolor{currentstroke}{rgb}{0.001462,0.000466,0.013866}%
\pgfsetstrokecolor{currentstroke}%
\pgfsetdash{}{0pt}%
\pgfpathmoveto{\pgfqpoint{0.800193in}{1.131819in}}%
\pgfpathlineto{\pgfqpoint{0.939082in}{1.131819in}}%
\pgfpathlineto{\pgfqpoint{1.077971in}{1.131819in}}%
\pgfusepath{stroke}%
\end{pgfscope}%
\begin{pgfscope}%
\pgfsetbuttcap%
\pgfsetroundjoin%
\definecolor{currentfill}{rgb}{0.001462,0.000466,0.013866}%
\pgfsetfillcolor{currentfill}%
\pgfsetlinewidth{1.003750pt}%
\definecolor{currentstroke}{rgb}{0.001462,0.000466,0.013866}%
\pgfsetstrokecolor{currentstroke}%
\pgfsetdash{}{0pt}%
\pgfsys@defobject{currentmarker}{\pgfqpoint{-0.020833in}{-0.020833in}}{\pgfqpoint{0.020833in}{0.020833in}}{%
\pgfpathmoveto{\pgfqpoint{0.000000in}{-0.020833in}}%
\pgfpathcurveto{\pgfqpoint{0.005525in}{-0.020833in}}{\pgfqpoint{0.010825in}{-0.018638in}}{\pgfqpoint{0.014731in}{-0.014731in}}%
\pgfpathcurveto{\pgfqpoint{0.018638in}{-0.010825in}}{\pgfqpoint{0.020833in}{-0.005525in}}{\pgfqpoint{0.020833in}{0.000000in}}%
\pgfpathcurveto{\pgfqpoint{0.020833in}{0.005525in}}{\pgfqpoint{0.018638in}{0.010825in}}{\pgfqpoint{0.014731in}{0.014731in}}%
\pgfpathcurveto{\pgfqpoint{0.010825in}{0.018638in}}{\pgfqpoint{0.005525in}{0.020833in}}{\pgfqpoint{0.000000in}{0.020833in}}%
\pgfpathcurveto{\pgfqpoint{-0.005525in}{0.020833in}}{\pgfqpoint{-0.010825in}{0.018638in}}{\pgfqpoint{-0.014731in}{0.014731in}}%
\pgfpathcurveto{\pgfqpoint{-0.018638in}{0.010825in}}{\pgfqpoint{-0.020833in}{0.005525in}}{\pgfqpoint{-0.020833in}{0.000000in}}%
\pgfpathcurveto{\pgfqpoint{-0.020833in}{-0.005525in}}{\pgfqpoint{-0.018638in}{-0.010825in}}{\pgfqpoint{-0.014731in}{-0.014731in}}%
\pgfpathcurveto{\pgfqpoint{-0.010825in}{-0.018638in}}{\pgfqpoint{-0.005525in}{-0.020833in}}{\pgfqpoint{0.000000in}{-0.020833in}}%
\pgfpathlineto{\pgfqpoint{0.000000in}{-0.020833in}}%
\pgfpathclose%
\pgfusepath{stroke,fill}%
}%
\begin{pgfscope}%
\pgfsys@transformshift{0.939082in}{1.131819in}%
\pgfsys@useobject{currentmarker}{}%
\end{pgfscope}%
\end{pgfscope}%
\begin{pgfscope}%
\definecolor{textcolor}{rgb}{0.000000,0.000000,0.000000}%
\pgfsetstrokecolor{textcolor}%
\pgfsetfillcolor{textcolor}%
\pgftext[x=1.189082in,y=1.083208in,left,base]{\color{textcolor}{\sffamily\fontsize{10.000000}{12.000000}\selectfont\catcode`\^=\active\def^{\ifmmode\sp\else\^{}\fi}\catcode`\%=\active\def%{\%}DGC}}%
\end{pgfscope}%
\begin{pgfscope}%
\pgfsetrectcap%
\pgfsetroundjoin%
\pgfsetlinewidth{1.003750pt}%
\definecolor{currentstroke}{rgb}{0.445163,0.122724,0.506901}%
\pgfsetstrokecolor{currentstroke}%
\pgfsetdash{}{0pt}%
\pgfpathmoveto{\pgfqpoint{0.800193in}{0.927961in}}%
\pgfpathlineto{\pgfqpoint{0.939082in}{0.927961in}}%
\pgfpathlineto{\pgfqpoint{1.077971in}{0.927961in}}%
\pgfusepath{stroke}%
\end{pgfscope}%
\begin{pgfscope}%
\pgfsetbuttcap%
\pgfsetroundjoin%
\definecolor{currentfill}{rgb}{0.445163,0.122724,0.506901}%
\pgfsetfillcolor{currentfill}%
\pgfsetlinewidth{1.003750pt}%
\definecolor{currentstroke}{rgb}{0.445163,0.122724,0.506901}%
\pgfsetstrokecolor{currentstroke}%
\pgfsetdash{}{0pt}%
\pgfsys@defobject{currentmarker}{\pgfqpoint{-0.020833in}{-0.020833in}}{\pgfqpoint{0.020833in}{0.020833in}}{%
\pgfpathmoveto{\pgfqpoint{0.000000in}{-0.020833in}}%
\pgfpathcurveto{\pgfqpoint{0.005525in}{-0.020833in}}{\pgfqpoint{0.010825in}{-0.018638in}}{\pgfqpoint{0.014731in}{-0.014731in}}%
\pgfpathcurveto{\pgfqpoint{0.018638in}{-0.010825in}}{\pgfqpoint{0.020833in}{-0.005525in}}{\pgfqpoint{0.020833in}{0.000000in}}%
\pgfpathcurveto{\pgfqpoint{0.020833in}{0.005525in}}{\pgfqpoint{0.018638in}{0.010825in}}{\pgfqpoint{0.014731in}{0.014731in}}%
\pgfpathcurveto{\pgfqpoint{0.010825in}{0.018638in}}{\pgfqpoint{0.005525in}{0.020833in}}{\pgfqpoint{0.000000in}{0.020833in}}%
\pgfpathcurveto{\pgfqpoint{-0.005525in}{0.020833in}}{\pgfqpoint{-0.010825in}{0.018638in}}{\pgfqpoint{-0.014731in}{0.014731in}}%
\pgfpathcurveto{\pgfqpoint{-0.018638in}{0.010825in}}{\pgfqpoint{-0.020833in}{0.005525in}}{\pgfqpoint{-0.020833in}{0.000000in}}%
\pgfpathcurveto{\pgfqpoint{-0.020833in}{-0.005525in}}{\pgfqpoint{-0.018638in}{-0.010825in}}{\pgfqpoint{-0.014731in}{-0.014731in}}%
\pgfpathcurveto{\pgfqpoint{-0.010825in}{-0.018638in}}{\pgfqpoint{-0.005525in}{-0.020833in}}{\pgfqpoint{0.000000in}{-0.020833in}}%
\pgfpathlineto{\pgfqpoint{0.000000in}{-0.020833in}}%
\pgfpathclose%
\pgfusepath{stroke,fill}%
}%
\begin{pgfscope}%
\pgfsys@transformshift{0.939082in}{0.927961in}%
\pgfsys@useobject{currentmarker}{}%
\end{pgfscope}%
\end{pgfscope}%
\begin{pgfscope}%
\definecolor{textcolor}{rgb}{0.000000,0.000000,0.000000}%
\pgfsetstrokecolor{textcolor}%
\pgfsetfillcolor{textcolor}%
\pgftext[x=1.189082in,y=0.879350in,left,base]{\color{textcolor}{\sffamily\fontsize{10.000000}{12.000000}\selectfont\catcode`\^=\active\def^{\ifmmode\sp\else\^{}\fi}\catcode`\%=\active\def%{\%}NC}}%
\end{pgfscope}%
\begin{pgfscope}%
\pgfsetrectcap%
\pgfsetroundjoin%
\pgfsetlinewidth{1.003750pt}%
\definecolor{currentstroke}{rgb}{0.944006,0.377643,0.365136}%
\pgfsetstrokecolor{currentstroke}%
\pgfsetdash{}{0pt}%
\pgfpathmoveto{\pgfqpoint{0.800193in}{0.724104in}}%
\pgfpathlineto{\pgfqpoint{0.939082in}{0.724104in}}%
\pgfpathlineto{\pgfqpoint{1.077971in}{0.724104in}}%
\pgfusepath{stroke}%
\end{pgfscope}%
\begin{pgfscope}%
\pgfsetbuttcap%
\pgfsetroundjoin%
\definecolor{currentfill}{rgb}{0.944006,0.377643,0.365136}%
\pgfsetfillcolor{currentfill}%
\pgfsetlinewidth{1.003750pt}%
\definecolor{currentstroke}{rgb}{0.944006,0.377643,0.365136}%
\pgfsetstrokecolor{currentstroke}%
\pgfsetdash{}{0pt}%
\pgfsys@defobject{currentmarker}{\pgfqpoint{-0.020833in}{-0.020833in}}{\pgfqpoint{0.020833in}{0.020833in}}{%
\pgfpathmoveto{\pgfqpoint{0.000000in}{-0.020833in}}%
\pgfpathcurveto{\pgfqpoint{0.005525in}{-0.020833in}}{\pgfqpoint{0.010825in}{-0.018638in}}{\pgfqpoint{0.014731in}{-0.014731in}}%
\pgfpathcurveto{\pgfqpoint{0.018638in}{-0.010825in}}{\pgfqpoint{0.020833in}{-0.005525in}}{\pgfqpoint{0.020833in}{0.000000in}}%
\pgfpathcurveto{\pgfqpoint{0.020833in}{0.005525in}}{\pgfqpoint{0.018638in}{0.010825in}}{\pgfqpoint{0.014731in}{0.014731in}}%
\pgfpathcurveto{\pgfqpoint{0.010825in}{0.018638in}}{\pgfqpoint{0.005525in}{0.020833in}}{\pgfqpoint{0.000000in}{0.020833in}}%
\pgfpathcurveto{\pgfqpoint{-0.005525in}{0.020833in}}{\pgfqpoint{-0.010825in}{0.018638in}}{\pgfqpoint{-0.014731in}{0.014731in}}%
\pgfpathcurveto{\pgfqpoint{-0.018638in}{0.010825in}}{\pgfqpoint{-0.020833in}{0.005525in}}{\pgfqpoint{-0.020833in}{0.000000in}}%
\pgfpathcurveto{\pgfqpoint{-0.020833in}{-0.005525in}}{\pgfqpoint{-0.018638in}{-0.010825in}}{\pgfqpoint{-0.014731in}{-0.014731in}}%
\pgfpathcurveto{\pgfqpoint{-0.010825in}{-0.018638in}}{\pgfqpoint{-0.005525in}{-0.020833in}}{\pgfqpoint{0.000000in}{-0.020833in}}%
\pgfpathlineto{\pgfqpoint{0.000000in}{-0.020833in}}%
\pgfpathclose%
\pgfusepath{stroke,fill}%
}%
\begin{pgfscope}%
\pgfsys@transformshift{0.939082in}{0.724104in}%
\pgfsys@useobject{currentmarker}{}%
\end{pgfscope}%
\end{pgfscope}%
\begin{pgfscope}%
\definecolor{textcolor}{rgb}{0.000000,0.000000,0.000000}%
\pgfsetstrokecolor{textcolor}%
\pgfsetfillcolor{textcolor}%
\pgftext[x=1.189082in,y=0.675493in,left,base]{\color{textcolor}{\sffamily\fontsize{10.000000}{12.000000}\selectfont\catcode`\^=\active\def^{\ifmmode\sp\else\^{}\fi}\catcode`\%=\active\def%{\%}NC++}}%
\end{pgfscope}%
\end{pgfpicture}%
\makeatother%
\endgroup%

        \caption{\gls{chebyshev-degree} $=2400$}
        \label{fig:5-experiments-electronic-structure-convergence-nv-m2400}
    \end{subfigure}
    \caption{For increasing values of \gls{sketch-size} $+$ \gls{num-hutchinson-queries}
    but fixed \gls{chebyshev-degree} we plot the $L^1$ relative approximation error \refequ{equ:5-experiments-L1-error}
    for the model problem with \gls{smoothing-parameter} $=0.05$.}
    \label{fig:5-experiments-electronic-structure-convergence-nv}
\end{figure}

\begin{figure}[ht]
    \centering
    \begin{subfigure}[b]{0.49\columnwidth}
        %% Creator: Matplotlib, PGF backend
%%
%% To include the figure in your LaTeX document, write
%%   \input{<filename>.pgf}
%%
%% Make sure the required packages are loaded in your preamble
%%   \usepackage{pgf}
%%
%% Also ensure that all the required font packages are loaded; for instance,
%% the lmodern package is sometimes necessary when using math font.
%%   \usepackage{lmodern}
%%
%% Figures using additional raster images can only be included by \input if
%% they are in the same directory as the main LaTeX file. For loading figures
%% from other directories you can use the `import` package
%%   \usepackage{import}
%%
%% and then include the figures with
%%   \import{<path to file>}{<filename>.pgf}
%%
%% Matplotlib used the following preamble
%%   \def\mathdefault#1{#1}
%%   \everymath=\expandafter{\the\everymath\displaystyle}
%%   
%%   \usepackage{fontspec}
%%   \setmainfont{DejaVuSans.ttf}[Path=\detokenize{C:/Users/fabio/AppData/Local/Programs/Python/Python311/Lib/site-packages/matplotlib/mpl-data/fonts/ttf/}]
%%   \setsansfont{DejaVuSans.ttf}[Path=\detokenize{C:/Users/fabio/AppData/Local/Programs/Python/Python311/Lib/site-packages/matplotlib/mpl-data/fonts/ttf/}]
%%   \setmonofont{DejaVuSansMono.ttf}[Path=\detokenize{C:/Users/fabio/AppData/Local/Programs/Python/Python311/Lib/site-packages/matplotlib/mpl-data/fonts/ttf/}]
%%   \makeatletter\@ifpackageloaded{underscore}{}{\usepackage[strings]{underscore}}\makeatother
%%
\begingroup%
\makeatletter%
\begin{pgfpicture}%
\pgfpathrectangle{\pgfpointorigin}{\pgfqpoint{2.772561in}{2.661984in}}%
\pgfusepath{use as bounding box, clip}%
\begin{pgfscope}%
\pgfsetbuttcap%
\pgfsetmiterjoin%
\definecolor{currentfill}{rgb}{1.000000,1.000000,1.000000}%
\pgfsetfillcolor{currentfill}%
\pgfsetlinewidth{0.000000pt}%
\definecolor{currentstroke}{rgb}{1.000000,1.000000,1.000000}%
\pgfsetstrokecolor{currentstroke}%
\pgfsetdash{}{0pt}%
\pgfpathmoveto{\pgfqpoint{0.000000in}{-0.000000in}}%
\pgfpathlineto{\pgfqpoint{2.772561in}{-0.000000in}}%
\pgfpathlineto{\pgfqpoint{2.772561in}{2.661984in}}%
\pgfpathlineto{\pgfqpoint{0.000000in}{2.661984in}}%
\pgfpathlineto{\pgfqpoint{0.000000in}{-0.000000in}}%
\pgfpathclose%
\pgfusepath{fill}%
\end{pgfscope}%
\begin{pgfscope}%
\pgfsetbuttcap%
\pgfsetmiterjoin%
\definecolor{currentfill}{rgb}{1.000000,1.000000,1.000000}%
\pgfsetfillcolor{currentfill}%
\pgfsetlinewidth{0.000000pt}%
\definecolor{currentstroke}{rgb}{0.000000,0.000000,0.000000}%
\pgfsetstrokecolor{currentstroke}%
\pgfsetstrokeopacity{0.000000}%
\pgfsetdash{}{0pt}%
\pgfpathmoveto{\pgfqpoint{0.735061in}{0.575369in}}%
\pgfpathlineto{\pgfqpoint{2.672561in}{0.575369in}}%
\pgfpathlineto{\pgfqpoint{2.672561in}{2.500369in}}%
\pgfpathlineto{\pgfqpoint{0.735061in}{2.500369in}}%
\pgfpathlineto{\pgfqpoint{0.735061in}{0.575369in}}%
\pgfpathclose%
\pgfusepath{fill}%
\end{pgfscope}%
\begin{pgfscope}%
\pgfsetbuttcap%
\pgfsetroundjoin%
\definecolor{currentfill}{rgb}{0.000000,0.000000,0.000000}%
\pgfsetfillcolor{currentfill}%
\pgfsetlinewidth{0.803000pt}%
\definecolor{currentstroke}{rgb}{0.000000,0.000000,0.000000}%
\pgfsetstrokecolor{currentstroke}%
\pgfsetdash{}{0pt}%
\pgfsys@defobject{currentmarker}{\pgfqpoint{0.000000in}{-0.048611in}}{\pgfqpoint{0.000000in}{0.000000in}}{%
\pgfpathmoveto{\pgfqpoint{0.000000in}{0.000000in}}%
\pgfpathlineto{\pgfqpoint{0.000000in}{-0.048611in}}%
\pgfusepath{stroke,fill}%
}%
\begin{pgfscope}%
\pgfsys@transformshift{1.773721in}{0.575369in}%
\pgfsys@useobject{currentmarker}{}%
\end{pgfscope}%
\end{pgfscope}%
\begin{pgfscope}%
\definecolor{textcolor}{rgb}{0.000000,0.000000,0.000000}%
\pgfsetstrokecolor{textcolor}%
\pgfsetfillcolor{textcolor}%
\pgftext[x=1.773721in,y=0.478146in,,top]{\color{textcolor}{\rmfamily\fontsize{12.000000}{14.400000}\selectfont\catcode`\^=\active\def^{\ifmmode\sp\else\^{}\fi}\catcode`\%=\active\def%{\%}$\mathdefault{10^{3}}$}}%
\end{pgfscope}%
\begin{pgfscope}%
\pgfsetbuttcap%
\pgfsetroundjoin%
\definecolor{currentfill}{rgb}{0.000000,0.000000,0.000000}%
\pgfsetfillcolor{currentfill}%
\pgfsetlinewidth{0.602250pt}%
\definecolor{currentstroke}{rgb}{0.000000,0.000000,0.000000}%
\pgfsetstrokecolor{currentstroke}%
\pgfsetdash{}{0pt}%
\pgfsys@defobject{currentmarker}{\pgfqpoint{0.000000in}{-0.027778in}}{\pgfqpoint{0.000000in}{0.000000in}}{%
\pgfpathmoveto{\pgfqpoint{0.000000in}{0.000000in}}%
\pgfpathlineto{\pgfqpoint{0.000000in}{-0.027778in}}%
\pgfusepath{stroke,fill}%
}%
\begin{pgfscope}%
\pgfsys@transformshift{0.829029in}{0.575369in}%
\pgfsys@useobject{currentmarker}{}%
\end{pgfscope}%
\end{pgfscope}%
\begin{pgfscope}%
\pgfsetbuttcap%
\pgfsetroundjoin%
\definecolor{currentfill}{rgb}{0.000000,0.000000,0.000000}%
\pgfsetfillcolor{currentfill}%
\pgfsetlinewidth{0.602250pt}%
\definecolor{currentstroke}{rgb}{0.000000,0.000000,0.000000}%
\pgfsetstrokecolor{currentstroke}%
\pgfsetdash{}{0pt}%
\pgfsys@defobject{currentmarker}{\pgfqpoint{0.000000in}{-0.027778in}}{\pgfqpoint{0.000000in}{0.000000in}}{%
\pgfpathmoveto{\pgfqpoint{0.000000in}{0.000000in}}%
\pgfpathlineto{\pgfqpoint{0.000000in}{-0.027778in}}%
\pgfusepath{stroke,fill}%
}%
\begin{pgfscope}%
\pgfsys@transformshift{1.067025in}{0.575369in}%
\pgfsys@useobject{currentmarker}{}%
\end{pgfscope}%
\end{pgfscope}%
\begin{pgfscope}%
\pgfsetbuttcap%
\pgfsetroundjoin%
\definecolor{currentfill}{rgb}{0.000000,0.000000,0.000000}%
\pgfsetfillcolor{currentfill}%
\pgfsetlinewidth{0.602250pt}%
\definecolor{currentstroke}{rgb}{0.000000,0.000000,0.000000}%
\pgfsetstrokecolor{currentstroke}%
\pgfsetdash{}{0pt}%
\pgfsys@defobject{currentmarker}{\pgfqpoint{0.000000in}{-0.027778in}}{\pgfqpoint{0.000000in}{0.000000in}}{%
\pgfpathmoveto{\pgfqpoint{0.000000in}{0.000000in}}%
\pgfpathlineto{\pgfqpoint{0.000000in}{-0.027778in}}%
\pgfusepath{stroke,fill}%
}%
\begin{pgfscope}%
\pgfsys@transformshift{1.235886in}{0.575369in}%
\pgfsys@useobject{currentmarker}{}%
\end{pgfscope}%
\end{pgfscope}%
\begin{pgfscope}%
\pgfsetbuttcap%
\pgfsetroundjoin%
\definecolor{currentfill}{rgb}{0.000000,0.000000,0.000000}%
\pgfsetfillcolor{currentfill}%
\pgfsetlinewidth{0.602250pt}%
\definecolor{currentstroke}{rgb}{0.000000,0.000000,0.000000}%
\pgfsetstrokecolor{currentstroke}%
\pgfsetdash{}{0pt}%
\pgfsys@defobject{currentmarker}{\pgfqpoint{0.000000in}{-0.027778in}}{\pgfqpoint{0.000000in}{0.000000in}}{%
\pgfpathmoveto{\pgfqpoint{0.000000in}{0.000000in}}%
\pgfpathlineto{\pgfqpoint{0.000000in}{-0.027778in}}%
\pgfusepath{stroke,fill}%
}%
\begin{pgfscope}%
\pgfsys@transformshift{1.366864in}{0.575369in}%
\pgfsys@useobject{currentmarker}{}%
\end{pgfscope}%
\end{pgfscope}%
\begin{pgfscope}%
\pgfsetbuttcap%
\pgfsetroundjoin%
\definecolor{currentfill}{rgb}{0.000000,0.000000,0.000000}%
\pgfsetfillcolor{currentfill}%
\pgfsetlinewidth{0.602250pt}%
\definecolor{currentstroke}{rgb}{0.000000,0.000000,0.000000}%
\pgfsetstrokecolor{currentstroke}%
\pgfsetdash{}{0pt}%
\pgfsys@defobject{currentmarker}{\pgfqpoint{0.000000in}{-0.027778in}}{\pgfqpoint{0.000000in}{0.000000in}}{%
\pgfpathmoveto{\pgfqpoint{0.000000in}{0.000000in}}%
\pgfpathlineto{\pgfqpoint{0.000000in}{-0.027778in}}%
\pgfusepath{stroke,fill}%
}%
\begin{pgfscope}%
\pgfsys@transformshift{1.473882in}{0.575369in}%
\pgfsys@useobject{currentmarker}{}%
\end{pgfscope}%
\end{pgfscope}%
\begin{pgfscope}%
\pgfsetbuttcap%
\pgfsetroundjoin%
\definecolor{currentfill}{rgb}{0.000000,0.000000,0.000000}%
\pgfsetfillcolor{currentfill}%
\pgfsetlinewidth{0.602250pt}%
\definecolor{currentstroke}{rgb}{0.000000,0.000000,0.000000}%
\pgfsetstrokecolor{currentstroke}%
\pgfsetdash{}{0pt}%
\pgfsys@defobject{currentmarker}{\pgfqpoint{0.000000in}{-0.027778in}}{\pgfqpoint{0.000000in}{0.000000in}}{%
\pgfpathmoveto{\pgfqpoint{0.000000in}{0.000000in}}%
\pgfpathlineto{\pgfqpoint{0.000000in}{-0.027778in}}%
\pgfusepath{stroke,fill}%
}%
\begin{pgfscope}%
\pgfsys@transformshift{1.564364in}{0.575369in}%
\pgfsys@useobject{currentmarker}{}%
\end{pgfscope}%
\end{pgfscope}%
\begin{pgfscope}%
\pgfsetbuttcap%
\pgfsetroundjoin%
\definecolor{currentfill}{rgb}{0.000000,0.000000,0.000000}%
\pgfsetfillcolor{currentfill}%
\pgfsetlinewidth{0.602250pt}%
\definecolor{currentstroke}{rgb}{0.000000,0.000000,0.000000}%
\pgfsetstrokecolor{currentstroke}%
\pgfsetdash{}{0pt}%
\pgfsys@defobject{currentmarker}{\pgfqpoint{0.000000in}{-0.027778in}}{\pgfqpoint{0.000000in}{0.000000in}}{%
\pgfpathmoveto{\pgfqpoint{0.000000in}{0.000000in}}%
\pgfpathlineto{\pgfqpoint{0.000000in}{-0.027778in}}%
\pgfusepath{stroke,fill}%
}%
\begin{pgfscope}%
\pgfsys@transformshift{1.642743in}{0.575369in}%
\pgfsys@useobject{currentmarker}{}%
\end{pgfscope}%
\end{pgfscope}%
\begin{pgfscope}%
\pgfsetbuttcap%
\pgfsetroundjoin%
\definecolor{currentfill}{rgb}{0.000000,0.000000,0.000000}%
\pgfsetfillcolor{currentfill}%
\pgfsetlinewidth{0.602250pt}%
\definecolor{currentstroke}{rgb}{0.000000,0.000000,0.000000}%
\pgfsetstrokecolor{currentstroke}%
\pgfsetdash{}{0pt}%
\pgfsys@defobject{currentmarker}{\pgfqpoint{0.000000in}{-0.027778in}}{\pgfqpoint{0.000000in}{0.000000in}}{%
\pgfpathmoveto{\pgfqpoint{0.000000in}{0.000000in}}%
\pgfpathlineto{\pgfqpoint{0.000000in}{-0.027778in}}%
\pgfusepath{stroke,fill}%
}%
\begin{pgfscope}%
\pgfsys@transformshift{1.711878in}{0.575369in}%
\pgfsys@useobject{currentmarker}{}%
\end{pgfscope}%
\end{pgfscope}%
\begin{pgfscope}%
\pgfsetbuttcap%
\pgfsetroundjoin%
\definecolor{currentfill}{rgb}{0.000000,0.000000,0.000000}%
\pgfsetfillcolor{currentfill}%
\pgfsetlinewidth{0.602250pt}%
\definecolor{currentstroke}{rgb}{0.000000,0.000000,0.000000}%
\pgfsetstrokecolor{currentstroke}%
\pgfsetdash{}{0pt}%
\pgfsys@defobject{currentmarker}{\pgfqpoint{0.000000in}{-0.027778in}}{\pgfqpoint{0.000000in}{0.000000in}}{%
\pgfpathmoveto{\pgfqpoint{0.000000in}{0.000000in}}%
\pgfpathlineto{\pgfqpoint{0.000000in}{-0.027778in}}%
\pgfusepath{stroke,fill}%
}%
\begin{pgfscope}%
\pgfsys@transformshift{2.180578in}{0.575369in}%
\pgfsys@useobject{currentmarker}{}%
\end{pgfscope}%
\end{pgfscope}%
\begin{pgfscope}%
\pgfsetbuttcap%
\pgfsetroundjoin%
\definecolor{currentfill}{rgb}{0.000000,0.000000,0.000000}%
\pgfsetfillcolor{currentfill}%
\pgfsetlinewidth{0.602250pt}%
\definecolor{currentstroke}{rgb}{0.000000,0.000000,0.000000}%
\pgfsetstrokecolor{currentstroke}%
\pgfsetdash{}{0pt}%
\pgfsys@defobject{currentmarker}{\pgfqpoint{0.000000in}{-0.027778in}}{\pgfqpoint{0.000000in}{0.000000in}}{%
\pgfpathmoveto{\pgfqpoint{0.000000in}{0.000000in}}%
\pgfpathlineto{\pgfqpoint{0.000000in}{-0.027778in}}%
\pgfusepath{stroke,fill}%
}%
\begin{pgfscope}%
\pgfsys@transformshift{2.418575in}{0.575369in}%
\pgfsys@useobject{currentmarker}{}%
\end{pgfscope}%
\end{pgfscope}%
\begin{pgfscope}%
\pgfsetbuttcap%
\pgfsetroundjoin%
\definecolor{currentfill}{rgb}{0.000000,0.000000,0.000000}%
\pgfsetfillcolor{currentfill}%
\pgfsetlinewidth{0.602250pt}%
\definecolor{currentstroke}{rgb}{0.000000,0.000000,0.000000}%
\pgfsetstrokecolor{currentstroke}%
\pgfsetdash{}{0pt}%
\pgfsys@defobject{currentmarker}{\pgfqpoint{0.000000in}{-0.027778in}}{\pgfqpoint{0.000000in}{0.000000in}}{%
\pgfpathmoveto{\pgfqpoint{0.000000in}{0.000000in}}%
\pgfpathlineto{\pgfqpoint{0.000000in}{-0.027778in}}%
\pgfusepath{stroke,fill}%
}%
\begin{pgfscope}%
\pgfsys@transformshift{2.587435in}{0.575369in}%
\pgfsys@useobject{currentmarker}{}%
\end{pgfscope}%
\end{pgfscope}%
\begin{pgfscope}%
\definecolor{textcolor}{rgb}{0.000000,0.000000,0.000000}%
\pgfsetstrokecolor{textcolor}%
\pgfsetfillcolor{textcolor}%
\pgftext[x=1.703811in,y=0.261295in,,top]{\color{textcolor}{\rmfamily\fontsize{12.000000}{14.400000}\selectfont\catcode`\^=\active\def^{\ifmmode\sp\else\^{}\fi}\catcode`\%=\active\def%{\%}$m$}}%
\end{pgfscope}%
\begin{pgfscope}%
\pgfsetbuttcap%
\pgfsetroundjoin%
\definecolor{currentfill}{rgb}{0.000000,0.000000,0.000000}%
\pgfsetfillcolor{currentfill}%
\pgfsetlinewidth{0.803000pt}%
\definecolor{currentstroke}{rgb}{0.000000,0.000000,0.000000}%
\pgfsetstrokecolor{currentstroke}%
\pgfsetdash{}{0pt}%
\pgfsys@defobject{currentmarker}{\pgfqpoint{-0.048611in}{0.000000in}}{\pgfqpoint{-0.000000in}{0.000000in}}{%
\pgfpathmoveto{\pgfqpoint{-0.000000in}{0.000000in}}%
\pgfpathlineto{\pgfqpoint{-0.048611in}{0.000000in}}%
\pgfusepath{stroke,fill}%
}%
\begin{pgfscope}%
\pgfsys@transformshift{0.735061in}{0.614155in}%
\pgfsys@useobject{currentmarker}{}%
\end{pgfscope}%
\end{pgfscope}%
\begin{pgfscope}%
\definecolor{textcolor}{rgb}{0.000000,0.000000,0.000000}%
\pgfsetstrokecolor{textcolor}%
\pgfsetfillcolor{textcolor}%
\pgftext[x=0.316851in, y=0.550842in, left, base]{\color{textcolor}{\rmfamily\fontsize{12.000000}{14.400000}\selectfont\catcode`\^=\active\def^{\ifmmode\sp\else\^{}\fi}\catcode`\%=\active\def%{\%}$\mathdefault{10^{-2}}$}}%
\end{pgfscope}%
\begin{pgfscope}%
\pgfsetbuttcap%
\pgfsetroundjoin%
\definecolor{currentfill}{rgb}{0.000000,0.000000,0.000000}%
\pgfsetfillcolor{currentfill}%
\pgfsetlinewidth{0.803000pt}%
\definecolor{currentstroke}{rgb}{0.000000,0.000000,0.000000}%
\pgfsetstrokecolor{currentstroke}%
\pgfsetdash{}{0pt}%
\pgfsys@defobject{currentmarker}{\pgfqpoint{-0.048611in}{0.000000in}}{\pgfqpoint{-0.000000in}{0.000000in}}{%
\pgfpathmoveto{\pgfqpoint{-0.000000in}{0.000000in}}%
\pgfpathlineto{\pgfqpoint{-0.048611in}{0.000000in}}%
\pgfusepath{stroke,fill}%
}%
\begin{pgfscope}%
\pgfsys@transformshift{0.735061in}{1.556413in}%
\pgfsys@useobject{currentmarker}{}%
\end{pgfscope}%
\end{pgfscope}%
\begin{pgfscope}%
\definecolor{textcolor}{rgb}{0.000000,0.000000,0.000000}%
\pgfsetstrokecolor{textcolor}%
\pgfsetfillcolor{textcolor}%
\pgftext[x=0.316851in, y=1.493099in, left, base]{\color{textcolor}{\rmfamily\fontsize{12.000000}{14.400000}\selectfont\catcode`\^=\active\def^{\ifmmode\sp\else\^{}\fi}\catcode`\%=\active\def%{\%}$\mathdefault{10^{-1}}$}}%
\end{pgfscope}%
\begin{pgfscope}%
\pgfsetbuttcap%
\pgfsetroundjoin%
\definecolor{currentfill}{rgb}{0.000000,0.000000,0.000000}%
\pgfsetfillcolor{currentfill}%
\pgfsetlinewidth{0.803000pt}%
\definecolor{currentstroke}{rgb}{0.000000,0.000000,0.000000}%
\pgfsetstrokecolor{currentstroke}%
\pgfsetdash{}{0pt}%
\pgfsys@defobject{currentmarker}{\pgfqpoint{-0.048611in}{0.000000in}}{\pgfqpoint{-0.000000in}{0.000000in}}{%
\pgfpathmoveto{\pgfqpoint{-0.000000in}{0.000000in}}%
\pgfpathlineto{\pgfqpoint{-0.048611in}{0.000000in}}%
\pgfusepath{stroke,fill}%
}%
\begin{pgfscope}%
\pgfsys@transformshift{0.735061in}{2.498670in}%
\pgfsys@useobject{currentmarker}{}%
\end{pgfscope}%
\end{pgfscope}%
\begin{pgfscope}%
\definecolor{textcolor}{rgb}{0.000000,0.000000,0.000000}%
\pgfsetstrokecolor{textcolor}%
\pgfsetfillcolor{textcolor}%
\pgftext[x=0.408673in, y=2.435356in, left, base]{\color{textcolor}{\rmfamily\fontsize{12.000000}{14.400000}\selectfont\catcode`\^=\active\def^{\ifmmode\sp\else\^{}\fi}\catcode`\%=\active\def%{\%}$\mathdefault{10^{0}}$}}%
\end{pgfscope}%
\begin{pgfscope}%
\pgfsetbuttcap%
\pgfsetroundjoin%
\definecolor{currentfill}{rgb}{0.000000,0.000000,0.000000}%
\pgfsetfillcolor{currentfill}%
\pgfsetlinewidth{0.602250pt}%
\definecolor{currentstroke}{rgb}{0.000000,0.000000,0.000000}%
\pgfsetstrokecolor{currentstroke}%
\pgfsetdash{}{0pt}%
\pgfsys@defobject{currentmarker}{\pgfqpoint{-0.027778in}{0.000000in}}{\pgfqpoint{-0.000000in}{0.000000in}}{%
\pgfpathmoveto{\pgfqpoint{-0.000000in}{0.000000in}}%
\pgfpathlineto{\pgfqpoint{-0.027778in}{0.000000in}}%
\pgfusepath{stroke,fill}%
}%
\begin{pgfscope}%
\pgfsys@transformshift{0.735061in}{0.897803in}%
\pgfsys@useobject{currentmarker}{}%
\end{pgfscope}%
\end{pgfscope}%
\begin{pgfscope}%
\pgfsetbuttcap%
\pgfsetroundjoin%
\definecolor{currentfill}{rgb}{0.000000,0.000000,0.000000}%
\pgfsetfillcolor{currentfill}%
\pgfsetlinewidth{0.602250pt}%
\definecolor{currentstroke}{rgb}{0.000000,0.000000,0.000000}%
\pgfsetstrokecolor{currentstroke}%
\pgfsetdash{}{0pt}%
\pgfsys@defobject{currentmarker}{\pgfqpoint{-0.027778in}{0.000000in}}{\pgfqpoint{-0.000000in}{0.000000in}}{%
\pgfpathmoveto{\pgfqpoint{-0.000000in}{0.000000in}}%
\pgfpathlineto{\pgfqpoint{-0.027778in}{0.000000in}}%
\pgfusepath{stroke,fill}%
}%
\begin{pgfscope}%
\pgfsys@transformshift{0.735061in}{1.063726in}%
\pgfsys@useobject{currentmarker}{}%
\end{pgfscope}%
\end{pgfscope}%
\begin{pgfscope}%
\pgfsetbuttcap%
\pgfsetroundjoin%
\definecolor{currentfill}{rgb}{0.000000,0.000000,0.000000}%
\pgfsetfillcolor{currentfill}%
\pgfsetlinewidth{0.602250pt}%
\definecolor{currentstroke}{rgb}{0.000000,0.000000,0.000000}%
\pgfsetstrokecolor{currentstroke}%
\pgfsetdash{}{0pt}%
\pgfsys@defobject{currentmarker}{\pgfqpoint{-0.027778in}{0.000000in}}{\pgfqpoint{-0.000000in}{0.000000in}}{%
\pgfpathmoveto{\pgfqpoint{-0.000000in}{0.000000in}}%
\pgfpathlineto{\pgfqpoint{-0.027778in}{0.000000in}}%
\pgfusepath{stroke,fill}%
}%
\begin{pgfscope}%
\pgfsys@transformshift{0.735061in}{1.181451in}%
\pgfsys@useobject{currentmarker}{}%
\end{pgfscope}%
\end{pgfscope}%
\begin{pgfscope}%
\pgfsetbuttcap%
\pgfsetroundjoin%
\definecolor{currentfill}{rgb}{0.000000,0.000000,0.000000}%
\pgfsetfillcolor{currentfill}%
\pgfsetlinewidth{0.602250pt}%
\definecolor{currentstroke}{rgb}{0.000000,0.000000,0.000000}%
\pgfsetstrokecolor{currentstroke}%
\pgfsetdash{}{0pt}%
\pgfsys@defobject{currentmarker}{\pgfqpoint{-0.027778in}{0.000000in}}{\pgfqpoint{-0.000000in}{0.000000in}}{%
\pgfpathmoveto{\pgfqpoint{-0.000000in}{0.000000in}}%
\pgfpathlineto{\pgfqpoint{-0.027778in}{0.000000in}}%
\pgfusepath{stroke,fill}%
}%
\begin{pgfscope}%
\pgfsys@transformshift{0.735061in}{1.272765in}%
\pgfsys@useobject{currentmarker}{}%
\end{pgfscope}%
\end{pgfscope}%
\begin{pgfscope}%
\pgfsetbuttcap%
\pgfsetroundjoin%
\definecolor{currentfill}{rgb}{0.000000,0.000000,0.000000}%
\pgfsetfillcolor{currentfill}%
\pgfsetlinewidth{0.602250pt}%
\definecolor{currentstroke}{rgb}{0.000000,0.000000,0.000000}%
\pgfsetstrokecolor{currentstroke}%
\pgfsetdash{}{0pt}%
\pgfsys@defobject{currentmarker}{\pgfqpoint{-0.027778in}{0.000000in}}{\pgfqpoint{-0.000000in}{0.000000in}}{%
\pgfpathmoveto{\pgfqpoint{-0.000000in}{0.000000in}}%
\pgfpathlineto{\pgfqpoint{-0.027778in}{0.000000in}}%
\pgfusepath{stroke,fill}%
}%
\begin{pgfscope}%
\pgfsys@transformshift{0.735061in}{1.347374in}%
\pgfsys@useobject{currentmarker}{}%
\end{pgfscope}%
\end{pgfscope}%
\begin{pgfscope}%
\pgfsetbuttcap%
\pgfsetroundjoin%
\definecolor{currentfill}{rgb}{0.000000,0.000000,0.000000}%
\pgfsetfillcolor{currentfill}%
\pgfsetlinewidth{0.602250pt}%
\definecolor{currentstroke}{rgb}{0.000000,0.000000,0.000000}%
\pgfsetstrokecolor{currentstroke}%
\pgfsetdash{}{0pt}%
\pgfsys@defobject{currentmarker}{\pgfqpoint{-0.027778in}{0.000000in}}{\pgfqpoint{-0.000000in}{0.000000in}}{%
\pgfpathmoveto{\pgfqpoint{-0.000000in}{0.000000in}}%
\pgfpathlineto{\pgfqpoint{-0.027778in}{0.000000in}}%
\pgfusepath{stroke,fill}%
}%
\begin{pgfscope}%
\pgfsys@transformshift{0.735061in}{1.410455in}%
\pgfsys@useobject{currentmarker}{}%
\end{pgfscope}%
\end{pgfscope}%
\begin{pgfscope}%
\pgfsetbuttcap%
\pgfsetroundjoin%
\definecolor{currentfill}{rgb}{0.000000,0.000000,0.000000}%
\pgfsetfillcolor{currentfill}%
\pgfsetlinewidth{0.602250pt}%
\definecolor{currentstroke}{rgb}{0.000000,0.000000,0.000000}%
\pgfsetstrokecolor{currentstroke}%
\pgfsetdash{}{0pt}%
\pgfsys@defobject{currentmarker}{\pgfqpoint{-0.027778in}{0.000000in}}{\pgfqpoint{-0.000000in}{0.000000in}}{%
\pgfpathmoveto{\pgfqpoint{-0.000000in}{0.000000in}}%
\pgfpathlineto{\pgfqpoint{-0.027778in}{0.000000in}}%
\pgfusepath{stroke,fill}%
}%
\begin{pgfscope}%
\pgfsys@transformshift{0.735061in}{1.465099in}%
\pgfsys@useobject{currentmarker}{}%
\end{pgfscope}%
\end{pgfscope}%
\begin{pgfscope}%
\pgfsetbuttcap%
\pgfsetroundjoin%
\definecolor{currentfill}{rgb}{0.000000,0.000000,0.000000}%
\pgfsetfillcolor{currentfill}%
\pgfsetlinewidth{0.602250pt}%
\definecolor{currentstroke}{rgb}{0.000000,0.000000,0.000000}%
\pgfsetstrokecolor{currentstroke}%
\pgfsetdash{}{0pt}%
\pgfsys@defobject{currentmarker}{\pgfqpoint{-0.027778in}{0.000000in}}{\pgfqpoint{-0.000000in}{0.000000in}}{%
\pgfpathmoveto{\pgfqpoint{-0.000000in}{0.000000in}}%
\pgfpathlineto{\pgfqpoint{-0.027778in}{0.000000in}}%
\pgfusepath{stroke,fill}%
}%
\begin{pgfscope}%
\pgfsys@transformshift{0.735061in}{1.513298in}%
\pgfsys@useobject{currentmarker}{}%
\end{pgfscope}%
\end{pgfscope}%
\begin{pgfscope}%
\pgfsetbuttcap%
\pgfsetroundjoin%
\definecolor{currentfill}{rgb}{0.000000,0.000000,0.000000}%
\pgfsetfillcolor{currentfill}%
\pgfsetlinewidth{0.602250pt}%
\definecolor{currentstroke}{rgb}{0.000000,0.000000,0.000000}%
\pgfsetstrokecolor{currentstroke}%
\pgfsetdash{}{0pt}%
\pgfsys@defobject{currentmarker}{\pgfqpoint{-0.027778in}{0.000000in}}{\pgfqpoint{-0.000000in}{0.000000in}}{%
\pgfpathmoveto{\pgfqpoint{-0.000000in}{0.000000in}}%
\pgfpathlineto{\pgfqpoint{-0.027778in}{0.000000in}}%
\pgfusepath{stroke,fill}%
}%
\begin{pgfscope}%
\pgfsys@transformshift{0.735061in}{1.840061in}%
\pgfsys@useobject{currentmarker}{}%
\end{pgfscope}%
\end{pgfscope}%
\begin{pgfscope}%
\pgfsetbuttcap%
\pgfsetroundjoin%
\definecolor{currentfill}{rgb}{0.000000,0.000000,0.000000}%
\pgfsetfillcolor{currentfill}%
\pgfsetlinewidth{0.602250pt}%
\definecolor{currentstroke}{rgb}{0.000000,0.000000,0.000000}%
\pgfsetstrokecolor{currentstroke}%
\pgfsetdash{}{0pt}%
\pgfsys@defobject{currentmarker}{\pgfqpoint{-0.027778in}{0.000000in}}{\pgfqpoint{-0.000000in}{0.000000in}}{%
\pgfpathmoveto{\pgfqpoint{-0.000000in}{0.000000in}}%
\pgfpathlineto{\pgfqpoint{-0.027778in}{0.000000in}}%
\pgfusepath{stroke,fill}%
}%
\begin{pgfscope}%
\pgfsys@transformshift{0.735061in}{2.005984in}%
\pgfsys@useobject{currentmarker}{}%
\end{pgfscope}%
\end{pgfscope}%
\begin{pgfscope}%
\pgfsetbuttcap%
\pgfsetroundjoin%
\definecolor{currentfill}{rgb}{0.000000,0.000000,0.000000}%
\pgfsetfillcolor{currentfill}%
\pgfsetlinewidth{0.602250pt}%
\definecolor{currentstroke}{rgb}{0.000000,0.000000,0.000000}%
\pgfsetstrokecolor{currentstroke}%
\pgfsetdash{}{0pt}%
\pgfsys@defobject{currentmarker}{\pgfqpoint{-0.027778in}{0.000000in}}{\pgfqpoint{-0.000000in}{0.000000in}}{%
\pgfpathmoveto{\pgfqpoint{-0.000000in}{0.000000in}}%
\pgfpathlineto{\pgfqpoint{-0.027778in}{0.000000in}}%
\pgfusepath{stroke,fill}%
}%
\begin{pgfscope}%
\pgfsys@transformshift{0.735061in}{2.123708in}%
\pgfsys@useobject{currentmarker}{}%
\end{pgfscope}%
\end{pgfscope}%
\begin{pgfscope}%
\pgfsetbuttcap%
\pgfsetroundjoin%
\definecolor{currentfill}{rgb}{0.000000,0.000000,0.000000}%
\pgfsetfillcolor{currentfill}%
\pgfsetlinewidth{0.602250pt}%
\definecolor{currentstroke}{rgb}{0.000000,0.000000,0.000000}%
\pgfsetstrokecolor{currentstroke}%
\pgfsetdash{}{0pt}%
\pgfsys@defobject{currentmarker}{\pgfqpoint{-0.027778in}{0.000000in}}{\pgfqpoint{-0.000000in}{0.000000in}}{%
\pgfpathmoveto{\pgfqpoint{-0.000000in}{0.000000in}}%
\pgfpathlineto{\pgfqpoint{-0.027778in}{0.000000in}}%
\pgfusepath{stroke,fill}%
}%
\begin{pgfscope}%
\pgfsys@transformshift{0.735061in}{2.215023in}%
\pgfsys@useobject{currentmarker}{}%
\end{pgfscope}%
\end{pgfscope}%
\begin{pgfscope}%
\pgfsetbuttcap%
\pgfsetroundjoin%
\definecolor{currentfill}{rgb}{0.000000,0.000000,0.000000}%
\pgfsetfillcolor{currentfill}%
\pgfsetlinewidth{0.602250pt}%
\definecolor{currentstroke}{rgb}{0.000000,0.000000,0.000000}%
\pgfsetstrokecolor{currentstroke}%
\pgfsetdash{}{0pt}%
\pgfsys@defobject{currentmarker}{\pgfqpoint{-0.027778in}{0.000000in}}{\pgfqpoint{-0.000000in}{0.000000in}}{%
\pgfpathmoveto{\pgfqpoint{-0.000000in}{0.000000in}}%
\pgfpathlineto{\pgfqpoint{-0.027778in}{0.000000in}}%
\pgfusepath{stroke,fill}%
}%
\begin{pgfscope}%
\pgfsys@transformshift{0.735061in}{2.289632in}%
\pgfsys@useobject{currentmarker}{}%
\end{pgfscope}%
\end{pgfscope}%
\begin{pgfscope}%
\pgfsetbuttcap%
\pgfsetroundjoin%
\definecolor{currentfill}{rgb}{0.000000,0.000000,0.000000}%
\pgfsetfillcolor{currentfill}%
\pgfsetlinewidth{0.602250pt}%
\definecolor{currentstroke}{rgb}{0.000000,0.000000,0.000000}%
\pgfsetstrokecolor{currentstroke}%
\pgfsetdash{}{0pt}%
\pgfsys@defobject{currentmarker}{\pgfqpoint{-0.027778in}{0.000000in}}{\pgfqpoint{-0.000000in}{0.000000in}}{%
\pgfpathmoveto{\pgfqpoint{-0.000000in}{0.000000in}}%
\pgfpathlineto{\pgfqpoint{-0.027778in}{0.000000in}}%
\pgfusepath{stroke,fill}%
}%
\begin{pgfscope}%
\pgfsys@transformshift{0.735061in}{2.352713in}%
\pgfsys@useobject{currentmarker}{}%
\end{pgfscope}%
\end{pgfscope}%
\begin{pgfscope}%
\pgfsetbuttcap%
\pgfsetroundjoin%
\definecolor{currentfill}{rgb}{0.000000,0.000000,0.000000}%
\pgfsetfillcolor{currentfill}%
\pgfsetlinewidth{0.602250pt}%
\definecolor{currentstroke}{rgb}{0.000000,0.000000,0.000000}%
\pgfsetstrokecolor{currentstroke}%
\pgfsetdash{}{0pt}%
\pgfsys@defobject{currentmarker}{\pgfqpoint{-0.027778in}{0.000000in}}{\pgfqpoint{-0.000000in}{0.000000in}}{%
\pgfpathmoveto{\pgfqpoint{-0.000000in}{0.000000in}}%
\pgfpathlineto{\pgfqpoint{-0.027778in}{0.000000in}}%
\pgfusepath{stroke,fill}%
}%
\begin{pgfscope}%
\pgfsys@transformshift{0.735061in}{2.407356in}%
\pgfsys@useobject{currentmarker}{}%
\end{pgfscope}%
\end{pgfscope}%
\begin{pgfscope}%
\pgfsetbuttcap%
\pgfsetroundjoin%
\definecolor{currentfill}{rgb}{0.000000,0.000000,0.000000}%
\pgfsetfillcolor{currentfill}%
\pgfsetlinewidth{0.602250pt}%
\definecolor{currentstroke}{rgb}{0.000000,0.000000,0.000000}%
\pgfsetstrokecolor{currentstroke}%
\pgfsetdash{}{0pt}%
\pgfsys@defobject{currentmarker}{\pgfqpoint{-0.027778in}{0.000000in}}{\pgfqpoint{-0.000000in}{0.000000in}}{%
\pgfpathmoveto{\pgfqpoint{-0.000000in}{0.000000in}}%
\pgfpathlineto{\pgfqpoint{-0.027778in}{0.000000in}}%
\pgfusepath{stroke,fill}%
}%
\begin{pgfscope}%
\pgfsys@transformshift{0.735061in}{2.455555in}%
\pgfsys@useobject{currentmarker}{}%
\end{pgfscope}%
\end{pgfscope}%
\begin{pgfscope}%
\definecolor{textcolor}{rgb}{0.000000,0.000000,0.000000}%
\pgfsetstrokecolor{textcolor}%
\pgfsetfillcolor{textcolor}%
\pgftext[x=0.261295in,y=1.537869in,,bottom,rotate=90.000000]{\color{textcolor}{\rmfamily\fontsize{12.000000}{14.400000}\selectfont\catcode`\^=\active\def^{\ifmmode\sp\else\^{}\fi}\catcode`\%=\active\def%{\%}$L^1$ relative error}}%
\end{pgfscope}%
\begin{pgfscope}%
\pgfpathrectangle{\pgfqpoint{0.735061in}{0.575369in}}{\pgfqpoint{1.937500in}{1.925000in}}%
\pgfusepath{clip}%
\pgfsetrectcap%
\pgfsetroundjoin%
\pgfsetlinewidth{1.003750pt}%
\definecolor{currentstroke}{rgb}{0.001462,0.000466,0.013866}%
\pgfsetstrokecolor{currentstroke}%
\pgfsetdash{}{0pt}%
\pgfpathmoveto{\pgfqpoint{0.823130in}{2.412869in}}%
\pgfpathlineto{\pgfqpoint{1.177294in}{1.997864in}}%
\pgfpathlineto{\pgfqpoint{1.529826in}{1.460633in}}%
\pgfpathlineto{\pgfqpoint{1.881716in}{1.266239in}}%
\pgfpathlineto{\pgfqpoint{2.232776in}{1.266111in}}%
\pgfpathlineto{\pgfqpoint{2.584493in}{1.266111in}}%
\pgfusepath{stroke}%
\end{pgfscope}%
\begin{pgfscope}%
\pgfpathrectangle{\pgfqpoint{0.735061in}{0.575369in}}{\pgfqpoint{1.937500in}{1.925000in}}%
\pgfusepath{clip}%
\pgfsetbuttcap%
\pgfsetroundjoin%
\definecolor{currentfill}{rgb}{0.001462,0.000466,0.013866}%
\pgfsetfillcolor{currentfill}%
\pgfsetlinewidth{1.003750pt}%
\definecolor{currentstroke}{rgb}{0.001462,0.000466,0.013866}%
\pgfsetstrokecolor{currentstroke}%
\pgfsetdash{}{0pt}%
\pgfsys@defobject{currentmarker}{\pgfqpoint{-0.020833in}{-0.020833in}}{\pgfqpoint{0.020833in}{0.020833in}}{%
\pgfpathmoveto{\pgfqpoint{0.000000in}{-0.020833in}}%
\pgfpathcurveto{\pgfqpoint{0.005525in}{-0.020833in}}{\pgfqpoint{0.010825in}{-0.018638in}}{\pgfqpoint{0.014731in}{-0.014731in}}%
\pgfpathcurveto{\pgfqpoint{0.018638in}{-0.010825in}}{\pgfqpoint{0.020833in}{-0.005525in}}{\pgfqpoint{0.020833in}{0.000000in}}%
\pgfpathcurveto{\pgfqpoint{0.020833in}{0.005525in}}{\pgfqpoint{0.018638in}{0.010825in}}{\pgfqpoint{0.014731in}{0.014731in}}%
\pgfpathcurveto{\pgfqpoint{0.010825in}{0.018638in}}{\pgfqpoint{0.005525in}{0.020833in}}{\pgfqpoint{0.000000in}{0.020833in}}%
\pgfpathcurveto{\pgfqpoint{-0.005525in}{0.020833in}}{\pgfqpoint{-0.010825in}{0.018638in}}{\pgfqpoint{-0.014731in}{0.014731in}}%
\pgfpathcurveto{\pgfqpoint{-0.018638in}{0.010825in}}{\pgfqpoint{-0.020833in}{0.005525in}}{\pgfqpoint{-0.020833in}{0.000000in}}%
\pgfpathcurveto{\pgfqpoint{-0.020833in}{-0.005525in}}{\pgfqpoint{-0.018638in}{-0.010825in}}{\pgfqpoint{-0.014731in}{-0.014731in}}%
\pgfpathcurveto{\pgfqpoint{-0.010825in}{-0.018638in}}{\pgfqpoint{-0.005525in}{-0.020833in}}{\pgfqpoint{0.000000in}{-0.020833in}}%
\pgfpathlineto{\pgfqpoint{0.000000in}{-0.020833in}}%
\pgfpathclose%
\pgfusepath{stroke,fill}%
}%
\begin{pgfscope}%
\pgfsys@transformshift{0.823130in}{2.412869in}%
\pgfsys@useobject{currentmarker}{}%
\end{pgfscope}%
\begin{pgfscope}%
\pgfsys@transformshift{1.177294in}{1.997864in}%
\pgfsys@useobject{currentmarker}{}%
\end{pgfscope}%
\begin{pgfscope}%
\pgfsys@transformshift{1.529826in}{1.460633in}%
\pgfsys@useobject{currentmarker}{}%
\end{pgfscope}%
\begin{pgfscope}%
\pgfsys@transformshift{1.881716in}{1.266239in}%
\pgfsys@useobject{currentmarker}{}%
\end{pgfscope}%
\begin{pgfscope}%
\pgfsys@transformshift{2.232776in}{1.266111in}%
\pgfsys@useobject{currentmarker}{}%
\end{pgfscope}%
\begin{pgfscope}%
\pgfsys@transformshift{2.584493in}{1.266111in}%
\pgfsys@useobject{currentmarker}{}%
\end{pgfscope}%
\end{pgfscope}%
\begin{pgfscope}%
\pgfpathrectangle{\pgfqpoint{0.735061in}{0.575369in}}{\pgfqpoint{1.937500in}{1.925000in}}%
\pgfusepath{clip}%
\pgfsetrectcap%
\pgfsetroundjoin%
\pgfsetlinewidth{1.003750pt}%
\definecolor{currentstroke}{rgb}{0.445163,0.122724,0.506901}%
\pgfsetstrokecolor{currentstroke}%
\pgfsetdash{}{0pt}%
\pgfpathmoveto{\pgfqpoint{0.823130in}{2.389332in}}%
\pgfpathlineto{\pgfqpoint{1.177294in}{2.167805in}}%
\pgfpathlineto{\pgfqpoint{1.529826in}{1.731079in}}%
\pgfpathlineto{\pgfqpoint{1.881716in}{1.313047in}}%
\pgfpathlineto{\pgfqpoint{2.232776in}{1.297082in}}%
\pgfpathlineto{\pgfqpoint{2.584493in}{1.297082in}}%
\pgfusepath{stroke}%
\end{pgfscope}%
\begin{pgfscope}%
\pgfpathrectangle{\pgfqpoint{0.735061in}{0.575369in}}{\pgfqpoint{1.937500in}{1.925000in}}%
\pgfusepath{clip}%
\pgfsetbuttcap%
\pgfsetroundjoin%
\definecolor{currentfill}{rgb}{0.445163,0.122724,0.506901}%
\pgfsetfillcolor{currentfill}%
\pgfsetlinewidth{1.003750pt}%
\definecolor{currentstroke}{rgb}{0.445163,0.122724,0.506901}%
\pgfsetstrokecolor{currentstroke}%
\pgfsetdash{}{0pt}%
\pgfsys@defobject{currentmarker}{\pgfqpoint{-0.020833in}{-0.020833in}}{\pgfqpoint{0.020833in}{0.020833in}}{%
\pgfpathmoveto{\pgfqpoint{0.000000in}{-0.020833in}}%
\pgfpathcurveto{\pgfqpoint{0.005525in}{-0.020833in}}{\pgfqpoint{0.010825in}{-0.018638in}}{\pgfqpoint{0.014731in}{-0.014731in}}%
\pgfpathcurveto{\pgfqpoint{0.018638in}{-0.010825in}}{\pgfqpoint{0.020833in}{-0.005525in}}{\pgfqpoint{0.020833in}{0.000000in}}%
\pgfpathcurveto{\pgfqpoint{0.020833in}{0.005525in}}{\pgfqpoint{0.018638in}{0.010825in}}{\pgfqpoint{0.014731in}{0.014731in}}%
\pgfpathcurveto{\pgfqpoint{0.010825in}{0.018638in}}{\pgfqpoint{0.005525in}{0.020833in}}{\pgfqpoint{0.000000in}{0.020833in}}%
\pgfpathcurveto{\pgfqpoint{-0.005525in}{0.020833in}}{\pgfqpoint{-0.010825in}{0.018638in}}{\pgfqpoint{-0.014731in}{0.014731in}}%
\pgfpathcurveto{\pgfqpoint{-0.018638in}{0.010825in}}{\pgfqpoint{-0.020833in}{0.005525in}}{\pgfqpoint{-0.020833in}{0.000000in}}%
\pgfpathcurveto{\pgfqpoint{-0.020833in}{-0.005525in}}{\pgfqpoint{-0.018638in}{-0.010825in}}{\pgfqpoint{-0.014731in}{-0.014731in}}%
\pgfpathcurveto{\pgfqpoint{-0.010825in}{-0.018638in}}{\pgfqpoint{-0.005525in}{-0.020833in}}{\pgfqpoint{0.000000in}{-0.020833in}}%
\pgfpathlineto{\pgfqpoint{0.000000in}{-0.020833in}}%
\pgfpathclose%
\pgfusepath{stroke,fill}%
}%
\begin{pgfscope}%
\pgfsys@transformshift{0.823130in}{2.389332in}%
\pgfsys@useobject{currentmarker}{}%
\end{pgfscope}%
\begin{pgfscope}%
\pgfsys@transformshift{1.177294in}{2.167805in}%
\pgfsys@useobject{currentmarker}{}%
\end{pgfscope}%
\begin{pgfscope}%
\pgfsys@transformshift{1.529826in}{1.731079in}%
\pgfsys@useobject{currentmarker}{}%
\end{pgfscope}%
\begin{pgfscope}%
\pgfsys@transformshift{1.881716in}{1.313047in}%
\pgfsys@useobject{currentmarker}{}%
\end{pgfscope}%
\begin{pgfscope}%
\pgfsys@transformshift{2.232776in}{1.297082in}%
\pgfsys@useobject{currentmarker}{}%
\end{pgfscope}%
\begin{pgfscope}%
\pgfsys@transformshift{2.584493in}{1.297082in}%
\pgfsys@useobject{currentmarker}{}%
\end{pgfscope}%
\end{pgfscope}%
\begin{pgfscope}%
\pgfpathrectangle{\pgfqpoint{0.735061in}{0.575369in}}{\pgfqpoint{1.937500in}{1.925000in}}%
\pgfusepath{clip}%
\pgfsetrectcap%
\pgfsetroundjoin%
\pgfsetlinewidth{1.003750pt}%
\definecolor{currentstroke}{rgb}{0.944006,0.377643,0.365136}%
\pgfsetstrokecolor{currentstroke}%
\pgfsetdash{}{0pt}%
\pgfpathmoveto{\pgfqpoint{0.823130in}{2.377649in}}%
\pgfpathlineto{\pgfqpoint{1.177294in}{1.959031in}}%
\pgfpathlineto{\pgfqpoint{1.529826in}{1.287600in}}%
\pgfpathlineto{\pgfqpoint{1.881716in}{0.675678in}}%
\pgfpathlineto{\pgfqpoint{2.232776in}{0.662869in}}%
\pgfpathlineto{\pgfqpoint{2.584493in}{0.662869in}}%
\pgfusepath{stroke}%
\end{pgfscope}%
\begin{pgfscope}%
\pgfpathrectangle{\pgfqpoint{0.735061in}{0.575369in}}{\pgfqpoint{1.937500in}{1.925000in}}%
\pgfusepath{clip}%
\pgfsetbuttcap%
\pgfsetroundjoin%
\definecolor{currentfill}{rgb}{0.944006,0.377643,0.365136}%
\pgfsetfillcolor{currentfill}%
\pgfsetlinewidth{1.003750pt}%
\definecolor{currentstroke}{rgb}{0.944006,0.377643,0.365136}%
\pgfsetstrokecolor{currentstroke}%
\pgfsetdash{}{0pt}%
\pgfsys@defobject{currentmarker}{\pgfqpoint{-0.020833in}{-0.020833in}}{\pgfqpoint{0.020833in}{0.020833in}}{%
\pgfpathmoveto{\pgfqpoint{0.000000in}{-0.020833in}}%
\pgfpathcurveto{\pgfqpoint{0.005525in}{-0.020833in}}{\pgfqpoint{0.010825in}{-0.018638in}}{\pgfqpoint{0.014731in}{-0.014731in}}%
\pgfpathcurveto{\pgfqpoint{0.018638in}{-0.010825in}}{\pgfqpoint{0.020833in}{-0.005525in}}{\pgfqpoint{0.020833in}{0.000000in}}%
\pgfpathcurveto{\pgfqpoint{0.020833in}{0.005525in}}{\pgfqpoint{0.018638in}{0.010825in}}{\pgfqpoint{0.014731in}{0.014731in}}%
\pgfpathcurveto{\pgfqpoint{0.010825in}{0.018638in}}{\pgfqpoint{0.005525in}{0.020833in}}{\pgfqpoint{0.000000in}{0.020833in}}%
\pgfpathcurveto{\pgfqpoint{-0.005525in}{0.020833in}}{\pgfqpoint{-0.010825in}{0.018638in}}{\pgfqpoint{-0.014731in}{0.014731in}}%
\pgfpathcurveto{\pgfqpoint{-0.018638in}{0.010825in}}{\pgfqpoint{-0.020833in}{0.005525in}}{\pgfqpoint{-0.020833in}{0.000000in}}%
\pgfpathcurveto{\pgfqpoint{-0.020833in}{-0.005525in}}{\pgfqpoint{-0.018638in}{-0.010825in}}{\pgfqpoint{-0.014731in}{-0.014731in}}%
\pgfpathcurveto{\pgfqpoint{-0.010825in}{-0.018638in}}{\pgfqpoint{-0.005525in}{-0.020833in}}{\pgfqpoint{0.000000in}{-0.020833in}}%
\pgfpathlineto{\pgfqpoint{0.000000in}{-0.020833in}}%
\pgfpathclose%
\pgfusepath{stroke,fill}%
}%
\begin{pgfscope}%
\pgfsys@transformshift{0.823130in}{2.377649in}%
\pgfsys@useobject{currentmarker}{}%
\end{pgfscope}%
\begin{pgfscope}%
\pgfsys@transformshift{1.177294in}{1.959031in}%
\pgfsys@useobject{currentmarker}{}%
\end{pgfscope}%
\begin{pgfscope}%
\pgfsys@transformshift{1.529826in}{1.287600in}%
\pgfsys@useobject{currentmarker}{}%
\end{pgfscope}%
\begin{pgfscope}%
\pgfsys@transformshift{1.881716in}{0.675678in}%
\pgfsys@useobject{currentmarker}{}%
\end{pgfscope}%
\begin{pgfscope}%
\pgfsys@transformshift{2.232776in}{0.662869in}%
\pgfsys@useobject{currentmarker}{}%
\end{pgfscope}%
\begin{pgfscope}%
\pgfsys@transformshift{2.584493in}{0.662869in}%
\pgfsys@useobject{currentmarker}{}%
\end{pgfscope}%
\end{pgfscope}%
\begin{pgfscope}%
\pgfsetrectcap%
\pgfsetmiterjoin%
\pgfsetlinewidth{0.803000pt}%
\definecolor{currentstroke}{rgb}{0.000000,0.000000,0.000000}%
\pgfsetstrokecolor{currentstroke}%
\pgfsetdash{}{0pt}%
\pgfpathmoveto{\pgfqpoint{0.735061in}{0.575369in}}%
\pgfpathlineto{\pgfqpoint{0.735061in}{2.500369in}}%
\pgfusepath{stroke}%
\end{pgfscope}%
\begin{pgfscope}%
\pgfsetrectcap%
\pgfsetmiterjoin%
\pgfsetlinewidth{0.803000pt}%
\definecolor{currentstroke}{rgb}{0.000000,0.000000,0.000000}%
\pgfsetstrokecolor{currentstroke}%
\pgfsetdash{}{0pt}%
\pgfpathmoveto{\pgfqpoint{2.672561in}{0.575369in}}%
\pgfpathlineto{\pgfqpoint{2.672561in}{2.500369in}}%
\pgfusepath{stroke}%
\end{pgfscope}%
\begin{pgfscope}%
\pgfsetrectcap%
\pgfsetmiterjoin%
\pgfsetlinewidth{0.803000pt}%
\definecolor{currentstroke}{rgb}{0.000000,0.000000,0.000000}%
\pgfsetstrokecolor{currentstroke}%
\pgfsetdash{}{0pt}%
\pgfpathmoveto{\pgfqpoint{0.735061in}{0.575369in}}%
\pgfpathlineto{\pgfqpoint{2.672561in}{0.575369in}}%
\pgfusepath{stroke}%
\end{pgfscope}%
\begin{pgfscope}%
\pgfsetrectcap%
\pgfsetmiterjoin%
\pgfsetlinewidth{0.803000pt}%
\definecolor{currentstroke}{rgb}{0.000000,0.000000,0.000000}%
\pgfsetstrokecolor{currentstroke}%
\pgfsetdash{}{0pt}%
\pgfpathmoveto{\pgfqpoint{0.735061in}{2.500369in}}%
\pgfpathlineto{\pgfqpoint{2.672561in}{2.500369in}}%
\pgfusepath{stroke}%
\end{pgfscope}%
\begin{pgfscope}%
\pgfsetbuttcap%
\pgfsetmiterjoin%
\definecolor{currentfill}{rgb}{1.000000,1.000000,1.000000}%
\pgfsetfillcolor{currentfill}%
\pgfsetfillopacity{0.800000}%
\pgfsetlinewidth{1.003750pt}%
\definecolor{currentstroke}{rgb}{0.800000,0.800000,0.800000}%
\pgfsetstrokecolor{currentstroke}%
\pgfsetstrokeopacity{0.800000}%
\pgfsetdash{}{0pt}%
\pgfpathmoveto{\pgfqpoint{1.502216in}{1.633149in}}%
\pgfpathlineto{\pgfqpoint{2.555895in}{1.633149in}}%
\pgfpathquadraticcurveto{\pgfqpoint{2.589228in}{1.633149in}}{\pgfqpoint{2.589228in}{1.666482in}}%
\pgfpathlineto{\pgfqpoint{2.589228in}{2.383702in}}%
\pgfpathquadraticcurveto{\pgfqpoint{2.589228in}{2.417035in}}{\pgfqpoint{2.555895in}{2.417035in}}%
\pgfpathlineto{\pgfqpoint{1.502216in}{2.417035in}}%
\pgfpathquadraticcurveto{\pgfqpoint{1.468883in}{2.417035in}}{\pgfqpoint{1.468883in}{2.383702in}}%
\pgfpathlineto{\pgfqpoint{1.468883in}{1.666482in}}%
\pgfpathquadraticcurveto{\pgfqpoint{1.468883in}{1.633149in}}{\pgfqpoint{1.502216in}{1.633149in}}%
\pgfpathlineto{\pgfqpoint{1.502216in}{1.633149in}}%
\pgfpathclose%
\pgfusepath{stroke,fill}%
\end{pgfscope}%
\begin{pgfscope}%
\pgfsetrectcap%
\pgfsetroundjoin%
\pgfsetlinewidth{1.003750pt}%
\definecolor{currentstroke}{rgb}{0.001462,0.000466,0.013866}%
\pgfsetstrokecolor{currentstroke}%
\pgfsetdash{}{0pt}%
\pgfpathmoveto{\pgfqpoint{1.535550in}{2.282074in}}%
\pgfpathlineto{\pgfqpoint{1.702216in}{2.282074in}}%
\pgfpathlineto{\pgfqpoint{1.868883in}{2.282074in}}%
\pgfusepath{stroke}%
\end{pgfscope}%
\begin{pgfscope}%
\pgfsetbuttcap%
\pgfsetroundjoin%
\definecolor{currentfill}{rgb}{0.001462,0.000466,0.013866}%
\pgfsetfillcolor{currentfill}%
\pgfsetlinewidth{1.003750pt}%
\definecolor{currentstroke}{rgb}{0.001462,0.000466,0.013866}%
\pgfsetstrokecolor{currentstroke}%
\pgfsetdash{}{0pt}%
\pgfsys@defobject{currentmarker}{\pgfqpoint{-0.020833in}{-0.020833in}}{\pgfqpoint{0.020833in}{0.020833in}}{%
\pgfpathmoveto{\pgfqpoint{0.000000in}{-0.020833in}}%
\pgfpathcurveto{\pgfqpoint{0.005525in}{-0.020833in}}{\pgfqpoint{0.010825in}{-0.018638in}}{\pgfqpoint{0.014731in}{-0.014731in}}%
\pgfpathcurveto{\pgfqpoint{0.018638in}{-0.010825in}}{\pgfqpoint{0.020833in}{-0.005525in}}{\pgfqpoint{0.020833in}{0.000000in}}%
\pgfpathcurveto{\pgfqpoint{0.020833in}{0.005525in}}{\pgfqpoint{0.018638in}{0.010825in}}{\pgfqpoint{0.014731in}{0.014731in}}%
\pgfpathcurveto{\pgfqpoint{0.010825in}{0.018638in}}{\pgfqpoint{0.005525in}{0.020833in}}{\pgfqpoint{0.000000in}{0.020833in}}%
\pgfpathcurveto{\pgfqpoint{-0.005525in}{0.020833in}}{\pgfqpoint{-0.010825in}{0.018638in}}{\pgfqpoint{-0.014731in}{0.014731in}}%
\pgfpathcurveto{\pgfqpoint{-0.018638in}{0.010825in}}{\pgfqpoint{-0.020833in}{0.005525in}}{\pgfqpoint{-0.020833in}{0.000000in}}%
\pgfpathcurveto{\pgfqpoint{-0.020833in}{-0.005525in}}{\pgfqpoint{-0.018638in}{-0.010825in}}{\pgfqpoint{-0.014731in}{-0.014731in}}%
\pgfpathcurveto{\pgfqpoint{-0.010825in}{-0.018638in}}{\pgfqpoint{-0.005525in}{-0.020833in}}{\pgfqpoint{0.000000in}{-0.020833in}}%
\pgfpathlineto{\pgfqpoint{0.000000in}{-0.020833in}}%
\pgfpathclose%
\pgfusepath{stroke,fill}%
}%
\begin{pgfscope}%
\pgfsys@transformshift{1.702216in}{2.282074in}%
\pgfsys@useobject{currentmarker}{}%
\end{pgfscope}%
\end{pgfscope}%
\begin{pgfscope}%
\definecolor{textcolor}{rgb}{0.000000,0.000000,0.000000}%
\pgfsetstrokecolor{textcolor}%
\pgfsetfillcolor{textcolor}%
\pgftext[x=2.002216in,y=2.223741in,left,base]{\color{textcolor}{\rmfamily\fontsize{12.000000}{14.400000}\selectfont\catcode`\^=\active\def^{\ifmmode\sp\else\^{}\fi}\catcode`\%=\active\def%{\%}DGC}}%
\end{pgfscope}%
\begin{pgfscope}%
\pgfsetrectcap%
\pgfsetroundjoin%
\pgfsetlinewidth{1.003750pt}%
\definecolor{currentstroke}{rgb}{0.445163,0.122724,0.506901}%
\pgfsetstrokecolor{currentstroke}%
\pgfsetdash{}{0pt}%
\pgfpathmoveto{\pgfqpoint{1.535550in}{2.037446in}}%
\pgfpathlineto{\pgfqpoint{1.702216in}{2.037446in}}%
\pgfpathlineto{\pgfqpoint{1.868883in}{2.037446in}}%
\pgfusepath{stroke}%
\end{pgfscope}%
\begin{pgfscope}%
\pgfsetbuttcap%
\pgfsetroundjoin%
\definecolor{currentfill}{rgb}{0.445163,0.122724,0.506901}%
\pgfsetfillcolor{currentfill}%
\pgfsetlinewidth{1.003750pt}%
\definecolor{currentstroke}{rgb}{0.445163,0.122724,0.506901}%
\pgfsetstrokecolor{currentstroke}%
\pgfsetdash{}{0pt}%
\pgfsys@defobject{currentmarker}{\pgfqpoint{-0.020833in}{-0.020833in}}{\pgfqpoint{0.020833in}{0.020833in}}{%
\pgfpathmoveto{\pgfqpoint{0.000000in}{-0.020833in}}%
\pgfpathcurveto{\pgfqpoint{0.005525in}{-0.020833in}}{\pgfqpoint{0.010825in}{-0.018638in}}{\pgfqpoint{0.014731in}{-0.014731in}}%
\pgfpathcurveto{\pgfqpoint{0.018638in}{-0.010825in}}{\pgfqpoint{0.020833in}{-0.005525in}}{\pgfqpoint{0.020833in}{0.000000in}}%
\pgfpathcurveto{\pgfqpoint{0.020833in}{0.005525in}}{\pgfqpoint{0.018638in}{0.010825in}}{\pgfqpoint{0.014731in}{0.014731in}}%
\pgfpathcurveto{\pgfqpoint{0.010825in}{0.018638in}}{\pgfqpoint{0.005525in}{0.020833in}}{\pgfqpoint{0.000000in}{0.020833in}}%
\pgfpathcurveto{\pgfqpoint{-0.005525in}{0.020833in}}{\pgfqpoint{-0.010825in}{0.018638in}}{\pgfqpoint{-0.014731in}{0.014731in}}%
\pgfpathcurveto{\pgfqpoint{-0.018638in}{0.010825in}}{\pgfqpoint{-0.020833in}{0.005525in}}{\pgfqpoint{-0.020833in}{0.000000in}}%
\pgfpathcurveto{\pgfqpoint{-0.020833in}{-0.005525in}}{\pgfqpoint{-0.018638in}{-0.010825in}}{\pgfqpoint{-0.014731in}{-0.014731in}}%
\pgfpathcurveto{\pgfqpoint{-0.010825in}{-0.018638in}}{\pgfqpoint{-0.005525in}{-0.020833in}}{\pgfqpoint{0.000000in}{-0.020833in}}%
\pgfpathlineto{\pgfqpoint{0.000000in}{-0.020833in}}%
\pgfpathclose%
\pgfusepath{stroke,fill}%
}%
\begin{pgfscope}%
\pgfsys@transformshift{1.702216in}{2.037446in}%
\pgfsys@useobject{currentmarker}{}%
\end{pgfscope}%
\end{pgfscope}%
\begin{pgfscope}%
\definecolor{textcolor}{rgb}{0.000000,0.000000,0.000000}%
\pgfsetstrokecolor{textcolor}%
\pgfsetfillcolor{textcolor}%
\pgftext[x=2.002216in,y=1.979112in,left,base]{\color{textcolor}{\rmfamily\fontsize{12.000000}{14.400000}\selectfont\catcode`\^=\active\def^{\ifmmode\sp\else\^{}\fi}\catcode`\%=\active\def%{\%}NC}}%
\end{pgfscope}%
\begin{pgfscope}%
\pgfsetrectcap%
\pgfsetroundjoin%
\pgfsetlinewidth{1.003750pt}%
\definecolor{currentstroke}{rgb}{0.944006,0.377643,0.365136}%
\pgfsetstrokecolor{currentstroke}%
\pgfsetdash{}{0pt}%
\pgfpathmoveto{\pgfqpoint{1.535550in}{1.792817in}}%
\pgfpathlineto{\pgfqpoint{1.702216in}{1.792817in}}%
\pgfpathlineto{\pgfqpoint{1.868883in}{1.792817in}}%
\pgfusepath{stroke}%
\end{pgfscope}%
\begin{pgfscope}%
\pgfsetbuttcap%
\pgfsetroundjoin%
\definecolor{currentfill}{rgb}{0.944006,0.377643,0.365136}%
\pgfsetfillcolor{currentfill}%
\pgfsetlinewidth{1.003750pt}%
\definecolor{currentstroke}{rgb}{0.944006,0.377643,0.365136}%
\pgfsetstrokecolor{currentstroke}%
\pgfsetdash{}{0pt}%
\pgfsys@defobject{currentmarker}{\pgfqpoint{-0.020833in}{-0.020833in}}{\pgfqpoint{0.020833in}{0.020833in}}{%
\pgfpathmoveto{\pgfqpoint{0.000000in}{-0.020833in}}%
\pgfpathcurveto{\pgfqpoint{0.005525in}{-0.020833in}}{\pgfqpoint{0.010825in}{-0.018638in}}{\pgfqpoint{0.014731in}{-0.014731in}}%
\pgfpathcurveto{\pgfqpoint{0.018638in}{-0.010825in}}{\pgfqpoint{0.020833in}{-0.005525in}}{\pgfqpoint{0.020833in}{0.000000in}}%
\pgfpathcurveto{\pgfqpoint{0.020833in}{0.005525in}}{\pgfqpoint{0.018638in}{0.010825in}}{\pgfqpoint{0.014731in}{0.014731in}}%
\pgfpathcurveto{\pgfqpoint{0.010825in}{0.018638in}}{\pgfqpoint{0.005525in}{0.020833in}}{\pgfqpoint{0.000000in}{0.020833in}}%
\pgfpathcurveto{\pgfqpoint{-0.005525in}{0.020833in}}{\pgfqpoint{-0.010825in}{0.018638in}}{\pgfqpoint{-0.014731in}{0.014731in}}%
\pgfpathcurveto{\pgfqpoint{-0.018638in}{0.010825in}}{\pgfqpoint{-0.020833in}{0.005525in}}{\pgfqpoint{-0.020833in}{0.000000in}}%
\pgfpathcurveto{\pgfqpoint{-0.020833in}{-0.005525in}}{\pgfqpoint{-0.018638in}{-0.010825in}}{\pgfqpoint{-0.014731in}{-0.014731in}}%
\pgfpathcurveto{\pgfqpoint{-0.010825in}{-0.018638in}}{\pgfqpoint{-0.005525in}{-0.020833in}}{\pgfqpoint{0.000000in}{-0.020833in}}%
\pgfpathlineto{\pgfqpoint{0.000000in}{-0.020833in}}%
\pgfpathclose%
\pgfusepath{stroke,fill}%
}%
\begin{pgfscope}%
\pgfsys@transformshift{1.702216in}{1.792817in}%
\pgfsys@useobject{currentmarker}{}%
\end{pgfscope}%
\end{pgfscope}%
\begin{pgfscope}%
\definecolor{textcolor}{rgb}{0.000000,0.000000,0.000000}%
\pgfsetstrokecolor{textcolor}%
\pgfsetfillcolor{textcolor}%
\pgftext[x=2.002216in,y=1.734483in,left,base]{\color{textcolor}{\rmfamily\fontsize{12.000000}{14.400000}\selectfont\catcode`\^=\active\def^{\ifmmode\sp\else\^{}\fi}\catcode`\%=\active\def%{\%}NC++}}%
\end{pgfscope}%
\end{pgfpicture}%
\makeatother%
\endgroup%

        \caption{\gls{sketch-size} $+$ \gls{num-hutchinson-queries} $=40$}
        \label{fig:5-experiments-electronic-structure-convergence-m-nv40}
    \end{subfigure}
    \begin{subfigure}[b]{0.49\columnwidth}
        %% Creator: Matplotlib, PGF backend
%%
%% To include the figure in your LaTeX document, write
%%   \input{<filename>.pgf}
%%
%% Make sure the required packages are loaded in your preamble
%%   \usepackage{pgf}
%%
%% Also ensure that all the required font packages are loaded; for instance,
%% the lmodern package is sometimes necessary when using math font.
%%   \usepackage{lmodern}
%%
%% Figures using additional raster images can only be included by \input if
%% they are in the same directory as the main LaTeX file. For loading figures
%% from other directories you can use the `import` package
%%   \usepackage{import}
%%
%% and then include the figures with
%%   \import{<path to file>}{<filename>.pgf}
%%
%% Matplotlib used the following preamble
%%   \def\mathdefault#1{#1}
%%   \everymath=\expandafter{\the\everymath\displaystyle}
%%   
%%   \makeatletter\@ifpackageloaded{underscore}{}{\usepackage[strings]{underscore}}\makeatother
%%
\begingroup%
\makeatletter%
\begin{pgfpicture}%
\pgfpathrectangle{\pgfpointorigin}{\pgfqpoint{2.759413in}{2.574073in}}%
\pgfusepath{use as bounding box, clip}%
\begin{pgfscope}%
\pgfsetbuttcap%
\pgfsetmiterjoin%
\definecolor{currentfill}{rgb}{1.000000,1.000000,1.000000}%
\pgfsetfillcolor{currentfill}%
\pgfsetlinewidth{0.000000pt}%
\definecolor{currentstroke}{rgb}{1.000000,1.000000,1.000000}%
\pgfsetstrokecolor{currentstroke}%
\pgfsetdash{}{0pt}%
\pgfpathmoveto{\pgfqpoint{0.000000in}{0.000000in}}%
\pgfpathlineto{\pgfqpoint{2.759413in}{0.000000in}}%
\pgfpathlineto{\pgfqpoint{2.759413in}{2.574073in}}%
\pgfpathlineto{\pgfqpoint{0.000000in}{2.574073in}}%
\pgfpathlineto{\pgfqpoint{0.000000in}{0.000000in}}%
\pgfpathclose%
\pgfusepath{fill}%
\end{pgfscope}%
\begin{pgfscope}%
\pgfsetbuttcap%
\pgfsetmiterjoin%
\definecolor{currentfill}{rgb}{1.000000,1.000000,1.000000}%
\pgfsetfillcolor{currentfill}%
\pgfsetlinewidth{0.000000pt}%
\definecolor{currentstroke}{rgb}{0.000000,0.000000,0.000000}%
\pgfsetstrokecolor{currentstroke}%
\pgfsetstrokeopacity{0.000000}%
\pgfsetdash{}{0pt}%
\pgfpathmoveto{\pgfqpoint{0.721913in}{0.549073in}}%
\pgfpathlineto{\pgfqpoint{2.659413in}{0.549073in}}%
\pgfpathlineto{\pgfqpoint{2.659413in}{2.474073in}}%
\pgfpathlineto{\pgfqpoint{0.721913in}{2.474073in}}%
\pgfpathlineto{\pgfqpoint{0.721913in}{0.549073in}}%
\pgfpathclose%
\pgfusepath{fill}%
\end{pgfscope}%
\begin{pgfscope}%
\pgfsetbuttcap%
\pgfsetroundjoin%
\definecolor{currentfill}{rgb}{0.000000,0.000000,0.000000}%
\pgfsetfillcolor{currentfill}%
\pgfsetlinewidth{0.803000pt}%
\definecolor{currentstroke}{rgb}{0.000000,0.000000,0.000000}%
\pgfsetstrokecolor{currentstroke}%
\pgfsetdash{}{0pt}%
\pgfsys@defobject{currentmarker}{\pgfqpoint{0.000000in}{-0.048611in}}{\pgfqpoint{0.000000in}{0.000000in}}{%
\pgfpathmoveto{\pgfqpoint{0.000000in}{0.000000in}}%
\pgfpathlineto{\pgfqpoint{0.000000in}{-0.048611in}}%
\pgfusepath{stroke,fill}%
}%
\begin{pgfscope}%
\pgfsys@transformshift{1.760574in}{0.549073in}%
\pgfsys@useobject{currentmarker}{}%
\end{pgfscope}%
\end{pgfscope}%
\begin{pgfscope}%
\definecolor{textcolor}{rgb}{0.000000,0.000000,0.000000}%
\pgfsetstrokecolor{textcolor}%
\pgfsetfillcolor{textcolor}%
\pgftext[x=1.760574in,y=0.451851in,,top]{\color{textcolor}{\rmfamily\fontsize{12.000000}{14.400000}\selectfont\catcode`\^=\active\def^{\ifmmode\sp\else\^{}\fi}\catcode`\%=\active\def%{\%}$\mathdefault{10^{3}}$}}%
\end{pgfscope}%
\begin{pgfscope}%
\pgfsetbuttcap%
\pgfsetroundjoin%
\definecolor{currentfill}{rgb}{0.000000,0.000000,0.000000}%
\pgfsetfillcolor{currentfill}%
\pgfsetlinewidth{0.602250pt}%
\definecolor{currentstroke}{rgb}{0.000000,0.000000,0.000000}%
\pgfsetstrokecolor{currentstroke}%
\pgfsetdash{}{0pt}%
\pgfsys@defobject{currentmarker}{\pgfqpoint{0.000000in}{-0.027778in}}{\pgfqpoint{0.000000in}{0.000000in}}{%
\pgfpathmoveto{\pgfqpoint{0.000000in}{0.000000in}}%
\pgfpathlineto{\pgfqpoint{0.000000in}{-0.027778in}}%
\pgfusepath{stroke,fill}%
}%
\begin{pgfscope}%
\pgfsys@transformshift{0.815881in}{0.549073in}%
\pgfsys@useobject{currentmarker}{}%
\end{pgfscope}%
\end{pgfscope}%
\begin{pgfscope}%
\pgfsetbuttcap%
\pgfsetroundjoin%
\definecolor{currentfill}{rgb}{0.000000,0.000000,0.000000}%
\pgfsetfillcolor{currentfill}%
\pgfsetlinewidth{0.602250pt}%
\definecolor{currentstroke}{rgb}{0.000000,0.000000,0.000000}%
\pgfsetstrokecolor{currentstroke}%
\pgfsetdash{}{0pt}%
\pgfsys@defobject{currentmarker}{\pgfqpoint{0.000000in}{-0.027778in}}{\pgfqpoint{0.000000in}{0.000000in}}{%
\pgfpathmoveto{\pgfqpoint{0.000000in}{0.000000in}}%
\pgfpathlineto{\pgfqpoint{0.000000in}{-0.027778in}}%
\pgfusepath{stroke,fill}%
}%
\begin{pgfscope}%
\pgfsys@transformshift{1.053877in}{0.549073in}%
\pgfsys@useobject{currentmarker}{}%
\end{pgfscope}%
\end{pgfscope}%
\begin{pgfscope}%
\pgfsetbuttcap%
\pgfsetroundjoin%
\definecolor{currentfill}{rgb}{0.000000,0.000000,0.000000}%
\pgfsetfillcolor{currentfill}%
\pgfsetlinewidth{0.602250pt}%
\definecolor{currentstroke}{rgb}{0.000000,0.000000,0.000000}%
\pgfsetstrokecolor{currentstroke}%
\pgfsetdash{}{0pt}%
\pgfsys@defobject{currentmarker}{\pgfqpoint{0.000000in}{-0.027778in}}{\pgfqpoint{0.000000in}{0.000000in}}{%
\pgfpathmoveto{\pgfqpoint{0.000000in}{0.000000in}}%
\pgfpathlineto{\pgfqpoint{0.000000in}{-0.027778in}}%
\pgfusepath{stroke,fill}%
}%
\begin{pgfscope}%
\pgfsys@transformshift{1.222738in}{0.549073in}%
\pgfsys@useobject{currentmarker}{}%
\end{pgfscope}%
\end{pgfscope}%
\begin{pgfscope}%
\pgfsetbuttcap%
\pgfsetroundjoin%
\definecolor{currentfill}{rgb}{0.000000,0.000000,0.000000}%
\pgfsetfillcolor{currentfill}%
\pgfsetlinewidth{0.602250pt}%
\definecolor{currentstroke}{rgb}{0.000000,0.000000,0.000000}%
\pgfsetstrokecolor{currentstroke}%
\pgfsetdash{}{0pt}%
\pgfsys@defobject{currentmarker}{\pgfqpoint{0.000000in}{-0.027778in}}{\pgfqpoint{0.000000in}{0.000000in}}{%
\pgfpathmoveto{\pgfqpoint{0.000000in}{0.000000in}}%
\pgfpathlineto{\pgfqpoint{0.000000in}{-0.027778in}}%
\pgfusepath{stroke,fill}%
}%
\begin{pgfscope}%
\pgfsys@transformshift{1.353717in}{0.549073in}%
\pgfsys@useobject{currentmarker}{}%
\end{pgfscope}%
\end{pgfscope}%
\begin{pgfscope}%
\pgfsetbuttcap%
\pgfsetroundjoin%
\definecolor{currentfill}{rgb}{0.000000,0.000000,0.000000}%
\pgfsetfillcolor{currentfill}%
\pgfsetlinewidth{0.602250pt}%
\definecolor{currentstroke}{rgb}{0.000000,0.000000,0.000000}%
\pgfsetstrokecolor{currentstroke}%
\pgfsetdash{}{0pt}%
\pgfsys@defobject{currentmarker}{\pgfqpoint{0.000000in}{-0.027778in}}{\pgfqpoint{0.000000in}{0.000000in}}{%
\pgfpathmoveto{\pgfqpoint{0.000000in}{0.000000in}}%
\pgfpathlineto{\pgfqpoint{0.000000in}{-0.027778in}}%
\pgfusepath{stroke,fill}%
}%
\begin{pgfscope}%
\pgfsys@transformshift{1.460734in}{0.549073in}%
\pgfsys@useobject{currentmarker}{}%
\end{pgfscope}%
\end{pgfscope}%
\begin{pgfscope}%
\pgfsetbuttcap%
\pgfsetroundjoin%
\definecolor{currentfill}{rgb}{0.000000,0.000000,0.000000}%
\pgfsetfillcolor{currentfill}%
\pgfsetlinewidth{0.602250pt}%
\definecolor{currentstroke}{rgb}{0.000000,0.000000,0.000000}%
\pgfsetstrokecolor{currentstroke}%
\pgfsetdash{}{0pt}%
\pgfsys@defobject{currentmarker}{\pgfqpoint{0.000000in}{-0.027778in}}{\pgfqpoint{0.000000in}{0.000000in}}{%
\pgfpathmoveto{\pgfqpoint{0.000000in}{0.000000in}}%
\pgfpathlineto{\pgfqpoint{0.000000in}{-0.027778in}}%
\pgfusepath{stroke,fill}%
}%
\begin{pgfscope}%
\pgfsys@transformshift{1.551216in}{0.549073in}%
\pgfsys@useobject{currentmarker}{}%
\end{pgfscope}%
\end{pgfscope}%
\begin{pgfscope}%
\pgfsetbuttcap%
\pgfsetroundjoin%
\definecolor{currentfill}{rgb}{0.000000,0.000000,0.000000}%
\pgfsetfillcolor{currentfill}%
\pgfsetlinewidth{0.602250pt}%
\definecolor{currentstroke}{rgb}{0.000000,0.000000,0.000000}%
\pgfsetstrokecolor{currentstroke}%
\pgfsetdash{}{0pt}%
\pgfsys@defobject{currentmarker}{\pgfqpoint{0.000000in}{-0.027778in}}{\pgfqpoint{0.000000in}{0.000000in}}{%
\pgfpathmoveto{\pgfqpoint{0.000000in}{0.000000in}}%
\pgfpathlineto{\pgfqpoint{0.000000in}{-0.027778in}}%
\pgfusepath{stroke,fill}%
}%
\begin{pgfscope}%
\pgfsys@transformshift{1.629595in}{0.549073in}%
\pgfsys@useobject{currentmarker}{}%
\end{pgfscope}%
\end{pgfscope}%
\begin{pgfscope}%
\pgfsetbuttcap%
\pgfsetroundjoin%
\definecolor{currentfill}{rgb}{0.000000,0.000000,0.000000}%
\pgfsetfillcolor{currentfill}%
\pgfsetlinewidth{0.602250pt}%
\definecolor{currentstroke}{rgb}{0.000000,0.000000,0.000000}%
\pgfsetstrokecolor{currentstroke}%
\pgfsetdash{}{0pt}%
\pgfsys@defobject{currentmarker}{\pgfqpoint{0.000000in}{-0.027778in}}{\pgfqpoint{0.000000in}{0.000000in}}{%
\pgfpathmoveto{\pgfqpoint{0.000000in}{0.000000in}}%
\pgfpathlineto{\pgfqpoint{0.000000in}{-0.027778in}}%
\pgfusepath{stroke,fill}%
}%
\begin{pgfscope}%
\pgfsys@transformshift{1.698730in}{0.549073in}%
\pgfsys@useobject{currentmarker}{}%
\end{pgfscope}%
\end{pgfscope}%
\begin{pgfscope}%
\pgfsetbuttcap%
\pgfsetroundjoin%
\definecolor{currentfill}{rgb}{0.000000,0.000000,0.000000}%
\pgfsetfillcolor{currentfill}%
\pgfsetlinewidth{0.602250pt}%
\definecolor{currentstroke}{rgb}{0.000000,0.000000,0.000000}%
\pgfsetstrokecolor{currentstroke}%
\pgfsetdash{}{0pt}%
\pgfsys@defobject{currentmarker}{\pgfqpoint{0.000000in}{-0.027778in}}{\pgfqpoint{0.000000in}{0.000000in}}{%
\pgfpathmoveto{\pgfqpoint{0.000000in}{0.000000in}}%
\pgfpathlineto{\pgfqpoint{0.000000in}{-0.027778in}}%
\pgfusepath{stroke,fill}%
}%
\begin{pgfscope}%
\pgfsys@transformshift{2.167431in}{0.549073in}%
\pgfsys@useobject{currentmarker}{}%
\end{pgfscope}%
\end{pgfscope}%
\begin{pgfscope}%
\pgfsetbuttcap%
\pgfsetroundjoin%
\definecolor{currentfill}{rgb}{0.000000,0.000000,0.000000}%
\pgfsetfillcolor{currentfill}%
\pgfsetlinewidth{0.602250pt}%
\definecolor{currentstroke}{rgb}{0.000000,0.000000,0.000000}%
\pgfsetstrokecolor{currentstroke}%
\pgfsetdash{}{0pt}%
\pgfsys@defobject{currentmarker}{\pgfqpoint{0.000000in}{-0.027778in}}{\pgfqpoint{0.000000in}{0.000000in}}{%
\pgfpathmoveto{\pgfqpoint{0.000000in}{0.000000in}}%
\pgfpathlineto{\pgfqpoint{0.000000in}{-0.027778in}}%
\pgfusepath{stroke,fill}%
}%
\begin{pgfscope}%
\pgfsys@transformshift{2.405427in}{0.549073in}%
\pgfsys@useobject{currentmarker}{}%
\end{pgfscope}%
\end{pgfscope}%
\begin{pgfscope}%
\pgfsetbuttcap%
\pgfsetroundjoin%
\definecolor{currentfill}{rgb}{0.000000,0.000000,0.000000}%
\pgfsetfillcolor{currentfill}%
\pgfsetlinewidth{0.602250pt}%
\definecolor{currentstroke}{rgb}{0.000000,0.000000,0.000000}%
\pgfsetstrokecolor{currentstroke}%
\pgfsetdash{}{0pt}%
\pgfsys@defobject{currentmarker}{\pgfqpoint{0.000000in}{-0.027778in}}{\pgfqpoint{0.000000in}{0.000000in}}{%
\pgfpathmoveto{\pgfqpoint{0.000000in}{0.000000in}}%
\pgfpathlineto{\pgfqpoint{0.000000in}{-0.027778in}}%
\pgfusepath{stroke,fill}%
}%
\begin{pgfscope}%
\pgfsys@transformshift{2.574288in}{0.549073in}%
\pgfsys@useobject{currentmarker}{}%
\end{pgfscope}%
\end{pgfscope}%
\begin{pgfscope}%
\definecolor{textcolor}{rgb}{0.000000,0.000000,0.000000}%
\pgfsetstrokecolor{textcolor}%
\pgfsetfillcolor{textcolor}%
\pgftext[x=1.690663in,y=0.248148in,,top]{\color{textcolor}{\rmfamily\fontsize{12.000000}{14.400000}\selectfont\catcode`\^=\active\def^{\ifmmode\sp\else\^{}\fi}\catcode`\%=\active\def%{\%}$m$}}%
\end{pgfscope}%
\begin{pgfscope}%
\pgfsetbuttcap%
\pgfsetroundjoin%
\definecolor{currentfill}{rgb}{0.000000,0.000000,0.000000}%
\pgfsetfillcolor{currentfill}%
\pgfsetlinewidth{0.803000pt}%
\definecolor{currentstroke}{rgb}{0.000000,0.000000,0.000000}%
\pgfsetstrokecolor{currentstroke}%
\pgfsetdash{}{0pt}%
\pgfsys@defobject{currentmarker}{\pgfqpoint{-0.048611in}{0.000000in}}{\pgfqpoint{-0.000000in}{0.000000in}}{%
\pgfpathmoveto{\pgfqpoint{-0.000000in}{0.000000in}}%
\pgfpathlineto{\pgfqpoint{-0.048611in}{0.000000in}}%
\pgfusepath{stroke,fill}%
}%
\begin{pgfscope}%
\pgfsys@transformshift{0.721913in}{0.891761in}%
\pgfsys@useobject{currentmarker}{}%
\end{pgfscope}%
\end{pgfscope}%
\begin{pgfscope}%
\definecolor{textcolor}{rgb}{0.000000,0.000000,0.000000}%
\pgfsetstrokecolor{textcolor}%
\pgfsetfillcolor{textcolor}%
\pgftext[x=0.303703in, y=0.833891in, left, base]{\color{textcolor}{\rmfamily\fontsize{12.000000}{14.400000}\selectfont\catcode`\^=\active\def^{\ifmmode\sp\else\^{}\fi}\catcode`\%=\active\def%{\%}$\mathdefault{10^{-6}}$}}%
\end{pgfscope}%
\begin{pgfscope}%
\pgfsetbuttcap%
\pgfsetroundjoin%
\definecolor{currentfill}{rgb}{0.000000,0.000000,0.000000}%
\pgfsetfillcolor{currentfill}%
\pgfsetlinewidth{0.803000pt}%
\definecolor{currentstroke}{rgb}{0.000000,0.000000,0.000000}%
\pgfsetstrokecolor{currentstroke}%
\pgfsetdash{}{0pt}%
\pgfsys@defobject{currentmarker}{\pgfqpoint{-0.048611in}{0.000000in}}{\pgfqpoint{-0.000000in}{0.000000in}}{%
\pgfpathmoveto{\pgfqpoint{-0.000000in}{0.000000in}}%
\pgfpathlineto{\pgfqpoint{-0.048611in}{0.000000in}}%
\pgfusepath{stroke,fill}%
}%
\begin{pgfscope}%
\pgfsys@transformshift{0.721913in}{1.380149in}%
\pgfsys@useobject{currentmarker}{}%
\end{pgfscope}%
\end{pgfscope}%
\begin{pgfscope}%
\definecolor{textcolor}{rgb}{0.000000,0.000000,0.000000}%
\pgfsetstrokecolor{textcolor}%
\pgfsetfillcolor{textcolor}%
\pgftext[x=0.303703in, y=1.322279in, left, base]{\color{textcolor}{\rmfamily\fontsize{12.000000}{14.400000}\selectfont\catcode`\^=\active\def^{\ifmmode\sp\else\^{}\fi}\catcode`\%=\active\def%{\%}$\mathdefault{10^{-4}}$}}%
\end{pgfscope}%
\begin{pgfscope}%
\pgfsetbuttcap%
\pgfsetroundjoin%
\definecolor{currentfill}{rgb}{0.000000,0.000000,0.000000}%
\pgfsetfillcolor{currentfill}%
\pgfsetlinewidth{0.803000pt}%
\definecolor{currentstroke}{rgb}{0.000000,0.000000,0.000000}%
\pgfsetstrokecolor{currentstroke}%
\pgfsetdash{}{0pt}%
\pgfsys@defobject{currentmarker}{\pgfqpoint{-0.048611in}{0.000000in}}{\pgfqpoint{-0.000000in}{0.000000in}}{%
\pgfpathmoveto{\pgfqpoint{-0.000000in}{0.000000in}}%
\pgfpathlineto{\pgfqpoint{-0.048611in}{0.000000in}}%
\pgfusepath{stroke,fill}%
}%
\begin{pgfscope}%
\pgfsys@transformshift{0.721913in}{1.868538in}%
\pgfsys@useobject{currentmarker}{}%
\end{pgfscope}%
\end{pgfscope}%
\begin{pgfscope}%
\definecolor{textcolor}{rgb}{0.000000,0.000000,0.000000}%
\pgfsetstrokecolor{textcolor}%
\pgfsetfillcolor{textcolor}%
\pgftext[x=0.303703in, y=1.810668in, left, base]{\color{textcolor}{\rmfamily\fontsize{12.000000}{14.400000}\selectfont\catcode`\^=\active\def^{\ifmmode\sp\else\^{}\fi}\catcode`\%=\active\def%{\%}$\mathdefault{10^{-2}}$}}%
\end{pgfscope}%
\begin{pgfscope}%
\pgfsetbuttcap%
\pgfsetroundjoin%
\definecolor{currentfill}{rgb}{0.000000,0.000000,0.000000}%
\pgfsetfillcolor{currentfill}%
\pgfsetlinewidth{0.803000pt}%
\definecolor{currentstroke}{rgb}{0.000000,0.000000,0.000000}%
\pgfsetstrokecolor{currentstroke}%
\pgfsetdash{}{0pt}%
\pgfsys@defobject{currentmarker}{\pgfqpoint{-0.048611in}{0.000000in}}{\pgfqpoint{-0.000000in}{0.000000in}}{%
\pgfpathmoveto{\pgfqpoint{-0.000000in}{0.000000in}}%
\pgfpathlineto{\pgfqpoint{-0.048611in}{0.000000in}}%
\pgfusepath{stroke,fill}%
}%
\begin{pgfscope}%
\pgfsys@transformshift{0.721913in}{2.356926in}%
\pgfsys@useobject{currentmarker}{}%
\end{pgfscope}%
\end{pgfscope}%
\begin{pgfscope}%
\definecolor{textcolor}{rgb}{0.000000,0.000000,0.000000}%
\pgfsetstrokecolor{textcolor}%
\pgfsetfillcolor{textcolor}%
\pgftext[x=0.395525in, y=2.299056in, left, base]{\color{textcolor}{\rmfamily\fontsize{12.000000}{14.400000}\selectfont\catcode`\^=\active\def^{\ifmmode\sp\else\^{}\fi}\catcode`\%=\active\def%{\%}$\mathdefault{10^{0}}$}}%
\end{pgfscope}%
\begin{pgfscope}%
\definecolor{textcolor}{rgb}{0.000000,0.000000,0.000000}%
\pgfsetstrokecolor{textcolor}%
\pgfsetfillcolor{textcolor}%
\pgftext[x=0.248148in,y=1.511573in,,bottom,rotate=90.000000]{\color{textcolor}{\rmfamily\fontsize{12.000000}{14.400000}\selectfont\catcode`\^=\active\def^{\ifmmode\sp\else\^{}\fi}\catcode`\%=\active\def%{\%}$L^1$ relative error}}%
\end{pgfscope}%
\begin{pgfscope}%
\pgfpathrectangle{\pgfqpoint{0.721913in}{0.549073in}}{\pgfqpoint{1.937500in}{1.925000in}}%
\pgfusepath{clip}%
\pgfsetrectcap%
\pgfsetroundjoin%
\pgfsetlinewidth{1.003750pt}%
\definecolor{currentstroke}{rgb}{0.537255,0.647059,0.760784}%
\pgfsetstrokecolor{currentstroke}%
\pgfsetdash{}{0pt}%
\pgfpathmoveto{\pgfqpoint{0.809982in}{2.334824in}}%
\pgfpathlineto{\pgfqpoint{1.164146in}{2.222692in}}%
\pgfpathlineto{\pgfqpoint{1.516678in}{2.055636in}}%
\pgfpathlineto{\pgfqpoint{1.868568in}{1.948700in}}%
\pgfpathlineto{\pgfqpoint{2.219628in}{1.948372in}}%
\pgfpathlineto{\pgfqpoint{2.571345in}{1.948372in}}%
\pgfusepath{stroke}%
\end{pgfscope}%
\begin{pgfscope}%
\pgfpathrectangle{\pgfqpoint{0.721913in}{0.549073in}}{\pgfqpoint{1.937500in}{1.925000in}}%
\pgfusepath{clip}%
\pgfsetbuttcap%
\pgfsetroundjoin%
\definecolor{currentfill}{rgb}{0.537255,0.647059,0.760784}%
\pgfsetfillcolor{currentfill}%
\pgfsetlinewidth{1.003750pt}%
\definecolor{currentstroke}{rgb}{0.537255,0.647059,0.760784}%
\pgfsetstrokecolor{currentstroke}%
\pgfsetdash{}{0pt}%
\pgfsys@defobject{currentmarker}{\pgfqpoint{-0.020833in}{-0.020833in}}{\pgfqpoint{0.020833in}{0.020833in}}{%
\pgfpathmoveto{\pgfqpoint{0.000000in}{-0.020833in}}%
\pgfpathcurveto{\pgfqpoint{0.005525in}{-0.020833in}}{\pgfqpoint{0.010825in}{-0.018638in}}{\pgfqpoint{0.014731in}{-0.014731in}}%
\pgfpathcurveto{\pgfqpoint{0.018638in}{-0.010825in}}{\pgfqpoint{0.020833in}{-0.005525in}}{\pgfqpoint{0.020833in}{0.000000in}}%
\pgfpathcurveto{\pgfqpoint{0.020833in}{0.005525in}}{\pgfqpoint{0.018638in}{0.010825in}}{\pgfqpoint{0.014731in}{0.014731in}}%
\pgfpathcurveto{\pgfqpoint{0.010825in}{0.018638in}}{\pgfqpoint{0.005525in}{0.020833in}}{\pgfqpoint{0.000000in}{0.020833in}}%
\pgfpathcurveto{\pgfqpoint{-0.005525in}{0.020833in}}{\pgfqpoint{-0.010825in}{0.018638in}}{\pgfqpoint{-0.014731in}{0.014731in}}%
\pgfpathcurveto{\pgfqpoint{-0.018638in}{0.010825in}}{\pgfqpoint{-0.020833in}{0.005525in}}{\pgfqpoint{-0.020833in}{0.000000in}}%
\pgfpathcurveto{\pgfqpoint{-0.020833in}{-0.005525in}}{\pgfqpoint{-0.018638in}{-0.010825in}}{\pgfqpoint{-0.014731in}{-0.014731in}}%
\pgfpathcurveto{\pgfqpoint{-0.010825in}{-0.018638in}}{\pgfqpoint{-0.005525in}{-0.020833in}}{\pgfqpoint{0.000000in}{-0.020833in}}%
\pgfpathlineto{\pgfqpoint{0.000000in}{-0.020833in}}%
\pgfpathclose%
\pgfusepath{stroke,fill}%
}%
\begin{pgfscope}%
\pgfsys@transformshift{0.809982in}{2.334824in}%
\pgfsys@useobject{currentmarker}{}%
\end{pgfscope}%
\begin{pgfscope}%
\pgfsys@transformshift{1.164146in}{2.222692in}%
\pgfsys@useobject{currentmarker}{}%
\end{pgfscope}%
\begin{pgfscope}%
\pgfsys@transformshift{1.516678in}{2.055636in}%
\pgfsys@useobject{currentmarker}{}%
\end{pgfscope}%
\begin{pgfscope}%
\pgfsys@transformshift{1.868568in}{1.948700in}%
\pgfsys@useobject{currentmarker}{}%
\end{pgfscope}%
\begin{pgfscope}%
\pgfsys@transformshift{2.219628in}{1.948372in}%
\pgfsys@useobject{currentmarker}{}%
\end{pgfscope}%
\begin{pgfscope}%
\pgfsys@transformshift{2.571345in}{1.948372in}%
\pgfsys@useobject{currentmarker}{}%
\end{pgfscope}%
\end{pgfscope}%
\begin{pgfscope}%
\pgfpathrectangle{\pgfqpoint{0.721913in}{0.549073in}}{\pgfqpoint{1.937500in}{1.925000in}}%
\pgfusepath{clip}%
\pgfsetrectcap%
\pgfsetroundjoin%
\pgfsetlinewidth{1.003750pt}%
\definecolor{currentstroke}{rgb}{0.184314,0.270588,0.360784}%
\pgfsetstrokecolor{currentstroke}%
\pgfsetdash{}{0pt}%
\pgfpathmoveto{\pgfqpoint{0.809982in}{2.386573in}}%
\pgfpathlineto{\pgfqpoint{1.164146in}{2.316626in}}%
\pgfpathlineto{\pgfqpoint{1.516678in}{2.183606in}}%
\pgfpathlineto{\pgfqpoint{1.868568in}{1.687104in}}%
\pgfpathlineto{\pgfqpoint{2.219628in}{0.777079in}}%
\pgfpathlineto{\pgfqpoint{2.571345in}{0.636573in}}%
\pgfusepath{stroke}%
\end{pgfscope}%
\begin{pgfscope}%
\pgfpathrectangle{\pgfqpoint{0.721913in}{0.549073in}}{\pgfqpoint{1.937500in}{1.925000in}}%
\pgfusepath{clip}%
\pgfsetbuttcap%
\pgfsetroundjoin%
\definecolor{currentfill}{rgb}{0.184314,0.270588,0.360784}%
\pgfsetfillcolor{currentfill}%
\pgfsetlinewidth{1.003750pt}%
\definecolor{currentstroke}{rgb}{0.184314,0.270588,0.360784}%
\pgfsetstrokecolor{currentstroke}%
\pgfsetdash{}{0pt}%
\pgfsys@defobject{currentmarker}{\pgfqpoint{-0.020833in}{-0.020833in}}{\pgfqpoint{0.020833in}{0.020833in}}{%
\pgfpathmoveto{\pgfqpoint{0.000000in}{-0.020833in}}%
\pgfpathcurveto{\pgfqpoint{0.005525in}{-0.020833in}}{\pgfqpoint{0.010825in}{-0.018638in}}{\pgfqpoint{0.014731in}{-0.014731in}}%
\pgfpathcurveto{\pgfqpoint{0.018638in}{-0.010825in}}{\pgfqpoint{0.020833in}{-0.005525in}}{\pgfqpoint{0.020833in}{0.000000in}}%
\pgfpathcurveto{\pgfqpoint{0.020833in}{0.005525in}}{\pgfqpoint{0.018638in}{0.010825in}}{\pgfqpoint{0.014731in}{0.014731in}}%
\pgfpathcurveto{\pgfqpoint{0.010825in}{0.018638in}}{\pgfqpoint{0.005525in}{0.020833in}}{\pgfqpoint{0.000000in}{0.020833in}}%
\pgfpathcurveto{\pgfqpoint{-0.005525in}{0.020833in}}{\pgfqpoint{-0.010825in}{0.018638in}}{\pgfqpoint{-0.014731in}{0.014731in}}%
\pgfpathcurveto{\pgfqpoint{-0.018638in}{0.010825in}}{\pgfqpoint{-0.020833in}{0.005525in}}{\pgfqpoint{-0.020833in}{0.000000in}}%
\pgfpathcurveto{\pgfqpoint{-0.020833in}{-0.005525in}}{\pgfqpoint{-0.018638in}{-0.010825in}}{\pgfqpoint{-0.014731in}{-0.014731in}}%
\pgfpathcurveto{\pgfqpoint{-0.010825in}{-0.018638in}}{\pgfqpoint{-0.005525in}{-0.020833in}}{\pgfqpoint{0.000000in}{-0.020833in}}%
\pgfpathlineto{\pgfqpoint{0.000000in}{-0.020833in}}%
\pgfpathclose%
\pgfusepath{stroke,fill}%
}%
\begin{pgfscope}%
\pgfsys@transformshift{0.809982in}{2.386573in}%
\pgfsys@useobject{currentmarker}{}%
\end{pgfscope}%
\begin{pgfscope}%
\pgfsys@transformshift{1.164146in}{2.316626in}%
\pgfsys@useobject{currentmarker}{}%
\end{pgfscope}%
\begin{pgfscope}%
\pgfsys@transformshift{1.516678in}{2.183606in}%
\pgfsys@useobject{currentmarker}{}%
\end{pgfscope}%
\begin{pgfscope}%
\pgfsys@transformshift{1.868568in}{1.687104in}%
\pgfsys@useobject{currentmarker}{}%
\end{pgfscope}%
\begin{pgfscope}%
\pgfsys@transformshift{2.219628in}{0.777079in}%
\pgfsys@useobject{currentmarker}{}%
\end{pgfscope}%
\begin{pgfscope}%
\pgfsys@transformshift{2.571345in}{0.636573in}%
\pgfsys@useobject{currentmarker}{}%
\end{pgfscope}%
\end{pgfscope}%
\begin{pgfscope}%
\pgfpathrectangle{\pgfqpoint{0.721913in}{0.549073in}}{\pgfqpoint{1.937500in}{1.925000in}}%
\pgfusepath{clip}%
\pgfsetrectcap%
\pgfsetroundjoin%
\pgfsetlinewidth{1.003750pt}%
\definecolor{currentstroke}{rgb}{0.976471,0.505882,0.145098}%
\pgfsetstrokecolor{currentstroke}%
\pgfsetdash{}{0pt}%
\pgfpathmoveto{\pgfqpoint{0.809982in}{2.326427in}}%
\pgfpathlineto{\pgfqpoint{1.164146in}{2.217542in}}%
\pgfpathlineto{\pgfqpoint{1.516678in}{2.027230in}}%
\pgfpathlineto{\pgfqpoint{1.868568in}{1.505345in}}%
\pgfpathlineto{\pgfqpoint{2.219628in}{0.856828in}}%
\pgfpathlineto{\pgfqpoint{2.571345in}{0.830200in}}%
\pgfusepath{stroke}%
\end{pgfscope}%
\begin{pgfscope}%
\pgfpathrectangle{\pgfqpoint{0.721913in}{0.549073in}}{\pgfqpoint{1.937500in}{1.925000in}}%
\pgfusepath{clip}%
\pgfsetbuttcap%
\pgfsetroundjoin%
\definecolor{currentfill}{rgb}{0.976471,0.505882,0.145098}%
\pgfsetfillcolor{currentfill}%
\pgfsetlinewidth{1.003750pt}%
\definecolor{currentstroke}{rgb}{0.976471,0.505882,0.145098}%
\pgfsetstrokecolor{currentstroke}%
\pgfsetdash{}{0pt}%
\pgfsys@defobject{currentmarker}{\pgfqpoint{-0.020833in}{-0.020833in}}{\pgfqpoint{0.020833in}{0.020833in}}{%
\pgfpathmoveto{\pgfqpoint{0.000000in}{-0.020833in}}%
\pgfpathcurveto{\pgfqpoint{0.005525in}{-0.020833in}}{\pgfqpoint{0.010825in}{-0.018638in}}{\pgfqpoint{0.014731in}{-0.014731in}}%
\pgfpathcurveto{\pgfqpoint{0.018638in}{-0.010825in}}{\pgfqpoint{0.020833in}{-0.005525in}}{\pgfqpoint{0.020833in}{0.000000in}}%
\pgfpathcurveto{\pgfqpoint{0.020833in}{0.005525in}}{\pgfqpoint{0.018638in}{0.010825in}}{\pgfqpoint{0.014731in}{0.014731in}}%
\pgfpathcurveto{\pgfqpoint{0.010825in}{0.018638in}}{\pgfqpoint{0.005525in}{0.020833in}}{\pgfqpoint{0.000000in}{0.020833in}}%
\pgfpathcurveto{\pgfqpoint{-0.005525in}{0.020833in}}{\pgfqpoint{-0.010825in}{0.018638in}}{\pgfqpoint{-0.014731in}{0.014731in}}%
\pgfpathcurveto{\pgfqpoint{-0.018638in}{0.010825in}}{\pgfqpoint{-0.020833in}{0.005525in}}{\pgfqpoint{-0.020833in}{0.000000in}}%
\pgfpathcurveto{\pgfqpoint{-0.020833in}{-0.005525in}}{\pgfqpoint{-0.018638in}{-0.010825in}}{\pgfqpoint{-0.014731in}{-0.014731in}}%
\pgfpathcurveto{\pgfqpoint{-0.010825in}{-0.018638in}}{\pgfqpoint{-0.005525in}{-0.020833in}}{\pgfqpoint{0.000000in}{-0.020833in}}%
\pgfpathlineto{\pgfqpoint{0.000000in}{-0.020833in}}%
\pgfpathclose%
\pgfusepath{stroke,fill}%
}%
\begin{pgfscope}%
\pgfsys@transformshift{0.809982in}{2.326427in}%
\pgfsys@useobject{currentmarker}{}%
\end{pgfscope}%
\begin{pgfscope}%
\pgfsys@transformshift{1.164146in}{2.217542in}%
\pgfsys@useobject{currentmarker}{}%
\end{pgfscope}%
\begin{pgfscope}%
\pgfsys@transformshift{1.516678in}{2.027230in}%
\pgfsys@useobject{currentmarker}{}%
\end{pgfscope}%
\begin{pgfscope}%
\pgfsys@transformshift{1.868568in}{1.505345in}%
\pgfsys@useobject{currentmarker}{}%
\end{pgfscope}%
\begin{pgfscope}%
\pgfsys@transformshift{2.219628in}{0.856828in}%
\pgfsys@useobject{currentmarker}{}%
\end{pgfscope}%
\begin{pgfscope}%
\pgfsys@transformshift{2.571345in}{0.830200in}%
\pgfsys@useobject{currentmarker}{}%
\end{pgfscope}%
\end{pgfscope}%
\begin{pgfscope}%
\pgfsetrectcap%
\pgfsetmiterjoin%
\pgfsetlinewidth{0.803000pt}%
\definecolor{currentstroke}{rgb}{0.000000,0.000000,0.000000}%
\pgfsetstrokecolor{currentstroke}%
\pgfsetdash{}{0pt}%
\pgfpathmoveto{\pgfqpoint{0.721913in}{0.549073in}}%
\pgfpathlineto{\pgfqpoint{0.721913in}{2.474073in}}%
\pgfusepath{stroke}%
\end{pgfscope}%
\begin{pgfscope}%
\pgfsetrectcap%
\pgfsetmiterjoin%
\pgfsetlinewidth{0.803000pt}%
\definecolor{currentstroke}{rgb}{0.000000,0.000000,0.000000}%
\pgfsetstrokecolor{currentstroke}%
\pgfsetdash{}{0pt}%
\pgfpathmoveto{\pgfqpoint{2.659413in}{0.549073in}}%
\pgfpathlineto{\pgfqpoint{2.659413in}{2.474073in}}%
\pgfusepath{stroke}%
\end{pgfscope}%
\begin{pgfscope}%
\pgfsetrectcap%
\pgfsetmiterjoin%
\pgfsetlinewidth{0.803000pt}%
\definecolor{currentstroke}{rgb}{0.000000,0.000000,0.000000}%
\pgfsetstrokecolor{currentstroke}%
\pgfsetdash{}{0pt}%
\pgfpathmoveto{\pgfqpoint{0.721913in}{0.549073in}}%
\pgfpathlineto{\pgfqpoint{2.659413in}{0.549073in}}%
\pgfusepath{stroke}%
\end{pgfscope}%
\begin{pgfscope}%
\pgfsetrectcap%
\pgfsetmiterjoin%
\pgfsetlinewidth{0.803000pt}%
\definecolor{currentstroke}{rgb}{0.000000,0.000000,0.000000}%
\pgfsetstrokecolor{currentstroke}%
\pgfsetdash{}{0pt}%
\pgfpathmoveto{\pgfqpoint{0.721913in}{2.474073in}}%
\pgfpathlineto{\pgfqpoint{2.659413in}{2.474073in}}%
\pgfusepath{stroke}%
\end{pgfscope}%
\begin{pgfscope}%
\pgfsetbuttcap%
\pgfsetmiterjoin%
\definecolor{currentfill}{rgb}{1.000000,1.000000,1.000000}%
\pgfsetfillcolor{currentfill}%
\pgfsetfillopacity{0.800000}%
\pgfsetlinewidth{1.003750pt}%
\definecolor{currentstroke}{rgb}{0.800000,0.800000,0.800000}%
\pgfsetstrokecolor{currentstroke}%
\pgfsetstrokeopacity{0.800000}%
\pgfsetdash{}{0pt}%
\pgfpathmoveto{\pgfqpoint{0.838580in}{0.632406in}}%
\pgfpathlineto{\pgfqpoint{1.865967in}{0.632406in}}%
\pgfpathquadraticcurveto{\pgfqpoint{1.899300in}{0.632406in}}{\pgfqpoint{1.899300in}{0.665739in}}%
\pgfpathlineto{\pgfqpoint{1.899300in}{1.346294in}}%
\pgfpathquadraticcurveto{\pgfqpoint{1.899300in}{1.379627in}}{\pgfqpoint{1.865967in}{1.379627in}}%
\pgfpathlineto{\pgfqpoint{0.838580in}{1.379627in}}%
\pgfpathquadraticcurveto{\pgfqpoint{0.805247in}{1.379627in}}{\pgfqpoint{0.805247in}{1.346294in}}%
\pgfpathlineto{\pgfqpoint{0.805247in}{0.665739in}}%
\pgfpathquadraticcurveto{\pgfqpoint{0.805247in}{0.632406in}}{\pgfqpoint{0.838580in}{0.632406in}}%
\pgfpathlineto{\pgfqpoint{0.838580in}{0.632406in}}%
\pgfpathclose%
\pgfusepath{stroke,fill}%
\end{pgfscope}%
\begin{pgfscope}%
\pgfsetrectcap%
\pgfsetroundjoin%
\pgfsetlinewidth{1.003750pt}%
\definecolor{currentstroke}{rgb}{0.537255,0.647059,0.760784}%
\pgfsetstrokecolor{currentstroke}%
\pgfsetdash{}{0pt}%
\pgfpathmoveto{\pgfqpoint{0.871913in}{1.254627in}}%
\pgfpathlineto{\pgfqpoint{1.038580in}{1.254627in}}%
\pgfpathlineto{\pgfqpoint{1.205247in}{1.254627in}}%
\pgfusepath{stroke}%
\end{pgfscope}%
\begin{pgfscope}%
\pgfsetbuttcap%
\pgfsetroundjoin%
\definecolor{currentfill}{rgb}{0.537255,0.647059,0.760784}%
\pgfsetfillcolor{currentfill}%
\pgfsetlinewidth{1.003750pt}%
\definecolor{currentstroke}{rgb}{0.537255,0.647059,0.760784}%
\pgfsetstrokecolor{currentstroke}%
\pgfsetdash{}{0pt}%
\pgfsys@defobject{currentmarker}{\pgfqpoint{-0.020833in}{-0.020833in}}{\pgfqpoint{0.020833in}{0.020833in}}{%
\pgfpathmoveto{\pgfqpoint{0.000000in}{-0.020833in}}%
\pgfpathcurveto{\pgfqpoint{0.005525in}{-0.020833in}}{\pgfqpoint{0.010825in}{-0.018638in}}{\pgfqpoint{0.014731in}{-0.014731in}}%
\pgfpathcurveto{\pgfqpoint{0.018638in}{-0.010825in}}{\pgfqpoint{0.020833in}{-0.005525in}}{\pgfqpoint{0.020833in}{0.000000in}}%
\pgfpathcurveto{\pgfqpoint{0.020833in}{0.005525in}}{\pgfqpoint{0.018638in}{0.010825in}}{\pgfqpoint{0.014731in}{0.014731in}}%
\pgfpathcurveto{\pgfqpoint{0.010825in}{0.018638in}}{\pgfqpoint{0.005525in}{0.020833in}}{\pgfqpoint{0.000000in}{0.020833in}}%
\pgfpathcurveto{\pgfqpoint{-0.005525in}{0.020833in}}{\pgfqpoint{-0.010825in}{0.018638in}}{\pgfqpoint{-0.014731in}{0.014731in}}%
\pgfpathcurveto{\pgfqpoint{-0.018638in}{0.010825in}}{\pgfqpoint{-0.020833in}{0.005525in}}{\pgfqpoint{-0.020833in}{0.000000in}}%
\pgfpathcurveto{\pgfqpoint{-0.020833in}{-0.005525in}}{\pgfqpoint{-0.018638in}{-0.010825in}}{\pgfqpoint{-0.014731in}{-0.014731in}}%
\pgfpathcurveto{\pgfqpoint{-0.010825in}{-0.018638in}}{\pgfqpoint{-0.005525in}{-0.020833in}}{\pgfqpoint{0.000000in}{-0.020833in}}%
\pgfpathlineto{\pgfqpoint{0.000000in}{-0.020833in}}%
\pgfpathclose%
\pgfusepath{stroke,fill}%
}%
\begin{pgfscope}%
\pgfsys@transformshift{1.038580in}{1.254627in}%
\pgfsys@useobject{currentmarker}{}%
\end{pgfscope}%
\end{pgfscope}%
\begin{pgfscope}%
\definecolor{textcolor}{rgb}{0.000000,0.000000,0.000000}%
\pgfsetstrokecolor{textcolor}%
\pgfsetfillcolor{textcolor}%
\pgftext[x=1.338580in,y=1.196294in,left,base]{\color{textcolor}{\rmfamily\fontsize{12.000000}{14.400000}\selectfont\catcode`\^=\active\def^{\ifmmode\sp\else\^{}\fi}\catcode`\%=\active\def%{\%}DGC}}%
\end{pgfscope}%
\begin{pgfscope}%
\pgfsetrectcap%
\pgfsetroundjoin%
\pgfsetlinewidth{1.003750pt}%
\definecolor{currentstroke}{rgb}{0.184314,0.270588,0.360784}%
\pgfsetstrokecolor{currentstroke}%
\pgfsetdash{}{0pt}%
\pgfpathmoveto{\pgfqpoint{0.871913in}{1.022220in}}%
\pgfpathlineto{\pgfqpoint{1.038580in}{1.022220in}}%
\pgfpathlineto{\pgfqpoint{1.205247in}{1.022220in}}%
\pgfusepath{stroke}%
\end{pgfscope}%
\begin{pgfscope}%
\pgfsetbuttcap%
\pgfsetroundjoin%
\definecolor{currentfill}{rgb}{0.184314,0.270588,0.360784}%
\pgfsetfillcolor{currentfill}%
\pgfsetlinewidth{1.003750pt}%
\definecolor{currentstroke}{rgb}{0.184314,0.270588,0.360784}%
\pgfsetstrokecolor{currentstroke}%
\pgfsetdash{}{0pt}%
\pgfsys@defobject{currentmarker}{\pgfqpoint{-0.020833in}{-0.020833in}}{\pgfqpoint{0.020833in}{0.020833in}}{%
\pgfpathmoveto{\pgfqpoint{0.000000in}{-0.020833in}}%
\pgfpathcurveto{\pgfqpoint{0.005525in}{-0.020833in}}{\pgfqpoint{0.010825in}{-0.018638in}}{\pgfqpoint{0.014731in}{-0.014731in}}%
\pgfpathcurveto{\pgfqpoint{0.018638in}{-0.010825in}}{\pgfqpoint{0.020833in}{-0.005525in}}{\pgfqpoint{0.020833in}{0.000000in}}%
\pgfpathcurveto{\pgfqpoint{0.020833in}{0.005525in}}{\pgfqpoint{0.018638in}{0.010825in}}{\pgfqpoint{0.014731in}{0.014731in}}%
\pgfpathcurveto{\pgfqpoint{0.010825in}{0.018638in}}{\pgfqpoint{0.005525in}{0.020833in}}{\pgfqpoint{0.000000in}{0.020833in}}%
\pgfpathcurveto{\pgfqpoint{-0.005525in}{0.020833in}}{\pgfqpoint{-0.010825in}{0.018638in}}{\pgfqpoint{-0.014731in}{0.014731in}}%
\pgfpathcurveto{\pgfqpoint{-0.018638in}{0.010825in}}{\pgfqpoint{-0.020833in}{0.005525in}}{\pgfqpoint{-0.020833in}{0.000000in}}%
\pgfpathcurveto{\pgfqpoint{-0.020833in}{-0.005525in}}{\pgfqpoint{-0.018638in}{-0.010825in}}{\pgfqpoint{-0.014731in}{-0.014731in}}%
\pgfpathcurveto{\pgfqpoint{-0.010825in}{-0.018638in}}{\pgfqpoint{-0.005525in}{-0.020833in}}{\pgfqpoint{0.000000in}{-0.020833in}}%
\pgfpathlineto{\pgfqpoint{0.000000in}{-0.020833in}}%
\pgfpathclose%
\pgfusepath{stroke,fill}%
}%
\begin{pgfscope}%
\pgfsys@transformshift{1.038580in}{1.022220in}%
\pgfsys@useobject{currentmarker}{}%
\end{pgfscope}%
\end{pgfscope}%
\begin{pgfscope}%
\definecolor{textcolor}{rgb}{0.000000,0.000000,0.000000}%
\pgfsetstrokecolor{textcolor}%
\pgfsetfillcolor{textcolor}%
\pgftext[x=1.338580in,y=0.963887in,left,base]{\color{textcolor}{\rmfamily\fontsize{12.000000}{14.400000}\selectfont\catcode`\^=\active\def^{\ifmmode\sp\else\^{}\fi}\catcode`\%=\active\def%{\%}NC}}%
\end{pgfscope}%
\begin{pgfscope}%
\pgfsetrectcap%
\pgfsetroundjoin%
\pgfsetlinewidth{1.003750pt}%
\definecolor{currentstroke}{rgb}{0.976471,0.505882,0.145098}%
\pgfsetstrokecolor{currentstroke}%
\pgfsetdash{}{0pt}%
\pgfpathmoveto{\pgfqpoint{0.871913in}{0.789813in}}%
\pgfpathlineto{\pgfqpoint{1.038580in}{0.789813in}}%
\pgfpathlineto{\pgfqpoint{1.205247in}{0.789813in}}%
\pgfusepath{stroke}%
\end{pgfscope}%
\begin{pgfscope}%
\pgfsetbuttcap%
\pgfsetroundjoin%
\definecolor{currentfill}{rgb}{0.976471,0.505882,0.145098}%
\pgfsetfillcolor{currentfill}%
\pgfsetlinewidth{1.003750pt}%
\definecolor{currentstroke}{rgb}{0.976471,0.505882,0.145098}%
\pgfsetstrokecolor{currentstroke}%
\pgfsetdash{}{0pt}%
\pgfsys@defobject{currentmarker}{\pgfqpoint{-0.020833in}{-0.020833in}}{\pgfqpoint{0.020833in}{0.020833in}}{%
\pgfpathmoveto{\pgfqpoint{0.000000in}{-0.020833in}}%
\pgfpathcurveto{\pgfqpoint{0.005525in}{-0.020833in}}{\pgfqpoint{0.010825in}{-0.018638in}}{\pgfqpoint{0.014731in}{-0.014731in}}%
\pgfpathcurveto{\pgfqpoint{0.018638in}{-0.010825in}}{\pgfqpoint{0.020833in}{-0.005525in}}{\pgfqpoint{0.020833in}{0.000000in}}%
\pgfpathcurveto{\pgfqpoint{0.020833in}{0.005525in}}{\pgfqpoint{0.018638in}{0.010825in}}{\pgfqpoint{0.014731in}{0.014731in}}%
\pgfpathcurveto{\pgfqpoint{0.010825in}{0.018638in}}{\pgfqpoint{0.005525in}{0.020833in}}{\pgfqpoint{0.000000in}{0.020833in}}%
\pgfpathcurveto{\pgfqpoint{-0.005525in}{0.020833in}}{\pgfqpoint{-0.010825in}{0.018638in}}{\pgfqpoint{-0.014731in}{0.014731in}}%
\pgfpathcurveto{\pgfqpoint{-0.018638in}{0.010825in}}{\pgfqpoint{-0.020833in}{0.005525in}}{\pgfqpoint{-0.020833in}{0.000000in}}%
\pgfpathcurveto{\pgfqpoint{-0.020833in}{-0.005525in}}{\pgfqpoint{-0.018638in}{-0.010825in}}{\pgfqpoint{-0.014731in}{-0.014731in}}%
\pgfpathcurveto{\pgfqpoint{-0.010825in}{-0.018638in}}{\pgfqpoint{-0.005525in}{-0.020833in}}{\pgfqpoint{0.000000in}{-0.020833in}}%
\pgfpathlineto{\pgfqpoint{0.000000in}{-0.020833in}}%
\pgfpathclose%
\pgfusepath{stroke,fill}%
}%
\begin{pgfscope}%
\pgfsys@transformshift{1.038580in}{0.789813in}%
\pgfsys@useobject{currentmarker}{}%
\end{pgfscope}%
\end{pgfscope}%
\begin{pgfscope}%
\definecolor{textcolor}{rgb}{0.000000,0.000000,0.000000}%
\pgfsetstrokecolor{textcolor}%
\pgfsetfillcolor{textcolor}%
\pgftext[x=1.338580in,y=0.731480in,left,base]{\color{textcolor}{\rmfamily\fontsize{12.000000}{14.400000}\selectfont\catcode`\^=\active\def^{\ifmmode\sp\else\^{}\fi}\catcode`\%=\active\def%{\%}NC++}}%
\end{pgfscope}%
\end{pgfpicture}%
\makeatother%
\endgroup%

        \caption{\gls{sketch-size} $+$ \gls{num-hutchinson-queries} $=160$}
        \label{fig:5-experiments-electronic-structure-convergence-m-nv160}
    \end{subfigure}
    \caption{For increasing values of \gls{chebyshev-degree} but fixed
    \gls{sketch-size} $+$ \gls{num-hutchinson-queries} we plot the $L^1$ relative
    approximation error \refequ{equ:5-experiments-L1-error}
    for the model problem with \gls{smoothing-parameter} $=0.05$.}
    \label{fig:5-experiments-electronic-structure-convergence-m}
\end{figure}

In \reffig{fig:5-experiments-electronic-structure-convergence-nv-m800} the
Chebyshev expansion is clearly not accurate enough for a good approximation
of the spectral density. This is confirmed by \reffig{fig:5-experiments-electronic-structure-convergence-m}:
unless a Chebyshev expansion of degree \gls{chebyshev-degree} $>1000$ is used,
we cannot hope for high accuracy approximations.
\Reffig{fig:5-experiments-electronic-structure-convergence-nv-m2400} allows us to
make an interesting observation: the approximation error for the \gls{NCPP} method
first decays quite slowly compared to the \gls{NC} method.
The convergence is approximately of order $\mathcal{\varepsilon^{-1}}$
as suggested by \refthm{thm:4-nystromchebyshev-trace-correction-parameter-dependent}.
However, after \gls{sketch-size} $+$ \gls{num-hutchinson-queries}
exceeds a certain value, the approximation error shoots down quickly to where it
saturates. The reason is that at this point \gls{sketch-size} starts exceeding
the \gls{numerical-rank} of the model matrix, which, as a consequence of
\refthm{thm:3-nystrom-frobenius-norm}, means that the approximation error is
expected to be significantly smaller than \refthm{thm:4-nystromchebyshev-final}
guarantees in general. In fact, it seems that after this point, \gls{NC} and \gls{NCPP}
behave almost identically, with the exception that the \gls{NC} uses 
a \gls{sketch-size} which is twice as large as the one in \gls{NCPP} by design
of the experiment, while the contribution from the Hutchinson's correction part
in \refequ{equ:4-nystromchebyshev-hutch-pp} seems to be insignificant.\\

In \reftab{tab:5-experiments-timing-DGC} we list the wall time each method
takes to compute an approximate \gls{spectral-density} at \gls{num-evaluation-points} $=100$ points
for different values of \gls{sketch-size} and \gls{chebyshev-degree}.\\

\begin{table}[ht]
    \caption{Comparison of the runtime in seconds of the algorithms applied to the model problem
        for approximating the \glsfirst{smooth-spectral-density} with 
        \gls{smoothing-parameter} $=0.05$ at \gls{num-evaluation-points} $=100$
        points for various choices of \gls{chebyshev-degree} and \gls{sketch-size} $+$ \gls{num-hutchinson-queries}.
        The mean and standard deviation of 7 runs is given.}
    \label{tab:5-experiments-timing-DGC}
    \centering
\renewcommand{\arraystretch}{1.2}
\begin{tabular}{@{}lcccc@{}}
\toprule
 & \shortstack[c]{$m=800$ \\ $n_{\Omega} + n_{\Psi}=40$} & \shortstack[c]{$m=2400$ \\ $n_{\Omega} + n_{\Psi}=40$} & \shortstack[c]{$m=800$ \\ $n_{\Omega} + n_{\Psi}=160$} & \shortstack[c]{$m=2400$ \\ $n_{\Omega} + n_{\Psi}=160$}\\
\midrule
DGC & 0.27 $\pm$ 0.01 & 0.80 $\pm$ 0.02 & 2.06 $\pm$ 0.05 & 6.06 $\pm$ 0.01 \\
NC & 1.85 $\pm$ 0.00 & 5.54 $\pm$ 0.01 & 7.24 $\pm$ 0.05 & 21.56 $\pm$ 0.09 \\
NC++ & 1.95 $\pm$ 0.00 & 5.83 $\pm$ 0.01 & 5.64 $\pm$ 0.43 & 16.21 $\pm$ 0.01 \\
\bottomrule
\end{tabular}

\end{table}

The \gls{NCPP} is a hybrid method between the \gls{DGC} and \gls{NC} methods.
In fact, for \gls{sketch-size} $=0$, the \gls{NCPP} is equivalent to the \gls{DGC}
method, while for \gls{num-hutchinson-queries} $=0$, it is equivalent to the \gls{NC} method.
Back in \refchp{chp:3-nystrom} we already saw that for small values of \gls{smoothing-parameter}
the Nystr\"om approximation will only need a small \gls{sketching-matrix} in order
to achieve an accurate approximation. On the other hand, for large choices of
\gls{smoothing-parameter} the low-rank approximation will by itself not suffice.
The interplay between the two parts which make up the \gls{NCPP},
on one hand the low-rank approximation and on the other hand the
trace estimation on the residual, is illustrated well in
\reffig{fig:5-experiments-electronic-structure-matvec-mixture}.
For various values of \gls{smoothing-parameter} and a simultaneously changing
\gls{chebyshev-degree} $=120 / \sigma$ to keep an approximately equal interpolation
accuracy, the behavior of the error for fixed \gls{sketch-size} $+$ \gls{num-hutchinson-queries} $=80$ is plotted.

\begin{figure}[ht]
    \centering
    %% Creator: Matplotlib, PGF backend
%%
%% To include the figure in your LaTeX document, write
%%   \input{<filename>.pgf}
%%
%% Make sure the required packages are loaded in your preamble
%%   \usepackage{pgf}
%%
%% Also ensure that all the required font packages are loaded; for instance,
%% the lmodern package is sometimes necessary when using math font.
%%   \usepackage{lmodern}
%%
%% Figures using additional raster images can only be included by \input if
%% they are in the same directory as the main LaTeX file. For loading figures
%% from other directories you can use the `import` package
%%   \usepackage{import}
%%
%% and then include the figures with
%%   \import{<path to file>}{<filename>.pgf}
%%
%% Matplotlib used the following preamble
%%   \def\mathdefault#1{#1}
%%   \everymath=\expandafter{\the\everymath\displaystyle}
%%   \IfFileExists{scrextend.sty}{
%%     \usepackage[fontsize=12.000000pt]{scrextend}
%%   }{
%%     \renewcommand{\normalsize}{\fontsize{12.000000}{14.400000}\selectfont}
%%     \normalsize
%%   }
%%   
%%   \ifdefined\pdftexversion\else  % non-pdftex case.
%%     \usepackage{fontspec}
%%     \setmainfont{DejaVuSans.ttf}[Path=\detokenize{/opt/hostedtoolcache/Python/3.12.9/x64/lib/python3.12/site-packages/matplotlib/mpl-data/fonts/ttf/}]
%%     \setsansfont{DejaVuSans.ttf}[Path=\detokenize{/opt/hostedtoolcache/Python/3.12.9/x64/lib/python3.12/site-packages/matplotlib/mpl-data/fonts/ttf/}]
%%     \setmonofont{DejaVuSansMono.ttf}[Path=\detokenize{/opt/hostedtoolcache/Python/3.12.9/x64/lib/python3.12/site-packages/matplotlib/mpl-data/fonts/ttf/}]
%%   \fi
%%   \makeatletter\@ifpackageloaded{underscore}{}{\usepackage[strings]{underscore}}\makeatother
%%
\begingroup%
\makeatletter%
\begin{pgfpicture}%
\pgfpathrectangle{\pgfpointorigin}{\pgfqpoint{5.471913in}{2.959073in}}%
\pgfusepath{use as bounding box, clip}%
\begin{pgfscope}%
\pgfsetbuttcap%
\pgfsetmiterjoin%
\definecolor{currentfill}{rgb}{1.000000,1.000000,1.000000}%
\pgfsetfillcolor{currentfill}%
\pgfsetlinewidth{0.000000pt}%
\definecolor{currentstroke}{rgb}{1.000000,1.000000,1.000000}%
\pgfsetstrokecolor{currentstroke}%
\pgfsetdash{}{0pt}%
\pgfpathmoveto{\pgfqpoint{0.000000in}{-0.000000in}}%
\pgfpathlineto{\pgfqpoint{5.471913in}{-0.000000in}}%
\pgfpathlineto{\pgfqpoint{5.471913in}{2.959073in}}%
\pgfpathlineto{\pgfqpoint{0.000000in}{2.959073in}}%
\pgfpathlineto{\pgfqpoint{0.000000in}{-0.000000in}}%
\pgfpathclose%
\pgfusepath{fill}%
\end{pgfscope}%
\begin{pgfscope}%
\pgfsetbuttcap%
\pgfsetmiterjoin%
\definecolor{currentfill}{rgb}{1.000000,1.000000,1.000000}%
\pgfsetfillcolor{currentfill}%
\pgfsetlinewidth{0.000000pt}%
\definecolor{currentstroke}{rgb}{0.000000,0.000000,0.000000}%
\pgfsetstrokecolor{currentstroke}%
\pgfsetstrokeopacity{0.000000}%
\pgfsetdash{}{0pt}%
\pgfpathmoveto{\pgfqpoint{0.721913in}{0.549073in}}%
\pgfpathlineto{\pgfqpoint{5.371913in}{0.549073in}}%
\pgfpathlineto{\pgfqpoint{5.371913in}{2.859073in}}%
\pgfpathlineto{\pgfqpoint{0.721913in}{2.859073in}}%
\pgfpathlineto{\pgfqpoint{0.721913in}{0.549073in}}%
\pgfpathclose%
\pgfusepath{fill}%
\end{pgfscope}%
\begin{pgfscope}%
\pgfsetbuttcap%
\pgfsetroundjoin%
\definecolor{currentfill}{rgb}{0.000000,0.000000,0.000000}%
\pgfsetfillcolor{currentfill}%
\pgfsetlinewidth{0.803000pt}%
\definecolor{currentstroke}{rgb}{0.000000,0.000000,0.000000}%
\pgfsetstrokecolor{currentstroke}%
\pgfsetdash{}{0pt}%
\pgfsys@defobject{currentmarker}{\pgfqpoint{0.000000in}{-0.048611in}}{\pgfqpoint{0.000000in}{0.000000in}}{%
\pgfpathmoveto{\pgfqpoint{0.000000in}{0.000000in}}%
\pgfpathlineto{\pgfqpoint{0.000000in}{-0.048611in}}%
\pgfusepath{stroke,fill}%
}%
\begin{pgfscope}%
\pgfsys@transformshift{0.933277in}{0.549073in}%
\pgfsys@useobject{currentmarker}{}%
\end{pgfscope}%
\end{pgfscope}%
\begin{pgfscope}%
\definecolor{textcolor}{rgb}{0.000000,0.000000,0.000000}%
\pgfsetstrokecolor{textcolor}%
\pgfsetfillcolor{textcolor}%
\pgftext[x=0.933277in,y=0.451851in,,top]{\color{textcolor}{\rmfamily\fontsize{12.000000}{14.400000}\selectfont\catcode`\^=\active\def^{\ifmmode\sp\else\^{}\fi}\catcode`\%=\active\def%{\%}$\mathdefault{10^{-2}}$}}%
\end{pgfscope}%
\begin{pgfscope}%
\pgfsetbuttcap%
\pgfsetroundjoin%
\definecolor{currentfill}{rgb}{0.000000,0.000000,0.000000}%
\pgfsetfillcolor{currentfill}%
\pgfsetlinewidth{0.803000pt}%
\definecolor{currentstroke}{rgb}{0.000000,0.000000,0.000000}%
\pgfsetstrokecolor{currentstroke}%
\pgfsetdash{}{0pt}%
\pgfsys@defobject{currentmarker}{\pgfqpoint{0.000000in}{-0.048611in}}{\pgfqpoint{0.000000in}{0.000000in}}{%
\pgfpathmoveto{\pgfqpoint{0.000000in}{0.000000in}}%
\pgfpathlineto{\pgfqpoint{0.000000in}{-0.048611in}}%
\pgfusepath{stroke,fill}%
}%
\begin{pgfscope}%
\pgfsys@transformshift{3.046913in}{0.549073in}%
\pgfsys@useobject{currentmarker}{}%
\end{pgfscope}%
\end{pgfscope}%
\begin{pgfscope}%
\definecolor{textcolor}{rgb}{0.000000,0.000000,0.000000}%
\pgfsetstrokecolor{textcolor}%
\pgfsetfillcolor{textcolor}%
\pgftext[x=3.046913in,y=0.451851in,,top]{\color{textcolor}{\rmfamily\fontsize{12.000000}{14.400000}\selectfont\catcode`\^=\active\def^{\ifmmode\sp\else\^{}\fi}\catcode`\%=\active\def%{\%}$\mathdefault{10^{-1}}$}}%
\end{pgfscope}%
\begin{pgfscope}%
\pgfsetbuttcap%
\pgfsetroundjoin%
\definecolor{currentfill}{rgb}{0.000000,0.000000,0.000000}%
\pgfsetfillcolor{currentfill}%
\pgfsetlinewidth{0.803000pt}%
\definecolor{currentstroke}{rgb}{0.000000,0.000000,0.000000}%
\pgfsetstrokecolor{currentstroke}%
\pgfsetdash{}{0pt}%
\pgfsys@defobject{currentmarker}{\pgfqpoint{0.000000in}{-0.048611in}}{\pgfqpoint{0.000000in}{0.000000in}}{%
\pgfpathmoveto{\pgfqpoint{0.000000in}{0.000000in}}%
\pgfpathlineto{\pgfqpoint{0.000000in}{-0.048611in}}%
\pgfusepath{stroke,fill}%
}%
\begin{pgfscope}%
\pgfsys@transformshift{5.160550in}{0.549073in}%
\pgfsys@useobject{currentmarker}{}%
\end{pgfscope}%
\end{pgfscope}%
\begin{pgfscope}%
\definecolor{textcolor}{rgb}{0.000000,0.000000,0.000000}%
\pgfsetstrokecolor{textcolor}%
\pgfsetfillcolor{textcolor}%
\pgftext[x=5.160550in,y=0.451851in,,top]{\color{textcolor}{\rmfamily\fontsize{12.000000}{14.400000}\selectfont\catcode`\^=\active\def^{\ifmmode\sp\else\^{}\fi}\catcode`\%=\active\def%{\%}$\mathdefault{10^{0}}$}}%
\end{pgfscope}%
\begin{pgfscope}%
\pgfsetbuttcap%
\pgfsetroundjoin%
\definecolor{currentfill}{rgb}{0.000000,0.000000,0.000000}%
\pgfsetfillcolor{currentfill}%
\pgfsetlinewidth{0.602250pt}%
\definecolor{currentstroke}{rgb}{0.000000,0.000000,0.000000}%
\pgfsetstrokecolor{currentstroke}%
\pgfsetdash{}{0pt}%
\pgfsys@defobject{currentmarker}{\pgfqpoint{0.000000in}{-0.027778in}}{\pgfqpoint{0.000000in}{0.000000in}}{%
\pgfpathmoveto{\pgfqpoint{0.000000in}{0.000000in}}%
\pgfpathlineto{\pgfqpoint{0.000000in}{-0.027778in}}%
\pgfusepath{stroke,fill}%
}%
\begin{pgfscope}%
\pgfsys@transformshift{0.728445in}{0.549073in}%
\pgfsys@useobject{currentmarker}{}%
\end{pgfscope}%
\end{pgfscope}%
\begin{pgfscope}%
\pgfsetbuttcap%
\pgfsetroundjoin%
\definecolor{currentfill}{rgb}{0.000000,0.000000,0.000000}%
\pgfsetfillcolor{currentfill}%
\pgfsetlinewidth{0.602250pt}%
\definecolor{currentstroke}{rgb}{0.000000,0.000000,0.000000}%
\pgfsetstrokecolor{currentstroke}%
\pgfsetdash{}{0pt}%
\pgfsys@defobject{currentmarker}{\pgfqpoint{0.000000in}{-0.027778in}}{\pgfqpoint{0.000000in}{0.000000in}}{%
\pgfpathmoveto{\pgfqpoint{0.000000in}{0.000000in}}%
\pgfpathlineto{\pgfqpoint{0.000000in}{-0.027778in}}%
\pgfusepath{stroke,fill}%
}%
\begin{pgfscope}%
\pgfsys@transformshift{0.836562in}{0.549073in}%
\pgfsys@useobject{currentmarker}{}%
\end{pgfscope}%
\end{pgfscope}%
\begin{pgfscope}%
\pgfsetbuttcap%
\pgfsetroundjoin%
\definecolor{currentfill}{rgb}{0.000000,0.000000,0.000000}%
\pgfsetfillcolor{currentfill}%
\pgfsetlinewidth{0.602250pt}%
\definecolor{currentstroke}{rgb}{0.000000,0.000000,0.000000}%
\pgfsetstrokecolor{currentstroke}%
\pgfsetdash{}{0pt}%
\pgfsys@defobject{currentmarker}{\pgfqpoint{0.000000in}{-0.027778in}}{\pgfqpoint{0.000000in}{0.000000in}}{%
\pgfpathmoveto{\pgfqpoint{0.000000in}{0.000000in}}%
\pgfpathlineto{\pgfqpoint{0.000000in}{-0.027778in}}%
\pgfusepath{stroke,fill}%
}%
\begin{pgfscope}%
\pgfsys@transformshift{1.569545in}{0.549073in}%
\pgfsys@useobject{currentmarker}{}%
\end{pgfscope}%
\end{pgfscope}%
\begin{pgfscope}%
\pgfsetbuttcap%
\pgfsetroundjoin%
\definecolor{currentfill}{rgb}{0.000000,0.000000,0.000000}%
\pgfsetfillcolor{currentfill}%
\pgfsetlinewidth{0.602250pt}%
\definecolor{currentstroke}{rgb}{0.000000,0.000000,0.000000}%
\pgfsetstrokecolor{currentstroke}%
\pgfsetdash{}{0pt}%
\pgfsys@defobject{currentmarker}{\pgfqpoint{0.000000in}{-0.027778in}}{\pgfqpoint{0.000000in}{0.000000in}}{%
\pgfpathmoveto{\pgfqpoint{0.000000in}{0.000000in}}%
\pgfpathlineto{\pgfqpoint{0.000000in}{-0.027778in}}%
\pgfusepath{stroke,fill}%
}%
\begin{pgfscope}%
\pgfsys@transformshift{1.941738in}{0.549073in}%
\pgfsys@useobject{currentmarker}{}%
\end{pgfscope}%
\end{pgfscope}%
\begin{pgfscope}%
\pgfsetbuttcap%
\pgfsetroundjoin%
\definecolor{currentfill}{rgb}{0.000000,0.000000,0.000000}%
\pgfsetfillcolor{currentfill}%
\pgfsetlinewidth{0.602250pt}%
\definecolor{currentstroke}{rgb}{0.000000,0.000000,0.000000}%
\pgfsetstrokecolor{currentstroke}%
\pgfsetdash{}{0pt}%
\pgfsys@defobject{currentmarker}{\pgfqpoint{0.000000in}{-0.027778in}}{\pgfqpoint{0.000000in}{0.000000in}}{%
\pgfpathmoveto{\pgfqpoint{0.000000in}{0.000000in}}%
\pgfpathlineto{\pgfqpoint{0.000000in}{-0.027778in}}%
\pgfusepath{stroke,fill}%
}%
\begin{pgfscope}%
\pgfsys@transformshift{2.205813in}{0.549073in}%
\pgfsys@useobject{currentmarker}{}%
\end{pgfscope}%
\end{pgfscope}%
\begin{pgfscope}%
\pgfsetbuttcap%
\pgfsetroundjoin%
\definecolor{currentfill}{rgb}{0.000000,0.000000,0.000000}%
\pgfsetfillcolor{currentfill}%
\pgfsetlinewidth{0.602250pt}%
\definecolor{currentstroke}{rgb}{0.000000,0.000000,0.000000}%
\pgfsetstrokecolor{currentstroke}%
\pgfsetdash{}{0pt}%
\pgfsys@defobject{currentmarker}{\pgfqpoint{0.000000in}{-0.027778in}}{\pgfqpoint{0.000000in}{0.000000in}}{%
\pgfpathmoveto{\pgfqpoint{0.000000in}{0.000000in}}%
\pgfpathlineto{\pgfqpoint{0.000000in}{-0.027778in}}%
\pgfusepath{stroke,fill}%
}%
\begin{pgfscope}%
\pgfsys@transformshift{2.410646in}{0.549073in}%
\pgfsys@useobject{currentmarker}{}%
\end{pgfscope}%
\end{pgfscope}%
\begin{pgfscope}%
\pgfsetbuttcap%
\pgfsetroundjoin%
\definecolor{currentfill}{rgb}{0.000000,0.000000,0.000000}%
\pgfsetfillcolor{currentfill}%
\pgfsetlinewidth{0.602250pt}%
\definecolor{currentstroke}{rgb}{0.000000,0.000000,0.000000}%
\pgfsetstrokecolor{currentstroke}%
\pgfsetdash{}{0pt}%
\pgfsys@defobject{currentmarker}{\pgfqpoint{0.000000in}{-0.027778in}}{\pgfqpoint{0.000000in}{0.000000in}}{%
\pgfpathmoveto{\pgfqpoint{0.000000in}{0.000000in}}%
\pgfpathlineto{\pgfqpoint{0.000000in}{-0.027778in}}%
\pgfusepath{stroke,fill}%
}%
\begin{pgfscope}%
\pgfsys@transformshift{2.578006in}{0.549073in}%
\pgfsys@useobject{currentmarker}{}%
\end{pgfscope}%
\end{pgfscope}%
\begin{pgfscope}%
\pgfsetbuttcap%
\pgfsetroundjoin%
\definecolor{currentfill}{rgb}{0.000000,0.000000,0.000000}%
\pgfsetfillcolor{currentfill}%
\pgfsetlinewidth{0.602250pt}%
\definecolor{currentstroke}{rgb}{0.000000,0.000000,0.000000}%
\pgfsetstrokecolor{currentstroke}%
\pgfsetdash{}{0pt}%
\pgfsys@defobject{currentmarker}{\pgfqpoint{0.000000in}{-0.027778in}}{\pgfqpoint{0.000000in}{0.000000in}}{%
\pgfpathmoveto{\pgfqpoint{0.000000in}{0.000000in}}%
\pgfpathlineto{\pgfqpoint{0.000000in}{-0.027778in}}%
\pgfusepath{stroke,fill}%
}%
\begin{pgfscope}%
\pgfsys@transformshift{2.719507in}{0.549073in}%
\pgfsys@useobject{currentmarker}{}%
\end{pgfscope}%
\end{pgfscope}%
\begin{pgfscope}%
\pgfsetbuttcap%
\pgfsetroundjoin%
\definecolor{currentfill}{rgb}{0.000000,0.000000,0.000000}%
\pgfsetfillcolor{currentfill}%
\pgfsetlinewidth{0.602250pt}%
\definecolor{currentstroke}{rgb}{0.000000,0.000000,0.000000}%
\pgfsetstrokecolor{currentstroke}%
\pgfsetdash{}{0pt}%
\pgfsys@defobject{currentmarker}{\pgfqpoint{0.000000in}{-0.027778in}}{\pgfqpoint{0.000000in}{0.000000in}}{%
\pgfpathmoveto{\pgfqpoint{0.000000in}{0.000000in}}%
\pgfpathlineto{\pgfqpoint{0.000000in}{-0.027778in}}%
\pgfusepath{stroke,fill}%
}%
\begin{pgfscope}%
\pgfsys@transformshift{2.842081in}{0.549073in}%
\pgfsys@useobject{currentmarker}{}%
\end{pgfscope}%
\end{pgfscope}%
\begin{pgfscope}%
\pgfsetbuttcap%
\pgfsetroundjoin%
\definecolor{currentfill}{rgb}{0.000000,0.000000,0.000000}%
\pgfsetfillcolor{currentfill}%
\pgfsetlinewidth{0.602250pt}%
\definecolor{currentstroke}{rgb}{0.000000,0.000000,0.000000}%
\pgfsetstrokecolor{currentstroke}%
\pgfsetdash{}{0pt}%
\pgfsys@defobject{currentmarker}{\pgfqpoint{0.000000in}{-0.027778in}}{\pgfqpoint{0.000000in}{0.000000in}}{%
\pgfpathmoveto{\pgfqpoint{0.000000in}{0.000000in}}%
\pgfpathlineto{\pgfqpoint{0.000000in}{-0.027778in}}%
\pgfusepath{stroke,fill}%
}%
\begin{pgfscope}%
\pgfsys@transformshift{2.950199in}{0.549073in}%
\pgfsys@useobject{currentmarker}{}%
\end{pgfscope}%
\end{pgfscope}%
\begin{pgfscope}%
\pgfsetbuttcap%
\pgfsetroundjoin%
\definecolor{currentfill}{rgb}{0.000000,0.000000,0.000000}%
\pgfsetfillcolor{currentfill}%
\pgfsetlinewidth{0.602250pt}%
\definecolor{currentstroke}{rgb}{0.000000,0.000000,0.000000}%
\pgfsetstrokecolor{currentstroke}%
\pgfsetdash{}{0pt}%
\pgfsys@defobject{currentmarker}{\pgfqpoint{0.000000in}{-0.027778in}}{\pgfqpoint{0.000000in}{0.000000in}}{%
\pgfpathmoveto{\pgfqpoint{0.000000in}{0.000000in}}%
\pgfpathlineto{\pgfqpoint{0.000000in}{-0.027778in}}%
\pgfusepath{stroke,fill}%
}%
\begin{pgfscope}%
\pgfsys@transformshift{3.683181in}{0.549073in}%
\pgfsys@useobject{currentmarker}{}%
\end{pgfscope}%
\end{pgfscope}%
\begin{pgfscope}%
\pgfsetbuttcap%
\pgfsetroundjoin%
\definecolor{currentfill}{rgb}{0.000000,0.000000,0.000000}%
\pgfsetfillcolor{currentfill}%
\pgfsetlinewidth{0.602250pt}%
\definecolor{currentstroke}{rgb}{0.000000,0.000000,0.000000}%
\pgfsetstrokecolor{currentstroke}%
\pgfsetdash{}{0pt}%
\pgfsys@defobject{currentmarker}{\pgfqpoint{0.000000in}{-0.027778in}}{\pgfqpoint{0.000000in}{0.000000in}}{%
\pgfpathmoveto{\pgfqpoint{0.000000in}{0.000000in}}%
\pgfpathlineto{\pgfqpoint{0.000000in}{-0.027778in}}%
\pgfusepath{stroke,fill}%
}%
\begin{pgfscope}%
\pgfsys@transformshift{4.055374in}{0.549073in}%
\pgfsys@useobject{currentmarker}{}%
\end{pgfscope}%
\end{pgfscope}%
\begin{pgfscope}%
\pgfsetbuttcap%
\pgfsetroundjoin%
\definecolor{currentfill}{rgb}{0.000000,0.000000,0.000000}%
\pgfsetfillcolor{currentfill}%
\pgfsetlinewidth{0.602250pt}%
\definecolor{currentstroke}{rgb}{0.000000,0.000000,0.000000}%
\pgfsetstrokecolor{currentstroke}%
\pgfsetdash{}{0pt}%
\pgfsys@defobject{currentmarker}{\pgfqpoint{0.000000in}{-0.027778in}}{\pgfqpoint{0.000000in}{0.000000in}}{%
\pgfpathmoveto{\pgfqpoint{0.000000in}{0.000000in}}%
\pgfpathlineto{\pgfqpoint{0.000000in}{-0.027778in}}%
\pgfusepath{stroke,fill}%
}%
\begin{pgfscope}%
\pgfsys@transformshift{4.319449in}{0.549073in}%
\pgfsys@useobject{currentmarker}{}%
\end{pgfscope}%
\end{pgfscope}%
\begin{pgfscope}%
\pgfsetbuttcap%
\pgfsetroundjoin%
\definecolor{currentfill}{rgb}{0.000000,0.000000,0.000000}%
\pgfsetfillcolor{currentfill}%
\pgfsetlinewidth{0.602250pt}%
\definecolor{currentstroke}{rgb}{0.000000,0.000000,0.000000}%
\pgfsetstrokecolor{currentstroke}%
\pgfsetdash{}{0pt}%
\pgfsys@defobject{currentmarker}{\pgfqpoint{0.000000in}{-0.027778in}}{\pgfqpoint{0.000000in}{0.000000in}}{%
\pgfpathmoveto{\pgfqpoint{0.000000in}{0.000000in}}%
\pgfpathlineto{\pgfqpoint{0.000000in}{-0.027778in}}%
\pgfusepath{stroke,fill}%
}%
\begin{pgfscope}%
\pgfsys@transformshift{4.524282in}{0.549073in}%
\pgfsys@useobject{currentmarker}{}%
\end{pgfscope}%
\end{pgfscope}%
\begin{pgfscope}%
\pgfsetbuttcap%
\pgfsetroundjoin%
\definecolor{currentfill}{rgb}{0.000000,0.000000,0.000000}%
\pgfsetfillcolor{currentfill}%
\pgfsetlinewidth{0.602250pt}%
\definecolor{currentstroke}{rgb}{0.000000,0.000000,0.000000}%
\pgfsetstrokecolor{currentstroke}%
\pgfsetdash{}{0pt}%
\pgfsys@defobject{currentmarker}{\pgfqpoint{0.000000in}{-0.027778in}}{\pgfqpoint{0.000000in}{0.000000in}}{%
\pgfpathmoveto{\pgfqpoint{0.000000in}{0.000000in}}%
\pgfpathlineto{\pgfqpoint{0.000000in}{-0.027778in}}%
\pgfusepath{stroke,fill}%
}%
\begin{pgfscope}%
\pgfsys@transformshift{4.691642in}{0.549073in}%
\pgfsys@useobject{currentmarker}{}%
\end{pgfscope}%
\end{pgfscope}%
\begin{pgfscope}%
\pgfsetbuttcap%
\pgfsetroundjoin%
\definecolor{currentfill}{rgb}{0.000000,0.000000,0.000000}%
\pgfsetfillcolor{currentfill}%
\pgfsetlinewidth{0.602250pt}%
\definecolor{currentstroke}{rgb}{0.000000,0.000000,0.000000}%
\pgfsetstrokecolor{currentstroke}%
\pgfsetdash{}{0pt}%
\pgfsys@defobject{currentmarker}{\pgfqpoint{0.000000in}{-0.027778in}}{\pgfqpoint{0.000000in}{0.000000in}}{%
\pgfpathmoveto{\pgfqpoint{0.000000in}{0.000000in}}%
\pgfpathlineto{\pgfqpoint{0.000000in}{-0.027778in}}%
\pgfusepath{stroke,fill}%
}%
\begin{pgfscope}%
\pgfsys@transformshift{4.833143in}{0.549073in}%
\pgfsys@useobject{currentmarker}{}%
\end{pgfscope}%
\end{pgfscope}%
\begin{pgfscope}%
\pgfsetbuttcap%
\pgfsetroundjoin%
\definecolor{currentfill}{rgb}{0.000000,0.000000,0.000000}%
\pgfsetfillcolor{currentfill}%
\pgfsetlinewidth{0.602250pt}%
\definecolor{currentstroke}{rgb}{0.000000,0.000000,0.000000}%
\pgfsetstrokecolor{currentstroke}%
\pgfsetdash{}{0pt}%
\pgfsys@defobject{currentmarker}{\pgfqpoint{0.000000in}{-0.027778in}}{\pgfqpoint{0.000000in}{0.000000in}}{%
\pgfpathmoveto{\pgfqpoint{0.000000in}{0.000000in}}%
\pgfpathlineto{\pgfqpoint{0.000000in}{-0.027778in}}%
\pgfusepath{stroke,fill}%
}%
\begin{pgfscope}%
\pgfsys@transformshift{4.955717in}{0.549073in}%
\pgfsys@useobject{currentmarker}{}%
\end{pgfscope}%
\end{pgfscope}%
\begin{pgfscope}%
\pgfsetbuttcap%
\pgfsetroundjoin%
\definecolor{currentfill}{rgb}{0.000000,0.000000,0.000000}%
\pgfsetfillcolor{currentfill}%
\pgfsetlinewidth{0.602250pt}%
\definecolor{currentstroke}{rgb}{0.000000,0.000000,0.000000}%
\pgfsetstrokecolor{currentstroke}%
\pgfsetdash{}{0pt}%
\pgfsys@defobject{currentmarker}{\pgfqpoint{0.000000in}{-0.027778in}}{\pgfqpoint{0.000000in}{0.000000in}}{%
\pgfpathmoveto{\pgfqpoint{0.000000in}{0.000000in}}%
\pgfpathlineto{\pgfqpoint{0.000000in}{-0.027778in}}%
\pgfusepath{stroke,fill}%
}%
\begin{pgfscope}%
\pgfsys@transformshift{5.063835in}{0.549073in}%
\pgfsys@useobject{currentmarker}{}%
\end{pgfscope}%
\end{pgfscope}%
\begin{pgfscope}%
\definecolor{textcolor}{rgb}{0.000000,0.000000,0.000000}%
\pgfsetstrokecolor{textcolor}%
\pgfsetfillcolor{textcolor}%
\pgftext[x=3.046913in,y=0.248148in,,top]{\color{textcolor}{\rmfamily\fontsize{12.000000}{14.400000}\selectfont\catcode`\^=\active\def^{\ifmmode\sp\else\^{}\fi}\catcode`\%=\active\def%{\%}$\sigma$}}%
\end{pgfscope}%
\begin{pgfscope}%
\pgfsetbuttcap%
\pgfsetroundjoin%
\definecolor{currentfill}{rgb}{0.000000,0.000000,0.000000}%
\pgfsetfillcolor{currentfill}%
\pgfsetlinewidth{0.803000pt}%
\definecolor{currentstroke}{rgb}{0.000000,0.000000,0.000000}%
\pgfsetstrokecolor{currentstroke}%
\pgfsetdash{}{0pt}%
\pgfsys@defobject{currentmarker}{\pgfqpoint{-0.048611in}{0.000000in}}{\pgfqpoint{-0.000000in}{0.000000in}}{%
\pgfpathmoveto{\pgfqpoint{-0.000000in}{0.000000in}}%
\pgfpathlineto{\pgfqpoint{-0.048611in}{0.000000in}}%
\pgfusepath{stroke,fill}%
}%
\begin{pgfscope}%
\pgfsys@transformshift{0.721913in}{0.986757in}%
\pgfsys@useobject{currentmarker}{}%
\end{pgfscope}%
\end{pgfscope}%
\begin{pgfscope}%
\definecolor{textcolor}{rgb}{0.000000,0.000000,0.000000}%
\pgfsetstrokecolor{textcolor}%
\pgfsetfillcolor{textcolor}%
\pgftext[x=0.303703in, y=0.928887in, left, base]{\color{textcolor}{\rmfamily\fontsize{12.000000}{14.400000}\selectfont\catcode`\^=\active\def^{\ifmmode\sp\else\^{}\fi}\catcode`\%=\active\def%{\%}$\mathdefault{10^{-6}}$}}%
\end{pgfscope}%
\begin{pgfscope}%
\pgfsetbuttcap%
\pgfsetroundjoin%
\definecolor{currentfill}{rgb}{0.000000,0.000000,0.000000}%
\pgfsetfillcolor{currentfill}%
\pgfsetlinewidth{0.803000pt}%
\definecolor{currentstroke}{rgb}{0.000000,0.000000,0.000000}%
\pgfsetstrokecolor{currentstroke}%
\pgfsetdash{}{0pt}%
\pgfsys@defobject{currentmarker}{\pgfqpoint{-0.048611in}{0.000000in}}{\pgfqpoint{-0.000000in}{0.000000in}}{%
\pgfpathmoveto{\pgfqpoint{-0.000000in}{0.000000in}}%
\pgfpathlineto{\pgfqpoint{-0.048611in}{0.000000in}}%
\pgfusepath{stroke,fill}%
}%
\begin{pgfscope}%
\pgfsys@transformshift{0.721913in}{1.611908in}%
\pgfsys@useobject{currentmarker}{}%
\end{pgfscope}%
\end{pgfscope}%
\begin{pgfscope}%
\definecolor{textcolor}{rgb}{0.000000,0.000000,0.000000}%
\pgfsetstrokecolor{textcolor}%
\pgfsetfillcolor{textcolor}%
\pgftext[x=0.303703in, y=1.554038in, left, base]{\color{textcolor}{\rmfamily\fontsize{12.000000}{14.400000}\selectfont\catcode`\^=\active\def^{\ifmmode\sp\else\^{}\fi}\catcode`\%=\active\def%{\%}$\mathdefault{10^{-4}}$}}%
\end{pgfscope}%
\begin{pgfscope}%
\pgfsetbuttcap%
\pgfsetroundjoin%
\definecolor{currentfill}{rgb}{0.000000,0.000000,0.000000}%
\pgfsetfillcolor{currentfill}%
\pgfsetlinewidth{0.803000pt}%
\definecolor{currentstroke}{rgb}{0.000000,0.000000,0.000000}%
\pgfsetstrokecolor{currentstroke}%
\pgfsetdash{}{0pt}%
\pgfsys@defobject{currentmarker}{\pgfqpoint{-0.048611in}{0.000000in}}{\pgfqpoint{-0.000000in}{0.000000in}}{%
\pgfpathmoveto{\pgfqpoint{-0.000000in}{0.000000in}}%
\pgfpathlineto{\pgfqpoint{-0.048611in}{0.000000in}}%
\pgfusepath{stroke,fill}%
}%
\begin{pgfscope}%
\pgfsys@transformshift{0.721913in}{2.237059in}%
\pgfsys@useobject{currentmarker}{}%
\end{pgfscope}%
\end{pgfscope}%
\begin{pgfscope}%
\definecolor{textcolor}{rgb}{0.000000,0.000000,0.000000}%
\pgfsetstrokecolor{textcolor}%
\pgfsetfillcolor{textcolor}%
\pgftext[x=0.303703in, y=2.179189in, left, base]{\color{textcolor}{\rmfamily\fontsize{12.000000}{14.400000}\selectfont\catcode`\^=\active\def^{\ifmmode\sp\else\^{}\fi}\catcode`\%=\active\def%{\%}$\mathdefault{10^{-2}}$}}%
\end{pgfscope}%
\begin{pgfscope}%
\definecolor{textcolor}{rgb}{0.000000,0.000000,0.000000}%
\pgfsetstrokecolor{textcolor}%
\pgfsetfillcolor{textcolor}%
\pgftext[x=0.248148in,y=1.704073in,,bottom,rotate=90.000000]{\color{textcolor}{\rmfamily\fontsize{12.000000}{14.400000}\selectfont\catcode`\^=\active\def^{\ifmmode\sp\else\^{}\fi}\catcode`\%=\active\def%{\%}$L^1$ relative error}}%
\end{pgfscope}%
\begin{pgfscope}%
\pgfpathrectangle{\pgfqpoint{0.721913in}{0.549073in}}{\pgfqpoint{4.650000in}{2.310000in}}%
\pgfusepath{clip}%
\pgfsetrectcap%
\pgfsetroundjoin%
\pgfsetlinewidth{1.003750pt}%
\definecolor{currentstroke}{rgb}{0.001462,0.000466,0.013866}%
\pgfsetstrokecolor{currentstroke}%
\pgfsetdash{}{0pt}%
\pgfpathmoveto{\pgfqpoint{0.933277in}{2.479602in}}%
\pgfpathlineto{\pgfqpoint{1.402974in}{2.462452in}}%
\pgfpathlineto{\pgfqpoint{1.872671in}{2.439960in}}%
\pgfpathlineto{\pgfqpoint{2.342368in}{2.416520in}}%
\pgfpathlineto{\pgfqpoint{2.812065in}{2.384949in}}%
\pgfpathlineto{\pgfqpoint{3.281762in}{2.368573in}}%
\pgfpathlineto{\pgfqpoint{3.751459in}{2.357554in}}%
\pgfpathlineto{\pgfqpoint{4.221156in}{2.341278in}}%
\pgfpathlineto{\pgfqpoint{4.690853in}{2.315259in}}%
\pgfpathlineto{\pgfqpoint{5.160550in}{2.277270in}}%
\pgfusepath{stroke}%
\end{pgfscope}%
\begin{pgfscope}%
\pgfpathrectangle{\pgfqpoint{0.721913in}{0.549073in}}{\pgfqpoint{4.650000in}{2.310000in}}%
\pgfusepath{clip}%
\pgfsetbuttcap%
\pgfsetroundjoin%
\definecolor{currentfill}{rgb}{0.001462,0.000466,0.013866}%
\pgfsetfillcolor{currentfill}%
\pgfsetlinewidth{1.003750pt}%
\definecolor{currentstroke}{rgb}{0.001462,0.000466,0.013866}%
\pgfsetstrokecolor{currentstroke}%
\pgfsetdash{}{0pt}%
\pgfsys@defobject{currentmarker}{\pgfqpoint{-0.020833in}{-0.020833in}}{\pgfqpoint{0.020833in}{0.020833in}}{%
\pgfpathmoveto{\pgfqpoint{0.000000in}{-0.020833in}}%
\pgfpathcurveto{\pgfqpoint{0.005525in}{-0.020833in}}{\pgfqpoint{0.010825in}{-0.018638in}}{\pgfqpoint{0.014731in}{-0.014731in}}%
\pgfpathcurveto{\pgfqpoint{0.018638in}{-0.010825in}}{\pgfqpoint{0.020833in}{-0.005525in}}{\pgfqpoint{0.020833in}{0.000000in}}%
\pgfpathcurveto{\pgfqpoint{0.020833in}{0.005525in}}{\pgfqpoint{0.018638in}{0.010825in}}{\pgfqpoint{0.014731in}{0.014731in}}%
\pgfpathcurveto{\pgfqpoint{0.010825in}{0.018638in}}{\pgfqpoint{0.005525in}{0.020833in}}{\pgfqpoint{0.000000in}{0.020833in}}%
\pgfpathcurveto{\pgfqpoint{-0.005525in}{0.020833in}}{\pgfqpoint{-0.010825in}{0.018638in}}{\pgfqpoint{-0.014731in}{0.014731in}}%
\pgfpathcurveto{\pgfqpoint{-0.018638in}{0.010825in}}{\pgfqpoint{-0.020833in}{0.005525in}}{\pgfqpoint{-0.020833in}{0.000000in}}%
\pgfpathcurveto{\pgfqpoint{-0.020833in}{-0.005525in}}{\pgfqpoint{-0.018638in}{-0.010825in}}{\pgfqpoint{-0.014731in}{-0.014731in}}%
\pgfpathcurveto{\pgfqpoint{-0.010825in}{-0.018638in}}{\pgfqpoint{-0.005525in}{-0.020833in}}{\pgfqpoint{0.000000in}{-0.020833in}}%
\pgfpathlineto{\pgfqpoint{0.000000in}{-0.020833in}}%
\pgfpathclose%
\pgfusepath{stroke,fill}%
}%
\begin{pgfscope}%
\pgfsys@transformshift{0.933277in}{2.479602in}%
\pgfsys@useobject{currentmarker}{}%
\end{pgfscope}%
\begin{pgfscope}%
\pgfsys@transformshift{1.402974in}{2.462452in}%
\pgfsys@useobject{currentmarker}{}%
\end{pgfscope}%
\begin{pgfscope}%
\pgfsys@transformshift{1.872671in}{2.439960in}%
\pgfsys@useobject{currentmarker}{}%
\end{pgfscope}%
\begin{pgfscope}%
\pgfsys@transformshift{2.342368in}{2.416520in}%
\pgfsys@useobject{currentmarker}{}%
\end{pgfscope}%
\begin{pgfscope}%
\pgfsys@transformshift{2.812065in}{2.384949in}%
\pgfsys@useobject{currentmarker}{}%
\end{pgfscope}%
\begin{pgfscope}%
\pgfsys@transformshift{3.281762in}{2.368573in}%
\pgfsys@useobject{currentmarker}{}%
\end{pgfscope}%
\begin{pgfscope}%
\pgfsys@transformshift{3.751459in}{2.357554in}%
\pgfsys@useobject{currentmarker}{}%
\end{pgfscope}%
\begin{pgfscope}%
\pgfsys@transformshift{4.221156in}{2.341278in}%
\pgfsys@useobject{currentmarker}{}%
\end{pgfscope}%
\begin{pgfscope}%
\pgfsys@transformshift{4.690853in}{2.315259in}%
\pgfsys@useobject{currentmarker}{}%
\end{pgfscope}%
\begin{pgfscope}%
\pgfsys@transformshift{5.160550in}{2.277270in}%
\pgfsys@useobject{currentmarker}{}%
\end{pgfscope}%
\end{pgfscope}%
\begin{pgfscope}%
\pgfpathrectangle{\pgfqpoint{0.721913in}{0.549073in}}{\pgfqpoint{4.650000in}{2.310000in}}%
\pgfusepath{clip}%
\pgfsetrectcap%
\pgfsetroundjoin%
\pgfsetlinewidth{1.003750pt}%
\definecolor{currentstroke}{rgb}{0.232077,0.059889,0.437695}%
\pgfsetstrokecolor{currentstroke}%
\pgfsetdash{}{0pt}%
\pgfpathmoveto{\pgfqpoint{0.933277in}{1.755814in}}%
\pgfpathlineto{\pgfqpoint{1.402974in}{2.007777in}}%
\pgfpathlineto{\pgfqpoint{1.872671in}{2.063382in}}%
\pgfpathlineto{\pgfqpoint{2.342368in}{2.137352in}}%
\pgfpathlineto{\pgfqpoint{2.812065in}{2.217118in}}%
\pgfpathlineto{\pgfqpoint{3.281762in}{2.251435in}}%
\pgfpathlineto{\pgfqpoint{3.751459in}{2.292012in}}%
\pgfpathlineto{\pgfqpoint{4.221156in}{2.298256in}}%
\pgfpathlineto{\pgfqpoint{4.690853in}{2.289837in}}%
\pgfpathlineto{\pgfqpoint{5.160550in}{2.265716in}}%
\pgfusepath{stroke}%
\end{pgfscope}%
\begin{pgfscope}%
\pgfpathrectangle{\pgfqpoint{0.721913in}{0.549073in}}{\pgfqpoint{4.650000in}{2.310000in}}%
\pgfusepath{clip}%
\pgfsetbuttcap%
\pgfsetroundjoin%
\definecolor{currentfill}{rgb}{0.232077,0.059889,0.437695}%
\pgfsetfillcolor{currentfill}%
\pgfsetlinewidth{1.003750pt}%
\definecolor{currentstroke}{rgb}{0.232077,0.059889,0.437695}%
\pgfsetstrokecolor{currentstroke}%
\pgfsetdash{}{0pt}%
\pgfsys@defobject{currentmarker}{\pgfqpoint{-0.020833in}{-0.020833in}}{\pgfqpoint{0.020833in}{0.020833in}}{%
\pgfpathmoveto{\pgfqpoint{0.000000in}{-0.020833in}}%
\pgfpathcurveto{\pgfqpoint{0.005525in}{-0.020833in}}{\pgfqpoint{0.010825in}{-0.018638in}}{\pgfqpoint{0.014731in}{-0.014731in}}%
\pgfpathcurveto{\pgfqpoint{0.018638in}{-0.010825in}}{\pgfqpoint{0.020833in}{-0.005525in}}{\pgfqpoint{0.020833in}{0.000000in}}%
\pgfpathcurveto{\pgfqpoint{0.020833in}{0.005525in}}{\pgfqpoint{0.018638in}{0.010825in}}{\pgfqpoint{0.014731in}{0.014731in}}%
\pgfpathcurveto{\pgfqpoint{0.010825in}{0.018638in}}{\pgfqpoint{0.005525in}{0.020833in}}{\pgfqpoint{0.000000in}{0.020833in}}%
\pgfpathcurveto{\pgfqpoint{-0.005525in}{0.020833in}}{\pgfqpoint{-0.010825in}{0.018638in}}{\pgfqpoint{-0.014731in}{0.014731in}}%
\pgfpathcurveto{\pgfqpoint{-0.018638in}{0.010825in}}{\pgfqpoint{-0.020833in}{0.005525in}}{\pgfqpoint{-0.020833in}{0.000000in}}%
\pgfpathcurveto{\pgfqpoint{-0.020833in}{-0.005525in}}{\pgfqpoint{-0.018638in}{-0.010825in}}{\pgfqpoint{-0.014731in}{-0.014731in}}%
\pgfpathcurveto{\pgfqpoint{-0.010825in}{-0.018638in}}{\pgfqpoint{-0.005525in}{-0.020833in}}{\pgfqpoint{0.000000in}{-0.020833in}}%
\pgfpathlineto{\pgfqpoint{0.000000in}{-0.020833in}}%
\pgfpathclose%
\pgfusepath{stroke,fill}%
}%
\begin{pgfscope}%
\pgfsys@transformshift{0.933277in}{1.755814in}%
\pgfsys@useobject{currentmarker}{}%
\end{pgfscope}%
\begin{pgfscope}%
\pgfsys@transformshift{1.402974in}{2.007777in}%
\pgfsys@useobject{currentmarker}{}%
\end{pgfscope}%
\begin{pgfscope}%
\pgfsys@transformshift{1.872671in}{2.063382in}%
\pgfsys@useobject{currentmarker}{}%
\end{pgfscope}%
\begin{pgfscope}%
\pgfsys@transformshift{2.342368in}{2.137352in}%
\pgfsys@useobject{currentmarker}{}%
\end{pgfscope}%
\begin{pgfscope}%
\pgfsys@transformshift{2.812065in}{2.217118in}%
\pgfsys@useobject{currentmarker}{}%
\end{pgfscope}%
\begin{pgfscope}%
\pgfsys@transformshift{3.281762in}{2.251435in}%
\pgfsys@useobject{currentmarker}{}%
\end{pgfscope}%
\begin{pgfscope}%
\pgfsys@transformshift{3.751459in}{2.292012in}%
\pgfsys@useobject{currentmarker}{}%
\end{pgfscope}%
\begin{pgfscope}%
\pgfsys@transformshift{4.221156in}{2.298256in}%
\pgfsys@useobject{currentmarker}{}%
\end{pgfscope}%
\begin{pgfscope}%
\pgfsys@transformshift{4.690853in}{2.289837in}%
\pgfsys@useobject{currentmarker}{}%
\end{pgfscope}%
\begin{pgfscope}%
\pgfsys@transformshift{5.160550in}{2.265716in}%
\pgfsys@useobject{currentmarker}{}%
\end{pgfscope}%
\end{pgfscope}%
\begin{pgfscope}%
\pgfpathrectangle{\pgfqpoint{0.721913in}{0.549073in}}{\pgfqpoint{4.650000in}{2.310000in}}%
\pgfusepath{clip}%
\pgfsetrectcap%
\pgfsetroundjoin%
\pgfsetlinewidth{1.003750pt}%
\definecolor{currentstroke}{rgb}{0.550287,0.161158,0.505719}%
\pgfsetstrokecolor{currentstroke}%
\pgfsetdash{}{0pt}%
\pgfpathmoveto{\pgfqpoint{0.933277in}{0.783267in}}%
\pgfpathlineto{\pgfqpoint{1.402974in}{0.935624in}}%
\pgfpathlineto{\pgfqpoint{1.872671in}{1.662346in}}%
\pgfpathlineto{\pgfqpoint{2.342368in}{2.085882in}}%
\pgfpathlineto{\pgfqpoint{2.812065in}{2.187484in}}%
\pgfpathlineto{\pgfqpoint{3.281762in}{2.171665in}}%
\pgfpathlineto{\pgfqpoint{3.751459in}{2.189342in}}%
\pgfpathlineto{\pgfqpoint{4.221156in}{2.214327in}}%
\pgfpathlineto{\pgfqpoint{4.690853in}{2.238628in}}%
\pgfpathlineto{\pgfqpoint{5.160550in}{2.256829in}}%
\pgfusepath{stroke}%
\end{pgfscope}%
\begin{pgfscope}%
\pgfpathrectangle{\pgfqpoint{0.721913in}{0.549073in}}{\pgfqpoint{4.650000in}{2.310000in}}%
\pgfusepath{clip}%
\pgfsetbuttcap%
\pgfsetroundjoin%
\definecolor{currentfill}{rgb}{0.550287,0.161158,0.505719}%
\pgfsetfillcolor{currentfill}%
\pgfsetlinewidth{1.003750pt}%
\definecolor{currentstroke}{rgb}{0.550287,0.161158,0.505719}%
\pgfsetstrokecolor{currentstroke}%
\pgfsetdash{}{0pt}%
\pgfsys@defobject{currentmarker}{\pgfqpoint{-0.020833in}{-0.020833in}}{\pgfqpoint{0.020833in}{0.020833in}}{%
\pgfpathmoveto{\pgfqpoint{0.000000in}{-0.020833in}}%
\pgfpathcurveto{\pgfqpoint{0.005525in}{-0.020833in}}{\pgfqpoint{0.010825in}{-0.018638in}}{\pgfqpoint{0.014731in}{-0.014731in}}%
\pgfpathcurveto{\pgfqpoint{0.018638in}{-0.010825in}}{\pgfqpoint{0.020833in}{-0.005525in}}{\pgfqpoint{0.020833in}{0.000000in}}%
\pgfpathcurveto{\pgfqpoint{0.020833in}{0.005525in}}{\pgfqpoint{0.018638in}{0.010825in}}{\pgfqpoint{0.014731in}{0.014731in}}%
\pgfpathcurveto{\pgfqpoint{0.010825in}{0.018638in}}{\pgfqpoint{0.005525in}{0.020833in}}{\pgfqpoint{0.000000in}{0.020833in}}%
\pgfpathcurveto{\pgfqpoint{-0.005525in}{0.020833in}}{\pgfqpoint{-0.010825in}{0.018638in}}{\pgfqpoint{-0.014731in}{0.014731in}}%
\pgfpathcurveto{\pgfqpoint{-0.018638in}{0.010825in}}{\pgfqpoint{-0.020833in}{0.005525in}}{\pgfqpoint{-0.020833in}{0.000000in}}%
\pgfpathcurveto{\pgfqpoint{-0.020833in}{-0.005525in}}{\pgfqpoint{-0.018638in}{-0.010825in}}{\pgfqpoint{-0.014731in}{-0.014731in}}%
\pgfpathcurveto{\pgfqpoint{-0.010825in}{-0.018638in}}{\pgfqpoint{-0.005525in}{-0.020833in}}{\pgfqpoint{0.000000in}{-0.020833in}}%
\pgfpathlineto{\pgfqpoint{0.000000in}{-0.020833in}}%
\pgfpathclose%
\pgfusepath{stroke,fill}%
}%
\begin{pgfscope}%
\pgfsys@transformshift{0.933277in}{0.783267in}%
\pgfsys@useobject{currentmarker}{}%
\end{pgfscope}%
\begin{pgfscope}%
\pgfsys@transformshift{1.402974in}{0.935624in}%
\pgfsys@useobject{currentmarker}{}%
\end{pgfscope}%
\begin{pgfscope}%
\pgfsys@transformshift{1.872671in}{1.662346in}%
\pgfsys@useobject{currentmarker}{}%
\end{pgfscope}%
\begin{pgfscope}%
\pgfsys@transformshift{2.342368in}{2.085882in}%
\pgfsys@useobject{currentmarker}{}%
\end{pgfscope}%
\begin{pgfscope}%
\pgfsys@transformshift{2.812065in}{2.187484in}%
\pgfsys@useobject{currentmarker}{}%
\end{pgfscope}%
\begin{pgfscope}%
\pgfsys@transformshift{3.281762in}{2.171665in}%
\pgfsys@useobject{currentmarker}{}%
\end{pgfscope}%
\begin{pgfscope}%
\pgfsys@transformshift{3.751459in}{2.189342in}%
\pgfsys@useobject{currentmarker}{}%
\end{pgfscope}%
\begin{pgfscope}%
\pgfsys@transformshift{4.221156in}{2.214327in}%
\pgfsys@useobject{currentmarker}{}%
\end{pgfscope}%
\begin{pgfscope}%
\pgfsys@transformshift{4.690853in}{2.238628in}%
\pgfsys@useobject{currentmarker}{}%
\end{pgfscope}%
\begin{pgfscope}%
\pgfsys@transformshift{5.160550in}{2.256829in}%
\pgfsys@useobject{currentmarker}{}%
\end{pgfscope}%
\end{pgfscope}%
\begin{pgfscope}%
\pgfpathrectangle{\pgfqpoint{0.721913in}{0.549073in}}{\pgfqpoint{4.650000in}{2.310000in}}%
\pgfusepath{clip}%
\pgfsetrectcap%
\pgfsetroundjoin%
\pgfsetlinewidth{1.003750pt}%
\definecolor{currentstroke}{rgb}{0.868793,0.287728,0.409303}%
\pgfsetstrokecolor{currentstroke}%
\pgfsetdash{}{0pt}%
\pgfpathmoveto{\pgfqpoint{0.933277in}{0.779709in}}%
\pgfpathlineto{\pgfqpoint{1.402974in}{0.654073in}}%
\pgfpathlineto{\pgfqpoint{1.872671in}{0.879340in}}%
\pgfpathlineto{\pgfqpoint{2.342368in}{1.670011in}}%
\pgfpathlineto{\pgfqpoint{2.812065in}{1.987993in}}%
\pgfpathlineto{\pgfqpoint{3.281762in}{2.048681in}}%
\pgfpathlineto{\pgfqpoint{3.751459in}{2.083093in}}%
\pgfpathlineto{\pgfqpoint{4.221156in}{2.165361in}}%
\pgfpathlineto{\pgfqpoint{4.690853in}{2.233862in}}%
\pgfpathlineto{\pgfqpoint{5.160550in}{2.251427in}}%
\pgfusepath{stroke}%
\end{pgfscope}%
\begin{pgfscope}%
\pgfpathrectangle{\pgfqpoint{0.721913in}{0.549073in}}{\pgfqpoint{4.650000in}{2.310000in}}%
\pgfusepath{clip}%
\pgfsetbuttcap%
\pgfsetroundjoin%
\definecolor{currentfill}{rgb}{0.868793,0.287728,0.409303}%
\pgfsetfillcolor{currentfill}%
\pgfsetlinewidth{1.003750pt}%
\definecolor{currentstroke}{rgb}{0.868793,0.287728,0.409303}%
\pgfsetstrokecolor{currentstroke}%
\pgfsetdash{}{0pt}%
\pgfsys@defobject{currentmarker}{\pgfqpoint{-0.020833in}{-0.020833in}}{\pgfqpoint{0.020833in}{0.020833in}}{%
\pgfpathmoveto{\pgfqpoint{0.000000in}{-0.020833in}}%
\pgfpathcurveto{\pgfqpoint{0.005525in}{-0.020833in}}{\pgfqpoint{0.010825in}{-0.018638in}}{\pgfqpoint{0.014731in}{-0.014731in}}%
\pgfpathcurveto{\pgfqpoint{0.018638in}{-0.010825in}}{\pgfqpoint{0.020833in}{-0.005525in}}{\pgfqpoint{0.020833in}{0.000000in}}%
\pgfpathcurveto{\pgfqpoint{0.020833in}{0.005525in}}{\pgfqpoint{0.018638in}{0.010825in}}{\pgfqpoint{0.014731in}{0.014731in}}%
\pgfpathcurveto{\pgfqpoint{0.010825in}{0.018638in}}{\pgfqpoint{0.005525in}{0.020833in}}{\pgfqpoint{0.000000in}{0.020833in}}%
\pgfpathcurveto{\pgfqpoint{-0.005525in}{0.020833in}}{\pgfqpoint{-0.010825in}{0.018638in}}{\pgfqpoint{-0.014731in}{0.014731in}}%
\pgfpathcurveto{\pgfqpoint{-0.018638in}{0.010825in}}{\pgfqpoint{-0.020833in}{0.005525in}}{\pgfqpoint{-0.020833in}{0.000000in}}%
\pgfpathcurveto{\pgfqpoint{-0.020833in}{-0.005525in}}{\pgfqpoint{-0.018638in}{-0.010825in}}{\pgfqpoint{-0.014731in}{-0.014731in}}%
\pgfpathcurveto{\pgfqpoint{-0.010825in}{-0.018638in}}{\pgfqpoint{-0.005525in}{-0.020833in}}{\pgfqpoint{0.000000in}{-0.020833in}}%
\pgfpathlineto{\pgfqpoint{0.000000in}{-0.020833in}}%
\pgfpathclose%
\pgfusepath{stroke,fill}%
}%
\begin{pgfscope}%
\pgfsys@transformshift{0.933277in}{0.779709in}%
\pgfsys@useobject{currentmarker}{}%
\end{pgfscope}%
\begin{pgfscope}%
\pgfsys@transformshift{1.402974in}{0.654073in}%
\pgfsys@useobject{currentmarker}{}%
\end{pgfscope}%
\begin{pgfscope}%
\pgfsys@transformshift{1.872671in}{0.879340in}%
\pgfsys@useobject{currentmarker}{}%
\end{pgfscope}%
\begin{pgfscope}%
\pgfsys@transformshift{2.342368in}{1.670011in}%
\pgfsys@useobject{currentmarker}{}%
\end{pgfscope}%
\begin{pgfscope}%
\pgfsys@transformshift{2.812065in}{1.987993in}%
\pgfsys@useobject{currentmarker}{}%
\end{pgfscope}%
\begin{pgfscope}%
\pgfsys@transformshift{3.281762in}{2.048681in}%
\pgfsys@useobject{currentmarker}{}%
\end{pgfscope}%
\begin{pgfscope}%
\pgfsys@transformshift{3.751459in}{2.083093in}%
\pgfsys@useobject{currentmarker}{}%
\end{pgfscope}%
\begin{pgfscope}%
\pgfsys@transformshift{4.221156in}{2.165361in}%
\pgfsys@useobject{currentmarker}{}%
\end{pgfscope}%
\begin{pgfscope}%
\pgfsys@transformshift{4.690853in}{2.233862in}%
\pgfsys@useobject{currentmarker}{}%
\end{pgfscope}%
\begin{pgfscope}%
\pgfsys@transformshift{5.160550in}{2.251427in}%
\pgfsys@useobject{currentmarker}{}%
\end{pgfscope}%
\end{pgfscope}%
\begin{pgfscope}%
\pgfpathrectangle{\pgfqpoint{0.721913in}{0.549073in}}{\pgfqpoint{4.650000in}{2.310000in}}%
\pgfusepath{clip}%
\pgfsetrectcap%
\pgfsetroundjoin%
\pgfsetlinewidth{1.003750pt}%
\definecolor{currentstroke}{rgb}{0.994738,0.624350,0.427397}%
\pgfsetstrokecolor{currentstroke}%
\pgfsetdash{}{0pt}%
\pgfpathmoveto{\pgfqpoint{0.933277in}{0.785874in}}%
\pgfpathlineto{\pgfqpoint{1.402974in}{0.678109in}}%
\pgfpathlineto{\pgfqpoint{1.872671in}{0.689838in}}%
\pgfpathlineto{\pgfqpoint{2.342368in}{1.059930in}}%
\pgfpathlineto{\pgfqpoint{2.812065in}{1.933786in}}%
\pgfpathlineto{\pgfqpoint{3.281762in}{2.271944in}}%
\pgfpathlineto{\pgfqpoint{3.751459in}{2.428285in}}%
\pgfpathlineto{\pgfqpoint{4.221156in}{2.569889in}}%
\pgfpathlineto{\pgfqpoint{4.690853in}{2.682484in}}%
\pgfpathlineto{\pgfqpoint{5.160550in}{2.754073in}}%
\pgfusepath{stroke}%
\end{pgfscope}%
\begin{pgfscope}%
\pgfpathrectangle{\pgfqpoint{0.721913in}{0.549073in}}{\pgfqpoint{4.650000in}{2.310000in}}%
\pgfusepath{clip}%
\pgfsetbuttcap%
\pgfsetroundjoin%
\definecolor{currentfill}{rgb}{0.994738,0.624350,0.427397}%
\pgfsetfillcolor{currentfill}%
\pgfsetlinewidth{1.003750pt}%
\definecolor{currentstroke}{rgb}{0.994738,0.624350,0.427397}%
\pgfsetstrokecolor{currentstroke}%
\pgfsetdash{}{0pt}%
\pgfsys@defobject{currentmarker}{\pgfqpoint{-0.020833in}{-0.020833in}}{\pgfqpoint{0.020833in}{0.020833in}}{%
\pgfpathmoveto{\pgfqpoint{0.000000in}{-0.020833in}}%
\pgfpathcurveto{\pgfqpoint{0.005525in}{-0.020833in}}{\pgfqpoint{0.010825in}{-0.018638in}}{\pgfqpoint{0.014731in}{-0.014731in}}%
\pgfpathcurveto{\pgfqpoint{0.018638in}{-0.010825in}}{\pgfqpoint{0.020833in}{-0.005525in}}{\pgfqpoint{0.020833in}{0.000000in}}%
\pgfpathcurveto{\pgfqpoint{0.020833in}{0.005525in}}{\pgfqpoint{0.018638in}{0.010825in}}{\pgfqpoint{0.014731in}{0.014731in}}%
\pgfpathcurveto{\pgfqpoint{0.010825in}{0.018638in}}{\pgfqpoint{0.005525in}{0.020833in}}{\pgfqpoint{0.000000in}{0.020833in}}%
\pgfpathcurveto{\pgfqpoint{-0.005525in}{0.020833in}}{\pgfqpoint{-0.010825in}{0.018638in}}{\pgfqpoint{-0.014731in}{0.014731in}}%
\pgfpathcurveto{\pgfqpoint{-0.018638in}{0.010825in}}{\pgfqpoint{-0.020833in}{0.005525in}}{\pgfqpoint{-0.020833in}{0.000000in}}%
\pgfpathcurveto{\pgfqpoint{-0.020833in}{-0.005525in}}{\pgfqpoint{-0.018638in}{-0.010825in}}{\pgfqpoint{-0.014731in}{-0.014731in}}%
\pgfpathcurveto{\pgfqpoint{-0.010825in}{-0.018638in}}{\pgfqpoint{-0.005525in}{-0.020833in}}{\pgfqpoint{0.000000in}{-0.020833in}}%
\pgfpathlineto{\pgfqpoint{0.000000in}{-0.020833in}}%
\pgfpathclose%
\pgfusepath{stroke,fill}%
}%
\begin{pgfscope}%
\pgfsys@transformshift{0.933277in}{0.785874in}%
\pgfsys@useobject{currentmarker}{}%
\end{pgfscope}%
\begin{pgfscope}%
\pgfsys@transformshift{1.402974in}{0.678109in}%
\pgfsys@useobject{currentmarker}{}%
\end{pgfscope}%
\begin{pgfscope}%
\pgfsys@transformshift{1.872671in}{0.689838in}%
\pgfsys@useobject{currentmarker}{}%
\end{pgfscope}%
\begin{pgfscope}%
\pgfsys@transformshift{2.342368in}{1.059930in}%
\pgfsys@useobject{currentmarker}{}%
\end{pgfscope}%
\begin{pgfscope}%
\pgfsys@transformshift{2.812065in}{1.933786in}%
\pgfsys@useobject{currentmarker}{}%
\end{pgfscope}%
\begin{pgfscope}%
\pgfsys@transformshift{3.281762in}{2.271944in}%
\pgfsys@useobject{currentmarker}{}%
\end{pgfscope}%
\begin{pgfscope}%
\pgfsys@transformshift{3.751459in}{2.428285in}%
\pgfsys@useobject{currentmarker}{}%
\end{pgfscope}%
\begin{pgfscope}%
\pgfsys@transformshift{4.221156in}{2.569889in}%
\pgfsys@useobject{currentmarker}{}%
\end{pgfscope}%
\begin{pgfscope}%
\pgfsys@transformshift{4.690853in}{2.682484in}%
\pgfsys@useobject{currentmarker}{}%
\end{pgfscope}%
\begin{pgfscope}%
\pgfsys@transformshift{5.160550in}{2.754073in}%
\pgfsys@useobject{currentmarker}{}%
\end{pgfscope}%
\end{pgfscope}%
\begin{pgfscope}%
\pgfsetrectcap%
\pgfsetmiterjoin%
\pgfsetlinewidth{0.803000pt}%
\definecolor{currentstroke}{rgb}{0.000000,0.000000,0.000000}%
\pgfsetstrokecolor{currentstroke}%
\pgfsetdash{}{0pt}%
\pgfpathmoveto{\pgfqpoint{0.721913in}{0.549073in}}%
\pgfpathlineto{\pgfqpoint{0.721913in}{2.859073in}}%
\pgfusepath{stroke}%
\end{pgfscope}%
\begin{pgfscope}%
\pgfsetrectcap%
\pgfsetmiterjoin%
\pgfsetlinewidth{0.803000pt}%
\definecolor{currentstroke}{rgb}{0.000000,0.000000,0.000000}%
\pgfsetstrokecolor{currentstroke}%
\pgfsetdash{}{0pt}%
\pgfpathmoveto{\pgfqpoint{5.371913in}{0.549073in}}%
\pgfpathlineto{\pgfqpoint{5.371913in}{2.859073in}}%
\pgfusepath{stroke}%
\end{pgfscope}%
\begin{pgfscope}%
\pgfsetrectcap%
\pgfsetmiterjoin%
\pgfsetlinewidth{0.803000pt}%
\definecolor{currentstroke}{rgb}{0.000000,0.000000,0.000000}%
\pgfsetstrokecolor{currentstroke}%
\pgfsetdash{}{0pt}%
\pgfpathmoveto{\pgfqpoint{0.721913in}{0.549073in}}%
\pgfpathlineto{\pgfqpoint{5.371913in}{0.549073in}}%
\pgfusepath{stroke}%
\end{pgfscope}%
\begin{pgfscope}%
\pgfsetrectcap%
\pgfsetmiterjoin%
\pgfsetlinewidth{0.803000pt}%
\definecolor{currentstroke}{rgb}{0.000000,0.000000,0.000000}%
\pgfsetstrokecolor{currentstroke}%
\pgfsetdash{}{0pt}%
\pgfpathmoveto{\pgfqpoint{0.721913in}{2.859073in}}%
\pgfpathlineto{\pgfqpoint{5.371913in}{2.859073in}}%
\pgfusepath{stroke}%
\end{pgfscope}%
\begin{pgfscope}%
\pgfsetbuttcap%
\pgfsetmiterjoin%
\definecolor{currentfill}{rgb}{1.000000,1.000000,1.000000}%
\pgfsetfillcolor{currentfill}%
\pgfsetfillopacity{0.800000}%
\pgfsetlinewidth{1.003750pt}%
\definecolor{currentstroke}{rgb}{0.800000,0.800000,0.800000}%
\pgfsetstrokecolor{currentstroke}%
\pgfsetstrokeopacity{0.800000}%
\pgfsetdash{}{0pt}%
\pgfpathmoveto{\pgfqpoint{2.795580in}{0.632406in}}%
\pgfpathlineto{\pgfqpoint{5.255247in}{0.632406in}}%
\pgfpathquadraticcurveto{\pgfqpoint{5.288580in}{0.632406in}}{\pgfqpoint{5.288580in}{0.665739in}}%
\pgfpathlineto{\pgfqpoint{5.288580in}{1.899073in}}%
\pgfpathquadraticcurveto{\pgfqpoint{5.288580in}{1.932406in}}{\pgfqpoint{5.255247in}{1.932406in}}%
\pgfpathlineto{\pgfqpoint{2.795580in}{1.932406in}}%
\pgfpathquadraticcurveto{\pgfqpoint{2.762246in}{1.932406in}}{\pgfqpoint{2.762246in}{1.899073in}}%
\pgfpathlineto{\pgfqpoint{2.762246in}{0.665739in}}%
\pgfpathquadraticcurveto{\pgfqpoint{2.762246in}{0.632406in}}{\pgfqpoint{2.795580in}{0.632406in}}%
\pgfpathlineto{\pgfqpoint{2.795580in}{0.632406in}}%
\pgfpathclose%
\pgfusepath{stroke,fill}%
\end{pgfscope}%
\begin{pgfscope}%
\pgfsetrectcap%
\pgfsetroundjoin%
\pgfsetlinewidth{1.003750pt}%
\definecolor{currentstroke}{rgb}{0.001462,0.000466,0.013866}%
\pgfsetstrokecolor{currentstroke}%
\pgfsetdash{}{0pt}%
\pgfpathmoveto{\pgfqpoint{2.828913in}{1.799073in}}%
\pgfpathlineto{\pgfqpoint{2.995580in}{1.799073in}}%
\pgfpathlineto{\pgfqpoint{3.162246in}{1.799073in}}%
\pgfusepath{stroke}%
\end{pgfscope}%
\begin{pgfscope}%
\pgfsetbuttcap%
\pgfsetroundjoin%
\definecolor{currentfill}{rgb}{0.001462,0.000466,0.013866}%
\pgfsetfillcolor{currentfill}%
\pgfsetlinewidth{1.003750pt}%
\definecolor{currentstroke}{rgb}{0.001462,0.000466,0.013866}%
\pgfsetstrokecolor{currentstroke}%
\pgfsetdash{}{0pt}%
\pgfsys@defobject{currentmarker}{\pgfqpoint{-0.020833in}{-0.020833in}}{\pgfqpoint{0.020833in}{0.020833in}}{%
\pgfpathmoveto{\pgfqpoint{0.000000in}{-0.020833in}}%
\pgfpathcurveto{\pgfqpoint{0.005525in}{-0.020833in}}{\pgfqpoint{0.010825in}{-0.018638in}}{\pgfqpoint{0.014731in}{-0.014731in}}%
\pgfpathcurveto{\pgfqpoint{0.018638in}{-0.010825in}}{\pgfqpoint{0.020833in}{-0.005525in}}{\pgfqpoint{0.020833in}{0.000000in}}%
\pgfpathcurveto{\pgfqpoint{0.020833in}{0.005525in}}{\pgfqpoint{0.018638in}{0.010825in}}{\pgfqpoint{0.014731in}{0.014731in}}%
\pgfpathcurveto{\pgfqpoint{0.010825in}{0.018638in}}{\pgfqpoint{0.005525in}{0.020833in}}{\pgfqpoint{0.000000in}{0.020833in}}%
\pgfpathcurveto{\pgfqpoint{-0.005525in}{0.020833in}}{\pgfqpoint{-0.010825in}{0.018638in}}{\pgfqpoint{-0.014731in}{0.014731in}}%
\pgfpathcurveto{\pgfqpoint{-0.018638in}{0.010825in}}{\pgfqpoint{-0.020833in}{0.005525in}}{\pgfqpoint{-0.020833in}{0.000000in}}%
\pgfpathcurveto{\pgfqpoint{-0.020833in}{-0.005525in}}{\pgfqpoint{-0.018638in}{-0.010825in}}{\pgfqpoint{-0.014731in}{-0.014731in}}%
\pgfpathcurveto{\pgfqpoint{-0.010825in}{-0.018638in}}{\pgfqpoint{-0.005525in}{-0.020833in}}{\pgfqpoint{0.000000in}{-0.020833in}}%
\pgfpathlineto{\pgfqpoint{0.000000in}{-0.020833in}}%
\pgfpathclose%
\pgfusepath{stroke,fill}%
}%
\begin{pgfscope}%
\pgfsys@transformshift{2.995580in}{1.799073in}%
\pgfsys@useobject{currentmarker}{}%
\end{pgfscope}%
\end{pgfscope}%
\begin{pgfscope}%
\definecolor{textcolor}{rgb}{0.000000,0.000000,0.000000}%
\pgfsetstrokecolor{textcolor}%
\pgfsetfillcolor{textcolor}%
\pgftext[x=3.295580in,y=1.740739in,left,base]{\color{textcolor}{\rmfamily\fontsize{12.000000}{14.400000}\selectfont\catcode`\^=\active\def^{\ifmmode\sp\else\^{}\fi}\catcode`\%=\active\def%{\%}$n_{\Omega}=0$, $n_{\Psi}=80$ (DGC)}}%
\end{pgfscope}%
\begin{pgfscope}%
\pgfsetrectcap%
\pgfsetroundjoin%
\pgfsetlinewidth{1.003750pt}%
\definecolor{currentstroke}{rgb}{0.232077,0.059889,0.437695}%
\pgfsetstrokecolor{currentstroke}%
\pgfsetdash{}{0pt}%
\pgfpathmoveto{\pgfqpoint{2.828913in}{1.549073in}}%
\pgfpathlineto{\pgfqpoint{2.995580in}{1.549073in}}%
\pgfpathlineto{\pgfqpoint{3.162246in}{1.549073in}}%
\pgfusepath{stroke}%
\end{pgfscope}%
\begin{pgfscope}%
\pgfsetbuttcap%
\pgfsetroundjoin%
\definecolor{currentfill}{rgb}{0.232077,0.059889,0.437695}%
\pgfsetfillcolor{currentfill}%
\pgfsetlinewidth{1.003750pt}%
\definecolor{currentstroke}{rgb}{0.232077,0.059889,0.437695}%
\pgfsetstrokecolor{currentstroke}%
\pgfsetdash{}{0pt}%
\pgfsys@defobject{currentmarker}{\pgfqpoint{-0.020833in}{-0.020833in}}{\pgfqpoint{0.020833in}{0.020833in}}{%
\pgfpathmoveto{\pgfqpoint{0.000000in}{-0.020833in}}%
\pgfpathcurveto{\pgfqpoint{0.005525in}{-0.020833in}}{\pgfqpoint{0.010825in}{-0.018638in}}{\pgfqpoint{0.014731in}{-0.014731in}}%
\pgfpathcurveto{\pgfqpoint{0.018638in}{-0.010825in}}{\pgfqpoint{0.020833in}{-0.005525in}}{\pgfqpoint{0.020833in}{0.000000in}}%
\pgfpathcurveto{\pgfqpoint{0.020833in}{0.005525in}}{\pgfqpoint{0.018638in}{0.010825in}}{\pgfqpoint{0.014731in}{0.014731in}}%
\pgfpathcurveto{\pgfqpoint{0.010825in}{0.018638in}}{\pgfqpoint{0.005525in}{0.020833in}}{\pgfqpoint{0.000000in}{0.020833in}}%
\pgfpathcurveto{\pgfqpoint{-0.005525in}{0.020833in}}{\pgfqpoint{-0.010825in}{0.018638in}}{\pgfqpoint{-0.014731in}{0.014731in}}%
\pgfpathcurveto{\pgfqpoint{-0.018638in}{0.010825in}}{\pgfqpoint{-0.020833in}{0.005525in}}{\pgfqpoint{-0.020833in}{0.000000in}}%
\pgfpathcurveto{\pgfqpoint{-0.020833in}{-0.005525in}}{\pgfqpoint{-0.018638in}{-0.010825in}}{\pgfqpoint{-0.014731in}{-0.014731in}}%
\pgfpathcurveto{\pgfqpoint{-0.010825in}{-0.018638in}}{\pgfqpoint{-0.005525in}{-0.020833in}}{\pgfqpoint{0.000000in}{-0.020833in}}%
\pgfpathlineto{\pgfqpoint{0.000000in}{-0.020833in}}%
\pgfpathclose%
\pgfusepath{stroke,fill}%
}%
\begin{pgfscope}%
\pgfsys@transformshift{2.995580in}{1.549073in}%
\pgfsys@useobject{currentmarker}{}%
\end{pgfscope}%
\end{pgfscope}%
\begin{pgfscope}%
\definecolor{textcolor}{rgb}{0.000000,0.000000,0.000000}%
\pgfsetstrokecolor{textcolor}%
\pgfsetfillcolor{textcolor}%
\pgftext[x=3.295580in,y=1.490739in,left,base]{\color{textcolor}{\rmfamily\fontsize{12.000000}{14.400000}\selectfont\catcode`\^=\active\def^{\ifmmode\sp\else\^{}\fi}\catcode`\%=\active\def%{\%}$n_{\Omega}=20$, $n_{\Psi}=60$ (NC++)}}%
\end{pgfscope}%
\begin{pgfscope}%
\pgfsetrectcap%
\pgfsetroundjoin%
\pgfsetlinewidth{1.003750pt}%
\definecolor{currentstroke}{rgb}{0.550287,0.161158,0.505719}%
\pgfsetstrokecolor{currentstroke}%
\pgfsetdash{}{0pt}%
\pgfpathmoveto{\pgfqpoint{2.828913in}{1.299073in}}%
\pgfpathlineto{\pgfqpoint{2.995580in}{1.299073in}}%
\pgfpathlineto{\pgfqpoint{3.162246in}{1.299073in}}%
\pgfusepath{stroke}%
\end{pgfscope}%
\begin{pgfscope}%
\pgfsetbuttcap%
\pgfsetroundjoin%
\definecolor{currentfill}{rgb}{0.550287,0.161158,0.505719}%
\pgfsetfillcolor{currentfill}%
\pgfsetlinewidth{1.003750pt}%
\definecolor{currentstroke}{rgb}{0.550287,0.161158,0.505719}%
\pgfsetstrokecolor{currentstroke}%
\pgfsetdash{}{0pt}%
\pgfsys@defobject{currentmarker}{\pgfqpoint{-0.020833in}{-0.020833in}}{\pgfqpoint{0.020833in}{0.020833in}}{%
\pgfpathmoveto{\pgfqpoint{0.000000in}{-0.020833in}}%
\pgfpathcurveto{\pgfqpoint{0.005525in}{-0.020833in}}{\pgfqpoint{0.010825in}{-0.018638in}}{\pgfqpoint{0.014731in}{-0.014731in}}%
\pgfpathcurveto{\pgfqpoint{0.018638in}{-0.010825in}}{\pgfqpoint{0.020833in}{-0.005525in}}{\pgfqpoint{0.020833in}{0.000000in}}%
\pgfpathcurveto{\pgfqpoint{0.020833in}{0.005525in}}{\pgfqpoint{0.018638in}{0.010825in}}{\pgfqpoint{0.014731in}{0.014731in}}%
\pgfpathcurveto{\pgfqpoint{0.010825in}{0.018638in}}{\pgfqpoint{0.005525in}{0.020833in}}{\pgfqpoint{0.000000in}{0.020833in}}%
\pgfpathcurveto{\pgfqpoint{-0.005525in}{0.020833in}}{\pgfqpoint{-0.010825in}{0.018638in}}{\pgfqpoint{-0.014731in}{0.014731in}}%
\pgfpathcurveto{\pgfqpoint{-0.018638in}{0.010825in}}{\pgfqpoint{-0.020833in}{0.005525in}}{\pgfqpoint{-0.020833in}{0.000000in}}%
\pgfpathcurveto{\pgfqpoint{-0.020833in}{-0.005525in}}{\pgfqpoint{-0.018638in}{-0.010825in}}{\pgfqpoint{-0.014731in}{-0.014731in}}%
\pgfpathcurveto{\pgfqpoint{-0.010825in}{-0.018638in}}{\pgfqpoint{-0.005525in}{-0.020833in}}{\pgfqpoint{0.000000in}{-0.020833in}}%
\pgfpathlineto{\pgfqpoint{0.000000in}{-0.020833in}}%
\pgfpathclose%
\pgfusepath{stroke,fill}%
}%
\begin{pgfscope}%
\pgfsys@transformshift{2.995580in}{1.299073in}%
\pgfsys@useobject{currentmarker}{}%
\end{pgfscope}%
\end{pgfscope}%
\begin{pgfscope}%
\definecolor{textcolor}{rgb}{0.000000,0.000000,0.000000}%
\pgfsetstrokecolor{textcolor}%
\pgfsetfillcolor{textcolor}%
\pgftext[x=3.295580in,y=1.240739in,left,base]{\color{textcolor}{\rmfamily\fontsize{12.000000}{14.400000}\selectfont\catcode`\^=\active\def^{\ifmmode\sp\else\^{}\fi}\catcode`\%=\active\def%{\%}$n_{\Omega}=40$, $n_{\Psi}=40$ (NC++)}}%
\end{pgfscope}%
\begin{pgfscope}%
\pgfsetrectcap%
\pgfsetroundjoin%
\pgfsetlinewidth{1.003750pt}%
\definecolor{currentstroke}{rgb}{0.868793,0.287728,0.409303}%
\pgfsetstrokecolor{currentstroke}%
\pgfsetdash{}{0pt}%
\pgfpathmoveto{\pgfqpoint{2.828913in}{1.049073in}}%
\pgfpathlineto{\pgfqpoint{2.995580in}{1.049073in}}%
\pgfpathlineto{\pgfqpoint{3.162246in}{1.049073in}}%
\pgfusepath{stroke}%
\end{pgfscope}%
\begin{pgfscope}%
\pgfsetbuttcap%
\pgfsetroundjoin%
\definecolor{currentfill}{rgb}{0.868793,0.287728,0.409303}%
\pgfsetfillcolor{currentfill}%
\pgfsetlinewidth{1.003750pt}%
\definecolor{currentstroke}{rgb}{0.868793,0.287728,0.409303}%
\pgfsetstrokecolor{currentstroke}%
\pgfsetdash{}{0pt}%
\pgfsys@defobject{currentmarker}{\pgfqpoint{-0.020833in}{-0.020833in}}{\pgfqpoint{0.020833in}{0.020833in}}{%
\pgfpathmoveto{\pgfqpoint{0.000000in}{-0.020833in}}%
\pgfpathcurveto{\pgfqpoint{0.005525in}{-0.020833in}}{\pgfqpoint{0.010825in}{-0.018638in}}{\pgfqpoint{0.014731in}{-0.014731in}}%
\pgfpathcurveto{\pgfqpoint{0.018638in}{-0.010825in}}{\pgfqpoint{0.020833in}{-0.005525in}}{\pgfqpoint{0.020833in}{0.000000in}}%
\pgfpathcurveto{\pgfqpoint{0.020833in}{0.005525in}}{\pgfqpoint{0.018638in}{0.010825in}}{\pgfqpoint{0.014731in}{0.014731in}}%
\pgfpathcurveto{\pgfqpoint{0.010825in}{0.018638in}}{\pgfqpoint{0.005525in}{0.020833in}}{\pgfqpoint{0.000000in}{0.020833in}}%
\pgfpathcurveto{\pgfqpoint{-0.005525in}{0.020833in}}{\pgfqpoint{-0.010825in}{0.018638in}}{\pgfqpoint{-0.014731in}{0.014731in}}%
\pgfpathcurveto{\pgfqpoint{-0.018638in}{0.010825in}}{\pgfqpoint{-0.020833in}{0.005525in}}{\pgfqpoint{-0.020833in}{0.000000in}}%
\pgfpathcurveto{\pgfqpoint{-0.020833in}{-0.005525in}}{\pgfqpoint{-0.018638in}{-0.010825in}}{\pgfqpoint{-0.014731in}{-0.014731in}}%
\pgfpathcurveto{\pgfqpoint{-0.010825in}{-0.018638in}}{\pgfqpoint{-0.005525in}{-0.020833in}}{\pgfqpoint{0.000000in}{-0.020833in}}%
\pgfpathlineto{\pgfqpoint{0.000000in}{-0.020833in}}%
\pgfpathclose%
\pgfusepath{stroke,fill}%
}%
\begin{pgfscope}%
\pgfsys@transformshift{2.995580in}{1.049073in}%
\pgfsys@useobject{currentmarker}{}%
\end{pgfscope}%
\end{pgfscope}%
\begin{pgfscope}%
\definecolor{textcolor}{rgb}{0.000000,0.000000,0.000000}%
\pgfsetstrokecolor{textcolor}%
\pgfsetfillcolor{textcolor}%
\pgftext[x=3.295580in,y=0.990739in,left,base]{\color{textcolor}{\rmfamily\fontsize{12.000000}{14.400000}\selectfont\catcode`\^=\active\def^{\ifmmode\sp\else\^{}\fi}\catcode`\%=\active\def%{\%}$n_{\Omega}=60$, $n_{\Psi}=20$ (NC++)}}%
\end{pgfscope}%
\begin{pgfscope}%
\pgfsetrectcap%
\pgfsetroundjoin%
\pgfsetlinewidth{1.003750pt}%
\definecolor{currentstroke}{rgb}{0.994738,0.624350,0.427397}%
\pgfsetstrokecolor{currentstroke}%
\pgfsetdash{}{0pt}%
\pgfpathmoveto{\pgfqpoint{2.828913in}{0.799073in}}%
\pgfpathlineto{\pgfqpoint{2.995580in}{0.799073in}}%
\pgfpathlineto{\pgfqpoint{3.162246in}{0.799073in}}%
\pgfusepath{stroke}%
\end{pgfscope}%
\begin{pgfscope}%
\pgfsetbuttcap%
\pgfsetroundjoin%
\definecolor{currentfill}{rgb}{0.994738,0.624350,0.427397}%
\pgfsetfillcolor{currentfill}%
\pgfsetlinewidth{1.003750pt}%
\definecolor{currentstroke}{rgb}{0.994738,0.624350,0.427397}%
\pgfsetstrokecolor{currentstroke}%
\pgfsetdash{}{0pt}%
\pgfsys@defobject{currentmarker}{\pgfqpoint{-0.020833in}{-0.020833in}}{\pgfqpoint{0.020833in}{0.020833in}}{%
\pgfpathmoveto{\pgfqpoint{0.000000in}{-0.020833in}}%
\pgfpathcurveto{\pgfqpoint{0.005525in}{-0.020833in}}{\pgfqpoint{0.010825in}{-0.018638in}}{\pgfqpoint{0.014731in}{-0.014731in}}%
\pgfpathcurveto{\pgfqpoint{0.018638in}{-0.010825in}}{\pgfqpoint{0.020833in}{-0.005525in}}{\pgfqpoint{0.020833in}{0.000000in}}%
\pgfpathcurveto{\pgfqpoint{0.020833in}{0.005525in}}{\pgfqpoint{0.018638in}{0.010825in}}{\pgfqpoint{0.014731in}{0.014731in}}%
\pgfpathcurveto{\pgfqpoint{0.010825in}{0.018638in}}{\pgfqpoint{0.005525in}{0.020833in}}{\pgfqpoint{0.000000in}{0.020833in}}%
\pgfpathcurveto{\pgfqpoint{-0.005525in}{0.020833in}}{\pgfqpoint{-0.010825in}{0.018638in}}{\pgfqpoint{-0.014731in}{0.014731in}}%
\pgfpathcurveto{\pgfqpoint{-0.018638in}{0.010825in}}{\pgfqpoint{-0.020833in}{0.005525in}}{\pgfqpoint{-0.020833in}{0.000000in}}%
\pgfpathcurveto{\pgfqpoint{-0.020833in}{-0.005525in}}{\pgfqpoint{-0.018638in}{-0.010825in}}{\pgfqpoint{-0.014731in}{-0.014731in}}%
\pgfpathcurveto{\pgfqpoint{-0.010825in}{-0.018638in}}{\pgfqpoint{-0.005525in}{-0.020833in}}{\pgfqpoint{0.000000in}{-0.020833in}}%
\pgfpathlineto{\pgfqpoint{0.000000in}{-0.020833in}}%
\pgfpathclose%
\pgfusepath{stroke,fill}%
}%
\begin{pgfscope}%
\pgfsys@transformshift{2.995580in}{0.799073in}%
\pgfsys@useobject{currentmarker}{}%
\end{pgfscope}%
\end{pgfscope}%
\begin{pgfscope}%
\definecolor{textcolor}{rgb}{0.000000,0.000000,0.000000}%
\pgfsetstrokecolor{textcolor}%
\pgfsetfillcolor{textcolor}%
\pgftext[x=3.295580in,y=0.740739in,left,base]{\color{textcolor}{\rmfamily\fontsize{12.000000}{14.400000}\selectfont\catcode`\^=\active\def^{\ifmmode\sp\else\^{}\fi}\catcode`\%=\active\def%{\%}$n_{\Omega}=80$, $n_{\Psi}=0$ (NC)}}%
\end{pgfscope}%
\end{pgfpicture}%
\makeatother%
\endgroup%

    \caption{The \gls{NCPP} method for different ways of allocations a 
    total of \gls{sketch-size} $+$ \gls{num-hutchinson-queries} $=80$ random vectors
    to either the Nystr\"om low-rank approximation or the Hutchinson's trace estimation
    for the Gaussian \gls{smoothing-kernel} with multiple different values of
    the \gls{smoothing-parameter}. We make the approximation error made in the
    Chebyshev expansion negligible by rescaling \gls{chebyshev-degree} $=120 / \sigma$.}
    \label{fig:5-experiments-electronic-structure-matvec-mixture}
\end{figure}

%%%%%%%%%%%%%%%%%%%%%%%%%%%%%%%%%%%%%%%%%%%%%%%%%%%%%%%%%%%%%%%%%%%%%%%%%%%%%%%%

\clearpage
\section{Benchmark against Haydock's method}
\label{sec:5-experiments-haydock-method}

Haydock's method \cite{haydock1972electronic,lin2016review} is a specialized technique for approximating \gls{smooth-spectral-density}
in the case where a Lorentzian smoothing kernel
\begin{equation}
    g_{\sigma}(s) = \frac{1}{\pi} \frac{\sigma}{s^2 + \sigma^2} = -\frac{1}{\pi} \Im\left\{ \frac{1}{s + \iota \sigma} \right\}
    \label{equ:5-experiments-cauchy-kernel}
\end{equation}
is used. A comparison of this kernel with the Gaussian kernel \refequ{equ:1-introduction-def-gaussian-kernel}
is provided in \reffig{fig:5-experiments-haydock-kernel}.\\
\begin{figure}[ht]
    \centering
    %% Creator: Matplotlib, PGF backend
%%
%% To include the figure in your LaTeX document, write
%%   \input{<filename>.pgf}
%%
%% Make sure the required packages are loaded in your preamble
%%   \usepackage{pgf}
%%
%% Also ensure that all the required font packages are loaded; for instance,
%% the lmodern package is sometimes necessary when using math font.
%%   \usepackage{lmodern}
%%
%% Figures using additional raster images can only be included by \input if
%% they are in the same directory as the main LaTeX file. For loading figures
%% from other directories you can use the `import` package
%%   \usepackage{import}
%%
%% and then include the figures with
%%   \import{<path to file>}{<filename>.pgf}
%%
%% Matplotlib used the following preamble
%%   \def\mathdefault#1{#1}
%%   \everymath=\expandafter{\the\everymath\displaystyle}
%%   
%%   \makeatletter\@ifpackageloaded{underscore}{}{\usepackage[strings]{underscore}}\makeatother
%%
\begingroup%
\makeatletter%
\begin{pgfpicture}%
\pgfpathrectangle{\pgfpointorigin}{\pgfqpoint{3.701041in}{1.804073in}}%
\pgfusepath{use as bounding box, clip}%
\begin{pgfscope}%
\pgfsetbuttcap%
\pgfsetmiterjoin%
\pgfsetlinewidth{0.000000pt}%
\definecolor{currentstroke}{rgb}{0.000000,0.000000,0.000000}%
\pgfsetstrokecolor{currentstroke}%
\pgfsetstrokeopacity{0.000000}%
\pgfsetdash{}{0pt}%
\pgfpathmoveto{\pgfqpoint{0.000000in}{0.000000in}}%
\pgfpathlineto{\pgfqpoint{3.701041in}{0.000000in}}%
\pgfpathlineto{\pgfqpoint{3.701041in}{1.804073in}}%
\pgfpathlineto{\pgfqpoint{0.000000in}{1.804073in}}%
\pgfpathlineto{\pgfqpoint{0.000000in}{0.000000in}}%
\pgfpathclose%
\pgfusepath{}%
\end{pgfscope}%
\begin{pgfscope}%
\pgfsetbuttcap%
\pgfsetmiterjoin%
\pgfsetlinewidth{0.000000pt}%
\definecolor{currentstroke}{rgb}{0.000000,0.000000,0.000000}%
\pgfsetstrokecolor{currentstroke}%
\pgfsetstrokeopacity{0.000000}%
\pgfsetdash{}{0pt}%
\pgfpathmoveto{\pgfqpoint{0.501041in}{0.549073in}}%
\pgfpathlineto{\pgfqpoint{3.601041in}{0.549073in}}%
\pgfpathlineto{\pgfqpoint{3.601041in}{1.704073in}}%
\pgfpathlineto{\pgfqpoint{0.501041in}{1.704073in}}%
\pgfpathlineto{\pgfqpoint{0.501041in}{0.549073in}}%
\pgfpathclose%
\pgfusepath{}%
\end{pgfscope}%
\begin{pgfscope}%
\pgfsetbuttcap%
\pgfsetroundjoin%
\definecolor{currentfill}{rgb}{0.000000,0.000000,0.000000}%
\pgfsetfillcolor{currentfill}%
\pgfsetlinewidth{0.803000pt}%
\definecolor{currentstroke}{rgb}{0.000000,0.000000,0.000000}%
\pgfsetstrokecolor{currentstroke}%
\pgfsetdash{}{0pt}%
\pgfsys@defobject{currentmarker}{\pgfqpoint{0.000000in}{-0.048611in}}{\pgfqpoint{0.000000in}{0.000000in}}{%
\pgfpathmoveto{\pgfqpoint{0.000000in}{0.000000in}}%
\pgfpathlineto{\pgfqpoint{0.000000in}{-0.048611in}}%
\pgfusepath{stroke,fill}%
}%
\begin{pgfscope}%
\pgfsys@transformshift{0.811041in}{0.549073in}%
\pgfsys@useobject{currentmarker}{}%
\end{pgfscope}%
\end{pgfscope}%
\begin{pgfscope}%
\definecolor{textcolor}{rgb}{0.000000,0.000000,0.000000}%
\pgfsetstrokecolor{textcolor}%
\pgfsetfillcolor{textcolor}%
\pgftext[x=0.811041in,y=0.451851in,,top]{\color{textcolor}{\rmfamily\fontsize{12.000000}{14.400000}\selectfont\catcode`\^=\active\def^{\ifmmode\sp\else\^{}\fi}\catcode`\%=\active\def%{\%}$\mathdefault{\ensuremath{-}0.4}$}}%
\end{pgfscope}%
\begin{pgfscope}%
\pgfsetbuttcap%
\pgfsetroundjoin%
\definecolor{currentfill}{rgb}{0.000000,0.000000,0.000000}%
\pgfsetfillcolor{currentfill}%
\pgfsetlinewidth{0.803000pt}%
\definecolor{currentstroke}{rgb}{0.000000,0.000000,0.000000}%
\pgfsetstrokecolor{currentstroke}%
\pgfsetdash{}{0pt}%
\pgfsys@defobject{currentmarker}{\pgfqpoint{0.000000in}{-0.048611in}}{\pgfqpoint{0.000000in}{0.000000in}}{%
\pgfpathmoveto{\pgfqpoint{0.000000in}{0.000000in}}%
\pgfpathlineto{\pgfqpoint{0.000000in}{-0.048611in}}%
\pgfusepath{stroke,fill}%
}%
\begin{pgfscope}%
\pgfsys@transformshift{1.431041in}{0.549073in}%
\pgfsys@useobject{currentmarker}{}%
\end{pgfscope}%
\end{pgfscope}%
\begin{pgfscope}%
\definecolor{textcolor}{rgb}{0.000000,0.000000,0.000000}%
\pgfsetstrokecolor{textcolor}%
\pgfsetfillcolor{textcolor}%
\pgftext[x=1.431041in,y=0.451851in,,top]{\color{textcolor}{\rmfamily\fontsize{12.000000}{14.400000}\selectfont\catcode`\^=\active\def^{\ifmmode\sp\else\^{}\fi}\catcode`\%=\active\def%{\%}$\mathdefault{\ensuremath{-}0.2}$}}%
\end{pgfscope}%
\begin{pgfscope}%
\pgfsetbuttcap%
\pgfsetroundjoin%
\definecolor{currentfill}{rgb}{0.000000,0.000000,0.000000}%
\pgfsetfillcolor{currentfill}%
\pgfsetlinewidth{0.803000pt}%
\definecolor{currentstroke}{rgb}{0.000000,0.000000,0.000000}%
\pgfsetstrokecolor{currentstroke}%
\pgfsetdash{}{0pt}%
\pgfsys@defobject{currentmarker}{\pgfqpoint{0.000000in}{-0.048611in}}{\pgfqpoint{0.000000in}{0.000000in}}{%
\pgfpathmoveto{\pgfqpoint{0.000000in}{0.000000in}}%
\pgfpathlineto{\pgfqpoint{0.000000in}{-0.048611in}}%
\pgfusepath{stroke,fill}%
}%
\begin{pgfscope}%
\pgfsys@transformshift{2.051041in}{0.549073in}%
\pgfsys@useobject{currentmarker}{}%
\end{pgfscope}%
\end{pgfscope}%
\begin{pgfscope}%
\definecolor{textcolor}{rgb}{0.000000,0.000000,0.000000}%
\pgfsetstrokecolor{textcolor}%
\pgfsetfillcolor{textcolor}%
\pgftext[x=2.051041in,y=0.451851in,,top]{\color{textcolor}{\rmfamily\fontsize{12.000000}{14.400000}\selectfont\catcode`\^=\active\def^{\ifmmode\sp\else\^{}\fi}\catcode`\%=\active\def%{\%}$\mathdefault{0.0}$}}%
\end{pgfscope}%
\begin{pgfscope}%
\pgfsetbuttcap%
\pgfsetroundjoin%
\definecolor{currentfill}{rgb}{0.000000,0.000000,0.000000}%
\pgfsetfillcolor{currentfill}%
\pgfsetlinewidth{0.803000pt}%
\definecolor{currentstroke}{rgb}{0.000000,0.000000,0.000000}%
\pgfsetstrokecolor{currentstroke}%
\pgfsetdash{}{0pt}%
\pgfsys@defobject{currentmarker}{\pgfqpoint{0.000000in}{-0.048611in}}{\pgfqpoint{0.000000in}{0.000000in}}{%
\pgfpathmoveto{\pgfqpoint{0.000000in}{0.000000in}}%
\pgfpathlineto{\pgfqpoint{0.000000in}{-0.048611in}}%
\pgfusepath{stroke,fill}%
}%
\begin{pgfscope}%
\pgfsys@transformshift{2.671041in}{0.549073in}%
\pgfsys@useobject{currentmarker}{}%
\end{pgfscope}%
\end{pgfscope}%
\begin{pgfscope}%
\definecolor{textcolor}{rgb}{0.000000,0.000000,0.000000}%
\pgfsetstrokecolor{textcolor}%
\pgfsetfillcolor{textcolor}%
\pgftext[x=2.671041in,y=0.451851in,,top]{\color{textcolor}{\rmfamily\fontsize{12.000000}{14.400000}\selectfont\catcode`\^=\active\def^{\ifmmode\sp\else\^{}\fi}\catcode`\%=\active\def%{\%}$\mathdefault{0.2}$}}%
\end{pgfscope}%
\begin{pgfscope}%
\pgfsetbuttcap%
\pgfsetroundjoin%
\definecolor{currentfill}{rgb}{0.000000,0.000000,0.000000}%
\pgfsetfillcolor{currentfill}%
\pgfsetlinewidth{0.803000pt}%
\definecolor{currentstroke}{rgb}{0.000000,0.000000,0.000000}%
\pgfsetstrokecolor{currentstroke}%
\pgfsetdash{}{0pt}%
\pgfsys@defobject{currentmarker}{\pgfqpoint{0.000000in}{-0.048611in}}{\pgfqpoint{0.000000in}{0.000000in}}{%
\pgfpathmoveto{\pgfqpoint{0.000000in}{0.000000in}}%
\pgfpathlineto{\pgfqpoint{0.000000in}{-0.048611in}}%
\pgfusepath{stroke,fill}%
}%
\begin{pgfscope}%
\pgfsys@transformshift{3.291041in}{0.549073in}%
\pgfsys@useobject{currentmarker}{}%
\end{pgfscope}%
\end{pgfscope}%
\begin{pgfscope}%
\definecolor{textcolor}{rgb}{0.000000,0.000000,0.000000}%
\pgfsetstrokecolor{textcolor}%
\pgfsetfillcolor{textcolor}%
\pgftext[x=3.291041in,y=0.451851in,,top]{\color{textcolor}{\rmfamily\fontsize{12.000000}{14.400000}\selectfont\catcode`\^=\active\def^{\ifmmode\sp\else\^{}\fi}\catcode`\%=\active\def%{\%}$\mathdefault{0.4}$}}%
\end{pgfscope}%
\begin{pgfscope}%
\definecolor{textcolor}{rgb}{0.000000,0.000000,0.000000}%
\pgfsetstrokecolor{textcolor}%
\pgfsetfillcolor{textcolor}%
\pgftext[x=2.051041in,y=0.248148in,,top]{\color{textcolor}{\rmfamily\fontsize{12.000000}{14.400000}\selectfont\catcode`\^=\active\def^{\ifmmode\sp\else\^{}\fi}\catcode`\%=\active\def%{\%}$s$}}%
\end{pgfscope}%
\begin{pgfscope}%
\pgfsetbuttcap%
\pgfsetroundjoin%
\definecolor{currentfill}{rgb}{0.000000,0.000000,0.000000}%
\pgfsetfillcolor{currentfill}%
\pgfsetlinewidth{0.803000pt}%
\definecolor{currentstroke}{rgb}{0.000000,0.000000,0.000000}%
\pgfsetstrokecolor{currentstroke}%
\pgfsetdash{}{0pt}%
\pgfsys@defobject{currentmarker}{\pgfqpoint{-0.048611in}{0.000000in}}{\pgfqpoint{-0.000000in}{0.000000in}}{%
\pgfpathmoveto{\pgfqpoint{-0.000000in}{0.000000in}}%
\pgfpathlineto{\pgfqpoint{-0.048611in}{0.000000in}}%
\pgfusepath{stroke,fill}%
}%
\begin{pgfscope}%
\pgfsys@transformshift{0.501041in}{0.549073in}%
\pgfsys@useobject{currentmarker}{}%
\end{pgfscope}%
\end{pgfscope}%
\begin{pgfscope}%
\definecolor{textcolor}{rgb}{0.000000,0.000000,0.000000}%
\pgfsetstrokecolor{textcolor}%
\pgfsetfillcolor{textcolor}%
\pgftext[x=0.322222in, y=0.491203in, left, base]{\color{textcolor}{\rmfamily\fontsize{12.000000}{14.400000}\selectfont\catcode`\^=\active\def^{\ifmmode\sp\else\^{}\fi}\catcode`\%=\active\def%{\%}$\mathdefault{0}$}}%
\end{pgfscope}%
\begin{pgfscope}%
\pgfsetbuttcap%
\pgfsetroundjoin%
\definecolor{currentfill}{rgb}{0.000000,0.000000,0.000000}%
\pgfsetfillcolor{currentfill}%
\pgfsetlinewidth{0.803000pt}%
\definecolor{currentstroke}{rgb}{0.000000,0.000000,0.000000}%
\pgfsetstrokecolor{currentstroke}%
\pgfsetdash{}{0pt}%
\pgfsys@defobject{currentmarker}{\pgfqpoint{-0.048611in}{0.000000in}}{\pgfqpoint{-0.000000in}{0.000000in}}{%
\pgfpathmoveto{\pgfqpoint{-0.000000in}{0.000000in}}%
\pgfpathlineto{\pgfqpoint{-0.048611in}{0.000000in}}%
\pgfusepath{stroke,fill}%
}%
\begin{pgfscope}%
\pgfsys@transformshift{0.501041in}{1.228485in}%
\pgfsys@useobject{currentmarker}{}%
\end{pgfscope}%
\end{pgfscope}%
\begin{pgfscope}%
\definecolor{textcolor}{rgb}{0.000000,0.000000,0.000000}%
\pgfsetstrokecolor{textcolor}%
\pgfsetfillcolor{textcolor}%
\pgftext[x=0.322222in, y=1.170614in, left, base]{\color{textcolor}{\rmfamily\fontsize{12.000000}{14.400000}\selectfont\catcode`\^=\active\def^{\ifmmode\sp\else\^{}\fi}\catcode`\%=\active\def%{\%}$\mathdefault{5}$}}%
\end{pgfscope}%
\begin{pgfscope}%
\definecolor{textcolor}{rgb}{0.000000,0.000000,0.000000}%
\pgfsetstrokecolor{textcolor}%
\pgfsetfillcolor{textcolor}%
\pgftext[x=0.266667in,y=1.126573in,,bottom,rotate=90.000000]{\color{textcolor}{\rmfamily\fontsize{12.000000}{14.400000}\selectfont\catcode`\^=\active\def^{\ifmmode\sp\else\^{}\fi}\catcode`\%=\active\def%{\%}$g_{\sigma}(s)$}}%
\end{pgfscope}%
\begin{pgfscope}%
\pgfpathrectangle{\pgfqpoint{0.501041in}{0.549073in}}{\pgfqpoint{3.100000in}{1.155000in}}%
\pgfusepath{clip}%
\pgfsetrectcap%
\pgfsetroundjoin%
\pgfsetlinewidth{1.505625pt}%
\definecolor{currentstroke}{rgb}{0.478431,0.701961,0.941176}%
\pgfsetstrokecolor{currentstroke}%
\pgfsetdash{}{0pt}%
\pgfpathmoveto{\pgfqpoint{0.501041in}{0.549073in}}%
\pgfpathlineto{\pgfqpoint{0.532354in}{0.549073in}}%
\pgfpathlineto{\pgfqpoint{0.563667in}{0.549073in}}%
\pgfpathlineto{\pgfqpoint{0.594980in}{0.549073in}}%
\pgfpathlineto{\pgfqpoint{0.626293in}{0.549073in}}%
\pgfpathlineto{\pgfqpoint{0.657606in}{0.549073in}}%
\pgfpathlineto{\pgfqpoint{0.688920in}{0.549073in}}%
\pgfpathlineto{\pgfqpoint{0.720233in}{0.549073in}}%
\pgfpathlineto{\pgfqpoint{0.751546in}{0.549073in}}%
\pgfpathlineto{\pgfqpoint{0.782859in}{0.549073in}}%
\pgfpathlineto{\pgfqpoint{0.814172in}{0.549073in}}%
\pgfpathlineto{\pgfqpoint{0.845485in}{0.549073in}}%
\pgfpathlineto{\pgfqpoint{0.876798in}{0.549073in}}%
\pgfpathlineto{\pgfqpoint{0.908112in}{0.549073in}}%
\pgfpathlineto{\pgfqpoint{0.939425in}{0.549073in}}%
\pgfpathlineto{\pgfqpoint{0.970738in}{0.549073in}}%
\pgfpathlineto{\pgfqpoint{1.002051in}{0.549073in}}%
\pgfpathlineto{\pgfqpoint{1.033364in}{0.549073in}}%
\pgfpathlineto{\pgfqpoint{1.064677in}{0.549073in}}%
\pgfpathlineto{\pgfqpoint{1.095990in}{0.549073in}}%
\pgfpathlineto{\pgfqpoint{1.127303in}{0.549073in}}%
\pgfpathlineto{\pgfqpoint{1.158617in}{0.549073in}}%
\pgfpathlineto{\pgfqpoint{1.189930in}{0.549073in}}%
\pgfpathlineto{\pgfqpoint{1.221243in}{0.549073in}}%
\pgfpathlineto{\pgfqpoint{1.252556in}{0.549075in}}%
\pgfpathlineto{\pgfqpoint{1.283869in}{0.549078in}}%
\pgfpathlineto{\pgfqpoint{1.315182in}{0.549087in}}%
\pgfpathlineto{\pgfqpoint{1.346495in}{0.549108in}}%
\pgfpathlineto{\pgfqpoint{1.377809in}{0.549160in}}%
\pgfpathlineto{\pgfqpoint{1.409122in}{0.549277in}}%
\pgfpathlineto{\pgfqpoint{1.440435in}{0.549535in}}%
\pgfpathlineto{\pgfqpoint{1.471748in}{0.550077in}}%
\pgfpathlineto{\pgfqpoint{1.503061in}{0.551167in}}%
\pgfpathlineto{\pgfqpoint{1.534374in}{0.553264in}}%
\pgfpathlineto{\pgfqpoint{1.565687in}{0.557126in}}%
\pgfpathlineto{\pgfqpoint{1.597000in}{0.563926in}}%
\pgfpathlineto{\pgfqpoint{1.628314in}{0.575373in}}%
\pgfpathlineto{\pgfqpoint{1.659627in}{0.593780in}}%
\pgfpathlineto{\pgfqpoint{1.690940in}{0.622030in}}%
\pgfpathlineto{\pgfqpoint{1.722253in}{0.663371in}}%
\pgfpathlineto{\pgfqpoint{1.753566in}{0.720976in}}%
\pgfpathlineto{\pgfqpoint{1.784879in}{0.797275in}}%
\pgfpathlineto{\pgfqpoint{1.816192in}{0.893107in}}%
\pgfpathlineto{\pgfqpoint{1.847505in}{1.006870in}}%
\pgfpathlineto{\pgfqpoint{1.878819in}{1.133890in}}%
\pgfpathlineto{\pgfqpoint{1.910132in}{1.266277in}}%
\pgfpathlineto{\pgfqpoint{1.941445in}{1.393458in}}%
\pgfpathlineto{\pgfqpoint{1.972758in}{1.503437in}}%
\pgfpathlineto{\pgfqpoint{2.004071in}{1.584604in}}%
\pgfpathlineto{\pgfqpoint{2.035384in}{1.627740in}}%
\pgfpathlineto{\pgfqpoint{2.066697in}{1.627740in}}%
\pgfpathlineto{\pgfqpoint{2.098011in}{1.584604in}}%
\pgfpathlineto{\pgfqpoint{2.129324in}{1.503437in}}%
\pgfpathlineto{\pgfqpoint{2.160637in}{1.393458in}}%
\pgfpathlineto{\pgfqpoint{2.191950in}{1.266277in}}%
\pgfpathlineto{\pgfqpoint{2.223263in}{1.133890in}}%
\pgfpathlineto{\pgfqpoint{2.254576in}{1.006870in}}%
\pgfpathlineto{\pgfqpoint{2.285889in}{0.893107in}}%
\pgfpathlineto{\pgfqpoint{2.317202in}{0.797275in}}%
\pgfpathlineto{\pgfqpoint{2.348516in}{0.720976in}}%
\pgfpathlineto{\pgfqpoint{2.379829in}{0.663371in}}%
\pgfpathlineto{\pgfqpoint{2.411142in}{0.622030in}}%
\pgfpathlineto{\pgfqpoint{2.442455in}{0.593780in}}%
\pgfpathlineto{\pgfqpoint{2.473768in}{0.575373in}}%
\pgfpathlineto{\pgfqpoint{2.505081in}{0.563926in}}%
\pgfpathlineto{\pgfqpoint{2.536394in}{0.557126in}}%
\pgfpathlineto{\pgfqpoint{2.567708in}{0.553264in}}%
\pgfpathlineto{\pgfqpoint{2.599021in}{0.551167in}}%
\pgfpathlineto{\pgfqpoint{2.630334in}{0.550077in}}%
\pgfpathlineto{\pgfqpoint{2.661647in}{0.549535in}}%
\pgfpathlineto{\pgfqpoint{2.692960in}{0.549277in}}%
\pgfpathlineto{\pgfqpoint{2.724273in}{0.549160in}}%
\pgfpathlineto{\pgfqpoint{2.755586in}{0.549108in}}%
\pgfpathlineto{\pgfqpoint{2.786899in}{0.549087in}}%
\pgfpathlineto{\pgfqpoint{2.818213in}{0.549078in}}%
\pgfpathlineto{\pgfqpoint{2.849526in}{0.549075in}}%
\pgfpathlineto{\pgfqpoint{2.880839in}{0.549073in}}%
\pgfpathlineto{\pgfqpoint{2.912152in}{0.549073in}}%
\pgfpathlineto{\pgfqpoint{2.943465in}{0.549073in}}%
\pgfpathlineto{\pgfqpoint{2.974778in}{0.549073in}}%
\pgfpathlineto{\pgfqpoint{3.006091in}{0.549073in}}%
\pgfpathlineto{\pgfqpoint{3.037404in}{0.549073in}}%
\pgfpathlineto{\pgfqpoint{3.068718in}{0.549073in}}%
\pgfpathlineto{\pgfqpoint{3.100031in}{0.549073in}}%
\pgfpathlineto{\pgfqpoint{3.131344in}{0.549073in}}%
\pgfpathlineto{\pgfqpoint{3.162657in}{0.549073in}}%
\pgfpathlineto{\pgfqpoint{3.193970in}{0.549073in}}%
\pgfpathlineto{\pgfqpoint{3.225283in}{0.549073in}}%
\pgfpathlineto{\pgfqpoint{3.256596in}{0.549073in}}%
\pgfpathlineto{\pgfqpoint{3.287910in}{0.549073in}}%
\pgfpathlineto{\pgfqpoint{3.319223in}{0.549073in}}%
\pgfpathlineto{\pgfqpoint{3.350536in}{0.549073in}}%
\pgfpathlineto{\pgfqpoint{3.381849in}{0.549073in}}%
\pgfpathlineto{\pgfqpoint{3.413162in}{0.549073in}}%
\pgfpathlineto{\pgfqpoint{3.444475in}{0.549073in}}%
\pgfpathlineto{\pgfqpoint{3.475788in}{0.549073in}}%
\pgfpathlineto{\pgfqpoint{3.507101in}{0.549073in}}%
\pgfpathlineto{\pgfqpoint{3.538415in}{0.549073in}}%
\pgfpathlineto{\pgfqpoint{3.569728in}{0.549073in}}%
\pgfpathlineto{\pgfqpoint{3.601041in}{0.549073in}}%
\pgfusepath{stroke}%
\end{pgfscope}%
\begin{pgfscope}%
\pgfpathrectangle{\pgfqpoint{0.501041in}{0.549073in}}{\pgfqpoint{3.100000in}{1.155000in}}%
\pgfusepath{clip}%
\pgfsetrectcap%
\pgfsetroundjoin%
\pgfsetlinewidth{1.505625pt}%
\definecolor{currentstroke}{rgb}{0.184314,0.270588,0.360784}%
\pgfsetstrokecolor{currentstroke}%
\pgfsetdash{}{0pt}%
\pgfpathmoveto{\pgfqpoint{0.501041in}{0.557638in}}%
\pgfpathlineto{\pgfqpoint{0.532354in}{0.557991in}}%
\pgfpathlineto{\pgfqpoint{0.563667in}{0.558366in}}%
\pgfpathlineto{\pgfqpoint{0.594980in}{0.558766in}}%
\pgfpathlineto{\pgfqpoint{0.626293in}{0.559191in}}%
\pgfpathlineto{\pgfqpoint{0.657606in}{0.559646in}}%
\pgfpathlineto{\pgfqpoint{0.688920in}{0.560131in}}%
\pgfpathlineto{\pgfqpoint{0.720233in}{0.560651in}}%
\pgfpathlineto{\pgfqpoint{0.751546in}{0.561207in}}%
\pgfpathlineto{\pgfqpoint{0.782859in}{0.561805in}}%
\pgfpathlineto{\pgfqpoint{0.814172in}{0.562448in}}%
\pgfpathlineto{\pgfqpoint{0.845485in}{0.563140in}}%
\pgfpathlineto{\pgfqpoint{0.876798in}{0.563887in}}%
\pgfpathlineto{\pgfqpoint{0.908112in}{0.564695in}}%
\pgfpathlineto{\pgfqpoint{0.939425in}{0.565571in}}%
\pgfpathlineto{\pgfqpoint{0.970738in}{0.566522in}}%
\pgfpathlineto{\pgfqpoint{1.002051in}{0.567556in}}%
\pgfpathlineto{\pgfqpoint{1.033364in}{0.568685in}}%
\pgfpathlineto{\pgfqpoint{1.064677in}{0.569920in}}%
\pgfpathlineto{\pgfqpoint{1.095990in}{0.571273in}}%
\pgfpathlineto{\pgfqpoint{1.127303in}{0.572762in}}%
\pgfpathlineto{\pgfqpoint{1.158617in}{0.574404in}}%
\pgfpathlineto{\pgfqpoint{1.189930in}{0.576221in}}%
\pgfpathlineto{\pgfqpoint{1.221243in}{0.578238in}}%
\pgfpathlineto{\pgfqpoint{1.252556in}{0.580486in}}%
\pgfpathlineto{\pgfqpoint{1.283869in}{0.583000in}}%
\pgfpathlineto{\pgfqpoint{1.315182in}{0.585823in}}%
\pgfpathlineto{\pgfqpoint{1.346495in}{0.589008in}}%
\pgfpathlineto{\pgfqpoint{1.377809in}{0.592618in}}%
\pgfpathlineto{\pgfqpoint{1.409122in}{0.596731in}}%
\pgfpathlineto{\pgfqpoint{1.440435in}{0.601441in}}%
\pgfpathlineto{\pgfqpoint{1.471748in}{0.606866in}}%
\pgfpathlineto{\pgfqpoint{1.503061in}{0.613157in}}%
\pgfpathlineto{\pgfqpoint{1.534374in}{0.620499in}}%
\pgfpathlineto{\pgfqpoint{1.565687in}{0.629132in}}%
\pgfpathlineto{\pgfqpoint{1.597000in}{0.639364in}}%
\pgfpathlineto{\pgfqpoint{1.628314in}{0.651591in}}%
\pgfpathlineto{\pgfqpoint{1.659627in}{0.666338in}}%
\pgfpathlineto{\pgfqpoint{1.690940in}{0.684292in}}%
\pgfpathlineto{\pgfqpoint{1.722253in}{0.706368in}}%
\pgfpathlineto{\pgfqpoint{1.753566in}{0.733783in}}%
\pgfpathlineto{\pgfqpoint{1.784879in}{0.768147in}}%
\pgfpathlineto{\pgfqpoint{1.816192in}{0.811554in}}%
\pgfpathlineto{\pgfqpoint{1.847505in}{0.866604in}}%
\pgfpathlineto{\pgfqpoint{1.878819in}{0.936196in}}%
\pgfpathlineto{\pgfqpoint{1.910132in}{1.022700in}}%
\pgfpathlineto{\pgfqpoint{1.941445in}{1.125795in}}%
\pgfpathlineto{\pgfqpoint{1.972758in}{1.238317in}}%
\pgfpathlineto{\pgfqpoint{2.004071in}{1.341372in}}%
\pgfpathlineto{\pgfqpoint{2.035384in}{1.405390in}}%
\pgfpathlineto{\pgfqpoint{2.066697in}{1.405390in}}%
\pgfpathlineto{\pgfqpoint{2.098011in}{1.341372in}}%
\pgfpathlineto{\pgfqpoint{2.129324in}{1.238317in}}%
\pgfpathlineto{\pgfqpoint{2.160637in}{1.125795in}}%
\pgfpathlineto{\pgfqpoint{2.191950in}{1.022700in}}%
\pgfpathlineto{\pgfqpoint{2.223263in}{0.936196in}}%
\pgfpathlineto{\pgfqpoint{2.254576in}{0.866604in}}%
\pgfpathlineto{\pgfqpoint{2.285889in}{0.811554in}}%
\pgfpathlineto{\pgfqpoint{2.317202in}{0.768147in}}%
\pgfpathlineto{\pgfqpoint{2.348516in}{0.733783in}}%
\pgfpathlineto{\pgfqpoint{2.379829in}{0.706368in}}%
\pgfpathlineto{\pgfqpoint{2.411142in}{0.684292in}}%
\pgfpathlineto{\pgfqpoint{2.442455in}{0.666338in}}%
\pgfpathlineto{\pgfqpoint{2.473768in}{0.651591in}}%
\pgfpathlineto{\pgfqpoint{2.505081in}{0.639364in}}%
\pgfpathlineto{\pgfqpoint{2.536394in}{0.629132in}}%
\pgfpathlineto{\pgfqpoint{2.567708in}{0.620499in}}%
\pgfpathlineto{\pgfqpoint{2.599021in}{0.613157in}}%
\pgfpathlineto{\pgfqpoint{2.630334in}{0.606866in}}%
\pgfpathlineto{\pgfqpoint{2.661647in}{0.601441in}}%
\pgfpathlineto{\pgfqpoint{2.692960in}{0.596731in}}%
\pgfpathlineto{\pgfqpoint{2.724273in}{0.592618in}}%
\pgfpathlineto{\pgfqpoint{2.755586in}{0.589008in}}%
\pgfpathlineto{\pgfqpoint{2.786899in}{0.585823in}}%
\pgfpathlineto{\pgfqpoint{2.818213in}{0.583000in}}%
\pgfpathlineto{\pgfqpoint{2.849526in}{0.580486in}}%
\pgfpathlineto{\pgfqpoint{2.880839in}{0.578238in}}%
\pgfpathlineto{\pgfqpoint{2.912152in}{0.576221in}}%
\pgfpathlineto{\pgfqpoint{2.943465in}{0.574404in}}%
\pgfpathlineto{\pgfqpoint{2.974778in}{0.572762in}}%
\pgfpathlineto{\pgfqpoint{3.006091in}{0.571273in}}%
\pgfpathlineto{\pgfqpoint{3.037404in}{0.569920in}}%
\pgfpathlineto{\pgfqpoint{3.068718in}{0.568685in}}%
\pgfpathlineto{\pgfqpoint{3.100031in}{0.567556in}}%
\pgfpathlineto{\pgfqpoint{3.131344in}{0.566522in}}%
\pgfpathlineto{\pgfqpoint{3.162657in}{0.565571in}}%
\pgfpathlineto{\pgfqpoint{3.193970in}{0.564695in}}%
\pgfpathlineto{\pgfqpoint{3.225283in}{0.563887in}}%
\pgfpathlineto{\pgfqpoint{3.256596in}{0.563140in}}%
\pgfpathlineto{\pgfqpoint{3.287910in}{0.562448in}}%
\pgfpathlineto{\pgfqpoint{3.319223in}{0.561805in}}%
\pgfpathlineto{\pgfqpoint{3.350536in}{0.561207in}}%
\pgfpathlineto{\pgfqpoint{3.381849in}{0.560651in}}%
\pgfpathlineto{\pgfqpoint{3.413162in}{0.560131in}}%
\pgfpathlineto{\pgfqpoint{3.444475in}{0.559646in}}%
\pgfpathlineto{\pgfqpoint{3.475788in}{0.559191in}}%
\pgfpathlineto{\pgfqpoint{3.507101in}{0.558766in}}%
\pgfpathlineto{\pgfqpoint{3.538415in}{0.558366in}}%
\pgfpathlineto{\pgfqpoint{3.569728in}{0.557991in}}%
\pgfpathlineto{\pgfqpoint{3.601041in}{0.557638in}}%
\pgfusepath{stroke}%
\end{pgfscope}%
\begin{pgfscope}%
\pgfsetrectcap%
\pgfsetmiterjoin%
\pgfsetlinewidth{0.803000pt}%
\definecolor{currentstroke}{rgb}{0.000000,0.000000,0.000000}%
\pgfsetstrokecolor{currentstroke}%
\pgfsetdash{}{0pt}%
\pgfpathmoveto{\pgfqpoint{0.501041in}{0.549073in}}%
\pgfpathlineto{\pgfqpoint{0.501041in}{1.704073in}}%
\pgfusepath{stroke}%
\end{pgfscope}%
\begin{pgfscope}%
\pgfsetrectcap%
\pgfsetmiterjoin%
\pgfsetlinewidth{0.803000pt}%
\definecolor{currentstroke}{rgb}{0.000000,0.000000,0.000000}%
\pgfsetstrokecolor{currentstroke}%
\pgfsetdash{}{0pt}%
\pgfpathmoveto{\pgfqpoint{3.601041in}{0.549073in}}%
\pgfpathlineto{\pgfqpoint{3.601041in}{1.704073in}}%
\pgfusepath{stroke}%
\end{pgfscope}%
\begin{pgfscope}%
\pgfsetrectcap%
\pgfsetmiterjoin%
\pgfsetlinewidth{0.803000pt}%
\definecolor{currentstroke}{rgb}{0.000000,0.000000,0.000000}%
\pgfsetstrokecolor{currentstroke}%
\pgfsetdash{}{0pt}%
\pgfpathmoveto{\pgfqpoint{0.501041in}{0.549073in}}%
\pgfpathlineto{\pgfqpoint{3.601041in}{0.549073in}}%
\pgfusepath{stroke}%
\end{pgfscope}%
\begin{pgfscope}%
\pgfsetrectcap%
\pgfsetmiterjoin%
\pgfsetlinewidth{0.803000pt}%
\definecolor{currentstroke}{rgb}{0.000000,0.000000,0.000000}%
\pgfsetstrokecolor{currentstroke}%
\pgfsetdash{}{0pt}%
\pgfpathmoveto{\pgfqpoint{0.501041in}{1.704073in}}%
\pgfpathlineto{\pgfqpoint{3.601041in}{1.704073in}}%
\pgfusepath{stroke}%
\end{pgfscope}%
\begin{pgfscope}%
\pgfsetbuttcap%
\pgfsetmiterjoin%
\definecolor{currentfill}{rgb}{1.000000,1.000000,1.000000}%
\pgfsetfillcolor{currentfill}%
\pgfsetfillopacity{0.800000}%
\pgfsetlinewidth{1.003750pt}%
\definecolor{currentstroke}{rgb}{0.800000,0.800000,0.800000}%
\pgfsetstrokecolor{currentstroke}%
\pgfsetstrokeopacity{0.800000}%
\pgfsetdash{}{0pt}%
\pgfpathmoveto{\pgfqpoint{2.191770in}{1.105925in}}%
\pgfpathlineto{\pgfqpoint{3.484374in}{1.105925in}}%
\pgfpathquadraticcurveto{\pgfqpoint{3.517708in}{1.105925in}}{\pgfqpoint{3.517708in}{1.139259in}}%
\pgfpathlineto{\pgfqpoint{3.517708in}{1.587406in}}%
\pgfpathquadraticcurveto{\pgfqpoint{3.517708in}{1.620739in}}{\pgfqpoint{3.484374in}{1.620739in}}%
\pgfpathlineto{\pgfqpoint{2.191770in}{1.620739in}}%
\pgfpathquadraticcurveto{\pgfqpoint{2.158437in}{1.620739in}}{\pgfqpoint{2.158437in}{1.587406in}}%
\pgfpathlineto{\pgfqpoint{2.158437in}{1.139259in}}%
\pgfpathquadraticcurveto{\pgfqpoint{2.158437in}{1.105925in}}{\pgfqpoint{2.191770in}{1.105925in}}%
\pgfpathlineto{\pgfqpoint{2.191770in}{1.105925in}}%
\pgfpathclose%
\pgfusepath{stroke,fill}%
\end{pgfscope}%
\begin{pgfscope}%
\pgfsetrectcap%
\pgfsetroundjoin%
\pgfsetlinewidth{1.505625pt}%
\definecolor{currentstroke}{rgb}{0.478431,0.701961,0.941176}%
\pgfsetstrokecolor{currentstroke}%
\pgfsetdash{}{0pt}%
\pgfpathmoveto{\pgfqpoint{2.225103in}{1.495739in}}%
\pgfpathlineto{\pgfqpoint{2.391770in}{1.495739in}}%
\pgfpathlineto{\pgfqpoint{2.558437in}{1.495739in}}%
\pgfusepath{stroke}%
\end{pgfscope}%
\begin{pgfscope}%
\definecolor{textcolor}{rgb}{0.000000,0.000000,0.000000}%
\pgfsetstrokecolor{textcolor}%
\pgfsetfillcolor{textcolor}%
\pgftext[x=2.691770in,y=1.437406in,left,base]{\color{textcolor}{\rmfamily\fontsize{12.000000}{14.400000}\selectfont\catcode`\^=\active\def^{\ifmmode\sp\else\^{}\fi}\catcode`\%=\active\def%{\%}Gaussian}}%
\end{pgfscope}%
\begin{pgfscope}%
\pgfsetrectcap%
\pgfsetroundjoin%
\pgfsetlinewidth{1.505625pt}%
\definecolor{currentstroke}{rgb}{0.184314,0.270588,0.360784}%
\pgfsetstrokecolor{currentstroke}%
\pgfsetdash{}{0pt}%
\pgfpathmoveto{\pgfqpoint{2.225103in}{1.263332in}}%
\pgfpathlineto{\pgfqpoint{2.391770in}{1.263332in}}%
\pgfpathlineto{\pgfqpoint{2.558437in}{1.263332in}}%
\pgfusepath{stroke}%
\end{pgfscope}%
\begin{pgfscope}%
\definecolor{textcolor}{rgb}{0.000000,0.000000,0.000000}%
\pgfsetstrokecolor{textcolor}%
\pgfsetfillcolor{textcolor}%
\pgftext[x=2.691770in,y=1.204999in,left,base]{\color{textcolor}{\rmfamily\fontsize{12.000000}{14.400000}\selectfont\catcode`\^=\active\def^{\ifmmode\sp\else\^{}\fi}\catcode`\%=\active\def%{\%}Lorentzian}}%
\end{pgfscope}%
\end{pgfpicture}%
\makeatother%
\endgroup%

    \caption{Comparison of the Gaussian with the Lorentzian \glsfirst{smoothing-kernel}
        for \gls{smoothing-parameter} $=0.05$.}
    \label{fig:5-experiments-haydock-kernel}
\end{figure}
%Estimating \gls{smooth-spectral-density} then becomes the
%trace estimation problem
%\begin{equation}
%    \phi_{\sigma}(t) = \Tr(g_{\sigma}(t\mtx{I}_n - \mtx{A})) = - \frac{1}{n \pi} \Im \left\{ \Tr\left[((t + i\sigma)I - A)^{-1}\right]  \right\}.
%    \label{equ:5-experiments-haydock-trace}
%\end{equation}
%Similarly to the \gls{DGC} method (see \refsec{sec:2-chebyshev-delta-gauss-chebyshev}),
%the Hutchinson's trace estimator with standard Gaussian random vectors
%$\vct{\psi} \in \mathbb{R}^n$ is used to
%approximate the trace
%\begin{equation}
%    \Tr\left[((t + i\sigma)I - A)^{-1}\right] \approx \frac{1}{n_{\Omega}} \sum_{j=1}^{n_{\Omega}} \left( \vct{\psi}_j \right)^{\top} ((t - i\sigma)\mtx{I}_n - \mtx{A})^{-1} \vct{\psi}_j.
%    \label{equ:5-experiments-haydock-hutchinson}
%\end{equation}
%It turns out that each summand in \refequ{5-experiments-haydock-hutchinson} can
%be efficiently evaluated for multiple $t$ by running Lanczos
%on $\mtx{A}$ with starting vector $\vct{\psi}_j$
%\begin{equation}
%    \vct{\psi}_j^{\top} ((t - i\sigma)\mtx{I}_n - \mtx{A})^{-1} \vct{\psi}_j% &\approx \vct{e}_1^{\top} ((t - i\sigma)\mtx{I} - \mtx{H}_k)^{-1} \vct{e}_1 \notag \\
%    \approx \cfrac{1}{(t - i\sigma) - \alpha_1 - \cfrac{\beta_2^2}{(t - i\sigma) - \alpha_2 - \dots}}
%    \label{equ:5-experiments-haydock-recursion}
%\end{equation}

We repeat the same experiments as in \refsec{sec:5-experiments-density-function}
but this time for a Lorentzian kernel with Haydock's method. We
plot the results in \reffig{fig:5-experiments-haydock-convergence-nv} and
\reffig{fig:5-experiments-haydock-convergence-m}, and compare the wall time
between the methods in \reftab{tab:5-experiments-timing-haydock}.\\

\begin{figure}[ht]
    \begin{subfigure}[b]{0.49\columnwidth}
        %% Creator: Matplotlib, PGF backend
%%
%% To include the figure in your LaTeX document, write
%%   \input{<filename>.pgf}
%%
%% Make sure the required packages are loaded in your preamble
%%   \usepackage{pgf}
%%
%% Also ensure that all the required font packages are loaded; for instance,
%% the lmodern package is sometimes necessary when using math font.
%%   \usepackage{lmodern}
%%
%% Figures using additional raster images can only be included by \input if
%% they are in the same directory as the main LaTeX file. For loading figures
%% from other directories you can use the `import` package
%%   \usepackage{import}
%%
%% and then include the figures with
%%   \import{<path to file>}{<filename>.pgf}
%%
%% Matplotlib used the following preamble
%%   \def\mathdefault#1{#1}
%%   \everymath=\expandafter{\the\everymath\displaystyle}
%%   
%%   \usepackage{fontspec}
%%   \setmainfont{DejaVuSerif.ttf}[Path=\detokenize{C:/Users/fabio/Documents/Work/MasterThesis/Rand-SD/.venv/Lib/site-packages/matplotlib/mpl-data/fonts/ttf/}]
%%   \setsansfont{DejaVuSans.ttf}[Path=\detokenize{C:/Users/fabio/Documents/Work/MasterThesis/Rand-SD/.venv/Lib/site-packages/matplotlib/mpl-data/fonts/ttf/}]
%%   \setmonofont{DejaVuSansMono.ttf}[Path=\detokenize{C:/Users/fabio/Documents/Work/MasterThesis/Rand-SD/.venv/Lib/site-packages/matplotlib/mpl-data/fonts/ttf/}]
%%   \makeatletter\@ifpackageloaded{underscore}{}{\usepackage[strings]{underscore}}\makeatother
%%
\begingroup%
\makeatletter%
\begin{pgfpicture}%
\pgfpathrectangle{\pgfpointorigin}{\pgfqpoint{2.712693in}{2.546603in}}%
\pgfusepath{use as bounding box, clip}%
\begin{pgfscope}%
\pgfsetbuttcap%
\pgfsetmiterjoin%
\definecolor{currentfill}{rgb}{1.000000,1.000000,1.000000}%
\pgfsetfillcolor{currentfill}%
\pgfsetlinewidth{0.000000pt}%
\definecolor{currentstroke}{rgb}{1.000000,1.000000,1.000000}%
\pgfsetstrokecolor{currentstroke}%
\pgfsetdash{}{0pt}%
\pgfpathmoveto{\pgfqpoint{0.000000in}{0.000000in}}%
\pgfpathlineto{\pgfqpoint{2.712693in}{0.000000in}}%
\pgfpathlineto{\pgfqpoint{2.712693in}{2.546603in}}%
\pgfpathlineto{\pgfqpoint{0.000000in}{2.546603in}}%
\pgfpathlineto{\pgfqpoint{0.000000in}{0.000000in}}%
\pgfpathclose%
\pgfusepath{fill}%
\end{pgfscope}%
\begin{pgfscope}%
\pgfsetbuttcap%
\pgfsetmiterjoin%
\definecolor{currentfill}{rgb}{1.000000,1.000000,1.000000}%
\pgfsetfillcolor{currentfill}%
\pgfsetlinewidth{0.000000pt}%
\definecolor{currentstroke}{rgb}{0.000000,0.000000,0.000000}%
\pgfsetstrokecolor{currentstroke}%
\pgfsetstrokeopacity{0.000000}%
\pgfsetdash{}{0pt}%
\pgfpathmoveto{\pgfqpoint{0.675193in}{0.521603in}}%
\pgfpathlineto{\pgfqpoint{2.612693in}{0.521603in}}%
\pgfpathlineto{\pgfqpoint{2.612693in}{2.446603in}}%
\pgfpathlineto{\pgfqpoint{0.675193in}{2.446603in}}%
\pgfpathlineto{\pgfqpoint{0.675193in}{0.521603in}}%
\pgfpathclose%
\pgfusepath{fill}%
\end{pgfscope}%
\begin{pgfscope}%
\pgfsetbuttcap%
\pgfsetroundjoin%
\definecolor{currentfill}{rgb}{0.000000,0.000000,0.000000}%
\pgfsetfillcolor{currentfill}%
\pgfsetlinewidth{0.803000pt}%
\definecolor{currentstroke}{rgb}{0.000000,0.000000,0.000000}%
\pgfsetstrokecolor{currentstroke}%
\pgfsetdash{}{0pt}%
\pgfsys@defobject{currentmarker}{\pgfqpoint{0.000000in}{-0.048611in}}{\pgfqpoint{0.000000in}{0.000000in}}{%
\pgfpathmoveto{\pgfqpoint{0.000000in}{0.000000in}}%
\pgfpathlineto{\pgfqpoint{0.000000in}{-0.048611in}}%
\pgfusepath{stroke,fill}%
}%
\begin{pgfscope}%
\pgfsys@transformshift{1.724845in}{0.521603in}%
\pgfsys@useobject{currentmarker}{}%
\end{pgfscope}%
\end{pgfscope}%
\begin{pgfscope}%
\definecolor{textcolor}{rgb}{0.000000,0.000000,0.000000}%
\pgfsetstrokecolor{textcolor}%
\pgfsetfillcolor{textcolor}%
\pgftext[x=1.724845in,y=0.424381in,,top]{\color{textcolor}{\sffamily\fontsize{10.000000}{12.000000}\selectfont\catcode`\^=\active\def^{\ifmmode\sp\else\^{}\fi}\catcode`\%=\active\def%{\%}$\mathdefault{10^{2}}$}}%
\end{pgfscope}%
\begin{pgfscope}%
\pgfsetbuttcap%
\pgfsetroundjoin%
\definecolor{currentfill}{rgb}{0.000000,0.000000,0.000000}%
\pgfsetfillcolor{currentfill}%
\pgfsetlinewidth{0.602250pt}%
\definecolor{currentstroke}{rgb}{0.000000,0.000000,0.000000}%
\pgfsetstrokecolor{currentstroke}%
\pgfsetdash{}{0pt}%
\pgfsys@defobject{currentmarker}{\pgfqpoint{0.000000in}{-0.027778in}}{\pgfqpoint{0.000000in}{0.000000in}}{%
\pgfpathmoveto{\pgfqpoint{0.000000in}{0.000000in}}%
\pgfpathlineto{\pgfqpoint{0.000000in}{-0.027778in}}%
\pgfusepath{stroke,fill}%
}%
\begin{pgfscope}%
\pgfsys@transformshift{0.792961in}{0.521603in}%
\pgfsys@useobject{currentmarker}{}%
\end{pgfscope}%
\end{pgfscope}%
\begin{pgfscope}%
\pgfsetbuttcap%
\pgfsetroundjoin%
\definecolor{currentfill}{rgb}{0.000000,0.000000,0.000000}%
\pgfsetfillcolor{currentfill}%
\pgfsetlinewidth{0.602250pt}%
\definecolor{currentstroke}{rgb}{0.000000,0.000000,0.000000}%
\pgfsetstrokecolor{currentstroke}%
\pgfsetdash{}{0pt}%
\pgfsys@defobject{currentmarker}{\pgfqpoint{0.000000in}{-0.027778in}}{\pgfqpoint{0.000000in}{0.000000in}}{%
\pgfpathmoveto{\pgfqpoint{0.000000in}{0.000000in}}%
\pgfpathlineto{\pgfqpoint{0.000000in}{-0.027778in}}%
\pgfusepath{stroke,fill}%
}%
\begin{pgfscope}%
\pgfsys@transformshift{1.027730in}{0.521603in}%
\pgfsys@useobject{currentmarker}{}%
\end{pgfscope}%
\end{pgfscope}%
\begin{pgfscope}%
\pgfsetbuttcap%
\pgfsetroundjoin%
\definecolor{currentfill}{rgb}{0.000000,0.000000,0.000000}%
\pgfsetfillcolor{currentfill}%
\pgfsetlinewidth{0.602250pt}%
\definecolor{currentstroke}{rgb}{0.000000,0.000000,0.000000}%
\pgfsetstrokecolor{currentstroke}%
\pgfsetdash{}{0pt}%
\pgfsys@defobject{currentmarker}{\pgfqpoint{0.000000in}{-0.027778in}}{\pgfqpoint{0.000000in}{0.000000in}}{%
\pgfpathmoveto{\pgfqpoint{0.000000in}{0.000000in}}%
\pgfpathlineto{\pgfqpoint{0.000000in}{-0.027778in}}%
\pgfusepath{stroke,fill}%
}%
\begin{pgfscope}%
\pgfsys@transformshift{1.194302in}{0.521603in}%
\pgfsys@useobject{currentmarker}{}%
\end{pgfscope}%
\end{pgfscope}%
\begin{pgfscope}%
\pgfsetbuttcap%
\pgfsetroundjoin%
\definecolor{currentfill}{rgb}{0.000000,0.000000,0.000000}%
\pgfsetfillcolor{currentfill}%
\pgfsetlinewidth{0.602250pt}%
\definecolor{currentstroke}{rgb}{0.000000,0.000000,0.000000}%
\pgfsetstrokecolor{currentstroke}%
\pgfsetdash{}{0pt}%
\pgfsys@defobject{currentmarker}{\pgfqpoint{0.000000in}{-0.027778in}}{\pgfqpoint{0.000000in}{0.000000in}}{%
\pgfpathmoveto{\pgfqpoint{0.000000in}{0.000000in}}%
\pgfpathlineto{\pgfqpoint{0.000000in}{-0.027778in}}%
\pgfusepath{stroke,fill}%
}%
\begin{pgfscope}%
\pgfsys@transformshift{1.323505in}{0.521603in}%
\pgfsys@useobject{currentmarker}{}%
\end{pgfscope}%
\end{pgfscope}%
\begin{pgfscope}%
\pgfsetbuttcap%
\pgfsetroundjoin%
\definecolor{currentfill}{rgb}{0.000000,0.000000,0.000000}%
\pgfsetfillcolor{currentfill}%
\pgfsetlinewidth{0.602250pt}%
\definecolor{currentstroke}{rgb}{0.000000,0.000000,0.000000}%
\pgfsetstrokecolor{currentstroke}%
\pgfsetdash{}{0pt}%
\pgfsys@defobject{currentmarker}{\pgfqpoint{0.000000in}{-0.027778in}}{\pgfqpoint{0.000000in}{0.000000in}}{%
\pgfpathmoveto{\pgfqpoint{0.000000in}{0.000000in}}%
\pgfpathlineto{\pgfqpoint{0.000000in}{-0.027778in}}%
\pgfusepath{stroke,fill}%
}%
\begin{pgfscope}%
\pgfsys@transformshift{1.429071in}{0.521603in}%
\pgfsys@useobject{currentmarker}{}%
\end{pgfscope}%
\end{pgfscope}%
\begin{pgfscope}%
\pgfsetbuttcap%
\pgfsetroundjoin%
\definecolor{currentfill}{rgb}{0.000000,0.000000,0.000000}%
\pgfsetfillcolor{currentfill}%
\pgfsetlinewidth{0.602250pt}%
\definecolor{currentstroke}{rgb}{0.000000,0.000000,0.000000}%
\pgfsetstrokecolor{currentstroke}%
\pgfsetdash{}{0pt}%
\pgfsys@defobject{currentmarker}{\pgfqpoint{0.000000in}{-0.027778in}}{\pgfqpoint{0.000000in}{0.000000in}}{%
\pgfpathmoveto{\pgfqpoint{0.000000in}{0.000000in}}%
\pgfpathlineto{\pgfqpoint{0.000000in}{-0.027778in}}%
\pgfusepath{stroke,fill}%
}%
\begin{pgfscope}%
\pgfsys@transformshift{1.518326in}{0.521603in}%
\pgfsys@useobject{currentmarker}{}%
\end{pgfscope}%
\end{pgfscope}%
\begin{pgfscope}%
\pgfsetbuttcap%
\pgfsetroundjoin%
\definecolor{currentfill}{rgb}{0.000000,0.000000,0.000000}%
\pgfsetfillcolor{currentfill}%
\pgfsetlinewidth{0.602250pt}%
\definecolor{currentstroke}{rgb}{0.000000,0.000000,0.000000}%
\pgfsetstrokecolor{currentstroke}%
\pgfsetdash{}{0pt}%
\pgfsys@defobject{currentmarker}{\pgfqpoint{0.000000in}{-0.027778in}}{\pgfqpoint{0.000000in}{0.000000in}}{%
\pgfpathmoveto{\pgfqpoint{0.000000in}{0.000000in}}%
\pgfpathlineto{\pgfqpoint{0.000000in}{-0.027778in}}%
\pgfusepath{stroke,fill}%
}%
\begin{pgfscope}%
\pgfsys@transformshift{1.595642in}{0.521603in}%
\pgfsys@useobject{currentmarker}{}%
\end{pgfscope}%
\end{pgfscope}%
\begin{pgfscope}%
\pgfsetbuttcap%
\pgfsetroundjoin%
\definecolor{currentfill}{rgb}{0.000000,0.000000,0.000000}%
\pgfsetfillcolor{currentfill}%
\pgfsetlinewidth{0.602250pt}%
\definecolor{currentstroke}{rgb}{0.000000,0.000000,0.000000}%
\pgfsetstrokecolor{currentstroke}%
\pgfsetdash{}{0pt}%
\pgfsys@defobject{currentmarker}{\pgfqpoint{0.000000in}{-0.027778in}}{\pgfqpoint{0.000000in}{0.000000in}}{%
\pgfpathmoveto{\pgfqpoint{0.000000in}{0.000000in}}%
\pgfpathlineto{\pgfqpoint{0.000000in}{-0.027778in}}%
\pgfusepath{stroke,fill}%
}%
\begin{pgfscope}%
\pgfsys@transformshift{1.663840in}{0.521603in}%
\pgfsys@useobject{currentmarker}{}%
\end{pgfscope}%
\end{pgfscope}%
\begin{pgfscope}%
\pgfsetbuttcap%
\pgfsetroundjoin%
\definecolor{currentfill}{rgb}{0.000000,0.000000,0.000000}%
\pgfsetfillcolor{currentfill}%
\pgfsetlinewidth{0.602250pt}%
\definecolor{currentstroke}{rgb}{0.000000,0.000000,0.000000}%
\pgfsetstrokecolor{currentstroke}%
\pgfsetdash{}{0pt}%
\pgfsys@defobject{currentmarker}{\pgfqpoint{0.000000in}{-0.027778in}}{\pgfqpoint{0.000000in}{0.000000in}}{%
\pgfpathmoveto{\pgfqpoint{0.000000in}{0.000000in}}%
\pgfpathlineto{\pgfqpoint{0.000000in}{-0.027778in}}%
\pgfusepath{stroke,fill}%
}%
\begin{pgfscope}%
\pgfsys@transformshift{2.126186in}{0.521603in}%
\pgfsys@useobject{currentmarker}{}%
\end{pgfscope}%
\end{pgfscope}%
\begin{pgfscope}%
\pgfsetbuttcap%
\pgfsetroundjoin%
\definecolor{currentfill}{rgb}{0.000000,0.000000,0.000000}%
\pgfsetfillcolor{currentfill}%
\pgfsetlinewidth{0.602250pt}%
\definecolor{currentstroke}{rgb}{0.000000,0.000000,0.000000}%
\pgfsetstrokecolor{currentstroke}%
\pgfsetdash{}{0pt}%
\pgfsys@defobject{currentmarker}{\pgfqpoint{0.000000in}{-0.027778in}}{\pgfqpoint{0.000000in}{0.000000in}}{%
\pgfpathmoveto{\pgfqpoint{0.000000in}{0.000000in}}%
\pgfpathlineto{\pgfqpoint{0.000000in}{-0.027778in}}%
\pgfusepath{stroke,fill}%
}%
\begin{pgfscope}%
\pgfsys@transformshift{2.360956in}{0.521603in}%
\pgfsys@useobject{currentmarker}{}%
\end{pgfscope}%
\end{pgfscope}%
\begin{pgfscope}%
\pgfsetbuttcap%
\pgfsetroundjoin%
\definecolor{currentfill}{rgb}{0.000000,0.000000,0.000000}%
\pgfsetfillcolor{currentfill}%
\pgfsetlinewidth{0.602250pt}%
\definecolor{currentstroke}{rgb}{0.000000,0.000000,0.000000}%
\pgfsetstrokecolor{currentstroke}%
\pgfsetdash{}{0pt}%
\pgfsys@defobject{currentmarker}{\pgfqpoint{0.000000in}{-0.027778in}}{\pgfqpoint{0.000000in}{0.000000in}}{%
\pgfpathmoveto{\pgfqpoint{0.000000in}{0.000000in}}%
\pgfpathlineto{\pgfqpoint{0.000000in}{-0.027778in}}%
\pgfusepath{stroke,fill}%
}%
\begin{pgfscope}%
\pgfsys@transformshift{2.527527in}{0.521603in}%
\pgfsys@useobject{currentmarker}{}%
\end{pgfscope}%
\end{pgfscope}%
\begin{pgfscope}%
\definecolor{textcolor}{rgb}{0.000000,0.000000,0.000000}%
\pgfsetstrokecolor{textcolor}%
\pgfsetfillcolor{textcolor}%
\pgftext[x=1.643943in,y=0.234413in,,top]{\color{textcolor}{\sffamily\fontsize{10.000000}{12.000000}\selectfont\catcode`\^=\active\def^{\ifmmode\sp\else\^{}\fi}\catcode`\%=\active\def%{\%}$n_v$}}%
\end{pgfscope}%
\begin{pgfscope}%
\pgfsetbuttcap%
\pgfsetroundjoin%
\definecolor{currentfill}{rgb}{0.000000,0.000000,0.000000}%
\pgfsetfillcolor{currentfill}%
\pgfsetlinewidth{0.803000pt}%
\definecolor{currentstroke}{rgb}{0.000000,0.000000,0.000000}%
\pgfsetstrokecolor{currentstroke}%
\pgfsetdash{}{0pt}%
\pgfsys@defobject{currentmarker}{\pgfqpoint{-0.048611in}{0.000000in}}{\pgfqpoint{-0.000000in}{0.000000in}}{%
\pgfpathmoveto{\pgfqpoint{-0.000000in}{0.000000in}}%
\pgfpathlineto{\pgfqpoint{-0.048611in}{0.000000in}}%
\pgfusepath{stroke,fill}%
}%
\begin{pgfscope}%
\pgfsys@transformshift{0.675193in}{1.648566in}%
\pgfsys@useobject{currentmarker}{}%
\end{pgfscope}%
\end{pgfscope}%
\begin{pgfscope}%
\definecolor{textcolor}{rgb}{0.000000,0.000000,0.000000}%
\pgfsetstrokecolor{textcolor}%
\pgfsetfillcolor{textcolor}%
\pgftext[x=0.289968in, y=1.595804in, left, base]{\color{textcolor}{\sffamily\fontsize{10.000000}{12.000000}\selectfont\catcode`\^=\active\def^{\ifmmode\sp\else\^{}\fi}\catcode`\%=\active\def%{\%}$\mathdefault{10^{-1}}$}}%
\end{pgfscope}%
\begin{pgfscope}%
\pgfsetbuttcap%
\pgfsetroundjoin%
\definecolor{currentfill}{rgb}{0.000000,0.000000,0.000000}%
\pgfsetfillcolor{currentfill}%
\pgfsetlinewidth{0.602250pt}%
\definecolor{currentstroke}{rgb}{0.000000,0.000000,0.000000}%
\pgfsetstrokecolor{currentstroke}%
\pgfsetdash{}{0pt}%
\pgfsys@defobject{currentmarker}{\pgfqpoint{-0.027778in}{0.000000in}}{\pgfqpoint{-0.000000in}{0.000000in}}{%
\pgfpathmoveto{\pgfqpoint{-0.000000in}{0.000000in}}%
\pgfpathlineto{\pgfqpoint{-0.027778in}{0.000000in}}%
\pgfusepath{stroke,fill}%
}%
\begin{pgfscope}%
\pgfsys@transformshift{0.675193in}{0.831517in}%
\pgfsys@useobject{currentmarker}{}%
\end{pgfscope}%
\end{pgfscope}%
\begin{pgfscope}%
\pgfsetbuttcap%
\pgfsetroundjoin%
\definecolor{currentfill}{rgb}{0.000000,0.000000,0.000000}%
\pgfsetfillcolor{currentfill}%
\pgfsetlinewidth{0.602250pt}%
\definecolor{currentstroke}{rgb}{0.000000,0.000000,0.000000}%
\pgfsetstrokecolor{currentstroke}%
\pgfsetdash{}{0pt}%
\pgfsys@defobject{currentmarker}{\pgfqpoint{-0.027778in}{0.000000in}}{\pgfqpoint{-0.000000in}{0.000000in}}{%
\pgfpathmoveto{\pgfqpoint{-0.000000in}{0.000000in}}%
\pgfpathlineto{\pgfqpoint{-0.027778in}{0.000000in}}%
\pgfusepath{stroke,fill}%
}%
\begin{pgfscope}%
\pgfsys@transformshift{0.675193in}{1.037356in}%
\pgfsys@useobject{currentmarker}{}%
\end{pgfscope}%
\end{pgfscope}%
\begin{pgfscope}%
\pgfsetbuttcap%
\pgfsetroundjoin%
\definecolor{currentfill}{rgb}{0.000000,0.000000,0.000000}%
\pgfsetfillcolor{currentfill}%
\pgfsetlinewidth{0.602250pt}%
\definecolor{currentstroke}{rgb}{0.000000,0.000000,0.000000}%
\pgfsetstrokecolor{currentstroke}%
\pgfsetdash{}{0pt}%
\pgfsys@defobject{currentmarker}{\pgfqpoint{-0.027778in}{0.000000in}}{\pgfqpoint{-0.000000in}{0.000000in}}{%
\pgfpathmoveto{\pgfqpoint{-0.000000in}{0.000000in}}%
\pgfpathlineto{\pgfqpoint{-0.027778in}{0.000000in}}%
\pgfusepath{stroke,fill}%
}%
\begin{pgfscope}%
\pgfsys@transformshift{0.675193in}{1.183401in}%
\pgfsys@useobject{currentmarker}{}%
\end{pgfscope}%
\end{pgfscope}%
\begin{pgfscope}%
\pgfsetbuttcap%
\pgfsetroundjoin%
\definecolor{currentfill}{rgb}{0.000000,0.000000,0.000000}%
\pgfsetfillcolor{currentfill}%
\pgfsetlinewidth{0.602250pt}%
\definecolor{currentstroke}{rgb}{0.000000,0.000000,0.000000}%
\pgfsetstrokecolor{currentstroke}%
\pgfsetdash{}{0pt}%
\pgfsys@defobject{currentmarker}{\pgfqpoint{-0.027778in}{0.000000in}}{\pgfqpoint{-0.000000in}{0.000000in}}{%
\pgfpathmoveto{\pgfqpoint{-0.000000in}{0.000000in}}%
\pgfpathlineto{\pgfqpoint{-0.027778in}{0.000000in}}%
\pgfusepath{stroke,fill}%
}%
\begin{pgfscope}%
\pgfsys@transformshift{0.675193in}{1.296682in}%
\pgfsys@useobject{currentmarker}{}%
\end{pgfscope}%
\end{pgfscope}%
\begin{pgfscope}%
\pgfsetbuttcap%
\pgfsetroundjoin%
\definecolor{currentfill}{rgb}{0.000000,0.000000,0.000000}%
\pgfsetfillcolor{currentfill}%
\pgfsetlinewidth{0.602250pt}%
\definecolor{currentstroke}{rgb}{0.000000,0.000000,0.000000}%
\pgfsetstrokecolor{currentstroke}%
\pgfsetdash{}{0pt}%
\pgfsys@defobject{currentmarker}{\pgfqpoint{-0.027778in}{0.000000in}}{\pgfqpoint{-0.000000in}{0.000000in}}{%
\pgfpathmoveto{\pgfqpoint{-0.000000in}{0.000000in}}%
\pgfpathlineto{\pgfqpoint{-0.027778in}{0.000000in}}%
\pgfusepath{stroke,fill}%
}%
\begin{pgfscope}%
\pgfsys@transformshift{0.675193in}{1.389240in}%
\pgfsys@useobject{currentmarker}{}%
\end{pgfscope}%
\end{pgfscope}%
\begin{pgfscope}%
\pgfsetbuttcap%
\pgfsetroundjoin%
\definecolor{currentfill}{rgb}{0.000000,0.000000,0.000000}%
\pgfsetfillcolor{currentfill}%
\pgfsetlinewidth{0.602250pt}%
\definecolor{currentstroke}{rgb}{0.000000,0.000000,0.000000}%
\pgfsetstrokecolor{currentstroke}%
\pgfsetdash{}{0pt}%
\pgfsys@defobject{currentmarker}{\pgfqpoint{-0.027778in}{0.000000in}}{\pgfqpoint{-0.000000in}{0.000000in}}{%
\pgfpathmoveto{\pgfqpoint{-0.000000in}{0.000000in}}%
\pgfpathlineto{\pgfqpoint{-0.027778in}{0.000000in}}%
\pgfusepath{stroke,fill}%
}%
\begin{pgfscope}%
\pgfsys@transformshift{0.675193in}{1.467496in}%
\pgfsys@useobject{currentmarker}{}%
\end{pgfscope}%
\end{pgfscope}%
\begin{pgfscope}%
\pgfsetbuttcap%
\pgfsetroundjoin%
\definecolor{currentfill}{rgb}{0.000000,0.000000,0.000000}%
\pgfsetfillcolor{currentfill}%
\pgfsetlinewidth{0.602250pt}%
\definecolor{currentstroke}{rgb}{0.000000,0.000000,0.000000}%
\pgfsetstrokecolor{currentstroke}%
\pgfsetdash{}{0pt}%
\pgfsys@defobject{currentmarker}{\pgfqpoint{-0.027778in}{0.000000in}}{\pgfqpoint{-0.000000in}{0.000000in}}{%
\pgfpathmoveto{\pgfqpoint{-0.000000in}{0.000000in}}%
\pgfpathlineto{\pgfqpoint{-0.027778in}{0.000000in}}%
\pgfusepath{stroke,fill}%
}%
\begin{pgfscope}%
\pgfsys@transformshift{0.675193in}{1.535285in}%
\pgfsys@useobject{currentmarker}{}%
\end{pgfscope}%
\end{pgfscope}%
\begin{pgfscope}%
\pgfsetbuttcap%
\pgfsetroundjoin%
\definecolor{currentfill}{rgb}{0.000000,0.000000,0.000000}%
\pgfsetfillcolor{currentfill}%
\pgfsetlinewidth{0.602250pt}%
\definecolor{currentstroke}{rgb}{0.000000,0.000000,0.000000}%
\pgfsetstrokecolor{currentstroke}%
\pgfsetdash{}{0pt}%
\pgfsys@defobject{currentmarker}{\pgfqpoint{-0.027778in}{0.000000in}}{\pgfqpoint{-0.000000in}{0.000000in}}{%
\pgfpathmoveto{\pgfqpoint{-0.000000in}{0.000000in}}%
\pgfpathlineto{\pgfqpoint{-0.027778in}{0.000000in}}%
\pgfusepath{stroke,fill}%
}%
\begin{pgfscope}%
\pgfsys@transformshift{0.675193in}{1.595078in}%
\pgfsys@useobject{currentmarker}{}%
\end{pgfscope}%
\end{pgfscope}%
\begin{pgfscope}%
\pgfsetbuttcap%
\pgfsetroundjoin%
\definecolor{currentfill}{rgb}{0.000000,0.000000,0.000000}%
\pgfsetfillcolor{currentfill}%
\pgfsetlinewidth{0.602250pt}%
\definecolor{currentstroke}{rgb}{0.000000,0.000000,0.000000}%
\pgfsetstrokecolor{currentstroke}%
\pgfsetdash{}{0pt}%
\pgfsys@defobject{currentmarker}{\pgfqpoint{-0.027778in}{0.000000in}}{\pgfqpoint{-0.000000in}{0.000000in}}{%
\pgfpathmoveto{\pgfqpoint{-0.000000in}{0.000000in}}%
\pgfpathlineto{\pgfqpoint{-0.027778in}{0.000000in}}%
\pgfusepath{stroke,fill}%
}%
\begin{pgfscope}%
\pgfsys@transformshift{0.675193in}{2.000450in}%
\pgfsys@useobject{currentmarker}{}%
\end{pgfscope}%
\end{pgfscope}%
\begin{pgfscope}%
\pgfsetbuttcap%
\pgfsetroundjoin%
\definecolor{currentfill}{rgb}{0.000000,0.000000,0.000000}%
\pgfsetfillcolor{currentfill}%
\pgfsetlinewidth{0.602250pt}%
\definecolor{currentstroke}{rgb}{0.000000,0.000000,0.000000}%
\pgfsetstrokecolor{currentstroke}%
\pgfsetdash{}{0pt}%
\pgfsys@defobject{currentmarker}{\pgfqpoint{-0.027778in}{0.000000in}}{\pgfqpoint{-0.000000in}{0.000000in}}{%
\pgfpathmoveto{\pgfqpoint{-0.000000in}{0.000000in}}%
\pgfpathlineto{\pgfqpoint{-0.027778in}{0.000000in}}%
\pgfusepath{stroke,fill}%
}%
\begin{pgfscope}%
\pgfsys@transformshift{0.675193in}{2.206289in}%
\pgfsys@useobject{currentmarker}{}%
\end{pgfscope}%
\end{pgfscope}%
\begin{pgfscope}%
\pgfsetbuttcap%
\pgfsetroundjoin%
\definecolor{currentfill}{rgb}{0.000000,0.000000,0.000000}%
\pgfsetfillcolor{currentfill}%
\pgfsetlinewidth{0.602250pt}%
\definecolor{currentstroke}{rgb}{0.000000,0.000000,0.000000}%
\pgfsetstrokecolor{currentstroke}%
\pgfsetdash{}{0pt}%
\pgfsys@defobject{currentmarker}{\pgfqpoint{-0.027778in}{0.000000in}}{\pgfqpoint{-0.000000in}{0.000000in}}{%
\pgfpathmoveto{\pgfqpoint{-0.000000in}{0.000000in}}%
\pgfpathlineto{\pgfqpoint{-0.027778in}{0.000000in}}%
\pgfusepath{stroke,fill}%
}%
\begin{pgfscope}%
\pgfsys@transformshift{0.675193in}{2.352334in}%
\pgfsys@useobject{currentmarker}{}%
\end{pgfscope}%
\end{pgfscope}%
\begin{pgfscope}%
\definecolor{textcolor}{rgb}{0.000000,0.000000,0.000000}%
\pgfsetstrokecolor{textcolor}%
\pgfsetfillcolor{textcolor}%
\pgftext[x=0.234413in,y=1.484103in,,bottom,rotate=90.000000]{\color{textcolor}{\sffamily\fontsize{10.000000}{12.000000}\selectfont\catcode`\^=\active\def^{\ifmmode\sp\else\^{}\fi}\catcode`\%=\active\def%{\%}$L^1$ error}}%
\end{pgfscope}%
\begin{pgfscope}%
\pgfpathrectangle{\pgfqpoint{0.675193in}{0.521603in}}{\pgfqpoint{1.937500in}{1.925000in}}%
\pgfusepath{clip}%
\pgfsetrectcap%
\pgfsetroundjoin%
\pgfsetlinewidth{1.003750pt}%
\definecolor{currentstroke}{rgb}{0.001462,0.000466,0.013866}%
\pgfsetstrokecolor{currentstroke}%
\pgfsetdash{}{0pt}%
\pgfpathmoveto{\pgfqpoint{0.763261in}{1.316425in}}%
\pgfpathlineto{\pgfqpoint{1.133297in}{1.155024in}}%
\pgfpathlineto{\pgfqpoint{1.484257in}{1.094590in}}%
\pgfpathlineto{\pgfqpoint{1.830412in}{0.902305in}}%
\pgfpathlineto{\pgfqpoint{2.176084in}{0.751658in}}%
\pgfpathlineto{\pgfqpoint{2.524625in}{0.609103in}}%
\pgfusepath{stroke}%
\end{pgfscope}%
\begin{pgfscope}%
\pgfpathrectangle{\pgfqpoint{0.675193in}{0.521603in}}{\pgfqpoint{1.937500in}{1.925000in}}%
\pgfusepath{clip}%
\pgfsetbuttcap%
\pgfsetroundjoin%
\definecolor{currentfill}{rgb}{0.001462,0.000466,0.013866}%
\pgfsetfillcolor{currentfill}%
\pgfsetlinewidth{1.003750pt}%
\definecolor{currentstroke}{rgb}{0.001462,0.000466,0.013866}%
\pgfsetstrokecolor{currentstroke}%
\pgfsetdash{}{0pt}%
\pgfsys@defobject{currentmarker}{\pgfqpoint{-0.020833in}{-0.020833in}}{\pgfqpoint{0.020833in}{0.020833in}}{%
\pgfpathmoveto{\pgfqpoint{0.000000in}{-0.020833in}}%
\pgfpathcurveto{\pgfqpoint{0.005525in}{-0.020833in}}{\pgfqpoint{0.010825in}{-0.018638in}}{\pgfqpoint{0.014731in}{-0.014731in}}%
\pgfpathcurveto{\pgfqpoint{0.018638in}{-0.010825in}}{\pgfqpoint{0.020833in}{-0.005525in}}{\pgfqpoint{0.020833in}{0.000000in}}%
\pgfpathcurveto{\pgfqpoint{0.020833in}{0.005525in}}{\pgfqpoint{0.018638in}{0.010825in}}{\pgfqpoint{0.014731in}{0.014731in}}%
\pgfpathcurveto{\pgfqpoint{0.010825in}{0.018638in}}{\pgfqpoint{0.005525in}{0.020833in}}{\pgfqpoint{0.000000in}{0.020833in}}%
\pgfpathcurveto{\pgfqpoint{-0.005525in}{0.020833in}}{\pgfqpoint{-0.010825in}{0.018638in}}{\pgfqpoint{-0.014731in}{0.014731in}}%
\pgfpathcurveto{\pgfqpoint{-0.018638in}{0.010825in}}{\pgfqpoint{-0.020833in}{0.005525in}}{\pgfqpoint{-0.020833in}{0.000000in}}%
\pgfpathcurveto{\pgfqpoint{-0.020833in}{-0.005525in}}{\pgfqpoint{-0.018638in}{-0.010825in}}{\pgfqpoint{-0.014731in}{-0.014731in}}%
\pgfpathcurveto{\pgfqpoint{-0.010825in}{-0.018638in}}{\pgfqpoint{-0.005525in}{-0.020833in}}{\pgfqpoint{0.000000in}{-0.020833in}}%
\pgfpathlineto{\pgfqpoint{0.000000in}{-0.020833in}}%
\pgfpathclose%
\pgfusepath{stroke,fill}%
}%
\begin{pgfscope}%
\pgfsys@transformshift{0.763261in}{1.316425in}%
\pgfsys@useobject{currentmarker}{}%
\end{pgfscope}%
\begin{pgfscope}%
\pgfsys@transformshift{1.133297in}{1.155024in}%
\pgfsys@useobject{currentmarker}{}%
\end{pgfscope}%
\begin{pgfscope}%
\pgfsys@transformshift{1.484257in}{1.094590in}%
\pgfsys@useobject{currentmarker}{}%
\end{pgfscope}%
\begin{pgfscope}%
\pgfsys@transformshift{1.830412in}{0.902305in}%
\pgfsys@useobject{currentmarker}{}%
\end{pgfscope}%
\begin{pgfscope}%
\pgfsys@transformshift{2.176084in}{0.751658in}%
\pgfsys@useobject{currentmarker}{}%
\end{pgfscope}%
\begin{pgfscope}%
\pgfsys@transformshift{2.524625in}{0.609103in}%
\pgfsys@useobject{currentmarker}{}%
\end{pgfscope}%
\end{pgfscope}%
\begin{pgfscope}%
\pgfpathrectangle{\pgfqpoint{0.675193in}{0.521603in}}{\pgfqpoint{1.937500in}{1.925000in}}%
\pgfusepath{clip}%
\pgfsetrectcap%
\pgfsetroundjoin%
\pgfsetlinewidth{1.003750pt}%
\definecolor{currentstroke}{rgb}{0.445163,0.122724,0.506901}%
\pgfsetstrokecolor{currentstroke}%
\pgfsetdash{}{0pt}%
\pgfpathmoveto{\pgfqpoint{0.763261in}{2.359103in}}%
\pgfpathlineto{\pgfqpoint{1.133297in}{2.053608in}}%
\pgfpathlineto{\pgfqpoint{1.484257in}{1.686153in}}%
\pgfpathlineto{\pgfqpoint{1.830412in}{1.653467in}}%
\pgfpathlineto{\pgfqpoint{2.176084in}{1.723462in}}%
\pgfpathlineto{\pgfqpoint{2.524625in}{1.783018in}}%
\pgfusepath{stroke}%
\end{pgfscope}%
\begin{pgfscope}%
\pgfpathrectangle{\pgfqpoint{0.675193in}{0.521603in}}{\pgfqpoint{1.937500in}{1.925000in}}%
\pgfusepath{clip}%
\pgfsetbuttcap%
\pgfsetroundjoin%
\definecolor{currentfill}{rgb}{0.445163,0.122724,0.506901}%
\pgfsetfillcolor{currentfill}%
\pgfsetlinewidth{1.003750pt}%
\definecolor{currentstroke}{rgb}{0.445163,0.122724,0.506901}%
\pgfsetstrokecolor{currentstroke}%
\pgfsetdash{}{0pt}%
\pgfsys@defobject{currentmarker}{\pgfqpoint{-0.020833in}{-0.020833in}}{\pgfqpoint{0.020833in}{0.020833in}}{%
\pgfpathmoveto{\pgfqpoint{0.000000in}{-0.020833in}}%
\pgfpathcurveto{\pgfqpoint{0.005525in}{-0.020833in}}{\pgfqpoint{0.010825in}{-0.018638in}}{\pgfqpoint{0.014731in}{-0.014731in}}%
\pgfpathcurveto{\pgfqpoint{0.018638in}{-0.010825in}}{\pgfqpoint{0.020833in}{-0.005525in}}{\pgfqpoint{0.020833in}{0.000000in}}%
\pgfpathcurveto{\pgfqpoint{0.020833in}{0.005525in}}{\pgfqpoint{0.018638in}{0.010825in}}{\pgfqpoint{0.014731in}{0.014731in}}%
\pgfpathcurveto{\pgfqpoint{0.010825in}{0.018638in}}{\pgfqpoint{0.005525in}{0.020833in}}{\pgfqpoint{0.000000in}{0.020833in}}%
\pgfpathcurveto{\pgfqpoint{-0.005525in}{0.020833in}}{\pgfqpoint{-0.010825in}{0.018638in}}{\pgfqpoint{-0.014731in}{0.014731in}}%
\pgfpathcurveto{\pgfqpoint{-0.018638in}{0.010825in}}{\pgfqpoint{-0.020833in}{0.005525in}}{\pgfqpoint{-0.020833in}{0.000000in}}%
\pgfpathcurveto{\pgfqpoint{-0.020833in}{-0.005525in}}{\pgfqpoint{-0.018638in}{-0.010825in}}{\pgfqpoint{-0.014731in}{-0.014731in}}%
\pgfpathcurveto{\pgfqpoint{-0.010825in}{-0.018638in}}{\pgfqpoint{-0.005525in}{-0.020833in}}{\pgfqpoint{0.000000in}{-0.020833in}}%
\pgfpathlineto{\pgfqpoint{0.000000in}{-0.020833in}}%
\pgfpathclose%
\pgfusepath{stroke,fill}%
}%
\begin{pgfscope}%
\pgfsys@transformshift{0.763261in}{2.359103in}%
\pgfsys@useobject{currentmarker}{}%
\end{pgfscope}%
\begin{pgfscope}%
\pgfsys@transformshift{1.133297in}{2.053608in}%
\pgfsys@useobject{currentmarker}{}%
\end{pgfscope}%
\begin{pgfscope}%
\pgfsys@transformshift{1.484257in}{1.686153in}%
\pgfsys@useobject{currentmarker}{}%
\end{pgfscope}%
\begin{pgfscope}%
\pgfsys@transformshift{1.830412in}{1.653467in}%
\pgfsys@useobject{currentmarker}{}%
\end{pgfscope}%
\begin{pgfscope}%
\pgfsys@transformshift{2.176084in}{1.723462in}%
\pgfsys@useobject{currentmarker}{}%
\end{pgfscope}%
\begin{pgfscope}%
\pgfsys@transformshift{2.524625in}{1.783018in}%
\pgfsys@useobject{currentmarker}{}%
\end{pgfscope}%
\end{pgfscope}%
\begin{pgfscope}%
\pgfpathrectangle{\pgfqpoint{0.675193in}{0.521603in}}{\pgfqpoint{1.937500in}{1.925000in}}%
\pgfusepath{clip}%
\pgfsetrectcap%
\pgfsetroundjoin%
\pgfsetlinewidth{1.003750pt}%
\definecolor{currentstroke}{rgb}{0.944006,0.377643,0.365136}%
\pgfsetstrokecolor{currentstroke}%
\pgfsetdash{}{0pt}%
\pgfpathmoveto{\pgfqpoint{0.763261in}{1.547985in}}%
\pgfpathlineto{\pgfqpoint{1.133297in}{1.375559in}}%
\pgfpathlineto{\pgfqpoint{1.484257in}{1.365442in}}%
\pgfpathlineto{\pgfqpoint{1.830412in}{1.389717in}}%
\pgfpathlineto{\pgfqpoint{2.176084in}{1.378706in}}%
\pgfpathlineto{\pgfqpoint{2.524625in}{1.375851in}}%
\pgfusepath{stroke}%
\end{pgfscope}%
\begin{pgfscope}%
\pgfpathrectangle{\pgfqpoint{0.675193in}{0.521603in}}{\pgfqpoint{1.937500in}{1.925000in}}%
\pgfusepath{clip}%
\pgfsetbuttcap%
\pgfsetroundjoin%
\definecolor{currentfill}{rgb}{0.944006,0.377643,0.365136}%
\pgfsetfillcolor{currentfill}%
\pgfsetlinewidth{1.003750pt}%
\definecolor{currentstroke}{rgb}{0.944006,0.377643,0.365136}%
\pgfsetstrokecolor{currentstroke}%
\pgfsetdash{}{0pt}%
\pgfsys@defobject{currentmarker}{\pgfqpoint{-0.020833in}{-0.020833in}}{\pgfqpoint{0.020833in}{0.020833in}}{%
\pgfpathmoveto{\pgfqpoint{0.000000in}{-0.020833in}}%
\pgfpathcurveto{\pgfqpoint{0.005525in}{-0.020833in}}{\pgfqpoint{0.010825in}{-0.018638in}}{\pgfqpoint{0.014731in}{-0.014731in}}%
\pgfpathcurveto{\pgfqpoint{0.018638in}{-0.010825in}}{\pgfqpoint{0.020833in}{-0.005525in}}{\pgfqpoint{0.020833in}{0.000000in}}%
\pgfpathcurveto{\pgfqpoint{0.020833in}{0.005525in}}{\pgfqpoint{0.018638in}{0.010825in}}{\pgfqpoint{0.014731in}{0.014731in}}%
\pgfpathcurveto{\pgfqpoint{0.010825in}{0.018638in}}{\pgfqpoint{0.005525in}{0.020833in}}{\pgfqpoint{0.000000in}{0.020833in}}%
\pgfpathcurveto{\pgfqpoint{-0.005525in}{0.020833in}}{\pgfqpoint{-0.010825in}{0.018638in}}{\pgfqpoint{-0.014731in}{0.014731in}}%
\pgfpathcurveto{\pgfqpoint{-0.018638in}{0.010825in}}{\pgfqpoint{-0.020833in}{0.005525in}}{\pgfqpoint{-0.020833in}{0.000000in}}%
\pgfpathcurveto{\pgfqpoint{-0.020833in}{-0.005525in}}{\pgfqpoint{-0.018638in}{-0.010825in}}{\pgfqpoint{-0.014731in}{-0.014731in}}%
\pgfpathcurveto{\pgfqpoint{-0.010825in}{-0.018638in}}{\pgfqpoint{-0.005525in}{-0.020833in}}{\pgfqpoint{0.000000in}{-0.020833in}}%
\pgfpathlineto{\pgfqpoint{0.000000in}{-0.020833in}}%
\pgfpathclose%
\pgfusepath{stroke,fill}%
}%
\begin{pgfscope}%
\pgfsys@transformshift{0.763261in}{1.547985in}%
\pgfsys@useobject{currentmarker}{}%
\end{pgfscope}%
\begin{pgfscope}%
\pgfsys@transformshift{1.133297in}{1.375559in}%
\pgfsys@useobject{currentmarker}{}%
\end{pgfscope}%
\begin{pgfscope}%
\pgfsys@transformshift{1.484257in}{1.365442in}%
\pgfsys@useobject{currentmarker}{}%
\end{pgfscope}%
\begin{pgfscope}%
\pgfsys@transformshift{1.830412in}{1.389717in}%
\pgfsys@useobject{currentmarker}{}%
\end{pgfscope}%
\begin{pgfscope}%
\pgfsys@transformshift{2.176084in}{1.378706in}%
\pgfsys@useobject{currentmarker}{}%
\end{pgfscope}%
\begin{pgfscope}%
\pgfsys@transformshift{2.524625in}{1.375851in}%
\pgfsys@useobject{currentmarker}{}%
\end{pgfscope}%
\end{pgfscope}%
\begin{pgfscope}%
\pgfsetrectcap%
\pgfsetmiterjoin%
\pgfsetlinewidth{0.803000pt}%
\definecolor{currentstroke}{rgb}{0.000000,0.000000,0.000000}%
\pgfsetstrokecolor{currentstroke}%
\pgfsetdash{}{0pt}%
\pgfpathmoveto{\pgfqpoint{0.675193in}{0.521603in}}%
\pgfpathlineto{\pgfqpoint{0.675193in}{2.446603in}}%
\pgfusepath{stroke}%
\end{pgfscope}%
\begin{pgfscope}%
\pgfsetrectcap%
\pgfsetmiterjoin%
\pgfsetlinewidth{0.803000pt}%
\definecolor{currentstroke}{rgb}{0.000000,0.000000,0.000000}%
\pgfsetstrokecolor{currentstroke}%
\pgfsetdash{}{0pt}%
\pgfpathmoveto{\pgfqpoint{2.612693in}{0.521603in}}%
\pgfpathlineto{\pgfqpoint{2.612693in}{2.446603in}}%
\pgfusepath{stroke}%
\end{pgfscope}%
\begin{pgfscope}%
\pgfsetrectcap%
\pgfsetmiterjoin%
\pgfsetlinewidth{0.803000pt}%
\definecolor{currentstroke}{rgb}{0.000000,0.000000,0.000000}%
\pgfsetstrokecolor{currentstroke}%
\pgfsetdash{}{0pt}%
\pgfpathmoveto{\pgfqpoint{0.675193in}{0.521603in}}%
\pgfpathlineto{\pgfqpoint{2.612693in}{0.521603in}}%
\pgfusepath{stroke}%
\end{pgfscope}%
\begin{pgfscope}%
\pgfsetrectcap%
\pgfsetmiterjoin%
\pgfsetlinewidth{0.803000pt}%
\definecolor{currentstroke}{rgb}{0.000000,0.000000,0.000000}%
\pgfsetstrokecolor{currentstroke}%
\pgfsetdash{}{0pt}%
\pgfpathmoveto{\pgfqpoint{0.675193in}{2.446603in}}%
\pgfpathlineto{\pgfqpoint{2.612693in}{2.446603in}}%
\pgfusepath{stroke}%
\end{pgfscope}%
\begin{pgfscope}%
\pgfsetbuttcap%
\pgfsetmiterjoin%
\definecolor{currentfill}{rgb}{1.000000,1.000000,1.000000}%
\pgfsetfillcolor{currentfill}%
\pgfsetfillopacity{0.800000}%
\pgfsetlinewidth{1.003750pt}%
\definecolor{currentstroke}{rgb}{0.800000,0.800000,0.800000}%
\pgfsetstrokecolor{currentstroke}%
\pgfsetstrokeopacity{0.800000}%
\pgfsetdash{}{0pt}%
\pgfpathmoveto{\pgfqpoint{1.469355in}{1.723921in}}%
\pgfpathlineto{\pgfqpoint{2.515471in}{1.723921in}}%
\pgfpathquadraticcurveto{\pgfqpoint{2.543249in}{1.723921in}}{\pgfqpoint{2.543249in}{1.751698in}}%
\pgfpathlineto{\pgfqpoint{2.543249in}{2.349381in}}%
\pgfpathquadraticcurveto{\pgfqpoint{2.543249in}{2.377159in}}{\pgfqpoint{2.515471in}{2.377159in}}%
\pgfpathlineto{\pgfqpoint{1.469355in}{2.377159in}}%
\pgfpathquadraticcurveto{\pgfqpoint{1.441578in}{2.377159in}}{\pgfqpoint{1.441578in}{2.349381in}}%
\pgfpathlineto{\pgfqpoint{1.441578in}{1.751698in}}%
\pgfpathquadraticcurveto{\pgfqpoint{1.441578in}{1.723921in}}{\pgfqpoint{1.469355in}{1.723921in}}%
\pgfpathlineto{\pgfqpoint{1.469355in}{1.723921in}}%
\pgfpathclose%
\pgfusepath{stroke,fill}%
\end{pgfscope}%
\begin{pgfscope}%
\pgfsetrectcap%
\pgfsetroundjoin%
\pgfsetlinewidth{1.003750pt}%
\definecolor{currentstroke}{rgb}{0.001462,0.000466,0.013866}%
\pgfsetstrokecolor{currentstroke}%
\pgfsetdash{}{0pt}%
\pgfpathmoveto{\pgfqpoint{1.497133in}{2.264691in}}%
\pgfpathlineto{\pgfqpoint{1.636022in}{2.264691in}}%
\pgfpathlineto{\pgfqpoint{1.774911in}{2.264691in}}%
\pgfusepath{stroke}%
\end{pgfscope}%
\begin{pgfscope}%
\pgfsetbuttcap%
\pgfsetroundjoin%
\definecolor{currentfill}{rgb}{0.001462,0.000466,0.013866}%
\pgfsetfillcolor{currentfill}%
\pgfsetlinewidth{1.003750pt}%
\definecolor{currentstroke}{rgb}{0.001462,0.000466,0.013866}%
\pgfsetstrokecolor{currentstroke}%
\pgfsetdash{}{0pt}%
\pgfsys@defobject{currentmarker}{\pgfqpoint{-0.020833in}{-0.020833in}}{\pgfqpoint{0.020833in}{0.020833in}}{%
\pgfpathmoveto{\pgfqpoint{0.000000in}{-0.020833in}}%
\pgfpathcurveto{\pgfqpoint{0.005525in}{-0.020833in}}{\pgfqpoint{0.010825in}{-0.018638in}}{\pgfqpoint{0.014731in}{-0.014731in}}%
\pgfpathcurveto{\pgfqpoint{0.018638in}{-0.010825in}}{\pgfqpoint{0.020833in}{-0.005525in}}{\pgfqpoint{0.020833in}{0.000000in}}%
\pgfpathcurveto{\pgfqpoint{0.020833in}{0.005525in}}{\pgfqpoint{0.018638in}{0.010825in}}{\pgfqpoint{0.014731in}{0.014731in}}%
\pgfpathcurveto{\pgfqpoint{0.010825in}{0.018638in}}{\pgfqpoint{0.005525in}{0.020833in}}{\pgfqpoint{0.000000in}{0.020833in}}%
\pgfpathcurveto{\pgfqpoint{-0.005525in}{0.020833in}}{\pgfqpoint{-0.010825in}{0.018638in}}{\pgfqpoint{-0.014731in}{0.014731in}}%
\pgfpathcurveto{\pgfqpoint{-0.018638in}{0.010825in}}{\pgfqpoint{-0.020833in}{0.005525in}}{\pgfqpoint{-0.020833in}{0.000000in}}%
\pgfpathcurveto{\pgfqpoint{-0.020833in}{-0.005525in}}{\pgfqpoint{-0.018638in}{-0.010825in}}{\pgfqpoint{-0.014731in}{-0.014731in}}%
\pgfpathcurveto{\pgfqpoint{-0.010825in}{-0.018638in}}{\pgfqpoint{-0.005525in}{-0.020833in}}{\pgfqpoint{0.000000in}{-0.020833in}}%
\pgfpathlineto{\pgfqpoint{0.000000in}{-0.020833in}}%
\pgfpathclose%
\pgfusepath{stroke,fill}%
}%
\begin{pgfscope}%
\pgfsys@transformshift{1.636022in}{2.264691in}%
\pgfsys@useobject{currentmarker}{}%
\end{pgfscope}%
\end{pgfscope}%
\begin{pgfscope}%
\definecolor{textcolor}{rgb}{0.000000,0.000000,0.000000}%
\pgfsetstrokecolor{textcolor}%
\pgfsetfillcolor{textcolor}%
\pgftext[x=1.886022in,y=2.216080in,left,base]{\color{textcolor}{\sffamily\fontsize{10.000000}{12.000000}\selectfont\catcode`\^=\active\def^{\ifmmode\sp\else\^{}\fi}\catcode`\%=\active\def%{\%}Haydock}}%
\end{pgfscope}%
\begin{pgfscope}%
\pgfsetrectcap%
\pgfsetroundjoin%
\pgfsetlinewidth{1.003750pt}%
\definecolor{currentstroke}{rgb}{0.445163,0.122724,0.506901}%
\pgfsetstrokecolor{currentstroke}%
\pgfsetdash{}{0pt}%
\pgfpathmoveto{\pgfqpoint{1.497133in}{2.060834in}}%
\pgfpathlineto{\pgfqpoint{1.636022in}{2.060834in}}%
\pgfpathlineto{\pgfqpoint{1.774911in}{2.060834in}}%
\pgfusepath{stroke}%
\end{pgfscope}%
\begin{pgfscope}%
\pgfsetbuttcap%
\pgfsetroundjoin%
\definecolor{currentfill}{rgb}{0.445163,0.122724,0.506901}%
\pgfsetfillcolor{currentfill}%
\pgfsetlinewidth{1.003750pt}%
\definecolor{currentstroke}{rgb}{0.445163,0.122724,0.506901}%
\pgfsetstrokecolor{currentstroke}%
\pgfsetdash{}{0pt}%
\pgfsys@defobject{currentmarker}{\pgfqpoint{-0.020833in}{-0.020833in}}{\pgfqpoint{0.020833in}{0.020833in}}{%
\pgfpathmoveto{\pgfqpoint{0.000000in}{-0.020833in}}%
\pgfpathcurveto{\pgfqpoint{0.005525in}{-0.020833in}}{\pgfqpoint{0.010825in}{-0.018638in}}{\pgfqpoint{0.014731in}{-0.014731in}}%
\pgfpathcurveto{\pgfqpoint{0.018638in}{-0.010825in}}{\pgfqpoint{0.020833in}{-0.005525in}}{\pgfqpoint{0.020833in}{0.000000in}}%
\pgfpathcurveto{\pgfqpoint{0.020833in}{0.005525in}}{\pgfqpoint{0.018638in}{0.010825in}}{\pgfqpoint{0.014731in}{0.014731in}}%
\pgfpathcurveto{\pgfqpoint{0.010825in}{0.018638in}}{\pgfqpoint{0.005525in}{0.020833in}}{\pgfqpoint{0.000000in}{0.020833in}}%
\pgfpathcurveto{\pgfqpoint{-0.005525in}{0.020833in}}{\pgfqpoint{-0.010825in}{0.018638in}}{\pgfqpoint{-0.014731in}{0.014731in}}%
\pgfpathcurveto{\pgfqpoint{-0.018638in}{0.010825in}}{\pgfqpoint{-0.020833in}{0.005525in}}{\pgfqpoint{-0.020833in}{0.000000in}}%
\pgfpathcurveto{\pgfqpoint{-0.020833in}{-0.005525in}}{\pgfqpoint{-0.018638in}{-0.010825in}}{\pgfqpoint{-0.014731in}{-0.014731in}}%
\pgfpathcurveto{\pgfqpoint{-0.010825in}{-0.018638in}}{\pgfqpoint{-0.005525in}{-0.020833in}}{\pgfqpoint{0.000000in}{-0.020833in}}%
\pgfpathlineto{\pgfqpoint{0.000000in}{-0.020833in}}%
\pgfpathclose%
\pgfusepath{stroke,fill}%
}%
\begin{pgfscope}%
\pgfsys@transformshift{1.636022in}{2.060834in}%
\pgfsys@useobject{currentmarker}{}%
\end{pgfscope}%
\end{pgfscope}%
\begin{pgfscope}%
\definecolor{textcolor}{rgb}{0.000000,0.000000,0.000000}%
\pgfsetstrokecolor{textcolor}%
\pgfsetfillcolor{textcolor}%
\pgftext[x=1.886022in,y=2.012223in,left,base]{\color{textcolor}{\sffamily\fontsize{10.000000}{12.000000}\selectfont\catcode`\^=\active\def^{\ifmmode\sp\else\^{}\fi}\catcode`\%=\active\def%{\%}NC}}%
\end{pgfscope}%
\begin{pgfscope}%
\pgfsetrectcap%
\pgfsetroundjoin%
\pgfsetlinewidth{1.003750pt}%
\definecolor{currentstroke}{rgb}{0.944006,0.377643,0.365136}%
\pgfsetstrokecolor{currentstroke}%
\pgfsetdash{}{0pt}%
\pgfpathmoveto{\pgfqpoint{1.497133in}{1.856977in}}%
\pgfpathlineto{\pgfqpoint{1.636022in}{1.856977in}}%
\pgfpathlineto{\pgfqpoint{1.774911in}{1.856977in}}%
\pgfusepath{stroke}%
\end{pgfscope}%
\begin{pgfscope}%
\pgfsetbuttcap%
\pgfsetroundjoin%
\definecolor{currentfill}{rgb}{0.944006,0.377643,0.365136}%
\pgfsetfillcolor{currentfill}%
\pgfsetlinewidth{1.003750pt}%
\definecolor{currentstroke}{rgb}{0.944006,0.377643,0.365136}%
\pgfsetstrokecolor{currentstroke}%
\pgfsetdash{}{0pt}%
\pgfsys@defobject{currentmarker}{\pgfqpoint{-0.020833in}{-0.020833in}}{\pgfqpoint{0.020833in}{0.020833in}}{%
\pgfpathmoveto{\pgfqpoint{0.000000in}{-0.020833in}}%
\pgfpathcurveto{\pgfqpoint{0.005525in}{-0.020833in}}{\pgfqpoint{0.010825in}{-0.018638in}}{\pgfqpoint{0.014731in}{-0.014731in}}%
\pgfpathcurveto{\pgfqpoint{0.018638in}{-0.010825in}}{\pgfqpoint{0.020833in}{-0.005525in}}{\pgfqpoint{0.020833in}{0.000000in}}%
\pgfpathcurveto{\pgfqpoint{0.020833in}{0.005525in}}{\pgfqpoint{0.018638in}{0.010825in}}{\pgfqpoint{0.014731in}{0.014731in}}%
\pgfpathcurveto{\pgfqpoint{0.010825in}{0.018638in}}{\pgfqpoint{0.005525in}{0.020833in}}{\pgfqpoint{0.000000in}{0.020833in}}%
\pgfpathcurveto{\pgfqpoint{-0.005525in}{0.020833in}}{\pgfqpoint{-0.010825in}{0.018638in}}{\pgfqpoint{-0.014731in}{0.014731in}}%
\pgfpathcurveto{\pgfqpoint{-0.018638in}{0.010825in}}{\pgfqpoint{-0.020833in}{0.005525in}}{\pgfqpoint{-0.020833in}{0.000000in}}%
\pgfpathcurveto{\pgfqpoint{-0.020833in}{-0.005525in}}{\pgfqpoint{-0.018638in}{-0.010825in}}{\pgfqpoint{-0.014731in}{-0.014731in}}%
\pgfpathcurveto{\pgfqpoint{-0.010825in}{-0.018638in}}{\pgfqpoint{-0.005525in}{-0.020833in}}{\pgfqpoint{0.000000in}{-0.020833in}}%
\pgfpathlineto{\pgfqpoint{0.000000in}{-0.020833in}}%
\pgfpathclose%
\pgfusepath{stroke,fill}%
}%
\begin{pgfscope}%
\pgfsys@transformshift{1.636022in}{1.856977in}%
\pgfsys@useobject{currentmarker}{}%
\end{pgfscope}%
\end{pgfscope}%
\begin{pgfscope}%
\definecolor{textcolor}{rgb}{0.000000,0.000000,0.000000}%
\pgfsetstrokecolor{textcolor}%
\pgfsetfillcolor{textcolor}%
\pgftext[x=1.886022in,y=1.808366in,left,base]{\color{textcolor}{\sffamily\fontsize{10.000000}{12.000000}\selectfont\catcode`\^=\active\def^{\ifmmode\sp\else\^{}\fi}\catcode`\%=\active\def%{\%}NC++}}%
\end{pgfscope}%
\end{pgfpicture}%
\makeatother%
\endgroup%

        \caption{\gls{chebyshev-degree} $=800$}
        \label{fig:5-experiments-haydock-convergence-nv-m800}
    \end{subfigure}
    \begin{subfigure}[b]{0.49\columnwidth}
        %% Creator: Matplotlib, PGF backend
%%
%% To include the figure in your LaTeX document, write
%%   \input{<filename>.pgf}
%%
%% Make sure the required packages are loaded in your preamble
%%   \usepackage{pgf}
%%
%% Also ensure that all the required font packages are loaded; for instance,
%% the lmodern package is sometimes necessary when using math font.
%%   \usepackage{lmodern}
%%
%% Figures using additional raster images can only be included by \input if
%% they are in the same directory as the main LaTeX file. For loading figures
%% from other directories you can use the `import` package
%%   \usepackage{import}
%%
%% and then include the figures with
%%   \import{<path to file>}{<filename>.pgf}
%%
%% Matplotlib used the following preamble
%%   \def\mathdefault#1{#1}
%%   \everymath=\expandafter{\the\everymath\displaystyle}
%%   
%%   \makeatletter\@ifpackageloaded{underscore}{}{\usepackage[strings]{underscore}}\makeatother
%%
\begingroup%
\makeatletter%
\begin{pgfpicture}%
\pgfpathrectangle{\pgfpointorigin}{\pgfqpoint{2.759413in}{2.574073in}}%
\pgfusepath{use as bounding box, clip}%
\begin{pgfscope}%
\pgfsetbuttcap%
\pgfsetmiterjoin%
\definecolor{currentfill}{rgb}{1.000000,1.000000,1.000000}%
\pgfsetfillcolor{currentfill}%
\pgfsetlinewidth{0.000000pt}%
\definecolor{currentstroke}{rgb}{1.000000,1.000000,1.000000}%
\pgfsetstrokecolor{currentstroke}%
\pgfsetdash{}{0pt}%
\pgfpathmoveto{\pgfqpoint{0.000000in}{0.000000in}}%
\pgfpathlineto{\pgfqpoint{2.759413in}{0.000000in}}%
\pgfpathlineto{\pgfqpoint{2.759413in}{2.574073in}}%
\pgfpathlineto{\pgfqpoint{0.000000in}{2.574073in}}%
\pgfpathlineto{\pgfqpoint{0.000000in}{0.000000in}}%
\pgfpathclose%
\pgfusepath{fill}%
\end{pgfscope}%
\begin{pgfscope}%
\pgfsetbuttcap%
\pgfsetmiterjoin%
\definecolor{currentfill}{rgb}{1.000000,1.000000,1.000000}%
\pgfsetfillcolor{currentfill}%
\pgfsetlinewidth{0.000000pt}%
\definecolor{currentstroke}{rgb}{0.000000,0.000000,0.000000}%
\pgfsetstrokecolor{currentstroke}%
\pgfsetstrokeopacity{0.000000}%
\pgfsetdash{}{0pt}%
\pgfpathmoveto{\pgfqpoint{0.721913in}{0.549073in}}%
\pgfpathlineto{\pgfqpoint{2.659413in}{0.549073in}}%
\pgfpathlineto{\pgfqpoint{2.659413in}{2.474073in}}%
\pgfpathlineto{\pgfqpoint{0.721913in}{2.474073in}}%
\pgfpathlineto{\pgfqpoint{0.721913in}{0.549073in}}%
\pgfpathclose%
\pgfusepath{fill}%
\end{pgfscope}%
\begin{pgfscope}%
\pgfsetbuttcap%
\pgfsetroundjoin%
\definecolor{currentfill}{rgb}{0.000000,0.000000,0.000000}%
\pgfsetfillcolor{currentfill}%
\pgfsetlinewidth{0.803000pt}%
\definecolor{currentstroke}{rgb}{0.000000,0.000000,0.000000}%
\pgfsetstrokecolor{currentstroke}%
\pgfsetdash{}{0pt}%
\pgfsys@defobject{currentmarker}{\pgfqpoint{0.000000in}{-0.048611in}}{\pgfqpoint{0.000000in}{0.000000in}}{%
\pgfpathmoveto{\pgfqpoint{0.000000in}{0.000000in}}%
\pgfpathlineto{\pgfqpoint{0.000000in}{-0.048611in}}%
\pgfusepath{stroke,fill}%
}%
\begin{pgfscope}%
\pgfsys@transformshift{1.771566in}{0.549073in}%
\pgfsys@useobject{currentmarker}{}%
\end{pgfscope}%
\end{pgfscope}%
\begin{pgfscope}%
\definecolor{textcolor}{rgb}{0.000000,0.000000,0.000000}%
\pgfsetstrokecolor{textcolor}%
\pgfsetfillcolor{textcolor}%
\pgftext[x=1.771566in,y=0.451851in,,top]{\color{textcolor}{\rmfamily\fontsize{12.000000}{14.400000}\selectfont\catcode`\^=\active\def^{\ifmmode\sp\else\^{}\fi}\catcode`\%=\active\def%{\%}$\mathdefault{10^{2}}$}}%
\end{pgfscope}%
\begin{pgfscope}%
\pgfsetbuttcap%
\pgfsetroundjoin%
\definecolor{currentfill}{rgb}{0.000000,0.000000,0.000000}%
\pgfsetfillcolor{currentfill}%
\pgfsetlinewidth{0.602250pt}%
\definecolor{currentstroke}{rgb}{0.000000,0.000000,0.000000}%
\pgfsetstrokecolor{currentstroke}%
\pgfsetdash{}{0pt}%
\pgfsys@defobject{currentmarker}{\pgfqpoint{0.000000in}{-0.027778in}}{\pgfqpoint{0.000000in}{0.000000in}}{%
\pgfpathmoveto{\pgfqpoint{0.000000in}{0.000000in}}%
\pgfpathlineto{\pgfqpoint{0.000000in}{-0.027778in}}%
\pgfusepath{stroke,fill}%
}%
\begin{pgfscope}%
\pgfsys@transformshift{0.839681in}{0.549073in}%
\pgfsys@useobject{currentmarker}{}%
\end{pgfscope}%
\end{pgfscope}%
\begin{pgfscope}%
\pgfsetbuttcap%
\pgfsetroundjoin%
\definecolor{currentfill}{rgb}{0.000000,0.000000,0.000000}%
\pgfsetfillcolor{currentfill}%
\pgfsetlinewidth{0.602250pt}%
\definecolor{currentstroke}{rgb}{0.000000,0.000000,0.000000}%
\pgfsetstrokecolor{currentstroke}%
\pgfsetdash{}{0pt}%
\pgfsys@defobject{currentmarker}{\pgfqpoint{0.000000in}{-0.027778in}}{\pgfqpoint{0.000000in}{0.000000in}}{%
\pgfpathmoveto{\pgfqpoint{0.000000in}{0.000000in}}%
\pgfpathlineto{\pgfqpoint{0.000000in}{-0.027778in}}%
\pgfusepath{stroke,fill}%
}%
\begin{pgfscope}%
\pgfsys@transformshift{1.074450in}{0.549073in}%
\pgfsys@useobject{currentmarker}{}%
\end{pgfscope}%
\end{pgfscope}%
\begin{pgfscope}%
\pgfsetbuttcap%
\pgfsetroundjoin%
\definecolor{currentfill}{rgb}{0.000000,0.000000,0.000000}%
\pgfsetfillcolor{currentfill}%
\pgfsetlinewidth{0.602250pt}%
\definecolor{currentstroke}{rgb}{0.000000,0.000000,0.000000}%
\pgfsetstrokecolor{currentstroke}%
\pgfsetdash{}{0pt}%
\pgfsys@defobject{currentmarker}{\pgfqpoint{0.000000in}{-0.027778in}}{\pgfqpoint{0.000000in}{0.000000in}}{%
\pgfpathmoveto{\pgfqpoint{0.000000in}{0.000000in}}%
\pgfpathlineto{\pgfqpoint{0.000000in}{-0.027778in}}%
\pgfusepath{stroke,fill}%
}%
\begin{pgfscope}%
\pgfsys@transformshift{1.241022in}{0.549073in}%
\pgfsys@useobject{currentmarker}{}%
\end{pgfscope}%
\end{pgfscope}%
\begin{pgfscope}%
\pgfsetbuttcap%
\pgfsetroundjoin%
\definecolor{currentfill}{rgb}{0.000000,0.000000,0.000000}%
\pgfsetfillcolor{currentfill}%
\pgfsetlinewidth{0.602250pt}%
\definecolor{currentstroke}{rgb}{0.000000,0.000000,0.000000}%
\pgfsetstrokecolor{currentstroke}%
\pgfsetdash{}{0pt}%
\pgfsys@defobject{currentmarker}{\pgfqpoint{0.000000in}{-0.027778in}}{\pgfqpoint{0.000000in}{0.000000in}}{%
\pgfpathmoveto{\pgfqpoint{0.000000in}{0.000000in}}%
\pgfpathlineto{\pgfqpoint{0.000000in}{-0.027778in}}%
\pgfusepath{stroke,fill}%
}%
\begin{pgfscope}%
\pgfsys@transformshift{1.370225in}{0.549073in}%
\pgfsys@useobject{currentmarker}{}%
\end{pgfscope}%
\end{pgfscope}%
\begin{pgfscope}%
\pgfsetbuttcap%
\pgfsetroundjoin%
\definecolor{currentfill}{rgb}{0.000000,0.000000,0.000000}%
\pgfsetfillcolor{currentfill}%
\pgfsetlinewidth{0.602250pt}%
\definecolor{currentstroke}{rgb}{0.000000,0.000000,0.000000}%
\pgfsetstrokecolor{currentstroke}%
\pgfsetdash{}{0pt}%
\pgfsys@defobject{currentmarker}{\pgfqpoint{0.000000in}{-0.027778in}}{\pgfqpoint{0.000000in}{0.000000in}}{%
\pgfpathmoveto{\pgfqpoint{0.000000in}{0.000000in}}%
\pgfpathlineto{\pgfqpoint{0.000000in}{-0.027778in}}%
\pgfusepath{stroke,fill}%
}%
\begin{pgfscope}%
\pgfsys@transformshift{1.475791in}{0.549073in}%
\pgfsys@useobject{currentmarker}{}%
\end{pgfscope}%
\end{pgfscope}%
\begin{pgfscope}%
\pgfsetbuttcap%
\pgfsetroundjoin%
\definecolor{currentfill}{rgb}{0.000000,0.000000,0.000000}%
\pgfsetfillcolor{currentfill}%
\pgfsetlinewidth{0.602250pt}%
\definecolor{currentstroke}{rgb}{0.000000,0.000000,0.000000}%
\pgfsetstrokecolor{currentstroke}%
\pgfsetdash{}{0pt}%
\pgfsys@defobject{currentmarker}{\pgfqpoint{0.000000in}{-0.027778in}}{\pgfqpoint{0.000000in}{0.000000in}}{%
\pgfpathmoveto{\pgfqpoint{0.000000in}{0.000000in}}%
\pgfpathlineto{\pgfqpoint{0.000000in}{-0.027778in}}%
\pgfusepath{stroke,fill}%
}%
\begin{pgfscope}%
\pgfsys@transformshift{1.565047in}{0.549073in}%
\pgfsys@useobject{currentmarker}{}%
\end{pgfscope}%
\end{pgfscope}%
\begin{pgfscope}%
\pgfsetbuttcap%
\pgfsetroundjoin%
\definecolor{currentfill}{rgb}{0.000000,0.000000,0.000000}%
\pgfsetfillcolor{currentfill}%
\pgfsetlinewidth{0.602250pt}%
\definecolor{currentstroke}{rgb}{0.000000,0.000000,0.000000}%
\pgfsetstrokecolor{currentstroke}%
\pgfsetdash{}{0pt}%
\pgfsys@defobject{currentmarker}{\pgfqpoint{0.000000in}{-0.027778in}}{\pgfqpoint{0.000000in}{0.000000in}}{%
\pgfpathmoveto{\pgfqpoint{0.000000in}{0.000000in}}%
\pgfpathlineto{\pgfqpoint{0.000000in}{-0.027778in}}%
\pgfusepath{stroke,fill}%
}%
\begin{pgfscope}%
\pgfsys@transformshift{1.642363in}{0.549073in}%
\pgfsys@useobject{currentmarker}{}%
\end{pgfscope}%
\end{pgfscope}%
\begin{pgfscope}%
\pgfsetbuttcap%
\pgfsetroundjoin%
\definecolor{currentfill}{rgb}{0.000000,0.000000,0.000000}%
\pgfsetfillcolor{currentfill}%
\pgfsetlinewidth{0.602250pt}%
\definecolor{currentstroke}{rgb}{0.000000,0.000000,0.000000}%
\pgfsetstrokecolor{currentstroke}%
\pgfsetdash{}{0pt}%
\pgfsys@defobject{currentmarker}{\pgfqpoint{0.000000in}{-0.027778in}}{\pgfqpoint{0.000000in}{0.000000in}}{%
\pgfpathmoveto{\pgfqpoint{0.000000in}{0.000000in}}%
\pgfpathlineto{\pgfqpoint{0.000000in}{-0.027778in}}%
\pgfusepath{stroke,fill}%
}%
\begin{pgfscope}%
\pgfsys@transformshift{1.710561in}{0.549073in}%
\pgfsys@useobject{currentmarker}{}%
\end{pgfscope}%
\end{pgfscope}%
\begin{pgfscope}%
\pgfsetbuttcap%
\pgfsetroundjoin%
\definecolor{currentfill}{rgb}{0.000000,0.000000,0.000000}%
\pgfsetfillcolor{currentfill}%
\pgfsetlinewidth{0.602250pt}%
\definecolor{currentstroke}{rgb}{0.000000,0.000000,0.000000}%
\pgfsetstrokecolor{currentstroke}%
\pgfsetdash{}{0pt}%
\pgfsys@defobject{currentmarker}{\pgfqpoint{0.000000in}{-0.027778in}}{\pgfqpoint{0.000000in}{0.000000in}}{%
\pgfpathmoveto{\pgfqpoint{0.000000in}{0.000000in}}%
\pgfpathlineto{\pgfqpoint{0.000000in}{-0.027778in}}%
\pgfusepath{stroke,fill}%
}%
\begin{pgfscope}%
\pgfsys@transformshift{2.172907in}{0.549073in}%
\pgfsys@useobject{currentmarker}{}%
\end{pgfscope}%
\end{pgfscope}%
\begin{pgfscope}%
\pgfsetbuttcap%
\pgfsetroundjoin%
\definecolor{currentfill}{rgb}{0.000000,0.000000,0.000000}%
\pgfsetfillcolor{currentfill}%
\pgfsetlinewidth{0.602250pt}%
\definecolor{currentstroke}{rgb}{0.000000,0.000000,0.000000}%
\pgfsetstrokecolor{currentstroke}%
\pgfsetdash{}{0pt}%
\pgfsys@defobject{currentmarker}{\pgfqpoint{0.000000in}{-0.027778in}}{\pgfqpoint{0.000000in}{0.000000in}}{%
\pgfpathmoveto{\pgfqpoint{0.000000in}{0.000000in}}%
\pgfpathlineto{\pgfqpoint{0.000000in}{-0.027778in}}%
\pgfusepath{stroke,fill}%
}%
\begin{pgfscope}%
\pgfsys@transformshift{2.407676in}{0.549073in}%
\pgfsys@useobject{currentmarker}{}%
\end{pgfscope}%
\end{pgfscope}%
\begin{pgfscope}%
\pgfsetbuttcap%
\pgfsetroundjoin%
\definecolor{currentfill}{rgb}{0.000000,0.000000,0.000000}%
\pgfsetfillcolor{currentfill}%
\pgfsetlinewidth{0.602250pt}%
\definecolor{currentstroke}{rgb}{0.000000,0.000000,0.000000}%
\pgfsetstrokecolor{currentstroke}%
\pgfsetdash{}{0pt}%
\pgfsys@defobject{currentmarker}{\pgfqpoint{0.000000in}{-0.027778in}}{\pgfqpoint{0.000000in}{0.000000in}}{%
\pgfpathmoveto{\pgfqpoint{0.000000in}{0.000000in}}%
\pgfpathlineto{\pgfqpoint{0.000000in}{-0.027778in}}%
\pgfusepath{stroke,fill}%
}%
\begin{pgfscope}%
\pgfsys@transformshift{2.574248in}{0.549073in}%
\pgfsys@useobject{currentmarker}{}%
\end{pgfscope}%
\end{pgfscope}%
\begin{pgfscope}%
\definecolor{textcolor}{rgb}{0.000000,0.000000,0.000000}%
\pgfsetstrokecolor{textcolor}%
\pgfsetfillcolor{textcolor}%
\pgftext[x=1.690663in,y=0.248148in,,top]{\color{textcolor}{\rmfamily\fontsize{12.000000}{14.400000}\selectfont\catcode`\^=\active\def^{\ifmmode\sp\else\^{}\fi}\catcode`\%=\active\def%{\%}$n_{\Omega} + n_{\Psi}$}}%
\end{pgfscope}%
\begin{pgfscope}%
\pgfsetbuttcap%
\pgfsetroundjoin%
\definecolor{currentfill}{rgb}{0.000000,0.000000,0.000000}%
\pgfsetfillcolor{currentfill}%
\pgfsetlinewidth{0.803000pt}%
\definecolor{currentstroke}{rgb}{0.000000,0.000000,0.000000}%
\pgfsetstrokecolor{currentstroke}%
\pgfsetdash{}{0pt}%
\pgfsys@defobject{currentmarker}{\pgfqpoint{-0.048611in}{0.000000in}}{\pgfqpoint{-0.000000in}{0.000000in}}{%
\pgfpathmoveto{\pgfqpoint{-0.000000in}{0.000000in}}%
\pgfpathlineto{\pgfqpoint{-0.048611in}{0.000000in}}%
\pgfusepath{stroke,fill}%
}%
\begin{pgfscope}%
\pgfsys@transformshift{0.721913in}{0.901962in}%
\pgfsys@useobject{currentmarker}{}%
\end{pgfscope}%
\end{pgfscope}%
\begin{pgfscope}%
\definecolor{textcolor}{rgb}{0.000000,0.000000,0.000000}%
\pgfsetstrokecolor{textcolor}%
\pgfsetfillcolor{textcolor}%
\pgftext[x=0.303703in, y=0.844092in, left, base]{\color{textcolor}{\rmfamily\fontsize{12.000000}{14.400000}\selectfont\catcode`\^=\active\def^{\ifmmode\sp\else\^{}\fi}\catcode`\%=\active\def%{\%}$\mathdefault{10^{-3}}$}}%
\end{pgfscope}%
\begin{pgfscope}%
\pgfsetbuttcap%
\pgfsetroundjoin%
\definecolor{currentfill}{rgb}{0.000000,0.000000,0.000000}%
\pgfsetfillcolor{currentfill}%
\pgfsetlinewidth{0.803000pt}%
\definecolor{currentstroke}{rgb}{0.000000,0.000000,0.000000}%
\pgfsetstrokecolor{currentstroke}%
\pgfsetdash{}{0pt}%
\pgfsys@defobject{currentmarker}{\pgfqpoint{-0.048611in}{0.000000in}}{\pgfqpoint{-0.000000in}{0.000000in}}{%
\pgfpathmoveto{\pgfqpoint{-0.000000in}{0.000000in}}%
\pgfpathlineto{\pgfqpoint{-0.048611in}{0.000000in}}%
\pgfusepath{stroke,fill}%
}%
\begin{pgfscope}%
\pgfsys@transformshift{0.721913in}{1.466781in}%
\pgfsys@useobject{currentmarker}{}%
\end{pgfscope}%
\end{pgfscope}%
\begin{pgfscope}%
\definecolor{textcolor}{rgb}{0.000000,0.000000,0.000000}%
\pgfsetstrokecolor{textcolor}%
\pgfsetfillcolor{textcolor}%
\pgftext[x=0.303703in, y=1.408911in, left, base]{\color{textcolor}{\rmfamily\fontsize{12.000000}{14.400000}\selectfont\catcode`\^=\active\def^{\ifmmode\sp\else\^{}\fi}\catcode`\%=\active\def%{\%}$\mathdefault{10^{-2}}$}}%
\end{pgfscope}%
\begin{pgfscope}%
\pgfsetbuttcap%
\pgfsetroundjoin%
\definecolor{currentfill}{rgb}{0.000000,0.000000,0.000000}%
\pgfsetfillcolor{currentfill}%
\pgfsetlinewidth{0.803000pt}%
\definecolor{currentstroke}{rgb}{0.000000,0.000000,0.000000}%
\pgfsetstrokecolor{currentstroke}%
\pgfsetdash{}{0pt}%
\pgfsys@defobject{currentmarker}{\pgfqpoint{-0.048611in}{0.000000in}}{\pgfqpoint{-0.000000in}{0.000000in}}{%
\pgfpathmoveto{\pgfqpoint{-0.000000in}{0.000000in}}%
\pgfpathlineto{\pgfqpoint{-0.048611in}{0.000000in}}%
\pgfusepath{stroke,fill}%
}%
\begin{pgfscope}%
\pgfsys@transformshift{0.721913in}{2.031600in}%
\pgfsys@useobject{currentmarker}{}%
\end{pgfscope}%
\end{pgfscope}%
\begin{pgfscope}%
\definecolor{textcolor}{rgb}{0.000000,0.000000,0.000000}%
\pgfsetstrokecolor{textcolor}%
\pgfsetfillcolor{textcolor}%
\pgftext[x=0.303703in, y=1.973730in, left, base]{\color{textcolor}{\rmfamily\fontsize{12.000000}{14.400000}\selectfont\catcode`\^=\active\def^{\ifmmode\sp\else\^{}\fi}\catcode`\%=\active\def%{\%}$\mathdefault{10^{-1}}$}}%
\end{pgfscope}%
\begin{pgfscope}%
\pgfsetbuttcap%
\pgfsetroundjoin%
\definecolor{currentfill}{rgb}{0.000000,0.000000,0.000000}%
\pgfsetfillcolor{currentfill}%
\pgfsetlinewidth{0.602250pt}%
\definecolor{currentstroke}{rgb}{0.000000,0.000000,0.000000}%
\pgfsetstrokecolor{currentstroke}%
\pgfsetdash{}{0pt}%
\pgfsys@defobject{currentmarker}{\pgfqpoint{-0.027778in}{0.000000in}}{\pgfqpoint{-0.000000in}{0.000000in}}{%
\pgfpathmoveto{\pgfqpoint{-0.000000in}{0.000000in}}%
\pgfpathlineto{\pgfqpoint{-0.027778in}{0.000000in}}%
\pgfusepath{stroke,fill}%
}%
\begin{pgfscope}%
\pgfsys@transformshift{0.721913in}{0.606630in}%
\pgfsys@useobject{currentmarker}{}%
\end{pgfscope}%
\end{pgfscope}%
\begin{pgfscope}%
\pgfsetbuttcap%
\pgfsetroundjoin%
\definecolor{currentfill}{rgb}{0.000000,0.000000,0.000000}%
\pgfsetfillcolor{currentfill}%
\pgfsetlinewidth{0.602250pt}%
\definecolor{currentstroke}{rgb}{0.000000,0.000000,0.000000}%
\pgfsetstrokecolor{currentstroke}%
\pgfsetdash{}{0pt}%
\pgfsys@defobject{currentmarker}{\pgfqpoint{-0.027778in}{0.000000in}}{\pgfqpoint{-0.000000in}{0.000000in}}{%
\pgfpathmoveto{\pgfqpoint{-0.000000in}{0.000000in}}%
\pgfpathlineto{\pgfqpoint{-0.027778in}{0.000000in}}%
\pgfusepath{stroke,fill}%
}%
\begin{pgfscope}%
\pgfsys@transformshift{0.721913in}{0.677198in}%
\pgfsys@useobject{currentmarker}{}%
\end{pgfscope}%
\end{pgfscope}%
\begin{pgfscope}%
\pgfsetbuttcap%
\pgfsetroundjoin%
\definecolor{currentfill}{rgb}{0.000000,0.000000,0.000000}%
\pgfsetfillcolor{currentfill}%
\pgfsetlinewidth{0.602250pt}%
\definecolor{currentstroke}{rgb}{0.000000,0.000000,0.000000}%
\pgfsetstrokecolor{currentstroke}%
\pgfsetdash{}{0pt}%
\pgfsys@defobject{currentmarker}{\pgfqpoint{-0.027778in}{0.000000in}}{\pgfqpoint{-0.000000in}{0.000000in}}{%
\pgfpathmoveto{\pgfqpoint{-0.000000in}{0.000000in}}%
\pgfpathlineto{\pgfqpoint{-0.027778in}{0.000000in}}%
\pgfusepath{stroke,fill}%
}%
\begin{pgfscope}%
\pgfsys@transformshift{0.721913in}{0.731935in}%
\pgfsys@useobject{currentmarker}{}%
\end{pgfscope}%
\end{pgfscope}%
\begin{pgfscope}%
\pgfsetbuttcap%
\pgfsetroundjoin%
\definecolor{currentfill}{rgb}{0.000000,0.000000,0.000000}%
\pgfsetfillcolor{currentfill}%
\pgfsetlinewidth{0.602250pt}%
\definecolor{currentstroke}{rgb}{0.000000,0.000000,0.000000}%
\pgfsetstrokecolor{currentstroke}%
\pgfsetdash{}{0pt}%
\pgfsys@defobject{currentmarker}{\pgfqpoint{-0.027778in}{0.000000in}}{\pgfqpoint{-0.000000in}{0.000000in}}{%
\pgfpathmoveto{\pgfqpoint{-0.000000in}{0.000000in}}%
\pgfpathlineto{\pgfqpoint{-0.027778in}{0.000000in}}%
\pgfusepath{stroke,fill}%
}%
\begin{pgfscope}%
\pgfsys@transformshift{0.721913in}{0.776658in}%
\pgfsys@useobject{currentmarker}{}%
\end{pgfscope}%
\end{pgfscope}%
\begin{pgfscope}%
\pgfsetbuttcap%
\pgfsetroundjoin%
\definecolor{currentfill}{rgb}{0.000000,0.000000,0.000000}%
\pgfsetfillcolor{currentfill}%
\pgfsetlinewidth{0.602250pt}%
\definecolor{currentstroke}{rgb}{0.000000,0.000000,0.000000}%
\pgfsetstrokecolor{currentstroke}%
\pgfsetdash{}{0pt}%
\pgfsys@defobject{currentmarker}{\pgfqpoint{-0.027778in}{0.000000in}}{\pgfqpoint{-0.000000in}{0.000000in}}{%
\pgfpathmoveto{\pgfqpoint{-0.000000in}{0.000000in}}%
\pgfpathlineto{\pgfqpoint{-0.027778in}{0.000000in}}%
\pgfusepath{stroke,fill}%
}%
\begin{pgfscope}%
\pgfsys@transformshift{0.721913in}{0.814470in}%
\pgfsys@useobject{currentmarker}{}%
\end{pgfscope}%
\end{pgfscope}%
\begin{pgfscope}%
\pgfsetbuttcap%
\pgfsetroundjoin%
\definecolor{currentfill}{rgb}{0.000000,0.000000,0.000000}%
\pgfsetfillcolor{currentfill}%
\pgfsetlinewidth{0.602250pt}%
\definecolor{currentstroke}{rgb}{0.000000,0.000000,0.000000}%
\pgfsetstrokecolor{currentstroke}%
\pgfsetdash{}{0pt}%
\pgfsys@defobject{currentmarker}{\pgfqpoint{-0.027778in}{0.000000in}}{\pgfqpoint{-0.000000in}{0.000000in}}{%
\pgfpathmoveto{\pgfqpoint{-0.000000in}{0.000000in}}%
\pgfpathlineto{\pgfqpoint{-0.027778in}{0.000000in}}%
\pgfusepath{stroke,fill}%
}%
\begin{pgfscope}%
\pgfsys@transformshift{0.721913in}{0.847225in}%
\pgfsys@useobject{currentmarker}{}%
\end{pgfscope}%
\end{pgfscope}%
\begin{pgfscope}%
\pgfsetbuttcap%
\pgfsetroundjoin%
\definecolor{currentfill}{rgb}{0.000000,0.000000,0.000000}%
\pgfsetfillcolor{currentfill}%
\pgfsetlinewidth{0.602250pt}%
\definecolor{currentstroke}{rgb}{0.000000,0.000000,0.000000}%
\pgfsetstrokecolor{currentstroke}%
\pgfsetdash{}{0pt}%
\pgfsys@defobject{currentmarker}{\pgfqpoint{-0.027778in}{0.000000in}}{\pgfqpoint{-0.000000in}{0.000000in}}{%
\pgfpathmoveto{\pgfqpoint{-0.000000in}{0.000000in}}%
\pgfpathlineto{\pgfqpoint{-0.027778in}{0.000000in}}%
\pgfusepath{stroke,fill}%
}%
\begin{pgfscope}%
\pgfsys@transformshift{0.721913in}{0.876117in}%
\pgfsys@useobject{currentmarker}{}%
\end{pgfscope}%
\end{pgfscope}%
\begin{pgfscope}%
\pgfsetbuttcap%
\pgfsetroundjoin%
\definecolor{currentfill}{rgb}{0.000000,0.000000,0.000000}%
\pgfsetfillcolor{currentfill}%
\pgfsetlinewidth{0.602250pt}%
\definecolor{currentstroke}{rgb}{0.000000,0.000000,0.000000}%
\pgfsetstrokecolor{currentstroke}%
\pgfsetdash{}{0pt}%
\pgfsys@defobject{currentmarker}{\pgfqpoint{-0.027778in}{0.000000in}}{\pgfqpoint{-0.000000in}{0.000000in}}{%
\pgfpathmoveto{\pgfqpoint{-0.000000in}{0.000000in}}%
\pgfpathlineto{\pgfqpoint{-0.027778in}{0.000000in}}%
\pgfusepath{stroke,fill}%
}%
\begin{pgfscope}%
\pgfsys@transformshift{0.721913in}{1.071989in}%
\pgfsys@useobject{currentmarker}{}%
\end{pgfscope}%
\end{pgfscope}%
\begin{pgfscope}%
\pgfsetbuttcap%
\pgfsetroundjoin%
\definecolor{currentfill}{rgb}{0.000000,0.000000,0.000000}%
\pgfsetfillcolor{currentfill}%
\pgfsetlinewidth{0.602250pt}%
\definecolor{currentstroke}{rgb}{0.000000,0.000000,0.000000}%
\pgfsetstrokecolor{currentstroke}%
\pgfsetdash{}{0pt}%
\pgfsys@defobject{currentmarker}{\pgfqpoint{-0.027778in}{0.000000in}}{\pgfqpoint{-0.000000in}{0.000000in}}{%
\pgfpathmoveto{\pgfqpoint{-0.000000in}{0.000000in}}%
\pgfpathlineto{\pgfqpoint{-0.027778in}{0.000000in}}%
\pgfusepath{stroke,fill}%
}%
\begin{pgfscope}%
\pgfsys@transformshift{0.721913in}{1.171449in}%
\pgfsys@useobject{currentmarker}{}%
\end{pgfscope}%
\end{pgfscope}%
\begin{pgfscope}%
\pgfsetbuttcap%
\pgfsetroundjoin%
\definecolor{currentfill}{rgb}{0.000000,0.000000,0.000000}%
\pgfsetfillcolor{currentfill}%
\pgfsetlinewidth{0.602250pt}%
\definecolor{currentstroke}{rgb}{0.000000,0.000000,0.000000}%
\pgfsetstrokecolor{currentstroke}%
\pgfsetdash{}{0pt}%
\pgfsys@defobject{currentmarker}{\pgfqpoint{-0.027778in}{0.000000in}}{\pgfqpoint{-0.000000in}{0.000000in}}{%
\pgfpathmoveto{\pgfqpoint{-0.000000in}{0.000000in}}%
\pgfpathlineto{\pgfqpoint{-0.027778in}{0.000000in}}%
\pgfusepath{stroke,fill}%
}%
\begin{pgfscope}%
\pgfsys@transformshift{0.721913in}{1.242017in}%
\pgfsys@useobject{currentmarker}{}%
\end{pgfscope}%
\end{pgfscope}%
\begin{pgfscope}%
\pgfsetbuttcap%
\pgfsetroundjoin%
\definecolor{currentfill}{rgb}{0.000000,0.000000,0.000000}%
\pgfsetfillcolor{currentfill}%
\pgfsetlinewidth{0.602250pt}%
\definecolor{currentstroke}{rgb}{0.000000,0.000000,0.000000}%
\pgfsetstrokecolor{currentstroke}%
\pgfsetdash{}{0pt}%
\pgfsys@defobject{currentmarker}{\pgfqpoint{-0.027778in}{0.000000in}}{\pgfqpoint{-0.000000in}{0.000000in}}{%
\pgfpathmoveto{\pgfqpoint{-0.000000in}{0.000000in}}%
\pgfpathlineto{\pgfqpoint{-0.027778in}{0.000000in}}%
\pgfusepath{stroke,fill}%
}%
\begin{pgfscope}%
\pgfsys@transformshift{0.721913in}{1.296754in}%
\pgfsys@useobject{currentmarker}{}%
\end{pgfscope}%
\end{pgfscope}%
\begin{pgfscope}%
\pgfsetbuttcap%
\pgfsetroundjoin%
\definecolor{currentfill}{rgb}{0.000000,0.000000,0.000000}%
\pgfsetfillcolor{currentfill}%
\pgfsetlinewidth{0.602250pt}%
\definecolor{currentstroke}{rgb}{0.000000,0.000000,0.000000}%
\pgfsetstrokecolor{currentstroke}%
\pgfsetdash{}{0pt}%
\pgfsys@defobject{currentmarker}{\pgfqpoint{-0.027778in}{0.000000in}}{\pgfqpoint{-0.000000in}{0.000000in}}{%
\pgfpathmoveto{\pgfqpoint{-0.000000in}{0.000000in}}%
\pgfpathlineto{\pgfqpoint{-0.027778in}{0.000000in}}%
\pgfusepath{stroke,fill}%
}%
\begin{pgfscope}%
\pgfsys@transformshift{0.721913in}{1.341477in}%
\pgfsys@useobject{currentmarker}{}%
\end{pgfscope}%
\end{pgfscope}%
\begin{pgfscope}%
\pgfsetbuttcap%
\pgfsetroundjoin%
\definecolor{currentfill}{rgb}{0.000000,0.000000,0.000000}%
\pgfsetfillcolor{currentfill}%
\pgfsetlinewidth{0.602250pt}%
\definecolor{currentstroke}{rgb}{0.000000,0.000000,0.000000}%
\pgfsetstrokecolor{currentstroke}%
\pgfsetdash{}{0pt}%
\pgfsys@defobject{currentmarker}{\pgfqpoint{-0.027778in}{0.000000in}}{\pgfqpoint{-0.000000in}{0.000000in}}{%
\pgfpathmoveto{\pgfqpoint{-0.000000in}{0.000000in}}%
\pgfpathlineto{\pgfqpoint{-0.027778in}{0.000000in}}%
\pgfusepath{stroke,fill}%
}%
\begin{pgfscope}%
\pgfsys@transformshift{0.721913in}{1.379289in}%
\pgfsys@useobject{currentmarker}{}%
\end{pgfscope}%
\end{pgfscope}%
\begin{pgfscope}%
\pgfsetbuttcap%
\pgfsetroundjoin%
\definecolor{currentfill}{rgb}{0.000000,0.000000,0.000000}%
\pgfsetfillcolor{currentfill}%
\pgfsetlinewidth{0.602250pt}%
\definecolor{currentstroke}{rgb}{0.000000,0.000000,0.000000}%
\pgfsetstrokecolor{currentstroke}%
\pgfsetdash{}{0pt}%
\pgfsys@defobject{currentmarker}{\pgfqpoint{-0.027778in}{0.000000in}}{\pgfqpoint{-0.000000in}{0.000000in}}{%
\pgfpathmoveto{\pgfqpoint{-0.000000in}{0.000000in}}%
\pgfpathlineto{\pgfqpoint{-0.027778in}{0.000000in}}%
\pgfusepath{stroke,fill}%
}%
\begin{pgfscope}%
\pgfsys@transformshift{0.721913in}{1.412044in}%
\pgfsys@useobject{currentmarker}{}%
\end{pgfscope}%
\end{pgfscope}%
\begin{pgfscope}%
\pgfsetbuttcap%
\pgfsetroundjoin%
\definecolor{currentfill}{rgb}{0.000000,0.000000,0.000000}%
\pgfsetfillcolor{currentfill}%
\pgfsetlinewidth{0.602250pt}%
\definecolor{currentstroke}{rgb}{0.000000,0.000000,0.000000}%
\pgfsetstrokecolor{currentstroke}%
\pgfsetdash{}{0pt}%
\pgfsys@defobject{currentmarker}{\pgfqpoint{-0.027778in}{0.000000in}}{\pgfqpoint{-0.000000in}{0.000000in}}{%
\pgfpathmoveto{\pgfqpoint{-0.000000in}{0.000000in}}%
\pgfpathlineto{\pgfqpoint{-0.027778in}{0.000000in}}%
\pgfusepath{stroke,fill}%
}%
\begin{pgfscope}%
\pgfsys@transformshift{0.721913in}{1.440936in}%
\pgfsys@useobject{currentmarker}{}%
\end{pgfscope}%
\end{pgfscope}%
\begin{pgfscope}%
\pgfsetbuttcap%
\pgfsetroundjoin%
\definecolor{currentfill}{rgb}{0.000000,0.000000,0.000000}%
\pgfsetfillcolor{currentfill}%
\pgfsetlinewidth{0.602250pt}%
\definecolor{currentstroke}{rgb}{0.000000,0.000000,0.000000}%
\pgfsetstrokecolor{currentstroke}%
\pgfsetdash{}{0pt}%
\pgfsys@defobject{currentmarker}{\pgfqpoint{-0.027778in}{0.000000in}}{\pgfqpoint{-0.000000in}{0.000000in}}{%
\pgfpathmoveto{\pgfqpoint{-0.000000in}{0.000000in}}%
\pgfpathlineto{\pgfqpoint{-0.027778in}{0.000000in}}%
\pgfusepath{stroke,fill}%
}%
\begin{pgfscope}%
\pgfsys@transformshift{0.721913in}{1.636808in}%
\pgfsys@useobject{currentmarker}{}%
\end{pgfscope}%
\end{pgfscope}%
\begin{pgfscope}%
\pgfsetbuttcap%
\pgfsetroundjoin%
\definecolor{currentfill}{rgb}{0.000000,0.000000,0.000000}%
\pgfsetfillcolor{currentfill}%
\pgfsetlinewidth{0.602250pt}%
\definecolor{currentstroke}{rgb}{0.000000,0.000000,0.000000}%
\pgfsetstrokecolor{currentstroke}%
\pgfsetdash{}{0pt}%
\pgfsys@defobject{currentmarker}{\pgfqpoint{-0.027778in}{0.000000in}}{\pgfqpoint{-0.000000in}{0.000000in}}{%
\pgfpathmoveto{\pgfqpoint{-0.000000in}{0.000000in}}%
\pgfpathlineto{\pgfqpoint{-0.027778in}{0.000000in}}%
\pgfusepath{stroke,fill}%
}%
\begin{pgfscope}%
\pgfsys@transformshift{0.721913in}{1.736268in}%
\pgfsys@useobject{currentmarker}{}%
\end{pgfscope}%
\end{pgfscope}%
\begin{pgfscope}%
\pgfsetbuttcap%
\pgfsetroundjoin%
\definecolor{currentfill}{rgb}{0.000000,0.000000,0.000000}%
\pgfsetfillcolor{currentfill}%
\pgfsetlinewidth{0.602250pt}%
\definecolor{currentstroke}{rgb}{0.000000,0.000000,0.000000}%
\pgfsetstrokecolor{currentstroke}%
\pgfsetdash{}{0pt}%
\pgfsys@defobject{currentmarker}{\pgfqpoint{-0.027778in}{0.000000in}}{\pgfqpoint{-0.000000in}{0.000000in}}{%
\pgfpathmoveto{\pgfqpoint{-0.000000in}{0.000000in}}%
\pgfpathlineto{\pgfqpoint{-0.027778in}{0.000000in}}%
\pgfusepath{stroke,fill}%
}%
\begin{pgfscope}%
\pgfsys@transformshift{0.721913in}{1.806836in}%
\pgfsys@useobject{currentmarker}{}%
\end{pgfscope}%
\end{pgfscope}%
\begin{pgfscope}%
\pgfsetbuttcap%
\pgfsetroundjoin%
\definecolor{currentfill}{rgb}{0.000000,0.000000,0.000000}%
\pgfsetfillcolor{currentfill}%
\pgfsetlinewidth{0.602250pt}%
\definecolor{currentstroke}{rgb}{0.000000,0.000000,0.000000}%
\pgfsetstrokecolor{currentstroke}%
\pgfsetdash{}{0pt}%
\pgfsys@defobject{currentmarker}{\pgfqpoint{-0.027778in}{0.000000in}}{\pgfqpoint{-0.000000in}{0.000000in}}{%
\pgfpathmoveto{\pgfqpoint{-0.000000in}{0.000000in}}%
\pgfpathlineto{\pgfqpoint{-0.027778in}{0.000000in}}%
\pgfusepath{stroke,fill}%
}%
\begin{pgfscope}%
\pgfsys@transformshift{0.721913in}{1.861572in}%
\pgfsys@useobject{currentmarker}{}%
\end{pgfscope}%
\end{pgfscope}%
\begin{pgfscope}%
\pgfsetbuttcap%
\pgfsetroundjoin%
\definecolor{currentfill}{rgb}{0.000000,0.000000,0.000000}%
\pgfsetfillcolor{currentfill}%
\pgfsetlinewidth{0.602250pt}%
\definecolor{currentstroke}{rgb}{0.000000,0.000000,0.000000}%
\pgfsetstrokecolor{currentstroke}%
\pgfsetdash{}{0pt}%
\pgfsys@defobject{currentmarker}{\pgfqpoint{-0.027778in}{0.000000in}}{\pgfqpoint{-0.000000in}{0.000000in}}{%
\pgfpathmoveto{\pgfqpoint{-0.000000in}{0.000000in}}%
\pgfpathlineto{\pgfqpoint{-0.027778in}{0.000000in}}%
\pgfusepath{stroke,fill}%
}%
\begin{pgfscope}%
\pgfsys@transformshift{0.721913in}{1.906295in}%
\pgfsys@useobject{currentmarker}{}%
\end{pgfscope}%
\end{pgfscope}%
\begin{pgfscope}%
\pgfsetbuttcap%
\pgfsetroundjoin%
\definecolor{currentfill}{rgb}{0.000000,0.000000,0.000000}%
\pgfsetfillcolor{currentfill}%
\pgfsetlinewidth{0.602250pt}%
\definecolor{currentstroke}{rgb}{0.000000,0.000000,0.000000}%
\pgfsetstrokecolor{currentstroke}%
\pgfsetdash{}{0pt}%
\pgfsys@defobject{currentmarker}{\pgfqpoint{-0.027778in}{0.000000in}}{\pgfqpoint{-0.000000in}{0.000000in}}{%
\pgfpathmoveto{\pgfqpoint{-0.000000in}{0.000000in}}%
\pgfpathlineto{\pgfqpoint{-0.027778in}{0.000000in}}%
\pgfusepath{stroke,fill}%
}%
\begin{pgfscope}%
\pgfsys@transformshift{0.721913in}{1.944108in}%
\pgfsys@useobject{currentmarker}{}%
\end{pgfscope}%
\end{pgfscope}%
\begin{pgfscope}%
\pgfsetbuttcap%
\pgfsetroundjoin%
\definecolor{currentfill}{rgb}{0.000000,0.000000,0.000000}%
\pgfsetfillcolor{currentfill}%
\pgfsetlinewidth{0.602250pt}%
\definecolor{currentstroke}{rgb}{0.000000,0.000000,0.000000}%
\pgfsetstrokecolor{currentstroke}%
\pgfsetdash{}{0pt}%
\pgfsys@defobject{currentmarker}{\pgfqpoint{-0.027778in}{0.000000in}}{\pgfqpoint{-0.000000in}{0.000000in}}{%
\pgfpathmoveto{\pgfqpoint{-0.000000in}{0.000000in}}%
\pgfpathlineto{\pgfqpoint{-0.027778in}{0.000000in}}%
\pgfusepath{stroke,fill}%
}%
\begin{pgfscope}%
\pgfsys@transformshift{0.721913in}{1.976863in}%
\pgfsys@useobject{currentmarker}{}%
\end{pgfscope}%
\end{pgfscope}%
\begin{pgfscope}%
\pgfsetbuttcap%
\pgfsetroundjoin%
\definecolor{currentfill}{rgb}{0.000000,0.000000,0.000000}%
\pgfsetfillcolor{currentfill}%
\pgfsetlinewidth{0.602250pt}%
\definecolor{currentstroke}{rgb}{0.000000,0.000000,0.000000}%
\pgfsetstrokecolor{currentstroke}%
\pgfsetdash{}{0pt}%
\pgfsys@defobject{currentmarker}{\pgfqpoint{-0.027778in}{0.000000in}}{\pgfqpoint{-0.000000in}{0.000000in}}{%
\pgfpathmoveto{\pgfqpoint{-0.000000in}{0.000000in}}%
\pgfpathlineto{\pgfqpoint{-0.027778in}{0.000000in}}%
\pgfusepath{stroke,fill}%
}%
\begin{pgfscope}%
\pgfsys@transformshift{0.721913in}{2.005755in}%
\pgfsys@useobject{currentmarker}{}%
\end{pgfscope}%
\end{pgfscope}%
\begin{pgfscope}%
\pgfsetbuttcap%
\pgfsetroundjoin%
\definecolor{currentfill}{rgb}{0.000000,0.000000,0.000000}%
\pgfsetfillcolor{currentfill}%
\pgfsetlinewidth{0.602250pt}%
\definecolor{currentstroke}{rgb}{0.000000,0.000000,0.000000}%
\pgfsetstrokecolor{currentstroke}%
\pgfsetdash{}{0pt}%
\pgfsys@defobject{currentmarker}{\pgfqpoint{-0.027778in}{0.000000in}}{\pgfqpoint{-0.000000in}{0.000000in}}{%
\pgfpathmoveto{\pgfqpoint{-0.000000in}{0.000000in}}%
\pgfpathlineto{\pgfqpoint{-0.027778in}{0.000000in}}%
\pgfusepath{stroke,fill}%
}%
\begin{pgfscope}%
\pgfsys@transformshift{0.721913in}{2.201627in}%
\pgfsys@useobject{currentmarker}{}%
\end{pgfscope}%
\end{pgfscope}%
\begin{pgfscope}%
\pgfsetbuttcap%
\pgfsetroundjoin%
\definecolor{currentfill}{rgb}{0.000000,0.000000,0.000000}%
\pgfsetfillcolor{currentfill}%
\pgfsetlinewidth{0.602250pt}%
\definecolor{currentstroke}{rgb}{0.000000,0.000000,0.000000}%
\pgfsetstrokecolor{currentstroke}%
\pgfsetdash{}{0pt}%
\pgfsys@defobject{currentmarker}{\pgfqpoint{-0.027778in}{0.000000in}}{\pgfqpoint{-0.000000in}{0.000000in}}{%
\pgfpathmoveto{\pgfqpoint{-0.000000in}{0.000000in}}%
\pgfpathlineto{\pgfqpoint{-0.027778in}{0.000000in}}%
\pgfusepath{stroke,fill}%
}%
\begin{pgfscope}%
\pgfsys@transformshift{0.721913in}{2.301087in}%
\pgfsys@useobject{currentmarker}{}%
\end{pgfscope}%
\end{pgfscope}%
\begin{pgfscope}%
\pgfsetbuttcap%
\pgfsetroundjoin%
\definecolor{currentfill}{rgb}{0.000000,0.000000,0.000000}%
\pgfsetfillcolor{currentfill}%
\pgfsetlinewidth{0.602250pt}%
\definecolor{currentstroke}{rgb}{0.000000,0.000000,0.000000}%
\pgfsetstrokecolor{currentstroke}%
\pgfsetdash{}{0pt}%
\pgfsys@defobject{currentmarker}{\pgfqpoint{-0.027778in}{0.000000in}}{\pgfqpoint{-0.000000in}{0.000000in}}{%
\pgfpathmoveto{\pgfqpoint{-0.000000in}{0.000000in}}%
\pgfpathlineto{\pgfqpoint{-0.027778in}{0.000000in}}%
\pgfusepath{stroke,fill}%
}%
\begin{pgfscope}%
\pgfsys@transformshift{0.721913in}{2.371655in}%
\pgfsys@useobject{currentmarker}{}%
\end{pgfscope}%
\end{pgfscope}%
\begin{pgfscope}%
\pgfsetbuttcap%
\pgfsetroundjoin%
\definecolor{currentfill}{rgb}{0.000000,0.000000,0.000000}%
\pgfsetfillcolor{currentfill}%
\pgfsetlinewidth{0.602250pt}%
\definecolor{currentstroke}{rgb}{0.000000,0.000000,0.000000}%
\pgfsetstrokecolor{currentstroke}%
\pgfsetdash{}{0pt}%
\pgfsys@defobject{currentmarker}{\pgfqpoint{-0.027778in}{0.000000in}}{\pgfqpoint{-0.000000in}{0.000000in}}{%
\pgfpathmoveto{\pgfqpoint{-0.000000in}{0.000000in}}%
\pgfpathlineto{\pgfqpoint{-0.027778in}{0.000000in}}%
\pgfusepath{stroke,fill}%
}%
\begin{pgfscope}%
\pgfsys@transformshift{0.721913in}{2.426391in}%
\pgfsys@useobject{currentmarker}{}%
\end{pgfscope}%
\end{pgfscope}%
\begin{pgfscope}%
\pgfsetbuttcap%
\pgfsetroundjoin%
\definecolor{currentfill}{rgb}{0.000000,0.000000,0.000000}%
\pgfsetfillcolor{currentfill}%
\pgfsetlinewidth{0.602250pt}%
\definecolor{currentstroke}{rgb}{0.000000,0.000000,0.000000}%
\pgfsetstrokecolor{currentstroke}%
\pgfsetdash{}{0pt}%
\pgfsys@defobject{currentmarker}{\pgfqpoint{-0.027778in}{0.000000in}}{\pgfqpoint{-0.000000in}{0.000000in}}{%
\pgfpathmoveto{\pgfqpoint{-0.000000in}{0.000000in}}%
\pgfpathlineto{\pgfqpoint{-0.027778in}{0.000000in}}%
\pgfusepath{stroke,fill}%
}%
\begin{pgfscope}%
\pgfsys@transformshift{0.721913in}{2.471114in}%
\pgfsys@useobject{currentmarker}{}%
\end{pgfscope}%
\end{pgfscope}%
\begin{pgfscope}%
\definecolor{textcolor}{rgb}{0.000000,0.000000,0.000000}%
\pgfsetstrokecolor{textcolor}%
\pgfsetfillcolor{textcolor}%
\pgftext[x=0.248148in,y=1.511573in,,bottom,rotate=90.000000]{\color{textcolor}{\rmfamily\fontsize{12.000000}{14.400000}\selectfont\catcode`\^=\active\def^{\ifmmode\sp\else\^{}\fi}\catcode`\%=\active\def%{\%}$L^1$ relative error}}%
\end{pgfscope}%
\begin{pgfscope}%
\pgfpathrectangle{\pgfqpoint{0.721913in}{0.549073in}}{\pgfqpoint{1.937500in}{1.925000in}}%
\pgfusepath{clip}%
\pgfsetbuttcap%
\pgfsetroundjoin%
\pgfsetlinewidth{1.505625pt}%
\definecolor{currentstroke}{rgb}{0.478431,0.478431,0.478431}%
\pgfsetstrokecolor{currentstroke}%
\pgfsetstrokeopacity{0.500000}%
\pgfsetdash{{5.550000pt}{2.400000pt}}{0.000000pt}%
\pgfpathmoveto{\pgfqpoint{1.180017in}{1.611720in}}%
\pgfpathlineto{\pgfqpoint{1.530977in}{1.463036in}}%
\pgfpathlineto{\pgfqpoint{1.877132in}{1.316388in}}%
\pgfpathlineto{\pgfqpoint{2.222805in}{1.169944in}}%
\pgfusepath{stroke}%
\end{pgfscope}%
\begin{pgfscope}%
\pgfpathrectangle{\pgfqpoint{0.721913in}{0.549073in}}{\pgfqpoint{1.937500in}{1.925000in}}%
\pgfusepath{clip}%
\pgfsetbuttcap%
\pgfsetroundjoin%
\pgfsetlinewidth{1.505625pt}%
\definecolor{currentstroke}{rgb}{0.478431,0.478431,0.478431}%
\pgfsetstrokecolor{currentstroke}%
\pgfsetstrokeopacity{0.500000}%
\pgfsetdash{{5.550000pt}{2.400000pt}}{0.000000pt}%
\pgfpathmoveto{\pgfqpoint{1.180017in}{1.861572in}}%
\pgfpathlineto{\pgfqpoint{1.530977in}{1.787231in}}%
\pgfpathlineto{\pgfqpoint{1.877132in}{1.713907in}}%
\pgfpathlineto{\pgfqpoint{2.222805in}{1.640685in}}%
\pgfusepath{stroke}%
\end{pgfscope}%
\begin{pgfscope}%
\pgfpathrectangle{\pgfqpoint{0.721913in}{0.549073in}}{\pgfqpoint{1.937500in}{1.925000in}}%
\pgfusepath{clip}%
\pgfsetrectcap%
\pgfsetroundjoin%
\pgfsetlinewidth{1.003750pt}%
\definecolor{currentstroke}{rgb}{0.537255,0.647059,0.760784}%
\pgfsetstrokecolor{currentstroke}%
\pgfsetdash{}{0pt}%
\pgfpathmoveto{\pgfqpoint{0.809982in}{1.871112in}}%
\pgfpathlineto{\pgfqpoint{1.180017in}{1.793124in}}%
\pgfpathlineto{\pgfqpoint{1.530977in}{1.763923in}}%
\pgfpathlineto{\pgfqpoint{1.877132in}{1.671013in}}%
\pgfpathlineto{\pgfqpoint{2.222805in}{1.598221in}}%
\pgfpathlineto{\pgfqpoint{2.571345in}{1.529340in}}%
\pgfusepath{stroke}%
\end{pgfscope}%
\begin{pgfscope}%
\pgfpathrectangle{\pgfqpoint{0.721913in}{0.549073in}}{\pgfqpoint{1.937500in}{1.925000in}}%
\pgfusepath{clip}%
\pgfsetbuttcap%
\pgfsetroundjoin%
\definecolor{currentfill}{rgb}{0.537255,0.647059,0.760784}%
\pgfsetfillcolor{currentfill}%
\pgfsetlinewidth{1.003750pt}%
\definecolor{currentstroke}{rgb}{0.537255,0.647059,0.760784}%
\pgfsetstrokecolor{currentstroke}%
\pgfsetdash{}{0pt}%
\pgfsys@defobject{currentmarker}{\pgfqpoint{-0.020833in}{-0.020833in}}{\pgfqpoint{0.020833in}{0.020833in}}{%
\pgfpathmoveto{\pgfqpoint{0.000000in}{-0.020833in}}%
\pgfpathcurveto{\pgfqpoint{0.005525in}{-0.020833in}}{\pgfqpoint{0.010825in}{-0.018638in}}{\pgfqpoint{0.014731in}{-0.014731in}}%
\pgfpathcurveto{\pgfqpoint{0.018638in}{-0.010825in}}{\pgfqpoint{0.020833in}{-0.005525in}}{\pgfqpoint{0.020833in}{0.000000in}}%
\pgfpathcurveto{\pgfqpoint{0.020833in}{0.005525in}}{\pgfqpoint{0.018638in}{0.010825in}}{\pgfqpoint{0.014731in}{0.014731in}}%
\pgfpathcurveto{\pgfqpoint{0.010825in}{0.018638in}}{\pgfqpoint{0.005525in}{0.020833in}}{\pgfqpoint{0.000000in}{0.020833in}}%
\pgfpathcurveto{\pgfqpoint{-0.005525in}{0.020833in}}{\pgfqpoint{-0.010825in}{0.018638in}}{\pgfqpoint{-0.014731in}{0.014731in}}%
\pgfpathcurveto{\pgfqpoint{-0.018638in}{0.010825in}}{\pgfqpoint{-0.020833in}{0.005525in}}{\pgfqpoint{-0.020833in}{0.000000in}}%
\pgfpathcurveto{\pgfqpoint{-0.020833in}{-0.005525in}}{\pgfqpoint{-0.018638in}{-0.010825in}}{\pgfqpoint{-0.014731in}{-0.014731in}}%
\pgfpathcurveto{\pgfqpoint{-0.010825in}{-0.018638in}}{\pgfqpoint{-0.005525in}{-0.020833in}}{\pgfqpoint{0.000000in}{-0.020833in}}%
\pgfpathlineto{\pgfqpoint{0.000000in}{-0.020833in}}%
\pgfpathclose%
\pgfusepath{stroke,fill}%
}%
\begin{pgfscope}%
\pgfsys@transformshift{0.809982in}{1.871112in}%
\pgfsys@useobject{currentmarker}{}%
\end{pgfscope}%
\begin{pgfscope}%
\pgfsys@transformshift{1.180017in}{1.793124in}%
\pgfsys@useobject{currentmarker}{}%
\end{pgfscope}%
\begin{pgfscope}%
\pgfsys@transformshift{1.530977in}{1.763923in}%
\pgfsys@useobject{currentmarker}{}%
\end{pgfscope}%
\begin{pgfscope}%
\pgfsys@transformshift{1.877132in}{1.671013in}%
\pgfsys@useobject{currentmarker}{}%
\end{pgfscope}%
\begin{pgfscope}%
\pgfsys@transformshift{2.222805in}{1.598221in}%
\pgfsys@useobject{currentmarker}{}%
\end{pgfscope}%
\begin{pgfscope}%
\pgfsys@transformshift{2.571345in}{1.529340in}%
\pgfsys@useobject{currentmarker}{}%
\end{pgfscope}%
\end{pgfscope}%
\begin{pgfscope}%
\pgfpathrectangle{\pgfqpoint{0.721913in}{0.549073in}}{\pgfqpoint{1.937500in}{1.925000in}}%
\pgfusepath{clip}%
\pgfsetrectcap%
\pgfsetroundjoin%
\pgfsetlinewidth{1.003750pt}%
\definecolor{currentstroke}{rgb}{0.184314,0.270588,0.360784}%
\pgfsetstrokecolor{currentstroke}%
\pgfsetdash{}{0pt}%
\pgfpathmoveto{\pgfqpoint{0.809982in}{2.386573in}}%
\pgfpathlineto{\pgfqpoint{1.180017in}{2.253154in}}%
\pgfpathlineto{\pgfqpoint{1.530977in}{2.070706in}}%
\pgfpathlineto{\pgfqpoint{1.877132in}{1.855636in}}%
\pgfpathlineto{\pgfqpoint{2.222805in}{1.665629in}}%
\pgfpathlineto{\pgfqpoint{2.571345in}{1.447004in}}%
\pgfusepath{stroke}%
\end{pgfscope}%
\begin{pgfscope}%
\pgfpathrectangle{\pgfqpoint{0.721913in}{0.549073in}}{\pgfqpoint{1.937500in}{1.925000in}}%
\pgfusepath{clip}%
\pgfsetbuttcap%
\pgfsetroundjoin%
\definecolor{currentfill}{rgb}{0.184314,0.270588,0.360784}%
\pgfsetfillcolor{currentfill}%
\pgfsetlinewidth{1.003750pt}%
\definecolor{currentstroke}{rgb}{0.184314,0.270588,0.360784}%
\pgfsetstrokecolor{currentstroke}%
\pgfsetdash{}{0pt}%
\pgfsys@defobject{currentmarker}{\pgfqpoint{-0.020833in}{-0.020833in}}{\pgfqpoint{0.020833in}{0.020833in}}{%
\pgfpathmoveto{\pgfqpoint{0.000000in}{-0.020833in}}%
\pgfpathcurveto{\pgfqpoint{0.005525in}{-0.020833in}}{\pgfqpoint{0.010825in}{-0.018638in}}{\pgfqpoint{0.014731in}{-0.014731in}}%
\pgfpathcurveto{\pgfqpoint{0.018638in}{-0.010825in}}{\pgfqpoint{0.020833in}{-0.005525in}}{\pgfqpoint{0.020833in}{0.000000in}}%
\pgfpathcurveto{\pgfqpoint{0.020833in}{0.005525in}}{\pgfqpoint{0.018638in}{0.010825in}}{\pgfqpoint{0.014731in}{0.014731in}}%
\pgfpathcurveto{\pgfqpoint{0.010825in}{0.018638in}}{\pgfqpoint{0.005525in}{0.020833in}}{\pgfqpoint{0.000000in}{0.020833in}}%
\pgfpathcurveto{\pgfqpoint{-0.005525in}{0.020833in}}{\pgfqpoint{-0.010825in}{0.018638in}}{\pgfqpoint{-0.014731in}{0.014731in}}%
\pgfpathcurveto{\pgfqpoint{-0.018638in}{0.010825in}}{\pgfqpoint{-0.020833in}{0.005525in}}{\pgfqpoint{-0.020833in}{0.000000in}}%
\pgfpathcurveto{\pgfqpoint{-0.020833in}{-0.005525in}}{\pgfqpoint{-0.018638in}{-0.010825in}}{\pgfqpoint{-0.014731in}{-0.014731in}}%
\pgfpathcurveto{\pgfqpoint{-0.010825in}{-0.018638in}}{\pgfqpoint{-0.005525in}{-0.020833in}}{\pgfqpoint{0.000000in}{-0.020833in}}%
\pgfpathlineto{\pgfqpoint{0.000000in}{-0.020833in}}%
\pgfpathclose%
\pgfusepath{stroke,fill}%
}%
\begin{pgfscope}%
\pgfsys@transformshift{0.809982in}{2.386573in}%
\pgfsys@useobject{currentmarker}{}%
\end{pgfscope}%
\begin{pgfscope}%
\pgfsys@transformshift{1.180017in}{2.253154in}%
\pgfsys@useobject{currentmarker}{}%
\end{pgfscope}%
\begin{pgfscope}%
\pgfsys@transformshift{1.530977in}{2.070706in}%
\pgfsys@useobject{currentmarker}{}%
\end{pgfscope}%
\begin{pgfscope}%
\pgfsys@transformshift{1.877132in}{1.855636in}%
\pgfsys@useobject{currentmarker}{}%
\end{pgfscope}%
\begin{pgfscope}%
\pgfsys@transformshift{2.222805in}{1.665629in}%
\pgfsys@useobject{currentmarker}{}%
\end{pgfscope}%
\begin{pgfscope}%
\pgfsys@transformshift{2.571345in}{1.447004in}%
\pgfsys@useobject{currentmarker}{}%
\end{pgfscope}%
\end{pgfscope}%
\begin{pgfscope}%
\pgfpathrectangle{\pgfqpoint{0.721913in}{0.549073in}}{\pgfqpoint{1.937500in}{1.925000in}}%
\pgfusepath{clip}%
\pgfsetrectcap%
\pgfsetroundjoin%
\pgfsetlinewidth{1.003750pt}%
\definecolor{currentstroke}{rgb}{0.976471,0.505882,0.145098}%
\pgfsetstrokecolor{currentstroke}%
\pgfsetdash{}{0pt}%
\pgfpathmoveto{\pgfqpoint{0.809982in}{1.872825in}}%
\pgfpathlineto{\pgfqpoint{1.180017in}{1.557985in}}%
\pgfpathlineto{\pgfqpoint{1.530977in}{1.419006in}}%
\pgfpathlineto{\pgfqpoint{1.877132in}{1.151211in}}%
\pgfpathlineto{\pgfqpoint{2.222805in}{0.838795in}}%
\pgfpathlineto{\pgfqpoint{2.571345in}{0.636573in}}%
\pgfusepath{stroke}%
\end{pgfscope}%
\begin{pgfscope}%
\pgfpathrectangle{\pgfqpoint{0.721913in}{0.549073in}}{\pgfqpoint{1.937500in}{1.925000in}}%
\pgfusepath{clip}%
\pgfsetbuttcap%
\pgfsetroundjoin%
\definecolor{currentfill}{rgb}{0.976471,0.505882,0.145098}%
\pgfsetfillcolor{currentfill}%
\pgfsetlinewidth{1.003750pt}%
\definecolor{currentstroke}{rgb}{0.976471,0.505882,0.145098}%
\pgfsetstrokecolor{currentstroke}%
\pgfsetdash{}{0pt}%
\pgfsys@defobject{currentmarker}{\pgfqpoint{-0.020833in}{-0.020833in}}{\pgfqpoint{0.020833in}{0.020833in}}{%
\pgfpathmoveto{\pgfqpoint{0.000000in}{-0.020833in}}%
\pgfpathcurveto{\pgfqpoint{0.005525in}{-0.020833in}}{\pgfqpoint{0.010825in}{-0.018638in}}{\pgfqpoint{0.014731in}{-0.014731in}}%
\pgfpathcurveto{\pgfqpoint{0.018638in}{-0.010825in}}{\pgfqpoint{0.020833in}{-0.005525in}}{\pgfqpoint{0.020833in}{0.000000in}}%
\pgfpathcurveto{\pgfqpoint{0.020833in}{0.005525in}}{\pgfqpoint{0.018638in}{0.010825in}}{\pgfqpoint{0.014731in}{0.014731in}}%
\pgfpathcurveto{\pgfqpoint{0.010825in}{0.018638in}}{\pgfqpoint{0.005525in}{0.020833in}}{\pgfqpoint{0.000000in}{0.020833in}}%
\pgfpathcurveto{\pgfqpoint{-0.005525in}{0.020833in}}{\pgfqpoint{-0.010825in}{0.018638in}}{\pgfqpoint{-0.014731in}{0.014731in}}%
\pgfpathcurveto{\pgfqpoint{-0.018638in}{0.010825in}}{\pgfqpoint{-0.020833in}{0.005525in}}{\pgfqpoint{-0.020833in}{0.000000in}}%
\pgfpathcurveto{\pgfqpoint{-0.020833in}{-0.005525in}}{\pgfqpoint{-0.018638in}{-0.010825in}}{\pgfqpoint{-0.014731in}{-0.014731in}}%
\pgfpathcurveto{\pgfqpoint{-0.010825in}{-0.018638in}}{\pgfqpoint{-0.005525in}{-0.020833in}}{\pgfqpoint{0.000000in}{-0.020833in}}%
\pgfpathlineto{\pgfqpoint{0.000000in}{-0.020833in}}%
\pgfpathclose%
\pgfusepath{stroke,fill}%
}%
\begin{pgfscope}%
\pgfsys@transformshift{0.809982in}{1.872825in}%
\pgfsys@useobject{currentmarker}{}%
\end{pgfscope}%
\begin{pgfscope}%
\pgfsys@transformshift{1.180017in}{1.557985in}%
\pgfsys@useobject{currentmarker}{}%
\end{pgfscope}%
\begin{pgfscope}%
\pgfsys@transformshift{1.530977in}{1.419006in}%
\pgfsys@useobject{currentmarker}{}%
\end{pgfscope}%
\begin{pgfscope}%
\pgfsys@transformshift{1.877132in}{1.151211in}%
\pgfsys@useobject{currentmarker}{}%
\end{pgfscope}%
\begin{pgfscope}%
\pgfsys@transformshift{2.222805in}{0.838795in}%
\pgfsys@useobject{currentmarker}{}%
\end{pgfscope}%
\begin{pgfscope}%
\pgfsys@transformshift{2.571345in}{0.636573in}%
\pgfsys@useobject{currentmarker}{}%
\end{pgfscope}%
\end{pgfscope}%
\begin{pgfscope}%
\pgfsetrectcap%
\pgfsetmiterjoin%
\pgfsetlinewidth{0.803000pt}%
\definecolor{currentstroke}{rgb}{0.000000,0.000000,0.000000}%
\pgfsetstrokecolor{currentstroke}%
\pgfsetdash{}{0pt}%
\pgfpathmoveto{\pgfqpoint{0.721913in}{0.549073in}}%
\pgfpathlineto{\pgfqpoint{0.721913in}{2.474073in}}%
\pgfusepath{stroke}%
\end{pgfscope}%
\begin{pgfscope}%
\pgfsetrectcap%
\pgfsetmiterjoin%
\pgfsetlinewidth{0.803000pt}%
\definecolor{currentstroke}{rgb}{0.000000,0.000000,0.000000}%
\pgfsetstrokecolor{currentstroke}%
\pgfsetdash{}{0pt}%
\pgfpathmoveto{\pgfqpoint{2.659413in}{0.549073in}}%
\pgfpathlineto{\pgfqpoint{2.659413in}{2.474073in}}%
\pgfusepath{stroke}%
\end{pgfscope}%
\begin{pgfscope}%
\pgfsetrectcap%
\pgfsetmiterjoin%
\pgfsetlinewidth{0.803000pt}%
\definecolor{currentstroke}{rgb}{0.000000,0.000000,0.000000}%
\pgfsetstrokecolor{currentstroke}%
\pgfsetdash{}{0pt}%
\pgfpathmoveto{\pgfqpoint{0.721913in}{0.549073in}}%
\pgfpathlineto{\pgfqpoint{2.659413in}{0.549073in}}%
\pgfusepath{stroke}%
\end{pgfscope}%
\begin{pgfscope}%
\pgfsetrectcap%
\pgfsetmiterjoin%
\pgfsetlinewidth{0.803000pt}%
\definecolor{currentstroke}{rgb}{0.000000,0.000000,0.000000}%
\pgfsetstrokecolor{currentstroke}%
\pgfsetdash{}{0pt}%
\pgfpathmoveto{\pgfqpoint{0.721913in}{2.474073in}}%
\pgfpathlineto{\pgfqpoint{2.659413in}{2.474073in}}%
\pgfusepath{stroke}%
\end{pgfscope}%
\begin{pgfscope}%
\definecolor{textcolor}{rgb}{0.478431,0.478431,0.478431}%
\pgfsetstrokecolor{textcolor}%
\pgfsetfillcolor{textcolor}%
\pgftext[x=1.642363in,y=1.440936in,left,base]{\color{textcolor}{\rmfamily\fontsize{12.000000}{14.400000}\selectfont\catcode`\^=\active\def^{\ifmmode\sp\else\^{}\fi}\catcode`\%=\active\def%{\%}$\mathcal{O}(\varepsilon^{-1})$}}%
\end{pgfscope}%
\begin{pgfscope}%
\definecolor{textcolor}{rgb}{0.478431,0.478431,0.478431}%
\pgfsetstrokecolor{textcolor}%
\pgfsetfillcolor{textcolor}%
\pgftext[x=0.991593in,y=1.925930in,left,base]{\color{textcolor}{\rmfamily\fontsize{12.000000}{14.400000}\selectfont\catcode`\^=\active\def^{\ifmmode\sp\else\^{}\fi}\catcode`\%=\active\def%{\%}$\mathcal{O}(\varepsilon^{-2})$}}%
\end{pgfscope}%
\begin{pgfscope}%
\pgfsetbuttcap%
\pgfsetmiterjoin%
\definecolor{currentfill}{rgb}{1.000000,1.000000,1.000000}%
\pgfsetfillcolor{currentfill}%
\pgfsetfillopacity{0.800000}%
\pgfsetlinewidth{1.003750pt}%
\definecolor{currentstroke}{rgb}{0.800000,0.800000,0.800000}%
\pgfsetstrokecolor{currentstroke}%
\pgfsetstrokeopacity{0.800000}%
\pgfsetdash{}{0pt}%
\pgfpathmoveto{\pgfqpoint{0.838580in}{0.632406in}}%
\pgfpathlineto{\pgfqpoint{1.988362in}{0.632406in}}%
\pgfpathquadraticcurveto{\pgfqpoint{2.021695in}{0.632406in}}{\pgfqpoint{2.021695in}{0.665739in}}%
\pgfpathlineto{\pgfqpoint{2.021695in}{1.346294in}}%
\pgfpathquadraticcurveto{\pgfqpoint{2.021695in}{1.379627in}}{\pgfqpoint{1.988362in}{1.379627in}}%
\pgfpathlineto{\pgfqpoint{0.838580in}{1.379627in}}%
\pgfpathquadraticcurveto{\pgfqpoint{0.805247in}{1.379627in}}{\pgfqpoint{0.805247in}{1.346294in}}%
\pgfpathlineto{\pgfqpoint{0.805247in}{0.665739in}}%
\pgfpathquadraticcurveto{\pgfqpoint{0.805247in}{0.632406in}}{\pgfqpoint{0.838580in}{0.632406in}}%
\pgfpathlineto{\pgfqpoint{0.838580in}{0.632406in}}%
\pgfpathclose%
\pgfusepath{stroke,fill}%
\end{pgfscope}%
\begin{pgfscope}%
\pgfsetrectcap%
\pgfsetroundjoin%
\pgfsetlinewidth{1.003750pt}%
\definecolor{currentstroke}{rgb}{0.537255,0.647059,0.760784}%
\pgfsetstrokecolor{currentstroke}%
\pgfsetdash{}{0pt}%
\pgfpathmoveto{\pgfqpoint{0.871913in}{1.254627in}}%
\pgfpathlineto{\pgfqpoint{1.038580in}{1.254627in}}%
\pgfpathlineto{\pgfqpoint{1.205247in}{1.254627in}}%
\pgfusepath{stroke}%
\end{pgfscope}%
\begin{pgfscope}%
\pgfsetbuttcap%
\pgfsetroundjoin%
\definecolor{currentfill}{rgb}{0.537255,0.647059,0.760784}%
\pgfsetfillcolor{currentfill}%
\pgfsetlinewidth{1.003750pt}%
\definecolor{currentstroke}{rgb}{0.537255,0.647059,0.760784}%
\pgfsetstrokecolor{currentstroke}%
\pgfsetdash{}{0pt}%
\pgfsys@defobject{currentmarker}{\pgfqpoint{-0.020833in}{-0.020833in}}{\pgfqpoint{0.020833in}{0.020833in}}{%
\pgfpathmoveto{\pgfqpoint{0.000000in}{-0.020833in}}%
\pgfpathcurveto{\pgfqpoint{0.005525in}{-0.020833in}}{\pgfqpoint{0.010825in}{-0.018638in}}{\pgfqpoint{0.014731in}{-0.014731in}}%
\pgfpathcurveto{\pgfqpoint{0.018638in}{-0.010825in}}{\pgfqpoint{0.020833in}{-0.005525in}}{\pgfqpoint{0.020833in}{0.000000in}}%
\pgfpathcurveto{\pgfqpoint{0.020833in}{0.005525in}}{\pgfqpoint{0.018638in}{0.010825in}}{\pgfqpoint{0.014731in}{0.014731in}}%
\pgfpathcurveto{\pgfqpoint{0.010825in}{0.018638in}}{\pgfqpoint{0.005525in}{0.020833in}}{\pgfqpoint{0.000000in}{0.020833in}}%
\pgfpathcurveto{\pgfqpoint{-0.005525in}{0.020833in}}{\pgfqpoint{-0.010825in}{0.018638in}}{\pgfqpoint{-0.014731in}{0.014731in}}%
\pgfpathcurveto{\pgfqpoint{-0.018638in}{0.010825in}}{\pgfqpoint{-0.020833in}{0.005525in}}{\pgfqpoint{-0.020833in}{0.000000in}}%
\pgfpathcurveto{\pgfqpoint{-0.020833in}{-0.005525in}}{\pgfqpoint{-0.018638in}{-0.010825in}}{\pgfqpoint{-0.014731in}{-0.014731in}}%
\pgfpathcurveto{\pgfqpoint{-0.010825in}{-0.018638in}}{\pgfqpoint{-0.005525in}{-0.020833in}}{\pgfqpoint{0.000000in}{-0.020833in}}%
\pgfpathlineto{\pgfqpoint{0.000000in}{-0.020833in}}%
\pgfpathclose%
\pgfusepath{stroke,fill}%
}%
\begin{pgfscope}%
\pgfsys@transformshift{1.038580in}{1.254627in}%
\pgfsys@useobject{currentmarker}{}%
\end{pgfscope}%
\end{pgfscope}%
\begin{pgfscope}%
\definecolor{textcolor}{rgb}{0.000000,0.000000,0.000000}%
\pgfsetstrokecolor{textcolor}%
\pgfsetfillcolor{textcolor}%
\pgftext[x=1.338580in,y=1.196294in,left,base]{\color{textcolor}{\rmfamily\fontsize{12.000000}{14.400000}\selectfont\catcode`\^=\active\def^{\ifmmode\sp\else\^{}\fi}\catcode`\%=\active\def%{\%}Haydock}}%
\end{pgfscope}%
\begin{pgfscope}%
\pgfsetrectcap%
\pgfsetroundjoin%
\pgfsetlinewidth{1.003750pt}%
\definecolor{currentstroke}{rgb}{0.184314,0.270588,0.360784}%
\pgfsetstrokecolor{currentstroke}%
\pgfsetdash{}{0pt}%
\pgfpathmoveto{\pgfqpoint{0.871913in}{1.022220in}}%
\pgfpathlineto{\pgfqpoint{1.038580in}{1.022220in}}%
\pgfpathlineto{\pgfqpoint{1.205247in}{1.022220in}}%
\pgfusepath{stroke}%
\end{pgfscope}%
\begin{pgfscope}%
\pgfsetbuttcap%
\pgfsetroundjoin%
\definecolor{currentfill}{rgb}{0.184314,0.270588,0.360784}%
\pgfsetfillcolor{currentfill}%
\pgfsetlinewidth{1.003750pt}%
\definecolor{currentstroke}{rgb}{0.184314,0.270588,0.360784}%
\pgfsetstrokecolor{currentstroke}%
\pgfsetdash{}{0pt}%
\pgfsys@defobject{currentmarker}{\pgfqpoint{-0.020833in}{-0.020833in}}{\pgfqpoint{0.020833in}{0.020833in}}{%
\pgfpathmoveto{\pgfqpoint{0.000000in}{-0.020833in}}%
\pgfpathcurveto{\pgfqpoint{0.005525in}{-0.020833in}}{\pgfqpoint{0.010825in}{-0.018638in}}{\pgfqpoint{0.014731in}{-0.014731in}}%
\pgfpathcurveto{\pgfqpoint{0.018638in}{-0.010825in}}{\pgfqpoint{0.020833in}{-0.005525in}}{\pgfqpoint{0.020833in}{0.000000in}}%
\pgfpathcurveto{\pgfqpoint{0.020833in}{0.005525in}}{\pgfqpoint{0.018638in}{0.010825in}}{\pgfqpoint{0.014731in}{0.014731in}}%
\pgfpathcurveto{\pgfqpoint{0.010825in}{0.018638in}}{\pgfqpoint{0.005525in}{0.020833in}}{\pgfqpoint{0.000000in}{0.020833in}}%
\pgfpathcurveto{\pgfqpoint{-0.005525in}{0.020833in}}{\pgfqpoint{-0.010825in}{0.018638in}}{\pgfqpoint{-0.014731in}{0.014731in}}%
\pgfpathcurveto{\pgfqpoint{-0.018638in}{0.010825in}}{\pgfqpoint{-0.020833in}{0.005525in}}{\pgfqpoint{-0.020833in}{0.000000in}}%
\pgfpathcurveto{\pgfqpoint{-0.020833in}{-0.005525in}}{\pgfqpoint{-0.018638in}{-0.010825in}}{\pgfqpoint{-0.014731in}{-0.014731in}}%
\pgfpathcurveto{\pgfqpoint{-0.010825in}{-0.018638in}}{\pgfqpoint{-0.005525in}{-0.020833in}}{\pgfqpoint{0.000000in}{-0.020833in}}%
\pgfpathlineto{\pgfqpoint{0.000000in}{-0.020833in}}%
\pgfpathclose%
\pgfusepath{stroke,fill}%
}%
\begin{pgfscope}%
\pgfsys@transformshift{1.038580in}{1.022220in}%
\pgfsys@useobject{currentmarker}{}%
\end{pgfscope}%
\end{pgfscope}%
\begin{pgfscope}%
\definecolor{textcolor}{rgb}{0.000000,0.000000,0.000000}%
\pgfsetstrokecolor{textcolor}%
\pgfsetfillcolor{textcolor}%
\pgftext[x=1.338580in,y=0.963887in,left,base]{\color{textcolor}{\rmfamily\fontsize{12.000000}{14.400000}\selectfont\catcode`\^=\active\def^{\ifmmode\sp\else\^{}\fi}\catcode`\%=\active\def%{\%}NC}}%
\end{pgfscope}%
\begin{pgfscope}%
\pgfsetrectcap%
\pgfsetroundjoin%
\pgfsetlinewidth{1.003750pt}%
\definecolor{currentstroke}{rgb}{0.976471,0.505882,0.145098}%
\pgfsetstrokecolor{currentstroke}%
\pgfsetdash{}{0pt}%
\pgfpathmoveto{\pgfqpoint{0.871913in}{0.789813in}}%
\pgfpathlineto{\pgfqpoint{1.038580in}{0.789813in}}%
\pgfpathlineto{\pgfqpoint{1.205247in}{0.789813in}}%
\pgfusepath{stroke}%
\end{pgfscope}%
\begin{pgfscope}%
\pgfsetbuttcap%
\pgfsetroundjoin%
\definecolor{currentfill}{rgb}{0.976471,0.505882,0.145098}%
\pgfsetfillcolor{currentfill}%
\pgfsetlinewidth{1.003750pt}%
\definecolor{currentstroke}{rgb}{0.976471,0.505882,0.145098}%
\pgfsetstrokecolor{currentstroke}%
\pgfsetdash{}{0pt}%
\pgfsys@defobject{currentmarker}{\pgfqpoint{-0.020833in}{-0.020833in}}{\pgfqpoint{0.020833in}{0.020833in}}{%
\pgfpathmoveto{\pgfqpoint{0.000000in}{-0.020833in}}%
\pgfpathcurveto{\pgfqpoint{0.005525in}{-0.020833in}}{\pgfqpoint{0.010825in}{-0.018638in}}{\pgfqpoint{0.014731in}{-0.014731in}}%
\pgfpathcurveto{\pgfqpoint{0.018638in}{-0.010825in}}{\pgfqpoint{0.020833in}{-0.005525in}}{\pgfqpoint{0.020833in}{0.000000in}}%
\pgfpathcurveto{\pgfqpoint{0.020833in}{0.005525in}}{\pgfqpoint{0.018638in}{0.010825in}}{\pgfqpoint{0.014731in}{0.014731in}}%
\pgfpathcurveto{\pgfqpoint{0.010825in}{0.018638in}}{\pgfqpoint{0.005525in}{0.020833in}}{\pgfqpoint{0.000000in}{0.020833in}}%
\pgfpathcurveto{\pgfqpoint{-0.005525in}{0.020833in}}{\pgfqpoint{-0.010825in}{0.018638in}}{\pgfqpoint{-0.014731in}{0.014731in}}%
\pgfpathcurveto{\pgfqpoint{-0.018638in}{0.010825in}}{\pgfqpoint{-0.020833in}{0.005525in}}{\pgfqpoint{-0.020833in}{0.000000in}}%
\pgfpathcurveto{\pgfqpoint{-0.020833in}{-0.005525in}}{\pgfqpoint{-0.018638in}{-0.010825in}}{\pgfqpoint{-0.014731in}{-0.014731in}}%
\pgfpathcurveto{\pgfqpoint{-0.010825in}{-0.018638in}}{\pgfqpoint{-0.005525in}{-0.020833in}}{\pgfqpoint{0.000000in}{-0.020833in}}%
\pgfpathlineto{\pgfqpoint{0.000000in}{-0.020833in}}%
\pgfpathclose%
\pgfusepath{stroke,fill}%
}%
\begin{pgfscope}%
\pgfsys@transformshift{1.038580in}{0.789813in}%
\pgfsys@useobject{currentmarker}{}%
\end{pgfscope}%
\end{pgfscope}%
\begin{pgfscope}%
\definecolor{textcolor}{rgb}{0.000000,0.000000,0.000000}%
\pgfsetstrokecolor{textcolor}%
\pgfsetfillcolor{textcolor}%
\pgftext[x=1.338580in,y=0.731480in,left,base]{\color{textcolor}{\rmfamily\fontsize{12.000000}{14.400000}\selectfont\catcode`\^=\active\def^{\ifmmode\sp\else\^{}\fi}\catcode`\%=\active\def%{\%}NC++}}%
\end{pgfscope}%
\end{pgfpicture}%
\makeatother%
\endgroup%

        \caption{\gls{chebyshev-degree} $=2400$}
        \label{fig:5-experiments-haydock-convergence-nv-m2400}
    \end{subfigure}
    \caption{For increasing values of \gls{sketch-size} $+$ \gls{num-hutchinson-queries}
    but fixed \gls{chebyshev-degree} we plot the $L^1$ relative approximation error \refequ{equ:5-experiments-L1-error}
    for the model problem from \refsec{sec:5-experiments-density-function} with
    the Lorentzian kernel with \gls{smoothing-parameter} $=0.05$.}
    \label{fig:5-experiments-haydock-convergence-nv}
\end{figure}

\begin{figure}[ht]
    \centering
    \begin{subfigure}[b]{0.49\columnwidth}
        %% Creator: Matplotlib, PGF backend
%%
%% To include the figure in your LaTeX document, write
%%   \input{<filename>.pgf}
%%
%% Make sure the required packages are loaded in your preamble
%%   \usepackage{pgf}
%%
%% Also ensure that all the required font packages are loaded; for instance,
%% the lmodern package is sometimes necessary when using math font.
%%   \usepackage{lmodern}
%%
%% Figures using additional raster images can only be included by \input if
%% they are in the same directory as the main LaTeX file. For loading figures
%% from other directories you can use the `import` package
%%   \usepackage{import}
%%
%% and then include the figures with
%%   \import{<path to file>}{<filename>.pgf}
%%
%% Matplotlib used the following preamble
%%   \def\mathdefault#1{#1}
%%   \everymath=\expandafter{\the\everymath\displaystyle}
%%   
%%   \usepackage{fontspec}
%%   \setmainfont{DejaVuSerif.ttf}[Path=\detokenize{C:/Users/fabio/Documents/Work/MasterThesis/Rand-SD/.venv/Lib/site-packages/matplotlib/mpl-data/fonts/ttf/}]
%%   \setsansfont{DejaVuSans.ttf}[Path=\detokenize{C:/Users/fabio/Documents/Work/MasterThesis/Rand-SD/.venv/Lib/site-packages/matplotlib/mpl-data/fonts/ttf/}]
%%   \setmonofont{DejaVuSansMono.ttf}[Path=\detokenize{C:/Users/fabio/Documents/Work/MasterThesis/Rand-SD/.venv/Lib/site-packages/matplotlib/mpl-data/fonts/ttf/}]
%%   \makeatletter\@ifpackageloaded{underscore}{}{\usepackage[strings]{underscore}}\makeatother
%%
\begingroup%
\makeatletter%
\begin{pgfpicture}%
\pgfpathrectangle{\pgfpointorigin}{\pgfqpoint{2.712693in}{2.546603in}}%
\pgfusepath{use as bounding box, clip}%
\begin{pgfscope}%
\pgfsetbuttcap%
\pgfsetmiterjoin%
\definecolor{currentfill}{rgb}{1.000000,1.000000,1.000000}%
\pgfsetfillcolor{currentfill}%
\pgfsetlinewidth{0.000000pt}%
\definecolor{currentstroke}{rgb}{1.000000,1.000000,1.000000}%
\pgfsetstrokecolor{currentstroke}%
\pgfsetdash{}{0pt}%
\pgfpathmoveto{\pgfqpoint{0.000000in}{0.000000in}}%
\pgfpathlineto{\pgfqpoint{2.712693in}{0.000000in}}%
\pgfpathlineto{\pgfqpoint{2.712693in}{2.546603in}}%
\pgfpathlineto{\pgfqpoint{0.000000in}{2.546603in}}%
\pgfpathlineto{\pgfqpoint{0.000000in}{0.000000in}}%
\pgfpathclose%
\pgfusepath{fill}%
\end{pgfscope}%
\begin{pgfscope}%
\pgfsetbuttcap%
\pgfsetmiterjoin%
\definecolor{currentfill}{rgb}{1.000000,1.000000,1.000000}%
\pgfsetfillcolor{currentfill}%
\pgfsetlinewidth{0.000000pt}%
\definecolor{currentstroke}{rgb}{0.000000,0.000000,0.000000}%
\pgfsetstrokecolor{currentstroke}%
\pgfsetstrokeopacity{0.000000}%
\pgfsetdash{}{0pt}%
\pgfpathmoveto{\pgfqpoint{0.675193in}{0.521603in}}%
\pgfpathlineto{\pgfqpoint{2.612693in}{0.521603in}}%
\pgfpathlineto{\pgfqpoint{2.612693in}{2.446603in}}%
\pgfpathlineto{\pgfqpoint{0.675193in}{2.446603in}}%
\pgfpathlineto{\pgfqpoint{0.675193in}{0.521603in}}%
\pgfpathclose%
\pgfusepath{fill}%
\end{pgfscope}%
\begin{pgfscope}%
\pgfsetbuttcap%
\pgfsetroundjoin%
\definecolor{currentfill}{rgb}{0.000000,0.000000,0.000000}%
\pgfsetfillcolor{currentfill}%
\pgfsetlinewidth{0.803000pt}%
\definecolor{currentstroke}{rgb}{0.000000,0.000000,0.000000}%
\pgfsetstrokecolor{currentstroke}%
\pgfsetdash{}{0pt}%
\pgfsys@defobject{currentmarker}{\pgfqpoint{0.000000in}{-0.048611in}}{\pgfqpoint{0.000000in}{0.000000in}}{%
\pgfpathmoveto{\pgfqpoint{0.000000in}{0.000000in}}%
\pgfpathlineto{\pgfqpoint{0.000000in}{-0.048611in}}%
\pgfusepath{stroke,fill}%
}%
\begin{pgfscope}%
\pgfsys@transformshift{1.713853in}{0.521603in}%
\pgfsys@useobject{currentmarker}{}%
\end{pgfscope}%
\end{pgfscope}%
\begin{pgfscope}%
\definecolor{textcolor}{rgb}{0.000000,0.000000,0.000000}%
\pgfsetstrokecolor{textcolor}%
\pgfsetfillcolor{textcolor}%
\pgftext[x=1.713853in,y=0.424381in,,top]{\color{textcolor}{\sffamily\fontsize{10.000000}{12.000000}\selectfont\catcode`\^=\active\def^{\ifmmode\sp\else\^{}\fi}\catcode`\%=\active\def%{\%}$\mathdefault{10^{3}}$}}%
\end{pgfscope}%
\begin{pgfscope}%
\pgfsetbuttcap%
\pgfsetroundjoin%
\definecolor{currentfill}{rgb}{0.000000,0.000000,0.000000}%
\pgfsetfillcolor{currentfill}%
\pgfsetlinewidth{0.602250pt}%
\definecolor{currentstroke}{rgb}{0.000000,0.000000,0.000000}%
\pgfsetstrokecolor{currentstroke}%
\pgfsetdash{}{0pt}%
\pgfsys@defobject{currentmarker}{\pgfqpoint{0.000000in}{-0.027778in}}{\pgfqpoint{0.000000in}{0.000000in}}{%
\pgfpathmoveto{\pgfqpoint{0.000000in}{0.000000in}}%
\pgfpathlineto{\pgfqpoint{0.000000in}{-0.027778in}}%
\pgfusepath{stroke,fill}%
}%
\begin{pgfscope}%
\pgfsys@transformshift{0.769160in}{0.521603in}%
\pgfsys@useobject{currentmarker}{}%
\end{pgfscope}%
\end{pgfscope}%
\begin{pgfscope}%
\pgfsetbuttcap%
\pgfsetroundjoin%
\definecolor{currentfill}{rgb}{0.000000,0.000000,0.000000}%
\pgfsetfillcolor{currentfill}%
\pgfsetlinewidth{0.602250pt}%
\definecolor{currentstroke}{rgb}{0.000000,0.000000,0.000000}%
\pgfsetstrokecolor{currentstroke}%
\pgfsetdash{}{0pt}%
\pgfsys@defobject{currentmarker}{\pgfqpoint{0.000000in}{-0.027778in}}{\pgfqpoint{0.000000in}{0.000000in}}{%
\pgfpathmoveto{\pgfqpoint{0.000000in}{0.000000in}}%
\pgfpathlineto{\pgfqpoint{0.000000in}{-0.027778in}}%
\pgfusepath{stroke,fill}%
}%
\begin{pgfscope}%
\pgfsys@transformshift{1.007157in}{0.521603in}%
\pgfsys@useobject{currentmarker}{}%
\end{pgfscope}%
\end{pgfscope}%
\begin{pgfscope}%
\pgfsetbuttcap%
\pgfsetroundjoin%
\definecolor{currentfill}{rgb}{0.000000,0.000000,0.000000}%
\pgfsetfillcolor{currentfill}%
\pgfsetlinewidth{0.602250pt}%
\definecolor{currentstroke}{rgb}{0.000000,0.000000,0.000000}%
\pgfsetstrokecolor{currentstroke}%
\pgfsetdash{}{0pt}%
\pgfsys@defobject{currentmarker}{\pgfqpoint{0.000000in}{-0.027778in}}{\pgfqpoint{0.000000in}{0.000000in}}{%
\pgfpathmoveto{\pgfqpoint{0.000000in}{0.000000in}}%
\pgfpathlineto{\pgfqpoint{0.000000in}{-0.027778in}}%
\pgfusepath{stroke,fill}%
}%
\begin{pgfscope}%
\pgfsys@transformshift{1.176017in}{0.521603in}%
\pgfsys@useobject{currentmarker}{}%
\end{pgfscope}%
\end{pgfscope}%
\begin{pgfscope}%
\pgfsetbuttcap%
\pgfsetroundjoin%
\definecolor{currentfill}{rgb}{0.000000,0.000000,0.000000}%
\pgfsetfillcolor{currentfill}%
\pgfsetlinewidth{0.602250pt}%
\definecolor{currentstroke}{rgb}{0.000000,0.000000,0.000000}%
\pgfsetstrokecolor{currentstroke}%
\pgfsetdash{}{0pt}%
\pgfsys@defobject{currentmarker}{\pgfqpoint{0.000000in}{-0.027778in}}{\pgfqpoint{0.000000in}{0.000000in}}{%
\pgfpathmoveto{\pgfqpoint{0.000000in}{0.000000in}}%
\pgfpathlineto{\pgfqpoint{0.000000in}{-0.027778in}}%
\pgfusepath{stroke,fill}%
}%
\begin{pgfscope}%
\pgfsys@transformshift{1.306996in}{0.521603in}%
\pgfsys@useobject{currentmarker}{}%
\end{pgfscope}%
\end{pgfscope}%
\begin{pgfscope}%
\pgfsetbuttcap%
\pgfsetroundjoin%
\definecolor{currentfill}{rgb}{0.000000,0.000000,0.000000}%
\pgfsetfillcolor{currentfill}%
\pgfsetlinewidth{0.602250pt}%
\definecolor{currentstroke}{rgb}{0.000000,0.000000,0.000000}%
\pgfsetstrokecolor{currentstroke}%
\pgfsetdash{}{0pt}%
\pgfsys@defobject{currentmarker}{\pgfqpoint{0.000000in}{-0.027778in}}{\pgfqpoint{0.000000in}{0.000000in}}{%
\pgfpathmoveto{\pgfqpoint{0.000000in}{0.000000in}}%
\pgfpathlineto{\pgfqpoint{0.000000in}{-0.027778in}}%
\pgfusepath{stroke,fill}%
}%
\begin{pgfscope}%
\pgfsys@transformshift{1.414014in}{0.521603in}%
\pgfsys@useobject{currentmarker}{}%
\end{pgfscope}%
\end{pgfscope}%
\begin{pgfscope}%
\pgfsetbuttcap%
\pgfsetroundjoin%
\definecolor{currentfill}{rgb}{0.000000,0.000000,0.000000}%
\pgfsetfillcolor{currentfill}%
\pgfsetlinewidth{0.602250pt}%
\definecolor{currentstroke}{rgb}{0.000000,0.000000,0.000000}%
\pgfsetstrokecolor{currentstroke}%
\pgfsetdash{}{0pt}%
\pgfsys@defobject{currentmarker}{\pgfqpoint{0.000000in}{-0.027778in}}{\pgfqpoint{0.000000in}{0.000000in}}{%
\pgfpathmoveto{\pgfqpoint{0.000000in}{0.000000in}}%
\pgfpathlineto{\pgfqpoint{0.000000in}{-0.027778in}}%
\pgfusepath{stroke,fill}%
}%
\begin{pgfscope}%
\pgfsys@transformshift{1.504495in}{0.521603in}%
\pgfsys@useobject{currentmarker}{}%
\end{pgfscope}%
\end{pgfscope}%
\begin{pgfscope}%
\pgfsetbuttcap%
\pgfsetroundjoin%
\definecolor{currentfill}{rgb}{0.000000,0.000000,0.000000}%
\pgfsetfillcolor{currentfill}%
\pgfsetlinewidth{0.602250pt}%
\definecolor{currentstroke}{rgb}{0.000000,0.000000,0.000000}%
\pgfsetstrokecolor{currentstroke}%
\pgfsetdash{}{0pt}%
\pgfsys@defobject{currentmarker}{\pgfqpoint{0.000000in}{-0.027778in}}{\pgfqpoint{0.000000in}{0.000000in}}{%
\pgfpathmoveto{\pgfqpoint{0.000000in}{0.000000in}}%
\pgfpathlineto{\pgfqpoint{0.000000in}{-0.027778in}}%
\pgfusepath{stroke,fill}%
}%
\begin{pgfscope}%
\pgfsys@transformshift{1.582874in}{0.521603in}%
\pgfsys@useobject{currentmarker}{}%
\end{pgfscope}%
\end{pgfscope}%
\begin{pgfscope}%
\pgfsetbuttcap%
\pgfsetroundjoin%
\definecolor{currentfill}{rgb}{0.000000,0.000000,0.000000}%
\pgfsetfillcolor{currentfill}%
\pgfsetlinewidth{0.602250pt}%
\definecolor{currentstroke}{rgb}{0.000000,0.000000,0.000000}%
\pgfsetstrokecolor{currentstroke}%
\pgfsetdash{}{0pt}%
\pgfsys@defobject{currentmarker}{\pgfqpoint{0.000000in}{-0.027778in}}{\pgfqpoint{0.000000in}{0.000000in}}{%
\pgfpathmoveto{\pgfqpoint{0.000000in}{0.000000in}}%
\pgfpathlineto{\pgfqpoint{0.000000in}{-0.027778in}}%
\pgfusepath{stroke,fill}%
}%
\begin{pgfscope}%
\pgfsys@transformshift{1.652010in}{0.521603in}%
\pgfsys@useobject{currentmarker}{}%
\end{pgfscope}%
\end{pgfscope}%
\begin{pgfscope}%
\pgfsetbuttcap%
\pgfsetroundjoin%
\definecolor{currentfill}{rgb}{0.000000,0.000000,0.000000}%
\pgfsetfillcolor{currentfill}%
\pgfsetlinewidth{0.602250pt}%
\definecolor{currentstroke}{rgb}{0.000000,0.000000,0.000000}%
\pgfsetstrokecolor{currentstroke}%
\pgfsetdash{}{0pt}%
\pgfsys@defobject{currentmarker}{\pgfqpoint{0.000000in}{-0.027778in}}{\pgfqpoint{0.000000in}{0.000000in}}{%
\pgfpathmoveto{\pgfqpoint{0.000000in}{0.000000in}}%
\pgfpathlineto{\pgfqpoint{0.000000in}{-0.027778in}}%
\pgfusepath{stroke,fill}%
}%
\begin{pgfscope}%
\pgfsys@transformshift{2.120710in}{0.521603in}%
\pgfsys@useobject{currentmarker}{}%
\end{pgfscope}%
\end{pgfscope}%
\begin{pgfscope}%
\pgfsetbuttcap%
\pgfsetroundjoin%
\definecolor{currentfill}{rgb}{0.000000,0.000000,0.000000}%
\pgfsetfillcolor{currentfill}%
\pgfsetlinewidth{0.602250pt}%
\definecolor{currentstroke}{rgb}{0.000000,0.000000,0.000000}%
\pgfsetstrokecolor{currentstroke}%
\pgfsetdash{}{0pt}%
\pgfsys@defobject{currentmarker}{\pgfqpoint{0.000000in}{-0.027778in}}{\pgfqpoint{0.000000in}{0.000000in}}{%
\pgfpathmoveto{\pgfqpoint{0.000000in}{0.000000in}}%
\pgfpathlineto{\pgfqpoint{0.000000in}{-0.027778in}}%
\pgfusepath{stroke,fill}%
}%
\begin{pgfscope}%
\pgfsys@transformshift{2.358706in}{0.521603in}%
\pgfsys@useobject{currentmarker}{}%
\end{pgfscope}%
\end{pgfscope}%
\begin{pgfscope}%
\pgfsetbuttcap%
\pgfsetroundjoin%
\definecolor{currentfill}{rgb}{0.000000,0.000000,0.000000}%
\pgfsetfillcolor{currentfill}%
\pgfsetlinewidth{0.602250pt}%
\definecolor{currentstroke}{rgb}{0.000000,0.000000,0.000000}%
\pgfsetstrokecolor{currentstroke}%
\pgfsetdash{}{0pt}%
\pgfsys@defobject{currentmarker}{\pgfqpoint{0.000000in}{-0.027778in}}{\pgfqpoint{0.000000in}{0.000000in}}{%
\pgfpathmoveto{\pgfqpoint{0.000000in}{0.000000in}}%
\pgfpathlineto{\pgfqpoint{0.000000in}{-0.027778in}}%
\pgfusepath{stroke,fill}%
}%
\begin{pgfscope}%
\pgfsys@transformshift{2.527567in}{0.521603in}%
\pgfsys@useobject{currentmarker}{}%
\end{pgfscope}%
\end{pgfscope}%
\begin{pgfscope}%
\definecolor{textcolor}{rgb}{0.000000,0.000000,0.000000}%
\pgfsetstrokecolor{textcolor}%
\pgfsetfillcolor{textcolor}%
\pgftext[x=1.643943in,y=0.234413in,,top]{\color{textcolor}{\sffamily\fontsize{10.000000}{12.000000}\selectfont\catcode`\^=\active\def^{\ifmmode\sp\else\^{}\fi}\catcode`\%=\active\def%{\%}$m$}}%
\end{pgfscope}%
\begin{pgfscope}%
\pgfsetbuttcap%
\pgfsetroundjoin%
\definecolor{currentfill}{rgb}{0.000000,0.000000,0.000000}%
\pgfsetfillcolor{currentfill}%
\pgfsetlinewidth{0.803000pt}%
\definecolor{currentstroke}{rgb}{0.000000,0.000000,0.000000}%
\pgfsetstrokecolor{currentstroke}%
\pgfsetdash{}{0pt}%
\pgfsys@defobject{currentmarker}{\pgfqpoint{-0.048611in}{0.000000in}}{\pgfqpoint{-0.000000in}{0.000000in}}{%
\pgfpathmoveto{\pgfqpoint{-0.000000in}{0.000000in}}%
\pgfpathlineto{\pgfqpoint{-0.048611in}{0.000000in}}%
\pgfusepath{stroke,fill}%
}%
\begin{pgfscope}%
\pgfsys@transformshift{0.675193in}{1.463418in}%
\pgfsys@useobject{currentmarker}{}%
\end{pgfscope}%
\end{pgfscope}%
\begin{pgfscope}%
\definecolor{textcolor}{rgb}{0.000000,0.000000,0.000000}%
\pgfsetstrokecolor{textcolor}%
\pgfsetfillcolor{textcolor}%
\pgftext[x=0.289968in, y=1.410656in, left, base]{\color{textcolor}{\sffamily\fontsize{10.000000}{12.000000}\selectfont\catcode`\^=\active\def^{\ifmmode\sp\else\^{}\fi}\catcode`\%=\active\def%{\%}$\mathdefault{10^{-1}}$}}%
\end{pgfscope}%
\begin{pgfscope}%
\pgfsetbuttcap%
\pgfsetroundjoin%
\definecolor{currentfill}{rgb}{0.000000,0.000000,0.000000}%
\pgfsetfillcolor{currentfill}%
\pgfsetlinewidth{0.602250pt}%
\definecolor{currentstroke}{rgb}{0.000000,0.000000,0.000000}%
\pgfsetstrokecolor{currentstroke}%
\pgfsetdash{}{0pt}%
\pgfsys@defobject{currentmarker}{\pgfqpoint{-0.027778in}{0.000000in}}{\pgfqpoint{-0.000000in}{0.000000in}}{%
\pgfpathmoveto{\pgfqpoint{-0.000000in}{0.000000in}}%
\pgfpathlineto{\pgfqpoint{-0.027778in}{0.000000in}}%
\pgfusepath{stroke,fill}%
}%
\begin{pgfscope}%
\pgfsys@transformshift{0.675193in}{0.638741in}%
\pgfsys@useobject{currentmarker}{}%
\end{pgfscope}%
\end{pgfscope}%
\begin{pgfscope}%
\pgfsetbuttcap%
\pgfsetroundjoin%
\definecolor{currentfill}{rgb}{0.000000,0.000000,0.000000}%
\pgfsetfillcolor{currentfill}%
\pgfsetlinewidth{0.602250pt}%
\definecolor{currentstroke}{rgb}{0.000000,0.000000,0.000000}%
\pgfsetstrokecolor{currentstroke}%
\pgfsetdash{}{0pt}%
\pgfsys@defobject{currentmarker}{\pgfqpoint{-0.027778in}{0.000000in}}{\pgfqpoint{-0.000000in}{0.000000in}}{%
\pgfpathmoveto{\pgfqpoint{-0.000000in}{0.000000in}}%
\pgfpathlineto{\pgfqpoint{-0.027778in}{0.000000in}}%
\pgfusepath{stroke,fill}%
}%
\begin{pgfscope}%
\pgfsys@transformshift{0.675193in}{0.846501in}%
\pgfsys@useobject{currentmarker}{}%
\end{pgfscope}%
\end{pgfscope}%
\begin{pgfscope}%
\pgfsetbuttcap%
\pgfsetroundjoin%
\definecolor{currentfill}{rgb}{0.000000,0.000000,0.000000}%
\pgfsetfillcolor{currentfill}%
\pgfsetlinewidth{0.602250pt}%
\definecolor{currentstroke}{rgb}{0.000000,0.000000,0.000000}%
\pgfsetstrokecolor{currentstroke}%
\pgfsetdash{}{0pt}%
\pgfsys@defobject{currentmarker}{\pgfqpoint{-0.027778in}{0.000000in}}{\pgfqpoint{-0.000000in}{0.000000in}}{%
\pgfpathmoveto{\pgfqpoint{-0.000000in}{0.000000in}}%
\pgfpathlineto{\pgfqpoint{-0.027778in}{0.000000in}}%
\pgfusepath{stroke,fill}%
}%
\begin{pgfscope}%
\pgfsys@transformshift{0.675193in}{0.993910in}%
\pgfsys@useobject{currentmarker}{}%
\end{pgfscope}%
\end{pgfscope}%
\begin{pgfscope}%
\pgfsetbuttcap%
\pgfsetroundjoin%
\definecolor{currentfill}{rgb}{0.000000,0.000000,0.000000}%
\pgfsetfillcolor{currentfill}%
\pgfsetlinewidth{0.602250pt}%
\definecolor{currentstroke}{rgb}{0.000000,0.000000,0.000000}%
\pgfsetstrokecolor{currentstroke}%
\pgfsetdash{}{0pt}%
\pgfsys@defobject{currentmarker}{\pgfqpoint{-0.027778in}{0.000000in}}{\pgfqpoint{-0.000000in}{0.000000in}}{%
\pgfpathmoveto{\pgfqpoint{-0.000000in}{0.000000in}}%
\pgfpathlineto{\pgfqpoint{-0.027778in}{0.000000in}}%
\pgfusepath{stroke,fill}%
}%
\begin{pgfscope}%
\pgfsys@transformshift{0.675193in}{1.108249in}%
\pgfsys@useobject{currentmarker}{}%
\end{pgfscope}%
\end{pgfscope}%
\begin{pgfscope}%
\pgfsetbuttcap%
\pgfsetroundjoin%
\definecolor{currentfill}{rgb}{0.000000,0.000000,0.000000}%
\pgfsetfillcolor{currentfill}%
\pgfsetlinewidth{0.602250pt}%
\definecolor{currentstroke}{rgb}{0.000000,0.000000,0.000000}%
\pgfsetstrokecolor{currentstroke}%
\pgfsetdash{}{0pt}%
\pgfsys@defobject{currentmarker}{\pgfqpoint{-0.027778in}{0.000000in}}{\pgfqpoint{-0.000000in}{0.000000in}}{%
\pgfpathmoveto{\pgfqpoint{-0.000000in}{0.000000in}}%
\pgfpathlineto{\pgfqpoint{-0.027778in}{0.000000in}}%
\pgfusepath{stroke,fill}%
}%
\begin{pgfscope}%
\pgfsys@transformshift{0.675193in}{1.201671in}%
\pgfsys@useobject{currentmarker}{}%
\end{pgfscope}%
\end{pgfscope}%
\begin{pgfscope}%
\pgfsetbuttcap%
\pgfsetroundjoin%
\definecolor{currentfill}{rgb}{0.000000,0.000000,0.000000}%
\pgfsetfillcolor{currentfill}%
\pgfsetlinewidth{0.602250pt}%
\definecolor{currentstroke}{rgb}{0.000000,0.000000,0.000000}%
\pgfsetstrokecolor{currentstroke}%
\pgfsetdash{}{0pt}%
\pgfsys@defobject{currentmarker}{\pgfqpoint{-0.027778in}{0.000000in}}{\pgfqpoint{-0.000000in}{0.000000in}}{%
\pgfpathmoveto{\pgfqpoint{-0.000000in}{0.000000in}}%
\pgfpathlineto{\pgfqpoint{-0.027778in}{0.000000in}}%
\pgfusepath{stroke,fill}%
}%
\begin{pgfscope}%
\pgfsys@transformshift{0.675193in}{1.280657in}%
\pgfsys@useobject{currentmarker}{}%
\end{pgfscope}%
\end{pgfscope}%
\begin{pgfscope}%
\pgfsetbuttcap%
\pgfsetroundjoin%
\definecolor{currentfill}{rgb}{0.000000,0.000000,0.000000}%
\pgfsetfillcolor{currentfill}%
\pgfsetlinewidth{0.602250pt}%
\definecolor{currentstroke}{rgb}{0.000000,0.000000,0.000000}%
\pgfsetstrokecolor{currentstroke}%
\pgfsetdash{}{0pt}%
\pgfsys@defobject{currentmarker}{\pgfqpoint{-0.027778in}{0.000000in}}{\pgfqpoint{-0.000000in}{0.000000in}}{%
\pgfpathmoveto{\pgfqpoint{-0.000000in}{0.000000in}}%
\pgfpathlineto{\pgfqpoint{-0.027778in}{0.000000in}}%
\pgfusepath{stroke,fill}%
}%
\begin{pgfscope}%
\pgfsys@transformshift{0.675193in}{1.349079in}%
\pgfsys@useobject{currentmarker}{}%
\end{pgfscope}%
\end{pgfscope}%
\begin{pgfscope}%
\pgfsetbuttcap%
\pgfsetroundjoin%
\definecolor{currentfill}{rgb}{0.000000,0.000000,0.000000}%
\pgfsetfillcolor{currentfill}%
\pgfsetlinewidth{0.602250pt}%
\definecolor{currentstroke}{rgb}{0.000000,0.000000,0.000000}%
\pgfsetstrokecolor{currentstroke}%
\pgfsetdash{}{0pt}%
\pgfsys@defobject{currentmarker}{\pgfqpoint{-0.027778in}{0.000000in}}{\pgfqpoint{-0.000000in}{0.000000in}}{%
\pgfpathmoveto{\pgfqpoint{-0.000000in}{0.000000in}}%
\pgfpathlineto{\pgfqpoint{-0.027778in}{0.000000in}}%
\pgfusepath{stroke,fill}%
}%
\begin{pgfscope}%
\pgfsys@transformshift{0.675193in}{1.409431in}%
\pgfsys@useobject{currentmarker}{}%
\end{pgfscope}%
\end{pgfscope}%
\begin{pgfscope}%
\pgfsetbuttcap%
\pgfsetroundjoin%
\definecolor{currentfill}{rgb}{0.000000,0.000000,0.000000}%
\pgfsetfillcolor{currentfill}%
\pgfsetlinewidth{0.602250pt}%
\definecolor{currentstroke}{rgb}{0.000000,0.000000,0.000000}%
\pgfsetstrokecolor{currentstroke}%
\pgfsetdash{}{0pt}%
\pgfsys@defobject{currentmarker}{\pgfqpoint{-0.027778in}{0.000000in}}{\pgfqpoint{-0.000000in}{0.000000in}}{%
\pgfpathmoveto{\pgfqpoint{-0.000000in}{0.000000in}}%
\pgfpathlineto{\pgfqpoint{-0.027778in}{0.000000in}}%
\pgfusepath{stroke,fill}%
}%
\begin{pgfscope}%
\pgfsys@transformshift{0.675193in}{1.818587in}%
\pgfsys@useobject{currentmarker}{}%
\end{pgfscope}%
\end{pgfscope}%
\begin{pgfscope}%
\pgfsetbuttcap%
\pgfsetroundjoin%
\definecolor{currentfill}{rgb}{0.000000,0.000000,0.000000}%
\pgfsetfillcolor{currentfill}%
\pgfsetlinewidth{0.602250pt}%
\definecolor{currentstroke}{rgb}{0.000000,0.000000,0.000000}%
\pgfsetstrokecolor{currentstroke}%
\pgfsetdash{}{0pt}%
\pgfsys@defobject{currentmarker}{\pgfqpoint{-0.027778in}{0.000000in}}{\pgfqpoint{-0.000000in}{0.000000in}}{%
\pgfpathmoveto{\pgfqpoint{-0.000000in}{0.000000in}}%
\pgfpathlineto{\pgfqpoint{-0.027778in}{0.000000in}}%
\pgfusepath{stroke,fill}%
}%
\begin{pgfscope}%
\pgfsys@transformshift{0.675193in}{2.026348in}%
\pgfsys@useobject{currentmarker}{}%
\end{pgfscope}%
\end{pgfscope}%
\begin{pgfscope}%
\pgfsetbuttcap%
\pgfsetroundjoin%
\definecolor{currentfill}{rgb}{0.000000,0.000000,0.000000}%
\pgfsetfillcolor{currentfill}%
\pgfsetlinewidth{0.602250pt}%
\definecolor{currentstroke}{rgb}{0.000000,0.000000,0.000000}%
\pgfsetstrokecolor{currentstroke}%
\pgfsetdash{}{0pt}%
\pgfsys@defobject{currentmarker}{\pgfqpoint{-0.027778in}{0.000000in}}{\pgfqpoint{-0.000000in}{0.000000in}}{%
\pgfpathmoveto{\pgfqpoint{-0.000000in}{0.000000in}}%
\pgfpathlineto{\pgfqpoint{-0.027778in}{0.000000in}}%
\pgfusepath{stroke,fill}%
}%
\begin{pgfscope}%
\pgfsys@transformshift{0.675193in}{2.173756in}%
\pgfsys@useobject{currentmarker}{}%
\end{pgfscope}%
\end{pgfscope}%
\begin{pgfscope}%
\pgfsetbuttcap%
\pgfsetroundjoin%
\definecolor{currentfill}{rgb}{0.000000,0.000000,0.000000}%
\pgfsetfillcolor{currentfill}%
\pgfsetlinewidth{0.602250pt}%
\definecolor{currentstroke}{rgb}{0.000000,0.000000,0.000000}%
\pgfsetstrokecolor{currentstroke}%
\pgfsetdash{}{0pt}%
\pgfsys@defobject{currentmarker}{\pgfqpoint{-0.027778in}{0.000000in}}{\pgfqpoint{-0.000000in}{0.000000in}}{%
\pgfpathmoveto{\pgfqpoint{-0.000000in}{0.000000in}}%
\pgfpathlineto{\pgfqpoint{-0.027778in}{0.000000in}}%
\pgfusepath{stroke,fill}%
}%
\begin{pgfscope}%
\pgfsys@transformshift{0.675193in}{2.288095in}%
\pgfsys@useobject{currentmarker}{}%
\end{pgfscope}%
\end{pgfscope}%
\begin{pgfscope}%
\pgfsetbuttcap%
\pgfsetroundjoin%
\definecolor{currentfill}{rgb}{0.000000,0.000000,0.000000}%
\pgfsetfillcolor{currentfill}%
\pgfsetlinewidth{0.602250pt}%
\definecolor{currentstroke}{rgb}{0.000000,0.000000,0.000000}%
\pgfsetstrokecolor{currentstroke}%
\pgfsetdash{}{0pt}%
\pgfsys@defobject{currentmarker}{\pgfqpoint{-0.027778in}{0.000000in}}{\pgfqpoint{-0.000000in}{0.000000in}}{%
\pgfpathmoveto{\pgfqpoint{-0.000000in}{0.000000in}}%
\pgfpathlineto{\pgfqpoint{-0.027778in}{0.000000in}}%
\pgfusepath{stroke,fill}%
}%
\begin{pgfscope}%
\pgfsys@transformshift{0.675193in}{2.381517in}%
\pgfsys@useobject{currentmarker}{}%
\end{pgfscope}%
\end{pgfscope}%
\begin{pgfscope}%
\definecolor{textcolor}{rgb}{0.000000,0.000000,0.000000}%
\pgfsetstrokecolor{textcolor}%
\pgfsetfillcolor{textcolor}%
\pgftext[x=0.234413in,y=1.484103in,,bottom,rotate=90.000000]{\color{textcolor}{\sffamily\fontsize{10.000000}{12.000000}\selectfont\catcode`\^=\active\def^{\ifmmode\sp\else\^{}\fi}\catcode`\%=\active\def%{\%}$L^1$ error}}%
\end{pgfscope}%
\begin{pgfscope}%
\pgfpathrectangle{\pgfqpoint{0.675193in}{0.521603in}}{\pgfqpoint{1.937500in}{1.925000in}}%
\pgfusepath{clip}%
\pgfsetrectcap%
\pgfsetroundjoin%
\pgfsetlinewidth{1.003750pt}%
\definecolor{currentstroke}{rgb}{0.001462,0.000466,0.013866}%
\pgfsetstrokecolor{currentstroke}%
\pgfsetdash{}{0pt}%
\pgfpathmoveto{\pgfqpoint{0.763261in}{0.966084in}}%
\pgfpathlineto{\pgfqpoint{1.117426in}{0.949331in}}%
\pgfpathlineto{\pgfqpoint{1.469958in}{0.949406in}}%
\pgfpathlineto{\pgfqpoint{1.821848in}{0.949406in}}%
\pgfpathlineto{\pgfqpoint{2.172907in}{0.949406in}}%
\pgfpathlineto{\pgfqpoint{2.524625in}{0.949406in}}%
\pgfusepath{stroke}%
\end{pgfscope}%
\begin{pgfscope}%
\pgfpathrectangle{\pgfqpoint{0.675193in}{0.521603in}}{\pgfqpoint{1.937500in}{1.925000in}}%
\pgfusepath{clip}%
\pgfsetbuttcap%
\pgfsetroundjoin%
\definecolor{currentfill}{rgb}{0.001462,0.000466,0.013866}%
\pgfsetfillcolor{currentfill}%
\pgfsetlinewidth{1.003750pt}%
\definecolor{currentstroke}{rgb}{0.001462,0.000466,0.013866}%
\pgfsetstrokecolor{currentstroke}%
\pgfsetdash{}{0pt}%
\pgfsys@defobject{currentmarker}{\pgfqpoint{-0.020833in}{-0.020833in}}{\pgfqpoint{0.020833in}{0.020833in}}{%
\pgfpathmoveto{\pgfqpoint{0.000000in}{-0.020833in}}%
\pgfpathcurveto{\pgfqpoint{0.005525in}{-0.020833in}}{\pgfqpoint{0.010825in}{-0.018638in}}{\pgfqpoint{0.014731in}{-0.014731in}}%
\pgfpathcurveto{\pgfqpoint{0.018638in}{-0.010825in}}{\pgfqpoint{0.020833in}{-0.005525in}}{\pgfqpoint{0.020833in}{0.000000in}}%
\pgfpathcurveto{\pgfqpoint{0.020833in}{0.005525in}}{\pgfqpoint{0.018638in}{0.010825in}}{\pgfqpoint{0.014731in}{0.014731in}}%
\pgfpathcurveto{\pgfqpoint{0.010825in}{0.018638in}}{\pgfqpoint{0.005525in}{0.020833in}}{\pgfqpoint{0.000000in}{0.020833in}}%
\pgfpathcurveto{\pgfqpoint{-0.005525in}{0.020833in}}{\pgfqpoint{-0.010825in}{0.018638in}}{\pgfqpoint{-0.014731in}{0.014731in}}%
\pgfpathcurveto{\pgfqpoint{-0.018638in}{0.010825in}}{\pgfqpoint{-0.020833in}{0.005525in}}{\pgfqpoint{-0.020833in}{0.000000in}}%
\pgfpathcurveto{\pgfqpoint{-0.020833in}{-0.005525in}}{\pgfqpoint{-0.018638in}{-0.010825in}}{\pgfqpoint{-0.014731in}{-0.014731in}}%
\pgfpathcurveto{\pgfqpoint{-0.010825in}{-0.018638in}}{\pgfqpoint{-0.005525in}{-0.020833in}}{\pgfqpoint{0.000000in}{-0.020833in}}%
\pgfpathlineto{\pgfqpoint{0.000000in}{-0.020833in}}%
\pgfpathclose%
\pgfusepath{stroke,fill}%
}%
\begin{pgfscope}%
\pgfsys@transformshift{0.763261in}{0.966084in}%
\pgfsys@useobject{currentmarker}{}%
\end{pgfscope}%
\begin{pgfscope}%
\pgfsys@transformshift{1.117426in}{0.949331in}%
\pgfsys@useobject{currentmarker}{}%
\end{pgfscope}%
\begin{pgfscope}%
\pgfsys@transformshift{1.469958in}{0.949406in}%
\pgfsys@useobject{currentmarker}{}%
\end{pgfscope}%
\begin{pgfscope}%
\pgfsys@transformshift{1.821848in}{0.949406in}%
\pgfsys@useobject{currentmarker}{}%
\end{pgfscope}%
\begin{pgfscope}%
\pgfsys@transformshift{2.172907in}{0.949406in}%
\pgfsys@useobject{currentmarker}{}%
\end{pgfscope}%
\begin{pgfscope}%
\pgfsys@transformshift{2.524625in}{0.949406in}%
\pgfsys@useobject{currentmarker}{}%
\end{pgfscope}%
\end{pgfscope}%
\begin{pgfscope}%
\pgfpathrectangle{\pgfqpoint{0.675193in}{0.521603in}}{\pgfqpoint{1.937500in}{1.925000in}}%
\pgfusepath{clip}%
\pgfsetrectcap%
\pgfsetroundjoin%
\pgfsetlinewidth{1.003750pt}%
\definecolor{currentstroke}{rgb}{0.445163,0.122724,0.506901}%
\pgfsetstrokecolor{currentstroke}%
\pgfsetdash{}{0pt}%
\pgfpathmoveto{\pgfqpoint{0.763261in}{2.359103in}}%
\pgfpathlineto{\pgfqpoint{1.117426in}{2.117823in}}%
\pgfpathlineto{\pgfqpoint{1.469958in}{1.833395in}}%
\pgfpathlineto{\pgfqpoint{1.821848in}{1.862624in}}%
\pgfpathlineto{\pgfqpoint{2.172907in}{1.871278in}}%
\pgfpathlineto{\pgfqpoint{2.524625in}{1.871491in}}%
\pgfusepath{stroke}%
\end{pgfscope}%
\begin{pgfscope}%
\pgfpathrectangle{\pgfqpoint{0.675193in}{0.521603in}}{\pgfqpoint{1.937500in}{1.925000in}}%
\pgfusepath{clip}%
\pgfsetbuttcap%
\pgfsetroundjoin%
\definecolor{currentfill}{rgb}{0.445163,0.122724,0.506901}%
\pgfsetfillcolor{currentfill}%
\pgfsetlinewidth{1.003750pt}%
\definecolor{currentstroke}{rgb}{0.445163,0.122724,0.506901}%
\pgfsetstrokecolor{currentstroke}%
\pgfsetdash{}{0pt}%
\pgfsys@defobject{currentmarker}{\pgfqpoint{-0.020833in}{-0.020833in}}{\pgfqpoint{0.020833in}{0.020833in}}{%
\pgfpathmoveto{\pgfqpoint{0.000000in}{-0.020833in}}%
\pgfpathcurveto{\pgfqpoint{0.005525in}{-0.020833in}}{\pgfqpoint{0.010825in}{-0.018638in}}{\pgfqpoint{0.014731in}{-0.014731in}}%
\pgfpathcurveto{\pgfqpoint{0.018638in}{-0.010825in}}{\pgfqpoint{0.020833in}{-0.005525in}}{\pgfqpoint{0.020833in}{0.000000in}}%
\pgfpathcurveto{\pgfqpoint{0.020833in}{0.005525in}}{\pgfqpoint{0.018638in}{0.010825in}}{\pgfqpoint{0.014731in}{0.014731in}}%
\pgfpathcurveto{\pgfqpoint{0.010825in}{0.018638in}}{\pgfqpoint{0.005525in}{0.020833in}}{\pgfqpoint{0.000000in}{0.020833in}}%
\pgfpathcurveto{\pgfqpoint{-0.005525in}{0.020833in}}{\pgfqpoint{-0.010825in}{0.018638in}}{\pgfqpoint{-0.014731in}{0.014731in}}%
\pgfpathcurveto{\pgfqpoint{-0.018638in}{0.010825in}}{\pgfqpoint{-0.020833in}{0.005525in}}{\pgfqpoint{-0.020833in}{0.000000in}}%
\pgfpathcurveto{\pgfqpoint{-0.020833in}{-0.005525in}}{\pgfqpoint{-0.018638in}{-0.010825in}}{\pgfqpoint{-0.014731in}{-0.014731in}}%
\pgfpathcurveto{\pgfqpoint{-0.010825in}{-0.018638in}}{\pgfqpoint{-0.005525in}{-0.020833in}}{\pgfqpoint{0.000000in}{-0.020833in}}%
\pgfpathlineto{\pgfqpoint{0.000000in}{-0.020833in}}%
\pgfpathclose%
\pgfusepath{stroke,fill}%
}%
\begin{pgfscope}%
\pgfsys@transformshift{0.763261in}{2.359103in}%
\pgfsys@useobject{currentmarker}{}%
\end{pgfscope}%
\begin{pgfscope}%
\pgfsys@transformshift{1.117426in}{2.117823in}%
\pgfsys@useobject{currentmarker}{}%
\end{pgfscope}%
\begin{pgfscope}%
\pgfsys@transformshift{1.469958in}{1.833395in}%
\pgfsys@useobject{currentmarker}{}%
\end{pgfscope}%
\begin{pgfscope}%
\pgfsys@transformshift{1.821848in}{1.862624in}%
\pgfsys@useobject{currentmarker}{}%
\end{pgfscope}%
\begin{pgfscope}%
\pgfsys@transformshift{2.172907in}{1.871278in}%
\pgfsys@useobject{currentmarker}{}%
\end{pgfscope}%
\begin{pgfscope}%
\pgfsys@transformshift{2.524625in}{1.871491in}%
\pgfsys@useobject{currentmarker}{}%
\end{pgfscope}%
\end{pgfscope}%
\begin{pgfscope}%
\pgfpathrectangle{\pgfqpoint{0.675193in}{0.521603in}}{\pgfqpoint{1.937500in}{1.925000in}}%
\pgfusepath{clip}%
\pgfsetrectcap%
\pgfsetroundjoin%
\pgfsetlinewidth{1.003750pt}%
\definecolor{currentstroke}{rgb}{0.944006,0.377643,0.365136}%
\pgfsetstrokecolor{currentstroke}%
\pgfsetdash{}{0pt}%
\pgfpathmoveto{\pgfqpoint{0.763261in}{2.290986in}}%
\pgfpathlineto{\pgfqpoint{1.117426in}{1.885941in}}%
\pgfpathlineto{\pgfqpoint{1.469958in}{1.343470in}}%
\pgfpathlineto{\pgfqpoint{1.821848in}{0.697695in}}%
\pgfpathlineto{\pgfqpoint{2.172907in}{0.611690in}}%
\pgfpathlineto{\pgfqpoint{2.524625in}{0.609103in}}%
\pgfusepath{stroke}%
\end{pgfscope}%
\begin{pgfscope}%
\pgfpathrectangle{\pgfqpoint{0.675193in}{0.521603in}}{\pgfqpoint{1.937500in}{1.925000in}}%
\pgfusepath{clip}%
\pgfsetbuttcap%
\pgfsetroundjoin%
\definecolor{currentfill}{rgb}{0.944006,0.377643,0.365136}%
\pgfsetfillcolor{currentfill}%
\pgfsetlinewidth{1.003750pt}%
\definecolor{currentstroke}{rgb}{0.944006,0.377643,0.365136}%
\pgfsetstrokecolor{currentstroke}%
\pgfsetdash{}{0pt}%
\pgfsys@defobject{currentmarker}{\pgfqpoint{-0.020833in}{-0.020833in}}{\pgfqpoint{0.020833in}{0.020833in}}{%
\pgfpathmoveto{\pgfqpoint{0.000000in}{-0.020833in}}%
\pgfpathcurveto{\pgfqpoint{0.005525in}{-0.020833in}}{\pgfqpoint{0.010825in}{-0.018638in}}{\pgfqpoint{0.014731in}{-0.014731in}}%
\pgfpathcurveto{\pgfqpoint{0.018638in}{-0.010825in}}{\pgfqpoint{0.020833in}{-0.005525in}}{\pgfqpoint{0.020833in}{0.000000in}}%
\pgfpathcurveto{\pgfqpoint{0.020833in}{0.005525in}}{\pgfqpoint{0.018638in}{0.010825in}}{\pgfqpoint{0.014731in}{0.014731in}}%
\pgfpathcurveto{\pgfqpoint{0.010825in}{0.018638in}}{\pgfqpoint{0.005525in}{0.020833in}}{\pgfqpoint{0.000000in}{0.020833in}}%
\pgfpathcurveto{\pgfqpoint{-0.005525in}{0.020833in}}{\pgfqpoint{-0.010825in}{0.018638in}}{\pgfqpoint{-0.014731in}{0.014731in}}%
\pgfpathcurveto{\pgfqpoint{-0.018638in}{0.010825in}}{\pgfqpoint{-0.020833in}{0.005525in}}{\pgfqpoint{-0.020833in}{0.000000in}}%
\pgfpathcurveto{\pgfqpoint{-0.020833in}{-0.005525in}}{\pgfqpoint{-0.018638in}{-0.010825in}}{\pgfqpoint{-0.014731in}{-0.014731in}}%
\pgfpathcurveto{\pgfqpoint{-0.010825in}{-0.018638in}}{\pgfqpoint{-0.005525in}{-0.020833in}}{\pgfqpoint{0.000000in}{-0.020833in}}%
\pgfpathlineto{\pgfqpoint{0.000000in}{-0.020833in}}%
\pgfpathclose%
\pgfusepath{stroke,fill}%
}%
\begin{pgfscope}%
\pgfsys@transformshift{0.763261in}{2.290986in}%
\pgfsys@useobject{currentmarker}{}%
\end{pgfscope}%
\begin{pgfscope}%
\pgfsys@transformshift{1.117426in}{1.885941in}%
\pgfsys@useobject{currentmarker}{}%
\end{pgfscope}%
\begin{pgfscope}%
\pgfsys@transformshift{1.469958in}{1.343470in}%
\pgfsys@useobject{currentmarker}{}%
\end{pgfscope}%
\begin{pgfscope}%
\pgfsys@transformshift{1.821848in}{0.697695in}%
\pgfsys@useobject{currentmarker}{}%
\end{pgfscope}%
\begin{pgfscope}%
\pgfsys@transformshift{2.172907in}{0.611690in}%
\pgfsys@useobject{currentmarker}{}%
\end{pgfscope}%
\begin{pgfscope}%
\pgfsys@transformshift{2.524625in}{0.609103in}%
\pgfsys@useobject{currentmarker}{}%
\end{pgfscope}%
\end{pgfscope}%
\begin{pgfscope}%
\pgfsetrectcap%
\pgfsetmiterjoin%
\pgfsetlinewidth{0.803000pt}%
\definecolor{currentstroke}{rgb}{0.000000,0.000000,0.000000}%
\pgfsetstrokecolor{currentstroke}%
\pgfsetdash{}{0pt}%
\pgfpathmoveto{\pgfqpoint{0.675193in}{0.521603in}}%
\pgfpathlineto{\pgfqpoint{0.675193in}{2.446603in}}%
\pgfusepath{stroke}%
\end{pgfscope}%
\begin{pgfscope}%
\pgfsetrectcap%
\pgfsetmiterjoin%
\pgfsetlinewidth{0.803000pt}%
\definecolor{currentstroke}{rgb}{0.000000,0.000000,0.000000}%
\pgfsetstrokecolor{currentstroke}%
\pgfsetdash{}{0pt}%
\pgfpathmoveto{\pgfqpoint{2.612693in}{0.521603in}}%
\pgfpathlineto{\pgfqpoint{2.612693in}{2.446603in}}%
\pgfusepath{stroke}%
\end{pgfscope}%
\begin{pgfscope}%
\pgfsetrectcap%
\pgfsetmiterjoin%
\pgfsetlinewidth{0.803000pt}%
\definecolor{currentstroke}{rgb}{0.000000,0.000000,0.000000}%
\pgfsetstrokecolor{currentstroke}%
\pgfsetdash{}{0pt}%
\pgfpathmoveto{\pgfqpoint{0.675193in}{0.521603in}}%
\pgfpathlineto{\pgfqpoint{2.612693in}{0.521603in}}%
\pgfusepath{stroke}%
\end{pgfscope}%
\begin{pgfscope}%
\pgfsetrectcap%
\pgfsetmiterjoin%
\pgfsetlinewidth{0.803000pt}%
\definecolor{currentstroke}{rgb}{0.000000,0.000000,0.000000}%
\pgfsetstrokecolor{currentstroke}%
\pgfsetdash{}{0pt}%
\pgfpathmoveto{\pgfqpoint{0.675193in}{2.446603in}}%
\pgfpathlineto{\pgfqpoint{2.612693in}{2.446603in}}%
\pgfusepath{stroke}%
\end{pgfscope}%
\begin{pgfscope}%
\pgfsetbuttcap%
\pgfsetmiterjoin%
\definecolor{currentfill}{rgb}{1.000000,1.000000,1.000000}%
\pgfsetfillcolor{currentfill}%
\pgfsetfillopacity{0.800000}%
\pgfsetlinewidth{1.003750pt}%
\definecolor{currentstroke}{rgb}{0.800000,0.800000,0.800000}%
\pgfsetstrokecolor{currentstroke}%
\pgfsetstrokeopacity{0.800000}%
\pgfsetdash{}{0pt}%
\pgfpathmoveto{\pgfqpoint{1.469355in}{1.157484in}}%
\pgfpathlineto{\pgfqpoint{2.515471in}{1.157484in}}%
\pgfpathquadraticcurveto{\pgfqpoint{2.543249in}{1.157484in}}{\pgfqpoint{2.543249in}{1.185262in}}%
\pgfpathlineto{\pgfqpoint{2.543249in}{1.782945in}}%
\pgfpathquadraticcurveto{\pgfqpoint{2.543249in}{1.810723in}}{\pgfqpoint{2.515471in}{1.810723in}}%
\pgfpathlineto{\pgfqpoint{1.469355in}{1.810723in}}%
\pgfpathquadraticcurveto{\pgfqpoint{1.441578in}{1.810723in}}{\pgfqpoint{1.441578in}{1.782945in}}%
\pgfpathlineto{\pgfqpoint{1.441578in}{1.185262in}}%
\pgfpathquadraticcurveto{\pgfqpoint{1.441578in}{1.157484in}}{\pgfqpoint{1.469355in}{1.157484in}}%
\pgfpathlineto{\pgfqpoint{1.469355in}{1.157484in}}%
\pgfpathclose%
\pgfusepath{stroke,fill}%
\end{pgfscope}%
\begin{pgfscope}%
\pgfsetrectcap%
\pgfsetroundjoin%
\pgfsetlinewidth{1.003750pt}%
\definecolor{currentstroke}{rgb}{0.001462,0.000466,0.013866}%
\pgfsetstrokecolor{currentstroke}%
\pgfsetdash{}{0pt}%
\pgfpathmoveto{\pgfqpoint{1.497133in}{1.698255in}}%
\pgfpathlineto{\pgfqpoint{1.636022in}{1.698255in}}%
\pgfpathlineto{\pgfqpoint{1.774911in}{1.698255in}}%
\pgfusepath{stroke}%
\end{pgfscope}%
\begin{pgfscope}%
\pgfsetbuttcap%
\pgfsetroundjoin%
\definecolor{currentfill}{rgb}{0.001462,0.000466,0.013866}%
\pgfsetfillcolor{currentfill}%
\pgfsetlinewidth{1.003750pt}%
\definecolor{currentstroke}{rgb}{0.001462,0.000466,0.013866}%
\pgfsetstrokecolor{currentstroke}%
\pgfsetdash{}{0pt}%
\pgfsys@defobject{currentmarker}{\pgfqpoint{-0.020833in}{-0.020833in}}{\pgfqpoint{0.020833in}{0.020833in}}{%
\pgfpathmoveto{\pgfqpoint{0.000000in}{-0.020833in}}%
\pgfpathcurveto{\pgfqpoint{0.005525in}{-0.020833in}}{\pgfqpoint{0.010825in}{-0.018638in}}{\pgfqpoint{0.014731in}{-0.014731in}}%
\pgfpathcurveto{\pgfqpoint{0.018638in}{-0.010825in}}{\pgfqpoint{0.020833in}{-0.005525in}}{\pgfqpoint{0.020833in}{0.000000in}}%
\pgfpathcurveto{\pgfqpoint{0.020833in}{0.005525in}}{\pgfqpoint{0.018638in}{0.010825in}}{\pgfqpoint{0.014731in}{0.014731in}}%
\pgfpathcurveto{\pgfqpoint{0.010825in}{0.018638in}}{\pgfqpoint{0.005525in}{0.020833in}}{\pgfqpoint{0.000000in}{0.020833in}}%
\pgfpathcurveto{\pgfqpoint{-0.005525in}{0.020833in}}{\pgfqpoint{-0.010825in}{0.018638in}}{\pgfqpoint{-0.014731in}{0.014731in}}%
\pgfpathcurveto{\pgfqpoint{-0.018638in}{0.010825in}}{\pgfqpoint{-0.020833in}{0.005525in}}{\pgfqpoint{-0.020833in}{0.000000in}}%
\pgfpathcurveto{\pgfqpoint{-0.020833in}{-0.005525in}}{\pgfqpoint{-0.018638in}{-0.010825in}}{\pgfqpoint{-0.014731in}{-0.014731in}}%
\pgfpathcurveto{\pgfqpoint{-0.010825in}{-0.018638in}}{\pgfqpoint{-0.005525in}{-0.020833in}}{\pgfqpoint{0.000000in}{-0.020833in}}%
\pgfpathlineto{\pgfqpoint{0.000000in}{-0.020833in}}%
\pgfpathclose%
\pgfusepath{stroke,fill}%
}%
\begin{pgfscope}%
\pgfsys@transformshift{1.636022in}{1.698255in}%
\pgfsys@useobject{currentmarker}{}%
\end{pgfscope}%
\end{pgfscope}%
\begin{pgfscope}%
\definecolor{textcolor}{rgb}{0.000000,0.000000,0.000000}%
\pgfsetstrokecolor{textcolor}%
\pgfsetfillcolor{textcolor}%
\pgftext[x=1.886022in,y=1.649644in,left,base]{\color{textcolor}{\sffamily\fontsize{10.000000}{12.000000}\selectfont\catcode`\^=\active\def^{\ifmmode\sp\else\^{}\fi}\catcode`\%=\active\def%{\%}Haydock}}%
\end{pgfscope}%
\begin{pgfscope}%
\pgfsetrectcap%
\pgfsetroundjoin%
\pgfsetlinewidth{1.003750pt}%
\definecolor{currentstroke}{rgb}{0.445163,0.122724,0.506901}%
\pgfsetstrokecolor{currentstroke}%
\pgfsetdash{}{0pt}%
\pgfpathmoveto{\pgfqpoint{1.497133in}{1.494398in}}%
\pgfpathlineto{\pgfqpoint{1.636022in}{1.494398in}}%
\pgfpathlineto{\pgfqpoint{1.774911in}{1.494398in}}%
\pgfusepath{stroke}%
\end{pgfscope}%
\begin{pgfscope}%
\pgfsetbuttcap%
\pgfsetroundjoin%
\definecolor{currentfill}{rgb}{0.445163,0.122724,0.506901}%
\pgfsetfillcolor{currentfill}%
\pgfsetlinewidth{1.003750pt}%
\definecolor{currentstroke}{rgb}{0.445163,0.122724,0.506901}%
\pgfsetstrokecolor{currentstroke}%
\pgfsetdash{}{0pt}%
\pgfsys@defobject{currentmarker}{\pgfqpoint{-0.020833in}{-0.020833in}}{\pgfqpoint{0.020833in}{0.020833in}}{%
\pgfpathmoveto{\pgfqpoint{0.000000in}{-0.020833in}}%
\pgfpathcurveto{\pgfqpoint{0.005525in}{-0.020833in}}{\pgfqpoint{0.010825in}{-0.018638in}}{\pgfqpoint{0.014731in}{-0.014731in}}%
\pgfpathcurveto{\pgfqpoint{0.018638in}{-0.010825in}}{\pgfqpoint{0.020833in}{-0.005525in}}{\pgfqpoint{0.020833in}{0.000000in}}%
\pgfpathcurveto{\pgfqpoint{0.020833in}{0.005525in}}{\pgfqpoint{0.018638in}{0.010825in}}{\pgfqpoint{0.014731in}{0.014731in}}%
\pgfpathcurveto{\pgfqpoint{0.010825in}{0.018638in}}{\pgfqpoint{0.005525in}{0.020833in}}{\pgfqpoint{0.000000in}{0.020833in}}%
\pgfpathcurveto{\pgfqpoint{-0.005525in}{0.020833in}}{\pgfqpoint{-0.010825in}{0.018638in}}{\pgfqpoint{-0.014731in}{0.014731in}}%
\pgfpathcurveto{\pgfqpoint{-0.018638in}{0.010825in}}{\pgfqpoint{-0.020833in}{0.005525in}}{\pgfqpoint{-0.020833in}{0.000000in}}%
\pgfpathcurveto{\pgfqpoint{-0.020833in}{-0.005525in}}{\pgfqpoint{-0.018638in}{-0.010825in}}{\pgfqpoint{-0.014731in}{-0.014731in}}%
\pgfpathcurveto{\pgfqpoint{-0.010825in}{-0.018638in}}{\pgfqpoint{-0.005525in}{-0.020833in}}{\pgfqpoint{0.000000in}{-0.020833in}}%
\pgfpathlineto{\pgfqpoint{0.000000in}{-0.020833in}}%
\pgfpathclose%
\pgfusepath{stroke,fill}%
}%
\begin{pgfscope}%
\pgfsys@transformshift{1.636022in}{1.494398in}%
\pgfsys@useobject{currentmarker}{}%
\end{pgfscope}%
\end{pgfscope}%
\begin{pgfscope}%
\definecolor{textcolor}{rgb}{0.000000,0.000000,0.000000}%
\pgfsetstrokecolor{textcolor}%
\pgfsetfillcolor{textcolor}%
\pgftext[x=1.886022in,y=1.445787in,left,base]{\color{textcolor}{\sffamily\fontsize{10.000000}{12.000000}\selectfont\catcode`\^=\active\def^{\ifmmode\sp\else\^{}\fi}\catcode`\%=\active\def%{\%}NC}}%
\end{pgfscope}%
\begin{pgfscope}%
\pgfsetrectcap%
\pgfsetroundjoin%
\pgfsetlinewidth{1.003750pt}%
\definecolor{currentstroke}{rgb}{0.944006,0.377643,0.365136}%
\pgfsetstrokecolor{currentstroke}%
\pgfsetdash{}{0pt}%
\pgfpathmoveto{\pgfqpoint{1.497133in}{1.290541in}}%
\pgfpathlineto{\pgfqpoint{1.636022in}{1.290541in}}%
\pgfpathlineto{\pgfqpoint{1.774911in}{1.290541in}}%
\pgfusepath{stroke}%
\end{pgfscope}%
\begin{pgfscope}%
\pgfsetbuttcap%
\pgfsetroundjoin%
\definecolor{currentfill}{rgb}{0.944006,0.377643,0.365136}%
\pgfsetfillcolor{currentfill}%
\pgfsetlinewidth{1.003750pt}%
\definecolor{currentstroke}{rgb}{0.944006,0.377643,0.365136}%
\pgfsetstrokecolor{currentstroke}%
\pgfsetdash{}{0pt}%
\pgfsys@defobject{currentmarker}{\pgfqpoint{-0.020833in}{-0.020833in}}{\pgfqpoint{0.020833in}{0.020833in}}{%
\pgfpathmoveto{\pgfqpoint{0.000000in}{-0.020833in}}%
\pgfpathcurveto{\pgfqpoint{0.005525in}{-0.020833in}}{\pgfqpoint{0.010825in}{-0.018638in}}{\pgfqpoint{0.014731in}{-0.014731in}}%
\pgfpathcurveto{\pgfqpoint{0.018638in}{-0.010825in}}{\pgfqpoint{0.020833in}{-0.005525in}}{\pgfqpoint{0.020833in}{0.000000in}}%
\pgfpathcurveto{\pgfqpoint{0.020833in}{0.005525in}}{\pgfqpoint{0.018638in}{0.010825in}}{\pgfqpoint{0.014731in}{0.014731in}}%
\pgfpathcurveto{\pgfqpoint{0.010825in}{0.018638in}}{\pgfqpoint{0.005525in}{0.020833in}}{\pgfqpoint{0.000000in}{0.020833in}}%
\pgfpathcurveto{\pgfqpoint{-0.005525in}{0.020833in}}{\pgfqpoint{-0.010825in}{0.018638in}}{\pgfqpoint{-0.014731in}{0.014731in}}%
\pgfpathcurveto{\pgfqpoint{-0.018638in}{0.010825in}}{\pgfqpoint{-0.020833in}{0.005525in}}{\pgfqpoint{-0.020833in}{0.000000in}}%
\pgfpathcurveto{\pgfqpoint{-0.020833in}{-0.005525in}}{\pgfqpoint{-0.018638in}{-0.010825in}}{\pgfqpoint{-0.014731in}{-0.014731in}}%
\pgfpathcurveto{\pgfqpoint{-0.010825in}{-0.018638in}}{\pgfqpoint{-0.005525in}{-0.020833in}}{\pgfqpoint{0.000000in}{-0.020833in}}%
\pgfpathlineto{\pgfqpoint{0.000000in}{-0.020833in}}%
\pgfpathclose%
\pgfusepath{stroke,fill}%
}%
\begin{pgfscope}%
\pgfsys@transformshift{1.636022in}{1.290541in}%
\pgfsys@useobject{currentmarker}{}%
\end{pgfscope}%
\end{pgfscope}%
\begin{pgfscope}%
\definecolor{textcolor}{rgb}{0.000000,0.000000,0.000000}%
\pgfsetstrokecolor{textcolor}%
\pgfsetfillcolor{textcolor}%
\pgftext[x=1.886022in,y=1.241929in,left,base]{\color{textcolor}{\sffamily\fontsize{10.000000}{12.000000}\selectfont\catcode`\^=\active\def^{\ifmmode\sp\else\^{}\fi}\catcode`\%=\active\def%{\%}NC++}}%
\end{pgfscope}%
\end{pgfpicture}%
\makeatother%
\endgroup%

        \caption{\gls{sketch-size} $+$ \gls{num-hutchinson-queries} $=40$}
        \label{fig:5-experiments-haydock-convergence-m-nv40}
    \end{subfigure}
    \begin{subfigure}[b]{0.49\columnwidth}
        %% Creator: Matplotlib, PGF backend
%%
%% To include the figure in your LaTeX document, write
%%   \input{<filename>.pgf}
%%
%% Make sure the required packages are loaded in your preamble
%%   \usepackage{pgf}
%%
%% Also ensure that all the required font packages are loaded; for instance,
%% the lmodern package is sometimes necessary when using math font.
%%   \usepackage{lmodern}
%%
%% Figures using additional raster images can only be included by \input if
%% they are in the same directory as the main LaTeX file. For loading figures
%% from other directories you can use the `import` package
%%   \usepackage{import}
%%
%% and then include the figures with
%%   \import{<path to file>}{<filename>.pgf}
%%
%% Matplotlib used the following preamble
%%   \def\mathdefault#1{#1}
%%   \everymath=\expandafter{\the\everymath\displaystyle}
%%   
%%   \makeatletter\@ifpackageloaded{underscore}{}{\usepackage[strings]{underscore}}\makeatother
%%
\begingroup%
\makeatletter%
\begin{pgfpicture}%
\pgfpathrectangle{\pgfpointorigin}{\pgfqpoint{2.759413in}{2.574073in}}%
\pgfusepath{use as bounding box, clip}%
\begin{pgfscope}%
\pgfsetbuttcap%
\pgfsetmiterjoin%
\definecolor{currentfill}{rgb}{1.000000,1.000000,1.000000}%
\pgfsetfillcolor{currentfill}%
\pgfsetlinewidth{0.000000pt}%
\definecolor{currentstroke}{rgb}{1.000000,1.000000,1.000000}%
\pgfsetstrokecolor{currentstroke}%
\pgfsetdash{}{0pt}%
\pgfpathmoveto{\pgfqpoint{0.000000in}{0.000000in}}%
\pgfpathlineto{\pgfqpoint{2.759413in}{0.000000in}}%
\pgfpathlineto{\pgfqpoint{2.759413in}{2.574073in}}%
\pgfpathlineto{\pgfqpoint{0.000000in}{2.574073in}}%
\pgfpathlineto{\pgfqpoint{0.000000in}{0.000000in}}%
\pgfpathclose%
\pgfusepath{fill}%
\end{pgfscope}%
\begin{pgfscope}%
\pgfsetbuttcap%
\pgfsetmiterjoin%
\definecolor{currentfill}{rgb}{1.000000,1.000000,1.000000}%
\pgfsetfillcolor{currentfill}%
\pgfsetlinewidth{0.000000pt}%
\definecolor{currentstroke}{rgb}{0.000000,0.000000,0.000000}%
\pgfsetstrokecolor{currentstroke}%
\pgfsetstrokeopacity{0.000000}%
\pgfsetdash{}{0pt}%
\pgfpathmoveto{\pgfqpoint{0.721913in}{0.549073in}}%
\pgfpathlineto{\pgfqpoint{2.659413in}{0.549073in}}%
\pgfpathlineto{\pgfqpoint{2.659413in}{2.474073in}}%
\pgfpathlineto{\pgfqpoint{0.721913in}{2.474073in}}%
\pgfpathlineto{\pgfqpoint{0.721913in}{0.549073in}}%
\pgfpathclose%
\pgfusepath{fill}%
\end{pgfscope}%
\begin{pgfscope}%
\pgfsetbuttcap%
\pgfsetroundjoin%
\definecolor{currentfill}{rgb}{0.000000,0.000000,0.000000}%
\pgfsetfillcolor{currentfill}%
\pgfsetlinewidth{0.803000pt}%
\definecolor{currentstroke}{rgb}{0.000000,0.000000,0.000000}%
\pgfsetstrokecolor{currentstroke}%
\pgfsetdash{}{0pt}%
\pgfsys@defobject{currentmarker}{\pgfqpoint{0.000000in}{-0.048611in}}{\pgfqpoint{0.000000in}{0.000000in}}{%
\pgfpathmoveto{\pgfqpoint{0.000000in}{0.000000in}}%
\pgfpathlineto{\pgfqpoint{0.000000in}{-0.048611in}}%
\pgfusepath{stroke,fill}%
}%
\begin{pgfscope}%
\pgfsys@transformshift{1.760574in}{0.549073in}%
\pgfsys@useobject{currentmarker}{}%
\end{pgfscope}%
\end{pgfscope}%
\begin{pgfscope}%
\definecolor{textcolor}{rgb}{0.000000,0.000000,0.000000}%
\pgfsetstrokecolor{textcolor}%
\pgfsetfillcolor{textcolor}%
\pgftext[x=1.760574in,y=0.451851in,,top]{\color{textcolor}{\rmfamily\fontsize{12.000000}{14.400000}\selectfont\catcode`\^=\active\def^{\ifmmode\sp\else\^{}\fi}\catcode`\%=\active\def%{\%}$\mathdefault{10^{3}}$}}%
\end{pgfscope}%
\begin{pgfscope}%
\pgfsetbuttcap%
\pgfsetroundjoin%
\definecolor{currentfill}{rgb}{0.000000,0.000000,0.000000}%
\pgfsetfillcolor{currentfill}%
\pgfsetlinewidth{0.602250pt}%
\definecolor{currentstroke}{rgb}{0.000000,0.000000,0.000000}%
\pgfsetstrokecolor{currentstroke}%
\pgfsetdash{}{0pt}%
\pgfsys@defobject{currentmarker}{\pgfqpoint{0.000000in}{-0.027778in}}{\pgfqpoint{0.000000in}{0.000000in}}{%
\pgfpathmoveto{\pgfqpoint{0.000000in}{0.000000in}}%
\pgfpathlineto{\pgfqpoint{0.000000in}{-0.027778in}}%
\pgfusepath{stroke,fill}%
}%
\begin{pgfscope}%
\pgfsys@transformshift{0.815881in}{0.549073in}%
\pgfsys@useobject{currentmarker}{}%
\end{pgfscope}%
\end{pgfscope}%
\begin{pgfscope}%
\pgfsetbuttcap%
\pgfsetroundjoin%
\definecolor{currentfill}{rgb}{0.000000,0.000000,0.000000}%
\pgfsetfillcolor{currentfill}%
\pgfsetlinewidth{0.602250pt}%
\definecolor{currentstroke}{rgb}{0.000000,0.000000,0.000000}%
\pgfsetstrokecolor{currentstroke}%
\pgfsetdash{}{0pt}%
\pgfsys@defobject{currentmarker}{\pgfqpoint{0.000000in}{-0.027778in}}{\pgfqpoint{0.000000in}{0.000000in}}{%
\pgfpathmoveto{\pgfqpoint{0.000000in}{0.000000in}}%
\pgfpathlineto{\pgfqpoint{0.000000in}{-0.027778in}}%
\pgfusepath{stroke,fill}%
}%
\begin{pgfscope}%
\pgfsys@transformshift{1.053877in}{0.549073in}%
\pgfsys@useobject{currentmarker}{}%
\end{pgfscope}%
\end{pgfscope}%
\begin{pgfscope}%
\pgfsetbuttcap%
\pgfsetroundjoin%
\definecolor{currentfill}{rgb}{0.000000,0.000000,0.000000}%
\pgfsetfillcolor{currentfill}%
\pgfsetlinewidth{0.602250pt}%
\definecolor{currentstroke}{rgb}{0.000000,0.000000,0.000000}%
\pgfsetstrokecolor{currentstroke}%
\pgfsetdash{}{0pt}%
\pgfsys@defobject{currentmarker}{\pgfqpoint{0.000000in}{-0.027778in}}{\pgfqpoint{0.000000in}{0.000000in}}{%
\pgfpathmoveto{\pgfqpoint{0.000000in}{0.000000in}}%
\pgfpathlineto{\pgfqpoint{0.000000in}{-0.027778in}}%
\pgfusepath{stroke,fill}%
}%
\begin{pgfscope}%
\pgfsys@transformshift{1.222738in}{0.549073in}%
\pgfsys@useobject{currentmarker}{}%
\end{pgfscope}%
\end{pgfscope}%
\begin{pgfscope}%
\pgfsetbuttcap%
\pgfsetroundjoin%
\definecolor{currentfill}{rgb}{0.000000,0.000000,0.000000}%
\pgfsetfillcolor{currentfill}%
\pgfsetlinewidth{0.602250pt}%
\definecolor{currentstroke}{rgb}{0.000000,0.000000,0.000000}%
\pgfsetstrokecolor{currentstroke}%
\pgfsetdash{}{0pt}%
\pgfsys@defobject{currentmarker}{\pgfqpoint{0.000000in}{-0.027778in}}{\pgfqpoint{0.000000in}{0.000000in}}{%
\pgfpathmoveto{\pgfqpoint{0.000000in}{0.000000in}}%
\pgfpathlineto{\pgfqpoint{0.000000in}{-0.027778in}}%
\pgfusepath{stroke,fill}%
}%
\begin{pgfscope}%
\pgfsys@transformshift{1.353717in}{0.549073in}%
\pgfsys@useobject{currentmarker}{}%
\end{pgfscope}%
\end{pgfscope}%
\begin{pgfscope}%
\pgfsetbuttcap%
\pgfsetroundjoin%
\definecolor{currentfill}{rgb}{0.000000,0.000000,0.000000}%
\pgfsetfillcolor{currentfill}%
\pgfsetlinewidth{0.602250pt}%
\definecolor{currentstroke}{rgb}{0.000000,0.000000,0.000000}%
\pgfsetstrokecolor{currentstroke}%
\pgfsetdash{}{0pt}%
\pgfsys@defobject{currentmarker}{\pgfqpoint{0.000000in}{-0.027778in}}{\pgfqpoint{0.000000in}{0.000000in}}{%
\pgfpathmoveto{\pgfqpoint{0.000000in}{0.000000in}}%
\pgfpathlineto{\pgfqpoint{0.000000in}{-0.027778in}}%
\pgfusepath{stroke,fill}%
}%
\begin{pgfscope}%
\pgfsys@transformshift{1.460734in}{0.549073in}%
\pgfsys@useobject{currentmarker}{}%
\end{pgfscope}%
\end{pgfscope}%
\begin{pgfscope}%
\pgfsetbuttcap%
\pgfsetroundjoin%
\definecolor{currentfill}{rgb}{0.000000,0.000000,0.000000}%
\pgfsetfillcolor{currentfill}%
\pgfsetlinewidth{0.602250pt}%
\definecolor{currentstroke}{rgb}{0.000000,0.000000,0.000000}%
\pgfsetstrokecolor{currentstroke}%
\pgfsetdash{}{0pt}%
\pgfsys@defobject{currentmarker}{\pgfqpoint{0.000000in}{-0.027778in}}{\pgfqpoint{0.000000in}{0.000000in}}{%
\pgfpathmoveto{\pgfqpoint{0.000000in}{0.000000in}}%
\pgfpathlineto{\pgfqpoint{0.000000in}{-0.027778in}}%
\pgfusepath{stroke,fill}%
}%
\begin{pgfscope}%
\pgfsys@transformshift{1.551216in}{0.549073in}%
\pgfsys@useobject{currentmarker}{}%
\end{pgfscope}%
\end{pgfscope}%
\begin{pgfscope}%
\pgfsetbuttcap%
\pgfsetroundjoin%
\definecolor{currentfill}{rgb}{0.000000,0.000000,0.000000}%
\pgfsetfillcolor{currentfill}%
\pgfsetlinewidth{0.602250pt}%
\definecolor{currentstroke}{rgb}{0.000000,0.000000,0.000000}%
\pgfsetstrokecolor{currentstroke}%
\pgfsetdash{}{0pt}%
\pgfsys@defobject{currentmarker}{\pgfqpoint{0.000000in}{-0.027778in}}{\pgfqpoint{0.000000in}{0.000000in}}{%
\pgfpathmoveto{\pgfqpoint{0.000000in}{0.000000in}}%
\pgfpathlineto{\pgfqpoint{0.000000in}{-0.027778in}}%
\pgfusepath{stroke,fill}%
}%
\begin{pgfscope}%
\pgfsys@transformshift{1.629595in}{0.549073in}%
\pgfsys@useobject{currentmarker}{}%
\end{pgfscope}%
\end{pgfscope}%
\begin{pgfscope}%
\pgfsetbuttcap%
\pgfsetroundjoin%
\definecolor{currentfill}{rgb}{0.000000,0.000000,0.000000}%
\pgfsetfillcolor{currentfill}%
\pgfsetlinewidth{0.602250pt}%
\definecolor{currentstroke}{rgb}{0.000000,0.000000,0.000000}%
\pgfsetstrokecolor{currentstroke}%
\pgfsetdash{}{0pt}%
\pgfsys@defobject{currentmarker}{\pgfqpoint{0.000000in}{-0.027778in}}{\pgfqpoint{0.000000in}{0.000000in}}{%
\pgfpathmoveto{\pgfqpoint{0.000000in}{0.000000in}}%
\pgfpathlineto{\pgfqpoint{0.000000in}{-0.027778in}}%
\pgfusepath{stroke,fill}%
}%
\begin{pgfscope}%
\pgfsys@transformshift{1.698730in}{0.549073in}%
\pgfsys@useobject{currentmarker}{}%
\end{pgfscope}%
\end{pgfscope}%
\begin{pgfscope}%
\pgfsetbuttcap%
\pgfsetroundjoin%
\definecolor{currentfill}{rgb}{0.000000,0.000000,0.000000}%
\pgfsetfillcolor{currentfill}%
\pgfsetlinewidth{0.602250pt}%
\definecolor{currentstroke}{rgb}{0.000000,0.000000,0.000000}%
\pgfsetstrokecolor{currentstroke}%
\pgfsetdash{}{0pt}%
\pgfsys@defobject{currentmarker}{\pgfqpoint{0.000000in}{-0.027778in}}{\pgfqpoint{0.000000in}{0.000000in}}{%
\pgfpathmoveto{\pgfqpoint{0.000000in}{0.000000in}}%
\pgfpathlineto{\pgfqpoint{0.000000in}{-0.027778in}}%
\pgfusepath{stroke,fill}%
}%
\begin{pgfscope}%
\pgfsys@transformshift{2.167431in}{0.549073in}%
\pgfsys@useobject{currentmarker}{}%
\end{pgfscope}%
\end{pgfscope}%
\begin{pgfscope}%
\pgfsetbuttcap%
\pgfsetroundjoin%
\definecolor{currentfill}{rgb}{0.000000,0.000000,0.000000}%
\pgfsetfillcolor{currentfill}%
\pgfsetlinewidth{0.602250pt}%
\definecolor{currentstroke}{rgb}{0.000000,0.000000,0.000000}%
\pgfsetstrokecolor{currentstroke}%
\pgfsetdash{}{0pt}%
\pgfsys@defobject{currentmarker}{\pgfqpoint{0.000000in}{-0.027778in}}{\pgfqpoint{0.000000in}{0.000000in}}{%
\pgfpathmoveto{\pgfqpoint{0.000000in}{0.000000in}}%
\pgfpathlineto{\pgfqpoint{0.000000in}{-0.027778in}}%
\pgfusepath{stroke,fill}%
}%
\begin{pgfscope}%
\pgfsys@transformshift{2.405427in}{0.549073in}%
\pgfsys@useobject{currentmarker}{}%
\end{pgfscope}%
\end{pgfscope}%
\begin{pgfscope}%
\pgfsetbuttcap%
\pgfsetroundjoin%
\definecolor{currentfill}{rgb}{0.000000,0.000000,0.000000}%
\pgfsetfillcolor{currentfill}%
\pgfsetlinewidth{0.602250pt}%
\definecolor{currentstroke}{rgb}{0.000000,0.000000,0.000000}%
\pgfsetstrokecolor{currentstroke}%
\pgfsetdash{}{0pt}%
\pgfsys@defobject{currentmarker}{\pgfqpoint{0.000000in}{-0.027778in}}{\pgfqpoint{0.000000in}{0.000000in}}{%
\pgfpathmoveto{\pgfqpoint{0.000000in}{0.000000in}}%
\pgfpathlineto{\pgfqpoint{0.000000in}{-0.027778in}}%
\pgfusepath{stroke,fill}%
}%
\begin{pgfscope}%
\pgfsys@transformshift{2.574288in}{0.549073in}%
\pgfsys@useobject{currentmarker}{}%
\end{pgfscope}%
\end{pgfscope}%
\begin{pgfscope}%
\definecolor{textcolor}{rgb}{0.000000,0.000000,0.000000}%
\pgfsetstrokecolor{textcolor}%
\pgfsetfillcolor{textcolor}%
\pgftext[x=1.690663in,y=0.248148in,,top]{\color{textcolor}{\rmfamily\fontsize{12.000000}{14.400000}\selectfont\catcode`\^=\active\def^{\ifmmode\sp\else\^{}\fi}\catcode`\%=\active\def%{\%}$m$}}%
\end{pgfscope}%
\begin{pgfscope}%
\pgfsetbuttcap%
\pgfsetroundjoin%
\definecolor{currentfill}{rgb}{0.000000,0.000000,0.000000}%
\pgfsetfillcolor{currentfill}%
\pgfsetlinewidth{0.803000pt}%
\definecolor{currentstroke}{rgb}{0.000000,0.000000,0.000000}%
\pgfsetstrokecolor{currentstroke}%
\pgfsetdash{}{0pt}%
\pgfsys@defobject{currentmarker}{\pgfqpoint{-0.048611in}{0.000000in}}{\pgfqpoint{-0.000000in}{0.000000in}}{%
\pgfpathmoveto{\pgfqpoint{-0.000000in}{0.000000in}}%
\pgfpathlineto{\pgfqpoint{-0.048611in}{0.000000in}}%
\pgfusepath{stroke,fill}%
}%
\begin{pgfscope}%
\pgfsys@transformshift{0.721913in}{1.097883in}%
\pgfsys@useobject{currentmarker}{}%
\end{pgfscope}%
\end{pgfscope}%
\begin{pgfscope}%
\definecolor{textcolor}{rgb}{0.000000,0.000000,0.000000}%
\pgfsetstrokecolor{textcolor}%
\pgfsetfillcolor{textcolor}%
\pgftext[x=0.303703in, y=1.040013in, left, base]{\color{textcolor}{\rmfamily\fontsize{12.000000}{14.400000}\selectfont\catcode`\^=\active\def^{\ifmmode\sp\else\^{}\fi}\catcode`\%=\active\def%{\%}$\mathdefault{10^{-2}}$}}%
\end{pgfscope}%
\begin{pgfscope}%
\pgfsetbuttcap%
\pgfsetroundjoin%
\definecolor{currentfill}{rgb}{0.000000,0.000000,0.000000}%
\pgfsetfillcolor{currentfill}%
\pgfsetlinewidth{0.803000pt}%
\definecolor{currentstroke}{rgb}{0.000000,0.000000,0.000000}%
\pgfsetstrokecolor{currentstroke}%
\pgfsetdash{}{0pt}%
\pgfsys@defobject{currentmarker}{\pgfqpoint{-0.048611in}{0.000000in}}{\pgfqpoint{-0.000000in}{0.000000in}}{%
\pgfpathmoveto{\pgfqpoint{-0.000000in}{0.000000in}}%
\pgfpathlineto{\pgfqpoint{-0.048611in}{0.000000in}}%
\pgfusepath{stroke,fill}%
}%
\begin{pgfscope}%
\pgfsys@transformshift{0.721913in}{1.706378in}%
\pgfsys@useobject{currentmarker}{}%
\end{pgfscope}%
\end{pgfscope}%
\begin{pgfscope}%
\definecolor{textcolor}{rgb}{0.000000,0.000000,0.000000}%
\pgfsetstrokecolor{textcolor}%
\pgfsetfillcolor{textcolor}%
\pgftext[x=0.303703in, y=1.648507in, left, base]{\color{textcolor}{\rmfamily\fontsize{12.000000}{14.400000}\selectfont\catcode`\^=\active\def^{\ifmmode\sp\else\^{}\fi}\catcode`\%=\active\def%{\%}$\mathdefault{10^{-1}}$}}%
\end{pgfscope}%
\begin{pgfscope}%
\pgfsetbuttcap%
\pgfsetroundjoin%
\definecolor{currentfill}{rgb}{0.000000,0.000000,0.000000}%
\pgfsetfillcolor{currentfill}%
\pgfsetlinewidth{0.803000pt}%
\definecolor{currentstroke}{rgb}{0.000000,0.000000,0.000000}%
\pgfsetstrokecolor{currentstroke}%
\pgfsetdash{}{0pt}%
\pgfsys@defobject{currentmarker}{\pgfqpoint{-0.048611in}{0.000000in}}{\pgfqpoint{-0.000000in}{0.000000in}}{%
\pgfpathmoveto{\pgfqpoint{-0.000000in}{0.000000in}}%
\pgfpathlineto{\pgfqpoint{-0.048611in}{0.000000in}}%
\pgfusepath{stroke,fill}%
}%
\begin{pgfscope}%
\pgfsys@transformshift{0.721913in}{2.314872in}%
\pgfsys@useobject{currentmarker}{}%
\end{pgfscope}%
\end{pgfscope}%
\begin{pgfscope}%
\definecolor{textcolor}{rgb}{0.000000,0.000000,0.000000}%
\pgfsetstrokecolor{textcolor}%
\pgfsetfillcolor{textcolor}%
\pgftext[x=0.395525in, y=2.257002in, left, base]{\color{textcolor}{\rmfamily\fontsize{12.000000}{14.400000}\selectfont\catcode`\^=\active\def^{\ifmmode\sp\else\^{}\fi}\catcode`\%=\active\def%{\%}$\mathdefault{10^{0}}$}}%
\end{pgfscope}%
\begin{pgfscope}%
\pgfsetbuttcap%
\pgfsetroundjoin%
\definecolor{currentfill}{rgb}{0.000000,0.000000,0.000000}%
\pgfsetfillcolor{currentfill}%
\pgfsetlinewidth{0.602250pt}%
\definecolor{currentstroke}{rgb}{0.000000,0.000000,0.000000}%
\pgfsetstrokecolor{currentstroke}%
\pgfsetdash{}{0pt}%
\pgfsys@defobject{currentmarker}{\pgfqpoint{-0.027778in}{0.000000in}}{\pgfqpoint{-0.000000in}{0.000000in}}{%
\pgfpathmoveto{\pgfqpoint{-0.000000in}{0.000000in}}%
\pgfpathlineto{\pgfqpoint{-0.027778in}{0.000000in}}%
\pgfusepath{stroke,fill}%
}%
\begin{pgfscope}%
\pgfsys@transformshift{0.721913in}{0.672563in}%
\pgfsys@useobject{currentmarker}{}%
\end{pgfscope}%
\end{pgfscope}%
\begin{pgfscope}%
\pgfsetbuttcap%
\pgfsetroundjoin%
\definecolor{currentfill}{rgb}{0.000000,0.000000,0.000000}%
\pgfsetfillcolor{currentfill}%
\pgfsetlinewidth{0.602250pt}%
\definecolor{currentstroke}{rgb}{0.000000,0.000000,0.000000}%
\pgfsetstrokecolor{currentstroke}%
\pgfsetdash{}{0pt}%
\pgfsys@defobject{currentmarker}{\pgfqpoint{-0.027778in}{0.000000in}}{\pgfqpoint{-0.000000in}{0.000000in}}{%
\pgfpathmoveto{\pgfqpoint{-0.000000in}{0.000000in}}%
\pgfpathlineto{\pgfqpoint{-0.027778in}{0.000000in}}%
\pgfusepath{stroke,fill}%
}%
\begin{pgfscope}%
\pgfsys@transformshift{0.721913in}{0.779714in}%
\pgfsys@useobject{currentmarker}{}%
\end{pgfscope}%
\end{pgfscope}%
\begin{pgfscope}%
\pgfsetbuttcap%
\pgfsetroundjoin%
\definecolor{currentfill}{rgb}{0.000000,0.000000,0.000000}%
\pgfsetfillcolor{currentfill}%
\pgfsetlinewidth{0.602250pt}%
\definecolor{currentstroke}{rgb}{0.000000,0.000000,0.000000}%
\pgfsetstrokecolor{currentstroke}%
\pgfsetdash{}{0pt}%
\pgfsys@defobject{currentmarker}{\pgfqpoint{-0.027778in}{0.000000in}}{\pgfqpoint{-0.000000in}{0.000000in}}{%
\pgfpathmoveto{\pgfqpoint{-0.000000in}{0.000000in}}%
\pgfpathlineto{\pgfqpoint{-0.027778in}{0.000000in}}%
\pgfusepath{stroke,fill}%
}%
\begin{pgfscope}%
\pgfsys@transformshift{0.721913in}{0.855738in}%
\pgfsys@useobject{currentmarker}{}%
\end{pgfscope}%
\end{pgfscope}%
\begin{pgfscope}%
\pgfsetbuttcap%
\pgfsetroundjoin%
\definecolor{currentfill}{rgb}{0.000000,0.000000,0.000000}%
\pgfsetfillcolor{currentfill}%
\pgfsetlinewidth{0.602250pt}%
\definecolor{currentstroke}{rgb}{0.000000,0.000000,0.000000}%
\pgfsetstrokecolor{currentstroke}%
\pgfsetdash{}{0pt}%
\pgfsys@defobject{currentmarker}{\pgfqpoint{-0.027778in}{0.000000in}}{\pgfqpoint{-0.000000in}{0.000000in}}{%
\pgfpathmoveto{\pgfqpoint{-0.000000in}{0.000000in}}%
\pgfpathlineto{\pgfqpoint{-0.027778in}{0.000000in}}%
\pgfusepath{stroke,fill}%
}%
\begin{pgfscope}%
\pgfsys@transformshift{0.721913in}{0.914708in}%
\pgfsys@useobject{currentmarker}{}%
\end{pgfscope}%
\end{pgfscope}%
\begin{pgfscope}%
\pgfsetbuttcap%
\pgfsetroundjoin%
\definecolor{currentfill}{rgb}{0.000000,0.000000,0.000000}%
\pgfsetfillcolor{currentfill}%
\pgfsetlinewidth{0.602250pt}%
\definecolor{currentstroke}{rgb}{0.000000,0.000000,0.000000}%
\pgfsetstrokecolor{currentstroke}%
\pgfsetdash{}{0pt}%
\pgfsys@defobject{currentmarker}{\pgfqpoint{-0.027778in}{0.000000in}}{\pgfqpoint{-0.000000in}{0.000000in}}{%
\pgfpathmoveto{\pgfqpoint{-0.000000in}{0.000000in}}%
\pgfpathlineto{\pgfqpoint{-0.027778in}{0.000000in}}%
\pgfusepath{stroke,fill}%
}%
\begin{pgfscope}%
\pgfsys@transformshift{0.721913in}{0.962889in}%
\pgfsys@useobject{currentmarker}{}%
\end{pgfscope}%
\end{pgfscope}%
\begin{pgfscope}%
\pgfsetbuttcap%
\pgfsetroundjoin%
\definecolor{currentfill}{rgb}{0.000000,0.000000,0.000000}%
\pgfsetfillcolor{currentfill}%
\pgfsetlinewidth{0.602250pt}%
\definecolor{currentstroke}{rgb}{0.000000,0.000000,0.000000}%
\pgfsetstrokecolor{currentstroke}%
\pgfsetdash{}{0pt}%
\pgfsys@defobject{currentmarker}{\pgfqpoint{-0.027778in}{0.000000in}}{\pgfqpoint{-0.000000in}{0.000000in}}{%
\pgfpathmoveto{\pgfqpoint{-0.000000in}{0.000000in}}%
\pgfpathlineto{\pgfqpoint{-0.027778in}{0.000000in}}%
\pgfusepath{stroke,fill}%
}%
\begin{pgfscope}%
\pgfsys@transformshift{0.721913in}{1.003626in}%
\pgfsys@useobject{currentmarker}{}%
\end{pgfscope}%
\end{pgfscope}%
\begin{pgfscope}%
\pgfsetbuttcap%
\pgfsetroundjoin%
\definecolor{currentfill}{rgb}{0.000000,0.000000,0.000000}%
\pgfsetfillcolor{currentfill}%
\pgfsetlinewidth{0.602250pt}%
\definecolor{currentstroke}{rgb}{0.000000,0.000000,0.000000}%
\pgfsetstrokecolor{currentstroke}%
\pgfsetdash{}{0pt}%
\pgfsys@defobject{currentmarker}{\pgfqpoint{-0.027778in}{0.000000in}}{\pgfqpoint{-0.000000in}{0.000000in}}{%
\pgfpathmoveto{\pgfqpoint{-0.000000in}{0.000000in}}%
\pgfpathlineto{\pgfqpoint{-0.027778in}{0.000000in}}%
\pgfusepath{stroke,fill}%
}%
\begin{pgfscope}%
\pgfsys@transformshift{0.721913in}{1.038914in}%
\pgfsys@useobject{currentmarker}{}%
\end{pgfscope}%
\end{pgfscope}%
\begin{pgfscope}%
\pgfsetbuttcap%
\pgfsetroundjoin%
\definecolor{currentfill}{rgb}{0.000000,0.000000,0.000000}%
\pgfsetfillcolor{currentfill}%
\pgfsetlinewidth{0.602250pt}%
\definecolor{currentstroke}{rgb}{0.000000,0.000000,0.000000}%
\pgfsetstrokecolor{currentstroke}%
\pgfsetdash{}{0pt}%
\pgfsys@defobject{currentmarker}{\pgfqpoint{-0.027778in}{0.000000in}}{\pgfqpoint{-0.000000in}{0.000000in}}{%
\pgfpathmoveto{\pgfqpoint{-0.000000in}{0.000000in}}%
\pgfpathlineto{\pgfqpoint{-0.027778in}{0.000000in}}%
\pgfusepath{stroke,fill}%
}%
\begin{pgfscope}%
\pgfsys@transformshift{0.721913in}{1.070040in}%
\pgfsys@useobject{currentmarker}{}%
\end{pgfscope}%
\end{pgfscope}%
\begin{pgfscope}%
\pgfsetbuttcap%
\pgfsetroundjoin%
\definecolor{currentfill}{rgb}{0.000000,0.000000,0.000000}%
\pgfsetfillcolor{currentfill}%
\pgfsetlinewidth{0.602250pt}%
\definecolor{currentstroke}{rgb}{0.000000,0.000000,0.000000}%
\pgfsetstrokecolor{currentstroke}%
\pgfsetdash{}{0pt}%
\pgfsys@defobject{currentmarker}{\pgfqpoint{-0.027778in}{0.000000in}}{\pgfqpoint{-0.000000in}{0.000000in}}{%
\pgfpathmoveto{\pgfqpoint{-0.000000in}{0.000000in}}%
\pgfpathlineto{\pgfqpoint{-0.027778in}{0.000000in}}%
\pgfusepath{stroke,fill}%
}%
\begin{pgfscope}%
\pgfsys@transformshift{0.721913in}{1.281058in}%
\pgfsys@useobject{currentmarker}{}%
\end{pgfscope}%
\end{pgfscope}%
\begin{pgfscope}%
\pgfsetbuttcap%
\pgfsetroundjoin%
\definecolor{currentfill}{rgb}{0.000000,0.000000,0.000000}%
\pgfsetfillcolor{currentfill}%
\pgfsetlinewidth{0.602250pt}%
\definecolor{currentstroke}{rgb}{0.000000,0.000000,0.000000}%
\pgfsetstrokecolor{currentstroke}%
\pgfsetdash{}{0pt}%
\pgfsys@defobject{currentmarker}{\pgfqpoint{-0.027778in}{0.000000in}}{\pgfqpoint{-0.000000in}{0.000000in}}{%
\pgfpathmoveto{\pgfqpoint{-0.000000in}{0.000000in}}%
\pgfpathlineto{\pgfqpoint{-0.027778in}{0.000000in}}%
\pgfusepath{stroke,fill}%
}%
\begin{pgfscope}%
\pgfsys@transformshift{0.721913in}{1.388209in}%
\pgfsys@useobject{currentmarker}{}%
\end{pgfscope}%
\end{pgfscope}%
\begin{pgfscope}%
\pgfsetbuttcap%
\pgfsetroundjoin%
\definecolor{currentfill}{rgb}{0.000000,0.000000,0.000000}%
\pgfsetfillcolor{currentfill}%
\pgfsetlinewidth{0.602250pt}%
\definecolor{currentstroke}{rgb}{0.000000,0.000000,0.000000}%
\pgfsetstrokecolor{currentstroke}%
\pgfsetdash{}{0pt}%
\pgfsys@defobject{currentmarker}{\pgfqpoint{-0.027778in}{0.000000in}}{\pgfqpoint{-0.000000in}{0.000000in}}{%
\pgfpathmoveto{\pgfqpoint{-0.000000in}{0.000000in}}%
\pgfpathlineto{\pgfqpoint{-0.027778in}{0.000000in}}%
\pgfusepath{stroke,fill}%
}%
\begin{pgfscope}%
\pgfsys@transformshift{0.721913in}{1.464233in}%
\pgfsys@useobject{currentmarker}{}%
\end{pgfscope}%
\end{pgfscope}%
\begin{pgfscope}%
\pgfsetbuttcap%
\pgfsetroundjoin%
\definecolor{currentfill}{rgb}{0.000000,0.000000,0.000000}%
\pgfsetfillcolor{currentfill}%
\pgfsetlinewidth{0.602250pt}%
\definecolor{currentstroke}{rgb}{0.000000,0.000000,0.000000}%
\pgfsetstrokecolor{currentstroke}%
\pgfsetdash{}{0pt}%
\pgfsys@defobject{currentmarker}{\pgfqpoint{-0.027778in}{0.000000in}}{\pgfqpoint{-0.000000in}{0.000000in}}{%
\pgfpathmoveto{\pgfqpoint{-0.000000in}{0.000000in}}%
\pgfpathlineto{\pgfqpoint{-0.027778in}{0.000000in}}%
\pgfusepath{stroke,fill}%
}%
\begin{pgfscope}%
\pgfsys@transformshift{0.721913in}{1.523202in}%
\pgfsys@useobject{currentmarker}{}%
\end{pgfscope}%
\end{pgfscope}%
\begin{pgfscope}%
\pgfsetbuttcap%
\pgfsetroundjoin%
\definecolor{currentfill}{rgb}{0.000000,0.000000,0.000000}%
\pgfsetfillcolor{currentfill}%
\pgfsetlinewidth{0.602250pt}%
\definecolor{currentstroke}{rgb}{0.000000,0.000000,0.000000}%
\pgfsetstrokecolor{currentstroke}%
\pgfsetdash{}{0pt}%
\pgfsys@defobject{currentmarker}{\pgfqpoint{-0.027778in}{0.000000in}}{\pgfqpoint{-0.000000in}{0.000000in}}{%
\pgfpathmoveto{\pgfqpoint{-0.000000in}{0.000000in}}%
\pgfpathlineto{\pgfqpoint{-0.027778in}{0.000000in}}%
\pgfusepath{stroke,fill}%
}%
\begin{pgfscope}%
\pgfsys@transformshift{0.721913in}{1.571384in}%
\pgfsys@useobject{currentmarker}{}%
\end{pgfscope}%
\end{pgfscope}%
\begin{pgfscope}%
\pgfsetbuttcap%
\pgfsetroundjoin%
\definecolor{currentfill}{rgb}{0.000000,0.000000,0.000000}%
\pgfsetfillcolor{currentfill}%
\pgfsetlinewidth{0.602250pt}%
\definecolor{currentstroke}{rgb}{0.000000,0.000000,0.000000}%
\pgfsetstrokecolor{currentstroke}%
\pgfsetdash{}{0pt}%
\pgfsys@defobject{currentmarker}{\pgfqpoint{-0.027778in}{0.000000in}}{\pgfqpoint{-0.000000in}{0.000000in}}{%
\pgfpathmoveto{\pgfqpoint{-0.000000in}{0.000000in}}%
\pgfpathlineto{\pgfqpoint{-0.027778in}{0.000000in}}%
\pgfusepath{stroke,fill}%
}%
\begin{pgfscope}%
\pgfsys@transformshift{0.721913in}{1.612121in}%
\pgfsys@useobject{currentmarker}{}%
\end{pgfscope}%
\end{pgfscope}%
\begin{pgfscope}%
\pgfsetbuttcap%
\pgfsetroundjoin%
\definecolor{currentfill}{rgb}{0.000000,0.000000,0.000000}%
\pgfsetfillcolor{currentfill}%
\pgfsetlinewidth{0.602250pt}%
\definecolor{currentstroke}{rgb}{0.000000,0.000000,0.000000}%
\pgfsetstrokecolor{currentstroke}%
\pgfsetdash{}{0pt}%
\pgfsys@defobject{currentmarker}{\pgfqpoint{-0.027778in}{0.000000in}}{\pgfqpoint{-0.000000in}{0.000000in}}{%
\pgfpathmoveto{\pgfqpoint{-0.000000in}{0.000000in}}%
\pgfpathlineto{\pgfqpoint{-0.027778in}{0.000000in}}%
\pgfusepath{stroke,fill}%
}%
\begin{pgfscope}%
\pgfsys@transformshift{0.721913in}{1.647408in}%
\pgfsys@useobject{currentmarker}{}%
\end{pgfscope}%
\end{pgfscope}%
\begin{pgfscope}%
\pgfsetbuttcap%
\pgfsetroundjoin%
\definecolor{currentfill}{rgb}{0.000000,0.000000,0.000000}%
\pgfsetfillcolor{currentfill}%
\pgfsetlinewidth{0.602250pt}%
\definecolor{currentstroke}{rgb}{0.000000,0.000000,0.000000}%
\pgfsetstrokecolor{currentstroke}%
\pgfsetdash{}{0pt}%
\pgfsys@defobject{currentmarker}{\pgfqpoint{-0.027778in}{0.000000in}}{\pgfqpoint{-0.000000in}{0.000000in}}{%
\pgfpathmoveto{\pgfqpoint{-0.000000in}{0.000000in}}%
\pgfpathlineto{\pgfqpoint{-0.027778in}{0.000000in}}%
\pgfusepath{stroke,fill}%
}%
\begin{pgfscope}%
\pgfsys@transformshift{0.721913in}{1.678534in}%
\pgfsys@useobject{currentmarker}{}%
\end{pgfscope}%
\end{pgfscope}%
\begin{pgfscope}%
\pgfsetbuttcap%
\pgfsetroundjoin%
\definecolor{currentfill}{rgb}{0.000000,0.000000,0.000000}%
\pgfsetfillcolor{currentfill}%
\pgfsetlinewidth{0.602250pt}%
\definecolor{currentstroke}{rgb}{0.000000,0.000000,0.000000}%
\pgfsetstrokecolor{currentstroke}%
\pgfsetdash{}{0pt}%
\pgfsys@defobject{currentmarker}{\pgfqpoint{-0.027778in}{0.000000in}}{\pgfqpoint{-0.000000in}{0.000000in}}{%
\pgfpathmoveto{\pgfqpoint{-0.000000in}{0.000000in}}%
\pgfpathlineto{\pgfqpoint{-0.027778in}{0.000000in}}%
\pgfusepath{stroke,fill}%
}%
\begin{pgfscope}%
\pgfsys@transformshift{0.721913in}{1.889553in}%
\pgfsys@useobject{currentmarker}{}%
\end{pgfscope}%
\end{pgfscope}%
\begin{pgfscope}%
\pgfsetbuttcap%
\pgfsetroundjoin%
\definecolor{currentfill}{rgb}{0.000000,0.000000,0.000000}%
\pgfsetfillcolor{currentfill}%
\pgfsetlinewidth{0.602250pt}%
\definecolor{currentstroke}{rgb}{0.000000,0.000000,0.000000}%
\pgfsetstrokecolor{currentstroke}%
\pgfsetdash{}{0pt}%
\pgfsys@defobject{currentmarker}{\pgfqpoint{-0.027778in}{0.000000in}}{\pgfqpoint{-0.000000in}{0.000000in}}{%
\pgfpathmoveto{\pgfqpoint{-0.000000in}{0.000000in}}%
\pgfpathlineto{\pgfqpoint{-0.027778in}{0.000000in}}%
\pgfusepath{stroke,fill}%
}%
\begin{pgfscope}%
\pgfsys@transformshift{0.721913in}{1.996703in}%
\pgfsys@useobject{currentmarker}{}%
\end{pgfscope}%
\end{pgfscope}%
\begin{pgfscope}%
\pgfsetbuttcap%
\pgfsetroundjoin%
\definecolor{currentfill}{rgb}{0.000000,0.000000,0.000000}%
\pgfsetfillcolor{currentfill}%
\pgfsetlinewidth{0.602250pt}%
\definecolor{currentstroke}{rgb}{0.000000,0.000000,0.000000}%
\pgfsetstrokecolor{currentstroke}%
\pgfsetdash{}{0pt}%
\pgfsys@defobject{currentmarker}{\pgfqpoint{-0.027778in}{0.000000in}}{\pgfqpoint{-0.000000in}{0.000000in}}{%
\pgfpathmoveto{\pgfqpoint{-0.000000in}{0.000000in}}%
\pgfpathlineto{\pgfqpoint{-0.027778in}{0.000000in}}%
\pgfusepath{stroke,fill}%
}%
\begin{pgfscope}%
\pgfsys@transformshift{0.721913in}{2.072728in}%
\pgfsys@useobject{currentmarker}{}%
\end{pgfscope}%
\end{pgfscope}%
\begin{pgfscope}%
\pgfsetbuttcap%
\pgfsetroundjoin%
\definecolor{currentfill}{rgb}{0.000000,0.000000,0.000000}%
\pgfsetfillcolor{currentfill}%
\pgfsetlinewidth{0.602250pt}%
\definecolor{currentstroke}{rgb}{0.000000,0.000000,0.000000}%
\pgfsetstrokecolor{currentstroke}%
\pgfsetdash{}{0pt}%
\pgfsys@defobject{currentmarker}{\pgfqpoint{-0.027778in}{0.000000in}}{\pgfqpoint{-0.000000in}{0.000000in}}{%
\pgfpathmoveto{\pgfqpoint{-0.000000in}{0.000000in}}%
\pgfpathlineto{\pgfqpoint{-0.027778in}{0.000000in}}%
\pgfusepath{stroke,fill}%
}%
\begin{pgfscope}%
\pgfsys@transformshift{0.721913in}{2.131697in}%
\pgfsys@useobject{currentmarker}{}%
\end{pgfscope}%
\end{pgfscope}%
\begin{pgfscope}%
\pgfsetbuttcap%
\pgfsetroundjoin%
\definecolor{currentfill}{rgb}{0.000000,0.000000,0.000000}%
\pgfsetfillcolor{currentfill}%
\pgfsetlinewidth{0.602250pt}%
\definecolor{currentstroke}{rgb}{0.000000,0.000000,0.000000}%
\pgfsetstrokecolor{currentstroke}%
\pgfsetdash{}{0pt}%
\pgfsys@defobject{currentmarker}{\pgfqpoint{-0.027778in}{0.000000in}}{\pgfqpoint{-0.000000in}{0.000000in}}{%
\pgfpathmoveto{\pgfqpoint{-0.000000in}{0.000000in}}%
\pgfpathlineto{\pgfqpoint{-0.027778in}{0.000000in}}%
\pgfusepath{stroke,fill}%
}%
\begin{pgfscope}%
\pgfsys@transformshift{0.721913in}{2.179879in}%
\pgfsys@useobject{currentmarker}{}%
\end{pgfscope}%
\end{pgfscope}%
\begin{pgfscope}%
\pgfsetbuttcap%
\pgfsetroundjoin%
\definecolor{currentfill}{rgb}{0.000000,0.000000,0.000000}%
\pgfsetfillcolor{currentfill}%
\pgfsetlinewidth{0.602250pt}%
\definecolor{currentstroke}{rgb}{0.000000,0.000000,0.000000}%
\pgfsetstrokecolor{currentstroke}%
\pgfsetdash{}{0pt}%
\pgfsys@defobject{currentmarker}{\pgfqpoint{-0.027778in}{0.000000in}}{\pgfqpoint{-0.000000in}{0.000000in}}{%
\pgfpathmoveto{\pgfqpoint{-0.000000in}{0.000000in}}%
\pgfpathlineto{\pgfqpoint{-0.027778in}{0.000000in}}%
\pgfusepath{stroke,fill}%
}%
\begin{pgfscope}%
\pgfsys@transformshift{0.721913in}{2.220615in}%
\pgfsys@useobject{currentmarker}{}%
\end{pgfscope}%
\end{pgfscope}%
\begin{pgfscope}%
\pgfsetbuttcap%
\pgfsetroundjoin%
\definecolor{currentfill}{rgb}{0.000000,0.000000,0.000000}%
\pgfsetfillcolor{currentfill}%
\pgfsetlinewidth{0.602250pt}%
\definecolor{currentstroke}{rgb}{0.000000,0.000000,0.000000}%
\pgfsetstrokecolor{currentstroke}%
\pgfsetdash{}{0pt}%
\pgfsys@defobject{currentmarker}{\pgfqpoint{-0.027778in}{0.000000in}}{\pgfqpoint{-0.000000in}{0.000000in}}{%
\pgfpathmoveto{\pgfqpoint{-0.000000in}{0.000000in}}%
\pgfpathlineto{\pgfqpoint{-0.027778in}{0.000000in}}%
\pgfusepath{stroke,fill}%
}%
\begin{pgfscope}%
\pgfsys@transformshift{0.721913in}{2.255903in}%
\pgfsys@useobject{currentmarker}{}%
\end{pgfscope}%
\end{pgfscope}%
\begin{pgfscope}%
\pgfsetbuttcap%
\pgfsetroundjoin%
\definecolor{currentfill}{rgb}{0.000000,0.000000,0.000000}%
\pgfsetfillcolor{currentfill}%
\pgfsetlinewidth{0.602250pt}%
\definecolor{currentstroke}{rgb}{0.000000,0.000000,0.000000}%
\pgfsetstrokecolor{currentstroke}%
\pgfsetdash{}{0pt}%
\pgfsys@defobject{currentmarker}{\pgfqpoint{-0.027778in}{0.000000in}}{\pgfqpoint{-0.000000in}{0.000000in}}{%
\pgfpathmoveto{\pgfqpoint{-0.000000in}{0.000000in}}%
\pgfpathlineto{\pgfqpoint{-0.027778in}{0.000000in}}%
\pgfusepath{stroke,fill}%
}%
\begin{pgfscope}%
\pgfsys@transformshift{0.721913in}{2.287029in}%
\pgfsys@useobject{currentmarker}{}%
\end{pgfscope}%
\end{pgfscope}%
\begin{pgfscope}%
\definecolor{textcolor}{rgb}{0.000000,0.000000,0.000000}%
\pgfsetstrokecolor{textcolor}%
\pgfsetfillcolor{textcolor}%
\pgftext[x=0.248148in,y=1.511573in,,bottom,rotate=90.000000]{\color{textcolor}{\rmfamily\fontsize{12.000000}{14.400000}\selectfont\catcode`\^=\active\def^{\ifmmode\sp\else\^{}\fi}\catcode`\%=\active\def%{\%}$L^1$ relative error}}%
\end{pgfscope}%
\begin{pgfscope}%
\pgfpathrectangle{\pgfqpoint{0.721913in}{0.549073in}}{\pgfqpoint{1.937500in}{1.925000in}}%
\pgfusepath{clip}%
\pgfsetrectcap%
\pgfsetroundjoin%
\pgfsetlinewidth{1.003750pt}%
\definecolor{currentstroke}{rgb}{0.537255,0.647059,0.760784}%
\pgfsetstrokecolor{currentstroke}%
\pgfsetdash{}{0pt}%
\pgfpathmoveto{\pgfqpoint{0.809982in}{1.347823in}}%
\pgfpathlineto{\pgfqpoint{1.164146in}{1.319398in}}%
\pgfpathlineto{\pgfqpoint{1.516678in}{1.319468in}}%
\pgfpathlineto{\pgfqpoint{1.868568in}{1.319468in}}%
\pgfpathlineto{\pgfqpoint{2.219628in}{1.319468in}}%
\pgfpathlineto{\pgfqpoint{2.571345in}{1.319468in}}%
\pgfusepath{stroke}%
\end{pgfscope}%
\begin{pgfscope}%
\pgfpathrectangle{\pgfqpoint{0.721913in}{0.549073in}}{\pgfqpoint{1.937500in}{1.925000in}}%
\pgfusepath{clip}%
\pgfsetbuttcap%
\pgfsetroundjoin%
\definecolor{currentfill}{rgb}{0.537255,0.647059,0.760784}%
\pgfsetfillcolor{currentfill}%
\pgfsetlinewidth{1.003750pt}%
\definecolor{currentstroke}{rgb}{0.537255,0.647059,0.760784}%
\pgfsetstrokecolor{currentstroke}%
\pgfsetdash{}{0pt}%
\pgfsys@defobject{currentmarker}{\pgfqpoint{-0.020833in}{-0.020833in}}{\pgfqpoint{0.020833in}{0.020833in}}{%
\pgfpathmoveto{\pgfqpoint{0.000000in}{-0.020833in}}%
\pgfpathcurveto{\pgfqpoint{0.005525in}{-0.020833in}}{\pgfqpoint{0.010825in}{-0.018638in}}{\pgfqpoint{0.014731in}{-0.014731in}}%
\pgfpathcurveto{\pgfqpoint{0.018638in}{-0.010825in}}{\pgfqpoint{0.020833in}{-0.005525in}}{\pgfqpoint{0.020833in}{0.000000in}}%
\pgfpathcurveto{\pgfqpoint{0.020833in}{0.005525in}}{\pgfqpoint{0.018638in}{0.010825in}}{\pgfqpoint{0.014731in}{0.014731in}}%
\pgfpathcurveto{\pgfqpoint{0.010825in}{0.018638in}}{\pgfqpoint{0.005525in}{0.020833in}}{\pgfqpoint{0.000000in}{0.020833in}}%
\pgfpathcurveto{\pgfqpoint{-0.005525in}{0.020833in}}{\pgfqpoint{-0.010825in}{0.018638in}}{\pgfqpoint{-0.014731in}{0.014731in}}%
\pgfpathcurveto{\pgfqpoint{-0.018638in}{0.010825in}}{\pgfqpoint{-0.020833in}{0.005525in}}{\pgfqpoint{-0.020833in}{0.000000in}}%
\pgfpathcurveto{\pgfqpoint{-0.020833in}{-0.005525in}}{\pgfqpoint{-0.018638in}{-0.010825in}}{\pgfqpoint{-0.014731in}{-0.014731in}}%
\pgfpathcurveto{\pgfqpoint{-0.010825in}{-0.018638in}}{\pgfqpoint{-0.005525in}{-0.020833in}}{\pgfqpoint{0.000000in}{-0.020833in}}%
\pgfpathlineto{\pgfqpoint{0.000000in}{-0.020833in}}%
\pgfpathclose%
\pgfusepath{stroke,fill}%
}%
\begin{pgfscope}%
\pgfsys@transformshift{0.809982in}{1.347823in}%
\pgfsys@useobject{currentmarker}{}%
\end{pgfscope}%
\begin{pgfscope}%
\pgfsys@transformshift{1.164146in}{1.319398in}%
\pgfsys@useobject{currentmarker}{}%
\end{pgfscope}%
\begin{pgfscope}%
\pgfsys@transformshift{1.516678in}{1.319468in}%
\pgfsys@useobject{currentmarker}{}%
\end{pgfscope}%
\begin{pgfscope}%
\pgfsys@transformshift{1.868568in}{1.319468in}%
\pgfsys@useobject{currentmarker}{}%
\end{pgfscope}%
\begin{pgfscope}%
\pgfsys@transformshift{2.219628in}{1.319468in}%
\pgfsys@useobject{currentmarker}{}%
\end{pgfscope}%
\begin{pgfscope}%
\pgfsys@transformshift{2.571345in}{1.319468in}%
\pgfsys@useobject{currentmarker}{}%
\end{pgfscope}%
\end{pgfscope}%
\begin{pgfscope}%
\pgfpathrectangle{\pgfqpoint{0.721913in}{0.549073in}}{\pgfqpoint{1.937500in}{1.925000in}}%
\pgfusepath{clip}%
\pgfsetrectcap%
\pgfsetroundjoin%
\pgfsetlinewidth{1.003750pt}%
\definecolor{currentstroke}{rgb}{0.184314,0.270588,0.360784}%
\pgfsetstrokecolor{currentstroke}%
\pgfsetdash{}{0pt}%
\pgfpathmoveto{\pgfqpoint{0.809982in}{2.386573in}}%
\pgfpathlineto{\pgfqpoint{1.164146in}{2.233803in}}%
\pgfpathlineto{\pgfqpoint{1.516678in}{1.997671in}}%
\pgfpathlineto{\pgfqpoint{1.868568in}{1.451888in}}%
\pgfpathlineto{\pgfqpoint{2.219628in}{1.417110in}}%
\pgfpathlineto{\pgfqpoint{2.571345in}{1.417843in}}%
\pgfusepath{stroke}%
\end{pgfscope}%
\begin{pgfscope}%
\pgfpathrectangle{\pgfqpoint{0.721913in}{0.549073in}}{\pgfqpoint{1.937500in}{1.925000in}}%
\pgfusepath{clip}%
\pgfsetbuttcap%
\pgfsetroundjoin%
\definecolor{currentfill}{rgb}{0.184314,0.270588,0.360784}%
\pgfsetfillcolor{currentfill}%
\pgfsetlinewidth{1.003750pt}%
\definecolor{currentstroke}{rgb}{0.184314,0.270588,0.360784}%
\pgfsetstrokecolor{currentstroke}%
\pgfsetdash{}{0pt}%
\pgfsys@defobject{currentmarker}{\pgfqpoint{-0.020833in}{-0.020833in}}{\pgfqpoint{0.020833in}{0.020833in}}{%
\pgfpathmoveto{\pgfqpoint{0.000000in}{-0.020833in}}%
\pgfpathcurveto{\pgfqpoint{0.005525in}{-0.020833in}}{\pgfqpoint{0.010825in}{-0.018638in}}{\pgfqpoint{0.014731in}{-0.014731in}}%
\pgfpathcurveto{\pgfqpoint{0.018638in}{-0.010825in}}{\pgfqpoint{0.020833in}{-0.005525in}}{\pgfqpoint{0.020833in}{0.000000in}}%
\pgfpathcurveto{\pgfqpoint{0.020833in}{0.005525in}}{\pgfqpoint{0.018638in}{0.010825in}}{\pgfqpoint{0.014731in}{0.014731in}}%
\pgfpathcurveto{\pgfqpoint{0.010825in}{0.018638in}}{\pgfqpoint{0.005525in}{0.020833in}}{\pgfqpoint{0.000000in}{0.020833in}}%
\pgfpathcurveto{\pgfqpoint{-0.005525in}{0.020833in}}{\pgfqpoint{-0.010825in}{0.018638in}}{\pgfqpoint{-0.014731in}{0.014731in}}%
\pgfpathcurveto{\pgfqpoint{-0.018638in}{0.010825in}}{\pgfqpoint{-0.020833in}{0.005525in}}{\pgfqpoint{-0.020833in}{0.000000in}}%
\pgfpathcurveto{\pgfqpoint{-0.020833in}{-0.005525in}}{\pgfqpoint{-0.018638in}{-0.010825in}}{\pgfqpoint{-0.014731in}{-0.014731in}}%
\pgfpathcurveto{\pgfqpoint{-0.010825in}{-0.018638in}}{\pgfqpoint{-0.005525in}{-0.020833in}}{\pgfqpoint{0.000000in}{-0.020833in}}%
\pgfpathlineto{\pgfqpoint{0.000000in}{-0.020833in}}%
\pgfpathclose%
\pgfusepath{stroke,fill}%
}%
\begin{pgfscope}%
\pgfsys@transformshift{0.809982in}{2.386573in}%
\pgfsys@useobject{currentmarker}{}%
\end{pgfscope}%
\begin{pgfscope}%
\pgfsys@transformshift{1.164146in}{2.233803in}%
\pgfsys@useobject{currentmarker}{}%
\end{pgfscope}%
\begin{pgfscope}%
\pgfsys@transformshift{1.516678in}{1.997671in}%
\pgfsys@useobject{currentmarker}{}%
\end{pgfscope}%
\begin{pgfscope}%
\pgfsys@transformshift{1.868568in}{1.451888in}%
\pgfsys@useobject{currentmarker}{}%
\end{pgfscope}%
\begin{pgfscope}%
\pgfsys@transformshift{2.219628in}{1.417110in}%
\pgfsys@useobject{currentmarker}{}%
\end{pgfscope}%
\begin{pgfscope}%
\pgfsys@transformshift{2.571345in}{1.417843in}%
\pgfsys@useobject{currentmarker}{}%
\end{pgfscope}%
\end{pgfscope}%
\begin{pgfscope}%
\pgfpathrectangle{\pgfqpoint{0.721913in}{0.549073in}}{\pgfqpoint{1.937500in}{1.925000in}}%
\pgfusepath{clip}%
\pgfsetrectcap%
\pgfsetroundjoin%
\pgfsetlinewidth{1.003750pt}%
\definecolor{currentstroke}{rgb}{0.976471,0.505882,0.145098}%
\pgfsetstrokecolor{currentstroke}%
\pgfsetdash{}{0pt}%
\pgfpathmoveto{\pgfqpoint{0.809982in}{2.154716in}}%
\pgfpathlineto{\pgfqpoint{1.164146in}{2.074881in}}%
\pgfpathlineto{\pgfqpoint{1.516678in}{1.743830in}}%
\pgfpathlineto{\pgfqpoint{1.868568in}{1.129851in}}%
\pgfpathlineto{\pgfqpoint{2.219628in}{0.662125in}}%
\pgfpathlineto{\pgfqpoint{2.571345in}{0.636573in}}%
\pgfusepath{stroke}%
\end{pgfscope}%
\begin{pgfscope}%
\pgfpathrectangle{\pgfqpoint{0.721913in}{0.549073in}}{\pgfqpoint{1.937500in}{1.925000in}}%
\pgfusepath{clip}%
\pgfsetbuttcap%
\pgfsetroundjoin%
\definecolor{currentfill}{rgb}{0.976471,0.505882,0.145098}%
\pgfsetfillcolor{currentfill}%
\pgfsetlinewidth{1.003750pt}%
\definecolor{currentstroke}{rgb}{0.976471,0.505882,0.145098}%
\pgfsetstrokecolor{currentstroke}%
\pgfsetdash{}{0pt}%
\pgfsys@defobject{currentmarker}{\pgfqpoint{-0.020833in}{-0.020833in}}{\pgfqpoint{0.020833in}{0.020833in}}{%
\pgfpathmoveto{\pgfqpoint{0.000000in}{-0.020833in}}%
\pgfpathcurveto{\pgfqpoint{0.005525in}{-0.020833in}}{\pgfqpoint{0.010825in}{-0.018638in}}{\pgfqpoint{0.014731in}{-0.014731in}}%
\pgfpathcurveto{\pgfqpoint{0.018638in}{-0.010825in}}{\pgfqpoint{0.020833in}{-0.005525in}}{\pgfqpoint{0.020833in}{0.000000in}}%
\pgfpathcurveto{\pgfqpoint{0.020833in}{0.005525in}}{\pgfqpoint{0.018638in}{0.010825in}}{\pgfqpoint{0.014731in}{0.014731in}}%
\pgfpathcurveto{\pgfqpoint{0.010825in}{0.018638in}}{\pgfqpoint{0.005525in}{0.020833in}}{\pgfqpoint{0.000000in}{0.020833in}}%
\pgfpathcurveto{\pgfqpoint{-0.005525in}{0.020833in}}{\pgfqpoint{-0.010825in}{0.018638in}}{\pgfqpoint{-0.014731in}{0.014731in}}%
\pgfpathcurveto{\pgfqpoint{-0.018638in}{0.010825in}}{\pgfqpoint{-0.020833in}{0.005525in}}{\pgfqpoint{-0.020833in}{0.000000in}}%
\pgfpathcurveto{\pgfqpoint{-0.020833in}{-0.005525in}}{\pgfqpoint{-0.018638in}{-0.010825in}}{\pgfqpoint{-0.014731in}{-0.014731in}}%
\pgfpathcurveto{\pgfqpoint{-0.010825in}{-0.018638in}}{\pgfqpoint{-0.005525in}{-0.020833in}}{\pgfqpoint{0.000000in}{-0.020833in}}%
\pgfpathlineto{\pgfqpoint{0.000000in}{-0.020833in}}%
\pgfpathclose%
\pgfusepath{stroke,fill}%
}%
\begin{pgfscope}%
\pgfsys@transformshift{0.809982in}{2.154716in}%
\pgfsys@useobject{currentmarker}{}%
\end{pgfscope}%
\begin{pgfscope}%
\pgfsys@transformshift{1.164146in}{2.074881in}%
\pgfsys@useobject{currentmarker}{}%
\end{pgfscope}%
\begin{pgfscope}%
\pgfsys@transformshift{1.516678in}{1.743830in}%
\pgfsys@useobject{currentmarker}{}%
\end{pgfscope}%
\begin{pgfscope}%
\pgfsys@transformshift{1.868568in}{1.129851in}%
\pgfsys@useobject{currentmarker}{}%
\end{pgfscope}%
\begin{pgfscope}%
\pgfsys@transformshift{2.219628in}{0.662125in}%
\pgfsys@useobject{currentmarker}{}%
\end{pgfscope}%
\begin{pgfscope}%
\pgfsys@transformshift{2.571345in}{0.636573in}%
\pgfsys@useobject{currentmarker}{}%
\end{pgfscope}%
\end{pgfscope}%
\begin{pgfscope}%
\pgfsetrectcap%
\pgfsetmiterjoin%
\pgfsetlinewidth{0.803000pt}%
\definecolor{currentstroke}{rgb}{0.000000,0.000000,0.000000}%
\pgfsetstrokecolor{currentstroke}%
\pgfsetdash{}{0pt}%
\pgfpathmoveto{\pgfqpoint{0.721913in}{0.549073in}}%
\pgfpathlineto{\pgfqpoint{0.721913in}{2.474073in}}%
\pgfusepath{stroke}%
\end{pgfscope}%
\begin{pgfscope}%
\pgfsetrectcap%
\pgfsetmiterjoin%
\pgfsetlinewidth{0.803000pt}%
\definecolor{currentstroke}{rgb}{0.000000,0.000000,0.000000}%
\pgfsetstrokecolor{currentstroke}%
\pgfsetdash{}{0pt}%
\pgfpathmoveto{\pgfqpoint{2.659413in}{0.549073in}}%
\pgfpathlineto{\pgfqpoint{2.659413in}{2.474073in}}%
\pgfusepath{stroke}%
\end{pgfscope}%
\begin{pgfscope}%
\pgfsetrectcap%
\pgfsetmiterjoin%
\pgfsetlinewidth{0.803000pt}%
\definecolor{currentstroke}{rgb}{0.000000,0.000000,0.000000}%
\pgfsetstrokecolor{currentstroke}%
\pgfsetdash{}{0pt}%
\pgfpathmoveto{\pgfqpoint{0.721913in}{0.549073in}}%
\pgfpathlineto{\pgfqpoint{2.659413in}{0.549073in}}%
\pgfusepath{stroke}%
\end{pgfscope}%
\begin{pgfscope}%
\pgfsetrectcap%
\pgfsetmiterjoin%
\pgfsetlinewidth{0.803000pt}%
\definecolor{currentstroke}{rgb}{0.000000,0.000000,0.000000}%
\pgfsetstrokecolor{currentstroke}%
\pgfsetdash{}{0pt}%
\pgfpathmoveto{\pgfqpoint{0.721913in}{2.474073in}}%
\pgfpathlineto{\pgfqpoint{2.659413in}{2.474073in}}%
\pgfusepath{stroke}%
\end{pgfscope}%
\begin{pgfscope}%
\pgfsetbuttcap%
\pgfsetmiterjoin%
\definecolor{currentfill}{rgb}{1.000000,1.000000,1.000000}%
\pgfsetfillcolor{currentfill}%
\pgfsetfillopacity{0.800000}%
\pgfsetlinewidth{1.003750pt}%
\definecolor{currentstroke}{rgb}{0.800000,0.800000,0.800000}%
\pgfsetstrokecolor{currentstroke}%
\pgfsetstrokeopacity{0.800000}%
\pgfsetdash{}{0pt}%
\pgfpathmoveto{\pgfqpoint{1.392965in}{1.643518in}}%
\pgfpathlineto{\pgfqpoint{2.542747in}{1.643518in}}%
\pgfpathquadraticcurveto{\pgfqpoint{2.576080in}{1.643518in}}{\pgfqpoint{2.576080in}{1.676852in}}%
\pgfpathlineto{\pgfqpoint{2.576080in}{2.357406in}}%
\pgfpathquadraticcurveto{\pgfqpoint{2.576080in}{2.390739in}}{\pgfqpoint{2.542747in}{2.390739in}}%
\pgfpathlineto{\pgfqpoint{1.392965in}{2.390739in}}%
\pgfpathquadraticcurveto{\pgfqpoint{1.359632in}{2.390739in}}{\pgfqpoint{1.359632in}{2.357406in}}%
\pgfpathlineto{\pgfqpoint{1.359632in}{1.676852in}}%
\pgfpathquadraticcurveto{\pgfqpoint{1.359632in}{1.643518in}}{\pgfqpoint{1.392965in}{1.643518in}}%
\pgfpathlineto{\pgfqpoint{1.392965in}{1.643518in}}%
\pgfpathclose%
\pgfusepath{stroke,fill}%
\end{pgfscope}%
\begin{pgfscope}%
\pgfsetrectcap%
\pgfsetroundjoin%
\pgfsetlinewidth{1.003750pt}%
\definecolor{currentstroke}{rgb}{0.537255,0.647059,0.760784}%
\pgfsetstrokecolor{currentstroke}%
\pgfsetdash{}{0pt}%
\pgfpathmoveto{\pgfqpoint{1.426299in}{2.265739in}}%
\pgfpathlineto{\pgfqpoint{1.592965in}{2.265739in}}%
\pgfpathlineto{\pgfqpoint{1.759632in}{2.265739in}}%
\pgfusepath{stroke}%
\end{pgfscope}%
\begin{pgfscope}%
\pgfsetbuttcap%
\pgfsetroundjoin%
\definecolor{currentfill}{rgb}{0.537255,0.647059,0.760784}%
\pgfsetfillcolor{currentfill}%
\pgfsetlinewidth{1.003750pt}%
\definecolor{currentstroke}{rgb}{0.537255,0.647059,0.760784}%
\pgfsetstrokecolor{currentstroke}%
\pgfsetdash{}{0pt}%
\pgfsys@defobject{currentmarker}{\pgfqpoint{-0.020833in}{-0.020833in}}{\pgfqpoint{0.020833in}{0.020833in}}{%
\pgfpathmoveto{\pgfqpoint{0.000000in}{-0.020833in}}%
\pgfpathcurveto{\pgfqpoint{0.005525in}{-0.020833in}}{\pgfqpoint{0.010825in}{-0.018638in}}{\pgfqpoint{0.014731in}{-0.014731in}}%
\pgfpathcurveto{\pgfqpoint{0.018638in}{-0.010825in}}{\pgfqpoint{0.020833in}{-0.005525in}}{\pgfqpoint{0.020833in}{0.000000in}}%
\pgfpathcurveto{\pgfqpoint{0.020833in}{0.005525in}}{\pgfqpoint{0.018638in}{0.010825in}}{\pgfqpoint{0.014731in}{0.014731in}}%
\pgfpathcurveto{\pgfqpoint{0.010825in}{0.018638in}}{\pgfqpoint{0.005525in}{0.020833in}}{\pgfqpoint{0.000000in}{0.020833in}}%
\pgfpathcurveto{\pgfqpoint{-0.005525in}{0.020833in}}{\pgfqpoint{-0.010825in}{0.018638in}}{\pgfqpoint{-0.014731in}{0.014731in}}%
\pgfpathcurveto{\pgfqpoint{-0.018638in}{0.010825in}}{\pgfqpoint{-0.020833in}{0.005525in}}{\pgfqpoint{-0.020833in}{0.000000in}}%
\pgfpathcurveto{\pgfqpoint{-0.020833in}{-0.005525in}}{\pgfqpoint{-0.018638in}{-0.010825in}}{\pgfqpoint{-0.014731in}{-0.014731in}}%
\pgfpathcurveto{\pgfqpoint{-0.010825in}{-0.018638in}}{\pgfqpoint{-0.005525in}{-0.020833in}}{\pgfqpoint{0.000000in}{-0.020833in}}%
\pgfpathlineto{\pgfqpoint{0.000000in}{-0.020833in}}%
\pgfpathclose%
\pgfusepath{stroke,fill}%
}%
\begin{pgfscope}%
\pgfsys@transformshift{1.592965in}{2.265739in}%
\pgfsys@useobject{currentmarker}{}%
\end{pgfscope}%
\end{pgfscope}%
\begin{pgfscope}%
\definecolor{textcolor}{rgb}{0.000000,0.000000,0.000000}%
\pgfsetstrokecolor{textcolor}%
\pgfsetfillcolor{textcolor}%
\pgftext[x=1.892965in,y=2.207406in,left,base]{\color{textcolor}{\rmfamily\fontsize{12.000000}{14.400000}\selectfont\catcode`\^=\active\def^{\ifmmode\sp\else\^{}\fi}\catcode`\%=\active\def%{\%}Haydock}}%
\end{pgfscope}%
\begin{pgfscope}%
\pgfsetrectcap%
\pgfsetroundjoin%
\pgfsetlinewidth{1.003750pt}%
\definecolor{currentstroke}{rgb}{0.184314,0.270588,0.360784}%
\pgfsetstrokecolor{currentstroke}%
\pgfsetdash{}{0pt}%
\pgfpathmoveto{\pgfqpoint{1.426299in}{2.033332in}}%
\pgfpathlineto{\pgfqpoint{1.592965in}{2.033332in}}%
\pgfpathlineto{\pgfqpoint{1.759632in}{2.033332in}}%
\pgfusepath{stroke}%
\end{pgfscope}%
\begin{pgfscope}%
\pgfsetbuttcap%
\pgfsetroundjoin%
\definecolor{currentfill}{rgb}{0.184314,0.270588,0.360784}%
\pgfsetfillcolor{currentfill}%
\pgfsetlinewidth{1.003750pt}%
\definecolor{currentstroke}{rgb}{0.184314,0.270588,0.360784}%
\pgfsetstrokecolor{currentstroke}%
\pgfsetdash{}{0pt}%
\pgfsys@defobject{currentmarker}{\pgfqpoint{-0.020833in}{-0.020833in}}{\pgfqpoint{0.020833in}{0.020833in}}{%
\pgfpathmoveto{\pgfqpoint{0.000000in}{-0.020833in}}%
\pgfpathcurveto{\pgfqpoint{0.005525in}{-0.020833in}}{\pgfqpoint{0.010825in}{-0.018638in}}{\pgfqpoint{0.014731in}{-0.014731in}}%
\pgfpathcurveto{\pgfqpoint{0.018638in}{-0.010825in}}{\pgfqpoint{0.020833in}{-0.005525in}}{\pgfqpoint{0.020833in}{0.000000in}}%
\pgfpathcurveto{\pgfqpoint{0.020833in}{0.005525in}}{\pgfqpoint{0.018638in}{0.010825in}}{\pgfqpoint{0.014731in}{0.014731in}}%
\pgfpathcurveto{\pgfqpoint{0.010825in}{0.018638in}}{\pgfqpoint{0.005525in}{0.020833in}}{\pgfqpoint{0.000000in}{0.020833in}}%
\pgfpathcurveto{\pgfqpoint{-0.005525in}{0.020833in}}{\pgfqpoint{-0.010825in}{0.018638in}}{\pgfqpoint{-0.014731in}{0.014731in}}%
\pgfpathcurveto{\pgfqpoint{-0.018638in}{0.010825in}}{\pgfqpoint{-0.020833in}{0.005525in}}{\pgfqpoint{-0.020833in}{0.000000in}}%
\pgfpathcurveto{\pgfqpoint{-0.020833in}{-0.005525in}}{\pgfqpoint{-0.018638in}{-0.010825in}}{\pgfqpoint{-0.014731in}{-0.014731in}}%
\pgfpathcurveto{\pgfqpoint{-0.010825in}{-0.018638in}}{\pgfqpoint{-0.005525in}{-0.020833in}}{\pgfqpoint{0.000000in}{-0.020833in}}%
\pgfpathlineto{\pgfqpoint{0.000000in}{-0.020833in}}%
\pgfpathclose%
\pgfusepath{stroke,fill}%
}%
\begin{pgfscope}%
\pgfsys@transformshift{1.592965in}{2.033332in}%
\pgfsys@useobject{currentmarker}{}%
\end{pgfscope}%
\end{pgfscope}%
\begin{pgfscope}%
\definecolor{textcolor}{rgb}{0.000000,0.000000,0.000000}%
\pgfsetstrokecolor{textcolor}%
\pgfsetfillcolor{textcolor}%
\pgftext[x=1.892965in,y=1.974999in,left,base]{\color{textcolor}{\rmfamily\fontsize{12.000000}{14.400000}\selectfont\catcode`\^=\active\def^{\ifmmode\sp\else\^{}\fi}\catcode`\%=\active\def%{\%}NC}}%
\end{pgfscope}%
\begin{pgfscope}%
\pgfsetrectcap%
\pgfsetroundjoin%
\pgfsetlinewidth{1.003750pt}%
\definecolor{currentstroke}{rgb}{0.976471,0.505882,0.145098}%
\pgfsetstrokecolor{currentstroke}%
\pgfsetdash{}{0pt}%
\pgfpathmoveto{\pgfqpoint{1.426299in}{1.800925in}}%
\pgfpathlineto{\pgfqpoint{1.592965in}{1.800925in}}%
\pgfpathlineto{\pgfqpoint{1.759632in}{1.800925in}}%
\pgfusepath{stroke}%
\end{pgfscope}%
\begin{pgfscope}%
\pgfsetbuttcap%
\pgfsetroundjoin%
\definecolor{currentfill}{rgb}{0.976471,0.505882,0.145098}%
\pgfsetfillcolor{currentfill}%
\pgfsetlinewidth{1.003750pt}%
\definecolor{currentstroke}{rgb}{0.976471,0.505882,0.145098}%
\pgfsetstrokecolor{currentstroke}%
\pgfsetdash{}{0pt}%
\pgfsys@defobject{currentmarker}{\pgfqpoint{-0.020833in}{-0.020833in}}{\pgfqpoint{0.020833in}{0.020833in}}{%
\pgfpathmoveto{\pgfqpoint{0.000000in}{-0.020833in}}%
\pgfpathcurveto{\pgfqpoint{0.005525in}{-0.020833in}}{\pgfqpoint{0.010825in}{-0.018638in}}{\pgfqpoint{0.014731in}{-0.014731in}}%
\pgfpathcurveto{\pgfqpoint{0.018638in}{-0.010825in}}{\pgfqpoint{0.020833in}{-0.005525in}}{\pgfqpoint{0.020833in}{0.000000in}}%
\pgfpathcurveto{\pgfqpoint{0.020833in}{0.005525in}}{\pgfqpoint{0.018638in}{0.010825in}}{\pgfqpoint{0.014731in}{0.014731in}}%
\pgfpathcurveto{\pgfqpoint{0.010825in}{0.018638in}}{\pgfqpoint{0.005525in}{0.020833in}}{\pgfqpoint{0.000000in}{0.020833in}}%
\pgfpathcurveto{\pgfqpoint{-0.005525in}{0.020833in}}{\pgfqpoint{-0.010825in}{0.018638in}}{\pgfqpoint{-0.014731in}{0.014731in}}%
\pgfpathcurveto{\pgfqpoint{-0.018638in}{0.010825in}}{\pgfqpoint{-0.020833in}{0.005525in}}{\pgfqpoint{-0.020833in}{0.000000in}}%
\pgfpathcurveto{\pgfqpoint{-0.020833in}{-0.005525in}}{\pgfqpoint{-0.018638in}{-0.010825in}}{\pgfqpoint{-0.014731in}{-0.014731in}}%
\pgfpathcurveto{\pgfqpoint{-0.010825in}{-0.018638in}}{\pgfqpoint{-0.005525in}{-0.020833in}}{\pgfqpoint{0.000000in}{-0.020833in}}%
\pgfpathlineto{\pgfqpoint{0.000000in}{-0.020833in}}%
\pgfpathclose%
\pgfusepath{stroke,fill}%
}%
\begin{pgfscope}%
\pgfsys@transformshift{1.592965in}{1.800925in}%
\pgfsys@useobject{currentmarker}{}%
\end{pgfscope}%
\end{pgfscope}%
\begin{pgfscope}%
\definecolor{textcolor}{rgb}{0.000000,0.000000,0.000000}%
\pgfsetstrokecolor{textcolor}%
\pgfsetfillcolor{textcolor}%
\pgftext[x=1.892965in,y=1.742592in,left,base]{\color{textcolor}{\rmfamily\fontsize{12.000000}{14.400000}\selectfont\catcode`\^=\active\def^{\ifmmode\sp\else\^{}\fi}\catcode`\%=\active\def%{\%}NC++}}%
\end{pgfscope}%
\end{pgfpicture}%
\makeatother%
\endgroup%

        \caption{\gls{sketch-size} $+$ \gls{num-hutchinson-queries} $=160$}
        \label{fig:5-experiments-haydock-convergence-m-nv160}
    \end{subfigure}
    \caption{For increasing values of \gls{chebyshev-degree} but fixed
    \gls{sketch-size} $+$ \gls{num-hutchinson-queries} we plot the $L^1$ relative
    approximation error \refequ{equ:5-experiments-L1-error}
    for the model problem from \refsec{sec:5-experiments-density-function} with
    the Lorentzian kernel with \gls{smoothing-parameter} $=0.05$.}
    \label{fig:5-experiments-haydock-convergence-m}
\end{figure}

On one hand the low-rank factorization for the Lorentzian \gls{smoothing-kernel}
is not as effective as it was for the Gaussian case, since the decay to zero
is noticeably slower (see \reffig{fig:5-experiments-haydock-kernel}). On the
other, the Lorentzian \gls{smoothing-kernel} has a pole at $s = \pm \iota$, which
has as a consequence that the Chebyshev expansion is not guaranteed to converge
as fast as it does in the Gaussian case according to \refthm{thm:2-chebyshev-bernstein}.
Due to these reasons, the convergence of the \gls{NC} and \gls{NCPP} methods
are slower than they used to be in \refsec{sec:5-experiments-density-function}.
Nevertheless, this choice of \gls{smoothing-kernel} exhibits perfectly the 
$\mathcal{O}(\varepsilon^{-1})$ and $\mathcal{O}(\varepsilon^{-2})$ convergence
orders of the $L^1$ approximation error, which the \gls{DGC} and \gls{NCPP}
methods respectively guarantee with respect to
\gls{sketch-size} $+$ \gls{num-hutchinson-queries} in \reffig{fig:5-experiments-haydock-convergence-nv-m2400}.\\

\begin{table}[ht]
    \caption{Comparison of the runtime in seconds of the algorithms applied to the model problem
    from \refsec{sec:5-experiments-density-function}
    for approximating the \glsfirst{smooth-spectral-density} with a Lorentzian kernel with
    \gls{smoothing-parameter} $=0.05$ at \gls{num-evaluation-points} $=100$
    points for various choices of \gls{chebyshev-degree} and \gls{sketch-size} $+$ \gls{num-hutchinson-queries}.
    The mean and standard deviation of 7 runs is given.}
    \label{tab:5-experiments-timing-haydock}
   \centering
\renewcommand{\arraystretch}{1.2}
\begin{tabular}{@{}lcccc@{}}
\toprule
 & \shortstack[c]{$m=800$ \\ $n_{\Omega} + n_{\Psi}=40$} & \shortstack[c]{$m=2400$ \\ $n_{\Omega} + n_{\Psi}=40$} & \shortstack[c]{$m=800$ \\ $n_{\Omega} + n_{\Psi}=160$} & \shortstack[c]{$m=2400$ \\ $n_{\Omega} + n_{\Psi}=160$}\\
\midrule
Haydock & $4.544$ $\pm$ $0.022$ & $9.887$ $\pm$ $0.028$ & $18.274$ $\pm$ $0.134$ & $39.137$ $\pm$ $0.237$ \\
NC & $0.913$ $\pm$ $0.013$ & $2.713$ $\pm$ $0.031$ & $3.761$ $\pm$ $0.022$ & $11.322$ $\pm$ $0.155$ \\
NC++ & $0.807$ $\pm$ $0.004$ & $2.392$ $\pm$ $0.027$ & $2.658$ $\pm$ $0.003$ & $7.970$ $\pm$ $0.038$ \\
\bottomrule
\end{tabular}

\end{table}

%%%%%%%%%%%%%%%%%%%%%%%%%%%%%%%%%%%%%%%%%%%%%%%%%%%%%%%%%%%%%%%%%%%%%%%%%%%%%%%%

\clearpage
\section{Application to various matrices}
\label{sec:5-experiments-various-matrices}

We test the algorithms on various problems encountered in literature.
We take a synthetic sparse matrix with $2000$ uniformly distributed eigenvalues
in $[-1, 1]$ \cite{chen2021slq};
GOE, a matrix $\mtx{A} = (\mtx{G} + \mtx{G}^{\top})/\sqrt{2}$ with standard
normal $\mtx{G} \in \mathbb{R}^{1000 \times 1000}$ from the Gaussian Orthogonal Ensemble;
the matrix ModES3D\_8, an $8000 \times 8000$ sparse matrix resulting
from the same problem as in \refsec{sec:5-experiments-density-function} but with
a larger computational domain \cite{lin2017randomized}; and
Erdos992\footnote{\url{https://sparse.tamu.edu/Pajek/Erdos992}},
an $6100 \times 6100$ sparse matrix representing the collaboration network of the
Hungarian mathematician P\'al Erd\H{o}s from \cite{chen2021slq}.
%the matrix nd3k\footnote{\url{https://sparse.tamu.edu/ND/nd3k}},
%an $9000 \times 9000$ matrix
%which originates from modelling the vibrational modes of a polyethylene molecule
%with 3000 atoms \cite{lin2016review},
All these
matrices are symmetric. For all these matrices, we compute, for
fixed \gls{chebyshev-degree} $=2400$ and increasing \gls{sketch-size} $+$ \gls{num-hutchinson-queries},
the relative $L^1$ approximation error
of the spectral density for a Gaussian \glsfirst{smoothing-kernel} with
\gls{smoothing-parameter} $=0.05$. The resulting plots are printed in 
\reffig{fig:5-experiments-multi-matrix-convergence}.\\ 

We observe that for the two matrices which have an evenly distributed spectrum
(\reffig{fig:5-experiments-multi-matrix-convergence-uniform}) or an approximately
evenly distributed spectrum (\reffig{fig:5-experiments-multi-matrix-convergence-ModES3D}),
the \gls{NC} method by itself can achieve a good approximation once \gls{sketch-size}
exceeds the numerical rank of the matrix. On the other hand, for matrices where
the spectrum is very concentrated around a certain point (\reffig{fig:5-experiments-multi-matrix-convergence-Erdos})
or approximately describes a semi-circle (\reffig{fig:5-experiments-multi-matrix-convergence-goe}) \cite{wigner1958distribution},
%or exhibits large gaps (\reffig{fig:5-experiments-multi-matrix-convergence-nd3k}),
the \gls{NC} is not as effective, and the correction part in the \gls{NCPP} makes
a significant difference.

\begin{figure}[ht]
    \centering
    \begin{subfigure}[b]{0.49\columnwidth}
        %% Creator: Matplotlib, PGF backend
%%
%% To include the figure in your LaTeX document, write
%%   \input{<filename>.pgf}
%%
%% Make sure the required packages are loaded in your preamble
%%   \usepackage{pgf}
%%
%% Also ensure that all the required font packages are loaded; for instance,
%% the lmodern package is sometimes necessary when using math font.
%%   \usepackage{lmodern}
%%
%% Figures using additional raster images can only be included by \input if
%% they are in the same directory as the main LaTeX file. For loading figures
%% from other directories you can use the `import` package
%%   \usepackage{import}
%%
%% and then include the figures with
%%   \import{<path to file>}{<filename>.pgf}
%%
%% Matplotlib used the following preamble
%%   \def\mathdefault#1{#1}
%%   \everymath=\expandafter{\the\everymath\displaystyle}
%%   
%%   \makeatletter\@ifpackageloaded{underscore}{}{\usepackage[strings]{underscore}}\makeatother
%%
\begingroup%
\makeatletter%
\begin{pgfpicture}%
\pgfpathrectangle{\pgfpointorigin}{\pgfqpoint{2.759413in}{2.574073in}}%
\pgfusepath{use as bounding box, clip}%
\begin{pgfscope}%
\pgfsetbuttcap%
\pgfsetmiterjoin%
\definecolor{currentfill}{rgb}{1.000000,1.000000,1.000000}%
\pgfsetfillcolor{currentfill}%
\pgfsetlinewidth{0.000000pt}%
\definecolor{currentstroke}{rgb}{1.000000,1.000000,1.000000}%
\pgfsetstrokecolor{currentstroke}%
\pgfsetdash{}{0pt}%
\pgfpathmoveto{\pgfqpoint{0.000000in}{0.000000in}}%
\pgfpathlineto{\pgfqpoint{2.759413in}{0.000000in}}%
\pgfpathlineto{\pgfqpoint{2.759413in}{2.574073in}}%
\pgfpathlineto{\pgfqpoint{0.000000in}{2.574073in}}%
\pgfpathlineto{\pgfqpoint{0.000000in}{0.000000in}}%
\pgfpathclose%
\pgfusepath{fill}%
\end{pgfscope}%
\begin{pgfscope}%
\pgfsetbuttcap%
\pgfsetmiterjoin%
\definecolor{currentfill}{rgb}{1.000000,1.000000,1.000000}%
\pgfsetfillcolor{currentfill}%
\pgfsetlinewidth{0.000000pt}%
\definecolor{currentstroke}{rgb}{0.000000,0.000000,0.000000}%
\pgfsetstrokecolor{currentstroke}%
\pgfsetstrokeopacity{0.000000}%
\pgfsetdash{}{0pt}%
\pgfpathmoveto{\pgfqpoint{0.721913in}{0.549073in}}%
\pgfpathlineto{\pgfqpoint{2.659413in}{0.549073in}}%
\pgfpathlineto{\pgfqpoint{2.659413in}{2.474073in}}%
\pgfpathlineto{\pgfqpoint{0.721913in}{2.474073in}}%
\pgfpathlineto{\pgfqpoint{0.721913in}{0.549073in}}%
\pgfpathclose%
\pgfusepath{fill}%
\end{pgfscope}%
\begin{pgfscope}%
\pgfsetbuttcap%
\pgfsetroundjoin%
\definecolor{currentfill}{rgb}{0.000000,0.000000,0.000000}%
\pgfsetfillcolor{currentfill}%
\pgfsetlinewidth{0.803000pt}%
\definecolor{currentstroke}{rgb}{0.000000,0.000000,0.000000}%
\pgfsetstrokecolor{currentstroke}%
\pgfsetdash{}{0pt}%
\pgfsys@defobject{currentmarker}{\pgfqpoint{0.000000in}{-0.048611in}}{\pgfqpoint{0.000000in}{0.000000in}}{%
\pgfpathmoveto{\pgfqpoint{0.000000in}{0.000000in}}%
\pgfpathlineto{\pgfqpoint{0.000000in}{-0.048611in}}%
\pgfusepath{stroke,fill}%
}%
\begin{pgfscope}%
\pgfsys@transformshift{1.771566in}{0.549073in}%
\pgfsys@useobject{currentmarker}{}%
\end{pgfscope}%
\end{pgfscope}%
\begin{pgfscope}%
\definecolor{textcolor}{rgb}{0.000000,0.000000,0.000000}%
\pgfsetstrokecolor{textcolor}%
\pgfsetfillcolor{textcolor}%
\pgftext[x=1.771566in,y=0.451851in,,top]{\color{textcolor}{\rmfamily\fontsize{12.000000}{14.400000}\selectfont\catcode`\^=\active\def^{\ifmmode\sp\else\^{}\fi}\catcode`\%=\active\def%{\%}$\mathdefault{10^{2}}$}}%
\end{pgfscope}%
\begin{pgfscope}%
\pgfsetbuttcap%
\pgfsetroundjoin%
\definecolor{currentfill}{rgb}{0.000000,0.000000,0.000000}%
\pgfsetfillcolor{currentfill}%
\pgfsetlinewidth{0.602250pt}%
\definecolor{currentstroke}{rgb}{0.000000,0.000000,0.000000}%
\pgfsetstrokecolor{currentstroke}%
\pgfsetdash{}{0pt}%
\pgfsys@defobject{currentmarker}{\pgfqpoint{0.000000in}{-0.027778in}}{\pgfqpoint{0.000000in}{0.000000in}}{%
\pgfpathmoveto{\pgfqpoint{0.000000in}{0.000000in}}%
\pgfpathlineto{\pgfqpoint{0.000000in}{-0.027778in}}%
\pgfusepath{stroke,fill}%
}%
\begin{pgfscope}%
\pgfsys@transformshift{0.839681in}{0.549073in}%
\pgfsys@useobject{currentmarker}{}%
\end{pgfscope}%
\end{pgfscope}%
\begin{pgfscope}%
\pgfsetbuttcap%
\pgfsetroundjoin%
\definecolor{currentfill}{rgb}{0.000000,0.000000,0.000000}%
\pgfsetfillcolor{currentfill}%
\pgfsetlinewidth{0.602250pt}%
\definecolor{currentstroke}{rgb}{0.000000,0.000000,0.000000}%
\pgfsetstrokecolor{currentstroke}%
\pgfsetdash{}{0pt}%
\pgfsys@defobject{currentmarker}{\pgfqpoint{0.000000in}{-0.027778in}}{\pgfqpoint{0.000000in}{0.000000in}}{%
\pgfpathmoveto{\pgfqpoint{0.000000in}{0.000000in}}%
\pgfpathlineto{\pgfqpoint{0.000000in}{-0.027778in}}%
\pgfusepath{stroke,fill}%
}%
\begin{pgfscope}%
\pgfsys@transformshift{1.074450in}{0.549073in}%
\pgfsys@useobject{currentmarker}{}%
\end{pgfscope}%
\end{pgfscope}%
\begin{pgfscope}%
\pgfsetbuttcap%
\pgfsetroundjoin%
\definecolor{currentfill}{rgb}{0.000000,0.000000,0.000000}%
\pgfsetfillcolor{currentfill}%
\pgfsetlinewidth{0.602250pt}%
\definecolor{currentstroke}{rgb}{0.000000,0.000000,0.000000}%
\pgfsetstrokecolor{currentstroke}%
\pgfsetdash{}{0pt}%
\pgfsys@defobject{currentmarker}{\pgfqpoint{0.000000in}{-0.027778in}}{\pgfqpoint{0.000000in}{0.000000in}}{%
\pgfpathmoveto{\pgfqpoint{0.000000in}{0.000000in}}%
\pgfpathlineto{\pgfqpoint{0.000000in}{-0.027778in}}%
\pgfusepath{stroke,fill}%
}%
\begin{pgfscope}%
\pgfsys@transformshift{1.241022in}{0.549073in}%
\pgfsys@useobject{currentmarker}{}%
\end{pgfscope}%
\end{pgfscope}%
\begin{pgfscope}%
\pgfsetbuttcap%
\pgfsetroundjoin%
\definecolor{currentfill}{rgb}{0.000000,0.000000,0.000000}%
\pgfsetfillcolor{currentfill}%
\pgfsetlinewidth{0.602250pt}%
\definecolor{currentstroke}{rgb}{0.000000,0.000000,0.000000}%
\pgfsetstrokecolor{currentstroke}%
\pgfsetdash{}{0pt}%
\pgfsys@defobject{currentmarker}{\pgfqpoint{0.000000in}{-0.027778in}}{\pgfqpoint{0.000000in}{0.000000in}}{%
\pgfpathmoveto{\pgfqpoint{0.000000in}{0.000000in}}%
\pgfpathlineto{\pgfqpoint{0.000000in}{-0.027778in}}%
\pgfusepath{stroke,fill}%
}%
\begin{pgfscope}%
\pgfsys@transformshift{1.370225in}{0.549073in}%
\pgfsys@useobject{currentmarker}{}%
\end{pgfscope}%
\end{pgfscope}%
\begin{pgfscope}%
\pgfsetbuttcap%
\pgfsetroundjoin%
\definecolor{currentfill}{rgb}{0.000000,0.000000,0.000000}%
\pgfsetfillcolor{currentfill}%
\pgfsetlinewidth{0.602250pt}%
\definecolor{currentstroke}{rgb}{0.000000,0.000000,0.000000}%
\pgfsetstrokecolor{currentstroke}%
\pgfsetdash{}{0pt}%
\pgfsys@defobject{currentmarker}{\pgfqpoint{0.000000in}{-0.027778in}}{\pgfqpoint{0.000000in}{0.000000in}}{%
\pgfpathmoveto{\pgfqpoint{0.000000in}{0.000000in}}%
\pgfpathlineto{\pgfqpoint{0.000000in}{-0.027778in}}%
\pgfusepath{stroke,fill}%
}%
\begin{pgfscope}%
\pgfsys@transformshift{1.475791in}{0.549073in}%
\pgfsys@useobject{currentmarker}{}%
\end{pgfscope}%
\end{pgfscope}%
\begin{pgfscope}%
\pgfsetbuttcap%
\pgfsetroundjoin%
\definecolor{currentfill}{rgb}{0.000000,0.000000,0.000000}%
\pgfsetfillcolor{currentfill}%
\pgfsetlinewidth{0.602250pt}%
\definecolor{currentstroke}{rgb}{0.000000,0.000000,0.000000}%
\pgfsetstrokecolor{currentstroke}%
\pgfsetdash{}{0pt}%
\pgfsys@defobject{currentmarker}{\pgfqpoint{0.000000in}{-0.027778in}}{\pgfqpoint{0.000000in}{0.000000in}}{%
\pgfpathmoveto{\pgfqpoint{0.000000in}{0.000000in}}%
\pgfpathlineto{\pgfqpoint{0.000000in}{-0.027778in}}%
\pgfusepath{stroke,fill}%
}%
\begin{pgfscope}%
\pgfsys@transformshift{1.565047in}{0.549073in}%
\pgfsys@useobject{currentmarker}{}%
\end{pgfscope}%
\end{pgfscope}%
\begin{pgfscope}%
\pgfsetbuttcap%
\pgfsetroundjoin%
\definecolor{currentfill}{rgb}{0.000000,0.000000,0.000000}%
\pgfsetfillcolor{currentfill}%
\pgfsetlinewidth{0.602250pt}%
\definecolor{currentstroke}{rgb}{0.000000,0.000000,0.000000}%
\pgfsetstrokecolor{currentstroke}%
\pgfsetdash{}{0pt}%
\pgfsys@defobject{currentmarker}{\pgfqpoint{0.000000in}{-0.027778in}}{\pgfqpoint{0.000000in}{0.000000in}}{%
\pgfpathmoveto{\pgfqpoint{0.000000in}{0.000000in}}%
\pgfpathlineto{\pgfqpoint{0.000000in}{-0.027778in}}%
\pgfusepath{stroke,fill}%
}%
\begin{pgfscope}%
\pgfsys@transformshift{1.642363in}{0.549073in}%
\pgfsys@useobject{currentmarker}{}%
\end{pgfscope}%
\end{pgfscope}%
\begin{pgfscope}%
\pgfsetbuttcap%
\pgfsetroundjoin%
\definecolor{currentfill}{rgb}{0.000000,0.000000,0.000000}%
\pgfsetfillcolor{currentfill}%
\pgfsetlinewidth{0.602250pt}%
\definecolor{currentstroke}{rgb}{0.000000,0.000000,0.000000}%
\pgfsetstrokecolor{currentstroke}%
\pgfsetdash{}{0pt}%
\pgfsys@defobject{currentmarker}{\pgfqpoint{0.000000in}{-0.027778in}}{\pgfqpoint{0.000000in}{0.000000in}}{%
\pgfpathmoveto{\pgfqpoint{0.000000in}{0.000000in}}%
\pgfpathlineto{\pgfqpoint{0.000000in}{-0.027778in}}%
\pgfusepath{stroke,fill}%
}%
\begin{pgfscope}%
\pgfsys@transformshift{1.710561in}{0.549073in}%
\pgfsys@useobject{currentmarker}{}%
\end{pgfscope}%
\end{pgfscope}%
\begin{pgfscope}%
\pgfsetbuttcap%
\pgfsetroundjoin%
\definecolor{currentfill}{rgb}{0.000000,0.000000,0.000000}%
\pgfsetfillcolor{currentfill}%
\pgfsetlinewidth{0.602250pt}%
\definecolor{currentstroke}{rgb}{0.000000,0.000000,0.000000}%
\pgfsetstrokecolor{currentstroke}%
\pgfsetdash{}{0pt}%
\pgfsys@defobject{currentmarker}{\pgfqpoint{0.000000in}{-0.027778in}}{\pgfqpoint{0.000000in}{0.000000in}}{%
\pgfpathmoveto{\pgfqpoint{0.000000in}{0.000000in}}%
\pgfpathlineto{\pgfqpoint{0.000000in}{-0.027778in}}%
\pgfusepath{stroke,fill}%
}%
\begin{pgfscope}%
\pgfsys@transformshift{2.172907in}{0.549073in}%
\pgfsys@useobject{currentmarker}{}%
\end{pgfscope}%
\end{pgfscope}%
\begin{pgfscope}%
\pgfsetbuttcap%
\pgfsetroundjoin%
\definecolor{currentfill}{rgb}{0.000000,0.000000,0.000000}%
\pgfsetfillcolor{currentfill}%
\pgfsetlinewidth{0.602250pt}%
\definecolor{currentstroke}{rgb}{0.000000,0.000000,0.000000}%
\pgfsetstrokecolor{currentstroke}%
\pgfsetdash{}{0pt}%
\pgfsys@defobject{currentmarker}{\pgfqpoint{0.000000in}{-0.027778in}}{\pgfqpoint{0.000000in}{0.000000in}}{%
\pgfpathmoveto{\pgfqpoint{0.000000in}{0.000000in}}%
\pgfpathlineto{\pgfqpoint{0.000000in}{-0.027778in}}%
\pgfusepath{stroke,fill}%
}%
\begin{pgfscope}%
\pgfsys@transformshift{2.407676in}{0.549073in}%
\pgfsys@useobject{currentmarker}{}%
\end{pgfscope}%
\end{pgfscope}%
\begin{pgfscope}%
\pgfsetbuttcap%
\pgfsetroundjoin%
\definecolor{currentfill}{rgb}{0.000000,0.000000,0.000000}%
\pgfsetfillcolor{currentfill}%
\pgfsetlinewidth{0.602250pt}%
\definecolor{currentstroke}{rgb}{0.000000,0.000000,0.000000}%
\pgfsetstrokecolor{currentstroke}%
\pgfsetdash{}{0pt}%
\pgfsys@defobject{currentmarker}{\pgfqpoint{0.000000in}{-0.027778in}}{\pgfqpoint{0.000000in}{0.000000in}}{%
\pgfpathmoveto{\pgfqpoint{0.000000in}{0.000000in}}%
\pgfpathlineto{\pgfqpoint{0.000000in}{-0.027778in}}%
\pgfusepath{stroke,fill}%
}%
\begin{pgfscope}%
\pgfsys@transformshift{2.574248in}{0.549073in}%
\pgfsys@useobject{currentmarker}{}%
\end{pgfscope}%
\end{pgfscope}%
\begin{pgfscope}%
\definecolor{textcolor}{rgb}{0.000000,0.000000,0.000000}%
\pgfsetstrokecolor{textcolor}%
\pgfsetfillcolor{textcolor}%
\pgftext[x=1.690663in,y=0.248148in,,top]{\color{textcolor}{\rmfamily\fontsize{12.000000}{14.400000}\selectfont\catcode`\^=\active\def^{\ifmmode\sp\else\^{}\fi}\catcode`\%=\active\def%{\%}$n_{\Omega} + n_{\Psi}$}}%
\end{pgfscope}%
\begin{pgfscope}%
\pgfsetbuttcap%
\pgfsetroundjoin%
\definecolor{currentfill}{rgb}{0.000000,0.000000,0.000000}%
\pgfsetfillcolor{currentfill}%
\pgfsetlinewidth{0.803000pt}%
\definecolor{currentstroke}{rgb}{0.000000,0.000000,0.000000}%
\pgfsetstrokecolor{currentstroke}%
\pgfsetdash{}{0pt}%
\pgfsys@defobject{currentmarker}{\pgfqpoint{-0.048611in}{0.000000in}}{\pgfqpoint{-0.000000in}{0.000000in}}{%
\pgfpathmoveto{\pgfqpoint{-0.000000in}{0.000000in}}%
\pgfpathlineto{\pgfqpoint{-0.048611in}{0.000000in}}%
\pgfusepath{stroke,fill}%
}%
\begin{pgfscope}%
\pgfsys@transformshift{0.721913in}{1.107737in}%
\pgfsys@useobject{currentmarker}{}%
\end{pgfscope}%
\end{pgfscope}%
\begin{pgfscope}%
\definecolor{textcolor}{rgb}{0.000000,0.000000,0.000000}%
\pgfsetstrokecolor{textcolor}%
\pgfsetfillcolor{textcolor}%
\pgftext[x=0.303703in, y=1.049867in, left, base]{\color{textcolor}{\rmfamily\fontsize{12.000000}{14.400000}\selectfont\catcode`\^=\active\def^{\ifmmode\sp\else\^{}\fi}\catcode`\%=\active\def%{\%}$\mathdefault{10^{-2}}$}}%
\end{pgfscope}%
\begin{pgfscope}%
\pgfsetbuttcap%
\pgfsetroundjoin%
\definecolor{currentfill}{rgb}{0.000000,0.000000,0.000000}%
\pgfsetfillcolor{currentfill}%
\pgfsetlinewidth{0.803000pt}%
\definecolor{currentstroke}{rgb}{0.000000,0.000000,0.000000}%
\pgfsetstrokecolor{currentstroke}%
\pgfsetdash{}{0pt}%
\pgfsys@defobject{currentmarker}{\pgfqpoint{-0.048611in}{0.000000in}}{\pgfqpoint{-0.000000in}{0.000000in}}{%
\pgfpathmoveto{\pgfqpoint{-0.000000in}{0.000000in}}%
\pgfpathlineto{\pgfqpoint{-0.048611in}{0.000000in}}%
\pgfusepath{stroke,fill}%
}%
\begin{pgfscope}%
\pgfsys@transformshift{0.721913in}{1.753458in}%
\pgfsys@useobject{currentmarker}{}%
\end{pgfscope}%
\end{pgfscope}%
\begin{pgfscope}%
\definecolor{textcolor}{rgb}{0.000000,0.000000,0.000000}%
\pgfsetstrokecolor{textcolor}%
\pgfsetfillcolor{textcolor}%
\pgftext[x=0.303703in, y=1.695588in, left, base]{\color{textcolor}{\rmfamily\fontsize{12.000000}{14.400000}\selectfont\catcode`\^=\active\def^{\ifmmode\sp\else\^{}\fi}\catcode`\%=\active\def%{\%}$\mathdefault{10^{-1}}$}}%
\end{pgfscope}%
\begin{pgfscope}%
\pgfsetbuttcap%
\pgfsetroundjoin%
\definecolor{currentfill}{rgb}{0.000000,0.000000,0.000000}%
\pgfsetfillcolor{currentfill}%
\pgfsetlinewidth{0.803000pt}%
\definecolor{currentstroke}{rgb}{0.000000,0.000000,0.000000}%
\pgfsetstrokecolor{currentstroke}%
\pgfsetdash{}{0pt}%
\pgfsys@defobject{currentmarker}{\pgfqpoint{-0.048611in}{0.000000in}}{\pgfqpoint{-0.000000in}{0.000000in}}{%
\pgfpathmoveto{\pgfqpoint{-0.000000in}{0.000000in}}%
\pgfpathlineto{\pgfqpoint{-0.048611in}{0.000000in}}%
\pgfusepath{stroke,fill}%
}%
\begin{pgfscope}%
\pgfsys@transformshift{0.721913in}{2.399179in}%
\pgfsys@useobject{currentmarker}{}%
\end{pgfscope}%
\end{pgfscope}%
\begin{pgfscope}%
\definecolor{textcolor}{rgb}{0.000000,0.000000,0.000000}%
\pgfsetstrokecolor{textcolor}%
\pgfsetfillcolor{textcolor}%
\pgftext[x=0.395525in, y=2.341309in, left, base]{\color{textcolor}{\rmfamily\fontsize{12.000000}{14.400000}\selectfont\catcode`\^=\active\def^{\ifmmode\sp\else\^{}\fi}\catcode`\%=\active\def%{\%}$\mathdefault{10^{0}}$}}%
\end{pgfscope}%
\begin{pgfscope}%
\pgfsetbuttcap%
\pgfsetroundjoin%
\definecolor{currentfill}{rgb}{0.000000,0.000000,0.000000}%
\pgfsetfillcolor{currentfill}%
\pgfsetlinewidth{0.602250pt}%
\definecolor{currentstroke}{rgb}{0.000000,0.000000,0.000000}%
\pgfsetstrokecolor{currentstroke}%
\pgfsetdash{}{0pt}%
\pgfsys@defobject{currentmarker}{\pgfqpoint{-0.027778in}{0.000000in}}{\pgfqpoint{-0.000000in}{0.000000in}}{%
\pgfpathmoveto{\pgfqpoint{-0.000000in}{0.000000in}}%
\pgfpathlineto{\pgfqpoint{-0.027778in}{0.000000in}}%
\pgfusepath{stroke,fill}%
}%
\begin{pgfscope}%
\pgfsys@transformshift{0.721913in}{0.656397in}%
\pgfsys@useobject{currentmarker}{}%
\end{pgfscope}%
\end{pgfscope}%
\begin{pgfscope}%
\pgfsetbuttcap%
\pgfsetroundjoin%
\definecolor{currentfill}{rgb}{0.000000,0.000000,0.000000}%
\pgfsetfillcolor{currentfill}%
\pgfsetlinewidth{0.602250pt}%
\definecolor{currentstroke}{rgb}{0.000000,0.000000,0.000000}%
\pgfsetstrokecolor{currentstroke}%
\pgfsetdash{}{0pt}%
\pgfsys@defobject{currentmarker}{\pgfqpoint{-0.027778in}{0.000000in}}{\pgfqpoint{-0.000000in}{0.000000in}}{%
\pgfpathmoveto{\pgfqpoint{-0.000000in}{0.000000in}}%
\pgfpathlineto{\pgfqpoint{-0.027778in}{0.000000in}}%
\pgfusepath{stroke,fill}%
}%
\begin{pgfscope}%
\pgfsys@transformshift{0.721913in}{0.770103in}%
\pgfsys@useobject{currentmarker}{}%
\end{pgfscope}%
\end{pgfscope}%
\begin{pgfscope}%
\pgfsetbuttcap%
\pgfsetroundjoin%
\definecolor{currentfill}{rgb}{0.000000,0.000000,0.000000}%
\pgfsetfillcolor{currentfill}%
\pgfsetlinewidth{0.602250pt}%
\definecolor{currentstroke}{rgb}{0.000000,0.000000,0.000000}%
\pgfsetstrokecolor{currentstroke}%
\pgfsetdash{}{0pt}%
\pgfsys@defobject{currentmarker}{\pgfqpoint{-0.027778in}{0.000000in}}{\pgfqpoint{-0.000000in}{0.000000in}}{%
\pgfpathmoveto{\pgfqpoint{-0.000000in}{0.000000in}}%
\pgfpathlineto{\pgfqpoint{-0.027778in}{0.000000in}}%
\pgfusepath{stroke,fill}%
}%
\begin{pgfscope}%
\pgfsys@transformshift{0.721913in}{0.850779in}%
\pgfsys@useobject{currentmarker}{}%
\end{pgfscope}%
\end{pgfscope}%
\begin{pgfscope}%
\pgfsetbuttcap%
\pgfsetroundjoin%
\definecolor{currentfill}{rgb}{0.000000,0.000000,0.000000}%
\pgfsetfillcolor{currentfill}%
\pgfsetlinewidth{0.602250pt}%
\definecolor{currentstroke}{rgb}{0.000000,0.000000,0.000000}%
\pgfsetstrokecolor{currentstroke}%
\pgfsetdash{}{0pt}%
\pgfsys@defobject{currentmarker}{\pgfqpoint{-0.027778in}{0.000000in}}{\pgfqpoint{-0.000000in}{0.000000in}}{%
\pgfpathmoveto{\pgfqpoint{-0.000000in}{0.000000in}}%
\pgfpathlineto{\pgfqpoint{-0.027778in}{0.000000in}}%
\pgfusepath{stroke,fill}%
}%
\begin{pgfscope}%
\pgfsys@transformshift{0.721913in}{0.913356in}%
\pgfsys@useobject{currentmarker}{}%
\end{pgfscope}%
\end{pgfscope}%
\begin{pgfscope}%
\pgfsetbuttcap%
\pgfsetroundjoin%
\definecolor{currentfill}{rgb}{0.000000,0.000000,0.000000}%
\pgfsetfillcolor{currentfill}%
\pgfsetlinewidth{0.602250pt}%
\definecolor{currentstroke}{rgb}{0.000000,0.000000,0.000000}%
\pgfsetstrokecolor{currentstroke}%
\pgfsetdash{}{0pt}%
\pgfsys@defobject{currentmarker}{\pgfqpoint{-0.027778in}{0.000000in}}{\pgfqpoint{-0.000000in}{0.000000in}}{%
\pgfpathmoveto{\pgfqpoint{-0.000000in}{0.000000in}}%
\pgfpathlineto{\pgfqpoint{-0.027778in}{0.000000in}}%
\pgfusepath{stroke,fill}%
}%
\begin{pgfscope}%
\pgfsys@transformshift{0.721913in}{0.964485in}%
\pgfsys@useobject{currentmarker}{}%
\end{pgfscope}%
\end{pgfscope}%
\begin{pgfscope}%
\pgfsetbuttcap%
\pgfsetroundjoin%
\definecolor{currentfill}{rgb}{0.000000,0.000000,0.000000}%
\pgfsetfillcolor{currentfill}%
\pgfsetlinewidth{0.602250pt}%
\definecolor{currentstroke}{rgb}{0.000000,0.000000,0.000000}%
\pgfsetstrokecolor{currentstroke}%
\pgfsetdash{}{0pt}%
\pgfsys@defobject{currentmarker}{\pgfqpoint{-0.027778in}{0.000000in}}{\pgfqpoint{-0.000000in}{0.000000in}}{%
\pgfpathmoveto{\pgfqpoint{-0.000000in}{0.000000in}}%
\pgfpathlineto{\pgfqpoint{-0.027778in}{0.000000in}}%
\pgfusepath{stroke,fill}%
}%
\begin{pgfscope}%
\pgfsys@transformshift{0.721913in}{1.007714in}%
\pgfsys@useobject{currentmarker}{}%
\end{pgfscope}%
\end{pgfscope}%
\begin{pgfscope}%
\pgfsetbuttcap%
\pgfsetroundjoin%
\definecolor{currentfill}{rgb}{0.000000,0.000000,0.000000}%
\pgfsetfillcolor{currentfill}%
\pgfsetlinewidth{0.602250pt}%
\definecolor{currentstroke}{rgb}{0.000000,0.000000,0.000000}%
\pgfsetstrokecolor{currentstroke}%
\pgfsetdash{}{0pt}%
\pgfsys@defobject{currentmarker}{\pgfqpoint{-0.027778in}{0.000000in}}{\pgfqpoint{-0.000000in}{0.000000in}}{%
\pgfpathmoveto{\pgfqpoint{-0.000000in}{0.000000in}}%
\pgfpathlineto{\pgfqpoint{-0.027778in}{0.000000in}}%
\pgfusepath{stroke,fill}%
}%
\begin{pgfscope}%
\pgfsys@transformshift{0.721913in}{1.045160in}%
\pgfsys@useobject{currentmarker}{}%
\end{pgfscope}%
\end{pgfscope}%
\begin{pgfscope}%
\pgfsetbuttcap%
\pgfsetroundjoin%
\definecolor{currentfill}{rgb}{0.000000,0.000000,0.000000}%
\pgfsetfillcolor{currentfill}%
\pgfsetlinewidth{0.602250pt}%
\definecolor{currentstroke}{rgb}{0.000000,0.000000,0.000000}%
\pgfsetstrokecolor{currentstroke}%
\pgfsetdash{}{0pt}%
\pgfsys@defobject{currentmarker}{\pgfqpoint{-0.027778in}{0.000000in}}{\pgfqpoint{-0.000000in}{0.000000in}}{%
\pgfpathmoveto{\pgfqpoint{-0.000000in}{0.000000in}}%
\pgfpathlineto{\pgfqpoint{-0.027778in}{0.000000in}}%
\pgfusepath{stroke,fill}%
}%
\begin{pgfscope}%
\pgfsys@transformshift{0.721913in}{1.078191in}%
\pgfsys@useobject{currentmarker}{}%
\end{pgfscope}%
\end{pgfscope}%
\begin{pgfscope}%
\pgfsetbuttcap%
\pgfsetroundjoin%
\definecolor{currentfill}{rgb}{0.000000,0.000000,0.000000}%
\pgfsetfillcolor{currentfill}%
\pgfsetlinewidth{0.602250pt}%
\definecolor{currentstroke}{rgb}{0.000000,0.000000,0.000000}%
\pgfsetstrokecolor{currentstroke}%
\pgfsetdash{}{0pt}%
\pgfsys@defobject{currentmarker}{\pgfqpoint{-0.027778in}{0.000000in}}{\pgfqpoint{-0.000000in}{0.000000in}}{%
\pgfpathmoveto{\pgfqpoint{-0.000000in}{0.000000in}}%
\pgfpathlineto{\pgfqpoint{-0.027778in}{0.000000in}}%
\pgfusepath{stroke,fill}%
}%
\begin{pgfscope}%
\pgfsys@transformshift{0.721913in}{1.302119in}%
\pgfsys@useobject{currentmarker}{}%
\end{pgfscope}%
\end{pgfscope}%
\begin{pgfscope}%
\pgfsetbuttcap%
\pgfsetroundjoin%
\definecolor{currentfill}{rgb}{0.000000,0.000000,0.000000}%
\pgfsetfillcolor{currentfill}%
\pgfsetlinewidth{0.602250pt}%
\definecolor{currentstroke}{rgb}{0.000000,0.000000,0.000000}%
\pgfsetstrokecolor{currentstroke}%
\pgfsetdash{}{0pt}%
\pgfsys@defobject{currentmarker}{\pgfqpoint{-0.027778in}{0.000000in}}{\pgfqpoint{-0.000000in}{0.000000in}}{%
\pgfpathmoveto{\pgfqpoint{-0.000000in}{0.000000in}}%
\pgfpathlineto{\pgfqpoint{-0.027778in}{0.000000in}}%
\pgfusepath{stroke,fill}%
}%
\begin{pgfscope}%
\pgfsys@transformshift{0.721913in}{1.415824in}%
\pgfsys@useobject{currentmarker}{}%
\end{pgfscope}%
\end{pgfscope}%
\begin{pgfscope}%
\pgfsetbuttcap%
\pgfsetroundjoin%
\definecolor{currentfill}{rgb}{0.000000,0.000000,0.000000}%
\pgfsetfillcolor{currentfill}%
\pgfsetlinewidth{0.602250pt}%
\definecolor{currentstroke}{rgb}{0.000000,0.000000,0.000000}%
\pgfsetstrokecolor{currentstroke}%
\pgfsetdash{}{0pt}%
\pgfsys@defobject{currentmarker}{\pgfqpoint{-0.027778in}{0.000000in}}{\pgfqpoint{-0.000000in}{0.000000in}}{%
\pgfpathmoveto{\pgfqpoint{-0.000000in}{0.000000in}}%
\pgfpathlineto{\pgfqpoint{-0.027778in}{0.000000in}}%
\pgfusepath{stroke,fill}%
}%
\begin{pgfscope}%
\pgfsys@transformshift{0.721913in}{1.496500in}%
\pgfsys@useobject{currentmarker}{}%
\end{pgfscope}%
\end{pgfscope}%
\begin{pgfscope}%
\pgfsetbuttcap%
\pgfsetroundjoin%
\definecolor{currentfill}{rgb}{0.000000,0.000000,0.000000}%
\pgfsetfillcolor{currentfill}%
\pgfsetlinewidth{0.602250pt}%
\definecolor{currentstroke}{rgb}{0.000000,0.000000,0.000000}%
\pgfsetstrokecolor{currentstroke}%
\pgfsetdash{}{0pt}%
\pgfsys@defobject{currentmarker}{\pgfqpoint{-0.027778in}{0.000000in}}{\pgfqpoint{-0.000000in}{0.000000in}}{%
\pgfpathmoveto{\pgfqpoint{-0.000000in}{0.000000in}}%
\pgfpathlineto{\pgfqpoint{-0.027778in}{0.000000in}}%
\pgfusepath{stroke,fill}%
}%
\begin{pgfscope}%
\pgfsys@transformshift{0.721913in}{1.559077in}%
\pgfsys@useobject{currentmarker}{}%
\end{pgfscope}%
\end{pgfscope}%
\begin{pgfscope}%
\pgfsetbuttcap%
\pgfsetroundjoin%
\definecolor{currentfill}{rgb}{0.000000,0.000000,0.000000}%
\pgfsetfillcolor{currentfill}%
\pgfsetlinewidth{0.602250pt}%
\definecolor{currentstroke}{rgb}{0.000000,0.000000,0.000000}%
\pgfsetstrokecolor{currentstroke}%
\pgfsetdash{}{0pt}%
\pgfsys@defobject{currentmarker}{\pgfqpoint{-0.027778in}{0.000000in}}{\pgfqpoint{-0.000000in}{0.000000in}}{%
\pgfpathmoveto{\pgfqpoint{-0.000000in}{0.000000in}}%
\pgfpathlineto{\pgfqpoint{-0.027778in}{0.000000in}}%
\pgfusepath{stroke,fill}%
}%
\begin{pgfscope}%
\pgfsys@transformshift{0.721913in}{1.610206in}%
\pgfsys@useobject{currentmarker}{}%
\end{pgfscope}%
\end{pgfscope}%
\begin{pgfscope}%
\pgfsetbuttcap%
\pgfsetroundjoin%
\definecolor{currentfill}{rgb}{0.000000,0.000000,0.000000}%
\pgfsetfillcolor{currentfill}%
\pgfsetlinewidth{0.602250pt}%
\definecolor{currentstroke}{rgb}{0.000000,0.000000,0.000000}%
\pgfsetstrokecolor{currentstroke}%
\pgfsetdash{}{0pt}%
\pgfsys@defobject{currentmarker}{\pgfqpoint{-0.027778in}{0.000000in}}{\pgfqpoint{-0.000000in}{0.000000in}}{%
\pgfpathmoveto{\pgfqpoint{-0.000000in}{0.000000in}}%
\pgfpathlineto{\pgfqpoint{-0.027778in}{0.000000in}}%
\pgfusepath{stroke,fill}%
}%
\begin{pgfscope}%
\pgfsys@transformshift{0.721913in}{1.653435in}%
\pgfsys@useobject{currentmarker}{}%
\end{pgfscope}%
\end{pgfscope}%
\begin{pgfscope}%
\pgfsetbuttcap%
\pgfsetroundjoin%
\definecolor{currentfill}{rgb}{0.000000,0.000000,0.000000}%
\pgfsetfillcolor{currentfill}%
\pgfsetlinewidth{0.602250pt}%
\definecolor{currentstroke}{rgb}{0.000000,0.000000,0.000000}%
\pgfsetstrokecolor{currentstroke}%
\pgfsetdash{}{0pt}%
\pgfsys@defobject{currentmarker}{\pgfqpoint{-0.027778in}{0.000000in}}{\pgfqpoint{-0.000000in}{0.000000in}}{%
\pgfpathmoveto{\pgfqpoint{-0.000000in}{0.000000in}}%
\pgfpathlineto{\pgfqpoint{-0.027778in}{0.000000in}}%
\pgfusepath{stroke,fill}%
}%
\begin{pgfscope}%
\pgfsys@transformshift{0.721913in}{1.690881in}%
\pgfsys@useobject{currentmarker}{}%
\end{pgfscope}%
\end{pgfscope}%
\begin{pgfscope}%
\pgfsetbuttcap%
\pgfsetroundjoin%
\definecolor{currentfill}{rgb}{0.000000,0.000000,0.000000}%
\pgfsetfillcolor{currentfill}%
\pgfsetlinewidth{0.602250pt}%
\definecolor{currentstroke}{rgb}{0.000000,0.000000,0.000000}%
\pgfsetstrokecolor{currentstroke}%
\pgfsetdash{}{0pt}%
\pgfsys@defobject{currentmarker}{\pgfqpoint{-0.027778in}{0.000000in}}{\pgfqpoint{-0.000000in}{0.000000in}}{%
\pgfpathmoveto{\pgfqpoint{-0.000000in}{0.000000in}}%
\pgfpathlineto{\pgfqpoint{-0.027778in}{0.000000in}}%
\pgfusepath{stroke,fill}%
}%
\begin{pgfscope}%
\pgfsys@transformshift{0.721913in}{1.723912in}%
\pgfsys@useobject{currentmarker}{}%
\end{pgfscope}%
\end{pgfscope}%
\begin{pgfscope}%
\pgfsetbuttcap%
\pgfsetroundjoin%
\definecolor{currentfill}{rgb}{0.000000,0.000000,0.000000}%
\pgfsetfillcolor{currentfill}%
\pgfsetlinewidth{0.602250pt}%
\definecolor{currentstroke}{rgb}{0.000000,0.000000,0.000000}%
\pgfsetstrokecolor{currentstroke}%
\pgfsetdash{}{0pt}%
\pgfsys@defobject{currentmarker}{\pgfqpoint{-0.027778in}{0.000000in}}{\pgfqpoint{-0.000000in}{0.000000in}}{%
\pgfpathmoveto{\pgfqpoint{-0.000000in}{0.000000in}}%
\pgfpathlineto{\pgfqpoint{-0.027778in}{0.000000in}}%
\pgfusepath{stroke,fill}%
}%
\begin{pgfscope}%
\pgfsys@transformshift{0.721913in}{1.947840in}%
\pgfsys@useobject{currentmarker}{}%
\end{pgfscope}%
\end{pgfscope}%
\begin{pgfscope}%
\pgfsetbuttcap%
\pgfsetroundjoin%
\definecolor{currentfill}{rgb}{0.000000,0.000000,0.000000}%
\pgfsetfillcolor{currentfill}%
\pgfsetlinewidth{0.602250pt}%
\definecolor{currentstroke}{rgb}{0.000000,0.000000,0.000000}%
\pgfsetstrokecolor{currentstroke}%
\pgfsetdash{}{0pt}%
\pgfsys@defobject{currentmarker}{\pgfqpoint{-0.027778in}{0.000000in}}{\pgfqpoint{-0.000000in}{0.000000in}}{%
\pgfpathmoveto{\pgfqpoint{-0.000000in}{0.000000in}}%
\pgfpathlineto{\pgfqpoint{-0.027778in}{0.000000in}}%
\pgfusepath{stroke,fill}%
}%
\begin{pgfscope}%
\pgfsys@transformshift{0.721913in}{2.061545in}%
\pgfsys@useobject{currentmarker}{}%
\end{pgfscope}%
\end{pgfscope}%
\begin{pgfscope}%
\pgfsetbuttcap%
\pgfsetroundjoin%
\definecolor{currentfill}{rgb}{0.000000,0.000000,0.000000}%
\pgfsetfillcolor{currentfill}%
\pgfsetlinewidth{0.602250pt}%
\definecolor{currentstroke}{rgb}{0.000000,0.000000,0.000000}%
\pgfsetstrokecolor{currentstroke}%
\pgfsetdash{}{0pt}%
\pgfsys@defobject{currentmarker}{\pgfqpoint{-0.027778in}{0.000000in}}{\pgfqpoint{-0.000000in}{0.000000in}}{%
\pgfpathmoveto{\pgfqpoint{-0.000000in}{0.000000in}}%
\pgfpathlineto{\pgfqpoint{-0.027778in}{0.000000in}}%
\pgfusepath{stroke,fill}%
}%
\begin{pgfscope}%
\pgfsys@transformshift{0.721913in}{2.142221in}%
\pgfsys@useobject{currentmarker}{}%
\end{pgfscope}%
\end{pgfscope}%
\begin{pgfscope}%
\pgfsetbuttcap%
\pgfsetroundjoin%
\definecolor{currentfill}{rgb}{0.000000,0.000000,0.000000}%
\pgfsetfillcolor{currentfill}%
\pgfsetlinewidth{0.602250pt}%
\definecolor{currentstroke}{rgb}{0.000000,0.000000,0.000000}%
\pgfsetstrokecolor{currentstroke}%
\pgfsetdash{}{0pt}%
\pgfsys@defobject{currentmarker}{\pgfqpoint{-0.027778in}{0.000000in}}{\pgfqpoint{-0.000000in}{0.000000in}}{%
\pgfpathmoveto{\pgfqpoint{-0.000000in}{0.000000in}}%
\pgfpathlineto{\pgfqpoint{-0.027778in}{0.000000in}}%
\pgfusepath{stroke,fill}%
}%
\begin{pgfscope}%
\pgfsys@transformshift{0.721913in}{2.204798in}%
\pgfsys@useobject{currentmarker}{}%
\end{pgfscope}%
\end{pgfscope}%
\begin{pgfscope}%
\pgfsetbuttcap%
\pgfsetroundjoin%
\definecolor{currentfill}{rgb}{0.000000,0.000000,0.000000}%
\pgfsetfillcolor{currentfill}%
\pgfsetlinewidth{0.602250pt}%
\definecolor{currentstroke}{rgb}{0.000000,0.000000,0.000000}%
\pgfsetstrokecolor{currentstroke}%
\pgfsetdash{}{0pt}%
\pgfsys@defobject{currentmarker}{\pgfqpoint{-0.027778in}{0.000000in}}{\pgfqpoint{-0.000000in}{0.000000in}}{%
\pgfpathmoveto{\pgfqpoint{-0.000000in}{0.000000in}}%
\pgfpathlineto{\pgfqpoint{-0.027778in}{0.000000in}}%
\pgfusepath{stroke,fill}%
}%
\begin{pgfscope}%
\pgfsys@transformshift{0.721913in}{2.255927in}%
\pgfsys@useobject{currentmarker}{}%
\end{pgfscope}%
\end{pgfscope}%
\begin{pgfscope}%
\pgfsetbuttcap%
\pgfsetroundjoin%
\definecolor{currentfill}{rgb}{0.000000,0.000000,0.000000}%
\pgfsetfillcolor{currentfill}%
\pgfsetlinewidth{0.602250pt}%
\definecolor{currentstroke}{rgb}{0.000000,0.000000,0.000000}%
\pgfsetstrokecolor{currentstroke}%
\pgfsetdash{}{0pt}%
\pgfsys@defobject{currentmarker}{\pgfqpoint{-0.027778in}{0.000000in}}{\pgfqpoint{-0.000000in}{0.000000in}}{%
\pgfpathmoveto{\pgfqpoint{-0.000000in}{0.000000in}}%
\pgfpathlineto{\pgfqpoint{-0.027778in}{0.000000in}}%
\pgfusepath{stroke,fill}%
}%
\begin{pgfscope}%
\pgfsys@transformshift{0.721913in}{2.299156in}%
\pgfsys@useobject{currentmarker}{}%
\end{pgfscope}%
\end{pgfscope}%
\begin{pgfscope}%
\pgfsetbuttcap%
\pgfsetroundjoin%
\definecolor{currentfill}{rgb}{0.000000,0.000000,0.000000}%
\pgfsetfillcolor{currentfill}%
\pgfsetlinewidth{0.602250pt}%
\definecolor{currentstroke}{rgb}{0.000000,0.000000,0.000000}%
\pgfsetstrokecolor{currentstroke}%
\pgfsetdash{}{0pt}%
\pgfsys@defobject{currentmarker}{\pgfqpoint{-0.027778in}{0.000000in}}{\pgfqpoint{-0.000000in}{0.000000in}}{%
\pgfpathmoveto{\pgfqpoint{-0.000000in}{0.000000in}}%
\pgfpathlineto{\pgfqpoint{-0.027778in}{0.000000in}}%
\pgfusepath{stroke,fill}%
}%
\begin{pgfscope}%
\pgfsys@transformshift{0.721913in}{2.336602in}%
\pgfsys@useobject{currentmarker}{}%
\end{pgfscope}%
\end{pgfscope}%
\begin{pgfscope}%
\pgfsetbuttcap%
\pgfsetroundjoin%
\definecolor{currentfill}{rgb}{0.000000,0.000000,0.000000}%
\pgfsetfillcolor{currentfill}%
\pgfsetlinewidth{0.602250pt}%
\definecolor{currentstroke}{rgb}{0.000000,0.000000,0.000000}%
\pgfsetstrokecolor{currentstroke}%
\pgfsetdash{}{0pt}%
\pgfsys@defobject{currentmarker}{\pgfqpoint{-0.027778in}{0.000000in}}{\pgfqpoint{-0.000000in}{0.000000in}}{%
\pgfpathmoveto{\pgfqpoint{-0.000000in}{0.000000in}}%
\pgfpathlineto{\pgfqpoint{-0.027778in}{0.000000in}}%
\pgfusepath{stroke,fill}%
}%
\begin{pgfscope}%
\pgfsys@transformshift{0.721913in}{2.369633in}%
\pgfsys@useobject{currentmarker}{}%
\end{pgfscope}%
\end{pgfscope}%
\begin{pgfscope}%
\definecolor{textcolor}{rgb}{0.000000,0.000000,0.000000}%
\pgfsetstrokecolor{textcolor}%
\pgfsetfillcolor{textcolor}%
\pgftext[x=0.248148in,y=1.511573in,,bottom,rotate=90.000000]{\color{textcolor}{\rmfamily\fontsize{12.000000}{14.400000}\selectfont\catcode`\^=\active\def^{\ifmmode\sp\else\^{}\fi}\catcode`\%=\active\def%{\%}$L^1$ relative error}}%
\end{pgfscope}%
\begin{pgfscope}%
\pgfpathrectangle{\pgfqpoint{0.721913in}{0.549073in}}{\pgfqpoint{1.937500in}{1.925000in}}%
\pgfusepath{clip}%
\pgfsetrectcap%
\pgfsetroundjoin%
\pgfsetlinewidth{1.003750pt}%
\definecolor{currentstroke}{rgb}{0.001462,0.000466,0.013866}%
\pgfsetstrokecolor{currentstroke}%
\pgfsetdash{}{0pt}%
\pgfpathmoveto{\pgfqpoint{0.809982in}{1.070389in}}%
\pgfpathlineto{\pgfqpoint{1.180017in}{1.135422in}}%
\pgfpathlineto{\pgfqpoint{1.530977in}{0.928056in}}%
\pgfpathlineto{\pgfqpoint{1.877132in}{0.857019in}}%
\pgfpathlineto{\pgfqpoint{2.222805in}{0.821744in}}%
\pgfpathlineto{\pgfqpoint{2.571345in}{0.636573in}}%
\pgfusepath{stroke}%
\end{pgfscope}%
\begin{pgfscope}%
\pgfpathrectangle{\pgfqpoint{0.721913in}{0.549073in}}{\pgfqpoint{1.937500in}{1.925000in}}%
\pgfusepath{clip}%
\pgfsetbuttcap%
\pgfsetroundjoin%
\definecolor{currentfill}{rgb}{0.001462,0.000466,0.013866}%
\pgfsetfillcolor{currentfill}%
\pgfsetlinewidth{1.003750pt}%
\definecolor{currentstroke}{rgb}{0.001462,0.000466,0.013866}%
\pgfsetstrokecolor{currentstroke}%
\pgfsetdash{}{0pt}%
\pgfsys@defobject{currentmarker}{\pgfqpoint{-0.020833in}{-0.020833in}}{\pgfqpoint{0.020833in}{0.020833in}}{%
\pgfpathmoveto{\pgfqpoint{0.000000in}{-0.020833in}}%
\pgfpathcurveto{\pgfqpoint{0.005525in}{-0.020833in}}{\pgfqpoint{0.010825in}{-0.018638in}}{\pgfqpoint{0.014731in}{-0.014731in}}%
\pgfpathcurveto{\pgfqpoint{0.018638in}{-0.010825in}}{\pgfqpoint{0.020833in}{-0.005525in}}{\pgfqpoint{0.020833in}{0.000000in}}%
\pgfpathcurveto{\pgfqpoint{0.020833in}{0.005525in}}{\pgfqpoint{0.018638in}{0.010825in}}{\pgfqpoint{0.014731in}{0.014731in}}%
\pgfpathcurveto{\pgfqpoint{0.010825in}{0.018638in}}{\pgfqpoint{0.005525in}{0.020833in}}{\pgfqpoint{0.000000in}{0.020833in}}%
\pgfpathcurveto{\pgfqpoint{-0.005525in}{0.020833in}}{\pgfqpoint{-0.010825in}{0.018638in}}{\pgfqpoint{-0.014731in}{0.014731in}}%
\pgfpathcurveto{\pgfqpoint{-0.018638in}{0.010825in}}{\pgfqpoint{-0.020833in}{0.005525in}}{\pgfqpoint{-0.020833in}{0.000000in}}%
\pgfpathcurveto{\pgfqpoint{-0.020833in}{-0.005525in}}{\pgfqpoint{-0.018638in}{-0.010825in}}{\pgfqpoint{-0.014731in}{-0.014731in}}%
\pgfpathcurveto{\pgfqpoint{-0.010825in}{-0.018638in}}{\pgfqpoint{-0.005525in}{-0.020833in}}{\pgfqpoint{0.000000in}{-0.020833in}}%
\pgfpathlineto{\pgfqpoint{0.000000in}{-0.020833in}}%
\pgfpathclose%
\pgfusepath{stroke,fill}%
}%
\begin{pgfscope}%
\pgfsys@transformshift{0.809982in}{1.070389in}%
\pgfsys@useobject{currentmarker}{}%
\end{pgfscope}%
\begin{pgfscope}%
\pgfsys@transformshift{1.180017in}{1.135422in}%
\pgfsys@useobject{currentmarker}{}%
\end{pgfscope}%
\begin{pgfscope}%
\pgfsys@transformshift{1.530977in}{0.928056in}%
\pgfsys@useobject{currentmarker}{}%
\end{pgfscope}%
\begin{pgfscope}%
\pgfsys@transformshift{1.877132in}{0.857019in}%
\pgfsys@useobject{currentmarker}{}%
\end{pgfscope}%
\begin{pgfscope}%
\pgfsys@transformshift{2.222805in}{0.821744in}%
\pgfsys@useobject{currentmarker}{}%
\end{pgfscope}%
\begin{pgfscope}%
\pgfsys@transformshift{2.571345in}{0.636573in}%
\pgfsys@useobject{currentmarker}{}%
\end{pgfscope}%
\end{pgfscope}%
\begin{pgfscope}%
\pgfpathrectangle{\pgfqpoint{0.721913in}{0.549073in}}{\pgfqpoint{1.937500in}{1.925000in}}%
\pgfusepath{clip}%
\pgfsetrectcap%
\pgfsetroundjoin%
\pgfsetlinewidth{1.003750pt}%
\definecolor{currentstroke}{rgb}{0.445163,0.122724,0.506901}%
\pgfsetstrokecolor{currentstroke}%
\pgfsetdash{}{0pt}%
\pgfpathmoveto{\pgfqpoint{0.809982in}{2.386573in}}%
\pgfpathlineto{\pgfqpoint{1.180017in}{2.374943in}}%
\pgfpathlineto{\pgfqpoint{1.530977in}{2.353597in}}%
\pgfpathlineto{\pgfqpoint{1.877132in}{2.312264in}}%
\pgfpathlineto{\pgfqpoint{2.222805in}{2.225944in}}%
\pgfpathlineto{\pgfqpoint{2.571345in}{2.016371in}}%
\pgfusepath{stroke}%
\end{pgfscope}%
\begin{pgfscope}%
\pgfpathrectangle{\pgfqpoint{0.721913in}{0.549073in}}{\pgfqpoint{1.937500in}{1.925000in}}%
\pgfusepath{clip}%
\pgfsetbuttcap%
\pgfsetroundjoin%
\definecolor{currentfill}{rgb}{0.445163,0.122724,0.506901}%
\pgfsetfillcolor{currentfill}%
\pgfsetlinewidth{1.003750pt}%
\definecolor{currentstroke}{rgb}{0.445163,0.122724,0.506901}%
\pgfsetstrokecolor{currentstroke}%
\pgfsetdash{}{0pt}%
\pgfsys@defobject{currentmarker}{\pgfqpoint{-0.020833in}{-0.020833in}}{\pgfqpoint{0.020833in}{0.020833in}}{%
\pgfpathmoveto{\pgfqpoint{0.000000in}{-0.020833in}}%
\pgfpathcurveto{\pgfqpoint{0.005525in}{-0.020833in}}{\pgfqpoint{0.010825in}{-0.018638in}}{\pgfqpoint{0.014731in}{-0.014731in}}%
\pgfpathcurveto{\pgfqpoint{0.018638in}{-0.010825in}}{\pgfqpoint{0.020833in}{-0.005525in}}{\pgfqpoint{0.020833in}{0.000000in}}%
\pgfpathcurveto{\pgfqpoint{0.020833in}{0.005525in}}{\pgfqpoint{0.018638in}{0.010825in}}{\pgfqpoint{0.014731in}{0.014731in}}%
\pgfpathcurveto{\pgfqpoint{0.010825in}{0.018638in}}{\pgfqpoint{0.005525in}{0.020833in}}{\pgfqpoint{0.000000in}{0.020833in}}%
\pgfpathcurveto{\pgfqpoint{-0.005525in}{0.020833in}}{\pgfqpoint{-0.010825in}{0.018638in}}{\pgfqpoint{-0.014731in}{0.014731in}}%
\pgfpathcurveto{\pgfqpoint{-0.018638in}{0.010825in}}{\pgfqpoint{-0.020833in}{0.005525in}}{\pgfqpoint{-0.020833in}{0.000000in}}%
\pgfpathcurveto{\pgfqpoint{-0.020833in}{-0.005525in}}{\pgfqpoint{-0.018638in}{-0.010825in}}{\pgfqpoint{-0.014731in}{-0.014731in}}%
\pgfpathcurveto{\pgfqpoint{-0.010825in}{-0.018638in}}{\pgfqpoint{-0.005525in}{-0.020833in}}{\pgfqpoint{0.000000in}{-0.020833in}}%
\pgfpathlineto{\pgfqpoint{0.000000in}{-0.020833in}}%
\pgfpathclose%
\pgfusepath{stroke,fill}%
}%
\begin{pgfscope}%
\pgfsys@transformshift{0.809982in}{2.386573in}%
\pgfsys@useobject{currentmarker}{}%
\end{pgfscope}%
\begin{pgfscope}%
\pgfsys@transformshift{1.180017in}{2.374943in}%
\pgfsys@useobject{currentmarker}{}%
\end{pgfscope}%
\begin{pgfscope}%
\pgfsys@transformshift{1.530977in}{2.353597in}%
\pgfsys@useobject{currentmarker}{}%
\end{pgfscope}%
\begin{pgfscope}%
\pgfsys@transformshift{1.877132in}{2.312264in}%
\pgfsys@useobject{currentmarker}{}%
\end{pgfscope}%
\begin{pgfscope}%
\pgfsys@transformshift{2.222805in}{2.225944in}%
\pgfsys@useobject{currentmarker}{}%
\end{pgfscope}%
\begin{pgfscope}%
\pgfsys@transformshift{2.571345in}{2.016371in}%
\pgfsys@useobject{currentmarker}{}%
\end{pgfscope}%
\end{pgfscope}%
\begin{pgfscope}%
\pgfpathrectangle{\pgfqpoint{0.721913in}{0.549073in}}{\pgfqpoint{1.937500in}{1.925000in}}%
\pgfusepath{clip}%
\pgfsetrectcap%
\pgfsetroundjoin%
\pgfsetlinewidth{1.003750pt}%
\definecolor{currentstroke}{rgb}{0.944006,0.377643,0.365136}%
\pgfsetstrokecolor{currentstroke}%
\pgfsetdash{}{0pt}%
\pgfpathmoveto{\pgfqpoint{0.809982in}{1.265039in}}%
\pgfpathlineto{\pgfqpoint{1.180017in}{1.176798in}}%
\pgfpathlineto{\pgfqpoint{1.530977in}{1.030861in}}%
\pgfpathlineto{\pgfqpoint{1.877132in}{0.954357in}}%
\pgfpathlineto{\pgfqpoint{2.222805in}{0.863397in}}%
\pgfpathlineto{\pgfqpoint{2.571345in}{0.727314in}}%
\pgfusepath{stroke}%
\end{pgfscope}%
\begin{pgfscope}%
\pgfpathrectangle{\pgfqpoint{0.721913in}{0.549073in}}{\pgfqpoint{1.937500in}{1.925000in}}%
\pgfusepath{clip}%
\pgfsetbuttcap%
\pgfsetroundjoin%
\definecolor{currentfill}{rgb}{0.944006,0.377643,0.365136}%
\pgfsetfillcolor{currentfill}%
\pgfsetlinewidth{1.003750pt}%
\definecolor{currentstroke}{rgb}{0.944006,0.377643,0.365136}%
\pgfsetstrokecolor{currentstroke}%
\pgfsetdash{}{0pt}%
\pgfsys@defobject{currentmarker}{\pgfqpoint{-0.020833in}{-0.020833in}}{\pgfqpoint{0.020833in}{0.020833in}}{%
\pgfpathmoveto{\pgfqpoint{0.000000in}{-0.020833in}}%
\pgfpathcurveto{\pgfqpoint{0.005525in}{-0.020833in}}{\pgfqpoint{0.010825in}{-0.018638in}}{\pgfqpoint{0.014731in}{-0.014731in}}%
\pgfpathcurveto{\pgfqpoint{0.018638in}{-0.010825in}}{\pgfqpoint{0.020833in}{-0.005525in}}{\pgfqpoint{0.020833in}{0.000000in}}%
\pgfpathcurveto{\pgfqpoint{0.020833in}{0.005525in}}{\pgfqpoint{0.018638in}{0.010825in}}{\pgfqpoint{0.014731in}{0.014731in}}%
\pgfpathcurveto{\pgfqpoint{0.010825in}{0.018638in}}{\pgfqpoint{0.005525in}{0.020833in}}{\pgfqpoint{0.000000in}{0.020833in}}%
\pgfpathcurveto{\pgfqpoint{-0.005525in}{0.020833in}}{\pgfqpoint{-0.010825in}{0.018638in}}{\pgfqpoint{-0.014731in}{0.014731in}}%
\pgfpathcurveto{\pgfqpoint{-0.018638in}{0.010825in}}{\pgfqpoint{-0.020833in}{0.005525in}}{\pgfqpoint{-0.020833in}{0.000000in}}%
\pgfpathcurveto{\pgfqpoint{-0.020833in}{-0.005525in}}{\pgfqpoint{-0.018638in}{-0.010825in}}{\pgfqpoint{-0.014731in}{-0.014731in}}%
\pgfpathcurveto{\pgfqpoint{-0.010825in}{-0.018638in}}{\pgfqpoint{-0.005525in}{-0.020833in}}{\pgfqpoint{0.000000in}{-0.020833in}}%
\pgfpathlineto{\pgfqpoint{0.000000in}{-0.020833in}}%
\pgfpathclose%
\pgfusepath{stroke,fill}%
}%
\begin{pgfscope}%
\pgfsys@transformshift{0.809982in}{1.265039in}%
\pgfsys@useobject{currentmarker}{}%
\end{pgfscope}%
\begin{pgfscope}%
\pgfsys@transformshift{1.180017in}{1.176798in}%
\pgfsys@useobject{currentmarker}{}%
\end{pgfscope}%
\begin{pgfscope}%
\pgfsys@transformshift{1.530977in}{1.030861in}%
\pgfsys@useobject{currentmarker}{}%
\end{pgfscope}%
\begin{pgfscope}%
\pgfsys@transformshift{1.877132in}{0.954357in}%
\pgfsys@useobject{currentmarker}{}%
\end{pgfscope}%
\begin{pgfscope}%
\pgfsys@transformshift{2.222805in}{0.863397in}%
\pgfsys@useobject{currentmarker}{}%
\end{pgfscope}%
\begin{pgfscope}%
\pgfsys@transformshift{2.571345in}{0.727314in}%
\pgfsys@useobject{currentmarker}{}%
\end{pgfscope}%
\end{pgfscope}%
\begin{pgfscope}%
\pgfsetrectcap%
\pgfsetmiterjoin%
\pgfsetlinewidth{0.803000pt}%
\definecolor{currentstroke}{rgb}{0.000000,0.000000,0.000000}%
\pgfsetstrokecolor{currentstroke}%
\pgfsetdash{}{0pt}%
\pgfpathmoveto{\pgfqpoint{0.721913in}{0.549073in}}%
\pgfpathlineto{\pgfqpoint{0.721913in}{2.474073in}}%
\pgfusepath{stroke}%
\end{pgfscope}%
\begin{pgfscope}%
\pgfsetrectcap%
\pgfsetmiterjoin%
\pgfsetlinewidth{0.803000pt}%
\definecolor{currentstroke}{rgb}{0.000000,0.000000,0.000000}%
\pgfsetstrokecolor{currentstroke}%
\pgfsetdash{}{0pt}%
\pgfpathmoveto{\pgfqpoint{2.659413in}{0.549073in}}%
\pgfpathlineto{\pgfqpoint{2.659413in}{2.474073in}}%
\pgfusepath{stroke}%
\end{pgfscope}%
\begin{pgfscope}%
\pgfsetrectcap%
\pgfsetmiterjoin%
\pgfsetlinewidth{0.803000pt}%
\definecolor{currentstroke}{rgb}{0.000000,0.000000,0.000000}%
\pgfsetstrokecolor{currentstroke}%
\pgfsetdash{}{0pt}%
\pgfpathmoveto{\pgfqpoint{0.721913in}{0.549073in}}%
\pgfpathlineto{\pgfqpoint{2.659413in}{0.549073in}}%
\pgfusepath{stroke}%
\end{pgfscope}%
\begin{pgfscope}%
\pgfsetrectcap%
\pgfsetmiterjoin%
\pgfsetlinewidth{0.803000pt}%
\definecolor{currentstroke}{rgb}{0.000000,0.000000,0.000000}%
\pgfsetstrokecolor{currentstroke}%
\pgfsetdash{}{0pt}%
\pgfpathmoveto{\pgfqpoint{0.721913in}{2.474073in}}%
\pgfpathlineto{\pgfqpoint{2.659413in}{2.474073in}}%
\pgfusepath{stroke}%
\end{pgfscope}%
\begin{pgfscope}%
\pgfsetbuttcap%
\pgfsetmiterjoin%
\definecolor{currentfill}{rgb}{1.000000,1.000000,1.000000}%
\pgfsetfillcolor{currentfill}%
\pgfsetfillopacity{0.800000}%
\pgfsetlinewidth{1.003750pt}%
\definecolor{currentstroke}{rgb}{0.800000,0.800000,0.800000}%
\pgfsetstrokecolor{currentstroke}%
\pgfsetstrokeopacity{0.800000}%
\pgfsetdash{}{0pt}%
\pgfpathmoveto{\pgfqpoint{1.515360in}{1.137962in}}%
\pgfpathlineto{\pgfqpoint{2.542747in}{1.137962in}}%
\pgfpathquadraticcurveto{\pgfqpoint{2.576080in}{1.137962in}}{\pgfqpoint{2.576080in}{1.171295in}}%
\pgfpathlineto{\pgfqpoint{2.576080in}{1.851850in}}%
\pgfpathquadraticcurveto{\pgfqpoint{2.576080in}{1.885183in}}{\pgfqpoint{2.542747in}{1.885183in}}%
\pgfpathlineto{\pgfqpoint{1.515360in}{1.885183in}}%
\pgfpathquadraticcurveto{\pgfqpoint{1.482027in}{1.885183in}}{\pgfqpoint{1.482027in}{1.851850in}}%
\pgfpathlineto{\pgfqpoint{1.482027in}{1.171295in}}%
\pgfpathquadraticcurveto{\pgfqpoint{1.482027in}{1.137962in}}{\pgfqpoint{1.515360in}{1.137962in}}%
\pgfpathlineto{\pgfqpoint{1.515360in}{1.137962in}}%
\pgfpathclose%
\pgfusepath{stroke,fill}%
\end{pgfscope}%
\begin{pgfscope}%
\pgfsetrectcap%
\pgfsetroundjoin%
\pgfsetlinewidth{1.003750pt}%
\definecolor{currentstroke}{rgb}{0.001462,0.000466,0.013866}%
\pgfsetstrokecolor{currentstroke}%
\pgfsetdash{}{0pt}%
\pgfpathmoveto{\pgfqpoint{1.548693in}{1.760183in}}%
\pgfpathlineto{\pgfqpoint{1.715360in}{1.760183in}}%
\pgfpathlineto{\pgfqpoint{1.882027in}{1.760183in}}%
\pgfusepath{stroke}%
\end{pgfscope}%
\begin{pgfscope}%
\pgfsetbuttcap%
\pgfsetroundjoin%
\definecolor{currentfill}{rgb}{0.001462,0.000466,0.013866}%
\pgfsetfillcolor{currentfill}%
\pgfsetlinewidth{1.003750pt}%
\definecolor{currentstroke}{rgb}{0.001462,0.000466,0.013866}%
\pgfsetstrokecolor{currentstroke}%
\pgfsetdash{}{0pt}%
\pgfsys@defobject{currentmarker}{\pgfqpoint{-0.020833in}{-0.020833in}}{\pgfqpoint{0.020833in}{0.020833in}}{%
\pgfpathmoveto{\pgfqpoint{0.000000in}{-0.020833in}}%
\pgfpathcurveto{\pgfqpoint{0.005525in}{-0.020833in}}{\pgfqpoint{0.010825in}{-0.018638in}}{\pgfqpoint{0.014731in}{-0.014731in}}%
\pgfpathcurveto{\pgfqpoint{0.018638in}{-0.010825in}}{\pgfqpoint{0.020833in}{-0.005525in}}{\pgfqpoint{0.020833in}{0.000000in}}%
\pgfpathcurveto{\pgfqpoint{0.020833in}{0.005525in}}{\pgfqpoint{0.018638in}{0.010825in}}{\pgfqpoint{0.014731in}{0.014731in}}%
\pgfpathcurveto{\pgfqpoint{0.010825in}{0.018638in}}{\pgfqpoint{0.005525in}{0.020833in}}{\pgfqpoint{0.000000in}{0.020833in}}%
\pgfpathcurveto{\pgfqpoint{-0.005525in}{0.020833in}}{\pgfqpoint{-0.010825in}{0.018638in}}{\pgfqpoint{-0.014731in}{0.014731in}}%
\pgfpathcurveto{\pgfqpoint{-0.018638in}{0.010825in}}{\pgfqpoint{-0.020833in}{0.005525in}}{\pgfqpoint{-0.020833in}{0.000000in}}%
\pgfpathcurveto{\pgfqpoint{-0.020833in}{-0.005525in}}{\pgfqpoint{-0.018638in}{-0.010825in}}{\pgfqpoint{-0.014731in}{-0.014731in}}%
\pgfpathcurveto{\pgfqpoint{-0.010825in}{-0.018638in}}{\pgfqpoint{-0.005525in}{-0.020833in}}{\pgfqpoint{0.000000in}{-0.020833in}}%
\pgfpathlineto{\pgfqpoint{0.000000in}{-0.020833in}}%
\pgfpathclose%
\pgfusepath{stroke,fill}%
}%
\begin{pgfscope}%
\pgfsys@transformshift{1.715360in}{1.760183in}%
\pgfsys@useobject{currentmarker}{}%
\end{pgfscope}%
\end{pgfscope}%
\begin{pgfscope}%
\definecolor{textcolor}{rgb}{0.000000,0.000000,0.000000}%
\pgfsetstrokecolor{textcolor}%
\pgfsetfillcolor{textcolor}%
\pgftext[x=2.015360in,y=1.701850in,left,base]{\color{textcolor}{\rmfamily\fontsize{12.000000}{14.400000}\selectfont\catcode`\^=\active\def^{\ifmmode\sp\else\^{}\fi}\catcode`\%=\active\def%{\%}DGC}}%
\end{pgfscope}%
\begin{pgfscope}%
\pgfsetrectcap%
\pgfsetroundjoin%
\pgfsetlinewidth{1.003750pt}%
\definecolor{currentstroke}{rgb}{0.445163,0.122724,0.506901}%
\pgfsetstrokecolor{currentstroke}%
\pgfsetdash{}{0pt}%
\pgfpathmoveto{\pgfqpoint{1.548693in}{1.527776in}}%
\pgfpathlineto{\pgfqpoint{1.715360in}{1.527776in}}%
\pgfpathlineto{\pgfqpoint{1.882027in}{1.527776in}}%
\pgfusepath{stroke}%
\end{pgfscope}%
\begin{pgfscope}%
\pgfsetbuttcap%
\pgfsetroundjoin%
\definecolor{currentfill}{rgb}{0.445163,0.122724,0.506901}%
\pgfsetfillcolor{currentfill}%
\pgfsetlinewidth{1.003750pt}%
\definecolor{currentstroke}{rgb}{0.445163,0.122724,0.506901}%
\pgfsetstrokecolor{currentstroke}%
\pgfsetdash{}{0pt}%
\pgfsys@defobject{currentmarker}{\pgfqpoint{-0.020833in}{-0.020833in}}{\pgfqpoint{0.020833in}{0.020833in}}{%
\pgfpathmoveto{\pgfqpoint{0.000000in}{-0.020833in}}%
\pgfpathcurveto{\pgfqpoint{0.005525in}{-0.020833in}}{\pgfqpoint{0.010825in}{-0.018638in}}{\pgfqpoint{0.014731in}{-0.014731in}}%
\pgfpathcurveto{\pgfqpoint{0.018638in}{-0.010825in}}{\pgfqpoint{0.020833in}{-0.005525in}}{\pgfqpoint{0.020833in}{0.000000in}}%
\pgfpathcurveto{\pgfqpoint{0.020833in}{0.005525in}}{\pgfqpoint{0.018638in}{0.010825in}}{\pgfqpoint{0.014731in}{0.014731in}}%
\pgfpathcurveto{\pgfqpoint{0.010825in}{0.018638in}}{\pgfqpoint{0.005525in}{0.020833in}}{\pgfqpoint{0.000000in}{0.020833in}}%
\pgfpathcurveto{\pgfqpoint{-0.005525in}{0.020833in}}{\pgfqpoint{-0.010825in}{0.018638in}}{\pgfqpoint{-0.014731in}{0.014731in}}%
\pgfpathcurveto{\pgfqpoint{-0.018638in}{0.010825in}}{\pgfqpoint{-0.020833in}{0.005525in}}{\pgfqpoint{-0.020833in}{0.000000in}}%
\pgfpathcurveto{\pgfqpoint{-0.020833in}{-0.005525in}}{\pgfqpoint{-0.018638in}{-0.010825in}}{\pgfqpoint{-0.014731in}{-0.014731in}}%
\pgfpathcurveto{\pgfqpoint{-0.010825in}{-0.018638in}}{\pgfqpoint{-0.005525in}{-0.020833in}}{\pgfqpoint{0.000000in}{-0.020833in}}%
\pgfpathlineto{\pgfqpoint{0.000000in}{-0.020833in}}%
\pgfpathclose%
\pgfusepath{stroke,fill}%
}%
\begin{pgfscope}%
\pgfsys@transformshift{1.715360in}{1.527776in}%
\pgfsys@useobject{currentmarker}{}%
\end{pgfscope}%
\end{pgfscope}%
\begin{pgfscope}%
\definecolor{textcolor}{rgb}{0.000000,0.000000,0.000000}%
\pgfsetstrokecolor{textcolor}%
\pgfsetfillcolor{textcolor}%
\pgftext[x=2.015360in,y=1.469443in,left,base]{\color{textcolor}{\rmfamily\fontsize{12.000000}{14.400000}\selectfont\catcode`\^=\active\def^{\ifmmode\sp\else\^{}\fi}\catcode`\%=\active\def%{\%}NC}}%
\end{pgfscope}%
\begin{pgfscope}%
\pgfsetrectcap%
\pgfsetroundjoin%
\pgfsetlinewidth{1.003750pt}%
\definecolor{currentstroke}{rgb}{0.944006,0.377643,0.365136}%
\pgfsetstrokecolor{currentstroke}%
\pgfsetdash{}{0pt}%
\pgfpathmoveto{\pgfqpoint{1.548693in}{1.295369in}}%
\pgfpathlineto{\pgfqpoint{1.715360in}{1.295369in}}%
\pgfpathlineto{\pgfqpoint{1.882027in}{1.295369in}}%
\pgfusepath{stroke}%
\end{pgfscope}%
\begin{pgfscope}%
\pgfsetbuttcap%
\pgfsetroundjoin%
\definecolor{currentfill}{rgb}{0.944006,0.377643,0.365136}%
\pgfsetfillcolor{currentfill}%
\pgfsetlinewidth{1.003750pt}%
\definecolor{currentstroke}{rgb}{0.944006,0.377643,0.365136}%
\pgfsetstrokecolor{currentstroke}%
\pgfsetdash{}{0pt}%
\pgfsys@defobject{currentmarker}{\pgfqpoint{-0.020833in}{-0.020833in}}{\pgfqpoint{0.020833in}{0.020833in}}{%
\pgfpathmoveto{\pgfqpoint{0.000000in}{-0.020833in}}%
\pgfpathcurveto{\pgfqpoint{0.005525in}{-0.020833in}}{\pgfqpoint{0.010825in}{-0.018638in}}{\pgfqpoint{0.014731in}{-0.014731in}}%
\pgfpathcurveto{\pgfqpoint{0.018638in}{-0.010825in}}{\pgfqpoint{0.020833in}{-0.005525in}}{\pgfqpoint{0.020833in}{0.000000in}}%
\pgfpathcurveto{\pgfqpoint{0.020833in}{0.005525in}}{\pgfqpoint{0.018638in}{0.010825in}}{\pgfqpoint{0.014731in}{0.014731in}}%
\pgfpathcurveto{\pgfqpoint{0.010825in}{0.018638in}}{\pgfqpoint{0.005525in}{0.020833in}}{\pgfqpoint{0.000000in}{0.020833in}}%
\pgfpathcurveto{\pgfqpoint{-0.005525in}{0.020833in}}{\pgfqpoint{-0.010825in}{0.018638in}}{\pgfqpoint{-0.014731in}{0.014731in}}%
\pgfpathcurveto{\pgfqpoint{-0.018638in}{0.010825in}}{\pgfqpoint{-0.020833in}{0.005525in}}{\pgfqpoint{-0.020833in}{0.000000in}}%
\pgfpathcurveto{\pgfqpoint{-0.020833in}{-0.005525in}}{\pgfqpoint{-0.018638in}{-0.010825in}}{\pgfqpoint{-0.014731in}{-0.014731in}}%
\pgfpathcurveto{\pgfqpoint{-0.010825in}{-0.018638in}}{\pgfqpoint{-0.005525in}{-0.020833in}}{\pgfqpoint{0.000000in}{-0.020833in}}%
\pgfpathlineto{\pgfqpoint{0.000000in}{-0.020833in}}%
\pgfpathclose%
\pgfusepath{stroke,fill}%
}%
\begin{pgfscope}%
\pgfsys@transformshift{1.715360in}{1.295369in}%
\pgfsys@useobject{currentmarker}{}%
\end{pgfscope}%
\end{pgfscope}%
\begin{pgfscope}%
\definecolor{textcolor}{rgb}{0.000000,0.000000,0.000000}%
\pgfsetstrokecolor{textcolor}%
\pgfsetfillcolor{textcolor}%
\pgftext[x=2.015360in,y=1.237036in,left,base]{\color{textcolor}{\rmfamily\fontsize{12.000000}{14.400000}\selectfont\catcode`\^=\active\def^{\ifmmode\sp\else\^{}\fi}\catcode`\%=\active\def%{\%}NC++}}%
\end{pgfscope}%
\end{pgfpicture}%
\makeatother%
\endgroup%

        \caption{uniform}
        \label{fig:5-experiments-multi-matrix-convergence-uniform}
    \end{subfigure}
    \begin{subfigure}[b]{0.49\columnwidth}
        %% Creator: Matplotlib, PGF backend
%%
%% To include the figure in your LaTeX document, write
%%   \input{<filename>.pgf}
%%
%% Make sure the required packages are loaded in your preamble
%%   \usepackage{pgf}
%%
%% Also ensure that all the required font packages are loaded; for instance,
%% the lmodern package is sometimes necessary when using math font.
%%   \usepackage{lmodern}
%%
%% Figures using additional raster images can only be included by \input if
%% they are in the same directory as the main LaTeX file. For loading figures
%% from other directories you can use the `import` package
%%   \usepackage{import}
%%
%% and then include the figures with
%%   \import{<path to file>}{<filename>.pgf}
%%
%% Matplotlib used the following preamble
%%   \def\mathdefault#1{#1}
%%   \everymath=\expandafter{\the\everymath\displaystyle}
%%   
%%   \makeatletter\@ifpackageloaded{underscore}{}{\usepackage[strings]{underscore}}\makeatother
%%
\begingroup%
\makeatletter%
\begin{pgfpicture}%
\pgfpathrectangle{\pgfpointorigin}{\pgfqpoint{3.022491in}{2.621308in}}%
\pgfusepath{use as bounding box, clip}%
\begin{pgfscope}%
\pgfsetbuttcap%
\pgfsetmiterjoin%
\definecolor{currentfill}{rgb}{1.000000,1.000000,1.000000}%
\pgfsetfillcolor{currentfill}%
\pgfsetlinewidth{0.000000pt}%
\definecolor{currentstroke}{rgb}{1.000000,1.000000,1.000000}%
\pgfsetstrokecolor{currentstroke}%
\pgfsetdash{}{0pt}%
\pgfpathmoveto{\pgfqpoint{0.000000in}{0.000000in}}%
\pgfpathlineto{\pgfqpoint{3.022491in}{0.000000in}}%
\pgfpathlineto{\pgfqpoint{3.022491in}{2.621308in}}%
\pgfpathlineto{\pgfqpoint{0.000000in}{2.621308in}}%
\pgfpathlineto{\pgfqpoint{0.000000in}{0.000000in}}%
\pgfpathclose%
\pgfusepath{fill}%
\end{pgfscope}%
\begin{pgfscope}%
\pgfsetbuttcap%
\pgfsetmiterjoin%
\definecolor{currentfill}{rgb}{1.000000,1.000000,1.000000}%
\pgfsetfillcolor{currentfill}%
\pgfsetlinewidth{0.000000pt}%
\definecolor{currentstroke}{rgb}{0.000000,0.000000,0.000000}%
\pgfsetstrokecolor{currentstroke}%
\pgfsetstrokeopacity{0.000000}%
\pgfsetdash{}{0pt}%
\pgfpathmoveto{\pgfqpoint{0.984991in}{0.549073in}}%
\pgfpathlineto{\pgfqpoint{2.922491in}{0.549073in}}%
\pgfpathlineto{\pgfqpoint{2.922491in}{2.474073in}}%
\pgfpathlineto{\pgfqpoint{0.984991in}{2.474073in}}%
\pgfpathlineto{\pgfqpoint{0.984991in}{0.549073in}}%
\pgfpathclose%
\pgfusepath{fill}%
\end{pgfscope}%
\begin{pgfscope}%
\pgfsetbuttcap%
\pgfsetroundjoin%
\definecolor{currentfill}{rgb}{0.000000,0.000000,0.000000}%
\pgfsetfillcolor{currentfill}%
\pgfsetlinewidth{0.803000pt}%
\definecolor{currentstroke}{rgb}{0.000000,0.000000,0.000000}%
\pgfsetstrokecolor{currentstroke}%
\pgfsetdash{}{0pt}%
\pgfsys@defobject{currentmarker}{\pgfqpoint{0.000000in}{-0.048611in}}{\pgfqpoint{0.000000in}{0.000000in}}{%
\pgfpathmoveto{\pgfqpoint{0.000000in}{0.000000in}}%
\pgfpathlineto{\pgfqpoint{0.000000in}{-0.048611in}}%
\pgfusepath{stroke,fill}%
}%
\begin{pgfscope}%
\pgfsys@transformshift{2.034643in}{0.549073in}%
\pgfsys@useobject{currentmarker}{}%
\end{pgfscope}%
\end{pgfscope}%
\begin{pgfscope}%
\definecolor{textcolor}{rgb}{0.000000,0.000000,0.000000}%
\pgfsetstrokecolor{textcolor}%
\pgfsetfillcolor{textcolor}%
\pgftext[x=2.034643in,y=0.451851in,,top]{\color{textcolor}{\rmfamily\fontsize{12.000000}{14.400000}\selectfont\catcode`\^=\active\def^{\ifmmode\sp\else\^{}\fi}\catcode`\%=\active\def%{\%}$\mathdefault{10^{2}}$}}%
\end{pgfscope}%
\begin{pgfscope}%
\pgfsetbuttcap%
\pgfsetroundjoin%
\definecolor{currentfill}{rgb}{0.000000,0.000000,0.000000}%
\pgfsetfillcolor{currentfill}%
\pgfsetlinewidth{0.602250pt}%
\definecolor{currentstroke}{rgb}{0.000000,0.000000,0.000000}%
\pgfsetstrokecolor{currentstroke}%
\pgfsetdash{}{0pt}%
\pgfsys@defobject{currentmarker}{\pgfqpoint{0.000000in}{-0.027778in}}{\pgfqpoint{0.000000in}{0.000000in}}{%
\pgfpathmoveto{\pgfqpoint{0.000000in}{0.000000in}}%
\pgfpathlineto{\pgfqpoint{0.000000in}{-0.027778in}}%
\pgfusepath{stroke,fill}%
}%
\begin{pgfscope}%
\pgfsys@transformshift{1.102758in}{0.549073in}%
\pgfsys@useobject{currentmarker}{}%
\end{pgfscope}%
\end{pgfscope}%
\begin{pgfscope}%
\pgfsetbuttcap%
\pgfsetroundjoin%
\definecolor{currentfill}{rgb}{0.000000,0.000000,0.000000}%
\pgfsetfillcolor{currentfill}%
\pgfsetlinewidth{0.602250pt}%
\definecolor{currentstroke}{rgb}{0.000000,0.000000,0.000000}%
\pgfsetstrokecolor{currentstroke}%
\pgfsetdash{}{0pt}%
\pgfsys@defobject{currentmarker}{\pgfqpoint{0.000000in}{-0.027778in}}{\pgfqpoint{0.000000in}{0.000000in}}{%
\pgfpathmoveto{\pgfqpoint{0.000000in}{0.000000in}}%
\pgfpathlineto{\pgfqpoint{0.000000in}{-0.027778in}}%
\pgfusepath{stroke,fill}%
}%
\begin{pgfscope}%
\pgfsys@transformshift{1.337528in}{0.549073in}%
\pgfsys@useobject{currentmarker}{}%
\end{pgfscope}%
\end{pgfscope}%
\begin{pgfscope}%
\pgfsetbuttcap%
\pgfsetroundjoin%
\definecolor{currentfill}{rgb}{0.000000,0.000000,0.000000}%
\pgfsetfillcolor{currentfill}%
\pgfsetlinewidth{0.602250pt}%
\definecolor{currentstroke}{rgb}{0.000000,0.000000,0.000000}%
\pgfsetstrokecolor{currentstroke}%
\pgfsetdash{}{0pt}%
\pgfsys@defobject{currentmarker}{\pgfqpoint{0.000000in}{-0.027778in}}{\pgfqpoint{0.000000in}{0.000000in}}{%
\pgfpathmoveto{\pgfqpoint{0.000000in}{0.000000in}}%
\pgfpathlineto{\pgfqpoint{0.000000in}{-0.027778in}}%
\pgfusepath{stroke,fill}%
}%
\begin{pgfscope}%
\pgfsys@transformshift{1.504099in}{0.549073in}%
\pgfsys@useobject{currentmarker}{}%
\end{pgfscope}%
\end{pgfscope}%
\begin{pgfscope}%
\pgfsetbuttcap%
\pgfsetroundjoin%
\definecolor{currentfill}{rgb}{0.000000,0.000000,0.000000}%
\pgfsetfillcolor{currentfill}%
\pgfsetlinewidth{0.602250pt}%
\definecolor{currentstroke}{rgb}{0.000000,0.000000,0.000000}%
\pgfsetstrokecolor{currentstroke}%
\pgfsetdash{}{0pt}%
\pgfsys@defobject{currentmarker}{\pgfqpoint{0.000000in}{-0.027778in}}{\pgfqpoint{0.000000in}{0.000000in}}{%
\pgfpathmoveto{\pgfqpoint{0.000000in}{0.000000in}}%
\pgfpathlineto{\pgfqpoint{0.000000in}{-0.027778in}}%
\pgfusepath{stroke,fill}%
}%
\begin{pgfscope}%
\pgfsys@transformshift{1.633302in}{0.549073in}%
\pgfsys@useobject{currentmarker}{}%
\end{pgfscope}%
\end{pgfscope}%
\begin{pgfscope}%
\pgfsetbuttcap%
\pgfsetroundjoin%
\definecolor{currentfill}{rgb}{0.000000,0.000000,0.000000}%
\pgfsetfillcolor{currentfill}%
\pgfsetlinewidth{0.602250pt}%
\definecolor{currentstroke}{rgb}{0.000000,0.000000,0.000000}%
\pgfsetstrokecolor{currentstroke}%
\pgfsetdash{}{0pt}%
\pgfsys@defobject{currentmarker}{\pgfqpoint{0.000000in}{-0.027778in}}{\pgfqpoint{0.000000in}{0.000000in}}{%
\pgfpathmoveto{\pgfqpoint{0.000000in}{0.000000in}}%
\pgfpathlineto{\pgfqpoint{0.000000in}{-0.027778in}}%
\pgfusepath{stroke,fill}%
}%
\begin{pgfscope}%
\pgfsys@transformshift{1.738868in}{0.549073in}%
\pgfsys@useobject{currentmarker}{}%
\end{pgfscope}%
\end{pgfscope}%
\begin{pgfscope}%
\pgfsetbuttcap%
\pgfsetroundjoin%
\definecolor{currentfill}{rgb}{0.000000,0.000000,0.000000}%
\pgfsetfillcolor{currentfill}%
\pgfsetlinewidth{0.602250pt}%
\definecolor{currentstroke}{rgb}{0.000000,0.000000,0.000000}%
\pgfsetstrokecolor{currentstroke}%
\pgfsetdash{}{0pt}%
\pgfsys@defobject{currentmarker}{\pgfqpoint{0.000000in}{-0.027778in}}{\pgfqpoint{0.000000in}{0.000000in}}{%
\pgfpathmoveto{\pgfqpoint{0.000000in}{0.000000in}}%
\pgfpathlineto{\pgfqpoint{0.000000in}{-0.027778in}}%
\pgfusepath{stroke,fill}%
}%
\begin{pgfscope}%
\pgfsys@transformshift{1.828124in}{0.549073in}%
\pgfsys@useobject{currentmarker}{}%
\end{pgfscope}%
\end{pgfscope}%
\begin{pgfscope}%
\pgfsetbuttcap%
\pgfsetroundjoin%
\definecolor{currentfill}{rgb}{0.000000,0.000000,0.000000}%
\pgfsetfillcolor{currentfill}%
\pgfsetlinewidth{0.602250pt}%
\definecolor{currentstroke}{rgb}{0.000000,0.000000,0.000000}%
\pgfsetstrokecolor{currentstroke}%
\pgfsetdash{}{0pt}%
\pgfsys@defobject{currentmarker}{\pgfqpoint{0.000000in}{-0.027778in}}{\pgfqpoint{0.000000in}{0.000000in}}{%
\pgfpathmoveto{\pgfqpoint{0.000000in}{0.000000in}}%
\pgfpathlineto{\pgfqpoint{0.000000in}{-0.027778in}}%
\pgfusepath{stroke,fill}%
}%
\begin{pgfscope}%
\pgfsys@transformshift{1.905440in}{0.549073in}%
\pgfsys@useobject{currentmarker}{}%
\end{pgfscope}%
\end{pgfscope}%
\begin{pgfscope}%
\pgfsetbuttcap%
\pgfsetroundjoin%
\definecolor{currentfill}{rgb}{0.000000,0.000000,0.000000}%
\pgfsetfillcolor{currentfill}%
\pgfsetlinewidth{0.602250pt}%
\definecolor{currentstroke}{rgb}{0.000000,0.000000,0.000000}%
\pgfsetstrokecolor{currentstroke}%
\pgfsetdash{}{0pt}%
\pgfsys@defobject{currentmarker}{\pgfqpoint{0.000000in}{-0.027778in}}{\pgfqpoint{0.000000in}{0.000000in}}{%
\pgfpathmoveto{\pgfqpoint{0.000000in}{0.000000in}}%
\pgfpathlineto{\pgfqpoint{0.000000in}{-0.027778in}}%
\pgfusepath{stroke,fill}%
}%
\begin{pgfscope}%
\pgfsys@transformshift{1.973638in}{0.549073in}%
\pgfsys@useobject{currentmarker}{}%
\end{pgfscope}%
\end{pgfscope}%
\begin{pgfscope}%
\pgfsetbuttcap%
\pgfsetroundjoin%
\definecolor{currentfill}{rgb}{0.000000,0.000000,0.000000}%
\pgfsetfillcolor{currentfill}%
\pgfsetlinewidth{0.602250pt}%
\definecolor{currentstroke}{rgb}{0.000000,0.000000,0.000000}%
\pgfsetstrokecolor{currentstroke}%
\pgfsetdash{}{0pt}%
\pgfsys@defobject{currentmarker}{\pgfqpoint{0.000000in}{-0.027778in}}{\pgfqpoint{0.000000in}{0.000000in}}{%
\pgfpathmoveto{\pgfqpoint{0.000000in}{0.000000in}}%
\pgfpathlineto{\pgfqpoint{0.000000in}{-0.027778in}}%
\pgfusepath{stroke,fill}%
}%
\begin{pgfscope}%
\pgfsys@transformshift{2.435984in}{0.549073in}%
\pgfsys@useobject{currentmarker}{}%
\end{pgfscope}%
\end{pgfscope}%
\begin{pgfscope}%
\pgfsetbuttcap%
\pgfsetroundjoin%
\definecolor{currentfill}{rgb}{0.000000,0.000000,0.000000}%
\pgfsetfillcolor{currentfill}%
\pgfsetlinewidth{0.602250pt}%
\definecolor{currentstroke}{rgb}{0.000000,0.000000,0.000000}%
\pgfsetstrokecolor{currentstroke}%
\pgfsetdash{}{0pt}%
\pgfsys@defobject{currentmarker}{\pgfqpoint{0.000000in}{-0.027778in}}{\pgfqpoint{0.000000in}{0.000000in}}{%
\pgfpathmoveto{\pgfqpoint{0.000000in}{0.000000in}}%
\pgfpathlineto{\pgfqpoint{0.000000in}{-0.027778in}}%
\pgfusepath{stroke,fill}%
}%
\begin{pgfscope}%
\pgfsys@transformshift{2.670753in}{0.549073in}%
\pgfsys@useobject{currentmarker}{}%
\end{pgfscope}%
\end{pgfscope}%
\begin{pgfscope}%
\pgfsetbuttcap%
\pgfsetroundjoin%
\definecolor{currentfill}{rgb}{0.000000,0.000000,0.000000}%
\pgfsetfillcolor{currentfill}%
\pgfsetlinewidth{0.602250pt}%
\definecolor{currentstroke}{rgb}{0.000000,0.000000,0.000000}%
\pgfsetstrokecolor{currentstroke}%
\pgfsetdash{}{0pt}%
\pgfsys@defobject{currentmarker}{\pgfqpoint{0.000000in}{-0.027778in}}{\pgfqpoint{0.000000in}{0.000000in}}{%
\pgfpathmoveto{\pgfqpoint{0.000000in}{0.000000in}}%
\pgfpathlineto{\pgfqpoint{0.000000in}{-0.027778in}}%
\pgfusepath{stroke,fill}%
}%
\begin{pgfscope}%
\pgfsys@transformshift{2.837325in}{0.549073in}%
\pgfsys@useobject{currentmarker}{}%
\end{pgfscope}%
\end{pgfscope}%
\begin{pgfscope}%
\definecolor{textcolor}{rgb}{0.000000,0.000000,0.000000}%
\pgfsetstrokecolor{textcolor}%
\pgfsetfillcolor{textcolor}%
\pgftext[x=1.953741in,y=0.248148in,,top]{\color{textcolor}{\rmfamily\fontsize{12.000000}{14.400000}\selectfont\catcode`\^=\active\def^{\ifmmode\sp\else\^{}\fi}\catcode`\%=\active\def%{\%}$n_{\Omega} + n_{\Psi}$}}%
\end{pgfscope}%
\begin{pgfscope}%
\pgfsetbuttcap%
\pgfsetroundjoin%
\definecolor{currentfill}{rgb}{0.000000,0.000000,0.000000}%
\pgfsetfillcolor{currentfill}%
\pgfsetlinewidth{0.803000pt}%
\definecolor{currentstroke}{rgb}{0.000000,0.000000,0.000000}%
\pgfsetstrokecolor{currentstroke}%
\pgfsetdash{}{0pt}%
\pgfsys@defobject{currentmarker}{\pgfqpoint{-0.048611in}{0.000000in}}{\pgfqpoint{-0.000000in}{0.000000in}}{%
\pgfpathmoveto{\pgfqpoint{-0.000000in}{0.000000in}}%
\pgfpathlineto{\pgfqpoint{-0.048611in}{0.000000in}}%
\pgfusepath{stroke,fill}%
}%
\begin{pgfscope}%
\pgfsys@transformshift{0.984991in}{1.490216in}%
\pgfsys@useobject{currentmarker}{}%
\end{pgfscope}%
\end{pgfscope}%
\begin{pgfscope}%
\definecolor{textcolor}{rgb}{0.000000,0.000000,0.000000}%
\pgfsetstrokecolor{textcolor}%
\pgfsetfillcolor{textcolor}%
\pgftext[x=0.566780in, y=1.432346in, left, base]{\color{textcolor}{\rmfamily\fontsize{12.000000}{14.400000}\selectfont\catcode`\^=\active\def^{\ifmmode\sp\else\^{}\fi}\catcode`\%=\active\def%{\%}$\mathdefault{10^{-1}}$}}%
\end{pgfscope}%
\begin{pgfscope}%
\pgfsetbuttcap%
\pgfsetroundjoin%
\definecolor{currentfill}{rgb}{0.000000,0.000000,0.000000}%
\pgfsetfillcolor{currentfill}%
\pgfsetlinewidth{0.602250pt}%
\definecolor{currentstroke}{rgb}{0.000000,0.000000,0.000000}%
\pgfsetstrokecolor{currentstroke}%
\pgfsetdash{}{0pt}%
\pgfsys@defobject{currentmarker}{\pgfqpoint{-0.027778in}{0.000000in}}{\pgfqpoint{-0.000000in}{0.000000in}}{%
\pgfpathmoveto{\pgfqpoint{-0.000000in}{0.000000in}}%
\pgfpathlineto{\pgfqpoint{-0.027778in}{0.000000in}}%
\pgfusepath{stroke,fill}%
}%
\begin{pgfscope}%
\pgfsys@transformshift{0.984991in}{0.681076in}%
\pgfsys@useobject{currentmarker}{}%
\end{pgfscope}%
\end{pgfscope}%
\begin{pgfscope}%
\definecolor{textcolor}{rgb}{0.000000,0.000000,0.000000}%
\pgfsetstrokecolor{textcolor}%
\pgfsetfillcolor{textcolor}%
\pgftext[x=0.303703in, y=0.620125in, left, base]{\color{textcolor}{\rmfamily\fontsize{12.000000}{14.400000}\selectfont\catcode`\^=\active\def^{\ifmmode\sp\else\^{}\fi}\catcode`\%=\active\def%{\%}$\mathdefault{4\times10^{-2}}$}}%
\end{pgfscope}%
\begin{pgfscope}%
\pgfsetbuttcap%
\pgfsetroundjoin%
\definecolor{currentfill}{rgb}{0.000000,0.000000,0.000000}%
\pgfsetfillcolor{currentfill}%
\pgfsetlinewidth{0.602250pt}%
\definecolor{currentstroke}{rgb}{0.000000,0.000000,0.000000}%
\pgfsetstrokecolor{currentstroke}%
\pgfsetdash{}{0pt}%
\pgfsys@defobject{currentmarker}{\pgfqpoint{-0.027778in}{0.000000in}}{\pgfqpoint{-0.000000in}{0.000000in}}{%
\pgfpathmoveto{\pgfqpoint{-0.000000in}{0.000000in}}%
\pgfpathlineto{\pgfqpoint{-0.027778in}{0.000000in}}%
\pgfusepath{stroke,fill}%
}%
\begin{pgfscope}%
\pgfsys@transformshift{0.984991in}{0.878125in}%
\pgfsys@useobject{currentmarker}{}%
\end{pgfscope}%
\end{pgfscope}%
\begin{pgfscope}%
\pgfsetbuttcap%
\pgfsetroundjoin%
\definecolor{currentfill}{rgb}{0.000000,0.000000,0.000000}%
\pgfsetfillcolor{currentfill}%
\pgfsetlinewidth{0.602250pt}%
\definecolor{currentstroke}{rgb}{0.000000,0.000000,0.000000}%
\pgfsetstrokecolor{currentstroke}%
\pgfsetdash{}{0pt}%
\pgfsys@defobject{currentmarker}{\pgfqpoint{-0.027778in}{0.000000in}}{\pgfqpoint{-0.000000in}{0.000000in}}{%
\pgfpathmoveto{\pgfqpoint{-0.000000in}{0.000000in}}%
\pgfpathlineto{\pgfqpoint{-0.027778in}{0.000000in}}%
\pgfusepath{stroke,fill}%
}%
\begin{pgfscope}%
\pgfsys@transformshift{0.984991in}{1.039126in}%
\pgfsys@useobject{currentmarker}{}%
\end{pgfscope}%
\end{pgfscope}%
\begin{pgfscope}%
\definecolor{textcolor}{rgb}{0.000000,0.000000,0.000000}%
\pgfsetstrokecolor{textcolor}%
\pgfsetfillcolor{textcolor}%
\pgftext[x=0.303703in, y=0.978175in, left, base]{\color{textcolor}{\rmfamily\fontsize{12.000000}{14.400000}\selectfont\catcode`\^=\active\def^{\ifmmode\sp\else\^{}\fi}\catcode`\%=\active\def%{\%}$\mathdefault{6\times10^{-2}}$}}%
\end{pgfscope}%
\begin{pgfscope}%
\pgfsetbuttcap%
\pgfsetroundjoin%
\definecolor{currentfill}{rgb}{0.000000,0.000000,0.000000}%
\pgfsetfillcolor{currentfill}%
\pgfsetlinewidth{0.602250pt}%
\definecolor{currentstroke}{rgb}{0.000000,0.000000,0.000000}%
\pgfsetstrokecolor{currentstroke}%
\pgfsetdash{}{0pt}%
\pgfsys@defobject{currentmarker}{\pgfqpoint{-0.027778in}{0.000000in}}{\pgfqpoint{-0.000000in}{0.000000in}}{%
\pgfpathmoveto{\pgfqpoint{-0.000000in}{0.000000in}}%
\pgfpathlineto{\pgfqpoint{-0.027778in}{0.000000in}}%
\pgfusepath{stroke,fill}%
}%
\begin{pgfscope}%
\pgfsys@transformshift{0.984991in}{1.175250in}%
\pgfsys@useobject{currentmarker}{}%
\end{pgfscope}%
\end{pgfscope}%
\begin{pgfscope}%
\pgfsetbuttcap%
\pgfsetroundjoin%
\definecolor{currentfill}{rgb}{0.000000,0.000000,0.000000}%
\pgfsetfillcolor{currentfill}%
\pgfsetlinewidth{0.602250pt}%
\definecolor{currentstroke}{rgb}{0.000000,0.000000,0.000000}%
\pgfsetstrokecolor{currentstroke}%
\pgfsetdash{}{0pt}%
\pgfsys@defobject{currentmarker}{\pgfqpoint{-0.027778in}{0.000000in}}{\pgfqpoint{-0.000000in}{0.000000in}}{%
\pgfpathmoveto{\pgfqpoint{-0.000000in}{0.000000in}}%
\pgfpathlineto{\pgfqpoint{-0.027778in}{0.000000in}}%
\pgfusepath{stroke,fill}%
}%
\begin{pgfscope}%
\pgfsys@transformshift{0.984991in}{1.293167in}%
\pgfsys@useobject{currentmarker}{}%
\end{pgfscope}%
\end{pgfscope}%
\begin{pgfscope}%
\pgfsetbuttcap%
\pgfsetroundjoin%
\definecolor{currentfill}{rgb}{0.000000,0.000000,0.000000}%
\pgfsetfillcolor{currentfill}%
\pgfsetlinewidth{0.602250pt}%
\definecolor{currentstroke}{rgb}{0.000000,0.000000,0.000000}%
\pgfsetstrokecolor{currentstroke}%
\pgfsetdash{}{0pt}%
\pgfsys@defobject{currentmarker}{\pgfqpoint{-0.027778in}{0.000000in}}{\pgfqpoint{-0.000000in}{0.000000in}}{%
\pgfpathmoveto{\pgfqpoint{-0.000000in}{0.000000in}}%
\pgfpathlineto{\pgfqpoint{-0.027778in}{0.000000in}}%
\pgfusepath{stroke,fill}%
}%
\begin{pgfscope}%
\pgfsys@transformshift{0.984991in}{1.397176in}%
\pgfsys@useobject{currentmarker}{}%
\end{pgfscope}%
\end{pgfscope}%
\begin{pgfscope}%
\pgfsetbuttcap%
\pgfsetroundjoin%
\definecolor{currentfill}{rgb}{0.000000,0.000000,0.000000}%
\pgfsetfillcolor{currentfill}%
\pgfsetlinewidth{0.602250pt}%
\definecolor{currentstroke}{rgb}{0.000000,0.000000,0.000000}%
\pgfsetstrokecolor{currentstroke}%
\pgfsetdash{}{0pt}%
\pgfsys@defobject{currentmarker}{\pgfqpoint{-0.027778in}{0.000000in}}{\pgfqpoint{-0.000000in}{0.000000in}}{%
\pgfpathmoveto{\pgfqpoint{-0.000000in}{0.000000in}}%
\pgfpathlineto{\pgfqpoint{-0.027778in}{0.000000in}}%
\pgfusepath{stroke,fill}%
}%
\begin{pgfscope}%
\pgfsys@transformshift{0.984991in}{2.102307in}%
\pgfsys@useobject{currentmarker}{}%
\end{pgfscope}%
\end{pgfscope}%
\begin{pgfscope}%
\definecolor{textcolor}{rgb}{0.000000,0.000000,0.000000}%
\pgfsetstrokecolor{textcolor}%
\pgfsetfillcolor{textcolor}%
\pgftext[x=0.303703in, y=2.041356in, left, base]{\color{textcolor}{\rmfamily\fontsize{12.000000}{14.400000}\selectfont\catcode`\^=\active\def^{\ifmmode\sp\else\^{}\fi}\catcode`\%=\active\def%{\%}$\mathdefault{2\times10^{-1}}$}}%
\end{pgfscope}%
\begin{pgfscope}%
\pgfsetbuttcap%
\pgfsetroundjoin%
\definecolor{currentfill}{rgb}{0.000000,0.000000,0.000000}%
\pgfsetfillcolor{currentfill}%
\pgfsetlinewidth{0.602250pt}%
\definecolor{currentstroke}{rgb}{0.000000,0.000000,0.000000}%
\pgfsetstrokecolor{currentstroke}%
\pgfsetdash{}{0pt}%
\pgfsys@defobject{currentmarker}{\pgfqpoint{-0.027778in}{0.000000in}}{\pgfqpoint{-0.000000in}{0.000000in}}{%
\pgfpathmoveto{\pgfqpoint{-0.000000in}{0.000000in}}%
\pgfpathlineto{\pgfqpoint{-0.027778in}{0.000000in}}%
\pgfusepath{stroke,fill}%
}%
\begin{pgfscope}%
\pgfsys@transformshift{0.984991in}{2.460357in}%
\pgfsys@useobject{currentmarker}{}%
\end{pgfscope}%
\end{pgfscope}%
\begin{pgfscope}%
\definecolor{textcolor}{rgb}{0.000000,0.000000,0.000000}%
\pgfsetstrokecolor{textcolor}%
\pgfsetfillcolor{textcolor}%
\pgftext[x=0.303703in, y=2.399406in, left, base]{\color{textcolor}{\rmfamily\fontsize{12.000000}{14.400000}\selectfont\catcode`\^=\active\def^{\ifmmode\sp\else\^{}\fi}\catcode`\%=\active\def%{\%}$\mathdefault{3\times10^{-1}}$}}%
\end{pgfscope}%
\begin{pgfscope}%
\definecolor{textcolor}{rgb}{0.000000,0.000000,0.000000}%
\pgfsetstrokecolor{textcolor}%
\pgfsetfillcolor{textcolor}%
\pgftext[x=0.248148in,y=1.511573in,,bottom,rotate=90.000000]{\color{textcolor}{\rmfamily\fontsize{12.000000}{14.400000}\selectfont\catcode`\^=\active\def^{\ifmmode\sp\else\^{}\fi}\catcode`\%=\active\def%{\%}$L^1$ relative error}}%
\end{pgfscope}%
\begin{pgfscope}%
\pgfpathrectangle{\pgfqpoint{0.984991in}{0.549073in}}{\pgfqpoint{1.937500in}{1.925000in}}%
\pgfusepath{clip}%
\pgfsetrectcap%
\pgfsetroundjoin%
\pgfsetlinewidth{1.003750pt}%
\definecolor{currentstroke}{rgb}{0.537255,0.647059,0.760784}%
\pgfsetstrokecolor{currentstroke}%
\pgfsetdash{}{0pt}%
\pgfpathmoveto{\pgfqpoint{1.073059in}{2.116964in}}%
\pgfpathlineto{\pgfqpoint{1.443094in}{1.863215in}}%
\pgfpathlineto{\pgfqpoint{1.794054in}{1.595523in}}%
\pgfpathlineto{\pgfqpoint{2.140209in}{1.379860in}}%
\pgfpathlineto{\pgfqpoint{2.485882in}{1.218567in}}%
\pgfpathlineto{\pgfqpoint{2.834422in}{1.166856in}}%
\pgfusepath{stroke}%
\end{pgfscope}%
\begin{pgfscope}%
\pgfpathrectangle{\pgfqpoint{0.984991in}{0.549073in}}{\pgfqpoint{1.937500in}{1.925000in}}%
\pgfusepath{clip}%
\pgfsetbuttcap%
\pgfsetroundjoin%
\definecolor{currentfill}{rgb}{0.537255,0.647059,0.760784}%
\pgfsetfillcolor{currentfill}%
\pgfsetlinewidth{1.003750pt}%
\definecolor{currentstroke}{rgb}{0.537255,0.647059,0.760784}%
\pgfsetstrokecolor{currentstroke}%
\pgfsetdash{}{0pt}%
\pgfsys@defobject{currentmarker}{\pgfqpoint{-0.020833in}{-0.020833in}}{\pgfqpoint{0.020833in}{0.020833in}}{%
\pgfpathmoveto{\pgfqpoint{0.000000in}{-0.020833in}}%
\pgfpathcurveto{\pgfqpoint{0.005525in}{-0.020833in}}{\pgfqpoint{0.010825in}{-0.018638in}}{\pgfqpoint{0.014731in}{-0.014731in}}%
\pgfpathcurveto{\pgfqpoint{0.018638in}{-0.010825in}}{\pgfqpoint{0.020833in}{-0.005525in}}{\pgfqpoint{0.020833in}{0.000000in}}%
\pgfpathcurveto{\pgfqpoint{0.020833in}{0.005525in}}{\pgfqpoint{0.018638in}{0.010825in}}{\pgfqpoint{0.014731in}{0.014731in}}%
\pgfpathcurveto{\pgfqpoint{0.010825in}{0.018638in}}{\pgfqpoint{0.005525in}{0.020833in}}{\pgfqpoint{0.000000in}{0.020833in}}%
\pgfpathcurveto{\pgfqpoint{-0.005525in}{0.020833in}}{\pgfqpoint{-0.010825in}{0.018638in}}{\pgfqpoint{-0.014731in}{0.014731in}}%
\pgfpathcurveto{\pgfqpoint{-0.018638in}{0.010825in}}{\pgfqpoint{-0.020833in}{0.005525in}}{\pgfqpoint{-0.020833in}{0.000000in}}%
\pgfpathcurveto{\pgfqpoint{-0.020833in}{-0.005525in}}{\pgfqpoint{-0.018638in}{-0.010825in}}{\pgfqpoint{-0.014731in}{-0.014731in}}%
\pgfpathcurveto{\pgfqpoint{-0.010825in}{-0.018638in}}{\pgfqpoint{-0.005525in}{-0.020833in}}{\pgfqpoint{0.000000in}{-0.020833in}}%
\pgfpathlineto{\pgfqpoint{0.000000in}{-0.020833in}}%
\pgfpathclose%
\pgfusepath{stroke,fill}%
}%
\begin{pgfscope}%
\pgfsys@transformshift{1.073059in}{2.116964in}%
\pgfsys@useobject{currentmarker}{}%
\end{pgfscope}%
\begin{pgfscope}%
\pgfsys@transformshift{1.443094in}{1.863215in}%
\pgfsys@useobject{currentmarker}{}%
\end{pgfscope}%
\begin{pgfscope}%
\pgfsys@transformshift{1.794054in}{1.595523in}%
\pgfsys@useobject{currentmarker}{}%
\end{pgfscope}%
\begin{pgfscope}%
\pgfsys@transformshift{2.140209in}{1.379860in}%
\pgfsys@useobject{currentmarker}{}%
\end{pgfscope}%
\begin{pgfscope}%
\pgfsys@transformshift{2.485882in}{1.218567in}%
\pgfsys@useobject{currentmarker}{}%
\end{pgfscope}%
\begin{pgfscope}%
\pgfsys@transformshift{2.834422in}{1.166856in}%
\pgfsys@useobject{currentmarker}{}%
\end{pgfscope}%
\end{pgfscope}%
\begin{pgfscope}%
\pgfpathrectangle{\pgfqpoint{0.984991in}{0.549073in}}{\pgfqpoint{1.937500in}{1.925000in}}%
\pgfusepath{clip}%
\pgfsetrectcap%
\pgfsetroundjoin%
\pgfsetlinewidth{1.003750pt}%
\definecolor{currentstroke}{rgb}{0.184314,0.270588,0.360784}%
\pgfsetstrokecolor{currentstroke}%
\pgfsetdash{}{0pt}%
\pgfpathmoveto{\pgfqpoint{1.073059in}{1.866520in}}%
\pgfpathlineto{\pgfqpoint{1.443094in}{2.205571in}}%
\pgfpathlineto{\pgfqpoint{1.794054in}{2.212066in}}%
\pgfpathlineto{\pgfqpoint{2.140209in}{2.285590in}}%
\pgfpathlineto{\pgfqpoint{2.485882in}{2.288427in}}%
\pgfpathlineto{\pgfqpoint{2.834422in}{2.386573in}}%
\pgfusepath{stroke}%
\end{pgfscope}%
\begin{pgfscope}%
\pgfpathrectangle{\pgfqpoint{0.984991in}{0.549073in}}{\pgfqpoint{1.937500in}{1.925000in}}%
\pgfusepath{clip}%
\pgfsetbuttcap%
\pgfsetroundjoin%
\definecolor{currentfill}{rgb}{0.184314,0.270588,0.360784}%
\pgfsetfillcolor{currentfill}%
\pgfsetlinewidth{1.003750pt}%
\definecolor{currentstroke}{rgb}{0.184314,0.270588,0.360784}%
\pgfsetstrokecolor{currentstroke}%
\pgfsetdash{}{0pt}%
\pgfsys@defobject{currentmarker}{\pgfqpoint{-0.020833in}{-0.020833in}}{\pgfqpoint{0.020833in}{0.020833in}}{%
\pgfpathmoveto{\pgfqpoint{0.000000in}{-0.020833in}}%
\pgfpathcurveto{\pgfqpoint{0.005525in}{-0.020833in}}{\pgfqpoint{0.010825in}{-0.018638in}}{\pgfqpoint{0.014731in}{-0.014731in}}%
\pgfpathcurveto{\pgfqpoint{0.018638in}{-0.010825in}}{\pgfqpoint{0.020833in}{-0.005525in}}{\pgfqpoint{0.020833in}{0.000000in}}%
\pgfpathcurveto{\pgfqpoint{0.020833in}{0.005525in}}{\pgfqpoint{0.018638in}{0.010825in}}{\pgfqpoint{0.014731in}{0.014731in}}%
\pgfpathcurveto{\pgfqpoint{0.010825in}{0.018638in}}{\pgfqpoint{0.005525in}{0.020833in}}{\pgfqpoint{0.000000in}{0.020833in}}%
\pgfpathcurveto{\pgfqpoint{-0.005525in}{0.020833in}}{\pgfqpoint{-0.010825in}{0.018638in}}{\pgfqpoint{-0.014731in}{0.014731in}}%
\pgfpathcurveto{\pgfqpoint{-0.018638in}{0.010825in}}{\pgfqpoint{-0.020833in}{0.005525in}}{\pgfqpoint{-0.020833in}{0.000000in}}%
\pgfpathcurveto{\pgfqpoint{-0.020833in}{-0.005525in}}{\pgfqpoint{-0.018638in}{-0.010825in}}{\pgfqpoint{-0.014731in}{-0.014731in}}%
\pgfpathcurveto{\pgfqpoint{-0.010825in}{-0.018638in}}{\pgfqpoint{-0.005525in}{-0.020833in}}{\pgfqpoint{0.000000in}{-0.020833in}}%
\pgfpathlineto{\pgfqpoint{0.000000in}{-0.020833in}}%
\pgfpathclose%
\pgfusepath{stroke,fill}%
}%
\begin{pgfscope}%
\pgfsys@transformshift{1.073059in}{1.866520in}%
\pgfsys@useobject{currentmarker}{}%
\end{pgfscope}%
\begin{pgfscope}%
\pgfsys@transformshift{1.443094in}{2.205571in}%
\pgfsys@useobject{currentmarker}{}%
\end{pgfscope}%
\begin{pgfscope}%
\pgfsys@transformshift{1.794054in}{2.212066in}%
\pgfsys@useobject{currentmarker}{}%
\end{pgfscope}%
\begin{pgfscope}%
\pgfsys@transformshift{2.140209in}{2.285590in}%
\pgfsys@useobject{currentmarker}{}%
\end{pgfscope}%
\begin{pgfscope}%
\pgfsys@transformshift{2.485882in}{2.288427in}%
\pgfsys@useobject{currentmarker}{}%
\end{pgfscope}%
\begin{pgfscope}%
\pgfsys@transformshift{2.834422in}{2.386573in}%
\pgfsys@useobject{currentmarker}{}%
\end{pgfscope}%
\end{pgfscope}%
\begin{pgfscope}%
\pgfpathrectangle{\pgfqpoint{0.984991in}{0.549073in}}{\pgfqpoint{1.937500in}{1.925000in}}%
\pgfusepath{clip}%
\pgfsetrectcap%
\pgfsetroundjoin%
\pgfsetlinewidth{1.003750pt}%
\definecolor{currentstroke}{rgb}{0.976471,0.505882,0.145098}%
\pgfsetstrokecolor{currentstroke}%
\pgfsetdash{}{0pt}%
\pgfpathmoveto{\pgfqpoint{1.073059in}{1.018405in}}%
\pgfpathlineto{\pgfqpoint{1.443094in}{0.789327in}}%
\pgfpathlineto{\pgfqpoint{1.794054in}{0.636573in}}%
\pgfpathlineto{\pgfqpoint{2.140209in}{0.667503in}}%
\pgfpathlineto{\pgfqpoint{2.485882in}{0.720151in}}%
\pgfpathlineto{\pgfqpoint{2.834422in}{0.722717in}}%
\pgfusepath{stroke}%
\end{pgfscope}%
\begin{pgfscope}%
\pgfpathrectangle{\pgfqpoint{0.984991in}{0.549073in}}{\pgfqpoint{1.937500in}{1.925000in}}%
\pgfusepath{clip}%
\pgfsetbuttcap%
\pgfsetroundjoin%
\definecolor{currentfill}{rgb}{0.976471,0.505882,0.145098}%
\pgfsetfillcolor{currentfill}%
\pgfsetlinewidth{1.003750pt}%
\definecolor{currentstroke}{rgb}{0.976471,0.505882,0.145098}%
\pgfsetstrokecolor{currentstroke}%
\pgfsetdash{}{0pt}%
\pgfsys@defobject{currentmarker}{\pgfqpoint{-0.020833in}{-0.020833in}}{\pgfqpoint{0.020833in}{0.020833in}}{%
\pgfpathmoveto{\pgfqpoint{0.000000in}{-0.020833in}}%
\pgfpathcurveto{\pgfqpoint{0.005525in}{-0.020833in}}{\pgfqpoint{0.010825in}{-0.018638in}}{\pgfqpoint{0.014731in}{-0.014731in}}%
\pgfpathcurveto{\pgfqpoint{0.018638in}{-0.010825in}}{\pgfqpoint{0.020833in}{-0.005525in}}{\pgfqpoint{0.020833in}{0.000000in}}%
\pgfpathcurveto{\pgfqpoint{0.020833in}{0.005525in}}{\pgfqpoint{0.018638in}{0.010825in}}{\pgfqpoint{0.014731in}{0.014731in}}%
\pgfpathcurveto{\pgfqpoint{0.010825in}{0.018638in}}{\pgfqpoint{0.005525in}{0.020833in}}{\pgfqpoint{0.000000in}{0.020833in}}%
\pgfpathcurveto{\pgfqpoint{-0.005525in}{0.020833in}}{\pgfqpoint{-0.010825in}{0.018638in}}{\pgfqpoint{-0.014731in}{0.014731in}}%
\pgfpathcurveto{\pgfqpoint{-0.018638in}{0.010825in}}{\pgfqpoint{-0.020833in}{0.005525in}}{\pgfqpoint{-0.020833in}{0.000000in}}%
\pgfpathcurveto{\pgfqpoint{-0.020833in}{-0.005525in}}{\pgfqpoint{-0.018638in}{-0.010825in}}{\pgfqpoint{-0.014731in}{-0.014731in}}%
\pgfpathcurveto{\pgfqpoint{-0.010825in}{-0.018638in}}{\pgfqpoint{-0.005525in}{-0.020833in}}{\pgfqpoint{0.000000in}{-0.020833in}}%
\pgfpathlineto{\pgfqpoint{0.000000in}{-0.020833in}}%
\pgfpathclose%
\pgfusepath{stroke,fill}%
}%
\begin{pgfscope}%
\pgfsys@transformshift{1.073059in}{1.018405in}%
\pgfsys@useobject{currentmarker}{}%
\end{pgfscope}%
\begin{pgfscope}%
\pgfsys@transformshift{1.443094in}{0.789327in}%
\pgfsys@useobject{currentmarker}{}%
\end{pgfscope}%
\begin{pgfscope}%
\pgfsys@transformshift{1.794054in}{0.636573in}%
\pgfsys@useobject{currentmarker}{}%
\end{pgfscope}%
\begin{pgfscope}%
\pgfsys@transformshift{2.140209in}{0.667503in}%
\pgfsys@useobject{currentmarker}{}%
\end{pgfscope}%
\begin{pgfscope}%
\pgfsys@transformshift{2.485882in}{0.720151in}%
\pgfsys@useobject{currentmarker}{}%
\end{pgfscope}%
\begin{pgfscope}%
\pgfsys@transformshift{2.834422in}{0.722717in}%
\pgfsys@useobject{currentmarker}{}%
\end{pgfscope}%
\end{pgfscope}%
\begin{pgfscope}%
\pgfsetrectcap%
\pgfsetmiterjoin%
\pgfsetlinewidth{0.803000pt}%
\definecolor{currentstroke}{rgb}{0.000000,0.000000,0.000000}%
\pgfsetstrokecolor{currentstroke}%
\pgfsetdash{}{0pt}%
\pgfpathmoveto{\pgfqpoint{0.984991in}{0.549073in}}%
\pgfpathlineto{\pgfqpoint{0.984991in}{2.474073in}}%
\pgfusepath{stroke}%
\end{pgfscope}%
\begin{pgfscope}%
\pgfsetrectcap%
\pgfsetmiterjoin%
\pgfsetlinewidth{0.803000pt}%
\definecolor{currentstroke}{rgb}{0.000000,0.000000,0.000000}%
\pgfsetstrokecolor{currentstroke}%
\pgfsetdash{}{0pt}%
\pgfpathmoveto{\pgfqpoint{2.922491in}{0.549073in}}%
\pgfpathlineto{\pgfqpoint{2.922491in}{2.474073in}}%
\pgfusepath{stroke}%
\end{pgfscope}%
\begin{pgfscope}%
\pgfsetrectcap%
\pgfsetmiterjoin%
\pgfsetlinewidth{0.803000pt}%
\definecolor{currentstroke}{rgb}{0.000000,0.000000,0.000000}%
\pgfsetstrokecolor{currentstroke}%
\pgfsetdash{}{0pt}%
\pgfpathmoveto{\pgfqpoint{0.984991in}{0.549073in}}%
\pgfpathlineto{\pgfqpoint{2.922491in}{0.549073in}}%
\pgfusepath{stroke}%
\end{pgfscope}%
\begin{pgfscope}%
\pgfsetrectcap%
\pgfsetmiterjoin%
\pgfsetlinewidth{0.803000pt}%
\definecolor{currentstroke}{rgb}{0.000000,0.000000,0.000000}%
\pgfsetstrokecolor{currentstroke}%
\pgfsetdash{}{0pt}%
\pgfpathmoveto{\pgfqpoint{0.984991in}{2.474073in}}%
\pgfpathlineto{\pgfqpoint{2.922491in}{2.474073in}}%
\pgfusepath{stroke}%
\end{pgfscope}%
\begin{pgfscope}%
\pgfsetbuttcap%
\pgfsetmiterjoin%
\definecolor{currentfill}{rgb}{1.000000,1.000000,1.000000}%
\pgfsetfillcolor{currentfill}%
\pgfsetfillopacity{0.800000}%
\pgfsetlinewidth{1.003750pt}%
\definecolor{currentstroke}{rgb}{0.800000,0.800000,0.800000}%
\pgfsetstrokecolor{currentstroke}%
\pgfsetstrokeopacity{0.800000}%
\pgfsetdash{}{0pt}%
\pgfpathmoveto{\pgfqpoint{1.778437in}{1.643518in}}%
\pgfpathlineto{\pgfqpoint{2.805824in}{1.643518in}}%
\pgfpathquadraticcurveto{\pgfqpoint{2.839157in}{1.643518in}}{\pgfqpoint{2.839157in}{1.676852in}}%
\pgfpathlineto{\pgfqpoint{2.839157in}{2.357406in}}%
\pgfpathquadraticcurveto{\pgfqpoint{2.839157in}{2.390739in}}{\pgfqpoint{2.805824in}{2.390739in}}%
\pgfpathlineto{\pgfqpoint{1.778437in}{2.390739in}}%
\pgfpathquadraticcurveto{\pgfqpoint{1.745104in}{2.390739in}}{\pgfqpoint{1.745104in}{2.357406in}}%
\pgfpathlineto{\pgfqpoint{1.745104in}{1.676852in}}%
\pgfpathquadraticcurveto{\pgfqpoint{1.745104in}{1.643518in}}{\pgfqpoint{1.778437in}{1.643518in}}%
\pgfpathlineto{\pgfqpoint{1.778437in}{1.643518in}}%
\pgfpathclose%
\pgfusepath{stroke,fill}%
\end{pgfscope}%
\begin{pgfscope}%
\pgfsetrectcap%
\pgfsetroundjoin%
\pgfsetlinewidth{1.003750pt}%
\definecolor{currentstroke}{rgb}{0.537255,0.647059,0.760784}%
\pgfsetstrokecolor{currentstroke}%
\pgfsetdash{}{0pt}%
\pgfpathmoveto{\pgfqpoint{1.811770in}{2.265739in}}%
\pgfpathlineto{\pgfqpoint{1.978437in}{2.265739in}}%
\pgfpathlineto{\pgfqpoint{2.145104in}{2.265739in}}%
\pgfusepath{stroke}%
\end{pgfscope}%
\begin{pgfscope}%
\pgfsetbuttcap%
\pgfsetroundjoin%
\definecolor{currentfill}{rgb}{0.537255,0.647059,0.760784}%
\pgfsetfillcolor{currentfill}%
\pgfsetlinewidth{1.003750pt}%
\definecolor{currentstroke}{rgb}{0.537255,0.647059,0.760784}%
\pgfsetstrokecolor{currentstroke}%
\pgfsetdash{}{0pt}%
\pgfsys@defobject{currentmarker}{\pgfqpoint{-0.020833in}{-0.020833in}}{\pgfqpoint{0.020833in}{0.020833in}}{%
\pgfpathmoveto{\pgfqpoint{0.000000in}{-0.020833in}}%
\pgfpathcurveto{\pgfqpoint{0.005525in}{-0.020833in}}{\pgfqpoint{0.010825in}{-0.018638in}}{\pgfqpoint{0.014731in}{-0.014731in}}%
\pgfpathcurveto{\pgfqpoint{0.018638in}{-0.010825in}}{\pgfqpoint{0.020833in}{-0.005525in}}{\pgfqpoint{0.020833in}{0.000000in}}%
\pgfpathcurveto{\pgfqpoint{0.020833in}{0.005525in}}{\pgfqpoint{0.018638in}{0.010825in}}{\pgfqpoint{0.014731in}{0.014731in}}%
\pgfpathcurveto{\pgfqpoint{0.010825in}{0.018638in}}{\pgfqpoint{0.005525in}{0.020833in}}{\pgfqpoint{0.000000in}{0.020833in}}%
\pgfpathcurveto{\pgfqpoint{-0.005525in}{0.020833in}}{\pgfqpoint{-0.010825in}{0.018638in}}{\pgfqpoint{-0.014731in}{0.014731in}}%
\pgfpathcurveto{\pgfqpoint{-0.018638in}{0.010825in}}{\pgfqpoint{-0.020833in}{0.005525in}}{\pgfqpoint{-0.020833in}{0.000000in}}%
\pgfpathcurveto{\pgfqpoint{-0.020833in}{-0.005525in}}{\pgfqpoint{-0.018638in}{-0.010825in}}{\pgfqpoint{-0.014731in}{-0.014731in}}%
\pgfpathcurveto{\pgfqpoint{-0.010825in}{-0.018638in}}{\pgfqpoint{-0.005525in}{-0.020833in}}{\pgfqpoint{0.000000in}{-0.020833in}}%
\pgfpathlineto{\pgfqpoint{0.000000in}{-0.020833in}}%
\pgfpathclose%
\pgfusepath{stroke,fill}%
}%
\begin{pgfscope}%
\pgfsys@transformshift{1.978437in}{2.265739in}%
\pgfsys@useobject{currentmarker}{}%
\end{pgfscope}%
\end{pgfscope}%
\begin{pgfscope}%
\definecolor{textcolor}{rgb}{0.000000,0.000000,0.000000}%
\pgfsetstrokecolor{textcolor}%
\pgfsetfillcolor{textcolor}%
\pgftext[x=2.278437in,y=2.207406in,left,base]{\color{textcolor}{\rmfamily\fontsize{12.000000}{14.400000}\selectfont\catcode`\^=\active\def^{\ifmmode\sp\else\^{}\fi}\catcode`\%=\active\def%{\%}DGC}}%
\end{pgfscope}%
\begin{pgfscope}%
\pgfsetrectcap%
\pgfsetroundjoin%
\pgfsetlinewidth{1.003750pt}%
\definecolor{currentstroke}{rgb}{0.184314,0.270588,0.360784}%
\pgfsetstrokecolor{currentstroke}%
\pgfsetdash{}{0pt}%
\pgfpathmoveto{\pgfqpoint{1.811770in}{2.033332in}}%
\pgfpathlineto{\pgfqpoint{1.978437in}{2.033332in}}%
\pgfpathlineto{\pgfqpoint{2.145104in}{2.033332in}}%
\pgfusepath{stroke}%
\end{pgfscope}%
\begin{pgfscope}%
\pgfsetbuttcap%
\pgfsetroundjoin%
\definecolor{currentfill}{rgb}{0.184314,0.270588,0.360784}%
\pgfsetfillcolor{currentfill}%
\pgfsetlinewidth{1.003750pt}%
\definecolor{currentstroke}{rgb}{0.184314,0.270588,0.360784}%
\pgfsetstrokecolor{currentstroke}%
\pgfsetdash{}{0pt}%
\pgfsys@defobject{currentmarker}{\pgfqpoint{-0.020833in}{-0.020833in}}{\pgfqpoint{0.020833in}{0.020833in}}{%
\pgfpathmoveto{\pgfqpoint{0.000000in}{-0.020833in}}%
\pgfpathcurveto{\pgfqpoint{0.005525in}{-0.020833in}}{\pgfqpoint{0.010825in}{-0.018638in}}{\pgfqpoint{0.014731in}{-0.014731in}}%
\pgfpathcurveto{\pgfqpoint{0.018638in}{-0.010825in}}{\pgfqpoint{0.020833in}{-0.005525in}}{\pgfqpoint{0.020833in}{0.000000in}}%
\pgfpathcurveto{\pgfqpoint{0.020833in}{0.005525in}}{\pgfqpoint{0.018638in}{0.010825in}}{\pgfqpoint{0.014731in}{0.014731in}}%
\pgfpathcurveto{\pgfqpoint{0.010825in}{0.018638in}}{\pgfqpoint{0.005525in}{0.020833in}}{\pgfqpoint{0.000000in}{0.020833in}}%
\pgfpathcurveto{\pgfqpoint{-0.005525in}{0.020833in}}{\pgfqpoint{-0.010825in}{0.018638in}}{\pgfqpoint{-0.014731in}{0.014731in}}%
\pgfpathcurveto{\pgfqpoint{-0.018638in}{0.010825in}}{\pgfqpoint{-0.020833in}{0.005525in}}{\pgfqpoint{-0.020833in}{0.000000in}}%
\pgfpathcurveto{\pgfqpoint{-0.020833in}{-0.005525in}}{\pgfqpoint{-0.018638in}{-0.010825in}}{\pgfqpoint{-0.014731in}{-0.014731in}}%
\pgfpathcurveto{\pgfqpoint{-0.010825in}{-0.018638in}}{\pgfqpoint{-0.005525in}{-0.020833in}}{\pgfqpoint{0.000000in}{-0.020833in}}%
\pgfpathlineto{\pgfqpoint{0.000000in}{-0.020833in}}%
\pgfpathclose%
\pgfusepath{stroke,fill}%
}%
\begin{pgfscope}%
\pgfsys@transformshift{1.978437in}{2.033332in}%
\pgfsys@useobject{currentmarker}{}%
\end{pgfscope}%
\end{pgfscope}%
\begin{pgfscope}%
\definecolor{textcolor}{rgb}{0.000000,0.000000,0.000000}%
\pgfsetstrokecolor{textcolor}%
\pgfsetfillcolor{textcolor}%
\pgftext[x=2.278437in,y=1.974999in,left,base]{\color{textcolor}{\rmfamily\fontsize{12.000000}{14.400000}\selectfont\catcode`\^=\active\def^{\ifmmode\sp\else\^{}\fi}\catcode`\%=\active\def%{\%}NC}}%
\end{pgfscope}%
\begin{pgfscope}%
\pgfsetrectcap%
\pgfsetroundjoin%
\pgfsetlinewidth{1.003750pt}%
\definecolor{currentstroke}{rgb}{0.976471,0.505882,0.145098}%
\pgfsetstrokecolor{currentstroke}%
\pgfsetdash{}{0pt}%
\pgfpathmoveto{\pgfqpoint{1.811770in}{1.800925in}}%
\pgfpathlineto{\pgfqpoint{1.978437in}{1.800925in}}%
\pgfpathlineto{\pgfqpoint{2.145104in}{1.800925in}}%
\pgfusepath{stroke}%
\end{pgfscope}%
\begin{pgfscope}%
\pgfsetbuttcap%
\pgfsetroundjoin%
\definecolor{currentfill}{rgb}{0.976471,0.505882,0.145098}%
\pgfsetfillcolor{currentfill}%
\pgfsetlinewidth{1.003750pt}%
\definecolor{currentstroke}{rgb}{0.976471,0.505882,0.145098}%
\pgfsetstrokecolor{currentstroke}%
\pgfsetdash{}{0pt}%
\pgfsys@defobject{currentmarker}{\pgfqpoint{-0.020833in}{-0.020833in}}{\pgfqpoint{0.020833in}{0.020833in}}{%
\pgfpathmoveto{\pgfqpoint{0.000000in}{-0.020833in}}%
\pgfpathcurveto{\pgfqpoint{0.005525in}{-0.020833in}}{\pgfqpoint{0.010825in}{-0.018638in}}{\pgfqpoint{0.014731in}{-0.014731in}}%
\pgfpathcurveto{\pgfqpoint{0.018638in}{-0.010825in}}{\pgfqpoint{0.020833in}{-0.005525in}}{\pgfqpoint{0.020833in}{0.000000in}}%
\pgfpathcurveto{\pgfqpoint{0.020833in}{0.005525in}}{\pgfqpoint{0.018638in}{0.010825in}}{\pgfqpoint{0.014731in}{0.014731in}}%
\pgfpathcurveto{\pgfqpoint{0.010825in}{0.018638in}}{\pgfqpoint{0.005525in}{0.020833in}}{\pgfqpoint{0.000000in}{0.020833in}}%
\pgfpathcurveto{\pgfqpoint{-0.005525in}{0.020833in}}{\pgfqpoint{-0.010825in}{0.018638in}}{\pgfqpoint{-0.014731in}{0.014731in}}%
\pgfpathcurveto{\pgfqpoint{-0.018638in}{0.010825in}}{\pgfqpoint{-0.020833in}{0.005525in}}{\pgfqpoint{-0.020833in}{0.000000in}}%
\pgfpathcurveto{\pgfqpoint{-0.020833in}{-0.005525in}}{\pgfqpoint{-0.018638in}{-0.010825in}}{\pgfqpoint{-0.014731in}{-0.014731in}}%
\pgfpathcurveto{\pgfqpoint{-0.010825in}{-0.018638in}}{\pgfqpoint{-0.005525in}{-0.020833in}}{\pgfqpoint{0.000000in}{-0.020833in}}%
\pgfpathlineto{\pgfqpoint{0.000000in}{-0.020833in}}%
\pgfpathclose%
\pgfusepath{stroke,fill}%
}%
\begin{pgfscope}%
\pgfsys@transformshift{1.978437in}{1.800925in}%
\pgfsys@useobject{currentmarker}{}%
\end{pgfscope}%
\end{pgfscope}%
\begin{pgfscope}%
\definecolor{textcolor}{rgb}{0.000000,0.000000,0.000000}%
\pgfsetstrokecolor{textcolor}%
\pgfsetfillcolor{textcolor}%
\pgftext[x=2.278437in,y=1.742592in,left,base]{\color{textcolor}{\rmfamily\fontsize{12.000000}{14.400000}\selectfont\catcode`\^=\active\def^{\ifmmode\sp\else\^{}\fi}\catcode`\%=\active\def%{\%}NC++}}%
\end{pgfscope}%
\end{pgfpicture}%
\makeatother%
\endgroup%

        \caption{GOE}
        \label{fig:5-experiments-multi-matrix-convergence-goe}
    \end{subfigure}
    \begin{subfigure}[b]{0.49\columnwidth}
        %% Creator: Matplotlib, PGF backend
%%
%% To include the figure in your LaTeX document, write
%%   \input{<filename>.pgf}
%%
%% Make sure the required packages are loaded in your preamble
%%   \usepackage{pgf}
%%
%% Also ensure that all the required font packages are loaded; for instance,
%% the lmodern package is sometimes necessary when using math font.
%%   \usepackage{lmodern}
%%
%% Figures using additional raster images can only be included by \input if
%% they are in the same directory as the main LaTeX file. For loading figures
%% from other directories you can use the `import` package
%%   \usepackage{import}
%%
%% and then include the figures with
%%   \import{<path to file>}{<filename>.pgf}
%%
%% Matplotlib used the following preamble
%%   \def\mathdefault#1{#1}
%%   \everymath=\expandafter{\the\everymath\displaystyle}
%%   
%%   \makeatletter\@ifpackageloaded{underscore}{}{\usepackage[strings]{underscore}}\makeatother
%%
\begingroup%
\makeatletter%
\begin{pgfpicture}%
\pgfpathrectangle{\pgfpointorigin}{\pgfqpoint{2.759413in}{2.586282in}}%
\pgfusepath{use as bounding box, clip}%
\begin{pgfscope}%
\pgfsetbuttcap%
\pgfsetmiterjoin%
\definecolor{currentfill}{rgb}{1.000000,1.000000,1.000000}%
\pgfsetfillcolor{currentfill}%
\pgfsetlinewidth{0.000000pt}%
\definecolor{currentstroke}{rgb}{1.000000,1.000000,1.000000}%
\pgfsetstrokecolor{currentstroke}%
\pgfsetdash{}{0pt}%
\pgfpathmoveto{\pgfqpoint{0.000000in}{0.000000in}}%
\pgfpathlineto{\pgfqpoint{2.759413in}{0.000000in}}%
\pgfpathlineto{\pgfqpoint{2.759413in}{2.586282in}}%
\pgfpathlineto{\pgfqpoint{0.000000in}{2.586282in}}%
\pgfpathlineto{\pgfqpoint{0.000000in}{0.000000in}}%
\pgfpathclose%
\pgfusepath{fill}%
\end{pgfscope}%
\begin{pgfscope}%
\pgfsetbuttcap%
\pgfsetmiterjoin%
\definecolor{currentfill}{rgb}{1.000000,1.000000,1.000000}%
\pgfsetfillcolor{currentfill}%
\pgfsetlinewidth{0.000000pt}%
\definecolor{currentstroke}{rgb}{0.000000,0.000000,0.000000}%
\pgfsetstrokecolor{currentstroke}%
\pgfsetstrokeopacity{0.000000}%
\pgfsetdash{}{0pt}%
\pgfpathmoveto{\pgfqpoint{0.721913in}{0.549073in}}%
\pgfpathlineto{\pgfqpoint{2.659413in}{0.549073in}}%
\pgfpathlineto{\pgfqpoint{2.659413in}{2.474073in}}%
\pgfpathlineto{\pgfqpoint{0.721913in}{2.474073in}}%
\pgfpathlineto{\pgfqpoint{0.721913in}{0.549073in}}%
\pgfpathclose%
\pgfusepath{fill}%
\end{pgfscope}%
\begin{pgfscope}%
\pgfsetbuttcap%
\pgfsetroundjoin%
\definecolor{currentfill}{rgb}{0.000000,0.000000,0.000000}%
\pgfsetfillcolor{currentfill}%
\pgfsetlinewidth{0.803000pt}%
\definecolor{currentstroke}{rgb}{0.000000,0.000000,0.000000}%
\pgfsetstrokecolor{currentstroke}%
\pgfsetdash{}{0pt}%
\pgfsys@defobject{currentmarker}{\pgfqpoint{0.000000in}{-0.048611in}}{\pgfqpoint{0.000000in}{0.000000in}}{%
\pgfpathmoveto{\pgfqpoint{0.000000in}{0.000000in}}%
\pgfpathlineto{\pgfqpoint{0.000000in}{-0.048611in}}%
\pgfusepath{stroke,fill}%
}%
\begin{pgfscope}%
\pgfsys@transformshift{1.771566in}{0.549073in}%
\pgfsys@useobject{currentmarker}{}%
\end{pgfscope}%
\end{pgfscope}%
\begin{pgfscope}%
\definecolor{textcolor}{rgb}{0.000000,0.000000,0.000000}%
\pgfsetstrokecolor{textcolor}%
\pgfsetfillcolor{textcolor}%
\pgftext[x=1.771566in,y=0.451851in,,top]{\color{textcolor}{\rmfamily\fontsize{12.000000}{14.400000}\selectfont\catcode`\^=\active\def^{\ifmmode\sp\else\^{}\fi}\catcode`\%=\active\def%{\%}$\mathdefault{10^{2}}$}}%
\end{pgfscope}%
\begin{pgfscope}%
\pgfsetbuttcap%
\pgfsetroundjoin%
\definecolor{currentfill}{rgb}{0.000000,0.000000,0.000000}%
\pgfsetfillcolor{currentfill}%
\pgfsetlinewidth{0.602250pt}%
\definecolor{currentstroke}{rgb}{0.000000,0.000000,0.000000}%
\pgfsetstrokecolor{currentstroke}%
\pgfsetdash{}{0pt}%
\pgfsys@defobject{currentmarker}{\pgfqpoint{0.000000in}{-0.027778in}}{\pgfqpoint{0.000000in}{0.000000in}}{%
\pgfpathmoveto{\pgfqpoint{0.000000in}{0.000000in}}%
\pgfpathlineto{\pgfqpoint{0.000000in}{-0.027778in}}%
\pgfusepath{stroke,fill}%
}%
\begin{pgfscope}%
\pgfsys@transformshift{0.839681in}{0.549073in}%
\pgfsys@useobject{currentmarker}{}%
\end{pgfscope}%
\end{pgfscope}%
\begin{pgfscope}%
\pgfsetbuttcap%
\pgfsetroundjoin%
\definecolor{currentfill}{rgb}{0.000000,0.000000,0.000000}%
\pgfsetfillcolor{currentfill}%
\pgfsetlinewidth{0.602250pt}%
\definecolor{currentstroke}{rgb}{0.000000,0.000000,0.000000}%
\pgfsetstrokecolor{currentstroke}%
\pgfsetdash{}{0pt}%
\pgfsys@defobject{currentmarker}{\pgfqpoint{0.000000in}{-0.027778in}}{\pgfqpoint{0.000000in}{0.000000in}}{%
\pgfpathmoveto{\pgfqpoint{0.000000in}{0.000000in}}%
\pgfpathlineto{\pgfqpoint{0.000000in}{-0.027778in}}%
\pgfusepath{stroke,fill}%
}%
\begin{pgfscope}%
\pgfsys@transformshift{1.074450in}{0.549073in}%
\pgfsys@useobject{currentmarker}{}%
\end{pgfscope}%
\end{pgfscope}%
\begin{pgfscope}%
\pgfsetbuttcap%
\pgfsetroundjoin%
\definecolor{currentfill}{rgb}{0.000000,0.000000,0.000000}%
\pgfsetfillcolor{currentfill}%
\pgfsetlinewidth{0.602250pt}%
\definecolor{currentstroke}{rgb}{0.000000,0.000000,0.000000}%
\pgfsetstrokecolor{currentstroke}%
\pgfsetdash{}{0pt}%
\pgfsys@defobject{currentmarker}{\pgfqpoint{0.000000in}{-0.027778in}}{\pgfqpoint{0.000000in}{0.000000in}}{%
\pgfpathmoveto{\pgfqpoint{0.000000in}{0.000000in}}%
\pgfpathlineto{\pgfqpoint{0.000000in}{-0.027778in}}%
\pgfusepath{stroke,fill}%
}%
\begin{pgfscope}%
\pgfsys@transformshift{1.241022in}{0.549073in}%
\pgfsys@useobject{currentmarker}{}%
\end{pgfscope}%
\end{pgfscope}%
\begin{pgfscope}%
\pgfsetbuttcap%
\pgfsetroundjoin%
\definecolor{currentfill}{rgb}{0.000000,0.000000,0.000000}%
\pgfsetfillcolor{currentfill}%
\pgfsetlinewidth{0.602250pt}%
\definecolor{currentstroke}{rgb}{0.000000,0.000000,0.000000}%
\pgfsetstrokecolor{currentstroke}%
\pgfsetdash{}{0pt}%
\pgfsys@defobject{currentmarker}{\pgfqpoint{0.000000in}{-0.027778in}}{\pgfqpoint{0.000000in}{0.000000in}}{%
\pgfpathmoveto{\pgfqpoint{0.000000in}{0.000000in}}%
\pgfpathlineto{\pgfqpoint{0.000000in}{-0.027778in}}%
\pgfusepath{stroke,fill}%
}%
\begin{pgfscope}%
\pgfsys@transformshift{1.370225in}{0.549073in}%
\pgfsys@useobject{currentmarker}{}%
\end{pgfscope}%
\end{pgfscope}%
\begin{pgfscope}%
\pgfsetbuttcap%
\pgfsetroundjoin%
\definecolor{currentfill}{rgb}{0.000000,0.000000,0.000000}%
\pgfsetfillcolor{currentfill}%
\pgfsetlinewidth{0.602250pt}%
\definecolor{currentstroke}{rgb}{0.000000,0.000000,0.000000}%
\pgfsetstrokecolor{currentstroke}%
\pgfsetdash{}{0pt}%
\pgfsys@defobject{currentmarker}{\pgfqpoint{0.000000in}{-0.027778in}}{\pgfqpoint{0.000000in}{0.000000in}}{%
\pgfpathmoveto{\pgfqpoint{0.000000in}{0.000000in}}%
\pgfpathlineto{\pgfqpoint{0.000000in}{-0.027778in}}%
\pgfusepath{stroke,fill}%
}%
\begin{pgfscope}%
\pgfsys@transformshift{1.475791in}{0.549073in}%
\pgfsys@useobject{currentmarker}{}%
\end{pgfscope}%
\end{pgfscope}%
\begin{pgfscope}%
\pgfsetbuttcap%
\pgfsetroundjoin%
\definecolor{currentfill}{rgb}{0.000000,0.000000,0.000000}%
\pgfsetfillcolor{currentfill}%
\pgfsetlinewidth{0.602250pt}%
\definecolor{currentstroke}{rgb}{0.000000,0.000000,0.000000}%
\pgfsetstrokecolor{currentstroke}%
\pgfsetdash{}{0pt}%
\pgfsys@defobject{currentmarker}{\pgfqpoint{0.000000in}{-0.027778in}}{\pgfqpoint{0.000000in}{0.000000in}}{%
\pgfpathmoveto{\pgfqpoint{0.000000in}{0.000000in}}%
\pgfpathlineto{\pgfqpoint{0.000000in}{-0.027778in}}%
\pgfusepath{stroke,fill}%
}%
\begin{pgfscope}%
\pgfsys@transformshift{1.565047in}{0.549073in}%
\pgfsys@useobject{currentmarker}{}%
\end{pgfscope}%
\end{pgfscope}%
\begin{pgfscope}%
\pgfsetbuttcap%
\pgfsetroundjoin%
\definecolor{currentfill}{rgb}{0.000000,0.000000,0.000000}%
\pgfsetfillcolor{currentfill}%
\pgfsetlinewidth{0.602250pt}%
\definecolor{currentstroke}{rgb}{0.000000,0.000000,0.000000}%
\pgfsetstrokecolor{currentstroke}%
\pgfsetdash{}{0pt}%
\pgfsys@defobject{currentmarker}{\pgfqpoint{0.000000in}{-0.027778in}}{\pgfqpoint{0.000000in}{0.000000in}}{%
\pgfpathmoveto{\pgfqpoint{0.000000in}{0.000000in}}%
\pgfpathlineto{\pgfqpoint{0.000000in}{-0.027778in}}%
\pgfusepath{stroke,fill}%
}%
\begin{pgfscope}%
\pgfsys@transformshift{1.642363in}{0.549073in}%
\pgfsys@useobject{currentmarker}{}%
\end{pgfscope}%
\end{pgfscope}%
\begin{pgfscope}%
\pgfsetbuttcap%
\pgfsetroundjoin%
\definecolor{currentfill}{rgb}{0.000000,0.000000,0.000000}%
\pgfsetfillcolor{currentfill}%
\pgfsetlinewidth{0.602250pt}%
\definecolor{currentstroke}{rgb}{0.000000,0.000000,0.000000}%
\pgfsetstrokecolor{currentstroke}%
\pgfsetdash{}{0pt}%
\pgfsys@defobject{currentmarker}{\pgfqpoint{0.000000in}{-0.027778in}}{\pgfqpoint{0.000000in}{0.000000in}}{%
\pgfpathmoveto{\pgfqpoint{0.000000in}{0.000000in}}%
\pgfpathlineto{\pgfqpoint{0.000000in}{-0.027778in}}%
\pgfusepath{stroke,fill}%
}%
\begin{pgfscope}%
\pgfsys@transformshift{1.710561in}{0.549073in}%
\pgfsys@useobject{currentmarker}{}%
\end{pgfscope}%
\end{pgfscope}%
\begin{pgfscope}%
\pgfsetbuttcap%
\pgfsetroundjoin%
\definecolor{currentfill}{rgb}{0.000000,0.000000,0.000000}%
\pgfsetfillcolor{currentfill}%
\pgfsetlinewidth{0.602250pt}%
\definecolor{currentstroke}{rgb}{0.000000,0.000000,0.000000}%
\pgfsetstrokecolor{currentstroke}%
\pgfsetdash{}{0pt}%
\pgfsys@defobject{currentmarker}{\pgfqpoint{0.000000in}{-0.027778in}}{\pgfqpoint{0.000000in}{0.000000in}}{%
\pgfpathmoveto{\pgfqpoint{0.000000in}{0.000000in}}%
\pgfpathlineto{\pgfqpoint{0.000000in}{-0.027778in}}%
\pgfusepath{stroke,fill}%
}%
\begin{pgfscope}%
\pgfsys@transformshift{2.172907in}{0.549073in}%
\pgfsys@useobject{currentmarker}{}%
\end{pgfscope}%
\end{pgfscope}%
\begin{pgfscope}%
\pgfsetbuttcap%
\pgfsetroundjoin%
\definecolor{currentfill}{rgb}{0.000000,0.000000,0.000000}%
\pgfsetfillcolor{currentfill}%
\pgfsetlinewidth{0.602250pt}%
\definecolor{currentstroke}{rgb}{0.000000,0.000000,0.000000}%
\pgfsetstrokecolor{currentstroke}%
\pgfsetdash{}{0pt}%
\pgfsys@defobject{currentmarker}{\pgfqpoint{0.000000in}{-0.027778in}}{\pgfqpoint{0.000000in}{0.000000in}}{%
\pgfpathmoveto{\pgfqpoint{0.000000in}{0.000000in}}%
\pgfpathlineto{\pgfqpoint{0.000000in}{-0.027778in}}%
\pgfusepath{stroke,fill}%
}%
\begin{pgfscope}%
\pgfsys@transformshift{2.407676in}{0.549073in}%
\pgfsys@useobject{currentmarker}{}%
\end{pgfscope}%
\end{pgfscope}%
\begin{pgfscope}%
\pgfsetbuttcap%
\pgfsetroundjoin%
\definecolor{currentfill}{rgb}{0.000000,0.000000,0.000000}%
\pgfsetfillcolor{currentfill}%
\pgfsetlinewidth{0.602250pt}%
\definecolor{currentstroke}{rgb}{0.000000,0.000000,0.000000}%
\pgfsetstrokecolor{currentstroke}%
\pgfsetdash{}{0pt}%
\pgfsys@defobject{currentmarker}{\pgfqpoint{0.000000in}{-0.027778in}}{\pgfqpoint{0.000000in}{0.000000in}}{%
\pgfpathmoveto{\pgfqpoint{0.000000in}{0.000000in}}%
\pgfpathlineto{\pgfqpoint{0.000000in}{-0.027778in}}%
\pgfusepath{stroke,fill}%
}%
\begin{pgfscope}%
\pgfsys@transformshift{2.574248in}{0.549073in}%
\pgfsys@useobject{currentmarker}{}%
\end{pgfscope}%
\end{pgfscope}%
\begin{pgfscope}%
\definecolor{textcolor}{rgb}{0.000000,0.000000,0.000000}%
\pgfsetstrokecolor{textcolor}%
\pgfsetfillcolor{textcolor}%
\pgftext[x=1.690663in,y=0.248148in,,top]{\color{textcolor}{\rmfamily\fontsize{12.000000}{14.400000}\selectfont\catcode`\^=\active\def^{\ifmmode\sp\else\^{}\fi}\catcode`\%=\active\def%{\%}$n_{\Omega} + n_{\Psi}$}}%
\end{pgfscope}%
\begin{pgfscope}%
\pgfsetbuttcap%
\pgfsetroundjoin%
\definecolor{currentfill}{rgb}{0.000000,0.000000,0.000000}%
\pgfsetfillcolor{currentfill}%
\pgfsetlinewidth{0.803000pt}%
\definecolor{currentstroke}{rgb}{0.000000,0.000000,0.000000}%
\pgfsetstrokecolor{currentstroke}%
\pgfsetdash{}{0pt}%
\pgfsys@defobject{currentmarker}{\pgfqpoint{-0.048611in}{0.000000in}}{\pgfqpoint{-0.000000in}{0.000000in}}{%
\pgfpathmoveto{\pgfqpoint{-0.000000in}{0.000000in}}%
\pgfpathlineto{\pgfqpoint{-0.048611in}{0.000000in}}%
\pgfusepath{stroke,fill}%
}%
\begin{pgfscope}%
\pgfsys@transformshift{0.721913in}{0.913333in}%
\pgfsys@useobject{currentmarker}{}%
\end{pgfscope}%
\end{pgfscope}%
\begin{pgfscope}%
\definecolor{textcolor}{rgb}{0.000000,0.000000,0.000000}%
\pgfsetstrokecolor{textcolor}%
\pgfsetfillcolor{textcolor}%
\pgftext[x=0.303703in, y=0.855462in, left, base]{\color{textcolor}{\rmfamily\fontsize{12.000000}{14.400000}\selectfont\catcode`\^=\active\def^{\ifmmode\sp\else\^{}\fi}\catcode`\%=\active\def%{\%}$\mathdefault{10^{-6}}$}}%
\end{pgfscope}%
\begin{pgfscope}%
\pgfsetbuttcap%
\pgfsetroundjoin%
\definecolor{currentfill}{rgb}{0.000000,0.000000,0.000000}%
\pgfsetfillcolor{currentfill}%
\pgfsetlinewidth{0.803000pt}%
\definecolor{currentstroke}{rgb}{0.000000,0.000000,0.000000}%
\pgfsetstrokecolor{currentstroke}%
\pgfsetdash{}{0pt}%
\pgfsys@defobject{currentmarker}{\pgfqpoint{-0.048611in}{0.000000in}}{\pgfqpoint{-0.000000in}{0.000000in}}{%
\pgfpathmoveto{\pgfqpoint{-0.000000in}{0.000000in}}%
\pgfpathlineto{\pgfqpoint{-0.048611in}{0.000000in}}%
\pgfusepath{stroke,fill}%
}%
\begin{pgfscope}%
\pgfsys@transformshift{0.721913in}{1.418359in}%
\pgfsys@useobject{currentmarker}{}%
\end{pgfscope}%
\end{pgfscope}%
\begin{pgfscope}%
\definecolor{textcolor}{rgb}{0.000000,0.000000,0.000000}%
\pgfsetstrokecolor{textcolor}%
\pgfsetfillcolor{textcolor}%
\pgftext[x=0.303703in, y=1.360489in, left, base]{\color{textcolor}{\rmfamily\fontsize{12.000000}{14.400000}\selectfont\catcode`\^=\active\def^{\ifmmode\sp\else\^{}\fi}\catcode`\%=\active\def%{\%}$\mathdefault{10^{-4}}$}}%
\end{pgfscope}%
\begin{pgfscope}%
\pgfsetbuttcap%
\pgfsetroundjoin%
\definecolor{currentfill}{rgb}{0.000000,0.000000,0.000000}%
\pgfsetfillcolor{currentfill}%
\pgfsetlinewidth{0.803000pt}%
\definecolor{currentstroke}{rgb}{0.000000,0.000000,0.000000}%
\pgfsetstrokecolor{currentstroke}%
\pgfsetdash{}{0pt}%
\pgfsys@defobject{currentmarker}{\pgfqpoint{-0.048611in}{0.000000in}}{\pgfqpoint{-0.000000in}{0.000000in}}{%
\pgfpathmoveto{\pgfqpoint{-0.000000in}{0.000000in}}%
\pgfpathlineto{\pgfqpoint{-0.048611in}{0.000000in}}%
\pgfusepath{stroke,fill}%
}%
\begin{pgfscope}%
\pgfsys@transformshift{0.721913in}{1.923385in}%
\pgfsys@useobject{currentmarker}{}%
\end{pgfscope}%
\end{pgfscope}%
\begin{pgfscope}%
\definecolor{textcolor}{rgb}{0.000000,0.000000,0.000000}%
\pgfsetstrokecolor{textcolor}%
\pgfsetfillcolor{textcolor}%
\pgftext[x=0.303703in, y=1.865515in, left, base]{\color{textcolor}{\rmfamily\fontsize{12.000000}{14.400000}\selectfont\catcode`\^=\active\def^{\ifmmode\sp\else\^{}\fi}\catcode`\%=\active\def%{\%}$\mathdefault{10^{-2}}$}}%
\end{pgfscope}%
\begin{pgfscope}%
\pgfsetbuttcap%
\pgfsetroundjoin%
\definecolor{currentfill}{rgb}{0.000000,0.000000,0.000000}%
\pgfsetfillcolor{currentfill}%
\pgfsetlinewidth{0.803000pt}%
\definecolor{currentstroke}{rgb}{0.000000,0.000000,0.000000}%
\pgfsetstrokecolor{currentstroke}%
\pgfsetdash{}{0pt}%
\pgfsys@defobject{currentmarker}{\pgfqpoint{-0.048611in}{0.000000in}}{\pgfqpoint{-0.000000in}{0.000000in}}{%
\pgfpathmoveto{\pgfqpoint{-0.000000in}{0.000000in}}%
\pgfpathlineto{\pgfqpoint{-0.048611in}{0.000000in}}%
\pgfusepath{stroke,fill}%
}%
\begin{pgfscope}%
\pgfsys@transformshift{0.721913in}{2.428411in}%
\pgfsys@useobject{currentmarker}{}%
\end{pgfscope}%
\end{pgfscope}%
\begin{pgfscope}%
\definecolor{textcolor}{rgb}{0.000000,0.000000,0.000000}%
\pgfsetstrokecolor{textcolor}%
\pgfsetfillcolor{textcolor}%
\pgftext[x=0.395525in, y=2.370541in, left, base]{\color{textcolor}{\rmfamily\fontsize{12.000000}{14.400000}\selectfont\catcode`\^=\active\def^{\ifmmode\sp\else\^{}\fi}\catcode`\%=\active\def%{\%}$\mathdefault{10^{0}}$}}%
\end{pgfscope}%
\begin{pgfscope}%
\definecolor{textcolor}{rgb}{0.000000,0.000000,0.000000}%
\pgfsetstrokecolor{textcolor}%
\pgfsetfillcolor{textcolor}%
\pgftext[x=0.248148in,y=1.511573in,,bottom,rotate=90.000000]{\color{textcolor}{\rmfamily\fontsize{12.000000}{14.400000}\selectfont\catcode`\^=\active\def^{\ifmmode\sp\else\^{}\fi}\catcode`\%=\active\def%{\%}$L^1$ relative error}}%
\end{pgfscope}%
\begin{pgfscope}%
\pgfpathrectangle{\pgfqpoint{0.721913in}{0.549073in}}{\pgfqpoint{1.937500in}{1.925000in}}%
\pgfusepath{clip}%
\pgfsetrectcap%
\pgfsetroundjoin%
\pgfsetlinewidth{1.003750pt}%
\definecolor{currentstroke}{rgb}{0.001462,0.000466,0.013866}%
\pgfsetstrokecolor{currentstroke}%
\pgfsetdash{}{0pt}%
\pgfpathmoveto{\pgfqpoint{0.809982in}{2.055459in}}%
\pgfpathlineto{\pgfqpoint{1.180017in}{2.021849in}}%
\pgfpathlineto{\pgfqpoint{1.530977in}{1.995031in}}%
\pgfpathlineto{\pgfqpoint{1.877132in}{1.962044in}}%
\pgfpathlineto{\pgfqpoint{2.222805in}{1.922940in}}%
\pgfpathlineto{\pgfqpoint{2.571345in}{1.907338in}}%
\pgfusepath{stroke}%
\end{pgfscope}%
\begin{pgfscope}%
\pgfpathrectangle{\pgfqpoint{0.721913in}{0.549073in}}{\pgfqpoint{1.937500in}{1.925000in}}%
\pgfusepath{clip}%
\pgfsetbuttcap%
\pgfsetroundjoin%
\definecolor{currentfill}{rgb}{0.001462,0.000466,0.013866}%
\pgfsetfillcolor{currentfill}%
\pgfsetlinewidth{1.003750pt}%
\definecolor{currentstroke}{rgb}{0.001462,0.000466,0.013866}%
\pgfsetstrokecolor{currentstroke}%
\pgfsetdash{}{0pt}%
\pgfsys@defobject{currentmarker}{\pgfqpoint{-0.020833in}{-0.020833in}}{\pgfqpoint{0.020833in}{0.020833in}}{%
\pgfpathmoveto{\pgfqpoint{0.000000in}{-0.020833in}}%
\pgfpathcurveto{\pgfqpoint{0.005525in}{-0.020833in}}{\pgfqpoint{0.010825in}{-0.018638in}}{\pgfqpoint{0.014731in}{-0.014731in}}%
\pgfpathcurveto{\pgfqpoint{0.018638in}{-0.010825in}}{\pgfqpoint{0.020833in}{-0.005525in}}{\pgfqpoint{0.020833in}{0.000000in}}%
\pgfpathcurveto{\pgfqpoint{0.020833in}{0.005525in}}{\pgfqpoint{0.018638in}{0.010825in}}{\pgfqpoint{0.014731in}{0.014731in}}%
\pgfpathcurveto{\pgfqpoint{0.010825in}{0.018638in}}{\pgfqpoint{0.005525in}{0.020833in}}{\pgfqpoint{0.000000in}{0.020833in}}%
\pgfpathcurveto{\pgfqpoint{-0.005525in}{0.020833in}}{\pgfqpoint{-0.010825in}{0.018638in}}{\pgfqpoint{-0.014731in}{0.014731in}}%
\pgfpathcurveto{\pgfqpoint{-0.018638in}{0.010825in}}{\pgfqpoint{-0.020833in}{0.005525in}}{\pgfqpoint{-0.020833in}{0.000000in}}%
\pgfpathcurveto{\pgfqpoint{-0.020833in}{-0.005525in}}{\pgfqpoint{-0.018638in}{-0.010825in}}{\pgfqpoint{-0.014731in}{-0.014731in}}%
\pgfpathcurveto{\pgfqpoint{-0.010825in}{-0.018638in}}{\pgfqpoint{-0.005525in}{-0.020833in}}{\pgfqpoint{0.000000in}{-0.020833in}}%
\pgfpathlineto{\pgfqpoint{0.000000in}{-0.020833in}}%
\pgfpathclose%
\pgfusepath{stroke,fill}%
}%
\begin{pgfscope}%
\pgfsys@transformshift{0.809982in}{2.055459in}%
\pgfsys@useobject{currentmarker}{}%
\end{pgfscope}%
\begin{pgfscope}%
\pgfsys@transformshift{1.180017in}{2.021849in}%
\pgfsys@useobject{currentmarker}{}%
\end{pgfscope}%
\begin{pgfscope}%
\pgfsys@transformshift{1.530977in}{1.995031in}%
\pgfsys@useobject{currentmarker}{}%
\end{pgfscope}%
\begin{pgfscope}%
\pgfsys@transformshift{1.877132in}{1.962044in}%
\pgfsys@useobject{currentmarker}{}%
\end{pgfscope}%
\begin{pgfscope}%
\pgfsys@transformshift{2.222805in}{1.922940in}%
\pgfsys@useobject{currentmarker}{}%
\end{pgfscope}%
\begin{pgfscope}%
\pgfsys@transformshift{2.571345in}{1.907338in}%
\pgfsys@useobject{currentmarker}{}%
\end{pgfscope}%
\end{pgfscope}%
\begin{pgfscope}%
\pgfpathrectangle{\pgfqpoint{0.721913in}{0.549073in}}{\pgfqpoint{1.937500in}{1.925000in}}%
\pgfusepath{clip}%
\pgfsetrectcap%
\pgfsetroundjoin%
\pgfsetlinewidth{1.003750pt}%
\definecolor{currentstroke}{rgb}{0.445163,0.122724,0.506901}%
\pgfsetstrokecolor{currentstroke}%
\pgfsetdash{}{0pt}%
\pgfpathmoveto{\pgfqpoint{0.809982in}{2.386573in}}%
\pgfpathlineto{\pgfqpoint{1.180017in}{2.349814in}}%
\pgfpathlineto{\pgfqpoint{1.530977in}{2.283901in}}%
\pgfpathlineto{\pgfqpoint{1.877132in}{2.162841in}}%
\pgfpathlineto{\pgfqpoint{2.222805in}{1.817480in}}%
\pgfpathlineto{\pgfqpoint{2.571345in}{0.636573in}}%
\pgfusepath{stroke}%
\end{pgfscope}%
\begin{pgfscope}%
\pgfpathrectangle{\pgfqpoint{0.721913in}{0.549073in}}{\pgfqpoint{1.937500in}{1.925000in}}%
\pgfusepath{clip}%
\pgfsetbuttcap%
\pgfsetroundjoin%
\definecolor{currentfill}{rgb}{0.445163,0.122724,0.506901}%
\pgfsetfillcolor{currentfill}%
\pgfsetlinewidth{1.003750pt}%
\definecolor{currentstroke}{rgb}{0.445163,0.122724,0.506901}%
\pgfsetstrokecolor{currentstroke}%
\pgfsetdash{}{0pt}%
\pgfsys@defobject{currentmarker}{\pgfqpoint{-0.020833in}{-0.020833in}}{\pgfqpoint{0.020833in}{0.020833in}}{%
\pgfpathmoveto{\pgfqpoint{0.000000in}{-0.020833in}}%
\pgfpathcurveto{\pgfqpoint{0.005525in}{-0.020833in}}{\pgfqpoint{0.010825in}{-0.018638in}}{\pgfqpoint{0.014731in}{-0.014731in}}%
\pgfpathcurveto{\pgfqpoint{0.018638in}{-0.010825in}}{\pgfqpoint{0.020833in}{-0.005525in}}{\pgfqpoint{0.020833in}{0.000000in}}%
\pgfpathcurveto{\pgfqpoint{0.020833in}{0.005525in}}{\pgfqpoint{0.018638in}{0.010825in}}{\pgfqpoint{0.014731in}{0.014731in}}%
\pgfpathcurveto{\pgfqpoint{0.010825in}{0.018638in}}{\pgfqpoint{0.005525in}{0.020833in}}{\pgfqpoint{0.000000in}{0.020833in}}%
\pgfpathcurveto{\pgfqpoint{-0.005525in}{0.020833in}}{\pgfqpoint{-0.010825in}{0.018638in}}{\pgfqpoint{-0.014731in}{0.014731in}}%
\pgfpathcurveto{\pgfqpoint{-0.018638in}{0.010825in}}{\pgfqpoint{-0.020833in}{0.005525in}}{\pgfqpoint{-0.020833in}{0.000000in}}%
\pgfpathcurveto{\pgfqpoint{-0.020833in}{-0.005525in}}{\pgfqpoint{-0.018638in}{-0.010825in}}{\pgfqpoint{-0.014731in}{-0.014731in}}%
\pgfpathcurveto{\pgfqpoint{-0.010825in}{-0.018638in}}{\pgfqpoint{-0.005525in}{-0.020833in}}{\pgfqpoint{0.000000in}{-0.020833in}}%
\pgfpathlineto{\pgfqpoint{0.000000in}{-0.020833in}}%
\pgfpathclose%
\pgfusepath{stroke,fill}%
}%
\begin{pgfscope}%
\pgfsys@transformshift{0.809982in}{2.386573in}%
\pgfsys@useobject{currentmarker}{}%
\end{pgfscope}%
\begin{pgfscope}%
\pgfsys@transformshift{1.180017in}{2.349814in}%
\pgfsys@useobject{currentmarker}{}%
\end{pgfscope}%
\begin{pgfscope}%
\pgfsys@transformshift{1.530977in}{2.283901in}%
\pgfsys@useobject{currentmarker}{}%
\end{pgfscope}%
\begin{pgfscope}%
\pgfsys@transformshift{1.877132in}{2.162841in}%
\pgfsys@useobject{currentmarker}{}%
\end{pgfscope}%
\begin{pgfscope}%
\pgfsys@transformshift{2.222805in}{1.817480in}%
\pgfsys@useobject{currentmarker}{}%
\end{pgfscope}%
\begin{pgfscope}%
\pgfsys@transformshift{2.571345in}{0.636573in}%
\pgfsys@useobject{currentmarker}{}%
\end{pgfscope}%
\end{pgfscope}%
\begin{pgfscope}%
\pgfpathrectangle{\pgfqpoint{0.721913in}{0.549073in}}{\pgfqpoint{1.937500in}{1.925000in}}%
\pgfusepath{clip}%
\pgfsetrectcap%
\pgfsetroundjoin%
\pgfsetlinewidth{1.003750pt}%
\definecolor{currentstroke}{rgb}{0.944006,0.377643,0.365136}%
\pgfsetstrokecolor{currentstroke}%
\pgfsetdash{}{0pt}%
\pgfpathmoveto{\pgfqpoint{0.809982in}{2.091909in}}%
\pgfpathlineto{\pgfqpoint{1.180017in}{2.024101in}}%
\pgfpathlineto{\pgfqpoint{1.530977in}{1.943923in}}%
\pgfpathlineto{\pgfqpoint{1.877132in}{1.822271in}}%
\pgfpathlineto{\pgfqpoint{2.222805in}{1.716867in}}%
\pgfpathlineto{\pgfqpoint{2.571345in}{1.409788in}}%
\pgfusepath{stroke}%
\end{pgfscope}%
\begin{pgfscope}%
\pgfpathrectangle{\pgfqpoint{0.721913in}{0.549073in}}{\pgfqpoint{1.937500in}{1.925000in}}%
\pgfusepath{clip}%
\pgfsetbuttcap%
\pgfsetroundjoin%
\definecolor{currentfill}{rgb}{0.944006,0.377643,0.365136}%
\pgfsetfillcolor{currentfill}%
\pgfsetlinewidth{1.003750pt}%
\definecolor{currentstroke}{rgb}{0.944006,0.377643,0.365136}%
\pgfsetstrokecolor{currentstroke}%
\pgfsetdash{}{0pt}%
\pgfsys@defobject{currentmarker}{\pgfqpoint{-0.020833in}{-0.020833in}}{\pgfqpoint{0.020833in}{0.020833in}}{%
\pgfpathmoveto{\pgfqpoint{0.000000in}{-0.020833in}}%
\pgfpathcurveto{\pgfqpoint{0.005525in}{-0.020833in}}{\pgfqpoint{0.010825in}{-0.018638in}}{\pgfqpoint{0.014731in}{-0.014731in}}%
\pgfpathcurveto{\pgfqpoint{0.018638in}{-0.010825in}}{\pgfqpoint{0.020833in}{-0.005525in}}{\pgfqpoint{0.020833in}{0.000000in}}%
\pgfpathcurveto{\pgfqpoint{0.020833in}{0.005525in}}{\pgfqpoint{0.018638in}{0.010825in}}{\pgfqpoint{0.014731in}{0.014731in}}%
\pgfpathcurveto{\pgfqpoint{0.010825in}{0.018638in}}{\pgfqpoint{0.005525in}{0.020833in}}{\pgfqpoint{0.000000in}{0.020833in}}%
\pgfpathcurveto{\pgfqpoint{-0.005525in}{0.020833in}}{\pgfqpoint{-0.010825in}{0.018638in}}{\pgfqpoint{-0.014731in}{0.014731in}}%
\pgfpathcurveto{\pgfqpoint{-0.018638in}{0.010825in}}{\pgfqpoint{-0.020833in}{0.005525in}}{\pgfqpoint{-0.020833in}{0.000000in}}%
\pgfpathcurveto{\pgfqpoint{-0.020833in}{-0.005525in}}{\pgfqpoint{-0.018638in}{-0.010825in}}{\pgfqpoint{-0.014731in}{-0.014731in}}%
\pgfpathcurveto{\pgfqpoint{-0.010825in}{-0.018638in}}{\pgfqpoint{-0.005525in}{-0.020833in}}{\pgfqpoint{0.000000in}{-0.020833in}}%
\pgfpathlineto{\pgfqpoint{0.000000in}{-0.020833in}}%
\pgfpathclose%
\pgfusepath{stroke,fill}%
}%
\begin{pgfscope}%
\pgfsys@transformshift{0.809982in}{2.091909in}%
\pgfsys@useobject{currentmarker}{}%
\end{pgfscope}%
\begin{pgfscope}%
\pgfsys@transformshift{1.180017in}{2.024101in}%
\pgfsys@useobject{currentmarker}{}%
\end{pgfscope}%
\begin{pgfscope}%
\pgfsys@transformshift{1.530977in}{1.943923in}%
\pgfsys@useobject{currentmarker}{}%
\end{pgfscope}%
\begin{pgfscope}%
\pgfsys@transformshift{1.877132in}{1.822271in}%
\pgfsys@useobject{currentmarker}{}%
\end{pgfscope}%
\begin{pgfscope}%
\pgfsys@transformshift{2.222805in}{1.716867in}%
\pgfsys@useobject{currentmarker}{}%
\end{pgfscope}%
\begin{pgfscope}%
\pgfsys@transformshift{2.571345in}{1.409788in}%
\pgfsys@useobject{currentmarker}{}%
\end{pgfscope}%
\end{pgfscope}%
\begin{pgfscope}%
\pgfsetrectcap%
\pgfsetmiterjoin%
\pgfsetlinewidth{0.803000pt}%
\definecolor{currentstroke}{rgb}{0.000000,0.000000,0.000000}%
\pgfsetstrokecolor{currentstroke}%
\pgfsetdash{}{0pt}%
\pgfpathmoveto{\pgfqpoint{0.721913in}{0.549073in}}%
\pgfpathlineto{\pgfqpoint{0.721913in}{2.474073in}}%
\pgfusepath{stroke}%
\end{pgfscope}%
\begin{pgfscope}%
\pgfsetrectcap%
\pgfsetmiterjoin%
\pgfsetlinewidth{0.803000pt}%
\definecolor{currentstroke}{rgb}{0.000000,0.000000,0.000000}%
\pgfsetstrokecolor{currentstroke}%
\pgfsetdash{}{0pt}%
\pgfpathmoveto{\pgfqpoint{2.659413in}{0.549073in}}%
\pgfpathlineto{\pgfqpoint{2.659413in}{2.474073in}}%
\pgfusepath{stroke}%
\end{pgfscope}%
\begin{pgfscope}%
\pgfsetrectcap%
\pgfsetmiterjoin%
\pgfsetlinewidth{0.803000pt}%
\definecolor{currentstroke}{rgb}{0.000000,0.000000,0.000000}%
\pgfsetstrokecolor{currentstroke}%
\pgfsetdash{}{0pt}%
\pgfpathmoveto{\pgfqpoint{0.721913in}{0.549073in}}%
\pgfpathlineto{\pgfqpoint{2.659413in}{0.549073in}}%
\pgfusepath{stroke}%
\end{pgfscope}%
\begin{pgfscope}%
\pgfsetrectcap%
\pgfsetmiterjoin%
\pgfsetlinewidth{0.803000pt}%
\definecolor{currentstroke}{rgb}{0.000000,0.000000,0.000000}%
\pgfsetstrokecolor{currentstroke}%
\pgfsetdash{}{0pt}%
\pgfpathmoveto{\pgfqpoint{0.721913in}{2.474073in}}%
\pgfpathlineto{\pgfqpoint{2.659413in}{2.474073in}}%
\pgfusepath{stroke}%
\end{pgfscope}%
\begin{pgfscope}%
\pgfsetbuttcap%
\pgfsetmiterjoin%
\definecolor{currentfill}{rgb}{1.000000,1.000000,1.000000}%
\pgfsetfillcolor{currentfill}%
\pgfsetfillopacity{0.800000}%
\pgfsetlinewidth{1.003750pt}%
\definecolor{currentstroke}{rgb}{0.800000,0.800000,0.800000}%
\pgfsetstrokecolor{currentstroke}%
\pgfsetstrokeopacity{0.800000}%
\pgfsetdash{}{0pt}%
\pgfpathmoveto{\pgfqpoint{0.838580in}{0.632406in}}%
\pgfpathlineto{\pgfqpoint{1.865967in}{0.632406in}}%
\pgfpathquadraticcurveto{\pgfqpoint{1.899300in}{0.632406in}}{\pgfqpoint{1.899300in}{0.665739in}}%
\pgfpathlineto{\pgfqpoint{1.899300in}{1.346294in}}%
\pgfpathquadraticcurveto{\pgfqpoint{1.899300in}{1.379627in}}{\pgfqpoint{1.865967in}{1.379627in}}%
\pgfpathlineto{\pgfqpoint{0.838580in}{1.379627in}}%
\pgfpathquadraticcurveto{\pgfqpoint{0.805247in}{1.379627in}}{\pgfqpoint{0.805247in}{1.346294in}}%
\pgfpathlineto{\pgfqpoint{0.805247in}{0.665739in}}%
\pgfpathquadraticcurveto{\pgfqpoint{0.805247in}{0.632406in}}{\pgfqpoint{0.838580in}{0.632406in}}%
\pgfpathlineto{\pgfqpoint{0.838580in}{0.632406in}}%
\pgfpathclose%
\pgfusepath{stroke,fill}%
\end{pgfscope}%
\begin{pgfscope}%
\pgfsetrectcap%
\pgfsetroundjoin%
\pgfsetlinewidth{1.003750pt}%
\definecolor{currentstroke}{rgb}{0.001462,0.000466,0.013866}%
\pgfsetstrokecolor{currentstroke}%
\pgfsetdash{}{0pt}%
\pgfpathmoveto{\pgfqpoint{0.871913in}{1.254627in}}%
\pgfpathlineto{\pgfqpoint{1.038580in}{1.254627in}}%
\pgfpathlineto{\pgfqpoint{1.205247in}{1.254627in}}%
\pgfusepath{stroke}%
\end{pgfscope}%
\begin{pgfscope}%
\pgfsetbuttcap%
\pgfsetroundjoin%
\definecolor{currentfill}{rgb}{0.001462,0.000466,0.013866}%
\pgfsetfillcolor{currentfill}%
\pgfsetlinewidth{1.003750pt}%
\definecolor{currentstroke}{rgb}{0.001462,0.000466,0.013866}%
\pgfsetstrokecolor{currentstroke}%
\pgfsetdash{}{0pt}%
\pgfsys@defobject{currentmarker}{\pgfqpoint{-0.020833in}{-0.020833in}}{\pgfqpoint{0.020833in}{0.020833in}}{%
\pgfpathmoveto{\pgfqpoint{0.000000in}{-0.020833in}}%
\pgfpathcurveto{\pgfqpoint{0.005525in}{-0.020833in}}{\pgfqpoint{0.010825in}{-0.018638in}}{\pgfqpoint{0.014731in}{-0.014731in}}%
\pgfpathcurveto{\pgfqpoint{0.018638in}{-0.010825in}}{\pgfqpoint{0.020833in}{-0.005525in}}{\pgfqpoint{0.020833in}{0.000000in}}%
\pgfpathcurveto{\pgfqpoint{0.020833in}{0.005525in}}{\pgfqpoint{0.018638in}{0.010825in}}{\pgfqpoint{0.014731in}{0.014731in}}%
\pgfpathcurveto{\pgfqpoint{0.010825in}{0.018638in}}{\pgfqpoint{0.005525in}{0.020833in}}{\pgfqpoint{0.000000in}{0.020833in}}%
\pgfpathcurveto{\pgfqpoint{-0.005525in}{0.020833in}}{\pgfqpoint{-0.010825in}{0.018638in}}{\pgfqpoint{-0.014731in}{0.014731in}}%
\pgfpathcurveto{\pgfqpoint{-0.018638in}{0.010825in}}{\pgfqpoint{-0.020833in}{0.005525in}}{\pgfqpoint{-0.020833in}{0.000000in}}%
\pgfpathcurveto{\pgfqpoint{-0.020833in}{-0.005525in}}{\pgfqpoint{-0.018638in}{-0.010825in}}{\pgfqpoint{-0.014731in}{-0.014731in}}%
\pgfpathcurveto{\pgfqpoint{-0.010825in}{-0.018638in}}{\pgfqpoint{-0.005525in}{-0.020833in}}{\pgfqpoint{0.000000in}{-0.020833in}}%
\pgfpathlineto{\pgfqpoint{0.000000in}{-0.020833in}}%
\pgfpathclose%
\pgfusepath{stroke,fill}%
}%
\begin{pgfscope}%
\pgfsys@transformshift{1.038580in}{1.254627in}%
\pgfsys@useobject{currentmarker}{}%
\end{pgfscope}%
\end{pgfscope}%
\begin{pgfscope}%
\definecolor{textcolor}{rgb}{0.000000,0.000000,0.000000}%
\pgfsetstrokecolor{textcolor}%
\pgfsetfillcolor{textcolor}%
\pgftext[x=1.338580in,y=1.196294in,left,base]{\color{textcolor}{\rmfamily\fontsize{12.000000}{14.400000}\selectfont\catcode`\^=\active\def^{\ifmmode\sp\else\^{}\fi}\catcode`\%=\active\def%{\%}DGC}}%
\end{pgfscope}%
\begin{pgfscope}%
\pgfsetrectcap%
\pgfsetroundjoin%
\pgfsetlinewidth{1.003750pt}%
\definecolor{currentstroke}{rgb}{0.445163,0.122724,0.506901}%
\pgfsetstrokecolor{currentstroke}%
\pgfsetdash{}{0pt}%
\pgfpathmoveto{\pgfqpoint{0.871913in}{1.022220in}}%
\pgfpathlineto{\pgfqpoint{1.038580in}{1.022220in}}%
\pgfpathlineto{\pgfqpoint{1.205247in}{1.022220in}}%
\pgfusepath{stroke}%
\end{pgfscope}%
\begin{pgfscope}%
\pgfsetbuttcap%
\pgfsetroundjoin%
\definecolor{currentfill}{rgb}{0.445163,0.122724,0.506901}%
\pgfsetfillcolor{currentfill}%
\pgfsetlinewidth{1.003750pt}%
\definecolor{currentstroke}{rgb}{0.445163,0.122724,0.506901}%
\pgfsetstrokecolor{currentstroke}%
\pgfsetdash{}{0pt}%
\pgfsys@defobject{currentmarker}{\pgfqpoint{-0.020833in}{-0.020833in}}{\pgfqpoint{0.020833in}{0.020833in}}{%
\pgfpathmoveto{\pgfqpoint{0.000000in}{-0.020833in}}%
\pgfpathcurveto{\pgfqpoint{0.005525in}{-0.020833in}}{\pgfqpoint{0.010825in}{-0.018638in}}{\pgfqpoint{0.014731in}{-0.014731in}}%
\pgfpathcurveto{\pgfqpoint{0.018638in}{-0.010825in}}{\pgfqpoint{0.020833in}{-0.005525in}}{\pgfqpoint{0.020833in}{0.000000in}}%
\pgfpathcurveto{\pgfqpoint{0.020833in}{0.005525in}}{\pgfqpoint{0.018638in}{0.010825in}}{\pgfqpoint{0.014731in}{0.014731in}}%
\pgfpathcurveto{\pgfqpoint{0.010825in}{0.018638in}}{\pgfqpoint{0.005525in}{0.020833in}}{\pgfqpoint{0.000000in}{0.020833in}}%
\pgfpathcurveto{\pgfqpoint{-0.005525in}{0.020833in}}{\pgfqpoint{-0.010825in}{0.018638in}}{\pgfqpoint{-0.014731in}{0.014731in}}%
\pgfpathcurveto{\pgfqpoint{-0.018638in}{0.010825in}}{\pgfqpoint{-0.020833in}{0.005525in}}{\pgfqpoint{-0.020833in}{0.000000in}}%
\pgfpathcurveto{\pgfqpoint{-0.020833in}{-0.005525in}}{\pgfqpoint{-0.018638in}{-0.010825in}}{\pgfqpoint{-0.014731in}{-0.014731in}}%
\pgfpathcurveto{\pgfqpoint{-0.010825in}{-0.018638in}}{\pgfqpoint{-0.005525in}{-0.020833in}}{\pgfqpoint{0.000000in}{-0.020833in}}%
\pgfpathlineto{\pgfqpoint{0.000000in}{-0.020833in}}%
\pgfpathclose%
\pgfusepath{stroke,fill}%
}%
\begin{pgfscope}%
\pgfsys@transformshift{1.038580in}{1.022220in}%
\pgfsys@useobject{currentmarker}{}%
\end{pgfscope}%
\end{pgfscope}%
\begin{pgfscope}%
\definecolor{textcolor}{rgb}{0.000000,0.000000,0.000000}%
\pgfsetstrokecolor{textcolor}%
\pgfsetfillcolor{textcolor}%
\pgftext[x=1.338580in,y=0.963887in,left,base]{\color{textcolor}{\rmfamily\fontsize{12.000000}{14.400000}\selectfont\catcode`\^=\active\def^{\ifmmode\sp\else\^{}\fi}\catcode`\%=\active\def%{\%}NC}}%
\end{pgfscope}%
\begin{pgfscope}%
\pgfsetrectcap%
\pgfsetroundjoin%
\pgfsetlinewidth{1.003750pt}%
\definecolor{currentstroke}{rgb}{0.944006,0.377643,0.365136}%
\pgfsetstrokecolor{currentstroke}%
\pgfsetdash{}{0pt}%
\pgfpathmoveto{\pgfqpoint{0.871913in}{0.789813in}}%
\pgfpathlineto{\pgfqpoint{1.038580in}{0.789813in}}%
\pgfpathlineto{\pgfqpoint{1.205247in}{0.789813in}}%
\pgfusepath{stroke}%
\end{pgfscope}%
\begin{pgfscope}%
\pgfsetbuttcap%
\pgfsetroundjoin%
\definecolor{currentfill}{rgb}{0.944006,0.377643,0.365136}%
\pgfsetfillcolor{currentfill}%
\pgfsetlinewidth{1.003750pt}%
\definecolor{currentstroke}{rgb}{0.944006,0.377643,0.365136}%
\pgfsetstrokecolor{currentstroke}%
\pgfsetdash{}{0pt}%
\pgfsys@defobject{currentmarker}{\pgfqpoint{-0.020833in}{-0.020833in}}{\pgfqpoint{0.020833in}{0.020833in}}{%
\pgfpathmoveto{\pgfqpoint{0.000000in}{-0.020833in}}%
\pgfpathcurveto{\pgfqpoint{0.005525in}{-0.020833in}}{\pgfqpoint{0.010825in}{-0.018638in}}{\pgfqpoint{0.014731in}{-0.014731in}}%
\pgfpathcurveto{\pgfqpoint{0.018638in}{-0.010825in}}{\pgfqpoint{0.020833in}{-0.005525in}}{\pgfqpoint{0.020833in}{0.000000in}}%
\pgfpathcurveto{\pgfqpoint{0.020833in}{0.005525in}}{\pgfqpoint{0.018638in}{0.010825in}}{\pgfqpoint{0.014731in}{0.014731in}}%
\pgfpathcurveto{\pgfqpoint{0.010825in}{0.018638in}}{\pgfqpoint{0.005525in}{0.020833in}}{\pgfqpoint{0.000000in}{0.020833in}}%
\pgfpathcurveto{\pgfqpoint{-0.005525in}{0.020833in}}{\pgfqpoint{-0.010825in}{0.018638in}}{\pgfqpoint{-0.014731in}{0.014731in}}%
\pgfpathcurveto{\pgfqpoint{-0.018638in}{0.010825in}}{\pgfqpoint{-0.020833in}{0.005525in}}{\pgfqpoint{-0.020833in}{0.000000in}}%
\pgfpathcurveto{\pgfqpoint{-0.020833in}{-0.005525in}}{\pgfqpoint{-0.018638in}{-0.010825in}}{\pgfqpoint{-0.014731in}{-0.014731in}}%
\pgfpathcurveto{\pgfqpoint{-0.010825in}{-0.018638in}}{\pgfqpoint{-0.005525in}{-0.020833in}}{\pgfqpoint{0.000000in}{-0.020833in}}%
\pgfpathlineto{\pgfqpoint{0.000000in}{-0.020833in}}%
\pgfpathclose%
\pgfusepath{stroke,fill}%
}%
\begin{pgfscope}%
\pgfsys@transformshift{1.038580in}{0.789813in}%
\pgfsys@useobject{currentmarker}{}%
\end{pgfscope}%
\end{pgfscope}%
\begin{pgfscope}%
\definecolor{textcolor}{rgb}{0.000000,0.000000,0.000000}%
\pgfsetstrokecolor{textcolor}%
\pgfsetfillcolor{textcolor}%
\pgftext[x=1.338580in,y=0.731480in,left,base]{\color{textcolor}{\rmfamily\fontsize{12.000000}{14.400000}\selectfont\catcode`\^=\active\def^{\ifmmode\sp\else\^{}\fi}\catcode`\%=\active\def%{\%}NC++}}%
\end{pgfscope}%
\end{pgfpicture}%
\makeatother%
\endgroup%

        \caption{ModES3D\_8}
        \label{fig:5-experiments-multi-matrix-convergence-ModES3D}
    \end{subfigure}
    \begin{subfigure}[b]{0.49\columnwidth}
        %% Creator: Matplotlib, PGF backend
%%
%% To include the figure in your LaTeX document, write
%%   \input{<filename>.pgf}
%%
%% Make sure the required packages are loaded in your preamble
%%   \usepackage{pgf}
%%
%% Also ensure that all the required font packages are loaded; for instance,
%% the lmodern package is sometimes necessary when using math font.
%%   \usepackage{lmodern}
%%
%% Figures using additional raster images can only be included by \input if
%% they are in the same directory as the main LaTeX file. For loading figures
%% from other directories you can use the `import` package
%%   \usepackage{import}
%%
%% and then include the figures with
%%   \import{<path to file>}{<filename>.pgf}
%%
%% Matplotlib used the following preamble
%%   \def\mathdefault#1{#1}
%%   \everymath=\expandafter{\the\everymath\displaystyle}
%%   \IfFileExists{scrextend.sty}{
%%     \usepackage[fontsize=12.000000pt]{scrextend}
%%   }{
%%     \renewcommand{\normalsize}{\fontsize{12.000000}{14.400000}\selectfont}
%%     \normalsize
%%   }
%%   
%%   \ifdefined\pdftexversion\else  % non-pdftex case.
%%     \usepackage{fontspec}
%%     \setmainfont{DejaVuSans.ttf}[Path=\detokenize{/opt/hostedtoolcache/Python/3.12.9/x64/lib/python3.12/site-packages/matplotlib/mpl-data/fonts/ttf/}]
%%     \setsansfont{DejaVuSans.ttf}[Path=\detokenize{/opt/hostedtoolcache/Python/3.12.9/x64/lib/python3.12/site-packages/matplotlib/mpl-data/fonts/ttf/}]
%%     \setmonofont{DejaVuSansMono.ttf}[Path=\detokenize{/opt/hostedtoolcache/Python/3.12.9/x64/lib/python3.12/site-packages/matplotlib/mpl-data/fonts/ttf/}]
%%   \fi
%%   \makeatletter\@ifpackageloaded{underscore}{}{\usepackage[strings]{underscore}}\makeatother
%%
\begingroup%
\makeatletter%
\begin{pgfpicture}%
\pgfpathrectangle{\pgfpointorigin}{\pgfqpoint{2.759413in}{2.574073in}}%
\pgfusepath{use as bounding box, clip}%
\begin{pgfscope}%
\pgfsetbuttcap%
\pgfsetmiterjoin%
\definecolor{currentfill}{rgb}{1.000000,1.000000,1.000000}%
\pgfsetfillcolor{currentfill}%
\pgfsetlinewidth{0.000000pt}%
\definecolor{currentstroke}{rgb}{1.000000,1.000000,1.000000}%
\pgfsetstrokecolor{currentstroke}%
\pgfsetdash{}{0pt}%
\pgfpathmoveto{\pgfqpoint{0.000000in}{0.000000in}}%
\pgfpathlineto{\pgfqpoint{2.759413in}{0.000000in}}%
\pgfpathlineto{\pgfqpoint{2.759413in}{2.574073in}}%
\pgfpathlineto{\pgfqpoint{0.000000in}{2.574073in}}%
\pgfpathlineto{\pgfqpoint{0.000000in}{0.000000in}}%
\pgfpathclose%
\pgfusepath{fill}%
\end{pgfscope}%
\begin{pgfscope}%
\pgfsetbuttcap%
\pgfsetmiterjoin%
\definecolor{currentfill}{rgb}{1.000000,1.000000,1.000000}%
\pgfsetfillcolor{currentfill}%
\pgfsetlinewidth{0.000000pt}%
\definecolor{currentstroke}{rgb}{0.000000,0.000000,0.000000}%
\pgfsetstrokecolor{currentstroke}%
\pgfsetstrokeopacity{0.000000}%
\pgfsetdash{}{0pt}%
\pgfpathmoveto{\pgfqpoint{0.721913in}{0.549073in}}%
\pgfpathlineto{\pgfqpoint{2.659413in}{0.549073in}}%
\pgfpathlineto{\pgfqpoint{2.659413in}{2.474073in}}%
\pgfpathlineto{\pgfqpoint{0.721913in}{2.474073in}}%
\pgfpathlineto{\pgfqpoint{0.721913in}{0.549073in}}%
\pgfpathclose%
\pgfusepath{fill}%
\end{pgfscope}%
\begin{pgfscope}%
\pgfsetbuttcap%
\pgfsetroundjoin%
\definecolor{currentfill}{rgb}{0.000000,0.000000,0.000000}%
\pgfsetfillcolor{currentfill}%
\pgfsetlinewidth{0.803000pt}%
\definecolor{currentstroke}{rgb}{0.000000,0.000000,0.000000}%
\pgfsetstrokecolor{currentstroke}%
\pgfsetdash{}{0pt}%
\pgfsys@defobject{currentmarker}{\pgfqpoint{0.000000in}{-0.048611in}}{\pgfqpoint{0.000000in}{0.000000in}}{%
\pgfpathmoveto{\pgfqpoint{0.000000in}{0.000000in}}%
\pgfpathlineto{\pgfqpoint{0.000000in}{-0.048611in}}%
\pgfusepath{stroke,fill}%
}%
\begin{pgfscope}%
\pgfsys@transformshift{0.910580in}{0.549073in}%
\pgfsys@useobject{currentmarker}{}%
\end{pgfscope}%
\end{pgfscope}%
\begin{pgfscope}%
\definecolor{textcolor}{rgb}{0.000000,0.000000,0.000000}%
\pgfsetstrokecolor{textcolor}%
\pgfsetfillcolor{textcolor}%
\pgftext[x=0.910580in,y=0.451851in,,top]{\color{textcolor}{\rmfamily\fontsize{12.000000}{14.400000}\selectfont\catcode`\^=\active\def^{\ifmmode\sp\else\^{}\fi}\catcode`\%=\active\def%{\%}$\mathdefault{10^{1}}$}}%
\end{pgfscope}%
\begin{pgfscope}%
\pgfsetbuttcap%
\pgfsetroundjoin%
\definecolor{currentfill}{rgb}{0.000000,0.000000,0.000000}%
\pgfsetfillcolor{currentfill}%
\pgfsetlinewidth{0.803000pt}%
\definecolor{currentstroke}{rgb}{0.000000,0.000000,0.000000}%
\pgfsetstrokecolor{currentstroke}%
\pgfsetdash{}{0pt}%
\pgfsys@defobject{currentmarker}{\pgfqpoint{0.000000in}{-0.048611in}}{\pgfqpoint{0.000000in}{0.000000in}}{%
\pgfpathmoveto{\pgfqpoint{0.000000in}{0.000000in}}%
\pgfpathlineto{\pgfqpoint{0.000000in}{-0.048611in}}%
\pgfusepath{stroke,fill}%
}%
\begin{pgfscope}%
\pgfsys@transformshift{1.948634in}{0.549073in}%
\pgfsys@useobject{currentmarker}{}%
\end{pgfscope}%
\end{pgfscope}%
\begin{pgfscope}%
\definecolor{textcolor}{rgb}{0.000000,0.000000,0.000000}%
\pgfsetstrokecolor{textcolor}%
\pgfsetfillcolor{textcolor}%
\pgftext[x=1.948634in,y=0.451851in,,top]{\color{textcolor}{\rmfamily\fontsize{12.000000}{14.400000}\selectfont\catcode`\^=\active\def^{\ifmmode\sp\else\^{}\fi}\catcode`\%=\active\def%{\%}$\mathdefault{10^{2}}$}}%
\end{pgfscope}%
\begin{pgfscope}%
\pgfsetbuttcap%
\pgfsetroundjoin%
\definecolor{currentfill}{rgb}{0.000000,0.000000,0.000000}%
\pgfsetfillcolor{currentfill}%
\pgfsetlinewidth{0.602250pt}%
\definecolor{currentstroke}{rgb}{0.000000,0.000000,0.000000}%
\pgfsetstrokecolor{currentstroke}%
\pgfsetdash{}{0pt}%
\pgfsys@defobject{currentmarker}{\pgfqpoint{0.000000in}{-0.027778in}}{\pgfqpoint{0.000000in}{0.000000in}}{%
\pgfpathmoveto{\pgfqpoint{0.000000in}{0.000000in}}%
\pgfpathlineto{\pgfqpoint{0.000000in}{-0.027778in}}%
\pgfusepath{stroke,fill}%
}%
\begin{pgfscope}%
\pgfsys@transformshift{0.749783in}{0.549073in}%
\pgfsys@useobject{currentmarker}{}%
\end{pgfscope}%
\end{pgfscope}%
\begin{pgfscope}%
\pgfsetbuttcap%
\pgfsetroundjoin%
\definecolor{currentfill}{rgb}{0.000000,0.000000,0.000000}%
\pgfsetfillcolor{currentfill}%
\pgfsetlinewidth{0.602250pt}%
\definecolor{currentstroke}{rgb}{0.000000,0.000000,0.000000}%
\pgfsetstrokecolor{currentstroke}%
\pgfsetdash{}{0pt}%
\pgfsys@defobject{currentmarker}{\pgfqpoint{0.000000in}{-0.027778in}}{\pgfqpoint{0.000000in}{0.000000in}}{%
\pgfpathmoveto{\pgfqpoint{0.000000in}{0.000000in}}%
\pgfpathlineto{\pgfqpoint{0.000000in}{-0.027778in}}%
\pgfusepath{stroke,fill}%
}%
\begin{pgfscope}%
\pgfsys@transformshift{0.809982in}{0.549073in}%
\pgfsys@useobject{currentmarker}{}%
\end{pgfscope}%
\end{pgfscope}%
\begin{pgfscope}%
\pgfsetbuttcap%
\pgfsetroundjoin%
\definecolor{currentfill}{rgb}{0.000000,0.000000,0.000000}%
\pgfsetfillcolor{currentfill}%
\pgfsetlinewidth{0.602250pt}%
\definecolor{currentstroke}{rgb}{0.000000,0.000000,0.000000}%
\pgfsetstrokecolor{currentstroke}%
\pgfsetdash{}{0pt}%
\pgfsys@defobject{currentmarker}{\pgfqpoint{0.000000in}{-0.027778in}}{\pgfqpoint{0.000000in}{0.000000in}}{%
\pgfpathmoveto{\pgfqpoint{0.000000in}{0.000000in}}%
\pgfpathlineto{\pgfqpoint{0.000000in}{-0.027778in}}%
\pgfusepath{stroke,fill}%
}%
\begin{pgfscope}%
\pgfsys@transformshift{0.863081in}{0.549073in}%
\pgfsys@useobject{currentmarker}{}%
\end{pgfscope}%
\end{pgfscope}%
\begin{pgfscope}%
\pgfsetbuttcap%
\pgfsetroundjoin%
\definecolor{currentfill}{rgb}{0.000000,0.000000,0.000000}%
\pgfsetfillcolor{currentfill}%
\pgfsetlinewidth{0.602250pt}%
\definecolor{currentstroke}{rgb}{0.000000,0.000000,0.000000}%
\pgfsetstrokecolor{currentstroke}%
\pgfsetdash{}{0pt}%
\pgfsys@defobject{currentmarker}{\pgfqpoint{0.000000in}{-0.027778in}}{\pgfqpoint{0.000000in}{0.000000in}}{%
\pgfpathmoveto{\pgfqpoint{0.000000in}{0.000000in}}%
\pgfpathlineto{\pgfqpoint{0.000000in}{-0.027778in}}%
\pgfusepath{stroke,fill}%
}%
\begin{pgfscope}%
\pgfsys@transformshift{1.223065in}{0.549073in}%
\pgfsys@useobject{currentmarker}{}%
\end{pgfscope}%
\end{pgfscope}%
\begin{pgfscope}%
\pgfsetbuttcap%
\pgfsetroundjoin%
\definecolor{currentfill}{rgb}{0.000000,0.000000,0.000000}%
\pgfsetfillcolor{currentfill}%
\pgfsetlinewidth{0.602250pt}%
\definecolor{currentstroke}{rgb}{0.000000,0.000000,0.000000}%
\pgfsetstrokecolor{currentstroke}%
\pgfsetdash{}{0pt}%
\pgfsys@defobject{currentmarker}{\pgfqpoint{0.000000in}{-0.027778in}}{\pgfqpoint{0.000000in}{0.000000in}}{%
\pgfpathmoveto{\pgfqpoint{0.000000in}{0.000000in}}%
\pgfpathlineto{\pgfqpoint{0.000000in}{-0.027778in}}%
\pgfusepath{stroke,fill}%
}%
\begin{pgfscope}%
\pgfsys@transformshift{1.405857in}{0.549073in}%
\pgfsys@useobject{currentmarker}{}%
\end{pgfscope}%
\end{pgfscope}%
\begin{pgfscope}%
\pgfsetbuttcap%
\pgfsetroundjoin%
\definecolor{currentfill}{rgb}{0.000000,0.000000,0.000000}%
\pgfsetfillcolor{currentfill}%
\pgfsetlinewidth{0.602250pt}%
\definecolor{currentstroke}{rgb}{0.000000,0.000000,0.000000}%
\pgfsetstrokecolor{currentstroke}%
\pgfsetdash{}{0pt}%
\pgfsys@defobject{currentmarker}{\pgfqpoint{0.000000in}{-0.027778in}}{\pgfqpoint{0.000000in}{0.000000in}}{%
\pgfpathmoveto{\pgfqpoint{0.000000in}{0.000000in}}%
\pgfpathlineto{\pgfqpoint{0.000000in}{-0.027778in}}%
\pgfusepath{stroke,fill}%
}%
\begin{pgfscope}%
\pgfsys@transformshift{1.535551in}{0.549073in}%
\pgfsys@useobject{currentmarker}{}%
\end{pgfscope}%
\end{pgfscope}%
\begin{pgfscope}%
\pgfsetbuttcap%
\pgfsetroundjoin%
\definecolor{currentfill}{rgb}{0.000000,0.000000,0.000000}%
\pgfsetfillcolor{currentfill}%
\pgfsetlinewidth{0.602250pt}%
\definecolor{currentstroke}{rgb}{0.000000,0.000000,0.000000}%
\pgfsetstrokecolor{currentstroke}%
\pgfsetdash{}{0pt}%
\pgfsys@defobject{currentmarker}{\pgfqpoint{0.000000in}{-0.027778in}}{\pgfqpoint{0.000000in}{0.000000in}}{%
\pgfpathmoveto{\pgfqpoint{0.000000in}{0.000000in}}%
\pgfpathlineto{\pgfqpoint{0.000000in}{-0.027778in}}%
\pgfusepath{stroke,fill}%
}%
\begin{pgfscope}%
\pgfsys@transformshift{1.636148in}{0.549073in}%
\pgfsys@useobject{currentmarker}{}%
\end{pgfscope}%
\end{pgfscope}%
\begin{pgfscope}%
\pgfsetbuttcap%
\pgfsetroundjoin%
\definecolor{currentfill}{rgb}{0.000000,0.000000,0.000000}%
\pgfsetfillcolor{currentfill}%
\pgfsetlinewidth{0.602250pt}%
\definecolor{currentstroke}{rgb}{0.000000,0.000000,0.000000}%
\pgfsetstrokecolor{currentstroke}%
\pgfsetdash{}{0pt}%
\pgfsys@defobject{currentmarker}{\pgfqpoint{0.000000in}{-0.027778in}}{\pgfqpoint{0.000000in}{0.000000in}}{%
\pgfpathmoveto{\pgfqpoint{0.000000in}{0.000000in}}%
\pgfpathlineto{\pgfqpoint{0.000000in}{-0.027778in}}%
\pgfusepath{stroke,fill}%
}%
\begin{pgfscope}%
\pgfsys@transformshift{1.718343in}{0.549073in}%
\pgfsys@useobject{currentmarker}{}%
\end{pgfscope}%
\end{pgfscope}%
\begin{pgfscope}%
\pgfsetbuttcap%
\pgfsetroundjoin%
\definecolor{currentfill}{rgb}{0.000000,0.000000,0.000000}%
\pgfsetfillcolor{currentfill}%
\pgfsetlinewidth{0.602250pt}%
\definecolor{currentstroke}{rgb}{0.000000,0.000000,0.000000}%
\pgfsetstrokecolor{currentstroke}%
\pgfsetdash{}{0pt}%
\pgfsys@defobject{currentmarker}{\pgfqpoint{0.000000in}{-0.027778in}}{\pgfqpoint{0.000000in}{0.000000in}}{%
\pgfpathmoveto{\pgfqpoint{0.000000in}{0.000000in}}%
\pgfpathlineto{\pgfqpoint{0.000000in}{-0.027778in}}%
\pgfusepath{stroke,fill}%
}%
\begin{pgfscope}%
\pgfsys@transformshift{1.787837in}{0.549073in}%
\pgfsys@useobject{currentmarker}{}%
\end{pgfscope}%
\end{pgfscope}%
\begin{pgfscope}%
\pgfsetbuttcap%
\pgfsetroundjoin%
\definecolor{currentfill}{rgb}{0.000000,0.000000,0.000000}%
\pgfsetfillcolor{currentfill}%
\pgfsetlinewidth{0.602250pt}%
\definecolor{currentstroke}{rgb}{0.000000,0.000000,0.000000}%
\pgfsetstrokecolor{currentstroke}%
\pgfsetdash{}{0pt}%
\pgfsys@defobject{currentmarker}{\pgfqpoint{0.000000in}{-0.027778in}}{\pgfqpoint{0.000000in}{0.000000in}}{%
\pgfpathmoveto{\pgfqpoint{0.000000in}{0.000000in}}%
\pgfpathlineto{\pgfqpoint{0.000000in}{-0.027778in}}%
\pgfusepath{stroke,fill}%
}%
\begin{pgfscope}%
\pgfsys@transformshift{1.848036in}{0.549073in}%
\pgfsys@useobject{currentmarker}{}%
\end{pgfscope}%
\end{pgfscope}%
\begin{pgfscope}%
\pgfsetbuttcap%
\pgfsetroundjoin%
\definecolor{currentfill}{rgb}{0.000000,0.000000,0.000000}%
\pgfsetfillcolor{currentfill}%
\pgfsetlinewidth{0.602250pt}%
\definecolor{currentstroke}{rgb}{0.000000,0.000000,0.000000}%
\pgfsetstrokecolor{currentstroke}%
\pgfsetdash{}{0pt}%
\pgfsys@defobject{currentmarker}{\pgfqpoint{0.000000in}{-0.027778in}}{\pgfqpoint{0.000000in}{0.000000in}}{%
\pgfpathmoveto{\pgfqpoint{0.000000in}{0.000000in}}%
\pgfpathlineto{\pgfqpoint{0.000000in}{-0.027778in}}%
\pgfusepath{stroke,fill}%
}%
\begin{pgfscope}%
\pgfsys@transformshift{1.901135in}{0.549073in}%
\pgfsys@useobject{currentmarker}{}%
\end{pgfscope}%
\end{pgfscope}%
\begin{pgfscope}%
\pgfsetbuttcap%
\pgfsetroundjoin%
\definecolor{currentfill}{rgb}{0.000000,0.000000,0.000000}%
\pgfsetfillcolor{currentfill}%
\pgfsetlinewidth{0.602250pt}%
\definecolor{currentstroke}{rgb}{0.000000,0.000000,0.000000}%
\pgfsetstrokecolor{currentstroke}%
\pgfsetdash{}{0pt}%
\pgfsys@defobject{currentmarker}{\pgfqpoint{0.000000in}{-0.027778in}}{\pgfqpoint{0.000000in}{0.000000in}}{%
\pgfpathmoveto{\pgfqpoint{0.000000in}{0.000000in}}%
\pgfpathlineto{\pgfqpoint{0.000000in}{-0.027778in}}%
\pgfusepath{stroke,fill}%
}%
\begin{pgfscope}%
\pgfsys@transformshift{2.261120in}{0.549073in}%
\pgfsys@useobject{currentmarker}{}%
\end{pgfscope}%
\end{pgfscope}%
\begin{pgfscope}%
\pgfsetbuttcap%
\pgfsetroundjoin%
\definecolor{currentfill}{rgb}{0.000000,0.000000,0.000000}%
\pgfsetfillcolor{currentfill}%
\pgfsetlinewidth{0.602250pt}%
\definecolor{currentstroke}{rgb}{0.000000,0.000000,0.000000}%
\pgfsetstrokecolor{currentstroke}%
\pgfsetdash{}{0pt}%
\pgfsys@defobject{currentmarker}{\pgfqpoint{0.000000in}{-0.027778in}}{\pgfqpoint{0.000000in}{0.000000in}}{%
\pgfpathmoveto{\pgfqpoint{0.000000in}{0.000000in}}%
\pgfpathlineto{\pgfqpoint{0.000000in}{-0.027778in}}%
\pgfusepath{stroke,fill}%
}%
\begin{pgfscope}%
\pgfsys@transformshift{2.443912in}{0.549073in}%
\pgfsys@useobject{currentmarker}{}%
\end{pgfscope}%
\end{pgfscope}%
\begin{pgfscope}%
\pgfsetbuttcap%
\pgfsetroundjoin%
\definecolor{currentfill}{rgb}{0.000000,0.000000,0.000000}%
\pgfsetfillcolor{currentfill}%
\pgfsetlinewidth{0.602250pt}%
\definecolor{currentstroke}{rgb}{0.000000,0.000000,0.000000}%
\pgfsetstrokecolor{currentstroke}%
\pgfsetdash{}{0pt}%
\pgfsys@defobject{currentmarker}{\pgfqpoint{0.000000in}{-0.027778in}}{\pgfqpoint{0.000000in}{0.000000in}}{%
\pgfpathmoveto{\pgfqpoint{0.000000in}{0.000000in}}%
\pgfpathlineto{\pgfqpoint{0.000000in}{-0.027778in}}%
\pgfusepath{stroke,fill}%
}%
\begin{pgfscope}%
\pgfsys@transformshift{2.573605in}{0.549073in}%
\pgfsys@useobject{currentmarker}{}%
\end{pgfscope}%
\end{pgfscope}%
\begin{pgfscope}%
\definecolor{textcolor}{rgb}{0.000000,0.000000,0.000000}%
\pgfsetstrokecolor{textcolor}%
\pgfsetfillcolor{textcolor}%
\pgftext[x=1.690663in,y=0.248148in,,top]{\color{textcolor}{\rmfamily\fontsize{12.000000}{14.400000}\selectfont\catcode`\^=\active\def^{\ifmmode\sp\else\^{}\fi}\catcode`\%=\active\def%{\%}$n_{\Omega} + n_{\Psi}$}}%
\end{pgfscope}%
\begin{pgfscope}%
\pgfsetbuttcap%
\pgfsetroundjoin%
\definecolor{currentfill}{rgb}{0.000000,0.000000,0.000000}%
\pgfsetfillcolor{currentfill}%
\pgfsetlinewidth{0.803000pt}%
\definecolor{currentstroke}{rgb}{0.000000,0.000000,0.000000}%
\pgfsetstrokecolor{currentstroke}%
\pgfsetdash{}{0pt}%
\pgfsys@defobject{currentmarker}{\pgfqpoint{-0.048611in}{0.000000in}}{\pgfqpoint{-0.000000in}{0.000000in}}{%
\pgfpathmoveto{\pgfqpoint{-0.000000in}{0.000000in}}%
\pgfpathlineto{\pgfqpoint{-0.048611in}{0.000000in}}%
\pgfusepath{stroke,fill}%
}%
\begin{pgfscope}%
\pgfsys@transformshift{0.721913in}{0.866478in}%
\pgfsys@useobject{currentmarker}{}%
\end{pgfscope}%
\end{pgfscope}%
\begin{pgfscope}%
\definecolor{textcolor}{rgb}{0.000000,0.000000,0.000000}%
\pgfsetstrokecolor{textcolor}%
\pgfsetfillcolor{textcolor}%
\pgftext[x=0.303703in, y=0.808607in, left, base]{\color{textcolor}{\rmfamily\fontsize{12.000000}{14.400000}\selectfont\catcode`\^=\active\def^{\ifmmode\sp\else\^{}\fi}\catcode`\%=\active\def%{\%}$\mathdefault{10^{-3}}$}}%
\end{pgfscope}%
\begin{pgfscope}%
\pgfsetbuttcap%
\pgfsetroundjoin%
\definecolor{currentfill}{rgb}{0.000000,0.000000,0.000000}%
\pgfsetfillcolor{currentfill}%
\pgfsetlinewidth{0.803000pt}%
\definecolor{currentstroke}{rgb}{0.000000,0.000000,0.000000}%
\pgfsetstrokecolor{currentstroke}%
\pgfsetdash{}{0pt}%
\pgfsys@defobject{currentmarker}{\pgfqpoint{-0.048611in}{0.000000in}}{\pgfqpoint{-0.000000in}{0.000000in}}{%
\pgfpathmoveto{\pgfqpoint{-0.000000in}{0.000000in}}%
\pgfpathlineto{\pgfqpoint{-0.048611in}{0.000000in}}%
\pgfusepath{stroke,fill}%
}%
\begin{pgfscope}%
\pgfsys@transformshift{0.721913in}{1.378851in}%
\pgfsys@useobject{currentmarker}{}%
\end{pgfscope}%
\end{pgfscope}%
\begin{pgfscope}%
\definecolor{textcolor}{rgb}{0.000000,0.000000,0.000000}%
\pgfsetstrokecolor{textcolor}%
\pgfsetfillcolor{textcolor}%
\pgftext[x=0.303703in, y=1.320980in, left, base]{\color{textcolor}{\rmfamily\fontsize{12.000000}{14.400000}\selectfont\catcode`\^=\active\def^{\ifmmode\sp\else\^{}\fi}\catcode`\%=\active\def%{\%}$\mathdefault{10^{-2}}$}}%
\end{pgfscope}%
\begin{pgfscope}%
\pgfsetbuttcap%
\pgfsetroundjoin%
\definecolor{currentfill}{rgb}{0.000000,0.000000,0.000000}%
\pgfsetfillcolor{currentfill}%
\pgfsetlinewidth{0.803000pt}%
\definecolor{currentstroke}{rgb}{0.000000,0.000000,0.000000}%
\pgfsetstrokecolor{currentstroke}%
\pgfsetdash{}{0pt}%
\pgfsys@defobject{currentmarker}{\pgfqpoint{-0.048611in}{0.000000in}}{\pgfqpoint{-0.000000in}{0.000000in}}{%
\pgfpathmoveto{\pgfqpoint{-0.000000in}{0.000000in}}%
\pgfpathlineto{\pgfqpoint{-0.048611in}{0.000000in}}%
\pgfusepath{stroke,fill}%
}%
\begin{pgfscope}%
\pgfsys@transformshift{0.721913in}{1.891224in}%
\pgfsys@useobject{currentmarker}{}%
\end{pgfscope}%
\end{pgfscope}%
\begin{pgfscope}%
\definecolor{textcolor}{rgb}{0.000000,0.000000,0.000000}%
\pgfsetstrokecolor{textcolor}%
\pgfsetfillcolor{textcolor}%
\pgftext[x=0.303703in, y=1.833353in, left, base]{\color{textcolor}{\rmfamily\fontsize{12.000000}{14.400000}\selectfont\catcode`\^=\active\def^{\ifmmode\sp\else\^{}\fi}\catcode`\%=\active\def%{\%}$\mathdefault{10^{-1}}$}}%
\end{pgfscope}%
\begin{pgfscope}%
\pgfsetbuttcap%
\pgfsetroundjoin%
\definecolor{currentfill}{rgb}{0.000000,0.000000,0.000000}%
\pgfsetfillcolor{currentfill}%
\pgfsetlinewidth{0.803000pt}%
\definecolor{currentstroke}{rgb}{0.000000,0.000000,0.000000}%
\pgfsetstrokecolor{currentstroke}%
\pgfsetdash{}{0pt}%
\pgfsys@defobject{currentmarker}{\pgfqpoint{-0.048611in}{0.000000in}}{\pgfqpoint{-0.000000in}{0.000000in}}{%
\pgfpathmoveto{\pgfqpoint{-0.000000in}{0.000000in}}%
\pgfpathlineto{\pgfqpoint{-0.048611in}{0.000000in}}%
\pgfusepath{stroke,fill}%
}%
\begin{pgfscope}%
\pgfsys@transformshift{0.721913in}{2.403597in}%
\pgfsys@useobject{currentmarker}{}%
\end{pgfscope}%
\end{pgfscope}%
\begin{pgfscope}%
\definecolor{textcolor}{rgb}{0.000000,0.000000,0.000000}%
\pgfsetstrokecolor{textcolor}%
\pgfsetfillcolor{textcolor}%
\pgftext[x=0.395525in, y=2.345726in, left, base]{\color{textcolor}{\rmfamily\fontsize{12.000000}{14.400000}\selectfont\catcode`\^=\active\def^{\ifmmode\sp\else\^{}\fi}\catcode`\%=\active\def%{\%}$\mathdefault{10^{0}}$}}%
\end{pgfscope}%
\begin{pgfscope}%
\pgfsetbuttcap%
\pgfsetroundjoin%
\definecolor{currentfill}{rgb}{0.000000,0.000000,0.000000}%
\pgfsetfillcolor{currentfill}%
\pgfsetlinewidth{0.602250pt}%
\definecolor{currentstroke}{rgb}{0.000000,0.000000,0.000000}%
\pgfsetstrokecolor{currentstroke}%
\pgfsetdash{}{0pt}%
\pgfsys@defobject{currentmarker}{\pgfqpoint{-0.027778in}{0.000000in}}{\pgfqpoint{-0.000000in}{0.000000in}}{%
\pgfpathmoveto{\pgfqpoint{-0.000000in}{0.000000in}}%
\pgfpathlineto{\pgfqpoint{-0.027778in}{0.000000in}}%
\pgfusepath{stroke,fill}%
}%
\begin{pgfscope}%
\pgfsys@transformshift{0.721913in}{0.598569in}%
\pgfsys@useobject{currentmarker}{}%
\end{pgfscope}%
\end{pgfscope}%
\begin{pgfscope}%
\pgfsetbuttcap%
\pgfsetroundjoin%
\definecolor{currentfill}{rgb}{0.000000,0.000000,0.000000}%
\pgfsetfillcolor{currentfill}%
\pgfsetlinewidth{0.602250pt}%
\definecolor{currentstroke}{rgb}{0.000000,0.000000,0.000000}%
\pgfsetstrokecolor{currentstroke}%
\pgfsetdash{}{0pt}%
\pgfsys@defobject{currentmarker}{\pgfqpoint{-0.027778in}{0.000000in}}{\pgfqpoint{-0.000000in}{0.000000in}}{%
\pgfpathmoveto{\pgfqpoint{-0.000000in}{0.000000in}}%
\pgfpathlineto{\pgfqpoint{-0.027778in}{0.000000in}}%
\pgfusepath{stroke,fill}%
}%
\begin{pgfscope}%
\pgfsys@transformshift{0.721913in}{0.662584in}%
\pgfsys@useobject{currentmarker}{}%
\end{pgfscope}%
\end{pgfscope}%
\begin{pgfscope}%
\pgfsetbuttcap%
\pgfsetroundjoin%
\definecolor{currentfill}{rgb}{0.000000,0.000000,0.000000}%
\pgfsetfillcolor{currentfill}%
\pgfsetlinewidth{0.602250pt}%
\definecolor{currentstroke}{rgb}{0.000000,0.000000,0.000000}%
\pgfsetstrokecolor{currentstroke}%
\pgfsetdash{}{0pt}%
\pgfsys@defobject{currentmarker}{\pgfqpoint{-0.027778in}{0.000000in}}{\pgfqpoint{-0.000000in}{0.000000in}}{%
\pgfpathmoveto{\pgfqpoint{-0.000000in}{0.000000in}}%
\pgfpathlineto{\pgfqpoint{-0.027778in}{0.000000in}}%
\pgfusepath{stroke,fill}%
}%
\begin{pgfscope}%
\pgfsys@transformshift{0.721913in}{0.712238in}%
\pgfsys@useobject{currentmarker}{}%
\end{pgfscope}%
\end{pgfscope}%
\begin{pgfscope}%
\pgfsetbuttcap%
\pgfsetroundjoin%
\definecolor{currentfill}{rgb}{0.000000,0.000000,0.000000}%
\pgfsetfillcolor{currentfill}%
\pgfsetlinewidth{0.602250pt}%
\definecolor{currentstroke}{rgb}{0.000000,0.000000,0.000000}%
\pgfsetstrokecolor{currentstroke}%
\pgfsetdash{}{0pt}%
\pgfsys@defobject{currentmarker}{\pgfqpoint{-0.027778in}{0.000000in}}{\pgfqpoint{-0.000000in}{0.000000in}}{%
\pgfpathmoveto{\pgfqpoint{-0.000000in}{0.000000in}}%
\pgfpathlineto{\pgfqpoint{-0.027778in}{0.000000in}}%
\pgfusepath{stroke,fill}%
}%
\begin{pgfscope}%
\pgfsys@transformshift{0.721913in}{0.752808in}%
\pgfsys@useobject{currentmarker}{}%
\end{pgfscope}%
\end{pgfscope}%
\begin{pgfscope}%
\pgfsetbuttcap%
\pgfsetroundjoin%
\definecolor{currentfill}{rgb}{0.000000,0.000000,0.000000}%
\pgfsetfillcolor{currentfill}%
\pgfsetlinewidth{0.602250pt}%
\definecolor{currentstroke}{rgb}{0.000000,0.000000,0.000000}%
\pgfsetstrokecolor{currentstroke}%
\pgfsetdash{}{0pt}%
\pgfsys@defobject{currentmarker}{\pgfqpoint{-0.027778in}{0.000000in}}{\pgfqpoint{-0.000000in}{0.000000in}}{%
\pgfpathmoveto{\pgfqpoint{-0.000000in}{0.000000in}}%
\pgfpathlineto{\pgfqpoint{-0.027778in}{0.000000in}}%
\pgfusepath{stroke,fill}%
}%
\begin{pgfscope}%
\pgfsys@transformshift{0.721913in}{0.787110in}%
\pgfsys@useobject{currentmarker}{}%
\end{pgfscope}%
\end{pgfscope}%
\begin{pgfscope}%
\pgfsetbuttcap%
\pgfsetroundjoin%
\definecolor{currentfill}{rgb}{0.000000,0.000000,0.000000}%
\pgfsetfillcolor{currentfill}%
\pgfsetlinewidth{0.602250pt}%
\definecolor{currentstroke}{rgb}{0.000000,0.000000,0.000000}%
\pgfsetstrokecolor{currentstroke}%
\pgfsetdash{}{0pt}%
\pgfsys@defobject{currentmarker}{\pgfqpoint{-0.027778in}{0.000000in}}{\pgfqpoint{-0.000000in}{0.000000in}}{%
\pgfpathmoveto{\pgfqpoint{-0.000000in}{0.000000in}}%
\pgfpathlineto{\pgfqpoint{-0.027778in}{0.000000in}}%
\pgfusepath{stroke,fill}%
}%
\begin{pgfscope}%
\pgfsys@transformshift{0.721913in}{0.816824in}%
\pgfsys@useobject{currentmarker}{}%
\end{pgfscope}%
\end{pgfscope}%
\begin{pgfscope}%
\pgfsetbuttcap%
\pgfsetroundjoin%
\definecolor{currentfill}{rgb}{0.000000,0.000000,0.000000}%
\pgfsetfillcolor{currentfill}%
\pgfsetlinewidth{0.602250pt}%
\definecolor{currentstroke}{rgb}{0.000000,0.000000,0.000000}%
\pgfsetstrokecolor{currentstroke}%
\pgfsetdash{}{0pt}%
\pgfsys@defobject{currentmarker}{\pgfqpoint{-0.027778in}{0.000000in}}{\pgfqpoint{-0.000000in}{0.000000in}}{%
\pgfpathmoveto{\pgfqpoint{-0.000000in}{0.000000in}}%
\pgfpathlineto{\pgfqpoint{-0.027778in}{0.000000in}}%
\pgfusepath{stroke,fill}%
}%
\begin{pgfscope}%
\pgfsys@transformshift{0.721913in}{0.843033in}%
\pgfsys@useobject{currentmarker}{}%
\end{pgfscope}%
\end{pgfscope}%
\begin{pgfscope}%
\pgfsetbuttcap%
\pgfsetroundjoin%
\definecolor{currentfill}{rgb}{0.000000,0.000000,0.000000}%
\pgfsetfillcolor{currentfill}%
\pgfsetlinewidth{0.602250pt}%
\definecolor{currentstroke}{rgb}{0.000000,0.000000,0.000000}%
\pgfsetstrokecolor{currentstroke}%
\pgfsetdash{}{0pt}%
\pgfsys@defobject{currentmarker}{\pgfqpoint{-0.027778in}{0.000000in}}{\pgfqpoint{-0.000000in}{0.000000in}}{%
\pgfpathmoveto{\pgfqpoint{-0.000000in}{0.000000in}}%
\pgfpathlineto{\pgfqpoint{-0.027778in}{0.000000in}}%
\pgfusepath{stroke,fill}%
}%
\begin{pgfscope}%
\pgfsys@transformshift{0.721913in}{1.020717in}%
\pgfsys@useobject{currentmarker}{}%
\end{pgfscope}%
\end{pgfscope}%
\begin{pgfscope}%
\pgfsetbuttcap%
\pgfsetroundjoin%
\definecolor{currentfill}{rgb}{0.000000,0.000000,0.000000}%
\pgfsetfillcolor{currentfill}%
\pgfsetlinewidth{0.602250pt}%
\definecolor{currentstroke}{rgb}{0.000000,0.000000,0.000000}%
\pgfsetstrokecolor{currentstroke}%
\pgfsetdash{}{0pt}%
\pgfsys@defobject{currentmarker}{\pgfqpoint{-0.027778in}{0.000000in}}{\pgfqpoint{-0.000000in}{0.000000in}}{%
\pgfpathmoveto{\pgfqpoint{-0.000000in}{0.000000in}}%
\pgfpathlineto{\pgfqpoint{-0.027778in}{0.000000in}}%
\pgfusepath{stroke,fill}%
}%
\begin{pgfscope}%
\pgfsys@transformshift{0.721913in}{1.110942in}%
\pgfsys@useobject{currentmarker}{}%
\end{pgfscope}%
\end{pgfscope}%
\begin{pgfscope}%
\pgfsetbuttcap%
\pgfsetroundjoin%
\definecolor{currentfill}{rgb}{0.000000,0.000000,0.000000}%
\pgfsetfillcolor{currentfill}%
\pgfsetlinewidth{0.602250pt}%
\definecolor{currentstroke}{rgb}{0.000000,0.000000,0.000000}%
\pgfsetstrokecolor{currentstroke}%
\pgfsetdash{}{0pt}%
\pgfsys@defobject{currentmarker}{\pgfqpoint{-0.027778in}{0.000000in}}{\pgfqpoint{-0.000000in}{0.000000in}}{%
\pgfpathmoveto{\pgfqpoint{-0.000000in}{0.000000in}}%
\pgfpathlineto{\pgfqpoint{-0.027778in}{0.000000in}}%
\pgfusepath{stroke,fill}%
}%
\begin{pgfscope}%
\pgfsys@transformshift{0.721913in}{1.174957in}%
\pgfsys@useobject{currentmarker}{}%
\end{pgfscope}%
\end{pgfscope}%
\begin{pgfscope}%
\pgfsetbuttcap%
\pgfsetroundjoin%
\definecolor{currentfill}{rgb}{0.000000,0.000000,0.000000}%
\pgfsetfillcolor{currentfill}%
\pgfsetlinewidth{0.602250pt}%
\definecolor{currentstroke}{rgb}{0.000000,0.000000,0.000000}%
\pgfsetstrokecolor{currentstroke}%
\pgfsetdash{}{0pt}%
\pgfsys@defobject{currentmarker}{\pgfqpoint{-0.027778in}{0.000000in}}{\pgfqpoint{-0.000000in}{0.000000in}}{%
\pgfpathmoveto{\pgfqpoint{-0.000000in}{0.000000in}}%
\pgfpathlineto{\pgfqpoint{-0.027778in}{0.000000in}}%
\pgfusepath{stroke,fill}%
}%
\begin{pgfscope}%
\pgfsys@transformshift{0.721913in}{1.224611in}%
\pgfsys@useobject{currentmarker}{}%
\end{pgfscope}%
\end{pgfscope}%
\begin{pgfscope}%
\pgfsetbuttcap%
\pgfsetroundjoin%
\definecolor{currentfill}{rgb}{0.000000,0.000000,0.000000}%
\pgfsetfillcolor{currentfill}%
\pgfsetlinewidth{0.602250pt}%
\definecolor{currentstroke}{rgb}{0.000000,0.000000,0.000000}%
\pgfsetstrokecolor{currentstroke}%
\pgfsetdash{}{0pt}%
\pgfsys@defobject{currentmarker}{\pgfqpoint{-0.027778in}{0.000000in}}{\pgfqpoint{-0.000000in}{0.000000in}}{%
\pgfpathmoveto{\pgfqpoint{-0.000000in}{0.000000in}}%
\pgfpathlineto{\pgfqpoint{-0.027778in}{0.000000in}}%
\pgfusepath{stroke,fill}%
}%
\begin{pgfscope}%
\pgfsys@transformshift{0.721913in}{1.265181in}%
\pgfsys@useobject{currentmarker}{}%
\end{pgfscope}%
\end{pgfscope}%
\begin{pgfscope}%
\pgfsetbuttcap%
\pgfsetroundjoin%
\definecolor{currentfill}{rgb}{0.000000,0.000000,0.000000}%
\pgfsetfillcolor{currentfill}%
\pgfsetlinewidth{0.602250pt}%
\definecolor{currentstroke}{rgb}{0.000000,0.000000,0.000000}%
\pgfsetstrokecolor{currentstroke}%
\pgfsetdash{}{0pt}%
\pgfsys@defobject{currentmarker}{\pgfqpoint{-0.027778in}{0.000000in}}{\pgfqpoint{-0.000000in}{0.000000in}}{%
\pgfpathmoveto{\pgfqpoint{-0.000000in}{0.000000in}}%
\pgfpathlineto{\pgfqpoint{-0.027778in}{0.000000in}}%
\pgfusepath{stroke,fill}%
}%
\begin{pgfscope}%
\pgfsys@transformshift{0.721913in}{1.299483in}%
\pgfsys@useobject{currentmarker}{}%
\end{pgfscope}%
\end{pgfscope}%
\begin{pgfscope}%
\pgfsetbuttcap%
\pgfsetroundjoin%
\definecolor{currentfill}{rgb}{0.000000,0.000000,0.000000}%
\pgfsetfillcolor{currentfill}%
\pgfsetlinewidth{0.602250pt}%
\definecolor{currentstroke}{rgb}{0.000000,0.000000,0.000000}%
\pgfsetstrokecolor{currentstroke}%
\pgfsetdash{}{0pt}%
\pgfsys@defobject{currentmarker}{\pgfqpoint{-0.027778in}{0.000000in}}{\pgfqpoint{-0.000000in}{0.000000in}}{%
\pgfpathmoveto{\pgfqpoint{-0.000000in}{0.000000in}}%
\pgfpathlineto{\pgfqpoint{-0.027778in}{0.000000in}}%
\pgfusepath{stroke,fill}%
}%
\begin{pgfscope}%
\pgfsys@transformshift{0.721913in}{1.329197in}%
\pgfsys@useobject{currentmarker}{}%
\end{pgfscope}%
\end{pgfscope}%
\begin{pgfscope}%
\pgfsetbuttcap%
\pgfsetroundjoin%
\definecolor{currentfill}{rgb}{0.000000,0.000000,0.000000}%
\pgfsetfillcolor{currentfill}%
\pgfsetlinewidth{0.602250pt}%
\definecolor{currentstroke}{rgb}{0.000000,0.000000,0.000000}%
\pgfsetstrokecolor{currentstroke}%
\pgfsetdash{}{0pt}%
\pgfsys@defobject{currentmarker}{\pgfqpoint{-0.027778in}{0.000000in}}{\pgfqpoint{-0.000000in}{0.000000in}}{%
\pgfpathmoveto{\pgfqpoint{-0.000000in}{0.000000in}}%
\pgfpathlineto{\pgfqpoint{-0.027778in}{0.000000in}}%
\pgfusepath{stroke,fill}%
}%
\begin{pgfscope}%
\pgfsys@transformshift{0.721913in}{1.355406in}%
\pgfsys@useobject{currentmarker}{}%
\end{pgfscope}%
\end{pgfscope}%
\begin{pgfscope}%
\pgfsetbuttcap%
\pgfsetroundjoin%
\definecolor{currentfill}{rgb}{0.000000,0.000000,0.000000}%
\pgfsetfillcolor{currentfill}%
\pgfsetlinewidth{0.602250pt}%
\definecolor{currentstroke}{rgb}{0.000000,0.000000,0.000000}%
\pgfsetstrokecolor{currentstroke}%
\pgfsetdash{}{0pt}%
\pgfsys@defobject{currentmarker}{\pgfqpoint{-0.027778in}{0.000000in}}{\pgfqpoint{-0.000000in}{0.000000in}}{%
\pgfpathmoveto{\pgfqpoint{-0.000000in}{0.000000in}}%
\pgfpathlineto{\pgfqpoint{-0.027778in}{0.000000in}}%
\pgfusepath{stroke,fill}%
}%
\begin{pgfscope}%
\pgfsys@transformshift{0.721913in}{1.533090in}%
\pgfsys@useobject{currentmarker}{}%
\end{pgfscope}%
\end{pgfscope}%
\begin{pgfscope}%
\pgfsetbuttcap%
\pgfsetroundjoin%
\definecolor{currentfill}{rgb}{0.000000,0.000000,0.000000}%
\pgfsetfillcolor{currentfill}%
\pgfsetlinewidth{0.602250pt}%
\definecolor{currentstroke}{rgb}{0.000000,0.000000,0.000000}%
\pgfsetstrokecolor{currentstroke}%
\pgfsetdash{}{0pt}%
\pgfsys@defobject{currentmarker}{\pgfqpoint{-0.027778in}{0.000000in}}{\pgfqpoint{-0.000000in}{0.000000in}}{%
\pgfpathmoveto{\pgfqpoint{-0.000000in}{0.000000in}}%
\pgfpathlineto{\pgfqpoint{-0.027778in}{0.000000in}}%
\pgfusepath{stroke,fill}%
}%
\begin{pgfscope}%
\pgfsys@transformshift{0.721913in}{1.623315in}%
\pgfsys@useobject{currentmarker}{}%
\end{pgfscope}%
\end{pgfscope}%
\begin{pgfscope}%
\pgfsetbuttcap%
\pgfsetroundjoin%
\definecolor{currentfill}{rgb}{0.000000,0.000000,0.000000}%
\pgfsetfillcolor{currentfill}%
\pgfsetlinewidth{0.602250pt}%
\definecolor{currentstroke}{rgb}{0.000000,0.000000,0.000000}%
\pgfsetstrokecolor{currentstroke}%
\pgfsetdash{}{0pt}%
\pgfsys@defobject{currentmarker}{\pgfqpoint{-0.027778in}{0.000000in}}{\pgfqpoint{-0.000000in}{0.000000in}}{%
\pgfpathmoveto{\pgfqpoint{-0.000000in}{0.000000in}}%
\pgfpathlineto{\pgfqpoint{-0.027778in}{0.000000in}}%
\pgfusepath{stroke,fill}%
}%
\begin{pgfscope}%
\pgfsys@transformshift{0.721913in}{1.687330in}%
\pgfsys@useobject{currentmarker}{}%
\end{pgfscope}%
\end{pgfscope}%
\begin{pgfscope}%
\pgfsetbuttcap%
\pgfsetroundjoin%
\definecolor{currentfill}{rgb}{0.000000,0.000000,0.000000}%
\pgfsetfillcolor{currentfill}%
\pgfsetlinewidth{0.602250pt}%
\definecolor{currentstroke}{rgb}{0.000000,0.000000,0.000000}%
\pgfsetstrokecolor{currentstroke}%
\pgfsetdash{}{0pt}%
\pgfsys@defobject{currentmarker}{\pgfqpoint{-0.027778in}{0.000000in}}{\pgfqpoint{-0.000000in}{0.000000in}}{%
\pgfpathmoveto{\pgfqpoint{-0.000000in}{0.000000in}}%
\pgfpathlineto{\pgfqpoint{-0.027778in}{0.000000in}}%
\pgfusepath{stroke,fill}%
}%
\begin{pgfscope}%
\pgfsys@transformshift{0.721913in}{1.736984in}%
\pgfsys@useobject{currentmarker}{}%
\end{pgfscope}%
\end{pgfscope}%
\begin{pgfscope}%
\pgfsetbuttcap%
\pgfsetroundjoin%
\definecolor{currentfill}{rgb}{0.000000,0.000000,0.000000}%
\pgfsetfillcolor{currentfill}%
\pgfsetlinewidth{0.602250pt}%
\definecolor{currentstroke}{rgb}{0.000000,0.000000,0.000000}%
\pgfsetstrokecolor{currentstroke}%
\pgfsetdash{}{0pt}%
\pgfsys@defobject{currentmarker}{\pgfqpoint{-0.027778in}{0.000000in}}{\pgfqpoint{-0.000000in}{0.000000in}}{%
\pgfpathmoveto{\pgfqpoint{-0.000000in}{0.000000in}}%
\pgfpathlineto{\pgfqpoint{-0.027778in}{0.000000in}}%
\pgfusepath{stroke,fill}%
}%
\begin{pgfscope}%
\pgfsys@transformshift{0.721913in}{1.777554in}%
\pgfsys@useobject{currentmarker}{}%
\end{pgfscope}%
\end{pgfscope}%
\begin{pgfscope}%
\pgfsetbuttcap%
\pgfsetroundjoin%
\definecolor{currentfill}{rgb}{0.000000,0.000000,0.000000}%
\pgfsetfillcolor{currentfill}%
\pgfsetlinewidth{0.602250pt}%
\definecolor{currentstroke}{rgb}{0.000000,0.000000,0.000000}%
\pgfsetstrokecolor{currentstroke}%
\pgfsetdash{}{0pt}%
\pgfsys@defobject{currentmarker}{\pgfqpoint{-0.027778in}{0.000000in}}{\pgfqpoint{-0.000000in}{0.000000in}}{%
\pgfpathmoveto{\pgfqpoint{-0.000000in}{0.000000in}}%
\pgfpathlineto{\pgfqpoint{-0.027778in}{0.000000in}}%
\pgfusepath{stroke,fill}%
}%
\begin{pgfscope}%
\pgfsys@transformshift{0.721913in}{1.811856in}%
\pgfsys@useobject{currentmarker}{}%
\end{pgfscope}%
\end{pgfscope}%
\begin{pgfscope}%
\pgfsetbuttcap%
\pgfsetroundjoin%
\definecolor{currentfill}{rgb}{0.000000,0.000000,0.000000}%
\pgfsetfillcolor{currentfill}%
\pgfsetlinewidth{0.602250pt}%
\definecolor{currentstroke}{rgb}{0.000000,0.000000,0.000000}%
\pgfsetstrokecolor{currentstroke}%
\pgfsetdash{}{0pt}%
\pgfsys@defobject{currentmarker}{\pgfqpoint{-0.027778in}{0.000000in}}{\pgfqpoint{-0.000000in}{0.000000in}}{%
\pgfpathmoveto{\pgfqpoint{-0.000000in}{0.000000in}}%
\pgfpathlineto{\pgfqpoint{-0.027778in}{0.000000in}}%
\pgfusepath{stroke,fill}%
}%
\begin{pgfscope}%
\pgfsys@transformshift{0.721913in}{1.841570in}%
\pgfsys@useobject{currentmarker}{}%
\end{pgfscope}%
\end{pgfscope}%
\begin{pgfscope}%
\pgfsetbuttcap%
\pgfsetroundjoin%
\definecolor{currentfill}{rgb}{0.000000,0.000000,0.000000}%
\pgfsetfillcolor{currentfill}%
\pgfsetlinewidth{0.602250pt}%
\definecolor{currentstroke}{rgb}{0.000000,0.000000,0.000000}%
\pgfsetstrokecolor{currentstroke}%
\pgfsetdash{}{0pt}%
\pgfsys@defobject{currentmarker}{\pgfqpoint{-0.027778in}{0.000000in}}{\pgfqpoint{-0.000000in}{0.000000in}}{%
\pgfpathmoveto{\pgfqpoint{-0.000000in}{0.000000in}}%
\pgfpathlineto{\pgfqpoint{-0.027778in}{0.000000in}}%
\pgfusepath{stroke,fill}%
}%
\begin{pgfscope}%
\pgfsys@transformshift{0.721913in}{1.867779in}%
\pgfsys@useobject{currentmarker}{}%
\end{pgfscope}%
\end{pgfscope}%
\begin{pgfscope}%
\pgfsetbuttcap%
\pgfsetroundjoin%
\definecolor{currentfill}{rgb}{0.000000,0.000000,0.000000}%
\pgfsetfillcolor{currentfill}%
\pgfsetlinewidth{0.602250pt}%
\definecolor{currentstroke}{rgb}{0.000000,0.000000,0.000000}%
\pgfsetstrokecolor{currentstroke}%
\pgfsetdash{}{0pt}%
\pgfsys@defobject{currentmarker}{\pgfqpoint{-0.027778in}{0.000000in}}{\pgfqpoint{-0.000000in}{0.000000in}}{%
\pgfpathmoveto{\pgfqpoint{-0.000000in}{0.000000in}}%
\pgfpathlineto{\pgfqpoint{-0.027778in}{0.000000in}}%
\pgfusepath{stroke,fill}%
}%
\begin{pgfscope}%
\pgfsys@transformshift{0.721913in}{2.045463in}%
\pgfsys@useobject{currentmarker}{}%
\end{pgfscope}%
\end{pgfscope}%
\begin{pgfscope}%
\pgfsetbuttcap%
\pgfsetroundjoin%
\definecolor{currentfill}{rgb}{0.000000,0.000000,0.000000}%
\pgfsetfillcolor{currentfill}%
\pgfsetlinewidth{0.602250pt}%
\definecolor{currentstroke}{rgb}{0.000000,0.000000,0.000000}%
\pgfsetstrokecolor{currentstroke}%
\pgfsetdash{}{0pt}%
\pgfsys@defobject{currentmarker}{\pgfqpoint{-0.027778in}{0.000000in}}{\pgfqpoint{-0.000000in}{0.000000in}}{%
\pgfpathmoveto{\pgfqpoint{-0.000000in}{0.000000in}}%
\pgfpathlineto{\pgfqpoint{-0.027778in}{0.000000in}}%
\pgfusepath{stroke,fill}%
}%
\begin{pgfscope}%
\pgfsys@transformshift{0.721913in}{2.135688in}%
\pgfsys@useobject{currentmarker}{}%
\end{pgfscope}%
\end{pgfscope}%
\begin{pgfscope}%
\pgfsetbuttcap%
\pgfsetroundjoin%
\definecolor{currentfill}{rgb}{0.000000,0.000000,0.000000}%
\pgfsetfillcolor{currentfill}%
\pgfsetlinewidth{0.602250pt}%
\definecolor{currentstroke}{rgb}{0.000000,0.000000,0.000000}%
\pgfsetstrokecolor{currentstroke}%
\pgfsetdash{}{0pt}%
\pgfsys@defobject{currentmarker}{\pgfqpoint{-0.027778in}{0.000000in}}{\pgfqpoint{-0.000000in}{0.000000in}}{%
\pgfpathmoveto{\pgfqpoint{-0.000000in}{0.000000in}}%
\pgfpathlineto{\pgfqpoint{-0.027778in}{0.000000in}}%
\pgfusepath{stroke,fill}%
}%
\begin{pgfscope}%
\pgfsys@transformshift{0.721913in}{2.199703in}%
\pgfsys@useobject{currentmarker}{}%
\end{pgfscope}%
\end{pgfscope}%
\begin{pgfscope}%
\pgfsetbuttcap%
\pgfsetroundjoin%
\definecolor{currentfill}{rgb}{0.000000,0.000000,0.000000}%
\pgfsetfillcolor{currentfill}%
\pgfsetlinewidth{0.602250pt}%
\definecolor{currentstroke}{rgb}{0.000000,0.000000,0.000000}%
\pgfsetstrokecolor{currentstroke}%
\pgfsetdash{}{0pt}%
\pgfsys@defobject{currentmarker}{\pgfqpoint{-0.027778in}{0.000000in}}{\pgfqpoint{-0.000000in}{0.000000in}}{%
\pgfpathmoveto{\pgfqpoint{-0.000000in}{0.000000in}}%
\pgfpathlineto{\pgfqpoint{-0.027778in}{0.000000in}}%
\pgfusepath{stroke,fill}%
}%
\begin{pgfscope}%
\pgfsys@transformshift{0.721913in}{2.249357in}%
\pgfsys@useobject{currentmarker}{}%
\end{pgfscope}%
\end{pgfscope}%
\begin{pgfscope}%
\pgfsetbuttcap%
\pgfsetroundjoin%
\definecolor{currentfill}{rgb}{0.000000,0.000000,0.000000}%
\pgfsetfillcolor{currentfill}%
\pgfsetlinewidth{0.602250pt}%
\definecolor{currentstroke}{rgb}{0.000000,0.000000,0.000000}%
\pgfsetstrokecolor{currentstroke}%
\pgfsetdash{}{0pt}%
\pgfsys@defobject{currentmarker}{\pgfqpoint{-0.027778in}{0.000000in}}{\pgfqpoint{-0.000000in}{0.000000in}}{%
\pgfpathmoveto{\pgfqpoint{-0.000000in}{0.000000in}}%
\pgfpathlineto{\pgfqpoint{-0.027778in}{0.000000in}}%
\pgfusepath{stroke,fill}%
}%
\begin{pgfscope}%
\pgfsys@transformshift{0.721913in}{2.289927in}%
\pgfsys@useobject{currentmarker}{}%
\end{pgfscope}%
\end{pgfscope}%
\begin{pgfscope}%
\pgfsetbuttcap%
\pgfsetroundjoin%
\definecolor{currentfill}{rgb}{0.000000,0.000000,0.000000}%
\pgfsetfillcolor{currentfill}%
\pgfsetlinewidth{0.602250pt}%
\definecolor{currentstroke}{rgb}{0.000000,0.000000,0.000000}%
\pgfsetstrokecolor{currentstroke}%
\pgfsetdash{}{0pt}%
\pgfsys@defobject{currentmarker}{\pgfqpoint{-0.027778in}{0.000000in}}{\pgfqpoint{-0.000000in}{0.000000in}}{%
\pgfpathmoveto{\pgfqpoint{-0.000000in}{0.000000in}}%
\pgfpathlineto{\pgfqpoint{-0.027778in}{0.000000in}}%
\pgfusepath{stroke,fill}%
}%
\begin{pgfscope}%
\pgfsys@transformshift{0.721913in}{2.324229in}%
\pgfsys@useobject{currentmarker}{}%
\end{pgfscope}%
\end{pgfscope}%
\begin{pgfscope}%
\pgfsetbuttcap%
\pgfsetroundjoin%
\definecolor{currentfill}{rgb}{0.000000,0.000000,0.000000}%
\pgfsetfillcolor{currentfill}%
\pgfsetlinewidth{0.602250pt}%
\definecolor{currentstroke}{rgb}{0.000000,0.000000,0.000000}%
\pgfsetstrokecolor{currentstroke}%
\pgfsetdash{}{0pt}%
\pgfsys@defobject{currentmarker}{\pgfqpoint{-0.027778in}{0.000000in}}{\pgfqpoint{-0.000000in}{0.000000in}}{%
\pgfpathmoveto{\pgfqpoint{-0.000000in}{0.000000in}}%
\pgfpathlineto{\pgfqpoint{-0.027778in}{0.000000in}}%
\pgfusepath{stroke,fill}%
}%
\begin{pgfscope}%
\pgfsys@transformshift{0.721913in}{2.353943in}%
\pgfsys@useobject{currentmarker}{}%
\end{pgfscope}%
\end{pgfscope}%
\begin{pgfscope}%
\pgfsetbuttcap%
\pgfsetroundjoin%
\definecolor{currentfill}{rgb}{0.000000,0.000000,0.000000}%
\pgfsetfillcolor{currentfill}%
\pgfsetlinewidth{0.602250pt}%
\definecolor{currentstroke}{rgb}{0.000000,0.000000,0.000000}%
\pgfsetstrokecolor{currentstroke}%
\pgfsetdash{}{0pt}%
\pgfsys@defobject{currentmarker}{\pgfqpoint{-0.027778in}{0.000000in}}{\pgfqpoint{-0.000000in}{0.000000in}}{%
\pgfpathmoveto{\pgfqpoint{-0.000000in}{0.000000in}}%
\pgfpathlineto{\pgfqpoint{-0.027778in}{0.000000in}}%
\pgfusepath{stroke,fill}%
}%
\begin{pgfscope}%
\pgfsys@transformshift{0.721913in}{2.380152in}%
\pgfsys@useobject{currentmarker}{}%
\end{pgfscope}%
\end{pgfscope}%
\begin{pgfscope}%
\definecolor{textcolor}{rgb}{0.000000,0.000000,0.000000}%
\pgfsetstrokecolor{textcolor}%
\pgfsetfillcolor{textcolor}%
\pgftext[x=0.248148in,y=1.511573in,,bottom,rotate=90.000000]{\color{textcolor}{\rmfamily\fontsize{12.000000}{14.400000}\selectfont\catcode`\^=\active\def^{\ifmmode\sp\else\^{}\fi}\catcode`\%=\active\def%{\%}$L^1$ relative error}}%
\end{pgfscope}%
\begin{pgfscope}%
\pgfpathrectangle{\pgfqpoint{0.721913in}{0.549073in}}{\pgfqpoint{1.937500in}{1.925000in}}%
\pgfusepath{clip}%
\pgfsetrectcap%
\pgfsetroundjoin%
\pgfsetlinewidth{1.003750pt}%
\definecolor{currentstroke}{rgb}{0.537255,0.647059,0.760784}%
\pgfsetstrokecolor{currentstroke}%
\pgfsetdash{}{0pt}%
\pgfpathmoveto{\pgfqpoint{0.809982in}{1.547882in}}%
\pgfpathlineto{\pgfqpoint{1.122467in}{1.446984in}}%
\pgfpathlineto{\pgfqpoint{1.420640in}{1.370887in}}%
\pgfpathlineto{\pgfqpoint{1.710766in}{1.336590in}}%
\pgfpathlineto{\pgfqpoint{1.999725in}{1.187797in}}%
\pgfpathlineto{\pgfqpoint{2.285257in}{1.162165in}}%
\pgfpathlineto{\pgfqpoint{2.571345in}{1.081331in}}%
\pgfusepath{stroke}%
\end{pgfscope}%
\begin{pgfscope}%
\pgfpathrectangle{\pgfqpoint{0.721913in}{0.549073in}}{\pgfqpoint{1.937500in}{1.925000in}}%
\pgfusepath{clip}%
\pgfsetbuttcap%
\pgfsetroundjoin%
\definecolor{currentfill}{rgb}{0.537255,0.647059,0.760784}%
\pgfsetfillcolor{currentfill}%
\pgfsetlinewidth{1.003750pt}%
\definecolor{currentstroke}{rgb}{0.537255,0.647059,0.760784}%
\pgfsetstrokecolor{currentstroke}%
\pgfsetdash{}{0pt}%
\pgfsys@defobject{currentmarker}{\pgfqpoint{-0.020833in}{-0.020833in}}{\pgfqpoint{0.020833in}{0.020833in}}{%
\pgfpathmoveto{\pgfqpoint{0.000000in}{-0.020833in}}%
\pgfpathcurveto{\pgfqpoint{0.005525in}{-0.020833in}}{\pgfqpoint{0.010825in}{-0.018638in}}{\pgfqpoint{0.014731in}{-0.014731in}}%
\pgfpathcurveto{\pgfqpoint{0.018638in}{-0.010825in}}{\pgfqpoint{0.020833in}{-0.005525in}}{\pgfqpoint{0.020833in}{0.000000in}}%
\pgfpathcurveto{\pgfqpoint{0.020833in}{0.005525in}}{\pgfqpoint{0.018638in}{0.010825in}}{\pgfqpoint{0.014731in}{0.014731in}}%
\pgfpathcurveto{\pgfqpoint{0.010825in}{0.018638in}}{\pgfqpoint{0.005525in}{0.020833in}}{\pgfqpoint{0.000000in}{0.020833in}}%
\pgfpathcurveto{\pgfqpoint{-0.005525in}{0.020833in}}{\pgfqpoint{-0.010825in}{0.018638in}}{\pgfqpoint{-0.014731in}{0.014731in}}%
\pgfpathcurveto{\pgfqpoint{-0.018638in}{0.010825in}}{\pgfqpoint{-0.020833in}{0.005525in}}{\pgfqpoint{-0.020833in}{0.000000in}}%
\pgfpathcurveto{\pgfqpoint{-0.020833in}{-0.005525in}}{\pgfqpoint{-0.018638in}{-0.010825in}}{\pgfqpoint{-0.014731in}{-0.014731in}}%
\pgfpathcurveto{\pgfqpoint{-0.010825in}{-0.018638in}}{\pgfqpoint{-0.005525in}{-0.020833in}}{\pgfqpoint{0.000000in}{-0.020833in}}%
\pgfpathlineto{\pgfqpoint{0.000000in}{-0.020833in}}%
\pgfpathclose%
\pgfusepath{stroke,fill}%
}%
\begin{pgfscope}%
\pgfsys@transformshift{0.809982in}{1.547882in}%
\pgfsys@useobject{currentmarker}{}%
\end{pgfscope}%
\begin{pgfscope}%
\pgfsys@transformshift{1.122467in}{1.446984in}%
\pgfsys@useobject{currentmarker}{}%
\end{pgfscope}%
\begin{pgfscope}%
\pgfsys@transformshift{1.420640in}{1.370887in}%
\pgfsys@useobject{currentmarker}{}%
\end{pgfscope}%
\begin{pgfscope}%
\pgfsys@transformshift{1.710766in}{1.336590in}%
\pgfsys@useobject{currentmarker}{}%
\end{pgfscope}%
\begin{pgfscope}%
\pgfsys@transformshift{1.999725in}{1.187797in}%
\pgfsys@useobject{currentmarker}{}%
\end{pgfscope}%
\begin{pgfscope}%
\pgfsys@transformshift{2.285257in}{1.162165in}%
\pgfsys@useobject{currentmarker}{}%
\end{pgfscope}%
\begin{pgfscope}%
\pgfsys@transformshift{2.571345in}{1.081331in}%
\pgfsys@useobject{currentmarker}{}%
\end{pgfscope}%
\end{pgfscope}%
\begin{pgfscope}%
\pgfpathrectangle{\pgfqpoint{0.721913in}{0.549073in}}{\pgfqpoint{1.937500in}{1.925000in}}%
\pgfusepath{clip}%
\pgfsetrectcap%
\pgfsetroundjoin%
\pgfsetlinewidth{1.003750pt}%
\definecolor{currentstroke}{rgb}{0.184314,0.270588,0.360784}%
\pgfsetstrokecolor{currentstroke}%
\pgfsetdash{}{0pt}%
\pgfpathmoveto{\pgfqpoint{0.809982in}{2.386573in}}%
\pgfpathlineto{\pgfqpoint{1.122467in}{2.377453in}}%
\pgfpathlineto{\pgfqpoint{1.420640in}{2.371539in}}%
\pgfpathlineto{\pgfqpoint{1.710766in}{2.370058in}}%
\pgfpathlineto{\pgfqpoint{1.999725in}{2.367742in}}%
\pgfpathlineto{\pgfqpoint{2.285257in}{2.363351in}}%
\pgfpathlineto{\pgfqpoint{2.571345in}{2.354812in}}%
\pgfusepath{stroke}%
\end{pgfscope}%
\begin{pgfscope}%
\pgfpathrectangle{\pgfqpoint{0.721913in}{0.549073in}}{\pgfqpoint{1.937500in}{1.925000in}}%
\pgfusepath{clip}%
\pgfsetbuttcap%
\pgfsetroundjoin%
\definecolor{currentfill}{rgb}{0.184314,0.270588,0.360784}%
\pgfsetfillcolor{currentfill}%
\pgfsetlinewidth{1.003750pt}%
\definecolor{currentstroke}{rgb}{0.184314,0.270588,0.360784}%
\pgfsetstrokecolor{currentstroke}%
\pgfsetdash{}{0pt}%
\pgfsys@defobject{currentmarker}{\pgfqpoint{-0.020833in}{-0.020833in}}{\pgfqpoint{0.020833in}{0.020833in}}{%
\pgfpathmoveto{\pgfqpoint{0.000000in}{-0.020833in}}%
\pgfpathcurveto{\pgfqpoint{0.005525in}{-0.020833in}}{\pgfqpoint{0.010825in}{-0.018638in}}{\pgfqpoint{0.014731in}{-0.014731in}}%
\pgfpathcurveto{\pgfqpoint{0.018638in}{-0.010825in}}{\pgfqpoint{0.020833in}{-0.005525in}}{\pgfqpoint{0.020833in}{0.000000in}}%
\pgfpathcurveto{\pgfqpoint{0.020833in}{0.005525in}}{\pgfqpoint{0.018638in}{0.010825in}}{\pgfqpoint{0.014731in}{0.014731in}}%
\pgfpathcurveto{\pgfqpoint{0.010825in}{0.018638in}}{\pgfqpoint{0.005525in}{0.020833in}}{\pgfqpoint{0.000000in}{0.020833in}}%
\pgfpathcurveto{\pgfqpoint{-0.005525in}{0.020833in}}{\pgfqpoint{-0.010825in}{0.018638in}}{\pgfqpoint{-0.014731in}{0.014731in}}%
\pgfpathcurveto{\pgfqpoint{-0.018638in}{0.010825in}}{\pgfqpoint{-0.020833in}{0.005525in}}{\pgfqpoint{-0.020833in}{0.000000in}}%
\pgfpathcurveto{\pgfqpoint{-0.020833in}{-0.005525in}}{\pgfqpoint{-0.018638in}{-0.010825in}}{\pgfqpoint{-0.014731in}{-0.014731in}}%
\pgfpathcurveto{\pgfqpoint{-0.010825in}{-0.018638in}}{\pgfqpoint{-0.005525in}{-0.020833in}}{\pgfqpoint{0.000000in}{-0.020833in}}%
\pgfpathlineto{\pgfqpoint{0.000000in}{-0.020833in}}%
\pgfpathclose%
\pgfusepath{stroke,fill}%
}%
\begin{pgfscope}%
\pgfsys@transformshift{0.809982in}{2.386573in}%
\pgfsys@useobject{currentmarker}{}%
\end{pgfscope}%
\begin{pgfscope}%
\pgfsys@transformshift{1.122467in}{2.377453in}%
\pgfsys@useobject{currentmarker}{}%
\end{pgfscope}%
\begin{pgfscope}%
\pgfsys@transformshift{1.420640in}{2.371539in}%
\pgfsys@useobject{currentmarker}{}%
\end{pgfscope}%
\begin{pgfscope}%
\pgfsys@transformshift{1.710766in}{2.370058in}%
\pgfsys@useobject{currentmarker}{}%
\end{pgfscope}%
\begin{pgfscope}%
\pgfsys@transformshift{1.999725in}{2.367742in}%
\pgfsys@useobject{currentmarker}{}%
\end{pgfscope}%
\begin{pgfscope}%
\pgfsys@transformshift{2.285257in}{2.363351in}%
\pgfsys@useobject{currentmarker}{}%
\end{pgfscope}%
\begin{pgfscope}%
\pgfsys@transformshift{2.571345in}{2.354812in}%
\pgfsys@useobject{currentmarker}{}%
\end{pgfscope}%
\end{pgfscope}%
\begin{pgfscope}%
\pgfpathrectangle{\pgfqpoint{0.721913in}{0.549073in}}{\pgfqpoint{1.937500in}{1.925000in}}%
\pgfusepath{clip}%
\pgfsetrectcap%
\pgfsetroundjoin%
\pgfsetlinewidth{1.003750pt}%
\definecolor{currentstroke}{rgb}{0.976471,0.505882,0.145098}%
\pgfsetstrokecolor{currentstroke}%
\pgfsetdash{}{0pt}%
\pgfpathmoveto{\pgfqpoint{0.809982in}{1.560082in}}%
\pgfpathlineto{\pgfqpoint{1.122467in}{1.297043in}}%
\pgfpathlineto{\pgfqpoint{1.420640in}{1.234130in}}%
\pgfpathlineto{\pgfqpoint{1.710766in}{0.702079in}}%
\pgfpathlineto{\pgfqpoint{1.999725in}{0.636573in}}%
\pgfpathlineto{\pgfqpoint{2.285257in}{0.676326in}}%
\pgfpathlineto{\pgfqpoint{2.571345in}{0.832523in}}%
\pgfusepath{stroke}%
\end{pgfscope}%
\begin{pgfscope}%
\pgfpathrectangle{\pgfqpoint{0.721913in}{0.549073in}}{\pgfqpoint{1.937500in}{1.925000in}}%
\pgfusepath{clip}%
\pgfsetbuttcap%
\pgfsetroundjoin%
\definecolor{currentfill}{rgb}{0.976471,0.505882,0.145098}%
\pgfsetfillcolor{currentfill}%
\pgfsetlinewidth{1.003750pt}%
\definecolor{currentstroke}{rgb}{0.976471,0.505882,0.145098}%
\pgfsetstrokecolor{currentstroke}%
\pgfsetdash{}{0pt}%
\pgfsys@defobject{currentmarker}{\pgfqpoint{-0.020833in}{-0.020833in}}{\pgfqpoint{0.020833in}{0.020833in}}{%
\pgfpathmoveto{\pgfqpoint{0.000000in}{-0.020833in}}%
\pgfpathcurveto{\pgfqpoint{0.005525in}{-0.020833in}}{\pgfqpoint{0.010825in}{-0.018638in}}{\pgfqpoint{0.014731in}{-0.014731in}}%
\pgfpathcurveto{\pgfqpoint{0.018638in}{-0.010825in}}{\pgfqpoint{0.020833in}{-0.005525in}}{\pgfqpoint{0.020833in}{0.000000in}}%
\pgfpathcurveto{\pgfqpoint{0.020833in}{0.005525in}}{\pgfqpoint{0.018638in}{0.010825in}}{\pgfqpoint{0.014731in}{0.014731in}}%
\pgfpathcurveto{\pgfqpoint{0.010825in}{0.018638in}}{\pgfqpoint{0.005525in}{0.020833in}}{\pgfqpoint{0.000000in}{0.020833in}}%
\pgfpathcurveto{\pgfqpoint{-0.005525in}{0.020833in}}{\pgfqpoint{-0.010825in}{0.018638in}}{\pgfqpoint{-0.014731in}{0.014731in}}%
\pgfpathcurveto{\pgfqpoint{-0.018638in}{0.010825in}}{\pgfqpoint{-0.020833in}{0.005525in}}{\pgfqpoint{-0.020833in}{0.000000in}}%
\pgfpathcurveto{\pgfqpoint{-0.020833in}{-0.005525in}}{\pgfqpoint{-0.018638in}{-0.010825in}}{\pgfqpoint{-0.014731in}{-0.014731in}}%
\pgfpathcurveto{\pgfqpoint{-0.010825in}{-0.018638in}}{\pgfqpoint{-0.005525in}{-0.020833in}}{\pgfqpoint{0.000000in}{-0.020833in}}%
\pgfpathlineto{\pgfqpoint{0.000000in}{-0.020833in}}%
\pgfpathclose%
\pgfusepath{stroke,fill}%
}%
\begin{pgfscope}%
\pgfsys@transformshift{0.809982in}{1.560082in}%
\pgfsys@useobject{currentmarker}{}%
\end{pgfscope}%
\begin{pgfscope}%
\pgfsys@transformshift{1.122467in}{1.297043in}%
\pgfsys@useobject{currentmarker}{}%
\end{pgfscope}%
\begin{pgfscope}%
\pgfsys@transformshift{1.420640in}{1.234130in}%
\pgfsys@useobject{currentmarker}{}%
\end{pgfscope}%
\begin{pgfscope}%
\pgfsys@transformshift{1.710766in}{0.702079in}%
\pgfsys@useobject{currentmarker}{}%
\end{pgfscope}%
\begin{pgfscope}%
\pgfsys@transformshift{1.999725in}{0.636573in}%
\pgfsys@useobject{currentmarker}{}%
\end{pgfscope}%
\begin{pgfscope}%
\pgfsys@transformshift{2.285257in}{0.676326in}%
\pgfsys@useobject{currentmarker}{}%
\end{pgfscope}%
\begin{pgfscope}%
\pgfsys@transformshift{2.571345in}{0.832523in}%
\pgfsys@useobject{currentmarker}{}%
\end{pgfscope}%
\end{pgfscope}%
\begin{pgfscope}%
\pgfsetrectcap%
\pgfsetmiterjoin%
\pgfsetlinewidth{0.803000pt}%
\definecolor{currentstroke}{rgb}{0.000000,0.000000,0.000000}%
\pgfsetstrokecolor{currentstroke}%
\pgfsetdash{}{0pt}%
\pgfpathmoveto{\pgfqpoint{0.721913in}{0.549073in}}%
\pgfpathlineto{\pgfqpoint{0.721913in}{2.474073in}}%
\pgfusepath{stroke}%
\end{pgfscope}%
\begin{pgfscope}%
\pgfsetrectcap%
\pgfsetmiterjoin%
\pgfsetlinewidth{0.803000pt}%
\definecolor{currentstroke}{rgb}{0.000000,0.000000,0.000000}%
\pgfsetstrokecolor{currentstroke}%
\pgfsetdash{}{0pt}%
\pgfpathmoveto{\pgfqpoint{2.659413in}{0.549073in}}%
\pgfpathlineto{\pgfqpoint{2.659413in}{2.474073in}}%
\pgfusepath{stroke}%
\end{pgfscope}%
\begin{pgfscope}%
\pgfsetrectcap%
\pgfsetmiterjoin%
\pgfsetlinewidth{0.803000pt}%
\definecolor{currentstroke}{rgb}{0.000000,0.000000,0.000000}%
\pgfsetstrokecolor{currentstroke}%
\pgfsetdash{}{0pt}%
\pgfpathmoveto{\pgfqpoint{0.721913in}{0.549073in}}%
\pgfpathlineto{\pgfqpoint{2.659413in}{0.549073in}}%
\pgfusepath{stroke}%
\end{pgfscope}%
\begin{pgfscope}%
\pgfsetrectcap%
\pgfsetmiterjoin%
\pgfsetlinewidth{0.803000pt}%
\definecolor{currentstroke}{rgb}{0.000000,0.000000,0.000000}%
\pgfsetstrokecolor{currentstroke}%
\pgfsetdash{}{0pt}%
\pgfpathmoveto{\pgfqpoint{0.721913in}{2.474073in}}%
\pgfpathlineto{\pgfqpoint{2.659413in}{2.474073in}}%
\pgfusepath{stroke}%
\end{pgfscope}%
\begin{pgfscope}%
\pgfsetbuttcap%
\pgfsetmiterjoin%
\definecolor{currentfill}{rgb}{1.000000,1.000000,1.000000}%
\pgfsetfillcolor{currentfill}%
\pgfsetfillopacity{0.800000}%
\pgfsetlinewidth{1.003750pt}%
\definecolor{currentstroke}{rgb}{0.800000,0.800000,0.800000}%
\pgfsetstrokecolor{currentstroke}%
\pgfsetstrokeopacity{0.800000}%
\pgfsetdash{}{0pt}%
\pgfpathmoveto{\pgfqpoint{1.515360in}{1.137962in}}%
\pgfpathlineto{\pgfqpoint{2.542747in}{1.137962in}}%
\pgfpathquadraticcurveto{\pgfqpoint{2.576080in}{1.137962in}}{\pgfqpoint{2.576080in}{1.171295in}}%
\pgfpathlineto{\pgfqpoint{2.576080in}{1.851850in}}%
\pgfpathquadraticcurveto{\pgfqpoint{2.576080in}{1.885183in}}{\pgfqpoint{2.542747in}{1.885183in}}%
\pgfpathlineto{\pgfqpoint{1.515360in}{1.885183in}}%
\pgfpathquadraticcurveto{\pgfqpoint{1.482027in}{1.885183in}}{\pgfqpoint{1.482027in}{1.851850in}}%
\pgfpathlineto{\pgfqpoint{1.482027in}{1.171295in}}%
\pgfpathquadraticcurveto{\pgfqpoint{1.482027in}{1.137962in}}{\pgfqpoint{1.515360in}{1.137962in}}%
\pgfpathlineto{\pgfqpoint{1.515360in}{1.137962in}}%
\pgfpathclose%
\pgfusepath{stroke,fill}%
\end{pgfscope}%
\begin{pgfscope}%
\pgfsetrectcap%
\pgfsetroundjoin%
\pgfsetlinewidth{1.003750pt}%
\definecolor{currentstroke}{rgb}{0.537255,0.647059,0.760784}%
\pgfsetstrokecolor{currentstroke}%
\pgfsetdash{}{0pt}%
\pgfpathmoveto{\pgfqpoint{1.548693in}{1.760183in}}%
\pgfpathlineto{\pgfqpoint{1.715360in}{1.760183in}}%
\pgfpathlineto{\pgfqpoint{1.882027in}{1.760183in}}%
\pgfusepath{stroke}%
\end{pgfscope}%
\begin{pgfscope}%
\pgfsetbuttcap%
\pgfsetroundjoin%
\definecolor{currentfill}{rgb}{0.537255,0.647059,0.760784}%
\pgfsetfillcolor{currentfill}%
\pgfsetlinewidth{1.003750pt}%
\definecolor{currentstroke}{rgb}{0.537255,0.647059,0.760784}%
\pgfsetstrokecolor{currentstroke}%
\pgfsetdash{}{0pt}%
\pgfsys@defobject{currentmarker}{\pgfqpoint{-0.020833in}{-0.020833in}}{\pgfqpoint{0.020833in}{0.020833in}}{%
\pgfpathmoveto{\pgfqpoint{0.000000in}{-0.020833in}}%
\pgfpathcurveto{\pgfqpoint{0.005525in}{-0.020833in}}{\pgfqpoint{0.010825in}{-0.018638in}}{\pgfqpoint{0.014731in}{-0.014731in}}%
\pgfpathcurveto{\pgfqpoint{0.018638in}{-0.010825in}}{\pgfqpoint{0.020833in}{-0.005525in}}{\pgfqpoint{0.020833in}{0.000000in}}%
\pgfpathcurveto{\pgfqpoint{0.020833in}{0.005525in}}{\pgfqpoint{0.018638in}{0.010825in}}{\pgfqpoint{0.014731in}{0.014731in}}%
\pgfpathcurveto{\pgfqpoint{0.010825in}{0.018638in}}{\pgfqpoint{0.005525in}{0.020833in}}{\pgfqpoint{0.000000in}{0.020833in}}%
\pgfpathcurveto{\pgfqpoint{-0.005525in}{0.020833in}}{\pgfqpoint{-0.010825in}{0.018638in}}{\pgfqpoint{-0.014731in}{0.014731in}}%
\pgfpathcurveto{\pgfqpoint{-0.018638in}{0.010825in}}{\pgfqpoint{-0.020833in}{0.005525in}}{\pgfqpoint{-0.020833in}{0.000000in}}%
\pgfpathcurveto{\pgfqpoint{-0.020833in}{-0.005525in}}{\pgfqpoint{-0.018638in}{-0.010825in}}{\pgfqpoint{-0.014731in}{-0.014731in}}%
\pgfpathcurveto{\pgfqpoint{-0.010825in}{-0.018638in}}{\pgfqpoint{-0.005525in}{-0.020833in}}{\pgfqpoint{0.000000in}{-0.020833in}}%
\pgfpathlineto{\pgfqpoint{0.000000in}{-0.020833in}}%
\pgfpathclose%
\pgfusepath{stroke,fill}%
}%
\begin{pgfscope}%
\pgfsys@transformshift{1.715360in}{1.760183in}%
\pgfsys@useobject{currentmarker}{}%
\end{pgfscope}%
\end{pgfscope}%
\begin{pgfscope}%
\definecolor{textcolor}{rgb}{0.000000,0.000000,0.000000}%
\pgfsetstrokecolor{textcolor}%
\pgfsetfillcolor{textcolor}%
\pgftext[x=2.015360in,y=1.701850in,left,base]{\color{textcolor}{\rmfamily\fontsize{12.000000}{14.400000}\selectfont\catcode`\^=\active\def^{\ifmmode\sp\else\^{}\fi}\catcode`\%=\active\def%{\%}DGC}}%
\end{pgfscope}%
\begin{pgfscope}%
\pgfsetrectcap%
\pgfsetroundjoin%
\pgfsetlinewidth{1.003750pt}%
\definecolor{currentstroke}{rgb}{0.184314,0.270588,0.360784}%
\pgfsetstrokecolor{currentstroke}%
\pgfsetdash{}{0pt}%
\pgfpathmoveto{\pgfqpoint{1.548693in}{1.527776in}}%
\pgfpathlineto{\pgfqpoint{1.715360in}{1.527776in}}%
\pgfpathlineto{\pgfqpoint{1.882027in}{1.527776in}}%
\pgfusepath{stroke}%
\end{pgfscope}%
\begin{pgfscope}%
\pgfsetbuttcap%
\pgfsetroundjoin%
\definecolor{currentfill}{rgb}{0.184314,0.270588,0.360784}%
\pgfsetfillcolor{currentfill}%
\pgfsetlinewidth{1.003750pt}%
\definecolor{currentstroke}{rgb}{0.184314,0.270588,0.360784}%
\pgfsetstrokecolor{currentstroke}%
\pgfsetdash{}{0pt}%
\pgfsys@defobject{currentmarker}{\pgfqpoint{-0.020833in}{-0.020833in}}{\pgfqpoint{0.020833in}{0.020833in}}{%
\pgfpathmoveto{\pgfqpoint{0.000000in}{-0.020833in}}%
\pgfpathcurveto{\pgfqpoint{0.005525in}{-0.020833in}}{\pgfqpoint{0.010825in}{-0.018638in}}{\pgfqpoint{0.014731in}{-0.014731in}}%
\pgfpathcurveto{\pgfqpoint{0.018638in}{-0.010825in}}{\pgfqpoint{0.020833in}{-0.005525in}}{\pgfqpoint{0.020833in}{0.000000in}}%
\pgfpathcurveto{\pgfqpoint{0.020833in}{0.005525in}}{\pgfqpoint{0.018638in}{0.010825in}}{\pgfqpoint{0.014731in}{0.014731in}}%
\pgfpathcurveto{\pgfqpoint{0.010825in}{0.018638in}}{\pgfqpoint{0.005525in}{0.020833in}}{\pgfqpoint{0.000000in}{0.020833in}}%
\pgfpathcurveto{\pgfqpoint{-0.005525in}{0.020833in}}{\pgfqpoint{-0.010825in}{0.018638in}}{\pgfqpoint{-0.014731in}{0.014731in}}%
\pgfpathcurveto{\pgfqpoint{-0.018638in}{0.010825in}}{\pgfqpoint{-0.020833in}{0.005525in}}{\pgfqpoint{-0.020833in}{0.000000in}}%
\pgfpathcurveto{\pgfqpoint{-0.020833in}{-0.005525in}}{\pgfqpoint{-0.018638in}{-0.010825in}}{\pgfqpoint{-0.014731in}{-0.014731in}}%
\pgfpathcurveto{\pgfqpoint{-0.010825in}{-0.018638in}}{\pgfqpoint{-0.005525in}{-0.020833in}}{\pgfqpoint{0.000000in}{-0.020833in}}%
\pgfpathlineto{\pgfqpoint{0.000000in}{-0.020833in}}%
\pgfpathclose%
\pgfusepath{stroke,fill}%
}%
\begin{pgfscope}%
\pgfsys@transformshift{1.715360in}{1.527776in}%
\pgfsys@useobject{currentmarker}{}%
\end{pgfscope}%
\end{pgfscope}%
\begin{pgfscope}%
\definecolor{textcolor}{rgb}{0.000000,0.000000,0.000000}%
\pgfsetstrokecolor{textcolor}%
\pgfsetfillcolor{textcolor}%
\pgftext[x=2.015360in,y=1.469443in,left,base]{\color{textcolor}{\rmfamily\fontsize{12.000000}{14.400000}\selectfont\catcode`\^=\active\def^{\ifmmode\sp\else\^{}\fi}\catcode`\%=\active\def%{\%}NC}}%
\end{pgfscope}%
\begin{pgfscope}%
\pgfsetrectcap%
\pgfsetroundjoin%
\pgfsetlinewidth{1.003750pt}%
\definecolor{currentstroke}{rgb}{0.976471,0.505882,0.145098}%
\pgfsetstrokecolor{currentstroke}%
\pgfsetdash{}{0pt}%
\pgfpathmoveto{\pgfqpoint{1.548693in}{1.295369in}}%
\pgfpathlineto{\pgfqpoint{1.715360in}{1.295369in}}%
\pgfpathlineto{\pgfqpoint{1.882027in}{1.295369in}}%
\pgfusepath{stroke}%
\end{pgfscope}%
\begin{pgfscope}%
\pgfsetbuttcap%
\pgfsetroundjoin%
\definecolor{currentfill}{rgb}{0.976471,0.505882,0.145098}%
\pgfsetfillcolor{currentfill}%
\pgfsetlinewidth{1.003750pt}%
\definecolor{currentstroke}{rgb}{0.976471,0.505882,0.145098}%
\pgfsetstrokecolor{currentstroke}%
\pgfsetdash{}{0pt}%
\pgfsys@defobject{currentmarker}{\pgfqpoint{-0.020833in}{-0.020833in}}{\pgfqpoint{0.020833in}{0.020833in}}{%
\pgfpathmoveto{\pgfqpoint{0.000000in}{-0.020833in}}%
\pgfpathcurveto{\pgfqpoint{0.005525in}{-0.020833in}}{\pgfqpoint{0.010825in}{-0.018638in}}{\pgfqpoint{0.014731in}{-0.014731in}}%
\pgfpathcurveto{\pgfqpoint{0.018638in}{-0.010825in}}{\pgfqpoint{0.020833in}{-0.005525in}}{\pgfqpoint{0.020833in}{0.000000in}}%
\pgfpathcurveto{\pgfqpoint{0.020833in}{0.005525in}}{\pgfqpoint{0.018638in}{0.010825in}}{\pgfqpoint{0.014731in}{0.014731in}}%
\pgfpathcurveto{\pgfqpoint{0.010825in}{0.018638in}}{\pgfqpoint{0.005525in}{0.020833in}}{\pgfqpoint{0.000000in}{0.020833in}}%
\pgfpathcurveto{\pgfqpoint{-0.005525in}{0.020833in}}{\pgfqpoint{-0.010825in}{0.018638in}}{\pgfqpoint{-0.014731in}{0.014731in}}%
\pgfpathcurveto{\pgfqpoint{-0.018638in}{0.010825in}}{\pgfqpoint{-0.020833in}{0.005525in}}{\pgfqpoint{-0.020833in}{0.000000in}}%
\pgfpathcurveto{\pgfqpoint{-0.020833in}{-0.005525in}}{\pgfqpoint{-0.018638in}{-0.010825in}}{\pgfqpoint{-0.014731in}{-0.014731in}}%
\pgfpathcurveto{\pgfqpoint{-0.010825in}{-0.018638in}}{\pgfqpoint{-0.005525in}{-0.020833in}}{\pgfqpoint{0.000000in}{-0.020833in}}%
\pgfpathlineto{\pgfqpoint{0.000000in}{-0.020833in}}%
\pgfpathclose%
\pgfusepath{stroke,fill}%
}%
\begin{pgfscope}%
\pgfsys@transformshift{1.715360in}{1.295369in}%
\pgfsys@useobject{currentmarker}{}%
\end{pgfscope}%
\end{pgfscope}%
\begin{pgfscope}%
\definecolor{textcolor}{rgb}{0.000000,0.000000,0.000000}%
\pgfsetstrokecolor{textcolor}%
\pgfsetfillcolor{textcolor}%
\pgftext[x=2.015360in,y=1.237036in,left,base]{\color{textcolor}{\rmfamily\fontsize{12.000000}{14.400000}\selectfont\catcode`\^=\active\def^{\ifmmode\sp\else\^{}\fi}\catcode`\%=\active\def%{\%}NC++}}%
\end{pgfscope}%
\end{pgfpicture}%
\makeatother%
\endgroup%

        \caption{Erdos992}
        \label{fig:5-experiments-multi-matrix-convergence-Erdos}
    \end{subfigure}
    \caption{For increasing values of \gls{sketch-size} $+$ \gls{num-hutchinson-queries}
    but fixed \gls{chebyshev-degree} $=2400$ we plot the $L^1$ relative approximation error \refequ{equ:5-experiments-L1-error}
    for multiple different matrices. As usual, we use
    Gaussian smoothing with \gls{smoothing-parameter} $=0.05$.}
    \label{fig:5-experiments-multi-matrix-convergence}
\end{figure}
