\chapter{Numerical experiments}
\label{chp:5-experiments}

\todo{Discuss the results from the plot, precise all parameters...}

In the previous chapters we have introduced multiple methods for approximating the
spectral density of a symmetric matrix $\mtx{A} \in \mathbb{R}^{n \times n}$. Now,
our goal is to compare these methods with each other and with other related methods
in terms of their accuracy and speed. In order to do so, we apply these algorithms
in multiple scenarios. We consider a specific example from density functional theory
\cite{lin2017randomized} for two different \gls{smoothing-kernel}, and
subsequently test the methods on various other matrices.\\

The accuracy is measured in terms of the discrete relative $L^1$ error of the approximated
spectral density $\widetilde{\phi}_{\sigma}^m$ from the spectral density $\phi_{\sigma}$
which we obtain using standard eigenvalue solvers\footnote{\url{https://numpy.org/doc/stable/reference/generated/numpy.linalg.eigvalsh.html}}.
\begin{equation}
    \frac{\sum_{i=1}^{n_t} |\widetilde{\phi}_{\sigma}^m(t_i) - \phi_{\sigma}(t_i)|}{\sum_{i=1}^{n_t} |\phi_{\sigma}(t_i)|}.
    \label{equ:5-experiments-L1-error}
\end{equation}
We use $n_t=100$ evenly spaced evaluation points covering the whole spectrum of
$\mtx{A}$. The choice of this metric can be justified by the fact that this
error is the midpoint quadrature rule applied to \todo{refer to choice of metric} \cite[chapter~9.2.1]{quarteroni2007numerical}. 

%%%%%%%%%%%%%%%%%%%%%%%%%%%%%%%%%%%%%%%%%%%%%%%%%%%%%%%%%%%%%%%%%%%%%%%%%%%%%%%%

\section{Model problem from density functional theory}
\label{sec:5-experiments-density-function}

For our first example, we assemble the matrix which arises from the second order
finite difference discretization of the differential operator
\begin{equation}
    \mathcal{A} u(\vct{x}) = - \Delta u(\vct{x}) + V(\vct{x}) u(\vct{x})
    \label{equ:5-experiments-electronic-hamiltonian}
\end{equation}
for a uniform mesh of size $h=0.6$. The potential $V$ results from a
lattice whose primitive cell is of side-length $L=6$ and in whose center the
charge
\begin{equation}
    \alpha \exp(-\frac{\lVert \vct{x} \rVert _2^2}{ 2 \beta^2 })
    \label{equ:5-experiments-gaussian-cell}
\end{equation}
with $\alpha = 4$, $\beta = 2$ is located. The computational domain is chosen
to span $c$ primitive cells in every spatial dimension, hence, yielding
discretization matrices which are growing in size with $c$. In our experiments
we consider the three-dimensional case, but for visualization purposes, we
illustrate the potential in \reffig{fig:5-experiments-periodic-gaussian-well}
in two dimensions.\\
\begin{figure}[ht]
    \begin{subfigure}[b]{0.32\columnwidth}
        %% Creator: Matplotlib, PGF backend
%%
%% To include the figure in your LaTeX document, write
%%   \input{<filename>.pgf}
%%
%% Make sure the required packages are loaded in your preamble
%%   \usepackage{pgf}
%%
%% Also ensure that all the required font packages are loaded; for instance,
%% the lmodern package is sometimes necessary when using math font.
%%   \usepackage{lmodern}
%%
%% Figures using additional raster images can only be included by \input if
%% they are in the same directory as the main LaTeX file. For loading figures
%% from other directories you can use the `import` package
%%   \usepackage{import}
%%
%% and then include the figures with
%%   \import{<path to file>}{<filename>.pgf}
%%
%% Matplotlib used the following preamble
%%   \def\mathdefault#1{#1}
%%   \everymath=\expandafter{\the\everymath\displaystyle}
%%   
%%   \makeatletter\@ifpackageloaded{underscore}{}{\usepackage[strings]{underscore}}\makeatother
%%
\begingroup%
\makeatletter%
\begin{pgfpicture}%
\pgfpathrectangle{\pgfpointorigin}{\pgfqpoint{1.969617in}{1.850740in}}%
\pgfusepath{use as bounding box, clip}%
\begin{pgfscope}%
\pgfsetbuttcap%
\pgfsetmiterjoin%
\definecolor{currentfill}{rgb}{1.000000,1.000000,1.000000}%
\pgfsetfillcolor{currentfill}%
\pgfsetlinewidth{0.000000pt}%
\definecolor{currentstroke}{rgb}{1.000000,1.000000,1.000000}%
\pgfsetstrokecolor{currentstroke}%
\pgfsetdash{}{0pt}%
\pgfpathmoveto{\pgfqpoint{0.000000in}{0.000000in}}%
\pgfpathlineto{\pgfqpoint{1.969617in}{0.000000in}}%
\pgfpathlineto{\pgfqpoint{1.969617in}{1.850740in}}%
\pgfpathlineto{\pgfqpoint{0.000000in}{1.850740in}}%
\pgfpathlineto{\pgfqpoint{0.000000in}{0.000000in}}%
\pgfpathclose%
\pgfusepath{fill}%
\end{pgfscope}%
\begin{pgfscope}%
\pgfsetbuttcap%
\pgfsetmiterjoin%
\definecolor{currentfill}{rgb}{1.000000,1.000000,1.000000}%
\pgfsetfillcolor{currentfill}%
\pgfsetlinewidth{0.000000pt}%
\definecolor{currentstroke}{rgb}{0.000000,0.000000,0.000000}%
\pgfsetstrokecolor{currentstroke}%
\pgfsetstrokeopacity{0.000000}%
\pgfsetdash{}{0pt}%
\pgfpathmoveto{\pgfqpoint{0.278819in}{0.345370in}}%
\pgfpathlineto{\pgfqpoint{1.828819in}{0.345370in}}%
\pgfpathlineto{\pgfqpoint{1.828819in}{1.692870in}}%
\pgfpathlineto{\pgfqpoint{0.278819in}{1.692870in}}%
\pgfpathlineto{\pgfqpoint{0.278819in}{0.345370in}}%
\pgfpathclose%
\pgfusepath{fill}%
\end{pgfscope}%
\begin{pgfscope}%
\pgfpathrectangle{\pgfqpoint{0.278819in}{0.345370in}}{\pgfqpoint{1.550000in}{1.347500in}}%
\pgfusepath{clip}%
\pgfsetbuttcap%
\pgfsetroundjoin%
\definecolor{currentfill}{rgb}{0.993545,0.862859,0.619299}%
\pgfsetfillcolor{currentfill}%
\pgfsetlinewidth{0.000000pt}%
\definecolor{currentstroke}{rgb}{0.000000,0.000000,0.000000}%
\pgfsetstrokecolor{currentstroke}%
\pgfsetdash{}{0pt}%
\pgfpathmoveto{\pgfqpoint{1.014677in}{0.873327in}}%
\pgfpathlineto{\pgfqpoint{1.030334in}{0.870750in}}%
\pgfpathlineto{\pgfqpoint{1.045990in}{0.869462in}}%
\pgfpathlineto{\pgfqpoint{1.061647in}{0.869462in}}%
\pgfpathlineto{\pgfqpoint{1.077303in}{0.870750in}}%
\pgfpathlineto{\pgfqpoint{1.092960in}{0.873327in}}%
\pgfpathlineto{\pgfqpoint{1.104615in}{0.876203in}}%
\pgfpathlineto{\pgfqpoint{1.108617in}{0.877276in}}%
\pgfpathlineto{\pgfqpoint{1.124273in}{0.882865in}}%
\pgfpathlineto{\pgfqpoint{1.139841in}{0.889814in}}%
\pgfpathlineto{\pgfqpoint{1.139930in}{0.889858in}}%
\pgfpathlineto{\pgfqpoint{1.155586in}{0.899059in}}%
\pgfpathlineto{\pgfqpoint{1.161972in}{0.903425in}}%
\pgfpathlineto{\pgfqpoint{1.171243in}{0.910510in}}%
\pgfpathlineto{\pgfqpoint{1.178750in}{0.917036in}}%
\pgfpathlineto{\pgfqpoint{1.186899in}{0.925096in}}%
\pgfpathlineto{\pgfqpoint{1.191922in}{0.930648in}}%
\pgfpathlineto{\pgfqpoint{1.202506in}{0.944259in}}%
\pgfpathlineto{\pgfqpoint{1.202556in}{0.944336in}}%
\pgfpathlineto{\pgfqpoint{1.210550in}{0.957870in}}%
\pgfpathlineto{\pgfqpoint{1.216979in}{0.971481in}}%
\pgfpathlineto{\pgfqpoint{1.218213in}{0.974959in}}%
\pgfpathlineto{\pgfqpoint{1.221521in}{0.985092in}}%
\pgfpathlineto{\pgfqpoint{1.224485in}{0.998703in}}%
\pgfpathlineto{\pgfqpoint{1.225967in}{1.012314in}}%
\pgfpathlineto{\pgfqpoint{1.225967in}{1.025925in}}%
\pgfpathlineto{\pgfqpoint{1.224485in}{1.039536in}}%
\pgfpathlineto{\pgfqpoint{1.221521in}{1.053148in}}%
\pgfpathlineto{\pgfqpoint{1.218213in}{1.063280in}}%
\pgfpathlineto{\pgfqpoint{1.216979in}{1.066759in}}%
\pgfpathlineto{\pgfqpoint{1.210550in}{1.080370in}}%
\pgfpathlineto{\pgfqpoint{1.202556in}{1.093903in}}%
\pgfpathlineto{\pgfqpoint{1.202506in}{1.093981in}}%
\pgfpathlineto{\pgfqpoint{1.191922in}{1.107592in}}%
\pgfpathlineto{\pgfqpoint{1.186899in}{1.113143in}}%
\pgfpathlineto{\pgfqpoint{1.178750in}{1.121203in}}%
\pgfpathlineto{\pgfqpoint{1.171243in}{1.127729in}}%
\pgfpathlineto{\pgfqpoint{1.161972in}{1.134814in}}%
\pgfpathlineto{\pgfqpoint{1.155586in}{1.139181in}}%
\pgfpathlineto{\pgfqpoint{1.139930in}{1.148381in}}%
\pgfpathlineto{\pgfqpoint{1.139841in}{1.148425in}}%
\pgfpathlineto{\pgfqpoint{1.124273in}{1.155375in}}%
\pgfpathlineto{\pgfqpoint{1.108617in}{1.160964in}}%
\pgfpathlineto{\pgfqpoint{1.104615in}{1.162036in}}%
\pgfpathlineto{\pgfqpoint{1.092960in}{1.164913in}}%
\pgfpathlineto{\pgfqpoint{1.077303in}{1.167489in}}%
\pgfpathlineto{\pgfqpoint{1.061647in}{1.168777in}}%
\pgfpathlineto{\pgfqpoint{1.045990in}{1.168777in}}%
\pgfpathlineto{\pgfqpoint{1.030334in}{1.167489in}}%
\pgfpathlineto{\pgfqpoint{1.014677in}{1.164913in}}%
\pgfpathlineto{\pgfqpoint{1.003022in}{1.162036in}}%
\pgfpathlineto{\pgfqpoint{0.999021in}{1.160964in}}%
\pgfpathlineto{\pgfqpoint{0.983364in}{1.155375in}}%
\pgfpathlineto{\pgfqpoint{0.967797in}{1.148425in}}%
\pgfpathlineto{\pgfqpoint{0.967708in}{1.148381in}}%
\pgfpathlineto{\pgfqpoint{0.952051in}{1.139181in}}%
\pgfpathlineto{\pgfqpoint{0.945666in}{1.134814in}}%
\pgfpathlineto{\pgfqpoint{0.936394in}{1.127729in}}%
\pgfpathlineto{\pgfqpoint{0.928888in}{1.121203in}}%
\pgfpathlineto{\pgfqpoint{0.920738in}{1.113143in}}%
\pgfpathlineto{\pgfqpoint{0.915715in}{1.107592in}}%
\pgfpathlineto{\pgfqpoint{0.905132in}{1.093981in}}%
\pgfpathlineto{\pgfqpoint{0.905081in}{1.093903in}}%
\pgfpathlineto{\pgfqpoint{0.897088in}{1.080370in}}%
\pgfpathlineto{\pgfqpoint{0.890659in}{1.066759in}}%
\pgfpathlineto{\pgfqpoint{0.889425in}{1.063280in}}%
\pgfpathlineto{\pgfqpoint{0.886116in}{1.053148in}}%
\pgfpathlineto{\pgfqpoint{0.883152in}{1.039536in}}%
\pgfpathlineto{\pgfqpoint{0.881671in}{1.025925in}}%
\pgfpathlineto{\pgfqpoint{0.881671in}{1.012314in}}%
\pgfpathlineto{\pgfqpoint{0.883152in}{0.998703in}}%
\pgfpathlineto{\pgfqpoint{0.886116in}{0.985092in}}%
\pgfpathlineto{\pgfqpoint{0.889425in}{0.974959in}}%
\pgfpathlineto{\pgfqpoint{0.890659in}{0.971481in}}%
\pgfpathlineto{\pgfqpoint{0.897088in}{0.957870in}}%
\pgfpathlineto{\pgfqpoint{0.905081in}{0.944336in}}%
\pgfpathlineto{\pgfqpoint{0.905132in}{0.944259in}}%
\pgfpathlineto{\pgfqpoint{0.915715in}{0.930648in}}%
\pgfpathlineto{\pgfqpoint{0.920738in}{0.925096in}}%
\pgfpathlineto{\pgfqpoint{0.928888in}{0.917036in}}%
\pgfpathlineto{\pgfqpoint{0.936394in}{0.910510in}}%
\pgfpathlineto{\pgfqpoint{0.945666in}{0.903425in}}%
\pgfpathlineto{\pgfqpoint{0.952051in}{0.899059in}}%
\pgfpathlineto{\pgfqpoint{0.967708in}{0.889858in}}%
\pgfpathlineto{\pgfqpoint{0.967797in}{0.889814in}}%
\pgfpathlineto{\pgfqpoint{0.983364in}{0.882865in}}%
\pgfpathlineto{\pgfqpoint{0.999021in}{0.877276in}}%
\pgfpathlineto{\pgfqpoint{1.003022in}{0.876203in}}%
\pgfpathlineto{\pgfqpoint{1.014677in}{0.873327in}}%
\pgfpathclose%
\pgfusepath{fill}%
\end{pgfscope}%
\begin{pgfscope}%
\pgfpathrectangle{\pgfqpoint{0.278819in}{0.345370in}}{\pgfqpoint{1.550000in}{1.347500in}}%
\pgfusepath{clip}%
\pgfsetbuttcap%
\pgfsetroundjoin%
\definecolor{currentfill}{rgb}{0.993326,0.602275,0.414390}%
\pgfsetfillcolor{currentfill}%
\pgfsetlinewidth{0.000000pt}%
\definecolor{currentstroke}{rgb}{0.000000,0.000000,0.000000}%
\pgfsetstrokecolor{currentstroke}%
\pgfsetdash{}{0pt}%
\pgfpathmoveto{\pgfqpoint{0.999021in}{0.739415in}}%
\pgfpathlineto{\pgfqpoint{1.014677in}{0.737106in}}%
\pgfpathlineto{\pgfqpoint{1.030334in}{0.735568in}}%
\pgfpathlineto{\pgfqpoint{1.045990in}{0.734799in}}%
\pgfpathlineto{\pgfqpoint{1.061647in}{0.734799in}}%
\pgfpathlineto{\pgfqpoint{1.077303in}{0.735568in}}%
\pgfpathlineto{\pgfqpoint{1.092960in}{0.737106in}}%
\pgfpathlineto{\pgfqpoint{1.108617in}{0.739415in}}%
\pgfpathlineto{\pgfqpoint{1.112069in}{0.740092in}}%
\pgfpathlineto{\pgfqpoint{1.124273in}{0.742535in}}%
\pgfpathlineto{\pgfqpoint{1.139930in}{0.746452in}}%
\pgfpathlineto{\pgfqpoint{1.155586in}{0.751154in}}%
\pgfpathlineto{\pgfqpoint{1.162877in}{0.753703in}}%
\pgfpathlineto{\pgfqpoint{1.171243in}{0.756707in}}%
\pgfpathlineto{\pgfqpoint{1.186899in}{0.763121in}}%
\pgfpathlineto{\pgfqpoint{1.196020in}{0.767314in}}%
\pgfpathlineto{\pgfqpoint{1.202556in}{0.770419in}}%
\pgfpathlineto{\pgfqpoint{1.218213in}{0.778656in}}%
\pgfpathlineto{\pgfqpoint{1.222157in}{0.780925in}}%
\pgfpathlineto{\pgfqpoint{1.233869in}{0.787933in}}%
\pgfpathlineto{\pgfqpoint{1.244037in}{0.794536in}}%
\pgfpathlineto{\pgfqpoint{1.249526in}{0.798265in}}%
\pgfpathlineto{\pgfqpoint{1.263051in}{0.808148in}}%
\pgfpathlineto{\pgfqpoint{1.265182in}{0.809786in}}%
\pgfpathlineto{\pgfqpoint{1.279772in}{0.821759in}}%
\pgfpathlineto{\pgfqpoint{1.280839in}{0.822686in}}%
\pgfpathlineto{\pgfqpoint{1.294610in}{0.835370in}}%
\pgfpathlineto{\pgfqpoint{1.296495in}{0.837223in}}%
\pgfpathlineto{\pgfqpoint{1.307864in}{0.848981in}}%
\pgfpathlineto{\pgfqpoint{1.312152in}{0.853753in}}%
\pgfpathlineto{\pgfqpoint{1.319748in}{0.862592in}}%
\pgfpathlineto{\pgfqpoint{1.327809in}{0.872774in}}%
\pgfpathlineto{\pgfqpoint{1.330419in}{0.876203in}}%
\pgfpathlineto{\pgfqpoint{1.339893in}{0.889814in}}%
\pgfpathlineto{\pgfqpoint{1.343465in}{0.895496in}}%
\pgfpathlineto{\pgfqpoint{1.348289in}{0.903425in}}%
\pgfpathlineto{\pgfqpoint{1.355666in}{0.917036in}}%
\pgfpathlineto{\pgfqpoint{1.359122in}{0.924309in}}%
\pgfpathlineto{\pgfqpoint{1.362054in}{0.930647in}}%
\pgfpathlineto{\pgfqpoint{1.367463in}{0.944259in}}%
\pgfpathlineto{\pgfqpoint{1.371968in}{0.957870in}}%
\pgfpathlineto{\pgfqpoint{1.374778in}{0.968479in}}%
\pgfpathlineto{\pgfqpoint{1.375557in}{0.971481in}}%
\pgfpathlineto{\pgfqpoint{1.378212in}{0.985092in}}%
\pgfpathlineto{\pgfqpoint{1.379982in}{0.998703in}}%
\pgfpathlineto{\pgfqpoint{1.380866in}{1.012314in}}%
\pgfpathlineto{\pgfqpoint{1.380866in}{1.025925in}}%
\pgfpathlineto{\pgfqpoint{1.379982in}{1.039536in}}%
\pgfpathlineto{\pgfqpoint{1.378212in}{1.053148in}}%
\pgfpathlineto{\pgfqpoint{1.375557in}{1.066759in}}%
\pgfpathlineto{\pgfqpoint{1.374778in}{1.069760in}}%
\pgfpathlineto{\pgfqpoint{1.371968in}{1.080370in}}%
\pgfpathlineto{\pgfqpoint{1.367463in}{1.093981in}}%
\pgfpathlineto{\pgfqpoint{1.362054in}{1.107592in}}%
\pgfpathlineto{\pgfqpoint{1.359122in}{1.113930in}}%
\pgfpathlineto{\pgfqpoint{1.355666in}{1.121203in}}%
\pgfpathlineto{\pgfqpoint{1.348289in}{1.134814in}}%
\pgfpathlineto{\pgfqpoint{1.343465in}{1.142743in}}%
\pgfpathlineto{\pgfqpoint{1.339893in}{1.148425in}}%
\pgfpathlineto{\pgfqpoint{1.330419in}{1.162036in}}%
\pgfpathlineto{\pgfqpoint{1.327809in}{1.165465in}}%
\pgfpathlineto{\pgfqpoint{1.319748in}{1.175647in}}%
\pgfpathlineto{\pgfqpoint{1.312152in}{1.184487in}}%
\pgfpathlineto{\pgfqpoint{1.307864in}{1.189259in}}%
\pgfpathlineto{\pgfqpoint{1.296495in}{1.201017in}}%
\pgfpathlineto{\pgfqpoint{1.294610in}{1.202870in}}%
\pgfpathlineto{\pgfqpoint{1.280839in}{1.215553in}}%
\pgfpathlineto{\pgfqpoint{1.279772in}{1.216481in}}%
\pgfpathlineto{\pgfqpoint{1.265182in}{1.228453in}}%
\pgfpathlineto{\pgfqpoint{1.263051in}{1.230092in}}%
\pgfpathlineto{\pgfqpoint{1.249526in}{1.239975in}}%
\pgfpathlineto{\pgfqpoint{1.244037in}{1.243703in}}%
\pgfpathlineto{\pgfqpoint{1.233869in}{1.250307in}}%
\pgfpathlineto{\pgfqpoint{1.222157in}{1.257314in}}%
\pgfpathlineto{\pgfqpoint{1.218213in}{1.259584in}}%
\pgfpathlineto{\pgfqpoint{1.202556in}{1.267820in}}%
\pgfpathlineto{\pgfqpoint{1.196020in}{1.270925in}}%
\pgfpathlineto{\pgfqpoint{1.186899in}{1.275119in}}%
\pgfpathlineto{\pgfqpoint{1.171243in}{1.281532in}}%
\pgfpathlineto{\pgfqpoint{1.162877in}{1.284536in}}%
\pgfpathlineto{\pgfqpoint{1.155586in}{1.287085in}}%
\pgfpathlineto{\pgfqpoint{1.139930in}{1.291788in}}%
\pgfpathlineto{\pgfqpoint{1.124273in}{1.295705in}}%
\pgfpathlineto{\pgfqpoint{1.112069in}{1.298148in}}%
\pgfpathlineto{\pgfqpoint{1.108617in}{1.298825in}}%
\pgfpathlineto{\pgfqpoint{1.092960in}{1.301133in}}%
\pgfpathlineto{\pgfqpoint{1.077303in}{1.302671in}}%
\pgfpathlineto{\pgfqpoint{1.061647in}{1.303440in}}%
\pgfpathlineto{\pgfqpoint{1.045990in}{1.303440in}}%
\pgfpathlineto{\pgfqpoint{1.030334in}{1.302671in}}%
\pgfpathlineto{\pgfqpoint{1.014677in}{1.301133in}}%
\pgfpathlineto{\pgfqpoint{0.999021in}{1.298825in}}%
\pgfpathlineto{\pgfqpoint{0.995568in}{1.298148in}}%
\pgfpathlineto{\pgfqpoint{0.983364in}{1.295705in}}%
\pgfpathlineto{\pgfqpoint{0.967708in}{1.291788in}}%
\pgfpathlineto{\pgfqpoint{0.952051in}{1.287085in}}%
\pgfpathlineto{\pgfqpoint{0.944760in}{1.284536in}}%
\pgfpathlineto{\pgfqpoint{0.936394in}{1.281532in}}%
\pgfpathlineto{\pgfqpoint{0.920738in}{1.275119in}}%
\pgfpathlineto{\pgfqpoint{0.911617in}{1.270925in}}%
\pgfpathlineto{\pgfqpoint{0.905081in}{1.267820in}}%
\pgfpathlineto{\pgfqpoint{0.889425in}{1.259584in}}%
\pgfpathlineto{\pgfqpoint{0.885481in}{1.257314in}}%
\pgfpathlineto{\pgfqpoint{0.873768in}{1.250307in}}%
\pgfpathlineto{\pgfqpoint{0.863600in}{1.243703in}}%
\pgfpathlineto{\pgfqpoint{0.858112in}{1.239975in}}%
\pgfpathlineto{\pgfqpoint{0.844586in}{1.230092in}}%
\pgfpathlineto{\pgfqpoint{0.842455in}{1.228453in}}%
\pgfpathlineto{\pgfqpoint{0.827865in}{1.216481in}}%
\pgfpathlineto{\pgfqpoint{0.826798in}{1.215553in}}%
\pgfpathlineto{\pgfqpoint{0.813027in}{1.202870in}}%
\pgfpathlineto{\pgfqpoint{0.811142in}{1.201017in}}%
\pgfpathlineto{\pgfqpoint{0.799774in}{1.189259in}}%
\pgfpathlineto{\pgfqpoint{0.795485in}{1.184487in}}%
\pgfpathlineto{\pgfqpoint{0.787889in}{1.175647in}}%
\pgfpathlineto{\pgfqpoint{0.779829in}{1.165465in}}%
\pgfpathlineto{\pgfqpoint{0.777218in}{1.162036in}}%
\pgfpathlineto{\pgfqpoint{0.767744in}{1.148425in}}%
\pgfpathlineto{\pgfqpoint{0.764172in}{1.142743in}}%
\pgfpathlineto{\pgfqpoint{0.759348in}{1.134814in}}%
\pgfpathlineto{\pgfqpoint{0.751971in}{1.121203in}}%
\pgfpathlineto{\pgfqpoint{0.748516in}{1.113930in}}%
\pgfpathlineto{\pgfqpoint{0.745583in}{1.107592in}}%
\pgfpathlineto{\pgfqpoint{0.740174in}{1.093981in}}%
\pgfpathlineto{\pgfqpoint{0.735669in}{1.080370in}}%
\pgfpathlineto{\pgfqpoint{0.732859in}{1.069760in}}%
\pgfpathlineto{\pgfqpoint{0.732080in}{1.066759in}}%
\pgfpathlineto{\pgfqpoint{0.729425in}{1.053148in}}%
\pgfpathlineto{\pgfqpoint{0.727655in}{1.039536in}}%
\pgfpathlineto{\pgfqpoint{0.726771in}{1.025925in}}%
\pgfpathlineto{\pgfqpoint{0.726771in}{1.012314in}}%
\pgfpathlineto{\pgfqpoint{0.727655in}{0.998703in}}%
\pgfpathlineto{\pgfqpoint{0.729425in}{0.985092in}}%
\pgfpathlineto{\pgfqpoint{0.732080in}{0.971481in}}%
\pgfpathlineto{\pgfqpoint{0.732859in}{0.968479in}}%
\pgfpathlineto{\pgfqpoint{0.735669in}{0.957870in}}%
\pgfpathlineto{\pgfqpoint{0.740174in}{0.944259in}}%
\pgfpathlineto{\pgfqpoint{0.745583in}{0.930647in}}%
\pgfpathlineto{\pgfqpoint{0.748516in}{0.924309in}}%
\pgfpathlineto{\pgfqpoint{0.751971in}{0.917036in}}%
\pgfpathlineto{\pgfqpoint{0.759348in}{0.903425in}}%
\pgfpathlineto{\pgfqpoint{0.764172in}{0.895496in}}%
\pgfpathlineto{\pgfqpoint{0.767744in}{0.889814in}}%
\pgfpathlineto{\pgfqpoint{0.777218in}{0.876203in}}%
\pgfpathlineto{\pgfqpoint{0.779829in}{0.872774in}}%
\pgfpathlineto{\pgfqpoint{0.787889in}{0.862592in}}%
\pgfpathlineto{\pgfqpoint{0.795485in}{0.853753in}}%
\pgfpathlineto{\pgfqpoint{0.799774in}{0.848981in}}%
\pgfpathlineto{\pgfqpoint{0.811142in}{0.837223in}}%
\pgfpathlineto{\pgfqpoint{0.813027in}{0.835370in}}%
\pgfpathlineto{\pgfqpoint{0.826798in}{0.822686in}}%
\pgfpathlineto{\pgfqpoint{0.827865in}{0.821759in}}%
\pgfpathlineto{\pgfqpoint{0.842455in}{0.809786in}}%
\pgfpathlineto{\pgfqpoint{0.844586in}{0.808148in}}%
\pgfpathlineto{\pgfqpoint{0.858112in}{0.798265in}}%
\pgfpathlineto{\pgfqpoint{0.863600in}{0.794536in}}%
\pgfpathlineto{\pgfqpoint{0.873768in}{0.787933in}}%
\pgfpathlineto{\pgfqpoint{0.885481in}{0.780925in}}%
\pgfpathlineto{\pgfqpoint{0.889425in}{0.778656in}}%
\pgfpathlineto{\pgfqpoint{0.905081in}{0.770419in}}%
\pgfpathlineto{\pgfqpoint{0.911617in}{0.767314in}}%
\pgfpathlineto{\pgfqpoint{0.920738in}{0.763121in}}%
\pgfpathlineto{\pgfqpoint{0.936394in}{0.756707in}}%
\pgfpathlineto{\pgfqpoint{0.944760in}{0.753703in}}%
\pgfpathlineto{\pgfqpoint{0.952051in}{0.751154in}}%
\pgfpathlineto{\pgfqpoint{0.967708in}{0.746452in}}%
\pgfpathlineto{\pgfqpoint{0.983364in}{0.742535in}}%
\pgfpathlineto{\pgfqpoint{0.995568in}{0.740092in}}%
\pgfpathlineto{\pgfqpoint{0.999021in}{0.739415in}}%
\pgfpathclose%
\pgfpathmoveto{\pgfqpoint{1.003022in}{0.876203in}}%
\pgfpathlineto{\pgfqpoint{0.999021in}{0.877276in}}%
\pgfpathlineto{\pgfqpoint{0.983364in}{0.882865in}}%
\pgfpathlineto{\pgfqpoint{0.967797in}{0.889814in}}%
\pgfpathlineto{\pgfqpoint{0.967708in}{0.889858in}}%
\pgfpathlineto{\pgfqpoint{0.952051in}{0.899059in}}%
\pgfpathlineto{\pgfqpoint{0.945666in}{0.903425in}}%
\pgfpathlineto{\pgfqpoint{0.936394in}{0.910510in}}%
\pgfpathlineto{\pgfqpoint{0.928888in}{0.917036in}}%
\pgfpathlineto{\pgfqpoint{0.920738in}{0.925096in}}%
\pgfpathlineto{\pgfqpoint{0.915715in}{0.930648in}}%
\pgfpathlineto{\pgfqpoint{0.905132in}{0.944259in}}%
\pgfpathlineto{\pgfqpoint{0.905081in}{0.944336in}}%
\pgfpathlineto{\pgfqpoint{0.897088in}{0.957870in}}%
\pgfpathlineto{\pgfqpoint{0.890659in}{0.971481in}}%
\pgfpathlineto{\pgfqpoint{0.889425in}{0.974959in}}%
\pgfpathlineto{\pgfqpoint{0.886116in}{0.985092in}}%
\pgfpathlineto{\pgfqpoint{0.883152in}{0.998703in}}%
\pgfpathlineto{\pgfqpoint{0.881671in}{1.012314in}}%
\pgfpathlineto{\pgfqpoint{0.881671in}{1.025925in}}%
\pgfpathlineto{\pgfqpoint{0.883152in}{1.039536in}}%
\pgfpathlineto{\pgfqpoint{0.886116in}{1.053148in}}%
\pgfpathlineto{\pgfqpoint{0.889425in}{1.063280in}}%
\pgfpathlineto{\pgfqpoint{0.890659in}{1.066759in}}%
\pgfpathlineto{\pgfqpoint{0.897088in}{1.080370in}}%
\pgfpathlineto{\pgfqpoint{0.905081in}{1.093903in}}%
\pgfpathlineto{\pgfqpoint{0.905132in}{1.093981in}}%
\pgfpathlineto{\pgfqpoint{0.915715in}{1.107592in}}%
\pgfpathlineto{\pgfqpoint{0.920738in}{1.113143in}}%
\pgfpathlineto{\pgfqpoint{0.928888in}{1.121203in}}%
\pgfpathlineto{\pgfqpoint{0.936394in}{1.127729in}}%
\pgfpathlineto{\pgfqpoint{0.945666in}{1.134814in}}%
\pgfpathlineto{\pgfqpoint{0.952051in}{1.139181in}}%
\pgfpathlineto{\pgfqpoint{0.967708in}{1.148381in}}%
\pgfpathlineto{\pgfqpoint{0.967797in}{1.148425in}}%
\pgfpathlineto{\pgfqpoint{0.983364in}{1.155375in}}%
\pgfpathlineto{\pgfqpoint{0.999021in}{1.160964in}}%
\pgfpathlineto{\pgfqpoint{1.003022in}{1.162036in}}%
\pgfpathlineto{\pgfqpoint{1.014677in}{1.164913in}}%
\pgfpathlineto{\pgfqpoint{1.030334in}{1.167489in}}%
\pgfpathlineto{\pgfqpoint{1.045990in}{1.168777in}}%
\pgfpathlineto{\pgfqpoint{1.061647in}{1.168777in}}%
\pgfpathlineto{\pgfqpoint{1.077303in}{1.167489in}}%
\pgfpathlineto{\pgfqpoint{1.092960in}{1.164913in}}%
\pgfpathlineto{\pgfqpoint{1.104615in}{1.162036in}}%
\pgfpathlineto{\pgfqpoint{1.108617in}{1.160964in}}%
\pgfpathlineto{\pgfqpoint{1.124273in}{1.155375in}}%
\pgfpathlineto{\pgfqpoint{1.139841in}{1.148425in}}%
\pgfpathlineto{\pgfqpoint{1.139930in}{1.148381in}}%
\pgfpathlineto{\pgfqpoint{1.155586in}{1.139181in}}%
\pgfpathlineto{\pgfqpoint{1.161972in}{1.134814in}}%
\pgfpathlineto{\pgfqpoint{1.171243in}{1.127729in}}%
\pgfpathlineto{\pgfqpoint{1.178750in}{1.121203in}}%
\pgfpathlineto{\pgfqpoint{1.186899in}{1.113143in}}%
\pgfpathlineto{\pgfqpoint{1.191922in}{1.107592in}}%
\pgfpathlineto{\pgfqpoint{1.202506in}{1.093981in}}%
\pgfpathlineto{\pgfqpoint{1.202556in}{1.093903in}}%
\pgfpathlineto{\pgfqpoint{1.210550in}{1.080370in}}%
\pgfpathlineto{\pgfqpoint{1.216979in}{1.066759in}}%
\pgfpathlineto{\pgfqpoint{1.218213in}{1.063280in}}%
\pgfpathlineto{\pgfqpoint{1.221521in}{1.053148in}}%
\pgfpathlineto{\pgfqpoint{1.224485in}{1.039536in}}%
\pgfpathlineto{\pgfqpoint{1.225967in}{1.025925in}}%
\pgfpathlineto{\pgfqpoint{1.225967in}{1.012314in}}%
\pgfpathlineto{\pgfqpoint{1.224485in}{0.998703in}}%
\pgfpathlineto{\pgfqpoint{1.221521in}{0.985092in}}%
\pgfpathlineto{\pgfqpoint{1.218213in}{0.974959in}}%
\pgfpathlineto{\pgfqpoint{1.216979in}{0.971481in}}%
\pgfpathlineto{\pgfqpoint{1.210550in}{0.957870in}}%
\pgfpathlineto{\pgfqpoint{1.202556in}{0.944336in}}%
\pgfpathlineto{\pgfqpoint{1.202506in}{0.944259in}}%
\pgfpathlineto{\pgfqpoint{1.191922in}{0.930648in}}%
\pgfpathlineto{\pgfqpoint{1.186899in}{0.925096in}}%
\pgfpathlineto{\pgfqpoint{1.178750in}{0.917036in}}%
\pgfpathlineto{\pgfqpoint{1.171243in}{0.910510in}}%
\pgfpathlineto{\pgfqpoint{1.161972in}{0.903425in}}%
\pgfpathlineto{\pgfqpoint{1.155586in}{0.899059in}}%
\pgfpathlineto{\pgfqpoint{1.139930in}{0.889858in}}%
\pgfpathlineto{\pgfqpoint{1.139841in}{0.889814in}}%
\pgfpathlineto{\pgfqpoint{1.124273in}{0.882865in}}%
\pgfpathlineto{\pgfqpoint{1.108617in}{0.877276in}}%
\pgfpathlineto{\pgfqpoint{1.104615in}{0.876203in}}%
\pgfpathlineto{\pgfqpoint{1.092960in}{0.873327in}}%
\pgfpathlineto{\pgfqpoint{1.077303in}{0.870750in}}%
\pgfpathlineto{\pgfqpoint{1.061647in}{0.869462in}}%
\pgfpathlineto{\pgfqpoint{1.045990in}{0.869462in}}%
\pgfpathlineto{\pgfqpoint{1.030334in}{0.870750in}}%
\pgfpathlineto{\pgfqpoint{1.014677in}{0.873327in}}%
\pgfpathlineto{\pgfqpoint{1.003022in}{0.876203in}}%
\pgfpathclose%
\pgfusepath{fill}%
\end{pgfscope}%
\begin{pgfscope}%
\pgfpathrectangle{\pgfqpoint{0.278819in}{0.345370in}}{\pgfqpoint{1.550000in}{1.347500in}}%
\pgfusepath{clip}%
\pgfsetbuttcap%
\pgfsetroundjoin%
\definecolor{currentfill}{rgb}{0.921884,0.341098,0.377376}%
\pgfsetfillcolor{currentfill}%
\pgfsetlinewidth{0.000000pt}%
\definecolor{currentstroke}{rgb}{0.000000,0.000000,0.000000}%
\pgfsetstrokecolor{currentstroke}%
\pgfsetdash{}{0pt}%
\pgfpathmoveto{\pgfqpoint{0.967708in}{0.631202in}}%
\pgfpathlineto{\pgfqpoint{0.983364in}{0.627741in}}%
\pgfpathlineto{\pgfqpoint{0.999021in}{0.624975in}}%
\pgfpathlineto{\pgfqpoint{1.014677in}{0.622900in}}%
\pgfpathlineto{\pgfqpoint{1.030334in}{0.621518in}}%
\pgfpathlineto{\pgfqpoint{1.045990in}{0.620826in}}%
\pgfpathlineto{\pgfqpoint{1.061647in}{0.620826in}}%
\pgfpathlineto{\pgfqpoint{1.077303in}{0.621518in}}%
\pgfpathlineto{\pgfqpoint{1.092960in}{0.622900in}}%
\pgfpathlineto{\pgfqpoint{1.108617in}{0.624975in}}%
\pgfpathlineto{\pgfqpoint{1.124273in}{0.627741in}}%
\pgfpathlineto{\pgfqpoint{1.139930in}{0.631202in}}%
\pgfpathlineto{\pgfqpoint{1.139935in}{0.631203in}}%
\pgfpathlineto{\pgfqpoint{1.155586in}{0.635291in}}%
\pgfpathlineto{\pgfqpoint{1.171243in}{0.640064in}}%
\pgfpathlineto{\pgfqpoint{1.184880in}{0.644814in}}%
\pgfpathlineto{\pgfqpoint{1.186899in}{0.645513in}}%
\pgfpathlineto{\pgfqpoint{1.202556in}{0.651585in}}%
\pgfpathlineto{\pgfqpoint{1.218213in}{0.658335in}}%
\pgfpathlineto{\pgfqpoint{1.218405in}{0.658425in}}%
\pgfpathlineto{\pgfqpoint{1.233869in}{0.665706in}}%
\pgfpathlineto{\pgfqpoint{1.246218in}{0.672036in}}%
\pgfpathlineto{\pgfqpoint{1.249526in}{0.673741in}}%
\pgfpathlineto{\pgfqpoint{1.265182in}{0.682419in}}%
\pgfpathlineto{\pgfqpoint{1.270627in}{0.685648in}}%
\pgfpathlineto{\pgfqpoint{1.280839in}{0.691764in}}%
\pgfpathlineto{\pgfqpoint{1.292598in}{0.699259in}}%
\pgfpathlineto{\pgfqpoint{1.296495in}{0.701780in}}%
\pgfpathlineto{\pgfqpoint{1.312152in}{0.712485in}}%
\pgfpathlineto{\pgfqpoint{1.312688in}{0.712870in}}%
\pgfpathlineto{\pgfqpoint{1.327809in}{0.723931in}}%
\pgfpathlineto{\pgfqpoint{1.331141in}{0.726481in}}%
\pgfpathlineto{\pgfqpoint{1.343465in}{0.736144in}}%
\pgfpathlineto{\pgfqpoint{1.348306in}{0.740092in}}%
\pgfpathlineto{\pgfqpoint{1.359122in}{0.749177in}}%
\pgfpathlineto{\pgfqpoint{1.364327in}{0.753703in}}%
\pgfpathlineto{\pgfqpoint{1.374778in}{0.763106in}}%
\pgfpathlineto{\pgfqpoint{1.379320in}{0.767314in}}%
\pgfpathlineto{\pgfqpoint{1.390435in}{0.778028in}}%
\pgfpathlineto{\pgfqpoint{1.393367in}{0.780925in}}%
\pgfpathlineto{\pgfqpoint{1.406091in}{0.794070in}}%
\pgfpathlineto{\pgfqpoint{1.406534in}{0.794536in}}%
\pgfpathlineto{\pgfqpoint{1.418847in}{0.808148in}}%
\pgfpathlineto{\pgfqpoint{1.421748in}{0.811536in}}%
\pgfpathlineto{\pgfqpoint{1.430369in}{0.821759in}}%
\pgfpathlineto{\pgfqpoint{1.437404in}{0.830636in}}%
\pgfpathlineto{\pgfqpoint{1.441118in}{0.835370in}}%
\pgfpathlineto{\pgfqpoint{1.451101in}{0.848981in}}%
\pgfpathlineto{\pgfqpoint{1.453061in}{0.851856in}}%
\pgfpathlineto{\pgfqpoint{1.460343in}{0.862592in}}%
\pgfpathlineto{\pgfqpoint{1.468718in}{0.876036in}}%
\pgfpathlineto{\pgfqpoint{1.468822in}{0.876203in}}%
\pgfpathlineto{\pgfqpoint{1.476585in}{0.889814in}}%
\pgfpathlineto{\pgfqpoint{1.483570in}{0.903425in}}%
\pgfpathlineto{\pgfqpoint{1.484374in}{0.905180in}}%
\pgfpathlineto{\pgfqpoint{1.489838in}{0.917036in}}%
\pgfpathlineto{\pgfqpoint{1.495329in}{0.930648in}}%
\pgfpathlineto{\pgfqpoint{1.500031in}{0.944254in}}%
\pgfpathlineto{\pgfqpoint{1.500032in}{0.944259in}}%
\pgfpathlineto{\pgfqpoint{1.504013in}{0.957870in}}%
\pgfpathlineto{\pgfqpoint{1.507195in}{0.971481in}}%
\pgfpathlineto{\pgfqpoint{1.509581in}{0.985092in}}%
\pgfpathlineto{\pgfqpoint{1.511172in}{0.998703in}}%
\pgfpathlineto{\pgfqpoint{1.511967in}{1.012314in}}%
\pgfpathlineto{\pgfqpoint{1.511967in}{1.025925in}}%
\pgfpathlineto{\pgfqpoint{1.511172in}{1.039536in}}%
\pgfpathlineto{\pgfqpoint{1.509581in}{1.053148in}}%
\pgfpathlineto{\pgfqpoint{1.507195in}{1.066759in}}%
\pgfpathlineto{\pgfqpoint{1.504013in}{1.080370in}}%
\pgfpathlineto{\pgfqpoint{1.500032in}{1.093981in}}%
\pgfpathlineto{\pgfqpoint{1.500031in}{1.093986in}}%
\pgfpathlineto{\pgfqpoint{1.495329in}{1.107592in}}%
\pgfpathlineto{\pgfqpoint{1.489838in}{1.121203in}}%
\pgfpathlineto{\pgfqpoint{1.484374in}{1.133059in}}%
\pgfpathlineto{\pgfqpoint{1.483570in}{1.134814in}}%
\pgfpathlineto{\pgfqpoint{1.476585in}{1.148425in}}%
\pgfpathlineto{\pgfqpoint{1.468822in}{1.162036in}}%
\pgfpathlineto{\pgfqpoint{1.468718in}{1.162204in}}%
\pgfpathlineto{\pgfqpoint{1.460343in}{1.175647in}}%
\pgfpathlineto{\pgfqpoint{1.453061in}{1.186383in}}%
\pgfpathlineto{\pgfqpoint{1.451101in}{1.189259in}}%
\pgfpathlineto{\pgfqpoint{1.441118in}{1.202870in}}%
\pgfpathlineto{\pgfqpoint{1.437404in}{1.207603in}}%
\pgfpathlineto{\pgfqpoint{1.430369in}{1.216481in}}%
\pgfpathlineto{\pgfqpoint{1.421748in}{1.226703in}}%
\pgfpathlineto{\pgfqpoint{1.418847in}{1.230092in}}%
\pgfpathlineto{\pgfqpoint{1.406534in}{1.243703in}}%
\pgfpathlineto{\pgfqpoint{1.406091in}{1.244169in}}%
\pgfpathlineto{\pgfqpoint{1.393367in}{1.257314in}}%
\pgfpathlineto{\pgfqpoint{1.390435in}{1.260212in}}%
\pgfpathlineto{\pgfqpoint{1.379320in}{1.270925in}}%
\pgfpathlineto{\pgfqpoint{1.374778in}{1.275134in}}%
\pgfpathlineto{\pgfqpoint{1.364327in}{1.284536in}}%
\pgfpathlineto{\pgfqpoint{1.359122in}{1.289062in}}%
\pgfpathlineto{\pgfqpoint{1.348306in}{1.298148in}}%
\pgfpathlineto{\pgfqpoint{1.343465in}{1.302096in}}%
\pgfpathlineto{\pgfqpoint{1.331141in}{1.311759in}}%
\pgfpathlineto{\pgfqpoint{1.327809in}{1.314308in}}%
\pgfpathlineto{\pgfqpoint{1.312688in}{1.325370in}}%
\pgfpathlineto{\pgfqpoint{1.312152in}{1.325754in}}%
\pgfpathlineto{\pgfqpoint{1.296495in}{1.336459in}}%
\pgfpathlineto{\pgfqpoint{1.292598in}{1.338981in}}%
\pgfpathlineto{\pgfqpoint{1.280839in}{1.346475in}}%
\pgfpathlineto{\pgfqpoint{1.270627in}{1.352592in}}%
\pgfpathlineto{\pgfqpoint{1.265182in}{1.355820in}}%
\pgfpathlineto{\pgfqpoint{1.249526in}{1.364499in}}%
\pgfpathlineto{\pgfqpoint{1.246218in}{1.366203in}}%
\pgfpathlineto{\pgfqpoint{1.233869in}{1.372534in}}%
\pgfpathlineto{\pgfqpoint{1.218405in}{1.379814in}}%
\pgfpathlineto{\pgfqpoint{1.218213in}{1.379905in}}%
\pgfpathlineto{\pgfqpoint{1.202556in}{1.386654in}}%
\pgfpathlineto{\pgfqpoint{1.186899in}{1.392726in}}%
\pgfpathlineto{\pgfqpoint{1.184880in}{1.393425in}}%
\pgfpathlineto{\pgfqpoint{1.171243in}{1.398176in}}%
\pgfpathlineto{\pgfqpoint{1.155586in}{1.402949in}}%
\pgfpathlineto{\pgfqpoint{1.139935in}{1.407036in}}%
\pgfpathlineto{\pgfqpoint{1.139930in}{1.407038in}}%
\pgfpathlineto{\pgfqpoint{1.124273in}{1.410498in}}%
\pgfpathlineto{\pgfqpoint{1.108617in}{1.413265in}}%
\pgfpathlineto{\pgfqpoint{1.092960in}{1.415339in}}%
\pgfpathlineto{\pgfqpoint{1.077303in}{1.416722in}}%
\pgfpathlineto{\pgfqpoint{1.061647in}{1.417413in}}%
\pgfpathlineto{\pgfqpoint{1.045990in}{1.417413in}}%
\pgfpathlineto{\pgfqpoint{1.030334in}{1.416722in}}%
\pgfpathlineto{\pgfqpoint{1.014677in}{1.415339in}}%
\pgfpathlineto{\pgfqpoint{0.999021in}{1.413265in}}%
\pgfpathlineto{\pgfqpoint{0.983364in}{1.410498in}}%
\pgfpathlineto{\pgfqpoint{0.967707in}{1.407038in}}%
\pgfpathlineto{\pgfqpoint{0.967702in}{1.407036in}}%
\pgfpathlineto{\pgfqpoint{0.952051in}{1.402949in}}%
\pgfpathlineto{\pgfqpoint{0.936394in}{1.398176in}}%
\pgfpathlineto{\pgfqpoint{0.922757in}{1.393425in}}%
\pgfpathlineto{\pgfqpoint{0.920738in}{1.392726in}}%
\pgfpathlineto{\pgfqpoint{0.905081in}{1.386654in}}%
\pgfpathlineto{\pgfqpoint{0.889425in}{1.379905in}}%
\pgfpathlineto{\pgfqpoint{0.889232in}{1.379814in}}%
\pgfpathlineto{\pgfqpoint{0.873768in}{1.372534in}}%
\pgfpathlineto{\pgfqpoint{0.861419in}{1.366203in}}%
\pgfpathlineto{\pgfqpoint{0.858112in}{1.364499in}}%
\pgfpathlineto{\pgfqpoint{0.842455in}{1.355820in}}%
\pgfpathlineto{\pgfqpoint{0.837010in}{1.352592in}}%
\pgfpathlineto{\pgfqpoint{0.826798in}{1.346475in}}%
\pgfpathlineto{\pgfqpoint{0.815040in}{1.338981in}}%
\pgfpathlineto{\pgfqpoint{0.811142in}{1.336459in}}%
\pgfpathlineto{\pgfqpoint{0.795485in}{1.325754in}}%
\pgfpathlineto{\pgfqpoint{0.794949in}{1.325370in}}%
\pgfpathlineto{\pgfqpoint{0.779829in}{1.314308in}}%
\pgfpathlineto{\pgfqpoint{0.776496in}{1.311759in}}%
\pgfpathlineto{\pgfqpoint{0.764172in}{1.302096in}}%
\pgfpathlineto{\pgfqpoint{0.759331in}{1.298148in}}%
\pgfpathlineto{\pgfqpoint{0.748516in}{1.289062in}}%
\pgfpathlineto{\pgfqpoint{0.743310in}{1.284536in}}%
\pgfpathlineto{\pgfqpoint{0.732859in}{1.275134in}}%
\pgfpathlineto{\pgfqpoint{0.728318in}{1.270925in}}%
\pgfpathlineto{\pgfqpoint{0.717202in}{1.260212in}}%
\pgfpathlineto{\pgfqpoint{0.714270in}{1.257314in}}%
\pgfpathlineto{\pgfqpoint{0.701546in}{1.244169in}}%
\pgfpathlineto{\pgfqpoint{0.701104in}{1.243703in}}%
\pgfpathlineto{\pgfqpoint{0.688790in}{1.230092in}}%
\pgfpathlineto{\pgfqpoint{0.685889in}{1.226703in}}%
\pgfpathlineto{\pgfqpoint{0.677269in}{1.216481in}}%
\pgfpathlineto{\pgfqpoint{0.670233in}{1.207603in}}%
\pgfpathlineto{\pgfqpoint{0.666520in}{1.202870in}}%
\pgfpathlineto{\pgfqpoint{0.656536in}{1.189259in}}%
\pgfpathlineto{\pgfqpoint{0.654576in}{1.186383in}}%
\pgfpathlineto{\pgfqpoint{0.647294in}{1.175647in}}%
\pgfpathlineto{\pgfqpoint{0.638920in}{1.162204in}}%
\pgfpathlineto{\pgfqpoint{0.638815in}{1.162036in}}%
\pgfpathlineto{\pgfqpoint{0.631052in}{1.148425in}}%
\pgfpathlineto{\pgfqpoint{0.624067in}{1.134814in}}%
\pgfpathlineto{\pgfqpoint{0.623263in}{1.133059in}}%
\pgfpathlineto{\pgfqpoint{0.617799in}{1.121203in}}%
\pgfpathlineto{\pgfqpoint{0.612308in}{1.107592in}}%
\pgfpathlineto{\pgfqpoint{0.607606in}{1.093986in}}%
\pgfpathlineto{\pgfqpoint{0.607605in}{1.093981in}}%
\pgfpathlineto{\pgfqpoint{0.603625in}{1.080370in}}%
\pgfpathlineto{\pgfqpoint{0.600442in}{1.066759in}}%
\pgfpathlineto{\pgfqpoint{0.598056in}{1.053148in}}%
\pgfpathlineto{\pgfqpoint{0.596466in}{1.039536in}}%
\pgfpathlineto{\pgfqpoint{0.595671in}{1.025925in}}%
\pgfpathlineto{\pgfqpoint{0.595671in}{1.012314in}}%
\pgfpathlineto{\pgfqpoint{0.596466in}{0.998703in}}%
\pgfpathlineto{\pgfqpoint{0.598056in}{0.985092in}}%
\pgfpathlineto{\pgfqpoint{0.600442in}{0.971481in}}%
\pgfpathlineto{\pgfqpoint{0.603625in}{0.957870in}}%
\pgfpathlineto{\pgfqpoint{0.607605in}{0.944259in}}%
\pgfpathlineto{\pgfqpoint{0.607606in}{0.944254in}}%
\pgfpathlineto{\pgfqpoint{0.612308in}{0.930648in}}%
\pgfpathlineto{\pgfqpoint{0.617799in}{0.917036in}}%
\pgfpathlineto{\pgfqpoint{0.623263in}{0.905180in}}%
\pgfpathlineto{\pgfqpoint{0.624067in}{0.903425in}}%
\pgfpathlineto{\pgfqpoint{0.631052in}{0.889814in}}%
\pgfpathlineto{\pgfqpoint{0.638815in}{0.876203in}}%
\pgfpathlineto{\pgfqpoint{0.638920in}{0.876036in}}%
\pgfpathlineto{\pgfqpoint{0.647294in}{0.862592in}}%
\pgfpathlineto{\pgfqpoint{0.654576in}{0.851856in}}%
\pgfpathlineto{\pgfqpoint{0.656536in}{0.848981in}}%
\pgfpathlineto{\pgfqpoint{0.666520in}{0.835370in}}%
\pgfpathlineto{\pgfqpoint{0.670233in}{0.830636in}}%
\pgfpathlineto{\pgfqpoint{0.677269in}{0.821759in}}%
\pgfpathlineto{\pgfqpoint{0.685889in}{0.811536in}}%
\pgfpathlineto{\pgfqpoint{0.688790in}{0.808148in}}%
\pgfpathlineto{\pgfqpoint{0.701104in}{0.794536in}}%
\pgfpathlineto{\pgfqpoint{0.701546in}{0.794070in}}%
\pgfpathlineto{\pgfqpoint{0.714270in}{0.780925in}}%
\pgfpathlineto{\pgfqpoint{0.717202in}{0.778028in}}%
\pgfpathlineto{\pgfqpoint{0.728318in}{0.767314in}}%
\pgfpathlineto{\pgfqpoint{0.732859in}{0.763106in}}%
\pgfpathlineto{\pgfqpoint{0.743310in}{0.753703in}}%
\pgfpathlineto{\pgfqpoint{0.748516in}{0.749177in}}%
\pgfpathlineto{\pgfqpoint{0.759331in}{0.740092in}}%
\pgfpathlineto{\pgfqpoint{0.764172in}{0.736144in}}%
\pgfpathlineto{\pgfqpoint{0.776496in}{0.726481in}}%
\pgfpathlineto{\pgfqpoint{0.779829in}{0.723931in}}%
\pgfpathlineto{\pgfqpoint{0.794949in}{0.712870in}}%
\pgfpathlineto{\pgfqpoint{0.795485in}{0.712485in}}%
\pgfpathlineto{\pgfqpoint{0.811142in}{0.701780in}}%
\pgfpathlineto{\pgfqpoint{0.815040in}{0.699259in}}%
\pgfpathlineto{\pgfqpoint{0.826798in}{0.691764in}}%
\pgfpathlineto{\pgfqpoint{0.837010in}{0.685648in}}%
\pgfpathlineto{\pgfqpoint{0.842455in}{0.682419in}}%
\pgfpathlineto{\pgfqpoint{0.858112in}{0.673741in}}%
\pgfpathlineto{\pgfqpoint{0.861419in}{0.672036in}}%
\pgfpathlineto{\pgfqpoint{0.873768in}{0.665706in}}%
\pgfpathlineto{\pgfqpoint{0.889232in}{0.658425in}}%
\pgfpathlineto{\pgfqpoint{0.889425in}{0.658335in}}%
\pgfpathlineto{\pgfqpoint{0.905081in}{0.651585in}}%
\pgfpathlineto{\pgfqpoint{0.920738in}{0.645513in}}%
\pgfpathlineto{\pgfqpoint{0.922757in}{0.644814in}}%
\pgfpathlineto{\pgfqpoint{0.936394in}{0.640064in}}%
\pgfpathlineto{\pgfqpoint{0.952051in}{0.635291in}}%
\pgfpathlineto{\pgfqpoint{0.967702in}{0.631203in}}%
\pgfpathlineto{\pgfqpoint{0.967708in}{0.631202in}}%
\pgfpathclose%
\pgfpathmoveto{\pgfqpoint{0.995568in}{0.740092in}}%
\pgfpathlineto{\pgfqpoint{0.983364in}{0.742535in}}%
\pgfpathlineto{\pgfqpoint{0.967708in}{0.746452in}}%
\pgfpathlineto{\pgfqpoint{0.952051in}{0.751154in}}%
\pgfpathlineto{\pgfqpoint{0.944760in}{0.753703in}}%
\pgfpathlineto{\pgfqpoint{0.936394in}{0.756707in}}%
\pgfpathlineto{\pgfqpoint{0.920738in}{0.763121in}}%
\pgfpathlineto{\pgfqpoint{0.911617in}{0.767314in}}%
\pgfpathlineto{\pgfqpoint{0.905081in}{0.770419in}}%
\pgfpathlineto{\pgfqpoint{0.889425in}{0.778656in}}%
\pgfpathlineto{\pgfqpoint{0.885481in}{0.780925in}}%
\pgfpathlineto{\pgfqpoint{0.873768in}{0.787933in}}%
\pgfpathlineto{\pgfqpoint{0.863600in}{0.794536in}}%
\pgfpathlineto{\pgfqpoint{0.858112in}{0.798265in}}%
\pgfpathlineto{\pgfqpoint{0.844586in}{0.808148in}}%
\pgfpathlineto{\pgfqpoint{0.842455in}{0.809786in}}%
\pgfpathlineto{\pgfqpoint{0.827865in}{0.821759in}}%
\pgfpathlineto{\pgfqpoint{0.826798in}{0.822686in}}%
\pgfpathlineto{\pgfqpoint{0.813027in}{0.835370in}}%
\pgfpathlineto{\pgfqpoint{0.811142in}{0.837223in}}%
\pgfpathlineto{\pgfqpoint{0.799774in}{0.848981in}}%
\pgfpathlineto{\pgfqpoint{0.795485in}{0.853753in}}%
\pgfpathlineto{\pgfqpoint{0.787889in}{0.862592in}}%
\pgfpathlineto{\pgfqpoint{0.779829in}{0.872774in}}%
\pgfpathlineto{\pgfqpoint{0.777218in}{0.876203in}}%
\pgfpathlineto{\pgfqpoint{0.767744in}{0.889814in}}%
\pgfpathlineto{\pgfqpoint{0.764172in}{0.895496in}}%
\pgfpathlineto{\pgfqpoint{0.759348in}{0.903425in}}%
\pgfpathlineto{\pgfqpoint{0.751971in}{0.917036in}}%
\pgfpathlineto{\pgfqpoint{0.748516in}{0.924309in}}%
\pgfpathlineto{\pgfqpoint{0.745583in}{0.930648in}}%
\pgfpathlineto{\pgfqpoint{0.740174in}{0.944259in}}%
\pgfpathlineto{\pgfqpoint{0.735669in}{0.957870in}}%
\pgfpathlineto{\pgfqpoint{0.732859in}{0.968479in}}%
\pgfpathlineto{\pgfqpoint{0.732080in}{0.971481in}}%
\pgfpathlineto{\pgfqpoint{0.729425in}{0.985092in}}%
\pgfpathlineto{\pgfqpoint{0.727655in}{0.998703in}}%
\pgfpathlineto{\pgfqpoint{0.726771in}{1.012314in}}%
\pgfpathlineto{\pgfqpoint{0.726771in}{1.025925in}}%
\pgfpathlineto{\pgfqpoint{0.727655in}{1.039536in}}%
\pgfpathlineto{\pgfqpoint{0.729425in}{1.053148in}}%
\pgfpathlineto{\pgfqpoint{0.732080in}{1.066759in}}%
\pgfpathlineto{\pgfqpoint{0.732859in}{1.069760in}}%
\pgfpathlineto{\pgfqpoint{0.735669in}{1.080370in}}%
\pgfpathlineto{\pgfqpoint{0.740174in}{1.093981in}}%
\pgfpathlineto{\pgfqpoint{0.745583in}{1.107592in}}%
\pgfpathlineto{\pgfqpoint{0.748516in}{1.113930in}}%
\pgfpathlineto{\pgfqpoint{0.751971in}{1.121203in}}%
\pgfpathlineto{\pgfqpoint{0.759348in}{1.134814in}}%
\pgfpathlineto{\pgfqpoint{0.764172in}{1.142743in}}%
\pgfpathlineto{\pgfqpoint{0.767744in}{1.148425in}}%
\pgfpathlineto{\pgfqpoint{0.777218in}{1.162036in}}%
\pgfpathlineto{\pgfqpoint{0.779829in}{1.165465in}}%
\pgfpathlineto{\pgfqpoint{0.787889in}{1.175647in}}%
\pgfpathlineto{\pgfqpoint{0.795485in}{1.184487in}}%
\pgfpathlineto{\pgfqpoint{0.799774in}{1.189259in}}%
\pgfpathlineto{\pgfqpoint{0.811142in}{1.201017in}}%
\pgfpathlineto{\pgfqpoint{0.813027in}{1.202870in}}%
\pgfpathlineto{\pgfqpoint{0.826798in}{1.215553in}}%
\pgfpathlineto{\pgfqpoint{0.827865in}{1.216481in}}%
\pgfpathlineto{\pgfqpoint{0.842455in}{1.228453in}}%
\pgfpathlineto{\pgfqpoint{0.844586in}{1.230092in}}%
\pgfpathlineto{\pgfqpoint{0.858112in}{1.239975in}}%
\pgfpathlineto{\pgfqpoint{0.863600in}{1.243703in}}%
\pgfpathlineto{\pgfqpoint{0.873768in}{1.250307in}}%
\pgfpathlineto{\pgfqpoint{0.885481in}{1.257314in}}%
\pgfpathlineto{\pgfqpoint{0.889425in}{1.259584in}}%
\pgfpathlineto{\pgfqpoint{0.905081in}{1.267820in}}%
\pgfpathlineto{\pgfqpoint{0.911617in}{1.270925in}}%
\pgfpathlineto{\pgfqpoint{0.920738in}{1.275119in}}%
\pgfpathlineto{\pgfqpoint{0.936394in}{1.281532in}}%
\pgfpathlineto{\pgfqpoint{0.944760in}{1.284536in}}%
\pgfpathlineto{\pgfqpoint{0.952051in}{1.287085in}}%
\pgfpathlineto{\pgfqpoint{0.967708in}{1.291788in}}%
\pgfpathlineto{\pgfqpoint{0.983364in}{1.295705in}}%
\pgfpathlineto{\pgfqpoint{0.995568in}{1.298148in}}%
\pgfpathlineto{\pgfqpoint{0.999021in}{1.298825in}}%
\pgfpathlineto{\pgfqpoint{1.014677in}{1.301133in}}%
\pgfpathlineto{\pgfqpoint{1.030334in}{1.302671in}}%
\pgfpathlineto{\pgfqpoint{1.045990in}{1.303440in}}%
\pgfpathlineto{\pgfqpoint{1.061647in}{1.303440in}}%
\pgfpathlineto{\pgfqpoint{1.077303in}{1.302671in}}%
\pgfpathlineto{\pgfqpoint{1.092960in}{1.301133in}}%
\pgfpathlineto{\pgfqpoint{1.108617in}{1.298825in}}%
\pgfpathlineto{\pgfqpoint{1.112069in}{1.298148in}}%
\pgfpathlineto{\pgfqpoint{1.124273in}{1.295705in}}%
\pgfpathlineto{\pgfqpoint{1.139930in}{1.291788in}}%
\pgfpathlineto{\pgfqpoint{1.155586in}{1.287085in}}%
\pgfpathlineto{\pgfqpoint{1.162877in}{1.284536in}}%
\pgfpathlineto{\pgfqpoint{1.171243in}{1.281532in}}%
\pgfpathlineto{\pgfqpoint{1.186899in}{1.275119in}}%
\pgfpathlineto{\pgfqpoint{1.196020in}{1.270925in}}%
\pgfpathlineto{\pgfqpoint{1.202556in}{1.267820in}}%
\pgfpathlineto{\pgfqpoint{1.218213in}{1.259584in}}%
\pgfpathlineto{\pgfqpoint{1.222157in}{1.257314in}}%
\pgfpathlineto{\pgfqpoint{1.233869in}{1.250307in}}%
\pgfpathlineto{\pgfqpoint{1.244037in}{1.243703in}}%
\pgfpathlineto{\pgfqpoint{1.249526in}{1.239975in}}%
\pgfpathlineto{\pgfqpoint{1.263051in}{1.230092in}}%
\pgfpathlineto{\pgfqpoint{1.265182in}{1.228453in}}%
\pgfpathlineto{\pgfqpoint{1.279772in}{1.216481in}}%
\pgfpathlineto{\pgfqpoint{1.280839in}{1.215553in}}%
\pgfpathlineto{\pgfqpoint{1.294610in}{1.202870in}}%
\pgfpathlineto{\pgfqpoint{1.296495in}{1.201017in}}%
\pgfpathlineto{\pgfqpoint{1.307864in}{1.189259in}}%
\pgfpathlineto{\pgfqpoint{1.312152in}{1.184487in}}%
\pgfpathlineto{\pgfqpoint{1.319748in}{1.175647in}}%
\pgfpathlineto{\pgfqpoint{1.327809in}{1.165465in}}%
\pgfpathlineto{\pgfqpoint{1.330419in}{1.162036in}}%
\pgfpathlineto{\pgfqpoint{1.339893in}{1.148425in}}%
\pgfpathlineto{\pgfqpoint{1.343465in}{1.142743in}}%
\pgfpathlineto{\pgfqpoint{1.348289in}{1.134814in}}%
\pgfpathlineto{\pgfqpoint{1.355666in}{1.121203in}}%
\pgfpathlineto{\pgfqpoint{1.359122in}{1.113930in}}%
\pgfpathlineto{\pgfqpoint{1.362054in}{1.107592in}}%
\pgfpathlineto{\pgfqpoint{1.367463in}{1.093981in}}%
\pgfpathlineto{\pgfqpoint{1.371968in}{1.080370in}}%
\pgfpathlineto{\pgfqpoint{1.374778in}{1.069760in}}%
\pgfpathlineto{\pgfqpoint{1.375557in}{1.066759in}}%
\pgfpathlineto{\pgfqpoint{1.378212in}{1.053148in}}%
\pgfpathlineto{\pgfqpoint{1.379982in}{1.039536in}}%
\pgfpathlineto{\pgfqpoint{1.380866in}{1.025925in}}%
\pgfpathlineto{\pgfqpoint{1.380866in}{1.012314in}}%
\pgfpathlineto{\pgfqpoint{1.379982in}{0.998703in}}%
\pgfpathlineto{\pgfqpoint{1.378212in}{0.985092in}}%
\pgfpathlineto{\pgfqpoint{1.375557in}{0.971481in}}%
\pgfpathlineto{\pgfqpoint{1.374778in}{0.968479in}}%
\pgfpathlineto{\pgfqpoint{1.371968in}{0.957870in}}%
\pgfpathlineto{\pgfqpoint{1.367463in}{0.944259in}}%
\pgfpathlineto{\pgfqpoint{1.362054in}{0.930648in}}%
\pgfpathlineto{\pgfqpoint{1.359122in}{0.924309in}}%
\pgfpathlineto{\pgfqpoint{1.355666in}{0.917036in}}%
\pgfpathlineto{\pgfqpoint{1.348289in}{0.903425in}}%
\pgfpathlineto{\pgfqpoint{1.343465in}{0.895496in}}%
\pgfpathlineto{\pgfqpoint{1.339893in}{0.889814in}}%
\pgfpathlineto{\pgfqpoint{1.330419in}{0.876203in}}%
\pgfpathlineto{\pgfqpoint{1.327809in}{0.872774in}}%
\pgfpathlineto{\pgfqpoint{1.319748in}{0.862592in}}%
\pgfpathlineto{\pgfqpoint{1.312152in}{0.853753in}}%
\pgfpathlineto{\pgfqpoint{1.307864in}{0.848981in}}%
\pgfpathlineto{\pgfqpoint{1.296495in}{0.837223in}}%
\pgfpathlineto{\pgfqpoint{1.294610in}{0.835370in}}%
\pgfpathlineto{\pgfqpoint{1.280839in}{0.822686in}}%
\pgfpathlineto{\pgfqpoint{1.279772in}{0.821759in}}%
\pgfpathlineto{\pgfqpoint{1.265182in}{0.809786in}}%
\pgfpathlineto{\pgfqpoint{1.263051in}{0.808148in}}%
\pgfpathlineto{\pgfqpoint{1.249526in}{0.798265in}}%
\pgfpathlineto{\pgfqpoint{1.244037in}{0.794536in}}%
\pgfpathlineto{\pgfqpoint{1.233869in}{0.787933in}}%
\pgfpathlineto{\pgfqpoint{1.222157in}{0.780925in}}%
\pgfpathlineto{\pgfqpoint{1.218213in}{0.778656in}}%
\pgfpathlineto{\pgfqpoint{1.202556in}{0.770419in}}%
\pgfpathlineto{\pgfqpoint{1.196020in}{0.767314in}}%
\pgfpathlineto{\pgfqpoint{1.186899in}{0.763121in}}%
\pgfpathlineto{\pgfqpoint{1.171243in}{0.756707in}}%
\pgfpathlineto{\pgfqpoint{1.162877in}{0.753703in}}%
\pgfpathlineto{\pgfqpoint{1.155586in}{0.751154in}}%
\pgfpathlineto{\pgfqpoint{1.139930in}{0.746452in}}%
\pgfpathlineto{\pgfqpoint{1.124273in}{0.742535in}}%
\pgfpathlineto{\pgfqpoint{1.112069in}{0.740092in}}%
\pgfpathlineto{\pgfqpoint{1.108617in}{0.739415in}}%
\pgfpathlineto{\pgfqpoint{1.092960in}{0.737106in}}%
\pgfpathlineto{\pgfqpoint{1.077303in}{0.735568in}}%
\pgfpathlineto{\pgfqpoint{1.061647in}{0.734799in}}%
\pgfpathlineto{\pgfqpoint{1.045990in}{0.734799in}}%
\pgfpathlineto{\pgfqpoint{1.030334in}{0.735568in}}%
\pgfpathlineto{\pgfqpoint{1.014677in}{0.737106in}}%
\pgfpathlineto{\pgfqpoint{0.999021in}{0.739415in}}%
\pgfpathlineto{\pgfqpoint{0.995568in}{0.740092in}}%
\pgfpathclose%
\pgfusepath{fill}%
\end{pgfscope}%
\begin{pgfscope}%
\pgfpathrectangle{\pgfqpoint{0.278819in}{0.345370in}}{\pgfqpoint{1.550000in}{1.347500in}}%
\pgfusepath{clip}%
\pgfsetbuttcap%
\pgfsetroundjoin%
\definecolor{currentfill}{rgb}{0.709962,0.212797,0.477201}%
\pgfsetfillcolor{currentfill}%
\pgfsetlinewidth{0.000000pt}%
\definecolor{currentstroke}{rgb}{0.000000,0.000000,0.000000}%
\pgfsetstrokecolor{currentstroke}%
\pgfsetdash{}{0pt}%
\pgfpathmoveto{\pgfqpoint{1.030334in}{0.481176in}}%
\pgfpathlineto{\pgfqpoint{1.045990in}{0.480145in}}%
\pgfpathlineto{\pgfqpoint{1.061647in}{0.480145in}}%
\pgfpathlineto{\pgfqpoint{1.077303in}{0.481176in}}%
\pgfpathlineto{\pgfqpoint{1.079619in}{0.481481in}}%
\pgfpathlineto{\pgfqpoint{1.092960in}{0.483093in}}%
\pgfpathlineto{\pgfqpoint{1.108617in}{0.485931in}}%
\pgfpathlineto{\pgfqpoint{1.124273in}{0.489717in}}%
\pgfpathlineto{\pgfqpoint{1.139930in}{0.494452in}}%
\pgfpathlineto{\pgfqpoint{1.141701in}{0.495092in}}%
\pgfpathlineto{\pgfqpoint{1.155586in}{0.499767in}}%
\pgfpathlineto{\pgfqpoint{1.171243in}{0.505916in}}%
\pgfpathlineto{\pgfqpoint{1.177470in}{0.508703in}}%
\pgfpathlineto{\pgfqpoint{1.186899in}{0.512673in}}%
\pgfpathlineto{\pgfqpoint{1.202556in}{0.520075in}}%
\pgfpathlineto{\pgfqpoint{1.206839in}{0.522314in}}%
\pgfpathlineto{\pgfqpoint{1.218213in}{0.527960in}}%
\pgfpathlineto{\pgfqpoint{1.232834in}{0.535925in}}%
\pgfpathlineto{\pgfqpoint{1.233869in}{0.536465in}}%
\pgfpathlineto{\pgfqpoint{1.249526in}{0.545304in}}%
\pgfpathlineto{\pgfqpoint{1.256482in}{0.549536in}}%
\pgfpathlineto{\pgfqpoint{1.265182in}{0.554640in}}%
\pgfpathlineto{\pgfqpoint{1.278730in}{0.563148in}}%
\pgfpathlineto{\pgfqpoint{1.280839in}{0.564433in}}%
\pgfpathlineto{\pgfqpoint{1.296495in}{0.574560in}}%
\pgfpathlineto{\pgfqpoint{1.299714in}{0.576759in}}%
\pgfpathlineto{\pgfqpoint{1.312152in}{0.585056in}}%
\pgfpathlineto{\pgfqpoint{1.319726in}{0.590370in}}%
\pgfpathlineto{\pgfqpoint{1.327809in}{0.595939in}}%
\pgfpathlineto{\pgfqpoint{1.338960in}{0.603981in}}%
\pgfpathlineto{\pgfqpoint{1.343465in}{0.607188in}}%
\pgfpathlineto{\pgfqpoint{1.357495in}{0.617592in}}%
\pgfpathlineto{\pgfqpoint{1.359122in}{0.618789in}}%
\pgfpathlineto{\pgfqpoint{1.374778in}{0.630726in}}%
\pgfpathlineto{\pgfqpoint{1.375385in}{0.631203in}}%
\pgfpathlineto{\pgfqpoint{1.390435in}{0.643000in}}%
\pgfpathlineto{\pgfqpoint{1.392686in}{0.644814in}}%
\pgfpathlineto{\pgfqpoint{1.406091in}{0.655636in}}%
\pgfpathlineto{\pgfqpoint{1.409467in}{0.658425in}}%
\pgfpathlineto{\pgfqpoint{1.421748in}{0.668636in}}%
\pgfpathlineto{\pgfqpoint{1.425760in}{0.672036in}}%
\pgfpathlineto{\pgfqpoint{1.437404in}{0.682010in}}%
\pgfpathlineto{\pgfqpoint{1.441588in}{0.685648in}}%
\pgfpathlineto{\pgfqpoint{1.453061in}{0.695771in}}%
\pgfpathlineto{\pgfqpoint{1.456972in}{0.699259in}}%
\pgfpathlineto{\pgfqpoint{1.468718in}{0.709935in}}%
\pgfpathlineto{\pgfqpoint{1.471926in}{0.712870in}}%
\pgfpathlineto{\pgfqpoint{1.484374in}{0.724524in}}%
\pgfpathlineto{\pgfqpoint{1.486461in}{0.726481in}}%
\pgfpathlineto{\pgfqpoint{1.500031in}{0.739564in}}%
\pgfpathlineto{\pgfqpoint{1.500580in}{0.740092in}}%
\pgfpathlineto{\pgfqpoint{1.514311in}{0.753703in}}%
\pgfpathlineto{\pgfqpoint{1.515687in}{0.755117in}}%
\pgfpathlineto{\pgfqpoint{1.527654in}{0.767314in}}%
\pgfpathlineto{\pgfqpoint{1.531344in}{0.771231in}}%
\pgfpathlineto{\pgfqpoint{1.540594in}{0.780925in}}%
\pgfpathlineto{\pgfqpoint{1.547000in}{0.787952in}}%
\pgfpathlineto{\pgfqpoint{1.553113in}{0.794536in}}%
\pgfpathlineto{\pgfqpoint{1.562657in}{0.805349in}}%
\pgfpathlineto{\pgfqpoint{1.565186in}{0.808148in}}%
\pgfpathlineto{\pgfqpoint{1.576835in}{0.821759in}}%
\pgfpathlineto{\pgfqpoint{1.578314in}{0.823592in}}%
\pgfpathlineto{\pgfqpoint{1.588100in}{0.835370in}}%
\pgfpathlineto{\pgfqpoint{1.593970in}{0.842933in}}%
\pgfpathlineto{\pgfqpoint{1.598838in}{0.848981in}}%
\pgfpathlineto{\pgfqpoint{1.609006in}{0.862592in}}%
\pgfpathlineto{\pgfqpoint{1.609627in}{0.863492in}}%
\pgfpathlineto{\pgfqpoint{1.618789in}{0.876203in}}%
\pgfpathlineto{\pgfqpoint{1.625283in}{0.886090in}}%
\pgfpathlineto{\pgfqpoint{1.627859in}{0.889814in}}%
\pgfpathlineto{\pgfqpoint{1.636373in}{0.903425in}}%
\pgfpathlineto{\pgfqpoint{1.640940in}{0.911623in}}%
\pgfpathlineto{\pgfqpoint{1.644146in}{0.917036in}}%
\pgfpathlineto{\pgfqpoint{1.651219in}{0.930648in}}%
\pgfpathlineto{\pgfqpoint{1.656596in}{0.942719in}}%
\pgfpathlineto{\pgfqpoint{1.657333in}{0.944259in}}%
\pgfpathlineto{\pgfqpoint{1.662779in}{0.957870in}}%
\pgfpathlineto{\pgfqpoint{1.667133in}{0.971481in}}%
\pgfpathlineto{\pgfqpoint{1.670398in}{0.985092in}}%
\pgfpathlineto{\pgfqpoint{1.672253in}{0.996690in}}%
\pgfpathlineto{\pgfqpoint{1.672603in}{0.998703in}}%
\pgfpathlineto{\pgfqpoint{1.673790in}{1.012314in}}%
\pgfpathlineto{\pgfqpoint{1.673790in}{1.025925in}}%
\pgfpathlineto{\pgfqpoint{1.672603in}{1.039536in}}%
\pgfpathlineto{\pgfqpoint{1.672253in}{1.041549in}}%
\pgfpathlineto{\pgfqpoint{1.670398in}{1.053148in}}%
\pgfpathlineto{\pgfqpoint{1.667133in}{1.066759in}}%
\pgfpathlineto{\pgfqpoint{1.662779in}{1.080370in}}%
\pgfpathlineto{\pgfqpoint{1.657333in}{1.093981in}}%
\pgfpathlineto{\pgfqpoint{1.656596in}{1.095520in}}%
\pgfpathlineto{\pgfqpoint{1.651219in}{1.107592in}}%
\pgfpathlineto{\pgfqpoint{1.644146in}{1.121203in}}%
\pgfpathlineto{\pgfqpoint{1.640940in}{1.126617in}}%
\pgfpathlineto{\pgfqpoint{1.636373in}{1.134814in}}%
\pgfpathlineto{\pgfqpoint{1.627859in}{1.148425in}}%
\pgfpathlineto{\pgfqpoint{1.625283in}{1.152149in}}%
\pgfpathlineto{\pgfqpoint{1.618789in}{1.162036in}}%
\pgfpathlineto{\pgfqpoint{1.609627in}{1.174747in}}%
\pgfpathlineto{\pgfqpoint{1.609006in}{1.175647in}}%
\pgfpathlineto{\pgfqpoint{1.598838in}{1.189259in}}%
\pgfpathlineto{\pgfqpoint{1.593970in}{1.195306in}}%
\pgfpathlineto{\pgfqpoint{1.588100in}{1.202870in}}%
\pgfpathlineto{\pgfqpoint{1.578314in}{1.214647in}}%
\pgfpathlineto{\pgfqpoint{1.576835in}{1.216481in}}%
\pgfpathlineto{\pgfqpoint{1.565186in}{1.230092in}}%
\pgfpathlineto{\pgfqpoint{1.562657in}{1.232890in}}%
\pgfpathlineto{\pgfqpoint{1.553113in}{1.243703in}}%
\pgfpathlineto{\pgfqpoint{1.547000in}{1.250287in}}%
\pgfpathlineto{\pgfqpoint{1.540594in}{1.257314in}}%
\pgfpathlineto{\pgfqpoint{1.531344in}{1.267009in}}%
\pgfpathlineto{\pgfqpoint{1.527654in}{1.270925in}}%
\pgfpathlineto{\pgfqpoint{1.515687in}{1.283122in}}%
\pgfpathlineto{\pgfqpoint{1.514311in}{1.284536in}}%
\pgfpathlineto{\pgfqpoint{1.500580in}{1.298148in}}%
\pgfpathlineto{\pgfqpoint{1.500031in}{1.298675in}}%
\pgfpathlineto{\pgfqpoint{1.486461in}{1.311759in}}%
\pgfpathlineto{\pgfqpoint{1.484374in}{1.313716in}}%
\pgfpathlineto{\pgfqpoint{1.471926in}{1.325370in}}%
\pgfpathlineto{\pgfqpoint{1.468718in}{1.328305in}}%
\pgfpathlineto{\pgfqpoint{1.456972in}{1.338981in}}%
\pgfpathlineto{\pgfqpoint{1.453061in}{1.342469in}}%
\pgfpathlineto{\pgfqpoint{1.441588in}{1.352592in}}%
\pgfpathlineto{\pgfqpoint{1.437404in}{1.356229in}}%
\pgfpathlineto{\pgfqpoint{1.425760in}{1.366203in}}%
\pgfpathlineto{\pgfqpoint{1.421748in}{1.369603in}}%
\pgfpathlineto{\pgfqpoint{1.409467in}{1.379814in}}%
\pgfpathlineto{\pgfqpoint{1.406091in}{1.382604in}}%
\pgfpathlineto{\pgfqpoint{1.392686in}{1.393425in}}%
\pgfpathlineto{\pgfqpoint{1.390435in}{1.395239in}}%
\pgfpathlineto{\pgfqpoint{1.375385in}{1.407036in}}%
\pgfpathlineto{\pgfqpoint{1.374778in}{1.407514in}}%
\pgfpathlineto{\pgfqpoint{1.359122in}{1.419451in}}%
\pgfpathlineto{\pgfqpoint{1.357495in}{1.420648in}}%
\pgfpathlineto{\pgfqpoint{1.343465in}{1.431051in}}%
\pgfpathlineto{\pgfqpoint{1.338960in}{1.434259in}}%
\pgfpathlineto{\pgfqpoint{1.327809in}{1.442300in}}%
\pgfpathlineto{\pgfqpoint{1.319726in}{1.447870in}}%
\pgfpathlineto{\pgfqpoint{1.312152in}{1.453184in}}%
\pgfpathlineto{\pgfqpoint{1.299714in}{1.461481in}}%
\pgfpathlineto{\pgfqpoint{1.296495in}{1.463679in}}%
\pgfpathlineto{\pgfqpoint{1.280839in}{1.473807in}}%
\pgfpathlineto{\pgfqpoint{1.278730in}{1.475092in}}%
\pgfpathlineto{\pgfqpoint{1.265182in}{1.483600in}}%
\pgfpathlineto{\pgfqpoint{1.256482in}{1.488703in}}%
\pgfpathlineto{\pgfqpoint{1.249526in}{1.492935in}}%
\pgfpathlineto{\pgfqpoint{1.233869in}{1.501774in}}%
\pgfpathlineto{\pgfqpoint{1.232834in}{1.502314in}}%
\pgfpathlineto{\pgfqpoint{1.218213in}{1.510280in}}%
\pgfpathlineto{\pgfqpoint{1.206839in}{1.515925in}}%
\pgfpathlineto{\pgfqpoint{1.202556in}{1.518164in}}%
\pgfpathlineto{\pgfqpoint{1.186899in}{1.525566in}}%
\pgfpathlineto{\pgfqpoint{1.177470in}{1.529536in}}%
\pgfpathlineto{\pgfqpoint{1.171243in}{1.532323in}}%
\pgfpathlineto{\pgfqpoint{1.155586in}{1.538473in}}%
\pgfpathlineto{\pgfqpoint{1.141701in}{1.543148in}}%
\pgfpathlineto{\pgfqpoint{1.139930in}{1.543788in}}%
\pgfpathlineto{\pgfqpoint{1.124273in}{1.548522in}}%
\pgfpathlineto{\pgfqpoint{1.108617in}{1.552308in}}%
\pgfpathlineto{\pgfqpoint{1.092960in}{1.555146in}}%
\pgfpathlineto{\pgfqpoint{1.079619in}{1.556759in}}%
\pgfpathlineto{\pgfqpoint{1.077303in}{1.557063in}}%
\pgfpathlineto{\pgfqpoint{1.061647in}{1.558094in}}%
\pgfpathlineto{\pgfqpoint{1.045990in}{1.558094in}}%
\pgfpathlineto{\pgfqpoint{1.030334in}{1.557063in}}%
\pgfpathlineto{\pgfqpoint{1.028018in}{1.556759in}}%
\pgfpathlineto{\pgfqpoint{1.014677in}{1.555146in}}%
\pgfpathlineto{\pgfqpoint{0.999021in}{1.552308in}}%
\pgfpathlineto{\pgfqpoint{0.983364in}{1.548522in}}%
\pgfpathlineto{\pgfqpoint{0.967708in}{1.543788in}}%
\pgfpathlineto{\pgfqpoint{0.965937in}{1.543148in}}%
\pgfpathlineto{\pgfqpoint{0.952051in}{1.538473in}}%
\pgfpathlineto{\pgfqpoint{0.936394in}{1.532323in}}%
\pgfpathlineto{\pgfqpoint{0.930167in}{1.529536in}}%
\pgfpathlineto{\pgfqpoint{0.920738in}{1.525566in}}%
\pgfpathlineto{\pgfqpoint{0.905081in}{1.518164in}}%
\pgfpathlineto{\pgfqpoint{0.900798in}{1.515925in}}%
\pgfpathlineto{\pgfqpoint{0.889425in}{1.510280in}}%
\pgfpathlineto{\pgfqpoint{0.874803in}{1.502314in}}%
\pgfpathlineto{\pgfqpoint{0.873768in}{1.501774in}}%
\pgfpathlineto{\pgfqpoint{0.858112in}{1.492935in}}%
\pgfpathlineto{\pgfqpoint{0.851155in}{1.488703in}}%
\pgfpathlineto{\pgfqpoint{0.842455in}{1.483600in}}%
\pgfpathlineto{\pgfqpoint{0.828907in}{1.475092in}}%
\pgfpathlineto{\pgfqpoint{0.826798in}{1.473807in}}%
\pgfpathlineto{\pgfqpoint{0.811142in}{1.463679in}}%
\pgfpathlineto{\pgfqpoint{0.807923in}{1.461481in}}%
\pgfpathlineto{\pgfqpoint{0.795485in}{1.453184in}}%
\pgfpathlineto{\pgfqpoint{0.787911in}{1.447870in}}%
\pgfpathlineto{\pgfqpoint{0.779829in}{1.442300in}}%
\pgfpathlineto{\pgfqpoint{0.768677in}{1.434259in}}%
\pgfpathlineto{\pgfqpoint{0.764172in}{1.431051in}}%
\pgfpathlineto{\pgfqpoint{0.750142in}{1.420648in}}%
\pgfpathlineto{\pgfqpoint{0.748516in}{1.419451in}}%
\pgfpathlineto{\pgfqpoint{0.732859in}{1.407514in}}%
\pgfpathlineto{\pgfqpoint{0.732252in}{1.407036in}}%
\pgfpathlineto{\pgfqpoint{0.717202in}{1.395239in}}%
\pgfpathlineto{\pgfqpoint{0.714951in}{1.393425in}}%
\pgfpathlineto{\pgfqpoint{0.701546in}{1.382604in}}%
\pgfpathlineto{\pgfqpoint{0.698170in}{1.379814in}}%
\pgfpathlineto{\pgfqpoint{0.685889in}{1.369603in}}%
\pgfpathlineto{\pgfqpoint{0.681877in}{1.366203in}}%
\pgfpathlineto{\pgfqpoint{0.670233in}{1.356229in}}%
\pgfpathlineto{\pgfqpoint{0.666049in}{1.352592in}}%
\pgfpathlineto{\pgfqpoint{0.654576in}{1.342469in}}%
\pgfpathlineto{\pgfqpoint{0.650665in}{1.338981in}}%
\pgfpathlineto{\pgfqpoint{0.638920in}{1.328305in}}%
\pgfpathlineto{\pgfqpoint{0.635711in}{1.325370in}}%
\pgfpathlineto{\pgfqpoint{0.623263in}{1.313716in}}%
\pgfpathlineto{\pgfqpoint{0.621177in}{1.311759in}}%
\pgfpathlineto{\pgfqpoint{0.607606in}{1.298675in}}%
\pgfpathlineto{\pgfqpoint{0.607057in}{1.298148in}}%
\pgfpathlineto{\pgfqpoint{0.593326in}{1.284536in}}%
\pgfpathlineto{\pgfqpoint{0.591950in}{1.283122in}}%
\pgfpathlineto{\pgfqpoint{0.579983in}{1.270925in}}%
\pgfpathlineto{\pgfqpoint{0.576293in}{1.267009in}}%
\pgfpathlineto{\pgfqpoint{0.567044in}{1.257314in}}%
\pgfpathlineto{\pgfqpoint{0.560637in}{1.250287in}}%
\pgfpathlineto{\pgfqpoint{0.554524in}{1.243703in}}%
\pgfpathlineto{\pgfqpoint{0.544980in}{1.232890in}}%
\pgfpathlineto{\pgfqpoint{0.542451in}{1.230092in}}%
\pgfpathlineto{\pgfqpoint{0.530802in}{1.216481in}}%
\pgfpathlineto{\pgfqpoint{0.529324in}{1.214647in}}%
\pgfpathlineto{\pgfqpoint{0.519537in}{1.202870in}}%
\pgfpathlineto{\pgfqpoint{0.513667in}{1.195306in}}%
\pgfpathlineto{\pgfqpoint{0.508799in}{1.189259in}}%
\pgfpathlineto{\pgfqpoint{0.498631in}{1.175647in}}%
\pgfpathlineto{\pgfqpoint{0.498011in}{1.174747in}}%
\pgfpathlineto{\pgfqpoint{0.488848in}{1.162036in}}%
\pgfpathlineto{\pgfqpoint{0.482354in}{1.152149in}}%
\pgfpathlineto{\pgfqpoint{0.479778in}{1.148425in}}%
\pgfpathlineto{\pgfqpoint{0.471264in}{1.134814in}}%
\pgfpathlineto{\pgfqpoint{0.466697in}{1.126617in}}%
\pgfpathlineto{\pgfqpoint{0.463492in}{1.121203in}}%
\pgfpathlineto{\pgfqpoint{0.456418in}{1.107592in}}%
\pgfpathlineto{\pgfqpoint{0.451041in}{1.095520in}}%
\pgfpathlineto{\pgfqpoint{0.450304in}{1.093981in}}%
\pgfpathlineto{\pgfqpoint{0.444858in}{1.080370in}}%
\pgfpathlineto{\pgfqpoint{0.440504in}{1.066759in}}%
\pgfpathlineto{\pgfqpoint{0.437239in}{1.053148in}}%
\pgfpathlineto{\pgfqpoint{0.435384in}{1.041549in}}%
\pgfpathlineto{\pgfqpoint{0.435034in}{1.039536in}}%
\pgfpathlineto{\pgfqpoint{0.433848in}{1.025925in}}%
\pgfpathlineto{\pgfqpoint{0.433848in}{1.012314in}}%
\pgfpathlineto{\pgfqpoint{0.435034in}{0.998703in}}%
\pgfpathlineto{\pgfqpoint{0.435384in}{0.996690in}}%
\pgfpathlineto{\pgfqpoint{0.437239in}{0.985092in}}%
\pgfpathlineto{\pgfqpoint{0.440504in}{0.971481in}}%
\pgfpathlineto{\pgfqpoint{0.444858in}{0.957870in}}%
\pgfpathlineto{\pgfqpoint{0.450304in}{0.944259in}}%
\pgfpathlineto{\pgfqpoint{0.451041in}{0.942719in}}%
\pgfpathlineto{\pgfqpoint{0.456418in}{0.930648in}}%
\pgfpathlineto{\pgfqpoint{0.463492in}{0.917036in}}%
\pgfpathlineto{\pgfqpoint{0.466697in}{0.911623in}}%
\pgfpathlineto{\pgfqpoint{0.471264in}{0.903425in}}%
\pgfpathlineto{\pgfqpoint{0.479778in}{0.889814in}}%
\pgfpathlineto{\pgfqpoint{0.482354in}{0.886090in}}%
\pgfpathlineto{\pgfqpoint{0.488848in}{0.876203in}}%
\pgfpathlineto{\pgfqpoint{0.498011in}{0.863492in}}%
\pgfpathlineto{\pgfqpoint{0.498631in}{0.862592in}}%
\pgfpathlineto{\pgfqpoint{0.508799in}{0.848981in}}%
\pgfpathlineto{\pgfqpoint{0.513667in}{0.842933in}}%
\pgfpathlineto{\pgfqpoint{0.519537in}{0.835370in}}%
\pgfpathlineto{\pgfqpoint{0.529324in}{0.823592in}}%
\pgfpathlineto{\pgfqpoint{0.530802in}{0.821759in}}%
\pgfpathlineto{\pgfqpoint{0.542451in}{0.808148in}}%
\pgfpathlineto{\pgfqpoint{0.544980in}{0.805349in}}%
\pgfpathlineto{\pgfqpoint{0.554524in}{0.794536in}}%
\pgfpathlineto{\pgfqpoint{0.560637in}{0.787952in}}%
\pgfpathlineto{\pgfqpoint{0.567044in}{0.780925in}}%
\pgfpathlineto{\pgfqpoint{0.576293in}{0.771231in}}%
\pgfpathlineto{\pgfqpoint{0.579983in}{0.767314in}}%
\pgfpathlineto{\pgfqpoint{0.591950in}{0.755117in}}%
\pgfpathlineto{\pgfqpoint{0.593326in}{0.753703in}}%
\pgfpathlineto{\pgfqpoint{0.607057in}{0.740092in}}%
\pgfpathlineto{\pgfqpoint{0.607606in}{0.739564in}}%
\pgfpathlineto{\pgfqpoint{0.621177in}{0.726481in}}%
\pgfpathlineto{\pgfqpoint{0.623263in}{0.724524in}}%
\pgfpathlineto{\pgfqpoint{0.635711in}{0.712870in}}%
\pgfpathlineto{\pgfqpoint{0.638920in}{0.709935in}}%
\pgfpathlineto{\pgfqpoint{0.650665in}{0.699259in}}%
\pgfpathlineto{\pgfqpoint{0.654576in}{0.695771in}}%
\pgfpathlineto{\pgfqpoint{0.666049in}{0.685648in}}%
\pgfpathlineto{\pgfqpoint{0.670233in}{0.682010in}}%
\pgfpathlineto{\pgfqpoint{0.681877in}{0.672036in}}%
\pgfpathlineto{\pgfqpoint{0.685889in}{0.668636in}}%
\pgfpathlineto{\pgfqpoint{0.698170in}{0.658425in}}%
\pgfpathlineto{\pgfqpoint{0.701546in}{0.655636in}}%
\pgfpathlineto{\pgfqpoint{0.714951in}{0.644814in}}%
\pgfpathlineto{\pgfqpoint{0.717202in}{0.643000in}}%
\pgfpathlineto{\pgfqpoint{0.732252in}{0.631203in}}%
\pgfpathlineto{\pgfqpoint{0.732859in}{0.630726in}}%
\pgfpathlineto{\pgfqpoint{0.748516in}{0.618789in}}%
\pgfpathlineto{\pgfqpoint{0.750142in}{0.617592in}}%
\pgfpathlineto{\pgfqpoint{0.764172in}{0.607188in}}%
\pgfpathlineto{\pgfqpoint{0.768677in}{0.603981in}}%
\pgfpathlineto{\pgfqpoint{0.779829in}{0.595939in}}%
\pgfpathlineto{\pgfqpoint{0.787911in}{0.590370in}}%
\pgfpathlineto{\pgfqpoint{0.795485in}{0.585056in}}%
\pgfpathlineto{\pgfqpoint{0.807923in}{0.576759in}}%
\pgfpathlineto{\pgfqpoint{0.811142in}{0.574560in}}%
\pgfpathlineto{\pgfqpoint{0.826798in}{0.564433in}}%
\pgfpathlineto{\pgfqpoint{0.828907in}{0.563148in}}%
\pgfpathlineto{\pgfqpoint{0.842455in}{0.554640in}}%
\pgfpathlineto{\pgfqpoint{0.851155in}{0.549536in}}%
\pgfpathlineto{\pgfqpoint{0.858112in}{0.545304in}}%
\pgfpathlineto{\pgfqpoint{0.873768in}{0.536465in}}%
\pgfpathlineto{\pgfqpoint{0.874803in}{0.535925in}}%
\pgfpathlineto{\pgfqpoint{0.889425in}{0.527960in}}%
\pgfpathlineto{\pgfqpoint{0.900798in}{0.522314in}}%
\pgfpathlineto{\pgfqpoint{0.905081in}{0.520075in}}%
\pgfpathlineto{\pgfqpoint{0.920738in}{0.512673in}}%
\pgfpathlineto{\pgfqpoint{0.930167in}{0.508703in}}%
\pgfpathlineto{\pgfqpoint{0.936394in}{0.505916in}}%
\pgfpathlineto{\pgfqpoint{0.952051in}{0.499767in}}%
\pgfpathlineto{\pgfqpoint{0.965937in}{0.495092in}}%
\pgfpathlineto{\pgfqpoint{0.967708in}{0.494452in}}%
\pgfpathlineto{\pgfqpoint{0.983364in}{0.489717in}}%
\pgfpathlineto{\pgfqpoint{0.999021in}{0.485931in}}%
\pgfpathlineto{\pgfqpoint{1.014677in}{0.483093in}}%
\pgfpathlineto{\pgfqpoint{1.028018in}{0.481481in}}%
\pgfpathlineto{\pgfqpoint{1.030334in}{0.481176in}}%
\pgfpathclose%
\pgfpathmoveto{\pgfqpoint{0.967702in}{0.631203in}}%
\pgfpathlineto{\pgfqpoint{0.952051in}{0.635291in}}%
\pgfpathlineto{\pgfqpoint{0.936394in}{0.640064in}}%
\pgfpathlineto{\pgfqpoint{0.922757in}{0.644814in}}%
\pgfpathlineto{\pgfqpoint{0.920738in}{0.645513in}}%
\pgfpathlineto{\pgfqpoint{0.905081in}{0.651585in}}%
\pgfpathlineto{\pgfqpoint{0.889425in}{0.658335in}}%
\pgfpathlineto{\pgfqpoint{0.889232in}{0.658425in}}%
\pgfpathlineto{\pgfqpoint{0.873768in}{0.665706in}}%
\pgfpathlineto{\pgfqpoint{0.861419in}{0.672036in}}%
\pgfpathlineto{\pgfqpoint{0.858112in}{0.673741in}}%
\pgfpathlineto{\pgfqpoint{0.842455in}{0.682419in}}%
\pgfpathlineto{\pgfqpoint{0.837010in}{0.685648in}}%
\pgfpathlineto{\pgfqpoint{0.826798in}{0.691764in}}%
\pgfpathlineto{\pgfqpoint{0.815040in}{0.699259in}}%
\pgfpathlineto{\pgfqpoint{0.811142in}{0.701780in}}%
\pgfpathlineto{\pgfqpoint{0.795485in}{0.712485in}}%
\pgfpathlineto{\pgfqpoint{0.794949in}{0.712870in}}%
\pgfpathlineto{\pgfqpoint{0.779829in}{0.723931in}}%
\pgfpathlineto{\pgfqpoint{0.776496in}{0.726481in}}%
\pgfpathlineto{\pgfqpoint{0.764172in}{0.736144in}}%
\pgfpathlineto{\pgfqpoint{0.759331in}{0.740092in}}%
\pgfpathlineto{\pgfqpoint{0.748516in}{0.749177in}}%
\pgfpathlineto{\pgfqpoint{0.743310in}{0.753703in}}%
\pgfpathlineto{\pgfqpoint{0.732859in}{0.763106in}}%
\pgfpathlineto{\pgfqpoint{0.728318in}{0.767314in}}%
\pgfpathlineto{\pgfqpoint{0.717202in}{0.778028in}}%
\pgfpathlineto{\pgfqpoint{0.714270in}{0.780925in}}%
\pgfpathlineto{\pgfqpoint{0.701546in}{0.794070in}}%
\pgfpathlineto{\pgfqpoint{0.701104in}{0.794536in}}%
\pgfpathlineto{\pgfqpoint{0.688790in}{0.808148in}}%
\pgfpathlineto{\pgfqpoint{0.685889in}{0.811536in}}%
\pgfpathlineto{\pgfqpoint{0.677269in}{0.821759in}}%
\pgfpathlineto{\pgfqpoint{0.670233in}{0.830636in}}%
\pgfpathlineto{\pgfqpoint{0.666520in}{0.835370in}}%
\pgfpathlineto{\pgfqpoint{0.656536in}{0.848981in}}%
\pgfpathlineto{\pgfqpoint{0.654576in}{0.851856in}}%
\pgfpathlineto{\pgfqpoint{0.647294in}{0.862592in}}%
\pgfpathlineto{\pgfqpoint{0.638920in}{0.876036in}}%
\pgfpathlineto{\pgfqpoint{0.638815in}{0.876203in}}%
\pgfpathlineto{\pgfqpoint{0.631052in}{0.889814in}}%
\pgfpathlineto{\pgfqpoint{0.624067in}{0.903425in}}%
\pgfpathlineto{\pgfqpoint{0.623263in}{0.905180in}}%
\pgfpathlineto{\pgfqpoint{0.617799in}{0.917036in}}%
\pgfpathlineto{\pgfqpoint{0.612308in}{0.930648in}}%
\pgfpathlineto{\pgfqpoint{0.607606in}{0.944254in}}%
\pgfpathlineto{\pgfqpoint{0.607605in}{0.944259in}}%
\pgfpathlineto{\pgfqpoint{0.603625in}{0.957870in}}%
\pgfpathlineto{\pgfqpoint{0.600442in}{0.971481in}}%
\pgfpathlineto{\pgfqpoint{0.598056in}{0.985092in}}%
\pgfpathlineto{\pgfqpoint{0.596466in}{0.998703in}}%
\pgfpathlineto{\pgfqpoint{0.595671in}{1.012314in}}%
\pgfpathlineto{\pgfqpoint{0.595671in}{1.025925in}}%
\pgfpathlineto{\pgfqpoint{0.596466in}{1.039536in}}%
\pgfpathlineto{\pgfqpoint{0.598056in}{1.053148in}}%
\pgfpathlineto{\pgfqpoint{0.600442in}{1.066759in}}%
\pgfpathlineto{\pgfqpoint{0.603625in}{1.080370in}}%
\pgfpathlineto{\pgfqpoint{0.607605in}{1.093981in}}%
\pgfpathlineto{\pgfqpoint{0.607606in}{1.093986in}}%
\pgfpathlineto{\pgfqpoint{0.612308in}{1.107592in}}%
\pgfpathlineto{\pgfqpoint{0.617799in}{1.121203in}}%
\pgfpathlineto{\pgfqpoint{0.623263in}{1.133059in}}%
\pgfpathlineto{\pgfqpoint{0.624067in}{1.134814in}}%
\pgfpathlineto{\pgfqpoint{0.631052in}{1.148425in}}%
\pgfpathlineto{\pgfqpoint{0.638815in}{1.162036in}}%
\pgfpathlineto{\pgfqpoint{0.638920in}{1.162204in}}%
\pgfpathlineto{\pgfqpoint{0.647294in}{1.175647in}}%
\pgfpathlineto{\pgfqpoint{0.654576in}{1.186383in}}%
\pgfpathlineto{\pgfqpoint{0.656536in}{1.189259in}}%
\pgfpathlineto{\pgfqpoint{0.666520in}{1.202870in}}%
\pgfpathlineto{\pgfqpoint{0.670233in}{1.207603in}}%
\pgfpathlineto{\pgfqpoint{0.677269in}{1.216481in}}%
\pgfpathlineto{\pgfqpoint{0.685889in}{1.226703in}}%
\pgfpathlineto{\pgfqpoint{0.688790in}{1.230092in}}%
\pgfpathlineto{\pgfqpoint{0.701104in}{1.243703in}}%
\pgfpathlineto{\pgfqpoint{0.701546in}{1.244169in}}%
\pgfpathlineto{\pgfqpoint{0.714270in}{1.257314in}}%
\pgfpathlineto{\pgfqpoint{0.717202in}{1.260212in}}%
\pgfpathlineto{\pgfqpoint{0.728318in}{1.270925in}}%
\pgfpathlineto{\pgfqpoint{0.732859in}{1.275134in}}%
\pgfpathlineto{\pgfqpoint{0.743310in}{1.284536in}}%
\pgfpathlineto{\pgfqpoint{0.748516in}{1.289062in}}%
\pgfpathlineto{\pgfqpoint{0.759331in}{1.298148in}}%
\pgfpathlineto{\pgfqpoint{0.764172in}{1.302096in}}%
\pgfpathlineto{\pgfqpoint{0.776496in}{1.311759in}}%
\pgfpathlineto{\pgfqpoint{0.779829in}{1.314308in}}%
\pgfpathlineto{\pgfqpoint{0.794949in}{1.325370in}}%
\pgfpathlineto{\pgfqpoint{0.795485in}{1.325754in}}%
\pgfpathlineto{\pgfqpoint{0.811142in}{1.336459in}}%
\pgfpathlineto{\pgfqpoint{0.815040in}{1.338981in}}%
\pgfpathlineto{\pgfqpoint{0.826798in}{1.346475in}}%
\pgfpathlineto{\pgfqpoint{0.837010in}{1.352592in}}%
\pgfpathlineto{\pgfqpoint{0.842455in}{1.355820in}}%
\pgfpathlineto{\pgfqpoint{0.858112in}{1.364499in}}%
\pgfpathlineto{\pgfqpoint{0.861419in}{1.366203in}}%
\pgfpathlineto{\pgfqpoint{0.873768in}{1.372534in}}%
\pgfpathlineto{\pgfqpoint{0.889232in}{1.379814in}}%
\pgfpathlineto{\pgfqpoint{0.889425in}{1.379905in}}%
\pgfpathlineto{\pgfqpoint{0.905081in}{1.386654in}}%
\pgfpathlineto{\pgfqpoint{0.920738in}{1.392726in}}%
\pgfpathlineto{\pgfqpoint{0.922757in}{1.393425in}}%
\pgfpathlineto{\pgfqpoint{0.936394in}{1.398176in}}%
\pgfpathlineto{\pgfqpoint{0.952051in}{1.402949in}}%
\pgfpathlineto{\pgfqpoint{0.967702in}{1.407036in}}%
\pgfpathlineto{\pgfqpoint{0.967708in}{1.407038in}}%
\pgfpathlineto{\pgfqpoint{0.983364in}{1.410498in}}%
\pgfpathlineto{\pgfqpoint{0.999021in}{1.413265in}}%
\pgfpathlineto{\pgfqpoint{1.014677in}{1.415339in}}%
\pgfpathlineto{\pgfqpoint{1.030334in}{1.416722in}}%
\pgfpathlineto{\pgfqpoint{1.045990in}{1.417413in}}%
\pgfpathlineto{\pgfqpoint{1.061647in}{1.417413in}}%
\pgfpathlineto{\pgfqpoint{1.077303in}{1.416722in}}%
\pgfpathlineto{\pgfqpoint{1.092960in}{1.415339in}}%
\pgfpathlineto{\pgfqpoint{1.108617in}{1.413265in}}%
\pgfpathlineto{\pgfqpoint{1.124273in}{1.410498in}}%
\pgfpathlineto{\pgfqpoint{1.139930in}{1.407038in}}%
\pgfpathlineto{\pgfqpoint{1.139935in}{1.407036in}}%
\pgfpathlineto{\pgfqpoint{1.155586in}{1.402949in}}%
\pgfpathlineto{\pgfqpoint{1.171243in}{1.398176in}}%
\pgfpathlineto{\pgfqpoint{1.184880in}{1.393425in}}%
\pgfpathlineto{\pgfqpoint{1.186899in}{1.392726in}}%
\pgfpathlineto{\pgfqpoint{1.202556in}{1.386654in}}%
\pgfpathlineto{\pgfqpoint{1.218213in}{1.379905in}}%
\pgfpathlineto{\pgfqpoint{1.218405in}{1.379814in}}%
\pgfpathlineto{\pgfqpoint{1.233869in}{1.372534in}}%
\pgfpathlineto{\pgfqpoint{1.246218in}{1.366203in}}%
\pgfpathlineto{\pgfqpoint{1.249526in}{1.364499in}}%
\pgfpathlineto{\pgfqpoint{1.265182in}{1.355820in}}%
\pgfpathlineto{\pgfqpoint{1.270627in}{1.352592in}}%
\pgfpathlineto{\pgfqpoint{1.280839in}{1.346475in}}%
\pgfpathlineto{\pgfqpoint{1.292598in}{1.338981in}}%
\pgfpathlineto{\pgfqpoint{1.296495in}{1.336459in}}%
\pgfpathlineto{\pgfqpoint{1.312152in}{1.325754in}}%
\pgfpathlineto{\pgfqpoint{1.312688in}{1.325370in}}%
\pgfpathlineto{\pgfqpoint{1.327809in}{1.314308in}}%
\pgfpathlineto{\pgfqpoint{1.331141in}{1.311759in}}%
\pgfpathlineto{\pgfqpoint{1.343465in}{1.302096in}}%
\pgfpathlineto{\pgfqpoint{1.348306in}{1.298148in}}%
\pgfpathlineto{\pgfqpoint{1.359122in}{1.289062in}}%
\pgfpathlineto{\pgfqpoint{1.364327in}{1.284536in}}%
\pgfpathlineto{\pgfqpoint{1.374778in}{1.275134in}}%
\pgfpathlineto{\pgfqpoint{1.379320in}{1.270925in}}%
\pgfpathlineto{\pgfqpoint{1.390435in}{1.260212in}}%
\pgfpathlineto{\pgfqpoint{1.393367in}{1.257314in}}%
\pgfpathlineto{\pgfqpoint{1.406091in}{1.244169in}}%
\pgfpathlineto{\pgfqpoint{1.406534in}{1.243703in}}%
\pgfpathlineto{\pgfqpoint{1.418847in}{1.230092in}}%
\pgfpathlineto{\pgfqpoint{1.421748in}{1.226703in}}%
\pgfpathlineto{\pgfqpoint{1.430369in}{1.216481in}}%
\pgfpathlineto{\pgfqpoint{1.437404in}{1.207603in}}%
\pgfpathlineto{\pgfqpoint{1.441118in}{1.202870in}}%
\pgfpathlineto{\pgfqpoint{1.451101in}{1.189259in}}%
\pgfpathlineto{\pgfqpoint{1.453061in}{1.186383in}}%
\pgfpathlineto{\pgfqpoint{1.460343in}{1.175647in}}%
\pgfpathlineto{\pgfqpoint{1.468718in}{1.162204in}}%
\pgfpathlineto{\pgfqpoint{1.468822in}{1.162036in}}%
\pgfpathlineto{\pgfqpoint{1.476585in}{1.148425in}}%
\pgfpathlineto{\pgfqpoint{1.483570in}{1.134814in}}%
\pgfpathlineto{\pgfqpoint{1.484374in}{1.133059in}}%
\pgfpathlineto{\pgfqpoint{1.489838in}{1.121203in}}%
\pgfpathlineto{\pgfqpoint{1.495329in}{1.107592in}}%
\pgfpathlineto{\pgfqpoint{1.500031in}{1.093986in}}%
\pgfpathlineto{\pgfqpoint{1.500032in}{1.093981in}}%
\pgfpathlineto{\pgfqpoint{1.504013in}{1.080370in}}%
\pgfpathlineto{\pgfqpoint{1.507195in}{1.066759in}}%
\pgfpathlineto{\pgfqpoint{1.509581in}{1.053148in}}%
\pgfpathlineto{\pgfqpoint{1.511172in}{1.039536in}}%
\pgfpathlineto{\pgfqpoint{1.511967in}{1.025925in}}%
\pgfpathlineto{\pgfqpoint{1.511967in}{1.012314in}}%
\pgfpathlineto{\pgfqpoint{1.511172in}{0.998703in}}%
\pgfpathlineto{\pgfqpoint{1.509581in}{0.985092in}}%
\pgfpathlineto{\pgfqpoint{1.507195in}{0.971481in}}%
\pgfpathlineto{\pgfqpoint{1.504013in}{0.957870in}}%
\pgfpathlineto{\pgfqpoint{1.500032in}{0.944259in}}%
\pgfpathlineto{\pgfqpoint{1.500031in}{0.944254in}}%
\pgfpathlineto{\pgfqpoint{1.495329in}{0.930647in}}%
\pgfpathlineto{\pgfqpoint{1.489838in}{0.917036in}}%
\pgfpathlineto{\pgfqpoint{1.484374in}{0.905180in}}%
\pgfpathlineto{\pgfqpoint{1.483570in}{0.903425in}}%
\pgfpathlineto{\pgfqpoint{1.476585in}{0.889814in}}%
\pgfpathlineto{\pgfqpoint{1.468822in}{0.876203in}}%
\pgfpathlineto{\pgfqpoint{1.468718in}{0.876036in}}%
\pgfpathlineto{\pgfqpoint{1.460343in}{0.862592in}}%
\pgfpathlineto{\pgfqpoint{1.453061in}{0.851856in}}%
\pgfpathlineto{\pgfqpoint{1.451101in}{0.848981in}}%
\pgfpathlineto{\pgfqpoint{1.441118in}{0.835370in}}%
\pgfpathlineto{\pgfqpoint{1.437404in}{0.830636in}}%
\pgfpathlineto{\pgfqpoint{1.430369in}{0.821759in}}%
\pgfpathlineto{\pgfqpoint{1.421748in}{0.811536in}}%
\pgfpathlineto{\pgfqpoint{1.418847in}{0.808148in}}%
\pgfpathlineto{\pgfqpoint{1.406534in}{0.794536in}}%
\pgfpathlineto{\pgfqpoint{1.406091in}{0.794070in}}%
\pgfpathlineto{\pgfqpoint{1.393367in}{0.780925in}}%
\pgfpathlineto{\pgfqpoint{1.390435in}{0.778028in}}%
\pgfpathlineto{\pgfqpoint{1.379320in}{0.767314in}}%
\pgfpathlineto{\pgfqpoint{1.374778in}{0.763106in}}%
\pgfpathlineto{\pgfqpoint{1.364327in}{0.753703in}}%
\pgfpathlineto{\pgfqpoint{1.359122in}{0.749177in}}%
\pgfpathlineto{\pgfqpoint{1.348306in}{0.740092in}}%
\pgfpathlineto{\pgfqpoint{1.343465in}{0.736144in}}%
\pgfpathlineto{\pgfqpoint{1.331141in}{0.726481in}}%
\pgfpathlineto{\pgfqpoint{1.327809in}{0.723931in}}%
\pgfpathlineto{\pgfqpoint{1.312688in}{0.712870in}}%
\pgfpathlineto{\pgfqpoint{1.312152in}{0.712485in}}%
\pgfpathlineto{\pgfqpoint{1.296495in}{0.701780in}}%
\pgfpathlineto{\pgfqpoint{1.292598in}{0.699259in}}%
\pgfpathlineto{\pgfqpoint{1.280839in}{0.691764in}}%
\pgfpathlineto{\pgfqpoint{1.270627in}{0.685648in}}%
\pgfpathlineto{\pgfqpoint{1.265182in}{0.682419in}}%
\pgfpathlineto{\pgfqpoint{1.249526in}{0.673741in}}%
\pgfpathlineto{\pgfqpoint{1.246218in}{0.672036in}}%
\pgfpathlineto{\pgfqpoint{1.233869in}{0.665706in}}%
\pgfpathlineto{\pgfqpoint{1.218405in}{0.658425in}}%
\pgfpathlineto{\pgfqpoint{1.218213in}{0.658335in}}%
\pgfpathlineto{\pgfqpoint{1.202556in}{0.651585in}}%
\pgfpathlineto{\pgfqpoint{1.186899in}{0.645513in}}%
\pgfpathlineto{\pgfqpoint{1.184880in}{0.644814in}}%
\pgfpathlineto{\pgfqpoint{1.171243in}{0.640064in}}%
\pgfpathlineto{\pgfqpoint{1.155586in}{0.635291in}}%
\pgfpathlineto{\pgfqpoint{1.139935in}{0.631203in}}%
\pgfpathlineto{\pgfqpoint{1.139930in}{0.631202in}}%
\pgfpathlineto{\pgfqpoint{1.124273in}{0.627741in}}%
\pgfpathlineto{\pgfqpoint{1.108617in}{0.624975in}}%
\pgfpathlineto{\pgfqpoint{1.092960in}{0.622900in}}%
\pgfpathlineto{\pgfqpoint{1.077303in}{0.621518in}}%
\pgfpathlineto{\pgfqpoint{1.061647in}{0.620826in}}%
\pgfpathlineto{\pgfqpoint{1.045990in}{0.620826in}}%
\pgfpathlineto{\pgfqpoint{1.030334in}{0.621518in}}%
\pgfpathlineto{\pgfqpoint{1.014677in}{0.622900in}}%
\pgfpathlineto{\pgfqpoint{0.999021in}{0.624975in}}%
\pgfpathlineto{\pgfqpoint{0.983364in}{0.627741in}}%
\pgfpathlineto{\pgfqpoint{0.967708in}{0.631202in}}%
\pgfpathlineto{\pgfqpoint{0.967702in}{0.631203in}}%
\pgfpathclose%
\pgfusepath{fill}%
\end{pgfscope}%
\begin{pgfscope}%
\pgfpathrectangle{\pgfqpoint{0.278819in}{0.345370in}}{\pgfqpoint{1.550000in}{1.347500in}}%
\pgfusepath{clip}%
\pgfsetbuttcap%
\pgfsetroundjoin%
\definecolor{currentfill}{rgb}{0.481929,0.136891,0.507989}%
\pgfsetfillcolor{currentfill}%
\pgfsetlinewidth{0.000000pt}%
\definecolor{currentstroke}{rgb}{0.000000,0.000000,0.000000}%
\pgfsetstrokecolor{currentstroke}%
\pgfsetdash{}{0pt}%
\pgfpathmoveto{\pgfqpoint{0.795485in}{0.345370in}}%
\pgfpathlineto{\pgfqpoint{0.811142in}{0.345370in}}%
\pgfpathlineto{\pgfqpoint{0.826798in}{0.345370in}}%
\pgfpathlineto{\pgfqpoint{0.842455in}{0.345370in}}%
\pgfpathlineto{\pgfqpoint{0.858112in}{0.345370in}}%
\pgfpathlineto{\pgfqpoint{0.873768in}{0.345370in}}%
\pgfpathlineto{\pgfqpoint{0.889425in}{0.345370in}}%
\pgfpathlineto{\pgfqpoint{0.905081in}{0.345370in}}%
\pgfpathlineto{\pgfqpoint{0.920738in}{0.345370in}}%
\pgfpathlineto{\pgfqpoint{0.936394in}{0.345370in}}%
\pgfpathlineto{\pgfqpoint{0.952051in}{0.345370in}}%
\pgfpathlineto{\pgfqpoint{0.967708in}{0.345370in}}%
\pgfpathlineto{\pgfqpoint{0.983364in}{0.345370in}}%
\pgfpathlineto{\pgfqpoint{0.999021in}{0.345370in}}%
\pgfpathlineto{\pgfqpoint{1.014677in}{0.345370in}}%
\pgfpathlineto{\pgfqpoint{1.030334in}{0.345370in}}%
\pgfpathlineto{\pgfqpoint{1.045990in}{0.345370in}}%
\pgfpathlineto{\pgfqpoint{1.061647in}{0.345370in}}%
\pgfpathlineto{\pgfqpoint{1.077303in}{0.345370in}}%
\pgfpathlineto{\pgfqpoint{1.092960in}{0.345370in}}%
\pgfpathlineto{\pgfqpoint{1.108617in}{0.345370in}}%
\pgfpathlineto{\pgfqpoint{1.124273in}{0.345370in}}%
\pgfpathlineto{\pgfqpoint{1.139930in}{0.345370in}}%
\pgfpathlineto{\pgfqpoint{1.155586in}{0.345370in}}%
\pgfpathlineto{\pgfqpoint{1.171243in}{0.345370in}}%
\pgfpathlineto{\pgfqpoint{1.186899in}{0.345370in}}%
\pgfpathlineto{\pgfqpoint{1.202556in}{0.345370in}}%
\pgfpathlineto{\pgfqpoint{1.218213in}{0.345370in}}%
\pgfpathlineto{\pgfqpoint{1.233869in}{0.345370in}}%
\pgfpathlineto{\pgfqpoint{1.249526in}{0.345370in}}%
\pgfpathlineto{\pgfqpoint{1.265182in}{0.345370in}}%
\pgfpathlineto{\pgfqpoint{1.280839in}{0.345370in}}%
\pgfpathlineto{\pgfqpoint{1.296495in}{0.345370in}}%
\pgfpathlineto{\pgfqpoint{1.312152in}{0.345370in}}%
\pgfpathlineto{\pgfqpoint{1.319147in}{0.345370in}}%
\pgfpathlineto{\pgfqpoint{1.319942in}{0.358981in}}%
\pgfpathlineto{\pgfqpoint{1.322318in}{0.372592in}}%
\pgfpathlineto{\pgfqpoint{1.326249in}{0.386203in}}%
\pgfpathlineto{\pgfqpoint{1.327809in}{0.390091in}}%
\pgfpathlineto{\pgfqpoint{1.331575in}{0.399814in}}%
\pgfpathlineto{\pgfqpoint{1.338264in}{0.413425in}}%
\pgfpathlineto{\pgfqpoint{1.343465in}{0.422220in}}%
\pgfpathlineto{\pgfqpoint{1.346223in}{0.427036in}}%
\pgfpathlineto{\pgfqpoint{1.355275in}{0.440648in}}%
\pgfpathlineto{\pgfqpoint{1.359122in}{0.445759in}}%
\pgfpathlineto{\pgfqpoint{1.365331in}{0.454259in}}%
\pgfpathlineto{\pgfqpoint{1.374778in}{0.465940in}}%
\pgfpathlineto{\pgfqpoint{1.376299in}{0.467870in}}%
\pgfpathlineto{\pgfqpoint{1.388009in}{0.481481in}}%
\pgfpathlineto{\pgfqpoint{1.390435in}{0.484098in}}%
\pgfpathlineto{\pgfqpoint{1.400390in}{0.495092in}}%
\pgfpathlineto{\pgfqpoint{1.406091in}{0.501007in}}%
\pgfpathlineto{\pgfqpoint{1.413363in}{0.508703in}}%
\pgfpathlineto{\pgfqpoint{1.421748in}{0.517120in}}%
\pgfpathlineto{\pgfqpoint{1.426838in}{0.522314in}}%
\pgfpathlineto{\pgfqpoint{1.437404in}{0.532624in}}%
\pgfpathlineto{\pgfqpoint{1.440744in}{0.535925in}}%
\pgfpathlineto{\pgfqpoint{1.453061in}{0.547652in}}%
\pgfpathlineto{\pgfqpoint{1.455021in}{0.549536in}}%
\pgfpathlineto{\pgfqpoint{1.468718in}{0.562299in}}%
\pgfpathlineto{\pgfqpoint{1.469623in}{0.563148in}}%
\pgfpathlineto{\pgfqpoint{1.484374in}{0.576631in}}%
\pgfpathlineto{\pgfqpoint{1.484514in}{0.576759in}}%
\pgfpathlineto{\pgfqpoint{1.499674in}{0.590370in}}%
\pgfpathlineto{\pgfqpoint{1.500031in}{0.590685in}}%
\pgfpathlineto{\pgfqpoint{1.515086in}{0.603981in}}%
\pgfpathlineto{\pgfqpoint{1.515687in}{0.604507in}}%
\pgfpathlineto{\pgfqpoint{1.530739in}{0.617592in}}%
\pgfpathlineto{\pgfqpoint{1.531344in}{0.618115in}}%
\pgfpathlineto{\pgfqpoint{1.546638in}{0.631203in}}%
\pgfpathlineto{\pgfqpoint{1.547000in}{0.631513in}}%
\pgfpathlineto{\pgfqpoint{1.562657in}{0.644693in}}%
\pgfpathlineto{\pgfqpoint{1.562804in}{0.644814in}}%
\pgfpathlineto{\pgfqpoint{1.578314in}{0.657638in}}%
\pgfpathlineto{\pgfqpoint{1.579290in}{0.658425in}}%
\pgfpathlineto{\pgfqpoint{1.593970in}{0.670333in}}%
\pgfpathlineto{\pgfqpoint{1.596137in}{0.672036in}}%
\pgfpathlineto{\pgfqpoint{1.609627in}{0.682744in}}%
\pgfpathlineto{\pgfqpoint{1.613424in}{0.685648in}}%
\pgfpathlineto{\pgfqpoint{1.625283in}{0.694834in}}%
\pgfpathlineto{\pgfqpoint{1.631259in}{0.699259in}}%
\pgfpathlineto{\pgfqpoint{1.640940in}{0.706548in}}%
\pgfpathlineto{\pgfqpoint{1.649793in}{0.712870in}}%
\pgfpathlineto{\pgfqpoint{1.656596in}{0.717826in}}%
\pgfpathlineto{\pgfqpoint{1.669243in}{0.726481in}}%
\pgfpathlineto{\pgfqpoint{1.672253in}{0.728590in}}%
\pgfpathlineto{\pgfqpoint{1.687910in}{0.738770in}}%
\pgfpathlineto{\pgfqpoint{1.690129in}{0.740092in}}%
\pgfpathlineto{\pgfqpoint{1.703566in}{0.748305in}}%
\pgfpathlineto{\pgfqpoint{1.713343in}{0.753703in}}%
\pgfpathlineto{\pgfqpoint{1.719223in}{0.757047in}}%
\pgfpathlineto{\pgfqpoint{1.734879in}{0.764917in}}%
\pgfpathlineto{\pgfqpoint{1.740419in}{0.767314in}}%
\pgfpathlineto{\pgfqpoint{1.750536in}{0.771836in}}%
\pgfpathlineto{\pgfqpoint{1.766192in}{0.777651in}}%
\pgfpathlineto{\pgfqpoint{1.777377in}{0.780925in}}%
\pgfpathlineto{\pgfqpoint{1.781849in}{0.782281in}}%
\pgfpathlineto{\pgfqpoint{1.797505in}{0.785699in}}%
\pgfpathlineto{\pgfqpoint{1.813162in}{0.787764in}}%
\pgfpathlineto{\pgfqpoint{1.828819in}{0.788455in}}%
\pgfpathlineto{\pgfqpoint{1.828819in}{0.794536in}}%
\pgfpathlineto{\pgfqpoint{1.828819in}{0.808148in}}%
\pgfpathlineto{\pgfqpoint{1.828819in}{0.821759in}}%
\pgfpathlineto{\pgfqpoint{1.828819in}{0.835370in}}%
\pgfpathlineto{\pgfqpoint{1.828819in}{0.848981in}}%
\pgfpathlineto{\pgfqpoint{1.828819in}{0.862592in}}%
\pgfpathlineto{\pgfqpoint{1.828819in}{0.876203in}}%
\pgfpathlineto{\pgfqpoint{1.828819in}{0.889814in}}%
\pgfpathlineto{\pgfqpoint{1.828819in}{0.903425in}}%
\pgfpathlineto{\pgfqpoint{1.828819in}{0.917036in}}%
\pgfpathlineto{\pgfqpoint{1.828819in}{0.930648in}}%
\pgfpathlineto{\pgfqpoint{1.828819in}{0.944259in}}%
\pgfpathlineto{\pgfqpoint{1.828819in}{0.957870in}}%
\pgfpathlineto{\pgfqpoint{1.828819in}{0.971481in}}%
\pgfpathlineto{\pgfqpoint{1.828819in}{0.985092in}}%
\pgfpathlineto{\pgfqpoint{1.828819in}{0.998703in}}%
\pgfpathlineto{\pgfqpoint{1.828819in}{1.012314in}}%
\pgfpathlineto{\pgfqpoint{1.828819in}{1.025925in}}%
\pgfpathlineto{\pgfqpoint{1.828819in}{1.039536in}}%
\pgfpathlineto{\pgfqpoint{1.828819in}{1.053148in}}%
\pgfpathlineto{\pgfqpoint{1.828819in}{1.066759in}}%
\pgfpathlineto{\pgfqpoint{1.828819in}{1.080370in}}%
\pgfpathlineto{\pgfqpoint{1.828819in}{1.093981in}}%
\pgfpathlineto{\pgfqpoint{1.828819in}{1.107592in}}%
\pgfpathlineto{\pgfqpoint{1.828819in}{1.121203in}}%
\pgfpathlineto{\pgfqpoint{1.828819in}{1.134814in}}%
\pgfpathlineto{\pgfqpoint{1.828819in}{1.148425in}}%
\pgfpathlineto{\pgfqpoint{1.828819in}{1.162036in}}%
\pgfpathlineto{\pgfqpoint{1.828819in}{1.175647in}}%
\pgfpathlineto{\pgfqpoint{1.828819in}{1.189259in}}%
\pgfpathlineto{\pgfqpoint{1.828819in}{1.202870in}}%
\pgfpathlineto{\pgfqpoint{1.828819in}{1.216481in}}%
\pgfpathlineto{\pgfqpoint{1.828819in}{1.230092in}}%
\pgfpathlineto{\pgfqpoint{1.828819in}{1.243703in}}%
\pgfpathlineto{\pgfqpoint{1.828819in}{1.249784in}}%
\pgfpathlineto{\pgfqpoint{1.813162in}{1.250475in}}%
\pgfpathlineto{\pgfqpoint{1.797505in}{1.252541in}}%
\pgfpathlineto{\pgfqpoint{1.781849in}{1.255958in}}%
\pgfpathlineto{\pgfqpoint{1.777377in}{1.257314in}}%
\pgfpathlineto{\pgfqpoint{1.766192in}{1.260588in}}%
\pgfpathlineto{\pgfqpoint{1.750536in}{1.266404in}}%
\pgfpathlineto{\pgfqpoint{1.740419in}{1.270925in}}%
\pgfpathlineto{\pgfqpoint{1.734879in}{1.273323in}}%
\pgfpathlineto{\pgfqpoint{1.719223in}{1.281192in}}%
\pgfpathlineto{\pgfqpoint{1.713343in}{1.284536in}}%
\pgfpathlineto{\pgfqpoint{1.703566in}{1.289935in}}%
\pgfpathlineto{\pgfqpoint{1.690129in}{1.298148in}}%
\pgfpathlineto{\pgfqpoint{1.687910in}{1.299469in}}%
\pgfpathlineto{\pgfqpoint{1.672253in}{1.309650in}}%
\pgfpathlineto{\pgfqpoint{1.669243in}{1.311759in}}%
\pgfpathlineto{\pgfqpoint{1.656596in}{1.320413in}}%
\pgfpathlineto{\pgfqpoint{1.649793in}{1.325370in}}%
\pgfpathlineto{\pgfqpoint{1.640940in}{1.331691in}}%
\pgfpathlineto{\pgfqpoint{1.631259in}{1.338981in}}%
\pgfpathlineto{\pgfqpoint{1.625283in}{1.343406in}}%
\pgfpathlineto{\pgfqpoint{1.613424in}{1.352592in}}%
\pgfpathlineto{\pgfqpoint{1.609627in}{1.355495in}}%
\pgfpathlineto{\pgfqpoint{1.596137in}{1.366203in}}%
\pgfpathlineto{\pgfqpoint{1.593970in}{1.367907in}}%
\pgfpathlineto{\pgfqpoint{1.579290in}{1.379814in}}%
\pgfpathlineto{\pgfqpoint{1.578314in}{1.380601in}}%
\pgfpathlineto{\pgfqpoint{1.562804in}{1.393425in}}%
\pgfpathlineto{\pgfqpoint{1.562657in}{1.393547in}}%
\pgfpathlineto{\pgfqpoint{1.547000in}{1.406727in}}%
\pgfpathlineto{\pgfqpoint{1.546638in}{1.407036in}}%
\pgfpathlineto{\pgfqpoint{1.531344in}{1.420124in}}%
\pgfpathlineto{\pgfqpoint{1.530739in}{1.420648in}}%
\pgfpathlineto{\pgfqpoint{1.515687in}{1.433733in}}%
\pgfpathlineto{\pgfqpoint{1.515086in}{1.434259in}}%
\pgfpathlineto{\pgfqpoint{1.500031in}{1.447555in}}%
\pgfpathlineto{\pgfqpoint{1.499674in}{1.447870in}}%
\pgfpathlineto{\pgfqpoint{1.484514in}{1.461481in}}%
\pgfpathlineto{\pgfqpoint{1.484374in}{1.461609in}}%
\pgfpathlineto{\pgfqpoint{1.469623in}{1.475092in}}%
\pgfpathlineto{\pgfqpoint{1.468718in}{1.475940in}}%
\pgfpathlineto{\pgfqpoint{1.455021in}{1.488703in}}%
\pgfpathlineto{\pgfqpoint{1.453061in}{1.490587in}}%
\pgfpathlineto{\pgfqpoint{1.440744in}{1.502314in}}%
\pgfpathlineto{\pgfqpoint{1.437404in}{1.505615in}}%
\pgfpathlineto{\pgfqpoint{1.426838in}{1.515925in}}%
\pgfpathlineto{\pgfqpoint{1.421748in}{1.521120in}}%
\pgfpathlineto{\pgfqpoint{1.413363in}{1.529536in}}%
\pgfpathlineto{\pgfqpoint{1.406091in}{1.537233in}}%
\pgfpathlineto{\pgfqpoint{1.400390in}{1.543148in}}%
\pgfpathlineto{\pgfqpoint{1.390435in}{1.554142in}}%
\pgfpathlineto{\pgfqpoint{1.388009in}{1.556759in}}%
\pgfpathlineto{\pgfqpoint{1.376299in}{1.570370in}}%
\pgfpathlineto{\pgfqpoint{1.374778in}{1.572300in}}%
\pgfpathlineto{\pgfqpoint{1.365331in}{1.583981in}}%
\pgfpathlineto{\pgfqpoint{1.359122in}{1.592480in}}%
\pgfpathlineto{\pgfqpoint{1.355275in}{1.597592in}}%
\pgfpathlineto{\pgfqpoint{1.346223in}{1.611203in}}%
\pgfpathlineto{\pgfqpoint{1.343465in}{1.616020in}}%
\pgfpathlineto{\pgfqpoint{1.338264in}{1.624814in}}%
\pgfpathlineto{\pgfqpoint{1.331575in}{1.638425in}}%
\pgfpathlineto{\pgfqpoint{1.327809in}{1.648149in}}%
\pgfpathlineto{\pgfqpoint{1.326249in}{1.652036in}}%
\pgfpathlineto{\pgfqpoint{1.322318in}{1.665648in}}%
\pgfpathlineto{\pgfqpoint{1.319942in}{1.679259in}}%
\pgfpathlineto{\pgfqpoint{1.319147in}{1.692870in}}%
\pgfpathlineto{\pgfqpoint{1.312152in}{1.692870in}}%
\pgfpathlineto{\pgfqpoint{1.296495in}{1.692870in}}%
\pgfpathlineto{\pgfqpoint{1.280839in}{1.692870in}}%
\pgfpathlineto{\pgfqpoint{1.265182in}{1.692870in}}%
\pgfpathlineto{\pgfqpoint{1.249526in}{1.692870in}}%
\pgfpathlineto{\pgfqpoint{1.233869in}{1.692870in}}%
\pgfpathlineto{\pgfqpoint{1.218213in}{1.692870in}}%
\pgfpathlineto{\pgfqpoint{1.202556in}{1.692870in}}%
\pgfpathlineto{\pgfqpoint{1.186899in}{1.692870in}}%
\pgfpathlineto{\pgfqpoint{1.171243in}{1.692870in}}%
\pgfpathlineto{\pgfqpoint{1.155586in}{1.692870in}}%
\pgfpathlineto{\pgfqpoint{1.139930in}{1.692870in}}%
\pgfpathlineto{\pgfqpoint{1.124273in}{1.692870in}}%
\pgfpathlineto{\pgfqpoint{1.108617in}{1.692870in}}%
\pgfpathlineto{\pgfqpoint{1.092960in}{1.692870in}}%
\pgfpathlineto{\pgfqpoint{1.077303in}{1.692870in}}%
\pgfpathlineto{\pgfqpoint{1.061647in}{1.692870in}}%
\pgfpathlineto{\pgfqpoint{1.045990in}{1.692870in}}%
\pgfpathlineto{\pgfqpoint{1.030334in}{1.692870in}}%
\pgfpathlineto{\pgfqpoint{1.014677in}{1.692870in}}%
\pgfpathlineto{\pgfqpoint{0.999021in}{1.692870in}}%
\pgfpathlineto{\pgfqpoint{0.983364in}{1.692870in}}%
\pgfpathlineto{\pgfqpoint{0.967708in}{1.692870in}}%
\pgfpathlineto{\pgfqpoint{0.952051in}{1.692870in}}%
\pgfpathlineto{\pgfqpoint{0.936394in}{1.692870in}}%
\pgfpathlineto{\pgfqpoint{0.920738in}{1.692870in}}%
\pgfpathlineto{\pgfqpoint{0.905081in}{1.692870in}}%
\pgfpathlineto{\pgfqpoint{0.889425in}{1.692870in}}%
\pgfpathlineto{\pgfqpoint{0.873768in}{1.692870in}}%
\pgfpathlineto{\pgfqpoint{0.858112in}{1.692870in}}%
\pgfpathlineto{\pgfqpoint{0.842455in}{1.692870in}}%
\pgfpathlineto{\pgfqpoint{0.826798in}{1.692870in}}%
\pgfpathlineto{\pgfqpoint{0.811142in}{1.692870in}}%
\pgfpathlineto{\pgfqpoint{0.795485in}{1.692870in}}%
\pgfpathlineto{\pgfqpoint{0.788490in}{1.692870in}}%
\pgfpathlineto{\pgfqpoint{0.787696in}{1.679259in}}%
\pgfpathlineto{\pgfqpoint{0.785320in}{1.665648in}}%
\pgfpathlineto{\pgfqpoint{0.781388in}{1.652036in}}%
\pgfpathlineto{\pgfqpoint{0.779829in}{1.648149in}}%
\pgfpathlineto{\pgfqpoint{0.776063in}{1.638425in}}%
\pgfpathlineto{\pgfqpoint{0.769373in}{1.624814in}}%
\pgfpathlineto{\pgfqpoint{0.764172in}{1.616020in}}%
\pgfpathlineto{\pgfqpoint{0.761415in}{1.611203in}}%
\pgfpathlineto{\pgfqpoint{0.752362in}{1.597592in}}%
\pgfpathlineto{\pgfqpoint{0.748516in}{1.592480in}}%
\pgfpathlineto{\pgfqpoint{0.742306in}{1.583981in}}%
\pgfpathlineto{\pgfqpoint{0.732859in}{1.572300in}}%
\pgfpathlineto{\pgfqpoint{0.731339in}{1.570370in}}%
\pgfpathlineto{\pgfqpoint{0.719628in}{1.556759in}}%
\pgfpathlineto{\pgfqpoint{0.717202in}{1.554142in}}%
\pgfpathlineto{\pgfqpoint{0.707247in}{1.543148in}}%
\pgfpathlineto{\pgfqpoint{0.701546in}{1.537233in}}%
\pgfpathlineto{\pgfqpoint{0.694275in}{1.529536in}}%
\pgfpathlineto{\pgfqpoint{0.685889in}{1.521120in}}%
\pgfpathlineto{\pgfqpoint{0.680799in}{1.515925in}}%
\pgfpathlineto{\pgfqpoint{0.670233in}{1.505615in}}%
\pgfpathlineto{\pgfqpoint{0.666893in}{1.502314in}}%
\pgfpathlineto{\pgfqpoint{0.654576in}{1.490587in}}%
\pgfpathlineto{\pgfqpoint{0.652616in}{1.488703in}}%
\pgfpathlineto{\pgfqpoint{0.638920in}{1.475940in}}%
\pgfpathlineto{\pgfqpoint{0.638014in}{1.475092in}}%
\pgfpathlineto{\pgfqpoint{0.623263in}{1.461609in}}%
\pgfpathlineto{\pgfqpoint{0.623124in}{1.461481in}}%
\pgfpathlineto{\pgfqpoint{0.607963in}{1.447870in}}%
\pgfpathlineto{\pgfqpoint{0.607606in}{1.447555in}}%
\pgfpathlineto{\pgfqpoint{0.592552in}{1.434259in}}%
\pgfpathlineto{\pgfqpoint{0.591950in}{1.433733in}}%
\pgfpathlineto{\pgfqpoint{0.576899in}{1.420648in}}%
\pgfpathlineto{\pgfqpoint{0.576293in}{1.420124in}}%
\pgfpathlineto{\pgfqpoint{0.560999in}{1.407036in}}%
\pgfpathlineto{\pgfqpoint{0.560637in}{1.406727in}}%
\pgfpathlineto{\pgfqpoint{0.544980in}{1.393547in}}%
\pgfpathlineto{\pgfqpoint{0.544833in}{1.393425in}}%
\pgfpathlineto{\pgfqpoint{0.529324in}{1.380601in}}%
\pgfpathlineto{\pgfqpoint{0.528348in}{1.379814in}}%
\pgfpathlineto{\pgfqpoint{0.513667in}{1.367907in}}%
\pgfpathlineto{\pgfqpoint{0.511500in}{1.366203in}}%
\pgfpathlineto{\pgfqpoint{0.498011in}{1.355495in}}%
\pgfpathlineto{\pgfqpoint{0.494213in}{1.352592in}}%
\pgfpathlineto{\pgfqpoint{0.482354in}{1.343406in}}%
\pgfpathlineto{\pgfqpoint{0.476379in}{1.338981in}}%
\pgfpathlineto{\pgfqpoint{0.466697in}{1.331691in}}%
\pgfpathlineto{\pgfqpoint{0.457844in}{1.325370in}}%
\pgfpathlineto{\pgfqpoint{0.451041in}{1.320413in}}%
\pgfpathlineto{\pgfqpoint{0.438394in}{1.311759in}}%
\pgfpathlineto{\pgfqpoint{0.435384in}{1.309650in}}%
\pgfpathlineto{\pgfqpoint{0.419728in}{1.299469in}}%
\pgfpathlineto{\pgfqpoint{0.417508in}{1.298148in}}%
\pgfpathlineto{\pgfqpoint{0.404071in}{1.289935in}}%
\pgfpathlineto{\pgfqpoint{0.394294in}{1.284536in}}%
\pgfpathlineto{\pgfqpoint{0.388415in}{1.281192in}}%
\pgfpathlineto{\pgfqpoint{0.372758in}{1.273323in}}%
\pgfpathlineto{\pgfqpoint{0.367218in}{1.270925in}}%
\pgfpathlineto{\pgfqpoint{0.357101in}{1.266404in}}%
\pgfpathlineto{\pgfqpoint{0.341445in}{1.260588in}}%
\pgfpathlineto{\pgfqpoint{0.330260in}{1.257314in}}%
\pgfpathlineto{\pgfqpoint{0.325788in}{1.255958in}}%
\pgfpathlineto{\pgfqpoint{0.310132in}{1.252541in}}%
\pgfpathlineto{\pgfqpoint{0.294475in}{1.250475in}}%
\pgfpathlineto{\pgfqpoint{0.278819in}{1.249784in}}%
\pgfpathlineto{\pgfqpoint{0.278819in}{1.243703in}}%
\pgfpathlineto{\pgfqpoint{0.278819in}{1.230092in}}%
\pgfpathlineto{\pgfqpoint{0.278819in}{1.216481in}}%
\pgfpathlineto{\pgfqpoint{0.278819in}{1.202870in}}%
\pgfpathlineto{\pgfqpoint{0.278819in}{1.189259in}}%
\pgfpathlineto{\pgfqpoint{0.278819in}{1.175647in}}%
\pgfpathlineto{\pgfqpoint{0.278819in}{1.162036in}}%
\pgfpathlineto{\pgfqpoint{0.278819in}{1.148425in}}%
\pgfpathlineto{\pgfqpoint{0.278819in}{1.134814in}}%
\pgfpathlineto{\pgfqpoint{0.278819in}{1.121203in}}%
\pgfpathlineto{\pgfqpoint{0.278819in}{1.107592in}}%
\pgfpathlineto{\pgfqpoint{0.278819in}{1.093981in}}%
\pgfpathlineto{\pgfqpoint{0.278819in}{1.080370in}}%
\pgfpathlineto{\pgfqpoint{0.278819in}{1.066759in}}%
\pgfpathlineto{\pgfqpoint{0.278819in}{1.053148in}}%
\pgfpathlineto{\pgfqpoint{0.278819in}{1.039536in}}%
\pgfpathlineto{\pgfqpoint{0.278819in}{1.025925in}}%
\pgfpathlineto{\pgfqpoint{0.278819in}{1.012314in}}%
\pgfpathlineto{\pgfqpoint{0.278819in}{0.998703in}}%
\pgfpathlineto{\pgfqpoint{0.278819in}{0.985092in}}%
\pgfpathlineto{\pgfqpoint{0.278819in}{0.971481in}}%
\pgfpathlineto{\pgfqpoint{0.278819in}{0.957870in}}%
\pgfpathlineto{\pgfqpoint{0.278819in}{0.944259in}}%
\pgfpathlineto{\pgfqpoint{0.278819in}{0.930648in}}%
\pgfpathlineto{\pgfqpoint{0.278819in}{0.917036in}}%
\pgfpathlineto{\pgfqpoint{0.278819in}{0.903425in}}%
\pgfpathlineto{\pgfqpoint{0.278819in}{0.889814in}}%
\pgfpathlineto{\pgfqpoint{0.278819in}{0.876203in}}%
\pgfpathlineto{\pgfqpoint{0.278819in}{0.862592in}}%
\pgfpathlineto{\pgfqpoint{0.278819in}{0.848981in}}%
\pgfpathlineto{\pgfqpoint{0.278819in}{0.835370in}}%
\pgfpathlineto{\pgfqpoint{0.278819in}{0.821759in}}%
\pgfpathlineto{\pgfqpoint{0.278819in}{0.808148in}}%
\pgfpathlineto{\pgfqpoint{0.278819in}{0.794536in}}%
\pgfpathlineto{\pgfqpoint{0.278819in}{0.788455in}}%
\pgfpathlineto{\pgfqpoint{0.294475in}{0.787764in}}%
\pgfpathlineto{\pgfqpoint{0.310132in}{0.785699in}}%
\pgfpathlineto{\pgfqpoint{0.325788in}{0.782281in}}%
\pgfpathlineto{\pgfqpoint{0.330260in}{0.780925in}}%
\pgfpathlineto{\pgfqpoint{0.341445in}{0.777651in}}%
\pgfpathlineto{\pgfqpoint{0.357101in}{0.771836in}}%
\pgfpathlineto{\pgfqpoint{0.367218in}{0.767314in}}%
\pgfpathlineto{\pgfqpoint{0.372758in}{0.764917in}}%
\pgfpathlineto{\pgfqpoint{0.388415in}{0.757047in}}%
\pgfpathlineto{\pgfqpoint{0.394294in}{0.753703in}}%
\pgfpathlineto{\pgfqpoint{0.404071in}{0.748305in}}%
\pgfpathlineto{\pgfqpoint{0.417508in}{0.740092in}}%
\pgfpathlineto{\pgfqpoint{0.419728in}{0.738770in}}%
\pgfpathlineto{\pgfqpoint{0.435384in}{0.728590in}}%
\pgfpathlineto{\pgfqpoint{0.438394in}{0.726481in}}%
\pgfpathlineto{\pgfqpoint{0.451041in}{0.717826in}}%
\pgfpathlineto{\pgfqpoint{0.457844in}{0.712870in}}%
\pgfpathlineto{\pgfqpoint{0.466697in}{0.706548in}}%
\pgfpathlineto{\pgfqpoint{0.476379in}{0.699259in}}%
\pgfpathlineto{\pgfqpoint{0.482354in}{0.694834in}}%
\pgfpathlineto{\pgfqpoint{0.494213in}{0.685648in}}%
\pgfpathlineto{\pgfqpoint{0.498011in}{0.682744in}}%
\pgfpathlineto{\pgfqpoint{0.511500in}{0.672036in}}%
\pgfpathlineto{\pgfqpoint{0.513667in}{0.670333in}}%
\pgfpathlineto{\pgfqpoint{0.528348in}{0.658425in}}%
\pgfpathlineto{\pgfqpoint{0.529324in}{0.657638in}}%
\pgfpathlineto{\pgfqpoint{0.544833in}{0.644814in}}%
\pgfpathlineto{\pgfqpoint{0.544980in}{0.644693in}}%
\pgfpathlineto{\pgfqpoint{0.560637in}{0.631513in}}%
\pgfpathlineto{\pgfqpoint{0.560999in}{0.631203in}}%
\pgfpathlineto{\pgfqpoint{0.576293in}{0.618115in}}%
\pgfpathlineto{\pgfqpoint{0.576899in}{0.617592in}}%
\pgfpathlineto{\pgfqpoint{0.591950in}{0.604507in}}%
\pgfpathlineto{\pgfqpoint{0.592552in}{0.603981in}}%
\pgfpathlineto{\pgfqpoint{0.607606in}{0.590685in}}%
\pgfpathlineto{\pgfqpoint{0.607963in}{0.590370in}}%
\pgfpathlineto{\pgfqpoint{0.623124in}{0.576759in}}%
\pgfpathlineto{\pgfqpoint{0.623263in}{0.576631in}}%
\pgfpathlineto{\pgfqpoint{0.638014in}{0.563148in}}%
\pgfpathlineto{\pgfqpoint{0.638920in}{0.562299in}}%
\pgfpathlineto{\pgfqpoint{0.652616in}{0.549536in}}%
\pgfpathlineto{\pgfqpoint{0.654576in}{0.547652in}}%
\pgfpathlineto{\pgfqpoint{0.666893in}{0.535925in}}%
\pgfpathlineto{\pgfqpoint{0.670233in}{0.532624in}}%
\pgfpathlineto{\pgfqpoint{0.680799in}{0.522314in}}%
\pgfpathlineto{\pgfqpoint{0.685889in}{0.517120in}}%
\pgfpathlineto{\pgfqpoint{0.694275in}{0.508703in}}%
\pgfpathlineto{\pgfqpoint{0.701546in}{0.501007in}}%
\pgfpathlineto{\pgfqpoint{0.707247in}{0.495092in}}%
\pgfpathlineto{\pgfqpoint{0.717202in}{0.484098in}}%
\pgfpathlineto{\pgfqpoint{0.719628in}{0.481481in}}%
\pgfpathlineto{\pgfqpoint{0.731339in}{0.467870in}}%
\pgfpathlineto{\pgfqpoint{0.732859in}{0.465940in}}%
\pgfpathlineto{\pgfqpoint{0.742306in}{0.454259in}}%
\pgfpathlineto{\pgfqpoint{0.748516in}{0.445759in}}%
\pgfpathlineto{\pgfqpoint{0.752362in}{0.440648in}}%
\pgfpathlineto{\pgfqpoint{0.761415in}{0.427036in}}%
\pgfpathlineto{\pgfqpoint{0.764172in}{0.422220in}}%
\pgfpathlineto{\pgfqpoint{0.769373in}{0.413425in}}%
\pgfpathlineto{\pgfqpoint{0.776063in}{0.399814in}}%
\pgfpathlineto{\pgfqpoint{0.779829in}{0.390091in}}%
\pgfpathlineto{\pgfqpoint{0.781388in}{0.386203in}}%
\pgfpathlineto{\pgfqpoint{0.785320in}{0.372592in}}%
\pgfpathlineto{\pgfqpoint{0.787696in}{0.358981in}}%
\pgfpathlineto{\pgfqpoint{0.788490in}{0.345370in}}%
\pgfpathlineto{\pgfqpoint{0.795485in}{0.345370in}}%
\pgfpathclose%
\pgfpathmoveto{\pgfqpoint{1.028018in}{0.481481in}}%
\pgfpathlineto{\pgfqpoint{1.014677in}{0.483093in}}%
\pgfpathlineto{\pgfqpoint{0.999021in}{0.485931in}}%
\pgfpathlineto{\pgfqpoint{0.983364in}{0.489717in}}%
\pgfpathlineto{\pgfqpoint{0.967708in}{0.494452in}}%
\pgfpathlineto{\pgfqpoint{0.965937in}{0.495092in}}%
\pgfpathlineto{\pgfqpoint{0.952051in}{0.499767in}}%
\pgfpathlineto{\pgfqpoint{0.936394in}{0.505916in}}%
\pgfpathlineto{\pgfqpoint{0.930167in}{0.508703in}}%
\pgfpathlineto{\pgfqpoint{0.920738in}{0.512673in}}%
\pgfpathlineto{\pgfqpoint{0.905081in}{0.520075in}}%
\pgfpathlineto{\pgfqpoint{0.900798in}{0.522314in}}%
\pgfpathlineto{\pgfqpoint{0.889425in}{0.527960in}}%
\pgfpathlineto{\pgfqpoint{0.874803in}{0.535925in}}%
\pgfpathlineto{\pgfqpoint{0.873768in}{0.536465in}}%
\pgfpathlineto{\pgfqpoint{0.858112in}{0.545304in}}%
\pgfpathlineto{\pgfqpoint{0.851155in}{0.549536in}}%
\pgfpathlineto{\pgfqpoint{0.842455in}{0.554640in}}%
\pgfpathlineto{\pgfqpoint{0.828907in}{0.563148in}}%
\pgfpathlineto{\pgfqpoint{0.826798in}{0.564433in}}%
\pgfpathlineto{\pgfqpoint{0.811142in}{0.574560in}}%
\pgfpathlineto{\pgfqpoint{0.807923in}{0.576759in}}%
\pgfpathlineto{\pgfqpoint{0.795485in}{0.585056in}}%
\pgfpathlineto{\pgfqpoint{0.787911in}{0.590370in}}%
\pgfpathlineto{\pgfqpoint{0.779829in}{0.595939in}}%
\pgfpathlineto{\pgfqpoint{0.768677in}{0.603981in}}%
\pgfpathlineto{\pgfqpoint{0.764172in}{0.607188in}}%
\pgfpathlineto{\pgfqpoint{0.750142in}{0.617592in}}%
\pgfpathlineto{\pgfqpoint{0.748516in}{0.618789in}}%
\pgfpathlineto{\pgfqpoint{0.732859in}{0.630726in}}%
\pgfpathlineto{\pgfqpoint{0.732252in}{0.631203in}}%
\pgfpathlineto{\pgfqpoint{0.717202in}{0.643000in}}%
\pgfpathlineto{\pgfqpoint{0.714951in}{0.644814in}}%
\pgfpathlineto{\pgfqpoint{0.701546in}{0.655636in}}%
\pgfpathlineto{\pgfqpoint{0.698170in}{0.658425in}}%
\pgfpathlineto{\pgfqpoint{0.685889in}{0.668636in}}%
\pgfpathlineto{\pgfqpoint{0.681877in}{0.672036in}}%
\pgfpathlineto{\pgfqpoint{0.670233in}{0.682010in}}%
\pgfpathlineto{\pgfqpoint{0.666049in}{0.685648in}}%
\pgfpathlineto{\pgfqpoint{0.654576in}{0.695771in}}%
\pgfpathlineto{\pgfqpoint{0.650665in}{0.699259in}}%
\pgfpathlineto{\pgfqpoint{0.638920in}{0.709935in}}%
\pgfpathlineto{\pgfqpoint{0.635711in}{0.712870in}}%
\pgfpathlineto{\pgfqpoint{0.623263in}{0.724524in}}%
\pgfpathlineto{\pgfqpoint{0.621177in}{0.726481in}}%
\pgfpathlineto{\pgfqpoint{0.607606in}{0.739564in}}%
\pgfpathlineto{\pgfqpoint{0.607057in}{0.740092in}}%
\pgfpathlineto{\pgfqpoint{0.593326in}{0.753703in}}%
\pgfpathlineto{\pgfqpoint{0.591950in}{0.755117in}}%
\pgfpathlineto{\pgfqpoint{0.579983in}{0.767314in}}%
\pgfpathlineto{\pgfqpoint{0.576293in}{0.771231in}}%
\pgfpathlineto{\pgfqpoint{0.567044in}{0.780925in}}%
\pgfpathlineto{\pgfqpoint{0.560637in}{0.787952in}}%
\pgfpathlineto{\pgfqpoint{0.554524in}{0.794536in}}%
\pgfpathlineto{\pgfqpoint{0.544980in}{0.805349in}}%
\pgfpathlineto{\pgfqpoint{0.542451in}{0.808148in}}%
\pgfpathlineto{\pgfqpoint{0.530802in}{0.821759in}}%
\pgfpathlineto{\pgfqpoint{0.529324in}{0.823592in}}%
\pgfpathlineto{\pgfqpoint{0.519537in}{0.835370in}}%
\pgfpathlineto{\pgfqpoint{0.513667in}{0.842933in}}%
\pgfpathlineto{\pgfqpoint{0.508799in}{0.848981in}}%
\pgfpathlineto{\pgfqpoint{0.498631in}{0.862592in}}%
\pgfpathlineto{\pgfqpoint{0.498011in}{0.863492in}}%
\pgfpathlineto{\pgfqpoint{0.488848in}{0.876203in}}%
\pgfpathlineto{\pgfqpoint{0.482354in}{0.886090in}}%
\pgfpathlineto{\pgfqpoint{0.479778in}{0.889814in}}%
\pgfpathlineto{\pgfqpoint{0.471264in}{0.903425in}}%
\pgfpathlineto{\pgfqpoint{0.466697in}{0.911623in}}%
\pgfpathlineto{\pgfqpoint{0.463492in}{0.917036in}}%
\pgfpathlineto{\pgfqpoint{0.456418in}{0.930648in}}%
\pgfpathlineto{\pgfqpoint{0.451041in}{0.942719in}}%
\pgfpathlineto{\pgfqpoint{0.450304in}{0.944259in}}%
\pgfpathlineto{\pgfqpoint{0.444858in}{0.957870in}}%
\pgfpathlineto{\pgfqpoint{0.440504in}{0.971481in}}%
\pgfpathlineto{\pgfqpoint{0.437239in}{0.985092in}}%
\pgfpathlineto{\pgfqpoint{0.435384in}{0.996690in}}%
\pgfpathlineto{\pgfqpoint{0.435034in}{0.998703in}}%
\pgfpathlineto{\pgfqpoint{0.433848in}{1.012314in}}%
\pgfpathlineto{\pgfqpoint{0.433848in}{1.025925in}}%
\pgfpathlineto{\pgfqpoint{0.435034in}{1.039536in}}%
\pgfpathlineto{\pgfqpoint{0.435384in}{1.041549in}}%
\pgfpathlineto{\pgfqpoint{0.437239in}{1.053148in}}%
\pgfpathlineto{\pgfqpoint{0.440504in}{1.066759in}}%
\pgfpathlineto{\pgfqpoint{0.444858in}{1.080370in}}%
\pgfpathlineto{\pgfqpoint{0.450304in}{1.093981in}}%
\pgfpathlineto{\pgfqpoint{0.451041in}{1.095520in}}%
\pgfpathlineto{\pgfqpoint{0.456418in}{1.107592in}}%
\pgfpathlineto{\pgfqpoint{0.463492in}{1.121203in}}%
\pgfpathlineto{\pgfqpoint{0.466697in}{1.126617in}}%
\pgfpathlineto{\pgfqpoint{0.471264in}{1.134814in}}%
\pgfpathlineto{\pgfqpoint{0.479778in}{1.148425in}}%
\pgfpathlineto{\pgfqpoint{0.482354in}{1.152149in}}%
\pgfpathlineto{\pgfqpoint{0.488848in}{1.162036in}}%
\pgfpathlineto{\pgfqpoint{0.498011in}{1.174747in}}%
\pgfpathlineto{\pgfqpoint{0.498631in}{1.175647in}}%
\pgfpathlineto{\pgfqpoint{0.508799in}{1.189259in}}%
\pgfpathlineto{\pgfqpoint{0.513667in}{1.195306in}}%
\pgfpathlineto{\pgfqpoint{0.519537in}{1.202870in}}%
\pgfpathlineto{\pgfqpoint{0.529324in}{1.214647in}}%
\pgfpathlineto{\pgfqpoint{0.530802in}{1.216481in}}%
\pgfpathlineto{\pgfqpoint{0.542451in}{1.230092in}}%
\pgfpathlineto{\pgfqpoint{0.544980in}{1.232890in}}%
\pgfpathlineto{\pgfqpoint{0.554524in}{1.243703in}}%
\pgfpathlineto{\pgfqpoint{0.560637in}{1.250287in}}%
\pgfpathlineto{\pgfqpoint{0.567044in}{1.257314in}}%
\pgfpathlineto{\pgfqpoint{0.576293in}{1.267009in}}%
\pgfpathlineto{\pgfqpoint{0.579983in}{1.270925in}}%
\pgfpathlineto{\pgfqpoint{0.591950in}{1.283122in}}%
\pgfpathlineto{\pgfqpoint{0.593326in}{1.284536in}}%
\pgfpathlineto{\pgfqpoint{0.607057in}{1.298148in}}%
\pgfpathlineto{\pgfqpoint{0.607606in}{1.298675in}}%
\pgfpathlineto{\pgfqpoint{0.621177in}{1.311759in}}%
\pgfpathlineto{\pgfqpoint{0.623263in}{1.313716in}}%
\pgfpathlineto{\pgfqpoint{0.635711in}{1.325370in}}%
\pgfpathlineto{\pgfqpoint{0.638920in}{1.328305in}}%
\pgfpathlineto{\pgfqpoint{0.650665in}{1.338981in}}%
\pgfpathlineto{\pgfqpoint{0.654576in}{1.342469in}}%
\pgfpathlineto{\pgfqpoint{0.666049in}{1.352592in}}%
\pgfpathlineto{\pgfqpoint{0.670233in}{1.356229in}}%
\pgfpathlineto{\pgfqpoint{0.681877in}{1.366203in}}%
\pgfpathlineto{\pgfqpoint{0.685889in}{1.369603in}}%
\pgfpathlineto{\pgfqpoint{0.698170in}{1.379814in}}%
\pgfpathlineto{\pgfqpoint{0.701546in}{1.382604in}}%
\pgfpathlineto{\pgfqpoint{0.714951in}{1.393425in}}%
\pgfpathlineto{\pgfqpoint{0.717202in}{1.395239in}}%
\pgfpathlineto{\pgfqpoint{0.732252in}{1.407036in}}%
\pgfpathlineto{\pgfqpoint{0.732859in}{1.407514in}}%
\pgfpathlineto{\pgfqpoint{0.748516in}{1.419451in}}%
\pgfpathlineto{\pgfqpoint{0.750142in}{1.420648in}}%
\pgfpathlineto{\pgfqpoint{0.764172in}{1.431051in}}%
\pgfpathlineto{\pgfqpoint{0.768677in}{1.434259in}}%
\pgfpathlineto{\pgfqpoint{0.779829in}{1.442300in}}%
\pgfpathlineto{\pgfqpoint{0.787911in}{1.447870in}}%
\pgfpathlineto{\pgfqpoint{0.795485in}{1.453184in}}%
\pgfpathlineto{\pgfqpoint{0.807923in}{1.461481in}}%
\pgfpathlineto{\pgfqpoint{0.811142in}{1.463679in}}%
\pgfpathlineto{\pgfqpoint{0.826798in}{1.473807in}}%
\pgfpathlineto{\pgfqpoint{0.828907in}{1.475092in}}%
\pgfpathlineto{\pgfqpoint{0.842455in}{1.483600in}}%
\pgfpathlineto{\pgfqpoint{0.851155in}{1.488703in}}%
\pgfpathlineto{\pgfqpoint{0.858112in}{1.492935in}}%
\pgfpathlineto{\pgfqpoint{0.873768in}{1.501774in}}%
\pgfpathlineto{\pgfqpoint{0.874803in}{1.502314in}}%
\pgfpathlineto{\pgfqpoint{0.889425in}{1.510280in}}%
\pgfpathlineto{\pgfqpoint{0.900798in}{1.515925in}}%
\pgfpathlineto{\pgfqpoint{0.905081in}{1.518164in}}%
\pgfpathlineto{\pgfqpoint{0.920738in}{1.525566in}}%
\pgfpathlineto{\pgfqpoint{0.930167in}{1.529536in}}%
\pgfpathlineto{\pgfqpoint{0.936394in}{1.532323in}}%
\pgfpathlineto{\pgfqpoint{0.952051in}{1.538473in}}%
\pgfpathlineto{\pgfqpoint{0.965937in}{1.543148in}}%
\pgfpathlineto{\pgfqpoint{0.967708in}{1.543788in}}%
\pgfpathlineto{\pgfqpoint{0.983364in}{1.548522in}}%
\pgfpathlineto{\pgfqpoint{0.999021in}{1.552308in}}%
\pgfpathlineto{\pgfqpoint{1.014677in}{1.555146in}}%
\pgfpathlineto{\pgfqpoint{1.028018in}{1.556759in}}%
\pgfpathlineto{\pgfqpoint{1.030334in}{1.557063in}}%
\pgfpathlineto{\pgfqpoint{1.045990in}{1.558094in}}%
\pgfpathlineto{\pgfqpoint{1.061647in}{1.558094in}}%
\pgfpathlineto{\pgfqpoint{1.077303in}{1.557063in}}%
\pgfpathlineto{\pgfqpoint{1.079619in}{1.556759in}}%
\pgfpathlineto{\pgfqpoint{1.092960in}{1.555146in}}%
\pgfpathlineto{\pgfqpoint{1.108617in}{1.552308in}}%
\pgfpathlineto{\pgfqpoint{1.124273in}{1.548522in}}%
\pgfpathlineto{\pgfqpoint{1.139930in}{1.543788in}}%
\pgfpathlineto{\pgfqpoint{1.141701in}{1.543148in}}%
\pgfpathlineto{\pgfqpoint{1.155586in}{1.538473in}}%
\pgfpathlineto{\pgfqpoint{1.171243in}{1.532323in}}%
\pgfpathlineto{\pgfqpoint{1.177470in}{1.529536in}}%
\pgfpathlineto{\pgfqpoint{1.186899in}{1.525566in}}%
\pgfpathlineto{\pgfqpoint{1.202556in}{1.518164in}}%
\pgfpathlineto{\pgfqpoint{1.206839in}{1.515925in}}%
\pgfpathlineto{\pgfqpoint{1.218213in}{1.510280in}}%
\pgfpathlineto{\pgfqpoint{1.232834in}{1.502314in}}%
\pgfpathlineto{\pgfqpoint{1.233869in}{1.501774in}}%
\pgfpathlineto{\pgfqpoint{1.249526in}{1.492935in}}%
\pgfpathlineto{\pgfqpoint{1.256482in}{1.488703in}}%
\pgfpathlineto{\pgfqpoint{1.265182in}{1.483600in}}%
\pgfpathlineto{\pgfqpoint{1.278730in}{1.475092in}}%
\pgfpathlineto{\pgfqpoint{1.280839in}{1.473807in}}%
\pgfpathlineto{\pgfqpoint{1.296495in}{1.463679in}}%
\pgfpathlineto{\pgfqpoint{1.299714in}{1.461481in}}%
\pgfpathlineto{\pgfqpoint{1.312152in}{1.453184in}}%
\pgfpathlineto{\pgfqpoint{1.319726in}{1.447870in}}%
\pgfpathlineto{\pgfqpoint{1.327809in}{1.442300in}}%
\pgfpathlineto{\pgfqpoint{1.338960in}{1.434259in}}%
\pgfpathlineto{\pgfqpoint{1.343465in}{1.431051in}}%
\pgfpathlineto{\pgfqpoint{1.357495in}{1.420648in}}%
\pgfpathlineto{\pgfqpoint{1.359122in}{1.419451in}}%
\pgfpathlineto{\pgfqpoint{1.374778in}{1.407514in}}%
\pgfpathlineto{\pgfqpoint{1.375385in}{1.407036in}}%
\pgfpathlineto{\pgfqpoint{1.390435in}{1.395239in}}%
\pgfpathlineto{\pgfqpoint{1.392686in}{1.393425in}}%
\pgfpathlineto{\pgfqpoint{1.406091in}{1.382604in}}%
\pgfpathlineto{\pgfqpoint{1.409467in}{1.379814in}}%
\pgfpathlineto{\pgfqpoint{1.421748in}{1.369603in}}%
\pgfpathlineto{\pgfqpoint{1.425760in}{1.366203in}}%
\pgfpathlineto{\pgfqpoint{1.437404in}{1.356229in}}%
\pgfpathlineto{\pgfqpoint{1.441588in}{1.352592in}}%
\pgfpathlineto{\pgfqpoint{1.453061in}{1.342469in}}%
\pgfpathlineto{\pgfqpoint{1.456972in}{1.338981in}}%
\pgfpathlineto{\pgfqpoint{1.468718in}{1.328305in}}%
\pgfpathlineto{\pgfqpoint{1.471926in}{1.325370in}}%
\pgfpathlineto{\pgfqpoint{1.484374in}{1.313716in}}%
\pgfpathlineto{\pgfqpoint{1.486461in}{1.311759in}}%
\pgfpathlineto{\pgfqpoint{1.500031in}{1.298675in}}%
\pgfpathlineto{\pgfqpoint{1.500580in}{1.298148in}}%
\pgfpathlineto{\pgfqpoint{1.514311in}{1.284536in}}%
\pgfpathlineto{\pgfqpoint{1.515687in}{1.283122in}}%
\pgfpathlineto{\pgfqpoint{1.527654in}{1.270925in}}%
\pgfpathlineto{\pgfqpoint{1.531344in}{1.267009in}}%
\pgfpathlineto{\pgfqpoint{1.540594in}{1.257314in}}%
\pgfpathlineto{\pgfqpoint{1.547000in}{1.250287in}}%
\pgfpathlineto{\pgfqpoint{1.553113in}{1.243703in}}%
\pgfpathlineto{\pgfqpoint{1.562657in}{1.232890in}}%
\pgfpathlineto{\pgfqpoint{1.565186in}{1.230092in}}%
\pgfpathlineto{\pgfqpoint{1.576835in}{1.216481in}}%
\pgfpathlineto{\pgfqpoint{1.578314in}{1.214647in}}%
\pgfpathlineto{\pgfqpoint{1.588100in}{1.202870in}}%
\pgfpathlineto{\pgfqpoint{1.593970in}{1.195306in}}%
\pgfpathlineto{\pgfqpoint{1.598838in}{1.189259in}}%
\pgfpathlineto{\pgfqpoint{1.609006in}{1.175647in}}%
\pgfpathlineto{\pgfqpoint{1.609627in}{1.174747in}}%
\pgfpathlineto{\pgfqpoint{1.618789in}{1.162036in}}%
\pgfpathlineto{\pgfqpoint{1.625283in}{1.152149in}}%
\pgfpathlineto{\pgfqpoint{1.627859in}{1.148425in}}%
\pgfpathlineto{\pgfqpoint{1.636373in}{1.134814in}}%
\pgfpathlineto{\pgfqpoint{1.640940in}{1.126617in}}%
\pgfpathlineto{\pgfqpoint{1.644146in}{1.121203in}}%
\pgfpathlineto{\pgfqpoint{1.651219in}{1.107592in}}%
\pgfpathlineto{\pgfqpoint{1.656596in}{1.095520in}}%
\pgfpathlineto{\pgfqpoint{1.657333in}{1.093981in}}%
\pgfpathlineto{\pgfqpoint{1.662779in}{1.080370in}}%
\pgfpathlineto{\pgfqpoint{1.667133in}{1.066759in}}%
\pgfpathlineto{\pgfqpoint{1.670398in}{1.053148in}}%
\pgfpathlineto{\pgfqpoint{1.672253in}{1.041549in}}%
\pgfpathlineto{\pgfqpoint{1.672603in}{1.039536in}}%
\pgfpathlineto{\pgfqpoint{1.673790in}{1.025925in}}%
\pgfpathlineto{\pgfqpoint{1.673790in}{1.012314in}}%
\pgfpathlineto{\pgfqpoint{1.672603in}{0.998703in}}%
\pgfpathlineto{\pgfqpoint{1.672253in}{0.996690in}}%
\pgfpathlineto{\pgfqpoint{1.670398in}{0.985092in}}%
\pgfpathlineto{\pgfqpoint{1.667133in}{0.971481in}}%
\pgfpathlineto{\pgfqpoint{1.662779in}{0.957870in}}%
\pgfpathlineto{\pgfqpoint{1.657333in}{0.944259in}}%
\pgfpathlineto{\pgfqpoint{1.656596in}{0.942719in}}%
\pgfpathlineto{\pgfqpoint{1.651219in}{0.930648in}}%
\pgfpathlineto{\pgfqpoint{1.644146in}{0.917036in}}%
\pgfpathlineto{\pgfqpoint{1.640940in}{0.911623in}}%
\pgfpathlineto{\pgfqpoint{1.636373in}{0.903425in}}%
\pgfpathlineto{\pgfqpoint{1.627859in}{0.889814in}}%
\pgfpathlineto{\pgfqpoint{1.625283in}{0.886090in}}%
\pgfpathlineto{\pgfqpoint{1.618789in}{0.876203in}}%
\pgfpathlineto{\pgfqpoint{1.609627in}{0.863492in}}%
\pgfpathlineto{\pgfqpoint{1.609006in}{0.862592in}}%
\pgfpathlineto{\pgfqpoint{1.598838in}{0.848981in}}%
\pgfpathlineto{\pgfqpoint{1.593970in}{0.842933in}}%
\pgfpathlineto{\pgfqpoint{1.588100in}{0.835370in}}%
\pgfpathlineto{\pgfqpoint{1.578314in}{0.823592in}}%
\pgfpathlineto{\pgfqpoint{1.576835in}{0.821759in}}%
\pgfpathlineto{\pgfqpoint{1.565186in}{0.808148in}}%
\pgfpathlineto{\pgfqpoint{1.562657in}{0.805349in}}%
\pgfpathlineto{\pgfqpoint{1.553113in}{0.794536in}}%
\pgfpathlineto{\pgfqpoint{1.547000in}{0.787952in}}%
\pgfpathlineto{\pgfqpoint{1.540594in}{0.780925in}}%
\pgfpathlineto{\pgfqpoint{1.531344in}{0.771231in}}%
\pgfpathlineto{\pgfqpoint{1.527654in}{0.767314in}}%
\pgfpathlineto{\pgfqpoint{1.515687in}{0.755117in}}%
\pgfpathlineto{\pgfqpoint{1.514311in}{0.753703in}}%
\pgfpathlineto{\pgfqpoint{1.500580in}{0.740092in}}%
\pgfpathlineto{\pgfqpoint{1.500031in}{0.739564in}}%
\pgfpathlineto{\pgfqpoint{1.486461in}{0.726481in}}%
\pgfpathlineto{\pgfqpoint{1.484374in}{0.724524in}}%
\pgfpathlineto{\pgfqpoint{1.471926in}{0.712870in}}%
\pgfpathlineto{\pgfqpoint{1.468718in}{0.709935in}}%
\pgfpathlineto{\pgfqpoint{1.456972in}{0.699259in}}%
\pgfpathlineto{\pgfqpoint{1.453061in}{0.695771in}}%
\pgfpathlineto{\pgfqpoint{1.441588in}{0.685648in}}%
\pgfpathlineto{\pgfqpoint{1.437404in}{0.682010in}}%
\pgfpathlineto{\pgfqpoint{1.425760in}{0.672036in}}%
\pgfpathlineto{\pgfqpoint{1.421748in}{0.668636in}}%
\pgfpathlineto{\pgfqpoint{1.409467in}{0.658425in}}%
\pgfpathlineto{\pgfqpoint{1.406091in}{0.655636in}}%
\pgfpathlineto{\pgfqpoint{1.392686in}{0.644814in}}%
\pgfpathlineto{\pgfqpoint{1.390435in}{0.643000in}}%
\pgfpathlineto{\pgfqpoint{1.375385in}{0.631203in}}%
\pgfpathlineto{\pgfqpoint{1.374778in}{0.630726in}}%
\pgfpathlineto{\pgfqpoint{1.359122in}{0.618789in}}%
\pgfpathlineto{\pgfqpoint{1.357495in}{0.617592in}}%
\pgfpathlineto{\pgfqpoint{1.343465in}{0.607188in}}%
\pgfpathlineto{\pgfqpoint{1.338960in}{0.603981in}}%
\pgfpathlineto{\pgfqpoint{1.327809in}{0.595939in}}%
\pgfpathlineto{\pgfqpoint{1.319726in}{0.590370in}}%
\pgfpathlineto{\pgfqpoint{1.312152in}{0.585056in}}%
\pgfpathlineto{\pgfqpoint{1.299714in}{0.576759in}}%
\pgfpathlineto{\pgfqpoint{1.296495in}{0.574560in}}%
\pgfpathlineto{\pgfqpoint{1.280839in}{0.564433in}}%
\pgfpathlineto{\pgfqpoint{1.278730in}{0.563148in}}%
\pgfpathlineto{\pgfqpoint{1.265182in}{0.554640in}}%
\pgfpathlineto{\pgfqpoint{1.256482in}{0.549536in}}%
\pgfpathlineto{\pgfqpoint{1.249526in}{0.545304in}}%
\pgfpathlineto{\pgfqpoint{1.233869in}{0.536465in}}%
\pgfpathlineto{\pgfqpoint{1.232834in}{0.535925in}}%
\pgfpathlineto{\pgfqpoint{1.218213in}{0.527960in}}%
\pgfpathlineto{\pgfqpoint{1.206839in}{0.522314in}}%
\pgfpathlineto{\pgfqpoint{1.202556in}{0.520075in}}%
\pgfpathlineto{\pgfqpoint{1.186899in}{0.512673in}}%
\pgfpathlineto{\pgfqpoint{1.177470in}{0.508703in}}%
\pgfpathlineto{\pgfqpoint{1.171243in}{0.505916in}}%
\pgfpathlineto{\pgfqpoint{1.155586in}{0.499767in}}%
\pgfpathlineto{\pgfqpoint{1.141701in}{0.495092in}}%
\pgfpathlineto{\pgfqpoint{1.139930in}{0.494452in}}%
\pgfpathlineto{\pgfqpoint{1.124273in}{0.489717in}}%
\pgfpathlineto{\pgfqpoint{1.108617in}{0.485931in}}%
\pgfpathlineto{\pgfqpoint{1.092960in}{0.483093in}}%
\pgfpathlineto{\pgfqpoint{1.079619in}{0.481481in}}%
\pgfpathlineto{\pgfqpoint{1.077303in}{0.481176in}}%
\pgfpathlineto{\pgfqpoint{1.061647in}{0.480145in}}%
\pgfpathlineto{\pgfqpoint{1.045990in}{0.480145in}}%
\pgfpathlineto{\pgfqpoint{1.030334in}{0.481176in}}%
\pgfpathlineto{\pgfqpoint{1.028018in}{0.481481in}}%
\pgfpathclose%
\pgfusepath{fill}%
\end{pgfscope}%
\begin{pgfscope}%
\pgfpathrectangle{\pgfqpoint{0.278819in}{0.345370in}}{\pgfqpoint{1.550000in}{1.347500in}}%
\pgfusepath{clip}%
\pgfsetbuttcap%
\pgfsetroundjoin%
\definecolor{currentfill}{rgb}{0.252220,0.059415,0.453248}%
\pgfsetfillcolor{currentfill}%
\pgfsetlinewidth{0.000000pt}%
\definecolor{currentstroke}{rgb}{0.000000,0.000000,0.000000}%
\pgfsetstrokecolor{currentstroke}%
\pgfsetdash{}{0pt}%
\pgfpathmoveto{\pgfqpoint{0.591950in}{0.345370in}}%
\pgfpathlineto{\pgfqpoint{0.607606in}{0.345370in}}%
\pgfpathlineto{\pgfqpoint{0.623263in}{0.345370in}}%
\pgfpathlineto{\pgfqpoint{0.638920in}{0.345370in}}%
\pgfpathlineto{\pgfqpoint{0.654576in}{0.345370in}}%
\pgfpathlineto{\pgfqpoint{0.670233in}{0.345370in}}%
\pgfpathlineto{\pgfqpoint{0.685889in}{0.345370in}}%
\pgfpathlineto{\pgfqpoint{0.701546in}{0.345370in}}%
\pgfpathlineto{\pgfqpoint{0.717202in}{0.345370in}}%
\pgfpathlineto{\pgfqpoint{0.732859in}{0.345370in}}%
\pgfpathlineto{\pgfqpoint{0.748516in}{0.345370in}}%
\pgfpathlineto{\pgfqpoint{0.764172in}{0.345370in}}%
\pgfpathlineto{\pgfqpoint{0.779829in}{0.345370in}}%
\pgfpathlineto{\pgfqpoint{0.788490in}{0.345370in}}%
\pgfpathlineto{\pgfqpoint{0.787696in}{0.358981in}}%
\pgfpathlineto{\pgfqpoint{0.785320in}{0.372592in}}%
\pgfpathlineto{\pgfqpoint{0.781388in}{0.386203in}}%
\pgfpathlineto{\pgfqpoint{0.779829in}{0.390091in}}%
\pgfpathlineto{\pgfqpoint{0.776063in}{0.399814in}}%
\pgfpathlineto{\pgfqpoint{0.769373in}{0.413425in}}%
\pgfpathlineto{\pgfqpoint{0.764172in}{0.422220in}}%
\pgfpathlineto{\pgfqpoint{0.761415in}{0.427036in}}%
\pgfpathlineto{\pgfqpoint{0.752362in}{0.440648in}}%
\pgfpathlineto{\pgfqpoint{0.748516in}{0.445759in}}%
\pgfpathlineto{\pgfqpoint{0.742306in}{0.454259in}}%
\pgfpathlineto{\pgfqpoint{0.732859in}{0.465940in}}%
\pgfpathlineto{\pgfqpoint{0.731339in}{0.467870in}}%
\pgfpathlineto{\pgfqpoint{0.719628in}{0.481481in}}%
\pgfpathlineto{\pgfqpoint{0.717202in}{0.484098in}}%
\pgfpathlineto{\pgfqpoint{0.707247in}{0.495092in}}%
\pgfpathlineto{\pgfqpoint{0.701546in}{0.501007in}}%
\pgfpathlineto{\pgfqpoint{0.694275in}{0.508703in}}%
\pgfpathlineto{\pgfqpoint{0.685889in}{0.517120in}}%
\pgfpathlineto{\pgfqpoint{0.680799in}{0.522314in}}%
\pgfpathlineto{\pgfqpoint{0.670233in}{0.532624in}}%
\pgfpathlineto{\pgfqpoint{0.666893in}{0.535925in}}%
\pgfpathlineto{\pgfqpoint{0.654576in}{0.547652in}}%
\pgfpathlineto{\pgfqpoint{0.652616in}{0.549536in}}%
\pgfpathlineto{\pgfqpoint{0.638920in}{0.562299in}}%
\pgfpathlineto{\pgfqpoint{0.638014in}{0.563148in}}%
\pgfpathlineto{\pgfqpoint{0.623263in}{0.576631in}}%
\pgfpathlineto{\pgfqpoint{0.623124in}{0.576759in}}%
\pgfpathlineto{\pgfqpoint{0.607963in}{0.590370in}}%
\pgfpathlineto{\pgfqpoint{0.607606in}{0.590685in}}%
\pgfpathlineto{\pgfqpoint{0.592552in}{0.603981in}}%
\pgfpathlineto{\pgfqpoint{0.591950in}{0.604507in}}%
\pgfpathlineto{\pgfqpoint{0.576899in}{0.617592in}}%
\pgfpathlineto{\pgfqpoint{0.576293in}{0.618115in}}%
\pgfpathlineto{\pgfqpoint{0.560999in}{0.631203in}}%
\pgfpathlineto{\pgfqpoint{0.560637in}{0.631513in}}%
\pgfpathlineto{\pgfqpoint{0.544980in}{0.644693in}}%
\pgfpathlineto{\pgfqpoint{0.544833in}{0.644814in}}%
\pgfpathlineto{\pgfqpoint{0.529324in}{0.657638in}}%
\pgfpathlineto{\pgfqpoint{0.528348in}{0.658425in}}%
\pgfpathlineto{\pgfqpoint{0.513667in}{0.670333in}}%
\pgfpathlineto{\pgfqpoint{0.511500in}{0.672036in}}%
\pgfpathlineto{\pgfqpoint{0.498011in}{0.682744in}}%
\pgfpathlineto{\pgfqpoint{0.494213in}{0.685648in}}%
\pgfpathlineto{\pgfqpoint{0.482354in}{0.694834in}}%
\pgfpathlineto{\pgfqpoint{0.476379in}{0.699259in}}%
\pgfpathlineto{\pgfqpoint{0.466697in}{0.706548in}}%
\pgfpathlineto{\pgfqpoint{0.457844in}{0.712870in}}%
\pgfpathlineto{\pgfqpoint{0.451041in}{0.717826in}}%
\pgfpathlineto{\pgfqpoint{0.438394in}{0.726481in}}%
\pgfpathlineto{\pgfqpoint{0.435384in}{0.728590in}}%
\pgfpathlineto{\pgfqpoint{0.419728in}{0.738770in}}%
\pgfpathlineto{\pgfqpoint{0.417508in}{0.740092in}}%
\pgfpathlineto{\pgfqpoint{0.404071in}{0.748305in}}%
\pgfpathlineto{\pgfqpoint{0.394294in}{0.753703in}}%
\pgfpathlineto{\pgfqpoint{0.388415in}{0.757047in}}%
\pgfpathlineto{\pgfqpoint{0.372758in}{0.764917in}}%
\pgfpathlineto{\pgfqpoint{0.367218in}{0.767314in}}%
\pgfpathlineto{\pgfqpoint{0.357101in}{0.771836in}}%
\pgfpathlineto{\pgfqpoint{0.341445in}{0.777651in}}%
\pgfpathlineto{\pgfqpoint{0.330260in}{0.780925in}}%
\pgfpathlineto{\pgfqpoint{0.325788in}{0.782281in}}%
\pgfpathlineto{\pgfqpoint{0.310132in}{0.785699in}}%
\pgfpathlineto{\pgfqpoint{0.294475in}{0.787764in}}%
\pgfpathlineto{\pgfqpoint{0.278819in}{0.788455in}}%
\pgfpathlineto{\pgfqpoint{0.278819in}{0.780925in}}%
\pgfpathlineto{\pgfqpoint{0.278819in}{0.767314in}}%
\pgfpathlineto{\pgfqpoint{0.278819in}{0.753703in}}%
\pgfpathlineto{\pgfqpoint{0.278819in}{0.740092in}}%
\pgfpathlineto{\pgfqpoint{0.278819in}{0.726481in}}%
\pgfpathlineto{\pgfqpoint{0.278819in}{0.712870in}}%
\pgfpathlineto{\pgfqpoint{0.278819in}{0.699259in}}%
\pgfpathlineto{\pgfqpoint{0.278819in}{0.685648in}}%
\pgfpathlineto{\pgfqpoint{0.278819in}{0.672036in}}%
\pgfpathlineto{\pgfqpoint{0.278819in}{0.658425in}}%
\pgfpathlineto{\pgfqpoint{0.278819in}{0.644814in}}%
\pgfpathlineto{\pgfqpoint{0.278819in}{0.631203in}}%
\pgfpathlineto{\pgfqpoint{0.278819in}{0.617592in}}%
\pgfpathlineto{\pgfqpoint{0.278819in}{0.605370in}}%
\pgfpathlineto{\pgfqpoint{0.294475in}{0.604830in}}%
\pgfpathlineto{\pgfqpoint{0.302717in}{0.603981in}}%
\pgfpathlineto{\pgfqpoint{0.310132in}{0.603199in}}%
\pgfpathlineto{\pgfqpoint{0.325788in}{0.600464in}}%
\pgfpathlineto{\pgfqpoint{0.341445in}{0.596676in}}%
\pgfpathlineto{\pgfqpoint{0.357101in}{0.591875in}}%
\pgfpathlineto{\pgfqpoint{0.361173in}{0.590370in}}%
\pgfpathlineto{\pgfqpoint{0.372758in}{0.585985in}}%
\pgfpathlineto{\pgfqpoint{0.388415in}{0.579117in}}%
\pgfpathlineto{\pgfqpoint{0.393163in}{0.576759in}}%
\pgfpathlineto{\pgfqpoint{0.404071in}{0.571197in}}%
\pgfpathlineto{\pgfqpoint{0.418346in}{0.563148in}}%
\pgfpathlineto{\pgfqpoint{0.419728in}{0.562344in}}%
\pgfpathlineto{\pgfqpoint{0.435384in}{0.552398in}}%
\pgfpathlineto{\pgfqpoint{0.439569in}{0.549536in}}%
\pgfpathlineto{\pgfqpoint{0.451041in}{0.541411in}}%
\pgfpathlineto{\pgfqpoint{0.458313in}{0.535925in}}%
\pgfpathlineto{\pgfqpoint{0.466697in}{0.529339in}}%
\pgfpathlineto{\pgfqpoint{0.475167in}{0.522314in}}%
\pgfpathlineto{\pgfqpoint{0.482354in}{0.516066in}}%
\pgfpathlineto{\pgfqpoint{0.490434in}{0.508703in}}%
\pgfpathlineto{\pgfqpoint{0.498011in}{0.501414in}}%
\pgfpathlineto{\pgfqpoint{0.504320in}{0.495092in}}%
\pgfpathlineto{\pgfqpoint{0.513667in}{0.485118in}}%
\pgfpathlineto{\pgfqpoint{0.516959in}{0.481481in}}%
\pgfpathlineto{\pgfqpoint{0.528400in}{0.467870in}}%
\pgfpathlineto{\pgfqpoint{0.529324in}{0.466669in}}%
\pgfpathlineto{\pgfqpoint{0.538583in}{0.454259in}}%
\pgfpathlineto{\pgfqpoint{0.544980in}{0.444775in}}%
\pgfpathlineto{\pgfqpoint{0.547693in}{0.440648in}}%
\pgfpathlineto{\pgfqpoint{0.555593in}{0.427036in}}%
\pgfpathlineto{\pgfqpoint{0.560637in}{0.416965in}}%
\pgfpathlineto{\pgfqpoint{0.562369in}{0.413425in}}%
\pgfpathlineto{\pgfqpoint{0.567891in}{0.399814in}}%
\pgfpathlineto{\pgfqpoint{0.572248in}{0.386203in}}%
\pgfpathlineto{\pgfqpoint{0.575394in}{0.372592in}}%
\pgfpathlineto{\pgfqpoint{0.576293in}{0.366146in}}%
\pgfpathlineto{\pgfqpoint{0.577271in}{0.358981in}}%
\pgfpathlineto{\pgfqpoint{0.577891in}{0.345370in}}%
\pgfpathlineto{\pgfqpoint{0.591950in}{0.345370in}}%
\pgfpathclose%
\pgfpathmoveto{\pgfqpoint{1.327809in}{0.345370in}}%
\pgfpathlineto{\pgfqpoint{1.343465in}{0.345370in}}%
\pgfpathlineto{\pgfqpoint{1.359122in}{0.345370in}}%
\pgfpathlineto{\pgfqpoint{1.374778in}{0.345370in}}%
\pgfpathlineto{\pgfqpoint{1.390435in}{0.345370in}}%
\pgfpathlineto{\pgfqpoint{1.406091in}{0.345370in}}%
\pgfpathlineto{\pgfqpoint{1.421748in}{0.345370in}}%
\pgfpathlineto{\pgfqpoint{1.437404in}{0.345370in}}%
\pgfpathlineto{\pgfqpoint{1.453061in}{0.345370in}}%
\pgfpathlineto{\pgfqpoint{1.468718in}{0.345370in}}%
\pgfpathlineto{\pgfqpoint{1.484374in}{0.345370in}}%
\pgfpathlineto{\pgfqpoint{1.500031in}{0.345370in}}%
\pgfpathlineto{\pgfqpoint{1.515687in}{0.345370in}}%
\pgfpathlineto{\pgfqpoint{1.529746in}{0.345370in}}%
\pgfpathlineto{\pgfqpoint{1.530367in}{0.358981in}}%
\pgfpathlineto{\pgfqpoint{1.531344in}{0.366146in}}%
\pgfpathlineto{\pgfqpoint{1.532243in}{0.372592in}}%
\pgfpathlineto{\pgfqpoint{1.535389in}{0.386203in}}%
\pgfpathlineto{\pgfqpoint{1.539746in}{0.399814in}}%
\pgfpathlineto{\pgfqpoint{1.545268in}{0.413425in}}%
\pgfpathlineto{\pgfqpoint{1.547000in}{0.416965in}}%
\pgfpathlineto{\pgfqpoint{1.552045in}{0.427036in}}%
\pgfpathlineto{\pgfqpoint{1.559944in}{0.440648in}}%
\pgfpathlineto{\pgfqpoint{1.562657in}{0.444775in}}%
\pgfpathlineto{\pgfqpoint{1.569055in}{0.454259in}}%
\pgfpathlineto{\pgfqpoint{1.578314in}{0.466669in}}%
\pgfpathlineto{\pgfqpoint{1.579237in}{0.467870in}}%
\pgfpathlineto{\pgfqpoint{1.590678in}{0.481481in}}%
\pgfpathlineto{\pgfqpoint{1.593970in}{0.485118in}}%
\pgfpathlineto{\pgfqpoint{1.603317in}{0.495092in}}%
\pgfpathlineto{\pgfqpoint{1.609627in}{0.501414in}}%
\pgfpathlineto{\pgfqpoint{1.617203in}{0.508703in}}%
\pgfpathlineto{\pgfqpoint{1.625283in}{0.516066in}}%
\pgfpathlineto{\pgfqpoint{1.632470in}{0.522314in}}%
\pgfpathlineto{\pgfqpoint{1.640940in}{0.529339in}}%
\pgfpathlineto{\pgfqpoint{1.649324in}{0.535925in}}%
\pgfpathlineto{\pgfqpoint{1.656596in}{0.541411in}}%
\pgfpathlineto{\pgfqpoint{1.668069in}{0.549536in}}%
\pgfpathlineto{\pgfqpoint{1.672253in}{0.552398in}}%
\pgfpathlineto{\pgfqpoint{1.687910in}{0.562344in}}%
\pgfpathlineto{\pgfqpoint{1.689291in}{0.563148in}}%
\pgfpathlineto{\pgfqpoint{1.703566in}{0.571197in}}%
\pgfpathlineto{\pgfqpoint{1.714474in}{0.576759in}}%
\pgfpathlineto{\pgfqpoint{1.719223in}{0.579117in}}%
\pgfpathlineto{\pgfqpoint{1.734879in}{0.585985in}}%
\pgfpathlineto{\pgfqpoint{1.746464in}{0.590370in}}%
\pgfpathlineto{\pgfqpoint{1.750536in}{0.591875in}}%
\pgfpathlineto{\pgfqpoint{1.766192in}{0.596676in}}%
\pgfpathlineto{\pgfqpoint{1.781849in}{0.600464in}}%
\pgfpathlineto{\pgfqpoint{1.797505in}{0.603199in}}%
\pgfpathlineto{\pgfqpoint{1.804921in}{0.603981in}}%
\pgfpathlineto{\pgfqpoint{1.813162in}{0.604830in}}%
\pgfpathlineto{\pgfqpoint{1.828819in}{0.605370in}}%
\pgfpathlineto{\pgfqpoint{1.828819in}{0.617592in}}%
\pgfpathlineto{\pgfqpoint{1.828819in}{0.631203in}}%
\pgfpathlineto{\pgfqpoint{1.828819in}{0.644814in}}%
\pgfpathlineto{\pgfqpoint{1.828819in}{0.658425in}}%
\pgfpathlineto{\pgfqpoint{1.828819in}{0.672036in}}%
\pgfpathlineto{\pgfqpoint{1.828819in}{0.685648in}}%
\pgfpathlineto{\pgfqpoint{1.828819in}{0.699259in}}%
\pgfpathlineto{\pgfqpoint{1.828819in}{0.712870in}}%
\pgfpathlineto{\pgfqpoint{1.828819in}{0.726481in}}%
\pgfpathlineto{\pgfqpoint{1.828819in}{0.740092in}}%
\pgfpathlineto{\pgfqpoint{1.828819in}{0.753703in}}%
\pgfpathlineto{\pgfqpoint{1.828819in}{0.767314in}}%
\pgfpathlineto{\pgfqpoint{1.828819in}{0.780925in}}%
\pgfpathlineto{\pgfqpoint{1.828819in}{0.788455in}}%
\pgfpathlineto{\pgfqpoint{1.813162in}{0.787764in}}%
\pgfpathlineto{\pgfqpoint{1.797505in}{0.785699in}}%
\pgfpathlineto{\pgfqpoint{1.781849in}{0.782281in}}%
\pgfpathlineto{\pgfqpoint{1.777377in}{0.780925in}}%
\pgfpathlineto{\pgfqpoint{1.766192in}{0.777651in}}%
\pgfpathlineto{\pgfqpoint{1.750536in}{0.771836in}}%
\pgfpathlineto{\pgfqpoint{1.740419in}{0.767314in}}%
\pgfpathlineto{\pgfqpoint{1.734879in}{0.764917in}}%
\pgfpathlineto{\pgfqpoint{1.719223in}{0.757047in}}%
\pgfpathlineto{\pgfqpoint{1.713343in}{0.753703in}}%
\pgfpathlineto{\pgfqpoint{1.703566in}{0.748305in}}%
\pgfpathlineto{\pgfqpoint{1.690129in}{0.740092in}}%
\pgfpathlineto{\pgfqpoint{1.687910in}{0.738770in}}%
\pgfpathlineto{\pgfqpoint{1.672253in}{0.728590in}}%
\pgfpathlineto{\pgfqpoint{1.669243in}{0.726481in}}%
\pgfpathlineto{\pgfqpoint{1.656596in}{0.717826in}}%
\pgfpathlineto{\pgfqpoint{1.649793in}{0.712870in}}%
\pgfpathlineto{\pgfqpoint{1.640940in}{0.706548in}}%
\pgfpathlineto{\pgfqpoint{1.631259in}{0.699259in}}%
\pgfpathlineto{\pgfqpoint{1.625283in}{0.694834in}}%
\pgfpathlineto{\pgfqpoint{1.613424in}{0.685648in}}%
\pgfpathlineto{\pgfqpoint{1.609627in}{0.682744in}}%
\pgfpathlineto{\pgfqpoint{1.596137in}{0.672036in}}%
\pgfpathlineto{\pgfqpoint{1.593970in}{0.670333in}}%
\pgfpathlineto{\pgfqpoint{1.579290in}{0.658425in}}%
\pgfpathlineto{\pgfqpoint{1.578314in}{0.657638in}}%
\pgfpathlineto{\pgfqpoint{1.562804in}{0.644814in}}%
\pgfpathlineto{\pgfqpoint{1.562657in}{0.644693in}}%
\pgfpathlineto{\pgfqpoint{1.547000in}{0.631513in}}%
\pgfpathlineto{\pgfqpoint{1.546638in}{0.631203in}}%
\pgfpathlineto{\pgfqpoint{1.531344in}{0.618115in}}%
\pgfpathlineto{\pgfqpoint{1.530739in}{0.617592in}}%
\pgfpathlineto{\pgfqpoint{1.515687in}{0.604507in}}%
\pgfpathlineto{\pgfqpoint{1.515086in}{0.603981in}}%
\pgfpathlineto{\pgfqpoint{1.500031in}{0.590685in}}%
\pgfpathlineto{\pgfqpoint{1.499674in}{0.590370in}}%
\pgfpathlineto{\pgfqpoint{1.484514in}{0.576759in}}%
\pgfpathlineto{\pgfqpoint{1.484374in}{0.576631in}}%
\pgfpathlineto{\pgfqpoint{1.469623in}{0.563148in}}%
\pgfpathlineto{\pgfqpoint{1.468718in}{0.562299in}}%
\pgfpathlineto{\pgfqpoint{1.455021in}{0.549536in}}%
\pgfpathlineto{\pgfqpoint{1.453061in}{0.547652in}}%
\pgfpathlineto{\pgfqpoint{1.440744in}{0.535925in}}%
\pgfpathlineto{\pgfqpoint{1.437404in}{0.532624in}}%
\pgfpathlineto{\pgfqpoint{1.426838in}{0.522314in}}%
\pgfpathlineto{\pgfqpoint{1.421748in}{0.517120in}}%
\pgfpathlineto{\pgfqpoint{1.413363in}{0.508703in}}%
\pgfpathlineto{\pgfqpoint{1.406091in}{0.501007in}}%
\pgfpathlineto{\pgfqpoint{1.400390in}{0.495092in}}%
\pgfpathlineto{\pgfqpoint{1.390435in}{0.484098in}}%
\pgfpathlineto{\pgfqpoint{1.388009in}{0.481481in}}%
\pgfpathlineto{\pgfqpoint{1.376299in}{0.467870in}}%
\pgfpathlineto{\pgfqpoint{1.374778in}{0.465940in}}%
\pgfpathlineto{\pgfqpoint{1.365331in}{0.454259in}}%
\pgfpathlineto{\pgfqpoint{1.359122in}{0.445759in}}%
\pgfpathlineto{\pgfqpoint{1.355275in}{0.440648in}}%
\pgfpathlineto{\pgfqpoint{1.346223in}{0.427036in}}%
\pgfpathlineto{\pgfqpoint{1.343465in}{0.422220in}}%
\pgfpathlineto{\pgfqpoint{1.338264in}{0.413425in}}%
\pgfpathlineto{\pgfqpoint{1.331575in}{0.399814in}}%
\pgfpathlineto{\pgfqpoint{1.327809in}{0.390091in}}%
\pgfpathlineto{\pgfqpoint{1.326249in}{0.386203in}}%
\pgfpathlineto{\pgfqpoint{1.322318in}{0.372592in}}%
\pgfpathlineto{\pgfqpoint{1.319942in}{0.358981in}}%
\pgfpathlineto{\pgfqpoint{1.319147in}{0.345370in}}%
\pgfpathlineto{\pgfqpoint{1.327809in}{0.345370in}}%
\pgfpathclose%
\pgfpathmoveto{\pgfqpoint{0.294475in}{1.250475in}}%
\pgfpathlineto{\pgfqpoint{0.310132in}{1.252541in}}%
\pgfpathlineto{\pgfqpoint{0.325788in}{1.255958in}}%
\pgfpathlineto{\pgfqpoint{0.330260in}{1.257314in}}%
\pgfpathlineto{\pgfqpoint{0.341445in}{1.260588in}}%
\pgfpathlineto{\pgfqpoint{0.357101in}{1.266404in}}%
\pgfpathlineto{\pgfqpoint{0.367218in}{1.270925in}}%
\pgfpathlineto{\pgfqpoint{0.372758in}{1.273323in}}%
\pgfpathlineto{\pgfqpoint{0.388415in}{1.281192in}}%
\pgfpathlineto{\pgfqpoint{0.394294in}{1.284536in}}%
\pgfpathlineto{\pgfqpoint{0.404071in}{1.289935in}}%
\pgfpathlineto{\pgfqpoint{0.417508in}{1.298148in}}%
\pgfpathlineto{\pgfqpoint{0.419728in}{1.299469in}}%
\pgfpathlineto{\pgfqpoint{0.435384in}{1.309650in}}%
\pgfpathlineto{\pgfqpoint{0.438394in}{1.311759in}}%
\pgfpathlineto{\pgfqpoint{0.451041in}{1.320413in}}%
\pgfpathlineto{\pgfqpoint{0.457844in}{1.325370in}}%
\pgfpathlineto{\pgfqpoint{0.466697in}{1.331691in}}%
\pgfpathlineto{\pgfqpoint{0.476379in}{1.338981in}}%
\pgfpathlineto{\pgfqpoint{0.482354in}{1.343406in}}%
\pgfpathlineto{\pgfqpoint{0.494213in}{1.352592in}}%
\pgfpathlineto{\pgfqpoint{0.498011in}{1.355495in}}%
\pgfpathlineto{\pgfqpoint{0.511500in}{1.366203in}}%
\pgfpathlineto{\pgfqpoint{0.513667in}{1.367907in}}%
\pgfpathlineto{\pgfqpoint{0.528348in}{1.379814in}}%
\pgfpathlineto{\pgfqpoint{0.529324in}{1.380601in}}%
\pgfpathlineto{\pgfqpoint{0.544833in}{1.393425in}}%
\pgfpathlineto{\pgfqpoint{0.544980in}{1.393547in}}%
\pgfpathlineto{\pgfqpoint{0.560637in}{1.406727in}}%
\pgfpathlineto{\pgfqpoint{0.560999in}{1.407036in}}%
\pgfpathlineto{\pgfqpoint{0.576293in}{1.420124in}}%
\pgfpathlineto{\pgfqpoint{0.576899in}{1.420648in}}%
\pgfpathlineto{\pgfqpoint{0.591950in}{1.433733in}}%
\pgfpathlineto{\pgfqpoint{0.592552in}{1.434259in}}%
\pgfpathlineto{\pgfqpoint{0.607606in}{1.447555in}}%
\pgfpathlineto{\pgfqpoint{0.607963in}{1.447870in}}%
\pgfpathlineto{\pgfqpoint{0.623124in}{1.461481in}}%
\pgfpathlineto{\pgfqpoint{0.623263in}{1.461609in}}%
\pgfpathlineto{\pgfqpoint{0.638014in}{1.475092in}}%
\pgfpathlineto{\pgfqpoint{0.638920in}{1.475940in}}%
\pgfpathlineto{\pgfqpoint{0.652616in}{1.488703in}}%
\pgfpathlineto{\pgfqpoint{0.654576in}{1.490587in}}%
\pgfpathlineto{\pgfqpoint{0.666893in}{1.502314in}}%
\pgfpathlineto{\pgfqpoint{0.670233in}{1.505615in}}%
\pgfpathlineto{\pgfqpoint{0.680799in}{1.515925in}}%
\pgfpathlineto{\pgfqpoint{0.685889in}{1.521120in}}%
\pgfpathlineto{\pgfqpoint{0.694275in}{1.529536in}}%
\pgfpathlineto{\pgfqpoint{0.701546in}{1.537233in}}%
\pgfpathlineto{\pgfqpoint{0.707247in}{1.543148in}}%
\pgfpathlineto{\pgfqpoint{0.717202in}{1.554142in}}%
\pgfpathlineto{\pgfqpoint{0.719628in}{1.556759in}}%
\pgfpathlineto{\pgfqpoint{0.731339in}{1.570370in}}%
\pgfpathlineto{\pgfqpoint{0.732859in}{1.572300in}}%
\pgfpathlineto{\pgfqpoint{0.742306in}{1.583981in}}%
\pgfpathlineto{\pgfqpoint{0.748516in}{1.592480in}}%
\pgfpathlineto{\pgfqpoint{0.752362in}{1.597592in}}%
\pgfpathlineto{\pgfqpoint{0.761415in}{1.611203in}}%
\pgfpathlineto{\pgfqpoint{0.764172in}{1.616020in}}%
\pgfpathlineto{\pgfqpoint{0.769373in}{1.624814in}}%
\pgfpathlineto{\pgfqpoint{0.776063in}{1.638425in}}%
\pgfpathlineto{\pgfqpoint{0.779829in}{1.648149in}}%
\pgfpathlineto{\pgfqpoint{0.781388in}{1.652036in}}%
\pgfpathlineto{\pgfqpoint{0.785320in}{1.665648in}}%
\pgfpathlineto{\pgfqpoint{0.787696in}{1.679259in}}%
\pgfpathlineto{\pgfqpoint{0.788490in}{1.692870in}}%
\pgfpathlineto{\pgfqpoint{0.779829in}{1.692870in}}%
\pgfpathlineto{\pgfqpoint{0.764172in}{1.692870in}}%
\pgfpathlineto{\pgfqpoint{0.748516in}{1.692870in}}%
\pgfpathlineto{\pgfqpoint{0.732859in}{1.692870in}}%
\pgfpathlineto{\pgfqpoint{0.717202in}{1.692870in}}%
\pgfpathlineto{\pgfqpoint{0.701546in}{1.692870in}}%
\pgfpathlineto{\pgfqpoint{0.685889in}{1.692870in}}%
\pgfpathlineto{\pgfqpoint{0.670233in}{1.692870in}}%
\pgfpathlineto{\pgfqpoint{0.654576in}{1.692870in}}%
\pgfpathlineto{\pgfqpoint{0.638920in}{1.692870in}}%
\pgfpathlineto{\pgfqpoint{0.623263in}{1.692870in}}%
\pgfpathlineto{\pgfqpoint{0.607606in}{1.692870in}}%
\pgfpathlineto{\pgfqpoint{0.591950in}{1.692870in}}%
\pgfpathlineto{\pgfqpoint{0.577891in}{1.692870in}}%
\pgfpathlineto{\pgfqpoint{0.577271in}{1.679259in}}%
\pgfpathlineto{\pgfqpoint{0.576293in}{1.672094in}}%
\pgfpathlineto{\pgfqpoint{0.575394in}{1.665648in}}%
\pgfpathlineto{\pgfqpoint{0.572248in}{1.652036in}}%
\pgfpathlineto{\pgfqpoint{0.567891in}{1.638425in}}%
\pgfpathlineto{\pgfqpoint{0.562369in}{1.624814in}}%
\pgfpathlineto{\pgfqpoint{0.560637in}{1.621275in}}%
\pgfpathlineto{\pgfqpoint{0.555593in}{1.611203in}}%
\pgfpathlineto{\pgfqpoint{0.547693in}{1.597592in}}%
\pgfpathlineto{\pgfqpoint{0.544980in}{1.593464in}}%
\pgfpathlineto{\pgfqpoint{0.538583in}{1.583981in}}%
\pgfpathlineto{\pgfqpoint{0.529324in}{1.571571in}}%
\pgfpathlineto{\pgfqpoint{0.528400in}{1.570370in}}%
\pgfpathlineto{\pgfqpoint{0.516959in}{1.556759in}}%
\pgfpathlineto{\pgfqpoint{0.513667in}{1.553121in}}%
\pgfpathlineto{\pgfqpoint{0.504320in}{1.543148in}}%
\pgfpathlineto{\pgfqpoint{0.498011in}{1.536825in}}%
\pgfpathlineto{\pgfqpoint{0.490434in}{1.529536in}}%
\pgfpathlineto{\pgfqpoint{0.482354in}{1.522173in}}%
\pgfpathlineto{\pgfqpoint{0.475167in}{1.515925in}}%
\pgfpathlineto{\pgfqpoint{0.466697in}{1.508900in}}%
\pgfpathlineto{\pgfqpoint{0.458313in}{1.502314in}}%
\pgfpathlineto{\pgfqpoint{0.451041in}{1.496829in}}%
\pgfpathlineto{\pgfqpoint{0.439569in}{1.488703in}}%
\pgfpathlineto{\pgfqpoint{0.435384in}{1.485841in}}%
\pgfpathlineto{\pgfqpoint{0.419728in}{1.475895in}}%
\pgfpathlineto{\pgfqpoint{0.418346in}{1.475092in}}%
\pgfpathlineto{\pgfqpoint{0.404071in}{1.467043in}}%
\pgfpathlineto{\pgfqpoint{0.393163in}{1.461481in}}%
\pgfpathlineto{\pgfqpoint{0.388415in}{1.459123in}}%
\pgfpathlineto{\pgfqpoint{0.372758in}{1.452255in}}%
\pgfpathlineto{\pgfqpoint{0.361173in}{1.447870in}}%
\pgfpathlineto{\pgfqpoint{0.357101in}{1.446364in}}%
\pgfpathlineto{\pgfqpoint{0.341445in}{1.441563in}}%
\pgfpathlineto{\pgfqpoint{0.325788in}{1.437776in}}%
\pgfpathlineto{\pgfqpoint{0.310132in}{1.435041in}}%
\pgfpathlineto{\pgfqpoint{0.302717in}{1.434259in}}%
\pgfpathlineto{\pgfqpoint{0.294475in}{1.433409in}}%
\pgfpathlineto{\pgfqpoint{0.278819in}{1.432870in}}%
\pgfpathlineto{\pgfqpoint{0.278819in}{1.420648in}}%
\pgfpathlineto{\pgfqpoint{0.278819in}{1.407036in}}%
\pgfpathlineto{\pgfqpoint{0.278819in}{1.393425in}}%
\pgfpathlineto{\pgfqpoint{0.278819in}{1.379814in}}%
\pgfpathlineto{\pgfqpoint{0.278819in}{1.366203in}}%
\pgfpathlineto{\pgfqpoint{0.278819in}{1.352592in}}%
\pgfpathlineto{\pgfqpoint{0.278819in}{1.338981in}}%
\pgfpathlineto{\pgfqpoint{0.278819in}{1.325370in}}%
\pgfpathlineto{\pgfqpoint{0.278819in}{1.311759in}}%
\pgfpathlineto{\pgfqpoint{0.278819in}{1.298148in}}%
\pgfpathlineto{\pgfqpoint{0.278819in}{1.284536in}}%
\pgfpathlineto{\pgfqpoint{0.278819in}{1.270925in}}%
\pgfpathlineto{\pgfqpoint{0.278819in}{1.257314in}}%
\pgfpathlineto{\pgfqpoint{0.278819in}{1.249784in}}%
\pgfpathlineto{\pgfqpoint{0.294475in}{1.250475in}}%
\pgfpathclose%
\pgfpathmoveto{\pgfqpoint{1.781849in}{1.255958in}}%
\pgfpathlineto{\pgfqpoint{1.797505in}{1.252541in}}%
\pgfpathlineto{\pgfqpoint{1.813162in}{1.250475in}}%
\pgfpathlineto{\pgfqpoint{1.828819in}{1.249784in}}%
\pgfpathlineto{\pgfqpoint{1.828819in}{1.257314in}}%
\pgfpathlineto{\pgfqpoint{1.828819in}{1.270925in}}%
\pgfpathlineto{\pgfqpoint{1.828819in}{1.284536in}}%
\pgfpathlineto{\pgfqpoint{1.828819in}{1.298148in}}%
\pgfpathlineto{\pgfqpoint{1.828819in}{1.311759in}}%
\pgfpathlineto{\pgfqpoint{1.828819in}{1.325370in}}%
\pgfpathlineto{\pgfqpoint{1.828819in}{1.338981in}}%
\pgfpathlineto{\pgfqpoint{1.828819in}{1.352592in}}%
\pgfpathlineto{\pgfqpoint{1.828819in}{1.366203in}}%
\pgfpathlineto{\pgfqpoint{1.828819in}{1.379814in}}%
\pgfpathlineto{\pgfqpoint{1.828819in}{1.393425in}}%
\pgfpathlineto{\pgfqpoint{1.828819in}{1.407036in}}%
\pgfpathlineto{\pgfqpoint{1.828819in}{1.420648in}}%
\pgfpathlineto{\pgfqpoint{1.828819in}{1.432870in}}%
\pgfpathlineto{\pgfqpoint{1.813162in}{1.433409in}}%
\pgfpathlineto{\pgfqpoint{1.804921in}{1.434259in}}%
\pgfpathlineto{\pgfqpoint{1.797505in}{1.435041in}}%
\pgfpathlineto{\pgfqpoint{1.781849in}{1.437776in}}%
\pgfpathlineto{\pgfqpoint{1.766192in}{1.441563in}}%
\pgfpathlineto{\pgfqpoint{1.750536in}{1.446364in}}%
\pgfpathlineto{\pgfqpoint{1.746464in}{1.447870in}}%
\pgfpathlineto{\pgfqpoint{1.734879in}{1.452255in}}%
\pgfpathlineto{\pgfqpoint{1.719223in}{1.459123in}}%
\pgfpathlineto{\pgfqpoint{1.714474in}{1.461481in}}%
\pgfpathlineto{\pgfqpoint{1.703566in}{1.467043in}}%
\pgfpathlineto{\pgfqpoint{1.689291in}{1.475092in}}%
\pgfpathlineto{\pgfqpoint{1.687910in}{1.475895in}}%
\pgfpathlineto{\pgfqpoint{1.672253in}{1.485841in}}%
\pgfpathlineto{\pgfqpoint{1.668069in}{1.488703in}}%
\pgfpathlineto{\pgfqpoint{1.656596in}{1.496829in}}%
\pgfpathlineto{\pgfqpoint{1.649324in}{1.502314in}}%
\pgfpathlineto{\pgfqpoint{1.640940in}{1.508900in}}%
\pgfpathlineto{\pgfqpoint{1.632470in}{1.515925in}}%
\pgfpathlineto{\pgfqpoint{1.625283in}{1.522173in}}%
\pgfpathlineto{\pgfqpoint{1.617203in}{1.529536in}}%
\pgfpathlineto{\pgfqpoint{1.609627in}{1.536825in}}%
\pgfpathlineto{\pgfqpoint{1.603317in}{1.543148in}}%
\pgfpathlineto{\pgfqpoint{1.593970in}{1.553121in}}%
\pgfpathlineto{\pgfqpoint{1.590678in}{1.556759in}}%
\pgfpathlineto{\pgfqpoint{1.579237in}{1.570370in}}%
\pgfpathlineto{\pgfqpoint{1.578314in}{1.571571in}}%
\pgfpathlineto{\pgfqpoint{1.569055in}{1.583981in}}%
\pgfpathlineto{\pgfqpoint{1.562657in}{1.593464in}}%
\pgfpathlineto{\pgfqpoint{1.559944in}{1.597592in}}%
\pgfpathlineto{\pgfqpoint{1.552045in}{1.611203in}}%
\pgfpathlineto{\pgfqpoint{1.547000in}{1.621275in}}%
\pgfpathlineto{\pgfqpoint{1.545268in}{1.624814in}}%
\pgfpathlineto{\pgfqpoint{1.539746in}{1.638425in}}%
\pgfpathlineto{\pgfqpoint{1.535389in}{1.652036in}}%
\pgfpathlineto{\pgfqpoint{1.532243in}{1.665648in}}%
\pgfpathlineto{\pgfqpoint{1.531344in}{1.672094in}}%
\pgfpathlineto{\pgfqpoint{1.530367in}{1.679259in}}%
\pgfpathlineto{\pgfqpoint{1.529746in}{1.692870in}}%
\pgfpathlineto{\pgfqpoint{1.515687in}{1.692870in}}%
\pgfpathlineto{\pgfqpoint{1.500031in}{1.692870in}}%
\pgfpathlineto{\pgfqpoint{1.484374in}{1.692870in}}%
\pgfpathlineto{\pgfqpoint{1.468718in}{1.692870in}}%
\pgfpathlineto{\pgfqpoint{1.453061in}{1.692870in}}%
\pgfpathlineto{\pgfqpoint{1.437404in}{1.692870in}}%
\pgfpathlineto{\pgfqpoint{1.421748in}{1.692870in}}%
\pgfpathlineto{\pgfqpoint{1.406091in}{1.692870in}}%
\pgfpathlineto{\pgfqpoint{1.390435in}{1.692870in}}%
\pgfpathlineto{\pgfqpoint{1.374778in}{1.692870in}}%
\pgfpathlineto{\pgfqpoint{1.359122in}{1.692870in}}%
\pgfpathlineto{\pgfqpoint{1.343465in}{1.692870in}}%
\pgfpathlineto{\pgfqpoint{1.327809in}{1.692870in}}%
\pgfpathlineto{\pgfqpoint{1.319147in}{1.692870in}}%
\pgfpathlineto{\pgfqpoint{1.319942in}{1.679259in}}%
\pgfpathlineto{\pgfqpoint{1.322318in}{1.665648in}}%
\pgfpathlineto{\pgfqpoint{1.326249in}{1.652036in}}%
\pgfpathlineto{\pgfqpoint{1.327809in}{1.648149in}}%
\pgfpathlineto{\pgfqpoint{1.331575in}{1.638425in}}%
\pgfpathlineto{\pgfqpoint{1.338264in}{1.624814in}}%
\pgfpathlineto{\pgfqpoint{1.343465in}{1.616020in}}%
\pgfpathlineto{\pgfqpoint{1.346223in}{1.611203in}}%
\pgfpathlineto{\pgfqpoint{1.355275in}{1.597592in}}%
\pgfpathlineto{\pgfqpoint{1.359122in}{1.592480in}}%
\pgfpathlineto{\pgfqpoint{1.365331in}{1.583981in}}%
\pgfpathlineto{\pgfqpoint{1.374778in}{1.572300in}}%
\pgfpathlineto{\pgfqpoint{1.376299in}{1.570370in}}%
\pgfpathlineto{\pgfqpoint{1.388009in}{1.556759in}}%
\pgfpathlineto{\pgfqpoint{1.390435in}{1.554142in}}%
\pgfpathlineto{\pgfqpoint{1.400390in}{1.543148in}}%
\pgfpathlineto{\pgfqpoint{1.406091in}{1.537233in}}%
\pgfpathlineto{\pgfqpoint{1.413363in}{1.529536in}}%
\pgfpathlineto{\pgfqpoint{1.421748in}{1.521120in}}%
\pgfpathlineto{\pgfqpoint{1.426838in}{1.515925in}}%
\pgfpathlineto{\pgfqpoint{1.437404in}{1.505615in}}%
\pgfpathlineto{\pgfqpoint{1.440744in}{1.502314in}}%
\pgfpathlineto{\pgfqpoint{1.453061in}{1.490587in}}%
\pgfpathlineto{\pgfqpoint{1.455021in}{1.488703in}}%
\pgfpathlineto{\pgfqpoint{1.468718in}{1.475940in}}%
\pgfpathlineto{\pgfqpoint{1.469623in}{1.475092in}}%
\pgfpathlineto{\pgfqpoint{1.484374in}{1.461609in}}%
\pgfpathlineto{\pgfqpoint{1.484514in}{1.461481in}}%
\pgfpathlineto{\pgfqpoint{1.499674in}{1.447870in}}%
\pgfpathlineto{\pgfqpoint{1.500031in}{1.447555in}}%
\pgfpathlineto{\pgfqpoint{1.515086in}{1.434259in}}%
\pgfpathlineto{\pgfqpoint{1.515687in}{1.433733in}}%
\pgfpathlineto{\pgfqpoint{1.530739in}{1.420648in}}%
\pgfpathlineto{\pgfqpoint{1.531344in}{1.420124in}}%
\pgfpathlineto{\pgfqpoint{1.546638in}{1.407036in}}%
\pgfpathlineto{\pgfqpoint{1.547000in}{1.406727in}}%
\pgfpathlineto{\pgfqpoint{1.562657in}{1.393547in}}%
\pgfpathlineto{\pgfqpoint{1.562804in}{1.393425in}}%
\pgfpathlineto{\pgfqpoint{1.578314in}{1.380601in}}%
\pgfpathlineto{\pgfqpoint{1.579290in}{1.379814in}}%
\pgfpathlineto{\pgfqpoint{1.593970in}{1.367907in}}%
\pgfpathlineto{\pgfqpoint{1.596137in}{1.366203in}}%
\pgfpathlineto{\pgfqpoint{1.609627in}{1.355495in}}%
\pgfpathlineto{\pgfqpoint{1.613424in}{1.352592in}}%
\pgfpathlineto{\pgfqpoint{1.625283in}{1.343406in}}%
\pgfpathlineto{\pgfqpoint{1.631259in}{1.338981in}}%
\pgfpathlineto{\pgfqpoint{1.640940in}{1.331691in}}%
\pgfpathlineto{\pgfqpoint{1.649793in}{1.325370in}}%
\pgfpathlineto{\pgfqpoint{1.656596in}{1.320413in}}%
\pgfpathlineto{\pgfqpoint{1.669243in}{1.311759in}}%
\pgfpathlineto{\pgfqpoint{1.672253in}{1.309650in}}%
\pgfpathlineto{\pgfqpoint{1.687910in}{1.299469in}}%
\pgfpathlineto{\pgfqpoint{1.690129in}{1.298148in}}%
\pgfpathlineto{\pgfqpoint{1.703566in}{1.289935in}}%
\pgfpathlineto{\pgfqpoint{1.713343in}{1.284536in}}%
\pgfpathlineto{\pgfqpoint{1.719223in}{1.281192in}}%
\pgfpathlineto{\pgfqpoint{1.734879in}{1.273323in}}%
\pgfpathlineto{\pgfqpoint{1.740419in}{1.270925in}}%
\pgfpathlineto{\pgfqpoint{1.750536in}{1.266404in}}%
\pgfpathlineto{\pgfqpoint{1.766192in}{1.260588in}}%
\pgfpathlineto{\pgfqpoint{1.777377in}{1.257314in}}%
\pgfpathlineto{\pgfqpoint{1.781849in}{1.255958in}}%
\pgfpathclose%
\pgfusepath{fill}%
\end{pgfscope}%
\begin{pgfscope}%
\pgfpathrectangle{\pgfqpoint{0.278819in}{0.345370in}}{\pgfqpoint{1.550000in}{1.347500in}}%
\pgfusepath{clip}%
\pgfsetbuttcap%
\pgfsetroundjoin%
\definecolor{currentfill}{rgb}{0.048062,0.036607,0.150327}%
\pgfsetfillcolor{currentfill}%
\pgfsetlinewidth{0.000000pt}%
\definecolor{currentstroke}{rgb}{0.000000,0.000000,0.000000}%
\pgfsetstrokecolor{currentstroke}%
\pgfsetdash{}{0pt}%
\pgfpathmoveto{\pgfqpoint{0.294475in}{0.345370in}}%
\pgfpathlineto{\pgfqpoint{0.310132in}{0.345370in}}%
\pgfpathlineto{\pgfqpoint{0.325788in}{0.345370in}}%
\pgfpathlineto{\pgfqpoint{0.341445in}{0.345370in}}%
\pgfpathlineto{\pgfqpoint{0.357101in}{0.345370in}}%
\pgfpathlineto{\pgfqpoint{0.372758in}{0.345370in}}%
\pgfpathlineto{\pgfqpoint{0.388415in}{0.345370in}}%
\pgfpathlineto{\pgfqpoint{0.404071in}{0.345370in}}%
\pgfpathlineto{\pgfqpoint{0.419728in}{0.345370in}}%
\pgfpathlineto{\pgfqpoint{0.435384in}{0.345370in}}%
\pgfpathlineto{\pgfqpoint{0.451041in}{0.345370in}}%
\pgfpathlineto{\pgfqpoint{0.466697in}{0.345370in}}%
\pgfpathlineto{\pgfqpoint{0.482354in}{0.345370in}}%
\pgfpathlineto{\pgfqpoint{0.498011in}{0.345370in}}%
\pgfpathlineto{\pgfqpoint{0.513667in}{0.345370in}}%
\pgfpathlineto{\pgfqpoint{0.529324in}{0.345370in}}%
\pgfpathlineto{\pgfqpoint{0.544980in}{0.345370in}}%
\pgfpathlineto{\pgfqpoint{0.560637in}{0.345370in}}%
\pgfpathlineto{\pgfqpoint{0.576293in}{0.345370in}}%
\pgfpathlineto{\pgfqpoint{0.577891in}{0.345370in}}%
\pgfpathlineto{\pgfqpoint{0.577271in}{0.358981in}}%
\pgfpathlineto{\pgfqpoint{0.576293in}{0.366146in}}%
\pgfpathlineto{\pgfqpoint{0.575394in}{0.372592in}}%
\pgfpathlineto{\pgfqpoint{0.572248in}{0.386203in}}%
\pgfpathlineto{\pgfqpoint{0.567891in}{0.399814in}}%
\pgfpathlineto{\pgfqpoint{0.562369in}{0.413425in}}%
\pgfpathlineto{\pgfqpoint{0.560637in}{0.416965in}}%
\pgfpathlineto{\pgfqpoint{0.555593in}{0.427036in}}%
\pgfpathlineto{\pgfqpoint{0.547693in}{0.440648in}}%
\pgfpathlineto{\pgfqpoint{0.544980in}{0.444775in}}%
\pgfpathlineto{\pgfqpoint{0.538583in}{0.454259in}}%
\pgfpathlineto{\pgfqpoint{0.529324in}{0.466669in}}%
\pgfpathlineto{\pgfqpoint{0.528400in}{0.467870in}}%
\pgfpathlineto{\pgfqpoint{0.516959in}{0.481481in}}%
\pgfpathlineto{\pgfqpoint{0.513667in}{0.485118in}}%
\pgfpathlineto{\pgfqpoint{0.504320in}{0.495092in}}%
\pgfpathlineto{\pgfqpoint{0.498011in}{0.501414in}}%
\pgfpathlineto{\pgfqpoint{0.490434in}{0.508703in}}%
\pgfpathlineto{\pgfqpoint{0.482354in}{0.516066in}}%
\pgfpathlineto{\pgfqpoint{0.475167in}{0.522314in}}%
\pgfpathlineto{\pgfqpoint{0.466697in}{0.529339in}}%
\pgfpathlineto{\pgfqpoint{0.458313in}{0.535925in}}%
\pgfpathlineto{\pgfqpoint{0.451041in}{0.541411in}}%
\pgfpathlineto{\pgfqpoint{0.439569in}{0.549536in}}%
\pgfpathlineto{\pgfqpoint{0.435384in}{0.552398in}}%
\pgfpathlineto{\pgfqpoint{0.419728in}{0.562344in}}%
\pgfpathlineto{\pgfqpoint{0.418346in}{0.563148in}}%
\pgfpathlineto{\pgfqpoint{0.404071in}{0.571197in}}%
\pgfpathlineto{\pgfqpoint{0.393163in}{0.576759in}}%
\pgfpathlineto{\pgfqpoint{0.388415in}{0.579117in}}%
\pgfpathlineto{\pgfqpoint{0.372758in}{0.585985in}}%
\pgfpathlineto{\pgfqpoint{0.361173in}{0.590370in}}%
\pgfpathlineto{\pgfqpoint{0.357101in}{0.591875in}}%
\pgfpathlineto{\pgfqpoint{0.341445in}{0.596676in}}%
\pgfpathlineto{\pgfqpoint{0.325788in}{0.600464in}}%
\pgfpathlineto{\pgfqpoint{0.310132in}{0.603199in}}%
\pgfpathlineto{\pgfqpoint{0.302717in}{0.603981in}}%
\pgfpathlineto{\pgfqpoint{0.294475in}{0.604830in}}%
\pgfpathlineto{\pgfqpoint{0.278819in}{0.605370in}}%
\pgfpathlineto{\pgfqpoint{0.278819in}{0.603981in}}%
\pgfpathlineto{\pgfqpoint{0.278819in}{0.590370in}}%
\pgfpathlineto{\pgfqpoint{0.278819in}{0.576759in}}%
\pgfpathlineto{\pgfqpoint{0.278819in}{0.563148in}}%
\pgfpathlineto{\pgfqpoint{0.278819in}{0.549536in}}%
\pgfpathlineto{\pgfqpoint{0.278819in}{0.535925in}}%
\pgfpathlineto{\pgfqpoint{0.278819in}{0.522314in}}%
\pgfpathlineto{\pgfqpoint{0.278819in}{0.508703in}}%
\pgfpathlineto{\pgfqpoint{0.278819in}{0.495092in}}%
\pgfpathlineto{\pgfqpoint{0.278819in}{0.481481in}}%
\pgfpathlineto{\pgfqpoint{0.278819in}{0.467870in}}%
\pgfpathlineto{\pgfqpoint{0.278819in}{0.454259in}}%
\pgfpathlineto{\pgfqpoint{0.278819in}{0.440648in}}%
\pgfpathlineto{\pgfqpoint{0.278819in}{0.427036in}}%
\pgfpathlineto{\pgfqpoint{0.278819in}{0.413425in}}%
\pgfpathlineto{\pgfqpoint{0.278819in}{0.399814in}}%
\pgfpathlineto{\pgfqpoint{0.278819in}{0.386203in}}%
\pgfpathlineto{\pgfqpoint{0.278819in}{0.372592in}}%
\pgfpathlineto{\pgfqpoint{0.278819in}{0.358981in}}%
\pgfpathlineto{\pgfqpoint{0.278819in}{0.345370in}}%
\pgfpathlineto{\pgfqpoint{0.294475in}{0.345370in}}%
\pgfpathclose%
\pgfpathmoveto{\pgfqpoint{1.531344in}{0.345370in}}%
\pgfpathlineto{\pgfqpoint{1.547000in}{0.345370in}}%
\pgfpathlineto{\pgfqpoint{1.562657in}{0.345370in}}%
\pgfpathlineto{\pgfqpoint{1.578314in}{0.345370in}}%
\pgfpathlineto{\pgfqpoint{1.593970in}{0.345370in}}%
\pgfpathlineto{\pgfqpoint{1.609627in}{0.345370in}}%
\pgfpathlineto{\pgfqpoint{1.625283in}{0.345370in}}%
\pgfpathlineto{\pgfqpoint{1.640940in}{0.345370in}}%
\pgfpathlineto{\pgfqpoint{1.656596in}{0.345370in}}%
\pgfpathlineto{\pgfqpoint{1.672253in}{0.345370in}}%
\pgfpathlineto{\pgfqpoint{1.687910in}{0.345370in}}%
\pgfpathlineto{\pgfqpoint{1.703566in}{0.345370in}}%
\pgfpathlineto{\pgfqpoint{1.719223in}{0.345370in}}%
\pgfpathlineto{\pgfqpoint{1.734879in}{0.345370in}}%
\pgfpathlineto{\pgfqpoint{1.750536in}{0.345370in}}%
\pgfpathlineto{\pgfqpoint{1.766192in}{0.345370in}}%
\pgfpathlineto{\pgfqpoint{1.781849in}{0.345370in}}%
\pgfpathlineto{\pgfqpoint{1.797505in}{0.345370in}}%
\pgfpathlineto{\pgfqpoint{1.813162in}{0.345370in}}%
\pgfpathlineto{\pgfqpoint{1.828819in}{0.345370in}}%
\pgfpathlineto{\pgfqpoint{1.828819in}{0.358981in}}%
\pgfpathlineto{\pgfqpoint{1.828819in}{0.372592in}}%
\pgfpathlineto{\pgfqpoint{1.828819in}{0.386203in}}%
\pgfpathlineto{\pgfqpoint{1.828819in}{0.399814in}}%
\pgfpathlineto{\pgfqpoint{1.828819in}{0.413425in}}%
\pgfpathlineto{\pgfqpoint{1.828819in}{0.427036in}}%
\pgfpathlineto{\pgfqpoint{1.828819in}{0.440648in}}%
\pgfpathlineto{\pgfqpoint{1.828819in}{0.454259in}}%
\pgfpathlineto{\pgfqpoint{1.828819in}{0.467870in}}%
\pgfpathlineto{\pgfqpoint{1.828819in}{0.481481in}}%
\pgfpathlineto{\pgfqpoint{1.828819in}{0.495092in}}%
\pgfpathlineto{\pgfqpoint{1.828819in}{0.508703in}}%
\pgfpathlineto{\pgfqpoint{1.828819in}{0.522314in}}%
\pgfpathlineto{\pgfqpoint{1.828819in}{0.535925in}}%
\pgfpathlineto{\pgfqpoint{1.828819in}{0.549536in}}%
\pgfpathlineto{\pgfqpoint{1.828819in}{0.563148in}}%
\pgfpathlineto{\pgfqpoint{1.828819in}{0.576759in}}%
\pgfpathlineto{\pgfqpoint{1.828819in}{0.590370in}}%
\pgfpathlineto{\pgfqpoint{1.828819in}{0.603981in}}%
\pgfpathlineto{\pgfqpoint{1.828819in}{0.605370in}}%
\pgfpathlineto{\pgfqpoint{1.813162in}{0.604830in}}%
\pgfpathlineto{\pgfqpoint{1.804921in}{0.603981in}}%
\pgfpathlineto{\pgfqpoint{1.797505in}{0.603199in}}%
\pgfpathlineto{\pgfqpoint{1.781849in}{0.600464in}}%
\pgfpathlineto{\pgfqpoint{1.766192in}{0.596676in}}%
\pgfpathlineto{\pgfqpoint{1.750536in}{0.591875in}}%
\pgfpathlineto{\pgfqpoint{1.746464in}{0.590370in}}%
\pgfpathlineto{\pgfqpoint{1.734879in}{0.585985in}}%
\pgfpathlineto{\pgfqpoint{1.719223in}{0.579117in}}%
\pgfpathlineto{\pgfqpoint{1.714474in}{0.576759in}}%
\pgfpathlineto{\pgfqpoint{1.703566in}{0.571197in}}%
\pgfpathlineto{\pgfqpoint{1.689291in}{0.563148in}}%
\pgfpathlineto{\pgfqpoint{1.687910in}{0.562344in}}%
\pgfpathlineto{\pgfqpoint{1.672253in}{0.552398in}}%
\pgfpathlineto{\pgfqpoint{1.668069in}{0.549536in}}%
\pgfpathlineto{\pgfqpoint{1.656596in}{0.541411in}}%
\pgfpathlineto{\pgfqpoint{1.649324in}{0.535925in}}%
\pgfpathlineto{\pgfqpoint{1.640940in}{0.529339in}}%
\pgfpathlineto{\pgfqpoint{1.632470in}{0.522314in}}%
\pgfpathlineto{\pgfqpoint{1.625283in}{0.516066in}}%
\pgfpathlineto{\pgfqpoint{1.617203in}{0.508703in}}%
\pgfpathlineto{\pgfqpoint{1.609627in}{0.501414in}}%
\pgfpathlineto{\pgfqpoint{1.603317in}{0.495092in}}%
\pgfpathlineto{\pgfqpoint{1.593970in}{0.485118in}}%
\pgfpathlineto{\pgfqpoint{1.590678in}{0.481481in}}%
\pgfpathlineto{\pgfqpoint{1.579237in}{0.467870in}}%
\pgfpathlineto{\pgfqpoint{1.578314in}{0.466669in}}%
\pgfpathlineto{\pgfqpoint{1.569055in}{0.454259in}}%
\pgfpathlineto{\pgfqpoint{1.562657in}{0.444775in}}%
\pgfpathlineto{\pgfqpoint{1.559944in}{0.440648in}}%
\pgfpathlineto{\pgfqpoint{1.552045in}{0.427036in}}%
\pgfpathlineto{\pgfqpoint{1.547000in}{0.416965in}}%
\pgfpathlineto{\pgfqpoint{1.545268in}{0.413425in}}%
\pgfpathlineto{\pgfqpoint{1.539746in}{0.399814in}}%
\pgfpathlineto{\pgfqpoint{1.535389in}{0.386203in}}%
\pgfpathlineto{\pgfqpoint{1.532243in}{0.372592in}}%
\pgfpathlineto{\pgfqpoint{1.531344in}{0.366146in}}%
\pgfpathlineto{\pgfqpoint{1.530367in}{0.358981in}}%
\pgfpathlineto{\pgfqpoint{1.529746in}{0.345370in}}%
\pgfpathlineto{\pgfqpoint{1.531344in}{0.345370in}}%
\pgfpathclose%
\pgfpathmoveto{\pgfqpoint{0.294475in}{1.433409in}}%
\pgfpathlineto{\pgfqpoint{0.302717in}{1.434259in}}%
\pgfpathlineto{\pgfqpoint{0.310132in}{1.435041in}}%
\pgfpathlineto{\pgfqpoint{0.325788in}{1.437776in}}%
\pgfpathlineto{\pgfqpoint{0.341445in}{1.441563in}}%
\pgfpathlineto{\pgfqpoint{0.357101in}{1.446364in}}%
\pgfpathlineto{\pgfqpoint{0.361173in}{1.447870in}}%
\pgfpathlineto{\pgfqpoint{0.372758in}{1.452255in}}%
\pgfpathlineto{\pgfqpoint{0.388415in}{1.459123in}}%
\pgfpathlineto{\pgfqpoint{0.393163in}{1.461481in}}%
\pgfpathlineto{\pgfqpoint{0.404071in}{1.467043in}}%
\pgfpathlineto{\pgfqpoint{0.418346in}{1.475092in}}%
\pgfpathlineto{\pgfqpoint{0.419728in}{1.475895in}}%
\pgfpathlineto{\pgfqpoint{0.435384in}{1.485841in}}%
\pgfpathlineto{\pgfqpoint{0.439569in}{1.488703in}}%
\pgfpathlineto{\pgfqpoint{0.451041in}{1.496829in}}%
\pgfpathlineto{\pgfqpoint{0.458313in}{1.502314in}}%
\pgfpathlineto{\pgfqpoint{0.466697in}{1.508900in}}%
\pgfpathlineto{\pgfqpoint{0.475167in}{1.515925in}}%
\pgfpathlineto{\pgfqpoint{0.482354in}{1.522173in}}%
\pgfpathlineto{\pgfqpoint{0.490434in}{1.529536in}}%
\pgfpathlineto{\pgfqpoint{0.498011in}{1.536825in}}%
\pgfpathlineto{\pgfqpoint{0.504320in}{1.543148in}}%
\pgfpathlineto{\pgfqpoint{0.513667in}{1.553121in}}%
\pgfpathlineto{\pgfqpoint{0.516959in}{1.556759in}}%
\pgfpathlineto{\pgfqpoint{0.528400in}{1.570370in}}%
\pgfpathlineto{\pgfqpoint{0.529324in}{1.571571in}}%
\pgfpathlineto{\pgfqpoint{0.538583in}{1.583981in}}%
\pgfpathlineto{\pgfqpoint{0.544980in}{1.593464in}}%
\pgfpathlineto{\pgfqpoint{0.547693in}{1.597592in}}%
\pgfpathlineto{\pgfqpoint{0.555593in}{1.611203in}}%
\pgfpathlineto{\pgfqpoint{0.560637in}{1.621275in}}%
\pgfpathlineto{\pgfqpoint{0.562369in}{1.624814in}}%
\pgfpathlineto{\pgfqpoint{0.567891in}{1.638425in}}%
\pgfpathlineto{\pgfqpoint{0.572248in}{1.652036in}}%
\pgfpathlineto{\pgfqpoint{0.575394in}{1.665648in}}%
\pgfpathlineto{\pgfqpoint{0.576293in}{1.672094in}}%
\pgfpathlineto{\pgfqpoint{0.577271in}{1.679259in}}%
\pgfpathlineto{\pgfqpoint{0.577891in}{1.692870in}}%
\pgfpathlineto{\pgfqpoint{0.576293in}{1.692870in}}%
\pgfpathlineto{\pgfqpoint{0.560637in}{1.692870in}}%
\pgfpathlineto{\pgfqpoint{0.544980in}{1.692870in}}%
\pgfpathlineto{\pgfqpoint{0.529324in}{1.692870in}}%
\pgfpathlineto{\pgfqpoint{0.513667in}{1.692870in}}%
\pgfpathlineto{\pgfqpoint{0.498011in}{1.692870in}}%
\pgfpathlineto{\pgfqpoint{0.482354in}{1.692870in}}%
\pgfpathlineto{\pgfqpoint{0.466697in}{1.692870in}}%
\pgfpathlineto{\pgfqpoint{0.451041in}{1.692870in}}%
\pgfpathlineto{\pgfqpoint{0.435384in}{1.692870in}}%
\pgfpathlineto{\pgfqpoint{0.419728in}{1.692870in}}%
\pgfpathlineto{\pgfqpoint{0.404071in}{1.692870in}}%
\pgfpathlineto{\pgfqpoint{0.388415in}{1.692870in}}%
\pgfpathlineto{\pgfqpoint{0.372758in}{1.692870in}}%
\pgfpathlineto{\pgfqpoint{0.357101in}{1.692870in}}%
\pgfpathlineto{\pgfqpoint{0.341445in}{1.692870in}}%
\pgfpathlineto{\pgfqpoint{0.325788in}{1.692870in}}%
\pgfpathlineto{\pgfqpoint{0.310132in}{1.692870in}}%
\pgfpathlineto{\pgfqpoint{0.294475in}{1.692870in}}%
\pgfpathlineto{\pgfqpoint{0.278819in}{1.692870in}}%
\pgfpathlineto{\pgfqpoint{0.278819in}{1.679259in}}%
\pgfpathlineto{\pgfqpoint{0.278819in}{1.665648in}}%
\pgfpathlineto{\pgfqpoint{0.278819in}{1.652036in}}%
\pgfpathlineto{\pgfqpoint{0.278819in}{1.638425in}}%
\pgfpathlineto{\pgfqpoint{0.278819in}{1.624814in}}%
\pgfpathlineto{\pgfqpoint{0.278819in}{1.611203in}}%
\pgfpathlineto{\pgfqpoint{0.278819in}{1.597592in}}%
\pgfpathlineto{\pgfqpoint{0.278819in}{1.583981in}}%
\pgfpathlineto{\pgfqpoint{0.278819in}{1.570370in}}%
\pgfpathlineto{\pgfqpoint{0.278819in}{1.556759in}}%
\pgfpathlineto{\pgfqpoint{0.278819in}{1.543148in}}%
\pgfpathlineto{\pgfqpoint{0.278819in}{1.529536in}}%
\pgfpathlineto{\pgfqpoint{0.278819in}{1.515925in}}%
\pgfpathlineto{\pgfqpoint{0.278819in}{1.502314in}}%
\pgfpathlineto{\pgfqpoint{0.278819in}{1.488703in}}%
\pgfpathlineto{\pgfqpoint{0.278819in}{1.475092in}}%
\pgfpathlineto{\pgfqpoint{0.278819in}{1.461481in}}%
\pgfpathlineto{\pgfqpoint{0.278819in}{1.447870in}}%
\pgfpathlineto{\pgfqpoint{0.278819in}{1.434259in}}%
\pgfpathlineto{\pgfqpoint{0.278819in}{1.432870in}}%
\pgfpathlineto{\pgfqpoint{0.294475in}{1.433409in}}%
\pgfpathclose%
\pgfpathmoveto{\pgfqpoint{1.813162in}{1.433409in}}%
\pgfpathlineto{\pgfqpoint{1.828819in}{1.432870in}}%
\pgfpathlineto{\pgfqpoint{1.828819in}{1.434259in}}%
\pgfpathlineto{\pgfqpoint{1.828819in}{1.447870in}}%
\pgfpathlineto{\pgfqpoint{1.828819in}{1.461481in}}%
\pgfpathlineto{\pgfqpoint{1.828819in}{1.475092in}}%
\pgfpathlineto{\pgfqpoint{1.828819in}{1.488703in}}%
\pgfpathlineto{\pgfqpoint{1.828819in}{1.502314in}}%
\pgfpathlineto{\pgfqpoint{1.828819in}{1.515925in}}%
\pgfpathlineto{\pgfqpoint{1.828819in}{1.529536in}}%
\pgfpathlineto{\pgfqpoint{1.828819in}{1.543148in}}%
\pgfpathlineto{\pgfqpoint{1.828819in}{1.556759in}}%
\pgfpathlineto{\pgfqpoint{1.828819in}{1.570370in}}%
\pgfpathlineto{\pgfqpoint{1.828819in}{1.583981in}}%
\pgfpathlineto{\pgfqpoint{1.828819in}{1.597592in}}%
\pgfpathlineto{\pgfqpoint{1.828819in}{1.611203in}}%
\pgfpathlineto{\pgfqpoint{1.828819in}{1.624814in}}%
\pgfpathlineto{\pgfqpoint{1.828819in}{1.638425in}}%
\pgfpathlineto{\pgfqpoint{1.828819in}{1.652036in}}%
\pgfpathlineto{\pgfqpoint{1.828819in}{1.665648in}}%
\pgfpathlineto{\pgfqpoint{1.828819in}{1.679259in}}%
\pgfpathlineto{\pgfqpoint{1.828819in}{1.692870in}}%
\pgfpathlineto{\pgfqpoint{1.813162in}{1.692870in}}%
\pgfpathlineto{\pgfqpoint{1.797505in}{1.692870in}}%
\pgfpathlineto{\pgfqpoint{1.781849in}{1.692870in}}%
\pgfpathlineto{\pgfqpoint{1.766192in}{1.692870in}}%
\pgfpathlineto{\pgfqpoint{1.750536in}{1.692870in}}%
\pgfpathlineto{\pgfqpoint{1.734879in}{1.692870in}}%
\pgfpathlineto{\pgfqpoint{1.719223in}{1.692870in}}%
\pgfpathlineto{\pgfqpoint{1.703566in}{1.692870in}}%
\pgfpathlineto{\pgfqpoint{1.687910in}{1.692870in}}%
\pgfpathlineto{\pgfqpoint{1.672253in}{1.692870in}}%
\pgfpathlineto{\pgfqpoint{1.656596in}{1.692870in}}%
\pgfpathlineto{\pgfqpoint{1.640940in}{1.692870in}}%
\pgfpathlineto{\pgfqpoint{1.625283in}{1.692870in}}%
\pgfpathlineto{\pgfqpoint{1.609627in}{1.692870in}}%
\pgfpathlineto{\pgfqpoint{1.593970in}{1.692870in}}%
\pgfpathlineto{\pgfqpoint{1.578314in}{1.692870in}}%
\pgfpathlineto{\pgfqpoint{1.562657in}{1.692870in}}%
\pgfpathlineto{\pgfqpoint{1.547000in}{1.692870in}}%
\pgfpathlineto{\pgfqpoint{1.531344in}{1.692870in}}%
\pgfpathlineto{\pgfqpoint{1.529746in}{1.692870in}}%
\pgfpathlineto{\pgfqpoint{1.530367in}{1.679259in}}%
\pgfpathlineto{\pgfqpoint{1.531344in}{1.672094in}}%
\pgfpathlineto{\pgfqpoint{1.532243in}{1.665648in}}%
\pgfpathlineto{\pgfqpoint{1.535389in}{1.652036in}}%
\pgfpathlineto{\pgfqpoint{1.539746in}{1.638425in}}%
\pgfpathlineto{\pgfqpoint{1.545268in}{1.624814in}}%
\pgfpathlineto{\pgfqpoint{1.547000in}{1.621275in}}%
\pgfpathlineto{\pgfqpoint{1.552045in}{1.611203in}}%
\pgfpathlineto{\pgfqpoint{1.559944in}{1.597592in}}%
\pgfpathlineto{\pgfqpoint{1.562657in}{1.593464in}}%
\pgfpathlineto{\pgfqpoint{1.569055in}{1.583981in}}%
\pgfpathlineto{\pgfqpoint{1.578314in}{1.571571in}}%
\pgfpathlineto{\pgfqpoint{1.579237in}{1.570370in}}%
\pgfpathlineto{\pgfqpoint{1.590678in}{1.556759in}}%
\pgfpathlineto{\pgfqpoint{1.593970in}{1.553121in}}%
\pgfpathlineto{\pgfqpoint{1.603317in}{1.543148in}}%
\pgfpathlineto{\pgfqpoint{1.609627in}{1.536825in}}%
\pgfpathlineto{\pgfqpoint{1.617203in}{1.529536in}}%
\pgfpathlineto{\pgfqpoint{1.625283in}{1.522173in}}%
\pgfpathlineto{\pgfqpoint{1.632470in}{1.515925in}}%
\pgfpathlineto{\pgfqpoint{1.640940in}{1.508900in}}%
\pgfpathlineto{\pgfqpoint{1.649324in}{1.502314in}}%
\pgfpathlineto{\pgfqpoint{1.656596in}{1.496829in}}%
\pgfpathlineto{\pgfqpoint{1.668069in}{1.488703in}}%
\pgfpathlineto{\pgfqpoint{1.672253in}{1.485841in}}%
\pgfpathlineto{\pgfqpoint{1.687910in}{1.475895in}}%
\pgfpathlineto{\pgfqpoint{1.689291in}{1.475092in}}%
\pgfpathlineto{\pgfqpoint{1.703566in}{1.467043in}}%
\pgfpathlineto{\pgfqpoint{1.714474in}{1.461481in}}%
\pgfpathlineto{\pgfqpoint{1.719223in}{1.459123in}}%
\pgfpathlineto{\pgfqpoint{1.734879in}{1.452255in}}%
\pgfpathlineto{\pgfqpoint{1.746464in}{1.447870in}}%
\pgfpathlineto{\pgfqpoint{1.750536in}{1.446364in}}%
\pgfpathlineto{\pgfqpoint{1.766192in}{1.441563in}}%
\pgfpathlineto{\pgfqpoint{1.781849in}{1.437776in}}%
\pgfpathlineto{\pgfqpoint{1.797505in}{1.435041in}}%
\pgfpathlineto{\pgfqpoint{1.804921in}{1.434259in}}%
\pgfpathlineto{\pgfqpoint{1.813162in}{1.433409in}}%
\pgfpathclose%
\pgfusepath{fill}%
\end{pgfscope}%
\begin{pgfscope}%
\pgfsetbuttcap%
\pgfsetroundjoin%
\definecolor{currentfill}{rgb}{0.000000,0.000000,0.000000}%
\pgfsetfillcolor{currentfill}%
\pgfsetlinewidth{0.803000pt}%
\definecolor{currentstroke}{rgb}{0.000000,0.000000,0.000000}%
\pgfsetstrokecolor{currentstroke}%
\pgfsetdash{}{0pt}%
\pgfsys@defobject{currentmarker}{\pgfqpoint{0.000000in}{-0.048611in}}{\pgfqpoint{0.000000in}{0.000000in}}{%
\pgfpathmoveto{\pgfqpoint{0.000000in}{0.000000in}}%
\pgfpathlineto{\pgfqpoint{0.000000in}{-0.048611in}}%
\pgfusepath{stroke,fill}%
}%
\begin{pgfscope}%
\pgfsys@transformshift{0.278819in}{0.345370in}%
\pgfsys@useobject{currentmarker}{}%
\end{pgfscope}%
\end{pgfscope}%
\begin{pgfscope}%
\definecolor{textcolor}{rgb}{0.000000,0.000000,0.000000}%
\pgfsetstrokecolor{textcolor}%
\pgfsetfillcolor{textcolor}%
\pgftext[x=0.278819in,y=0.248148in,,top]{\color{textcolor}{\rmfamily\fontsize{12.000000}{14.400000}\selectfont\catcode`\^=\active\def^{\ifmmode\sp\else\^{}\fi}\catcode`\%=\active\def%{\%}$\mathdefault{0}$}}%
\end{pgfscope}%
\begin{pgfscope}%
\pgfsetbuttcap%
\pgfsetroundjoin%
\definecolor{currentfill}{rgb}{0.000000,0.000000,0.000000}%
\pgfsetfillcolor{currentfill}%
\pgfsetlinewidth{0.803000pt}%
\definecolor{currentstroke}{rgb}{0.000000,0.000000,0.000000}%
\pgfsetstrokecolor{currentstroke}%
\pgfsetdash{}{0pt}%
\pgfsys@defobject{currentmarker}{\pgfqpoint{0.000000in}{-0.048611in}}{\pgfqpoint{0.000000in}{0.000000in}}{%
\pgfpathmoveto{\pgfqpoint{0.000000in}{0.000000in}}%
\pgfpathlineto{\pgfqpoint{0.000000in}{-0.048611in}}%
\pgfusepath{stroke,fill}%
}%
\begin{pgfscope}%
\pgfsys@transformshift{0.795485in}{0.345370in}%
\pgfsys@useobject{currentmarker}{}%
\end{pgfscope}%
\end{pgfscope}%
\begin{pgfscope}%
\definecolor{textcolor}{rgb}{0.000000,0.000000,0.000000}%
\pgfsetstrokecolor{textcolor}%
\pgfsetfillcolor{textcolor}%
\pgftext[x=0.795485in,y=0.248148in,,top]{\color{textcolor}{\rmfamily\fontsize{12.000000}{14.400000}\selectfont\catcode`\^=\active\def^{\ifmmode\sp\else\^{}\fi}\catcode`\%=\active\def%{\%}$\mathdefault{2}$}}%
\end{pgfscope}%
\begin{pgfscope}%
\pgfsetbuttcap%
\pgfsetroundjoin%
\definecolor{currentfill}{rgb}{0.000000,0.000000,0.000000}%
\pgfsetfillcolor{currentfill}%
\pgfsetlinewidth{0.803000pt}%
\definecolor{currentstroke}{rgb}{0.000000,0.000000,0.000000}%
\pgfsetstrokecolor{currentstroke}%
\pgfsetdash{}{0pt}%
\pgfsys@defobject{currentmarker}{\pgfqpoint{0.000000in}{-0.048611in}}{\pgfqpoint{0.000000in}{0.000000in}}{%
\pgfpathmoveto{\pgfqpoint{0.000000in}{0.000000in}}%
\pgfpathlineto{\pgfqpoint{0.000000in}{-0.048611in}}%
\pgfusepath{stroke,fill}%
}%
\begin{pgfscope}%
\pgfsys@transformshift{1.312152in}{0.345370in}%
\pgfsys@useobject{currentmarker}{}%
\end{pgfscope}%
\end{pgfscope}%
\begin{pgfscope}%
\definecolor{textcolor}{rgb}{0.000000,0.000000,0.000000}%
\pgfsetstrokecolor{textcolor}%
\pgfsetfillcolor{textcolor}%
\pgftext[x=1.312152in,y=0.248148in,,top]{\color{textcolor}{\rmfamily\fontsize{12.000000}{14.400000}\selectfont\catcode`\^=\active\def^{\ifmmode\sp\else\^{}\fi}\catcode`\%=\active\def%{\%}$\mathdefault{4}$}}%
\end{pgfscope}%
\begin{pgfscope}%
\pgfsetbuttcap%
\pgfsetroundjoin%
\definecolor{currentfill}{rgb}{0.000000,0.000000,0.000000}%
\pgfsetfillcolor{currentfill}%
\pgfsetlinewidth{0.803000pt}%
\definecolor{currentstroke}{rgb}{0.000000,0.000000,0.000000}%
\pgfsetstrokecolor{currentstroke}%
\pgfsetdash{}{0pt}%
\pgfsys@defobject{currentmarker}{\pgfqpoint{0.000000in}{-0.048611in}}{\pgfqpoint{0.000000in}{0.000000in}}{%
\pgfpathmoveto{\pgfqpoint{0.000000in}{0.000000in}}%
\pgfpathlineto{\pgfqpoint{0.000000in}{-0.048611in}}%
\pgfusepath{stroke,fill}%
}%
\begin{pgfscope}%
\pgfsys@transformshift{1.828819in}{0.345370in}%
\pgfsys@useobject{currentmarker}{}%
\end{pgfscope}%
\end{pgfscope}%
\begin{pgfscope}%
\definecolor{textcolor}{rgb}{0.000000,0.000000,0.000000}%
\pgfsetstrokecolor{textcolor}%
\pgfsetfillcolor{textcolor}%
\pgftext[x=1.828819in,y=0.248148in,,top]{\color{textcolor}{\rmfamily\fontsize{12.000000}{14.400000}\selectfont\catcode`\^=\active\def^{\ifmmode\sp\else\^{}\fi}\catcode`\%=\active\def%{\%}$\mathdefault{6}$}}%
\end{pgfscope}%
\begin{pgfscope}%
\pgfsetbuttcap%
\pgfsetroundjoin%
\definecolor{currentfill}{rgb}{0.000000,0.000000,0.000000}%
\pgfsetfillcolor{currentfill}%
\pgfsetlinewidth{0.803000pt}%
\definecolor{currentstroke}{rgb}{0.000000,0.000000,0.000000}%
\pgfsetstrokecolor{currentstroke}%
\pgfsetdash{}{0pt}%
\pgfsys@defobject{currentmarker}{\pgfqpoint{-0.048611in}{0.000000in}}{\pgfqpoint{-0.000000in}{0.000000in}}{%
\pgfpathmoveto{\pgfqpoint{-0.000000in}{0.000000in}}%
\pgfpathlineto{\pgfqpoint{-0.048611in}{0.000000in}}%
\pgfusepath{stroke,fill}%
}%
\begin{pgfscope}%
\pgfsys@transformshift{0.278819in}{0.345370in}%
\pgfsys@useobject{currentmarker}{}%
\end{pgfscope}%
\end{pgfscope}%
\begin{pgfscope}%
\definecolor{textcolor}{rgb}{0.000000,0.000000,0.000000}%
\pgfsetstrokecolor{textcolor}%
\pgfsetfillcolor{textcolor}%
\pgftext[x=0.100000in, y=0.287500in, left, base]{\color{textcolor}{\rmfamily\fontsize{12.000000}{14.400000}\selectfont\catcode`\^=\active\def^{\ifmmode\sp\else\^{}\fi}\catcode`\%=\active\def%{\%}$\mathdefault{0}$}}%
\end{pgfscope}%
\begin{pgfscope}%
\pgfsetbuttcap%
\pgfsetroundjoin%
\definecolor{currentfill}{rgb}{0.000000,0.000000,0.000000}%
\pgfsetfillcolor{currentfill}%
\pgfsetlinewidth{0.803000pt}%
\definecolor{currentstroke}{rgb}{0.000000,0.000000,0.000000}%
\pgfsetstrokecolor{currentstroke}%
\pgfsetdash{}{0pt}%
\pgfsys@defobject{currentmarker}{\pgfqpoint{-0.048611in}{0.000000in}}{\pgfqpoint{-0.000000in}{0.000000in}}{%
\pgfpathmoveto{\pgfqpoint{-0.000000in}{0.000000in}}%
\pgfpathlineto{\pgfqpoint{-0.048611in}{0.000000in}}%
\pgfusepath{stroke,fill}%
}%
\begin{pgfscope}%
\pgfsys@transformshift{0.278819in}{0.794536in}%
\pgfsys@useobject{currentmarker}{}%
\end{pgfscope}%
\end{pgfscope}%
\begin{pgfscope}%
\definecolor{textcolor}{rgb}{0.000000,0.000000,0.000000}%
\pgfsetstrokecolor{textcolor}%
\pgfsetfillcolor{textcolor}%
\pgftext[x=0.100000in, y=0.736666in, left, base]{\color{textcolor}{\rmfamily\fontsize{12.000000}{14.400000}\selectfont\catcode`\^=\active\def^{\ifmmode\sp\else\^{}\fi}\catcode`\%=\active\def%{\%}$\mathdefault{2}$}}%
\end{pgfscope}%
\begin{pgfscope}%
\pgfsetbuttcap%
\pgfsetroundjoin%
\definecolor{currentfill}{rgb}{0.000000,0.000000,0.000000}%
\pgfsetfillcolor{currentfill}%
\pgfsetlinewidth{0.803000pt}%
\definecolor{currentstroke}{rgb}{0.000000,0.000000,0.000000}%
\pgfsetstrokecolor{currentstroke}%
\pgfsetdash{}{0pt}%
\pgfsys@defobject{currentmarker}{\pgfqpoint{-0.048611in}{0.000000in}}{\pgfqpoint{-0.000000in}{0.000000in}}{%
\pgfpathmoveto{\pgfqpoint{-0.000000in}{0.000000in}}%
\pgfpathlineto{\pgfqpoint{-0.048611in}{0.000000in}}%
\pgfusepath{stroke,fill}%
}%
\begin{pgfscope}%
\pgfsys@transformshift{0.278819in}{1.243703in}%
\pgfsys@useobject{currentmarker}{}%
\end{pgfscope}%
\end{pgfscope}%
\begin{pgfscope}%
\definecolor{textcolor}{rgb}{0.000000,0.000000,0.000000}%
\pgfsetstrokecolor{textcolor}%
\pgfsetfillcolor{textcolor}%
\pgftext[x=0.100000in, y=1.185833in, left, base]{\color{textcolor}{\rmfamily\fontsize{12.000000}{14.400000}\selectfont\catcode`\^=\active\def^{\ifmmode\sp\else\^{}\fi}\catcode`\%=\active\def%{\%}$\mathdefault{4}$}}%
\end{pgfscope}%
\begin{pgfscope}%
\pgfsetbuttcap%
\pgfsetroundjoin%
\definecolor{currentfill}{rgb}{0.000000,0.000000,0.000000}%
\pgfsetfillcolor{currentfill}%
\pgfsetlinewidth{0.803000pt}%
\definecolor{currentstroke}{rgb}{0.000000,0.000000,0.000000}%
\pgfsetstrokecolor{currentstroke}%
\pgfsetdash{}{0pt}%
\pgfsys@defobject{currentmarker}{\pgfqpoint{-0.048611in}{0.000000in}}{\pgfqpoint{-0.000000in}{0.000000in}}{%
\pgfpathmoveto{\pgfqpoint{-0.000000in}{0.000000in}}%
\pgfpathlineto{\pgfqpoint{-0.048611in}{0.000000in}}%
\pgfusepath{stroke,fill}%
}%
\begin{pgfscope}%
\pgfsys@transformshift{0.278819in}{1.692870in}%
\pgfsys@useobject{currentmarker}{}%
\end{pgfscope}%
\end{pgfscope}%
\begin{pgfscope}%
\definecolor{textcolor}{rgb}{0.000000,0.000000,0.000000}%
\pgfsetstrokecolor{textcolor}%
\pgfsetfillcolor{textcolor}%
\pgftext[x=0.100000in, y=1.635000in, left, base]{\color{textcolor}{\rmfamily\fontsize{12.000000}{14.400000}\selectfont\catcode`\^=\active\def^{\ifmmode\sp\else\^{}\fi}\catcode`\%=\active\def%{\%}$\mathdefault{6}$}}%
\end{pgfscope}%
\begin{pgfscope}%
\pgfsetrectcap%
\pgfsetmiterjoin%
\pgfsetlinewidth{0.803000pt}%
\definecolor{currentstroke}{rgb}{0.000000,0.000000,0.000000}%
\pgfsetstrokecolor{currentstroke}%
\pgfsetdash{}{0pt}%
\pgfpathmoveto{\pgfqpoint{0.278819in}{0.345370in}}%
\pgfpathlineto{\pgfqpoint{0.278819in}{1.692870in}}%
\pgfusepath{stroke}%
\end{pgfscope}%
\begin{pgfscope}%
\pgfsetrectcap%
\pgfsetmiterjoin%
\pgfsetlinewidth{0.803000pt}%
\definecolor{currentstroke}{rgb}{0.000000,0.000000,0.000000}%
\pgfsetstrokecolor{currentstroke}%
\pgfsetdash{}{0pt}%
\pgfpathmoveto{\pgfqpoint{1.828819in}{0.345370in}}%
\pgfpathlineto{\pgfqpoint{1.828819in}{1.692870in}}%
\pgfusepath{stroke}%
\end{pgfscope}%
\begin{pgfscope}%
\pgfsetrectcap%
\pgfsetmiterjoin%
\pgfsetlinewidth{0.803000pt}%
\definecolor{currentstroke}{rgb}{0.000000,0.000000,0.000000}%
\pgfsetstrokecolor{currentstroke}%
\pgfsetdash{}{0pt}%
\pgfpathmoveto{\pgfqpoint{0.278819in}{0.345370in}}%
\pgfpathlineto{\pgfqpoint{1.828819in}{0.345370in}}%
\pgfusepath{stroke}%
\end{pgfscope}%
\begin{pgfscope}%
\pgfsetrectcap%
\pgfsetmiterjoin%
\pgfsetlinewidth{0.803000pt}%
\definecolor{currentstroke}{rgb}{0.000000,0.000000,0.000000}%
\pgfsetstrokecolor{currentstroke}%
\pgfsetdash{}{0pt}%
\pgfpathmoveto{\pgfqpoint{0.278819in}{1.692870in}}%
\pgfpathlineto{\pgfqpoint{1.828819in}{1.692870in}}%
\pgfusepath{stroke}%
\end{pgfscope}%
\end{pgfpicture}%
\makeatother%
\endgroup%

        \caption{$c=1$}
        \label{fig:5-experiments-periodic-gaussian-well-1}
    \end{subfigure}
    \begin{subfigure}[b]{0.32\columnwidth}
        %% Creator: Matplotlib, PGF backend
%%
%% To include the figure in your LaTeX document, write
%%   \input{<filename>.pgf}
%%
%% Make sure the required packages are loaded in your preamble
%%   \usepackage{pgf}
%%
%% Also ensure that all the required font packages are loaded; for instance,
%% the lmodern package is sometimes necessary when using math font.
%%   \usepackage{lmodern}
%%
%% Figures using additional raster images can only be included by \input if
%% they are in the same directory as the main LaTeX file. For loading figures
%% from other directories you can use the `import` package
%%   \usepackage{import}
%%
%% and then include the figures with
%%   \import{<path to file>}{<filename>.pgf}
%%
%% Matplotlib used the following preamble
%%   \def\mathdefault#1{#1}
%%   \everymath=\expandafter{\the\everymath\displaystyle}
%%   
%%   \usepackage{fontspec}
%%   \setmainfont{DejaVuSerif.ttf}[Path=\detokenize{C:/Users/fabio/Documents/Work/MasterThesis/Rand-SD/.venv/Lib/site-packages/matplotlib/mpl-data/fonts/ttf/}]
%%   \setsansfont{DejaVuSans.ttf}[Path=\detokenize{C:/Users/fabio/Documents/Work/MasterThesis/Rand-SD/.venv/Lib/site-packages/matplotlib/mpl-data/fonts/ttf/}]
%%   \setmonofont{DejaVuSansMono.ttf}[Path=\detokenize{C:/Users/fabio/Documents/Work/MasterThesis/Rand-SD/.venv/Lib/site-packages/matplotlib/mpl-data/fonts/ttf/}]
%%   \makeatletter\@ifpackageloaded{underscore}{}{\usepackage[strings]{underscore}}\makeatother
%%
\begingroup%
\makeatletter%
\begin{pgfpicture}%
\pgfpathrectangle{\pgfpointorigin}{\pgfqpoint{2.023953in}{1.779135in}}%
\pgfusepath{use as bounding box, clip}%
\begin{pgfscope}%
\pgfsetbuttcap%
\pgfsetmiterjoin%
\definecolor{currentfill}{rgb}{1.000000,1.000000,1.000000}%
\pgfsetfillcolor{currentfill}%
\pgfsetlinewidth{0.000000pt}%
\definecolor{currentstroke}{rgb}{1.000000,1.000000,1.000000}%
\pgfsetstrokecolor{currentstroke}%
\pgfsetdash{}{0pt}%
\pgfpathmoveto{\pgfqpoint{0.000000in}{0.000000in}}%
\pgfpathlineto{\pgfqpoint{2.023953in}{0.000000in}}%
\pgfpathlineto{\pgfqpoint{2.023953in}{1.779135in}}%
\pgfpathlineto{\pgfqpoint{0.000000in}{1.779135in}}%
\pgfpathlineto{\pgfqpoint{0.000000in}{0.000000in}}%
\pgfpathclose%
\pgfusepath{fill}%
\end{pgfscope}%
\begin{pgfscope}%
\pgfsetbuttcap%
\pgfsetmiterjoin%
\definecolor{currentfill}{rgb}{1.000000,1.000000,1.000000}%
\pgfsetfillcolor{currentfill}%
\pgfsetlinewidth{0.000000pt}%
\definecolor{currentstroke}{rgb}{0.000000,0.000000,0.000000}%
\pgfsetstrokecolor{currentstroke}%
\pgfsetstrokeopacity{0.000000}%
\pgfsetdash{}{0pt}%
\pgfpathmoveto{\pgfqpoint{0.373953in}{0.331635in}}%
\pgfpathlineto{\pgfqpoint{1.923953in}{0.331635in}}%
\pgfpathlineto{\pgfqpoint{1.923953in}{1.679135in}}%
\pgfpathlineto{\pgfqpoint{0.373953in}{1.679135in}}%
\pgfpathlineto{\pgfqpoint{0.373953in}{0.331635in}}%
\pgfpathclose%
\pgfusepath{fill}%
\end{pgfscope}%
\begin{pgfscope}%
\pgfpathrectangle{\pgfqpoint{0.373953in}{0.331635in}}{\pgfqpoint{1.550000in}{1.347500in}}%
\pgfusepath{clip}%
\pgfsetbuttcap%
\pgfsetroundjoin%
\definecolor{currentfill}{rgb}{0.913264,0.948881,0.982284}%
\pgfsetfillcolor{currentfill}%
\pgfsetlinewidth{0.000000pt}%
\definecolor{currentstroke}{rgb}{0.000000,0.000000,0.000000}%
\pgfsetstrokecolor{currentstroke}%
\pgfsetdash{}{0pt}%
\pgfpathmoveto{\pgfqpoint{0.734054in}{0.597840in}}%
\pgfpathlineto{\pgfqpoint{0.749710in}{0.594489in}}%
\pgfpathlineto{\pgfqpoint{0.765367in}{0.593819in}}%
\pgfpathlineto{\pgfqpoint{0.781024in}{0.595830in}}%
\pgfpathlineto{\pgfqpoint{0.796680in}{0.600522in}}%
\pgfpathlineto{\pgfqpoint{0.803786in}{0.603857in}}%
\pgfpathlineto{\pgfqpoint{0.812337in}{0.608724in}}%
\pgfpathlineto{\pgfqpoint{0.823668in}{0.617468in}}%
\pgfpathlineto{\pgfqpoint{0.827993in}{0.621790in}}%
\pgfpathlineto{\pgfqpoint{0.835437in}{0.631079in}}%
\pgfpathlineto{\pgfqpoint{0.843005in}{0.644691in}}%
\pgfpathlineto{\pgfqpoint{0.843650in}{0.646775in}}%
\pgfpathlineto{\pgfqpoint{0.846675in}{0.658302in}}%
\pgfpathlineto{\pgfqpoint{0.847390in}{0.671913in}}%
\pgfpathlineto{\pgfqpoint{0.845246in}{0.685524in}}%
\pgfpathlineto{\pgfqpoint{0.843650in}{0.689878in}}%
\pgfpathlineto{\pgfqpoint{0.839642in}{0.699135in}}%
\pgfpathlineto{\pgfqpoint{0.830389in}{0.712746in}}%
\pgfpathlineto{\pgfqpoint{0.827993in}{0.715351in}}%
\pgfpathlineto{\pgfqpoint{0.815333in}{0.726357in}}%
\pgfpathlineto{\pgfqpoint{0.812337in}{0.728440in}}%
\pgfpathlineto{\pgfqpoint{0.796680in}{0.736484in}}%
\pgfpathlineto{\pgfqpoint{0.786032in}{0.739968in}}%
\pgfpathlineto{\pgfqpoint{0.781024in}{0.741356in}}%
\pgfpathlineto{\pgfqpoint{0.765367in}{0.743220in}}%
\pgfpathlineto{\pgfqpoint{0.749710in}{0.742598in}}%
\pgfpathlineto{\pgfqpoint{0.736451in}{0.739968in}}%
\pgfpathlineto{\pgfqpoint{0.734054in}{0.739407in}}%
\pgfpathlineto{\pgfqpoint{0.718397in}{0.732829in}}%
\pgfpathlineto{\pgfqpoint{0.707712in}{0.726357in}}%
\pgfpathlineto{\pgfqpoint{0.702741in}{0.722597in}}%
\pgfpathlineto{\pgfqpoint{0.692682in}{0.712746in}}%
\pgfpathlineto{\pgfqpoint{0.687084in}{0.705312in}}%
\pgfpathlineto{\pgfqpoint{0.683248in}{0.699135in}}%
\pgfpathlineto{\pgfqpoint{0.677850in}{0.685524in}}%
\pgfpathlineto{\pgfqpoint{0.675538in}{0.671913in}}%
\pgfpathlineto{\pgfqpoint{0.676309in}{0.658302in}}%
\pgfpathlineto{\pgfqpoint{0.680163in}{0.644691in}}%
\pgfpathlineto{\pgfqpoint{0.687084in}{0.631122in}}%
\pgfpathlineto{\pgfqpoint{0.687111in}{0.631079in}}%
\pgfpathlineto{\pgfqpoint{0.699186in}{0.617468in}}%
\pgfpathlineto{\pgfqpoint{0.702741in}{0.614378in}}%
\pgfpathlineto{\pgfqpoint{0.718397in}{0.603880in}}%
\pgfpathlineto{\pgfqpoint{0.718447in}{0.603857in}}%
\pgfpathlineto{\pgfqpoint{0.734054in}{0.597840in}}%
\pgfpathclose%
\pgfpathmoveto{\pgfqpoint{1.501226in}{0.600522in}}%
\pgfpathlineto{\pgfqpoint{1.516882in}{0.595830in}}%
\pgfpathlineto{\pgfqpoint{1.532539in}{0.593819in}}%
\pgfpathlineto{\pgfqpoint{1.548195in}{0.594489in}}%
\pgfpathlineto{\pgfqpoint{1.563852in}{0.597840in}}%
\pgfpathlineto{\pgfqpoint{1.579459in}{0.603857in}}%
\pgfpathlineto{\pgfqpoint{1.579508in}{0.603880in}}%
\pgfpathlineto{\pgfqpoint{1.595165in}{0.614378in}}%
\pgfpathlineto{\pgfqpoint{1.598720in}{0.617468in}}%
\pgfpathlineto{\pgfqpoint{1.610795in}{0.631079in}}%
\pgfpathlineto{\pgfqpoint{1.610822in}{0.631122in}}%
\pgfpathlineto{\pgfqpoint{1.617743in}{0.644691in}}%
\pgfpathlineto{\pgfqpoint{1.621597in}{0.658302in}}%
\pgfpathlineto{\pgfqpoint{1.622368in}{0.671913in}}%
\pgfpathlineto{\pgfqpoint{1.620056in}{0.685524in}}%
\pgfpathlineto{\pgfqpoint{1.614658in}{0.699135in}}%
\pgfpathlineto{\pgfqpoint{1.610822in}{0.705312in}}%
\pgfpathlineto{\pgfqpoint{1.605223in}{0.712746in}}%
\pgfpathlineto{\pgfqpoint{1.595165in}{0.722597in}}%
\pgfpathlineto{\pgfqpoint{1.590194in}{0.726357in}}%
\pgfpathlineto{\pgfqpoint{1.579508in}{0.732829in}}%
\pgfpathlineto{\pgfqpoint{1.563852in}{0.739407in}}%
\pgfpathlineto{\pgfqpoint{1.561454in}{0.739968in}}%
\pgfpathlineto{\pgfqpoint{1.548195in}{0.742598in}}%
\pgfpathlineto{\pgfqpoint{1.532539in}{0.743220in}}%
\pgfpathlineto{\pgfqpoint{1.516882in}{0.741356in}}%
\pgfpathlineto{\pgfqpoint{1.511873in}{0.739968in}}%
\pgfpathlineto{\pgfqpoint{1.501226in}{0.736484in}}%
\pgfpathlineto{\pgfqpoint{1.485569in}{0.728440in}}%
\pgfpathlineto{\pgfqpoint{1.482573in}{0.726357in}}%
\pgfpathlineto{\pgfqpoint{1.469913in}{0.715351in}}%
\pgfpathlineto{\pgfqpoint{1.467517in}{0.712746in}}%
\pgfpathlineto{\pgfqpoint{1.458264in}{0.699135in}}%
\pgfpathlineto{\pgfqpoint{1.454256in}{0.689878in}}%
\pgfpathlineto{\pgfqpoint{1.452660in}{0.685524in}}%
\pgfpathlineto{\pgfqpoint{1.450516in}{0.671913in}}%
\pgfpathlineto{\pgfqpoint{1.451231in}{0.658302in}}%
\pgfpathlineto{\pgfqpoint{1.454256in}{0.646775in}}%
\pgfpathlineto{\pgfqpoint{1.454901in}{0.644691in}}%
\pgfpathlineto{\pgfqpoint{1.462468in}{0.631079in}}%
\pgfpathlineto{\pgfqpoint{1.469913in}{0.621790in}}%
\pgfpathlineto{\pgfqpoint{1.474238in}{0.617468in}}%
\pgfpathlineto{\pgfqpoint{1.485569in}{0.608724in}}%
\pgfpathlineto{\pgfqpoint{1.494120in}{0.603857in}}%
\pgfpathlineto{\pgfqpoint{1.501226in}{0.600522in}}%
\pgfpathclose%
\pgfpathmoveto{\pgfqpoint{0.749710in}{1.268172in}}%
\pgfpathlineto{\pgfqpoint{0.765367in}{1.267550in}}%
\pgfpathlineto{\pgfqpoint{0.781024in}{1.269414in}}%
\pgfpathlineto{\pgfqpoint{0.786032in}{1.270802in}}%
\pgfpathlineto{\pgfqpoint{0.796680in}{1.274286in}}%
\pgfpathlineto{\pgfqpoint{0.812337in}{1.282330in}}%
\pgfpathlineto{\pgfqpoint{0.815333in}{1.284413in}}%
\pgfpathlineto{\pgfqpoint{0.827993in}{1.295419in}}%
\pgfpathlineto{\pgfqpoint{0.830389in}{1.298024in}}%
\pgfpathlineto{\pgfqpoint{0.839642in}{1.311635in}}%
\pgfpathlineto{\pgfqpoint{0.843650in}{1.320892in}}%
\pgfpathlineto{\pgfqpoint{0.845246in}{1.325246in}}%
\pgfpathlineto{\pgfqpoint{0.847390in}{1.338857in}}%
\pgfpathlineto{\pgfqpoint{0.846675in}{1.352468in}}%
\pgfpathlineto{\pgfqpoint{0.843650in}{1.363995in}}%
\pgfpathlineto{\pgfqpoint{0.843005in}{1.366079in}}%
\pgfpathlineto{\pgfqpoint{0.835437in}{1.379691in}}%
\pgfpathlineto{\pgfqpoint{0.827993in}{1.388980in}}%
\pgfpathlineto{\pgfqpoint{0.823668in}{1.393302in}}%
\pgfpathlineto{\pgfqpoint{0.812337in}{1.402046in}}%
\pgfpathlineto{\pgfqpoint{0.803786in}{1.406913in}}%
\pgfpathlineto{\pgfqpoint{0.796680in}{1.410248in}}%
\pgfpathlineto{\pgfqpoint{0.781024in}{1.414940in}}%
\pgfpathlineto{\pgfqpoint{0.765367in}{1.416951in}}%
\pgfpathlineto{\pgfqpoint{0.749710in}{1.416281in}}%
\pgfpathlineto{\pgfqpoint{0.734054in}{1.412930in}}%
\pgfpathlineto{\pgfqpoint{0.718447in}{1.406913in}}%
\pgfpathlineto{\pgfqpoint{0.718397in}{1.406890in}}%
\pgfpathlineto{\pgfqpoint{0.702741in}{1.396392in}}%
\pgfpathlineto{\pgfqpoint{0.699186in}{1.393302in}}%
\pgfpathlineto{\pgfqpoint{0.687111in}{1.379691in}}%
\pgfpathlineto{\pgfqpoint{0.687084in}{1.379648in}}%
\pgfpathlineto{\pgfqpoint{0.680163in}{1.366079in}}%
\pgfpathlineto{\pgfqpoint{0.676309in}{1.352468in}}%
\pgfpathlineto{\pgfqpoint{0.675538in}{1.338857in}}%
\pgfpathlineto{\pgfqpoint{0.677850in}{1.325246in}}%
\pgfpathlineto{\pgfqpoint{0.683248in}{1.311635in}}%
\pgfpathlineto{\pgfqpoint{0.687084in}{1.305458in}}%
\pgfpathlineto{\pgfqpoint{0.692682in}{1.298024in}}%
\pgfpathlineto{\pgfqpoint{0.702741in}{1.288173in}}%
\pgfpathlineto{\pgfqpoint{0.707712in}{1.284413in}}%
\pgfpathlineto{\pgfqpoint{0.718397in}{1.277941in}}%
\pgfpathlineto{\pgfqpoint{0.734054in}{1.271363in}}%
\pgfpathlineto{\pgfqpoint{0.736451in}{1.270802in}}%
\pgfpathlineto{\pgfqpoint{0.749710in}{1.268172in}}%
\pgfpathclose%
\pgfpathmoveto{\pgfqpoint{1.516882in}{1.269414in}}%
\pgfpathlineto{\pgfqpoint{1.532539in}{1.267550in}}%
\pgfpathlineto{\pgfqpoint{1.548195in}{1.268172in}}%
\pgfpathlineto{\pgfqpoint{1.561454in}{1.270802in}}%
\pgfpathlineto{\pgfqpoint{1.563852in}{1.271363in}}%
\pgfpathlineto{\pgfqpoint{1.579508in}{1.277941in}}%
\pgfpathlineto{\pgfqpoint{1.590194in}{1.284413in}}%
\pgfpathlineto{\pgfqpoint{1.595165in}{1.288173in}}%
\pgfpathlineto{\pgfqpoint{1.605223in}{1.298024in}}%
\pgfpathlineto{\pgfqpoint{1.610822in}{1.305458in}}%
\pgfpathlineto{\pgfqpoint{1.614658in}{1.311635in}}%
\pgfpathlineto{\pgfqpoint{1.620056in}{1.325246in}}%
\pgfpathlineto{\pgfqpoint{1.622368in}{1.338857in}}%
\pgfpathlineto{\pgfqpoint{1.621597in}{1.352468in}}%
\pgfpathlineto{\pgfqpoint{1.617743in}{1.366079in}}%
\pgfpathlineto{\pgfqpoint{1.610822in}{1.379648in}}%
\pgfpathlineto{\pgfqpoint{1.610795in}{1.379691in}}%
\pgfpathlineto{\pgfqpoint{1.598720in}{1.393302in}}%
\pgfpathlineto{\pgfqpoint{1.595165in}{1.396392in}}%
\pgfpathlineto{\pgfqpoint{1.579508in}{1.406890in}}%
\pgfpathlineto{\pgfqpoint{1.579459in}{1.406913in}}%
\pgfpathlineto{\pgfqpoint{1.563852in}{1.412930in}}%
\pgfpathlineto{\pgfqpoint{1.548195in}{1.416281in}}%
\pgfpathlineto{\pgfqpoint{1.532539in}{1.416951in}}%
\pgfpathlineto{\pgfqpoint{1.516882in}{1.414940in}}%
\pgfpathlineto{\pgfqpoint{1.501226in}{1.410248in}}%
\pgfpathlineto{\pgfqpoint{1.494120in}{1.406913in}}%
\pgfpathlineto{\pgfqpoint{1.485569in}{1.402046in}}%
\pgfpathlineto{\pgfqpoint{1.474238in}{1.393302in}}%
\pgfpathlineto{\pgfqpoint{1.469913in}{1.388980in}}%
\pgfpathlineto{\pgfqpoint{1.462468in}{1.379691in}}%
\pgfpathlineto{\pgfqpoint{1.454901in}{1.366079in}}%
\pgfpathlineto{\pgfqpoint{1.454256in}{1.363995in}}%
\pgfpathlineto{\pgfqpoint{1.451231in}{1.352468in}}%
\pgfpathlineto{\pgfqpoint{1.450516in}{1.338857in}}%
\pgfpathlineto{\pgfqpoint{1.452660in}{1.325246in}}%
\pgfpathlineto{\pgfqpoint{1.454256in}{1.320892in}}%
\pgfpathlineto{\pgfqpoint{1.458264in}{1.311635in}}%
\pgfpathlineto{\pgfqpoint{1.467517in}{1.298024in}}%
\pgfpathlineto{\pgfqpoint{1.469913in}{1.295419in}}%
\pgfpathlineto{\pgfqpoint{1.482573in}{1.284413in}}%
\pgfpathlineto{\pgfqpoint{1.485569in}{1.282330in}}%
\pgfpathlineto{\pgfqpoint{1.501226in}{1.274286in}}%
\pgfpathlineto{\pgfqpoint{1.511873in}{1.270802in}}%
\pgfpathlineto{\pgfqpoint{1.516882in}{1.269414in}}%
\pgfpathclose%
\pgfusepath{fill}%
\end{pgfscope}%
\begin{pgfscope}%
\pgfpathrectangle{\pgfqpoint{0.373953in}{0.331635in}}{\pgfqpoint{1.550000in}{1.347500in}}%
\pgfusepath{clip}%
\pgfsetbuttcap%
\pgfsetroundjoin%
\definecolor{currentfill}{rgb}{0.805260,0.878016,0.946851}%
\pgfsetfillcolor{currentfill}%
\pgfsetlinewidth{0.000000pt}%
\definecolor{currentstroke}{rgb}{0.000000,0.000000,0.000000}%
\pgfsetstrokecolor{currentstroke}%
\pgfsetdash{}{0pt}%
\pgfpathmoveto{\pgfqpoint{0.718397in}{0.532207in}}%
\pgfpathlineto{\pgfqpoint{0.734054in}{0.528713in}}%
\pgfpathlineto{\pgfqpoint{0.749710in}{0.526773in}}%
\pgfpathlineto{\pgfqpoint{0.765367in}{0.526385in}}%
\pgfpathlineto{\pgfqpoint{0.781024in}{0.527549in}}%
\pgfpathlineto{\pgfqpoint{0.796680in}{0.530266in}}%
\pgfpathlineto{\pgfqpoint{0.812337in}{0.534538in}}%
\pgfpathlineto{\pgfqpoint{0.815758in}{0.535802in}}%
\pgfpathlineto{\pgfqpoint{0.827993in}{0.540572in}}%
\pgfpathlineto{\pgfqpoint{0.843650in}{0.548293in}}%
\pgfpathlineto{\pgfqpoint{0.845552in}{0.549413in}}%
\pgfpathlineto{\pgfqpoint{0.859306in}{0.558172in}}%
\pgfpathlineto{\pgfqpoint{0.865884in}{0.563024in}}%
\pgfpathlineto{\pgfqpoint{0.874963in}{0.570446in}}%
\pgfpathlineto{\pgfqpoint{0.881704in}{0.576635in}}%
\pgfpathlineto{\pgfqpoint{0.890620in}{0.585964in}}%
\pgfpathlineto{\pgfqpoint{0.894358in}{0.590246in}}%
\pgfpathlineto{\pgfqpoint{0.904462in}{0.603857in}}%
\pgfpathlineto{\pgfqpoint{0.906276in}{0.606849in}}%
\pgfpathlineto{\pgfqpoint{0.912311in}{0.617468in}}%
\pgfpathlineto{\pgfqpoint{0.918235in}{0.631079in}}%
\pgfpathlineto{\pgfqpoint{0.921933in}{0.643352in}}%
\pgfpathlineto{\pgfqpoint{0.922320in}{0.644691in}}%
\pgfpathlineto{\pgfqpoint{0.924518in}{0.658302in}}%
\pgfpathlineto{\pgfqpoint{0.924957in}{0.671913in}}%
\pgfpathlineto{\pgfqpoint{0.923639in}{0.685524in}}%
\pgfpathlineto{\pgfqpoint{0.921933in}{0.693085in}}%
\pgfpathlineto{\pgfqpoint{0.920512in}{0.699135in}}%
\pgfpathlineto{\pgfqpoint{0.915502in}{0.712746in}}%
\pgfpathlineto{\pgfqpoint{0.908662in}{0.726357in}}%
\pgfpathlineto{\pgfqpoint{0.906276in}{0.730139in}}%
\pgfpathlineto{\pgfqpoint{0.899652in}{0.739968in}}%
\pgfpathlineto{\pgfqpoint{0.890620in}{0.751109in}}%
\pgfpathlineto{\pgfqpoint{0.888430in}{0.753579in}}%
\pgfpathlineto{\pgfqpoint{0.874963in}{0.766714in}}%
\pgfpathlineto{\pgfqpoint{0.874415in}{0.767191in}}%
\pgfpathlineto{\pgfqpoint{0.859306in}{0.778898in}}%
\pgfpathlineto{\pgfqpoint{0.856465in}{0.780802in}}%
\pgfpathlineto{\pgfqpoint{0.843650in}{0.788654in}}%
\pgfpathlineto{\pgfqpoint{0.832344in}{0.794413in}}%
\pgfpathlineto{\pgfqpoint{0.827993in}{0.796487in}}%
\pgfpathlineto{\pgfqpoint{0.812337in}{0.802433in}}%
\pgfpathlineto{\pgfqpoint{0.796680in}{0.806789in}}%
\pgfpathlineto{\pgfqpoint{0.789721in}{0.808024in}}%
\pgfpathlineto{\pgfqpoint{0.781024in}{0.809507in}}%
\pgfpathlineto{\pgfqpoint{0.765367in}{0.810653in}}%
\pgfpathlineto{\pgfqpoint{0.749710in}{0.810271in}}%
\pgfpathlineto{\pgfqpoint{0.734054in}{0.808360in}}%
\pgfpathlineto{\pgfqpoint{0.732514in}{0.808024in}}%
\pgfpathlineto{\pgfqpoint{0.718397in}{0.804810in}}%
\pgfpathlineto{\pgfqpoint{0.702741in}{0.799659in}}%
\pgfpathlineto{\pgfqpoint{0.690525in}{0.794413in}}%
\pgfpathlineto{\pgfqpoint{0.687084in}{0.792836in}}%
\pgfpathlineto{\pgfqpoint{0.671428in}{0.784052in}}%
\pgfpathlineto{\pgfqpoint{0.666502in}{0.780802in}}%
\pgfpathlineto{\pgfqpoint{0.655771in}{0.773051in}}%
\pgfpathlineto{\pgfqpoint{0.648652in}{0.767191in}}%
\pgfpathlineto{\pgfqpoint{0.640115in}{0.759298in}}%
\pgfpathlineto{\pgfqpoint{0.634534in}{0.753579in}}%
\pgfpathlineto{\pgfqpoint{0.624458in}{0.741622in}}%
\pgfpathlineto{\pgfqpoint{0.623170in}{0.739968in}}%
\pgfpathlineto{\pgfqpoint{0.614289in}{0.726357in}}%
\pgfpathlineto{\pgfqpoint{0.608801in}{0.715720in}}%
\pgfpathlineto{\pgfqpoint{0.607348in}{0.712746in}}%
\pgfpathlineto{\pgfqpoint{0.602433in}{0.699135in}}%
\pgfpathlineto{\pgfqpoint{0.599309in}{0.685524in}}%
\pgfpathlineto{\pgfqpoint{0.597970in}{0.671913in}}%
\pgfpathlineto{\pgfqpoint{0.598416in}{0.658302in}}%
\pgfpathlineto{\pgfqpoint{0.600648in}{0.644691in}}%
\pgfpathlineto{\pgfqpoint{0.604667in}{0.631079in}}%
\pgfpathlineto{\pgfqpoint{0.608801in}{0.621368in}}%
\pgfpathlineto{\pgfqpoint{0.610552in}{0.617468in}}%
\pgfpathlineto{\pgfqpoint{0.618495in}{0.603857in}}%
\pgfpathlineto{\pgfqpoint{0.624458in}{0.595542in}}%
\pgfpathlineto{\pgfqpoint{0.628562in}{0.590246in}}%
\pgfpathlineto{\pgfqpoint{0.640115in}{0.577590in}}%
\pgfpathlineto{\pgfqpoint{0.641080in}{0.576635in}}%
\pgfpathlineto{\pgfqpoint{0.655771in}{0.563863in}}%
\pgfpathlineto{\pgfqpoint{0.656870in}{0.563024in}}%
\pgfpathlineto{\pgfqpoint{0.671428in}{0.552980in}}%
\pgfpathlineto{\pgfqpoint{0.677520in}{0.549413in}}%
\pgfpathlineto{\pgfqpoint{0.687084in}{0.544229in}}%
\pgfpathlineto{\pgfqpoint{0.702741in}{0.537323in}}%
\pgfpathlineto{\pgfqpoint{0.707227in}{0.535802in}}%
\pgfpathlineto{\pgfqpoint{0.718397in}{0.532207in}}%
\pgfpathclose%
\pgfpathmoveto{\pgfqpoint{0.718447in}{0.603857in}}%
\pgfpathlineto{\pgfqpoint{0.718397in}{0.603880in}}%
\pgfpathlineto{\pgfqpoint{0.702741in}{0.614378in}}%
\pgfpathlineto{\pgfqpoint{0.699186in}{0.617468in}}%
\pgfpathlineto{\pgfqpoint{0.687111in}{0.631079in}}%
\pgfpathlineto{\pgfqpoint{0.687084in}{0.631122in}}%
\pgfpathlineto{\pgfqpoint{0.680163in}{0.644691in}}%
\pgfpathlineto{\pgfqpoint{0.676309in}{0.658302in}}%
\pgfpathlineto{\pgfqpoint{0.675538in}{0.671913in}}%
\pgfpathlineto{\pgfqpoint{0.677850in}{0.685524in}}%
\pgfpathlineto{\pgfqpoint{0.683248in}{0.699135in}}%
\pgfpathlineto{\pgfqpoint{0.687084in}{0.705312in}}%
\pgfpathlineto{\pgfqpoint{0.692682in}{0.712746in}}%
\pgfpathlineto{\pgfqpoint{0.702741in}{0.722597in}}%
\pgfpathlineto{\pgfqpoint{0.707712in}{0.726357in}}%
\pgfpathlineto{\pgfqpoint{0.718397in}{0.732829in}}%
\pgfpathlineto{\pgfqpoint{0.734054in}{0.739407in}}%
\pgfpathlineto{\pgfqpoint{0.736451in}{0.739968in}}%
\pgfpathlineto{\pgfqpoint{0.749710in}{0.742598in}}%
\pgfpathlineto{\pgfqpoint{0.765367in}{0.743220in}}%
\pgfpathlineto{\pgfqpoint{0.781024in}{0.741356in}}%
\pgfpathlineto{\pgfqpoint{0.786032in}{0.739968in}}%
\pgfpathlineto{\pgfqpoint{0.796680in}{0.736484in}}%
\pgfpathlineto{\pgfqpoint{0.812337in}{0.728440in}}%
\pgfpathlineto{\pgfqpoint{0.815333in}{0.726357in}}%
\pgfpathlineto{\pgfqpoint{0.827993in}{0.715351in}}%
\pgfpathlineto{\pgfqpoint{0.830389in}{0.712746in}}%
\pgfpathlineto{\pgfqpoint{0.839642in}{0.699135in}}%
\pgfpathlineto{\pgfqpoint{0.843650in}{0.689878in}}%
\pgfpathlineto{\pgfqpoint{0.845246in}{0.685524in}}%
\pgfpathlineto{\pgfqpoint{0.847390in}{0.671913in}}%
\pgfpathlineto{\pgfqpoint{0.846675in}{0.658302in}}%
\pgfpathlineto{\pgfqpoint{0.843650in}{0.646775in}}%
\pgfpathlineto{\pgfqpoint{0.843005in}{0.644691in}}%
\pgfpathlineto{\pgfqpoint{0.835437in}{0.631079in}}%
\pgfpathlineto{\pgfqpoint{0.827993in}{0.621790in}}%
\pgfpathlineto{\pgfqpoint{0.823668in}{0.617468in}}%
\pgfpathlineto{\pgfqpoint{0.812337in}{0.608724in}}%
\pgfpathlineto{\pgfqpoint{0.803786in}{0.603857in}}%
\pgfpathlineto{\pgfqpoint{0.796680in}{0.600522in}}%
\pgfpathlineto{\pgfqpoint{0.781024in}{0.595830in}}%
\pgfpathlineto{\pgfqpoint{0.765367in}{0.593819in}}%
\pgfpathlineto{\pgfqpoint{0.749710in}{0.594489in}}%
\pgfpathlineto{\pgfqpoint{0.734054in}{0.597840in}}%
\pgfpathlineto{\pgfqpoint{0.718447in}{0.603857in}}%
\pgfpathclose%
\pgfpathmoveto{\pgfqpoint{1.485569in}{0.534538in}}%
\pgfpathlineto{\pgfqpoint{1.501226in}{0.530266in}}%
\pgfpathlineto{\pgfqpoint{1.516882in}{0.527549in}}%
\pgfpathlineto{\pgfqpoint{1.532539in}{0.526385in}}%
\pgfpathlineto{\pgfqpoint{1.548195in}{0.526773in}}%
\pgfpathlineto{\pgfqpoint{1.563852in}{0.528713in}}%
\pgfpathlineto{\pgfqpoint{1.579508in}{0.532207in}}%
\pgfpathlineto{\pgfqpoint{1.590679in}{0.535802in}}%
\pgfpathlineto{\pgfqpoint{1.595165in}{0.537323in}}%
\pgfpathlineto{\pgfqpoint{1.610822in}{0.544229in}}%
\pgfpathlineto{\pgfqpoint{1.620386in}{0.549413in}}%
\pgfpathlineto{\pgfqpoint{1.626478in}{0.552980in}}%
\pgfpathlineto{\pgfqpoint{1.641036in}{0.563024in}}%
\pgfpathlineto{\pgfqpoint{1.642135in}{0.563863in}}%
\pgfpathlineto{\pgfqpoint{1.656826in}{0.576635in}}%
\pgfpathlineto{\pgfqpoint{1.657791in}{0.577590in}}%
\pgfpathlineto{\pgfqpoint{1.669344in}{0.590246in}}%
\pgfpathlineto{\pgfqpoint{1.673448in}{0.595542in}}%
\pgfpathlineto{\pgfqpoint{1.679411in}{0.603857in}}%
\pgfpathlineto{\pgfqpoint{1.687354in}{0.617468in}}%
\pgfpathlineto{\pgfqpoint{1.689104in}{0.621368in}}%
\pgfpathlineto{\pgfqpoint{1.693239in}{0.631079in}}%
\pgfpathlineto{\pgfqpoint{1.697258in}{0.644691in}}%
\pgfpathlineto{\pgfqpoint{1.699490in}{0.658302in}}%
\pgfpathlineto{\pgfqpoint{1.699936in}{0.671913in}}%
\pgfpathlineto{\pgfqpoint{1.698597in}{0.685524in}}%
\pgfpathlineto{\pgfqpoint{1.695472in}{0.699135in}}%
\pgfpathlineto{\pgfqpoint{1.690558in}{0.712746in}}%
\pgfpathlineto{\pgfqpoint{1.689104in}{0.715720in}}%
\pgfpathlineto{\pgfqpoint{1.683617in}{0.726357in}}%
\pgfpathlineto{\pgfqpoint{1.674736in}{0.739968in}}%
\pgfpathlineto{\pgfqpoint{1.673448in}{0.741622in}}%
\pgfpathlineto{\pgfqpoint{1.663372in}{0.753579in}}%
\pgfpathlineto{\pgfqpoint{1.657791in}{0.759298in}}%
\pgfpathlineto{\pgfqpoint{1.649254in}{0.767191in}}%
\pgfpathlineto{\pgfqpoint{1.642135in}{0.773051in}}%
\pgfpathlineto{\pgfqpoint{1.631404in}{0.780802in}}%
\pgfpathlineto{\pgfqpoint{1.626478in}{0.784052in}}%
\pgfpathlineto{\pgfqpoint{1.610822in}{0.792836in}}%
\pgfpathlineto{\pgfqpoint{1.607381in}{0.794413in}}%
\pgfpathlineto{\pgfqpoint{1.595165in}{0.799659in}}%
\pgfpathlineto{\pgfqpoint{1.579508in}{0.804810in}}%
\pgfpathlineto{\pgfqpoint{1.565392in}{0.808024in}}%
\pgfpathlineto{\pgfqpoint{1.563852in}{0.808360in}}%
\pgfpathlineto{\pgfqpoint{1.548195in}{0.810271in}}%
\pgfpathlineto{\pgfqpoint{1.532539in}{0.810653in}}%
\pgfpathlineto{\pgfqpoint{1.516882in}{0.809507in}}%
\pgfpathlineto{\pgfqpoint{1.508185in}{0.808024in}}%
\pgfpathlineto{\pgfqpoint{1.501226in}{0.806789in}}%
\pgfpathlineto{\pgfqpoint{1.485569in}{0.802433in}}%
\pgfpathlineto{\pgfqpoint{1.469913in}{0.796487in}}%
\pgfpathlineto{\pgfqpoint{1.465562in}{0.794413in}}%
\pgfpathlineto{\pgfqpoint{1.454256in}{0.788654in}}%
\pgfpathlineto{\pgfqpoint{1.441441in}{0.780802in}}%
\pgfpathlineto{\pgfqpoint{1.438599in}{0.778898in}}%
\pgfpathlineto{\pgfqpoint{1.423491in}{0.767191in}}%
\pgfpathlineto{\pgfqpoint{1.422943in}{0.766714in}}%
\pgfpathlineto{\pgfqpoint{1.409475in}{0.753579in}}%
\pgfpathlineto{\pgfqpoint{1.407286in}{0.751109in}}%
\pgfpathlineto{\pgfqpoint{1.398254in}{0.739968in}}%
\pgfpathlineto{\pgfqpoint{1.391630in}{0.730139in}}%
\pgfpathlineto{\pgfqpoint{1.389244in}{0.726357in}}%
\pgfpathlineto{\pgfqpoint{1.382404in}{0.712746in}}%
\pgfpathlineto{\pgfqpoint{1.377393in}{0.699135in}}%
\pgfpathlineto{\pgfqpoint{1.375973in}{0.693085in}}%
\pgfpathlineto{\pgfqpoint{1.374267in}{0.685524in}}%
\pgfpathlineto{\pgfqpoint{1.372949in}{0.671913in}}%
\pgfpathlineto{\pgfqpoint{1.373388in}{0.658302in}}%
\pgfpathlineto{\pgfqpoint{1.375586in}{0.644691in}}%
\pgfpathlineto{\pgfqpoint{1.375973in}{0.643352in}}%
\pgfpathlineto{\pgfqpoint{1.379670in}{0.631079in}}%
\pgfpathlineto{\pgfqpoint{1.385595in}{0.617468in}}%
\pgfpathlineto{\pgfqpoint{1.391630in}{0.606849in}}%
\pgfpathlineto{\pgfqpoint{1.393443in}{0.603857in}}%
\pgfpathlineto{\pgfqpoint{1.403547in}{0.590246in}}%
\pgfpathlineto{\pgfqpoint{1.407286in}{0.585964in}}%
\pgfpathlineto{\pgfqpoint{1.416202in}{0.576635in}}%
\pgfpathlineto{\pgfqpoint{1.422943in}{0.570446in}}%
\pgfpathlineto{\pgfqpoint{1.432022in}{0.563024in}}%
\pgfpathlineto{\pgfqpoint{1.438599in}{0.558172in}}%
\pgfpathlineto{\pgfqpoint{1.452354in}{0.549413in}}%
\pgfpathlineto{\pgfqpoint{1.454256in}{0.548293in}}%
\pgfpathlineto{\pgfqpoint{1.469913in}{0.540572in}}%
\pgfpathlineto{\pgfqpoint{1.482148in}{0.535802in}}%
\pgfpathlineto{\pgfqpoint{1.485569in}{0.534538in}}%
\pgfpathclose%
\pgfpathmoveto{\pgfqpoint{1.494120in}{0.603857in}}%
\pgfpathlineto{\pgfqpoint{1.485569in}{0.608724in}}%
\pgfpathlineto{\pgfqpoint{1.474238in}{0.617468in}}%
\pgfpathlineto{\pgfqpoint{1.469913in}{0.621790in}}%
\pgfpathlineto{\pgfqpoint{1.462468in}{0.631079in}}%
\pgfpathlineto{\pgfqpoint{1.454901in}{0.644691in}}%
\pgfpathlineto{\pgfqpoint{1.454256in}{0.646775in}}%
\pgfpathlineto{\pgfqpoint{1.451231in}{0.658302in}}%
\pgfpathlineto{\pgfqpoint{1.450516in}{0.671913in}}%
\pgfpathlineto{\pgfqpoint{1.452660in}{0.685524in}}%
\pgfpathlineto{\pgfqpoint{1.454256in}{0.689878in}}%
\pgfpathlineto{\pgfqpoint{1.458264in}{0.699135in}}%
\pgfpathlineto{\pgfqpoint{1.467517in}{0.712746in}}%
\pgfpathlineto{\pgfqpoint{1.469913in}{0.715351in}}%
\pgfpathlineto{\pgfqpoint{1.482573in}{0.726357in}}%
\pgfpathlineto{\pgfqpoint{1.485569in}{0.728440in}}%
\pgfpathlineto{\pgfqpoint{1.501226in}{0.736484in}}%
\pgfpathlineto{\pgfqpoint{1.511873in}{0.739968in}}%
\pgfpathlineto{\pgfqpoint{1.516882in}{0.741356in}}%
\pgfpathlineto{\pgfqpoint{1.532539in}{0.743220in}}%
\pgfpathlineto{\pgfqpoint{1.548195in}{0.742598in}}%
\pgfpathlineto{\pgfqpoint{1.561454in}{0.739968in}}%
\pgfpathlineto{\pgfqpoint{1.563852in}{0.739407in}}%
\pgfpathlineto{\pgfqpoint{1.579508in}{0.732829in}}%
\pgfpathlineto{\pgfqpoint{1.590194in}{0.726357in}}%
\pgfpathlineto{\pgfqpoint{1.595165in}{0.722597in}}%
\pgfpathlineto{\pgfqpoint{1.605223in}{0.712746in}}%
\pgfpathlineto{\pgfqpoint{1.610822in}{0.705312in}}%
\pgfpathlineto{\pgfqpoint{1.614658in}{0.699135in}}%
\pgfpathlineto{\pgfqpoint{1.620056in}{0.685524in}}%
\pgfpathlineto{\pgfqpoint{1.622368in}{0.671913in}}%
\pgfpathlineto{\pgfqpoint{1.621597in}{0.658302in}}%
\pgfpathlineto{\pgfqpoint{1.617743in}{0.644691in}}%
\pgfpathlineto{\pgfqpoint{1.610822in}{0.631122in}}%
\pgfpathlineto{\pgfqpoint{1.610795in}{0.631079in}}%
\pgfpathlineto{\pgfqpoint{1.598720in}{0.617468in}}%
\pgfpathlineto{\pgfqpoint{1.595165in}{0.614378in}}%
\pgfpathlineto{\pgfqpoint{1.579508in}{0.603880in}}%
\pgfpathlineto{\pgfqpoint{1.579459in}{0.603857in}}%
\pgfpathlineto{\pgfqpoint{1.563852in}{0.597840in}}%
\pgfpathlineto{\pgfqpoint{1.548195in}{0.594489in}}%
\pgfpathlineto{\pgfqpoint{1.532539in}{0.593819in}}%
\pgfpathlineto{\pgfqpoint{1.516882in}{0.595830in}}%
\pgfpathlineto{\pgfqpoint{1.501226in}{0.600522in}}%
\pgfpathlineto{\pgfqpoint{1.494120in}{0.603857in}}%
\pgfpathclose%
\pgfpathmoveto{\pgfqpoint{0.734054in}{1.202410in}}%
\pgfpathlineto{\pgfqpoint{0.749710in}{1.200499in}}%
\pgfpathlineto{\pgfqpoint{0.765367in}{1.200117in}}%
\pgfpathlineto{\pgfqpoint{0.781024in}{1.201263in}}%
\pgfpathlineto{\pgfqpoint{0.789721in}{1.202746in}}%
\pgfpathlineto{\pgfqpoint{0.796680in}{1.203981in}}%
\pgfpathlineto{\pgfqpoint{0.812337in}{1.208337in}}%
\pgfpathlineto{\pgfqpoint{0.827993in}{1.214283in}}%
\pgfpathlineto{\pgfqpoint{0.832344in}{1.216357in}}%
\pgfpathlineto{\pgfqpoint{0.843650in}{1.222116in}}%
\pgfpathlineto{\pgfqpoint{0.856465in}{1.229968in}}%
\pgfpathlineto{\pgfqpoint{0.859306in}{1.231872in}}%
\pgfpathlineto{\pgfqpoint{0.874415in}{1.243579in}}%
\pgfpathlineto{\pgfqpoint{0.874963in}{1.244056in}}%
\pgfpathlineto{\pgfqpoint{0.888430in}{1.257191in}}%
\pgfpathlineto{\pgfqpoint{0.890620in}{1.259661in}}%
\pgfpathlineto{\pgfqpoint{0.899652in}{1.270802in}}%
\pgfpathlineto{\pgfqpoint{0.906276in}{1.280631in}}%
\pgfpathlineto{\pgfqpoint{0.908662in}{1.284413in}}%
\pgfpathlineto{\pgfqpoint{0.915502in}{1.298024in}}%
\pgfpathlineto{\pgfqpoint{0.920512in}{1.311635in}}%
\pgfpathlineto{\pgfqpoint{0.921933in}{1.317685in}}%
\pgfpathlineto{\pgfqpoint{0.923639in}{1.325246in}}%
\pgfpathlineto{\pgfqpoint{0.924957in}{1.338857in}}%
\pgfpathlineto{\pgfqpoint{0.924518in}{1.352468in}}%
\pgfpathlineto{\pgfqpoint{0.922320in}{1.366079in}}%
\pgfpathlineto{\pgfqpoint{0.921933in}{1.367418in}}%
\pgfpathlineto{\pgfqpoint{0.918235in}{1.379691in}}%
\pgfpathlineto{\pgfqpoint{0.912311in}{1.393302in}}%
\pgfpathlineto{\pgfqpoint{0.906276in}{1.403921in}}%
\pgfpathlineto{\pgfqpoint{0.904462in}{1.406913in}}%
\pgfpathlineto{\pgfqpoint{0.894358in}{1.420524in}}%
\pgfpathlineto{\pgfqpoint{0.890620in}{1.424806in}}%
\pgfpathlineto{\pgfqpoint{0.881704in}{1.434135in}}%
\pgfpathlineto{\pgfqpoint{0.874963in}{1.440324in}}%
\pgfpathlineto{\pgfqpoint{0.865884in}{1.447746in}}%
\pgfpathlineto{\pgfqpoint{0.859306in}{1.452598in}}%
\pgfpathlineto{\pgfqpoint{0.845552in}{1.461357in}}%
\pgfpathlineto{\pgfqpoint{0.843650in}{1.462477in}}%
\pgfpathlineto{\pgfqpoint{0.827993in}{1.470198in}}%
\pgfpathlineto{\pgfqpoint{0.815758in}{1.474968in}}%
\pgfpathlineto{\pgfqpoint{0.812337in}{1.476232in}}%
\pgfpathlineto{\pgfqpoint{0.796680in}{1.480504in}}%
\pgfpathlineto{\pgfqpoint{0.781024in}{1.483221in}}%
\pgfpathlineto{\pgfqpoint{0.765367in}{1.484385in}}%
\pgfpathlineto{\pgfqpoint{0.749710in}{1.483997in}}%
\pgfpathlineto{\pgfqpoint{0.734054in}{1.482057in}}%
\pgfpathlineto{\pgfqpoint{0.718397in}{1.478563in}}%
\pgfpathlineto{\pgfqpoint{0.707227in}{1.474968in}}%
\pgfpathlineto{\pgfqpoint{0.702741in}{1.473447in}}%
\pgfpathlineto{\pgfqpoint{0.687084in}{1.466541in}}%
\pgfpathlineto{\pgfqpoint{0.677520in}{1.461357in}}%
\pgfpathlineto{\pgfqpoint{0.671428in}{1.457790in}}%
\pgfpathlineto{\pgfqpoint{0.656870in}{1.447746in}}%
\pgfpathlineto{\pgfqpoint{0.655771in}{1.446907in}}%
\pgfpathlineto{\pgfqpoint{0.641080in}{1.434135in}}%
\pgfpathlineto{\pgfqpoint{0.640115in}{1.433180in}}%
\pgfpathlineto{\pgfqpoint{0.628562in}{1.420524in}}%
\pgfpathlineto{\pgfqpoint{0.624458in}{1.415228in}}%
\pgfpathlineto{\pgfqpoint{0.618495in}{1.406913in}}%
\pgfpathlineto{\pgfqpoint{0.610552in}{1.393302in}}%
\pgfpathlineto{\pgfqpoint{0.608801in}{1.389402in}}%
\pgfpathlineto{\pgfqpoint{0.604667in}{1.379691in}}%
\pgfpathlineto{\pgfqpoint{0.600648in}{1.366079in}}%
\pgfpathlineto{\pgfqpoint{0.598416in}{1.352468in}}%
\pgfpathlineto{\pgfqpoint{0.597970in}{1.338857in}}%
\pgfpathlineto{\pgfqpoint{0.599309in}{1.325246in}}%
\pgfpathlineto{\pgfqpoint{0.602433in}{1.311635in}}%
\pgfpathlineto{\pgfqpoint{0.607348in}{1.298024in}}%
\pgfpathlineto{\pgfqpoint{0.608801in}{1.295050in}}%
\pgfpathlineto{\pgfqpoint{0.614289in}{1.284413in}}%
\pgfpathlineto{\pgfqpoint{0.623170in}{1.270802in}}%
\pgfpathlineto{\pgfqpoint{0.624458in}{1.269148in}}%
\pgfpathlineto{\pgfqpoint{0.634534in}{1.257191in}}%
\pgfpathlineto{\pgfqpoint{0.640115in}{1.251472in}}%
\pgfpathlineto{\pgfqpoint{0.648652in}{1.243579in}}%
\pgfpathlineto{\pgfqpoint{0.655771in}{1.237719in}}%
\pgfpathlineto{\pgfqpoint{0.666502in}{1.229968in}}%
\pgfpathlineto{\pgfqpoint{0.671428in}{1.226718in}}%
\pgfpathlineto{\pgfqpoint{0.687084in}{1.217934in}}%
\pgfpathlineto{\pgfqpoint{0.690525in}{1.216357in}}%
\pgfpathlineto{\pgfqpoint{0.702741in}{1.211111in}}%
\pgfpathlineto{\pgfqpoint{0.718397in}{1.205960in}}%
\pgfpathlineto{\pgfqpoint{0.732514in}{1.202746in}}%
\pgfpathlineto{\pgfqpoint{0.734054in}{1.202410in}}%
\pgfpathclose%
\pgfpathmoveto{\pgfqpoint{0.736451in}{1.270802in}}%
\pgfpathlineto{\pgfqpoint{0.734054in}{1.271363in}}%
\pgfpathlineto{\pgfqpoint{0.718397in}{1.277941in}}%
\pgfpathlineto{\pgfqpoint{0.707712in}{1.284413in}}%
\pgfpathlineto{\pgfqpoint{0.702741in}{1.288173in}}%
\pgfpathlineto{\pgfqpoint{0.692682in}{1.298024in}}%
\pgfpathlineto{\pgfqpoint{0.687084in}{1.305458in}}%
\pgfpathlineto{\pgfqpoint{0.683248in}{1.311635in}}%
\pgfpathlineto{\pgfqpoint{0.677850in}{1.325246in}}%
\pgfpathlineto{\pgfqpoint{0.675538in}{1.338857in}}%
\pgfpathlineto{\pgfqpoint{0.676309in}{1.352468in}}%
\pgfpathlineto{\pgfqpoint{0.680163in}{1.366079in}}%
\pgfpathlineto{\pgfqpoint{0.687084in}{1.379648in}}%
\pgfpathlineto{\pgfqpoint{0.687111in}{1.379691in}}%
\pgfpathlineto{\pgfqpoint{0.699186in}{1.393302in}}%
\pgfpathlineto{\pgfqpoint{0.702741in}{1.396392in}}%
\pgfpathlineto{\pgfqpoint{0.718397in}{1.406890in}}%
\pgfpathlineto{\pgfqpoint{0.718447in}{1.406913in}}%
\pgfpathlineto{\pgfqpoint{0.734054in}{1.412930in}}%
\pgfpathlineto{\pgfqpoint{0.749710in}{1.416281in}}%
\pgfpathlineto{\pgfqpoint{0.765367in}{1.416951in}}%
\pgfpathlineto{\pgfqpoint{0.781024in}{1.414940in}}%
\pgfpathlineto{\pgfqpoint{0.796680in}{1.410248in}}%
\pgfpathlineto{\pgfqpoint{0.803786in}{1.406913in}}%
\pgfpathlineto{\pgfqpoint{0.812337in}{1.402046in}}%
\pgfpathlineto{\pgfqpoint{0.823668in}{1.393302in}}%
\pgfpathlineto{\pgfqpoint{0.827993in}{1.388980in}}%
\pgfpathlineto{\pgfqpoint{0.835437in}{1.379691in}}%
\pgfpathlineto{\pgfqpoint{0.843005in}{1.366079in}}%
\pgfpathlineto{\pgfqpoint{0.843650in}{1.363995in}}%
\pgfpathlineto{\pgfqpoint{0.846675in}{1.352468in}}%
\pgfpathlineto{\pgfqpoint{0.847390in}{1.338857in}}%
\pgfpathlineto{\pgfqpoint{0.845246in}{1.325246in}}%
\pgfpathlineto{\pgfqpoint{0.843650in}{1.320892in}}%
\pgfpathlineto{\pgfqpoint{0.839642in}{1.311635in}}%
\pgfpathlineto{\pgfqpoint{0.830389in}{1.298024in}}%
\pgfpathlineto{\pgfqpoint{0.827993in}{1.295419in}}%
\pgfpathlineto{\pgfqpoint{0.815333in}{1.284413in}}%
\pgfpathlineto{\pgfqpoint{0.812337in}{1.282330in}}%
\pgfpathlineto{\pgfqpoint{0.796680in}{1.274286in}}%
\pgfpathlineto{\pgfqpoint{0.786032in}{1.270802in}}%
\pgfpathlineto{\pgfqpoint{0.781024in}{1.269414in}}%
\pgfpathlineto{\pgfqpoint{0.765367in}{1.267550in}}%
\pgfpathlineto{\pgfqpoint{0.749710in}{1.268172in}}%
\pgfpathlineto{\pgfqpoint{0.736451in}{1.270802in}}%
\pgfpathclose%
\pgfpathmoveto{\pgfqpoint{1.516882in}{1.201263in}}%
\pgfpathlineto{\pgfqpoint{1.532539in}{1.200117in}}%
\pgfpathlineto{\pgfqpoint{1.548195in}{1.200499in}}%
\pgfpathlineto{\pgfqpoint{1.563852in}{1.202410in}}%
\pgfpathlineto{\pgfqpoint{1.565392in}{1.202746in}}%
\pgfpathlineto{\pgfqpoint{1.579508in}{1.205960in}}%
\pgfpathlineto{\pgfqpoint{1.595165in}{1.211111in}}%
\pgfpathlineto{\pgfqpoint{1.607381in}{1.216357in}}%
\pgfpathlineto{\pgfqpoint{1.610822in}{1.217934in}}%
\pgfpathlineto{\pgfqpoint{1.626478in}{1.226718in}}%
\pgfpathlineto{\pgfqpoint{1.631404in}{1.229968in}}%
\pgfpathlineto{\pgfqpoint{1.642135in}{1.237719in}}%
\pgfpathlineto{\pgfqpoint{1.649254in}{1.243579in}}%
\pgfpathlineto{\pgfqpoint{1.657791in}{1.251472in}}%
\pgfpathlineto{\pgfqpoint{1.663372in}{1.257191in}}%
\pgfpathlineto{\pgfqpoint{1.673448in}{1.269148in}}%
\pgfpathlineto{\pgfqpoint{1.674736in}{1.270802in}}%
\pgfpathlineto{\pgfqpoint{1.683617in}{1.284413in}}%
\pgfpathlineto{\pgfqpoint{1.689104in}{1.295050in}}%
\pgfpathlineto{\pgfqpoint{1.690558in}{1.298024in}}%
\pgfpathlineto{\pgfqpoint{1.695472in}{1.311635in}}%
\pgfpathlineto{\pgfqpoint{1.698597in}{1.325246in}}%
\pgfpathlineto{\pgfqpoint{1.699936in}{1.338857in}}%
\pgfpathlineto{\pgfqpoint{1.699490in}{1.352468in}}%
\pgfpathlineto{\pgfqpoint{1.697258in}{1.366079in}}%
\pgfpathlineto{\pgfqpoint{1.693239in}{1.379691in}}%
\pgfpathlineto{\pgfqpoint{1.689104in}{1.389402in}}%
\pgfpathlineto{\pgfqpoint{1.687354in}{1.393302in}}%
\pgfpathlineto{\pgfqpoint{1.679411in}{1.406913in}}%
\pgfpathlineto{\pgfqpoint{1.673448in}{1.415228in}}%
\pgfpathlineto{\pgfqpoint{1.669344in}{1.420524in}}%
\pgfpathlineto{\pgfqpoint{1.657791in}{1.433180in}}%
\pgfpathlineto{\pgfqpoint{1.656826in}{1.434135in}}%
\pgfpathlineto{\pgfqpoint{1.642135in}{1.446907in}}%
\pgfpathlineto{\pgfqpoint{1.641036in}{1.447746in}}%
\pgfpathlineto{\pgfqpoint{1.626478in}{1.457790in}}%
\pgfpathlineto{\pgfqpoint{1.620386in}{1.461357in}}%
\pgfpathlineto{\pgfqpoint{1.610822in}{1.466541in}}%
\pgfpathlineto{\pgfqpoint{1.595165in}{1.473447in}}%
\pgfpathlineto{\pgfqpoint{1.590679in}{1.474968in}}%
\pgfpathlineto{\pgfqpoint{1.579508in}{1.478563in}}%
\pgfpathlineto{\pgfqpoint{1.563852in}{1.482057in}}%
\pgfpathlineto{\pgfqpoint{1.548195in}{1.483997in}}%
\pgfpathlineto{\pgfqpoint{1.532539in}{1.484385in}}%
\pgfpathlineto{\pgfqpoint{1.516882in}{1.483221in}}%
\pgfpathlineto{\pgfqpoint{1.501226in}{1.480504in}}%
\pgfpathlineto{\pgfqpoint{1.485569in}{1.476232in}}%
\pgfpathlineto{\pgfqpoint{1.482148in}{1.474968in}}%
\pgfpathlineto{\pgfqpoint{1.469913in}{1.470198in}}%
\pgfpathlineto{\pgfqpoint{1.454256in}{1.462477in}}%
\pgfpathlineto{\pgfqpoint{1.452354in}{1.461357in}}%
\pgfpathlineto{\pgfqpoint{1.438599in}{1.452598in}}%
\pgfpathlineto{\pgfqpoint{1.432022in}{1.447746in}}%
\pgfpathlineto{\pgfqpoint{1.422943in}{1.440324in}}%
\pgfpathlineto{\pgfqpoint{1.416202in}{1.434135in}}%
\pgfpathlineto{\pgfqpoint{1.407286in}{1.424806in}}%
\pgfpathlineto{\pgfqpoint{1.403547in}{1.420524in}}%
\pgfpathlineto{\pgfqpoint{1.393443in}{1.406913in}}%
\pgfpathlineto{\pgfqpoint{1.391630in}{1.403921in}}%
\pgfpathlineto{\pgfqpoint{1.385595in}{1.393302in}}%
\pgfpathlineto{\pgfqpoint{1.379670in}{1.379691in}}%
\pgfpathlineto{\pgfqpoint{1.375973in}{1.367418in}}%
\pgfpathlineto{\pgfqpoint{1.375586in}{1.366079in}}%
\pgfpathlineto{\pgfqpoint{1.373388in}{1.352468in}}%
\pgfpathlineto{\pgfqpoint{1.372949in}{1.338857in}}%
\pgfpathlineto{\pgfqpoint{1.374267in}{1.325246in}}%
\pgfpathlineto{\pgfqpoint{1.375973in}{1.317685in}}%
\pgfpathlineto{\pgfqpoint{1.377393in}{1.311635in}}%
\pgfpathlineto{\pgfqpoint{1.382404in}{1.298024in}}%
\pgfpathlineto{\pgfqpoint{1.389244in}{1.284413in}}%
\pgfpathlineto{\pgfqpoint{1.391630in}{1.280631in}}%
\pgfpathlineto{\pgfqpoint{1.398254in}{1.270802in}}%
\pgfpathlineto{\pgfqpoint{1.407286in}{1.259661in}}%
\pgfpathlineto{\pgfqpoint{1.409475in}{1.257191in}}%
\pgfpathlineto{\pgfqpoint{1.422943in}{1.244056in}}%
\pgfpathlineto{\pgfqpoint{1.423491in}{1.243579in}}%
\pgfpathlineto{\pgfqpoint{1.438599in}{1.231872in}}%
\pgfpathlineto{\pgfqpoint{1.441441in}{1.229968in}}%
\pgfpathlineto{\pgfqpoint{1.454256in}{1.222116in}}%
\pgfpathlineto{\pgfqpoint{1.465562in}{1.216357in}}%
\pgfpathlineto{\pgfqpoint{1.469913in}{1.214283in}}%
\pgfpathlineto{\pgfqpoint{1.485569in}{1.208337in}}%
\pgfpathlineto{\pgfqpoint{1.501226in}{1.203981in}}%
\pgfpathlineto{\pgfqpoint{1.508185in}{1.202746in}}%
\pgfpathlineto{\pgfqpoint{1.516882in}{1.201263in}}%
\pgfpathclose%
\pgfpathmoveto{\pgfqpoint{1.511873in}{1.270802in}}%
\pgfpathlineto{\pgfqpoint{1.501226in}{1.274286in}}%
\pgfpathlineto{\pgfqpoint{1.485569in}{1.282330in}}%
\pgfpathlineto{\pgfqpoint{1.482573in}{1.284413in}}%
\pgfpathlineto{\pgfqpoint{1.469913in}{1.295419in}}%
\pgfpathlineto{\pgfqpoint{1.467517in}{1.298024in}}%
\pgfpathlineto{\pgfqpoint{1.458264in}{1.311635in}}%
\pgfpathlineto{\pgfqpoint{1.454256in}{1.320892in}}%
\pgfpathlineto{\pgfqpoint{1.452660in}{1.325246in}}%
\pgfpathlineto{\pgfqpoint{1.450516in}{1.338857in}}%
\pgfpathlineto{\pgfqpoint{1.451231in}{1.352468in}}%
\pgfpathlineto{\pgfqpoint{1.454256in}{1.363995in}}%
\pgfpathlineto{\pgfqpoint{1.454901in}{1.366079in}}%
\pgfpathlineto{\pgfqpoint{1.462468in}{1.379691in}}%
\pgfpathlineto{\pgfqpoint{1.469913in}{1.388980in}}%
\pgfpathlineto{\pgfqpoint{1.474238in}{1.393302in}}%
\pgfpathlineto{\pgfqpoint{1.485569in}{1.402046in}}%
\pgfpathlineto{\pgfqpoint{1.494120in}{1.406913in}}%
\pgfpathlineto{\pgfqpoint{1.501226in}{1.410248in}}%
\pgfpathlineto{\pgfqpoint{1.516882in}{1.414940in}}%
\pgfpathlineto{\pgfqpoint{1.532539in}{1.416951in}}%
\pgfpathlineto{\pgfqpoint{1.548195in}{1.416281in}}%
\pgfpathlineto{\pgfqpoint{1.563852in}{1.412930in}}%
\pgfpathlineto{\pgfqpoint{1.579459in}{1.406913in}}%
\pgfpathlineto{\pgfqpoint{1.579508in}{1.406890in}}%
\pgfpathlineto{\pgfqpoint{1.595165in}{1.396392in}}%
\pgfpathlineto{\pgfqpoint{1.598720in}{1.393302in}}%
\pgfpathlineto{\pgfqpoint{1.610795in}{1.379691in}}%
\pgfpathlineto{\pgfqpoint{1.610822in}{1.379648in}}%
\pgfpathlineto{\pgfqpoint{1.617743in}{1.366079in}}%
\pgfpathlineto{\pgfqpoint{1.621597in}{1.352468in}}%
\pgfpathlineto{\pgfqpoint{1.622368in}{1.338857in}}%
\pgfpathlineto{\pgfqpoint{1.620056in}{1.325246in}}%
\pgfpathlineto{\pgfqpoint{1.614658in}{1.311635in}}%
\pgfpathlineto{\pgfqpoint{1.610822in}{1.305458in}}%
\pgfpathlineto{\pgfqpoint{1.605223in}{1.298024in}}%
\pgfpathlineto{\pgfqpoint{1.595165in}{1.288173in}}%
\pgfpathlineto{\pgfqpoint{1.590194in}{1.284413in}}%
\pgfpathlineto{\pgfqpoint{1.579508in}{1.277941in}}%
\pgfpathlineto{\pgfqpoint{1.563852in}{1.271363in}}%
\pgfpathlineto{\pgfqpoint{1.561454in}{1.270802in}}%
\pgfpathlineto{\pgfqpoint{1.548195in}{1.268172in}}%
\pgfpathlineto{\pgfqpoint{1.532539in}{1.267550in}}%
\pgfpathlineto{\pgfqpoint{1.516882in}{1.269414in}}%
\pgfpathlineto{\pgfqpoint{1.511873in}{1.270802in}}%
\pgfpathclose%
\pgfusepath{fill}%
\end{pgfscope}%
\begin{pgfscope}%
\pgfpathrectangle{\pgfqpoint{0.373953in}{0.331635in}}{\pgfqpoint{1.550000in}{1.347500in}}%
\pgfusepath{clip}%
\pgfsetbuttcap%
\pgfsetroundjoin%
\definecolor{currentfill}{rgb}{0.642368,0.801830,0.890319}%
\pgfsetfillcolor{currentfill}%
\pgfsetlinewidth{0.000000pt}%
\definecolor{currentstroke}{rgb}{0.000000,0.000000,0.000000}%
\pgfsetstrokecolor{currentstroke}%
\pgfsetdash{}{0pt}%
\pgfpathmoveto{\pgfqpoint{0.702741in}{0.478963in}}%
\pgfpathlineto{\pgfqpoint{0.718397in}{0.474497in}}%
\pgfpathlineto{\pgfqpoint{0.734054in}{0.471408in}}%
\pgfpathlineto{\pgfqpoint{0.749710in}{0.469693in}}%
\pgfpathlineto{\pgfqpoint{0.765367in}{0.469351in}}%
\pgfpathlineto{\pgfqpoint{0.781024in}{0.470379in}}%
\pgfpathlineto{\pgfqpoint{0.796680in}{0.472781in}}%
\pgfpathlineto{\pgfqpoint{0.812337in}{0.476558in}}%
\pgfpathlineto{\pgfqpoint{0.826922in}{0.481357in}}%
\pgfpathlineto{\pgfqpoint{0.827993in}{0.481706in}}%
\pgfpathlineto{\pgfqpoint{0.843650in}{0.488092in}}%
\pgfpathlineto{\pgfqpoint{0.857594in}{0.494968in}}%
\pgfpathlineto{\pgfqpoint{0.859306in}{0.495821in}}%
\pgfpathlineto{\pgfqpoint{0.874963in}{0.504833in}}%
\pgfpathlineto{\pgfqpoint{0.880719in}{0.508579in}}%
\pgfpathlineto{\pgfqpoint{0.890620in}{0.515222in}}%
\pgfpathlineto{\pgfqpoint{0.900018in}{0.522191in}}%
\pgfpathlineto{\pgfqpoint{0.906276in}{0.527064in}}%
\pgfpathlineto{\pgfqpoint{0.916651in}{0.535802in}}%
\pgfpathlineto{\pgfqpoint{0.921933in}{0.540565in}}%
\pgfpathlineto{\pgfqpoint{0.931186in}{0.549413in}}%
\pgfpathlineto{\pgfqpoint{0.937589in}{0.556105in}}%
\pgfpathlineto{\pgfqpoint{0.943952in}{0.563024in}}%
\pgfpathlineto{\pgfqpoint{0.953246in}{0.574325in}}%
\pgfpathlineto{\pgfqpoint{0.955106in}{0.576635in}}%
\pgfpathlineto{\pgfqpoint{0.964722in}{0.590246in}}%
\pgfpathlineto{\pgfqpoint{0.968902in}{0.597238in}}%
\pgfpathlineto{\pgfqpoint{0.972859in}{0.603857in}}%
\pgfpathlineto{\pgfqpoint{0.979492in}{0.617468in}}%
\pgfpathlineto{\pgfqpoint{0.984559in}{0.631077in}}%
\pgfpathlineto{\pgfqpoint{0.984560in}{0.631079in}}%
\pgfpathlineto{\pgfqpoint{0.988177in}{0.644691in}}%
\pgfpathlineto{\pgfqpoint{0.990185in}{0.658302in}}%
\pgfpathlineto{\pgfqpoint{0.990587in}{0.671913in}}%
\pgfpathlineto{\pgfqpoint{0.989382in}{0.685524in}}%
\pgfpathlineto{\pgfqpoint{0.986570in}{0.699135in}}%
\pgfpathlineto{\pgfqpoint{0.984559in}{0.705350in}}%
\pgfpathlineto{\pgfqpoint{0.982222in}{0.712746in}}%
\pgfpathlineto{\pgfqpoint{0.976371in}{0.726357in}}%
\pgfpathlineto{\pgfqpoint{0.968955in}{0.739968in}}%
\pgfpathlineto{\pgfqpoint{0.968902in}{0.740049in}}%
\pgfpathlineto{\pgfqpoint{0.960106in}{0.753579in}}%
\pgfpathlineto{\pgfqpoint{0.953246in}{0.762630in}}%
\pgfpathlineto{\pgfqpoint{0.949720in}{0.767191in}}%
\pgfpathlineto{\pgfqpoint{0.937810in}{0.780802in}}%
\pgfpathlineto{\pgfqpoint{0.937589in}{0.781031in}}%
\pgfpathlineto{\pgfqpoint{0.924189in}{0.794413in}}%
\pgfpathlineto{\pgfqpoint{0.921933in}{0.796494in}}%
\pgfpathlineto{\pgfqpoint{0.908670in}{0.808024in}}%
\pgfpathlineto{\pgfqpoint{0.906276in}{0.809985in}}%
\pgfpathlineto{\pgfqpoint{0.890884in}{0.821635in}}%
\pgfpathlineto{\pgfqpoint{0.890620in}{0.821827in}}%
\pgfpathlineto{\pgfqpoint{0.874963in}{0.832181in}}%
\pgfpathlineto{\pgfqpoint{0.869717in}{0.835246in}}%
\pgfpathlineto{\pgfqpoint{0.859306in}{0.841210in}}%
\pgfpathlineto{\pgfqpoint{0.843743in}{0.848857in}}%
\pgfpathlineto{\pgfqpoint{0.843650in}{0.848903in}}%
\pgfpathlineto{\pgfqpoint{0.827993in}{0.855350in}}%
\pgfpathlineto{\pgfqpoint{0.812337in}{0.860436in}}%
\pgfpathlineto{\pgfqpoint{0.803829in}{0.862468in}}%
\pgfpathlineto{\pgfqpoint{0.796680in}{0.864216in}}%
\pgfpathlineto{\pgfqpoint{0.781024in}{0.866661in}}%
\pgfpathlineto{\pgfqpoint{0.765367in}{0.867709in}}%
\pgfpathlineto{\pgfqpoint{0.749710in}{0.867360in}}%
\pgfpathlineto{\pgfqpoint{0.734054in}{0.865614in}}%
\pgfpathlineto{\pgfqpoint{0.718397in}{0.862469in}}%
\pgfpathlineto{\pgfqpoint{0.718395in}{0.862468in}}%
\pgfpathlineto{\pgfqpoint{0.702741in}{0.858063in}}%
\pgfpathlineto{\pgfqpoint{0.687084in}{0.852297in}}%
\pgfpathlineto{\pgfqpoint{0.679471in}{0.848857in}}%
\pgfpathlineto{\pgfqpoint{0.671428in}{0.845223in}}%
\pgfpathlineto{\pgfqpoint{0.655771in}{0.836863in}}%
\pgfpathlineto{\pgfqpoint{0.653114in}{0.835246in}}%
\pgfpathlineto{\pgfqpoint{0.640115in}{0.827167in}}%
\pgfpathlineto{\pgfqpoint{0.632155in}{0.821635in}}%
\pgfpathlineto{\pgfqpoint{0.624458in}{0.816068in}}%
\pgfpathlineto{\pgfqpoint{0.614280in}{0.808024in}}%
\pgfpathlineto{\pgfqpoint{0.608801in}{0.803432in}}%
\pgfpathlineto{\pgfqpoint{0.598750in}{0.794413in}}%
\pgfpathlineto{\pgfqpoint{0.593145in}{0.788973in}}%
\pgfpathlineto{\pgfqpoint{0.585129in}{0.780802in}}%
\pgfpathlineto{\pgfqpoint{0.577488in}{0.772195in}}%
\pgfpathlineto{\pgfqpoint{0.573179in}{0.767191in}}%
\pgfpathlineto{\pgfqpoint{0.562812in}{0.753579in}}%
\pgfpathlineto{\pgfqpoint{0.561832in}{0.752091in}}%
\pgfpathlineto{\pgfqpoint{0.553922in}{0.739968in}}%
\pgfpathlineto{\pgfqpoint{0.546576in}{0.726357in}}%
\pgfpathlineto{\pgfqpoint{0.546175in}{0.725426in}}%
\pgfpathlineto{\pgfqpoint{0.540655in}{0.712746in}}%
\pgfpathlineto{\pgfqpoint{0.536310in}{0.699135in}}%
\pgfpathlineto{\pgfqpoint{0.533548in}{0.685524in}}%
\pgfpathlineto{\pgfqpoint{0.532364in}{0.671913in}}%
\pgfpathlineto{\pgfqpoint{0.532759in}{0.658302in}}%
\pgfpathlineto{\pgfqpoint{0.534731in}{0.644691in}}%
\pgfpathlineto{\pgfqpoint{0.538284in}{0.631079in}}%
\pgfpathlineto{\pgfqpoint{0.543422in}{0.617468in}}%
\pgfpathlineto{\pgfqpoint{0.546175in}{0.611854in}}%
\pgfpathlineto{\pgfqpoint{0.550055in}{0.603857in}}%
\pgfpathlineto{\pgfqpoint{0.558177in}{0.590246in}}%
\pgfpathlineto{\pgfqpoint{0.561832in}{0.585045in}}%
\pgfpathlineto{\pgfqpoint{0.567805in}{0.576635in}}%
\pgfpathlineto{\pgfqpoint{0.577488in}{0.564759in}}%
\pgfpathlineto{\pgfqpoint{0.578945in}{0.563024in}}%
\pgfpathlineto{\pgfqpoint{0.591685in}{0.549413in}}%
\pgfpathlineto{\pgfqpoint{0.593145in}{0.547983in}}%
\pgfpathlineto{\pgfqpoint{0.606221in}{0.535802in}}%
\pgfpathlineto{\pgfqpoint{0.608801in}{0.533558in}}%
\pgfpathlineto{\pgfqpoint{0.622813in}{0.522191in}}%
\pgfpathlineto{\pgfqpoint{0.624458in}{0.520921in}}%
\pgfpathlineto{\pgfqpoint{0.640115in}{0.509846in}}%
\pgfpathlineto{\pgfqpoint{0.642110in}{0.508579in}}%
\pgfpathlineto{\pgfqpoint{0.655771in}{0.500162in}}%
\pgfpathlineto{\pgfqpoint{0.665445in}{0.494968in}}%
\pgfpathlineto{\pgfqpoint{0.671428in}{0.491791in}}%
\pgfpathlineto{\pgfqpoint{0.687084in}{0.484730in}}%
\pgfpathlineto{\pgfqpoint{0.696283in}{0.481357in}}%
\pgfpathlineto{\pgfqpoint{0.702741in}{0.478963in}}%
\pgfpathclose%
\pgfpathmoveto{\pgfqpoint{0.707227in}{0.535802in}}%
\pgfpathlineto{\pgfqpoint{0.702741in}{0.537323in}}%
\pgfpathlineto{\pgfqpoint{0.687084in}{0.544229in}}%
\pgfpathlineto{\pgfqpoint{0.677520in}{0.549413in}}%
\pgfpathlineto{\pgfqpoint{0.671428in}{0.552980in}}%
\pgfpathlineto{\pgfqpoint{0.656870in}{0.563024in}}%
\pgfpathlineto{\pgfqpoint{0.655771in}{0.563863in}}%
\pgfpathlineto{\pgfqpoint{0.641080in}{0.576635in}}%
\pgfpathlineto{\pgfqpoint{0.640115in}{0.577590in}}%
\pgfpathlineto{\pgfqpoint{0.628562in}{0.590246in}}%
\pgfpathlineto{\pgfqpoint{0.624458in}{0.595542in}}%
\pgfpathlineto{\pgfqpoint{0.618495in}{0.603857in}}%
\pgfpathlineto{\pgfqpoint{0.610552in}{0.617468in}}%
\pgfpathlineto{\pgfqpoint{0.608801in}{0.621368in}}%
\pgfpathlineto{\pgfqpoint{0.604667in}{0.631079in}}%
\pgfpathlineto{\pgfqpoint{0.600648in}{0.644691in}}%
\pgfpathlineto{\pgfqpoint{0.598416in}{0.658302in}}%
\pgfpathlineto{\pgfqpoint{0.597970in}{0.671913in}}%
\pgfpathlineto{\pgfqpoint{0.599309in}{0.685524in}}%
\pgfpathlineto{\pgfqpoint{0.602433in}{0.699135in}}%
\pgfpathlineto{\pgfqpoint{0.607348in}{0.712746in}}%
\pgfpathlineto{\pgfqpoint{0.608801in}{0.715720in}}%
\pgfpathlineto{\pgfqpoint{0.614289in}{0.726357in}}%
\pgfpathlineto{\pgfqpoint{0.623170in}{0.739968in}}%
\pgfpathlineto{\pgfqpoint{0.624458in}{0.741622in}}%
\pgfpathlineto{\pgfqpoint{0.634534in}{0.753579in}}%
\pgfpathlineto{\pgfqpoint{0.640115in}{0.759298in}}%
\pgfpathlineto{\pgfqpoint{0.648652in}{0.767191in}}%
\pgfpathlineto{\pgfqpoint{0.655771in}{0.773051in}}%
\pgfpathlineto{\pgfqpoint{0.666502in}{0.780802in}}%
\pgfpathlineto{\pgfqpoint{0.671428in}{0.784052in}}%
\pgfpathlineto{\pgfqpoint{0.687084in}{0.792836in}}%
\pgfpathlineto{\pgfqpoint{0.690525in}{0.794413in}}%
\pgfpathlineto{\pgfqpoint{0.702741in}{0.799659in}}%
\pgfpathlineto{\pgfqpoint{0.718397in}{0.804810in}}%
\pgfpathlineto{\pgfqpoint{0.732514in}{0.808024in}}%
\pgfpathlineto{\pgfqpoint{0.734054in}{0.808360in}}%
\pgfpathlineto{\pgfqpoint{0.749710in}{0.810271in}}%
\pgfpathlineto{\pgfqpoint{0.765367in}{0.810653in}}%
\pgfpathlineto{\pgfqpoint{0.781024in}{0.809507in}}%
\pgfpathlineto{\pgfqpoint{0.789721in}{0.808024in}}%
\pgfpathlineto{\pgfqpoint{0.796680in}{0.806789in}}%
\pgfpathlineto{\pgfqpoint{0.812337in}{0.802433in}}%
\pgfpathlineto{\pgfqpoint{0.827993in}{0.796487in}}%
\pgfpathlineto{\pgfqpoint{0.832344in}{0.794413in}}%
\pgfpathlineto{\pgfqpoint{0.843650in}{0.788654in}}%
\pgfpathlineto{\pgfqpoint{0.856465in}{0.780802in}}%
\pgfpathlineto{\pgfqpoint{0.859306in}{0.778898in}}%
\pgfpathlineto{\pgfqpoint{0.874415in}{0.767191in}}%
\pgfpathlineto{\pgfqpoint{0.874963in}{0.766714in}}%
\pgfpathlineto{\pgfqpoint{0.888430in}{0.753579in}}%
\pgfpathlineto{\pgfqpoint{0.890620in}{0.751109in}}%
\pgfpathlineto{\pgfqpoint{0.899652in}{0.739968in}}%
\pgfpathlineto{\pgfqpoint{0.906276in}{0.730139in}}%
\pgfpathlineto{\pgfqpoint{0.908662in}{0.726357in}}%
\pgfpathlineto{\pgfqpoint{0.915502in}{0.712746in}}%
\pgfpathlineto{\pgfqpoint{0.920512in}{0.699135in}}%
\pgfpathlineto{\pgfqpoint{0.921933in}{0.693085in}}%
\pgfpathlineto{\pgfqpoint{0.923639in}{0.685524in}}%
\pgfpathlineto{\pgfqpoint{0.924957in}{0.671913in}}%
\pgfpathlineto{\pgfqpoint{0.924518in}{0.658302in}}%
\pgfpathlineto{\pgfqpoint{0.922320in}{0.644691in}}%
\pgfpathlineto{\pgfqpoint{0.921933in}{0.643352in}}%
\pgfpathlineto{\pgfqpoint{0.918235in}{0.631079in}}%
\pgfpathlineto{\pgfqpoint{0.912311in}{0.617468in}}%
\pgfpathlineto{\pgfqpoint{0.906276in}{0.606849in}}%
\pgfpathlineto{\pgfqpoint{0.904462in}{0.603857in}}%
\pgfpathlineto{\pgfqpoint{0.894358in}{0.590246in}}%
\pgfpathlineto{\pgfqpoint{0.890620in}{0.585964in}}%
\pgfpathlineto{\pgfqpoint{0.881704in}{0.576635in}}%
\pgfpathlineto{\pgfqpoint{0.874963in}{0.570446in}}%
\pgfpathlineto{\pgfqpoint{0.865884in}{0.563024in}}%
\pgfpathlineto{\pgfqpoint{0.859306in}{0.558172in}}%
\pgfpathlineto{\pgfqpoint{0.845552in}{0.549413in}}%
\pgfpathlineto{\pgfqpoint{0.843650in}{0.548293in}}%
\pgfpathlineto{\pgfqpoint{0.827993in}{0.540572in}}%
\pgfpathlineto{\pgfqpoint{0.815758in}{0.535802in}}%
\pgfpathlineto{\pgfqpoint{0.812337in}{0.534538in}}%
\pgfpathlineto{\pgfqpoint{0.796680in}{0.530266in}}%
\pgfpathlineto{\pgfqpoint{0.781024in}{0.527549in}}%
\pgfpathlineto{\pgfqpoint{0.765367in}{0.526385in}}%
\pgfpathlineto{\pgfqpoint{0.749710in}{0.526773in}}%
\pgfpathlineto{\pgfqpoint{0.734054in}{0.528713in}}%
\pgfpathlineto{\pgfqpoint{0.718397in}{0.532207in}}%
\pgfpathlineto{\pgfqpoint{0.707227in}{0.535802in}}%
\pgfpathclose%
\pgfpathmoveto{\pgfqpoint{1.485569in}{0.476558in}}%
\pgfpathlineto{\pgfqpoint{1.501226in}{0.472781in}}%
\pgfpathlineto{\pgfqpoint{1.516882in}{0.470379in}}%
\pgfpathlineto{\pgfqpoint{1.532539in}{0.469351in}}%
\pgfpathlineto{\pgfqpoint{1.548195in}{0.469693in}}%
\pgfpathlineto{\pgfqpoint{1.563852in}{0.471408in}}%
\pgfpathlineto{\pgfqpoint{1.579508in}{0.474497in}}%
\pgfpathlineto{\pgfqpoint{1.595165in}{0.478963in}}%
\pgfpathlineto{\pgfqpoint{1.601623in}{0.481357in}}%
\pgfpathlineto{\pgfqpoint{1.610822in}{0.484730in}}%
\pgfpathlineto{\pgfqpoint{1.626478in}{0.491791in}}%
\pgfpathlineto{\pgfqpoint{1.632461in}{0.494968in}}%
\pgfpathlineto{\pgfqpoint{1.642135in}{0.500162in}}%
\pgfpathlineto{\pgfqpoint{1.655796in}{0.508579in}}%
\pgfpathlineto{\pgfqpoint{1.657791in}{0.509846in}}%
\pgfpathlineto{\pgfqpoint{1.673448in}{0.520921in}}%
\pgfpathlineto{\pgfqpoint{1.675092in}{0.522191in}}%
\pgfpathlineto{\pgfqpoint{1.689104in}{0.533558in}}%
\pgfpathlineto{\pgfqpoint{1.691685in}{0.535802in}}%
\pgfpathlineto{\pgfqpoint{1.704761in}{0.547983in}}%
\pgfpathlineto{\pgfqpoint{1.706221in}{0.549413in}}%
\pgfpathlineto{\pgfqpoint{1.718961in}{0.563024in}}%
\pgfpathlineto{\pgfqpoint{1.720418in}{0.564759in}}%
\pgfpathlineto{\pgfqpoint{1.730100in}{0.576635in}}%
\pgfpathlineto{\pgfqpoint{1.736074in}{0.585045in}}%
\pgfpathlineto{\pgfqpoint{1.739729in}{0.590246in}}%
\pgfpathlineto{\pgfqpoint{1.747851in}{0.603857in}}%
\pgfpathlineto{\pgfqpoint{1.751731in}{0.611854in}}%
\pgfpathlineto{\pgfqpoint{1.754484in}{0.617468in}}%
\pgfpathlineto{\pgfqpoint{1.759621in}{0.631079in}}%
\pgfpathlineto{\pgfqpoint{1.763175in}{0.644691in}}%
\pgfpathlineto{\pgfqpoint{1.765147in}{0.658302in}}%
\pgfpathlineto{\pgfqpoint{1.765542in}{0.671913in}}%
\pgfpathlineto{\pgfqpoint{1.764358in}{0.685524in}}%
\pgfpathlineto{\pgfqpoint{1.761596in}{0.699135in}}%
\pgfpathlineto{\pgfqpoint{1.757251in}{0.712746in}}%
\pgfpathlineto{\pgfqpoint{1.751731in}{0.725426in}}%
\pgfpathlineto{\pgfqpoint{1.751330in}{0.726357in}}%
\pgfpathlineto{\pgfqpoint{1.743983in}{0.739968in}}%
\pgfpathlineto{\pgfqpoint{1.736074in}{0.752091in}}%
\pgfpathlineto{\pgfqpoint{1.735094in}{0.753579in}}%
\pgfpathlineto{\pgfqpoint{1.724726in}{0.767191in}}%
\pgfpathlineto{\pgfqpoint{1.720418in}{0.772195in}}%
\pgfpathlineto{\pgfqpoint{1.712777in}{0.780802in}}%
\pgfpathlineto{\pgfqpoint{1.704761in}{0.788973in}}%
\pgfpathlineto{\pgfqpoint{1.699156in}{0.794413in}}%
\pgfpathlineto{\pgfqpoint{1.689104in}{0.803432in}}%
\pgfpathlineto{\pgfqpoint{1.683625in}{0.808024in}}%
\pgfpathlineto{\pgfqpoint{1.673448in}{0.816068in}}%
\pgfpathlineto{\pgfqpoint{1.665750in}{0.821635in}}%
\pgfpathlineto{\pgfqpoint{1.657791in}{0.827167in}}%
\pgfpathlineto{\pgfqpoint{1.644792in}{0.835246in}}%
\pgfpathlineto{\pgfqpoint{1.642135in}{0.836863in}}%
\pgfpathlineto{\pgfqpoint{1.626478in}{0.845223in}}%
\pgfpathlineto{\pgfqpoint{1.618435in}{0.848857in}}%
\pgfpathlineto{\pgfqpoint{1.610822in}{0.852297in}}%
\pgfpathlineto{\pgfqpoint{1.595165in}{0.858063in}}%
\pgfpathlineto{\pgfqpoint{1.579511in}{0.862468in}}%
\pgfpathlineto{\pgfqpoint{1.579508in}{0.862469in}}%
\pgfpathlineto{\pgfqpoint{1.563852in}{0.865614in}}%
\pgfpathlineto{\pgfqpoint{1.548195in}{0.867360in}}%
\pgfpathlineto{\pgfqpoint{1.532539in}{0.867709in}}%
\pgfpathlineto{\pgfqpoint{1.516882in}{0.866661in}}%
\pgfpathlineto{\pgfqpoint{1.501226in}{0.864216in}}%
\pgfpathlineto{\pgfqpoint{1.494077in}{0.862468in}}%
\pgfpathlineto{\pgfqpoint{1.485569in}{0.860436in}}%
\pgfpathlineto{\pgfqpoint{1.469913in}{0.855350in}}%
\pgfpathlineto{\pgfqpoint{1.454256in}{0.848903in}}%
\pgfpathlineto{\pgfqpoint{1.454163in}{0.848857in}}%
\pgfpathlineto{\pgfqpoint{1.438599in}{0.841210in}}%
\pgfpathlineto{\pgfqpoint{1.428188in}{0.835246in}}%
\pgfpathlineto{\pgfqpoint{1.422943in}{0.832181in}}%
\pgfpathlineto{\pgfqpoint{1.407286in}{0.821827in}}%
\pgfpathlineto{\pgfqpoint{1.407022in}{0.821635in}}%
\pgfpathlineto{\pgfqpoint{1.391630in}{0.809985in}}%
\pgfpathlineto{\pgfqpoint{1.389236in}{0.808024in}}%
\pgfpathlineto{\pgfqpoint{1.375973in}{0.796494in}}%
\pgfpathlineto{\pgfqpoint{1.373717in}{0.794413in}}%
\pgfpathlineto{\pgfqpoint{1.360317in}{0.781031in}}%
\pgfpathlineto{\pgfqpoint{1.360096in}{0.780802in}}%
\pgfpathlineto{\pgfqpoint{1.348186in}{0.767191in}}%
\pgfpathlineto{\pgfqpoint{1.344660in}{0.762630in}}%
\pgfpathlineto{\pgfqpoint{1.337800in}{0.753579in}}%
\pgfpathlineto{\pgfqpoint{1.329003in}{0.740049in}}%
\pgfpathlineto{\pgfqpoint{1.328951in}{0.739968in}}%
\pgfpathlineto{\pgfqpoint{1.321534in}{0.726357in}}%
\pgfpathlineto{\pgfqpoint{1.315684in}{0.712746in}}%
\pgfpathlineto{\pgfqpoint{1.313347in}{0.705350in}}%
\pgfpathlineto{\pgfqpoint{1.311336in}{0.699135in}}%
\pgfpathlineto{\pgfqpoint{1.308524in}{0.685524in}}%
\pgfpathlineto{\pgfqpoint{1.307319in}{0.671913in}}%
\pgfpathlineto{\pgfqpoint{1.307720in}{0.658302in}}%
\pgfpathlineto{\pgfqpoint{1.309729in}{0.644691in}}%
\pgfpathlineto{\pgfqpoint{1.313346in}{0.631079in}}%
\pgfpathlineto{\pgfqpoint{1.313347in}{0.631077in}}%
\pgfpathlineto{\pgfqpoint{1.318414in}{0.617468in}}%
\pgfpathlineto{\pgfqpoint{1.325047in}{0.603857in}}%
\pgfpathlineto{\pgfqpoint{1.329003in}{0.597238in}}%
\pgfpathlineto{\pgfqpoint{1.333184in}{0.590246in}}%
\pgfpathlineto{\pgfqpoint{1.342800in}{0.576635in}}%
\pgfpathlineto{\pgfqpoint{1.344660in}{0.574325in}}%
\pgfpathlineto{\pgfqpoint{1.353954in}{0.563024in}}%
\pgfpathlineto{\pgfqpoint{1.360317in}{0.556105in}}%
\pgfpathlineto{\pgfqpoint{1.366720in}{0.549413in}}%
\pgfpathlineto{\pgfqpoint{1.375973in}{0.540565in}}%
\pgfpathlineto{\pgfqpoint{1.381255in}{0.535802in}}%
\pgfpathlineto{\pgfqpoint{1.391630in}{0.527064in}}%
\pgfpathlineto{\pgfqpoint{1.397887in}{0.522191in}}%
\pgfpathlineto{\pgfqpoint{1.407286in}{0.515222in}}%
\pgfpathlineto{\pgfqpoint{1.417187in}{0.508579in}}%
\pgfpathlineto{\pgfqpoint{1.422943in}{0.504833in}}%
\pgfpathlineto{\pgfqpoint{1.438599in}{0.495821in}}%
\pgfpathlineto{\pgfqpoint{1.440312in}{0.494968in}}%
\pgfpathlineto{\pgfqpoint{1.454256in}{0.488092in}}%
\pgfpathlineto{\pgfqpoint{1.469913in}{0.481706in}}%
\pgfpathlineto{\pgfqpoint{1.470984in}{0.481357in}}%
\pgfpathlineto{\pgfqpoint{1.485569in}{0.476558in}}%
\pgfpathclose%
\pgfpathmoveto{\pgfqpoint{1.482148in}{0.535802in}}%
\pgfpathlineto{\pgfqpoint{1.469913in}{0.540572in}}%
\pgfpathlineto{\pgfqpoint{1.454256in}{0.548293in}}%
\pgfpathlineto{\pgfqpoint{1.452354in}{0.549413in}}%
\pgfpathlineto{\pgfqpoint{1.438599in}{0.558172in}}%
\pgfpathlineto{\pgfqpoint{1.432022in}{0.563024in}}%
\pgfpathlineto{\pgfqpoint{1.422943in}{0.570446in}}%
\pgfpathlineto{\pgfqpoint{1.416202in}{0.576635in}}%
\pgfpathlineto{\pgfqpoint{1.407286in}{0.585964in}}%
\pgfpathlineto{\pgfqpoint{1.403547in}{0.590246in}}%
\pgfpathlineto{\pgfqpoint{1.393443in}{0.603857in}}%
\pgfpathlineto{\pgfqpoint{1.391630in}{0.606849in}}%
\pgfpathlineto{\pgfqpoint{1.385595in}{0.617468in}}%
\pgfpathlineto{\pgfqpoint{1.379670in}{0.631079in}}%
\pgfpathlineto{\pgfqpoint{1.375973in}{0.643352in}}%
\pgfpathlineto{\pgfqpoint{1.375586in}{0.644691in}}%
\pgfpathlineto{\pgfqpoint{1.373388in}{0.658302in}}%
\pgfpathlineto{\pgfqpoint{1.372949in}{0.671913in}}%
\pgfpathlineto{\pgfqpoint{1.374267in}{0.685524in}}%
\pgfpathlineto{\pgfqpoint{1.375973in}{0.693085in}}%
\pgfpathlineto{\pgfqpoint{1.377393in}{0.699135in}}%
\pgfpathlineto{\pgfqpoint{1.382404in}{0.712746in}}%
\pgfpathlineto{\pgfqpoint{1.389244in}{0.726357in}}%
\pgfpathlineto{\pgfqpoint{1.391630in}{0.730139in}}%
\pgfpathlineto{\pgfqpoint{1.398254in}{0.739968in}}%
\pgfpathlineto{\pgfqpoint{1.407286in}{0.751109in}}%
\pgfpathlineto{\pgfqpoint{1.409475in}{0.753579in}}%
\pgfpathlineto{\pgfqpoint{1.422943in}{0.766714in}}%
\pgfpathlineto{\pgfqpoint{1.423491in}{0.767191in}}%
\pgfpathlineto{\pgfqpoint{1.438599in}{0.778898in}}%
\pgfpathlineto{\pgfqpoint{1.441441in}{0.780802in}}%
\pgfpathlineto{\pgfqpoint{1.454256in}{0.788654in}}%
\pgfpathlineto{\pgfqpoint{1.465562in}{0.794413in}}%
\pgfpathlineto{\pgfqpoint{1.469913in}{0.796487in}}%
\pgfpathlineto{\pgfqpoint{1.485569in}{0.802433in}}%
\pgfpathlineto{\pgfqpoint{1.501226in}{0.806789in}}%
\pgfpathlineto{\pgfqpoint{1.508185in}{0.808024in}}%
\pgfpathlineto{\pgfqpoint{1.516882in}{0.809507in}}%
\pgfpathlineto{\pgfqpoint{1.532539in}{0.810653in}}%
\pgfpathlineto{\pgfqpoint{1.548195in}{0.810271in}}%
\pgfpathlineto{\pgfqpoint{1.563852in}{0.808360in}}%
\pgfpathlineto{\pgfqpoint{1.565392in}{0.808024in}}%
\pgfpathlineto{\pgfqpoint{1.579508in}{0.804810in}}%
\pgfpathlineto{\pgfqpoint{1.595165in}{0.799659in}}%
\pgfpathlineto{\pgfqpoint{1.607381in}{0.794413in}}%
\pgfpathlineto{\pgfqpoint{1.610822in}{0.792836in}}%
\pgfpathlineto{\pgfqpoint{1.626478in}{0.784052in}}%
\pgfpathlineto{\pgfqpoint{1.631404in}{0.780802in}}%
\pgfpathlineto{\pgfqpoint{1.642135in}{0.773051in}}%
\pgfpathlineto{\pgfqpoint{1.649254in}{0.767191in}}%
\pgfpathlineto{\pgfqpoint{1.657791in}{0.759298in}}%
\pgfpathlineto{\pgfqpoint{1.663372in}{0.753579in}}%
\pgfpathlineto{\pgfqpoint{1.673448in}{0.741622in}}%
\pgfpathlineto{\pgfqpoint{1.674736in}{0.739968in}}%
\pgfpathlineto{\pgfqpoint{1.683617in}{0.726357in}}%
\pgfpathlineto{\pgfqpoint{1.689104in}{0.715720in}}%
\pgfpathlineto{\pgfqpoint{1.690558in}{0.712746in}}%
\pgfpathlineto{\pgfqpoint{1.695472in}{0.699135in}}%
\pgfpathlineto{\pgfqpoint{1.698597in}{0.685524in}}%
\pgfpathlineto{\pgfqpoint{1.699936in}{0.671913in}}%
\pgfpathlineto{\pgfqpoint{1.699490in}{0.658302in}}%
\pgfpathlineto{\pgfqpoint{1.697258in}{0.644691in}}%
\pgfpathlineto{\pgfqpoint{1.693239in}{0.631079in}}%
\pgfpathlineto{\pgfqpoint{1.689104in}{0.621368in}}%
\pgfpathlineto{\pgfqpoint{1.687354in}{0.617468in}}%
\pgfpathlineto{\pgfqpoint{1.679411in}{0.603857in}}%
\pgfpathlineto{\pgfqpoint{1.673448in}{0.595542in}}%
\pgfpathlineto{\pgfqpoint{1.669344in}{0.590246in}}%
\pgfpathlineto{\pgfqpoint{1.657791in}{0.577590in}}%
\pgfpathlineto{\pgfqpoint{1.656826in}{0.576635in}}%
\pgfpathlineto{\pgfqpoint{1.642135in}{0.563863in}}%
\pgfpathlineto{\pgfqpoint{1.641036in}{0.563024in}}%
\pgfpathlineto{\pgfqpoint{1.626478in}{0.552980in}}%
\pgfpathlineto{\pgfqpoint{1.620386in}{0.549413in}}%
\pgfpathlineto{\pgfqpoint{1.610822in}{0.544229in}}%
\pgfpathlineto{\pgfqpoint{1.595165in}{0.537323in}}%
\pgfpathlineto{\pgfqpoint{1.590679in}{0.535802in}}%
\pgfpathlineto{\pgfqpoint{1.579508in}{0.532207in}}%
\pgfpathlineto{\pgfqpoint{1.563852in}{0.528713in}}%
\pgfpathlineto{\pgfqpoint{1.548195in}{0.526773in}}%
\pgfpathlineto{\pgfqpoint{1.532539in}{0.526385in}}%
\pgfpathlineto{\pgfqpoint{1.516882in}{0.527549in}}%
\pgfpathlineto{\pgfqpoint{1.501226in}{0.530266in}}%
\pgfpathlineto{\pgfqpoint{1.485569in}{0.534538in}}%
\pgfpathlineto{\pgfqpoint{1.482148in}{0.535802in}}%
\pgfpathclose%
\pgfpathmoveto{\pgfqpoint{0.718397in}{1.148301in}}%
\pgfpathlineto{\pgfqpoint{0.734054in}{1.145156in}}%
\pgfpathlineto{\pgfqpoint{0.749710in}{1.143410in}}%
\pgfpathlineto{\pgfqpoint{0.765367in}{1.143061in}}%
\pgfpathlineto{\pgfqpoint{0.781024in}{1.144109in}}%
\pgfpathlineto{\pgfqpoint{0.796680in}{1.146554in}}%
\pgfpathlineto{\pgfqpoint{0.803829in}{1.148302in}}%
\pgfpathlineto{\pgfqpoint{0.812337in}{1.150334in}}%
\pgfpathlineto{\pgfqpoint{0.827993in}{1.155420in}}%
\pgfpathlineto{\pgfqpoint{0.843650in}{1.161867in}}%
\pgfpathlineto{\pgfqpoint{0.843743in}{1.161913in}}%
\pgfpathlineto{\pgfqpoint{0.859306in}{1.169560in}}%
\pgfpathlineto{\pgfqpoint{0.869717in}{1.175524in}}%
\pgfpathlineto{\pgfqpoint{0.874963in}{1.178589in}}%
\pgfpathlineto{\pgfqpoint{0.890620in}{1.188943in}}%
\pgfpathlineto{\pgfqpoint{0.890884in}{1.189135in}}%
\pgfpathlineto{\pgfqpoint{0.906276in}{1.200785in}}%
\pgfpathlineto{\pgfqpoint{0.908670in}{1.202746in}}%
\pgfpathlineto{\pgfqpoint{0.921933in}{1.214276in}}%
\pgfpathlineto{\pgfqpoint{0.924189in}{1.216357in}}%
\pgfpathlineto{\pgfqpoint{0.937589in}{1.229739in}}%
\pgfpathlineto{\pgfqpoint{0.937810in}{1.229968in}}%
\pgfpathlineto{\pgfqpoint{0.949720in}{1.243579in}}%
\pgfpathlineto{\pgfqpoint{0.953246in}{1.248140in}}%
\pgfpathlineto{\pgfqpoint{0.960106in}{1.257191in}}%
\pgfpathlineto{\pgfqpoint{0.968902in}{1.270721in}}%
\pgfpathlineto{\pgfqpoint{0.968955in}{1.270802in}}%
\pgfpathlineto{\pgfqpoint{0.976371in}{1.284413in}}%
\pgfpathlineto{\pgfqpoint{0.982222in}{1.298024in}}%
\pgfpathlineto{\pgfqpoint{0.984559in}{1.305420in}}%
\pgfpathlineto{\pgfqpoint{0.986570in}{1.311635in}}%
\pgfpathlineto{\pgfqpoint{0.989382in}{1.325246in}}%
\pgfpathlineto{\pgfqpoint{0.990587in}{1.338857in}}%
\pgfpathlineto{\pgfqpoint{0.990185in}{1.352468in}}%
\pgfpathlineto{\pgfqpoint{0.988177in}{1.366079in}}%
\pgfpathlineto{\pgfqpoint{0.984560in}{1.379691in}}%
\pgfpathlineto{\pgfqpoint{0.984559in}{1.379693in}}%
\pgfpathlineto{\pgfqpoint{0.979492in}{1.393302in}}%
\pgfpathlineto{\pgfqpoint{0.972859in}{1.406913in}}%
\pgfpathlineto{\pgfqpoint{0.968902in}{1.413532in}}%
\pgfpathlineto{\pgfqpoint{0.964722in}{1.420524in}}%
\pgfpathlineto{\pgfqpoint{0.955106in}{1.434135in}}%
\pgfpathlineto{\pgfqpoint{0.953246in}{1.436445in}}%
\pgfpathlineto{\pgfqpoint{0.943952in}{1.447746in}}%
\pgfpathlineto{\pgfqpoint{0.937589in}{1.454665in}}%
\pgfpathlineto{\pgfqpoint{0.931186in}{1.461357in}}%
\pgfpathlineto{\pgfqpoint{0.921933in}{1.470205in}}%
\pgfpathlineto{\pgfqpoint{0.916651in}{1.474968in}}%
\pgfpathlineto{\pgfqpoint{0.906276in}{1.483706in}}%
\pgfpathlineto{\pgfqpoint{0.900018in}{1.488579in}}%
\pgfpathlineto{\pgfqpoint{0.890620in}{1.495548in}}%
\pgfpathlineto{\pgfqpoint{0.880719in}{1.502191in}}%
\pgfpathlineto{\pgfqpoint{0.874963in}{1.505937in}}%
\pgfpathlineto{\pgfqpoint{0.859306in}{1.514949in}}%
\pgfpathlineto{\pgfqpoint{0.857594in}{1.515802in}}%
\pgfpathlineto{\pgfqpoint{0.843650in}{1.522678in}}%
\pgfpathlineto{\pgfqpoint{0.827993in}{1.529064in}}%
\pgfpathlineto{\pgfqpoint{0.826922in}{1.529413in}}%
\pgfpathlineto{\pgfqpoint{0.812337in}{1.534212in}}%
\pgfpathlineto{\pgfqpoint{0.796680in}{1.537989in}}%
\pgfpathlineto{\pgfqpoint{0.781024in}{1.540391in}}%
\pgfpathlineto{\pgfqpoint{0.765367in}{1.541419in}}%
\pgfpathlineto{\pgfqpoint{0.749710in}{1.541077in}}%
\pgfpathlineto{\pgfqpoint{0.734054in}{1.539362in}}%
\pgfpathlineto{\pgfqpoint{0.718397in}{1.536273in}}%
\pgfpathlineto{\pgfqpoint{0.702741in}{1.531807in}}%
\pgfpathlineto{\pgfqpoint{0.696283in}{1.529413in}}%
\pgfpathlineto{\pgfqpoint{0.687084in}{1.526040in}}%
\pgfpathlineto{\pgfqpoint{0.671428in}{1.518979in}}%
\pgfpathlineto{\pgfqpoint{0.665445in}{1.515802in}}%
\pgfpathlineto{\pgfqpoint{0.655771in}{1.510608in}}%
\pgfpathlineto{\pgfqpoint{0.642110in}{1.502191in}}%
\pgfpathlineto{\pgfqpoint{0.640115in}{1.500924in}}%
\pgfpathlineto{\pgfqpoint{0.624458in}{1.489849in}}%
\pgfpathlineto{\pgfqpoint{0.622813in}{1.488579in}}%
\pgfpathlineto{\pgfqpoint{0.608801in}{1.477212in}}%
\pgfpathlineto{\pgfqpoint{0.606221in}{1.474968in}}%
\pgfpathlineto{\pgfqpoint{0.593145in}{1.462787in}}%
\pgfpathlineto{\pgfqpoint{0.591685in}{1.461357in}}%
\pgfpathlineto{\pgfqpoint{0.578945in}{1.447746in}}%
\pgfpathlineto{\pgfqpoint{0.577488in}{1.446011in}}%
\pgfpathlineto{\pgfqpoint{0.567805in}{1.434135in}}%
\pgfpathlineto{\pgfqpoint{0.561832in}{1.425725in}}%
\pgfpathlineto{\pgfqpoint{0.558177in}{1.420524in}}%
\pgfpathlineto{\pgfqpoint{0.550055in}{1.406913in}}%
\pgfpathlineto{\pgfqpoint{0.546175in}{1.398916in}}%
\pgfpathlineto{\pgfqpoint{0.543422in}{1.393302in}}%
\pgfpathlineto{\pgfqpoint{0.538284in}{1.379691in}}%
\pgfpathlineto{\pgfqpoint{0.534731in}{1.366079in}}%
\pgfpathlineto{\pgfqpoint{0.532759in}{1.352468in}}%
\pgfpathlineto{\pgfqpoint{0.532364in}{1.338857in}}%
\pgfpathlineto{\pgfqpoint{0.533548in}{1.325246in}}%
\pgfpathlineto{\pgfqpoint{0.536310in}{1.311635in}}%
\pgfpathlineto{\pgfqpoint{0.540655in}{1.298024in}}%
\pgfpathlineto{\pgfqpoint{0.546175in}{1.285344in}}%
\pgfpathlineto{\pgfqpoint{0.546576in}{1.284413in}}%
\pgfpathlineto{\pgfqpoint{0.553922in}{1.270802in}}%
\pgfpathlineto{\pgfqpoint{0.561832in}{1.258679in}}%
\pgfpathlineto{\pgfqpoint{0.562812in}{1.257191in}}%
\pgfpathlineto{\pgfqpoint{0.573179in}{1.243579in}}%
\pgfpathlineto{\pgfqpoint{0.577488in}{1.238575in}}%
\pgfpathlineto{\pgfqpoint{0.585129in}{1.229968in}}%
\pgfpathlineto{\pgfqpoint{0.593145in}{1.221797in}}%
\pgfpathlineto{\pgfqpoint{0.598750in}{1.216357in}}%
\pgfpathlineto{\pgfqpoint{0.608801in}{1.207338in}}%
\pgfpathlineto{\pgfqpoint{0.614280in}{1.202746in}}%
\pgfpathlineto{\pgfqpoint{0.624458in}{1.194702in}}%
\pgfpathlineto{\pgfqpoint{0.632155in}{1.189135in}}%
\pgfpathlineto{\pgfqpoint{0.640115in}{1.183603in}}%
\pgfpathlineto{\pgfqpoint{0.653114in}{1.175524in}}%
\pgfpathlineto{\pgfqpoint{0.655771in}{1.173907in}}%
\pgfpathlineto{\pgfqpoint{0.671428in}{1.165547in}}%
\pgfpathlineto{\pgfqpoint{0.679471in}{1.161913in}}%
\pgfpathlineto{\pgfqpoint{0.687084in}{1.158473in}}%
\pgfpathlineto{\pgfqpoint{0.702741in}{1.152707in}}%
\pgfpathlineto{\pgfqpoint{0.718395in}{1.148302in}}%
\pgfpathlineto{\pgfqpoint{0.718397in}{1.148301in}}%
\pgfpathclose%
\pgfpathmoveto{\pgfqpoint{0.732514in}{1.202746in}}%
\pgfpathlineto{\pgfqpoint{0.718397in}{1.205960in}}%
\pgfpathlineto{\pgfqpoint{0.702741in}{1.211111in}}%
\pgfpathlineto{\pgfqpoint{0.690525in}{1.216357in}}%
\pgfpathlineto{\pgfqpoint{0.687084in}{1.217934in}}%
\pgfpathlineto{\pgfqpoint{0.671428in}{1.226718in}}%
\pgfpathlineto{\pgfqpoint{0.666502in}{1.229968in}}%
\pgfpathlineto{\pgfqpoint{0.655771in}{1.237719in}}%
\pgfpathlineto{\pgfqpoint{0.648652in}{1.243579in}}%
\pgfpathlineto{\pgfqpoint{0.640115in}{1.251472in}}%
\pgfpathlineto{\pgfqpoint{0.634534in}{1.257191in}}%
\pgfpathlineto{\pgfqpoint{0.624458in}{1.269148in}}%
\pgfpathlineto{\pgfqpoint{0.623170in}{1.270802in}}%
\pgfpathlineto{\pgfqpoint{0.614289in}{1.284413in}}%
\pgfpathlineto{\pgfqpoint{0.608801in}{1.295050in}}%
\pgfpathlineto{\pgfqpoint{0.607348in}{1.298024in}}%
\pgfpathlineto{\pgfqpoint{0.602433in}{1.311635in}}%
\pgfpathlineto{\pgfqpoint{0.599309in}{1.325246in}}%
\pgfpathlineto{\pgfqpoint{0.597970in}{1.338857in}}%
\pgfpathlineto{\pgfqpoint{0.598416in}{1.352468in}}%
\pgfpathlineto{\pgfqpoint{0.600648in}{1.366079in}}%
\pgfpathlineto{\pgfqpoint{0.604667in}{1.379691in}}%
\pgfpathlineto{\pgfqpoint{0.608801in}{1.389402in}}%
\pgfpathlineto{\pgfqpoint{0.610552in}{1.393302in}}%
\pgfpathlineto{\pgfqpoint{0.618495in}{1.406913in}}%
\pgfpathlineto{\pgfqpoint{0.624458in}{1.415228in}}%
\pgfpathlineto{\pgfqpoint{0.628562in}{1.420524in}}%
\pgfpathlineto{\pgfqpoint{0.640115in}{1.433180in}}%
\pgfpathlineto{\pgfqpoint{0.641080in}{1.434135in}}%
\pgfpathlineto{\pgfqpoint{0.655771in}{1.446907in}}%
\pgfpathlineto{\pgfqpoint{0.656870in}{1.447746in}}%
\pgfpathlineto{\pgfqpoint{0.671428in}{1.457790in}}%
\pgfpathlineto{\pgfqpoint{0.677520in}{1.461357in}}%
\pgfpathlineto{\pgfqpoint{0.687084in}{1.466541in}}%
\pgfpathlineto{\pgfqpoint{0.702741in}{1.473447in}}%
\pgfpathlineto{\pgfqpoint{0.707227in}{1.474968in}}%
\pgfpathlineto{\pgfqpoint{0.718397in}{1.478563in}}%
\pgfpathlineto{\pgfqpoint{0.734054in}{1.482057in}}%
\pgfpathlineto{\pgfqpoint{0.749710in}{1.483997in}}%
\pgfpathlineto{\pgfqpoint{0.765367in}{1.484385in}}%
\pgfpathlineto{\pgfqpoint{0.781024in}{1.483221in}}%
\pgfpathlineto{\pgfqpoint{0.796680in}{1.480504in}}%
\pgfpathlineto{\pgfqpoint{0.812337in}{1.476232in}}%
\pgfpathlineto{\pgfqpoint{0.815758in}{1.474968in}}%
\pgfpathlineto{\pgfqpoint{0.827993in}{1.470198in}}%
\pgfpathlineto{\pgfqpoint{0.843650in}{1.462477in}}%
\pgfpathlineto{\pgfqpoint{0.845552in}{1.461357in}}%
\pgfpathlineto{\pgfqpoint{0.859306in}{1.452598in}}%
\pgfpathlineto{\pgfqpoint{0.865884in}{1.447746in}}%
\pgfpathlineto{\pgfqpoint{0.874963in}{1.440324in}}%
\pgfpathlineto{\pgfqpoint{0.881704in}{1.434135in}}%
\pgfpathlineto{\pgfqpoint{0.890620in}{1.424806in}}%
\pgfpathlineto{\pgfqpoint{0.894358in}{1.420524in}}%
\pgfpathlineto{\pgfqpoint{0.904462in}{1.406913in}}%
\pgfpathlineto{\pgfqpoint{0.906276in}{1.403921in}}%
\pgfpathlineto{\pgfqpoint{0.912311in}{1.393302in}}%
\pgfpathlineto{\pgfqpoint{0.918235in}{1.379691in}}%
\pgfpathlineto{\pgfqpoint{0.921933in}{1.367418in}}%
\pgfpathlineto{\pgfqpoint{0.922320in}{1.366079in}}%
\pgfpathlineto{\pgfqpoint{0.924518in}{1.352468in}}%
\pgfpathlineto{\pgfqpoint{0.924957in}{1.338857in}}%
\pgfpathlineto{\pgfqpoint{0.923639in}{1.325246in}}%
\pgfpathlineto{\pgfqpoint{0.921933in}{1.317685in}}%
\pgfpathlineto{\pgfqpoint{0.920512in}{1.311635in}}%
\pgfpathlineto{\pgfqpoint{0.915502in}{1.298024in}}%
\pgfpathlineto{\pgfqpoint{0.908662in}{1.284413in}}%
\pgfpathlineto{\pgfqpoint{0.906276in}{1.280631in}}%
\pgfpathlineto{\pgfqpoint{0.899652in}{1.270802in}}%
\pgfpathlineto{\pgfqpoint{0.890620in}{1.259661in}}%
\pgfpathlineto{\pgfqpoint{0.888430in}{1.257191in}}%
\pgfpathlineto{\pgfqpoint{0.874963in}{1.244056in}}%
\pgfpathlineto{\pgfqpoint{0.874415in}{1.243579in}}%
\pgfpathlineto{\pgfqpoint{0.859306in}{1.231872in}}%
\pgfpathlineto{\pgfqpoint{0.856465in}{1.229968in}}%
\pgfpathlineto{\pgfqpoint{0.843650in}{1.222116in}}%
\pgfpathlineto{\pgfqpoint{0.832344in}{1.216357in}}%
\pgfpathlineto{\pgfqpoint{0.827993in}{1.214283in}}%
\pgfpathlineto{\pgfqpoint{0.812337in}{1.208337in}}%
\pgfpathlineto{\pgfqpoint{0.796680in}{1.203981in}}%
\pgfpathlineto{\pgfqpoint{0.789721in}{1.202746in}}%
\pgfpathlineto{\pgfqpoint{0.781024in}{1.201263in}}%
\pgfpathlineto{\pgfqpoint{0.765367in}{1.200117in}}%
\pgfpathlineto{\pgfqpoint{0.749710in}{1.200499in}}%
\pgfpathlineto{\pgfqpoint{0.734054in}{1.202410in}}%
\pgfpathlineto{\pgfqpoint{0.732514in}{1.202746in}}%
\pgfpathclose%
\pgfpathmoveto{\pgfqpoint{1.501226in}{1.146554in}}%
\pgfpathlineto{\pgfqpoint{1.516882in}{1.144109in}}%
\pgfpathlineto{\pgfqpoint{1.532539in}{1.143061in}}%
\pgfpathlineto{\pgfqpoint{1.548195in}{1.143410in}}%
\pgfpathlineto{\pgfqpoint{1.563852in}{1.145156in}}%
\pgfpathlineto{\pgfqpoint{1.579508in}{1.148301in}}%
\pgfpathlineto{\pgfqpoint{1.579511in}{1.148302in}}%
\pgfpathlineto{\pgfqpoint{1.595165in}{1.152707in}}%
\pgfpathlineto{\pgfqpoint{1.610822in}{1.158473in}}%
\pgfpathlineto{\pgfqpoint{1.618435in}{1.161913in}}%
\pgfpathlineto{\pgfqpoint{1.626478in}{1.165547in}}%
\pgfpathlineto{\pgfqpoint{1.642135in}{1.173907in}}%
\pgfpathlineto{\pgfqpoint{1.644792in}{1.175524in}}%
\pgfpathlineto{\pgfqpoint{1.657791in}{1.183603in}}%
\pgfpathlineto{\pgfqpoint{1.665750in}{1.189135in}}%
\pgfpathlineto{\pgfqpoint{1.673448in}{1.194702in}}%
\pgfpathlineto{\pgfqpoint{1.683625in}{1.202746in}}%
\pgfpathlineto{\pgfqpoint{1.689104in}{1.207338in}}%
\pgfpathlineto{\pgfqpoint{1.699156in}{1.216357in}}%
\pgfpathlineto{\pgfqpoint{1.704761in}{1.221797in}}%
\pgfpathlineto{\pgfqpoint{1.712777in}{1.229968in}}%
\pgfpathlineto{\pgfqpoint{1.720418in}{1.238575in}}%
\pgfpathlineto{\pgfqpoint{1.724726in}{1.243579in}}%
\pgfpathlineto{\pgfqpoint{1.735094in}{1.257191in}}%
\pgfpathlineto{\pgfqpoint{1.736074in}{1.258679in}}%
\pgfpathlineto{\pgfqpoint{1.743983in}{1.270802in}}%
\pgfpathlineto{\pgfqpoint{1.751330in}{1.284413in}}%
\pgfpathlineto{\pgfqpoint{1.751731in}{1.285344in}}%
\pgfpathlineto{\pgfqpoint{1.757251in}{1.298024in}}%
\pgfpathlineto{\pgfqpoint{1.761596in}{1.311635in}}%
\pgfpathlineto{\pgfqpoint{1.764358in}{1.325246in}}%
\pgfpathlineto{\pgfqpoint{1.765542in}{1.338857in}}%
\pgfpathlineto{\pgfqpoint{1.765147in}{1.352468in}}%
\pgfpathlineto{\pgfqpoint{1.763175in}{1.366079in}}%
\pgfpathlineto{\pgfqpoint{1.759621in}{1.379691in}}%
\pgfpathlineto{\pgfqpoint{1.754484in}{1.393302in}}%
\pgfpathlineto{\pgfqpoint{1.751731in}{1.398916in}}%
\pgfpathlineto{\pgfqpoint{1.747851in}{1.406913in}}%
\pgfpathlineto{\pgfqpoint{1.739729in}{1.420524in}}%
\pgfpathlineto{\pgfqpoint{1.736074in}{1.425725in}}%
\pgfpathlineto{\pgfqpoint{1.730100in}{1.434135in}}%
\pgfpathlineto{\pgfqpoint{1.720418in}{1.446011in}}%
\pgfpathlineto{\pgfqpoint{1.718961in}{1.447746in}}%
\pgfpathlineto{\pgfqpoint{1.706221in}{1.461357in}}%
\pgfpathlineto{\pgfqpoint{1.704761in}{1.462787in}}%
\pgfpathlineto{\pgfqpoint{1.691685in}{1.474968in}}%
\pgfpathlineto{\pgfqpoint{1.689104in}{1.477212in}}%
\pgfpathlineto{\pgfqpoint{1.675092in}{1.488579in}}%
\pgfpathlineto{\pgfqpoint{1.673448in}{1.489849in}}%
\pgfpathlineto{\pgfqpoint{1.657791in}{1.500924in}}%
\pgfpathlineto{\pgfqpoint{1.655796in}{1.502191in}}%
\pgfpathlineto{\pgfqpoint{1.642135in}{1.510608in}}%
\pgfpathlineto{\pgfqpoint{1.632461in}{1.515802in}}%
\pgfpathlineto{\pgfqpoint{1.626478in}{1.518979in}}%
\pgfpathlineto{\pgfqpoint{1.610822in}{1.526040in}}%
\pgfpathlineto{\pgfqpoint{1.601623in}{1.529413in}}%
\pgfpathlineto{\pgfqpoint{1.595165in}{1.531807in}}%
\pgfpathlineto{\pgfqpoint{1.579508in}{1.536273in}}%
\pgfpathlineto{\pgfqpoint{1.563852in}{1.539362in}}%
\pgfpathlineto{\pgfqpoint{1.548195in}{1.541077in}}%
\pgfpathlineto{\pgfqpoint{1.532539in}{1.541419in}}%
\pgfpathlineto{\pgfqpoint{1.516882in}{1.540391in}}%
\pgfpathlineto{\pgfqpoint{1.501226in}{1.537989in}}%
\pgfpathlineto{\pgfqpoint{1.485569in}{1.534212in}}%
\pgfpathlineto{\pgfqpoint{1.470984in}{1.529413in}}%
\pgfpathlineto{\pgfqpoint{1.469913in}{1.529064in}}%
\pgfpathlineto{\pgfqpoint{1.454256in}{1.522678in}}%
\pgfpathlineto{\pgfqpoint{1.440312in}{1.515802in}}%
\pgfpathlineto{\pgfqpoint{1.438599in}{1.514949in}}%
\pgfpathlineto{\pgfqpoint{1.422943in}{1.505937in}}%
\pgfpathlineto{\pgfqpoint{1.417187in}{1.502191in}}%
\pgfpathlineto{\pgfqpoint{1.407286in}{1.495548in}}%
\pgfpathlineto{\pgfqpoint{1.397887in}{1.488579in}}%
\pgfpathlineto{\pgfqpoint{1.391630in}{1.483706in}}%
\pgfpathlineto{\pgfqpoint{1.381255in}{1.474968in}}%
\pgfpathlineto{\pgfqpoint{1.375973in}{1.470205in}}%
\pgfpathlineto{\pgfqpoint{1.366720in}{1.461357in}}%
\pgfpathlineto{\pgfqpoint{1.360317in}{1.454665in}}%
\pgfpathlineto{\pgfqpoint{1.353954in}{1.447746in}}%
\pgfpathlineto{\pgfqpoint{1.344660in}{1.436445in}}%
\pgfpathlineto{\pgfqpoint{1.342800in}{1.434135in}}%
\pgfpathlineto{\pgfqpoint{1.333184in}{1.420524in}}%
\pgfpathlineto{\pgfqpoint{1.329003in}{1.413532in}}%
\pgfpathlineto{\pgfqpoint{1.325047in}{1.406913in}}%
\pgfpathlineto{\pgfqpoint{1.318414in}{1.393302in}}%
\pgfpathlineto{\pgfqpoint{1.313347in}{1.379693in}}%
\pgfpathlineto{\pgfqpoint{1.313346in}{1.379691in}}%
\pgfpathlineto{\pgfqpoint{1.309729in}{1.366079in}}%
\pgfpathlineto{\pgfqpoint{1.307720in}{1.352468in}}%
\pgfpathlineto{\pgfqpoint{1.307319in}{1.338857in}}%
\pgfpathlineto{\pgfqpoint{1.308524in}{1.325246in}}%
\pgfpathlineto{\pgfqpoint{1.311336in}{1.311635in}}%
\pgfpathlineto{\pgfqpoint{1.313347in}{1.305420in}}%
\pgfpathlineto{\pgfqpoint{1.315684in}{1.298024in}}%
\pgfpathlineto{\pgfqpoint{1.321534in}{1.284413in}}%
\pgfpathlineto{\pgfqpoint{1.328951in}{1.270802in}}%
\pgfpathlineto{\pgfqpoint{1.329003in}{1.270721in}}%
\pgfpathlineto{\pgfqpoint{1.337800in}{1.257191in}}%
\pgfpathlineto{\pgfqpoint{1.344660in}{1.248140in}}%
\pgfpathlineto{\pgfqpoint{1.348186in}{1.243579in}}%
\pgfpathlineto{\pgfqpoint{1.360096in}{1.229968in}}%
\pgfpathlineto{\pgfqpoint{1.360317in}{1.229739in}}%
\pgfpathlineto{\pgfqpoint{1.373717in}{1.216357in}}%
\pgfpathlineto{\pgfqpoint{1.375973in}{1.214276in}}%
\pgfpathlineto{\pgfqpoint{1.389236in}{1.202746in}}%
\pgfpathlineto{\pgfqpoint{1.391630in}{1.200785in}}%
\pgfpathlineto{\pgfqpoint{1.407022in}{1.189135in}}%
\pgfpathlineto{\pgfqpoint{1.407286in}{1.188943in}}%
\pgfpathlineto{\pgfqpoint{1.422943in}{1.178589in}}%
\pgfpathlineto{\pgfqpoint{1.428188in}{1.175524in}}%
\pgfpathlineto{\pgfqpoint{1.438599in}{1.169560in}}%
\pgfpathlineto{\pgfqpoint{1.454163in}{1.161913in}}%
\pgfpathlineto{\pgfqpoint{1.454256in}{1.161867in}}%
\pgfpathlineto{\pgfqpoint{1.469913in}{1.155420in}}%
\pgfpathlineto{\pgfqpoint{1.485569in}{1.150334in}}%
\pgfpathlineto{\pgfqpoint{1.494077in}{1.148302in}}%
\pgfpathlineto{\pgfqpoint{1.501226in}{1.146554in}}%
\pgfpathclose%
\pgfpathmoveto{\pgfqpoint{1.508185in}{1.202746in}}%
\pgfpathlineto{\pgfqpoint{1.501226in}{1.203981in}}%
\pgfpathlineto{\pgfqpoint{1.485569in}{1.208337in}}%
\pgfpathlineto{\pgfqpoint{1.469913in}{1.214283in}}%
\pgfpathlineto{\pgfqpoint{1.465562in}{1.216357in}}%
\pgfpathlineto{\pgfqpoint{1.454256in}{1.222116in}}%
\pgfpathlineto{\pgfqpoint{1.441441in}{1.229968in}}%
\pgfpathlineto{\pgfqpoint{1.438599in}{1.231872in}}%
\pgfpathlineto{\pgfqpoint{1.423491in}{1.243579in}}%
\pgfpathlineto{\pgfqpoint{1.422943in}{1.244056in}}%
\pgfpathlineto{\pgfqpoint{1.409475in}{1.257191in}}%
\pgfpathlineto{\pgfqpoint{1.407286in}{1.259661in}}%
\pgfpathlineto{\pgfqpoint{1.398254in}{1.270802in}}%
\pgfpathlineto{\pgfqpoint{1.391630in}{1.280631in}}%
\pgfpathlineto{\pgfqpoint{1.389244in}{1.284413in}}%
\pgfpathlineto{\pgfqpoint{1.382404in}{1.298024in}}%
\pgfpathlineto{\pgfqpoint{1.377393in}{1.311635in}}%
\pgfpathlineto{\pgfqpoint{1.375973in}{1.317685in}}%
\pgfpathlineto{\pgfqpoint{1.374267in}{1.325246in}}%
\pgfpathlineto{\pgfqpoint{1.372949in}{1.338857in}}%
\pgfpathlineto{\pgfqpoint{1.373388in}{1.352468in}}%
\pgfpathlineto{\pgfqpoint{1.375586in}{1.366079in}}%
\pgfpathlineto{\pgfqpoint{1.375973in}{1.367418in}}%
\pgfpathlineto{\pgfqpoint{1.379670in}{1.379691in}}%
\pgfpathlineto{\pgfqpoint{1.385595in}{1.393302in}}%
\pgfpathlineto{\pgfqpoint{1.391630in}{1.403921in}}%
\pgfpathlineto{\pgfqpoint{1.393443in}{1.406913in}}%
\pgfpathlineto{\pgfqpoint{1.403547in}{1.420524in}}%
\pgfpathlineto{\pgfqpoint{1.407286in}{1.424806in}}%
\pgfpathlineto{\pgfqpoint{1.416202in}{1.434135in}}%
\pgfpathlineto{\pgfqpoint{1.422943in}{1.440324in}}%
\pgfpathlineto{\pgfqpoint{1.432022in}{1.447746in}}%
\pgfpathlineto{\pgfqpoint{1.438599in}{1.452598in}}%
\pgfpathlineto{\pgfqpoint{1.452354in}{1.461357in}}%
\pgfpathlineto{\pgfqpoint{1.454256in}{1.462477in}}%
\pgfpathlineto{\pgfqpoint{1.469913in}{1.470198in}}%
\pgfpathlineto{\pgfqpoint{1.482148in}{1.474968in}}%
\pgfpathlineto{\pgfqpoint{1.485569in}{1.476232in}}%
\pgfpathlineto{\pgfqpoint{1.501226in}{1.480504in}}%
\pgfpathlineto{\pgfqpoint{1.516882in}{1.483221in}}%
\pgfpathlineto{\pgfqpoint{1.532539in}{1.484385in}}%
\pgfpathlineto{\pgfqpoint{1.548195in}{1.483997in}}%
\pgfpathlineto{\pgfqpoint{1.563852in}{1.482057in}}%
\pgfpathlineto{\pgfqpoint{1.579508in}{1.478563in}}%
\pgfpathlineto{\pgfqpoint{1.590679in}{1.474968in}}%
\pgfpathlineto{\pgfqpoint{1.595165in}{1.473447in}}%
\pgfpathlineto{\pgfqpoint{1.610822in}{1.466541in}}%
\pgfpathlineto{\pgfqpoint{1.620386in}{1.461357in}}%
\pgfpathlineto{\pgfqpoint{1.626478in}{1.457790in}}%
\pgfpathlineto{\pgfqpoint{1.641036in}{1.447746in}}%
\pgfpathlineto{\pgfqpoint{1.642135in}{1.446907in}}%
\pgfpathlineto{\pgfqpoint{1.656826in}{1.434135in}}%
\pgfpathlineto{\pgfqpoint{1.657791in}{1.433180in}}%
\pgfpathlineto{\pgfqpoint{1.669344in}{1.420524in}}%
\pgfpathlineto{\pgfqpoint{1.673448in}{1.415228in}}%
\pgfpathlineto{\pgfqpoint{1.679411in}{1.406913in}}%
\pgfpathlineto{\pgfqpoint{1.687354in}{1.393302in}}%
\pgfpathlineto{\pgfqpoint{1.689104in}{1.389402in}}%
\pgfpathlineto{\pgfqpoint{1.693239in}{1.379691in}}%
\pgfpathlineto{\pgfqpoint{1.697258in}{1.366079in}}%
\pgfpathlineto{\pgfqpoint{1.699490in}{1.352468in}}%
\pgfpathlineto{\pgfqpoint{1.699936in}{1.338857in}}%
\pgfpathlineto{\pgfqpoint{1.698597in}{1.325246in}}%
\pgfpathlineto{\pgfqpoint{1.695472in}{1.311635in}}%
\pgfpathlineto{\pgfqpoint{1.690558in}{1.298024in}}%
\pgfpathlineto{\pgfqpoint{1.689104in}{1.295050in}}%
\pgfpathlineto{\pgfqpoint{1.683617in}{1.284413in}}%
\pgfpathlineto{\pgfqpoint{1.674736in}{1.270802in}}%
\pgfpathlineto{\pgfqpoint{1.673448in}{1.269148in}}%
\pgfpathlineto{\pgfqpoint{1.663372in}{1.257191in}}%
\pgfpathlineto{\pgfqpoint{1.657791in}{1.251472in}}%
\pgfpathlineto{\pgfqpoint{1.649254in}{1.243579in}}%
\pgfpathlineto{\pgfqpoint{1.642135in}{1.237719in}}%
\pgfpathlineto{\pgfqpoint{1.631404in}{1.229968in}}%
\pgfpathlineto{\pgfqpoint{1.626478in}{1.226718in}}%
\pgfpathlineto{\pgfqpoint{1.610822in}{1.217934in}}%
\pgfpathlineto{\pgfqpoint{1.607381in}{1.216357in}}%
\pgfpathlineto{\pgfqpoint{1.595165in}{1.211111in}}%
\pgfpathlineto{\pgfqpoint{1.579508in}{1.205960in}}%
\pgfpathlineto{\pgfqpoint{1.565392in}{1.202746in}}%
\pgfpathlineto{\pgfqpoint{1.563852in}{1.202410in}}%
\pgfpathlineto{\pgfqpoint{1.548195in}{1.200499in}}%
\pgfpathlineto{\pgfqpoint{1.532539in}{1.200117in}}%
\pgfpathlineto{\pgfqpoint{1.516882in}{1.201263in}}%
\pgfpathlineto{\pgfqpoint{1.508185in}{1.202746in}}%
\pgfpathclose%
\pgfusepath{fill}%
\end{pgfscope}%
\begin{pgfscope}%
\pgfpathrectangle{\pgfqpoint{0.373953in}{0.331635in}}{\pgfqpoint{1.550000in}{1.347500in}}%
\pgfusepath{clip}%
\pgfsetbuttcap%
\pgfsetroundjoin%
\definecolor{currentfill}{rgb}{0.417086,0.680631,0.838231}%
\pgfsetfillcolor{currentfill}%
\pgfsetlinewidth{0.000000pt}%
\definecolor{currentstroke}{rgb}{0.000000,0.000000,0.000000}%
\pgfsetstrokecolor{currentstroke}%
\pgfsetdash{}{0pt}%
\pgfpathmoveto{\pgfqpoint{0.749710in}{0.399531in}}%
\pgfpathlineto{\pgfqpoint{0.765367in}{0.398990in}}%
\pgfpathlineto{\pgfqpoint{0.772136in}{0.399691in}}%
\pgfpathlineto{\pgfqpoint{0.781024in}{0.400466in}}%
\pgfpathlineto{\pgfqpoint{0.796680in}{0.403653in}}%
\pgfpathlineto{\pgfqpoint{0.812337in}{0.408664in}}%
\pgfpathlineto{\pgfqpoint{0.822989in}{0.413302in}}%
\pgfpathlineto{\pgfqpoint{0.827993in}{0.415228in}}%
\pgfpathlineto{\pgfqpoint{0.843650in}{0.422813in}}%
\pgfpathlineto{\pgfqpoint{0.850702in}{0.426913in}}%
\pgfpathlineto{\pgfqpoint{0.859306in}{0.431496in}}%
\pgfpathlineto{\pgfqpoint{0.873880in}{0.440524in}}%
\pgfpathlineto{\pgfqpoint{0.874963in}{0.441155in}}%
\pgfpathlineto{\pgfqpoint{0.890620in}{0.451432in}}%
\pgfpathlineto{\pgfqpoint{0.894348in}{0.454135in}}%
\pgfpathlineto{\pgfqpoint{0.906276in}{0.462480in}}%
\pgfpathlineto{\pgfqpoint{0.913215in}{0.467746in}}%
\pgfpathlineto{\pgfqpoint{0.921933in}{0.474261in}}%
\pgfpathlineto{\pgfqpoint{0.930843in}{0.481357in}}%
\pgfpathlineto{\pgfqpoint{0.937589in}{0.486748in}}%
\pgfpathlineto{\pgfqpoint{0.947410in}{0.494968in}}%
\pgfpathlineto{\pgfqpoint{0.953246in}{0.499957in}}%
\pgfpathlineto{\pgfqpoint{0.963040in}{0.508579in}}%
\pgfpathlineto{\pgfqpoint{0.968902in}{0.513941in}}%
\pgfpathlineto{\pgfqpoint{0.977813in}{0.522191in}}%
\pgfpathlineto{\pgfqpoint{0.984559in}{0.528788in}}%
\pgfpathlineto{\pgfqpoint{0.991771in}{0.535802in}}%
\pgfpathlineto{\pgfqpoint{1.000216in}{0.544629in}}%
\pgfpathlineto{\pgfqpoint{1.004908in}{0.549413in}}%
\pgfpathlineto{\pgfqpoint{1.015872in}{0.561650in}}%
\pgfpathlineto{\pgfqpoint{1.017162in}{0.563024in}}%
\pgfpathlineto{\pgfqpoint{1.028652in}{0.576635in}}%
\pgfpathlineto{\pgfqpoint{1.031529in}{0.580534in}}%
\pgfpathlineto{\pgfqpoint{1.039241in}{0.590246in}}%
\pgfpathlineto{\pgfqpoint{1.047185in}{0.602070in}}%
\pgfpathlineto{\pgfqpoint{1.048516in}{0.603857in}}%
\pgfpathlineto{\pgfqpoint{1.056824in}{0.617468in}}%
\pgfpathlineto{\pgfqpoint{1.062842in}{0.630365in}}%
\pgfpathlineto{\pgfqpoint{1.063226in}{0.631079in}}%
\pgfpathlineto{\pgfqpoint{1.068338in}{0.644691in}}%
\pgfpathlineto{\pgfqpoint{1.071176in}{0.658302in}}%
\pgfpathlineto{\pgfqpoint{1.071743in}{0.671913in}}%
\pgfpathlineto{\pgfqpoint{1.070041in}{0.685524in}}%
\pgfpathlineto{\pgfqpoint{1.066067in}{0.699135in}}%
\pgfpathlineto{\pgfqpoint{1.062842in}{0.706184in}}%
\pgfpathlineto{\pgfqpoint{1.060243in}{0.712746in}}%
\pgfpathlineto{\pgfqpoint{1.052916in}{0.726357in}}%
\pgfpathlineto{\pgfqpoint{1.047185in}{0.734797in}}%
\pgfpathlineto{\pgfqpoint{1.044022in}{0.739968in}}%
\pgfpathlineto{\pgfqpoint{1.034026in}{0.753579in}}%
\pgfpathlineto{\pgfqpoint{1.031529in}{0.756521in}}%
\pgfpathlineto{\pgfqpoint{1.023102in}{0.767191in}}%
\pgfpathlineto{\pgfqpoint{1.015872in}{0.775282in}}%
\pgfpathlineto{\pgfqpoint{1.011170in}{0.780802in}}%
\pgfpathlineto{\pgfqpoint{1.000216in}{0.792424in}}%
\pgfpathlineto{\pgfqpoint{0.998389in}{0.794413in}}%
\pgfpathlineto{\pgfqpoint{0.984836in}{0.808024in}}%
\pgfpathlineto{\pgfqpoint{0.984559in}{0.808286in}}%
\pgfpathlineto{\pgfqpoint{0.970516in}{0.821635in}}%
\pgfpathlineto{\pgfqpoint{0.968902in}{0.823099in}}%
\pgfpathlineto{\pgfqpoint{0.955341in}{0.835246in}}%
\pgfpathlineto{\pgfqpoint{0.953246in}{0.837068in}}%
\pgfpathlineto{\pgfqpoint{0.939273in}{0.848857in}}%
\pgfpathlineto{\pgfqpoint{0.937589in}{0.850260in}}%
\pgfpathlineto{\pgfqpoint{0.922234in}{0.862468in}}%
\pgfpathlineto{\pgfqpoint{0.921933in}{0.862709in}}%
\pgfpathlineto{\pgfqpoint{0.906276in}{0.874492in}}%
\pgfpathlineto{\pgfqpoint{0.903988in}{0.876079in}}%
\pgfpathlineto{\pgfqpoint{0.890620in}{0.885603in}}%
\pgfpathlineto{\pgfqpoint{0.884270in}{0.889691in}}%
\pgfpathlineto{\pgfqpoint{0.874963in}{0.895976in}}%
\pgfpathlineto{\pgfqpoint{0.862690in}{0.903302in}}%
\pgfpathlineto{\pgfqpoint{0.859306in}{0.905473in}}%
\pgfpathlineto{\pgfqpoint{0.843650in}{0.914163in}}%
\pgfpathlineto{\pgfqpoint{0.837702in}{0.916913in}}%
\pgfpathlineto{\pgfqpoint{0.827993in}{0.921894in}}%
\pgfpathlineto{\pgfqpoint{0.812337in}{0.928265in}}%
\pgfpathlineto{\pgfqpoint{0.804789in}{0.930524in}}%
\pgfpathlineto{\pgfqpoint{0.796680in}{0.933327in}}%
\pgfpathlineto{\pgfqpoint{0.781024in}{0.936783in}}%
\pgfpathlineto{\pgfqpoint{0.765367in}{0.938263in}}%
\pgfpathlineto{\pgfqpoint{0.749710in}{0.937769in}}%
\pgfpathlineto{\pgfqpoint{0.734054in}{0.935302in}}%
\pgfpathlineto{\pgfqpoint{0.718397in}{0.930858in}}%
\pgfpathlineto{\pgfqpoint{0.717576in}{0.930524in}}%
\pgfpathlineto{\pgfqpoint{0.702741in}{0.925293in}}%
\pgfpathlineto{\pgfqpoint{0.687084in}{0.918070in}}%
\pgfpathlineto{\pgfqpoint{0.685029in}{0.916913in}}%
\pgfpathlineto{\pgfqpoint{0.671428in}{0.910007in}}%
\pgfpathlineto{\pgfqpoint{0.660256in}{0.903302in}}%
\pgfpathlineto{\pgfqpoint{0.655771in}{0.900801in}}%
\pgfpathlineto{\pgfqpoint{0.640115in}{0.890812in}}%
\pgfpathlineto{\pgfqpoint{0.638534in}{0.889691in}}%
\pgfpathlineto{\pgfqpoint{0.624458in}{0.880159in}}%
\pgfpathlineto{\pgfqpoint{0.618955in}{0.876079in}}%
\pgfpathlineto{\pgfqpoint{0.608801in}{0.868738in}}%
\pgfpathlineto{\pgfqpoint{0.600734in}{0.862468in}}%
\pgfpathlineto{\pgfqpoint{0.593145in}{0.856604in}}%
\pgfpathlineto{\pgfqpoint{0.583655in}{0.848857in}}%
\pgfpathlineto{\pgfqpoint{0.577488in}{0.843760in}}%
\pgfpathlineto{\pgfqpoint{0.567570in}{0.835246in}}%
\pgfpathlineto{\pgfqpoint{0.561832in}{0.830173in}}%
\pgfpathlineto{\pgfqpoint{0.552376in}{0.821635in}}%
\pgfpathlineto{\pgfqpoint{0.546175in}{0.815770in}}%
\pgfpathlineto{\pgfqpoint{0.538013in}{0.808024in}}%
\pgfpathlineto{\pgfqpoint{0.530519in}{0.800445in}}%
\pgfpathlineto{\pgfqpoint{0.524461in}{0.794413in}}%
\pgfpathlineto{\pgfqpoint{0.514862in}{0.784043in}}%
\pgfpathlineto{\pgfqpoint{0.511753in}{0.780802in}}%
\pgfpathlineto{\pgfqpoint{0.499932in}{0.767191in}}%
\pgfpathlineto{\pgfqpoint{0.499205in}{0.766249in}}%
\pgfpathlineto{\pgfqpoint{0.488820in}{0.753579in}}%
\pgfpathlineto{\pgfqpoint{0.483549in}{0.746099in}}%
\pgfpathlineto{\pgfqpoint{0.478833in}{0.739968in}}%
\pgfpathlineto{\pgfqpoint{0.470108in}{0.726357in}}%
\pgfpathlineto{\pgfqpoint{0.467892in}{0.722006in}}%
\pgfpathlineto{\pgfqpoint{0.462558in}{0.712746in}}%
\pgfpathlineto{\pgfqpoint{0.456793in}{0.699135in}}%
\pgfpathlineto{\pgfqpoint{0.453128in}{0.685524in}}%
\pgfpathlineto{\pgfqpoint{0.452236in}{0.677797in}}%
\pgfpathlineto{\pgfqpoint{0.451429in}{0.671913in}}%
\pgfpathlineto{\pgfqpoint{0.452052in}{0.658302in}}%
\pgfpathlineto{\pgfqpoint{0.452236in}{0.657495in}}%
\pgfpathlineto{\pgfqpoint{0.454698in}{0.644691in}}%
\pgfpathlineto{\pgfqpoint{0.459413in}{0.631079in}}%
\pgfpathlineto{\pgfqpoint{0.466229in}{0.617468in}}%
\pgfpathlineto{\pgfqpoint{0.467892in}{0.614905in}}%
\pgfpathlineto{\pgfqpoint{0.474240in}{0.603857in}}%
\pgfpathlineto{\pgfqpoint{0.483549in}{0.590714in}}%
\pgfpathlineto{\pgfqpoint{0.483852in}{0.590246in}}%
\pgfpathlineto{\pgfqpoint{0.494201in}{0.576635in}}%
\pgfpathlineto{\pgfqpoint{0.499205in}{0.570888in}}%
\pgfpathlineto{\pgfqpoint{0.505657in}{0.563024in}}%
\pgfpathlineto{\pgfqpoint{0.514862in}{0.552990in}}%
\pgfpathlineto{\pgfqpoint{0.518026in}{0.549413in}}%
\pgfpathlineto{\pgfqpoint{0.530519in}{0.536518in}}%
\pgfpathlineto{\pgfqpoint{0.531201in}{0.535802in}}%
\pgfpathlineto{\pgfqpoint{0.545124in}{0.522191in}}%
\pgfpathlineto{\pgfqpoint{0.546175in}{0.521216in}}%
\pgfpathlineto{\pgfqpoint{0.559869in}{0.508579in}}%
\pgfpathlineto{\pgfqpoint{0.561832in}{0.506836in}}%
\pgfpathlineto{\pgfqpoint{0.575483in}{0.494968in}}%
\pgfpathlineto{\pgfqpoint{0.577488in}{0.493262in}}%
\pgfpathlineto{\pgfqpoint{0.592024in}{0.481357in}}%
\pgfpathlineto{\pgfqpoint{0.593145in}{0.480443in}}%
\pgfpathlineto{\pgfqpoint{0.608801in}{0.468340in}}%
\pgfpathlineto{\pgfqpoint{0.609625in}{0.467746in}}%
\pgfpathlineto{\pgfqpoint{0.624458in}{0.456886in}}%
\pgfpathlineto{\pgfqpoint{0.628573in}{0.454135in}}%
\pgfpathlineto{\pgfqpoint{0.640115in}{0.446132in}}%
\pgfpathlineto{\pgfqpoint{0.649160in}{0.440524in}}%
\pgfpathlineto{\pgfqpoint{0.655771in}{0.436173in}}%
\pgfpathlineto{\pgfqpoint{0.671428in}{0.427176in}}%
\pgfpathlineto{\pgfqpoint{0.671966in}{0.426913in}}%
\pgfpathlineto{\pgfqpoint{0.687084in}{0.418820in}}%
\pgfpathlineto{\pgfqpoint{0.699792in}{0.413302in}}%
\pgfpathlineto{\pgfqpoint{0.702741in}{0.411855in}}%
\pgfpathlineto{\pgfqpoint{0.718397in}{0.405930in}}%
\pgfpathlineto{\pgfqpoint{0.734054in}{0.401831in}}%
\pgfpathlineto{\pgfqpoint{0.748782in}{0.399691in}}%
\pgfpathlineto{\pgfqpoint{0.749710in}{0.399531in}}%
\pgfpathclose%
\pgfpathmoveto{\pgfqpoint{0.696283in}{0.481357in}}%
\pgfpathlineto{\pgfqpoint{0.687084in}{0.484730in}}%
\pgfpathlineto{\pgfqpoint{0.671428in}{0.491791in}}%
\pgfpathlineto{\pgfqpoint{0.665445in}{0.494968in}}%
\pgfpathlineto{\pgfqpoint{0.655771in}{0.500162in}}%
\pgfpathlineto{\pgfqpoint{0.642110in}{0.508579in}}%
\pgfpathlineto{\pgfqpoint{0.640115in}{0.509846in}}%
\pgfpathlineto{\pgfqpoint{0.624458in}{0.520921in}}%
\pgfpathlineto{\pgfqpoint{0.622813in}{0.522191in}}%
\pgfpathlineto{\pgfqpoint{0.608801in}{0.533558in}}%
\pgfpathlineto{\pgfqpoint{0.606221in}{0.535802in}}%
\pgfpathlineto{\pgfqpoint{0.593145in}{0.547983in}}%
\pgfpathlineto{\pgfqpoint{0.591685in}{0.549413in}}%
\pgfpathlineto{\pgfqpoint{0.578945in}{0.563024in}}%
\pgfpathlineto{\pgfqpoint{0.577488in}{0.564759in}}%
\pgfpathlineto{\pgfqpoint{0.567805in}{0.576635in}}%
\pgfpathlineto{\pgfqpoint{0.561832in}{0.585045in}}%
\pgfpathlineto{\pgfqpoint{0.558177in}{0.590246in}}%
\pgfpathlineto{\pgfqpoint{0.550055in}{0.603857in}}%
\pgfpathlineto{\pgfqpoint{0.546175in}{0.611854in}}%
\pgfpathlineto{\pgfqpoint{0.543422in}{0.617468in}}%
\pgfpathlineto{\pgfqpoint{0.538284in}{0.631079in}}%
\pgfpathlineto{\pgfqpoint{0.534731in}{0.644691in}}%
\pgfpathlineto{\pgfqpoint{0.532759in}{0.658302in}}%
\pgfpathlineto{\pgfqpoint{0.532364in}{0.671913in}}%
\pgfpathlineto{\pgfqpoint{0.533548in}{0.685524in}}%
\pgfpathlineto{\pgfqpoint{0.536310in}{0.699135in}}%
\pgfpathlineto{\pgfqpoint{0.540655in}{0.712746in}}%
\pgfpathlineto{\pgfqpoint{0.546175in}{0.725426in}}%
\pgfpathlineto{\pgfqpoint{0.546576in}{0.726357in}}%
\pgfpathlineto{\pgfqpoint{0.553922in}{0.739968in}}%
\pgfpathlineto{\pgfqpoint{0.561832in}{0.752091in}}%
\pgfpathlineto{\pgfqpoint{0.562812in}{0.753579in}}%
\pgfpathlineto{\pgfqpoint{0.573179in}{0.767191in}}%
\pgfpathlineto{\pgfqpoint{0.577488in}{0.772195in}}%
\pgfpathlineto{\pgfqpoint{0.585129in}{0.780802in}}%
\pgfpathlineto{\pgfqpoint{0.593145in}{0.788973in}}%
\pgfpathlineto{\pgfqpoint{0.598750in}{0.794413in}}%
\pgfpathlineto{\pgfqpoint{0.608801in}{0.803432in}}%
\pgfpathlineto{\pgfqpoint{0.614280in}{0.808024in}}%
\pgfpathlineto{\pgfqpoint{0.624458in}{0.816068in}}%
\pgfpathlineto{\pgfqpoint{0.632155in}{0.821635in}}%
\pgfpathlineto{\pgfqpoint{0.640115in}{0.827167in}}%
\pgfpathlineto{\pgfqpoint{0.653114in}{0.835246in}}%
\pgfpathlineto{\pgfqpoint{0.655771in}{0.836863in}}%
\pgfpathlineto{\pgfqpoint{0.671428in}{0.845223in}}%
\pgfpathlineto{\pgfqpoint{0.679471in}{0.848857in}}%
\pgfpathlineto{\pgfqpoint{0.687084in}{0.852297in}}%
\pgfpathlineto{\pgfqpoint{0.702741in}{0.858063in}}%
\pgfpathlineto{\pgfqpoint{0.718395in}{0.862468in}}%
\pgfpathlineto{\pgfqpoint{0.718397in}{0.862469in}}%
\pgfpathlineto{\pgfqpoint{0.734054in}{0.865614in}}%
\pgfpathlineto{\pgfqpoint{0.749710in}{0.867360in}}%
\pgfpathlineto{\pgfqpoint{0.765367in}{0.867709in}}%
\pgfpathlineto{\pgfqpoint{0.781024in}{0.866661in}}%
\pgfpathlineto{\pgfqpoint{0.796680in}{0.864216in}}%
\pgfpathlineto{\pgfqpoint{0.803829in}{0.862468in}}%
\pgfpathlineto{\pgfqpoint{0.812337in}{0.860436in}}%
\pgfpathlineto{\pgfqpoint{0.827993in}{0.855350in}}%
\pgfpathlineto{\pgfqpoint{0.843650in}{0.848903in}}%
\pgfpathlineto{\pgfqpoint{0.843743in}{0.848857in}}%
\pgfpathlineto{\pgfqpoint{0.859306in}{0.841210in}}%
\pgfpathlineto{\pgfqpoint{0.869717in}{0.835246in}}%
\pgfpathlineto{\pgfqpoint{0.874963in}{0.832181in}}%
\pgfpathlineto{\pgfqpoint{0.890620in}{0.821827in}}%
\pgfpathlineto{\pgfqpoint{0.890884in}{0.821635in}}%
\pgfpathlineto{\pgfqpoint{0.906276in}{0.809985in}}%
\pgfpathlineto{\pgfqpoint{0.908670in}{0.808024in}}%
\pgfpathlineto{\pgfqpoint{0.921933in}{0.796494in}}%
\pgfpathlineto{\pgfqpoint{0.924189in}{0.794413in}}%
\pgfpathlineto{\pgfqpoint{0.937589in}{0.781031in}}%
\pgfpathlineto{\pgfqpoint{0.937810in}{0.780802in}}%
\pgfpathlineto{\pgfqpoint{0.949720in}{0.767191in}}%
\pgfpathlineto{\pgfqpoint{0.953246in}{0.762630in}}%
\pgfpathlineto{\pgfqpoint{0.960106in}{0.753579in}}%
\pgfpathlineto{\pgfqpoint{0.968902in}{0.740049in}}%
\pgfpathlineto{\pgfqpoint{0.968955in}{0.739968in}}%
\pgfpathlineto{\pgfqpoint{0.976371in}{0.726357in}}%
\pgfpathlineto{\pgfqpoint{0.982222in}{0.712746in}}%
\pgfpathlineto{\pgfqpoint{0.984559in}{0.705350in}}%
\pgfpathlineto{\pgfqpoint{0.986570in}{0.699135in}}%
\pgfpathlineto{\pgfqpoint{0.989382in}{0.685524in}}%
\pgfpathlineto{\pgfqpoint{0.990587in}{0.671913in}}%
\pgfpathlineto{\pgfqpoint{0.990185in}{0.658302in}}%
\pgfpathlineto{\pgfqpoint{0.988177in}{0.644691in}}%
\pgfpathlineto{\pgfqpoint{0.984560in}{0.631079in}}%
\pgfpathlineto{\pgfqpoint{0.984559in}{0.631077in}}%
\pgfpathlineto{\pgfqpoint{0.979492in}{0.617468in}}%
\pgfpathlineto{\pgfqpoint{0.972859in}{0.603857in}}%
\pgfpathlineto{\pgfqpoint{0.968902in}{0.597238in}}%
\pgfpathlineto{\pgfqpoint{0.964722in}{0.590246in}}%
\pgfpathlineto{\pgfqpoint{0.955106in}{0.576635in}}%
\pgfpathlineto{\pgfqpoint{0.953246in}{0.574325in}}%
\pgfpathlineto{\pgfqpoint{0.943952in}{0.563024in}}%
\pgfpathlineto{\pgfqpoint{0.937589in}{0.556105in}}%
\pgfpathlineto{\pgfqpoint{0.931186in}{0.549413in}}%
\pgfpathlineto{\pgfqpoint{0.921933in}{0.540565in}}%
\pgfpathlineto{\pgfqpoint{0.916651in}{0.535802in}}%
\pgfpathlineto{\pgfqpoint{0.906276in}{0.527064in}}%
\pgfpathlineto{\pgfqpoint{0.900018in}{0.522191in}}%
\pgfpathlineto{\pgfqpoint{0.890620in}{0.515222in}}%
\pgfpathlineto{\pgfqpoint{0.880719in}{0.508579in}}%
\pgfpathlineto{\pgfqpoint{0.874963in}{0.504833in}}%
\pgfpathlineto{\pgfqpoint{0.859306in}{0.495821in}}%
\pgfpathlineto{\pgfqpoint{0.857594in}{0.494968in}}%
\pgfpathlineto{\pgfqpoint{0.843650in}{0.488092in}}%
\pgfpathlineto{\pgfqpoint{0.827993in}{0.481706in}}%
\pgfpathlineto{\pgfqpoint{0.826922in}{0.481357in}}%
\pgfpathlineto{\pgfqpoint{0.812337in}{0.476558in}}%
\pgfpathlineto{\pgfqpoint{0.796680in}{0.472781in}}%
\pgfpathlineto{\pgfqpoint{0.781024in}{0.470379in}}%
\pgfpathlineto{\pgfqpoint{0.765367in}{0.469351in}}%
\pgfpathlineto{\pgfqpoint{0.749710in}{0.469693in}}%
\pgfpathlineto{\pgfqpoint{0.734054in}{0.471408in}}%
\pgfpathlineto{\pgfqpoint{0.718397in}{0.474497in}}%
\pgfpathlineto{\pgfqpoint{0.702741in}{0.478963in}}%
\pgfpathlineto{\pgfqpoint{0.696283in}{0.481357in}}%
\pgfpathclose%
\pgfpathmoveto{\pgfqpoint{1.532539in}{0.398990in}}%
\pgfpathlineto{\pgfqpoint{1.548195in}{0.399531in}}%
\pgfpathlineto{\pgfqpoint{1.549123in}{0.399691in}}%
\pgfpathlineto{\pgfqpoint{1.563852in}{0.401831in}}%
\pgfpathlineto{\pgfqpoint{1.579508in}{0.405930in}}%
\pgfpathlineto{\pgfqpoint{1.595165in}{0.411855in}}%
\pgfpathlineto{\pgfqpoint{1.598114in}{0.413302in}}%
\pgfpathlineto{\pgfqpoint{1.610822in}{0.418820in}}%
\pgfpathlineto{\pgfqpoint{1.625940in}{0.426913in}}%
\pgfpathlineto{\pgfqpoint{1.626478in}{0.427176in}}%
\pgfpathlineto{\pgfqpoint{1.642135in}{0.436173in}}%
\pgfpathlineto{\pgfqpoint{1.648746in}{0.440524in}}%
\pgfpathlineto{\pgfqpoint{1.657791in}{0.446132in}}%
\pgfpathlineto{\pgfqpoint{1.669333in}{0.454135in}}%
\pgfpathlineto{\pgfqpoint{1.673448in}{0.456886in}}%
\pgfpathlineto{\pgfqpoint{1.688281in}{0.467746in}}%
\pgfpathlineto{\pgfqpoint{1.689104in}{0.468340in}}%
\pgfpathlineto{\pgfqpoint{1.704761in}{0.480443in}}%
\pgfpathlineto{\pgfqpoint{1.705882in}{0.481357in}}%
\pgfpathlineto{\pgfqpoint{1.720418in}{0.493262in}}%
\pgfpathlineto{\pgfqpoint{1.722423in}{0.494968in}}%
\pgfpathlineto{\pgfqpoint{1.736074in}{0.506836in}}%
\pgfpathlineto{\pgfqpoint{1.738037in}{0.508579in}}%
\pgfpathlineto{\pgfqpoint{1.751731in}{0.521216in}}%
\pgfpathlineto{\pgfqpoint{1.752782in}{0.522191in}}%
\pgfpathlineto{\pgfqpoint{1.766704in}{0.535802in}}%
\pgfpathlineto{\pgfqpoint{1.767387in}{0.536518in}}%
\pgfpathlineto{\pgfqpoint{1.779880in}{0.549413in}}%
\pgfpathlineto{\pgfqpoint{1.783044in}{0.552990in}}%
\pgfpathlineto{\pgfqpoint{1.792249in}{0.563024in}}%
\pgfpathlineto{\pgfqpoint{1.798700in}{0.570888in}}%
\pgfpathlineto{\pgfqpoint{1.803705in}{0.576635in}}%
\pgfpathlineto{\pgfqpoint{1.814054in}{0.590246in}}%
\pgfpathlineto{\pgfqpoint{1.814357in}{0.590714in}}%
\pgfpathlineto{\pgfqpoint{1.823666in}{0.603857in}}%
\pgfpathlineto{\pgfqpoint{1.830014in}{0.614905in}}%
\pgfpathlineto{\pgfqpoint{1.831677in}{0.617468in}}%
\pgfpathlineto{\pgfqpoint{1.838493in}{0.631079in}}%
\pgfpathlineto{\pgfqpoint{1.843207in}{0.644691in}}%
\pgfpathlineto{\pgfqpoint{1.845670in}{0.657495in}}%
\pgfpathlineto{\pgfqpoint{1.845854in}{0.658302in}}%
\pgfpathlineto{\pgfqpoint{1.846476in}{0.671913in}}%
\pgfpathlineto{\pgfqpoint{1.845670in}{0.677797in}}%
\pgfpathlineto{\pgfqpoint{1.844778in}{0.685524in}}%
\pgfpathlineto{\pgfqpoint{1.841113in}{0.699135in}}%
\pgfpathlineto{\pgfqpoint{1.835348in}{0.712746in}}%
\pgfpathlineto{\pgfqpoint{1.830014in}{0.722006in}}%
\pgfpathlineto{\pgfqpoint{1.827797in}{0.726357in}}%
\pgfpathlineto{\pgfqpoint{1.819073in}{0.739968in}}%
\pgfpathlineto{\pgfqpoint{1.814357in}{0.746099in}}%
\pgfpathlineto{\pgfqpoint{1.809085in}{0.753579in}}%
\pgfpathlineto{\pgfqpoint{1.798700in}{0.766249in}}%
\pgfpathlineto{\pgfqpoint{1.797974in}{0.767191in}}%
\pgfpathlineto{\pgfqpoint{1.786152in}{0.780802in}}%
\pgfpathlineto{\pgfqpoint{1.783044in}{0.784043in}}%
\pgfpathlineto{\pgfqpoint{1.773444in}{0.794413in}}%
\pgfpathlineto{\pgfqpoint{1.767387in}{0.800445in}}%
\pgfpathlineto{\pgfqpoint{1.759893in}{0.808024in}}%
\pgfpathlineto{\pgfqpoint{1.751731in}{0.815770in}}%
\pgfpathlineto{\pgfqpoint{1.745530in}{0.821635in}}%
\pgfpathlineto{\pgfqpoint{1.736074in}{0.830173in}}%
\pgfpathlineto{\pgfqpoint{1.730336in}{0.835246in}}%
\pgfpathlineto{\pgfqpoint{1.720418in}{0.843760in}}%
\pgfpathlineto{\pgfqpoint{1.714250in}{0.848857in}}%
\pgfpathlineto{\pgfqpoint{1.704761in}{0.856604in}}%
\pgfpathlineto{\pgfqpoint{1.697172in}{0.862468in}}%
\pgfpathlineto{\pgfqpoint{1.689104in}{0.868738in}}%
\pgfpathlineto{\pgfqpoint{1.678950in}{0.876079in}}%
\pgfpathlineto{\pgfqpoint{1.673448in}{0.880159in}}%
\pgfpathlineto{\pgfqpoint{1.659372in}{0.889691in}}%
\pgfpathlineto{\pgfqpoint{1.657791in}{0.890812in}}%
\pgfpathlineto{\pgfqpoint{1.642135in}{0.900801in}}%
\pgfpathlineto{\pgfqpoint{1.637649in}{0.903302in}}%
\pgfpathlineto{\pgfqpoint{1.626478in}{0.910007in}}%
\pgfpathlineto{\pgfqpoint{1.612877in}{0.916913in}}%
\pgfpathlineto{\pgfqpoint{1.610822in}{0.918070in}}%
\pgfpathlineto{\pgfqpoint{1.595165in}{0.925293in}}%
\pgfpathlineto{\pgfqpoint{1.580330in}{0.930524in}}%
\pgfpathlineto{\pgfqpoint{1.579508in}{0.930858in}}%
\pgfpathlineto{\pgfqpoint{1.563852in}{0.935302in}}%
\pgfpathlineto{\pgfqpoint{1.548195in}{0.937769in}}%
\pgfpathlineto{\pgfqpoint{1.532539in}{0.938263in}}%
\pgfpathlineto{\pgfqpoint{1.516882in}{0.936783in}}%
\pgfpathlineto{\pgfqpoint{1.501226in}{0.933327in}}%
\pgfpathlineto{\pgfqpoint{1.493117in}{0.930524in}}%
\pgfpathlineto{\pgfqpoint{1.485569in}{0.928265in}}%
\pgfpathlineto{\pgfqpoint{1.469913in}{0.921894in}}%
\pgfpathlineto{\pgfqpoint{1.460204in}{0.916913in}}%
\pgfpathlineto{\pgfqpoint{1.454256in}{0.914163in}}%
\pgfpathlineto{\pgfqpoint{1.438599in}{0.905473in}}%
\pgfpathlineto{\pgfqpoint{1.435216in}{0.903302in}}%
\pgfpathlineto{\pgfqpoint{1.422943in}{0.895976in}}%
\pgfpathlineto{\pgfqpoint{1.413635in}{0.889691in}}%
\pgfpathlineto{\pgfqpoint{1.407286in}{0.885603in}}%
\pgfpathlineto{\pgfqpoint{1.393917in}{0.876079in}}%
\pgfpathlineto{\pgfqpoint{1.391630in}{0.874492in}}%
\pgfpathlineto{\pgfqpoint{1.375973in}{0.862709in}}%
\pgfpathlineto{\pgfqpoint{1.375671in}{0.862468in}}%
\pgfpathlineto{\pgfqpoint{1.360317in}{0.850260in}}%
\pgfpathlineto{\pgfqpoint{1.358632in}{0.848857in}}%
\pgfpathlineto{\pgfqpoint{1.344660in}{0.837068in}}%
\pgfpathlineto{\pgfqpoint{1.342564in}{0.835246in}}%
\pgfpathlineto{\pgfqpoint{1.329003in}{0.823099in}}%
\pgfpathlineto{\pgfqpoint{1.327390in}{0.821635in}}%
\pgfpathlineto{\pgfqpoint{1.313347in}{0.808286in}}%
\pgfpathlineto{\pgfqpoint{1.313070in}{0.808024in}}%
\pgfpathlineto{\pgfqpoint{1.299517in}{0.794413in}}%
\pgfpathlineto{\pgfqpoint{1.297690in}{0.792424in}}%
\pgfpathlineto{\pgfqpoint{1.286736in}{0.780802in}}%
\pgfpathlineto{\pgfqpoint{1.282034in}{0.775282in}}%
\pgfpathlineto{\pgfqpoint{1.274804in}{0.767191in}}%
\pgfpathlineto{\pgfqpoint{1.266377in}{0.756521in}}%
\pgfpathlineto{\pgfqpoint{1.263880in}{0.753579in}}%
\pgfpathlineto{\pgfqpoint{1.253883in}{0.739968in}}%
\pgfpathlineto{\pgfqpoint{1.250721in}{0.734797in}}%
\pgfpathlineto{\pgfqpoint{1.244990in}{0.726357in}}%
\pgfpathlineto{\pgfqpoint{1.237663in}{0.712746in}}%
\pgfpathlineto{\pgfqpoint{1.235064in}{0.706184in}}%
\pgfpathlineto{\pgfqpoint{1.231839in}{0.699135in}}%
\pgfpathlineto{\pgfqpoint{1.227865in}{0.685524in}}%
\pgfpathlineto{\pgfqpoint{1.226162in}{0.671913in}}%
\pgfpathlineto{\pgfqpoint{1.226730in}{0.658302in}}%
\pgfpathlineto{\pgfqpoint{1.229568in}{0.644691in}}%
\pgfpathlineto{\pgfqpoint{1.234680in}{0.631079in}}%
\pgfpathlineto{\pgfqpoint{1.235064in}{0.630365in}}%
\pgfpathlineto{\pgfqpoint{1.241081in}{0.617468in}}%
\pgfpathlineto{\pgfqpoint{1.249390in}{0.603857in}}%
\pgfpathlineto{\pgfqpoint{1.250721in}{0.602070in}}%
\pgfpathlineto{\pgfqpoint{1.258664in}{0.590246in}}%
\pgfpathlineto{\pgfqpoint{1.266377in}{0.580534in}}%
\pgfpathlineto{\pgfqpoint{1.269254in}{0.576635in}}%
\pgfpathlineto{\pgfqpoint{1.280744in}{0.563024in}}%
\pgfpathlineto{\pgfqpoint{1.282034in}{0.561650in}}%
\pgfpathlineto{\pgfqpoint{1.292998in}{0.549413in}}%
\pgfpathlineto{\pgfqpoint{1.297690in}{0.544629in}}%
\pgfpathlineto{\pgfqpoint{1.306135in}{0.535802in}}%
\pgfpathlineto{\pgfqpoint{1.313347in}{0.528788in}}%
\pgfpathlineto{\pgfqpoint{1.320093in}{0.522191in}}%
\pgfpathlineto{\pgfqpoint{1.329003in}{0.513941in}}%
\pgfpathlineto{\pgfqpoint{1.334866in}{0.508579in}}%
\pgfpathlineto{\pgfqpoint{1.344660in}{0.499957in}}%
\pgfpathlineto{\pgfqpoint{1.350496in}{0.494968in}}%
\pgfpathlineto{\pgfqpoint{1.360317in}{0.486748in}}%
\pgfpathlineto{\pgfqpoint{1.367063in}{0.481357in}}%
\pgfpathlineto{\pgfqpoint{1.375973in}{0.474261in}}%
\pgfpathlineto{\pgfqpoint{1.384690in}{0.467746in}}%
\pgfpathlineto{\pgfqpoint{1.391630in}{0.462480in}}%
\pgfpathlineto{\pgfqpoint{1.403558in}{0.454135in}}%
\pgfpathlineto{\pgfqpoint{1.407286in}{0.451432in}}%
\pgfpathlineto{\pgfqpoint{1.422943in}{0.441155in}}%
\pgfpathlineto{\pgfqpoint{1.424026in}{0.440524in}}%
\pgfpathlineto{\pgfqpoint{1.438599in}{0.431496in}}%
\pgfpathlineto{\pgfqpoint{1.447204in}{0.426913in}}%
\pgfpathlineto{\pgfqpoint{1.454256in}{0.422813in}}%
\pgfpathlineto{\pgfqpoint{1.469913in}{0.415228in}}%
\pgfpathlineto{\pgfqpoint{1.474917in}{0.413302in}}%
\pgfpathlineto{\pgfqpoint{1.485569in}{0.408664in}}%
\pgfpathlineto{\pgfqpoint{1.501226in}{0.403653in}}%
\pgfpathlineto{\pgfqpoint{1.516882in}{0.400466in}}%
\pgfpathlineto{\pgfqpoint{1.525770in}{0.399691in}}%
\pgfpathlineto{\pgfqpoint{1.532539in}{0.398990in}}%
\pgfpathclose%
\pgfpathmoveto{\pgfqpoint{1.470984in}{0.481357in}}%
\pgfpathlineto{\pgfqpoint{1.469913in}{0.481706in}}%
\pgfpathlineto{\pgfqpoint{1.454256in}{0.488092in}}%
\pgfpathlineto{\pgfqpoint{1.440312in}{0.494968in}}%
\pgfpathlineto{\pgfqpoint{1.438599in}{0.495821in}}%
\pgfpathlineto{\pgfqpoint{1.422943in}{0.504833in}}%
\pgfpathlineto{\pgfqpoint{1.417187in}{0.508579in}}%
\pgfpathlineto{\pgfqpoint{1.407286in}{0.515222in}}%
\pgfpathlineto{\pgfqpoint{1.397887in}{0.522191in}}%
\pgfpathlineto{\pgfqpoint{1.391630in}{0.527064in}}%
\pgfpathlineto{\pgfqpoint{1.381255in}{0.535802in}}%
\pgfpathlineto{\pgfqpoint{1.375973in}{0.540565in}}%
\pgfpathlineto{\pgfqpoint{1.366720in}{0.549413in}}%
\pgfpathlineto{\pgfqpoint{1.360317in}{0.556105in}}%
\pgfpathlineto{\pgfqpoint{1.353954in}{0.563024in}}%
\pgfpathlineto{\pgfqpoint{1.344660in}{0.574325in}}%
\pgfpathlineto{\pgfqpoint{1.342800in}{0.576635in}}%
\pgfpathlineto{\pgfqpoint{1.333184in}{0.590246in}}%
\pgfpathlineto{\pgfqpoint{1.329003in}{0.597238in}}%
\pgfpathlineto{\pgfqpoint{1.325047in}{0.603857in}}%
\pgfpathlineto{\pgfqpoint{1.318414in}{0.617468in}}%
\pgfpathlineto{\pgfqpoint{1.313347in}{0.631077in}}%
\pgfpathlineto{\pgfqpoint{1.313346in}{0.631079in}}%
\pgfpathlineto{\pgfqpoint{1.309729in}{0.644691in}}%
\pgfpathlineto{\pgfqpoint{1.307720in}{0.658302in}}%
\pgfpathlineto{\pgfqpoint{1.307319in}{0.671913in}}%
\pgfpathlineto{\pgfqpoint{1.308524in}{0.685524in}}%
\pgfpathlineto{\pgfqpoint{1.311336in}{0.699135in}}%
\pgfpathlineto{\pgfqpoint{1.313347in}{0.705350in}}%
\pgfpathlineto{\pgfqpoint{1.315684in}{0.712746in}}%
\pgfpathlineto{\pgfqpoint{1.321534in}{0.726357in}}%
\pgfpathlineto{\pgfqpoint{1.328951in}{0.739968in}}%
\pgfpathlineto{\pgfqpoint{1.329003in}{0.740049in}}%
\pgfpathlineto{\pgfqpoint{1.337800in}{0.753579in}}%
\pgfpathlineto{\pgfqpoint{1.344660in}{0.762630in}}%
\pgfpathlineto{\pgfqpoint{1.348186in}{0.767191in}}%
\pgfpathlineto{\pgfqpoint{1.360096in}{0.780802in}}%
\pgfpathlineto{\pgfqpoint{1.360317in}{0.781031in}}%
\pgfpathlineto{\pgfqpoint{1.373717in}{0.794413in}}%
\pgfpathlineto{\pgfqpoint{1.375973in}{0.796494in}}%
\pgfpathlineto{\pgfqpoint{1.389236in}{0.808024in}}%
\pgfpathlineto{\pgfqpoint{1.391630in}{0.809985in}}%
\pgfpathlineto{\pgfqpoint{1.407022in}{0.821635in}}%
\pgfpathlineto{\pgfqpoint{1.407286in}{0.821827in}}%
\pgfpathlineto{\pgfqpoint{1.422943in}{0.832181in}}%
\pgfpathlineto{\pgfqpoint{1.428188in}{0.835246in}}%
\pgfpathlineto{\pgfqpoint{1.438599in}{0.841210in}}%
\pgfpathlineto{\pgfqpoint{1.454163in}{0.848857in}}%
\pgfpathlineto{\pgfqpoint{1.454256in}{0.848903in}}%
\pgfpathlineto{\pgfqpoint{1.469913in}{0.855350in}}%
\pgfpathlineto{\pgfqpoint{1.485569in}{0.860436in}}%
\pgfpathlineto{\pgfqpoint{1.494077in}{0.862468in}}%
\pgfpathlineto{\pgfqpoint{1.501226in}{0.864216in}}%
\pgfpathlineto{\pgfqpoint{1.516882in}{0.866661in}}%
\pgfpathlineto{\pgfqpoint{1.532539in}{0.867709in}}%
\pgfpathlineto{\pgfqpoint{1.548195in}{0.867360in}}%
\pgfpathlineto{\pgfqpoint{1.563852in}{0.865614in}}%
\pgfpathlineto{\pgfqpoint{1.579508in}{0.862469in}}%
\pgfpathlineto{\pgfqpoint{1.579511in}{0.862468in}}%
\pgfpathlineto{\pgfqpoint{1.595165in}{0.858063in}}%
\pgfpathlineto{\pgfqpoint{1.610822in}{0.852297in}}%
\pgfpathlineto{\pgfqpoint{1.618435in}{0.848857in}}%
\pgfpathlineto{\pgfqpoint{1.626478in}{0.845223in}}%
\pgfpathlineto{\pgfqpoint{1.642135in}{0.836863in}}%
\pgfpathlineto{\pgfqpoint{1.644792in}{0.835246in}}%
\pgfpathlineto{\pgfqpoint{1.657791in}{0.827167in}}%
\pgfpathlineto{\pgfqpoint{1.665750in}{0.821635in}}%
\pgfpathlineto{\pgfqpoint{1.673448in}{0.816068in}}%
\pgfpathlineto{\pgfqpoint{1.683625in}{0.808024in}}%
\pgfpathlineto{\pgfqpoint{1.689104in}{0.803432in}}%
\pgfpathlineto{\pgfqpoint{1.699156in}{0.794413in}}%
\pgfpathlineto{\pgfqpoint{1.704761in}{0.788973in}}%
\pgfpathlineto{\pgfqpoint{1.712777in}{0.780802in}}%
\pgfpathlineto{\pgfqpoint{1.720418in}{0.772195in}}%
\pgfpathlineto{\pgfqpoint{1.724726in}{0.767191in}}%
\pgfpathlineto{\pgfqpoint{1.735094in}{0.753579in}}%
\pgfpathlineto{\pgfqpoint{1.736074in}{0.752091in}}%
\pgfpathlineto{\pgfqpoint{1.743983in}{0.739968in}}%
\pgfpathlineto{\pgfqpoint{1.751330in}{0.726357in}}%
\pgfpathlineto{\pgfqpoint{1.751731in}{0.725426in}}%
\pgfpathlineto{\pgfqpoint{1.757251in}{0.712746in}}%
\pgfpathlineto{\pgfqpoint{1.761596in}{0.699135in}}%
\pgfpathlineto{\pgfqpoint{1.764358in}{0.685524in}}%
\pgfpathlineto{\pgfqpoint{1.765542in}{0.671913in}}%
\pgfpathlineto{\pgfqpoint{1.765147in}{0.658302in}}%
\pgfpathlineto{\pgfqpoint{1.763175in}{0.644691in}}%
\pgfpathlineto{\pgfqpoint{1.759621in}{0.631079in}}%
\pgfpathlineto{\pgfqpoint{1.754484in}{0.617468in}}%
\pgfpathlineto{\pgfqpoint{1.751731in}{0.611854in}}%
\pgfpathlineto{\pgfqpoint{1.747851in}{0.603857in}}%
\pgfpathlineto{\pgfqpoint{1.739729in}{0.590246in}}%
\pgfpathlineto{\pgfqpoint{1.736074in}{0.585045in}}%
\pgfpathlineto{\pgfqpoint{1.730100in}{0.576635in}}%
\pgfpathlineto{\pgfqpoint{1.720418in}{0.564759in}}%
\pgfpathlineto{\pgfqpoint{1.718961in}{0.563024in}}%
\pgfpathlineto{\pgfqpoint{1.706221in}{0.549413in}}%
\pgfpathlineto{\pgfqpoint{1.704761in}{0.547983in}}%
\pgfpathlineto{\pgfqpoint{1.691685in}{0.535802in}}%
\pgfpathlineto{\pgfqpoint{1.689104in}{0.533558in}}%
\pgfpathlineto{\pgfqpoint{1.675092in}{0.522191in}}%
\pgfpathlineto{\pgfqpoint{1.673448in}{0.520921in}}%
\pgfpathlineto{\pgfqpoint{1.657791in}{0.509846in}}%
\pgfpathlineto{\pgfqpoint{1.655796in}{0.508579in}}%
\pgfpathlineto{\pgfqpoint{1.642135in}{0.500162in}}%
\pgfpathlineto{\pgfqpoint{1.632461in}{0.494968in}}%
\pgfpathlineto{\pgfqpoint{1.626478in}{0.491791in}}%
\pgfpathlineto{\pgfqpoint{1.610822in}{0.484730in}}%
\pgfpathlineto{\pgfqpoint{1.601623in}{0.481357in}}%
\pgfpathlineto{\pgfqpoint{1.595165in}{0.478963in}}%
\pgfpathlineto{\pgfqpoint{1.579508in}{0.474497in}}%
\pgfpathlineto{\pgfqpoint{1.563852in}{0.471408in}}%
\pgfpathlineto{\pgfqpoint{1.548195in}{0.469693in}}%
\pgfpathlineto{\pgfqpoint{1.532539in}{0.469351in}}%
\pgfpathlineto{\pgfqpoint{1.516882in}{0.470379in}}%
\pgfpathlineto{\pgfqpoint{1.501226in}{0.472781in}}%
\pgfpathlineto{\pgfqpoint{1.485569in}{0.476558in}}%
\pgfpathlineto{\pgfqpoint{1.470984in}{0.481357in}}%
\pgfpathclose%
\pgfpathmoveto{\pgfqpoint{0.718397in}{1.079912in}}%
\pgfpathlineto{\pgfqpoint{0.734054in}{1.075468in}}%
\pgfpathlineto{\pgfqpoint{0.749710in}{1.073001in}}%
\pgfpathlineto{\pgfqpoint{0.765367in}{1.072507in}}%
\pgfpathlineto{\pgfqpoint{0.781024in}{1.073987in}}%
\pgfpathlineto{\pgfqpoint{0.796680in}{1.077443in}}%
\pgfpathlineto{\pgfqpoint{0.804789in}{1.080246in}}%
\pgfpathlineto{\pgfqpoint{0.812337in}{1.082505in}}%
\pgfpathlineto{\pgfqpoint{0.827993in}{1.088876in}}%
\pgfpathlineto{\pgfqpoint{0.837702in}{1.093857in}}%
\pgfpathlineto{\pgfqpoint{0.843650in}{1.096607in}}%
\pgfpathlineto{\pgfqpoint{0.859306in}{1.105297in}}%
\pgfpathlineto{\pgfqpoint{0.862690in}{1.107468in}}%
\pgfpathlineto{\pgfqpoint{0.874963in}{1.114794in}}%
\pgfpathlineto{\pgfqpoint{0.884270in}{1.121079in}}%
\pgfpathlineto{\pgfqpoint{0.890620in}{1.125167in}}%
\pgfpathlineto{\pgfqpoint{0.903988in}{1.134691in}}%
\pgfpathlineto{\pgfqpoint{0.906276in}{1.136278in}}%
\pgfpathlineto{\pgfqpoint{0.921933in}{1.148061in}}%
\pgfpathlineto{\pgfqpoint{0.922234in}{1.148302in}}%
\pgfpathlineto{\pgfqpoint{0.937589in}{1.160510in}}%
\pgfpathlineto{\pgfqpoint{0.939273in}{1.161913in}}%
\pgfpathlineto{\pgfqpoint{0.953246in}{1.173702in}}%
\pgfpathlineto{\pgfqpoint{0.955341in}{1.175524in}}%
\pgfpathlineto{\pgfqpoint{0.968902in}{1.187671in}}%
\pgfpathlineto{\pgfqpoint{0.970516in}{1.189135in}}%
\pgfpathlineto{\pgfqpoint{0.984559in}{1.202484in}}%
\pgfpathlineto{\pgfqpoint{0.984836in}{1.202746in}}%
\pgfpathlineto{\pgfqpoint{0.998389in}{1.216357in}}%
\pgfpathlineto{\pgfqpoint{1.000216in}{1.218346in}}%
\pgfpathlineto{\pgfqpoint{1.011170in}{1.229968in}}%
\pgfpathlineto{\pgfqpoint{1.015872in}{1.235488in}}%
\pgfpathlineto{\pgfqpoint{1.023102in}{1.243579in}}%
\pgfpathlineto{\pgfqpoint{1.031529in}{1.254249in}}%
\pgfpathlineto{\pgfqpoint{1.034026in}{1.257191in}}%
\pgfpathlineto{\pgfqpoint{1.044022in}{1.270802in}}%
\pgfpathlineto{\pgfqpoint{1.047185in}{1.275973in}}%
\pgfpathlineto{\pgfqpoint{1.052916in}{1.284413in}}%
\pgfpathlineto{\pgfqpoint{1.060243in}{1.298024in}}%
\pgfpathlineto{\pgfqpoint{1.062842in}{1.304586in}}%
\pgfpathlineto{\pgfqpoint{1.066067in}{1.311635in}}%
\pgfpathlineto{\pgfqpoint{1.070041in}{1.325246in}}%
\pgfpathlineto{\pgfqpoint{1.071743in}{1.338857in}}%
\pgfpathlineto{\pgfqpoint{1.071176in}{1.352468in}}%
\pgfpathlineto{\pgfqpoint{1.068338in}{1.366079in}}%
\pgfpathlineto{\pgfqpoint{1.063226in}{1.379691in}}%
\pgfpathlineto{\pgfqpoint{1.062842in}{1.380405in}}%
\pgfpathlineto{\pgfqpoint{1.056824in}{1.393302in}}%
\pgfpathlineto{\pgfqpoint{1.048516in}{1.406913in}}%
\pgfpathlineto{\pgfqpoint{1.047185in}{1.408700in}}%
\pgfpathlineto{\pgfqpoint{1.039241in}{1.420524in}}%
\pgfpathlineto{\pgfqpoint{1.031529in}{1.430236in}}%
\pgfpathlineto{\pgfqpoint{1.028652in}{1.434135in}}%
\pgfpathlineto{\pgfqpoint{1.017162in}{1.447746in}}%
\pgfpathlineto{\pgfqpoint{1.015872in}{1.449120in}}%
\pgfpathlineto{\pgfqpoint{1.004908in}{1.461357in}}%
\pgfpathlineto{\pgfqpoint{1.000216in}{1.466141in}}%
\pgfpathlineto{\pgfqpoint{0.991771in}{1.474968in}}%
\pgfpathlineto{\pgfqpoint{0.984559in}{1.481982in}}%
\pgfpathlineto{\pgfqpoint{0.977813in}{1.488579in}}%
\pgfpathlineto{\pgfqpoint{0.968902in}{1.496829in}}%
\pgfpathlineto{\pgfqpoint{0.963040in}{1.502191in}}%
\pgfpathlineto{\pgfqpoint{0.953246in}{1.510813in}}%
\pgfpathlineto{\pgfqpoint{0.947410in}{1.515802in}}%
\pgfpathlineto{\pgfqpoint{0.937589in}{1.524022in}}%
\pgfpathlineto{\pgfqpoint{0.930843in}{1.529413in}}%
\pgfpathlineto{\pgfqpoint{0.921933in}{1.536509in}}%
\pgfpathlineto{\pgfqpoint{0.913215in}{1.543024in}}%
\pgfpathlineto{\pgfqpoint{0.906276in}{1.548290in}}%
\pgfpathlineto{\pgfqpoint{0.894348in}{1.556635in}}%
\pgfpathlineto{\pgfqpoint{0.890620in}{1.559338in}}%
\pgfpathlineto{\pgfqpoint{0.874963in}{1.569615in}}%
\pgfpathlineto{\pgfqpoint{0.873880in}{1.570246in}}%
\pgfpathlineto{\pgfqpoint{0.859306in}{1.579274in}}%
\pgfpathlineto{\pgfqpoint{0.850702in}{1.583857in}}%
\pgfpathlineto{\pgfqpoint{0.843650in}{1.587957in}}%
\pgfpathlineto{\pgfqpoint{0.827993in}{1.595542in}}%
\pgfpathlineto{\pgfqpoint{0.822989in}{1.597468in}}%
\pgfpathlineto{\pgfqpoint{0.812337in}{1.602106in}}%
\pgfpathlineto{\pgfqpoint{0.796680in}{1.607117in}}%
\pgfpathlineto{\pgfqpoint{0.781024in}{1.610304in}}%
\pgfpathlineto{\pgfqpoint{0.772136in}{1.611079in}}%
\pgfpathlineto{\pgfqpoint{0.765367in}{1.611780in}}%
\pgfpathlineto{\pgfqpoint{0.749710in}{1.611239in}}%
\pgfpathlineto{\pgfqpoint{0.748782in}{1.611079in}}%
\pgfpathlineto{\pgfqpoint{0.734054in}{1.608939in}}%
\pgfpathlineto{\pgfqpoint{0.718397in}{1.604840in}}%
\pgfpathlineto{\pgfqpoint{0.702741in}{1.598915in}}%
\pgfpathlineto{\pgfqpoint{0.699792in}{1.597468in}}%
\pgfpathlineto{\pgfqpoint{0.687084in}{1.591950in}}%
\pgfpathlineto{\pgfqpoint{0.671966in}{1.583857in}}%
\pgfpathlineto{\pgfqpoint{0.671428in}{1.583594in}}%
\pgfpathlineto{\pgfqpoint{0.655771in}{1.574597in}}%
\pgfpathlineto{\pgfqpoint{0.649160in}{1.570246in}}%
\pgfpathlineto{\pgfqpoint{0.640115in}{1.564638in}}%
\pgfpathlineto{\pgfqpoint{0.628573in}{1.556635in}}%
\pgfpathlineto{\pgfqpoint{0.624458in}{1.553884in}}%
\pgfpathlineto{\pgfqpoint{0.609625in}{1.543024in}}%
\pgfpathlineto{\pgfqpoint{0.608801in}{1.542430in}}%
\pgfpathlineto{\pgfqpoint{0.593145in}{1.530327in}}%
\pgfpathlineto{\pgfqpoint{0.592024in}{1.529413in}}%
\pgfpathlineto{\pgfqpoint{0.577488in}{1.517508in}}%
\pgfpathlineto{\pgfqpoint{0.575483in}{1.515802in}}%
\pgfpathlineto{\pgfqpoint{0.561832in}{1.503934in}}%
\pgfpathlineto{\pgfqpoint{0.559869in}{1.502191in}}%
\pgfpathlineto{\pgfqpoint{0.546175in}{1.489554in}}%
\pgfpathlineto{\pgfqpoint{0.545124in}{1.488579in}}%
\pgfpathlineto{\pgfqpoint{0.531201in}{1.474968in}}%
\pgfpathlineto{\pgfqpoint{0.530519in}{1.474252in}}%
\pgfpathlineto{\pgfqpoint{0.518026in}{1.461357in}}%
\pgfpathlineto{\pgfqpoint{0.514862in}{1.457780in}}%
\pgfpathlineto{\pgfqpoint{0.505657in}{1.447746in}}%
\pgfpathlineto{\pgfqpoint{0.499205in}{1.439882in}}%
\pgfpathlineto{\pgfqpoint{0.494201in}{1.434135in}}%
\pgfpathlineto{\pgfqpoint{0.483852in}{1.420524in}}%
\pgfpathlineto{\pgfqpoint{0.483549in}{1.420056in}}%
\pgfpathlineto{\pgfqpoint{0.474240in}{1.406913in}}%
\pgfpathlineto{\pgfqpoint{0.467892in}{1.395865in}}%
\pgfpathlineto{\pgfqpoint{0.466229in}{1.393302in}}%
\pgfpathlineto{\pgfqpoint{0.459413in}{1.379691in}}%
\pgfpathlineto{\pgfqpoint{0.454698in}{1.366079in}}%
\pgfpathlineto{\pgfqpoint{0.452236in}{1.353275in}}%
\pgfpathlineto{\pgfqpoint{0.452052in}{1.352468in}}%
\pgfpathlineto{\pgfqpoint{0.451429in}{1.338857in}}%
\pgfpathlineto{\pgfqpoint{0.452236in}{1.332973in}}%
\pgfpathlineto{\pgfqpoint{0.453128in}{1.325246in}}%
\pgfpathlineto{\pgfqpoint{0.456793in}{1.311635in}}%
\pgfpathlineto{\pgfqpoint{0.462558in}{1.298024in}}%
\pgfpathlineto{\pgfqpoint{0.467892in}{1.288764in}}%
\pgfpathlineto{\pgfqpoint{0.470108in}{1.284413in}}%
\pgfpathlineto{\pgfqpoint{0.478833in}{1.270802in}}%
\pgfpathlineto{\pgfqpoint{0.483549in}{1.264671in}}%
\pgfpathlineto{\pgfqpoint{0.488820in}{1.257191in}}%
\pgfpathlineto{\pgfqpoint{0.499205in}{1.244521in}}%
\pgfpathlineto{\pgfqpoint{0.499932in}{1.243579in}}%
\pgfpathlineto{\pgfqpoint{0.511753in}{1.229968in}}%
\pgfpathlineto{\pgfqpoint{0.514862in}{1.226727in}}%
\pgfpathlineto{\pgfqpoint{0.524461in}{1.216357in}}%
\pgfpathlineto{\pgfqpoint{0.530519in}{1.210325in}}%
\pgfpathlineto{\pgfqpoint{0.538013in}{1.202746in}}%
\pgfpathlineto{\pgfqpoint{0.546175in}{1.195000in}}%
\pgfpathlineto{\pgfqpoint{0.552376in}{1.189135in}}%
\pgfpathlineto{\pgfqpoint{0.561832in}{1.180597in}}%
\pgfpathlineto{\pgfqpoint{0.567570in}{1.175524in}}%
\pgfpathlineto{\pgfqpoint{0.577488in}{1.167010in}}%
\pgfpathlineto{\pgfqpoint{0.583655in}{1.161913in}}%
\pgfpathlineto{\pgfqpoint{0.593145in}{1.154166in}}%
\pgfpathlineto{\pgfqpoint{0.600734in}{1.148302in}}%
\pgfpathlineto{\pgfqpoint{0.608801in}{1.142032in}}%
\pgfpathlineto{\pgfqpoint{0.618955in}{1.134691in}}%
\pgfpathlineto{\pgfqpoint{0.624458in}{1.130611in}}%
\pgfpathlineto{\pgfqpoint{0.638534in}{1.121079in}}%
\pgfpathlineto{\pgfqpoint{0.640115in}{1.119958in}}%
\pgfpathlineto{\pgfqpoint{0.655771in}{1.109969in}}%
\pgfpathlineto{\pgfqpoint{0.660256in}{1.107468in}}%
\pgfpathlineto{\pgfqpoint{0.671428in}{1.100763in}}%
\pgfpathlineto{\pgfqpoint{0.685029in}{1.093857in}}%
\pgfpathlineto{\pgfqpoint{0.687084in}{1.092700in}}%
\pgfpathlineto{\pgfqpoint{0.702741in}{1.085477in}}%
\pgfpathlineto{\pgfqpoint{0.717576in}{1.080246in}}%
\pgfpathlineto{\pgfqpoint{0.718397in}{1.079912in}}%
\pgfpathclose%
\pgfpathmoveto{\pgfqpoint{0.718395in}{1.148302in}}%
\pgfpathlineto{\pgfqpoint{0.702741in}{1.152707in}}%
\pgfpathlineto{\pgfqpoint{0.687084in}{1.158473in}}%
\pgfpathlineto{\pgfqpoint{0.679471in}{1.161913in}}%
\pgfpathlineto{\pgfqpoint{0.671428in}{1.165547in}}%
\pgfpathlineto{\pgfqpoint{0.655771in}{1.173907in}}%
\pgfpathlineto{\pgfqpoint{0.653114in}{1.175524in}}%
\pgfpathlineto{\pgfqpoint{0.640115in}{1.183603in}}%
\pgfpathlineto{\pgfqpoint{0.632155in}{1.189135in}}%
\pgfpathlineto{\pgfqpoint{0.624458in}{1.194702in}}%
\pgfpathlineto{\pgfqpoint{0.614280in}{1.202746in}}%
\pgfpathlineto{\pgfqpoint{0.608801in}{1.207338in}}%
\pgfpathlineto{\pgfqpoint{0.598750in}{1.216357in}}%
\pgfpathlineto{\pgfqpoint{0.593145in}{1.221797in}}%
\pgfpathlineto{\pgfqpoint{0.585129in}{1.229968in}}%
\pgfpathlineto{\pgfqpoint{0.577488in}{1.238575in}}%
\pgfpathlineto{\pgfqpoint{0.573179in}{1.243579in}}%
\pgfpathlineto{\pgfqpoint{0.562812in}{1.257191in}}%
\pgfpathlineto{\pgfqpoint{0.561832in}{1.258679in}}%
\pgfpathlineto{\pgfqpoint{0.553922in}{1.270802in}}%
\pgfpathlineto{\pgfqpoint{0.546576in}{1.284413in}}%
\pgfpathlineto{\pgfqpoint{0.546175in}{1.285344in}}%
\pgfpathlineto{\pgfqpoint{0.540655in}{1.298024in}}%
\pgfpathlineto{\pgfqpoint{0.536310in}{1.311635in}}%
\pgfpathlineto{\pgfqpoint{0.533548in}{1.325246in}}%
\pgfpathlineto{\pgfqpoint{0.532364in}{1.338857in}}%
\pgfpathlineto{\pgfqpoint{0.532759in}{1.352468in}}%
\pgfpathlineto{\pgfqpoint{0.534731in}{1.366079in}}%
\pgfpathlineto{\pgfqpoint{0.538284in}{1.379691in}}%
\pgfpathlineto{\pgfqpoint{0.543422in}{1.393302in}}%
\pgfpathlineto{\pgfqpoint{0.546175in}{1.398916in}}%
\pgfpathlineto{\pgfqpoint{0.550055in}{1.406913in}}%
\pgfpathlineto{\pgfqpoint{0.558177in}{1.420524in}}%
\pgfpathlineto{\pgfqpoint{0.561832in}{1.425725in}}%
\pgfpathlineto{\pgfqpoint{0.567805in}{1.434135in}}%
\pgfpathlineto{\pgfqpoint{0.577488in}{1.446011in}}%
\pgfpathlineto{\pgfqpoint{0.578945in}{1.447746in}}%
\pgfpathlineto{\pgfqpoint{0.591685in}{1.461357in}}%
\pgfpathlineto{\pgfqpoint{0.593145in}{1.462787in}}%
\pgfpathlineto{\pgfqpoint{0.606221in}{1.474968in}}%
\pgfpathlineto{\pgfqpoint{0.608801in}{1.477212in}}%
\pgfpathlineto{\pgfqpoint{0.622813in}{1.488579in}}%
\pgfpathlineto{\pgfqpoint{0.624458in}{1.489849in}}%
\pgfpathlineto{\pgfqpoint{0.640115in}{1.500924in}}%
\pgfpathlineto{\pgfqpoint{0.642110in}{1.502191in}}%
\pgfpathlineto{\pgfqpoint{0.655771in}{1.510608in}}%
\pgfpathlineto{\pgfqpoint{0.665445in}{1.515802in}}%
\pgfpathlineto{\pgfqpoint{0.671428in}{1.518979in}}%
\pgfpathlineto{\pgfqpoint{0.687084in}{1.526040in}}%
\pgfpathlineto{\pgfqpoint{0.696283in}{1.529413in}}%
\pgfpathlineto{\pgfqpoint{0.702741in}{1.531807in}}%
\pgfpathlineto{\pgfqpoint{0.718397in}{1.536273in}}%
\pgfpathlineto{\pgfqpoint{0.734054in}{1.539362in}}%
\pgfpathlineto{\pgfqpoint{0.749710in}{1.541077in}}%
\pgfpathlineto{\pgfqpoint{0.765367in}{1.541419in}}%
\pgfpathlineto{\pgfqpoint{0.781024in}{1.540391in}}%
\pgfpathlineto{\pgfqpoint{0.796680in}{1.537989in}}%
\pgfpathlineto{\pgfqpoint{0.812337in}{1.534212in}}%
\pgfpathlineto{\pgfqpoint{0.826922in}{1.529413in}}%
\pgfpathlineto{\pgfqpoint{0.827993in}{1.529064in}}%
\pgfpathlineto{\pgfqpoint{0.843650in}{1.522678in}}%
\pgfpathlineto{\pgfqpoint{0.857594in}{1.515802in}}%
\pgfpathlineto{\pgfqpoint{0.859306in}{1.514949in}}%
\pgfpathlineto{\pgfqpoint{0.874963in}{1.505937in}}%
\pgfpathlineto{\pgfqpoint{0.880719in}{1.502191in}}%
\pgfpathlineto{\pgfqpoint{0.890620in}{1.495548in}}%
\pgfpathlineto{\pgfqpoint{0.900018in}{1.488579in}}%
\pgfpathlineto{\pgfqpoint{0.906276in}{1.483706in}}%
\pgfpathlineto{\pgfqpoint{0.916651in}{1.474968in}}%
\pgfpathlineto{\pgfqpoint{0.921933in}{1.470205in}}%
\pgfpathlineto{\pgfqpoint{0.931186in}{1.461357in}}%
\pgfpathlineto{\pgfqpoint{0.937589in}{1.454665in}}%
\pgfpathlineto{\pgfqpoint{0.943952in}{1.447746in}}%
\pgfpathlineto{\pgfqpoint{0.953246in}{1.436445in}}%
\pgfpathlineto{\pgfqpoint{0.955106in}{1.434135in}}%
\pgfpathlineto{\pgfqpoint{0.964722in}{1.420524in}}%
\pgfpathlineto{\pgfqpoint{0.968902in}{1.413532in}}%
\pgfpathlineto{\pgfqpoint{0.972859in}{1.406913in}}%
\pgfpathlineto{\pgfqpoint{0.979492in}{1.393302in}}%
\pgfpathlineto{\pgfqpoint{0.984559in}{1.379693in}}%
\pgfpathlineto{\pgfqpoint{0.984560in}{1.379691in}}%
\pgfpathlineto{\pgfqpoint{0.988177in}{1.366079in}}%
\pgfpathlineto{\pgfqpoint{0.990185in}{1.352468in}}%
\pgfpathlineto{\pgfqpoint{0.990587in}{1.338857in}}%
\pgfpathlineto{\pgfqpoint{0.989382in}{1.325246in}}%
\pgfpathlineto{\pgfqpoint{0.986570in}{1.311635in}}%
\pgfpathlineto{\pgfqpoint{0.984559in}{1.305420in}}%
\pgfpathlineto{\pgfqpoint{0.982222in}{1.298024in}}%
\pgfpathlineto{\pgfqpoint{0.976371in}{1.284413in}}%
\pgfpathlineto{\pgfqpoint{0.968955in}{1.270802in}}%
\pgfpathlineto{\pgfqpoint{0.968902in}{1.270721in}}%
\pgfpathlineto{\pgfqpoint{0.960106in}{1.257191in}}%
\pgfpathlineto{\pgfqpoint{0.953246in}{1.248140in}}%
\pgfpathlineto{\pgfqpoint{0.949720in}{1.243579in}}%
\pgfpathlineto{\pgfqpoint{0.937810in}{1.229968in}}%
\pgfpathlineto{\pgfqpoint{0.937589in}{1.229739in}}%
\pgfpathlineto{\pgfqpoint{0.924189in}{1.216357in}}%
\pgfpathlineto{\pgfqpoint{0.921933in}{1.214276in}}%
\pgfpathlineto{\pgfqpoint{0.908670in}{1.202746in}}%
\pgfpathlineto{\pgfqpoint{0.906276in}{1.200785in}}%
\pgfpathlineto{\pgfqpoint{0.890884in}{1.189135in}}%
\pgfpathlineto{\pgfqpoint{0.890620in}{1.188943in}}%
\pgfpathlineto{\pgfqpoint{0.874963in}{1.178589in}}%
\pgfpathlineto{\pgfqpoint{0.869717in}{1.175524in}}%
\pgfpathlineto{\pgfqpoint{0.859306in}{1.169560in}}%
\pgfpathlineto{\pgfqpoint{0.843743in}{1.161913in}}%
\pgfpathlineto{\pgfqpoint{0.843650in}{1.161867in}}%
\pgfpathlineto{\pgfqpoint{0.827993in}{1.155420in}}%
\pgfpathlineto{\pgfqpoint{0.812337in}{1.150334in}}%
\pgfpathlineto{\pgfqpoint{0.803829in}{1.148302in}}%
\pgfpathlineto{\pgfqpoint{0.796680in}{1.146554in}}%
\pgfpathlineto{\pgfqpoint{0.781024in}{1.144109in}}%
\pgfpathlineto{\pgfqpoint{0.765367in}{1.143061in}}%
\pgfpathlineto{\pgfqpoint{0.749710in}{1.143410in}}%
\pgfpathlineto{\pgfqpoint{0.734054in}{1.145156in}}%
\pgfpathlineto{\pgfqpoint{0.718397in}{1.148301in}}%
\pgfpathlineto{\pgfqpoint{0.718395in}{1.148302in}}%
\pgfpathclose%
\pgfpathmoveto{\pgfqpoint{1.501226in}{1.077443in}}%
\pgfpathlineto{\pgfqpoint{1.516882in}{1.073987in}}%
\pgfpathlineto{\pgfqpoint{1.532539in}{1.072507in}}%
\pgfpathlineto{\pgfqpoint{1.548195in}{1.073001in}}%
\pgfpathlineto{\pgfqpoint{1.563852in}{1.075468in}}%
\pgfpathlineto{\pgfqpoint{1.579508in}{1.079912in}}%
\pgfpathlineto{\pgfqpoint{1.580330in}{1.080246in}}%
\pgfpathlineto{\pgfqpoint{1.595165in}{1.085477in}}%
\pgfpathlineto{\pgfqpoint{1.610822in}{1.092700in}}%
\pgfpathlineto{\pgfqpoint{1.612877in}{1.093857in}}%
\pgfpathlineto{\pgfqpoint{1.626478in}{1.100763in}}%
\pgfpathlineto{\pgfqpoint{1.637649in}{1.107468in}}%
\pgfpathlineto{\pgfqpoint{1.642135in}{1.109969in}}%
\pgfpathlineto{\pgfqpoint{1.657791in}{1.119958in}}%
\pgfpathlineto{\pgfqpoint{1.659372in}{1.121079in}}%
\pgfpathlineto{\pgfqpoint{1.673448in}{1.130611in}}%
\pgfpathlineto{\pgfqpoint{1.678950in}{1.134691in}}%
\pgfpathlineto{\pgfqpoint{1.689104in}{1.142032in}}%
\pgfpathlineto{\pgfqpoint{1.697172in}{1.148302in}}%
\pgfpathlineto{\pgfqpoint{1.704761in}{1.154166in}}%
\pgfpathlineto{\pgfqpoint{1.714250in}{1.161913in}}%
\pgfpathlineto{\pgfqpoint{1.720418in}{1.167010in}}%
\pgfpathlineto{\pgfqpoint{1.730336in}{1.175524in}}%
\pgfpathlineto{\pgfqpoint{1.736074in}{1.180597in}}%
\pgfpathlineto{\pgfqpoint{1.745530in}{1.189135in}}%
\pgfpathlineto{\pgfqpoint{1.751731in}{1.195000in}}%
\pgfpathlineto{\pgfqpoint{1.759893in}{1.202746in}}%
\pgfpathlineto{\pgfqpoint{1.767387in}{1.210325in}}%
\pgfpathlineto{\pgfqpoint{1.773444in}{1.216357in}}%
\pgfpathlineto{\pgfqpoint{1.783044in}{1.226727in}}%
\pgfpathlineto{\pgfqpoint{1.786152in}{1.229968in}}%
\pgfpathlineto{\pgfqpoint{1.797974in}{1.243579in}}%
\pgfpathlineto{\pgfqpoint{1.798700in}{1.244521in}}%
\pgfpathlineto{\pgfqpoint{1.809085in}{1.257191in}}%
\pgfpathlineto{\pgfqpoint{1.814357in}{1.264671in}}%
\pgfpathlineto{\pgfqpoint{1.819073in}{1.270802in}}%
\pgfpathlineto{\pgfqpoint{1.827797in}{1.284413in}}%
\pgfpathlineto{\pgfqpoint{1.830014in}{1.288764in}}%
\pgfpathlineto{\pgfqpoint{1.835348in}{1.298024in}}%
\pgfpathlineto{\pgfqpoint{1.841113in}{1.311635in}}%
\pgfpathlineto{\pgfqpoint{1.844778in}{1.325246in}}%
\pgfpathlineto{\pgfqpoint{1.845670in}{1.332973in}}%
\pgfpathlineto{\pgfqpoint{1.846476in}{1.338857in}}%
\pgfpathlineto{\pgfqpoint{1.845854in}{1.352468in}}%
\pgfpathlineto{\pgfqpoint{1.845670in}{1.353275in}}%
\pgfpathlineto{\pgfqpoint{1.843207in}{1.366079in}}%
\pgfpathlineto{\pgfqpoint{1.838493in}{1.379691in}}%
\pgfpathlineto{\pgfqpoint{1.831677in}{1.393302in}}%
\pgfpathlineto{\pgfqpoint{1.830014in}{1.395865in}}%
\pgfpathlineto{\pgfqpoint{1.823666in}{1.406913in}}%
\pgfpathlineto{\pgfqpoint{1.814357in}{1.420056in}}%
\pgfpathlineto{\pgfqpoint{1.814054in}{1.420524in}}%
\pgfpathlineto{\pgfqpoint{1.803705in}{1.434135in}}%
\pgfpathlineto{\pgfqpoint{1.798700in}{1.439882in}}%
\pgfpathlineto{\pgfqpoint{1.792249in}{1.447746in}}%
\pgfpathlineto{\pgfqpoint{1.783044in}{1.457780in}}%
\pgfpathlineto{\pgfqpoint{1.779880in}{1.461357in}}%
\pgfpathlineto{\pgfqpoint{1.767387in}{1.474252in}}%
\pgfpathlineto{\pgfqpoint{1.766704in}{1.474968in}}%
\pgfpathlineto{\pgfqpoint{1.752782in}{1.488579in}}%
\pgfpathlineto{\pgfqpoint{1.751731in}{1.489554in}}%
\pgfpathlineto{\pgfqpoint{1.738037in}{1.502191in}}%
\pgfpathlineto{\pgfqpoint{1.736074in}{1.503934in}}%
\pgfpathlineto{\pgfqpoint{1.722423in}{1.515802in}}%
\pgfpathlineto{\pgfqpoint{1.720418in}{1.517508in}}%
\pgfpathlineto{\pgfqpoint{1.705882in}{1.529413in}}%
\pgfpathlineto{\pgfqpoint{1.704761in}{1.530327in}}%
\pgfpathlineto{\pgfqpoint{1.689104in}{1.542430in}}%
\pgfpathlineto{\pgfqpoint{1.688281in}{1.543024in}}%
\pgfpathlineto{\pgfqpoint{1.673448in}{1.553884in}}%
\pgfpathlineto{\pgfqpoint{1.669333in}{1.556635in}}%
\pgfpathlineto{\pgfqpoint{1.657791in}{1.564638in}}%
\pgfpathlineto{\pgfqpoint{1.648746in}{1.570246in}}%
\pgfpathlineto{\pgfqpoint{1.642135in}{1.574597in}}%
\pgfpathlineto{\pgfqpoint{1.626478in}{1.583594in}}%
\pgfpathlineto{\pgfqpoint{1.625940in}{1.583857in}}%
\pgfpathlineto{\pgfqpoint{1.610822in}{1.591950in}}%
\pgfpathlineto{\pgfqpoint{1.598114in}{1.597468in}}%
\pgfpathlineto{\pgfqpoint{1.595165in}{1.598915in}}%
\pgfpathlineto{\pgfqpoint{1.579508in}{1.604840in}}%
\pgfpathlineto{\pgfqpoint{1.563852in}{1.608939in}}%
\pgfpathlineto{\pgfqpoint{1.549123in}{1.611079in}}%
\pgfpathlineto{\pgfqpoint{1.548195in}{1.611239in}}%
\pgfpathlineto{\pgfqpoint{1.532539in}{1.611780in}}%
\pgfpathlineto{\pgfqpoint{1.525770in}{1.611079in}}%
\pgfpathlineto{\pgfqpoint{1.516882in}{1.610304in}}%
\pgfpathlineto{\pgfqpoint{1.501226in}{1.607117in}}%
\pgfpathlineto{\pgfqpoint{1.485569in}{1.602106in}}%
\pgfpathlineto{\pgfqpoint{1.474917in}{1.597468in}}%
\pgfpathlineto{\pgfqpoint{1.469913in}{1.595542in}}%
\pgfpathlineto{\pgfqpoint{1.454256in}{1.587957in}}%
\pgfpathlineto{\pgfqpoint{1.447204in}{1.583857in}}%
\pgfpathlineto{\pgfqpoint{1.438599in}{1.579274in}}%
\pgfpathlineto{\pgfqpoint{1.424026in}{1.570246in}}%
\pgfpathlineto{\pgfqpoint{1.422943in}{1.569615in}}%
\pgfpathlineto{\pgfqpoint{1.407286in}{1.559338in}}%
\pgfpathlineto{\pgfqpoint{1.403558in}{1.556635in}}%
\pgfpathlineto{\pgfqpoint{1.391630in}{1.548290in}}%
\pgfpathlineto{\pgfqpoint{1.384690in}{1.543024in}}%
\pgfpathlineto{\pgfqpoint{1.375973in}{1.536509in}}%
\pgfpathlineto{\pgfqpoint{1.367063in}{1.529413in}}%
\pgfpathlineto{\pgfqpoint{1.360317in}{1.524022in}}%
\pgfpathlineto{\pgfqpoint{1.350496in}{1.515802in}}%
\pgfpathlineto{\pgfqpoint{1.344660in}{1.510813in}}%
\pgfpathlineto{\pgfqpoint{1.334866in}{1.502191in}}%
\pgfpathlineto{\pgfqpoint{1.329003in}{1.496829in}}%
\pgfpathlineto{\pgfqpoint{1.320093in}{1.488579in}}%
\pgfpathlineto{\pgfqpoint{1.313347in}{1.481982in}}%
\pgfpathlineto{\pgfqpoint{1.306135in}{1.474968in}}%
\pgfpathlineto{\pgfqpoint{1.297690in}{1.466141in}}%
\pgfpathlineto{\pgfqpoint{1.292998in}{1.461357in}}%
\pgfpathlineto{\pgfqpoint{1.282034in}{1.449120in}}%
\pgfpathlineto{\pgfqpoint{1.280744in}{1.447746in}}%
\pgfpathlineto{\pgfqpoint{1.269254in}{1.434135in}}%
\pgfpathlineto{\pgfqpoint{1.266377in}{1.430236in}}%
\pgfpathlineto{\pgfqpoint{1.258664in}{1.420524in}}%
\pgfpathlineto{\pgfqpoint{1.250721in}{1.408700in}}%
\pgfpathlineto{\pgfqpoint{1.249390in}{1.406913in}}%
\pgfpathlineto{\pgfqpoint{1.241081in}{1.393302in}}%
\pgfpathlineto{\pgfqpoint{1.235064in}{1.380405in}}%
\pgfpathlineto{\pgfqpoint{1.234680in}{1.379691in}}%
\pgfpathlineto{\pgfqpoint{1.229568in}{1.366079in}}%
\pgfpathlineto{\pgfqpoint{1.226730in}{1.352468in}}%
\pgfpathlineto{\pgfqpoint{1.226162in}{1.338857in}}%
\pgfpathlineto{\pgfqpoint{1.227865in}{1.325246in}}%
\pgfpathlineto{\pgfqpoint{1.231839in}{1.311635in}}%
\pgfpathlineto{\pgfqpoint{1.235064in}{1.304586in}}%
\pgfpathlineto{\pgfqpoint{1.237663in}{1.298024in}}%
\pgfpathlineto{\pgfqpoint{1.244990in}{1.284413in}}%
\pgfpathlineto{\pgfqpoint{1.250721in}{1.275973in}}%
\pgfpathlineto{\pgfqpoint{1.253883in}{1.270802in}}%
\pgfpathlineto{\pgfqpoint{1.263880in}{1.257191in}}%
\pgfpathlineto{\pgfqpoint{1.266377in}{1.254249in}}%
\pgfpathlineto{\pgfqpoint{1.274804in}{1.243579in}}%
\pgfpathlineto{\pgfqpoint{1.282034in}{1.235488in}}%
\pgfpathlineto{\pgfqpoint{1.286736in}{1.229968in}}%
\pgfpathlineto{\pgfqpoint{1.297690in}{1.218346in}}%
\pgfpathlineto{\pgfqpoint{1.299517in}{1.216357in}}%
\pgfpathlineto{\pgfqpoint{1.313070in}{1.202746in}}%
\pgfpathlineto{\pgfqpoint{1.313347in}{1.202484in}}%
\pgfpathlineto{\pgfqpoint{1.327390in}{1.189135in}}%
\pgfpathlineto{\pgfqpoint{1.329003in}{1.187671in}}%
\pgfpathlineto{\pgfqpoint{1.342564in}{1.175524in}}%
\pgfpathlineto{\pgfqpoint{1.344660in}{1.173702in}}%
\pgfpathlineto{\pgfqpoint{1.358632in}{1.161913in}}%
\pgfpathlineto{\pgfqpoint{1.360317in}{1.160510in}}%
\pgfpathlineto{\pgfqpoint{1.375671in}{1.148302in}}%
\pgfpathlineto{\pgfqpoint{1.375973in}{1.148061in}}%
\pgfpathlineto{\pgfqpoint{1.391630in}{1.136278in}}%
\pgfpathlineto{\pgfqpoint{1.393917in}{1.134691in}}%
\pgfpathlineto{\pgfqpoint{1.407286in}{1.125167in}}%
\pgfpathlineto{\pgfqpoint{1.413635in}{1.121079in}}%
\pgfpathlineto{\pgfqpoint{1.422943in}{1.114794in}}%
\pgfpathlineto{\pgfqpoint{1.435216in}{1.107468in}}%
\pgfpathlineto{\pgfqpoint{1.438599in}{1.105297in}}%
\pgfpathlineto{\pgfqpoint{1.454256in}{1.096607in}}%
\pgfpathlineto{\pgfqpoint{1.460204in}{1.093857in}}%
\pgfpathlineto{\pgfqpoint{1.469913in}{1.088876in}}%
\pgfpathlineto{\pgfqpoint{1.485569in}{1.082505in}}%
\pgfpathlineto{\pgfqpoint{1.493117in}{1.080246in}}%
\pgfpathlineto{\pgfqpoint{1.501226in}{1.077443in}}%
\pgfpathclose%
\pgfpathmoveto{\pgfqpoint{1.494077in}{1.148302in}}%
\pgfpathlineto{\pgfqpoint{1.485569in}{1.150334in}}%
\pgfpathlineto{\pgfqpoint{1.469913in}{1.155420in}}%
\pgfpathlineto{\pgfqpoint{1.454256in}{1.161867in}}%
\pgfpathlineto{\pgfqpoint{1.454163in}{1.161913in}}%
\pgfpathlineto{\pgfqpoint{1.438599in}{1.169560in}}%
\pgfpathlineto{\pgfqpoint{1.428188in}{1.175524in}}%
\pgfpathlineto{\pgfqpoint{1.422943in}{1.178589in}}%
\pgfpathlineto{\pgfqpoint{1.407286in}{1.188943in}}%
\pgfpathlineto{\pgfqpoint{1.407022in}{1.189135in}}%
\pgfpathlineto{\pgfqpoint{1.391630in}{1.200785in}}%
\pgfpathlineto{\pgfqpoint{1.389236in}{1.202746in}}%
\pgfpathlineto{\pgfqpoint{1.375973in}{1.214276in}}%
\pgfpathlineto{\pgfqpoint{1.373717in}{1.216357in}}%
\pgfpathlineto{\pgfqpoint{1.360317in}{1.229739in}}%
\pgfpathlineto{\pgfqpoint{1.360096in}{1.229968in}}%
\pgfpathlineto{\pgfqpoint{1.348186in}{1.243579in}}%
\pgfpathlineto{\pgfqpoint{1.344660in}{1.248140in}}%
\pgfpathlineto{\pgfqpoint{1.337800in}{1.257191in}}%
\pgfpathlineto{\pgfqpoint{1.329003in}{1.270721in}}%
\pgfpathlineto{\pgfqpoint{1.328951in}{1.270802in}}%
\pgfpathlineto{\pgfqpoint{1.321534in}{1.284413in}}%
\pgfpathlineto{\pgfqpoint{1.315684in}{1.298024in}}%
\pgfpathlineto{\pgfqpoint{1.313347in}{1.305420in}}%
\pgfpathlineto{\pgfqpoint{1.311336in}{1.311635in}}%
\pgfpathlineto{\pgfqpoint{1.308524in}{1.325246in}}%
\pgfpathlineto{\pgfqpoint{1.307319in}{1.338857in}}%
\pgfpathlineto{\pgfqpoint{1.307720in}{1.352468in}}%
\pgfpathlineto{\pgfqpoint{1.309729in}{1.366079in}}%
\pgfpathlineto{\pgfqpoint{1.313346in}{1.379691in}}%
\pgfpathlineto{\pgfqpoint{1.313347in}{1.379693in}}%
\pgfpathlineto{\pgfqpoint{1.318414in}{1.393302in}}%
\pgfpathlineto{\pgfqpoint{1.325047in}{1.406913in}}%
\pgfpathlineto{\pgfqpoint{1.329003in}{1.413532in}}%
\pgfpathlineto{\pgfqpoint{1.333184in}{1.420524in}}%
\pgfpathlineto{\pgfqpoint{1.342800in}{1.434135in}}%
\pgfpathlineto{\pgfqpoint{1.344660in}{1.436445in}}%
\pgfpathlineto{\pgfqpoint{1.353954in}{1.447746in}}%
\pgfpathlineto{\pgfqpoint{1.360317in}{1.454665in}}%
\pgfpathlineto{\pgfqpoint{1.366720in}{1.461357in}}%
\pgfpathlineto{\pgfqpoint{1.375973in}{1.470205in}}%
\pgfpathlineto{\pgfqpoint{1.381255in}{1.474968in}}%
\pgfpathlineto{\pgfqpoint{1.391630in}{1.483706in}}%
\pgfpathlineto{\pgfqpoint{1.397887in}{1.488579in}}%
\pgfpathlineto{\pgfqpoint{1.407286in}{1.495548in}}%
\pgfpathlineto{\pgfqpoint{1.417187in}{1.502191in}}%
\pgfpathlineto{\pgfqpoint{1.422943in}{1.505937in}}%
\pgfpathlineto{\pgfqpoint{1.438599in}{1.514949in}}%
\pgfpathlineto{\pgfqpoint{1.440312in}{1.515802in}}%
\pgfpathlineto{\pgfqpoint{1.454256in}{1.522678in}}%
\pgfpathlineto{\pgfqpoint{1.469913in}{1.529064in}}%
\pgfpathlineto{\pgfqpoint{1.470984in}{1.529413in}}%
\pgfpathlineto{\pgfqpoint{1.485569in}{1.534212in}}%
\pgfpathlineto{\pgfqpoint{1.501226in}{1.537989in}}%
\pgfpathlineto{\pgfqpoint{1.516882in}{1.540391in}}%
\pgfpathlineto{\pgfqpoint{1.532539in}{1.541419in}}%
\pgfpathlineto{\pgfqpoint{1.548195in}{1.541077in}}%
\pgfpathlineto{\pgfqpoint{1.563852in}{1.539362in}}%
\pgfpathlineto{\pgfqpoint{1.579508in}{1.536273in}}%
\pgfpathlineto{\pgfqpoint{1.595165in}{1.531807in}}%
\pgfpathlineto{\pgfqpoint{1.601623in}{1.529413in}}%
\pgfpathlineto{\pgfqpoint{1.610822in}{1.526040in}}%
\pgfpathlineto{\pgfqpoint{1.626478in}{1.518979in}}%
\pgfpathlineto{\pgfqpoint{1.632461in}{1.515802in}}%
\pgfpathlineto{\pgfqpoint{1.642135in}{1.510608in}}%
\pgfpathlineto{\pgfqpoint{1.655796in}{1.502191in}}%
\pgfpathlineto{\pgfqpoint{1.657791in}{1.500924in}}%
\pgfpathlineto{\pgfqpoint{1.673448in}{1.489849in}}%
\pgfpathlineto{\pgfqpoint{1.675092in}{1.488579in}}%
\pgfpathlineto{\pgfqpoint{1.689104in}{1.477212in}}%
\pgfpathlineto{\pgfqpoint{1.691685in}{1.474968in}}%
\pgfpathlineto{\pgfqpoint{1.704761in}{1.462787in}}%
\pgfpathlineto{\pgfqpoint{1.706221in}{1.461357in}}%
\pgfpathlineto{\pgfqpoint{1.718961in}{1.447746in}}%
\pgfpathlineto{\pgfqpoint{1.720418in}{1.446011in}}%
\pgfpathlineto{\pgfqpoint{1.730100in}{1.434135in}}%
\pgfpathlineto{\pgfqpoint{1.736074in}{1.425725in}}%
\pgfpathlineto{\pgfqpoint{1.739729in}{1.420524in}}%
\pgfpathlineto{\pgfqpoint{1.747851in}{1.406913in}}%
\pgfpathlineto{\pgfqpoint{1.751731in}{1.398916in}}%
\pgfpathlineto{\pgfqpoint{1.754484in}{1.393302in}}%
\pgfpathlineto{\pgfqpoint{1.759621in}{1.379691in}}%
\pgfpathlineto{\pgfqpoint{1.763175in}{1.366079in}}%
\pgfpathlineto{\pgfqpoint{1.765147in}{1.352468in}}%
\pgfpathlineto{\pgfqpoint{1.765542in}{1.338857in}}%
\pgfpathlineto{\pgfqpoint{1.764358in}{1.325246in}}%
\pgfpathlineto{\pgfqpoint{1.761596in}{1.311635in}}%
\pgfpathlineto{\pgfqpoint{1.757251in}{1.298024in}}%
\pgfpathlineto{\pgfqpoint{1.751731in}{1.285344in}}%
\pgfpathlineto{\pgfqpoint{1.751330in}{1.284413in}}%
\pgfpathlineto{\pgfqpoint{1.743983in}{1.270802in}}%
\pgfpathlineto{\pgfqpoint{1.736074in}{1.258679in}}%
\pgfpathlineto{\pgfqpoint{1.735094in}{1.257191in}}%
\pgfpathlineto{\pgfqpoint{1.724726in}{1.243579in}}%
\pgfpathlineto{\pgfqpoint{1.720418in}{1.238575in}}%
\pgfpathlineto{\pgfqpoint{1.712777in}{1.229968in}}%
\pgfpathlineto{\pgfqpoint{1.704761in}{1.221797in}}%
\pgfpathlineto{\pgfqpoint{1.699156in}{1.216357in}}%
\pgfpathlineto{\pgfqpoint{1.689104in}{1.207338in}}%
\pgfpathlineto{\pgfqpoint{1.683625in}{1.202746in}}%
\pgfpathlineto{\pgfqpoint{1.673448in}{1.194702in}}%
\pgfpathlineto{\pgfqpoint{1.665750in}{1.189135in}}%
\pgfpathlineto{\pgfqpoint{1.657791in}{1.183603in}}%
\pgfpathlineto{\pgfqpoint{1.644792in}{1.175524in}}%
\pgfpathlineto{\pgfqpoint{1.642135in}{1.173907in}}%
\pgfpathlineto{\pgfqpoint{1.626478in}{1.165547in}}%
\pgfpathlineto{\pgfqpoint{1.618435in}{1.161913in}}%
\pgfpathlineto{\pgfqpoint{1.610822in}{1.158473in}}%
\pgfpathlineto{\pgfqpoint{1.595165in}{1.152707in}}%
\pgfpathlineto{\pgfqpoint{1.579511in}{1.148302in}}%
\pgfpathlineto{\pgfqpoint{1.579508in}{1.148301in}}%
\pgfpathlineto{\pgfqpoint{1.563852in}{1.145156in}}%
\pgfpathlineto{\pgfqpoint{1.548195in}{1.143410in}}%
\pgfpathlineto{\pgfqpoint{1.532539in}{1.143061in}}%
\pgfpathlineto{\pgfqpoint{1.516882in}{1.144109in}}%
\pgfpathlineto{\pgfqpoint{1.501226in}{1.146554in}}%
\pgfpathlineto{\pgfqpoint{1.494077in}{1.148302in}}%
\pgfpathclose%
\pgfusepath{fill}%
\end{pgfscope}%
\begin{pgfscope}%
\pgfpathrectangle{\pgfqpoint{0.373953in}{0.331635in}}{\pgfqpoint{1.550000in}{1.347500in}}%
\pgfusepath{clip}%
\pgfsetbuttcap%
\pgfsetroundjoin%
\definecolor{currentfill}{rgb}{0.240046,0.553772,0.766797}%
\pgfsetfillcolor{currentfill}%
\pgfsetlinewidth{0.000000pt}%
\definecolor{currentstroke}{rgb}{0.000000,0.000000,0.000000}%
\pgfsetstrokecolor{currentstroke}%
\pgfsetdash{}{0pt}%
\pgfpathmoveto{\pgfqpoint{0.640115in}{0.331635in}}%
\pgfpathlineto{\pgfqpoint{0.655771in}{0.331635in}}%
\pgfpathlineto{\pgfqpoint{0.671428in}{0.331635in}}%
\pgfpathlineto{\pgfqpoint{0.687084in}{0.331635in}}%
\pgfpathlineto{\pgfqpoint{0.702741in}{0.331635in}}%
\pgfpathlineto{\pgfqpoint{0.718397in}{0.331635in}}%
\pgfpathlineto{\pgfqpoint{0.734054in}{0.331635in}}%
\pgfpathlineto{\pgfqpoint{0.749710in}{0.331635in}}%
\pgfpathlineto{\pgfqpoint{0.765367in}{0.331635in}}%
\pgfpathlineto{\pgfqpoint{0.781024in}{0.331635in}}%
\pgfpathlineto{\pgfqpoint{0.796680in}{0.331635in}}%
\pgfpathlineto{\pgfqpoint{0.812337in}{0.331635in}}%
\pgfpathlineto{\pgfqpoint{0.827993in}{0.331635in}}%
\pgfpathlineto{\pgfqpoint{0.843650in}{0.331635in}}%
\pgfpathlineto{\pgfqpoint{0.859306in}{0.331635in}}%
\pgfpathlineto{\pgfqpoint{0.874963in}{0.331635in}}%
\pgfpathlineto{\pgfqpoint{0.890620in}{0.331635in}}%
\pgfpathlineto{\pgfqpoint{0.894062in}{0.331635in}}%
\pgfpathlineto{\pgfqpoint{0.895623in}{0.345246in}}%
\pgfpathlineto{\pgfqpoint{0.900238in}{0.358857in}}%
\pgfpathlineto{\pgfqpoint{0.906276in}{0.369827in}}%
\pgfpathlineto{\pgfqpoint{0.907640in}{0.372468in}}%
\pgfpathlineto{\pgfqpoint{0.917158in}{0.386079in}}%
\pgfpathlineto{\pgfqpoint{0.921933in}{0.391631in}}%
\pgfpathlineto{\pgfqpoint{0.928505in}{0.399691in}}%
\pgfpathlineto{\pgfqpoint{0.937589in}{0.409307in}}%
\pgfpathlineto{\pgfqpoint{0.941216in}{0.413302in}}%
\pgfpathlineto{\pgfqpoint{0.953246in}{0.425214in}}%
\pgfpathlineto{\pgfqpoint{0.954918in}{0.426913in}}%
\pgfpathlineto{\pgfqpoint{0.968902in}{0.440090in}}%
\pgfpathlineto{\pgfqpoint{0.969358in}{0.440524in}}%
\pgfpathlineto{\pgfqpoint{0.984382in}{0.454135in}}%
\pgfpathlineto{\pgfqpoint{0.984559in}{0.454291in}}%
\pgfpathlineto{\pgfqpoint{0.999916in}{0.467746in}}%
\pgfpathlineto{\pgfqpoint{1.000216in}{0.468006in}}%
\pgfpathlineto{\pgfqpoint{1.015872in}{0.481296in}}%
\pgfpathlineto{\pgfqpoint{1.015947in}{0.481357in}}%
\pgfpathlineto{\pgfqpoint{1.031529in}{0.494113in}}%
\pgfpathlineto{\pgfqpoint{1.032640in}{0.494968in}}%
\pgfpathlineto{\pgfqpoint{1.047185in}{0.506368in}}%
\pgfpathlineto{\pgfqpoint{1.050273in}{0.508579in}}%
\pgfpathlineto{\pgfqpoint{1.062842in}{0.517882in}}%
\pgfpathlineto{\pgfqpoint{1.069438in}{0.522191in}}%
\pgfpathlineto{\pgfqpoint{1.078498in}{0.528388in}}%
\pgfpathlineto{\pgfqpoint{1.091383in}{0.535802in}}%
\pgfpathlineto{\pgfqpoint{1.094155in}{0.537496in}}%
\pgfpathlineto{\pgfqpoint{1.109812in}{0.544928in}}%
\pgfpathlineto{\pgfqpoint{1.123506in}{0.549413in}}%
\pgfpathlineto{\pgfqpoint{1.125468in}{0.550103in}}%
\pgfpathlineto{\pgfqpoint{1.141125in}{0.552895in}}%
\pgfpathlineto{\pgfqpoint{1.156781in}{0.552895in}}%
\pgfpathlineto{\pgfqpoint{1.172438in}{0.550103in}}%
\pgfpathlineto{\pgfqpoint{1.174400in}{0.549413in}}%
\pgfpathlineto{\pgfqpoint{1.188094in}{0.544928in}}%
\pgfpathlineto{\pgfqpoint{1.203751in}{0.537496in}}%
\pgfpathlineto{\pgfqpoint{1.206523in}{0.535802in}}%
\pgfpathlineto{\pgfqpoint{1.219407in}{0.528388in}}%
\pgfpathlineto{\pgfqpoint{1.228467in}{0.522191in}}%
\pgfpathlineto{\pgfqpoint{1.235064in}{0.517882in}}%
\pgfpathlineto{\pgfqpoint{1.247633in}{0.508579in}}%
\pgfpathlineto{\pgfqpoint{1.250721in}{0.506368in}}%
\pgfpathlineto{\pgfqpoint{1.265266in}{0.494968in}}%
\pgfpathlineto{\pgfqpoint{1.266377in}{0.494113in}}%
\pgfpathlineto{\pgfqpoint{1.281959in}{0.481357in}}%
\pgfpathlineto{\pgfqpoint{1.282034in}{0.481296in}}%
\pgfpathlineto{\pgfqpoint{1.297690in}{0.468006in}}%
\pgfpathlineto{\pgfqpoint{1.297990in}{0.467746in}}%
\pgfpathlineto{\pgfqpoint{1.313347in}{0.454291in}}%
\pgfpathlineto{\pgfqpoint{1.313524in}{0.454135in}}%
\pgfpathlineto{\pgfqpoint{1.328548in}{0.440524in}}%
\pgfpathlineto{\pgfqpoint{1.329003in}{0.440090in}}%
\pgfpathlineto{\pgfqpoint{1.342988in}{0.426913in}}%
\pgfpathlineto{\pgfqpoint{1.344660in}{0.425214in}}%
\pgfpathlineto{\pgfqpoint{1.356689in}{0.413302in}}%
\pgfpathlineto{\pgfqpoint{1.360317in}{0.409307in}}%
\pgfpathlineto{\pgfqpoint{1.369401in}{0.399691in}}%
\pgfpathlineto{\pgfqpoint{1.375973in}{0.391631in}}%
\pgfpathlineto{\pgfqpoint{1.380748in}{0.386079in}}%
\pgfpathlineto{\pgfqpoint{1.390266in}{0.372468in}}%
\pgfpathlineto{\pgfqpoint{1.391630in}{0.369827in}}%
\pgfpathlineto{\pgfqpoint{1.397667in}{0.358857in}}%
\pgfpathlineto{\pgfqpoint{1.402283in}{0.345246in}}%
\pgfpathlineto{\pgfqpoint{1.403843in}{0.331635in}}%
\pgfpathlineto{\pgfqpoint{1.407286in}{0.331635in}}%
\pgfpathlineto{\pgfqpoint{1.422943in}{0.331635in}}%
\pgfpathlineto{\pgfqpoint{1.438599in}{0.331635in}}%
\pgfpathlineto{\pgfqpoint{1.454256in}{0.331635in}}%
\pgfpathlineto{\pgfqpoint{1.469913in}{0.331635in}}%
\pgfpathlineto{\pgfqpoint{1.485569in}{0.331635in}}%
\pgfpathlineto{\pgfqpoint{1.501226in}{0.331635in}}%
\pgfpathlineto{\pgfqpoint{1.516882in}{0.331635in}}%
\pgfpathlineto{\pgfqpoint{1.532539in}{0.331635in}}%
\pgfpathlineto{\pgfqpoint{1.548195in}{0.331635in}}%
\pgfpathlineto{\pgfqpoint{1.563852in}{0.331635in}}%
\pgfpathlineto{\pgfqpoint{1.579508in}{0.331635in}}%
\pgfpathlineto{\pgfqpoint{1.595165in}{0.331635in}}%
\pgfpathlineto{\pgfqpoint{1.610822in}{0.331635in}}%
\pgfpathlineto{\pgfqpoint{1.626478in}{0.331635in}}%
\pgfpathlineto{\pgfqpoint{1.642135in}{0.331635in}}%
\pgfpathlineto{\pgfqpoint{1.657791in}{0.331635in}}%
\pgfpathlineto{\pgfqpoint{1.669038in}{0.331635in}}%
\pgfpathlineto{\pgfqpoint{1.670652in}{0.345246in}}%
\pgfpathlineto{\pgfqpoint{1.673448in}{0.353195in}}%
\pgfpathlineto{\pgfqpoint{1.675306in}{0.358857in}}%
\pgfpathlineto{\pgfqpoint{1.682570in}{0.372468in}}%
\pgfpathlineto{\pgfqpoint{1.689104in}{0.381546in}}%
\pgfpathlineto{\pgfqpoint{1.692182in}{0.386079in}}%
\pgfpathlineto{\pgfqpoint{1.703538in}{0.399691in}}%
\pgfpathlineto{\pgfqpoint{1.704761in}{0.400949in}}%
\pgfpathlineto{\pgfqpoint{1.716207in}{0.413302in}}%
\pgfpathlineto{\pgfqpoint{1.720418in}{0.417386in}}%
\pgfpathlineto{\pgfqpoint{1.729913in}{0.426913in}}%
\pgfpathlineto{\pgfqpoint{1.736074in}{0.432643in}}%
\pgfpathlineto{\pgfqpoint{1.744382in}{0.440524in}}%
\pgfpathlineto{\pgfqpoint{1.751731in}{0.447150in}}%
\pgfpathlineto{\pgfqpoint{1.759441in}{0.454135in}}%
\pgfpathlineto{\pgfqpoint{1.767387in}{0.461123in}}%
\pgfpathlineto{\pgfqpoint{1.775005in}{0.467746in}}%
\pgfpathlineto{\pgfqpoint{1.783044in}{0.474654in}}%
\pgfpathlineto{\pgfqpoint{1.791079in}{0.481357in}}%
\pgfpathlineto{\pgfqpoint{1.798700in}{0.487745in}}%
\pgfpathlineto{\pgfqpoint{1.807766in}{0.494968in}}%
\pgfpathlineto{\pgfqpoint{1.814357in}{0.500324in}}%
\pgfpathlineto{\pgfqpoint{1.825315in}{0.508579in}}%
\pgfpathlineto{\pgfqpoint{1.830014in}{0.512240in}}%
\pgfpathlineto{\pgfqpoint{1.844222in}{0.522191in}}%
\pgfpathlineto{\pgfqpoint{1.845670in}{0.523254in}}%
\pgfpathlineto{\pgfqpoint{1.861327in}{0.533126in}}%
\pgfpathlineto{\pgfqpoint{1.866541in}{0.535802in}}%
\pgfpathlineto{\pgfqpoint{1.876983in}{0.541482in}}%
\pgfpathlineto{\pgfqpoint{1.892640in}{0.547797in}}%
\pgfpathlineto{\pgfqpoint{1.899153in}{0.549413in}}%
\pgfpathlineto{\pgfqpoint{1.908296in}{0.551843in}}%
\pgfpathlineto{\pgfqpoint{1.923953in}{0.553246in}}%
\pgfpathlineto{\pgfqpoint{1.923953in}{0.563024in}}%
\pgfpathlineto{\pgfqpoint{1.923953in}{0.576635in}}%
\pgfpathlineto{\pgfqpoint{1.923953in}{0.590246in}}%
\pgfpathlineto{\pgfqpoint{1.923953in}{0.603857in}}%
\pgfpathlineto{\pgfqpoint{1.923953in}{0.617468in}}%
\pgfpathlineto{\pgfqpoint{1.923953in}{0.631079in}}%
\pgfpathlineto{\pgfqpoint{1.923953in}{0.644691in}}%
\pgfpathlineto{\pgfqpoint{1.923953in}{0.658302in}}%
\pgfpathlineto{\pgfqpoint{1.923953in}{0.671913in}}%
\pgfpathlineto{\pgfqpoint{1.923953in}{0.685524in}}%
\pgfpathlineto{\pgfqpoint{1.923953in}{0.699135in}}%
\pgfpathlineto{\pgfqpoint{1.923953in}{0.712746in}}%
\pgfpathlineto{\pgfqpoint{1.923953in}{0.726357in}}%
\pgfpathlineto{\pgfqpoint{1.923953in}{0.739968in}}%
\pgfpathlineto{\pgfqpoint{1.923953in}{0.753579in}}%
\pgfpathlineto{\pgfqpoint{1.923953in}{0.767191in}}%
\pgfpathlineto{\pgfqpoint{1.923953in}{0.780802in}}%
\pgfpathlineto{\pgfqpoint{1.923953in}{0.783795in}}%
\pgfpathlineto{\pgfqpoint{1.908296in}{0.785152in}}%
\pgfpathlineto{\pgfqpoint{1.892640in}{0.789164in}}%
\pgfpathlineto{\pgfqpoint{1.880021in}{0.794413in}}%
\pgfpathlineto{\pgfqpoint{1.876983in}{0.795598in}}%
\pgfpathlineto{\pgfqpoint{1.861327in}{0.803873in}}%
\pgfpathlineto{\pgfqpoint{1.854941in}{0.808024in}}%
\pgfpathlineto{\pgfqpoint{1.845670in}{0.813738in}}%
\pgfpathlineto{\pgfqpoint{1.834608in}{0.821635in}}%
\pgfpathlineto{\pgfqpoint{1.830014in}{0.824788in}}%
\pgfpathlineto{\pgfqpoint{1.816312in}{0.835246in}}%
\pgfpathlineto{\pgfqpoint{1.814357in}{0.836700in}}%
\pgfpathlineto{\pgfqpoint{1.799200in}{0.848857in}}%
\pgfpathlineto{\pgfqpoint{1.798700in}{0.849253in}}%
\pgfpathlineto{\pgfqpoint{1.783044in}{0.862314in}}%
\pgfpathlineto{\pgfqpoint{1.782865in}{0.862468in}}%
\pgfpathlineto{\pgfqpoint{1.767387in}{0.875819in}}%
\pgfpathlineto{\pgfqpoint{1.767089in}{0.876079in}}%
\pgfpathlineto{\pgfqpoint{1.751801in}{0.889691in}}%
\pgfpathlineto{\pgfqpoint{1.751731in}{0.889756in}}%
\pgfpathlineto{\pgfqpoint{1.737058in}{0.903302in}}%
\pgfpathlineto{\pgfqpoint{1.736074in}{0.904268in}}%
\pgfpathlineto{\pgfqpoint{1.722962in}{0.916913in}}%
\pgfpathlineto{\pgfqpoint{1.720418in}{0.919597in}}%
\pgfpathlineto{\pgfqpoint{1.709717in}{0.930524in}}%
\pgfpathlineto{\pgfqpoint{1.704761in}{0.936259in}}%
\pgfpathlineto{\pgfqpoint{1.697632in}{0.944135in}}%
\pgfpathlineto{\pgfqpoint{1.689104in}{0.955336in}}%
\pgfpathlineto{\pgfqpoint{1.687156in}{0.957746in}}%
\pgfpathlineto{\pgfqpoint{1.678607in}{0.971357in}}%
\pgfpathlineto{\pgfqpoint{1.673448in}{0.983262in}}%
\pgfpathlineto{\pgfqpoint{1.672654in}{0.984968in}}%
\pgfpathlineto{\pgfqpoint{1.669443in}{0.998579in}}%
\pgfpathlineto{\pgfqpoint{1.669443in}{1.012191in}}%
\pgfpathlineto{\pgfqpoint{1.672654in}{1.025802in}}%
\pgfpathlineto{\pgfqpoint{1.673448in}{1.027508in}}%
\pgfpathlineto{\pgfqpoint{1.678607in}{1.039413in}}%
\pgfpathlineto{\pgfqpoint{1.687156in}{1.053024in}}%
\pgfpathlineto{\pgfqpoint{1.689104in}{1.055434in}}%
\pgfpathlineto{\pgfqpoint{1.697632in}{1.066635in}}%
\pgfpathlineto{\pgfqpoint{1.704761in}{1.074511in}}%
\pgfpathlineto{\pgfqpoint{1.709717in}{1.080246in}}%
\pgfpathlineto{\pgfqpoint{1.720418in}{1.091173in}}%
\pgfpathlineto{\pgfqpoint{1.722962in}{1.093857in}}%
\pgfpathlineto{\pgfqpoint{1.736074in}{1.106502in}}%
\pgfpathlineto{\pgfqpoint{1.737058in}{1.107468in}}%
\pgfpathlineto{\pgfqpoint{1.751731in}{1.121014in}}%
\pgfpathlineto{\pgfqpoint{1.751801in}{1.121079in}}%
\pgfpathlineto{\pgfqpoint{1.767089in}{1.134691in}}%
\pgfpathlineto{\pgfqpoint{1.767387in}{1.134951in}}%
\pgfpathlineto{\pgfqpoint{1.782865in}{1.148302in}}%
\pgfpathlineto{\pgfqpoint{1.783044in}{1.148456in}}%
\pgfpathlineto{\pgfqpoint{1.798700in}{1.161517in}}%
\pgfpathlineto{\pgfqpoint{1.799200in}{1.161913in}}%
\pgfpathlineto{\pgfqpoint{1.814357in}{1.174070in}}%
\pgfpathlineto{\pgfqpoint{1.816312in}{1.175524in}}%
\pgfpathlineto{\pgfqpoint{1.830014in}{1.185982in}}%
\pgfpathlineto{\pgfqpoint{1.834608in}{1.189135in}}%
\pgfpathlineto{\pgfqpoint{1.845670in}{1.197032in}}%
\pgfpathlineto{\pgfqpoint{1.854941in}{1.202746in}}%
\pgfpathlineto{\pgfqpoint{1.861327in}{1.206897in}}%
\pgfpathlineto{\pgfqpoint{1.876983in}{1.215172in}}%
\pgfpathlineto{\pgfqpoint{1.880021in}{1.216357in}}%
\pgfpathlineto{\pgfqpoint{1.892640in}{1.221606in}}%
\pgfpathlineto{\pgfqpoint{1.908296in}{1.225618in}}%
\pgfpathlineto{\pgfqpoint{1.923953in}{1.226975in}}%
\pgfpathlineto{\pgfqpoint{1.923953in}{1.229968in}}%
\pgfpathlineto{\pgfqpoint{1.923953in}{1.243579in}}%
\pgfpathlineto{\pgfqpoint{1.923953in}{1.257191in}}%
\pgfpathlineto{\pgfqpoint{1.923953in}{1.270802in}}%
\pgfpathlineto{\pgfqpoint{1.923953in}{1.284413in}}%
\pgfpathlineto{\pgfqpoint{1.923953in}{1.298024in}}%
\pgfpathlineto{\pgfqpoint{1.923953in}{1.311635in}}%
\pgfpathlineto{\pgfqpoint{1.923953in}{1.325246in}}%
\pgfpathlineto{\pgfqpoint{1.923953in}{1.338857in}}%
\pgfpathlineto{\pgfqpoint{1.923953in}{1.352468in}}%
\pgfpathlineto{\pgfqpoint{1.923953in}{1.366079in}}%
\pgfpathlineto{\pgfqpoint{1.923953in}{1.379691in}}%
\pgfpathlineto{\pgfqpoint{1.923953in}{1.393302in}}%
\pgfpathlineto{\pgfqpoint{1.923953in}{1.406913in}}%
\pgfpathlineto{\pgfqpoint{1.923953in}{1.420524in}}%
\pgfpathlineto{\pgfqpoint{1.923953in}{1.434135in}}%
\pgfpathlineto{\pgfqpoint{1.923953in}{1.447746in}}%
\pgfpathlineto{\pgfqpoint{1.923953in}{1.457524in}}%
\pgfpathlineto{\pgfqpoint{1.908296in}{1.458927in}}%
\pgfpathlineto{\pgfqpoint{1.899153in}{1.461357in}}%
\pgfpathlineto{\pgfqpoint{1.892640in}{1.462973in}}%
\pgfpathlineto{\pgfqpoint{1.876983in}{1.469288in}}%
\pgfpathlineto{\pgfqpoint{1.866541in}{1.474968in}}%
\pgfpathlineto{\pgfqpoint{1.861327in}{1.477644in}}%
\pgfpathlineto{\pgfqpoint{1.845670in}{1.487516in}}%
\pgfpathlineto{\pgfqpoint{1.844222in}{1.488579in}}%
\pgfpathlineto{\pgfqpoint{1.830014in}{1.498530in}}%
\pgfpathlineto{\pgfqpoint{1.825315in}{1.502191in}}%
\pgfpathlineto{\pgfqpoint{1.814357in}{1.510446in}}%
\pgfpathlineto{\pgfqpoint{1.807766in}{1.515802in}}%
\pgfpathlineto{\pgfqpoint{1.798700in}{1.523025in}}%
\pgfpathlineto{\pgfqpoint{1.791079in}{1.529413in}}%
\pgfpathlineto{\pgfqpoint{1.783044in}{1.536116in}}%
\pgfpathlineto{\pgfqpoint{1.775005in}{1.543024in}}%
\pgfpathlineto{\pgfqpoint{1.767387in}{1.549647in}}%
\pgfpathlineto{\pgfqpoint{1.759441in}{1.556635in}}%
\pgfpathlineto{\pgfqpoint{1.751731in}{1.563620in}}%
\pgfpathlineto{\pgfqpoint{1.744382in}{1.570246in}}%
\pgfpathlineto{\pgfqpoint{1.736074in}{1.578127in}}%
\pgfpathlineto{\pgfqpoint{1.729913in}{1.583857in}}%
\pgfpathlineto{\pgfqpoint{1.720418in}{1.593384in}}%
\pgfpathlineto{\pgfqpoint{1.716207in}{1.597468in}}%
\pgfpathlineto{\pgfqpoint{1.704761in}{1.609821in}}%
\pgfpathlineto{\pgfqpoint{1.703538in}{1.611079in}}%
\pgfpathlineto{\pgfqpoint{1.692182in}{1.624691in}}%
\pgfpathlineto{\pgfqpoint{1.689104in}{1.629224in}}%
\pgfpathlineto{\pgfqpoint{1.682570in}{1.638302in}}%
\pgfpathlineto{\pgfqpoint{1.675306in}{1.651913in}}%
\pgfpathlineto{\pgfqpoint{1.673448in}{1.657575in}}%
\pgfpathlineto{\pgfqpoint{1.670652in}{1.665524in}}%
\pgfpathlineto{\pgfqpoint{1.669038in}{1.679135in}}%
\pgfpathlineto{\pgfqpoint{1.657791in}{1.679135in}}%
\pgfpathlineto{\pgfqpoint{1.642135in}{1.679135in}}%
\pgfpathlineto{\pgfqpoint{1.626478in}{1.679135in}}%
\pgfpathlineto{\pgfqpoint{1.610822in}{1.679135in}}%
\pgfpathlineto{\pgfqpoint{1.595165in}{1.679135in}}%
\pgfpathlineto{\pgfqpoint{1.579508in}{1.679135in}}%
\pgfpathlineto{\pgfqpoint{1.563852in}{1.679135in}}%
\pgfpathlineto{\pgfqpoint{1.548195in}{1.679135in}}%
\pgfpathlineto{\pgfqpoint{1.532539in}{1.679135in}}%
\pgfpathlineto{\pgfqpoint{1.516882in}{1.679135in}}%
\pgfpathlineto{\pgfqpoint{1.501226in}{1.679135in}}%
\pgfpathlineto{\pgfqpoint{1.485569in}{1.679135in}}%
\pgfpathlineto{\pgfqpoint{1.469913in}{1.679135in}}%
\pgfpathlineto{\pgfqpoint{1.454256in}{1.679135in}}%
\pgfpathlineto{\pgfqpoint{1.438599in}{1.679135in}}%
\pgfpathlineto{\pgfqpoint{1.422943in}{1.679135in}}%
\pgfpathlineto{\pgfqpoint{1.407286in}{1.679135in}}%
\pgfpathlineto{\pgfqpoint{1.403843in}{1.679135in}}%
\pgfpathlineto{\pgfqpoint{1.402283in}{1.665524in}}%
\pgfpathlineto{\pgfqpoint{1.397667in}{1.651913in}}%
\pgfpathlineto{\pgfqpoint{1.391630in}{1.640943in}}%
\pgfpathlineto{\pgfqpoint{1.390266in}{1.638302in}}%
\pgfpathlineto{\pgfqpoint{1.380748in}{1.624691in}}%
\pgfpathlineto{\pgfqpoint{1.375973in}{1.619139in}}%
\pgfpathlineto{\pgfqpoint{1.369401in}{1.611079in}}%
\pgfpathlineto{\pgfqpoint{1.360317in}{1.601463in}}%
\pgfpathlineto{\pgfqpoint{1.356689in}{1.597468in}}%
\pgfpathlineto{\pgfqpoint{1.344660in}{1.585556in}}%
\pgfpathlineto{\pgfqpoint{1.342988in}{1.583857in}}%
\pgfpathlineto{\pgfqpoint{1.329003in}{1.570680in}}%
\pgfpathlineto{\pgfqpoint{1.328548in}{1.570246in}}%
\pgfpathlineto{\pgfqpoint{1.313524in}{1.556635in}}%
\pgfpathlineto{\pgfqpoint{1.313347in}{1.556479in}}%
\pgfpathlineto{\pgfqpoint{1.297990in}{1.543024in}}%
\pgfpathlineto{\pgfqpoint{1.297690in}{1.542764in}}%
\pgfpathlineto{\pgfqpoint{1.282034in}{1.529474in}}%
\pgfpathlineto{\pgfqpoint{1.281959in}{1.529413in}}%
\pgfpathlineto{\pgfqpoint{1.266377in}{1.516657in}}%
\pgfpathlineto{\pgfqpoint{1.265266in}{1.515802in}}%
\pgfpathlineto{\pgfqpoint{1.250721in}{1.504402in}}%
\pgfpathlineto{\pgfqpoint{1.247633in}{1.502191in}}%
\pgfpathlineto{\pgfqpoint{1.235064in}{1.492888in}}%
\pgfpathlineto{\pgfqpoint{1.228467in}{1.488579in}}%
\pgfpathlineto{\pgfqpoint{1.219407in}{1.482382in}}%
\pgfpathlineto{\pgfqpoint{1.206523in}{1.474968in}}%
\pgfpathlineto{\pgfqpoint{1.203751in}{1.473274in}}%
\pgfpathlineto{\pgfqpoint{1.188094in}{1.465842in}}%
\pgfpathlineto{\pgfqpoint{1.174400in}{1.461357in}}%
\pgfpathlineto{\pgfqpoint{1.172438in}{1.460667in}}%
\pgfpathlineto{\pgfqpoint{1.156781in}{1.457875in}}%
\pgfpathlineto{\pgfqpoint{1.141125in}{1.457875in}}%
\pgfpathlineto{\pgfqpoint{1.125468in}{1.460667in}}%
\pgfpathlineto{\pgfqpoint{1.123506in}{1.461357in}}%
\pgfpathlineto{\pgfqpoint{1.109812in}{1.465842in}}%
\pgfpathlineto{\pgfqpoint{1.094155in}{1.473274in}}%
\pgfpathlineto{\pgfqpoint{1.091383in}{1.474968in}}%
\pgfpathlineto{\pgfqpoint{1.078498in}{1.482382in}}%
\pgfpathlineto{\pgfqpoint{1.069438in}{1.488579in}}%
\pgfpathlineto{\pgfqpoint{1.062842in}{1.492888in}}%
\pgfpathlineto{\pgfqpoint{1.050273in}{1.502191in}}%
\pgfpathlineto{\pgfqpoint{1.047185in}{1.504402in}}%
\pgfpathlineto{\pgfqpoint{1.032640in}{1.515802in}}%
\pgfpathlineto{\pgfqpoint{1.031529in}{1.516657in}}%
\pgfpathlineto{\pgfqpoint{1.015947in}{1.529413in}}%
\pgfpathlineto{\pgfqpoint{1.015872in}{1.529474in}}%
\pgfpathlineto{\pgfqpoint{1.000216in}{1.542764in}}%
\pgfpathlineto{\pgfqpoint{0.999916in}{1.543024in}}%
\pgfpathlineto{\pgfqpoint{0.984559in}{1.556479in}}%
\pgfpathlineto{\pgfqpoint{0.984382in}{1.556635in}}%
\pgfpathlineto{\pgfqpoint{0.969358in}{1.570246in}}%
\pgfpathlineto{\pgfqpoint{0.968902in}{1.570680in}}%
\pgfpathlineto{\pgfqpoint{0.954918in}{1.583857in}}%
\pgfpathlineto{\pgfqpoint{0.953246in}{1.585556in}}%
\pgfpathlineto{\pgfqpoint{0.941216in}{1.597468in}}%
\pgfpathlineto{\pgfqpoint{0.937589in}{1.601463in}}%
\pgfpathlineto{\pgfqpoint{0.928505in}{1.611079in}}%
\pgfpathlineto{\pgfqpoint{0.921933in}{1.619139in}}%
\pgfpathlineto{\pgfqpoint{0.917158in}{1.624691in}}%
\pgfpathlineto{\pgfqpoint{0.907640in}{1.638302in}}%
\pgfpathlineto{\pgfqpoint{0.906276in}{1.640943in}}%
\pgfpathlineto{\pgfqpoint{0.900238in}{1.651913in}}%
\pgfpathlineto{\pgfqpoint{0.895623in}{1.665524in}}%
\pgfpathlineto{\pgfqpoint{0.894062in}{1.679135in}}%
\pgfpathlineto{\pgfqpoint{0.890620in}{1.679135in}}%
\pgfpathlineto{\pgfqpoint{0.874963in}{1.679135in}}%
\pgfpathlineto{\pgfqpoint{0.859306in}{1.679135in}}%
\pgfpathlineto{\pgfqpoint{0.843650in}{1.679135in}}%
\pgfpathlineto{\pgfqpoint{0.827993in}{1.679135in}}%
\pgfpathlineto{\pgfqpoint{0.812337in}{1.679135in}}%
\pgfpathlineto{\pgfqpoint{0.796680in}{1.679135in}}%
\pgfpathlineto{\pgfqpoint{0.781024in}{1.679135in}}%
\pgfpathlineto{\pgfqpoint{0.765367in}{1.679135in}}%
\pgfpathlineto{\pgfqpoint{0.749710in}{1.679135in}}%
\pgfpathlineto{\pgfqpoint{0.734054in}{1.679135in}}%
\pgfpathlineto{\pgfqpoint{0.718397in}{1.679135in}}%
\pgfpathlineto{\pgfqpoint{0.702741in}{1.679135in}}%
\pgfpathlineto{\pgfqpoint{0.687084in}{1.679135in}}%
\pgfpathlineto{\pgfqpoint{0.671428in}{1.679135in}}%
\pgfpathlineto{\pgfqpoint{0.655771in}{1.679135in}}%
\pgfpathlineto{\pgfqpoint{0.640115in}{1.679135in}}%
\pgfpathlineto{\pgfqpoint{0.628868in}{1.679135in}}%
\pgfpathlineto{\pgfqpoint{0.627253in}{1.665524in}}%
\pgfpathlineto{\pgfqpoint{0.624458in}{1.657575in}}%
\pgfpathlineto{\pgfqpoint{0.622600in}{1.651913in}}%
\pgfpathlineto{\pgfqpoint{0.615336in}{1.638302in}}%
\pgfpathlineto{\pgfqpoint{0.608801in}{1.629224in}}%
\pgfpathlineto{\pgfqpoint{0.605723in}{1.624691in}}%
\pgfpathlineto{\pgfqpoint{0.594368in}{1.611079in}}%
\pgfpathlineto{\pgfqpoint{0.593145in}{1.609821in}}%
\pgfpathlineto{\pgfqpoint{0.581699in}{1.597468in}}%
\pgfpathlineto{\pgfqpoint{0.577488in}{1.593384in}}%
\pgfpathlineto{\pgfqpoint{0.567992in}{1.583857in}}%
\pgfpathlineto{\pgfqpoint{0.561832in}{1.578127in}}%
\pgfpathlineto{\pgfqpoint{0.553523in}{1.570246in}}%
\pgfpathlineto{\pgfqpoint{0.546175in}{1.563620in}}%
\pgfpathlineto{\pgfqpoint{0.538465in}{1.556635in}}%
\pgfpathlineto{\pgfqpoint{0.530519in}{1.549647in}}%
\pgfpathlineto{\pgfqpoint{0.522900in}{1.543024in}}%
\pgfpathlineto{\pgfqpoint{0.514862in}{1.536116in}}%
\pgfpathlineto{\pgfqpoint{0.506827in}{1.529413in}}%
\pgfpathlineto{\pgfqpoint{0.499205in}{1.523025in}}%
\pgfpathlineto{\pgfqpoint{0.490140in}{1.515802in}}%
\pgfpathlineto{\pgfqpoint{0.483549in}{1.510446in}}%
\pgfpathlineto{\pgfqpoint{0.472590in}{1.502191in}}%
\pgfpathlineto{\pgfqpoint{0.467892in}{1.498530in}}%
\pgfpathlineto{\pgfqpoint{0.453684in}{1.488579in}}%
\pgfpathlineto{\pgfqpoint{0.452236in}{1.487516in}}%
\pgfpathlineto{\pgfqpoint{0.436579in}{1.477644in}}%
\pgfpathlineto{\pgfqpoint{0.431364in}{1.474968in}}%
\pgfpathlineto{\pgfqpoint{0.420923in}{1.469288in}}%
\pgfpathlineto{\pgfqpoint{0.405266in}{1.462973in}}%
\pgfpathlineto{\pgfqpoint{0.398752in}{1.461357in}}%
\pgfpathlineto{\pgfqpoint{0.389609in}{1.458927in}}%
\pgfpathlineto{\pgfqpoint{0.373953in}{1.457524in}}%
\pgfpathlineto{\pgfqpoint{0.373953in}{1.447746in}}%
\pgfpathlineto{\pgfqpoint{0.373953in}{1.434135in}}%
\pgfpathlineto{\pgfqpoint{0.373953in}{1.420524in}}%
\pgfpathlineto{\pgfqpoint{0.373953in}{1.406913in}}%
\pgfpathlineto{\pgfqpoint{0.373953in}{1.393302in}}%
\pgfpathlineto{\pgfqpoint{0.373953in}{1.379691in}}%
\pgfpathlineto{\pgfqpoint{0.373953in}{1.366079in}}%
\pgfpathlineto{\pgfqpoint{0.373953in}{1.352468in}}%
\pgfpathlineto{\pgfqpoint{0.373953in}{1.338857in}}%
\pgfpathlineto{\pgfqpoint{0.373953in}{1.325246in}}%
\pgfpathlineto{\pgfqpoint{0.373953in}{1.311635in}}%
\pgfpathlineto{\pgfqpoint{0.373953in}{1.298024in}}%
\pgfpathlineto{\pgfqpoint{0.373953in}{1.284413in}}%
\pgfpathlineto{\pgfqpoint{0.373953in}{1.270802in}}%
\pgfpathlineto{\pgfqpoint{0.373953in}{1.257191in}}%
\pgfpathlineto{\pgfqpoint{0.373953in}{1.243579in}}%
\pgfpathlineto{\pgfqpoint{0.373953in}{1.229968in}}%
\pgfpathlineto{\pgfqpoint{0.373953in}{1.226975in}}%
\pgfpathlineto{\pgfqpoint{0.389609in}{1.225618in}}%
\pgfpathlineto{\pgfqpoint{0.405266in}{1.221606in}}%
\pgfpathlineto{\pgfqpoint{0.417885in}{1.216357in}}%
\pgfpathlineto{\pgfqpoint{0.420923in}{1.215172in}}%
\pgfpathlineto{\pgfqpoint{0.436579in}{1.206897in}}%
\pgfpathlineto{\pgfqpoint{0.442965in}{1.202746in}}%
\pgfpathlineto{\pgfqpoint{0.452236in}{1.197032in}}%
\pgfpathlineto{\pgfqpoint{0.463298in}{1.189135in}}%
\pgfpathlineto{\pgfqpoint{0.467892in}{1.185982in}}%
\pgfpathlineto{\pgfqpoint{0.481594in}{1.175524in}}%
\pgfpathlineto{\pgfqpoint{0.483549in}{1.174070in}}%
\pgfpathlineto{\pgfqpoint{0.498706in}{1.161913in}}%
\pgfpathlineto{\pgfqpoint{0.499205in}{1.161517in}}%
\pgfpathlineto{\pgfqpoint{0.514862in}{1.148456in}}%
\pgfpathlineto{\pgfqpoint{0.515041in}{1.148302in}}%
\pgfpathlineto{\pgfqpoint{0.530519in}{1.134951in}}%
\pgfpathlineto{\pgfqpoint{0.530817in}{1.134691in}}%
\pgfpathlineto{\pgfqpoint{0.546105in}{1.121079in}}%
\pgfpathlineto{\pgfqpoint{0.546175in}{1.121014in}}%
\pgfpathlineto{\pgfqpoint{0.560848in}{1.107468in}}%
\pgfpathlineto{\pgfqpoint{0.561832in}{1.106502in}}%
\pgfpathlineto{\pgfqpoint{0.574944in}{1.093857in}}%
\pgfpathlineto{\pgfqpoint{0.577488in}{1.091173in}}%
\pgfpathlineto{\pgfqpoint{0.588189in}{1.080246in}}%
\pgfpathlineto{\pgfqpoint{0.593145in}{1.074511in}}%
\pgfpathlineto{\pgfqpoint{0.600274in}{1.066635in}}%
\pgfpathlineto{\pgfqpoint{0.608801in}{1.055434in}}%
\pgfpathlineto{\pgfqpoint{0.610750in}{1.053024in}}%
\pgfpathlineto{\pgfqpoint{0.619299in}{1.039413in}}%
\pgfpathlineto{\pgfqpoint{0.624458in}{1.027508in}}%
\pgfpathlineto{\pgfqpoint{0.625252in}{1.025802in}}%
\pgfpathlineto{\pgfqpoint{0.628463in}{1.012191in}}%
\pgfpathlineto{\pgfqpoint{0.628463in}{0.998579in}}%
\pgfpathlineto{\pgfqpoint{0.625252in}{0.984968in}}%
\pgfpathlineto{\pgfqpoint{0.624458in}{0.983262in}}%
\pgfpathlineto{\pgfqpoint{0.619299in}{0.971357in}}%
\pgfpathlineto{\pgfqpoint{0.610750in}{0.957746in}}%
\pgfpathlineto{\pgfqpoint{0.608801in}{0.955336in}}%
\pgfpathlineto{\pgfqpoint{0.600274in}{0.944135in}}%
\pgfpathlineto{\pgfqpoint{0.593145in}{0.936259in}}%
\pgfpathlineto{\pgfqpoint{0.588189in}{0.930524in}}%
\pgfpathlineto{\pgfqpoint{0.577488in}{0.919597in}}%
\pgfpathlineto{\pgfqpoint{0.574944in}{0.916913in}}%
\pgfpathlineto{\pgfqpoint{0.561832in}{0.904268in}}%
\pgfpathlineto{\pgfqpoint{0.560848in}{0.903302in}}%
\pgfpathlineto{\pgfqpoint{0.546175in}{0.889756in}}%
\pgfpathlineto{\pgfqpoint{0.546105in}{0.889691in}}%
\pgfpathlineto{\pgfqpoint{0.530817in}{0.876079in}}%
\pgfpathlineto{\pgfqpoint{0.530519in}{0.875819in}}%
\pgfpathlineto{\pgfqpoint{0.515041in}{0.862468in}}%
\pgfpathlineto{\pgfqpoint{0.514862in}{0.862314in}}%
\pgfpathlineto{\pgfqpoint{0.499205in}{0.849253in}}%
\pgfpathlineto{\pgfqpoint{0.498706in}{0.848857in}}%
\pgfpathlineto{\pgfqpoint{0.483549in}{0.836700in}}%
\pgfpathlineto{\pgfqpoint{0.481594in}{0.835246in}}%
\pgfpathlineto{\pgfqpoint{0.467892in}{0.824788in}}%
\pgfpathlineto{\pgfqpoint{0.463298in}{0.821635in}}%
\pgfpathlineto{\pgfqpoint{0.452236in}{0.813738in}}%
\pgfpathlineto{\pgfqpoint{0.442965in}{0.808024in}}%
\pgfpathlineto{\pgfqpoint{0.436579in}{0.803873in}}%
\pgfpathlineto{\pgfqpoint{0.420923in}{0.795598in}}%
\pgfpathlineto{\pgfqpoint{0.417885in}{0.794413in}}%
\pgfpathlineto{\pgfqpoint{0.405266in}{0.789164in}}%
\pgfpathlineto{\pgfqpoint{0.389609in}{0.785152in}}%
\pgfpathlineto{\pgfqpoint{0.373953in}{0.783795in}}%
\pgfpathlineto{\pgfqpoint{0.373953in}{0.780802in}}%
\pgfpathlineto{\pgfqpoint{0.373953in}{0.767191in}}%
\pgfpathlineto{\pgfqpoint{0.373953in}{0.753579in}}%
\pgfpathlineto{\pgfqpoint{0.373953in}{0.739968in}}%
\pgfpathlineto{\pgfqpoint{0.373953in}{0.726357in}}%
\pgfpathlineto{\pgfqpoint{0.373953in}{0.712746in}}%
\pgfpathlineto{\pgfqpoint{0.373953in}{0.699135in}}%
\pgfpathlineto{\pgfqpoint{0.373953in}{0.685524in}}%
\pgfpathlineto{\pgfqpoint{0.373953in}{0.671913in}}%
\pgfpathlineto{\pgfqpoint{0.373953in}{0.658302in}}%
\pgfpathlineto{\pgfqpoint{0.373953in}{0.644691in}}%
\pgfpathlineto{\pgfqpoint{0.373953in}{0.631079in}}%
\pgfpathlineto{\pgfqpoint{0.373953in}{0.617468in}}%
\pgfpathlineto{\pgfqpoint{0.373953in}{0.603857in}}%
\pgfpathlineto{\pgfqpoint{0.373953in}{0.590246in}}%
\pgfpathlineto{\pgfqpoint{0.373953in}{0.576635in}}%
\pgfpathlineto{\pgfqpoint{0.373953in}{0.563024in}}%
\pgfpathlineto{\pgfqpoint{0.373953in}{0.553246in}}%
\pgfpathlineto{\pgfqpoint{0.389609in}{0.551843in}}%
\pgfpathlineto{\pgfqpoint{0.398752in}{0.549413in}}%
\pgfpathlineto{\pgfqpoint{0.405266in}{0.547797in}}%
\pgfpathlineto{\pgfqpoint{0.420923in}{0.541482in}}%
\pgfpathlineto{\pgfqpoint{0.431364in}{0.535802in}}%
\pgfpathlineto{\pgfqpoint{0.436579in}{0.533126in}}%
\pgfpathlineto{\pgfqpoint{0.452236in}{0.523254in}}%
\pgfpathlineto{\pgfqpoint{0.453684in}{0.522191in}}%
\pgfpathlineto{\pgfqpoint{0.467892in}{0.512240in}}%
\pgfpathlineto{\pgfqpoint{0.472590in}{0.508579in}}%
\pgfpathlineto{\pgfqpoint{0.483549in}{0.500324in}}%
\pgfpathlineto{\pgfqpoint{0.490140in}{0.494968in}}%
\pgfpathlineto{\pgfqpoint{0.499205in}{0.487745in}}%
\pgfpathlineto{\pgfqpoint{0.506827in}{0.481357in}}%
\pgfpathlineto{\pgfqpoint{0.514862in}{0.474654in}}%
\pgfpathlineto{\pgfqpoint{0.522900in}{0.467746in}}%
\pgfpathlineto{\pgfqpoint{0.530519in}{0.461123in}}%
\pgfpathlineto{\pgfqpoint{0.538465in}{0.454135in}}%
\pgfpathlineto{\pgfqpoint{0.546175in}{0.447150in}}%
\pgfpathlineto{\pgfqpoint{0.553523in}{0.440524in}}%
\pgfpathlineto{\pgfqpoint{0.561832in}{0.432643in}}%
\pgfpathlineto{\pgfqpoint{0.567992in}{0.426913in}}%
\pgfpathlineto{\pgfqpoint{0.577488in}{0.417386in}}%
\pgfpathlineto{\pgfqpoint{0.581699in}{0.413302in}}%
\pgfpathlineto{\pgfqpoint{0.593145in}{0.400949in}}%
\pgfpathlineto{\pgfqpoint{0.594368in}{0.399691in}}%
\pgfpathlineto{\pgfqpoint{0.605723in}{0.386079in}}%
\pgfpathlineto{\pgfqpoint{0.608801in}{0.381546in}}%
\pgfpathlineto{\pgfqpoint{0.615336in}{0.372468in}}%
\pgfpathlineto{\pgfqpoint{0.622600in}{0.358857in}}%
\pgfpathlineto{\pgfqpoint{0.624458in}{0.353195in}}%
\pgfpathlineto{\pgfqpoint{0.627253in}{0.345246in}}%
\pgfpathlineto{\pgfqpoint{0.628868in}{0.331635in}}%
\pgfpathlineto{\pgfqpoint{0.640115in}{0.331635in}}%
\pgfpathclose%
\pgfpathmoveto{\pgfqpoint{0.748782in}{0.399691in}}%
\pgfpathlineto{\pgfqpoint{0.734054in}{0.401831in}}%
\pgfpathlineto{\pgfqpoint{0.718397in}{0.405930in}}%
\pgfpathlineto{\pgfqpoint{0.702741in}{0.411855in}}%
\pgfpathlineto{\pgfqpoint{0.699792in}{0.413302in}}%
\pgfpathlineto{\pgfqpoint{0.687084in}{0.418820in}}%
\pgfpathlineto{\pgfqpoint{0.671966in}{0.426913in}}%
\pgfpathlineto{\pgfqpoint{0.671428in}{0.427176in}}%
\pgfpathlineto{\pgfqpoint{0.655771in}{0.436173in}}%
\pgfpathlineto{\pgfqpoint{0.649160in}{0.440524in}}%
\pgfpathlineto{\pgfqpoint{0.640115in}{0.446132in}}%
\pgfpathlineto{\pgfqpoint{0.628573in}{0.454135in}}%
\pgfpathlineto{\pgfqpoint{0.624458in}{0.456886in}}%
\pgfpathlineto{\pgfqpoint{0.609625in}{0.467746in}}%
\pgfpathlineto{\pgfqpoint{0.608801in}{0.468340in}}%
\pgfpathlineto{\pgfqpoint{0.593145in}{0.480443in}}%
\pgfpathlineto{\pgfqpoint{0.592024in}{0.481357in}}%
\pgfpathlineto{\pgfqpoint{0.577488in}{0.493262in}}%
\pgfpathlineto{\pgfqpoint{0.575483in}{0.494968in}}%
\pgfpathlineto{\pgfqpoint{0.561832in}{0.506836in}}%
\pgfpathlineto{\pgfqpoint{0.559869in}{0.508579in}}%
\pgfpathlineto{\pgfqpoint{0.546175in}{0.521216in}}%
\pgfpathlineto{\pgfqpoint{0.545124in}{0.522191in}}%
\pgfpathlineto{\pgfqpoint{0.531201in}{0.535802in}}%
\pgfpathlineto{\pgfqpoint{0.530519in}{0.536518in}}%
\pgfpathlineto{\pgfqpoint{0.518026in}{0.549413in}}%
\pgfpathlineto{\pgfqpoint{0.514862in}{0.552990in}}%
\pgfpathlineto{\pgfqpoint{0.505657in}{0.563024in}}%
\pgfpathlineto{\pgfqpoint{0.499205in}{0.570888in}}%
\pgfpathlineto{\pgfqpoint{0.494201in}{0.576635in}}%
\pgfpathlineto{\pgfqpoint{0.483852in}{0.590246in}}%
\pgfpathlineto{\pgfqpoint{0.483549in}{0.590714in}}%
\pgfpathlineto{\pgfqpoint{0.474240in}{0.603857in}}%
\pgfpathlineto{\pgfqpoint{0.467892in}{0.614905in}}%
\pgfpathlineto{\pgfqpoint{0.466229in}{0.617468in}}%
\pgfpathlineto{\pgfqpoint{0.459413in}{0.631079in}}%
\pgfpathlineto{\pgfqpoint{0.454698in}{0.644691in}}%
\pgfpathlineto{\pgfqpoint{0.452236in}{0.657495in}}%
\pgfpathlineto{\pgfqpoint{0.452052in}{0.658302in}}%
\pgfpathlineto{\pgfqpoint{0.451429in}{0.671913in}}%
\pgfpathlineto{\pgfqpoint{0.452236in}{0.677797in}}%
\pgfpathlineto{\pgfqpoint{0.453128in}{0.685524in}}%
\pgfpathlineto{\pgfqpoint{0.456793in}{0.699135in}}%
\pgfpathlineto{\pgfqpoint{0.462558in}{0.712746in}}%
\pgfpathlineto{\pgfqpoint{0.467892in}{0.722006in}}%
\pgfpathlineto{\pgfqpoint{0.470108in}{0.726357in}}%
\pgfpathlineto{\pgfqpoint{0.478833in}{0.739968in}}%
\pgfpathlineto{\pgfqpoint{0.483549in}{0.746099in}}%
\pgfpathlineto{\pgfqpoint{0.488820in}{0.753579in}}%
\pgfpathlineto{\pgfqpoint{0.499205in}{0.766249in}}%
\pgfpathlineto{\pgfqpoint{0.499932in}{0.767191in}}%
\pgfpathlineto{\pgfqpoint{0.511753in}{0.780802in}}%
\pgfpathlineto{\pgfqpoint{0.514862in}{0.784043in}}%
\pgfpathlineto{\pgfqpoint{0.524461in}{0.794413in}}%
\pgfpathlineto{\pgfqpoint{0.530519in}{0.800445in}}%
\pgfpathlineto{\pgfqpoint{0.538013in}{0.808024in}}%
\pgfpathlineto{\pgfqpoint{0.546175in}{0.815770in}}%
\pgfpathlineto{\pgfqpoint{0.552376in}{0.821635in}}%
\pgfpathlineto{\pgfqpoint{0.561832in}{0.830173in}}%
\pgfpathlineto{\pgfqpoint{0.567570in}{0.835246in}}%
\pgfpathlineto{\pgfqpoint{0.577488in}{0.843760in}}%
\pgfpathlineto{\pgfqpoint{0.583655in}{0.848857in}}%
\pgfpathlineto{\pgfqpoint{0.593145in}{0.856604in}}%
\pgfpathlineto{\pgfqpoint{0.600734in}{0.862468in}}%
\pgfpathlineto{\pgfqpoint{0.608801in}{0.868738in}}%
\pgfpathlineto{\pgfqpoint{0.618955in}{0.876079in}}%
\pgfpathlineto{\pgfqpoint{0.624458in}{0.880159in}}%
\pgfpathlineto{\pgfqpoint{0.638534in}{0.889691in}}%
\pgfpathlineto{\pgfqpoint{0.640115in}{0.890812in}}%
\pgfpathlineto{\pgfqpoint{0.655771in}{0.900801in}}%
\pgfpathlineto{\pgfqpoint{0.660256in}{0.903302in}}%
\pgfpathlineto{\pgfqpoint{0.671428in}{0.910007in}}%
\pgfpathlineto{\pgfqpoint{0.685029in}{0.916913in}}%
\pgfpathlineto{\pgfqpoint{0.687084in}{0.918070in}}%
\pgfpathlineto{\pgfqpoint{0.702741in}{0.925293in}}%
\pgfpathlineto{\pgfqpoint{0.717576in}{0.930524in}}%
\pgfpathlineto{\pgfqpoint{0.718397in}{0.930858in}}%
\pgfpathlineto{\pgfqpoint{0.734054in}{0.935302in}}%
\pgfpathlineto{\pgfqpoint{0.749710in}{0.937769in}}%
\pgfpathlineto{\pgfqpoint{0.765367in}{0.938263in}}%
\pgfpathlineto{\pgfqpoint{0.781024in}{0.936783in}}%
\pgfpathlineto{\pgfqpoint{0.796680in}{0.933327in}}%
\pgfpathlineto{\pgfqpoint{0.804789in}{0.930524in}}%
\pgfpathlineto{\pgfqpoint{0.812337in}{0.928265in}}%
\pgfpathlineto{\pgfqpoint{0.827993in}{0.921894in}}%
\pgfpathlineto{\pgfqpoint{0.837702in}{0.916913in}}%
\pgfpathlineto{\pgfqpoint{0.843650in}{0.914163in}}%
\pgfpathlineto{\pgfqpoint{0.859306in}{0.905473in}}%
\pgfpathlineto{\pgfqpoint{0.862690in}{0.903302in}}%
\pgfpathlineto{\pgfqpoint{0.874963in}{0.895976in}}%
\pgfpathlineto{\pgfqpoint{0.884270in}{0.889691in}}%
\pgfpathlineto{\pgfqpoint{0.890620in}{0.885603in}}%
\pgfpathlineto{\pgfqpoint{0.903988in}{0.876079in}}%
\pgfpathlineto{\pgfqpoint{0.906276in}{0.874492in}}%
\pgfpathlineto{\pgfqpoint{0.921933in}{0.862709in}}%
\pgfpathlineto{\pgfqpoint{0.922234in}{0.862468in}}%
\pgfpathlineto{\pgfqpoint{0.937589in}{0.850260in}}%
\pgfpathlineto{\pgfqpoint{0.939273in}{0.848857in}}%
\pgfpathlineto{\pgfqpoint{0.953246in}{0.837068in}}%
\pgfpathlineto{\pgfqpoint{0.955341in}{0.835246in}}%
\pgfpathlineto{\pgfqpoint{0.968902in}{0.823099in}}%
\pgfpathlineto{\pgfqpoint{0.970516in}{0.821635in}}%
\pgfpathlineto{\pgfqpoint{0.984559in}{0.808286in}}%
\pgfpathlineto{\pgfqpoint{0.984836in}{0.808024in}}%
\pgfpathlineto{\pgfqpoint{0.998389in}{0.794413in}}%
\pgfpathlineto{\pgfqpoint{1.000216in}{0.792424in}}%
\pgfpathlineto{\pgfqpoint{1.011170in}{0.780802in}}%
\pgfpathlineto{\pgfqpoint{1.015872in}{0.775282in}}%
\pgfpathlineto{\pgfqpoint{1.023102in}{0.767191in}}%
\pgfpathlineto{\pgfqpoint{1.031529in}{0.756521in}}%
\pgfpathlineto{\pgfqpoint{1.034026in}{0.753579in}}%
\pgfpathlineto{\pgfqpoint{1.044022in}{0.739968in}}%
\pgfpathlineto{\pgfqpoint{1.047185in}{0.734797in}}%
\pgfpathlineto{\pgfqpoint{1.052916in}{0.726357in}}%
\pgfpathlineto{\pgfqpoint{1.060243in}{0.712746in}}%
\pgfpathlineto{\pgfqpoint{1.062842in}{0.706184in}}%
\pgfpathlineto{\pgfqpoint{1.066067in}{0.699135in}}%
\pgfpathlineto{\pgfqpoint{1.070041in}{0.685524in}}%
\pgfpathlineto{\pgfqpoint{1.071743in}{0.671913in}}%
\pgfpathlineto{\pgfqpoint{1.071176in}{0.658302in}}%
\pgfpathlineto{\pgfqpoint{1.068338in}{0.644691in}}%
\pgfpathlineto{\pgfqpoint{1.063226in}{0.631079in}}%
\pgfpathlineto{\pgfqpoint{1.062842in}{0.630365in}}%
\pgfpathlineto{\pgfqpoint{1.056824in}{0.617468in}}%
\pgfpathlineto{\pgfqpoint{1.048516in}{0.603857in}}%
\pgfpathlineto{\pgfqpoint{1.047185in}{0.602070in}}%
\pgfpathlineto{\pgfqpoint{1.039241in}{0.590246in}}%
\pgfpathlineto{\pgfqpoint{1.031529in}{0.580534in}}%
\pgfpathlineto{\pgfqpoint{1.028652in}{0.576635in}}%
\pgfpathlineto{\pgfqpoint{1.017162in}{0.563024in}}%
\pgfpathlineto{\pgfqpoint{1.015872in}{0.561650in}}%
\pgfpathlineto{\pgfqpoint{1.004908in}{0.549413in}}%
\pgfpathlineto{\pgfqpoint{1.000216in}{0.544629in}}%
\pgfpathlineto{\pgfqpoint{0.991771in}{0.535802in}}%
\pgfpathlineto{\pgfqpoint{0.984559in}{0.528788in}}%
\pgfpathlineto{\pgfqpoint{0.977813in}{0.522191in}}%
\pgfpathlineto{\pgfqpoint{0.968902in}{0.513941in}}%
\pgfpathlineto{\pgfqpoint{0.963040in}{0.508579in}}%
\pgfpathlineto{\pgfqpoint{0.953246in}{0.499957in}}%
\pgfpathlineto{\pgfqpoint{0.947410in}{0.494968in}}%
\pgfpathlineto{\pgfqpoint{0.937589in}{0.486748in}}%
\pgfpathlineto{\pgfqpoint{0.930843in}{0.481357in}}%
\pgfpathlineto{\pgfqpoint{0.921933in}{0.474261in}}%
\pgfpathlineto{\pgfqpoint{0.913215in}{0.467746in}}%
\pgfpathlineto{\pgfqpoint{0.906276in}{0.462480in}}%
\pgfpathlineto{\pgfqpoint{0.894348in}{0.454135in}}%
\pgfpathlineto{\pgfqpoint{0.890620in}{0.451432in}}%
\pgfpathlineto{\pgfqpoint{0.874963in}{0.441155in}}%
\pgfpathlineto{\pgfqpoint{0.873880in}{0.440524in}}%
\pgfpathlineto{\pgfqpoint{0.859306in}{0.431496in}}%
\pgfpathlineto{\pgfqpoint{0.850702in}{0.426913in}}%
\pgfpathlineto{\pgfqpoint{0.843650in}{0.422813in}}%
\pgfpathlineto{\pgfqpoint{0.827993in}{0.415228in}}%
\pgfpathlineto{\pgfqpoint{0.822989in}{0.413302in}}%
\pgfpathlineto{\pgfqpoint{0.812337in}{0.408664in}}%
\pgfpathlineto{\pgfqpoint{0.796680in}{0.403653in}}%
\pgfpathlineto{\pgfqpoint{0.781024in}{0.400466in}}%
\pgfpathlineto{\pgfqpoint{0.772136in}{0.399691in}}%
\pgfpathlineto{\pgfqpoint{0.765367in}{0.398990in}}%
\pgfpathlineto{\pgfqpoint{0.749710in}{0.399531in}}%
\pgfpathlineto{\pgfqpoint{0.748782in}{0.399691in}}%
\pgfpathclose%
\pgfpathmoveto{\pgfqpoint{1.525770in}{0.399691in}}%
\pgfpathlineto{\pgfqpoint{1.516882in}{0.400466in}}%
\pgfpathlineto{\pgfqpoint{1.501226in}{0.403653in}}%
\pgfpathlineto{\pgfqpoint{1.485569in}{0.408664in}}%
\pgfpathlineto{\pgfqpoint{1.474917in}{0.413302in}}%
\pgfpathlineto{\pgfqpoint{1.469913in}{0.415228in}}%
\pgfpathlineto{\pgfqpoint{1.454256in}{0.422813in}}%
\pgfpathlineto{\pgfqpoint{1.447204in}{0.426913in}}%
\pgfpathlineto{\pgfqpoint{1.438599in}{0.431496in}}%
\pgfpathlineto{\pgfqpoint{1.424026in}{0.440524in}}%
\pgfpathlineto{\pgfqpoint{1.422943in}{0.441155in}}%
\pgfpathlineto{\pgfqpoint{1.407286in}{0.451432in}}%
\pgfpathlineto{\pgfqpoint{1.403558in}{0.454135in}}%
\pgfpathlineto{\pgfqpoint{1.391630in}{0.462480in}}%
\pgfpathlineto{\pgfqpoint{1.384690in}{0.467746in}}%
\pgfpathlineto{\pgfqpoint{1.375973in}{0.474261in}}%
\pgfpathlineto{\pgfqpoint{1.367063in}{0.481357in}}%
\pgfpathlineto{\pgfqpoint{1.360317in}{0.486748in}}%
\pgfpathlineto{\pgfqpoint{1.350496in}{0.494968in}}%
\pgfpathlineto{\pgfqpoint{1.344660in}{0.499957in}}%
\pgfpathlineto{\pgfqpoint{1.334866in}{0.508579in}}%
\pgfpathlineto{\pgfqpoint{1.329003in}{0.513941in}}%
\pgfpathlineto{\pgfqpoint{1.320093in}{0.522191in}}%
\pgfpathlineto{\pgfqpoint{1.313347in}{0.528788in}}%
\pgfpathlineto{\pgfqpoint{1.306135in}{0.535802in}}%
\pgfpathlineto{\pgfqpoint{1.297690in}{0.544629in}}%
\pgfpathlineto{\pgfqpoint{1.292998in}{0.549413in}}%
\pgfpathlineto{\pgfqpoint{1.282034in}{0.561650in}}%
\pgfpathlineto{\pgfqpoint{1.280744in}{0.563024in}}%
\pgfpathlineto{\pgfqpoint{1.269254in}{0.576635in}}%
\pgfpathlineto{\pgfqpoint{1.266377in}{0.580534in}}%
\pgfpathlineto{\pgfqpoint{1.258664in}{0.590246in}}%
\pgfpathlineto{\pgfqpoint{1.250721in}{0.602070in}}%
\pgfpathlineto{\pgfqpoint{1.249390in}{0.603857in}}%
\pgfpathlineto{\pgfqpoint{1.241081in}{0.617468in}}%
\pgfpathlineto{\pgfqpoint{1.235064in}{0.630365in}}%
\pgfpathlineto{\pgfqpoint{1.234680in}{0.631079in}}%
\pgfpathlineto{\pgfqpoint{1.229568in}{0.644691in}}%
\pgfpathlineto{\pgfqpoint{1.226730in}{0.658302in}}%
\pgfpathlineto{\pgfqpoint{1.226162in}{0.671913in}}%
\pgfpathlineto{\pgfqpoint{1.227865in}{0.685524in}}%
\pgfpathlineto{\pgfqpoint{1.231839in}{0.699135in}}%
\pgfpathlineto{\pgfqpoint{1.235064in}{0.706184in}}%
\pgfpathlineto{\pgfqpoint{1.237663in}{0.712746in}}%
\pgfpathlineto{\pgfqpoint{1.244990in}{0.726357in}}%
\pgfpathlineto{\pgfqpoint{1.250721in}{0.734797in}}%
\pgfpathlineto{\pgfqpoint{1.253883in}{0.739968in}}%
\pgfpathlineto{\pgfqpoint{1.263880in}{0.753579in}}%
\pgfpathlineto{\pgfqpoint{1.266377in}{0.756521in}}%
\pgfpathlineto{\pgfqpoint{1.274804in}{0.767191in}}%
\pgfpathlineto{\pgfqpoint{1.282034in}{0.775282in}}%
\pgfpathlineto{\pgfqpoint{1.286736in}{0.780802in}}%
\pgfpathlineto{\pgfqpoint{1.297690in}{0.792424in}}%
\pgfpathlineto{\pgfqpoint{1.299517in}{0.794413in}}%
\pgfpathlineto{\pgfqpoint{1.313070in}{0.808024in}}%
\pgfpathlineto{\pgfqpoint{1.313347in}{0.808286in}}%
\pgfpathlineto{\pgfqpoint{1.327390in}{0.821635in}}%
\pgfpathlineto{\pgfqpoint{1.329003in}{0.823099in}}%
\pgfpathlineto{\pgfqpoint{1.342564in}{0.835246in}}%
\pgfpathlineto{\pgfqpoint{1.344660in}{0.837068in}}%
\pgfpathlineto{\pgfqpoint{1.358632in}{0.848857in}}%
\pgfpathlineto{\pgfqpoint{1.360317in}{0.850260in}}%
\pgfpathlineto{\pgfqpoint{1.375671in}{0.862468in}}%
\pgfpathlineto{\pgfqpoint{1.375973in}{0.862709in}}%
\pgfpathlineto{\pgfqpoint{1.391630in}{0.874492in}}%
\pgfpathlineto{\pgfqpoint{1.393917in}{0.876079in}}%
\pgfpathlineto{\pgfqpoint{1.407286in}{0.885603in}}%
\pgfpathlineto{\pgfqpoint{1.413635in}{0.889691in}}%
\pgfpathlineto{\pgfqpoint{1.422943in}{0.895976in}}%
\pgfpathlineto{\pgfqpoint{1.435216in}{0.903302in}}%
\pgfpathlineto{\pgfqpoint{1.438599in}{0.905473in}}%
\pgfpathlineto{\pgfqpoint{1.454256in}{0.914163in}}%
\pgfpathlineto{\pgfqpoint{1.460204in}{0.916913in}}%
\pgfpathlineto{\pgfqpoint{1.469913in}{0.921894in}}%
\pgfpathlineto{\pgfqpoint{1.485569in}{0.928265in}}%
\pgfpathlineto{\pgfqpoint{1.493117in}{0.930524in}}%
\pgfpathlineto{\pgfqpoint{1.501226in}{0.933327in}}%
\pgfpathlineto{\pgfqpoint{1.516882in}{0.936783in}}%
\pgfpathlineto{\pgfqpoint{1.532539in}{0.938263in}}%
\pgfpathlineto{\pgfqpoint{1.548195in}{0.937769in}}%
\pgfpathlineto{\pgfqpoint{1.563852in}{0.935302in}}%
\pgfpathlineto{\pgfqpoint{1.579508in}{0.930858in}}%
\pgfpathlineto{\pgfqpoint{1.580330in}{0.930524in}}%
\pgfpathlineto{\pgfqpoint{1.595165in}{0.925293in}}%
\pgfpathlineto{\pgfqpoint{1.610822in}{0.918070in}}%
\pgfpathlineto{\pgfqpoint{1.612877in}{0.916913in}}%
\pgfpathlineto{\pgfqpoint{1.626478in}{0.910007in}}%
\pgfpathlineto{\pgfqpoint{1.637649in}{0.903302in}}%
\pgfpathlineto{\pgfqpoint{1.642135in}{0.900801in}}%
\pgfpathlineto{\pgfqpoint{1.657791in}{0.890812in}}%
\pgfpathlineto{\pgfqpoint{1.659372in}{0.889691in}}%
\pgfpathlineto{\pgfqpoint{1.673448in}{0.880159in}}%
\pgfpathlineto{\pgfqpoint{1.678950in}{0.876079in}}%
\pgfpathlineto{\pgfqpoint{1.689104in}{0.868738in}}%
\pgfpathlineto{\pgfqpoint{1.697172in}{0.862468in}}%
\pgfpathlineto{\pgfqpoint{1.704761in}{0.856604in}}%
\pgfpathlineto{\pgfqpoint{1.714250in}{0.848857in}}%
\pgfpathlineto{\pgfqpoint{1.720418in}{0.843760in}}%
\pgfpathlineto{\pgfqpoint{1.730336in}{0.835246in}}%
\pgfpathlineto{\pgfqpoint{1.736074in}{0.830173in}}%
\pgfpathlineto{\pgfqpoint{1.745530in}{0.821635in}}%
\pgfpathlineto{\pgfqpoint{1.751731in}{0.815770in}}%
\pgfpathlineto{\pgfqpoint{1.759893in}{0.808024in}}%
\pgfpathlineto{\pgfqpoint{1.767387in}{0.800445in}}%
\pgfpathlineto{\pgfqpoint{1.773444in}{0.794413in}}%
\pgfpathlineto{\pgfqpoint{1.783044in}{0.784043in}}%
\pgfpathlineto{\pgfqpoint{1.786152in}{0.780802in}}%
\pgfpathlineto{\pgfqpoint{1.797974in}{0.767191in}}%
\pgfpathlineto{\pgfqpoint{1.798700in}{0.766249in}}%
\pgfpathlineto{\pgfqpoint{1.809085in}{0.753579in}}%
\pgfpathlineto{\pgfqpoint{1.814357in}{0.746099in}}%
\pgfpathlineto{\pgfqpoint{1.819073in}{0.739968in}}%
\pgfpathlineto{\pgfqpoint{1.827797in}{0.726357in}}%
\pgfpathlineto{\pgfqpoint{1.830014in}{0.722006in}}%
\pgfpathlineto{\pgfqpoint{1.835348in}{0.712746in}}%
\pgfpathlineto{\pgfqpoint{1.841113in}{0.699135in}}%
\pgfpathlineto{\pgfqpoint{1.844778in}{0.685524in}}%
\pgfpathlineto{\pgfqpoint{1.845670in}{0.677797in}}%
\pgfpathlineto{\pgfqpoint{1.846476in}{0.671913in}}%
\pgfpathlineto{\pgfqpoint{1.845854in}{0.658302in}}%
\pgfpathlineto{\pgfqpoint{1.845670in}{0.657495in}}%
\pgfpathlineto{\pgfqpoint{1.843207in}{0.644691in}}%
\pgfpathlineto{\pgfqpoint{1.838493in}{0.631079in}}%
\pgfpathlineto{\pgfqpoint{1.831677in}{0.617468in}}%
\pgfpathlineto{\pgfqpoint{1.830014in}{0.614905in}}%
\pgfpathlineto{\pgfqpoint{1.823666in}{0.603857in}}%
\pgfpathlineto{\pgfqpoint{1.814357in}{0.590714in}}%
\pgfpathlineto{\pgfqpoint{1.814054in}{0.590246in}}%
\pgfpathlineto{\pgfqpoint{1.803705in}{0.576635in}}%
\pgfpathlineto{\pgfqpoint{1.798700in}{0.570888in}}%
\pgfpathlineto{\pgfqpoint{1.792249in}{0.563024in}}%
\pgfpathlineto{\pgfqpoint{1.783044in}{0.552990in}}%
\pgfpathlineto{\pgfqpoint{1.779880in}{0.549413in}}%
\pgfpathlineto{\pgfqpoint{1.767387in}{0.536518in}}%
\pgfpathlineto{\pgfqpoint{1.766704in}{0.535802in}}%
\pgfpathlineto{\pgfqpoint{1.752782in}{0.522191in}}%
\pgfpathlineto{\pgfqpoint{1.751731in}{0.521216in}}%
\pgfpathlineto{\pgfqpoint{1.738037in}{0.508579in}}%
\pgfpathlineto{\pgfqpoint{1.736074in}{0.506836in}}%
\pgfpathlineto{\pgfqpoint{1.722423in}{0.494968in}}%
\pgfpathlineto{\pgfqpoint{1.720418in}{0.493262in}}%
\pgfpathlineto{\pgfqpoint{1.705882in}{0.481357in}}%
\pgfpathlineto{\pgfqpoint{1.704761in}{0.480443in}}%
\pgfpathlineto{\pgfqpoint{1.689104in}{0.468340in}}%
\pgfpathlineto{\pgfqpoint{1.688281in}{0.467746in}}%
\pgfpathlineto{\pgfqpoint{1.673448in}{0.456886in}}%
\pgfpathlineto{\pgfqpoint{1.669333in}{0.454135in}}%
\pgfpathlineto{\pgfqpoint{1.657791in}{0.446132in}}%
\pgfpathlineto{\pgfqpoint{1.648746in}{0.440524in}}%
\pgfpathlineto{\pgfqpoint{1.642135in}{0.436173in}}%
\pgfpathlineto{\pgfqpoint{1.626478in}{0.427176in}}%
\pgfpathlineto{\pgfqpoint{1.625940in}{0.426913in}}%
\pgfpathlineto{\pgfqpoint{1.610822in}{0.418820in}}%
\pgfpathlineto{\pgfqpoint{1.598114in}{0.413302in}}%
\pgfpathlineto{\pgfqpoint{1.595165in}{0.411855in}}%
\pgfpathlineto{\pgfqpoint{1.579508in}{0.405930in}}%
\pgfpathlineto{\pgfqpoint{1.563852in}{0.401831in}}%
\pgfpathlineto{\pgfqpoint{1.549123in}{0.399691in}}%
\pgfpathlineto{\pgfqpoint{1.548195in}{0.399531in}}%
\pgfpathlineto{\pgfqpoint{1.532539in}{0.398990in}}%
\pgfpathlineto{\pgfqpoint{1.525770in}{0.399691in}}%
\pgfpathclose%
\pgfpathmoveto{\pgfqpoint{1.105155in}{0.794413in}}%
\pgfpathlineto{\pgfqpoint{1.094155in}{0.799491in}}%
\pgfpathlineto{\pgfqpoint{1.079672in}{0.808024in}}%
\pgfpathlineto{\pgfqpoint{1.078498in}{0.808681in}}%
\pgfpathlineto{\pgfqpoint{1.062842in}{0.819141in}}%
\pgfpathlineto{\pgfqpoint{1.059554in}{0.821635in}}%
\pgfpathlineto{\pgfqpoint{1.047185in}{0.830642in}}%
\pgfpathlineto{\pgfqpoint{1.041409in}{0.835246in}}%
\pgfpathlineto{\pgfqpoint{1.031529in}{0.842914in}}%
\pgfpathlineto{\pgfqpoint{1.024334in}{0.848857in}}%
\pgfpathlineto{\pgfqpoint{1.015872in}{0.855762in}}%
\pgfpathlineto{\pgfqpoint{1.007975in}{0.862468in}}%
\pgfpathlineto{\pgfqpoint{1.000216in}{0.869078in}}%
\pgfpathlineto{\pgfqpoint{0.992162in}{0.876079in}}%
\pgfpathlineto{\pgfqpoint{0.984559in}{0.882825in}}%
\pgfpathlineto{\pgfqpoint{0.976845in}{0.889691in}}%
\pgfpathlineto{\pgfqpoint{0.968902in}{0.897047in}}%
\pgfpathlineto{\pgfqpoint{0.962066in}{0.903302in}}%
\pgfpathlineto{\pgfqpoint{0.953246in}{0.911891in}}%
\pgfpathlineto{\pgfqpoint{0.947950in}{0.916913in}}%
\pgfpathlineto{\pgfqpoint{0.937589in}{0.927665in}}%
\pgfpathlineto{\pgfqpoint{0.934720in}{0.930524in}}%
\pgfpathlineto{\pgfqpoint{0.922688in}{0.944135in}}%
\pgfpathlineto{\pgfqpoint{0.921933in}{0.945155in}}%
\pgfpathlineto{\pgfqpoint{0.912117in}{0.957746in}}%
\pgfpathlineto{\pgfqpoint{0.906276in}{0.967309in}}%
\pgfpathlineto{\pgfqpoint{0.903635in}{0.971357in}}%
\pgfpathlineto{\pgfqpoint{0.897558in}{0.984968in}}%
\pgfpathlineto{\pgfqpoint{0.894454in}{0.998579in}}%
\pgfpathlineto{\pgfqpoint{0.894454in}{1.012191in}}%
\pgfpathlineto{\pgfqpoint{0.897558in}{1.025802in}}%
\pgfpathlineto{\pgfqpoint{0.903635in}{1.039413in}}%
\pgfpathlineto{\pgfqpoint{0.906276in}{1.043461in}}%
\pgfpathlineto{\pgfqpoint{0.912117in}{1.053024in}}%
\pgfpathlineto{\pgfqpoint{0.921933in}{1.065615in}}%
\pgfpathlineto{\pgfqpoint{0.922688in}{1.066635in}}%
\pgfpathlineto{\pgfqpoint{0.934720in}{1.080246in}}%
\pgfpathlineto{\pgfqpoint{0.937589in}{1.083105in}}%
\pgfpathlineto{\pgfqpoint{0.947950in}{1.093857in}}%
\pgfpathlineto{\pgfqpoint{0.953246in}{1.098879in}}%
\pgfpathlineto{\pgfqpoint{0.962066in}{1.107468in}}%
\pgfpathlineto{\pgfqpoint{0.968902in}{1.113723in}}%
\pgfpathlineto{\pgfqpoint{0.976845in}{1.121079in}}%
\pgfpathlineto{\pgfqpoint{0.984559in}{1.127945in}}%
\pgfpathlineto{\pgfqpoint{0.992162in}{1.134691in}}%
\pgfpathlineto{\pgfqpoint{1.000216in}{1.141692in}}%
\pgfpathlineto{\pgfqpoint{1.007975in}{1.148302in}}%
\pgfpathlineto{\pgfqpoint{1.015872in}{1.155008in}}%
\pgfpathlineto{\pgfqpoint{1.024334in}{1.161913in}}%
\pgfpathlineto{\pgfqpoint{1.031529in}{1.167856in}}%
\pgfpathlineto{\pgfqpoint{1.041409in}{1.175524in}}%
\pgfpathlineto{\pgfqpoint{1.047185in}{1.180128in}}%
\pgfpathlineto{\pgfqpoint{1.059554in}{1.189135in}}%
\pgfpathlineto{\pgfqpoint{1.062842in}{1.191629in}}%
\pgfpathlineto{\pgfqpoint{1.078498in}{1.202089in}}%
\pgfpathlineto{\pgfqpoint{1.079672in}{1.202746in}}%
\pgfpathlineto{\pgfqpoint{1.094155in}{1.211279in}}%
\pgfpathlineto{\pgfqpoint{1.105155in}{1.216357in}}%
\pgfpathlineto{\pgfqpoint{1.109812in}{1.218653in}}%
\pgfpathlineto{\pgfqpoint{1.125468in}{1.223936in}}%
\pgfpathlineto{\pgfqpoint{1.141125in}{1.226635in}}%
\pgfpathlineto{\pgfqpoint{1.156781in}{1.226635in}}%
\pgfpathlineto{\pgfqpoint{1.172438in}{1.223936in}}%
\pgfpathlineto{\pgfqpoint{1.188094in}{1.218653in}}%
\pgfpathlineto{\pgfqpoint{1.192751in}{1.216357in}}%
\pgfpathlineto{\pgfqpoint{1.203751in}{1.211279in}}%
\pgfpathlineto{\pgfqpoint{1.218234in}{1.202746in}}%
\pgfpathlineto{\pgfqpoint{1.219407in}{1.202089in}}%
\pgfpathlineto{\pgfqpoint{1.235064in}{1.191629in}}%
\pgfpathlineto{\pgfqpoint{1.238352in}{1.189135in}}%
\pgfpathlineto{\pgfqpoint{1.250721in}{1.180128in}}%
\pgfpathlineto{\pgfqpoint{1.256496in}{1.175524in}}%
\pgfpathlineto{\pgfqpoint{1.266377in}{1.167856in}}%
\pgfpathlineto{\pgfqpoint{1.273572in}{1.161913in}}%
\pgfpathlineto{\pgfqpoint{1.282034in}{1.155008in}}%
\pgfpathlineto{\pgfqpoint{1.289931in}{1.148302in}}%
\pgfpathlineto{\pgfqpoint{1.297690in}{1.141692in}}%
\pgfpathlineto{\pgfqpoint{1.305744in}{1.134691in}}%
\pgfpathlineto{\pgfqpoint{1.313347in}{1.127945in}}%
\pgfpathlineto{\pgfqpoint{1.321061in}{1.121079in}}%
\pgfpathlineto{\pgfqpoint{1.329003in}{1.113723in}}%
\pgfpathlineto{\pgfqpoint{1.335840in}{1.107468in}}%
\pgfpathlineto{\pgfqpoint{1.344660in}{1.098879in}}%
\pgfpathlineto{\pgfqpoint{1.349956in}{1.093857in}}%
\pgfpathlineto{\pgfqpoint{1.360317in}{1.083105in}}%
\pgfpathlineto{\pgfqpoint{1.363186in}{1.080246in}}%
\pgfpathlineto{\pgfqpoint{1.375218in}{1.066635in}}%
\pgfpathlineto{\pgfqpoint{1.375973in}{1.065615in}}%
\pgfpathlineto{\pgfqpoint{1.385789in}{1.053024in}}%
\pgfpathlineto{\pgfqpoint{1.391630in}{1.043461in}}%
\pgfpathlineto{\pgfqpoint{1.394271in}{1.039413in}}%
\pgfpathlineto{\pgfqpoint{1.400348in}{1.025802in}}%
\pgfpathlineto{\pgfqpoint{1.403452in}{1.012191in}}%
\pgfpathlineto{\pgfqpoint{1.403452in}{0.998579in}}%
\pgfpathlineto{\pgfqpoint{1.400348in}{0.984968in}}%
\pgfpathlineto{\pgfqpoint{1.394271in}{0.971357in}}%
\pgfpathlineto{\pgfqpoint{1.391630in}{0.967309in}}%
\pgfpathlineto{\pgfqpoint{1.385789in}{0.957746in}}%
\pgfpathlineto{\pgfqpoint{1.375973in}{0.945155in}}%
\pgfpathlineto{\pgfqpoint{1.375218in}{0.944135in}}%
\pgfpathlineto{\pgfqpoint{1.363186in}{0.930524in}}%
\pgfpathlineto{\pgfqpoint{1.360317in}{0.927665in}}%
\pgfpathlineto{\pgfqpoint{1.349956in}{0.916913in}}%
\pgfpathlineto{\pgfqpoint{1.344660in}{0.911891in}}%
\pgfpathlineto{\pgfqpoint{1.335840in}{0.903302in}}%
\pgfpathlineto{\pgfqpoint{1.329003in}{0.897047in}}%
\pgfpathlineto{\pgfqpoint{1.321061in}{0.889691in}}%
\pgfpathlineto{\pgfqpoint{1.313347in}{0.882825in}}%
\pgfpathlineto{\pgfqpoint{1.305744in}{0.876079in}}%
\pgfpathlineto{\pgfqpoint{1.297690in}{0.869078in}}%
\pgfpathlineto{\pgfqpoint{1.289931in}{0.862468in}}%
\pgfpathlineto{\pgfqpoint{1.282034in}{0.855762in}}%
\pgfpathlineto{\pgfqpoint{1.273572in}{0.848857in}}%
\pgfpathlineto{\pgfqpoint{1.266377in}{0.842914in}}%
\pgfpathlineto{\pgfqpoint{1.256496in}{0.835246in}}%
\pgfpathlineto{\pgfqpoint{1.250721in}{0.830642in}}%
\pgfpathlineto{\pgfqpoint{1.238352in}{0.821635in}}%
\pgfpathlineto{\pgfqpoint{1.235064in}{0.819141in}}%
\pgfpathlineto{\pgfqpoint{1.219407in}{0.808681in}}%
\pgfpathlineto{\pgfqpoint{1.218234in}{0.808024in}}%
\pgfpathlineto{\pgfqpoint{1.203751in}{0.799491in}}%
\pgfpathlineto{\pgfqpoint{1.192751in}{0.794413in}}%
\pgfpathlineto{\pgfqpoint{1.188094in}{0.792117in}}%
\pgfpathlineto{\pgfqpoint{1.172438in}{0.786834in}}%
\pgfpathlineto{\pgfqpoint{1.156781in}{0.784135in}}%
\pgfpathlineto{\pgfqpoint{1.141125in}{0.784135in}}%
\pgfpathlineto{\pgfqpoint{1.125468in}{0.786834in}}%
\pgfpathlineto{\pgfqpoint{1.109812in}{0.792117in}}%
\pgfpathlineto{\pgfqpoint{1.105155in}{0.794413in}}%
\pgfpathclose%
\pgfpathmoveto{\pgfqpoint{0.717576in}{1.080246in}}%
\pgfpathlineto{\pgfqpoint{0.702741in}{1.085477in}}%
\pgfpathlineto{\pgfqpoint{0.687084in}{1.092700in}}%
\pgfpathlineto{\pgfqpoint{0.685029in}{1.093857in}}%
\pgfpathlineto{\pgfqpoint{0.671428in}{1.100763in}}%
\pgfpathlineto{\pgfqpoint{0.660256in}{1.107468in}}%
\pgfpathlineto{\pgfqpoint{0.655771in}{1.109969in}}%
\pgfpathlineto{\pgfqpoint{0.640115in}{1.119958in}}%
\pgfpathlineto{\pgfqpoint{0.638534in}{1.121079in}}%
\pgfpathlineto{\pgfqpoint{0.624458in}{1.130611in}}%
\pgfpathlineto{\pgfqpoint{0.618955in}{1.134691in}}%
\pgfpathlineto{\pgfqpoint{0.608801in}{1.142032in}}%
\pgfpathlineto{\pgfqpoint{0.600734in}{1.148302in}}%
\pgfpathlineto{\pgfqpoint{0.593145in}{1.154166in}}%
\pgfpathlineto{\pgfqpoint{0.583655in}{1.161913in}}%
\pgfpathlineto{\pgfqpoint{0.577488in}{1.167010in}}%
\pgfpathlineto{\pgfqpoint{0.567570in}{1.175524in}}%
\pgfpathlineto{\pgfqpoint{0.561832in}{1.180597in}}%
\pgfpathlineto{\pgfqpoint{0.552376in}{1.189135in}}%
\pgfpathlineto{\pgfqpoint{0.546175in}{1.195000in}}%
\pgfpathlineto{\pgfqpoint{0.538013in}{1.202746in}}%
\pgfpathlineto{\pgfqpoint{0.530519in}{1.210325in}}%
\pgfpathlineto{\pgfqpoint{0.524461in}{1.216357in}}%
\pgfpathlineto{\pgfqpoint{0.514862in}{1.226727in}}%
\pgfpathlineto{\pgfqpoint{0.511753in}{1.229968in}}%
\pgfpathlineto{\pgfqpoint{0.499932in}{1.243579in}}%
\pgfpathlineto{\pgfqpoint{0.499205in}{1.244521in}}%
\pgfpathlineto{\pgfqpoint{0.488820in}{1.257191in}}%
\pgfpathlineto{\pgfqpoint{0.483549in}{1.264671in}}%
\pgfpathlineto{\pgfqpoint{0.478833in}{1.270802in}}%
\pgfpathlineto{\pgfqpoint{0.470108in}{1.284413in}}%
\pgfpathlineto{\pgfqpoint{0.467892in}{1.288764in}}%
\pgfpathlineto{\pgfqpoint{0.462558in}{1.298024in}}%
\pgfpathlineto{\pgfqpoint{0.456793in}{1.311635in}}%
\pgfpathlineto{\pgfqpoint{0.453128in}{1.325246in}}%
\pgfpathlineto{\pgfqpoint{0.452236in}{1.332973in}}%
\pgfpathlineto{\pgfqpoint{0.451429in}{1.338857in}}%
\pgfpathlineto{\pgfqpoint{0.452052in}{1.352468in}}%
\pgfpathlineto{\pgfqpoint{0.452236in}{1.353275in}}%
\pgfpathlineto{\pgfqpoint{0.454698in}{1.366079in}}%
\pgfpathlineto{\pgfqpoint{0.459413in}{1.379691in}}%
\pgfpathlineto{\pgfqpoint{0.466229in}{1.393302in}}%
\pgfpathlineto{\pgfqpoint{0.467892in}{1.395865in}}%
\pgfpathlineto{\pgfqpoint{0.474240in}{1.406913in}}%
\pgfpathlineto{\pgfqpoint{0.483549in}{1.420056in}}%
\pgfpathlineto{\pgfqpoint{0.483852in}{1.420524in}}%
\pgfpathlineto{\pgfqpoint{0.494201in}{1.434135in}}%
\pgfpathlineto{\pgfqpoint{0.499205in}{1.439882in}}%
\pgfpathlineto{\pgfqpoint{0.505657in}{1.447746in}}%
\pgfpathlineto{\pgfqpoint{0.514862in}{1.457780in}}%
\pgfpathlineto{\pgfqpoint{0.518026in}{1.461357in}}%
\pgfpathlineto{\pgfqpoint{0.530519in}{1.474252in}}%
\pgfpathlineto{\pgfqpoint{0.531201in}{1.474968in}}%
\pgfpathlineto{\pgfqpoint{0.545124in}{1.488579in}}%
\pgfpathlineto{\pgfqpoint{0.546175in}{1.489554in}}%
\pgfpathlineto{\pgfqpoint{0.559869in}{1.502191in}}%
\pgfpathlineto{\pgfqpoint{0.561832in}{1.503934in}}%
\pgfpathlineto{\pgfqpoint{0.575483in}{1.515802in}}%
\pgfpathlineto{\pgfqpoint{0.577488in}{1.517508in}}%
\pgfpathlineto{\pgfqpoint{0.592024in}{1.529413in}}%
\pgfpathlineto{\pgfqpoint{0.593145in}{1.530327in}}%
\pgfpathlineto{\pgfqpoint{0.608801in}{1.542430in}}%
\pgfpathlineto{\pgfqpoint{0.609625in}{1.543024in}}%
\pgfpathlineto{\pgfqpoint{0.624458in}{1.553884in}}%
\pgfpathlineto{\pgfqpoint{0.628573in}{1.556635in}}%
\pgfpathlineto{\pgfqpoint{0.640115in}{1.564638in}}%
\pgfpathlineto{\pgfqpoint{0.649160in}{1.570246in}}%
\pgfpathlineto{\pgfqpoint{0.655771in}{1.574597in}}%
\pgfpathlineto{\pgfqpoint{0.671428in}{1.583594in}}%
\pgfpathlineto{\pgfqpoint{0.671966in}{1.583857in}}%
\pgfpathlineto{\pgfqpoint{0.687084in}{1.591950in}}%
\pgfpathlineto{\pgfqpoint{0.699792in}{1.597468in}}%
\pgfpathlineto{\pgfqpoint{0.702741in}{1.598915in}}%
\pgfpathlineto{\pgfqpoint{0.718397in}{1.604840in}}%
\pgfpathlineto{\pgfqpoint{0.734054in}{1.608939in}}%
\pgfpathlineto{\pgfqpoint{0.748782in}{1.611079in}}%
\pgfpathlineto{\pgfqpoint{0.749710in}{1.611239in}}%
\pgfpathlineto{\pgfqpoint{0.765367in}{1.611780in}}%
\pgfpathlineto{\pgfqpoint{0.772136in}{1.611079in}}%
\pgfpathlineto{\pgfqpoint{0.781024in}{1.610304in}}%
\pgfpathlineto{\pgfqpoint{0.796680in}{1.607117in}}%
\pgfpathlineto{\pgfqpoint{0.812337in}{1.602106in}}%
\pgfpathlineto{\pgfqpoint{0.822989in}{1.597468in}}%
\pgfpathlineto{\pgfqpoint{0.827993in}{1.595542in}}%
\pgfpathlineto{\pgfqpoint{0.843650in}{1.587957in}}%
\pgfpathlineto{\pgfqpoint{0.850702in}{1.583857in}}%
\pgfpathlineto{\pgfqpoint{0.859306in}{1.579274in}}%
\pgfpathlineto{\pgfqpoint{0.873880in}{1.570246in}}%
\pgfpathlineto{\pgfqpoint{0.874963in}{1.569615in}}%
\pgfpathlineto{\pgfqpoint{0.890620in}{1.559338in}}%
\pgfpathlineto{\pgfqpoint{0.894348in}{1.556635in}}%
\pgfpathlineto{\pgfqpoint{0.906276in}{1.548290in}}%
\pgfpathlineto{\pgfqpoint{0.913215in}{1.543024in}}%
\pgfpathlineto{\pgfqpoint{0.921933in}{1.536509in}}%
\pgfpathlineto{\pgfqpoint{0.930843in}{1.529413in}}%
\pgfpathlineto{\pgfqpoint{0.937589in}{1.524022in}}%
\pgfpathlineto{\pgfqpoint{0.947410in}{1.515802in}}%
\pgfpathlineto{\pgfqpoint{0.953246in}{1.510813in}}%
\pgfpathlineto{\pgfqpoint{0.963040in}{1.502191in}}%
\pgfpathlineto{\pgfqpoint{0.968902in}{1.496829in}}%
\pgfpathlineto{\pgfqpoint{0.977813in}{1.488579in}}%
\pgfpathlineto{\pgfqpoint{0.984559in}{1.481982in}}%
\pgfpathlineto{\pgfqpoint{0.991771in}{1.474968in}}%
\pgfpathlineto{\pgfqpoint{1.000216in}{1.466141in}}%
\pgfpathlineto{\pgfqpoint{1.004908in}{1.461357in}}%
\pgfpathlineto{\pgfqpoint{1.015872in}{1.449120in}}%
\pgfpathlineto{\pgfqpoint{1.017162in}{1.447746in}}%
\pgfpathlineto{\pgfqpoint{1.028652in}{1.434135in}}%
\pgfpathlineto{\pgfqpoint{1.031529in}{1.430236in}}%
\pgfpathlineto{\pgfqpoint{1.039241in}{1.420524in}}%
\pgfpathlineto{\pgfqpoint{1.047185in}{1.408700in}}%
\pgfpathlineto{\pgfqpoint{1.048516in}{1.406913in}}%
\pgfpathlineto{\pgfqpoint{1.056824in}{1.393302in}}%
\pgfpathlineto{\pgfqpoint{1.062842in}{1.380405in}}%
\pgfpathlineto{\pgfqpoint{1.063226in}{1.379691in}}%
\pgfpathlineto{\pgfqpoint{1.068338in}{1.366079in}}%
\pgfpathlineto{\pgfqpoint{1.071176in}{1.352468in}}%
\pgfpathlineto{\pgfqpoint{1.071743in}{1.338857in}}%
\pgfpathlineto{\pgfqpoint{1.070041in}{1.325246in}}%
\pgfpathlineto{\pgfqpoint{1.066067in}{1.311635in}}%
\pgfpathlineto{\pgfqpoint{1.062842in}{1.304586in}}%
\pgfpathlineto{\pgfqpoint{1.060243in}{1.298024in}}%
\pgfpathlineto{\pgfqpoint{1.052916in}{1.284413in}}%
\pgfpathlineto{\pgfqpoint{1.047185in}{1.275973in}}%
\pgfpathlineto{\pgfqpoint{1.044022in}{1.270802in}}%
\pgfpathlineto{\pgfqpoint{1.034026in}{1.257191in}}%
\pgfpathlineto{\pgfqpoint{1.031529in}{1.254249in}}%
\pgfpathlineto{\pgfqpoint{1.023102in}{1.243579in}}%
\pgfpathlineto{\pgfqpoint{1.015872in}{1.235488in}}%
\pgfpathlineto{\pgfqpoint{1.011170in}{1.229968in}}%
\pgfpathlineto{\pgfqpoint{1.000216in}{1.218346in}}%
\pgfpathlineto{\pgfqpoint{0.998389in}{1.216357in}}%
\pgfpathlineto{\pgfqpoint{0.984836in}{1.202746in}}%
\pgfpathlineto{\pgfqpoint{0.984559in}{1.202484in}}%
\pgfpathlineto{\pgfqpoint{0.970516in}{1.189135in}}%
\pgfpathlineto{\pgfqpoint{0.968902in}{1.187671in}}%
\pgfpathlineto{\pgfqpoint{0.955341in}{1.175524in}}%
\pgfpathlineto{\pgfqpoint{0.953246in}{1.173702in}}%
\pgfpathlineto{\pgfqpoint{0.939273in}{1.161913in}}%
\pgfpathlineto{\pgfqpoint{0.937589in}{1.160510in}}%
\pgfpathlineto{\pgfqpoint{0.922234in}{1.148302in}}%
\pgfpathlineto{\pgfqpoint{0.921933in}{1.148061in}}%
\pgfpathlineto{\pgfqpoint{0.906276in}{1.136278in}}%
\pgfpathlineto{\pgfqpoint{0.903988in}{1.134691in}}%
\pgfpathlineto{\pgfqpoint{0.890620in}{1.125167in}}%
\pgfpathlineto{\pgfqpoint{0.884270in}{1.121079in}}%
\pgfpathlineto{\pgfqpoint{0.874963in}{1.114794in}}%
\pgfpathlineto{\pgfqpoint{0.862690in}{1.107468in}}%
\pgfpathlineto{\pgfqpoint{0.859306in}{1.105297in}}%
\pgfpathlineto{\pgfqpoint{0.843650in}{1.096607in}}%
\pgfpathlineto{\pgfqpoint{0.837702in}{1.093857in}}%
\pgfpathlineto{\pgfqpoint{0.827993in}{1.088876in}}%
\pgfpathlineto{\pgfqpoint{0.812337in}{1.082505in}}%
\pgfpathlineto{\pgfqpoint{0.804789in}{1.080246in}}%
\pgfpathlineto{\pgfqpoint{0.796680in}{1.077443in}}%
\pgfpathlineto{\pgfqpoint{0.781024in}{1.073987in}}%
\pgfpathlineto{\pgfqpoint{0.765367in}{1.072507in}}%
\pgfpathlineto{\pgfqpoint{0.749710in}{1.073001in}}%
\pgfpathlineto{\pgfqpoint{0.734054in}{1.075468in}}%
\pgfpathlineto{\pgfqpoint{0.718397in}{1.079912in}}%
\pgfpathlineto{\pgfqpoint{0.717576in}{1.080246in}}%
\pgfpathclose%
\pgfpathmoveto{\pgfqpoint{1.493117in}{1.080246in}}%
\pgfpathlineto{\pgfqpoint{1.485569in}{1.082505in}}%
\pgfpathlineto{\pgfqpoint{1.469913in}{1.088876in}}%
\pgfpathlineto{\pgfqpoint{1.460204in}{1.093857in}}%
\pgfpathlineto{\pgfqpoint{1.454256in}{1.096607in}}%
\pgfpathlineto{\pgfqpoint{1.438599in}{1.105297in}}%
\pgfpathlineto{\pgfqpoint{1.435216in}{1.107468in}}%
\pgfpathlineto{\pgfqpoint{1.422943in}{1.114794in}}%
\pgfpathlineto{\pgfqpoint{1.413635in}{1.121079in}}%
\pgfpathlineto{\pgfqpoint{1.407286in}{1.125167in}}%
\pgfpathlineto{\pgfqpoint{1.393917in}{1.134691in}}%
\pgfpathlineto{\pgfqpoint{1.391630in}{1.136278in}}%
\pgfpathlineto{\pgfqpoint{1.375973in}{1.148061in}}%
\pgfpathlineto{\pgfqpoint{1.375671in}{1.148302in}}%
\pgfpathlineto{\pgfqpoint{1.360317in}{1.160510in}}%
\pgfpathlineto{\pgfqpoint{1.358632in}{1.161913in}}%
\pgfpathlineto{\pgfqpoint{1.344660in}{1.173702in}}%
\pgfpathlineto{\pgfqpoint{1.342564in}{1.175524in}}%
\pgfpathlineto{\pgfqpoint{1.329003in}{1.187671in}}%
\pgfpathlineto{\pgfqpoint{1.327390in}{1.189135in}}%
\pgfpathlineto{\pgfqpoint{1.313347in}{1.202484in}}%
\pgfpathlineto{\pgfqpoint{1.313070in}{1.202746in}}%
\pgfpathlineto{\pgfqpoint{1.299517in}{1.216357in}}%
\pgfpathlineto{\pgfqpoint{1.297690in}{1.218346in}}%
\pgfpathlineto{\pgfqpoint{1.286736in}{1.229968in}}%
\pgfpathlineto{\pgfqpoint{1.282034in}{1.235488in}}%
\pgfpathlineto{\pgfqpoint{1.274804in}{1.243579in}}%
\pgfpathlineto{\pgfqpoint{1.266377in}{1.254249in}}%
\pgfpathlineto{\pgfqpoint{1.263880in}{1.257191in}}%
\pgfpathlineto{\pgfqpoint{1.253883in}{1.270802in}}%
\pgfpathlineto{\pgfqpoint{1.250721in}{1.275973in}}%
\pgfpathlineto{\pgfqpoint{1.244990in}{1.284413in}}%
\pgfpathlineto{\pgfqpoint{1.237663in}{1.298024in}}%
\pgfpathlineto{\pgfqpoint{1.235064in}{1.304586in}}%
\pgfpathlineto{\pgfqpoint{1.231839in}{1.311635in}}%
\pgfpathlineto{\pgfqpoint{1.227865in}{1.325246in}}%
\pgfpathlineto{\pgfqpoint{1.226162in}{1.338857in}}%
\pgfpathlineto{\pgfqpoint{1.226730in}{1.352468in}}%
\pgfpathlineto{\pgfqpoint{1.229568in}{1.366079in}}%
\pgfpathlineto{\pgfqpoint{1.234680in}{1.379691in}}%
\pgfpathlineto{\pgfqpoint{1.235064in}{1.380405in}}%
\pgfpathlineto{\pgfqpoint{1.241081in}{1.393302in}}%
\pgfpathlineto{\pgfqpoint{1.249390in}{1.406913in}}%
\pgfpathlineto{\pgfqpoint{1.250721in}{1.408700in}}%
\pgfpathlineto{\pgfqpoint{1.258664in}{1.420524in}}%
\pgfpathlineto{\pgfqpoint{1.266377in}{1.430236in}}%
\pgfpathlineto{\pgfqpoint{1.269254in}{1.434135in}}%
\pgfpathlineto{\pgfqpoint{1.280744in}{1.447746in}}%
\pgfpathlineto{\pgfqpoint{1.282034in}{1.449120in}}%
\pgfpathlineto{\pgfqpoint{1.292998in}{1.461357in}}%
\pgfpathlineto{\pgfqpoint{1.297690in}{1.466141in}}%
\pgfpathlineto{\pgfqpoint{1.306135in}{1.474968in}}%
\pgfpathlineto{\pgfqpoint{1.313347in}{1.481982in}}%
\pgfpathlineto{\pgfqpoint{1.320093in}{1.488579in}}%
\pgfpathlineto{\pgfqpoint{1.329003in}{1.496829in}}%
\pgfpathlineto{\pgfqpoint{1.334866in}{1.502191in}}%
\pgfpathlineto{\pgfqpoint{1.344660in}{1.510813in}}%
\pgfpathlineto{\pgfqpoint{1.350496in}{1.515802in}}%
\pgfpathlineto{\pgfqpoint{1.360317in}{1.524022in}}%
\pgfpathlineto{\pgfqpoint{1.367063in}{1.529413in}}%
\pgfpathlineto{\pgfqpoint{1.375973in}{1.536509in}}%
\pgfpathlineto{\pgfqpoint{1.384690in}{1.543024in}}%
\pgfpathlineto{\pgfqpoint{1.391630in}{1.548290in}}%
\pgfpathlineto{\pgfqpoint{1.403558in}{1.556635in}}%
\pgfpathlineto{\pgfqpoint{1.407286in}{1.559338in}}%
\pgfpathlineto{\pgfqpoint{1.422943in}{1.569615in}}%
\pgfpathlineto{\pgfqpoint{1.424026in}{1.570246in}}%
\pgfpathlineto{\pgfqpoint{1.438599in}{1.579274in}}%
\pgfpathlineto{\pgfqpoint{1.447204in}{1.583857in}}%
\pgfpathlineto{\pgfqpoint{1.454256in}{1.587957in}}%
\pgfpathlineto{\pgfqpoint{1.469913in}{1.595542in}}%
\pgfpathlineto{\pgfqpoint{1.474917in}{1.597468in}}%
\pgfpathlineto{\pgfqpoint{1.485569in}{1.602106in}}%
\pgfpathlineto{\pgfqpoint{1.501226in}{1.607117in}}%
\pgfpathlineto{\pgfqpoint{1.516882in}{1.610304in}}%
\pgfpathlineto{\pgfqpoint{1.525770in}{1.611079in}}%
\pgfpathlineto{\pgfqpoint{1.532539in}{1.611780in}}%
\pgfpathlineto{\pgfqpoint{1.548195in}{1.611239in}}%
\pgfpathlineto{\pgfqpoint{1.549123in}{1.611079in}}%
\pgfpathlineto{\pgfqpoint{1.563852in}{1.608939in}}%
\pgfpathlineto{\pgfqpoint{1.579508in}{1.604840in}}%
\pgfpathlineto{\pgfqpoint{1.595165in}{1.598915in}}%
\pgfpathlineto{\pgfqpoint{1.598114in}{1.597468in}}%
\pgfpathlineto{\pgfqpoint{1.610822in}{1.591950in}}%
\pgfpathlineto{\pgfqpoint{1.625940in}{1.583857in}}%
\pgfpathlineto{\pgfqpoint{1.626478in}{1.583594in}}%
\pgfpathlineto{\pgfqpoint{1.642135in}{1.574597in}}%
\pgfpathlineto{\pgfqpoint{1.648746in}{1.570246in}}%
\pgfpathlineto{\pgfqpoint{1.657791in}{1.564638in}}%
\pgfpathlineto{\pgfqpoint{1.669333in}{1.556635in}}%
\pgfpathlineto{\pgfqpoint{1.673448in}{1.553884in}}%
\pgfpathlineto{\pgfqpoint{1.688281in}{1.543024in}}%
\pgfpathlineto{\pgfqpoint{1.689104in}{1.542430in}}%
\pgfpathlineto{\pgfqpoint{1.704761in}{1.530327in}}%
\pgfpathlineto{\pgfqpoint{1.705882in}{1.529413in}}%
\pgfpathlineto{\pgfqpoint{1.720418in}{1.517508in}}%
\pgfpathlineto{\pgfqpoint{1.722423in}{1.515802in}}%
\pgfpathlineto{\pgfqpoint{1.736074in}{1.503934in}}%
\pgfpathlineto{\pgfqpoint{1.738037in}{1.502191in}}%
\pgfpathlineto{\pgfqpoint{1.751731in}{1.489554in}}%
\pgfpathlineto{\pgfqpoint{1.752782in}{1.488579in}}%
\pgfpathlineto{\pgfqpoint{1.766704in}{1.474968in}}%
\pgfpathlineto{\pgfqpoint{1.767387in}{1.474252in}}%
\pgfpathlineto{\pgfqpoint{1.779880in}{1.461357in}}%
\pgfpathlineto{\pgfqpoint{1.783044in}{1.457780in}}%
\pgfpathlineto{\pgfqpoint{1.792249in}{1.447746in}}%
\pgfpathlineto{\pgfqpoint{1.798700in}{1.439882in}}%
\pgfpathlineto{\pgfqpoint{1.803705in}{1.434135in}}%
\pgfpathlineto{\pgfqpoint{1.814054in}{1.420524in}}%
\pgfpathlineto{\pgfqpoint{1.814357in}{1.420056in}}%
\pgfpathlineto{\pgfqpoint{1.823666in}{1.406913in}}%
\pgfpathlineto{\pgfqpoint{1.830014in}{1.395865in}}%
\pgfpathlineto{\pgfqpoint{1.831677in}{1.393302in}}%
\pgfpathlineto{\pgfqpoint{1.838493in}{1.379691in}}%
\pgfpathlineto{\pgfqpoint{1.843207in}{1.366079in}}%
\pgfpathlineto{\pgfqpoint{1.845670in}{1.353275in}}%
\pgfpathlineto{\pgfqpoint{1.845854in}{1.352468in}}%
\pgfpathlineto{\pgfqpoint{1.846476in}{1.338857in}}%
\pgfpathlineto{\pgfqpoint{1.845670in}{1.332973in}}%
\pgfpathlineto{\pgfqpoint{1.844778in}{1.325246in}}%
\pgfpathlineto{\pgfqpoint{1.841113in}{1.311635in}}%
\pgfpathlineto{\pgfqpoint{1.835348in}{1.298024in}}%
\pgfpathlineto{\pgfqpoint{1.830014in}{1.288764in}}%
\pgfpathlineto{\pgfqpoint{1.827797in}{1.284413in}}%
\pgfpathlineto{\pgfqpoint{1.819073in}{1.270802in}}%
\pgfpathlineto{\pgfqpoint{1.814357in}{1.264671in}}%
\pgfpathlineto{\pgfqpoint{1.809085in}{1.257191in}}%
\pgfpathlineto{\pgfqpoint{1.798700in}{1.244521in}}%
\pgfpathlineto{\pgfqpoint{1.797974in}{1.243579in}}%
\pgfpathlineto{\pgfqpoint{1.786152in}{1.229968in}}%
\pgfpathlineto{\pgfqpoint{1.783044in}{1.226727in}}%
\pgfpathlineto{\pgfqpoint{1.773444in}{1.216357in}}%
\pgfpathlineto{\pgfqpoint{1.767387in}{1.210325in}}%
\pgfpathlineto{\pgfqpoint{1.759893in}{1.202746in}}%
\pgfpathlineto{\pgfqpoint{1.751731in}{1.195000in}}%
\pgfpathlineto{\pgfqpoint{1.745530in}{1.189135in}}%
\pgfpathlineto{\pgfqpoint{1.736074in}{1.180597in}}%
\pgfpathlineto{\pgfqpoint{1.730336in}{1.175524in}}%
\pgfpathlineto{\pgfqpoint{1.720418in}{1.167010in}}%
\pgfpathlineto{\pgfqpoint{1.714250in}{1.161913in}}%
\pgfpathlineto{\pgfqpoint{1.704761in}{1.154166in}}%
\pgfpathlineto{\pgfqpoint{1.697172in}{1.148302in}}%
\pgfpathlineto{\pgfqpoint{1.689104in}{1.142032in}}%
\pgfpathlineto{\pgfqpoint{1.678950in}{1.134691in}}%
\pgfpathlineto{\pgfqpoint{1.673448in}{1.130611in}}%
\pgfpathlineto{\pgfqpoint{1.659372in}{1.121079in}}%
\pgfpathlineto{\pgfqpoint{1.657791in}{1.119958in}}%
\pgfpathlineto{\pgfqpoint{1.642135in}{1.109969in}}%
\pgfpathlineto{\pgfqpoint{1.637649in}{1.107468in}}%
\pgfpathlineto{\pgfqpoint{1.626478in}{1.100763in}}%
\pgfpathlineto{\pgfqpoint{1.612877in}{1.093857in}}%
\pgfpathlineto{\pgfqpoint{1.610822in}{1.092700in}}%
\pgfpathlineto{\pgfqpoint{1.595165in}{1.085477in}}%
\pgfpathlineto{\pgfqpoint{1.580330in}{1.080246in}}%
\pgfpathlineto{\pgfqpoint{1.579508in}{1.079912in}}%
\pgfpathlineto{\pgfqpoint{1.563852in}{1.075468in}}%
\pgfpathlineto{\pgfqpoint{1.548195in}{1.073001in}}%
\pgfpathlineto{\pgfqpoint{1.532539in}{1.072507in}}%
\pgfpathlineto{\pgfqpoint{1.516882in}{1.073987in}}%
\pgfpathlineto{\pgfqpoint{1.501226in}{1.077443in}}%
\pgfpathlineto{\pgfqpoint{1.493117in}{1.080246in}}%
\pgfpathclose%
\pgfusepath{fill}%
\end{pgfscope}%
\begin{pgfscope}%
\pgfpathrectangle{\pgfqpoint{0.373953in}{0.331635in}}{\pgfqpoint{1.550000in}{1.347500in}}%
\pgfusepath{clip}%
\pgfsetbuttcap%
\pgfsetroundjoin%
\definecolor{currentfill}{rgb}{0.099423,0.404752,0.679815}%
\pgfsetfillcolor{currentfill}%
\pgfsetlinewidth{0.000000pt}%
\definecolor{currentstroke}{rgb}{0.000000,0.000000,0.000000}%
\pgfsetstrokecolor{currentstroke}%
\pgfsetdash{}{0pt}%
\pgfpathmoveto{\pgfqpoint{0.530519in}{0.331635in}}%
\pgfpathlineto{\pgfqpoint{0.546175in}{0.331635in}}%
\pgfpathlineto{\pgfqpoint{0.561832in}{0.331635in}}%
\pgfpathlineto{\pgfqpoint{0.577488in}{0.331635in}}%
\pgfpathlineto{\pgfqpoint{0.593145in}{0.331635in}}%
\pgfpathlineto{\pgfqpoint{0.608801in}{0.331635in}}%
\pgfpathlineto{\pgfqpoint{0.624458in}{0.331635in}}%
\pgfpathlineto{\pgfqpoint{0.628868in}{0.331635in}}%
\pgfpathlineto{\pgfqpoint{0.627253in}{0.345246in}}%
\pgfpathlineto{\pgfqpoint{0.624458in}{0.353195in}}%
\pgfpathlineto{\pgfqpoint{0.622600in}{0.358857in}}%
\pgfpathlineto{\pgfqpoint{0.615336in}{0.372468in}}%
\pgfpathlineto{\pgfqpoint{0.608801in}{0.381546in}}%
\pgfpathlineto{\pgfqpoint{0.605723in}{0.386079in}}%
\pgfpathlineto{\pgfqpoint{0.594368in}{0.399691in}}%
\pgfpathlineto{\pgfqpoint{0.593145in}{0.400949in}}%
\pgfpathlineto{\pgfqpoint{0.581699in}{0.413302in}}%
\pgfpathlineto{\pgfqpoint{0.577488in}{0.417386in}}%
\pgfpathlineto{\pgfqpoint{0.567992in}{0.426913in}}%
\pgfpathlineto{\pgfqpoint{0.561832in}{0.432643in}}%
\pgfpathlineto{\pgfqpoint{0.553523in}{0.440524in}}%
\pgfpathlineto{\pgfqpoint{0.546175in}{0.447150in}}%
\pgfpathlineto{\pgfqpoint{0.538465in}{0.454135in}}%
\pgfpathlineto{\pgfqpoint{0.530519in}{0.461123in}}%
\pgfpathlineto{\pgfqpoint{0.522900in}{0.467746in}}%
\pgfpathlineto{\pgfqpoint{0.514862in}{0.474654in}}%
\pgfpathlineto{\pgfqpoint{0.506827in}{0.481357in}}%
\pgfpathlineto{\pgfqpoint{0.499205in}{0.487745in}}%
\pgfpathlineto{\pgfqpoint{0.490140in}{0.494968in}}%
\pgfpathlineto{\pgfqpoint{0.483549in}{0.500324in}}%
\pgfpathlineto{\pgfqpoint{0.472590in}{0.508579in}}%
\pgfpathlineto{\pgfqpoint{0.467892in}{0.512240in}}%
\pgfpathlineto{\pgfqpoint{0.453684in}{0.522191in}}%
\pgfpathlineto{\pgfqpoint{0.452236in}{0.523254in}}%
\pgfpathlineto{\pgfqpoint{0.436579in}{0.533126in}}%
\pgfpathlineto{\pgfqpoint{0.431364in}{0.535802in}}%
\pgfpathlineto{\pgfqpoint{0.420923in}{0.541482in}}%
\pgfpathlineto{\pgfqpoint{0.405266in}{0.547797in}}%
\pgfpathlineto{\pgfqpoint{0.398752in}{0.549413in}}%
\pgfpathlineto{\pgfqpoint{0.389609in}{0.551843in}}%
\pgfpathlineto{\pgfqpoint{0.373953in}{0.553246in}}%
\pgfpathlineto{\pgfqpoint{0.373953in}{0.549413in}}%
\pgfpathlineto{\pgfqpoint{0.373953in}{0.535802in}}%
\pgfpathlineto{\pgfqpoint{0.373953in}{0.522191in}}%
\pgfpathlineto{\pgfqpoint{0.373953in}{0.508579in}}%
\pgfpathlineto{\pgfqpoint{0.373953in}{0.494968in}}%
\pgfpathlineto{\pgfqpoint{0.373953in}{0.481357in}}%
\pgfpathlineto{\pgfqpoint{0.373953in}{0.467746in}}%
\pgfpathlineto{\pgfqpoint{0.373953in}{0.461560in}}%
\pgfpathlineto{\pgfqpoint{0.389609in}{0.460471in}}%
\pgfpathlineto{\pgfqpoint{0.405266in}{0.457249in}}%
\pgfpathlineto{\pgfqpoint{0.414570in}{0.454135in}}%
\pgfpathlineto{\pgfqpoint{0.420923in}{0.451905in}}%
\pgfpathlineto{\pgfqpoint{0.436579in}{0.444479in}}%
\pgfpathlineto{\pgfqpoint{0.443364in}{0.440524in}}%
\pgfpathlineto{\pgfqpoint{0.452236in}{0.435027in}}%
\pgfpathlineto{\pgfqpoint{0.463541in}{0.426913in}}%
\pgfpathlineto{\pgfqpoint{0.467892in}{0.423523in}}%
\pgfpathlineto{\pgfqpoint{0.479649in}{0.413302in}}%
\pgfpathlineto{\pgfqpoint{0.483549in}{0.409519in}}%
\pgfpathlineto{\pgfqpoint{0.492883in}{0.399691in}}%
\pgfpathlineto{\pgfqpoint{0.499205in}{0.391977in}}%
\pgfpathlineto{\pgfqpoint{0.503755in}{0.386079in}}%
\pgfpathlineto{\pgfqpoint{0.512297in}{0.372468in}}%
\pgfpathlineto{\pgfqpoint{0.514862in}{0.366946in}}%
\pgfpathlineto{\pgfqpoint{0.518444in}{0.358857in}}%
\pgfpathlineto{\pgfqpoint{0.522150in}{0.345246in}}%
\pgfpathlineto{\pgfqpoint{0.523403in}{0.331635in}}%
\pgfpathlineto{\pgfqpoint{0.530519in}{0.331635in}}%
\pgfpathclose%
\pgfpathmoveto{\pgfqpoint{0.906276in}{0.331635in}}%
\pgfpathlineto{\pgfqpoint{0.921933in}{0.331635in}}%
\pgfpathlineto{\pgfqpoint{0.937589in}{0.331635in}}%
\pgfpathlineto{\pgfqpoint{0.953246in}{0.331635in}}%
\pgfpathlineto{\pgfqpoint{0.968902in}{0.331635in}}%
\pgfpathlineto{\pgfqpoint{0.984559in}{0.331635in}}%
\pgfpathlineto{\pgfqpoint{0.999425in}{0.331635in}}%
\pgfpathlineto{\pgfqpoint{1.000216in}{0.340413in}}%
\pgfpathlineto{\pgfqpoint{1.000672in}{0.345246in}}%
\pgfpathlineto{\pgfqpoint{1.004478in}{0.358857in}}%
\pgfpathlineto{\pgfqpoint{1.010644in}{0.372468in}}%
\pgfpathlineto{\pgfqpoint{1.015872in}{0.381021in}}%
\pgfpathlineto{\pgfqpoint{1.019134in}{0.386079in}}%
\pgfpathlineto{\pgfqpoint{1.029915in}{0.399691in}}%
\pgfpathlineto{\pgfqpoint{1.031529in}{0.401440in}}%
\pgfpathlineto{\pgfqpoint{1.043250in}{0.413302in}}%
\pgfpathlineto{\pgfqpoint{1.047185in}{0.416875in}}%
\pgfpathlineto{\pgfqpoint{1.059327in}{0.426913in}}%
\pgfpathlineto{\pgfqpoint{1.062842in}{0.429593in}}%
\pgfpathlineto{\pgfqpoint{1.078498in}{0.440113in}}%
\pgfpathlineto{\pgfqpoint{1.079228in}{0.440524in}}%
\pgfpathlineto{\pgfqpoint{1.094155in}{0.448412in}}%
\pgfpathlineto{\pgfqpoint{1.107937in}{0.454135in}}%
\pgfpathlineto{\pgfqpoint{1.109812in}{0.454879in}}%
\pgfpathlineto{\pgfqpoint{1.125468in}{0.459120in}}%
\pgfpathlineto{\pgfqpoint{1.141125in}{0.461287in}}%
\pgfpathlineto{\pgfqpoint{1.156781in}{0.461287in}}%
\pgfpathlineto{\pgfqpoint{1.172438in}{0.459120in}}%
\pgfpathlineto{\pgfqpoint{1.188094in}{0.454879in}}%
\pgfpathlineto{\pgfqpoint{1.189968in}{0.454135in}}%
\pgfpathlineto{\pgfqpoint{1.203751in}{0.448412in}}%
\pgfpathlineto{\pgfqpoint{1.218677in}{0.440524in}}%
\pgfpathlineto{\pgfqpoint{1.219407in}{0.440113in}}%
\pgfpathlineto{\pgfqpoint{1.235064in}{0.429593in}}%
\pgfpathlineto{\pgfqpoint{1.238579in}{0.426913in}}%
\pgfpathlineto{\pgfqpoint{1.250721in}{0.416875in}}%
\pgfpathlineto{\pgfqpoint{1.254656in}{0.413302in}}%
\pgfpathlineto{\pgfqpoint{1.266377in}{0.401440in}}%
\pgfpathlineto{\pgfqpoint{1.267990in}{0.399691in}}%
\pgfpathlineto{\pgfqpoint{1.278771in}{0.386079in}}%
\pgfpathlineto{\pgfqpoint{1.282034in}{0.381021in}}%
\pgfpathlineto{\pgfqpoint{1.287262in}{0.372468in}}%
\pgfpathlineto{\pgfqpoint{1.293427in}{0.358857in}}%
\pgfpathlineto{\pgfqpoint{1.297234in}{0.345246in}}%
\pgfpathlineto{\pgfqpoint{1.297690in}{0.340413in}}%
\pgfpathlineto{\pgfqpoint{1.298481in}{0.331635in}}%
\pgfpathlineto{\pgfqpoint{1.313347in}{0.331635in}}%
\pgfpathlineto{\pgfqpoint{1.329003in}{0.331635in}}%
\pgfpathlineto{\pgfqpoint{1.344660in}{0.331635in}}%
\pgfpathlineto{\pgfqpoint{1.360317in}{0.331635in}}%
\pgfpathlineto{\pgfqpoint{1.375973in}{0.331635in}}%
\pgfpathlineto{\pgfqpoint{1.391630in}{0.331635in}}%
\pgfpathlineto{\pgfqpoint{1.403843in}{0.331635in}}%
\pgfpathlineto{\pgfqpoint{1.402283in}{0.345246in}}%
\pgfpathlineto{\pgfqpoint{1.397667in}{0.358857in}}%
\pgfpathlineto{\pgfqpoint{1.391630in}{0.369827in}}%
\pgfpathlineto{\pgfqpoint{1.390266in}{0.372468in}}%
\pgfpathlineto{\pgfqpoint{1.380748in}{0.386079in}}%
\pgfpathlineto{\pgfqpoint{1.375973in}{0.391631in}}%
\pgfpathlineto{\pgfqpoint{1.369401in}{0.399691in}}%
\pgfpathlineto{\pgfqpoint{1.360317in}{0.409307in}}%
\pgfpathlineto{\pgfqpoint{1.356689in}{0.413302in}}%
\pgfpathlineto{\pgfqpoint{1.344660in}{0.425214in}}%
\pgfpathlineto{\pgfqpoint{1.342988in}{0.426913in}}%
\pgfpathlineto{\pgfqpoint{1.329003in}{0.440090in}}%
\pgfpathlineto{\pgfqpoint{1.328548in}{0.440524in}}%
\pgfpathlineto{\pgfqpoint{1.313524in}{0.454135in}}%
\pgfpathlineto{\pgfqpoint{1.313347in}{0.454291in}}%
\pgfpathlineto{\pgfqpoint{1.297990in}{0.467746in}}%
\pgfpathlineto{\pgfqpoint{1.297690in}{0.468006in}}%
\pgfpathlineto{\pgfqpoint{1.282034in}{0.481296in}}%
\pgfpathlineto{\pgfqpoint{1.281959in}{0.481357in}}%
\pgfpathlineto{\pgfqpoint{1.266377in}{0.494113in}}%
\pgfpathlineto{\pgfqpoint{1.265266in}{0.494968in}}%
\pgfpathlineto{\pgfqpoint{1.250721in}{0.506368in}}%
\pgfpathlineto{\pgfqpoint{1.247633in}{0.508579in}}%
\pgfpathlineto{\pgfqpoint{1.235064in}{0.517882in}}%
\pgfpathlineto{\pgfqpoint{1.228467in}{0.522191in}}%
\pgfpathlineto{\pgfqpoint{1.219407in}{0.528388in}}%
\pgfpathlineto{\pgfqpoint{1.206523in}{0.535802in}}%
\pgfpathlineto{\pgfqpoint{1.203751in}{0.537496in}}%
\pgfpathlineto{\pgfqpoint{1.188094in}{0.544928in}}%
\pgfpathlineto{\pgfqpoint{1.174400in}{0.549413in}}%
\pgfpathlineto{\pgfqpoint{1.172438in}{0.550103in}}%
\pgfpathlineto{\pgfqpoint{1.156781in}{0.552895in}}%
\pgfpathlineto{\pgfqpoint{1.141125in}{0.552895in}}%
\pgfpathlineto{\pgfqpoint{1.125468in}{0.550103in}}%
\pgfpathlineto{\pgfqpoint{1.123506in}{0.549413in}}%
\pgfpathlineto{\pgfqpoint{1.109812in}{0.544928in}}%
\pgfpathlineto{\pgfqpoint{1.094155in}{0.537496in}}%
\pgfpathlineto{\pgfqpoint{1.091383in}{0.535802in}}%
\pgfpathlineto{\pgfqpoint{1.078498in}{0.528388in}}%
\pgfpathlineto{\pgfqpoint{1.069438in}{0.522191in}}%
\pgfpathlineto{\pgfqpoint{1.062842in}{0.517882in}}%
\pgfpathlineto{\pgfqpoint{1.050273in}{0.508579in}}%
\pgfpathlineto{\pgfqpoint{1.047185in}{0.506368in}}%
\pgfpathlineto{\pgfqpoint{1.032640in}{0.494968in}}%
\pgfpathlineto{\pgfqpoint{1.031529in}{0.494113in}}%
\pgfpathlineto{\pgfqpoint{1.015947in}{0.481357in}}%
\pgfpathlineto{\pgfqpoint{1.015872in}{0.481296in}}%
\pgfpathlineto{\pgfqpoint{1.000216in}{0.468006in}}%
\pgfpathlineto{\pgfqpoint{0.999916in}{0.467746in}}%
\pgfpathlineto{\pgfqpoint{0.984559in}{0.454291in}}%
\pgfpathlineto{\pgfqpoint{0.984382in}{0.454135in}}%
\pgfpathlineto{\pgfqpoint{0.969358in}{0.440524in}}%
\pgfpathlineto{\pgfqpoint{0.968902in}{0.440090in}}%
\pgfpathlineto{\pgfqpoint{0.954918in}{0.426913in}}%
\pgfpathlineto{\pgfqpoint{0.953246in}{0.425214in}}%
\pgfpathlineto{\pgfqpoint{0.941216in}{0.413302in}}%
\pgfpathlineto{\pgfqpoint{0.937589in}{0.409307in}}%
\pgfpathlineto{\pgfqpoint{0.928505in}{0.399691in}}%
\pgfpathlineto{\pgfqpoint{0.921933in}{0.391631in}}%
\pgfpathlineto{\pgfqpoint{0.917158in}{0.386079in}}%
\pgfpathlineto{\pgfqpoint{0.907640in}{0.372468in}}%
\pgfpathlineto{\pgfqpoint{0.906276in}{0.369827in}}%
\pgfpathlineto{\pgfqpoint{0.900238in}{0.358857in}}%
\pgfpathlineto{\pgfqpoint{0.895623in}{0.345246in}}%
\pgfpathlineto{\pgfqpoint{0.894062in}{0.331635in}}%
\pgfpathlineto{\pgfqpoint{0.906276in}{0.331635in}}%
\pgfpathclose%
\pgfpathmoveto{\pgfqpoint{1.673448in}{0.331635in}}%
\pgfpathlineto{\pgfqpoint{1.689104in}{0.331635in}}%
\pgfpathlineto{\pgfqpoint{1.704761in}{0.331635in}}%
\pgfpathlineto{\pgfqpoint{1.720418in}{0.331635in}}%
\pgfpathlineto{\pgfqpoint{1.736074in}{0.331635in}}%
\pgfpathlineto{\pgfqpoint{1.751731in}{0.331635in}}%
\pgfpathlineto{\pgfqpoint{1.767387in}{0.331635in}}%
\pgfpathlineto{\pgfqpoint{1.774503in}{0.331635in}}%
\pgfpathlineto{\pgfqpoint{1.775756in}{0.345246in}}%
\pgfpathlineto{\pgfqpoint{1.779461in}{0.358857in}}%
\pgfpathlineto{\pgfqpoint{1.783044in}{0.366946in}}%
\pgfpathlineto{\pgfqpoint{1.785609in}{0.372468in}}%
\pgfpathlineto{\pgfqpoint{1.794150in}{0.386079in}}%
\pgfpathlineto{\pgfqpoint{1.798700in}{0.391977in}}%
\pgfpathlineto{\pgfqpoint{1.805023in}{0.399691in}}%
\pgfpathlineto{\pgfqpoint{1.814357in}{0.409519in}}%
\pgfpathlineto{\pgfqpoint{1.818257in}{0.413302in}}%
\pgfpathlineto{\pgfqpoint{1.830014in}{0.423523in}}%
\pgfpathlineto{\pgfqpoint{1.834365in}{0.426913in}}%
\pgfpathlineto{\pgfqpoint{1.845670in}{0.435027in}}%
\pgfpathlineto{\pgfqpoint{1.854542in}{0.440524in}}%
\pgfpathlineto{\pgfqpoint{1.861327in}{0.444479in}}%
\pgfpathlineto{\pgfqpoint{1.876983in}{0.451905in}}%
\pgfpathlineto{\pgfqpoint{1.883336in}{0.454135in}}%
\pgfpathlineto{\pgfqpoint{1.892640in}{0.457249in}}%
\pgfpathlineto{\pgfqpoint{1.908296in}{0.460471in}}%
\pgfpathlineto{\pgfqpoint{1.923953in}{0.461560in}}%
\pgfpathlineto{\pgfqpoint{1.923953in}{0.467746in}}%
\pgfpathlineto{\pgfqpoint{1.923953in}{0.481357in}}%
\pgfpathlineto{\pgfqpoint{1.923953in}{0.494968in}}%
\pgfpathlineto{\pgfqpoint{1.923953in}{0.508579in}}%
\pgfpathlineto{\pgfqpoint{1.923953in}{0.522191in}}%
\pgfpathlineto{\pgfqpoint{1.923953in}{0.535802in}}%
\pgfpathlineto{\pgfqpoint{1.923953in}{0.549413in}}%
\pgfpathlineto{\pgfqpoint{1.923953in}{0.553246in}}%
\pgfpathlineto{\pgfqpoint{1.908296in}{0.551843in}}%
\pgfpathlineto{\pgfqpoint{1.899153in}{0.549413in}}%
\pgfpathlineto{\pgfqpoint{1.892640in}{0.547797in}}%
\pgfpathlineto{\pgfqpoint{1.876983in}{0.541482in}}%
\pgfpathlineto{\pgfqpoint{1.866541in}{0.535802in}}%
\pgfpathlineto{\pgfqpoint{1.861327in}{0.533126in}}%
\pgfpathlineto{\pgfqpoint{1.845670in}{0.523254in}}%
\pgfpathlineto{\pgfqpoint{1.844222in}{0.522191in}}%
\pgfpathlineto{\pgfqpoint{1.830014in}{0.512240in}}%
\pgfpathlineto{\pgfqpoint{1.825315in}{0.508579in}}%
\pgfpathlineto{\pgfqpoint{1.814357in}{0.500324in}}%
\pgfpathlineto{\pgfqpoint{1.807766in}{0.494968in}}%
\pgfpathlineto{\pgfqpoint{1.798700in}{0.487745in}}%
\pgfpathlineto{\pgfqpoint{1.791079in}{0.481357in}}%
\pgfpathlineto{\pgfqpoint{1.783044in}{0.474654in}}%
\pgfpathlineto{\pgfqpoint{1.775005in}{0.467746in}}%
\pgfpathlineto{\pgfqpoint{1.767387in}{0.461123in}}%
\pgfpathlineto{\pgfqpoint{1.759441in}{0.454135in}}%
\pgfpathlineto{\pgfqpoint{1.751731in}{0.447150in}}%
\pgfpathlineto{\pgfqpoint{1.744382in}{0.440524in}}%
\pgfpathlineto{\pgfqpoint{1.736074in}{0.432643in}}%
\pgfpathlineto{\pgfqpoint{1.729913in}{0.426913in}}%
\pgfpathlineto{\pgfqpoint{1.720418in}{0.417386in}}%
\pgfpathlineto{\pgfqpoint{1.716207in}{0.413302in}}%
\pgfpathlineto{\pgfqpoint{1.704761in}{0.400949in}}%
\pgfpathlineto{\pgfqpoint{1.703538in}{0.399691in}}%
\pgfpathlineto{\pgfqpoint{1.692182in}{0.386079in}}%
\pgfpathlineto{\pgfqpoint{1.689104in}{0.381546in}}%
\pgfpathlineto{\pgfqpoint{1.682570in}{0.372468in}}%
\pgfpathlineto{\pgfqpoint{1.675306in}{0.358857in}}%
\pgfpathlineto{\pgfqpoint{1.673448in}{0.353195in}}%
\pgfpathlineto{\pgfqpoint{1.670652in}{0.345246in}}%
\pgfpathlineto{\pgfqpoint{1.669038in}{0.331635in}}%
\pgfpathlineto{\pgfqpoint{1.673448in}{0.331635in}}%
\pgfpathclose%
\pgfpathmoveto{\pgfqpoint{0.389609in}{0.785152in}}%
\pgfpathlineto{\pgfqpoint{0.405266in}{0.789164in}}%
\pgfpathlineto{\pgfqpoint{0.417885in}{0.794413in}}%
\pgfpathlineto{\pgfqpoint{0.420923in}{0.795598in}}%
\pgfpathlineto{\pgfqpoint{0.436579in}{0.803873in}}%
\pgfpathlineto{\pgfqpoint{0.442965in}{0.808024in}}%
\pgfpathlineto{\pgfqpoint{0.452236in}{0.813738in}}%
\pgfpathlineto{\pgfqpoint{0.463298in}{0.821635in}}%
\pgfpathlineto{\pgfqpoint{0.467892in}{0.824788in}}%
\pgfpathlineto{\pgfqpoint{0.481594in}{0.835246in}}%
\pgfpathlineto{\pgfqpoint{0.483549in}{0.836700in}}%
\pgfpathlineto{\pgfqpoint{0.498706in}{0.848857in}}%
\pgfpathlineto{\pgfqpoint{0.499205in}{0.849253in}}%
\pgfpathlineto{\pgfqpoint{0.514862in}{0.862314in}}%
\pgfpathlineto{\pgfqpoint{0.515041in}{0.862468in}}%
\pgfpathlineto{\pgfqpoint{0.530519in}{0.875819in}}%
\pgfpathlineto{\pgfqpoint{0.530817in}{0.876079in}}%
\pgfpathlineto{\pgfqpoint{0.546105in}{0.889691in}}%
\pgfpathlineto{\pgfqpoint{0.546175in}{0.889756in}}%
\pgfpathlineto{\pgfqpoint{0.560848in}{0.903302in}}%
\pgfpathlineto{\pgfqpoint{0.561832in}{0.904268in}}%
\pgfpathlineto{\pgfqpoint{0.574944in}{0.916913in}}%
\pgfpathlineto{\pgfqpoint{0.577488in}{0.919597in}}%
\pgfpathlineto{\pgfqpoint{0.588189in}{0.930524in}}%
\pgfpathlineto{\pgfqpoint{0.593145in}{0.936259in}}%
\pgfpathlineto{\pgfqpoint{0.600274in}{0.944135in}}%
\pgfpathlineto{\pgfqpoint{0.608801in}{0.955336in}}%
\pgfpathlineto{\pgfqpoint{0.610750in}{0.957746in}}%
\pgfpathlineto{\pgfqpoint{0.619299in}{0.971357in}}%
\pgfpathlineto{\pgfqpoint{0.624458in}{0.983262in}}%
\pgfpathlineto{\pgfqpoint{0.625252in}{0.984968in}}%
\pgfpathlineto{\pgfqpoint{0.628463in}{0.998579in}}%
\pgfpathlineto{\pgfqpoint{0.628463in}{1.012191in}}%
\pgfpathlineto{\pgfqpoint{0.625252in}{1.025802in}}%
\pgfpathlineto{\pgfqpoint{0.624458in}{1.027508in}}%
\pgfpathlineto{\pgfqpoint{0.619299in}{1.039413in}}%
\pgfpathlineto{\pgfqpoint{0.610750in}{1.053024in}}%
\pgfpathlineto{\pgfqpoint{0.608801in}{1.055434in}}%
\pgfpathlineto{\pgfqpoint{0.600274in}{1.066635in}}%
\pgfpathlineto{\pgfqpoint{0.593145in}{1.074511in}}%
\pgfpathlineto{\pgfqpoint{0.588189in}{1.080246in}}%
\pgfpathlineto{\pgfqpoint{0.577488in}{1.091173in}}%
\pgfpathlineto{\pgfqpoint{0.574944in}{1.093857in}}%
\pgfpathlineto{\pgfqpoint{0.561832in}{1.106502in}}%
\pgfpathlineto{\pgfqpoint{0.560848in}{1.107468in}}%
\pgfpathlineto{\pgfqpoint{0.546175in}{1.121014in}}%
\pgfpathlineto{\pgfqpoint{0.546105in}{1.121079in}}%
\pgfpathlineto{\pgfqpoint{0.530817in}{1.134691in}}%
\pgfpathlineto{\pgfqpoint{0.530519in}{1.134951in}}%
\pgfpathlineto{\pgfqpoint{0.515041in}{1.148302in}}%
\pgfpathlineto{\pgfqpoint{0.514862in}{1.148456in}}%
\pgfpathlineto{\pgfqpoint{0.499205in}{1.161517in}}%
\pgfpathlineto{\pgfqpoint{0.498706in}{1.161913in}}%
\pgfpathlineto{\pgfqpoint{0.483549in}{1.174070in}}%
\pgfpathlineto{\pgfqpoint{0.481594in}{1.175524in}}%
\pgfpathlineto{\pgfqpoint{0.467892in}{1.185982in}}%
\pgfpathlineto{\pgfqpoint{0.463298in}{1.189135in}}%
\pgfpathlineto{\pgfqpoint{0.452236in}{1.197032in}}%
\pgfpathlineto{\pgfqpoint{0.442965in}{1.202746in}}%
\pgfpathlineto{\pgfqpoint{0.436579in}{1.206897in}}%
\pgfpathlineto{\pgfqpoint{0.420923in}{1.215172in}}%
\pgfpathlineto{\pgfqpoint{0.417885in}{1.216357in}}%
\pgfpathlineto{\pgfqpoint{0.405266in}{1.221606in}}%
\pgfpathlineto{\pgfqpoint{0.389609in}{1.225618in}}%
\pgfpathlineto{\pgfqpoint{0.373953in}{1.226975in}}%
\pgfpathlineto{\pgfqpoint{0.373953in}{1.216357in}}%
\pgfpathlineto{\pgfqpoint{0.373953in}{1.202746in}}%
\pgfpathlineto{\pgfqpoint{0.373953in}{1.189135in}}%
\pgfpathlineto{\pgfqpoint{0.373953in}{1.175524in}}%
\pgfpathlineto{\pgfqpoint{0.373953in}{1.161913in}}%
\pgfpathlineto{\pgfqpoint{0.373953in}{1.148302in}}%
\pgfpathlineto{\pgfqpoint{0.373953in}{1.135378in}}%
\pgfpathlineto{\pgfqpoint{0.384050in}{1.134691in}}%
\pgfpathlineto{\pgfqpoint{0.389609in}{1.134294in}}%
\pgfpathlineto{\pgfqpoint{0.405266in}{1.130985in}}%
\pgfpathlineto{\pgfqpoint{0.420923in}{1.125625in}}%
\pgfpathlineto{\pgfqpoint{0.430761in}{1.121079in}}%
\pgfpathlineto{\pgfqpoint{0.436579in}{1.118243in}}%
\pgfpathlineto{\pgfqpoint{0.452236in}{1.108871in}}%
\pgfpathlineto{\pgfqpoint{0.454249in}{1.107468in}}%
\pgfpathlineto{\pgfqpoint{0.467892in}{1.097279in}}%
\pgfpathlineto{\pgfqpoint{0.472002in}{1.093857in}}%
\pgfpathlineto{\pgfqpoint{0.483549in}{1.083302in}}%
\pgfpathlineto{\pgfqpoint{0.486632in}{1.080246in}}%
\pgfpathlineto{\pgfqpoint{0.498733in}{1.066635in}}%
\pgfpathlineto{\pgfqpoint{0.499205in}{1.066000in}}%
\pgfpathlineto{\pgfqpoint{0.508279in}{1.053024in}}%
\pgfpathlineto{\pgfqpoint{0.514862in}{1.041042in}}%
\pgfpathlineto{\pgfqpoint{0.515717in}{1.039413in}}%
\pgfpathlineto{\pgfqpoint{0.520596in}{1.025802in}}%
\pgfpathlineto{\pgfqpoint{0.523089in}{1.012191in}}%
\pgfpathlineto{\pgfqpoint{0.523089in}{0.998579in}}%
\pgfpathlineto{\pgfqpoint{0.520596in}{0.984968in}}%
\pgfpathlineto{\pgfqpoint{0.515717in}{0.971357in}}%
\pgfpathlineto{\pgfqpoint{0.514862in}{0.969728in}}%
\pgfpathlineto{\pgfqpoint{0.508279in}{0.957746in}}%
\pgfpathlineto{\pgfqpoint{0.499205in}{0.944770in}}%
\pgfpathlineto{\pgfqpoint{0.498733in}{0.944135in}}%
\pgfpathlineto{\pgfqpoint{0.486632in}{0.930524in}}%
\pgfpathlineto{\pgfqpoint{0.483549in}{0.927468in}}%
\pgfpathlineto{\pgfqpoint{0.472002in}{0.916913in}}%
\pgfpathlineto{\pgfqpoint{0.467892in}{0.913491in}}%
\pgfpathlineto{\pgfqpoint{0.454249in}{0.903302in}}%
\pgfpathlineto{\pgfqpoint{0.452236in}{0.901899in}}%
\pgfpathlineto{\pgfqpoint{0.436579in}{0.892527in}}%
\pgfpathlineto{\pgfqpoint{0.430761in}{0.889691in}}%
\pgfpathlineto{\pgfqpoint{0.420923in}{0.885145in}}%
\pgfpathlineto{\pgfqpoint{0.405266in}{0.879785in}}%
\pgfpathlineto{\pgfqpoint{0.389609in}{0.876476in}}%
\pgfpathlineto{\pgfqpoint{0.384050in}{0.876079in}}%
\pgfpathlineto{\pgfqpoint{0.373953in}{0.875392in}}%
\pgfpathlineto{\pgfqpoint{0.373953in}{0.862468in}}%
\pgfpathlineto{\pgfqpoint{0.373953in}{0.848857in}}%
\pgfpathlineto{\pgfqpoint{0.373953in}{0.835246in}}%
\pgfpathlineto{\pgfqpoint{0.373953in}{0.821635in}}%
\pgfpathlineto{\pgfqpoint{0.373953in}{0.808024in}}%
\pgfpathlineto{\pgfqpoint{0.373953in}{0.794413in}}%
\pgfpathlineto{\pgfqpoint{0.373953in}{0.783795in}}%
\pgfpathlineto{\pgfqpoint{0.389609in}{0.785152in}}%
\pgfpathclose%
\pgfpathmoveto{\pgfqpoint{1.109812in}{0.792117in}}%
\pgfpathlineto{\pgfqpoint{1.125468in}{0.786834in}}%
\pgfpathlineto{\pgfqpoint{1.141125in}{0.784135in}}%
\pgfpathlineto{\pgfqpoint{1.156781in}{0.784135in}}%
\pgfpathlineto{\pgfqpoint{1.172438in}{0.786834in}}%
\pgfpathlineto{\pgfqpoint{1.188094in}{0.792117in}}%
\pgfpathlineto{\pgfqpoint{1.192751in}{0.794413in}}%
\pgfpathlineto{\pgfqpoint{1.203751in}{0.799491in}}%
\pgfpathlineto{\pgfqpoint{1.218234in}{0.808024in}}%
\pgfpathlineto{\pgfqpoint{1.219407in}{0.808681in}}%
\pgfpathlineto{\pgfqpoint{1.235064in}{0.819141in}}%
\pgfpathlineto{\pgfqpoint{1.238352in}{0.821635in}}%
\pgfpathlineto{\pgfqpoint{1.250721in}{0.830642in}}%
\pgfpathlineto{\pgfqpoint{1.256496in}{0.835246in}}%
\pgfpathlineto{\pgfqpoint{1.266377in}{0.842914in}}%
\pgfpathlineto{\pgfqpoint{1.273572in}{0.848857in}}%
\pgfpathlineto{\pgfqpoint{1.282034in}{0.855762in}}%
\pgfpathlineto{\pgfqpoint{1.289931in}{0.862468in}}%
\pgfpathlineto{\pgfqpoint{1.297690in}{0.869078in}}%
\pgfpathlineto{\pgfqpoint{1.305744in}{0.876079in}}%
\pgfpathlineto{\pgfqpoint{1.313347in}{0.882825in}}%
\pgfpathlineto{\pgfqpoint{1.321061in}{0.889691in}}%
\pgfpathlineto{\pgfqpoint{1.329003in}{0.897047in}}%
\pgfpathlineto{\pgfqpoint{1.335840in}{0.903302in}}%
\pgfpathlineto{\pgfqpoint{1.344660in}{0.911891in}}%
\pgfpathlineto{\pgfqpoint{1.349956in}{0.916913in}}%
\pgfpathlineto{\pgfqpoint{1.360317in}{0.927665in}}%
\pgfpathlineto{\pgfqpoint{1.363186in}{0.930524in}}%
\pgfpathlineto{\pgfqpoint{1.375218in}{0.944135in}}%
\pgfpathlineto{\pgfqpoint{1.375973in}{0.945155in}}%
\pgfpathlineto{\pgfqpoint{1.385789in}{0.957746in}}%
\pgfpathlineto{\pgfqpoint{1.391630in}{0.967309in}}%
\pgfpathlineto{\pgfqpoint{1.394271in}{0.971357in}}%
\pgfpathlineto{\pgfqpoint{1.400348in}{0.984968in}}%
\pgfpathlineto{\pgfqpoint{1.403452in}{0.998579in}}%
\pgfpathlineto{\pgfqpoint{1.403452in}{1.012191in}}%
\pgfpathlineto{\pgfqpoint{1.400348in}{1.025802in}}%
\pgfpathlineto{\pgfqpoint{1.394271in}{1.039413in}}%
\pgfpathlineto{\pgfqpoint{1.391630in}{1.043461in}}%
\pgfpathlineto{\pgfqpoint{1.385789in}{1.053024in}}%
\pgfpathlineto{\pgfqpoint{1.375973in}{1.065615in}}%
\pgfpathlineto{\pgfqpoint{1.375218in}{1.066635in}}%
\pgfpathlineto{\pgfqpoint{1.363186in}{1.080246in}}%
\pgfpathlineto{\pgfqpoint{1.360317in}{1.083105in}}%
\pgfpathlineto{\pgfqpoint{1.349956in}{1.093857in}}%
\pgfpathlineto{\pgfqpoint{1.344660in}{1.098879in}}%
\pgfpathlineto{\pgfqpoint{1.335840in}{1.107468in}}%
\pgfpathlineto{\pgfqpoint{1.329003in}{1.113723in}}%
\pgfpathlineto{\pgfqpoint{1.321061in}{1.121079in}}%
\pgfpathlineto{\pgfqpoint{1.313347in}{1.127945in}}%
\pgfpathlineto{\pgfqpoint{1.305744in}{1.134691in}}%
\pgfpathlineto{\pgfqpoint{1.297690in}{1.141692in}}%
\pgfpathlineto{\pgfqpoint{1.289931in}{1.148302in}}%
\pgfpathlineto{\pgfqpoint{1.282034in}{1.155008in}}%
\pgfpathlineto{\pgfqpoint{1.273572in}{1.161913in}}%
\pgfpathlineto{\pgfqpoint{1.266377in}{1.167856in}}%
\pgfpathlineto{\pgfqpoint{1.256496in}{1.175524in}}%
\pgfpathlineto{\pgfqpoint{1.250721in}{1.180128in}}%
\pgfpathlineto{\pgfqpoint{1.238352in}{1.189135in}}%
\pgfpathlineto{\pgfqpoint{1.235064in}{1.191629in}}%
\pgfpathlineto{\pgfqpoint{1.219407in}{1.202089in}}%
\pgfpathlineto{\pgfqpoint{1.218234in}{1.202746in}}%
\pgfpathlineto{\pgfqpoint{1.203751in}{1.211279in}}%
\pgfpathlineto{\pgfqpoint{1.192751in}{1.216357in}}%
\pgfpathlineto{\pgfqpoint{1.188094in}{1.218653in}}%
\pgfpathlineto{\pgfqpoint{1.172438in}{1.223936in}}%
\pgfpathlineto{\pgfqpoint{1.156781in}{1.226635in}}%
\pgfpathlineto{\pgfqpoint{1.141125in}{1.226635in}}%
\pgfpathlineto{\pgfqpoint{1.125468in}{1.223936in}}%
\pgfpathlineto{\pgfqpoint{1.109812in}{1.218653in}}%
\pgfpathlineto{\pgfqpoint{1.105155in}{1.216357in}}%
\pgfpathlineto{\pgfqpoint{1.094155in}{1.211279in}}%
\pgfpathlineto{\pgfqpoint{1.079672in}{1.202746in}}%
\pgfpathlineto{\pgfqpoint{1.078498in}{1.202089in}}%
\pgfpathlineto{\pgfqpoint{1.062842in}{1.191629in}}%
\pgfpathlineto{\pgfqpoint{1.059554in}{1.189135in}}%
\pgfpathlineto{\pgfqpoint{1.047185in}{1.180128in}}%
\pgfpathlineto{\pgfqpoint{1.041409in}{1.175524in}}%
\pgfpathlineto{\pgfqpoint{1.031529in}{1.167856in}}%
\pgfpathlineto{\pgfqpoint{1.024334in}{1.161913in}}%
\pgfpathlineto{\pgfqpoint{1.015872in}{1.155008in}}%
\pgfpathlineto{\pgfqpoint{1.007975in}{1.148302in}}%
\pgfpathlineto{\pgfqpoint{1.000216in}{1.141692in}}%
\pgfpathlineto{\pgfqpoint{0.992162in}{1.134691in}}%
\pgfpathlineto{\pgfqpoint{0.984559in}{1.127945in}}%
\pgfpathlineto{\pgfqpoint{0.976845in}{1.121079in}}%
\pgfpathlineto{\pgfqpoint{0.968902in}{1.113723in}}%
\pgfpathlineto{\pgfqpoint{0.962066in}{1.107468in}}%
\pgfpathlineto{\pgfqpoint{0.953246in}{1.098879in}}%
\pgfpathlineto{\pgfqpoint{0.947950in}{1.093857in}}%
\pgfpathlineto{\pgfqpoint{0.937589in}{1.083105in}}%
\pgfpathlineto{\pgfqpoint{0.934720in}{1.080246in}}%
\pgfpathlineto{\pgfqpoint{0.922688in}{1.066635in}}%
\pgfpathlineto{\pgfqpoint{0.921933in}{1.065615in}}%
\pgfpathlineto{\pgfqpoint{0.912117in}{1.053024in}}%
\pgfpathlineto{\pgfqpoint{0.906276in}{1.043461in}}%
\pgfpathlineto{\pgfqpoint{0.903635in}{1.039413in}}%
\pgfpathlineto{\pgfqpoint{0.897558in}{1.025802in}}%
\pgfpathlineto{\pgfqpoint{0.894454in}{1.012191in}}%
\pgfpathlineto{\pgfqpoint{0.894454in}{0.998579in}}%
\pgfpathlineto{\pgfqpoint{0.897558in}{0.984968in}}%
\pgfpathlineto{\pgfqpoint{0.903635in}{0.971357in}}%
\pgfpathlineto{\pgfqpoint{0.906276in}{0.967309in}}%
\pgfpathlineto{\pgfqpoint{0.912117in}{0.957746in}}%
\pgfpathlineto{\pgfqpoint{0.921933in}{0.945155in}}%
\pgfpathlineto{\pgfqpoint{0.922688in}{0.944135in}}%
\pgfpathlineto{\pgfqpoint{0.934720in}{0.930524in}}%
\pgfpathlineto{\pgfqpoint{0.937589in}{0.927665in}}%
\pgfpathlineto{\pgfqpoint{0.947950in}{0.916913in}}%
\pgfpathlineto{\pgfqpoint{0.953246in}{0.911891in}}%
\pgfpathlineto{\pgfqpoint{0.962066in}{0.903302in}}%
\pgfpathlineto{\pgfqpoint{0.968902in}{0.897047in}}%
\pgfpathlineto{\pgfqpoint{0.976845in}{0.889691in}}%
\pgfpathlineto{\pgfqpoint{0.984559in}{0.882825in}}%
\pgfpathlineto{\pgfqpoint{0.992162in}{0.876079in}}%
\pgfpathlineto{\pgfqpoint{1.000216in}{0.869078in}}%
\pgfpathlineto{\pgfqpoint{1.007975in}{0.862468in}}%
\pgfpathlineto{\pgfqpoint{1.015872in}{0.855762in}}%
\pgfpathlineto{\pgfqpoint{1.024334in}{0.848857in}}%
\pgfpathlineto{\pgfqpoint{1.031529in}{0.842914in}}%
\pgfpathlineto{\pgfqpoint{1.041409in}{0.835246in}}%
\pgfpathlineto{\pgfqpoint{1.047185in}{0.830642in}}%
\pgfpathlineto{\pgfqpoint{1.059554in}{0.821635in}}%
\pgfpathlineto{\pgfqpoint{1.062842in}{0.819141in}}%
\pgfpathlineto{\pgfqpoint{1.078498in}{0.808681in}}%
\pgfpathlineto{\pgfqpoint{1.079672in}{0.808024in}}%
\pgfpathlineto{\pgfqpoint{1.094155in}{0.799491in}}%
\pgfpathlineto{\pgfqpoint{1.105155in}{0.794413in}}%
\pgfpathlineto{\pgfqpoint{1.109812in}{0.792117in}}%
\pgfpathclose%
\pgfpathmoveto{\pgfqpoint{1.138029in}{0.876079in}}%
\pgfpathlineto{\pgfqpoint{1.125468in}{0.877864in}}%
\pgfpathlineto{\pgfqpoint{1.109812in}{0.882221in}}%
\pgfpathlineto{\pgfqpoint{1.094155in}{0.888529in}}%
\pgfpathlineto{\pgfqpoint{1.091916in}{0.889691in}}%
\pgfpathlineto{\pgfqpoint{1.078498in}{0.897025in}}%
\pgfpathlineto{\pgfqpoint{1.068826in}{0.903302in}}%
\pgfpathlineto{\pgfqpoint{1.062842in}{0.907470in}}%
\pgfpathlineto{\pgfqpoint{1.050899in}{0.916913in}}%
\pgfpathlineto{\pgfqpoint{1.047185in}{0.920141in}}%
\pgfpathlineto{\pgfqpoint{1.036323in}{0.930524in}}%
\pgfpathlineto{\pgfqpoint{1.031529in}{0.935726in}}%
\pgfpathlineto{\pgfqpoint{1.024309in}{0.944135in}}%
\pgfpathlineto{\pgfqpoint{1.015872in}{0.955800in}}%
\pgfpathlineto{\pgfqpoint{1.014536in}{0.957746in}}%
\pgfpathlineto{\pgfqpoint{1.007280in}{0.971357in}}%
\pgfpathlineto{\pgfqpoint{1.002268in}{0.984968in}}%
\pgfpathlineto{\pgfqpoint{1.000216in}{0.995888in}}%
\pgfpathlineto{\pgfqpoint{0.999732in}{0.998579in}}%
\pgfpathlineto{\pgfqpoint{0.999732in}{1.012191in}}%
\pgfpathlineto{\pgfqpoint{1.000216in}{1.014882in}}%
\pgfpathlineto{\pgfqpoint{1.002268in}{1.025802in}}%
\pgfpathlineto{\pgfqpoint{1.007280in}{1.039413in}}%
\pgfpathlineto{\pgfqpoint{1.014536in}{1.053024in}}%
\pgfpathlineto{\pgfqpoint{1.015872in}{1.054970in}}%
\pgfpathlineto{\pgfqpoint{1.024309in}{1.066635in}}%
\pgfpathlineto{\pgfqpoint{1.031529in}{1.075044in}}%
\pgfpathlineto{\pgfqpoint{1.036323in}{1.080246in}}%
\pgfpathlineto{\pgfqpoint{1.047185in}{1.090629in}}%
\pgfpathlineto{\pgfqpoint{1.050899in}{1.093857in}}%
\pgfpathlineto{\pgfqpoint{1.062842in}{1.103300in}}%
\pgfpathlineto{\pgfqpoint{1.068826in}{1.107468in}}%
\pgfpathlineto{\pgfqpoint{1.078498in}{1.113745in}}%
\pgfpathlineto{\pgfqpoint{1.091916in}{1.121079in}}%
\pgfpathlineto{\pgfqpoint{1.094155in}{1.122241in}}%
\pgfpathlineto{\pgfqpoint{1.109812in}{1.128549in}}%
\pgfpathlineto{\pgfqpoint{1.125468in}{1.132906in}}%
\pgfpathlineto{\pgfqpoint{1.138029in}{1.134691in}}%
\pgfpathlineto{\pgfqpoint{1.141125in}{1.135111in}}%
\pgfpathlineto{\pgfqpoint{1.156781in}{1.135111in}}%
\pgfpathlineto{\pgfqpoint{1.159877in}{1.134691in}}%
\pgfpathlineto{\pgfqpoint{1.172438in}{1.132906in}}%
\pgfpathlineto{\pgfqpoint{1.188094in}{1.128549in}}%
\pgfpathlineto{\pgfqpoint{1.203751in}{1.122241in}}%
\pgfpathlineto{\pgfqpoint{1.205989in}{1.121079in}}%
\pgfpathlineto{\pgfqpoint{1.219407in}{1.113745in}}%
\pgfpathlineto{\pgfqpoint{1.229080in}{1.107468in}}%
\pgfpathlineto{\pgfqpoint{1.235064in}{1.103300in}}%
\pgfpathlineto{\pgfqpoint{1.247007in}{1.093857in}}%
\pgfpathlineto{\pgfqpoint{1.250721in}{1.090629in}}%
\pgfpathlineto{\pgfqpoint{1.261583in}{1.080246in}}%
\pgfpathlineto{\pgfqpoint{1.266377in}{1.075044in}}%
\pgfpathlineto{\pgfqpoint{1.273597in}{1.066635in}}%
\pgfpathlineto{\pgfqpoint{1.282034in}{1.054970in}}%
\pgfpathlineto{\pgfqpoint{1.283370in}{1.053024in}}%
\pgfpathlineto{\pgfqpoint{1.290626in}{1.039413in}}%
\pgfpathlineto{\pgfqpoint{1.295638in}{1.025802in}}%
\pgfpathlineto{\pgfqpoint{1.297690in}{1.014882in}}%
\pgfpathlineto{\pgfqpoint{1.298174in}{1.012191in}}%
\pgfpathlineto{\pgfqpoint{1.298174in}{0.998579in}}%
\pgfpathlineto{\pgfqpoint{1.297690in}{0.995888in}}%
\pgfpathlineto{\pgfqpoint{1.295638in}{0.984968in}}%
\pgfpathlineto{\pgfqpoint{1.290626in}{0.971357in}}%
\pgfpathlineto{\pgfqpoint{1.283370in}{0.957746in}}%
\pgfpathlineto{\pgfqpoint{1.282034in}{0.955800in}}%
\pgfpathlineto{\pgfqpoint{1.273597in}{0.944135in}}%
\pgfpathlineto{\pgfqpoint{1.266377in}{0.935726in}}%
\pgfpathlineto{\pgfqpoint{1.261583in}{0.930524in}}%
\pgfpathlineto{\pgfqpoint{1.250721in}{0.920141in}}%
\pgfpathlineto{\pgfqpoint{1.247007in}{0.916913in}}%
\pgfpathlineto{\pgfqpoint{1.235064in}{0.907470in}}%
\pgfpathlineto{\pgfqpoint{1.229080in}{0.903302in}}%
\pgfpathlineto{\pgfqpoint{1.219407in}{0.897025in}}%
\pgfpathlineto{\pgfqpoint{1.205989in}{0.889691in}}%
\pgfpathlineto{\pgfqpoint{1.203751in}{0.888529in}}%
\pgfpathlineto{\pgfqpoint{1.188094in}{0.882221in}}%
\pgfpathlineto{\pgfqpoint{1.172438in}{0.877864in}}%
\pgfpathlineto{\pgfqpoint{1.159877in}{0.876079in}}%
\pgfpathlineto{\pgfqpoint{1.156781in}{0.875659in}}%
\pgfpathlineto{\pgfqpoint{1.141125in}{0.875659in}}%
\pgfpathlineto{\pgfqpoint{1.138029in}{0.876079in}}%
\pgfpathclose%
\pgfpathmoveto{\pgfqpoint{1.892640in}{0.789164in}}%
\pgfpathlineto{\pgfqpoint{1.908296in}{0.785152in}}%
\pgfpathlineto{\pgfqpoint{1.923953in}{0.783795in}}%
\pgfpathlineto{\pgfqpoint{1.923953in}{0.794413in}}%
\pgfpathlineto{\pgfqpoint{1.923953in}{0.808024in}}%
\pgfpathlineto{\pgfqpoint{1.923953in}{0.821635in}}%
\pgfpathlineto{\pgfqpoint{1.923953in}{0.835246in}}%
\pgfpathlineto{\pgfqpoint{1.923953in}{0.848857in}}%
\pgfpathlineto{\pgfqpoint{1.923953in}{0.862468in}}%
\pgfpathlineto{\pgfqpoint{1.923953in}{0.875392in}}%
\pgfpathlineto{\pgfqpoint{1.913856in}{0.876079in}}%
\pgfpathlineto{\pgfqpoint{1.908296in}{0.876476in}}%
\pgfpathlineto{\pgfqpoint{1.892640in}{0.879785in}}%
\pgfpathlineto{\pgfqpoint{1.876983in}{0.885145in}}%
\pgfpathlineto{\pgfqpoint{1.867145in}{0.889691in}}%
\pgfpathlineto{\pgfqpoint{1.861327in}{0.892527in}}%
\pgfpathlineto{\pgfqpoint{1.845670in}{0.901899in}}%
\pgfpathlineto{\pgfqpoint{1.843657in}{0.903302in}}%
\pgfpathlineto{\pgfqpoint{1.830014in}{0.913491in}}%
\pgfpathlineto{\pgfqpoint{1.825904in}{0.916913in}}%
\pgfpathlineto{\pgfqpoint{1.814357in}{0.927468in}}%
\pgfpathlineto{\pgfqpoint{1.811274in}{0.930524in}}%
\pgfpathlineto{\pgfqpoint{1.799173in}{0.944135in}}%
\pgfpathlineto{\pgfqpoint{1.798700in}{0.944770in}}%
\pgfpathlineto{\pgfqpoint{1.789627in}{0.957746in}}%
\pgfpathlineto{\pgfqpoint{1.783044in}{0.969728in}}%
\pgfpathlineto{\pgfqpoint{1.782188in}{0.971357in}}%
\pgfpathlineto{\pgfqpoint{1.777310in}{0.984968in}}%
\pgfpathlineto{\pgfqpoint{1.774817in}{0.998579in}}%
\pgfpathlineto{\pgfqpoint{1.774817in}{1.012191in}}%
\pgfpathlineto{\pgfqpoint{1.777310in}{1.025802in}}%
\pgfpathlineto{\pgfqpoint{1.782188in}{1.039413in}}%
\pgfpathlineto{\pgfqpoint{1.783044in}{1.041042in}}%
\pgfpathlineto{\pgfqpoint{1.789627in}{1.053024in}}%
\pgfpathlineto{\pgfqpoint{1.798700in}{1.066000in}}%
\pgfpathlineto{\pgfqpoint{1.799173in}{1.066635in}}%
\pgfpathlineto{\pgfqpoint{1.811274in}{1.080246in}}%
\pgfpathlineto{\pgfqpoint{1.814357in}{1.083302in}}%
\pgfpathlineto{\pgfqpoint{1.825904in}{1.093857in}}%
\pgfpathlineto{\pgfqpoint{1.830014in}{1.097279in}}%
\pgfpathlineto{\pgfqpoint{1.843657in}{1.107468in}}%
\pgfpathlineto{\pgfqpoint{1.845670in}{1.108871in}}%
\pgfpathlineto{\pgfqpoint{1.861327in}{1.118243in}}%
\pgfpathlineto{\pgfqpoint{1.867145in}{1.121079in}}%
\pgfpathlineto{\pgfqpoint{1.876983in}{1.125625in}}%
\pgfpathlineto{\pgfqpoint{1.892640in}{1.130985in}}%
\pgfpathlineto{\pgfqpoint{1.908296in}{1.134294in}}%
\pgfpathlineto{\pgfqpoint{1.913856in}{1.134691in}}%
\pgfpathlineto{\pgfqpoint{1.923953in}{1.135378in}}%
\pgfpathlineto{\pgfqpoint{1.923953in}{1.148302in}}%
\pgfpathlineto{\pgfqpoint{1.923953in}{1.161913in}}%
\pgfpathlineto{\pgfqpoint{1.923953in}{1.175524in}}%
\pgfpathlineto{\pgfqpoint{1.923953in}{1.189135in}}%
\pgfpathlineto{\pgfqpoint{1.923953in}{1.202746in}}%
\pgfpathlineto{\pgfqpoint{1.923953in}{1.216357in}}%
\pgfpathlineto{\pgfqpoint{1.923953in}{1.226975in}}%
\pgfpathlineto{\pgfqpoint{1.908296in}{1.225618in}}%
\pgfpathlineto{\pgfqpoint{1.892640in}{1.221606in}}%
\pgfpathlineto{\pgfqpoint{1.880021in}{1.216357in}}%
\pgfpathlineto{\pgfqpoint{1.876983in}{1.215172in}}%
\pgfpathlineto{\pgfqpoint{1.861327in}{1.206897in}}%
\pgfpathlineto{\pgfqpoint{1.854941in}{1.202746in}}%
\pgfpathlineto{\pgfqpoint{1.845670in}{1.197032in}}%
\pgfpathlineto{\pgfqpoint{1.834608in}{1.189135in}}%
\pgfpathlineto{\pgfqpoint{1.830014in}{1.185982in}}%
\pgfpathlineto{\pgfqpoint{1.816312in}{1.175524in}}%
\pgfpathlineto{\pgfqpoint{1.814357in}{1.174070in}}%
\pgfpathlineto{\pgfqpoint{1.799200in}{1.161913in}}%
\pgfpathlineto{\pgfqpoint{1.798700in}{1.161517in}}%
\pgfpathlineto{\pgfqpoint{1.783044in}{1.148456in}}%
\pgfpathlineto{\pgfqpoint{1.782865in}{1.148302in}}%
\pgfpathlineto{\pgfqpoint{1.767387in}{1.134951in}}%
\pgfpathlineto{\pgfqpoint{1.767089in}{1.134691in}}%
\pgfpathlineto{\pgfqpoint{1.751801in}{1.121079in}}%
\pgfpathlineto{\pgfqpoint{1.751731in}{1.121014in}}%
\pgfpathlineto{\pgfqpoint{1.737058in}{1.107468in}}%
\pgfpathlineto{\pgfqpoint{1.736074in}{1.106502in}}%
\pgfpathlineto{\pgfqpoint{1.722962in}{1.093857in}}%
\pgfpathlineto{\pgfqpoint{1.720418in}{1.091173in}}%
\pgfpathlineto{\pgfqpoint{1.709717in}{1.080246in}}%
\pgfpathlineto{\pgfqpoint{1.704761in}{1.074511in}}%
\pgfpathlineto{\pgfqpoint{1.697632in}{1.066635in}}%
\pgfpathlineto{\pgfqpoint{1.689104in}{1.055434in}}%
\pgfpathlineto{\pgfqpoint{1.687156in}{1.053024in}}%
\pgfpathlineto{\pgfqpoint{1.678607in}{1.039413in}}%
\pgfpathlineto{\pgfqpoint{1.673448in}{1.027508in}}%
\pgfpathlineto{\pgfqpoint{1.672654in}{1.025802in}}%
\pgfpathlineto{\pgfqpoint{1.669443in}{1.012191in}}%
\pgfpathlineto{\pgfqpoint{1.669443in}{0.998579in}}%
\pgfpathlineto{\pgfqpoint{1.672654in}{0.984968in}}%
\pgfpathlineto{\pgfqpoint{1.673448in}{0.983262in}}%
\pgfpathlineto{\pgfqpoint{1.678607in}{0.971357in}}%
\pgfpathlineto{\pgfqpoint{1.687156in}{0.957746in}}%
\pgfpathlineto{\pgfqpoint{1.689104in}{0.955336in}}%
\pgfpathlineto{\pgfqpoint{1.697632in}{0.944135in}}%
\pgfpathlineto{\pgfqpoint{1.704761in}{0.936259in}}%
\pgfpathlineto{\pgfqpoint{1.709717in}{0.930524in}}%
\pgfpathlineto{\pgfqpoint{1.720418in}{0.919597in}}%
\pgfpathlineto{\pgfqpoint{1.722962in}{0.916913in}}%
\pgfpathlineto{\pgfqpoint{1.736074in}{0.904268in}}%
\pgfpathlineto{\pgfqpoint{1.737058in}{0.903302in}}%
\pgfpathlineto{\pgfqpoint{1.751731in}{0.889756in}}%
\pgfpathlineto{\pgfqpoint{1.751801in}{0.889691in}}%
\pgfpathlineto{\pgfqpoint{1.767089in}{0.876079in}}%
\pgfpathlineto{\pgfqpoint{1.767387in}{0.875819in}}%
\pgfpathlineto{\pgfqpoint{1.782865in}{0.862468in}}%
\pgfpathlineto{\pgfqpoint{1.783044in}{0.862314in}}%
\pgfpathlineto{\pgfqpoint{1.798700in}{0.849253in}}%
\pgfpathlineto{\pgfqpoint{1.799200in}{0.848857in}}%
\pgfpathlineto{\pgfqpoint{1.814357in}{0.836700in}}%
\pgfpathlineto{\pgfqpoint{1.816312in}{0.835246in}}%
\pgfpathlineto{\pgfqpoint{1.830014in}{0.824788in}}%
\pgfpathlineto{\pgfqpoint{1.834608in}{0.821635in}}%
\pgfpathlineto{\pgfqpoint{1.845670in}{0.813738in}}%
\pgfpathlineto{\pgfqpoint{1.854941in}{0.808024in}}%
\pgfpathlineto{\pgfqpoint{1.861327in}{0.803873in}}%
\pgfpathlineto{\pgfqpoint{1.876983in}{0.795598in}}%
\pgfpathlineto{\pgfqpoint{1.880021in}{0.794413in}}%
\pgfpathlineto{\pgfqpoint{1.892640in}{0.789164in}}%
\pgfpathclose%
\pgfpathmoveto{\pgfqpoint{0.389609in}{1.458927in}}%
\pgfpathlineto{\pgfqpoint{0.398752in}{1.461357in}}%
\pgfpathlineto{\pgfqpoint{0.405266in}{1.462973in}}%
\pgfpathlineto{\pgfqpoint{0.420923in}{1.469288in}}%
\pgfpathlineto{\pgfqpoint{0.431364in}{1.474968in}}%
\pgfpathlineto{\pgfqpoint{0.436579in}{1.477644in}}%
\pgfpathlineto{\pgfqpoint{0.452236in}{1.487516in}}%
\pgfpathlineto{\pgfqpoint{0.453684in}{1.488579in}}%
\pgfpathlineto{\pgfqpoint{0.467892in}{1.498530in}}%
\pgfpathlineto{\pgfqpoint{0.472590in}{1.502191in}}%
\pgfpathlineto{\pgfqpoint{0.483549in}{1.510446in}}%
\pgfpathlineto{\pgfqpoint{0.490140in}{1.515802in}}%
\pgfpathlineto{\pgfqpoint{0.499205in}{1.523025in}}%
\pgfpathlineto{\pgfqpoint{0.506827in}{1.529413in}}%
\pgfpathlineto{\pgfqpoint{0.514862in}{1.536116in}}%
\pgfpathlineto{\pgfqpoint{0.522900in}{1.543024in}}%
\pgfpathlineto{\pgfqpoint{0.530519in}{1.549647in}}%
\pgfpathlineto{\pgfqpoint{0.538465in}{1.556635in}}%
\pgfpathlineto{\pgfqpoint{0.546175in}{1.563620in}}%
\pgfpathlineto{\pgfqpoint{0.553523in}{1.570246in}}%
\pgfpathlineto{\pgfqpoint{0.561832in}{1.578127in}}%
\pgfpathlineto{\pgfqpoint{0.567992in}{1.583857in}}%
\pgfpathlineto{\pgfqpoint{0.577488in}{1.593384in}}%
\pgfpathlineto{\pgfqpoint{0.581699in}{1.597468in}}%
\pgfpathlineto{\pgfqpoint{0.593145in}{1.609821in}}%
\pgfpathlineto{\pgfqpoint{0.594368in}{1.611079in}}%
\pgfpathlineto{\pgfqpoint{0.605723in}{1.624691in}}%
\pgfpathlineto{\pgfqpoint{0.608801in}{1.629224in}}%
\pgfpathlineto{\pgfqpoint{0.615336in}{1.638302in}}%
\pgfpathlineto{\pgfqpoint{0.622600in}{1.651913in}}%
\pgfpathlineto{\pgfqpoint{0.624458in}{1.657575in}}%
\pgfpathlineto{\pgfqpoint{0.627253in}{1.665524in}}%
\pgfpathlineto{\pgfqpoint{0.628868in}{1.679135in}}%
\pgfpathlineto{\pgfqpoint{0.624458in}{1.679135in}}%
\pgfpathlineto{\pgfqpoint{0.608801in}{1.679135in}}%
\pgfpathlineto{\pgfqpoint{0.593145in}{1.679135in}}%
\pgfpathlineto{\pgfqpoint{0.577488in}{1.679135in}}%
\pgfpathlineto{\pgfqpoint{0.561832in}{1.679135in}}%
\pgfpathlineto{\pgfqpoint{0.546175in}{1.679135in}}%
\pgfpathlineto{\pgfqpoint{0.530519in}{1.679135in}}%
\pgfpathlineto{\pgfqpoint{0.523403in}{1.679135in}}%
\pgfpathlineto{\pgfqpoint{0.522150in}{1.665524in}}%
\pgfpathlineto{\pgfqpoint{0.518444in}{1.651913in}}%
\pgfpathlineto{\pgfqpoint{0.514862in}{1.643824in}}%
\pgfpathlineto{\pgfqpoint{0.512297in}{1.638302in}}%
\pgfpathlineto{\pgfqpoint{0.503755in}{1.624691in}}%
\pgfpathlineto{\pgfqpoint{0.499205in}{1.618793in}}%
\pgfpathlineto{\pgfqpoint{0.492883in}{1.611079in}}%
\pgfpathlineto{\pgfqpoint{0.483549in}{1.601251in}}%
\pgfpathlineto{\pgfqpoint{0.479649in}{1.597468in}}%
\pgfpathlineto{\pgfqpoint{0.467892in}{1.587247in}}%
\pgfpathlineto{\pgfqpoint{0.463541in}{1.583857in}}%
\pgfpathlineto{\pgfqpoint{0.452236in}{1.575743in}}%
\pgfpathlineto{\pgfqpoint{0.443364in}{1.570246in}}%
\pgfpathlineto{\pgfqpoint{0.436579in}{1.566291in}}%
\pgfpathlineto{\pgfqpoint{0.420923in}{1.558865in}}%
\pgfpathlineto{\pgfqpoint{0.414570in}{1.556635in}}%
\pgfpathlineto{\pgfqpoint{0.405266in}{1.553521in}}%
\pgfpathlineto{\pgfqpoint{0.389609in}{1.550299in}}%
\pgfpathlineto{\pgfqpoint{0.373953in}{1.549210in}}%
\pgfpathlineto{\pgfqpoint{0.373953in}{1.543024in}}%
\pgfpathlineto{\pgfqpoint{0.373953in}{1.529413in}}%
\pgfpathlineto{\pgfqpoint{0.373953in}{1.515802in}}%
\pgfpathlineto{\pgfqpoint{0.373953in}{1.502191in}}%
\pgfpathlineto{\pgfqpoint{0.373953in}{1.488579in}}%
\pgfpathlineto{\pgfqpoint{0.373953in}{1.474968in}}%
\pgfpathlineto{\pgfqpoint{0.373953in}{1.461357in}}%
\pgfpathlineto{\pgfqpoint{0.373953in}{1.457524in}}%
\pgfpathlineto{\pgfqpoint{0.389609in}{1.458927in}}%
\pgfpathclose%
\pgfpathmoveto{\pgfqpoint{1.125468in}{1.460667in}}%
\pgfpathlineto{\pgfqpoint{1.141125in}{1.457875in}}%
\pgfpathlineto{\pgfqpoint{1.156781in}{1.457875in}}%
\pgfpathlineto{\pgfqpoint{1.172438in}{1.460667in}}%
\pgfpathlineto{\pgfqpoint{1.174400in}{1.461357in}}%
\pgfpathlineto{\pgfqpoint{1.188094in}{1.465842in}}%
\pgfpathlineto{\pgfqpoint{1.203751in}{1.473274in}}%
\pgfpathlineto{\pgfqpoint{1.206523in}{1.474968in}}%
\pgfpathlineto{\pgfqpoint{1.219407in}{1.482382in}}%
\pgfpathlineto{\pgfqpoint{1.228467in}{1.488579in}}%
\pgfpathlineto{\pgfqpoint{1.235064in}{1.492888in}}%
\pgfpathlineto{\pgfqpoint{1.247633in}{1.502191in}}%
\pgfpathlineto{\pgfqpoint{1.250721in}{1.504402in}}%
\pgfpathlineto{\pgfqpoint{1.265266in}{1.515802in}}%
\pgfpathlineto{\pgfqpoint{1.266377in}{1.516657in}}%
\pgfpathlineto{\pgfqpoint{1.281959in}{1.529413in}}%
\pgfpathlineto{\pgfqpoint{1.282034in}{1.529474in}}%
\pgfpathlineto{\pgfqpoint{1.297690in}{1.542764in}}%
\pgfpathlineto{\pgfqpoint{1.297990in}{1.543024in}}%
\pgfpathlineto{\pgfqpoint{1.313347in}{1.556479in}}%
\pgfpathlineto{\pgfqpoint{1.313524in}{1.556635in}}%
\pgfpathlineto{\pgfqpoint{1.328548in}{1.570246in}}%
\pgfpathlineto{\pgfqpoint{1.329003in}{1.570680in}}%
\pgfpathlineto{\pgfqpoint{1.342988in}{1.583857in}}%
\pgfpathlineto{\pgfqpoint{1.344660in}{1.585556in}}%
\pgfpathlineto{\pgfqpoint{1.356689in}{1.597468in}}%
\pgfpathlineto{\pgfqpoint{1.360317in}{1.601463in}}%
\pgfpathlineto{\pgfqpoint{1.369401in}{1.611079in}}%
\pgfpathlineto{\pgfqpoint{1.375973in}{1.619139in}}%
\pgfpathlineto{\pgfqpoint{1.380748in}{1.624691in}}%
\pgfpathlineto{\pgfqpoint{1.390266in}{1.638302in}}%
\pgfpathlineto{\pgfqpoint{1.391630in}{1.640943in}}%
\pgfpathlineto{\pgfqpoint{1.397667in}{1.651913in}}%
\pgfpathlineto{\pgfqpoint{1.402283in}{1.665524in}}%
\pgfpathlineto{\pgfqpoint{1.403843in}{1.679135in}}%
\pgfpathlineto{\pgfqpoint{1.391630in}{1.679135in}}%
\pgfpathlineto{\pgfqpoint{1.375973in}{1.679135in}}%
\pgfpathlineto{\pgfqpoint{1.360317in}{1.679135in}}%
\pgfpathlineto{\pgfqpoint{1.344660in}{1.679135in}}%
\pgfpathlineto{\pgfqpoint{1.329003in}{1.679135in}}%
\pgfpathlineto{\pgfqpoint{1.313347in}{1.679135in}}%
\pgfpathlineto{\pgfqpoint{1.298481in}{1.679135in}}%
\pgfpathlineto{\pgfqpoint{1.297690in}{1.670357in}}%
\pgfpathlineto{\pgfqpoint{1.297234in}{1.665524in}}%
\pgfpathlineto{\pgfqpoint{1.293427in}{1.651913in}}%
\pgfpathlineto{\pgfqpoint{1.287262in}{1.638302in}}%
\pgfpathlineto{\pgfqpoint{1.282034in}{1.629749in}}%
\pgfpathlineto{\pgfqpoint{1.278771in}{1.624691in}}%
\pgfpathlineto{\pgfqpoint{1.267990in}{1.611079in}}%
\pgfpathlineto{\pgfqpoint{1.266377in}{1.609330in}}%
\pgfpathlineto{\pgfqpoint{1.254656in}{1.597468in}}%
\pgfpathlineto{\pgfqpoint{1.250721in}{1.593895in}}%
\pgfpathlineto{\pgfqpoint{1.238579in}{1.583857in}}%
\pgfpathlineto{\pgfqpoint{1.235064in}{1.581177in}}%
\pgfpathlineto{\pgfqpoint{1.219407in}{1.570657in}}%
\pgfpathlineto{\pgfqpoint{1.218677in}{1.570246in}}%
\pgfpathlineto{\pgfqpoint{1.203751in}{1.562358in}}%
\pgfpathlineto{\pgfqpoint{1.189968in}{1.556635in}}%
\pgfpathlineto{\pgfqpoint{1.188094in}{1.555891in}}%
\pgfpathlineto{\pgfqpoint{1.172438in}{1.551650in}}%
\pgfpathlineto{\pgfqpoint{1.156781in}{1.549483in}}%
\pgfpathlineto{\pgfqpoint{1.141125in}{1.549483in}}%
\pgfpathlineto{\pgfqpoint{1.125468in}{1.551650in}}%
\pgfpathlineto{\pgfqpoint{1.109812in}{1.555891in}}%
\pgfpathlineto{\pgfqpoint{1.107937in}{1.556635in}}%
\pgfpathlineto{\pgfqpoint{1.094155in}{1.562358in}}%
\pgfpathlineto{\pgfqpoint{1.079228in}{1.570246in}}%
\pgfpathlineto{\pgfqpoint{1.078498in}{1.570657in}}%
\pgfpathlineto{\pgfqpoint{1.062842in}{1.581177in}}%
\pgfpathlineto{\pgfqpoint{1.059327in}{1.583857in}}%
\pgfpathlineto{\pgfqpoint{1.047185in}{1.593895in}}%
\pgfpathlineto{\pgfqpoint{1.043250in}{1.597468in}}%
\pgfpathlineto{\pgfqpoint{1.031529in}{1.609330in}}%
\pgfpathlineto{\pgfqpoint{1.029915in}{1.611079in}}%
\pgfpathlineto{\pgfqpoint{1.019134in}{1.624691in}}%
\pgfpathlineto{\pgfqpoint{1.015872in}{1.629749in}}%
\pgfpathlineto{\pgfqpoint{1.010644in}{1.638302in}}%
\pgfpathlineto{\pgfqpoint{1.004478in}{1.651913in}}%
\pgfpathlineto{\pgfqpoint{1.000672in}{1.665524in}}%
\pgfpathlineto{\pgfqpoint{1.000216in}{1.670357in}}%
\pgfpathlineto{\pgfqpoint{0.999425in}{1.679135in}}%
\pgfpathlineto{\pgfqpoint{0.984559in}{1.679135in}}%
\pgfpathlineto{\pgfqpoint{0.968902in}{1.679135in}}%
\pgfpathlineto{\pgfqpoint{0.953246in}{1.679135in}}%
\pgfpathlineto{\pgfqpoint{0.937589in}{1.679135in}}%
\pgfpathlineto{\pgfqpoint{0.921933in}{1.679135in}}%
\pgfpathlineto{\pgfqpoint{0.906276in}{1.679135in}}%
\pgfpathlineto{\pgfqpoint{0.894062in}{1.679135in}}%
\pgfpathlineto{\pgfqpoint{0.895623in}{1.665524in}}%
\pgfpathlineto{\pgfqpoint{0.900238in}{1.651913in}}%
\pgfpathlineto{\pgfqpoint{0.906276in}{1.640943in}}%
\pgfpathlineto{\pgfqpoint{0.907640in}{1.638302in}}%
\pgfpathlineto{\pgfqpoint{0.917158in}{1.624691in}}%
\pgfpathlineto{\pgfqpoint{0.921933in}{1.619139in}}%
\pgfpathlineto{\pgfqpoint{0.928505in}{1.611079in}}%
\pgfpathlineto{\pgfqpoint{0.937589in}{1.601463in}}%
\pgfpathlineto{\pgfqpoint{0.941216in}{1.597468in}}%
\pgfpathlineto{\pgfqpoint{0.953246in}{1.585556in}}%
\pgfpathlineto{\pgfqpoint{0.954918in}{1.583857in}}%
\pgfpathlineto{\pgfqpoint{0.968902in}{1.570680in}}%
\pgfpathlineto{\pgfqpoint{0.969358in}{1.570246in}}%
\pgfpathlineto{\pgfqpoint{0.984382in}{1.556635in}}%
\pgfpathlineto{\pgfqpoint{0.984559in}{1.556479in}}%
\pgfpathlineto{\pgfqpoint{0.999916in}{1.543024in}}%
\pgfpathlineto{\pgfqpoint{1.000216in}{1.542764in}}%
\pgfpathlineto{\pgfqpoint{1.015872in}{1.529474in}}%
\pgfpathlineto{\pgfqpoint{1.015947in}{1.529413in}}%
\pgfpathlineto{\pgfqpoint{1.031529in}{1.516657in}}%
\pgfpathlineto{\pgfqpoint{1.032640in}{1.515802in}}%
\pgfpathlineto{\pgfqpoint{1.047185in}{1.504402in}}%
\pgfpathlineto{\pgfqpoint{1.050273in}{1.502191in}}%
\pgfpathlineto{\pgfqpoint{1.062842in}{1.492888in}}%
\pgfpathlineto{\pgfqpoint{1.069438in}{1.488579in}}%
\pgfpathlineto{\pgfqpoint{1.078498in}{1.482382in}}%
\pgfpathlineto{\pgfqpoint{1.091383in}{1.474968in}}%
\pgfpathlineto{\pgfqpoint{1.094155in}{1.473274in}}%
\pgfpathlineto{\pgfqpoint{1.109812in}{1.465842in}}%
\pgfpathlineto{\pgfqpoint{1.123506in}{1.461357in}}%
\pgfpathlineto{\pgfqpoint{1.125468in}{1.460667in}}%
\pgfpathclose%
\pgfpathmoveto{\pgfqpoint{1.908296in}{1.458927in}}%
\pgfpathlineto{\pgfqpoint{1.923953in}{1.457524in}}%
\pgfpathlineto{\pgfqpoint{1.923953in}{1.461357in}}%
\pgfpathlineto{\pgfqpoint{1.923953in}{1.474968in}}%
\pgfpathlineto{\pgfqpoint{1.923953in}{1.488579in}}%
\pgfpathlineto{\pgfqpoint{1.923953in}{1.502191in}}%
\pgfpathlineto{\pgfqpoint{1.923953in}{1.515802in}}%
\pgfpathlineto{\pgfqpoint{1.923953in}{1.529413in}}%
\pgfpathlineto{\pgfqpoint{1.923953in}{1.543024in}}%
\pgfpathlineto{\pgfqpoint{1.923953in}{1.549210in}}%
\pgfpathlineto{\pgfqpoint{1.908296in}{1.550299in}}%
\pgfpathlineto{\pgfqpoint{1.892640in}{1.553521in}}%
\pgfpathlineto{\pgfqpoint{1.883336in}{1.556635in}}%
\pgfpathlineto{\pgfqpoint{1.876983in}{1.558865in}}%
\pgfpathlineto{\pgfqpoint{1.861327in}{1.566291in}}%
\pgfpathlineto{\pgfqpoint{1.854542in}{1.570246in}}%
\pgfpathlineto{\pgfqpoint{1.845670in}{1.575743in}}%
\pgfpathlineto{\pgfqpoint{1.834365in}{1.583857in}}%
\pgfpathlineto{\pgfqpoint{1.830014in}{1.587247in}}%
\pgfpathlineto{\pgfqpoint{1.818257in}{1.597468in}}%
\pgfpathlineto{\pgfqpoint{1.814357in}{1.601251in}}%
\pgfpathlineto{\pgfqpoint{1.805023in}{1.611079in}}%
\pgfpathlineto{\pgfqpoint{1.798700in}{1.618793in}}%
\pgfpathlineto{\pgfqpoint{1.794150in}{1.624691in}}%
\pgfpathlineto{\pgfqpoint{1.785609in}{1.638302in}}%
\pgfpathlineto{\pgfqpoint{1.783044in}{1.643824in}}%
\pgfpathlineto{\pgfqpoint{1.779461in}{1.651913in}}%
\pgfpathlineto{\pgfqpoint{1.775756in}{1.665524in}}%
\pgfpathlineto{\pgfqpoint{1.774503in}{1.679135in}}%
\pgfpathlineto{\pgfqpoint{1.767387in}{1.679135in}}%
\pgfpathlineto{\pgfqpoint{1.751731in}{1.679135in}}%
\pgfpathlineto{\pgfqpoint{1.736074in}{1.679135in}}%
\pgfpathlineto{\pgfqpoint{1.720418in}{1.679135in}}%
\pgfpathlineto{\pgfqpoint{1.704761in}{1.679135in}}%
\pgfpathlineto{\pgfqpoint{1.689104in}{1.679135in}}%
\pgfpathlineto{\pgfqpoint{1.673448in}{1.679135in}}%
\pgfpathlineto{\pgfqpoint{1.669038in}{1.679135in}}%
\pgfpathlineto{\pgfqpoint{1.670652in}{1.665524in}}%
\pgfpathlineto{\pgfqpoint{1.673448in}{1.657575in}}%
\pgfpathlineto{\pgfqpoint{1.675306in}{1.651913in}}%
\pgfpathlineto{\pgfqpoint{1.682570in}{1.638302in}}%
\pgfpathlineto{\pgfqpoint{1.689104in}{1.629224in}}%
\pgfpathlineto{\pgfqpoint{1.692182in}{1.624691in}}%
\pgfpathlineto{\pgfqpoint{1.703538in}{1.611079in}}%
\pgfpathlineto{\pgfqpoint{1.704761in}{1.609821in}}%
\pgfpathlineto{\pgfqpoint{1.716207in}{1.597468in}}%
\pgfpathlineto{\pgfqpoint{1.720418in}{1.593384in}}%
\pgfpathlineto{\pgfqpoint{1.729913in}{1.583857in}}%
\pgfpathlineto{\pgfqpoint{1.736074in}{1.578127in}}%
\pgfpathlineto{\pgfqpoint{1.744382in}{1.570246in}}%
\pgfpathlineto{\pgfqpoint{1.751731in}{1.563620in}}%
\pgfpathlineto{\pgfqpoint{1.759441in}{1.556635in}}%
\pgfpathlineto{\pgfqpoint{1.767387in}{1.549647in}}%
\pgfpathlineto{\pgfqpoint{1.775005in}{1.543024in}}%
\pgfpathlineto{\pgfqpoint{1.783044in}{1.536116in}}%
\pgfpathlineto{\pgfqpoint{1.791079in}{1.529413in}}%
\pgfpathlineto{\pgfqpoint{1.798700in}{1.523025in}}%
\pgfpathlineto{\pgfqpoint{1.807766in}{1.515802in}}%
\pgfpathlineto{\pgfqpoint{1.814357in}{1.510446in}}%
\pgfpathlineto{\pgfqpoint{1.825315in}{1.502191in}}%
\pgfpathlineto{\pgfqpoint{1.830014in}{1.498530in}}%
\pgfpathlineto{\pgfqpoint{1.844222in}{1.488579in}}%
\pgfpathlineto{\pgfqpoint{1.845670in}{1.487516in}}%
\pgfpathlineto{\pgfqpoint{1.861327in}{1.477644in}}%
\pgfpathlineto{\pgfqpoint{1.866541in}{1.474968in}}%
\pgfpathlineto{\pgfqpoint{1.876983in}{1.469288in}}%
\pgfpathlineto{\pgfqpoint{1.892640in}{1.462973in}}%
\pgfpathlineto{\pgfqpoint{1.899153in}{1.461357in}}%
\pgfpathlineto{\pgfqpoint{1.908296in}{1.458927in}}%
\pgfpathclose%
\pgfusepath{fill}%
\end{pgfscope}%
\begin{pgfscope}%
\pgfpathrectangle{\pgfqpoint{0.373953in}{0.331635in}}{\pgfqpoint{1.550000in}{1.347500in}}%
\pgfusepath{clip}%
\pgfsetbuttcap%
\pgfsetroundjoin%
\definecolor{currentfill}{rgb}{0.031373,0.261315,0.528120}%
\pgfsetfillcolor{currentfill}%
\pgfsetlinewidth{0.000000pt}%
\definecolor{currentstroke}{rgb}{0.000000,0.000000,0.000000}%
\pgfsetstrokecolor{currentstroke}%
\pgfsetdash{}{0pt}%
\pgfpathmoveto{\pgfqpoint{0.389609in}{0.331635in}}%
\pgfpathlineto{\pgfqpoint{0.405266in}{0.331635in}}%
\pgfpathlineto{\pgfqpoint{0.420923in}{0.331635in}}%
\pgfpathlineto{\pgfqpoint{0.436579in}{0.331635in}}%
\pgfpathlineto{\pgfqpoint{0.452236in}{0.331635in}}%
\pgfpathlineto{\pgfqpoint{0.467892in}{0.331635in}}%
\pgfpathlineto{\pgfqpoint{0.483549in}{0.331635in}}%
\pgfpathlineto{\pgfqpoint{0.499205in}{0.331635in}}%
\pgfpathlineto{\pgfqpoint{0.514862in}{0.331635in}}%
\pgfpathlineto{\pgfqpoint{0.523403in}{0.331635in}}%
\pgfpathlineto{\pgfqpoint{0.522150in}{0.345246in}}%
\pgfpathlineto{\pgfqpoint{0.518444in}{0.358857in}}%
\pgfpathlineto{\pgfqpoint{0.514862in}{0.366946in}}%
\pgfpathlineto{\pgfqpoint{0.512297in}{0.372468in}}%
\pgfpathlineto{\pgfqpoint{0.503755in}{0.386079in}}%
\pgfpathlineto{\pgfqpoint{0.499205in}{0.391977in}}%
\pgfpathlineto{\pgfqpoint{0.492883in}{0.399691in}}%
\pgfpathlineto{\pgfqpoint{0.483549in}{0.409519in}}%
\pgfpathlineto{\pgfqpoint{0.479649in}{0.413302in}}%
\pgfpathlineto{\pgfqpoint{0.467892in}{0.423523in}}%
\pgfpathlineto{\pgfqpoint{0.463541in}{0.426913in}}%
\pgfpathlineto{\pgfqpoint{0.452236in}{0.435027in}}%
\pgfpathlineto{\pgfqpoint{0.443364in}{0.440524in}}%
\pgfpathlineto{\pgfqpoint{0.436579in}{0.444479in}}%
\pgfpathlineto{\pgfqpoint{0.420923in}{0.451905in}}%
\pgfpathlineto{\pgfqpoint{0.414570in}{0.454135in}}%
\pgfpathlineto{\pgfqpoint{0.405266in}{0.457249in}}%
\pgfpathlineto{\pgfqpoint{0.389609in}{0.460471in}}%
\pgfpathlineto{\pgfqpoint{0.373953in}{0.461560in}}%
\pgfpathlineto{\pgfqpoint{0.373953in}{0.454135in}}%
\pgfpathlineto{\pgfqpoint{0.373953in}{0.440524in}}%
\pgfpathlineto{\pgfqpoint{0.373953in}{0.426913in}}%
\pgfpathlineto{\pgfqpoint{0.373953in}{0.413302in}}%
\pgfpathlineto{\pgfqpoint{0.373953in}{0.399691in}}%
\pgfpathlineto{\pgfqpoint{0.373953in}{0.386079in}}%
\pgfpathlineto{\pgfqpoint{0.373953in}{0.372468in}}%
\pgfpathlineto{\pgfqpoint{0.373953in}{0.358857in}}%
\pgfpathlineto{\pgfqpoint{0.373953in}{0.345246in}}%
\pgfpathlineto{\pgfqpoint{0.373953in}{0.331635in}}%
\pgfpathlineto{\pgfqpoint{0.389609in}{0.331635in}}%
\pgfpathclose%
\pgfpathmoveto{\pgfqpoint{1.000216in}{0.331635in}}%
\pgfpathlineto{\pgfqpoint{1.015872in}{0.331635in}}%
\pgfpathlineto{\pgfqpoint{1.031529in}{0.331635in}}%
\pgfpathlineto{\pgfqpoint{1.047185in}{0.331635in}}%
\pgfpathlineto{\pgfqpoint{1.062842in}{0.331635in}}%
\pgfpathlineto{\pgfqpoint{1.078498in}{0.331635in}}%
\pgfpathlineto{\pgfqpoint{1.094155in}{0.331635in}}%
\pgfpathlineto{\pgfqpoint{1.109812in}{0.331635in}}%
\pgfpathlineto{\pgfqpoint{1.125468in}{0.331635in}}%
\pgfpathlineto{\pgfqpoint{1.141125in}{0.331635in}}%
\pgfpathlineto{\pgfqpoint{1.156781in}{0.331635in}}%
\pgfpathlineto{\pgfqpoint{1.172438in}{0.331635in}}%
\pgfpathlineto{\pgfqpoint{1.188094in}{0.331635in}}%
\pgfpathlineto{\pgfqpoint{1.203751in}{0.331635in}}%
\pgfpathlineto{\pgfqpoint{1.219407in}{0.331635in}}%
\pgfpathlineto{\pgfqpoint{1.235064in}{0.331635in}}%
\pgfpathlineto{\pgfqpoint{1.250721in}{0.331635in}}%
\pgfpathlineto{\pgfqpoint{1.266377in}{0.331635in}}%
\pgfpathlineto{\pgfqpoint{1.282034in}{0.331635in}}%
\pgfpathlineto{\pgfqpoint{1.297690in}{0.331635in}}%
\pgfpathlineto{\pgfqpoint{1.298481in}{0.331635in}}%
\pgfpathlineto{\pgfqpoint{1.297690in}{0.340413in}}%
\pgfpathlineto{\pgfqpoint{1.297234in}{0.345246in}}%
\pgfpathlineto{\pgfqpoint{1.293427in}{0.358857in}}%
\pgfpathlineto{\pgfqpoint{1.287262in}{0.372468in}}%
\pgfpathlineto{\pgfqpoint{1.282034in}{0.381021in}}%
\pgfpathlineto{\pgfqpoint{1.278771in}{0.386079in}}%
\pgfpathlineto{\pgfqpoint{1.267990in}{0.399691in}}%
\pgfpathlineto{\pgfqpoint{1.266377in}{0.401440in}}%
\pgfpathlineto{\pgfqpoint{1.254656in}{0.413302in}}%
\pgfpathlineto{\pgfqpoint{1.250721in}{0.416875in}}%
\pgfpathlineto{\pgfqpoint{1.238579in}{0.426913in}}%
\pgfpathlineto{\pgfqpoint{1.235064in}{0.429593in}}%
\pgfpathlineto{\pgfqpoint{1.219407in}{0.440113in}}%
\pgfpathlineto{\pgfqpoint{1.218677in}{0.440524in}}%
\pgfpathlineto{\pgfqpoint{1.203751in}{0.448412in}}%
\pgfpathlineto{\pgfqpoint{1.189968in}{0.454135in}}%
\pgfpathlineto{\pgfqpoint{1.188094in}{0.454879in}}%
\pgfpathlineto{\pgfqpoint{1.172438in}{0.459120in}}%
\pgfpathlineto{\pgfqpoint{1.156781in}{0.461287in}}%
\pgfpathlineto{\pgfqpoint{1.141125in}{0.461287in}}%
\pgfpathlineto{\pgfqpoint{1.125468in}{0.459120in}}%
\pgfpathlineto{\pgfqpoint{1.109812in}{0.454879in}}%
\pgfpathlineto{\pgfqpoint{1.107937in}{0.454135in}}%
\pgfpathlineto{\pgfqpoint{1.094155in}{0.448412in}}%
\pgfpathlineto{\pgfqpoint{1.079228in}{0.440524in}}%
\pgfpathlineto{\pgfqpoint{1.078498in}{0.440113in}}%
\pgfpathlineto{\pgfqpoint{1.062842in}{0.429593in}}%
\pgfpathlineto{\pgfqpoint{1.059327in}{0.426913in}}%
\pgfpathlineto{\pgfqpoint{1.047185in}{0.416875in}}%
\pgfpathlineto{\pgfqpoint{1.043250in}{0.413302in}}%
\pgfpathlineto{\pgfqpoint{1.031529in}{0.401440in}}%
\pgfpathlineto{\pgfqpoint{1.029915in}{0.399691in}}%
\pgfpathlineto{\pgfqpoint{1.019134in}{0.386079in}}%
\pgfpathlineto{\pgfqpoint{1.015872in}{0.381021in}}%
\pgfpathlineto{\pgfqpoint{1.010644in}{0.372468in}}%
\pgfpathlineto{\pgfqpoint{1.004478in}{0.358857in}}%
\pgfpathlineto{\pgfqpoint{1.000672in}{0.345246in}}%
\pgfpathlineto{\pgfqpoint{1.000216in}{0.340413in}}%
\pgfpathlineto{\pgfqpoint{0.999425in}{0.331635in}}%
\pgfpathlineto{\pgfqpoint{1.000216in}{0.331635in}}%
\pgfpathclose%
\pgfpathmoveto{\pgfqpoint{1.783044in}{0.331635in}}%
\pgfpathlineto{\pgfqpoint{1.798700in}{0.331635in}}%
\pgfpathlineto{\pgfqpoint{1.814357in}{0.331635in}}%
\pgfpathlineto{\pgfqpoint{1.830014in}{0.331635in}}%
\pgfpathlineto{\pgfqpoint{1.845670in}{0.331635in}}%
\pgfpathlineto{\pgfqpoint{1.861327in}{0.331635in}}%
\pgfpathlineto{\pgfqpoint{1.876983in}{0.331635in}}%
\pgfpathlineto{\pgfqpoint{1.892640in}{0.331635in}}%
\pgfpathlineto{\pgfqpoint{1.908296in}{0.331635in}}%
\pgfpathlineto{\pgfqpoint{1.923953in}{0.331635in}}%
\pgfpathlineto{\pgfqpoint{1.923953in}{0.345246in}}%
\pgfpathlineto{\pgfqpoint{1.923953in}{0.358857in}}%
\pgfpathlineto{\pgfqpoint{1.923953in}{0.372468in}}%
\pgfpathlineto{\pgfqpoint{1.923953in}{0.386079in}}%
\pgfpathlineto{\pgfqpoint{1.923953in}{0.399691in}}%
\pgfpathlineto{\pgfqpoint{1.923953in}{0.413302in}}%
\pgfpathlineto{\pgfqpoint{1.923953in}{0.426913in}}%
\pgfpathlineto{\pgfqpoint{1.923953in}{0.440524in}}%
\pgfpathlineto{\pgfqpoint{1.923953in}{0.454135in}}%
\pgfpathlineto{\pgfqpoint{1.923953in}{0.461560in}}%
\pgfpathlineto{\pgfqpoint{1.908296in}{0.460471in}}%
\pgfpathlineto{\pgfqpoint{1.892640in}{0.457249in}}%
\pgfpathlineto{\pgfqpoint{1.883336in}{0.454135in}}%
\pgfpathlineto{\pgfqpoint{1.876983in}{0.451905in}}%
\pgfpathlineto{\pgfqpoint{1.861327in}{0.444479in}}%
\pgfpathlineto{\pgfqpoint{1.854542in}{0.440524in}}%
\pgfpathlineto{\pgfqpoint{1.845670in}{0.435027in}}%
\pgfpathlineto{\pgfqpoint{1.834365in}{0.426913in}}%
\pgfpathlineto{\pgfqpoint{1.830014in}{0.423523in}}%
\pgfpathlineto{\pgfqpoint{1.818257in}{0.413302in}}%
\pgfpathlineto{\pgfqpoint{1.814357in}{0.409519in}}%
\pgfpathlineto{\pgfqpoint{1.805023in}{0.399691in}}%
\pgfpathlineto{\pgfqpoint{1.798700in}{0.391977in}}%
\pgfpathlineto{\pgfqpoint{1.794150in}{0.386079in}}%
\pgfpathlineto{\pgfqpoint{1.785609in}{0.372468in}}%
\pgfpathlineto{\pgfqpoint{1.783044in}{0.366946in}}%
\pgfpathlineto{\pgfqpoint{1.779461in}{0.358857in}}%
\pgfpathlineto{\pgfqpoint{1.775756in}{0.345246in}}%
\pgfpathlineto{\pgfqpoint{1.774503in}{0.331635in}}%
\pgfpathlineto{\pgfqpoint{1.783044in}{0.331635in}}%
\pgfpathclose%
\pgfpathmoveto{\pgfqpoint{0.384050in}{0.876079in}}%
\pgfpathlineto{\pgfqpoint{0.389609in}{0.876476in}}%
\pgfpathlineto{\pgfqpoint{0.405266in}{0.879785in}}%
\pgfpathlineto{\pgfqpoint{0.420923in}{0.885145in}}%
\pgfpathlineto{\pgfqpoint{0.430761in}{0.889691in}}%
\pgfpathlineto{\pgfqpoint{0.436579in}{0.892527in}}%
\pgfpathlineto{\pgfqpoint{0.452236in}{0.901899in}}%
\pgfpathlineto{\pgfqpoint{0.454249in}{0.903302in}}%
\pgfpathlineto{\pgfqpoint{0.467892in}{0.913491in}}%
\pgfpathlineto{\pgfqpoint{0.472002in}{0.916913in}}%
\pgfpathlineto{\pgfqpoint{0.483549in}{0.927468in}}%
\pgfpathlineto{\pgfqpoint{0.486632in}{0.930524in}}%
\pgfpathlineto{\pgfqpoint{0.498733in}{0.944135in}}%
\pgfpathlineto{\pgfqpoint{0.499205in}{0.944770in}}%
\pgfpathlineto{\pgfqpoint{0.508279in}{0.957746in}}%
\pgfpathlineto{\pgfqpoint{0.514862in}{0.969728in}}%
\pgfpathlineto{\pgfqpoint{0.515717in}{0.971357in}}%
\pgfpathlineto{\pgfqpoint{0.520596in}{0.984968in}}%
\pgfpathlineto{\pgfqpoint{0.523089in}{0.998579in}}%
\pgfpathlineto{\pgfqpoint{0.523089in}{1.012191in}}%
\pgfpathlineto{\pgfqpoint{0.520596in}{1.025802in}}%
\pgfpathlineto{\pgfqpoint{0.515717in}{1.039413in}}%
\pgfpathlineto{\pgfqpoint{0.514862in}{1.041042in}}%
\pgfpathlineto{\pgfqpoint{0.508279in}{1.053024in}}%
\pgfpathlineto{\pgfqpoint{0.499205in}{1.066000in}}%
\pgfpathlineto{\pgfqpoint{0.498733in}{1.066635in}}%
\pgfpathlineto{\pgfqpoint{0.486632in}{1.080246in}}%
\pgfpathlineto{\pgfqpoint{0.483549in}{1.083302in}}%
\pgfpathlineto{\pgfqpoint{0.472002in}{1.093857in}}%
\pgfpathlineto{\pgfqpoint{0.467892in}{1.097279in}}%
\pgfpathlineto{\pgfqpoint{0.454249in}{1.107468in}}%
\pgfpathlineto{\pgfqpoint{0.452236in}{1.108871in}}%
\pgfpathlineto{\pgfqpoint{0.436579in}{1.118243in}}%
\pgfpathlineto{\pgfqpoint{0.430761in}{1.121079in}}%
\pgfpathlineto{\pgfqpoint{0.420923in}{1.125625in}}%
\pgfpathlineto{\pgfqpoint{0.405266in}{1.130985in}}%
\pgfpathlineto{\pgfqpoint{0.389609in}{1.134294in}}%
\pgfpathlineto{\pgfqpoint{0.384050in}{1.134691in}}%
\pgfpathlineto{\pgfqpoint{0.373953in}{1.135378in}}%
\pgfpathlineto{\pgfqpoint{0.373953in}{1.134691in}}%
\pgfpathlineto{\pgfqpoint{0.373953in}{1.121079in}}%
\pgfpathlineto{\pgfqpoint{0.373953in}{1.107468in}}%
\pgfpathlineto{\pgfqpoint{0.373953in}{1.093857in}}%
\pgfpathlineto{\pgfqpoint{0.373953in}{1.080246in}}%
\pgfpathlineto{\pgfqpoint{0.373953in}{1.066635in}}%
\pgfpathlineto{\pgfqpoint{0.373953in}{1.053024in}}%
\pgfpathlineto{\pgfqpoint{0.373953in}{1.039413in}}%
\pgfpathlineto{\pgfqpoint{0.373953in}{1.025802in}}%
\pgfpathlineto{\pgfqpoint{0.373953in}{1.012191in}}%
\pgfpathlineto{\pgfqpoint{0.373953in}{0.998579in}}%
\pgfpathlineto{\pgfqpoint{0.373953in}{0.984968in}}%
\pgfpathlineto{\pgfqpoint{0.373953in}{0.971357in}}%
\pgfpathlineto{\pgfqpoint{0.373953in}{0.957746in}}%
\pgfpathlineto{\pgfqpoint{0.373953in}{0.944135in}}%
\pgfpathlineto{\pgfqpoint{0.373953in}{0.930524in}}%
\pgfpathlineto{\pgfqpoint{0.373953in}{0.916913in}}%
\pgfpathlineto{\pgfqpoint{0.373953in}{0.903302in}}%
\pgfpathlineto{\pgfqpoint{0.373953in}{0.889691in}}%
\pgfpathlineto{\pgfqpoint{0.373953in}{0.876079in}}%
\pgfpathlineto{\pgfqpoint{0.373953in}{0.875392in}}%
\pgfpathlineto{\pgfqpoint{0.384050in}{0.876079in}}%
\pgfpathclose%
\pgfpathmoveto{\pgfqpoint{1.141125in}{0.875659in}}%
\pgfpathlineto{\pgfqpoint{1.156781in}{0.875659in}}%
\pgfpathlineto{\pgfqpoint{1.159877in}{0.876079in}}%
\pgfpathlineto{\pgfqpoint{1.172438in}{0.877864in}}%
\pgfpathlineto{\pgfqpoint{1.188094in}{0.882221in}}%
\pgfpathlineto{\pgfqpoint{1.203751in}{0.888529in}}%
\pgfpathlineto{\pgfqpoint{1.205989in}{0.889691in}}%
\pgfpathlineto{\pgfqpoint{1.219407in}{0.897025in}}%
\pgfpathlineto{\pgfqpoint{1.229080in}{0.903302in}}%
\pgfpathlineto{\pgfqpoint{1.235064in}{0.907470in}}%
\pgfpathlineto{\pgfqpoint{1.247007in}{0.916913in}}%
\pgfpathlineto{\pgfqpoint{1.250721in}{0.920141in}}%
\pgfpathlineto{\pgfqpoint{1.261583in}{0.930524in}}%
\pgfpathlineto{\pgfqpoint{1.266377in}{0.935726in}}%
\pgfpathlineto{\pgfqpoint{1.273597in}{0.944135in}}%
\pgfpathlineto{\pgfqpoint{1.282034in}{0.955800in}}%
\pgfpathlineto{\pgfqpoint{1.283370in}{0.957746in}}%
\pgfpathlineto{\pgfqpoint{1.290626in}{0.971357in}}%
\pgfpathlineto{\pgfqpoint{1.295638in}{0.984968in}}%
\pgfpathlineto{\pgfqpoint{1.297690in}{0.995888in}}%
\pgfpathlineto{\pgfqpoint{1.298174in}{0.998579in}}%
\pgfpathlineto{\pgfqpoint{1.298174in}{1.012191in}}%
\pgfpathlineto{\pgfqpoint{1.297690in}{1.014882in}}%
\pgfpathlineto{\pgfqpoint{1.295638in}{1.025802in}}%
\pgfpathlineto{\pgfqpoint{1.290626in}{1.039413in}}%
\pgfpathlineto{\pgfqpoint{1.283370in}{1.053024in}}%
\pgfpathlineto{\pgfqpoint{1.282034in}{1.054970in}}%
\pgfpathlineto{\pgfqpoint{1.273597in}{1.066635in}}%
\pgfpathlineto{\pgfqpoint{1.266377in}{1.075044in}}%
\pgfpathlineto{\pgfqpoint{1.261583in}{1.080246in}}%
\pgfpathlineto{\pgfqpoint{1.250721in}{1.090629in}}%
\pgfpathlineto{\pgfqpoint{1.247007in}{1.093857in}}%
\pgfpathlineto{\pgfqpoint{1.235064in}{1.103300in}}%
\pgfpathlineto{\pgfqpoint{1.229080in}{1.107468in}}%
\pgfpathlineto{\pgfqpoint{1.219407in}{1.113745in}}%
\pgfpathlineto{\pgfqpoint{1.205989in}{1.121079in}}%
\pgfpathlineto{\pgfqpoint{1.203751in}{1.122241in}}%
\pgfpathlineto{\pgfqpoint{1.188094in}{1.128549in}}%
\pgfpathlineto{\pgfqpoint{1.172438in}{1.132906in}}%
\pgfpathlineto{\pgfqpoint{1.159877in}{1.134691in}}%
\pgfpathlineto{\pgfqpoint{1.156781in}{1.135111in}}%
\pgfpathlineto{\pgfqpoint{1.141125in}{1.135111in}}%
\pgfpathlineto{\pgfqpoint{1.138029in}{1.134691in}}%
\pgfpathlineto{\pgfqpoint{1.125468in}{1.132906in}}%
\pgfpathlineto{\pgfqpoint{1.109812in}{1.128549in}}%
\pgfpathlineto{\pgfqpoint{1.094155in}{1.122241in}}%
\pgfpathlineto{\pgfqpoint{1.091916in}{1.121079in}}%
\pgfpathlineto{\pgfqpoint{1.078498in}{1.113745in}}%
\pgfpathlineto{\pgfqpoint{1.068826in}{1.107468in}}%
\pgfpathlineto{\pgfqpoint{1.062842in}{1.103300in}}%
\pgfpathlineto{\pgfqpoint{1.050899in}{1.093857in}}%
\pgfpathlineto{\pgfqpoint{1.047185in}{1.090629in}}%
\pgfpathlineto{\pgfqpoint{1.036323in}{1.080246in}}%
\pgfpathlineto{\pgfqpoint{1.031529in}{1.075044in}}%
\pgfpathlineto{\pgfqpoint{1.024309in}{1.066635in}}%
\pgfpathlineto{\pgfqpoint{1.015872in}{1.054970in}}%
\pgfpathlineto{\pgfqpoint{1.014536in}{1.053024in}}%
\pgfpathlineto{\pgfqpoint{1.007280in}{1.039413in}}%
\pgfpathlineto{\pgfqpoint{1.002268in}{1.025802in}}%
\pgfpathlineto{\pgfqpoint{1.000216in}{1.014882in}}%
\pgfpathlineto{\pgfqpoint{0.999732in}{1.012191in}}%
\pgfpathlineto{\pgfqpoint{0.999732in}{0.998579in}}%
\pgfpathlineto{\pgfqpoint{1.000216in}{0.995888in}}%
\pgfpathlineto{\pgfqpoint{1.002268in}{0.984968in}}%
\pgfpathlineto{\pgfqpoint{1.007280in}{0.971357in}}%
\pgfpathlineto{\pgfqpoint{1.014536in}{0.957746in}}%
\pgfpathlineto{\pgfqpoint{1.015872in}{0.955800in}}%
\pgfpathlineto{\pgfqpoint{1.024309in}{0.944135in}}%
\pgfpathlineto{\pgfqpoint{1.031529in}{0.935726in}}%
\pgfpathlineto{\pgfqpoint{1.036323in}{0.930524in}}%
\pgfpathlineto{\pgfqpoint{1.047185in}{0.920141in}}%
\pgfpathlineto{\pgfqpoint{1.050899in}{0.916913in}}%
\pgfpathlineto{\pgfqpoint{1.062842in}{0.907470in}}%
\pgfpathlineto{\pgfqpoint{1.068826in}{0.903302in}}%
\pgfpathlineto{\pgfqpoint{1.078498in}{0.897025in}}%
\pgfpathlineto{\pgfqpoint{1.091916in}{0.889691in}}%
\pgfpathlineto{\pgfqpoint{1.094155in}{0.888529in}}%
\pgfpathlineto{\pgfqpoint{1.109812in}{0.882221in}}%
\pgfpathlineto{\pgfqpoint{1.125468in}{0.877864in}}%
\pgfpathlineto{\pgfqpoint{1.138029in}{0.876079in}}%
\pgfpathlineto{\pgfqpoint{1.141125in}{0.875659in}}%
\pgfpathclose%
\pgfpathmoveto{\pgfqpoint{1.923953in}{0.875392in}}%
\pgfpathlineto{\pgfqpoint{1.923953in}{0.876079in}}%
\pgfpathlineto{\pgfqpoint{1.923953in}{0.889691in}}%
\pgfpathlineto{\pgfqpoint{1.923953in}{0.903302in}}%
\pgfpathlineto{\pgfqpoint{1.923953in}{0.916913in}}%
\pgfpathlineto{\pgfqpoint{1.923953in}{0.930524in}}%
\pgfpathlineto{\pgfqpoint{1.923953in}{0.944135in}}%
\pgfpathlineto{\pgfqpoint{1.923953in}{0.957746in}}%
\pgfpathlineto{\pgfqpoint{1.923953in}{0.971357in}}%
\pgfpathlineto{\pgfqpoint{1.923953in}{0.984968in}}%
\pgfpathlineto{\pgfqpoint{1.923953in}{0.998579in}}%
\pgfpathlineto{\pgfqpoint{1.923953in}{1.012191in}}%
\pgfpathlineto{\pgfqpoint{1.923953in}{1.025802in}}%
\pgfpathlineto{\pgfqpoint{1.923953in}{1.039413in}}%
\pgfpathlineto{\pgfqpoint{1.923953in}{1.053024in}}%
\pgfpathlineto{\pgfqpoint{1.923953in}{1.066635in}}%
\pgfpathlineto{\pgfqpoint{1.923953in}{1.080246in}}%
\pgfpathlineto{\pgfqpoint{1.923953in}{1.093857in}}%
\pgfpathlineto{\pgfqpoint{1.923953in}{1.107468in}}%
\pgfpathlineto{\pgfqpoint{1.923953in}{1.121079in}}%
\pgfpathlineto{\pgfqpoint{1.923953in}{1.134691in}}%
\pgfpathlineto{\pgfqpoint{1.923953in}{1.135378in}}%
\pgfpathlineto{\pgfqpoint{1.913856in}{1.134691in}}%
\pgfpathlineto{\pgfqpoint{1.908296in}{1.134294in}}%
\pgfpathlineto{\pgfqpoint{1.892640in}{1.130985in}}%
\pgfpathlineto{\pgfqpoint{1.876983in}{1.125625in}}%
\pgfpathlineto{\pgfqpoint{1.867145in}{1.121079in}}%
\pgfpathlineto{\pgfqpoint{1.861327in}{1.118243in}}%
\pgfpathlineto{\pgfqpoint{1.845670in}{1.108871in}}%
\pgfpathlineto{\pgfqpoint{1.843657in}{1.107468in}}%
\pgfpathlineto{\pgfqpoint{1.830014in}{1.097279in}}%
\pgfpathlineto{\pgfqpoint{1.825904in}{1.093857in}}%
\pgfpathlineto{\pgfqpoint{1.814357in}{1.083302in}}%
\pgfpathlineto{\pgfqpoint{1.811274in}{1.080246in}}%
\pgfpathlineto{\pgfqpoint{1.799173in}{1.066635in}}%
\pgfpathlineto{\pgfqpoint{1.798700in}{1.066000in}}%
\pgfpathlineto{\pgfqpoint{1.789627in}{1.053024in}}%
\pgfpathlineto{\pgfqpoint{1.783044in}{1.041042in}}%
\pgfpathlineto{\pgfqpoint{1.782188in}{1.039413in}}%
\pgfpathlineto{\pgfqpoint{1.777310in}{1.025802in}}%
\pgfpathlineto{\pgfqpoint{1.774817in}{1.012191in}}%
\pgfpathlineto{\pgfqpoint{1.774817in}{0.998579in}}%
\pgfpathlineto{\pgfqpoint{1.777310in}{0.984968in}}%
\pgfpathlineto{\pgfqpoint{1.782188in}{0.971357in}}%
\pgfpathlineto{\pgfqpoint{1.783044in}{0.969728in}}%
\pgfpathlineto{\pgfqpoint{1.789627in}{0.957746in}}%
\pgfpathlineto{\pgfqpoint{1.798700in}{0.944770in}}%
\pgfpathlineto{\pgfqpoint{1.799173in}{0.944135in}}%
\pgfpathlineto{\pgfqpoint{1.811274in}{0.930524in}}%
\pgfpathlineto{\pgfqpoint{1.814357in}{0.927468in}}%
\pgfpathlineto{\pgfqpoint{1.825904in}{0.916913in}}%
\pgfpathlineto{\pgfqpoint{1.830014in}{0.913491in}}%
\pgfpathlineto{\pgfqpoint{1.843657in}{0.903302in}}%
\pgfpathlineto{\pgfqpoint{1.845670in}{0.901899in}}%
\pgfpathlineto{\pgfqpoint{1.861327in}{0.892527in}}%
\pgfpathlineto{\pgfqpoint{1.867145in}{0.889691in}}%
\pgfpathlineto{\pgfqpoint{1.876983in}{0.885145in}}%
\pgfpathlineto{\pgfqpoint{1.892640in}{0.879785in}}%
\pgfpathlineto{\pgfqpoint{1.908296in}{0.876476in}}%
\pgfpathlineto{\pgfqpoint{1.913856in}{0.876079in}}%
\pgfpathlineto{\pgfqpoint{1.923953in}{0.875392in}}%
\pgfpathclose%
\pgfpathmoveto{\pgfqpoint{0.389609in}{1.550299in}}%
\pgfpathlineto{\pgfqpoint{0.405266in}{1.553521in}}%
\pgfpathlineto{\pgfqpoint{0.414570in}{1.556635in}}%
\pgfpathlineto{\pgfqpoint{0.420923in}{1.558865in}}%
\pgfpathlineto{\pgfqpoint{0.436579in}{1.566291in}}%
\pgfpathlineto{\pgfqpoint{0.443364in}{1.570246in}}%
\pgfpathlineto{\pgfqpoint{0.452236in}{1.575743in}}%
\pgfpathlineto{\pgfqpoint{0.463541in}{1.583857in}}%
\pgfpathlineto{\pgfqpoint{0.467892in}{1.587247in}}%
\pgfpathlineto{\pgfqpoint{0.479649in}{1.597468in}}%
\pgfpathlineto{\pgfqpoint{0.483549in}{1.601251in}}%
\pgfpathlineto{\pgfqpoint{0.492883in}{1.611079in}}%
\pgfpathlineto{\pgfqpoint{0.499205in}{1.618793in}}%
\pgfpathlineto{\pgfqpoint{0.503755in}{1.624691in}}%
\pgfpathlineto{\pgfqpoint{0.512297in}{1.638302in}}%
\pgfpathlineto{\pgfqpoint{0.514862in}{1.643824in}}%
\pgfpathlineto{\pgfqpoint{0.518444in}{1.651913in}}%
\pgfpathlineto{\pgfqpoint{0.522150in}{1.665524in}}%
\pgfpathlineto{\pgfqpoint{0.523403in}{1.679135in}}%
\pgfpathlineto{\pgfqpoint{0.514862in}{1.679135in}}%
\pgfpathlineto{\pgfqpoint{0.499205in}{1.679135in}}%
\pgfpathlineto{\pgfqpoint{0.483549in}{1.679135in}}%
\pgfpathlineto{\pgfqpoint{0.467892in}{1.679135in}}%
\pgfpathlineto{\pgfqpoint{0.452236in}{1.679135in}}%
\pgfpathlineto{\pgfqpoint{0.436579in}{1.679135in}}%
\pgfpathlineto{\pgfqpoint{0.420923in}{1.679135in}}%
\pgfpathlineto{\pgfqpoint{0.405266in}{1.679135in}}%
\pgfpathlineto{\pgfqpoint{0.389609in}{1.679135in}}%
\pgfpathlineto{\pgfqpoint{0.373953in}{1.679135in}}%
\pgfpathlineto{\pgfqpoint{0.373953in}{1.665524in}}%
\pgfpathlineto{\pgfqpoint{0.373953in}{1.651913in}}%
\pgfpathlineto{\pgfqpoint{0.373953in}{1.638302in}}%
\pgfpathlineto{\pgfqpoint{0.373953in}{1.624691in}}%
\pgfpathlineto{\pgfqpoint{0.373953in}{1.611079in}}%
\pgfpathlineto{\pgfqpoint{0.373953in}{1.597468in}}%
\pgfpathlineto{\pgfqpoint{0.373953in}{1.583857in}}%
\pgfpathlineto{\pgfqpoint{0.373953in}{1.570246in}}%
\pgfpathlineto{\pgfqpoint{0.373953in}{1.556635in}}%
\pgfpathlineto{\pgfqpoint{0.373953in}{1.549210in}}%
\pgfpathlineto{\pgfqpoint{0.389609in}{1.550299in}}%
\pgfpathclose%
\pgfpathmoveto{\pgfqpoint{1.109812in}{1.555891in}}%
\pgfpathlineto{\pgfqpoint{1.125468in}{1.551650in}}%
\pgfpathlineto{\pgfqpoint{1.141125in}{1.549483in}}%
\pgfpathlineto{\pgfqpoint{1.156781in}{1.549483in}}%
\pgfpathlineto{\pgfqpoint{1.172438in}{1.551650in}}%
\pgfpathlineto{\pgfqpoint{1.188094in}{1.555891in}}%
\pgfpathlineto{\pgfqpoint{1.189968in}{1.556635in}}%
\pgfpathlineto{\pgfqpoint{1.203751in}{1.562358in}}%
\pgfpathlineto{\pgfqpoint{1.218677in}{1.570246in}}%
\pgfpathlineto{\pgfqpoint{1.219407in}{1.570657in}}%
\pgfpathlineto{\pgfqpoint{1.235064in}{1.581177in}}%
\pgfpathlineto{\pgfqpoint{1.238579in}{1.583857in}}%
\pgfpathlineto{\pgfqpoint{1.250721in}{1.593895in}}%
\pgfpathlineto{\pgfqpoint{1.254656in}{1.597468in}}%
\pgfpathlineto{\pgfqpoint{1.266377in}{1.609330in}}%
\pgfpathlineto{\pgfqpoint{1.267990in}{1.611079in}}%
\pgfpathlineto{\pgfqpoint{1.278771in}{1.624691in}}%
\pgfpathlineto{\pgfqpoint{1.282034in}{1.629749in}}%
\pgfpathlineto{\pgfqpoint{1.287262in}{1.638302in}}%
\pgfpathlineto{\pgfqpoint{1.293427in}{1.651913in}}%
\pgfpathlineto{\pgfqpoint{1.297234in}{1.665524in}}%
\pgfpathlineto{\pgfqpoint{1.297690in}{1.670357in}}%
\pgfpathlineto{\pgfqpoint{1.298481in}{1.679135in}}%
\pgfpathlineto{\pgfqpoint{1.297690in}{1.679135in}}%
\pgfpathlineto{\pgfqpoint{1.282034in}{1.679135in}}%
\pgfpathlineto{\pgfqpoint{1.266377in}{1.679135in}}%
\pgfpathlineto{\pgfqpoint{1.250721in}{1.679135in}}%
\pgfpathlineto{\pgfqpoint{1.235064in}{1.679135in}}%
\pgfpathlineto{\pgfqpoint{1.219407in}{1.679135in}}%
\pgfpathlineto{\pgfqpoint{1.203751in}{1.679135in}}%
\pgfpathlineto{\pgfqpoint{1.188094in}{1.679135in}}%
\pgfpathlineto{\pgfqpoint{1.172438in}{1.679135in}}%
\pgfpathlineto{\pgfqpoint{1.156781in}{1.679135in}}%
\pgfpathlineto{\pgfqpoint{1.141125in}{1.679135in}}%
\pgfpathlineto{\pgfqpoint{1.125468in}{1.679135in}}%
\pgfpathlineto{\pgfqpoint{1.109812in}{1.679135in}}%
\pgfpathlineto{\pgfqpoint{1.094155in}{1.679135in}}%
\pgfpathlineto{\pgfqpoint{1.078498in}{1.679135in}}%
\pgfpathlineto{\pgfqpoint{1.062842in}{1.679135in}}%
\pgfpathlineto{\pgfqpoint{1.047185in}{1.679135in}}%
\pgfpathlineto{\pgfqpoint{1.031529in}{1.679135in}}%
\pgfpathlineto{\pgfqpoint{1.015872in}{1.679135in}}%
\pgfpathlineto{\pgfqpoint{1.000216in}{1.679135in}}%
\pgfpathlineto{\pgfqpoint{0.999425in}{1.679135in}}%
\pgfpathlineto{\pgfqpoint{1.000216in}{1.670357in}}%
\pgfpathlineto{\pgfqpoint{1.000672in}{1.665524in}}%
\pgfpathlineto{\pgfqpoint{1.004478in}{1.651913in}}%
\pgfpathlineto{\pgfqpoint{1.010644in}{1.638302in}}%
\pgfpathlineto{\pgfqpoint{1.015872in}{1.629749in}}%
\pgfpathlineto{\pgfqpoint{1.019134in}{1.624691in}}%
\pgfpathlineto{\pgfqpoint{1.029915in}{1.611079in}}%
\pgfpathlineto{\pgfqpoint{1.031529in}{1.609330in}}%
\pgfpathlineto{\pgfqpoint{1.043250in}{1.597468in}}%
\pgfpathlineto{\pgfqpoint{1.047185in}{1.593895in}}%
\pgfpathlineto{\pgfqpoint{1.059327in}{1.583857in}}%
\pgfpathlineto{\pgfqpoint{1.062842in}{1.581177in}}%
\pgfpathlineto{\pgfqpoint{1.078498in}{1.570657in}}%
\pgfpathlineto{\pgfqpoint{1.079228in}{1.570246in}}%
\pgfpathlineto{\pgfqpoint{1.094155in}{1.562358in}}%
\pgfpathlineto{\pgfqpoint{1.107937in}{1.556635in}}%
\pgfpathlineto{\pgfqpoint{1.109812in}{1.555891in}}%
\pgfpathclose%
\pgfpathmoveto{\pgfqpoint{1.892640in}{1.553521in}}%
\pgfpathlineto{\pgfqpoint{1.908296in}{1.550299in}}%
\pgfpathlineto{\pgfqpoint{1.923953in}{1.549210in}}%
\pgfpathlineto{\pgfqpoint{1.923953in}{1.556635in}}%
\pgfpathlineto{\pgfqpoint{1.923953in}{1.570246in}}%
\pgfpathlineto{\pgfqpoint{1.923953in}{1.583857in}}%
\pgfpathlineto{\pgfqpoint{1.923953in}{1.597468in}}%
\pgfpathlineto{\pgfqpoint{1.923953in}{1.611079in}}%
\pgfpathlineto{\pgfqpoint{1.923953in}{1.624691in}}%
\pgfpathlineto{\pgfqpoint{1.923953in}{1.638302in}}%
\pgfpathlineto{\pgfqpoint{1.923953in}{1.651913in}}%
\pgfpathlineto{\pgfqpoint{1.923953in}{1.665524in}}%
\pgfpathlineto{\pgfqpoint{1.923953in}{1.679135in}}%
\pgfpathlineto{\pgfqpoint{1.908296in}{1.679135in}}%
\pgfpathlineto{\pgfqpoint{1.892640in}{1.679135in}}%
\pgfpathlineto{\pgfqpoint{1.876983in}{1.679135in}}%
\pgfpathlineto{\pgfqpoint{1.861327in}{1.679135in}}%
\pgfpathlineto{\pgfqpoint{1.845670in}{1.679135in}}%
\pgfpathlineto{\pgfqpoint{1.830014in}{1.679135in}}%
\pgfpathlineto{\pgfqpoint{1.814357in}{1.679135in}}%
\pgfpathlineto{\pgfqpoint{1.798700in}{1.679135in}}%
\pgfpathlineto{\pgfqpoint{1.783044in}{1.679135in}}%
\pgfpathlineto{\pgfqpoint{1.774503in}{1.679135in}}%
\pgfpathlineto{\pgfqpoint{1.775756in}{1.665524in}}%
\pgfpathlineto{\pgfqpoint{1.779461in}{1.651913in}}%
\pgfpathlineto{\pgfqpoint{1.783044in}{1.643824in}}%
\pgfpathlineto{\pgfqpoint{1.785609in}{1.638302in}}%
\pgfpathlineto{\pgfqpoint{1.794150in}{1.624691in}}%
\pgfpathlineto{\pgfqpoint{1.798700in}{1.618793in}}%
\pgfpathlineto{\pgfqpoint{1.805023in}{1.611079in}}%
\pgfpathlineto{\pgfqpoint{1.814357in}{1.601251in}}%
\pgfpathlineto{\pgfqpoint{1.818257in}{1.597468in}}%
\pgfpathlineto{\pgfqpoint{1.830014in}{1.587247in}}%
\pgfpathlineto{\pgfqpoint{1.834365in}{1.583857in}}%
\pgfpathlineto{\pgfqpoint{1.845670in}{1.575743in}}%
\pgfpathlineto{\pgfqpoint{1.854542in}{1.570246in}}%
\pgfpathlineto{\pgfqpoint{1.861327in}{1.566291in}}%
\pgfpathlineto{\pgfqpoint{1.876983in}{1.558865in}}%
\pgfpathlineto{\pgfqpoint{1.883336in}{1.556635in}}%
\pgfpathlineto{\pgfqpoint{1.892640in}{1.553521in}}%
\pgfpathclose%
\pgfusepath{fill}%
\end{pgfscope}%
\begin{pgfscope}%
\pgfsetbuttcap%
\pgfsetroundjoin%
\definecolor{currentfill}{rgb}{0.000000,0.000000,0.000000}%
\pgfsetfillcolor{currentfill}%
\pgfsetlinewidth{0.803000pt}%
\definecolor{currentstroke}{rgb}{0.000000,0.000000,0.000000}%
\pgfsetstrokecolor{currentstroke}%
\pgfsetdash{}{0pt}%
\pgfsys@defobject{currentmarker}{\pgfqpoint{0.000000in}{-0.048611in}}{\pgfqpoint{0.000000in}{0.000000in}}{%
\pgfpathmoveto{\pgfqpoint{0.000000in}{0.000000in}}%
\pgfpathlineto{\pgfqpoint{0.000000in}{-0.048611in}}%
\pgfusepath{stroke,fill}%
}%
\begin{pgfscope}%
\pgfsys@transformshift{0.373953in}{0.331635in}%
\pgfsys@useobject{currentmarker}{}%
\end{pgfscope}%
\end{pgfscope}%
\begin{pgfscope}%
\definecolor{textcolor}{rgb}{0.000000,0.000000,0.000000}%
\pgfsetstrokecolor{textcolor}%
\pgfsetfillcolor{textcolor}%
\pgftext[x=0.373953in,y=0.234413in,,top]{\color{textcolor}{\sffamily\fontsize{10.000000}{12.000000}\selectfont\catcode`\^=\active\def^{\ifmmode\sp\else\^{}\fi}\catcode`\%=\active\def%{\%}0}}%
\end{pgfscope}%
\begin{pgfscope}%
\pgfsetbuttcap%
\pgfsetroundjoin%
\definecolor{currentfill}{rgb}{0.000000,0.000000,0.000000}%
\pgfsetfillcolor{currentfill}%
\pgfsetlinewidth{0.803000pt}%
\definecolor{currentstroke}{rgb}{0.000000,0.000000,0.000000}%
\pgfsetstrokecolor{currentstroke}%
\pgfsetdash{}{0pt}%
\pgfsys@defobject{currentmarker}{\pgfqpoint{0.000000in}{-0.048611in}}{\pgfqpoint{0.000000in}{0.000000in}}{%
\pgfpathmoveto{\pgfqpoint{0.000000in}{0.000000in}}%
\pgfpathlineto{\pgfqpoint{0.000000in}{-0.048611in}}%
\pgfusepath{stroke,fill}%
}%
\begin{pgfscope}%
\pgfsys@transformshift{1.019786in}{0.331635in}%
\pgfsys@useobject{currentmarker}{}%
\end{pgfscope}%
\end{pgfscope}%
\begin{pgfscope}%
\definecolor{textcolor}{rgb}{0.000000,0.000000,0.000000}%
\pgfsetstrokecolor{textcolor}%
\pgfsetfillcolor{textcolor}%
\pgftext[x=1.019786in,y=0.234413in,,top]{\color{textcolor}{\sffamily\fontsize{10.000000}{12.000000}\selectfont\catcode`\^=\active\def^{\ifmmode\sp\else\^{}\fi}\catcode`\%=\active\def%{\%}5}}%
\end{pgfscope}%
\begin{pgfscope}%
\pgfsetbuttcap%
\pgfsetroundjoin%
\definecolor{currentfill}{rgb}{0.000000,0.000000,0.000000}%
\pgfsetfillcolor{currentfill}%
\pgfsetlinewidth{0.803000pt}%
\definecolor{currentstroke}{rgb}{0.000000,0.000000,0.000000}%
\pgfsetstrokecolor{currentstroke}%
\pgfsetdash{}{0pt}%
\pgfsys@defobject{currentmarker}{\pgfqpoint{0.000000in}{-0.048611in}}{\pgfqpoint{0.000000in}{0.000000in}}{%
\pgfpathmoveto{\pgfqpoint{0.000000in}{0.000000in}}%
\pgfpathlineto{\pgfqpoint{0.000000in}{-0.048611in}}%
\pgfusepath{stroke,fill}%
}%
\begin{pgfscope}%
\pgfsys@transformshift{1.665620in}{0.331635in}%
\pgfsys@useobject{currentmarker}{}%
\end{pgfscope}%
\end{pgfscope}%
\begin{pgfscope}%
\definecolor{textcolor}{rgb}{0.000000,0.000000,0.000000}%
\pgfsetstrokecolor{textcolor}%
\pgfsetfillcolor{textcolor}%
\pgftext[x=1.665620in,y=0.234413in,,top]{\color{textcolor}{\sffamily\fontsize{10.000000}{12.000000}\selectfont\catcode`\^=\active\def^{\ifmmode\sp\else\^{}\fi}\catcode`\%=\active\def%{\%}10}}%
\end{pgfscope}%
\begin{pgfscope}%
\pgfsetbuttcap%
\pgfsetroundjoin%
\definecolor{currentfill}{rgb}{0.000000,0.000000,0.000000}%
\pgfsetfillcolor{currentfill}%
\pgfsetlinewidth{0.803000pt}%
\definecolor{currentstroke}{rgb}{0.000000,0.000000,0.000000}%
\pgfsetstrokecolor{currentstroke}%
\pgfsetdash{}{0pt}%
\pgfsys@defobject{currentmarker}{\pgfqpoint{-0.048611in}{0.000000in}}{\pgfqpoint{-0.000000in}{0.000000in}}{%
\pgfpathmoveto{\pgfqpoint{-0.000000in}{0.000000in}}%
\pgfpathlineto{\pgfqpoint{-0.048611in}{0.000000in}}%
\pgfusepath{stroke,fill}%
}%
\begin{pgfscope}%
\pgfsys@transformshift{0.373953in}{0.331635in}%
\pgfsys@useobject{currentmarker}{}%
\end{pgfscope}%
\end{pgfscope}%
\begin{pgfscope}%
\definecolor{textcolor}{rgb}{0.000000,0.000000,0.000000}%
\pgfsetstrokecolor{textcolor}%
\pgfsetfillcolor{textcolor}%
\pgftext[x=0.188365in, y=0.278873in, left, base]{\color{textcolor}{\sffamily\fontsize{10.000000}{12.000000}\selectfont\catcode`\^=\active\def^{\ifmmode\sp\else\^{}\fi}\catcode`\%=\active\def%{\%}0}}%
\end{pgfscope}%
\begin{pgfscope}%
\pgfsetbuttcap%
\pgfsetroundjoin%
\definecolor{currentfill}{rgb}{0.000000,0.000000,0.000000}%
\pgfsetfillcolor{currentfill}%
\pgfsetlinewidth{0.803000pt}%
\definecolor{currentstroke}{rgb}{0.000000,0.000000,0.000000}%
\pgfsetstrokecolor{currentstroke}%
\pgfsetdash{}{0pt}%
\pgfsys@defobject{currentmarker}{\pgfqpoint{-0.048611in}{0.000000in}}{\pgfqpoint{-0.000000in}{0.000000in}}{%
\pgfpathmoveto{\pgfqpoint{-0.000000in}{0.000000in}}%
\pgfpathlineto{\pgfqpoint{-0.048611in}{0.000000in}}%
\pgfusepath{stroke,fill}%
}%
\begin{pgfscope}%
\pgfsys@transformshift{0.373953in}{0.893093in}%
\pgfsys@useobject{currentmarker}{}%
\end{pgfscope}%
\end{pgfscope}%
\begin{pgfscope}%
\definecolor{textcolor}{rgb}{0.000000,0.000000,0.000000}%
\pgfsetstrokecolor{textcolor}%
\pgfsetfillcolor{textcolor}%
\pgftext[x=0.188365in, y=0.840332in, left, base]{\color{textcolor}{\sffamily\fontsize{10.000000}{12.000000}\selectfont\catcode`\^=\active\def^{\ifmmode\sp\else\^{}\fi}\catcode`\%=\active\def%{\%}5}}%
\end{pgfscope}%
\begin{pgfscope}%
\pgfsetbuttcap%
\pgfsetroundjoin%
\definecolor{currentfill}{rgb}{0.000000,0.000000,0.000000}%
\pgfsetfillcolor{currentfill}%
\pgfsetlinewidth{0.803000pt}%
\definecolor{currentstroke}{rgb}{0.000000,0.000000,0.000000}%
\pgfsetstrokecolor{currentstroke}%
\pgfsetdash{}{0pt}%
\pgfsys@defobject{currentmarker}{\pgfqpoint{-0.048611in}{0.000000in}}{\pgfqpoint{-0.000000in}{0.000000in}}{%
\pgfpathmoveto{\pgfqpoint{-0.000000in}{0.000000in}}%
\pgfpathlineto{\pgfqpoint{-0.048611in}{0.000000in}}%
\pgfusepath{stroke,fill}%
}%
\begin{pgfscope}%
\pgfsys@transformshift{0.373953in}{1.454552in}%
\pgfsys@useobject{currentmarker}{}%
\end{pgfscope}%
\end{pgfscope}%
\begin{pgfscope}%
\definecolor{textcolor}{rgb}{0.000000,0.000000,0.000000}%
\pgfsetstrokecolor{textcolor}%
\pgfsetfillcolor{textcolor}%
\pgftext[x=0.100000in, y=1.401790in, left, base]{\color{textcolor}{\sffamily\fontsize{10.000000}{12.000000}\selectfont\catcode`\^=\active\def^{\ifmmode\sp\else\^{}\fi}\catcode`\%=\active\def%{\%}10}}%
\end{pgfscope}%
\begin{pgfscope}%
\pgfsetrectcap%
\pgfsetmiterjoin%
\pgfsetlinewidth{0.803000pt}%
\definecolor{currentstroke}{rgb}{0.000000,0.000000,0.000000}%
\pgfsetstrokecolor{currentstroke}%
\pgfsetdash{}{0pt}%
\pgfpathmoveto{\pgfqpoint{0.373953in}{0.331635in}}%
\pgfpathlineto{\pgfqpoint{0.373953in}{1.679135in}}%
\pgfusepath{stroke}%
\end{pgfscope}%
\begin{pgfscope}%
\pgfsetrectcap%
\pgfsetmiterjoin%
\pgfsetlinewidth{0.803000pt}%
\definecolor{currentstroke}{rgb}{0.000000,0.000000,0.000000}%
\pgfsetstrokecolor{currentstroke}%
\pgfsetdash{}{0pt}%
\pgfpathmoveto{\pgfqpoint{1.923953in}{0.331635in}}%
\pgfpathlineto{\pgfqpoint{1.923953in}{1.679135in}}%
\pgfusepath{stroke}%
\end{pgfscope}%
\begin{pgfscope}%
\pgfsetrectcap%
\pgfsetmiterjoin%
\pgfsetlinewidth{0.803000pt}%
\definecolor{currentstroke}{rgb}{0.000000,0.000000,0.000000}%
\pgfsetstrokecolor{currentstroke}%
\pgfsetdash{}{0pt}%
\pgfpathmoveto{\pgfqpoint{0.373953in}{0.331635in}}%
\pgfpathlineto{\pgfqpoint{1.923953in}{0.331635in}}%
\pgfusepath{stroke}%
\end{pgfscope}%
\begin{pgfscope}%
\pgfsetrectcap%
\pgfsetmiterjoin%
\pgfsetlinewidth{0.803000pt}%
\definecolor{currentstroke}{rgb}{0.000000,0.000000,0.000000}%
\pgfsetstrokecolor{currentstroke}%
\pgfsetdash{}{0pt}%
\pgfpathmoveto{\pgfqpoint{0.373953in}{1.679135in}}%
\pgfpathlineto{\pgfqpoint{1.923953in}{1.679135in}}%
\pgfusepath{stroke}%
\end{pgfscope}%
\end{pgfpicture}%
\makeatother%
\endgroup%

        \caption{$c=2$}
        \label{fig:5-experiments-periodic-gaussian-well-2}
    \end{subfigure}
    \begin{subfigure}[b]{0.32\columnwidth}
        %% Creator: Matplotlib, PGF backend
%%
%% To include the figure in your LaTeX document, write
%%   \input{<filename>.pgf}
%%
%% Make sure the required packages are loaded in your preamble
%%   \usepackage{pgf}
%%
%% Also ensure that all the required font packages are loaded; for instance,
%% the lmodern package is sometimes necessary when using math font.
%%   \usepackage{lmodern}
%%
%% Figures using additional raster images can only be included by \input if
%% they are in the same directory as the main LaTeX file. For loading figures
%% from other directories you can use the `import` package
%%   \usepackage{import}
%%
%% and then include the figures with
%%   \import{<path to file>}{<filename>.pgf}
%%
%% Matplotlib used the following preamble
%%   \def\mathdefault#1{#1}
%%   \everymath=\expandafter{\the\everymath\displaystyle}
%%   
%%   \usepackage{fontspec}
%%   \setmainfont{DejaVuSerif.ttf}[Path=\detokenize{C:/Users/fabio/Documents/Work/MasterThesis/Rand-SD/.venv/Lib/site-packages/matplotlib/mpl-data/fonts/ttf/}]
%%   \setsansfont{DejaVuSans.ttf}[Path=\detokenize{C:/Users/fabio/Documents/Work/MasterThesis/Rand-SD/.venv/Lib/site-packages/matplotlib/mpl-data/fonts/ttf/}]
%%   \setmonofont{DejaVuSansMono.ttf}[Path=\detokenize{C:/Users/fabio/Documents/Work/MasterThesis/Rand-SD/.venv/Lib/site-packages/matplotlib/mpl-data/fonts/ttf/}]
%%   \makeatletter\@ifpackageloaded{underscore}{}{\usepackage[strings]{underscore}}\makeatother
%%
\begingroup%
\makeatletter%
\begin{pgfpicture}%
\pgfpathrectangle{\pgfpointorigin}{\pgfqpoint{2.112318in}{1.831897in}}%
\pgfusepath{use as bounding box, clip}%
\begin{pgfscope}%
\pgfsetbuttcap%
\pgfsetmiterjoin%
\definecolor{currentfill}{rgb}{1.000000,1.000000,1.000000}%
\pgfsetfillcolor{currentfill}%
\pgfsetlinewidth{0.000000pt}%
\definecolor{currentstroke}{rgb}{1.000000,1.000000,1.000000}%
\pgfsetstrokecolor{currentstroke}%
\pgfsetdash{}{0pt}%
\pgfpathmoveto{\pgfqpoint{0.000000in}{0.000000in}}%
\pgfpathlineto{\pgfqpoint{2.112318in}{0.000000in}}%
\pgfpathlineto{\pgfqpoint{2.112318in}{1.831897in}}%
\pgfpathlineto{\pgfqpoint{0.000000in}{1.831897in}}%
\pgfpathlineto{\pgfqpoint{0.000000in}{0.000000in}}%
\pgfpathclose%
\pgfusepath{fill}%
\end{pgfscope}%
\begin{pgfscope}%
\pgfsetbuttcap%
\pgfsetmiterjoin%
\definecolor{currentfill}{rgb}{1.000000,1.000000,1.000000}%
\pgfsetfillcolor{currentfill}%
\pgfsetlinewidth{0.000000pt}%
\definecolor{currentstroke}{rgb}{0.000000,0.000000,0.000000}%
\pgfsetstrokecolor{currentstroke}%
\pgfsetstrokeopacity{0.000000}%
\pgfsetdash{}{0pt}%
\pgfpathmoveto{\pgfqpoint{0.373953in}{0.331635in}}%
\pgfpathlineto{\pgfqpoint{1.923953in}{0.331635in}}%
\pgfpathlineto{\pgfqpoint{1.923953in}{1.679135in}}%
\pgfpathlineto{\pgfqpoint{0.373953in}{1.679135in}}%
\pgfpathlineto{\pgfqpoint{0.373953in}{0.331635in}}%
\pgfpathclose%
\pgfusepath{fill}%
\end{pgfscope}%
\begin{pgfscope}%
\pgfpathrectangle{\pgfqpoint{0.373953in}{0.331635in}}{\pgfqpoint{1.550000in}{1.347500in}}%
\pgfusepath{clip}%
\pgfsetbuttcap%
\pgfsetroundjoin%
\definecolor{currentfill}{rgb}{0.993545,0.862859,0.619299}%
\pgfsetfillcolor{currentfill}%
\pgfsetlinewidth{0.000000pt}%
\definecolor{currentstroke}{rgb}{0.000000,0.000000,0.000000}%
\pgfsetstrokecolor{currentstroke}%
\pgfsetdash{}{0pt}%
\pgfpathmoveto{\pgfqpoint{0.514862in}{0.439324in}}%
\pgfpathlineto{\pgfqpoint{0.530519in}{0.436913in}}%
\pgfpathlineto{\pgfqpoint{0.546146in}{0.440524in}}%
\pgfpathlineto{\pgfqpoint{0.546175in}{0.440535in}}%
\pgfpathlineto{\pgfqpoint{0.559956in}{0.454135in}}%
\pgfpathlineto{\pgfqpoint{0.561832in}{0.460601in}}%
\pgfpathlineto{\pgfqpoint{0.563195in}{0.467746in}}%
\pgfpathlineto{\pgfqpoint{0.561832in}{0.472526in}}%
\pgfpathlineto{\pgfqpoint{0.557988in}{0.481357in}}%
\pgfpathlineto{\pgfqpoint{0.546175in}{0.491627in}}%
\pgfpathlineto{\pgfqpoint{0.536016in}{0.494968in}}%
\pgfpathlineto{\pgfqpoint{0.530519in}{0.496153in}}%
\pgfpathlineto{\pgfqpoint{0.522300in}{0.494968in}}%
\pgfpathlineto{\pgfqpoint{0.514862in}{0.493338in}}%
\pgfpathlineto{\pgfqpoint{0.499218in}{0.481357in}}%
\pgfpathlineto{\pgfqpoint{0.499205in}{0.481332in}}%
\pgfpathlineto{\pgfqpoint{0.495052in}{0.467746in}}%
\pgfpathlineto{\pgfqpoint{0.497826in}{0.454135in}}%
\pgfpathlineto{\pgfqpoint{0.499205in}{0.452161in}}%
\pgfpathlineto{\pgfqpoint{0.512592in}{0.440524in}}%
\pgfpathlineto{\pgfqpoint{0.514862in}{0.439324in}}%
\pgfpathclose%
\pgfpathmoveto{\pgfqpoint{0.827993in}{0.438360in}}%
\pgfpathlineto{\pgfqpoint{0.843650in}{0.437154in}}%
\pgfpathlineto{\pgfqpoint{0.854632in}{0.440524in}}%
\pgfpathlineto{\pgfqpoint{0.859306in}{0.442858in}}%
\pgfpathlineto{\pgfqpoint{0.869539in}{0.454135in}}%
\pgfpathlineto{\pgfqpoint{0.873077in}{0.467746in}}%
\pgfpathlineto{\pgfqpoint{0.867769in}{0.481357in}}%
\pgfpathlineto{\pgfqpoint{0.859306in}{0.489572in}}%
\pgfpathlineto{\pgfqpoint{0.846999in}{0.494968in}}%
\pgfpathlineto{\pgfqpoint{0.843650in}{0.495927in}}%
\pgfpathlineto{\pgfqpoint{0.830391in}{0.494968in}}%
\pgfpathlineto{\pgfqpoint{0.827993in}{0.494706in}}%
\pgfpathlineto{\pgfqpoint{0.812337in}{0.484433in}}%
\pgfpathlineto{\pgfqpoint{0.809700in}{0.481357in}}%
\pgfpathlineto{\pgfqpoint{0.805233in}{0.467746in}}%
\pgfpathlineto{\pgfqpoint{0.808210in}{0.454135in}}%
\pgfpathlineto{\pgfqpoint{0.812337in}{0.448671in}}%
\pgfpathlineto{\pgfqpoint{0.823243in}{0.440524in}}%
\pgfpathlineto{\pgfqpoint{0.827993in}{0.438360in}}%
\pgfpathclose%
\pgfpathmoveto{\pgfqpoint{1.141125in}{0.437636in}}%
\pgfpathlineto{\pgfqpoint{1.156781in}{0.437636in}}%
\pgfpathlineto{\pgfqpoint{1.164346in}{0.440524in}}%
\pgfpathlineto{\pgfqpoint{1.172438in}{0.445570in}}%
\pgfpathlineto{\pgfqpoint{1.179490in}{0.454135in}}%
\pgfpathlineto{\pgfqpoint{1.182716in}{0.467746in}}%
\pgfpathlineto{\pgfqpoint{1.177875in}{0.481357in}}%
\pgfpathlineto{\pgfqpoint{1.172438in}{0.487175in}}%
\pgfpathlineto{\pgfqpoint{1.158201in}{0.494968in}}%
\pgfpathlineto{\pgfqpoint{1.156781in}{0.495474in}}%
\pgfpathlineto{\pgfqpoint{1.141125in}{0.495474in}}%
\pgfpathlineto{\pgfqpoint{1.139705in}{0.494968in}}%
\pgfpathlineto{\pgfqpoint{1.125468in}{0.487175in}}%
\pgfpathlineto{\pgfqpoint{1.120031in}{0.481357in}}%
\pgfpathlineto{\pgfqpoint{1.115189in}{0.467746in}}%
\pgfpathlineto{\pgfqpoint{1.118416in}{0.454135in}}%
\pgfpathlineto{\pgfqpoint{1.125468in}{0.445570in}}%
\pgfpathlineto{\pgfqpoint{1.133560in}{0.440524in}}%
\pgfpathlineto{\pgfqpoint{1.141125in}{0.437636in}}%
\pgfpathclose%
\pgfpathmoveto{\pgfqpoint{1.454256in}{0.437154in}}%
\pgfpathlineto{\pgfqpoint{1.469913in}{0.438360in}}%
\pgfpathlineto{\pgfqpoint{1.474663in}{0.440524in}}%
\pgfpathlineto{\pgfqpoint{1.485569in}{0.448671in}}%
\pgfpathlineto{\pgfqpoint{1.489696in}{0.454135in}}%
\pgfpathlineto{\pgfqpoint{1.492673in}{0.467746in}}%
\pgfpathlineto{\pgfqpoint{1.488206in}{0.481357in}}%
\pgfpathlineto{\pgfqpoint{1.485569in}{0.484433in}}%
\pgfpathlineto{\pgfqpoint{1.469913in}{0.494706in}}%
\pgfpathlineto{\pgfqpoint{1.467515in}{0.494968in}}%
\pgfpathlineto{\pgfqpoint{1.454256in}{0.495927in}}%
\pgfpathlineto{\pgfqpoint{1.450907in}{0.494968in}}%
\pgfpathlineto{\pgfqpoint{1.438599in}{0.489572in}}%
\pgfpathlineto{\pgfqpoint{1.430137in}{0.481357in}}%
\pgfpathlineto{\pgfqpoint{1.424829in}{0.467746in}}%
\pgfpathlineto{\pgfqpoint{1.428367in}{0.454135in}}%
\pgfpathlineto{\pgfqpoint{1.438599in}{0.442858in}}%
\pgfpathlineto{\pgfqpoint{1.443274in}{0.440524in}}%
\pgfpathlineto{\pgfqpoint{1.454256in}{0.437154in}}%
\pgfpathclose%
\pgfpathmoveto{\pgfqpoint{1.767387in}{0.436913in}}%
\pgfpathlineto{\pgfqpoint{1.783044in}{0.439324in}}%
\pgfpathlineto{\pgfqpoint{1.785314in}{0.440524in}}%
\pgfpathlineto{\pgfqpoint{1.798700in}{0.452161in}}%
\pgfpathlineto{\pgfqpoint{1.800080in}{0.454135in}}%
\pgfpathlineto{\pgfqpoint{1.802854in}{0.467746in}}%
\pgfpathlineto{\pgfqpoint{1.798700in}{0.481332in}}%
\pgfpathlineto{\pgfqpoint{1.798688in}{0.481357in}}%
\pgfpathlineto{\pgfqpoint{1.783044in}{0.493338in}}%
\pgfpathlineto{\pgfqpoint{1.775606in}{0.494968in}}%
\pgfpathlineto{\pgfqpoint{1.767387in}{0.496153in}}%
\pgfpathlineto{\pgfqpoint{1.761890in}{0.494968in}}%
\pgfpathlineto{\pgfqpoint{1.751731in}{0.491627in}}%
\pgfpathlineto{\pgfqpoint{1.739918in}{0.481357in}}%
\pgfpathlineto{\pgfqpoint{1.736074in}{0.472526in}}%
\pgfpathlineto{\pgfqpoint{1.734711in}{0.467746in}}%
\pgfpathlineto{\pgfqpoint{1.736074in}{0.460601in}}%
\pgfpathlineto{\pgfqpoint{1.737950in}{0.454135in}}%
\pgfpathlineto{\pgfqpoint{1.751731in}{0.440535in}}%
\pgfpathlineto{\pgfqpoint{1.751760in}{0.440524in}}%
\pgfpathlineto{\pgfqpoint{1.767387in}{0.436913in}}%
\pgfpathclose%
\pgfpathmoveto{\pgfqpoint{0.514862in}{0.709159in}}%
\pgfpathlineto{\pgfqpoint{0.530519in}{0.706571in}}%
\pgfpathlineto{\pgfqpoint{0.546175in}{0.710454in}}%
\pgfpathlineto{\pgfqpoint{0.549713in}{0.712746in}}%
\pgfpathlineto{\pgfqpoint{0.561530in}{0.726357in}}%
\pgfpathlineto{\pgfqpoint{0.561832in}{0.728442in}}%
\pgfpathlineto{\pgfqpoint{0.562934in}{0.739968in}}%
\pgfpathlineto{\pgfqpoint{0.561832in}{0.742880in}}%
\pgfpathlineto{\pgfqpoint{0.555625in}{0.753579in}}%
\pgfpathlineto{\pgfqpoint{0.546175in}{0.760936in}}%
\pgfpathlineto{\pgfqpoint{0.530519in}{0.765551in}}%
\pgfpathlineto{\pgfqpoint{0.514862in}{0.762475in}}%
\pgfpathlineto{\pgfqpoint{0.501891in}{0.753579in}}%
\pgfpathlineto{\pgfqpoint{0.499205in}{0.749516in}}%
\pgfpathlineto{\pgfqpoint{0.495329in}{0.739968in}}%
\pgfpathlineto{\pgfqpoint{0.496716in}{0.726357in}}%
\pgfpathlineto{\pgfqpoint{0.499205in}{0.722228in}}%
\pgfpathlineto{\pgfqpoint{0.508577in}{0.712746in}}%
\pgfpathlineto{\pgfqpoint{0.514862in}{0.709159in}}%
\pgfpathclose%
\pgfpathmoveto{\pgfqpoint{0.827993in}{0.708123in}}%
\pgfpathlineto{\pgfqpoint{0.843650in}{0.706829in}}%
\pgfpathlineto{\pgfqpoint{0.859306in}{0.712008in}}%
\pgfpathlineto{\pgfqpoint{0.860326in}{0.712746in}}%
\pgfpathlineto{\pgfqpoint{0.870955in}{0.726357in}}%
\pgfpathlineto{\pgfqpoint{0.872723in}{0.739968in}}%
\pgfpathlineto{\pgfqpoint{0.865644in}{0.753579in}}%
\pgfpathlineto{\pgfqpoint{0.859306in}{0.759089in}}%
\pgfpathlineto{\pgfqpoint{0.843650in}{0.765244in}}%
\pgfpathlineto{\pgfqpoint{0.827993in}{0.763706in}}%
\pgfpathlineto{\pgfqpoint{0.812337in}{0.754466in}}%
\pgfpathlineto{\pgfqpoint{0.811488in}{0.753579in}}%
\pgfpathlineto{\pgfqpoint{0.805531in}{0.739968in}}%
\pgfpathlineto{\pgfqpoint{0.807019in}{0.726357in}}%
\pgfpathlineto{\pgfqpoint{0.812337in}{0.718192in}}%
\pgfpathlineto{\pgfqpoint{0.818601in}{0.712746in}}%
\pgfpathlineto{\pgfqpoint{0.827993in}{0.708123in}}%
\pgfpathclose%
\pgfpathmoveto{\pgfqpoint{1.141125in}{0.707347in}}%
\pgfpathlineto{\pgfqpoint{1.156781in}{0.707347in}}%
\pgfpathlineto{\pgfqpoint{1.169874in}{0.712746in}}%
\pgfpathlineto{\pgfqpoint{1.172438in}{0.714607in}}%
\pgfpathlineto{\pgfqpoint{1.180781in}{0.726357in}}%
\pgfpathlineto{\pgfqpoint{1.182394in}{0.739968in}}%
\pgfpathlineto{\pgfqpoint{1.175937in}{0.753579in}}%
\pgfpathlineto{\pgfqpoint{1.172438in}{0.756932in}}%
\pgfpathlineto{\pgfqpoint{1.156781in}{0.764629in}}%
\pgfpathlineto{\pgfqpoint{1.141125in}{0.764629in}}%
\pgfpathlineto{\pgfqpoint{1.125468in}{0.756932in}}%
\pgfpathlineto{\pgfqpoint{1.121969in}{0.753579in}}%
\pgfpathlineto{\pgfqpoint{1.115512in}{0.739968in}}%
\pgfpathlineto{\pgfqpoint{1.117125in}{0.726357in}}%
\pgfpathlineto{\pgfqpoint{1.125468in}{0.714607in}}%
\pgfpathlineto{\pgfqpoint{1.128032in}{0.712746in}}%
\pgfpathlineto{\pgfqpoint{1.141125in}{0.707347in}}%
\pgfpathclose%
\pgfpathmoveto{\pgfqpoint{1.438599in}{0.712008in}}%
\pgfpathlineto{\pgfqpoint{1.454256in}{0.706829in}}%
\pgfpathlineto{\pgfqpoint{1.469913in}{0.708123in}}%
\pgfpathlineto{\pgfqpoint{1.479304in}{0.712746in}}%
\pgfpathlineto{\pgfqpoint{1.485569in}{0.718192in}}%
\pgfpathlineto{\pgfqpoint{1.490887in}{0.726357in}}%
\pgfpathlineto{\pgfqpoint{1.492375in}{0.739968in}}%
\pgfpathlineto{\pgfqpoint{1.486418in}{0.753579in}}%
\pgfpathlineto{\pgfqpoint{1.485569in}{0.754466in}}%
\pgfpathlineto{\pgfqpoint{1.469913in}{0.763706in}}%
\pgfpathlineto{\pgfqpoint{1.454256in}{0.765244in}}%
\pgfpathlineto{\pgfqpoint{1.438599in}{0.759089in}}%
\pgfpathlineto{\pgfqpoint{1.432262in}{0.753579in}}%
\pgfpathlineto{\pgfqpoint{1.425182in}{0.739968in}}%
\pgfpathlineto{\pgfqpoint{1.426951in}{0.726357in}}%
\pgfpathlineto{\pgfqpoint{1.437580in}{0.712746in}}%
\pgfpathlineto{\pgfqpoint{1.438599in}{0.712008in}}%
\pgfpathclose%
\pgfpathmoveto{\pgfqpoint{1.751731in}{0.710454in}}%
\pgfpathlineto{\pgfqpoint{1.767387in}{0.706571in}}%
\pgfpathlineto{\pgfqpoint{1.783044in}{0.709159in}}%
\pgfpathlineto{\pgfqpoint{1.789329in}{0.712746in}}%
\pgfpathlineto{\pgfqpoint{1.798700in}{0.722228in}}%
\pgfpathlineto{\pgfqpoint{1.801190in}{0.726357in}}%
\pgfpathlineto{\pgfqpoint{1.802577in}{0.739968in}}%
\pgfpathlineto{\pgfqpoint{1.798700in}{0.749516in}}%
\pgfpathlineto{\pgfqpoint{1.796015in}{0.753579in}}%
\pgfpathlineto{\pgfqpoint{1.783044in}{0.762475in}}%
\pgfpathlineto{\pgfqpoint{1.767387in}{0.765551in}}%
\pgfpathlineto{\pgfqpoint{1.751731in}{0.760936in}}%
\pgfpathlineto{\pgfqpoint{1.742281in}{0.753579in}}%
\pgfpathlineto{\pgfqpoint{1.736074in}{0.742880in}}%
\pgfpathlineto{\pgfqpoint{1.734971in}{0.739968in}}%
\pgfpathlineto{\pgfqpoint{1.736074in}{0.728442in}}%
\pgfpathlineto{\pgfqpoint{1.736376in}{0.726357in}}%
\pgfpathlineto{\pgfqpoint{1.748193in}{0.712746in}}%
\pgfpathlineto{\pgfqpoint{1.751731in}{0.710454in}}%
\pgfpathclose%
\pgfpathmoveto{\pgfqpoint{0.514862in}{0.978838in}}%
\pgfpathlineto{\pgfqpoint{0.530519in}{0.976033in}}%
\pgfpathlineto{\pgfqpoint{0.546175in}{0.980241in}}%
\pgfpathlineto{\pgfqpoint{0.552867in}{0.984968in}}%
\pgfpathlineto{\pgfqpoint{0.561832in}{0.997345in}}%
\pgfpathlineto{\pgfqpoint{0.562413in}{0.998579in}}%
\pgfpathlineto{\pgfqpoint{0.562413in}{1.012191in}}%
\pgfpathlineto{\pgfqpoint{0.561832in}{1.013425in}}%
\pgfpathlineto{\pgfqpoint{0.552867in}{1.025802in}}%
\pgfpathlineto{\pgfqpoint{0.546175in}{1.030529in}}%
\pgfpathlineto{\pgfqpoint{0.530519in}{1.034737in}}%
\pgfpathlineto{\pgfqpoint{0.514862in}{1.031932in}}%
\pgfpathlineto{\pgfqpoint{0.505010in}{1.025802in}}%
\pgfpathlineto{\pgfqpoint{0.499205in}{1.018767in}}%
\pgfpathlineto{\pgfqpoint{0.495884in}{1.012191in}}%
\pgfpathlineto{\pgfqpoint{0.495884in}{0.998579in}}%
\pgfpathlineto{\pgfqpoint{0.499205in}{0.992003in}}%
\pgfpathlineto{\pgfqpoint{0.505010in}{0.984968in}}%
\pgfpathlineto{\pgfqpoint{0.514862in}{0.978838in}}%
\pgfpathclose%
\pgfpathmoveto{\pgfqpoint{0.827993in}{0.977715in}}%
\pgfpathlineto{\pgfqpoint{0.843650in}{0.976313in}}%
\pgfpathlineto{\pgfqpoint{0.859306in}{0.981926in}}%
\pgfpathlineto{\pgfqpoint{0.863163in}{0.984968in}}%
\pgfpathlineto{\pgfqpoint{0.872016in}{0.998579in}}%
\pgfpathlineto{\pgfqpoint{0.872016in}{1.012191in}}%
\pgfpathlineto{\pgfqpoint{0.863163in}{1.025802in}}%
\pgfpathlineto{\pgfqpoint{0.859306in}{1.028844in}}%
\pgfpathlineto{\pgfqpoint{0.843650in}{1.034457in}}%
\pgfpathlineto{\pgfqpoint{0.827993in}{1.033055in}}%
\pgfpathlineto{\pgfqpoint{0.814477in}{1.025802in}}%
\pgfpathlineto{\pgfqpoint{0.812337in}{1.023573in}}%
\pgfpathlineto{\pgfqpoint{0.806126in}{1.012191in}}%
\pgfpathlineto{\pgfqpoint{0.806126in}{0.998579in}}%
\pgfpathlineto{\pgfqpoint{0.812337in}{0.987197in}}%
\pgfpathlineto{\pgfqpoint{0.814477in}{0.984968in}}%
\pgfpathlineto{\pgfqpoint{0.827993in}{0.977715in}}%
\pgfpathclose%
\pgfpathmoveto{\pgfqpoint{1.125468in}{0.983893in}}%
\pgfpathlineto{\pgfqpoint{1.141125in}{0.976874in}}%
\pgfpathlineto{\pgfqpoint{1.156781in}{0.976874in}}%
\pgfpathlineto{\pgfqpoint{1.172438in}{0.983893in}}%
\pgfpathlineto{\pgfqpoint{1.173674in}{0.984968in}}%
\pgfpathlineto{\pgfqpoint{1.181749in}{0.998579in}}%
\pgfpathlineto{\pgfqpoint{1.181749in}{1.012191in}}%
\pgfpathlineto{\pgfqpoint{1.173674in}{1.025802in}}%
\pgfpathlineto{\pgfqpoint{1.172438in}{1.026877in}}%
\pgfpathlineto{\pgfqpoint{1.156781in}{1.033896in}}%
\pgfpathlineto{\pgfqpoint{1.141125in}{1.033896in}}%
\pgfpathlineto{\pgfqpoint{1.125468in}{1.026877in}}%
\pgfpathlineto{\pgfqpoint{1.124231in}{1.025802in}}%
\pgfpathlineto{\pgfqpoint{1.116157in}{1.012191in}}%
\pgfpathlineto{\pgfqpoint{1.116157in}{0.998579in}}%
\pgfpathlineto{\pgfqpoint{1.124231in}{0.984968in}}%
\pgfpathlineto{\pgfqpoint{1.125468in}{0.983893in}}%
\pgfpathclose%
\pgfpathmoveto{\pgfqpoint{1.438599in}{0.981926in}}%
\pgfpathlineto{\pgfqpoint{1.454256in}{0.976313in}}%
\pgfpathlineto{\pgfqpoint{1.469913in}{0.977715in}}%
\pgfpathlineto{\pgfqpoint{1.483429in}{0.984968in}}%
\pgfpathlineto{\pgfqpoint{1.485569in}{0.987197in}}%
\pgfpathlineto{\pgfqpoint{1.491780in}{0.998579in}}%
\pgfpathlineto{\pgfqpoint{1.491780in}{1.012191in}}%
\pgfpathlineto{\pgfqpoint{1.485569in}{1.023573in}}%
\pgfpathlineto{\pgfqpoint{1.483429in}{1.025802in}}%
\pgfpathlineto{\pgfqpoint{1.469913in}{1.033055in}}%
\pgfpathlineto{\pgfqpoint{1.454256in}{1.034457in}}%
\pgfpathlineto{\pgfqpoint{1.438599in}{1.028844in}}%
\pgfpathlineto{\pgfqpoint{1.434743in}{1.025802in}}%
\pgfpathlineto{\pgfqpoint{1.425890in}{1.012191in}}%
\pgfpathlineto{\pgfqpoint{1.425890in}{0.998579in}}%
\pgfpathlineto{\pgfqpoint{1.434743in}{0.984968in}}%
\pgfpathlineto{\pgfqpoint{1.438599in}{0.981926in}}%
\pgfpathclose%
\pgfpathmoveto{\pgfqpoint{1.751731in}{0.980241in}}%
\pgfpathlineto{\pgfqpoint{1.767387in}{0.976033in}}%
\pgfpathlineto{\pgfqpoint{1.783044in}{0.978838in}}%
\pgfpathlineto{\pgfqpoint{1.792896in}{0.984968in}}%
\pgfpathlineto{\pgfqpoint{1.798700in}{0.992003in}}%
\pgfpathlineto{\pgfqpoint{1.802022in}{0.998579in}}%
\pgfpathlineto{\pgfqpoint{1.802022in}{1.012191in}}%
\pgfpathlineto{\pgfqpoint{1.798700in}{1.018767in}}%
\pgfpathlineto{\pgfqpoint{1.792896in}{1.025802in}}%
\pgfpathlineto{\pgfqpoint{1.783044in}{1.031932in}}%
\pgfpathlineto{\pgfqpoint{1.767387in}{1.034737in}}%
\pgfpathlineto{\pgfqpoint{1.751731in}{1.030529in}}%
\pgfpathlineto{\pgfqpoint{1.745039in}{1.025802in}}%
\pgfpathlineto{\pgfqpoint{1.736074in}{1.013425in}}%
\pgfpathlineto{\pgfqpoint{1.735492in}{1.012191in}}%
\pgfpathlineto{\pgfqpoint{1.735492in}{0.998579in}}%
\pgfpathlineto{\pgfqpoint{1.736074in}{0.997345in}}%
\pgfpathlineto{\pgfqpoint{1.745039in}{0.984968in}}%
\pgfpathlineto{\pgfqpoint{1.751731in}{0.980241in}}%
\pgfpathclose%
\pgfpathmoveto{\pgfqpoint{0.514862in}{1.248295in}}%
\pgfpathlineto{\pgfqpoint{0.530519in}{1.245219in}}%
\pgfpathlineto{\pgfqpoint{0.546175in}{1.249834in}}%
\pgfpathlineto{\pgfqpoint{0.555625in}{1.257191in}}%
\pgfpathlineto{\pgfqpoint{0.561832in}{1.267890in}}%
\pgfpathlineto{\pgfqpoint{0.562934in}{1.270802in}}%
\pgfpathlineto{\pgfqpoint{0.561832in}{1.282328in}}%
\pgfpathlineto{\pgfqpoint{0.561530in}{1.284413in}}%
\pgfpathlineto{\pgfqpoint{0.549713in}{1.298024in}}%
\pgfpathlineto{\pgfqpoint{0.546175in}{1.300316in}}%
\pgfpathlineto{\pgfqpoint{0.530519in}{1.304199in}}%
\pgfpathlineto{\pgfqpoint{0.514862in}{1.301611in}}%
\pgfpathlineto{\pgfqpoint{0.508577in}{1.298024in}}%
\pgfpathlineto{\pgfqpoint{0.499205in}{1.288542in}}%
\pgfpathlineto{\pgfqpoint{0.496716in}{1.284413in}}%
\pgfpathlineto{\pgfqpoint{0.495329in}{1.270802in}}%
\pgfpathlineto{\pgfqpoint{0.499205in}{1.261254in}}%
\pgfpathlineto{\pgfqpoint{0.501891in}{1.257191in}}%
\pgfpathlineto{\pgfqpoint{0.514862in}{1.248295in}}%
\pgfpathclose%
\pgfpathmoveto{\pgfqpoint{0.812337in}{1.256304in}}%
\pgfpathlineto{\pgfqpoint{0.827993in}{1.247064in}}%
\pgfpathlineto{\pgfqpoint{0.843650in}{1.245526in}}%
\pgfpathlineto{\pgfqpoint{0.859306in}{1.251681in}}%
\pgfpathlineto{\pgfqpoint{0.865644in}{1.257191in}}%
\pgfpathlineto{\pgfqpoint{0.872723in}{1.270802in}}%
\pgfpathlineto{\pgfqpoint{0.870955in}{1.284413in}}%
\pgfpathlineto{\pgfqpoint{0.860326in}{1.298024in}}%
\pgfpathlineto{\pgfqpoint{0.859306in}{1.298762in}}%
\pgfpathlineto{\pgfqpoint{0.843650in}{1.303941in}}%
\pgfpathlineto{\pgfqpoint{0.827993in}{1.302647in}}%
\pgfpathlineto{\pgfqpoint{0.818601in}{1.298024in}}%
\pgfpathlineto{\pgfqpoint{0.812337in}{1.292578in}}%
\pgfpathlineto{\pgfqpoint{0.807019in}{1.284413in}}%
\pgfpathlineto{\pgfqpoint{0.805531in}{1.270802in}}%
\pgfpathlineto{\pgfqpoint{0.811488in}{1.257191in}}%
\pgfpathlineto{\pgfqpoint{0.812337in}{1.256304in}}%
\pgfpathclose%
\pgfpathmoveto{\pgfqpoint{1.125468in}{1.253838in}}%
\pgfpathlineto{\pgfqpoint{1.141125in}{1.246141in}}%
\pgfpathlineto{\pgfqpoint{1.156781in}{1.246141in}}%
\pgfpathlineto{\pgfqpoint{1.172438in}{1.253838in}}%
\pgfpathlineto{\pgfqpoint{1.175937in}{1.257191in}}%
\pgfpathlineto{\pgfqpoint{1.182394in}{1.270802in}}%
\pgfpathlineto{\pgfqpoint{1.180781in}{1.284413in}}%
\pgfpathlineto{\pgfqpoint{1.172438in}{1.296163in}}%
\pgfpathlineto{\pgfqpoint{1.169874in}{1.298024in}}%
\pgfpathlineto{\pgfqpoint{1.156781in}{1.303423in}}%
\pgfpathlineto{\pgfqpoint{1.141125in}{1.303423in}}%
\pgfpathlineto{\pgfqpoint{1.128032in}{1.298024in}}%
\pgfpathlineto{\pgfqpoint{1.125468in}{1.296163in}}%
\pgfpathlineto{\pgfqpoint{1.117125in}{1.284413in}}%
\pgfpathlineto{\pgfqpoint{1.115512in}{1.270802in}}%
\pgfpathlineto{\pgfqpoint{1.121969in}{1.257191in}}%
\pgfpathlineto{\pgfqpoint{1.125468in}{1.253838in}}%
\pgfpathclose%
\pgfpathmoveto{\pgfqpoint{1.438599in}{1.251681in}}%
\pgfpathlineto{\pgfqpoint{1.454256in}{1.245526in}}%
\pgfpathlineto{\pgfqpoint{1.469913in}{1.247064in}}%
\pgfpathlineto{\pgfqpoint{1.485569in}{1.256304in}}%
\pgfpathlineto{\pgfqpoint{1.486418in}{1.257191in}}%
\pgfpathlineto{\pgfqpoint{1.492375in}{1.270802in}}%
\pgfpathlineto{\pgfqpoint{1.490887in}{1.284413in}}%
\pgfpathlineto{\pgfqpoint{1.485569in}{1.292578in}}%
\pgfpathlineto{\pgfqpoint{1.479304in}{1.298024in}}%
\pgfpathlineto{\pgfqpoint{1.469913in}{1.302647in}}%
\pgfpathlineto{\pgfqpoint{1.454256in}{1.303941in}}%
\pgfpathlineto{\pgfqpoint{1.438599in}{1.298762in}}%
\pgfpathlineto{\pgfqpoint{1.437580in}{1.298024in}}%
\pgfpathlineto{\pgfqpoint{1.426951in}{1.284413in}}%
\pgfpathlineto{\pgfqpoint{1.425182in}{1.270802in}}%
\pgfpathlineto{\pgfqpoint{1.432262in}{1.257191in}}%
\pgfpathlineto{\pgfqpoint{1.438599in}{1.251681in}}%
\pgfpathclose%
\pgfpathmoveto{\pgfqpoint{1.751731in}{1.249834in}}%
\pgfpathlineto{\pgfqpoint{1.767387in}{1.245219in}}%
\pgfpathlineto{\pgfqpoint{1.783044in}{1.248295in}}%
\pgfpathlineto{\pgfqpoint{1.796015in}{1.257191in}}%
\pgfpathlineto{\pgfqpoint{1.798700in}{1.261254in}}%
\pgfpathlineto{\pgfqpoint{1.802577in}{1.270802in}}%
\pgfpathlineto{\pgfqpoint{1.801190in}{1.284413in}}%
\pgfpathlineto{\pgfqpoint{1.798700in}{1.288542in}}%
\pgfpathlineto{\pgfqpoint{1.789329in}{1.298024in}}%
\pgfpathlineto{\pgfqpoint{1.783044in}{1.301611in}}%
\pgfpathlineto{\pgfqpoint{1.767387in}{1.304199in}}%
\pgfpathlineto{\pgfqpoint{1.751731in}{1.300316in}}%
\pgfpathlineto{\pgfqpoint{1.748193in}{1.298024in}}%
\pgfpathlineto{\pgfqpoint{1.736376in}{1.284413in}}%
\pgfpathlineto{\pgfqpoint{1.736074in}{1.282328in}}%
\pgfpathlineto{\pgfqpoint{1.734971in}{1.270802in}}%
\pgfpathlineto{\pgfqpoint{1.736074in}{1.267890in}}%
\pgfpathlineto{\pgfqpoint{1.742281in}{1.257191in}}%
\pgfpathlineto{\pgfqpoint{1.751731in}{1.249834in}}%
\pgfpathclose%
\pgfpathmoveto{\pgfqpoint{0.530519in}{1.514617in}}%
\pgfpathlineto{\pgfqpoint{0.536016in}{1.515802in}}%
\pgfpathlineto{\pgfqpoint{0.546175in}{1.519143in}}%
\pgfpathlineto{\pgfqpoint{0.557988in}{1.529413in}}%
\pgfpathlineto{\pgfqpoint{0.561832in}{1.538244in}}%
\pgfpathlineto{\pgfqpoint{0.563195in}{1.543024in}}%
\pgfpathlineto{\pgfqpoint{0.561832in}{1.550169in}}%
\pgfpathlineto{\pgfqpoint{0.559956in}{1.556635in}}%
\pgfpathlineto{\pgfqpoint{0.546175in}{1.570235in}}%
\pgfpathlineto{\pgfqpoint{0.546146in}{1.570246in}}%
\pgfpathlineto{\pgfqpoint{0.530519in}{1.573857in}}%
\pgfpathlineto{\pgfqpoint{0.514862in}{1.571446in}}%
\pgfpathlineto{\pgfqpoint{0.512592in}{1.570246in}}%
\pgfpathlineto{\pgfqpoint{0.499205in}{1.558609in}}%
\pgfpathlineto{\pgfqpoint{0.497826in}{1.556635in}}%
\pgfpathlineto{\pgfqpoint{0.495052in}{1.543024in}}%
\pgfpathlineto{\pgfqpoint{0.499205in}{1.529438in}}%
\pgfpathlineto{\pgfqpoint{0.499218in}{1.529413in}}%
\pgfpathlineto{\pgfqpoint{0.514862in}{1.517432in}}%
\pgfpathlineto{\pgfqpoint{0.522300in}{1.515802in}}%
\pgfpathlineto{\pgfqpoint{0.530519in}{1.514617in}}%
\pgfpathclose%
\pgfpathmoveto{\pgfqpoint{0.843650in}{1.514843in}}%
\pgfpathlineto{\pgfqpoint{0.846999in}{1.515802in}}%
\pgfpathlineto{\pgfqpoint{0.859306in}{1.521198in}}%
\pgfpathlineto{\pgfqpoint{0.867769in}{1.529413in}}%
\pgfpathlineto{\pgfqpoint{0.873077in}{1.543024in}}%
\pgfpathlineto{\pgfqpoint{0.869539in}{1.556635in}}%
\pgfpathlineto{\pgfqpoint{0.859306in}{1.567912in}}%
\pgfpathlineto{\pgfqpoint{0.854632in}{1.570246in}}%
\pgfpathlineto{\pgfqpoint{0.843650in}{1.573616in}}%
\pgfpathlineto{\pgfqpoint{0.827993in}{1.572410in}}%
\pgfpathlineto{\pgfqpoint{0.823243in}{1.570246in}}%
\pgfpathlineto{\pgfqpoint{0.812337in}{1.562099in}}%
\pgfpathlineto{\pgfqpoint{0.808210in}{1.556635in}}%
\pgfpathlineto{\pgfqpoint{0.805233in}{1.543024in}}%
\pgfpathlineto{\pgfqpoint{0.809700in}{1.529413in}}%
\pgfpathlineto{\pgfqpoint{0.812337in}{1.526337in}}%
\pgfpathlineto{\pgfqpoint{0.827993in}{1.516064in}}%
\pgfpathlineto{\pgfqpoint{0.830391in}{1.515802in}}%
\pgfpathlineto{\pgfqpoint{0.843650in}{1.514843in}}%
\pgfpathclose%
\pgfpathmoveto{\pgfqpoint{1.141125in}{1.515296in}}%
\pgfpathlineto{\pgfqpoint{1.156781in}{1.515296in}}%
\pgfpathlineto{\pgfqpoint{1.158201in}{1.515802in}}%
\pgfpathlineto{\pgfqpoint{1.172438in}{1.523595in}}%
\pgfpathlineto{\pgfqpoint{1.177875in}{1.529413in}}%
\pgfpathlineto{\pgfqpoint{1.182716in}{1.543024in}}%
\pgfpathlineto{\pgfqpoint{1.179490in}{1.556635in}}%
\pgfpathlineto{\pgfqpoint{1.172438in}{1.565200in}}%
\pgfpathlineto{\pgfqpoint{1.164346in}{1.570246in}}%
\pgfpathlineto{\pgfqpoint{1.156781in}{1.573134in}}%
\pgfpathlineto{\pgfqpoint{1.141125in}{1.573134in}}%
\pgfpathlineto{\pgfqpoint{1.133560in}{1.570246in}}%
\pgfpathlineto{\pgfqpoint{1.125468in}{1.565200in}}%
\pgfpathlineto{\pgfqpoint{1.118416in}{1.556635in}}%
\pgfpathlineto{\pgfqpoint{1.115189in}{1.543024in}}%
\pgfpathlineto{\pgfqpoint{1.120031in}{1.529413in}}%
\pgfpathlineto{\pgfqpoint{1.125468in}{1.523595in}}%
\pgfpathlineto{\pgfqpoint{1.139705in}{1.515802in}}%
\pgfpathlineto{\pgfqpoint{1.141125in}{1.515296in}}%
\pgfpathclose%
\pgfpathmoveto{\pgfqpoint{1.454256in}{1.514843in}}%
\pgfpathlineto{\pgfqpoint{1.467515in}{1.515802in}}%
\pgfpathlineto{\pgfqpoint{1.469913in}{1.516064in}}%
\pgfpathlineto{\pgfqpoint{1.485569in}{1.526337in}}%
\pgfpathlineto{\pgfqpoint{1.488206in}{1.529413in}}%
\pgfpathlineto{\pgfqpoint{1.492673in}{1.543024in}}%
\pgfpathlineto{\pgfqpoint{1.489696in}{1.556635in}}%
\pgfpathlineto{\pgfqpoint{1.485569in}{1.562099in}}%
\pgfpathlineto{\pgfqpoint{1.474663in}{1.570246in}}%
\pgfpathlineto{\pgfqpoint{1.469913in}{1.572410in}}%
\pgfpathlineto{\pgfqpoint{1.454256in}{1.573616in}}%
\pgfpathlineto{\pgfqpoint{1.443274in}{1.570246in}}%
\pgfpathlineto{\pgfqpoint{1.438599in}{1.567912in}}%
\pgfpathlineto{\pgfqpoint{1.428367in}{1.556635in}}%
\pgfpathlineto{\pgfqpoint{1.424829in}{1.543024in}}%
\pgfpathlineto{\pgfqpoint{1.430137in}{1.529413in}}%
\pgfpathlineto{\pgfqpoint{1.438599in}{1.521198in}}%
\pgfpathlineto{\pgfqpoint{1.450907in}{1.515802in}}%
\pgfpathlineto{\pgfqpoint{1.454256in}{1.514843in}}%
\pgfpathclose%
\pgfpathmoveto{\pgfqpoint{1.767387in}{1.514617in}}%
\pgfpathlineto{\pgfqpoint{1.775606in}{1.515802in}}%
\pgfpathlineto{\pgfqpoint{1.783044in}{1.517432in}}%
\pgfpathlineto{\pgfqpoint{1.798688in}{1.529413in}}%
\pgfpathlineto{\pgfqpoint{1.798700in}{1.529438in}}%
\pgfpathlineto{\pgfqpoint{1.802854in}{1.543024in}}%
\pgfpathlineto{\pgfqpoint{1.800080in}{1.556635in}}%
\pgfpathlineto{\pgfqpoint{1.798700in}{1.558609in}}%
\pgfpathlineto{\pgfqpoint{1.785314in}{1.570246in}}%
\pgfpathlineto{\pgfqpoint{1.783044in}{1.571446in}}%
\pgfpathlineto{\pgfqpoint{1.767387in}{1.573857in}}%
\pgfpathlineto{\pgfqpoint{1.751760in}{1.570246in}}%
\pgfpathlineto{\pgfqpoint{1.751731in}{1.570235in}}%
\pgfpathlineto{\pgfqpoint{1.737950in}{1.556635in}}%
\pgfpathlineto{\pgfqpoint{1.736074in}{1.550169in}}%
\pgfpathlineto{\pgfqpoint{1.734711in}{1.543024in}}%
\pgfpathlineto{\pgfqpoint{1.736074in}{1.538244in}}%
\pgfpathlineto{\pgfqpoint{1.739918in}{1.529413in}}%
\pgfpathlineto{\pgfqpoint{1.751731in}{1.519143in}}%
\pgfpathlineto{\pgfqpoint{1.761890in}{1.515802in}}%
\pgfpathlineto{\pgfqpoint{1.767387in}{1.514617in}}%
\pgfpathclose%
\pgfusepath{fill}%
\end{pgfscope}%
\begin{pgfscope}%
\pgfpathrectangle{\pgfqpoint{0.373953in}{0.331635in}}{\pgfqpoint{1.550000in}{1.347500in}}%
\pgfusepath{clip}%
\pgfsetbuttcap%
\pgfsetroundjoin%
\definecolor{currentfill}{rgb}{0.993326,0.602275,0.414390}%
\pgfsetfillcolor{currentfill}%
\pgfsetlinewidth{0.000000pt}%
\definecolor{currentstroke}{rgb}{0.000000,0.000000,0.000000}%
\pgfsetstrokecolor{currentstroke}%
\pgfsetdash{}{0pt}%
\pgfpathmoveto{\pgfqpoint{0.514862in}{0.411126in}}%
\pgfpathlineto{\pgfqpoint{0.530519in}{0.409600in}}%
\pgfpathlineto{\pgfqpoint{0.546175in}{0.411889in}}%
\pgfpathlineto{\pgfqpoint{0.549874in}{0.413302in}}%
\pgfpathlineto{\pgfqpoint{0.561832in}{0.418546in}}%
\pgfpathlineto{\pgfqpoint{0.573777in}{0.426913in}}%
\pgfpathlineto{\pgfqpoint{0.577488in}{0.430381in}}%
\pgfpathlineto{\pgfqpoint{0.585901in}{0.440524in}}%
\pgfpathlineto{\pgfqpoint{0.592553in}{0.454135in}}%
\pgfpathlineto{\pgfqpoint{0.593145in}{0.458361in}}%
\pgfpathlineto{\pgfqpoint{0.594341in}{0.467746in}}%
\pgfpathlineto{\pgfqpoint{0.593145in}{0.474024in}}%
\pgfpathlineto{\pgfqpoint{0.591604in}{0.481357in}}%
\pgfpathlineto{\pgfqpoint{0.583999in}{0.494968in}}%
\pgfpathlineto{\pgfqpoint{0.577488in}{0.502291in}}%
\pgfpathlineto{\pgfqpoint{0.570255in}{0.508579in}}%
\pgfpathlineto{\pgfqpoint{0.561832in}{0.514240in}}%
\pgfpathlineto{\pgfqpoint{0.546175in}{0.520851in}}%
\pgfpathlineto{\pgfqpoint{0.537740in}{0.522191in}}%
\pgfpathlineto{\pgfqpoint{0.530519in}{0.523231in}}%
\pgfpathlineto{\pgfqpoint{0.519724in}{0.522191in}}%
\pgfpathlineto{\pgfqpoint{0.514862in}{0.521676in}}%
\pgfpathlineto{\pgfqpoint{0.499205in}{0.515894in}}%
\pgfpathlineto{\pgfqpoint{0.487539in}{0.508579in}}%
\pgfpathlineto{\pgfqpoint{0.483549in}{0.505353in}}%
\pgfpathlineto{\pgfqpoint{0.473925in}{0.494968in}}%
\pgfpathlineto{\pgfqpoint{0.467892in}{0.484573in}}%
\pgfpathlineto{\pgfqpoint{0.466267in}{0.481357in}}%
\pgfpathlineto{\pgfqpoint{0.463635in}{0.467746in}}%
\pgfpathlineto{\pgfqpoint{0.465389in}{0.454135in}}%
\pgfpathlineto{\pgfqpoint{0.467892in}{0.448513in}}%
\pgfpathlineto{\pgfqpoint{0.471962in}{0.440524in}}%
\pgfpathlineto{\pgfqpoint{0.483549in}{0.427122in}}%
\pgfpathlineto{\pgfqpoint{0.483790in}{0.426913in}}%
\pgfpathlineto{\pgfqpoint{0.499205in}{0.416840in}}%
\pgfpathlineto{\pgfqpoint{0.508395in}{0.413302in}}%
\pgfpathlineto{\pgfqpoint{0.514862in}{0.411126in}}%
\pgfpathclose%
\pgfpathmoveto{\pgfqpoint{0.512592in}{0.440524in}}%
\pgfpathlineto{\pgfqpoint{0.499205in}{0.452161in}}%
\pgfpathlineto{\pgfqpoint{0.497826in}{0.454135in}}%
\pgfpathlineto{\pgfqpoint{0.495052in}{0.467746in}}%
\pgfpathlineto{\pgfqpoint{0.499205in}{0.481332in}}%
\pgfpathlineto{\pgfqpoint{0.499218in}{0.481357in}}%
\pgfpathlineto{\pgfqpoint{0.514862in}{0.493338in}}%
\pgfpathlineto{\pgfqpoint{0.522300in}{0.494968in}}%
\pgfpathlineto{\pgfqpoint{0.530519in}{0.496153in}}%
\pgfpathlineto{\pgfqpoint{0.536016in}{0.494968in}}%
\pgfpathlineto{\pgfqpoint{0.546175in}{0.491627in}}%
\pgfpathlineto{\pgfqpoint{0.557988in}{0.481357in}}%
\pgfpathlineto{\pgfqpoint{0.561832in}{0.472526in}}%
\pgfpathlineto{\pgfqpoint{0.563195in}{0.467746in}}%
\pgfpathlineto{\pgfqpoint{0.561832in}{0.460601in}}%
\pgfpathlineto{\pgfqpoint{0.559956in}{0.454135in}}%
\pgfpathlineto{\pgfqpoint{0.546175in}{0.440535in}}%
\pgfpathlineto{\pgfqpoint{0.546146in}{0.440524in}}%
\pgfpathlineto{\pgfqpoint{0.530519in}{0.436913in}}%
\pgfpathlineto{\pgfqpoint{0.514862in}{0.439324in}}%
\pgfpathlineto{\pgfqpoint{0.512592in}{0.440524in}}%
\pgfpathclose%
\pgfpathmoveto{\pgfqpoint{0.827993in}{0.410515in}}%
\pgfpathlineto{\pgfqpoint{0.843650in}{0.409753in}}%
\pgfpathlineto{\pgfqpoint{0.859306in}{0.412805in}}%
\pgfpathlineto{\pgfqpoint{0.860471in}{0.413302in}}%
\pgfpathlineto{\pgfqpoint{0.874963in}{0.420424in}}%
\pgfpathlineto{\pgfqpoint{0.883666in}{0.426913in}}%
\pgfpathlineto{\pgfqpoint{0.890620in}{0.433853in}}%
\pgfpathlineto{\pgfqpoint{0.895952in}{0.440524in}}%
\pgfpathlineto{\pgfqpoint{0.902431in}{0.454135in}}%
\pgfpathlineto{\pgfqpoint{0.904279in}{0.467746in}}%
\pgfpathlineto{\pgfqpoint{0.901506in}{0.481357in}}%
\pgfpathlineto{\pgfqpoint{0.894099in}{0.494968in}}%
\pgfpathlineto{\pgfqpoint{0.890620in}{0.499029in}}%
\pgfpathlineto{\pgfqpoint{0.880331in}{0.508579in}}%
\pgfpathlineto{\pgfqpoint{0.874963in}{0.512420in}}%
\pgfpathlineto{\pgfqpoint{0.859306in}{0.519860in}}%
\pgfpathlineto{\pgfqpoint{0.848298in}{0.522191in}}%
\pgfpathlineto{\pgfqpoint{0.843650in}{0.523079in}}%
\pgfpathlineto{\pgfqpoint{0.827993in}{0.522324in}}%
\pgfpathlineto{\pgfqpoint{0.827525in}{0.522191in}}%
\pgfpathlineto{\pgfqpoint{0.812337in}{0.517382in}}%
\pgfpathlineto{\pgfqpoint{0.797169in}{0.508579in}}%
\pgfpathlineto{\pgfqpoint{0.796680in}{0.508215in}}%
\pgfpathlineto{\pgfqpoint{0.783826in}{0.494968in}}%
\pgfpathlineto{\pgfqpoint{0.781024in}{0.490368in}}%
\pgfpathlineto{\pgfqpoint{0.776353in}{0.481357in}}%
\pgfpathlineto{\pgfqpoint{0.773686in}{0.467746in}}%
\pgfpathlineto{\pgfqpoint{0.775464in}{0.454135in}}%
\pgfpathlineto{\pgfqpoint{0.781024in}{0.441959in}}%
\pgfpathlineto{\pgfqpoint{0.781791in}{0.440524in}}%
\pgfpathlineto{\pgfqpoint{0.794003in}{0.426913in}}%
\pgfpathlineto{\pgfqpoint{0.796680in}{0.424690in}}%
\pgfpathlineto{\pgfqpoint{0.812337in}{0.415305in}}%
\pgfpathlineto{\pgfqpoint{0.818391in}{0.413302in}}%
\pgfpathlineto{\pgfqpoint{0.827993in}{0.410515in}}%
\pgfpathclose%
\pgfpathmoveto{\pgfqpoint{0.823243in}{0.440524in}}%
\pgfpathlineto{\pgfqpoint{0.812337in}{0.448671in}}%
\pgfpathlineto{\pgfqpoint{0.808210in}{0.454135in}}%
\pgfpathlineto{\pgfqpoint{0.805233in}{0.467746in}}%
\pgfpathlineto{\pgfqpoint{0.809700in}{0.481357in}}%
\pgfpathlineto{\pgfqpoint{0.812337in}{0.484433in}}%
\pgfpathlineto{\pgfqpoint{0.827993in}{0.494706in}}%
\pgfpathlineto{\pgfqpoint{0.830391in}{0.494968in}}%
\pgfpathlineto{\pgfqpoint{0.843650in}{0.495927in}}%
\pgfpathlineto{\pgfqpoint{0.846999in}{0.494968in}}%
\pgfpathlineto{\pgfqpoint{0.859306in}{0.489572in}}%
\pgfpathlineto{\pgfqpoint{0.867769in}{0.481357in}}%
\pgfpathlineto{\pgfqpoint{0.873077in}{0.467746in}}%
\pgfpathlineto{\pgfqpoint{0.869539in}{0.454135in}}%
\pgfpathlineto{\pgfqpoint{0.859306in}{0.442858in}}%
\pgfpathlineto{\pgfqpoint{0.854632in}{0.440524in}}%
\pgfpathlineto{\pgfqpoint{0.843650in}{0.437154in}}%
\pgfpathlineto{\pgfqpoint{0.827993in}{0.438360in}}%
\pgfpathlineto{\pgfqpoint{0.823243in}{0.440524in}}%
\pgfpathclose%
\pgfpathmoveto{\pgfqpoint{1.141125in}{0.410058in}}%
\pgfpathlineto{\pgfqpoint{1.156781in}{0.410058in}}%
\pgfpathlineto{\pgfqpoint{1.170124in}{0.413302in}}%
\pgfpathlineto{\pgfqpoint{1.172438in}{0.413941in}}%
\pgfpathlineto{\pgfqpoint{1.188094in}{0.422472in}}%
\pgfpathlineto{\pgfqpoint{1.193724in}{0.426913in}}%
\pgfpathlineto{\pgfqpoint{1.203751in}{0.437533in}}%
\pgfpathlineto{\pgfqpoint{1.206064in}{0.440524in}}%
\pgfpathlineto{\pgfqpoint{1.212404in}{0.454135in}}%
\pgfpathlineto{\pgfqpoint{1.214212in}{0.467746in}}%
\pgfpathlineto{\pgfqpoint{1.211500in}{0.481357in}}%
\pgfpathlineto{\pgfqpoint{1.204251in}{0.494968in}}%
\pgfpathlineto{\pgfqpoint{1.203751in}{0.495571in}}%
\pgfpathlineto{\pgfqpoint{1.190545in}{0.508579in}}%
\pgfpathlineto{\pgfqpoint{1.188094in}{0.510435in}}%
\pgfpathlineto{\pgfqpoint{1.172438in}{0.518704in}}%
\pgfpathlineto{\pgfqpoint{1.159247in}{0.522191in}}%
\pgfpathlineto{\pgfqpoint{1.156781in}{0.522777in}}%
\pgfpathlineto{\pgfqpoint{1.141125in}{0.522777in}}%
\pgfpathlineto{\pgfqpoint{1.138659in}{0.522191in}}%
\pgfpathlineto{\pgfqpoint{1.125468in}{0.518704in}}%
\pgfpathlineto{\pgfqpoint{1.109812in}{0.510435in}}%
\pgfpathlineto{\pgfqpoint{1.107361in}{0.508579in}}%
\pgfpathlineto{\pgfqpoint{1.094155in}{0.495571in}}%
\pgfpathlineto{\pgfqpoint{1.093655in}{0.494968in}}%
\pgfpathlineto{\pgfqpoint{1.086406in}{0.481357in}}%
\pgfpathlineto{\pgfqpoint{1.083693in}{0.467746in}}%
\pgfpathlineto{\pgfqpoint{1.085502in}{0.454135in}}%
\pgfpathlineto{\pgfqpoint{1.091842in}{0.440524in}}%
\pgfpathlineto{\pgfqpoint{1.094155in}{0.437533in}}%
\pgfpathlineto{\pgfqpoint{1.104182in}{0.426913in}}%
\pgfpathlineto{\pgfqpoint{1.109812in}{0.422472in}}%
\pgfpathlineto{\pgfqpoint{1.125468in}{0.413941in}}%
\pgfpathlineto{\pgfqpoint{1.127782in}{0.413302in}}%
\pgfpathlineto{\pgfqpoint{1.141125in}{0.410058in}}%
\pgfpathclose%
\pgfpathmoveto{\pgfqpoint{1.133560in}{0.440524in}}%
\pgfpathlineto{\pgfqpoint{1.125468in}{0.445570in}}%
\pgfpathlineto{\pgfqpoint{1.118416in}{0.454135in}}%
\pgfpathlineto{\pgfqpoint{1.115189in}{0.467746in}}%
\pgfpathlineto{\pgfqpoint{1.120031in}{0.481357in}}%
\pgfpathlineto{\pgfqpoint{1.125468in}{0.487175in}}%
\pgfpathlineto{\pgfqpoint{1.139705in}{0.494968in}}%
\pgfpathlineto{\pgfqpoint{1.141125in}{0.495474in}}%
\pgfpathlineto{\pgfqpoint{1.156781in}{0.495474in}}%
\pgfpathlineto{\pgfqpoint{1.158201in}{0.494968in}}%
\pgfpathlineto{\pgfqpoint{1.172438in}{0.487175in}}%
\pgfpathlineto{\pgfqpoint{1.177875in}{0.481357in}}%
\pgfpathlineto{\pgfqpoint{1.182716in}{0.467746in}}%
\pgfpathlineto{\pgfqpoint{1.179490in}{0.454135in}}%
\pgfpathlineto{\pgfqpoint{1.172438in}{0.445570in}}%
\pgfpathlineto{\pgfqpoint{1.164346in}{0.440524in}}%
\pgfpathlineto{\pgfqpoint{1.156781in}{0.437636in}}%
\pgfpathlineto{\pgfqpoint{1.141125in}{0.437636in}}%
\pgfpathlineto{\pgfqpoint{1.133560in}{0.440524in}}%
\pgfpathclose%
\pgfpathmoveto{\pgfqpoint{1.438599in}{0.412805in}}%
\pgfpathlineto{\pgfqpoint{1.454256in}{0.409753in}}%
\pgfpathlineto{\pgfqpoint{1.469913in}{0.410515in}}%
\pgfpathlineto{\pgfqpoint{1.479515in}{0.413302in}}%
\pgfpathlineto{\pgfqpoint{1.485569in}{0.415305in}}%
\pgfpathlineto{\pgfqpoint{1.501226in}{0.424690in}}%
\pgfpathlineto{\pgfqpoint{1.503903in}{0.426913in}}%
\pgfpathlineto{\pgfqpoint{1.516115in}{0.440524in}}%
\pgfpathlineto{\pgfqpoint{1.516882in}{0.441959in}}%
\pgfpathlineto{\pgfqpoint{1.522442in}{0.454135in}}%
\pgfpathlineto{\pgfqpoint{1.524219in}{0.467746in}}%
\pgfpathlineto{\pgfqpoint{1.521553in}{0.481357in}}%
\pgfpathlineto{\pgfqpoint{1.516882in}{0.490368in}}%
\pgfpathlineto{\pgfqpoint{1.514079in}{0.494968in}}%
\pgfpathlineto{\pgfqpoint{1.501226in}{0.508215in}}%
\pgfpathlineto{\pgfqpoint{1.500737in}{0.508579in}}%
\pgfpathlineto{\pgfqpoint{1.485569in}{0.517382in}}%
\pgfpathlineto{\pgfqpoint{1.470381in}{0.522191in}}%
\pgfpathlineto{\pgfqpoint{1.469913in}{0.522324in}}%
\pgfpathlineto{\pgfqpoint{1.454256in}{0.523079in}}%
\pgfpathlineto{\pgfqpoint{1.449608in}{0.522191in}}%
\pgfpathlineto{\pgfqpoint{1.438599in}{0.519860in}}%
\pgfpathlineto{\pgfqpoint{1.422943in}{0.512420in}}%
\pgfpathlineto{\pgfqpoint{1.417575in}{0.508579in}}%
\pgfpathlineto{\pgfqpoint{1.407286in}{0.499029in}}%
\pgfpathlineto{\pgfqpoint{1.403807in}{0.494968in}}%
\pgfpathlineto{\pgfqpoint{1.396399in}{0.481357in}}%
\pgfpathlineto{\pgfqpoint{1.393627in}{0.467746in}}%
\pgfpathlineto{\pgfqpoint{1.395475in}{0.454135in}}%
\pgfpathlineto{\pgfqpoint{1.401954in}{0.440524in}}%
\pgfpathlineto{\pgfqpoint{1.407286in}{0.433853in}}%
\pgfpathlineto{\pgfqpoint{1.414240in}{0.426913in}}%
\pgfpathlineto{\pgfqpoint{1.422943in}{0.420424in}}%
\pgfpathlineto{\pgfqpoint{1.437435in}{0.413302in}}%
\pgfpathlineto{\pgfqpoint{1.438599in}{0.412805in}}%
\pgfpathclose%
\pgfpathmoveto{\pgfqpoint{1.443274in}{0.440524in}}%
\pgfpathlineto{\pgfqpoint{1.438599in}{0.442858in}}%
\pgfpathlineto{\pgfqpoint{1.428367in}{0.454135in}}%
\pgfpathlineto{\pgfqpoint{1.424829in}{0.467746in}}%
\pgfpathlineto{\pgfqpoint{1.430137in}{0.481357in}}%
\pgfpathlineto{\pgfqpoint{1.438599in}{0.489572in}}%
\pgfpathlineto{\pgfqpoint{1.450907in}{0.494968in}}%
\pgfpathlineto{\pgfqpoint{1.454256in}{0.495927in}}%
\pgfpathlineto{\pgfqpoint{1.467515in}{0.494968in}}%
\pgfpathlineto{\pgfqpoint{1.469913in}{0.494706in}}%
\pgfpathlineto{\pgfqpoint{1.485569in}{0.484433in}}%
\pgfpathlineto{\pgfqpoint{1.488206in}{0.481357in}}%
\pgfpathlineto{\pgfqpoint{1.492673in}{0.467746in}}%
\pgfpathlineto{\pgfqpoint{1.489696in}{0.454135in}}%
\pgfpathlineto{\pgfqpoint{1.485569in}{0.448671in}}%
\pgfpathlineto{\pgfqpoint{1.474663in}{0.440524in}}%
\pgfpathlineto{\pgfqpoint{1.469913in}{0.438360in}}%
\pgfpathlineto{\pgfqpoint{1.454256in}{0.437154in}}%
\pgfpathlineto{\pgfqpoint{1.443274in}{0.440524in}}%
\pgfpathclose%
\pgfpathmoveto{\pgfqpoint{1.751731in}{0.411889in}}%
\pgfpathlineto{\pgfqpoint{1.767387in}{0.409600in}}%
\pgfpathlineto{\pgfqpoint{1.783044in}{0.411126in}}%
\pgfpathlineto{\pgfqpoint{1.789510in}{0.413302in}}%
\pgfpathlineto{\pgfqpoint{1.798700in}{0.416840in}}%
\pgfpathlineto{\pgfqpoint{1.814116in}{0.426913in}}%
\pgfpathlineto{\pgfqpoint{1.814357in}{0.427122in}}%
\pgfpathlineto{\pgfqpoint{1.825944in}{0.440524in}}%
\pgfpathlineto{\pgfqpoint{1.830014in}{0.448513in}}%
\pgfpathlineto{\pgfqpoint{1.832517in}{0.454135in}}%
\pgfpathlineto{\pgfqpoint{1.834271in}{0.467746in}}%
\pgfpathlineto{\pgfqpoint{1.831639in}{0.481357in}}%
\pgfpathlineto{\pgfqpoint{1.830014in}{0.484573in}}%
\pgfpathlineto{\pgfqpoint{1.823981in}{0.494968in}}%
\pgfpathlineto{\pgfqpoint{1.814357in}{0.505353in}}%
\pgfpathlineto{\pgfqpoint{1.810367in}{0.508579in}}%
\pgfpathlineto{\pgfqpoint{1.798700in}{0.515894in}}%
\pgfpathlineto{\pgfqpoint{1.783044in}{0.521676in}}%
\pgfpathlineto{\pgfqpoint{1.778182in}{0.522191in}}%
\pgfpathlineto{\pgfqpoint{1.767387in}{0.523231in}}%
\pgfpathlineto{\pgfqpoint{1.760166in}{0.522191in}}%
\pgfpathlineto{\pgfqpoint{1.751731in}{0.520851in}}%
\pgfpathlineto{\pgfqpoint{1.736074in}{0.514240in}}%
\pgfpathlineto{\pgfqpoint{1.727651in}{0.508579in}}%
\pgfpathlineto{\pgfqpoint{1.720418in}{0.502291in}}%
\pgfpathlineto{\pgfqpoint{1.713907in}{0.494968in}}%
\pgfpathlineto{\pgfqpoint{1.706302in}{0.481357in}}%
\pgfpathlineto{\pgfqpoint{1.704761in}{0.474024in}}%
\pgfpathlineto{\pgfqpoint{1.703565in}{0.467746in}}%
\pgfpathlineto{\pgfqpoint{1.704761in}{0.458361in}}%
\pgfpathlineto{\pgfqpoint{1.705353in}{0.454135in}}%
\pgfpathlineto{\pgfqpoint{1.712004in}{0.440524in}}%
\pgfpathlineto{\pgfqpoint{1.720418in}{0.430381in}}%
\pgfpathlineto{\pgfqpoint{1.724128in}{0.426913in}}%
\pgfpathlineto{\pgfqpoint{1.736074in}{0.418546in}}%
\pgfpathlineto{\pgfqpoint{1.748032in}{0.413302in}}%
\pgfpathlineto{\pgfqpoint{1.751731in}{0.411889in}}%
\pgfpathclose%
\pgfpathmoveto{\pgfqpoint{1.751760in}{0.440524in}}%
\pgfpathlineto{\pgfqpoint{1.751731in}{0.440535in}}%
\pgfpathlineto{\pgfqpoint{1.737950in}{0.454135in}}%
\pgfpathlineto{\pgfqpoint{1.736074in}{0.460601in}}%
\pgfpathlineto{\pgfqpoint{1.734711in}{0.467746in}}%
\pgfpathlineto{\pgfqpoint{1.736074in}{0.472526in}}%
\pgfpathlineto{\pgfqpoint{1.739918in}{0.481357in}}%
\pgfpathlineto{\pgfqpoint{1.751731in}{0.491627in}}%
\pgfpathlineto{\pgfqpoint{1.761890in}{0.494968in}}%
\pgfpathlineto{\pgfqpoint{1.767387in}{0.496153in}}%
\pgfpathlineto{\pgfqpoint{1.775606in}{0.494968in}}%
\pgfpathlineto{\pgfqpoint{1.783044in}{0.493338in}}%
\pgfpathlineto{\pgfqpoint{1.798688in}{0.481357in}}%
\pgfpathlineto{\pgfqpoint{1.798700in}{0.481332in}}%
\pgfpathlineto{\pgfqpoint{1.802854in}{0.467746in}}%
\pgfpathlineto{\pgfqpoint{1.800080in}{0.454135in}}%
\pgfpathlineto{\pgfqpoint{1.798700in}{0.452161in}}%
\pgfpathlineto{\pgfqpoint{1.785314in}{0.440524in}}%
\pgfpathlineto{\pgfqpoint{1.783044in}{0.439324in}}%
\pgfpathlineto{\pgfqpoint{1.767387in}{0.436913in}}%
\pgfpathlineto{\pgfqpoint{1.751760in}{0.440524in}}%
\pgfpathclose%
\pgfpathmoveto{\pgfqpoint{0.514862in}{0.680690in}}%
\pgfpathlineto{\pgfqpoint{0.530519in}{0.679145in}}%
\pgfpathlineto{\pgfqpoint{0.546175in}{0.681463in}}%
\pgfpathlineto{\pgfqpoint{0.556540in}{0.685524in}}%
\pgfpathlineto{\pgfqpoint{0.561832in}{0.687961in}}%
\pgfpathlineto{\pgfqpoint{0.577069in}{0.699135in}}%
\pgfpathlineto{\pgfqpoint{0.577488in}{0.699560in}}%
\pgfpathlineto{\pgfqpoint{0.587613in}{0.712746in}}%
\pgfpathlineto{\pgfqpoint{0.593145in}{0.725950in}}%
\pgfpathlineto{\pgfqpoint{0.593298in}{0.726357in}}%
\pgfpathlineto{\pgfqpoint{0.594167in}{0.739968in}}%
\pgfpathlineto{\pgfqpoint{0.593145in}{0.744009in}}%
\pgfpathlineto{\pgfqpoint{0.590464in}{0.753579in}}%
\pgfpathlineto{\pgfqpoint{0.581906in}{0.767191in}}%
\pgfpathlineto{\pgfqpoint{0.577488in}{0.771857in}}%
\pgfpathlineto{\pgfqpoint{0.566503in}{0.780802in}}%
\pgfpathlineto{\pgfqpoint{0.561832in}{0.783826in}}%
\pgfpathlineto{\pgfqpoint{0.546175in}{0.790266in}}%
\pgfpathlineto{\pgfqpoint{0.530519in}{0.792676in}}%
\pgfpathlineto{\pgfqpoint{0.514862in}{0.791070in}}%
\pgfpathlineto{\pgfqpoint{0.499205in}{0.785437in}}%
\pgfpathlineto{\pgfqpoint{0.491532in}{0.780802in}}%
\pgfpathlineto{\pgfqpoint{0.483549in}{0.774757in}}%
\pgfpathlineto{\pgfqpoint{0.476084in}{0.767191in}}%
\pgfpathlineto{\pgfqpoint{0.467892in}{0.754592in}}%
\pgfpathlineto{\pgfqpoint{0.467321in}{0.753579in}}%
\pgfpathlineto{\pgfqpoint{0.463810in}{0.739968in}}%
\pgfpathlineto{\pgfqpoint{0.464687in}{0.726357in}}%
\pgfpathlineto{\pgfqpoint{0.467892in}{0.718010in}}%
\pgfpathlineto{\pgfqpoint{0.470196in}{0.712746in}}%
\pgfpathlineto{\pgfqpoint{0.480992in}{0.699135in}}%
\pgfpathlineto{\pgfqpoint{0.483549in}{0.696807in}}%
\pgfpathlineto{\pgfqpoint{0.499205in}{0.686191in}}%
\pgfpathlineto{\pgfqpoint{0.500856in}{0.685524in}}%
\pgfpathlineto{\pgfqpoint{0.514862in}{0.680690in}}%
\pgfpathclose%
\pgfpathmoveto{\pgfqpoint{0.508577in}{0.712746in}}%
\pgfpathlineto{\pgfqpoint{0.499205in}{0.722228in}}%
\pgfpathlineto{\pgfqpoint{0.496716in}{0.726357in}}%
\pgfpathlineto{\pgfqpoint{0.495329in}{0.739968in}}%
\pgfpathlineto{\pgfqpoint{0.499205in}{0.749516in}}%
\pgfpathlineto{\pgfqpoint{0.501891in}{0.753579in}}%
\pgfpathlineto{\pgfqpoint{0.514862in}{0.762475in}}%
\pgfpathlineto{\pgfqpoint{0.530519in}{0.765551in}}%
\pgfpathlineto{\pgfqpoint{0.546175in}{0.760936in}}%
\pgfpathlineto{\pgfqpoint{0.555625in}{0.753579in}}%
\pgfpathlineto{\pgfqpoint{0.561832in}{0.742880in}}%
\pgfpathlineto{\pgfqpoint{0.562934in}{0.739968in}}%
\pgfpathlineto{\pgfqpoint{0.561832in}{0.728442in}}%
\pgfpathlineto{\pgfqpoint{0.561530in}{0.726357in}}%
\pgfpathlineto{\pgfqpoint{0.549713in}{0.712746in}}%
\pgfpathlineto{\pgfqpoint{0.546175in}{0.710454in}}%
\pgfpathlineto{\pgfqpoint{0.530519in}{0.706571in}}%
\pgfpathlineto{\pgfqpoint{0.514862in}{0.709159in}}%
\pgfpathlineto{\pgfqpoint{0.508577in}{0.712746in}}%
\pgfpathclose%
\pgfpathmoveto{\pgfqpoint{0.812337in}{0.684714in}}%
\pgfpathlineto{\pgfqpoint{0.827993in}{0.680072in}}%
\pgfpathlineto{\pgfqpoint{0.843650in}{0.679300in}}%
\pgfpathlineto{\pgfqpoint{0.859306in}{0.682392in}}%
\pgfpathlineto{\pgfqpoint{0.866466in}{0.685524in}}%
\pgfpathlineto{\pgfqpoint{0.874963in}{0.689907in}}%
\pgfpathlineto{\pgfqpoint{0.886782in}{0.699135in}}%
\pgfpathlineto{\pgfqpoint{0.890620in}{0.703286in}}%
\pgfpathlineto{\pgfqpoint{0.897619in}{0.712746in}}%
\pgfpathlineto{\pgfqpoint{0.903170in}{0.726357in}}%
\pgfpathlineto{\pgfqpoint{0.904094in}{0.739968in}}%
\pgfpathlineto{\pgfqpoint{0.900396in}{0.753579in}}%
\pgfpathlineto{\pgfqpoint{0.892060in}{0.767191in}}%
\pgfpathlineto{\pgfqpoint{0.890620in}{0.768770in}}%
\pgfpathlineto{\pgfqpoint{0.876779in}{0.780802in}}%
\pgfpathlineto{\pgfqpoint{0.874963in}{0.782054in}}%
\pgfpathlineto{\pgfqpoint{0.859306in}{0.789301in}}%
\pgfpathlineto{\pgfqpoint{0.843650in}{0.792516in}}%
\pgfpathlineto{\pgfqpoint{0.827993in}{0.791713in}}%
\pgfpathlineto{\pgfqpoint{0.812337in}{0.786887in}}%
\pgfpathlineto{\pgfqpoint{0.801455in}{0.780802in}}%
\pgfpathlineto{\pgfqpoint{0.796680in}{0.777466in}}%
\pgfpathlineto{\pgfqpoint{0.786066in}{0.767191in}}%
\pgfpathlineto{\pgfqpoint{0.781024in}{0.759804in}}%
\pgfpathlineto{\pgfqpoint{0.777421in}{0.753579in}}%
\pgfpathlineto{\pgfqpoint{0.773864in}{0.739968in}}%
\pgfpathlineto{\pgfqpoint{0.774753in}{0.726357in}}%
\pgfpathlineto{\pgfqpoint{0.780092in}{0.712746in}}%
\pgfpathlineto{\pgfqpoint{0.781024in}{0.711406in}}%
\pgfpathlineto{\pgfqpoint{0.791155in}{0.699135in}}%
\pgfpathlineto{\pgfqpoint{0.796680in}{0.694332in}}%
\pgfpathlineto{\pgfqpoint{0.810796in}{0.685524in}}%
\pgfpathlineto{\pgfqpoint{0.812337in}{0.684714in}}%
\pgfpathclose%
\pgfpathmoveto{\pgfqpoint{0.818601in}{0.712746in}}%
\pgfpathlineto{\pgfqpoint{0.812337in}{0.718192in}}%
\pgfpathlineto{\pgfqpoint{0.807019in}{0.726357in}}%
\pgfpathlineto{\pgfqpoint{0.805531in}{0.739968in}}%
\pgfpathlineto{\pgfqpoint{0.811488in}{0.753579in}}%
\pgfpathlineto{\pgfqpoint{0.812337in}{0.754466in}}%
\pgfpathlineto{\pgfqpoint{0.827993in}{0.763706in}}%
\pgfpathlineto{\pgfqpoint{0.843650in}{0.765244in}}%
\pgfpathlineto{\pgfqpoint{0.859306in}{0.759089in}}%
\pgfpathlineto{\pgfqpoint{0.865644in}{0.753579in}}%
\pgfpathlineto{\pgfqpoint{0.872723in}{0.739968in}}%
\pgfpathlineto{\pgfqpoint{0.870955in}{0.726357in}}%
\pgfpathlineto{\pgfqpoint{0.860326in}{0.712746in}}%
\pgfpathlineto{\pgfqpoint{0.859306in}{0.712008in}}%
\pgfpathlineto{\pgfqpoint{0.843650in}{0.706829in}}%
\pgfpathlineto{\pgfqpoint{0.827993in}{0.708123in}}%
\pgfpathlineto{\pgfqpoint{0.818601in}{0.712746in}}%
\pgfpathclose%
\pgfpathmoveto{\pgfqpoint{1.125468in}{0.683475in}}%
\pgfpathlineto{\pgfqpoint{1.141125in}{0.679609in}}%
\pgfpathlineto{\pgfqpoint{1.156781in}{0.679609in}}%
\pgfpathlineto{\pgfqpoint{1.172438in}{0.683475in}}%
\pgfpathlineto{\pgfqpoint{1.176687in}{0.685524in}}%
\pgfpathlineto{\pgfqpoint{1.188094in}{0.692031in}}%
\pgfpathlineto{\pgfqpoint{1.196693in}{0.699135in}}%
\pgfpathlineto{\pgfqpoint{1.203751in}{0.707236in}}%
\pgfpathlineto{\pgfqpoint{1.207696in}{0.712746in}}%
\pgfpathlineto{\pgfqpoint{1.213128in}{0.726357in}}%
\pgfpathlineto{\pgfqpoint{1.214032in}{0.739968in}}%
\pgfpathlineto{\pgfqpoint{1.210413in}{0.753579in}}%
\pgfpathlineto{\pgfqpoint{1.203751in}{0.764760in}}%
\pgfpathlineto{\pgfqpoint{1.202000in}{0.767191in}}%
\pgfpathlineto{\pgfqpoint{1.188094in}{0.779982in}}%
\pgfpathlineto{\pgfqpoint{1.186812in}{0.780802in}}%
\pgfpathlineto{\pgfqpoint{1.172438in}{0.788175in}}%
\pgfpathlineto{\pgfqpoint{1.156781in}{0.792194in}}%
\pgfpathlineto{\pgfqpoint{1.141125in}{0.792194in}}%
\pgfpathlineto{\pgfqpoint{1.125468in}{0.788175in}}%
\pgfpathlineto{\pgfqpoint{1.111094in}{0.780802in}}%
\pgfpathlineto{\pgfqpoint{1.109812in}{0.779982in}}%
\pgfpathlineto{\pgfqpoint{1.095905in}{0.767191in}}%
\pgfpathlineto{\pgfqpoint{1.094155in}{0.764760in}}%
\pgfpathlineto{\pgfqpoint{1.087492in}{0.753579in}}%
\pgfpathlineto{\pgfqpoint{1.083874in}{0.739968in}}%
\pgfpathlineto{\pgfqpoint{1.084778in}{0.726357in}}%
\pgfpathlineto{\pgfqpoint{1.090210in}{0.712746in}}%
\pgfpathlineto{\pgfqpoint{1.094155in}{0.707236in}}%
\pgfpathlineto{\pgfqpoint{1.101212in}{0.699135in}}%
\pgfpathlineto{\pgfqpoint{1.109812in}{0.692031in}}%
\pgfpathlineto{\pgfqpoint{1.121218in}{0.685524in}}%
\pgfpathlineto{\pgfqpoint{1.125468in}{0.683475in}}%
\pgfpathclose%
\pgfpathmoveto{\pgfqpoint{1.128032in}{0.712746in}}%
\pgfpathlineto{\pgfqpoint{1.125468in}{0.714607in}}%
\pgfpathlineto{\pgfqpoint{1.117125in}{0.726357in}}%
\pgfpathlineto{\pgfqpoint{1.115512in}{0.739968in}}%
\pgfpathlineto{\pgfqpoint{1.121969in}{0.753579in}}%
\pgfpathlineto{\pgfqpoint{1.125468in}{0.756932in}}%
\pgfpathlineto{\pgfqpoint{1.141125in}{0.764629in}}%
\pgfpathlineto{\pgfqpoint{1.156781in}{0.764629in}}%
\pgfpathlineto{\pgfqpoint{1.172438in}{0.756932in}}%
\pgfpathlineto{\pgfqpoint{1.175937in}{0.753579in}}%
\pgfpathlineto{\pgfqpoint{1.182394in}{0.739968in}}%
\pgfpathlineto{\pgfqpoint{1.180781in}{0.726357in}}%
\pgfpathlineto{\pgfqpoint{1.172438in}{0.714607in}}%
\pgfpathlineto{\pgfqpoint{1.169874in}{0.712746in}}%
\pgfpathlineto{\pgfqpoint{1.156781in}{0.707347in}}%
\pgfpathlineto{\pgfqpoint{1.141125in}{0.707347in}}%
\pgfpathlineto{\pgfqpoint{1.128032in}{0.712746in}}%
\pgfpathclose%
\pgfpathmoveto{\pgfqpoint{1.438599in}{0.682392in}}%
\pgfpathlineto{\pgfqpoint{1.454256in}{0.679300in}}%
\pgfpathlineto{\pgfqpoint{1.469913in}{0.680072in}}%
\pgfpathlineto{\pgfqpoint{1.485569in}{0.684714in}}%
\pgfpathlineto{\pgfqpoint{1.487110in}{0.685524in}}%
\pgfpathlineto{\pgfqpoint{1.501226in}{0.694332in}}%
\pgfpathlineto{\pgfqpoint{1.506751in}{0.699135in}}%
\pgfpathlineto{\pgfqpoint{1.516882in}{0.711406in}}%
\pgfpathlineto{\pgfqpoint{1.517814in}{0.712746in}}%
\pgfpathlineto{\pgfqpoint{1.523153in}{0.726357in}}%
\pgfpathlineto{\pgfqpoint{1.524042in}{0.739968in}}%
\pgfpathlineto{\pgfqpoint{1.520485in}{0.753579in}}%
\pgfpathlineto{\pgfqpoint{1.516882in}{0.759804in}}%
\pgfpathlineto{\pgfqpoint{1.511840in}{0.767191in}}%
\pgfpathlineto{\pgfqpoint{1.501226in}{0.777466in}}%
\pgfpathlineto{\pgfqpoint{1.496451in}{0.780802in}}%
\pgfpathlineto{\pgfqpoint{1.485569in}{0.786887in}}%
\pgfpathlineto{\pgfqpoint{1.469913in}{0.791713in}}%
\pgfpathlineto{\pgfqpoint{1.454256in}{0.792516in}}%
\pgfpathlineto{\pgfqpoint{1.438599in}{0.789301in}}%
\pgfpathlineto{\pgfqpoint{1.422943in}{0.782054in}}%
\pgfpathlineto{\pgfqpoint{1.421126in}{0.780802in}}%
\pgfpathlineto{\pgfqpoint{1.407286in}{0.768770in}}%
\pgfpathlineto{\pgfqpoint{1.405846in}{0.767191in}}%
\pgfpathlineto{\pgfqpoint{1.397509in}{0.753579in}}%
\pgfpathlineto{\pgfqpoint{1.393812in}{0.739968in}}%
\pgfpathlineto{\pgfqpoint{1.394736in}{0.726357in}}%
\pgfpathlineto{\pgfqpoint{1.400286in}{0.712746in}}%
\pgfpathlineto{\pgfqpoint{1.407286in}{0.703286in}}%
\pgfpathlineto{\pgfqpoint{1.411124in}{0.699135in}}%
\pgfpathlineto{\pgfqpoint{1.422943in}{0.689907in}}%
\pgfpathlineto{\pgfqpoint{1.431439in}{0.685524in}}%
\pgfpathlineto{\pgfqpoint{1.438599in}{0.682392in}}%
\pgfpathclose%
\pgfpathmoveto{\pgfqpoint{1.437580in}{0.712746in}}%
\pgfpathlineto{\pgfqpoint{1.426951in}{0.726357in}}%
\pgfpathlineto{\pgfqpoint{1.425182in}{0.739968in}}%
\pgfpathlineto{\pgfqpoint{1.432262in}{0.753579in}}%
\pgfpathlineto{\pgfqpoint{1.438599in}{0.759089in}}%
\pgfpathlineto{\pgfqpoint{1.454256in}{0.765244in}}%
\pgfpathlineto{\pgfqpoint{1.469913in}{0.763706in}}%
\pgfpathlineto{\pgfqpoint{1.485569in}{0.754466in}}%
\pgfpathlineto{\pgfqpoint{1.486418in}{0.753579in}}%
\pgfpathlineto{\pgfqpoint{1.492375in}{0.739968in}}%
\pgfpathlineto{\pgfqpoint{1.490887in}{0.726357in}}%
\pgfpathlineto{\pgfqpoint{1.485569in}{0.718192in}}%
\pgfpathlineto{\pgfqpoint{1.479304in}{0.712746in}}%
\pgfpathlineto{\pgfqpoint{1.469913in}{0.708123in}}%
\pgfpathlineto{\pgfqpoint{1.454256in}{0.706829in}}%
\pgfpathlineto{\pgfqpoint{1.438599in}{0.712008in}}%
\pgfpathlineto{\pgfqpoint{1.437580in}{0.712746in}}%
\pgfpathclose%
\pgfpathmoveto{\pgfqpoint{1.751731in}{0.681463in}}%
\pgfpathlineto{\pgfqpoint{1.767387in}{0.679145in}}%
\pgfpathlineto{\pgfqpoint{1.783044in}{0.680690in}}%
\pgfpathlineto{\pgfqpoint{1.797050in}{0.685524in}}%
\pgfpathlineto{\pgfqpoint{1.798700in}{0.686191in}}%
\pgfpathlineto{\pgfqpoint{1.814357in}{0.696807in}}%
\pgfpathlineto{\pgfqpoint{1.816914in}{0.699135in}}%
\pgfpathlineto{\pgfqpoint{1.827710in}{0.712746in}}%
\pgfpathlineto{\pgfqpoint{1.830014in}{0.718010in}}%
\pgfpathlineto{\pgfqpoint{1.833219in}{0.726357in}}%
\pgfpathlineto{\pgfqpoint{1.834096in}{0.739968in}}%
\pgfpathlineto{\pgfqpoint{1.830585in}{0.753579in}}%
\pgfpathlineto{\pgfqpoint{1.830014in}{0.754592in}}%
\pgfpathlineto{\pgfqpoint{1.821821in}{0.767191in}}%
\pgfpathlineto{\pgfqpoint{1.814357in}{0.774757in}}%
\pgfpathlineto{\pgfqpoint{1.806374in}{0.780802in}}%
\pgfpathlineto{\pgfqpoint{1.798700in}{0.785437in}}%
\pgfpathlineto{\pgfqpoint{1.783044in}{0.791070in}}%
\pgfpathlineto{\pgfqpoint{1.767387in}{0.792676in}}%
\pgfpathlineto{\pgfqpoint{1.751731in}{0.790266in}}%
\pgfpathlineto{\pgfqpoint{1.736074in}{0.783826in}}%
\pgfpathlineto{\pgfqpoint{1.731403in}{0.780802in}}%
\pgfpathlineto{\pgfqpoint{1.720418in}{0.771857in}}%
\pgfpathlineto{\pgfqpoint{1.716000in}{0.767191in}}%
\pgfpathlineto{\pgfqpoint{1.707442in}{0.753579in}}%
\pgfpathlineto{\pgfqpoint{1.704761in}{0.744009in}}%
\pgfpathlineto{\pgfqpoint{1.703739in}{0.739968in}}%
\pgfpathlineto{\pgfqpoint{1.704608in}{0.726357in}}%
\pgfpathlineto{\pgfqpoint{1.704761in}{0.725950in}}%
\pgfpathlineto{\pgfqpoint{1.710293in}{0.712746in}}%
\pgfpathlineto{\pgfqpoint{1.720418in}{0.699560in}}%
\pgfpathlineto{\pgfqpoint{1.720837in}{0.699135in}}%
\pgfpathlineto{\pgfqpoint{1.736074in}{0.687961in}}%
\pgfpathlineto{\pgfqpoint{1.741366in}{0.685524in}}%
\pgfpathlineto{\pgfqpoint{1.751731in}{0.681463in}}%
\pgfpathclose%
\pgfpathmoveto{\pgfqpoint{1.748193in}{0.712746in}}%
\pgfpathlineto{\pgfqpoint{1.736376in}{0.726357in}}%
\pgfpathlineto{\pgfqpoint{1.736074in}{0.728442in}}%
\pgfpathlineto{\pgfqpoint{1.734971in}{0.739968in}}%
\pgfpathlineto{\pgfqpoint{1.736074in}{0.742880in}}%
\pgfpathlineto{\pgfqpoint{1.742281in}{0.753579in}}%
\pgfpathlineto{\pgfqpoint{1.751731in}{0.760936in}}%
\pgfpathlineto{\pgfqpoint{1.767387in}{0.765551in}}%
\pgfpathlineto{\pgfqpoint{1.783044in}{0.762475in}}%
\pgfpathlineto{\pgfqpoint{1.796015in}{0.753579in}}%
\pgfpathlineto{\pgfqpoint{1.798700in}{0.749516in}}%
\pgfpathlineto{\pgfqpoint{1.802577in}{0.739968in}}%
\pgfpathlineto{\pgfqpoint{1.801190in}{0.726357in}}%
\pgfpathlineto{\pgfqpoint{1.798700in}{0.722228in}}%
\pgfpathlineto{\pgfqpoint{1.789329in}{0.712746in}}%
\pgfpathlineto{\pgfqpoint{1.783044in}{0.709159in}}%
\pgfpathlineto{\pgfqpoint{1.767387in}{0.706571in}}%
\pgfpathlineto{\pgfqpoint{1.751731in}{0.710454in}}%
\pgfpathlineto{\pgfqpoint{1.748193in}{0.712746in}}%
\pgfpathclose%
\pgfpathmoveto{\pgfqpoint{0.499205in}{0.955735in}}%
\pgfpathlineto{\pgfqpoint{0.514862in}{0.950223in}}%
\pgfpathlineto{\pgfqpoint{0.530519in}{0.948651in}}%
\pgfpathlineto{\pgfqpoint{0.546175in}{0.951010in}}%
\pgfpathlineto{\pgfqpoint{0.561832in}{0.957312in}}%
\pgfpathlineto{\pgfqpoint{0.562525in}{0.957746in}}%
\pgfpathlineto{\pgfqpoint{0.577488in}{0.969227in}}%
\pgfpathlineto{\pgfqpoint{0.579622in}{0.971357in}}%
\pgfpathlineto{\pgfqpoint{0.589134in}{0.984968in}}%
\pgfpathlineto{\pgfqpoint{0.593145in}{0.996436in}}%
\pgfpathlineto{\pgfqpoint{0.593820in}{0.998579in}}%
\pgfpathlineto{\pgfqpoint{0.593820in}{1.012191in}}%
\pgfpathlineto{\pgfqpoint{0.593145in}{1.014334in}}%
\pgfpathlineto{\pgfqpoint{0.589134in}{1.025802in}}%
\pgfpathlineto{\pgfqpoint{0.579622in}{1.039413in}}%
\pgfpathlineto{\pgfqpoint{0.577488in}{1.041543in}}%
\pgfpathlineto{\pgfqpoint{0.562525in}{1.053024in}}%
\pgfpathlineto{\pgfqpoint{0.561832in}{1.053458in}}%
\pgfpathlineto{\pgfqpoint{0.546175in}{1.059760in}}%
\pgfpathlineto{\pgfqpoint{0.530519in}{1.062119in}}%
\pgfpathlineto{\pgfqpoint{0.514862in}{1.060547in}}%
\pgfpathlineto{\pgfqpoint{0.499205in}{1.055035in}}%
\pgfpathlineto{\pgfqpoint{0.495765in}{1.053024in}}%
\pgfpathlineto{\pgfqpoint{0.483549in}{1.044307in}}%
\pgfpathlineto{\pgfqpoint{0.478440in}{1.039413in}}%
\pgfpathlineto{\pgfqpoint{0.468627in}{1.025802in}}%
\pgfpathlineto{\pgfqpoint{0.467892in}{1.023790in}}%
\pgfpathlineto{\pgfqpoint{0.464161in}{1.012191in}}%
\pgfpathlineto{\pgfqpoint{0.464161in}{0.998579in}}%
\pgfpathlineto{\pgfqpoint{0.467892in}{0.986980in}}%
\pgfpathlineto{\pgfqpoint{0.468627in}{0.984968in}}%
\pgfpathlineto{\pgfqpoint{0.478440in}{0.971357in}}%
\pgfpathlineto{\pgfqpoint{0.483549in}{0.966463in}}%
\pgfpathlineto{\pgfqpoint{0.495765in}{0.957746in}}%
\pgfpathlineto{\pgfqpoint{0.499205in}{0.955735in}}%
\pgfpathclose%
\pgfpathmoveto{\pgfqpoint{0.505010in}{0.984968in}}%
\pgfpathlineto{\pgfqpoint{0.499205in}{0.992003in}}%
\pgfpathlineto{\pgfqpoint{0.495884in}{0.998579in}}%
\pgfpathlineto{\pgfqpoint{0.495884in}{1.012191in}}%
\pgfpathlineto{\pgfqpoint{0.499205in}{1.018767in}}%
\pgfpathlineto{\pgfqpoint{0.505010in}{1.025802in}}%
\pgfpathlineto{\pgfqpoint{0.514862in}{1.031932in}}%
\pgfpathlineto{\pgfqpoint{0.530519in}{1.034737in}}%
\pgfpathlineto{\pgfqpoint{0.546175in}{1.030529in}}%
\pgfpathlineto{\pgfqpoint{0.552867in}{1.025802in}}%
\pgfpathlineto{\pgfqpoint{0.561832in}{1.013425in}}%
\pgfpathlineto{\pgfqpoint{0.562413in}{1.012191in}}%
\pgfpathlineto{\pgfqpoint{0.562413in}{0.998579in}}%
\pgfpathlineto{\pgfqpoint{0.561832in}{0.997345in}}%
\pgfpathlineto{\pgfqpoint{0.552867in}{0.984968in}}%
\pgfpathlineto{\pgfqpoint{0.546175in}{0.980241in}}%
\pgfpathlineto{\pgfqpoint{0.530519in}{0.976033in}}%
\pgfpathlineto{\pgfqpoint{0.514862in}{0.978838in}}%
\pgfpathlineto{\pgfqpoint{0.505010in}{0.984968in}}%
\pgfpathclose%
\pgfpathmoveto{\pgfqpoint{0.812337in}{0.954317in}}%
\pgfpathlineto{\pgfqpoint{0.827993in}{0.949594in}}%
\pgfpathlineto{\pgfqpoint{0.843650in}{0.948808in}}%
\pgfpathlineto{\pgfqpoint{0.859306in}{0.951954in}}%
\pgfpathlineto{\pgfqpoint{0.872168in}{0.957746in}}%
\pgfpathlineto{\pgfqpoint{0.874963in}{0.959268in}}%
\pgfpathlineto{\pgfqpoint{0.889677in}{0.971357in}}%
\pgfpathlineto{\pgfqpoint{0.890620in}{0.972472in}}%
\pgfpathlineto{\pgfqpoint{0.899101in}{0.984968in}}%
\pgfpathlineto{\pgfqpoint{0.903724in}{0.998579in}}%
\pgfpathlineto{\pgfqpoint{0.903724in}{1.012191in}}%
\pgfpathlineto{\pgfqpoint{0.899101in}{1.025802in}}%
\pgfpathlineto{\pgfqpoint{0.890620in}{1.038298in}}%
\pgfpathlineto{\pgfqpoint{0.889677in}{1.039413in}}%
\pgfpathlineto{\pgfqpoint{0.874963in}{1.051502in}}%
\pgfpathlineto{\pgfqpoint{0.872168in}{1.053024in}}%
\pgfpathlineto{\pgfqpoint{0.859306in}{1.058816in}}%
\pgfpathlineto{\pgfqpoint{0.843650in}{1.061962in}}%
\pgfpathlineto{\pgfqpoint{0.827993in}{1.061176in}}%
\pgfpathlineto{\pgfqpoint{0.812337in}{1.056453in}}%
\pgfpathlineto{\pgfqpoint{0.805998in}{1.053024in}}%
\pgfpathlineto{\pgfqpoint{0.796680in}{1.046888in}}%
\pgfpathlineto{\pgfqpoint{0.788509in}{1.039413in}}%
\pgfpathlineto{\pgfqpoint{0.781024in}{1.029496in}}%
\pgfpathlineto{\pgfqpoint{0.778667in}{1.025802in}}%
\pgfpathlineto{\pgfqpoint{0.774219in}{1.012191in}}%
\pgfpathlineto{\pgfqpoint{0.774219in}{0.998579in}}%
\pgfpathlineto{\pgfqpoint{0.778667in}{0.984968in}}%
\pgfpathlineto{\pgfqpoint{0.781024in}{0.981274in}}%
\pgfpathlineto{\pgfqpoint{0.788509in}{0.971357in}}%
\pgfpathlineto{\pgfqpoint{0.796680in}{0.963882in}}%
\pgfpathlineto{\pgfqpoint{0.805998in}{0.957746in}}%
\pgfpathlineto{\pgfqpoint{0.812337in}{0.954317in}}%
\pgfpathclose%
\pgfpathmoveto{\pgfqpoint{0.814477in}{0.984968in}}%
\pgfpathlineto{\pgfqpoint{0.812337in}{0.987197in}}%
\pgfpathlineto{\pgfqpoint{0.806126in}{0.998579in}}%
\pgfpathlineto{\pgfqpoint{0.806126in}{1.012191in}}%
\pgfpathlineto{\pgfqpoint{0.812337in}{1.023573in}}%
\pgfpathlineto{\pgfqpoint{0.814477in}{1.025802in}}%
\pgfpathlineto{\pgfqpoint{0.827993in}{1.033055in}}%
\pgfpathlineto{\pgfqpoint{0.843650in}{1.034457in}}%
\pgfpathlineto{\pgfqpoint{0.859306in}{1.028844in}}%
\pgfpathlineto{\pgfqpoint{0.863163in}{1.025802in}}%
\pgfpathlineto{\pgfqpoint{0.872016in}{1.012191in}}%
\pgfpathlineto{\pgfqpoint{0.872016in}{0.998579in}}%
\pgfpathlineto{\pgfqpoint{0.863163in}{0.984968in}}%
\pgfpathlineto{\pgfqpoint{0.859306in}{0.981926in}}%
\pgfpathlineto{\pgfqpoint{0.843650in}{0.976313in}}%
\pgfpathlineto{\pgfqpoint{0.827993in}{0.977715in}}%
\pgfpathlineto{\pgfqpoint{0.814477in}{0.984968in}}%
\pgfpathclose%
\pgfpathmoveto{\pgfqpoint{1.125468in}{0.953056in}}%
\pgfpathlineto{\pgfqpoint{1.141125in}{0.949123in}}%
\pgfpathlineto{\pgfqpoint{1.156781in}{0.949123in}}%
\pgfpathlineto{\pgfqpoint{1.172438in}{0.953056in}}%
\pgfpathlineto{\pgfqpoint{1.181887in}{0.957746in}}%
\pgfpathlineto{\pgfqpoint{1.188094in}{0.961483in}}%
\pgfpathlineto{\pgfqpoint{1.199453in}{0.971357in}}%
\pgfpathlineto{\pgfqpoint{1.203751in}{0.976753in}}%
\pgfpathlineto{\pgfqpoint{1.209145in}{0.984968in}}%
\pgfpathlineto{\pgfqpoint{1.213670in}{0.998579in}}%
\pgfpathlineto{\pgfqpoint{1.213670in}{1.012191in}}%
\pgfpathlineto{\pgfqpoint{1.209145in}{1.025802in}}%
\pgfpathlineto{\pgfqpoint{1.203751in}{1.034017in}}%
\pgfpathlineto{\pgfqpoint{1.199453in}{1.039413in}}%
\pgfpathlineto{\pgfqpoint{1.188094in}{1.049287in}}%
\pgfpathlineto{\pgfqpoint{1.181887in}{1.053024in}}%
\pgfpathlineto{\pgfqpoint{1.172438in}{1.057714in}}%
\pgfpathlineto{\pgfqpoint{1.156781in}{1.061647in}}%
\pgfpathlineto{\pgfqpoint{1.141125in}{1.061647in}}%
\pgfpathlineto{\pgfqpoint{1.125468in}{1.057714in}}%
\pgfpathlineto{\pgfqpoint{1.116019in}{1.053024in}}%
\pgfpathlineto{\pgfqpoint{1.109812in}{1.049287in}}%
\pgfpathlineto{\pgfqpoint{1.098453in}{1.039413in}}%
\pgfpathlineto{\pgfqpoint{1.094155in}{1.034017in}}%
\pgfpathlineto{\pgfqpoint{1.088760in}{1.025802in}}%
\pgfpathlineto{\pgfqpoint{1.084236in}{1.012191in}}%
\pgfpathlineto{\pgfqpoint{1.084236in}{0.998579in}}%
\pgfpathlineto{\pgfqpoint{1.088760in}{0.984968in}}%
\pgfpathlineto{\pgfqpoint{1.094155in}{0.976753in}}%
\pgfpathlineto{\pgfqpoint{1.098453in}{0.971357in}}%
\pgfpathlineto{\pgfqpoint{1.109812in}{0.961483in}}%
\pgfpathlineto{\pgfqpoint{1.116019in}{0.957746in}}%
\pgfpathlineto{\pgfqpoint{1.125468in}{0.953056in}}%
\pgfpathclose%
\pgfpathmoveto{\pgfqpoint{1.124231in}{0.984968in}}%
\pgfpathlineto{\pgfqpoint{1.116157in}{0.998579in}}%
\pgfpathlineto{\pgfqpoint{1.116157in}{1.012191in}}%
\pgfpathlineto{\pgfqpoint{1.124231in}{1.025802in}}%
\pgfpathlineto{\pgfqpoint{1.125468in}{1.026877in}}%
\pgfpathlineto{\pgfqpoint{1.141125in}{1.033896in}}%
\pgfpathlineto{\pgfqpoint{1.156781in}{1.033896in}}%
\pgfpathlineto{\pgfqpoint{1.172438in}{1.026877in}}%
\pgfpathlineto{\pgfqpoint{1.173674in}{1.025802in}}%
\pgfpathlineto{\pgfqpoint{1.181749in}{1.012191in}}%
\pgfpathlineto{\pgfqpoint{1.181749in}{0.998579in}}%
\pgfpathlineto{\pgfqpoint{1.173674in}{0.984968in}}%
\pgfpathlineto{\pgfqpoint{1.172438in}{0.983893in}}%
\pgfpathlineto{\pgfqpoint{1.156781in}{0.976874in}}%
\pgfpathlineto{\pgfqpoint{1.141125in}{0.976874in}}%
\pgfpathlineto{\pgfqpoint{1.125468in}{0.983893in}}%
\pgfpathlineto{\pgfqpoint{1.124231in}{0.984968in}}%
\pgfpathclose%
\pgfpathmoveto{\pgfqpoint{1.438599in}{0.951954in}}%
\pgfpathlineto{\pgfqpoint{1.454256in}{0.948808in}}%
\pgfpathlineto{\pgfqpoint{1.469913in}{0.949594in}}%
\pgfpathlineto{\pgfqpoint{1.485569in}{0.954317in}}%
\pgfpathlineto{\pgfqpoint{1.491907in}{0.957746in}}%
\pgfpathlineto{\pgfqpoint{1.501226in}{0.963882in}}%
\pgfpathlineto{\pgfqpoint{1.509397in}{0.971357in}}%
\pgfpathlineto{\pgfqpoint{1.516882in}{0.981274in}}%
\pgfpathlineto{\pgfqpoint{1.519239in}{0.984968in}}%
\pgfpathlineto{\pgfqpoint{1.523686in}{0.998579in}}%
\pgfpathlineto{\pgfqpoint{1.523686in}{1.012191in}}%
\pgfpathlineto{\pgfqpoint{1.519239in}{1.025802in}}%
\pgfpathlineto{\pgfqpoint{1.516882in}{1.029496in}}%
\pgfpathlineto{\pgfqpoint{1.509397in}{1.039413in}}%
\pgfpathlineto{\pgfqpoint{1.501226in}{1.046888in}}%
\pgfpathlineto{\pgfqpoint{1.491907in}{1.053024in}}%
\pgfpathlineto{\pgfqpoint{1.485569in}{1.056453in}}%
\pgfpathlineto{\pgfqpoint{1.469913in}{1.061176in}}%
\pgfpathlineto{\pgfqpoint{1.454256in}{1.061962in}}%
\pgfpathlineto{\pgfqpoint{1.438599in}{1.058816in}}%
\pgfpathlineto{\pgfqpoint{1.425738in}{1.053024in}}%
\pgfpathlineto{\pgfqpoint{1.422943in}{1.051502in}}%
\pgfpathlineto{\pgfqpoint{1.408229in}{1.039413in}}%
\pgfpathlineto{\pgfqpoint{1.407286in}{1.038298in}}%
\pgfpathlineto{\pgfqpoint{1.398805in}{1.025802in}}%
\pgfpathlineto{\pgfqpoint{1.394181in}{1.012191in}}%
\pgfpathlineto{\pgfqpoint{1.394181in}{0.998579in}}%
\pgfpathlineto{\pgfqpoint{1.398805in}{0.984968in}}%
\pgfpathlineto{\pgfqpoint{1.407286in}{0.972472in}}%
\pgfpathlineto{\pgfqpoint{1.408229in}{0.971357in}}%
\pgfpathlineto{\pgfqpoint{1.422943in}{0.959268in}}%
\pgfpathlineto{\pgfqpoint{1.425738in}{0.957746in}}%
\pgfpathlineto{\pgfqpoint{1.438599in}{0.951954in}}%
\pgfpathclose%
\pgfpathmoveto{\pgfqpoint{1.434743in}{0.984968in}}%
\pgfpathlineto{\pgfqpoint{1.425890in}{0.998579in}}%
\pgfpathlineto{\pgfqpoint{1.425890in}{1.012191in}}%
\pgfpathlineto{\pgfqpoint{1.434743in}{1.025802in}}%
\pgfpathlineto{\pgfqpoint{1.438599in}{1.028844in}}%
\pgfpathlineto{\pgfqpoint{1.454256in}{1.034457in}}%
\pgfpathlineto{\pgfqpoint{1.469913in}{1.033055in}}%
\pgfpathlineto{\pgfqpoint{1.483429in}{1.025802in}}%
\pgfpathlineto{\pgfqpoint{1.485569in}{1.023573in}}%
\pgfpathlineto{\pgfqpoint{1.491780in}{1.012191in}}%
\pgfpathlineto{\pgfqpoint{1.491780in}{0.998579in}}%
\pgfpathlineto{\pgfqpoint{1.485569in}{0.987197in}}%
\pgfpathlineto{\pgfqpoint{1.483429in}{0.984968in}}%
\pgfpathlineto{\pgfqpoint{1.469913in}{0.977715in}}%
\pgfpathlineto{\pgfqpoint{1.454256in}{0.976313in}}%
\pgfpathlineto{\pgfqpoint{1.438599in}{0.981926in}}%
\pgfpathlineto{\pgfqpoint{1.434743in}{0.984968in}}%
\pgfpathclose%
\pgfpathmoveto{\pgfqpoint{1.736074in}{0.957312in}}%
\pgfpathlineto{\pgfqpoint{1.751731in}{0.951010in}}%
\pgfpathlineto{\pgfqpoint{1.767387in}{0.948651in}}%
\pgfpathlineto{\pgfqpoint{1.783044in}{0.950223in}}%
\pgfpathlineto{\pgfqpoint{1.798700in}{0.955735in}}%
\pgfpathlineto{\pgfqpoint{1.802141in}{0.957746in}}%
\pgfpathlineto{\pgfqpoint{1.814357in}{0.966463in}}%
\pgfpathlineto{\pgfqpoint{1.819466in}{0.971357in}}%
\pgfpathlineto{\pgfqpoint{1.829279in}{0.984968in}}%
\pgfpathlineto{\pgfqpoint{1.830014in}{0.986980in}}%
\pgfpathlineto{\pgfqpoint{1.833745in}{0.998579in}}%
\pgfpathlineto{\pgfqpoint{1.833745in}{1.012191in}}%
\pgfpathlineto{\pgfqpoint{1.830014in}{1.023790in}}%
\pgfpathlineto{\pgfqpoint{1.829279in}{1.025802in}}%
\pgfpathlineto{\pgfqpoint{1.819466in}{1.039413in}}%
\pgfpathlineto{\pgfqpoint{1.814357in}{1.044307in}}%
\pgfpathlineto{\pgfqpoint{1.802141in}{1.053024in}}%
\pgfpathlineto{\pgfqpoint{1.798700in}{1.055035in}}%
\pgfpathlineto{\pgfqpoint{1.783044in}{1.060547in}}%
\pgfpathlineto{\pgfqpoint{1.767387in}{1.062119in}}%
\pgfpathlineto{\pgfqpoint{1.751731in}{1.059760in}}%
\pgfpathlineto{\pgfqpoint{1.736074in}{1.053458in}}%
\pgfpathlineto{\pgfqpoint{1.735381in}{1.053024in}}%
\pgfpathlineto{\pgfqpoint{1.720418in}{1.041543in}}%
\pgfpathlineto{\pgfqpoint{1.718284in}{1.039413in}}%
\pgfpathlineto{\pgfqpoint{1.708772in}{1.025802in}}%
\pgfpathlineto{\pgfqpoint{1.704761in}{1.014334in}}%
\pgfpathlineto{\pgfqpoint{1.704086in}{1.012191in}}%
\pgfpathlineto{\pgfqpoint{1.704086in}{0.998579in}}%
\pgfpathlineto{\pgfqpoint{1.704761in}{0.996436in}}%
\pgfpathlineto{\pgfqpoint{1.708772in}{0.984968in}}%
\pgfpathlineto{\pgfqpoint{1.718284in}{0.971357in}}%
\pgfpathlineto{\pgfqpoint{1.720418in}{0.969227in}}%
\pgfpathlineto{\pgfqpoint{1.735381in}{0.957746in}}%
\pgfpathlineto{\pgfqpoint{1.736074in}{0.957312in}}%
\pgfpathclose%
\pgfpathmoveto{\pgfqpoint{1.745039in}{0.984968in}}%
\pgfpathlineto{\pgfqpoint{1.736074in}{0.997345in}}%
\pgfpathlineto{\pgfqpoint{1.735492in}{0.998579in}}%
\pgfpathlineto{\pgfqpoint{1.735492in}{1.012191in}}%
\pgfpathlineto{\pgfqpoint{1.736074in}{1.013425in}}%
\pgfpathlineto{\pgfqpoint{1.745039in}{1.025802in}}%
\pgfpathlineto{\pgfqpoint{1.751731in}{1.030529in}}%
\pgfpathlineto{\pgfqpoint{1.767387in}{1.034737in}}%
\pgfpathlineto{\pgfqpoint{1.783044in}{1.031932in}}%
\pgfpathlineto{\pgfqpoint{1.792896in}{1.025802in}}%
\pgfpathlineto{\pgfqpoint{1.798700in}{1.018767in}}%
\pgfpathlineto{\pgfqpoint{1.802022in}{1.012191in}}%
\pgfpathlineto{\pgfqpoint{1.802022in}{0.998579in}}%
\pgfpathlineto{\pgfqpoint{1.798700in}{0.992003in}}%
\pgfpathlineto{\pgfqpoint{1.792896in}{0.984968in}}%
\pgfpathlineto{\pgfqpoint{1.783044in}{0.978838in}}%
\pgfpathlineto{\pgfqpoint{1.767387in}{0.976033in}}%
\pgfpathlineto{\pgfqpoint{1.751731in}{0.980241in}}%
\pgfpathlineto{\pgfqpoint{1.745039in}{0.984968in}}%
\pgfpathclose%
\pgfpathmoveto{\pgfqpoint{0.499205in}{1.225333in}}%
\pgfpathlineto{\pgfqpoint{0.514862in}{1.219700in}}%
\pgfpathlineto{\pgfqpoint{0.530519in}{1.218094in}}%
\pgfpathlineto{\pgfqpoint{0.546175in}{1.220504in}}%
\pgfpathlineto{\pgfqpoint{0.561832in}{1.226944in}}%
\pgfpathlineto{\pgfqpoint{0.566503in}{1.229968in}}%
\pgfpathlineto{\pgfqpoint{0.577488in}{1.238913in}}%
\pgfpathlineto{\pgfqpoint{0.581906in}{1.243579in}}%
\pgfpathlineto{\pgfqpoint{0.590464in}{1.257191in}}%
\pgfpathlineto{\pgfqpoint{0.593145in}{1.266761in}}%
\pgfpathlineto{\pgfqpoint{0.594167in}{1.270802in}}%
\pgfpathlineto{\pgfqpoint{0.593298in}{1.284413in}}%
\pgfpathlineto{\pgfqpoint{0.593145in}{1.284820in}}%
\pgfpathlineto{\pgfqpoint{0.587613in}{1.298024in}}%
\pgfpathlineto{\pgfqpoint{0.577488in}{1.311210in}}%
\pgfpathlineto{\pgfqpoint{0.577069in}{1.311635in}}%
\pgfpathlineto{\pgfqpoint{0.561832in}{1.322809in}}%
\pgfpathlineto{\pgfqpoint{0.556540in}{1.325246in}}%
\pgfpathlineto{\pgfqpoint{0.546175in}{1.329307in}}%
\pgfpathlineto{\pgfqpoint{0.530519in}{1.331625in}}%
\pgfpathlineto{\pgfqpoint{0.514862in}{1.330080in}}%
\pgfpathlineto{\pgfqpoint{0.500856in}{1.325246in}}%
\pgfpathlineto{\pgfqpoint{0.499205in}{1.324579in}}%
\pgfpathlineto{\pgfqpoint{0.483549in}{1.313963in}}%
\pgfpathlineto{\pgfqpoint{0.480992in}{1.311635in}}%
\pgfpathlineto{\pgfqpoint{0.470196in}{1.298024in}}%
\pgfpathlineto{\pgfqpoint{0.467892in}{1.292760in}}%
\pgfpathlineto{\pgfqpoint{0.464687in}{1.284413in}}%
\pgfpathlineto{\pgfqpoint{0.463810in}{1.270802in}}%
\pgfpathlineto{\pgfqpoint{0.467321in}{1.257191in}}%
\pgfpathlineto{\pgfqpoint{0.467892in}{1.256178in}}%
\pgfpathlineto{\pgfqpoint{0.476084in}{1.243579in}}%
\pgfpathlineto{\pgfqpoint{0.483549in}{1.236013in}}%
\pgfpathlineto{\pgfqpoint{0.491532in}{1.229968in}}%
\pgfpathlineto{\pgfqpoint{0.499205in}{1.225333in}}%
\pgfpathclose%
\pgfpathmoveto{\pgfqpoint{0.501891in}{1.257191in}}%
\pgfpathlineto{\pgfqpoint{0.499205in}{1.261254in}}%
\pgfpathlineto{\pgfqpoint{0.495329in}{1.270802in}}%
\pgfpathlineto{\pgfqpoint{0.496716in}{1.284413in}}%
\pgfpathlineto{\pgfqpoint{0.499205in}{1.288542in}}%
\pgfpathlineto{\pgfqpoint{0.508577in}{1.298024in}}%
\pgfpathlineto{\pgfqpoint{0.514862in}{1.301611in}}%
\pgfpathlineto{\pgfqpoint{0.530519in}{1.304199in}}%
\pgfpathlineto{\pgfqpoint{0.546175in}{1.300316in}}%
\pgfpathlineto{\pgfqpoint{0.549713in}{1.298024in}}%
\pgfpathlineto{\pgfqpoint{0.561530in}{1.284413in}}%
\pgfpathlineto{\pgfqpoint{0.561832in}{1.282328in}}%
\pgfpathlineto{\pgfqpoint{0.562934in}{1.270802in}}%
\pgfpathlineto{\pgfqpoint{0.561832in}{1.267890in}}%
\pgfpathlineto{\pgfqpoint{0.555625in}{1.257191in}}%
\pgfpathlineto{\pgfqpoint{0.546175in}{1.249834in}}%
\pgfpathlineto{\pgfqpoint{0.530519in}{1.245219in}}%
\pgfpathlineto{\pgfqpoint{0.514862in}{1.248295in}}%
\pgfpathlineto{\pgfqpoint{0.501891in}{1.257191in}}%
\pgfpathclose%
\pgfpathmoveto{\pgfqpoint{0.812337in}{1.223883in}}%
\pgfpathlineto{\pgfqpoint{0.827993in}{1.219057in}}%
\pgfpathlineto{\pgfqpoint{0.843650in}{1.218254in}}%
\pgfpathlineto{\pgfqpoint{0.859306in}{1.221469in}}%
\pgfpathlineto{\pgfqpoint{0.874963in}{1.228716in}}%
\pgfpathlineto{\pgfqpoint{0.876779in}{1.229968in}}%
\pgfpathlineto{\pgfqpoint{0.890620in}{1.242000in}}%
\pgfpathlineto{\pgfqpoint{0.892060in}{1.243579in}}%
\pgfpathlineto{\pgfqpoint{0.900396in}{1.257191in}}%
\pgfpathlineto{\pgfqpoint{0.904094in}{1.270802in}}%
\pgfpathlineto{\pgfqpoint{0.903170in}{1.284413in}}%
\pgfpathlineto{\pgfqpoint{0.897619in}{1.298024in}}%
\pgfpathlineto{\pgfqpoint{0.890620in}{1.307484in}}%
\pgfpathlineto{\pgfqpoint{0.886782in}{1.311635in}}%
\pgfpathlineto{\pgfqpoint{0.874963in}{1.320863in}}%
\pgfpathlineto{\pgfqpoint{0.866466in}{1.325246in}}%
\pgfpathlineto{\pgfqpoint{0.859306in}{1.328378in}}%
\pgfpathlineto{\pgfqpoint{0.843650in}{1.331470in}}%
\pgfpathlineto{\pgfqpoint{0.827993in}{1.330698in}}%
\pgfpathlineto{\pgfqpoint{0.812337in}{1.326056in}}%
\pgfpathlineto{\pgfqpoint{0.810796in}{1.325246in}}%
\pgfpathlineto{\pgfqpoint{0.796680in}{1.316438in}}%
\pgfpathlineto{\pgfqpoint{0.791155in}{1.311635in}}%
\pgfpathlineto{\pgfqpoint{0.781024in}{1.299364in}}%
\pgfpathlineto{\pgfqpoint{0.780092in}{1.298024in}}%
\pgfpathlineto{\pgfqpoint{0.774753in}{1.284413in}}%
\pgfpathlineto{\pgfqpoint{0.773864in}{1.270802in}}%
\pgfpathlineto{\pgfqpoint{0.777421in}{1.257191in}}%
\pgfpathlineto{\pgfqpoint{0.781024in}{1.250966in}}%
\pgfpathlineto{\pgfqpoint{0.786066in}{1.243579in}}%
\pgfpathlineto{\pgfqpoint{0.796680in}{1.233304in}}%
\pgfpathlineto{\pgfqpoint{0.801455in}{1.229968in}}%
\pgfpathlineto{\pgfqpoint{0.812337in}{1.223883in}}%
\pgfpathclose%
\pgfpathmoveto{\pgfqpoint{0.811488in}{1.257191in}}%
\pgfpathlineto{\pgfqpoint{0.805531in}{1.270802in}}%
\pgfpathlineto{\pgfqpoint{0.807019in}{1.284413in}}%
\pgfpathlineto{\pgfqpoint{0.812337in}{1.292578in}}%
\pgfpathlineto{\pgfqpoint{0.818601in}{1.298024in}}%
\pgfpathlineto{\pgfqpoint{0.827993in}{1.302647in}}%
\pgfpathlineto{\pgfqpoint{0.843650in}{1.303941in}}%
\pgfpathlineto{\pgfqpoint{0.859306in}{1.298762in}}%
\pgfpathlineto{\pgfqpoint{0.860326in}{1.298024in}}%
\pgfpathlineto{\pgfqpoint{0.870955in}{1.284413in}}%
\pgfpathlineto{\pgfqpoint{0.872723in}{1.270802in}}%
\pgfpathlineto{\pgfqpoint{0.865644in}{1.257191in}}%
\pgfpathlineto{\pgfqpoint{0.859306in}{1.251681in}}%
\pgfpathlineto{\pgfqpoint{0.843650in}{1.245526in}}%
\pgfpathlineto{\pgfqpoint{0.827993in}{1.247064in}}%
\pgfpathlineto{\pgfqpoint{0.812337in}{1.256304in}}%
\pgfpathlineto{\pgfqpoint{0.811488in}{1.257191in}}%
\pgfpathclose%
\pgfpathmoveto{\pgfqpoint{1.125468in}{1.222595in}}%
\pgfpathlineto{\pgfqpoint{1.141125in}{1.218576in}}%
\pgfpathlineto{\pgfqpoint{1.156781in}{1.218576in}}%
\pgfpathlineto{\pgfqpoint{1.172438in}{1.222595in}}%
\pgfpathlineto{\pgfqpoint{1.186812in}{1.229968in}}%
\pgfpathlineto{\pgfqpoint{1.188094in}{1.230788in}}%
\pgfpathlineto{\pgfqpoint{1.202000in}{1.243579in}}%
\pgfpathlineto{\pgfqpoint{1.203751in}{1.246010in}}%
\pgfpathlineto{\pgfqpoint{1.210413in}{1.257191in}}%
\pgfpathlineto{\pgfqpoint{1.214032in}{1.270802in}}%
\pgfpathlineto{\pgfqpoint{1.213128in}{1.284413in}}%
\pgfpathlineto{\pgfqpoint{1.207696in}{1.298024in}}%
\pgfpathlineto{\pgfqpoint{1.203751in}{1.303534in}}%
\pgfpathlineto{\pgfqpoint{1.196693in}{1.311635in}}%
\pgfpathlineto{\pgfqpoint{1.188094in}{1.318739in}}%
\pgfpathlineto{\pgfqpoint{1.176687in}{1.325246in}}%
\pgfpathlineto{\pgfqpoint{1.172438in}{1.327295in}}%
\pgfpathlineto{\pgfqpoint{1.156781in}{1.331161in}}%
\pgfpathlineto{\pgfqpoint{1.141125in}{1.331161in}}%
\pgfpathlineto{\pgfqpoint{1.125468in}{1.327295in}}%
\pgfpathlineto{\pgfqpoint{1.121218in}{1.325246in}}%
\pgfpathlineto{\pgfqpoint{1.109812in}{1.318739in}}%
\pgfpathlineto{\pgfqpoint{1.101212in}{1.311635in}}%
\pgfpathlineto{\pgfqpoint{1.094155in}{1.303534in}}%
\pgfpathlineto{\pgfqpoint{1.090210in}{1.298024in}}%
\pgfpathlineto{\pgfqpoint{1.084778in}{1.284413in}}%
\pgfpathlineto{\pgfqpoint{1.083874in}{1.270802in}}%
\pgfpathlineto{\pgfqpoint{1.087492in}{1.257191in}}%
\pgfpathlineto{\pgfqpoint{1.094155in}{1.246010in}}%
\pgfpathlineto{\pgfqpoint{1.095905in}{1.243579in}}%
\pgfpathlineto{\pgfqpoint{1.109812in}{1.230788in}}%
\pgfpathlineto{\pgfqpoint{1.111094in}{1.229968in}}%
\pgfpathlineto{\pgfqpoint{1.125468in}{1.222595in}}%
\pgfpathclose%
\pgfpathmoveto{\pgfqpoint{1.121969in}{1.257191in}}%
\pgfpathlineto{\pgfqpoint{1.115512in}{1.270802in}}%
\pgfpathlineto{\pgfqpoint{1.117125in}{1.284413in}}%
\pgfpathlineto{\pgfqpoint{1.125468in}{1.296163in}}%
\pgfpathlineto{\pgfqpoint{1.128032in}{1.298024in}}%
\pgfpathlineto{\pgfqpoint{1.141125in}{1.303423in}}%
\pgfpathlineto{\pgfqpoint{1.156781in}{1.303423in}}%
\pgfpathlineto{\pgfqpoint{1.169874in}{1.298024in}}%
\pgfpathlineto{\pgfqpoint{1.172438in}{1.296163in}}%
\pgfpathlineto{\pgfqpoint{1.180781in}{1.284413in}}%
\pgfpathlineto{\pgfqpoint{1.182394in}{1.270802in}}%
\pgfpathlineto{\pgfqpoint{1.175937in}{1.257191in}}%
\pgfpathlineto{\pgfqpoint{1.172438in}{1.253838in}}%
\pgfpathlineto{\pgfqpoint{1.156781in}{1.246141in}}%
\pgfpathlineto{\pgfqpoint{1.141125in}{1.246141in}}%
\pgfpathlineto{\pgfqpoint{1.125468in}{1.253838in}}%
\pgfpathlineto{\pgfqpoint{1.121969in}{1.257191in}}%
\pgfpathclose%
\pgfpathmoveto{\pgfqpoint{1.422943in}{1.228716in}}%
\pgfpathlineto{\pgfqpoint{1.438599in}{1.221469in}}%
\pgfpathlineto{\pgfqpoint{1.454256in}{1.218254in}}%
\pgfpathlineto{\pgfqpoint{1.469913in}{1.219057in}}%
\pgfpathlineto{\pgfqpoint{1.485569in}{1.223883in}}%
\pgfpathlineto{\pgfqpoint{1.496451in}{1.229968in}}%
\pgfpathlineto{\pgfqpoint{1.501226in}{1.233304in}}%
\pgfpathlineto{\pgfqpoint{1.511840in}{1.243579in}}%
\pgfpathlineto{\pgfqpoint{1.516882in}{1.250966in}}%
\pgfpathlineto{\pgfqpoint{1.520485in}{1.257191in}}%
\pgfpathlineto{\pgfqpoint{1.524042in}{1.270802in}}%
\pgfpathlineto{\pgfqpoint{1.523153in}{1.284413in}}%
\pgfpathlineto{\pgfqpoint{1.517814in}{1.298024in}}%
\pgfpathlineto{\pgfqpoint{1.516882in}{1.299364in}}%
\pgfpathlineto{\pgfqpoint{1.506751in}{1.311635in}}%
\pgfpathlineto{\pgfqpoint{1.501226in}{1.316438in}}%
\pgfpathlineto{\pgfqpoint{1.487110in}{1.325246in}}%
\pgfpathlineto{\pgfqpoint{1.485569in}{1.326056in}}%
\pgfpathlineto{\pgfqpoint{1.469913in}{1.330698in}}%
\pgfpathlineto{\pgfqpoint{1.454256in}{1.331470in}}%
\pgfpathlineto{\pgfqpoint{1.438599in}{1.328378in}}%
\pgfpathlineto{\pgfqpoint{1.431439in}{1.325246in}}%
\pgfpathlineto{\pgfqpoint{1.422943in}{1.320863in}}%
\pgfpathlineto{\pgfqpoint{1.411124in}{1.311635in}}%
\pgfpathlineto{\pgfqpoint{1.407286in}{1.307484in}}%
\pgfpathlineto{\pgfqpoint{1.400286in}{1.298024in}}%
\pgfpathlineto{\pgfqpoint{1.394736in}{1.284413in}}%
\pgfpathlineto{\pgfqpoint{1.393812in}{1.270802in}}%
\pgfpathlineto{\pgfqpoint{1.397509in}{1.257191in}}%
\pgfpathlineto{\pgfqpoint{1.405846in}{1.243579in}}%
\pgfpathlineto{\pgfqpoint{1.407286in}{1.242000in}}%
\pgfpathlineto{\pgfqpoint{1.421126in}{1.229968in}}%
\pgfpathlineto{\pgfqpoint{1.422943in}{1.228716in}}%
\pgfpathclose%
\pgfpathmoveto{\pgfqpoint{1.432262in}{1.257191in}}%
\pgfpathlineto{\pgfqpoint{1.425182in}{1.270802in}}%
\pgfpathlineto{\pgfqpoint{1.426951in}{1.284413in}}%
\pgfpathlineto{\pgfqpoint{1.437580in}{1.298024in}}%
\pgfpathlineto{\pgfqpoint{1.438599in}{1.298762in}}%
\pgfpathlineto{\pgfqpoint{1.454256in}{1.303941in}}%
\pgfpathlineto{\pgfqpoint{1.469913in}{1.302647in}}%
\pgfpathlineto{\pgfqpoint{1.479304in}{1.298024in}}%
\pgfpathlineto{\pgfqpoint{1.485569in}{1.292578in}}%
\pgfpathlineto{\pgfqpoint{1.490887in}{1.284413in}}%
\pgfpathlineto{\pgfqpoint{1.492375in}{1.270802in}}%
\pgfpathlineto{\pgfqpoint{1.486418in}{1.257191in}}%
\pgfpathlineto{\pgfqpoint{1.485569in}{1.256304in}}%
\pgfpathlineto{\pgfqpoint{1.469913in}{1.247064in}}%
\pgfpathlineto{\pgfqpoint{1.454256in}{1.245526in}}%
\pgfpathlineto{\pgfqpoint{1.438599in}{1.251681in}}%
\pgfpathlineto{\pgfqpoint{1.432262in}{1.257191in}}%
\pgfpathclose%
\pgfpathmoveto{\pgfqpoint{1.736074in}{1.226944in}}%
\pgfpathlineto{\pgfqpoint{1.751731in}{1.220504in}}%
\pgfpathlineto{\pgfqpoint{1.767387in}{1.218094in}}%
\pgfpathlineto{\pgfqpoint{1.783044in}{1.219700in}}%
\pgfpathlineto{\pgfqpoint{1.798700in}{1.225333in}}%
\pgfpathlineto{\pgfqpoint{1.806374in}{1.229968in}}%
\pgfpathlineto{\pgfqpoint{1.814357in}{1.236013in}}%
\pgfpathlineto{\pgfqpoint{1.821821in}{1.243579in}}%
\pgfpathlineto{\pgfqpoint{1.830014in}{1.256178in}}%
\pgfpathlineto{\pgfqpoint{1.830585in}{1.257191in}}%
\pgfpathlineto{\pgfqpoint{1.834096in}{1.270802in}}%
\pgfpathlineto{\pgfqpoint{1.833219in}{1.284413in}}%
\pgfpathlineto{\pgfqpoint{1.830014in}{1.292760in}}%
\pgfpathlineto{\pgfqpoint{1.827710in}{1.298024in}}%
\pgfpathlineto{\pgfqpoint{1.816914in}{1.311635in}}%
\pgfpathlineto{\pgfqpoint{1.814357in}{1.313963in}}%
\pgfpathlineto{\pgfqpoint{1.798700in}{1.324579in}}%
\pgfpathlineto{\pgfqpoint{1.797050in}{1.325246in}}%
\pgfpathlineto{\pgfqpoint{1.783044in}{1.330080in}}%
\pgfpathlineto{\pgfqpoint{1.767387in}{1.331625in}}%
\pgfpathlineto{\pgfqpoint{1.751731in}{1.329307in}}%
\pgfpathlineto{\pgfqpoint{1.741366in}{1.325246in}}%
\pgfpathlineto{\pgfqpoint{1.736074in}{1.322809in}}%
\pgfpathlineto{\pgfqpoint{1.720837in}{1.311635in}}%
\pgfpathlineto{\pgfqpoint{1.720418in}{1.311210in}}%
\pgfpathlineto{\pgfqpoint{1.710293in}{1.298024in}}%
\pgfpathlineto{\pgfqpoint{1.704761in}{1.284820in}}%
\pgfpathlineto{\pgfqpoint{1.704608in}{1.284413in}}%
\pgfpathlineto{\pgfqpoint{1.703739in}{1.270802in}}%
\pgfpathlineto{\pgfqpoint{1.704761in}{1.266761in}}%
\pgfpathlineto{\pgfqpoint{1.707442in}{1.257191in}}%
\pgfpathlineto{\pgfqpoint{1.716000in}{1.243579in}}%
\pgfpathlineto{\pgfqpoint{1.720418in}{1.238913in}}%
\pgfpathlineto{\pgfqpoint{1.731403in}{1.229968in}}%
\pgfpathlineto{\pgfqpoint{1.736074in}{1.226944in}}%
\pgfpathclose%
\pgfpathmoveto{\pgfqpoint{1.742281in}{1.257191in}}%
\pgfpathlineto{\pgfqpoint{1.736074in}{1.267890in}}%
\pgfpathlineto{\pgfqpoint{1.734971in}{1.270802in}}%
\pgfpathlineto{\pgfqpoint{1.736074in}{1.282328in}}%
\pgfpathlineto{\pgfqpoint{1.736376in}{1.284413in}}%
\pgfpathlineto{\pgfqpoint{1.748193in}{1.298024in}}%
\pgfpathlineto{\pgfqpoint{1.751731in}{1.300316in}}%
\pgfpathlineto{\pgfqpoint{1.767387in}{1.304199in}}%
\pgfpathlineto{\pgfqpoint{1.783044in}{1.301611in}}%
\pgfpathlineto{\pgfqpoint{1.789329in}{1.298024in}}%
\pgfpathlineto{\pgfqpoint{1.798700in}{1.288542in}}%
\pgfpathlineto{\pgfqpoint{1.801190in}{1.284413in}}%
\pgfpathlineto{\pgfqpoint{1.802577in}{1.270802in}}%
\pgfpathlineto{\pgfqpoint{1.798700in}{1.261254in}}%
\pgfpathlineto{\pgfqpoint{1.796015in}{1.257191in}}%
\pgfpathlineto{\pgfqpoint{1.783044in}{1.248295in}}%
\pgfpathlineto{\pgfqpoint{1.767387in}{1.245219in}}%
\pgfpathlineto{\pgfqpoint{1.751731in}{1.249834in}}%
\pgfpathlineto{\pgfqpoint{1.742281in}{1.257191in}}%
\pgfpathclose%
\pgfpathmoveto{\pgfqpoint{0.530519in}{1.487539in}}%
\pgfpathlineto{\pgfqpoint{0.537740in}{1.488579in}}%
\pgfpathlineto{\pgfqpoint{0.546175in}{1.489919in}}%
\pgfpathlineto{\pgfqpoint{0.561832in}{1.496530in}}%
\pgfpathlineto{\pgfqpoint{0.570255in}{1.502191in}}%
\pgfpathlineto{\pgfqpoint{0.577488in}{1.508479in}}%
\pgfpathlineto{\pgfqpoint{0.583999in}{1.515802in}}%
\pgfpathlineto{\pgfqpoint{0.591604in}{1.529413in}}%
\pgfpathlineto{\pgfqpoint{0.593145in}{1.536746in}}%
\pgfpathlineto{\pgfqpoint{0.594341in}{1.543024in}}%
\pgfpathlineto{\pgfqpoint{0.593145in}{1.552409in}}%
\pgfpathlineto{\pgfqpoint{0.592553in}{1.556635in}}%
\pgfpathlineto{\pgfqpoint{0.585901in}{1.570246in}}%
\pgfpathlineto{\pgfqpoint{0.577488in}{1.580389in}}%
\pgfpathlineto{\pgfqpoint{0.573777in}{1.583857in}}%
\pgfpathlineto{\pgfqpoint{0.561832in}{1.592224in}}%
\pgfpathlineto{\pgfqpoint{0.549874in}{1.597468in}}%
\pgfpathlineto{\pgfqpoint{0.546175in}{1.598881in}}%
\pgfpathlineto{\pgfqpoint{0.530519in}{1.601170in}}%
\pgfpathlineto{\pgfqpoint{0.514862in}{1.599644in}}%
\pgfpathlineto{\pgfqpoint{0.508395in}{1.597468in}}%
\pgfpathlineto{\pgfqpoint{0.499205in}{1.593930in}}%
\pgfpathlineto{\pgfqpoint{0.483790in}{1.583857in}}%
\pgfpathlineto{\pgfqpoint{0.483549in}{1.583648in}}%
\pgfpathlineto{\pgfqpoint{0.471962in}{1.570246in}}%
\pgfpathlineto{\pgfqpoint{0.467892in}{1.562257in}}%
\pgfpathlineto{\pgfqpoint{0.465389in}{1.556635in}}%
\pgfpathlineto{\pgfqpoint{0.463635in}{1.543024in}}%
\pgfpathlineto{\pgfqpoint{0.466267in}{1.529413in}}%
\pgfpathlineto{\pgfqpoint{0.467892in}{1.526197in}}%
\pgfpathlineto{\pgfqpoint{0.473925in}{1.515802in}}%
\pgfpathlineto{\pgfqpoint{0.483549in}{1.505417in}}%
\pgfpathlineto{\pgfqpoint{0.487539in}{1.502191in}}%
\pgfpathlineto{\pgfqpoint{0.499205in}{1.494876in}}%
\pgfpathlineto{\pgfqpoint{0.514862in}{1.489094in}}%
\pgfpathlineto{\pgfqpoint{0.519724in}{1.488579in}}%
\pgfpathlineto{\pgfqpoint{0.530519in}{1.487539in}}%
\pgfpathclose%
\pgfpathmoveto{\pgfqpoint{0.522300in}{1.515802in}}%
\pgfpathlineto{\pgfqpoint{0.514862in}{1.517432in}}%
\pgfpathlineto{\pgfqpoint{0.499218in}{1.529413in}}%
\pgfpathlineto{\pgfqpoint{0.499205in}{1.529438in}}%
\pgfpathlineto{\pgfqpoint{0.495052in}{1.543024in}}%
\pgfpathlineto{\pgfqpoint{0.497826in}{1.556635in}}%
\pgfpathlineto{\pgfqpoint{0.499205in}{1.558609in}}%
\pgfpathlineto{\pgfqpoint{0.512592in}{1.570246in}}%
\pgfpathlineto{\pgfqpoint{0.514862in}{1.571446in}}%
\pgfpathlineto{\pgfqpoint{0.530519in}{1.573857in}}%
\pgfpathlineto{\pgfqpoint{0.546146in}{1.570246in}}%
\pgfpathlineto{\pgfqpoint{0.546175in}{1.570235in}}%
\pgfpathlineto{\pgfqpoint{0.559956in}{1.556635in}}%
\pgfpathlineto{\pgfqpoint{0.561832in}{1.550169in}}%
\pgfpathlineto{\pgfqpoint{0.563195in}{1.543024in}}%
\pgfpathlineto{\pgfqpoint{0.561832in}{1.538244in}}%
\pgfpathlineto{\pgfqpoint{0.557988in}{1.529413in}}%
\pgfpathlineto{\pgfqpoint{0.546175in}{1.519143in}}%
\pgfpathlineto{\pgfqpoint{0.536016in}{1.515802in}}%
\pgfpathlineto{\pgfqpoint{0.530519in}{1.514617in}}%
\pgfpathlineto{\pgfqpoint{0.522300in}{1.515802in}}%
\pgfpathclose%
\pgfpathmoveto{\pgfqpoint{0.827993in}{1.488446in}}%
\pgfpathlineto{\pgfqpoint{0.843650in}{1.487691in}}%
\pgfpathlineto{\pgfqpoint{0.848298in}{1.488579in}}%
\pgfpathlineto{\pgfqpoint{0.859306in}{1.490910in}}%
\pgfpathlineto{\pgfqpoint{0.874963in}{1.498350in}}%
\pgfpathlineto{\pgfqpoint{0.880331in}{1.502191in}}%
\pgfpathlineto{\pgfqpoint{0.890620in}{1.511741in}}%
\pgfpathlineto{\pgfqpoint{0.894099in}{1.515802in}}%
\pgfpathlineto{\pgfqpoint{0.901506in}{1.529413in}}%
\pgfpathlineto{\pgfqpoint{0.904279in}{1.543024in}}%
\pgfpathlineto{\pgfqpoint{0.902431in}{1.556635in}}%
\pgfpathlineto{\pgfqpoint{0.895952in}{1.570246in}}%
\pgfpathlineto{\pgfqpoint{0.890620in}{1.576917in}}%
\pgfpathlineto{\pgfqpoint{0.883666in}{1.583857in}}%
\pgfpathlineto{\pgfqpoint{0.874963in}{1.590346in}}%
\pgfpathlineto{\pgfqpoint{0.860471in}{1.597468in}}%
\pgfpathlineto{\pgfqpoint{0.859306in}{1.597965in}}%
\pgfpathlineto{\pgfqpoint{0.843650in}{1.601017in}}%
\pgfpathlineto{\pgfqpoint{0.827993in}{1.600255in}}%
\pgfpathlineto{\pgfqpoint{0.818391in}{1.597468in}}%
\pgfpathlineto{\pgfqpoint{0.812337in}{1.595465in}}%
\pgfpathlineto{\pgfqpoint{0.796680in}{1.586080in}}%
\pgfpathlineto{\pgfqpoint{0.794003in}{1.583857in}}%
\pgfpathlineto{\pgfqpoint{0.781791in}{1.570246in}}%
\pgfpathlineto{\pgfqpoint{0.781024in}{1.568811in}}%
\pgfpathlineto{\pgfqpoint{0.775464in}{1.556635in}}%
\pgfpathlineto{\pgfqpoint{0.773686in}{1.543024in}}%
\pgfpathlineto{\pgfqpoint{0.776353in}{1.529413in}}%
\pgfpathlineto{\pgfqpoint{0.781024in}{1.520402in}}%
\pgfpathlineto{\pgfqpoint{0.783826in}{1.515802in}}%
\pgfpathlineto{\pgfqpoint{0.796680in}{1.502555in}}%
\pgfpathlineto{\pgfqpoint{0.797169in}{1.502191in}}%
\pgfpathlineto{\pgfqpoint{0.812337in}{1.493388in}}%
\pgfpathlineto{\pgfqpoint{0.827525in}{1.488579in}}%
\pgfpathlineto{\pgfqpoint{0.827993in}{1.488446in}}%
\pgfpathclose%
\pgfpathmoveto{\pgfqpoint{0.830391in}{1.515802in}}%
\pgfpathlineto{\pgfqpoint{0.827993in}{1.516064in}}%
\pgfpathlineto{\pgfqpoint{0.812337in}{1.526337in}}%
\pgfpathlineto{\pgfqpoint{0.809700in}{1.529413in}}%
\pgfpathlineto{\pgfqpoint{0.805233in}{1.543024in}}%
\pgfpathlineto{\pgfqpoint{0.808210in}{1.556635in}}%
\pgfpathlineto{\pgfqpoint{0.812337in}{1.562099in}}%
\pgfpathlineto{\pgfqpoint{0.823243in}{1.570246in}}%
\pgfpathlineto{\pgfqpoint{0.827993in}{1.572410in}}%
\pgfpathlineto{\pgfqpoint{0.843650in}{1.573616in}}%
\pgfpathlineto{\pgfqpoint{0.854632in}{1.570246in}}%
\pgfpathlineto{\pgfqpoint{0.859306in}{1.567912in}}%
\pgfpathlineto{\pgfqpoint{0.869539in}{1.556635in}}%
\pgfpathlineto{\pgfqpoint{0.873077in}{1.543024in}}%
\pgfpathlineto{\pgfqpoint{0.867769in}{1.529413in}}%
\pgfpathlineto{\pgfqpoint{0.859306in}{1.521198in}}%
\pgfpathlineto{\pgfqpoint{0.846999in}{1.515802in}}%
\pgfpathlineto{\pgfqpoint{0.843650in}{1.514843in}}%
\pgfpathlineto{\pgfqpoint{0.830391in}{1.515802in}}%
\pgfpathclose%
\pgfpathmoveto{\pgfqpoint{1.141125in}{1.487993in}}%
\pgfpathlineto{\pgfqpoint{1.156781in}{1.487993in}}%
\pgfpathlineto{\pgfqpoint{1.159247in}{1.488579in}}%
\pgfpathlineto{\pgfqpoint{1.172438in}{1.492066in}}%
\pgfpathlineto{\pgfqpoint{1.188094in}{1.500335in}}%
\pgfpathlineto{\pgfqpoint{1.190545in}{1.502191in}}%
\pgfpathlineto{\pgfqpoint{1.203751in}{1.515199in}}%
\pgfpathlineto{\pgfqpoint{1.204251in}{1.515802in}}%
\pgfpathlineto{\pgfqpoint{1.211500in}{1.529413in}}%
\pgfpathlineto{\pgfqpoint{1.214212in}{1.543024in}}%
\pgfpathlineto{\pgfqpoint{1.212404in}{1.556635in}}%
\pgfpathlineto{\pgfqpoint{1.206064in}{1.570246in}}%
\pgfpathlineto{\pgfqpoint{1.203751in}{1.573237in}}%
\pgfpathlineto{\pgfqpoint{1.193724in}{1.583857in}}%
\pgfpathlineto{\pgfqpoint{1.188094in}{1.588298in}}%
\pgfpathlineto{\pgfqpoint{1.172438in}{1.596829in}}%
\pgfpathlineto{\pgfqpoint{1.170124in}{1.597468in}}%
\pgfpathlineto{\pgfqpoint{1.156781in}{1.600712in}}%
\pgfpathlineto{\pgfqpoint{1.141125in}{1.600712in}}%
\pgfpathlineto{\pgfqpoint{1.127782in}{1.597468in}}%
\pgfpathlineto{\pgfqpoint{1.125468in}{1.596829in}}%
\pgfpathlineto{\pgfqpoint{1.109812in}{1.588298in}}%
\pgfpathlineto{\pgfqpoint{1.104182in}{1.583857in}}%
\pgfpathlineto{\pgfqpoint{1.094155in}{1.573237in}}%
\pgfpathlineto{\pgfqpoint{1.091842in}{1.570246in}}%
\pgfpathlineto{\pgfqpoint{1.085502in}{1.556635in}}%
\pgfpathlineto{\pgfqpoint{1.083693in}{1.543024in}}%
\pgfpathlineto{\pgfqpoint{1.086406in}{1.529413in}}%
\pgfpathlineto{\pgfqpoint{1.093655in}{1.515802in}}%
\pgfpathlineto{\pgfqpoint{1.094155in}{1.515199in}}%
\pgfpathlineto{\pgfqpoint{1.107361in}{1.502191in}}%
\pgfpathlineto{\pgfqpoint{1.109812in}{1.500335in}}%
\pgfpathlineto{\pgfqpoint{1.125468in}{1.492066in}}%
\pgfpathlineto{\pgfqpoint{1.138659in}{1.488579in}}%
\pgfpathlineto{\pgfqpoint{1.141125in}{1.487993in}}%
\pgfpathclose%
\pgfpathmoveto{\pgfqpoint{1.139705in}{1.515802in}}%
\pgfpathlineto{\pgfqpoint{1.125468in}{1.523595in}}%
\pgfpathlineto{\pgfqpoint{1.120031in}{1.529413in}}%
\pgfpathlineto{\pgfqpoint{1.115189in}{1.543024in}}%
\pgfpathlineto{\pgfqpoint{1.118416in}{1.556635in}}%
\pgfpathlineto{\pgfqpoint{1.125468in}{1.565200in}}%
\pgfpathlineto{\pgfqpoint{1.133560in}{1.570246in}}%
\pgfpathlineto{\pgfqpoint{1.141125in}{1.573134in}}%
\pgfpathlineto{\pgfqpoint{1.156781in}{1.573134in}}%
\pgfpathlineto{\pgfqpoint{1.164346in}{1.570246in}}%
\pgfpathlineto{\pgfqpoint{1.172438in}{1.565200in}}%
\pgfpathlineto{\pgfqpoint{1.179490in}{1.556635in}}%
\pgfpathlineto{\pgfqpoint{1.182716in}{1.543024in}}%
\pgfpathlineto{\pgfqpoint{1.177875in}{1.529413in}}%
\pgfpathlineto{\pgfqpoint{1.172438in}{1.523595in}}%
\pgfpathlineto{\pgfqpoint{1.158201in}{1.515802in}}%
\pgfpathlineto{\pgfqpoint{1.156781in}{1.515296in}}%
\pgfpathlineto{\pgfqpoint{1.141125in}{1.515296in}}%
\pgfpathlineto{\pgfqpoint{1.139705in}{1.515802in}}%
\pgfpathclose%
\pgfpathmoveto{\pgfqpoint{1.454256in}{1.487691in}}%
\pgfpathlineto{\pgfqpoint{1.469913in}{1.488446in}}%
\pgfpathlineto{\pgfqpoint{1.470381in}{1.488579in}}%
\pgfpathlineto{\pgfqpoint{1.485569in}{1.493388in}}%
\pgfpathlineto{\pgfqpoint{1.500737in}{1.502191in}}%
\pgfpathlineto{\pgfqpoint{1.501226in}{1.502555in}}%
\pgfpathlineto{\pgfqpoint{1.514079in}{1.515802in}}%
\pgfpathlineto{\pgfqpoint{1.516882in}{1.520402in}}%
\pgfpathlineto{\pgfqpoint{1.521553in}{1.529413in}}%
\pgfpathlineto{\pgfqpoint{1.524219in}{1.543024in}}%
\pgfpathlineto{\pgfqpoint{1.522442in}{1.556635in}}%
\pgfpathlineto{\pgfqpoint{1.516882in}{1.568811in}}%
\pgfpathlineto{\pgfqpoint{1.516115in}{1.570246in}}%
\pgfpathlineto{\pgfqpoint{1.503903in}{1.583857in}}%
\pgfpathlineto{\pgfqpoint{1.501226in}{1.586080in}}%
\pgfpathlineto{\pgfqpoint{1.485569in}{1.595465in}}%
\pgfpathlineto{\pgfqpoint{1.479515in}{1.597468in}}%
\pgfpathlineto{\pgfqpoint{1.469913in}{1.600255in}}%
\pgfpathlineto{\pgfqpoint{1.454256in}{1.601017in}}%
\pgfpathlineto{\pgfqpoint{1.438599in}{1.597965in}}%
\pgfpathlineto{\pgfqpoint{1.437435in}{1.597468in}}%
\pgfpathlineto{\pgfqpoint{1.422943in}{1.590346in}}%
\pgfpathlineto{\pgfqpoint{1.414240in}{1.583857in}}%
\pgfpathlineto{\pgfqpoint{1.407286in}{1.576917in}}%
\pgfpathlineto{\pgfqpoint{1.401954in}{1.570246in}}%
\pgfpathlineto{\pgfqpoint{1.395475in}{1.556635in}}%
\pgfpathlineto{\pgfqpoint{1.393627in}{1.543024in}}%
\pgfpathlineto{\pgfqpoint{1.396399in}{1.529413in}}%
\pgfpathlineto{\pgfqpoint{1.403807in}{1.515802in}}%
\pgfpathlineto{\pgfqpoint{1.407286in}{1.511741in}}%
\pgfpathlineto{\pgfqpoint{1.417575in}{1.502191in}}%
\pgfpathlineto{\pgfqpoint{1.422943in}{1.498350in}}%
\pgfpathlineto{\pgfqpoint{1.438599in}{1.490910in}}%
\pgfpathlineto{\pgfqpoint{1.449608in}{1.488579in}}%
\pgfpathlineto{\pgfqpoint{1.454256in}{1.487691in}}%
\pgfpathclose%
\pgfpathmoveto{\pgfqpoint{1.450907in}{1.515802in}}%
\pgfpathlineto{\pgfqpoint{1.438599in}{1.521198in}}%
\pgfpathlineto{\pgfqpoint{1.430137in}{1.529413in}}%
\pgfpathlineto{\pgfqpoint{1.424829in}{1.543024in}}%
\pgfpathlineto{\pgfqpoint{1.428367in}{1.556635in}}%
\pgfpathlineto{\pgfqpoint{1.438599in}{1.567912in}}%
\pgfpathlineto{\pgfqpoint{1.443274in}{1.570246in}}%
\pgfpathlineto{\pgfqpoint{1.454256in}{1.573616in}}%
\pgfpathlineto{\pgfqpoint{1.469913in}{1.572410in}}%
\pgfpathlineto{\pgfqpoint{1.474663in}{1.570246in}}%
\pgfpathlineto{\pgfqpoint{1.485569in}{1.562099in}}%
\pgfpathlineto{\pgfqpoint{1.489696in}{1.556635in}}%
\pgfpathlineto{\pgfqpoint{1.492673in}{1.543024in}}%
\pgfpathlineto{\pgfqpoint{1.488206in}{1.529413in}}%
\pgfpathlineto{\pgfqpoint{1.485569in}{1.526337in}}%
\pgfpathlineto{\pgfqpoint{1.469913in}{1.516064in}}%
\pgfpathlineto{\pgfqpoint{1.467515in}{1.515802in}}%
\pgfpathlineto{\pgfqpoint{1.454256in}{1.514843in}}%
\pgfpathlineto{\pgfqpoint{1.450907in}{1.515802in}}%
\pgfpathclose%
\pgfpathmoveto{\pgfqpoint{1.767387in}{1.487539in}}%
\pgfpathlineto{\pgfqpoint{1.778182in}{1.488579in}}%
\pgfpathlineto{\pgfqpoint{1.783044in}{1.489094in}}%
\pgfpathlineto{\pgfqpoint{1.798700in}{1.494876in}}%
\pgfpathlineto{\pgfqpoint{1.810367in}{1.502191in}}%
\pgfpathlineto{\pgfqpoint{1.814357in}{1.505417in}}%
\pgfpathlineto{\pgfqpoint{1.823981in}{1.515802in}}%
\pgfpathlineto{\pgfqpoint{1.830014in}{1.526197in}}%
\pgfpathlineto{\pgfqpoint{1.831639in}{1.529413in}}%
\pgfpathlineto{\pgfqpoint{1.834271in}{1.543024in}}%
\pgfpathlineto{\pgfqpoint{1.832517in}{1.556635in}}%
\pgfpathlineto{\pgfqpoint{1.830014in}{1.562257in}}%
\pgfpathlineto{\pgfqpoint{1.825944in}{1.570246in}}%
\pgfpathlineto{\pgfqpoint{1.814357in}{1.583648in}}%
\pgfpathlineto{\pgfqpoint{1.814116in}{1.583857in}}%
\pgfpathlineto{\pgfqpoint{1.798700in}{1.593930in}}%
\pgfpathlineto{\pgfqpoint{1.789510in}{1.597468in}}%
\pgfpathlineto{\pgfqpoint{1.783044in}{1.599644in}}%
\pgfpathlineto{\pgfqpoint{1.767387in}{1.601170in}}%
\pgfpathlineto{\pgfqpoint{1.751731in}{1.598881in}}%
\pgfpathlineto{\pgfqpoint{1.748032in}{1.597468in}}%
\pgfpathlineto{\pgfqpoint{1.736074in}{1.592224in}}%
\pgfpathlineto{\pgfqpoint{1.724128in}{1.583857in}}%
\pgfpathlineto{\pgfqpoint{1.720418in}{1.580389in}}%
\pgfpathlineto{\pgfqpoint{1.712004in}{1.570246in}}%
\pgfpathlineto{\pgfqpoint{1.705353in}{1.556635in}}%
\pgfpathlineto{\pgfqpoint{1.704761in}{1.552409in}}%
\pgfpathlineto{\pgfqpoint{1.703565in}{1.543024in}}%
\pgfpathlineto{\pgfqpoint{1.704761in}{1.536746in}}%
\pgfpathlineto{\pgfqpoint{1.706302in}{1.529413in}}%
\pgfpathlineto{\pgfqpoint{1.713907in}{1.515802in}}%
\pgfpathlineto{\pgfqpoint{1.720418in}{1.508479in}}%
\pgfpathlineto{\pgfqpoint{1.727651in}{1.502191in}}%
\pgfpathlineto{\pgfqpoint{1.736074in}{1.496530in}}%
\pgfpathlineto{\pgfqpoint{1.751731in}{1.489919in}}%
\pgfpathlineto{\pgfqpoint{1.760166in}{1.488579in}}%
\pgfpathlineto{\pgfqpoint{1.767387in}{1.487539in}}%
\pgfpathclose%
\pgfpathmoveto{\pgfqpoint{1.761890in}{1.515802in}}%
\pgfpathlineto{\pgfqpoint{1.751731in}{1.519143in}}%
\pgfpathlineto{\pgfqpoint{1.739918in}{1.529413in}}%
\pgfpathlineto{\pgfqpoint{1.736074in}{1.538244in}}%
\pgfpathlineto{\pgfqpoint{1.734711in}{1.543024in}}%
\pgfpathlineto{\pgfqpoint{1.736074in}{1.550169in}}%
\pgfpathlineto{\pgfqpoint{1.737950in}{1.556635in}}%
\pgfpathlineto{\pgfqpoint{1.751731in}{1.570235in}}%
\pgfpathlineto{\pgfqpoint{1.751760in}{1.570246in}}%
\pgfpathlineto{\pgfqpoint{1.767387in}{1.573857in}}%
\pgfpathlineto{\pgfqpoint{1.783044in}{1.571446in}}%
\pgfpathlineto{\pgfqpoint{1.785314in}{1.570246in}}%
\pgfpathlineto{\pgfqpoint{1.798700in}{1.558609in}}%
\pgfpathlineto{\pgfqpoint{1.800080in}{1.556635in}}%
\pgfpathlineto{\pgfqpoint{1.802854in}{1.543024in}}%
\pgfpathlineto{\pgfqpoint{1.798700in}{1.529438in}}%
\pgfpathlineto{\pgfqpoint{1.798688in}{1.529413in}}%
\pgfpathlineto{\pgfqpoint{1.783044in}{1.517432in}}%
\pgfpathlineto{\pgfqpoint{1.775606in}{1.515802in}}%
\pgfpathlineto{\pgfqpoint{1.767387in}{1.514617in}}%
\pgfpathlineto{\pgfqpoint{1.761890in}{1.515802in}}%
\pgfpathclose%
\pgfusepath{fill}%
\end{pgfscope}%
\begin{pgfscope}%
\pgfpathrectangle{\pgfqpoint{0.373953in}{0.331635in}}{\pgfqpoint{1.550000in}{1.347500in}}%
\pgfusepath{clip}%
\pgfsetbuttcap%
\pgfsetroundjoin%
\definecolor{currentfill}{rgb}{0.921884,0.341098,0.377376}%
\pgfsetfillcolor{currentfill}%
\pgfsetlinewidth{0.000000pt}%
\definecolor{currentstroke}{rgb}{0.000000,0.000000,0.000000}%
\pgfsetstrokecolor{currentstroke}%
\pgfsetdash{}{0pt}%
\pgfpathmoveto{\pgfqpoint{0.499205in}{0.392801in}}%
\pgfpathlineto{\pgfqpoint{0.514862in}{0.388063in}}%
\pgfpathlineto{\pgfqpoint{0.530519in}{0.386711in}}%
\pgfpathlineto{\pgfqpoint{0.546175in}{0.388739in}}%
\pgfpathlineto{\pgfqpoint{0.561832in}{0.394156in}}%
\pgfpathlineto{\pgfqpoint{0.571838in}{0.399691in}}%
\pgfpathlineto{\pgfqpoint{0.577488in}{0.402972in}}%
\pgfpathlineto{\pgfqpoint{0.590943in}{0.413302in}}%
\pgfpathlineto{\pgfqpoint{0.593145in}{0.415301in}}%
\pgfpathlineto{\pgfqpoint{0.604221in}{0.426913in}}%
\pgfpathlineto{\pgfqpoint{0.608801in}{0.433472in}}%
\pgfpathlineto{\pgfqpoint{0.613604in}{0.440524in}}%
\pgfpathlineto{\pgfqpoint{0.619117in}{0.454135in}}%
\pgfpathlineto{\pgfqpoint{0.620689in}{0.467746in}}%
\pgfpathlineto{\pgfqpoint{0.618330in}{0.481357in}}%
\pgfpathlineto{\pgfqpoint{0.612027in}{0.494968in}}%
\pgfpathlineto{\pgfqpoint{0.608801in}{0.499387in}}%
\pgfpathlineto{\pgfqpoint{0.601899in}{0.508579in}}%
\pgfpathlineto{\pgfqpoint{0.593145in}{0.517385in}}%
\pgfpathlineto{\pgfqpoint{0.587617in}{0.522191in}}%
\pgfpathlineto{\pgfqpoint{0.577488in}{0.529801in}}%
\pgfpathlineto{\pgfqpoint{0.566914in}{0.535802in}}%
\pgfpathlineto{\pgfqpoint{0.561832in}{0.538606in}}%
\pgfpathlineto{\pgfqpoint{0.546175in}{0.544086in}}%
\pgfpathlineto{\pgfqpoint{0.530519in}{0.546136in}}%
\pgfpathlineto{\pgfqpoint{0.514862in}{0.544770in}}%
\pgfpathlineto{\pgfqpoint{0.499205in}{0.539977in}}%
\pgfpathlineto{\pgfqpoint{0.491094in}{0.535802in}}%
\pgfpathlineto{\pgfqpoint{0.483549in}{0.531820in}}%
\pgfpathlineto{\pgfqpoint{0.470193in}{0.522191in}}%
\pgfpathlineto{\pgfqpoint{0.467892in}{0.520277in}}%
\pgfpathlineto{\pgfqpoint{0.456010in}{0.508579in}}%
\pgfpathlineto{\pgfqpoint{0.452236in}{0.503667in}}%
\pgfpathlineto{\pgfqpoint{0.445869in}{0.494968in}}%
\pgfpathlineto{\pgfqpoint{0.439638in}{0.481357in}}%
\pgfpathlineto{\pgfqpoint{0.437306in}{0.467746in}}%
\pgfpathlineto{\pgfqpoint{0.438860in}{0.454135in}}%
\pgfpathlineto{\pgfqpoint{0.444310in}{0.440524in}}%
\pgfpathlineto{\pgfqpoint{0.452236in}{0.428917in}}%
\pgfpathlineto{\pgfqpoint{0.453668in}{0.426913in}}%
\pgfpathlineto{\pgfqpoint{0.466878in}{0.413302in}}%
\pgfpathlineto{\pgfqpoint{0.467892in}{0.412420in}}%
\pgfpathlineto{\pgfqpoint{0.483549in}{0.400936in}}%
\pgfpathlineto{\pgfqpoint{0.485854in}{0.399691in}}%
\pgfpathlineto{\pgfqpoint{0.499205in}{0.392801in}}%
\pgfpathclose%
\pgfpathmoveto{\pgfqpoint{0.508395in}{0.413302in}}%
\pgfpathlineto{\pgfqpoint{0.499205in}{0.416840in}}%
\pgfpathlineto{\pgfqpoint{0.483790in}{0.426913in}}%
\pgfpathlineto{\pgfqpoint{0.483549in}{0.427122in}}%
\pgfpathlineto{\pgfqpoint{0.471962in}{0.440524in}}%
\pgfpathlineto{\pgfqpoint{0.467892in}{0.448513in}}%
\pgfpathlineto{\pgfqpoint{0.465389in}{0.454135in}}%
\pgfpathlineto{\pgfqpoint{0.463635in}{0.467746in}}%
\pgfpathlineto{\pgfqpoint{0.466267in}{0.481357in}}%
\pgfpathlineto{\pgfqpoint{0.467892in}{0.484573in}}%
\pgfpathlineto{\pgfqpoint{0.473925in}{0.494968in}}%
\pgfpathlineto{\pgfqpoint{0.483549in}{0.505353in}}%
\pgfpathlineto{\pgfqpoint{0.487539in}{0.508579in}}%
\pgfpathlineto{\pgfqpoint{0.499205in}{0.515894in}}%
\pgfpathlineto{\pgfqpoint{0.514862in}{0.521676in}}%
\pgfpathlineto{\pgfqpoint{0.519724in}{0.522191in}}%
\pgfpathlineto{\pgfqpoint{0.530519in}{0.523231in}}%
\pgfpathlineto{\pgfqpoint{0.537740in}{0.522191in}}%
\pgfpathlineto{\pgfqpoint{0.546175in}{0.520851in}}%
\pgfpathlineto{\pgfqpoint{0.561832in}{0.514240in}}%
\pgfpathlineto{\pgfqpoint{0.570255in}{0.508579in}}%
\pgfpathlineto{\pgfqpoint{0.577488in}{0.502291in}}%
\pgfpathlineto{\pgfqpoint{0.583999in}{0.494968in}}%
\pgfpathlineto{\pgfqpoint{0.591604in}{0.481357in}}%
\pgfpathlineto{\pgfqpoint{0.593145in}{0.474024in}}%
\pgfpathlineto{\pgfqpoint{0.594341in}{0.467746in}}%
\pgfpathlineto{\pgfqpoint{0.593145in}{0.458361in}}%
\pgfpathlineto{\pgfqpoint{0.592553in}{0.454135in}}%
\pgfpathlineto{\pgfqpoint{0.585901in}{0.440524in}}%
\pgfpathlineto{\pgfqpoint{0.577488in}{0.430381in}}%
\pgfpathlineto{\pgfqpoint{0.573777in}{0.426913in}}%
\pgfpathlineto{\pgfqpoint{0.561832in}{0.418546in}}%
\pgfpathlineto{\pgfqpoint{0.549874in}{0.413302in}}%
\pgfpathlineto{\pgfqpoint{0.546175in}{0.411889in}}%
\pgfpathlineto{\pgfqpoint{0.530519in}{0.409600in}}%
\pgfpathlineto{\pgfqpoint{0.514862in}{0.411126in}}%
\pgfpathlineto{\pgfqpoint{0.508395in}{0.413302in}}%
\pgfpathclose%
\pgfpathmoveto{\pgfqpoint{0.796680in}{0.399035in}}%
\pgfpathlineto{\pgfqpoint{0.812337in}{0.391581in}}%
\pgfpathlineto{\pgfqpoint{0.827993in}{0.387522in}}%
\pgfpathlineto{\pgfqpoint{0.843650in}{0.386847in}}%
\pgfpathlineto{\pgfqpoint{0.859306in}{0.389550in}}%
\pgfpathlineto{\pgfqpoint{0.874963in}{0.395647in}}%
\pgfpathlineto{\pgfqpoint{0.881830in}{0.399691in}}%
\pgfpathlineto{\pgfqpoint{0.890620in}{0.405140in}}%
\pgfpathlineto{\pgfqpoint{0.900863in}{0.413302in}}%
\pgfpathlineto{\pgfqpoint{0.906276in}{0.418416in}}%
\pgfpathlineto{\pgfqpoint{0.914238in}{0.426913in}}%
\pgfpathlineto{\pgfqpoint{0.921933in}{0.438143in}}%
\pgfpathlineto{\pgfqpoint{0.923556in}{0.440524in}}%
\pgfpathlineto{\pgfqpoint{0.929157in}{0.454135in}}%
\pgfpathlineto{\pgfqpoint{0.930754in}{0.467746in}}%
\pgfpathlineto{\pgfqpoint{0.928358in}{0.481357in}}%
\pgfpathlineto{\pgfqpoint{0.921954in}{0.494968in}}%
\pgfpathlineto{\pgfqpoint{0.921933in}{0.494998in}}%
\pgfpathlineto{\pgfqpoint{0.911928in}{0.508579in}}%
\pgfpathlineto{\pgfqpoint{0.906276in}{0.514366in}}%
\pgfpathlineto{\pgfqpoint{0.897623in}{0.522191in}}%
\pgfpathlineto{\pgfqpoint{0.890620in}{0.527652in}}%
\pgfpathlineto{\pgfqpoint{0.877169in}{0.535802in}}%
\pgfpathlineto{\pgfqpoint{0.874963in}{0.537097in}}%
\pgfpathlineto{\pgfqpoint{0.859306in}{0.543264in}}%
\pgfpathlineto{\pgfqpoint{0.843650in}{0.546000in}}%
\pgfpathlineto{\pgfqpoint{0.827993in}{0.545316in}}%
\pgfpathlineto{\pgfqpoint{0.812337in}{0.541210in}}%
\pgfpathlineto{\pgfqpoint{0.800985in}{0.535802in}}%
\pgfpathlineto{\pgfqpoint{0.796680in}{0.533705in}}%
\pgfpathlineto{\pgfqpoint{0.781024in}{0.522966in}}%
\pgfpathlineto{\pgfqpoint{0.780089in}{0.522191in}}%
\pgfpathlineto{\pgfqpoint{0.765962in}{0.508579in}}%
\pgfpathlineto{\pgfqpoint{0.765367in}{0.507825in}}%
\pgfpathlineto{\pgfqpoint{0.755895in}{0.494968in}}%
\pgfpathlineto{\pgfqpoint{0.749710in}{0.481358in}}%
\pgfpathlineto{\pgfqpoint{0.749710in}{0.481357in}}%
\pgfpathlineto{\pgfqpoint{0.747204in}{0.467746in}}%
\pgfpathlineto{\pgfqpoint{0.748874in}{0.454135in}}%
\pgfpathlineto{\pgfqpoint{0.749710in}{0.452149in}}%
\pgfpathlineto{\pgfqpoint{0.754348in}{0.440524in}}%
\pgfpathlineto{\pgfqpoint{0.763631in}{0.426913in}}%
\pgfpathlineto{\pgfqpoint{0.765367in}{0.425015in}}%
\pgfpathlineto{\pgfqpoint{0.776972in}{0.413302in}}%
\pgfpathlineto{\pgfqpoint{0.781024in}{0.409867in}}%
\pgfpathlineto{\pgfqpoint{0.795681in}{0.399691in}}%
\pgfpathlineto{\pgfqpoint{0.796680in}{0.399035in}}%
\pgfpathclose%
\pgfpathmoveto{\pgfqpoint{0.818391in}{0.413302in}}%
\pgfpathlineto{\pgfqpoint{0.812337in}{0.415305in}}%
\pgfpathlineto{\pgfqpoint{0.796680in}{0.424690in}}%
\pgfpathlineto{\pgfqpoint{0.794003in}{0.426913in}}%
\pgfpathlineto{\pgfqpoint{0.781791in}{0.440524in}}%
\pgfpathlineto{\pgfqpoint{0.781024in}{0.441959in}}%
\pgfpathlineto{\pgfqpoint{0.775464in}{0.454135in}}%
\pgfpathlineto{\pgfqpoint{0.773686in}{0.467746in}}%
\pgfpathlineto{\pgfqpoint{0.776353in}{0.481357in}}%
\pgfpathlineto{\pgfqpoint{0.781024in}{0.490368in}}%
\pgfpathlineto{\pgfqpoint{0.783826in}{0.494968in}}%
\pgfpathlineto{\pgfqpoint{0.796680in}{0.508215in}}%
\pgfpathlineto{\pgfqpoint{0.797169in}{0.508579in}}%
\pgfpathlineto{\pgfqpoint{0.812337in}{0.517382in}}%
\pgfpathlineto{\pgfqpoint{0.827525in}{0.522191in}}%
\pgfpathlineto{\pgfqpoint{0.827993in}{0.522324in}}%
\pgfpathlineto{\pgfqpoint{0.843650in}{0.523079in}}%
\pgfpathlineto{\pgfqpoint{0.848298in}{0.522191in}}%
\pgfpathlineto{\pgfqpoint{0.859306in}{0.519860in}}%
\pgfpathlineto{\pgfqpoint{0.874963in}{0.512420in}}%
\pgfpathlineto{\pgfqpoint{0.880331in}{0.508579in}}%
\pgfpathlineto{\pgfqpoint{0.890620in}{0.499029in}}%
\pgfpathlineto{\pgfqpoint{0.894099in}{0.494968in}}%
\pgfpathlineto{\pgfqpoint{0.901506in}{0.481357in}}%
\pgfpathlineto{\pgfqpoint{0.904279in}{0.467746in}}%
\pgfpathlineto{\pgfqpoint{0.902431in}{0.454135in}}%
\pgfpathlineto{\pgfqpoint{0.895952in}{0.440524in}}%
\pgfpathlineto{\pgfqpoint{0.890620in}{0.433853in}}%
\pgfpathlineto{\pgfqpoint{0.883666in}{0.426913in}}%
\pgfpathlineto{\pgfqpoint{0.874963in}{0.420424in}}%
\pgfpathlineto{\pgfqpoint{0.860471in}{0.413302in}}%
\pgfpathlineto{\pgfqpoint{0.859306in}{0.412805in}}%
\pgfpathlineto{\pgfqpoint{0.843650in}{0.409753in}}%
\pgfpathlineto{\pgfqpoint{0.827993in}{0.410515in}}%
\pgfpathlineto{\pgfqpoint{0.818391in}{0.413302in}}%
\pgfpathclose%
\pgfpathmoveto{\pgfqpoint{1.109812in}{0.397274in}}%
\pgfpathlineto{\pgfqpoint{1.125468in}{0.390498in}}%
\pgfpathlineto{\pgfqpoint{1.141125in}{0.387117in}}%
\pgfpathlineto{\pgfqpoint{1.156781in}{0.387117in}}%
\pgfpathlineto{\pgfqpoint{1.172438in}{0.390498in}}%
\pgfpathlineto{\pgfqpoint{1.188094in}{0.397274in}}%
\pgfpathlineto{\pgfqpoint{1.191973in}{0.399691in}}%
\pgfpathlineto{\pgfqpoint{1.203751in}{0.407440in}}%
\pgfpathlineto{\pgfqpoint{1.210870in}{0.413302in}}%
\pgfpathlineto{\pgfqpoint{1.219407in}{0.421656in}}%
\pgfpathlineto{\pgfqpoint{1.224265in}{0.426913in}}%
\pgfpathlineto{\pgfqpoint{1.233517in}{0.440524in}}%
\pgfpathlineto{\pgfqpoint{1.235064in}{0.444360in}}%
\pgfpathlineto{\pgfqpoint{1.239138in}{0.454135in}}%
\pgfpathlineto{\pgfqpoint{1.240768in}{0.467746in}}%
\pgfpathlineto{\pgfqpoint{1.238323in}{0.481357in}}%
\pgfpathlineto{\pgfqpoint{1.235064in}{0.488245in}}%
\pgfpathlineto{\pgfqpoint{1.231975in}{0.494968in}}%
\pgfpathlineto{\pgfqpoint{1.221955in}{0.508579in}}%
\pgfpathlineto{\pgfqpoint{1.219407in}{0.511225in}}%
\pgfpathlineto{\pgfqpoint{1.207699in}{0.522191in}}%
\pgfpathlineto{\pgfqpoint{1.203751in}{0.525373in}}%
\pgfpathlineto{\pgfqpoint{1.188094in}{0.535458in}}%
\pgfpathlineto{\pgfqpoint{1.187322in}{0.535802in}}%
\pgfpathlineto{\pgfqpoint{1.172438in}{0.542306in}}%
\pgfpathlineto{\pgfqpoint{1.156781in}{0.545727in}}%
\pgfpathlineto{\pgfqpoint{1.141125in}{0.545727in}}%
\pgfpathlineto{\pgfqpoint{1.125468in}{0.542306in}}%
\pgfpathlineto{\pgfqpoint{1.110584in}{0.535802in}}%
\pgfpathlineto{\pgfqpoint{1.109812in}{0.535458in}}%
\pgfpathlineto{\pgfqpoint{1.094155in}{0.525373in}}%
\pgfpathlineto{\pgfqpoint{1.090207in}{0.522191in}}%
\pgfpathlineto{\pgfqpoint{1.078498in}{0.511225in}}%
\pgfpathlineto{\pgfqpoint{1.075951in}{0.508579in}}%
\pgfpathlineto{\pgfqpoint{1.065931in}{0.494968in}}%
\pgfpathlineto{\pgfqpoint{1.062842in}{0.488245in}}%
\pgfpathlineto{\pgfqpoint{1.059583in}{0.481357in}}%
\pgfpathlineto{\pgfqpoint{1.057138in}{0.467746in}}%
\pgfpathlineto{\pgfqpoint{1.058767in}{0.454135in}}%
\pgfpathlineto{\pgfqpoint{1.062842in}{0.444360in}}%
\pgfpathlineto{\pgfqpoint{1.064389in}{0.440524in}}%
\pgfpathlineto{\pgfqpoint{1.073641in}{0.426913in}}%
\pgfpathlineto{\pgfqpoint{1.078498in}{0.421656in}}%
\pgfpathlineto{\pgfqpoint{1.087036in}{0.413302in}}%
\pgfpathlineto{\pgfqpoint{1.094155in}{0.407440in}}%
\pgfpathlineto{\pgfqpoint{1.105932in}{0.399691in}}%
\pgfpathlineto{\pgfqpoint{1.109812in}{0.397274in}}%
\pgfpathclose%
\pgfpathmoveto{\pgfqpoint{1.127782in}{0.413302in}}%
\pgfpathlineto{\pgfqpoint{1.125468in}{0.413941in}}%
\pgfpathlineto{\pgfqpoint{1.109812in}{0.422472in}}%
\pgfpathlineto{\pgfqpoint{1.104182in}{0.426913in}}%
\pgfpathlineto{\pgfqpoint{1.094155in}{0.437533in}}%
\pgfpathlineto{\pgfqpoint{1.091842in}{0.440524in}}%
\pgfpathlineto{\pgfqpoint{1.085502in}{0.454135in}}%
\pgfpathlineto{\pgfqpoint{1.083693in}{0.467746in}}%
\pgfpathlineto{\pgfqpoint{1.086406in}{0.481357in}}%
\pgfpathlineto{\pgfqpoint{1.093655in}{0.494968in}}%
\pgfpathlineto{\pgfqpoint{1.094155in}{0.495571in}}%
\pgfpathlineto{\pgfqpoint{1.107361in}{0.508579in}}%
\pgfpathlineto{\pgfqpoint{1.109812in}{0.510435in}}%
\pgfpathlineto{\pgfqpoint{1.125468in}{0.518704in}}%
\pgfpathlineto{\pgfqpoint{1.138659in}{0.522191in}}%
\pgfpathlineto{\pgfqpoint{1.141125in}{0.522777in}}%
\pgfpathlineto{\pgfqpoint{1.156781in}{0.522777in}}%
\pgfpathlineto{\pgfqpoint{1.159247in}{0.522191in}}%
\pgfpathlineto{\pgfqpoint{1.172438in}{0.518704in}}%
\pgfpathlineto{\pgfqpoint{1.188094in}{0.510435in}}%
\pgfpathlineto{\pgfqpoint{1.190545in}{0.508579in}}%
\pgfpathlineto{\pgfqpoint{1.203751in}{0.495571in}}%
\pgfpathlineto{\pgfqpoint{1.204251in}{0.494968in}}%
\pgfpathlineto{\pgfqpoint{1.211500in}{0.481357in}}%
\pgfpathlineto{\pgfqpoint{1.214212in}{0.467746in}}%
\pgfpathlineto{\pgfqpoint{1.212404in}{0.454135in}}%
\pgfpathlineto{\pgfqpoint{1.206064in}{0.440524in}}%
\pgfpathlineto{\pgfqpoint{1.203751in}{0.437533in}}%
\pgfpathlineto{\pgfqpoint{1.193724in}{0.426913in}}%
\pgfpathlineto{\pgfqpoint{1.188094in}{0.422472in}}%
\pgfpathlineto{\pgfqpoint{1.172438in}{0.413941in}}%
\pgfpathlineto{\pgfqpoint{1.170124in}{0.413302in}}%
\pgfpathlineto{\pgfqpoint{1.156781in}{0.410058in}}%
\pgfpathlineto{\pgfqpoint{1.141125in}{0.410058in}}%
\pgfpathlineto{\pgfqpoint{1.127782in}{0.413302in}}%
\pgfpathclose%
\pgfpathmoveto{\pgfqpoint{1.422943in}{0.395647in}}%
\pgfpathlineto{\pgfqpoint{1.438599in}{0.389550in}}%
\pgfpathlineto{\pgfqpoint{1.454256in}{0.386847in}}%
\pgfpathlineto{\pgfqpoint{1.469913in}{0.387522in}}%
\pgfpathlineto{\pgfqpoint{1.485569in}{0.391581in}}%
\pgfpathlineto{\pgfqpoint{1.501226in}{0.399035in}}%
\pgfpathlineto{\pgfqpoint{1.502225in}{0.399691in}}%
\pgfpathlineto{\pgfqpoint{1.516882in}{0.409867in}}%
\pgfpathlineto{\pgfqpoint{1.520934in}{0.413302in}}%
\pgfpathlineto{\pgfqpoint{1.532539in}{0.425015in}}%
\pgfpathlineto{\pgfqpoint{1.534275in}{0.426913in}}%
\pgfpathlineto{\pgfqpoint{1.543558in}{0.440524in}}%
\pgfpathlineto{\pgfqpoint{1.548195in}{0.452149in}}%
\pgfpathlineto{\pgfqpoint{1.549031in}{0.454135in}}%
\pgfpathlineto{\pgfqpoint{1.550702in}{0.467746in}}%
\pgfpathlineto{\pgfqpoint{1.548196in}{0.481357in}}%
\pgfpathlineto{\pgfqpoint{1.548195in}{0.481358in}}%
\pgfpathlineto{\pgfqpoint{1.542011in}{0.494968in}}%
\pgfpathlineto{\pgfqpoint{1.532539in}{0.507825in}}%
\pgfpathlineto{\pgfqpoint{1.531944in}{0.508579in}}%
\pgfpathlineto{\pgfqpoint{1.517817in}{0.522191in}}%
\pgfpathlineto{\pgfqpoint{1.516882in}{0.522966in}}%
\pgfpathlineto{\pgfqpoint{1.501226in}{0.533705in}}%
\pgfpathlineto{\pgfqpoint{1.496921in}{0.535802in}}%
\pgfpathlineto{\pgfqpoint{1.485569in}{0.541210in}}%
\pgfpathlineto{\pgfqpoint{1.469913in}{0.545316in}}%
\pgfpathlineto{\pgfqpoint{1.454256in}{0.546000in}}%
\pgfpathlineto{\pgfqpoint{1.438599in}{0.543264in}}%
\pgfpathlineto{\pgfqpoint{1.422943in}{0.537097in}}%
\pgfpathlineto{\pgfqpoint{1.420737in}{0.535802in}}%
\pgfpathlineto{\pgfqpoint{1.407286in}{0.527652in}}%
\pgfpathlineto{\pgfqpoint{1.400283in}{0.522191in}}%
\pgfpathlineto{\pgfqpoint{1.391630in}{0.514366in}}%
\pgfpathlineto{\pgfqpoint{1.385978in}{0.508579in}}%
\pgfpathlineto{\pgfqpoint{1.375973in}{0.494998in}}%
\pgfpathlineto{\pgfqpoint{1.375952in}{0.494968in}}%
\pgfpathlineto{\pgfqpoint{1.369548in}{0.481357in}}%
\pgfpathlineto{\pgfqpoint{1.367152in}{0.467746in}}%
\pgfpathlineto{\pgfqpoint{1.368749in}{0.454135in}}%
\pgfpathlineto{\pgfqpoint{1.374350in}{0.440524in}}%
\pgfpathlineto{\pgfqpoint{1.375973in}{0.438143in}}%
\pgfpathlineto{\pgfqpoint{1.383667in}{0.426913in}}%
\pgfpathlineto{\pgfqpoint{1.391630in}{0.418416in}}%
\pgfpathlineto{\pgfqpoint{1.397043in}{0.413302in}}%
\pgfpathlineto{\pgfqpoint{1.407286in}{0.405140in}}%
\pgfpathlineto{\pgfqpoint{1.416076in}{0.399691in}}%
\pgfpathlineto{\pgfqpoint{1.422943in}{0.395647in}}%
\pgfpathclose%
\pgfpathmoveto{\pgfqpoint{1.437435in}{0.413302in}}%
\pgfpathlineto{\pgfqpoint{1.422943in}{0.420424in}}%
\pgfpathlineto{\pgfqpoint{1.414240in}{0.426913in}}%
\pgfpathlineto{\pgfqpoint{1.407286in}{0.433853in}}%
\pgfpathlineto{\pgfqpoint{1.401954in}{0.440524in}}%
\pgfpathlineto{\pgfqpoint{1.395475in}{0.454135in}}%
\pgfpathlineto{\pgfqpoint{1.393627in}{0.467746in}}%
\pgfpathlineto{\pgfqpoint{1.396399in}{0.481357in}}%
\pgfpathlineto{\pgfqpoint{1.403807in}{0.494968in}}%
\pgfpathlineto{\pgfqpoint{1.407286in}{0.499029in}}%
\pgfpathlineto{\pgfqpoint{1.417575in}{0.508579in}}%
\pgfpathlineto{\pgfqpoint{1.422943in}{0.512420in}}%
\pgfpathlineto{\pgfqpoint{1.438599in}{0.519860in}}%
\pgfpathlineto{\pgfqpoint{1.449608in}{0.522191in}}%
\pgfpathlineto{\pgfqpoint{1.454256in}{0.523079in}}%
\pgfpathlineto{\pgfqpoint{1.469913in}{0.522324in}}%
\pgfpathlineto{\pgfqpoint{1.470381in}{0.522191in}}%
\pgfpathlineto{\pgfqpoint{1.485569in}{0.517382in}}%
\pgfpathlineto{\pgfqpoint{1.500737in}{0.508579in}}%
\pgfpathlineto{\pgfqpoint{1.501226in}{0.508215in}}%
\pgfpathlineto{\pgfqpoint{1.514079in}{0.494968in}}%
\pgfpathlineto{\pgfqpoint{1.516882in}{0.490368in}}%
\pgfpathlineto{\pgfqpoint{1.521553in}{0.481357in}}%
\pgfpathlineto{\pgfqpoint{1.524219in}{0.467746in}}%
\pgfpathlineto{\pgfqpoint{1.522442in}{0.454135in}}%
\pgfpathlineto{\pgfqpoint{1.516882in}{0.441959in}}%
\pgfpathlineto{\pgfqpoint{1.516115in}{0.440524in}}%
\pgfpathlineto{\pgfqpoint{1.503903in}{0.426913in}}%
\pgfpathlineto{\pgfqpoint{1.501226in}{0.424690in}}%
\pgfpathlineto{\pgfqpoint{1.485569in}{0.415305in}}%
\pgfpathlineto{\pgfqpoint{1.479515in}{0.413302in}}%
\pgfpathlineto{\pgfqpoint{1.469913in}{0.410515in}}%
\pgfpathlineto{\pgfqpoint{1.454256in}{0.409753in}}%
\pgfpathlineto{\pgfqpoint{1.438599in}{0.412805in}}%
\pgfpathlineto{\pgfqpoint{1.437435in}{0.413302in}}%
\pgfpathclose%
\pgfpathmoveto{\pgfqpoint{1.736074in}{0.394156in}}%
\pgfpathlineto{\pgfqpoint{1.751731in}{0.388739in}}%
\pgfpathlineto{\pgfqpoint{1.767387in}{0.386711in}}%
\pgfpathlineto{\pgfqpoint{1.783044in}{0.388063in}}%
\pgfpathlineto{\pgfqpoint{1.798700in}{0.392801in}}%
\pgfpathlineto{\pgfqpoint{1.812052in}{0.399691in}}%
\pgfpathlineto{\pgfqpoint{1.814357in}{0.400936in}}%
\pgfpathlineto{\pgfqpoint{1.830014in}{0.412420in}}%
\pgfpathlineto{\pgfqpoint{1.831028in}{0.413302in}}%
\pgfpathlineto{\pgfqpoint{1.844238in}{0.426913in}}%
\pgfpathlineto{\pgfqpoint{1.845670in}{0.428917in}}%
\pgfpathlineto{\pgfqpoint{1.853595in}{0.440524in}}%
\pgfpathlineto{\pgfqpoint{1.859045in}{0.454135in}}%
\pgfpathlineto{\pgfqpoint{1.860600in}{0.467746in}}%
\pgfpathlineto{\pgfqpoint{1.858268in}{0.481357in}}%
\pgfpathlineto{\pgfqpoint{1.852037in}{0.494968in}}%
\pgfpathlineto{\pgfqpoint{1.845670in}{0.503667in}}%
\pgfpathlineto{\pgfqpoint{1.841896in}{0.508579in}}%
\pgfpathlineto{\pgfqpoint{1.830014in}{0.520277in}}%
\pgfpathlineto{\pgfqpoint{1.827713in}{0.522191in}}%
\pgfpathlineto{\pgfqpoint{1.814357in}{0.531820in}}%
\pgfpathlineto{\pgfqpoint{1.806812in}{0.535802in}}%
\pgfpathlineto{\pgfqpoint{1.798700in}{0.539977in}}%
\pgfpathlineto{\pgfqpoint{1.783044in}{0.544770in}}%
\pgfpathlineto{\pgfqpoint{1.767387in}{0.546136in}}%
\pgfpathlineto{\pgfqpoint{1.751731in}{0.544086in}}%
\pgfpathlineto{\pgfqpoint{1.736074in}{0.538606in}}%
\pgfpathlineto{\pgfqpoint{1.730991in}{0.535802in}}%
\pgfpathlineto{\pgfqpoint{1.720418in}{0.529801in}}%
\pgfpathlineto{\pgfqpoint{1.710289in}{0.522191in}}%
\pgfpathlineto{\pgfqpoint{1.704761in}{0.517385in}}%
\pgfpathlineto{\pgfqpoint{1.696007in}{0.508579in}}%
\pgfpathlineto{\pgfqpoint{1.689104in}{0.499387in}}%
\pgfpathlineto{\pgfqpoint{1.685879in}{0.494968in}}%
\pgfpathlineto{\pgfqpoint{1.679576in}{0.481357in}}%
\pgfpathlineto{\pgfqpoint{1.677217in}{0.467746in}}%
\pgfpathlineto{\pgfqpoint{1.678789in}{0.454135in}}%
\pgfpathlineto{\pgfqpoint{1.684302in}{0.440524in}}%
\pgfpathlineto{\pgfqpoint{1.689104in}{0.433472in}}%
\pgfpathlineto{\pgfqpoint{1.693685in}{0.426913in}}%
\pgfpathlineto{\pgfqpoint{1.704761in}{0.415301in}}%
\pgfpathlineto{\pgfqpoint{1.706963in}{0.413302in}}%
\pgfpathlineto{\pgfqpoint{1.720418in}{0.402972in}}%
\pgfpathlineto{\pgfqpoint{1.726068in}{0.399691in}}%
\pgfpathlineto{\pgfqpoint{1.736074in}{0.394156in}}%
\pgfpathclose%
\pgfpathmoveto{\pgfqpoint{1.748032in}{0.413302in}}%
\pgfpathlineto{\pgfqpoint{1.736074in}{0.418546in}}%
\pgfpathlineto{\pgfqpoint{1.724128in}{0.426913in}}%
\pgfpathlineto{\pgfqpoint{1.720418in}{0.430381in}}%
\pgfpathlineto{\pgfqpoint{1.712004in}{0.440524in}}%
\pgfpathlineto{\pgfqpoint{1.705353in}{0.454135in}}%
\pgfpathlineto{\pgfqpoint{1.704761in}{0.458361in}}%
\pgfpathlineto{\pgfqpoint{1.703565in}{0.467746in}}%
\pgfpathlineto{\pgfqpoint{1.704761in}{0.474024in}}%
\pgfpathlineto{\pgfqpoint{1.706302in}{0.481357in}}%
\pgfpathlineto{\pgfqpoint{1.713907in}{0.494968in}}%
\pgfpathlineto{\pgfqpoint{1.720418in}{0.502291in}}%
\pgfpathlineto{\pgfqpoint{1.727651in}{0.508579in}}%
\pgfpathlineto{\pgfqpoint{1.736074in}{0.514240in}}%
\pgfpathlineto{\pgfqpoint{1.751731in}{0.520851in}}%
\pgfpathlineto{\pgfqpoint{1.760166in}{0.522191in}}%
\pgfpathlineto{\pgfqpoint{1.767387in}{0.523231in}}%
\pgfpathlineto{\pgfqpoint{1.778182in}{0.522191in}}%
\pgfpathlineto{\pgfqpoint{1.783044in}{0.521676in}}%
\pgfpathlineto{\pgfqpoint{1.798700in}{0.515894in}}%
\pgfpathlineto{\pgfqpoint{1.810367in}{0.508579in}}%
\pgfpathlineto{\pgfqpoint{1.814357in}{0.505353in}}%
\pgfpathlineto{\pgfqpoint{1.823981in}{0.494968in}}%
\pgfpathlineto{\pgfqpoint{1.830014in}{0.484573in}}%
\pgfpathlineto{\pgfqpoint{1.831639in}{0.481357in}}%
\pgfpathlineto{\pgfqpoint{1.834271in}{0.467746in}}%
\pgfpathlineto{\pgfqpoint{1.832517in}{0.454135in}}%
\pgfpathlineto{\pgfqpoint{1.830014in}{0.448513in}}%
\pgfpathlineto{\pgfqpoint{1.825944in}{0.440524in}}%
\pgfpathlineto{\pgfqpoint{1.814357in}{0.427122in}}%
\pgfpathlineto{\pgfqpoint{1.814116in}{0.426913in}}%
\pgfpathlineto{\pgfqpoint{1.798700in}{0.416840in}}%
\pgfpathlineto{\pgfqpoint{1.789510in}{0.413302in}}%
\pgfpathlineto{\pgfqpoint{1.783044in}{0.411126in}}%
\pgfpathlineto{\pgfqpoint{1.767387in}{0.409600in}}%
\pgfpathlineto{\pgfqpoint{1.751731in}{0.411889in}}%
\pgfpathlineto{\pgfqpoint{1.748032in}{0.413302in}}%
\pgfpathclose%
\pgfpathmoveto{\pgfqpoint{0.514862in}{0.657575in}}%
\pgfpathlineto{\pgfqpoint{0.530519in}{0.656123in}}%
\pgfpathlineto{\pgfqpoint{0.546175in}{0.658301in}}%
\pgfpathlineto{\pgfqpoint{0.546176in}{0.658302in}}%
\pgfpathlineto{\pgfqpoint{0.561832in}{0.663678in}}%
\pgfpathlineto{\pgfqpoint{0.576621in}{0.671913in}}%
\pgfpathlineto{\pgfqpoint{0.577488in}{0.672430in}}%
\pgfpathlineto{\pgfqpoint{0.593145in}{0.684711in}}%
\pgfpathlineto{\pgfqpoint{0.594037in}{0.685524in}}%
\pgfpathlineto{\pgfqpoint{0.606390in}{0.699135in}}%
\pgfpathlineto{\pgfqpoint{0.608801in}{0.702877in}}%
\pgfpathlineto{\pgfqpoint{0.615023in}{0.712746in}}%
\pgfpathlineto{\pgfqpoint{0.619746in}{0.726357in}}%
\pgfpathlineto{\pgfqpoint{0.620532in}{0.739968in}}%
\pgfpathlineto{\pgfqpoint{0.617386in}{0.753579in}}%
\pgfpathlineto{\pgfqpoint{0.610292in}{0.767191in}}%
\pgfpathlineto{\pgfqpoint{0.608801in}{0.769108in}}%
\pgfpathlineto{\pgfqpoint{0.599427in}{0.780802in}}%
\pgfpathlineto{\pgfqpoint{0.593145in}{0.786890in}}%
\pgfpathlineto{\pgfqpoint{0.584144in}{0.794413in}}%
\pgfpathlineto{\pgfqpoint{0.577488in}{0.799326in}}%
\pgfpathlineto{\pgfqpoint{0.561866in}{0.808024in}}%
\pgfpathlineto{\pgfqpoint{0.561832in}{0.808043in}}%
\pgfpathlineto{\pgfqpoint{0.546175in}{0.813610in}}%
\pgfpathlineto{\pgfqpoint{0.530519in}{0.815693in}}%
\pgfpathlineto{\pgfqpoint{0.514862in}{0.814304in}}%
\pgfpathlineto{\pgfqpoint{0.499205in}{0.809435in}}%
\pgfpathlineto{\pgfqpoint{0.496467in}{0.808024in}}%
\pgfpathlineto{\pgfqpoint{0.483549in}{0.801335in}}%
\pgfpathlineto{\pgfqpoint{0.473775in}{0.794413in}}%
\pgfpathlineto{\pgfqpoint{0.467892in}{0.789707in}}%
\pgfpathlineto{\pgfqpoint{0.458505in}{0.780802in}}%
\pgfpathlineto{\pgfqpoint{0.452236in}{0.773160in}}%
\pgfpathlineto{\pgfqpoint{0.447584in}{0.767191in}}%
\pgfpathlineto{\pgfqpoint{0.440572in}{0.753579in}}%
\pgfpathlineto{\pgfqpoint{0.437462in}{0.739968in}}%
\pgfpathlineto{\pgfqpoint{0.438239in}{0.726357in}}%
\pgfpathlineto{\pgfqpoint{0.442908in}{0.712746in}}%
\pgfpathlineto{\pgfqpoint{0.451482in}{0.699135in}}%
\pgfpathlineto{\pgfqpoint{0.452236in}{0.698266in}}%
\pgfpathlineto{\pgfqpoint{0.463941in}{0.685524in}}%
\pgfpathlineto{\pgfqpoint{0.467892in}{0.682001in}}%
\pgfpathlineto{\pgfqpoint{0.481365in}{0.671913in}}%
\pgfpathlineto{\pgfqpoint{0.483549in}{0.670404in}}%
\pgfpathlineto{\pgfqpoint{0.499205in}{0.662333in}}%
\pgfpathlineto{\pgfqpoint{0.512578in}{0.658302in}}%
\pgfpathlineto{\pgfqpoint{0.514862in}{0.657575in}}%
\pgfpathclose%
\pgfpathmoveto{\pgfqpoint{0.500856in}{0.685524in}}%
\pgfpathlineto{\pgfqpoint{0.499205in}{0.686191in}}%
\pgfpathlineto{\pgfqpoint{0.483549in}{0.696807in}}%
\pgfpathlineto{\pgfqpoint{0.480992in}{0.699135in}}%
\pgfpathlineto{\pgfqpoint{0.470196in}{0.712746in}}%
\pgfpathlineto{\pgfqpoint{0.467892in}{0.718010in}}%
\pgfpathlineto{\pgfqpoint{0.464687in}{0.726357in}}%
\pgfpathlineto{\pgfqpoint{0.463810in}{0.739968in}}%
\pgfpathlineto{\pgfqpoint{0.467321in}{0.753579in}}%
\pgfpathlineto{\pgfqpoint{0.467892in}{0.754592in}}%
\pgfpathlineto{\pgfqpoint{0.476084in}{0.767191in}}%
\pgfpathlineto{\pgfqpoint{0.483549in}{0.774757in}}%
\pgfpathlineto{\pgfqpoint{0.491532in}{0.780802in}}%
\pgfpathlineto{\pgfqpoint{0.499205in}{0.785437in}}%
\pgfpathlineto{\pgfqpoint{0.514862in}{0.791070in}}%
\pgfpathlineto{\pgfqpoint{0.530519in}{0.792676in}}%
\pgfpathlineto{\pgfqpoint{0.546175in}{0.790266in}}%
\pgfpathlineto{\pgfqpoint{0.561832in}{0.783826in}}%
\pgfpathlineto{\pgfqpoint{0.566503in}{0.780802in}}%
\pgfpathlineto{\pgfqpoint{0.577488in}{0.771857in}}%
\pgfpathlineto{\pgfqpoint{0.581906in}{0.767191in}}%
\pgfpathlineto{\pgfqpoint{0.590464in}{0.753579in}}%
\pgfpathlineto{\pgfqpoint{0.593145in}{0.744009in}}%
\pgfpathlineto{\pgfqpoint{0.594167in}{0.739968in}}%
\pgfpathlineto{\pgfqpoint{0.593298in}{0.726357in}}%
\pgfpathlineto{\pgfqpoint{0.593145in}{0.725950in}}%
\pgfpathlineto{\pgfqpoint{0.587613in}{0.712746in}}%
\pgfpathlineto{\pgfqpoint{0.577488in}{0.699560in}}%
\pgfpathlineto{\pgfqpoint{0.577069in}{0.699135in}}%
\pgfpathlineto{\pgfqpoint{0.561832in}{0.687961in}}%
\pgfpathlineto{\pgfqpoint{0.556540in}{0.685524in}}%
\pgfpathlineto{\pgfqpoint{0.546175in}{0.681463in}}%
\pgfpathlineto{\pgfqpoint{0.530519in}{0.679145in}}%
\pgfpathlineto{\pgfqpoint{0.514862in}{0.680690in}}%
\pgfpathlineto{\pgfqpoint{0.500856in}{0.685524in}}%
\pgfpathclose%
\pgfpathmoveto{\pgfqpoint{0.827993in}{0.656994in}}%
\pgfpathlineto{\pgfqpoint{0.843650in}{0.656268in}}%
\pgfpathlineto{\pgfqpoint{0.854656in}{0.658302in}}%
\pgfpathlineto{\pgfqpoint{0.859306in}{0.659107in}}%
\pgfpathlineto{\pgfqpoint{0.874963in}{0.665158in}}%
\pgfpathlineto{\pgfqpoint{0.886358in}{0.671913in}}%
\pgfpathlineto{\pgfqpoint{0.890620in}{0.674627in}}%
\pgfpathlineto{\pgfqpoint{0.903956in}{0.685524in}}%
\pgfpathlineto{\pgfqpoint{0.906276in}{0.687826in}}%
\pgfpathlineto{\pgfqpoint{0.916397in}{0.699135in}}%
\pgfpathlineto{\pgfqpoint{0.921933in}{0.707891in}}%
\pgfpathlineto{\pgfqpoint{0.924998in}{0.712746in}}%
\pgfpathlineto{\pgfqpoint{0.929796in}{0.726357in}}%
\pgfpathlineto{\pgfqpoint{0.930595in}{0.739968in}}%
\pgfpathlineto{\pgfqpoint{0.927398in}{0.753579in}}%
\pgfpathlineto{\pgfqpoint{0.921933in}{0.763982in}}%
\pgfpathlineto{\pgfqpoint{0.920256in}{0.767191in}}%
\pgfpathlineto{\pgfqpoint{0.909466in}{0.780802in}}%
\pgfpathlineto{\pgfqpoint{0.906276in}{0.783949in}}%
\pgfpathlineto{\pgfqpoint{0.894240in}{0.794413in}}%
\pgfpathlineto{\pgfqpoint{0.890620in}{0.797186in}}%
\pgfpathlineto{\pgfqpoint{0.874963in}{0.806566in}}%
\pgfpathlineto{\pgfqpoint{0.871272in}{0.808024in}}%
\pgfpathlineto{\pgfqpoint{0.859306in}{0.812775in}}%
\pgfpathlineto{\pgfqpoint{0.843650in}{0.815554in}}%
\pgfpathlineto{\pgfqpoint{0.827993in}{0.814860in}}%
\pgfpathlineto{\pgfqpoint{0.812337in}{0.810688in}}%
\pgfpathlineto{\pgfqpoint{0.806752in}{0.808024in}}%
\pgfpathlineto{\pgfqpoint{0.796680in}{0.803212in}}%
\pgfpathlineto{\pgfqpoint{0.783671in}{0.794413in}}%
\pgfpathlineto{\pgfqpoint{0.781024in}{0.792395in}}%
\pgfpathlineto{\pgfqpoint{0.768489in}{0.780802in}}%
\pgfpathlineto{\pgfqpoint{0.765367in}{0.777097in}}%
\pgfpathlineto{\pgfqpoint{0.757598in}{0.767191in}}%
\pgfpathlineto{\pgfqpoint{0.750637in}{0.753579in}}%
\pgfpathlineto{\pgfqpoint{0.749710in}{0.749537in}}%
\pgfpathlineto{\pgfqpoint{0.747371in}{0.739968in}}%
\pgfpathlineto{\pgfqpoint{0.748206in}{0.726357in}}%
\pgfpathlineto{\pgfqpoint{0.749710in}{0.722214in}}%
\pgfpathlineto{\pgfqpoint{0.752956in}{0.712746in}}%
\pgfpathlineto{\pgfqpoint{0.761466in}{0.699135in}}%
\pgfpathlineto{\pgfqpoint{0.765367in}{0.694668in}}%
\pgfpathlineto{\pgfqpoint{0.773997in}{0.685524in}}%
\pgfpathlineto{\pgfqpoint{0.781024in}{0.679415in}}%
\pgfpathlineto{\pgfqpoint{0.791542in}{0.671913in}}%
\pgfpathlineto{\pgfqpoint{0.796680in}{0.668522in}}%
\pgfpathlineto{\pgfqpoint{0.812337in}{0.661123in}}%
\pgfpathlineto{\pgfqpoint{0.823227in}{0.658302in}}%
\pgfpathlineto{\pgfqpoint{0.827993in}{0.656994in}}%
\pgfpathclose%
\pgfpathmoveto{\pgfqpoint{0.810796in}{0.685524in}}%
\pgfpathlineto{\pgfqpoint{0.796680in}{0.694332in}}%
\pgfpathlineto{\pgfqpoint{0.791155in}{0.699135in}}%
\pgfpathlineto{\pgfqpoint{0.781024in}{0.711406in}}%
\pgfpathlineto{\pgfqpoint{0.780092in}{0.712746in}}%
\pgfpathlineto{\pgfqpoint{0.774753in}{0.726357in}}%
\pgfpathlineto{\pgfqpoint{0.773864in}{0.739968in}}%
\pgfpathlineto{\pgfqpoint{0.777421in}{0.753579in}}%
\pgfpathlineto{\pgfqpoint{0.781024in}{0.759804in}}%
\pgfpathlineto{\pgfqpoint{0.786066in}{0.767191in}}%
\pgfpathlineto{\pgfqpoint{0.796680in}{0.777466in}}%
\pgfpathlineto{\pgfqpoint{0.801455in}{0.780802in}}%
\pgfpathlineto{\pgfqpoint{0.812337in}{0.786887in}}%
\pgfpathlineto{\pgfqpoint{0.827993in}{0.791713in}}%
\pgfpathlineto{\pgfqpoint{0.843650in}{0.792516in}}%
\pgfpathlineto{\pgfqpoint{0.859306in}{0.789301in}}%
\pgfpathlineto{\pgfqpoint{0.874963in}{0.782054in}}%
\pgfpathlineto{\pgfqpoint{0.876779in}{0.780802in}}%
\pgfpathlineto{\pgfqpoint{0.890620in}{0.768770in}}%
\pgfpathlineto{\pgfqpoint{0.892060in}{0.767191in}}%
\pgfpathlineto{\pgfqpoint{0.900396in}{0.753579in}}%
\pgfpathlineto{\pgfqpoint{0.904094in}{0.739968in}}%
\pgfpathlineto{\pgfqpoint{0.903170in}{0.726357in}}%
\pgfpathlineto{\pgfqpoint{0.897619in}{0.712746in}}%
\pgfpathlineto{\pgfqpoint{0.890620in}{0.703286in}}%
\pgfpathlineto{\pgfqpoint{0.886782in}{0.699135in}}%
\pgfpathlineto{\pgfqpoint{0.874963in}{0.689907in}}%
\pgfpathlineto{\pgfqpoint{0.866466in}{0.685524in}}%
\pgfpathlineto{\pgfqpoint{0.859306in}{0.682392in}}%
\pgfpathlineto{\pgfqpoint{0.843650in}{0.679300in}}%
\pgfpathlineto{\pgfqpoint{0.827993in}{0.680072in}}%
\pgfpathlineto{\pgfqpoint{0.812337in}{0.684714in}}%
\pgfpathlineto{\pgfqpoint{0.810796in}{0.685524in}}%
\pgfpathclose%
\pgfpathmoveto{\pgfqpoint{1.141125in}{0.656558in}}%
\pgfpathlineto{\pgfqpoint{1.156781in}{0.656558in}}%
\pgfpathlineto{\pgfqpoint{1.164365in}{0.658302in}}%
\pgfpathlineto{\pgfqpoint{1.172438in}{0.660048in}}%
\pgfpathlineto{\pgfqpoint{1.188094in}{0.666773in}}%
\pgfpathlineto{\pgfqpoint{1.196289in}{0.671913in}}%
\pgfpathlineto{\pgfqpoint{1.203751in}{0.676956in}}%
\pgfpathlineto{\pgfqpoint{1.213896in}{0.685524in}}%
\pgfpathlineto{\pgfqpoint{1.219407in}{0.691186in}}%
\pgfpathlineto{\pgfqpoint{1.226422in}{0.699135in}}%
\pgfpathlineto{\pgfqpoint{1.234904in}{0.712746in}}%
\pgfpathlineto{\pgfqpoint{1.235064in}{0.713207in}}%
\pgfpathlineto{\pgfqpoint{1.239790in}{0.726357in}}%
\pgfpathlineto{\pgfqpoint{1.240605in}{0.739968in}}%
\pgfpathlineto{\pgfqpoint{1.237344in}{0.753579in}}%
\pgfpathlineto{\pgfqpoint{1.235064in}{0.757895in}}%
\pgfpathlineto{\pgfqpoint{1.230278in}{0.767191in}}%
\pgfpathlineto{\pgfqpoint{1.219496in}{0.780802in}}%
\pgfpathlineto{\pgfqpoint{1.219407in}{0.780890in}}%
\pgfpathlineto{\pgfqpoint{1.204389in}{0.794413in}}%
\pgfpathlineto{\pgfqpoint{1.203751in}{0.794918in}}%
\pgfpathlineto{\pgfqpoint{1.188094in}{0.804956in}}%
\pgfpathlineto{\pgfqpoint{1.181071in}{0.808024in}}%
\pgfpathlineto{\pgfqpoint{1.172438in}{0.811802in}}%
\pgfpathlineto{\pgfqpoint{1.156781in}{0.815276in}}%
\pgfpathlineto{\pgfqpoint{1.141125in}{0.815276in}}%
\pgfpathlineto{\pgfqpoint{1.125468in}{0.811802in}}%
\pgfpathlineto{\pgfqpoint{1.116835in}{0.808024in}}%
\pgfpathlineto{\pgfqpoint{1.109812in}{0.804956in}}%
\pgfpathlineto{\pgfqpoint{1.094155in}{0.794918in}}%
\pgfpathlineto{\pgfqpoint{1.093517in}{0.794413in}}%
\pgfpathlineto{\pgfqpoint{1.078498in}{0.780890in}}%
\pgfpathlineto{\pgfqpoint{1.078410in}{0.780802in}}%
\pgfpathlineto{\pgfqpoint{1.067628in}{0.767191in}}%
\pgfpathlineto{\pgfqpoint{1.062842in}{0.757895in}}%
\pgfpathlineto{\pgfqpoint{1.060562in}{0.753579in}}%
\pgfpathlineto{\pgfqpoint{1.057301in}{0.739968in}}%
\pgfpathlineto{\pgfqpoint{1.058115in}{0.726357in}}%
\pgfpathlineto{\pgfqpoint{1.062842in}{0.713207in}}%
\pgfpathlineto{\pgfqpoint{1.063002in}{0.712746in}}%
\pgfpathlineto{\pgfqpoint{1.071484in}{0.699135in}}%
\pgfpathlineto{\pgfqpoint{1.078498in}{0.691186in}}%
\pgfpathlineto{\pgfqpoint{1.084009in}{0.685524in}}%
\pgfpathlineto{\pgfqpoint{1.094155in}{0.676956in}}%
\pgfpathlineto{\pgfqpoint{1.101617in}{0.671913in}}%
\pgfpathlineto{\pgfqpoint{1.109812in}{0.666773in}}%
\pgfpathlineto{\pgfqpoint{1.125468in}{0.660048in}}%
\pgfpathlineto{\pgfqpoint{1.133541in}{0.658302in}}%
\pgfpathlineto{\pgfqpoint{1.141125in}{0.656558in}}%
\pgfpathclose%
\pgfpathmoveto{\pgfqpoint{1.121218in}{0.685524in}}%
\pgfpathlineto{\pgfqpoint{1.109812in}{0.692031in}}%
\pgfpathlineto{\pgfqpoint{1.101212in}{0.699135in}}%
\pgfpathlineto{\pgfqpoint{1.094155in}{0.707236in}}%
\pgfpathlineto{\pgfqpoint{1.090210in}{0.712746in}}%
\pgfpathlineto{\pgfqpoint{1.084778in}{0.726357in}}%
\pgfpathlineto{\pgfqpoint{1.083874in}{0.739968in}}%
\pgfpathlineto{\pgfqpoint{1.087492in}{0.753579in}}%
\pgfpathlineto{\pgfqpoint{1.094155in}{0.764760in}}%
\pgfpathlineto{\pgfqpoint{1.095905in}{0.767191in}}%
\pgfpathlineto{\pgfqpoint{1.109812in}{0.779982in}}%
\pgfpathlineto{\pgfqpoint{1.111094in}{0.780802in}}%
\pgfpathlineto{\pgfqpoint{1.125468in}{0.788175in}}%
\pgfpathlineto{\pgfqpoint{1.141125in}{0.792194in}}%
\pgfpathlineto{\pgfqpoint{1.156781in}{0.792194in}}%
\pgfpathlineto{\pgfqpoint{1.172438in}{0.788175in}}%
\pgfpathlineto{\pgfqpoint{1.186812in}{0.780802in}}%
\pgfpathlineto{\pgfqpoint{1.188094in}{0.779982in}}%
\pgfpathlineto{\pgfqpoint{1.202000in}{0.767191in}}%
\pgfpathlineto{\pgfqpoint{1.203751in}{0.764760in}}%
\pgfpathlineto{\pgfqpoint{1.210413in}{0.753579in}}%
\pgfpathlineto{\pgfqpoint{1.214032in}{0.739968in}}%
\pgfpathlineto{\pgfqpoint{1.213128in}{0.726357in}}%
\pgfpathlineto{\pgfqpoint{1.207696in}{0.712746in}}%
\pgfpathlineto{\pgfqpoint{1.203751in}{0.707236in}}%
\pgfpathlineto{\pgfqpoint{1.196693in}{0.699135in}}%
\pgfpathlineto{\pgfqpoint{1.188094in}{0.692031in}}%
\pgfpathlineto{\pgfqpoint{1.176687in}{0.685524in}}%
\pgfpathlineto{\pgfqpoint{1.172438in}{0.683475in}}%
\pgfpathlineto{\pgfqpoint{1.156781in}{0.679609in}}%
\pgfpathlineto{\pgfqpoint{1.141125in}{0.679609in}}%
\pgfpathlineto{\pgfqpoint{1.125468in}{0.683475in}}%
\pgfpathlineto{\pgfqpoint{1.121218in}{0.685524in}}%
\pgfpathclose%
\pgfpathmoveto{\pgfqpoint{1.454256in}{0.656268in}}%
\pgfpathlineto{\pgfqpoint{1.469913in}{0.656994in}}%
\pgfpathlineto{\pgfqpoint{1.474679in}{0.658302in}}%
\pgfpathlineto{\pgfqpoint{1.485569in}{0.661123in}}%
\pgfpathlineto{\pgfqpoint{1.501226in}{0.668522in}}%
\pgfpathlineto{\pgfqpoint{1.506363in}{0.671913in}}%
\pgfpathlineto{\pgfqpoint{1.516882in}{0.679415in}}%
\pgfpathlineto{\pgfqpoint{1.523909in}{0.685524in}}%
\pgfpathlineto{\pgfqpoint{1.532539in}{0.694668in}}%
\pgfpathlineto{\pgfqpoint{1.536439in}{0.699135in}}%
\pgfpathlineto{\pgfqpoint{1.544950in}{0.712746in}}%
\pgfpathlineto{\pgfqpoint{1.548195in}{0.722214in}}%
\pgfpathlineto{\pgfqpoint{1.549700in}{0.726357in}}%
\pgfpathlineto{\pgfqpoint{1.550535in}{0.739968in}}%
\pgfpathlineto{\pgfqpoint{1.548195in}{0.749537in}}%
\pgfpathlineto{\pgfqpoint{1.547269in}{0.753579in}}%
\pgfpathlineto{\pgfqpoint{1.540308in}{0.767191in}}%
\pgfpathlineto{\pgfqpoint{1.532539in}{0.777097in}}%
\pgfpathlineto{\pgfqpoint{1.529417in}{0.780802in}}%
\pgfpathlineto{\pgfqpoint{1.516882in}{0.792395in}}%
\pgfpathlineto{\pgfqpoint{1.514235in}{0.794413in}}%
\pgfpathlineto{\pgfqpoint{1.501226in}{0.803212in}}%
\pgfpathlineto{\pgfqpoint{1.491154in}{0.808024in}}%
\pgfpathlineto{\pgfqpoint{1.485569in}{0.810688in}}%
\pgfpathlineto{\pgfqpoint{1.469913in}{0.814860in}}%
\pgfpathlineto{\pgfqpoint{1.454256in}{0.815554in}}%
\pgfpathlineto{\pgfqpoint{1.438599in}{0.812775in}}%
\pgfpathlineto{\pgfqpoint{1.426633in}{0.808024in}}%
\pgfpathlineto{\pgfqpoint{1.422943in}{0.806566in}}%
\pgfpathlineto{\pgfqpoint{1.407286in}{0.797186in}}%
\pgfpathlineto{\pgfqpoint{1.403666in}{0.794413in}}%
\pgfpathlineto{\pgfqpoint{1.391630in}{0.783949in}}%
\pgfpathlineto{\pgfqpoint{1.388439in}{0.780802in}}%
\pgfpathlineto{\pgfqpoint{1.377650in}{0.767191in}}%
\pgfpathlineto{\pgfqpoint{1.375973in}{0.763982in}}%
\pgfpathlineto{\pgfqpoint{1.370508in}{0.753579in}}%
\pgfpathlineto{\pgfqpoint{1.367311in}{0.739968in}}%
\pgfpathlineto{\pgfqpoint{1.368110in}{0.726357in}}%
\pgfpathlineto{\pgfqpoint{1.372908in}{0.712746in}}%
\pgfpathlineto{\pgfqpoint{1.375973in}{0.707891in}}%
\pgfpathlineto{\pgfqpoint{1.381508in}{0.699135in}}%
\pgfpathlineto{\pgfqpoint{1.391630in}{0.687826in}}%
\pgfpathlineto{\pgfqpoint{1.393950in}{0.685524in}}%
\pgfpathlineto{\pgfqpoint{1.407286in}{0.674627in}}%
\pgfpathlineto{\pgfqpoint{1.411548in}{0.671913in}}%
\pgfpathlineto{\pgfqpoint{1.422943in}{0.665158in}}%
\pgfpathlineto{\pgfqpoint{1.438599in}{0.659107in}}%
\pgfpathlineto{\pgfqpoint{1.443250in}{0.658302in}}%
\pgfpathlineto{\pgfqpoint{1.454256in}{0.656268in}}%
\pgfpathclose%
\pgfpathmoveto{\pgfqpoint{1.431439in}{0.685524in}}%
\pgfpathlineto{\pgfqpoint{1.422943in}{0.689907in}}%
\pgfpathlineto{\pgfqpoint{1.411124in}{0.699135in}}%
\pgfpathlineto{\pgfqpoint{1.407286in}{0.703286in}}%
\pgfpathlineto{\pgfqpoint{1.400286in}{0.712746in}}%
\pgfpathlineto{\pgfqpoint{1.394736in}{0.726357in}}%
\pgfpathlineto{\pgfqpoint{1.393812in}{0.739968in}}%
\pgfpathlineto{\pgfqpoint{1.397509in}{0.753579in}}%
\pgfpathlineto{\pgfqpoint{1.405846in}{0.767191in}}%
\pgfpathlineto{\pgfqpoint{1.407286in}{0.768770in}}%
\pgfpathlineto{\pgfqpoint{1.421126in}{0.780802in}}%
\pgfpathlineto{\pgfqpoint{1.422943in}{0.782054in}}%
\pgfpathlineto{\pgfqpoint{1.438599in}{0.789301in}}%
\pgfpathlineto{\pgfqpoint{1.454256in}{0.792516in}}%
\pgfpathlineto{\pgfqpoint{1.469913in}{0.791713in}}%
\pgfpathlineto{\pgfqpoint{1.485569in}{0.786887in}}%
\pgfpathlineto{\pgfqpoint{1.496451in}{0.780802in}}%
\pgfpathlineto{\pgfqpoint{1.501226in}{0.777466in}}%
\pgfpathlineto{\pgfqpoint{1.511840in}{0.767191in}}%
\pgfpathlineto{\pgfqpoint{1.516882in}{0.759804in}}%
\pgfpathlineto{\pgfqpoint{1.520485in}{0.753579in}}%
\pgfpathlineto{\pgfqpoint{1.524042in}{0.739968in}}%
\pgfpathlineto{\pgfqpoint{1.523153in}{0.726357in}}%
\pgfpathlineto{\pgfqpoint{1.517814in}{0.712746in}}%
\pgfpathlineto{\pgfqpoint{1.516882in}{0.711406in}}%
\pgfpathlineto{\pgfqpoint{1.506751in}{0.699135in}}%
\pgfpathlineto{\pgfqpoint{1.501226in}{0.694332in}}%
\pgfpathlineto{\pgfqpoint{1.487110in}{0.685524in}}%
\pgfpathlineto{\pgfqpoint{1.485569in}{0.684714in}}%
\pgfpathlineto{\pgfqpoint{1.469913in}{0.680072in}}%
\pgfpathlineto{\pgfqpoint{1.454256in}{0.679300in}}%
\pgfpathlineto{\pgfqpoint{1.438599in}{0.682392in}}%
\pgfpathlineto{\pgfqpoint{1.431439in}{0.685524in}}%
\pgfpathclose%
\pgfpathmoveto{\pgfqpoint{1.751731in}{0.658301in}}%
\pgfpathlineto{\pgfqpoint{1.767387in}{0.656123in}}%
\pgfpathlineto{\pgfqpoint{1.783044in}{0.657575in}}%
\pgfpathlineto{\pgfqpoint{1.785328in}{0.658302in}}%
\pgfpathlineto{\pgfqpoint{1.798700in}{0.662333in}}%
\pgfpathlineto{\pgfqpoint{1.814357in}{0.670404in}}%
\pgfpathlineto{\pgfqpoint{1.816540in}{0.671913in}}%
\pgfpathlineto{\pgfqpoint{1.830014in}{0.682001in}}%
\pgfpathlineto{\pgfqpoint{1.833964in}{0.685524in}}%
\pgfpathlineto{\pgfqpoint{1.845670in}{0.698266in}}%
\pgfpathlineto{\pgfqpoint{1.846424in}{0.699135in}}%
\pgfpathlineto{\pgfqpoint{1.854998in}{0.712746in}}%
\pgfpathlineto{\pgfqpoint{1.859667in}{0.726357in}}%
\pgfpathlineto{\pgfqpoint{1.860444in}{0.739968in}}%
\pgfpathlineto{\pgfqpoint{1.857334in}{0.753579in}}%
\pgfpathlineto{\pgfqpoint{1.850321in}{0.767191in}}%
\pgfpathlineto{\pgfqpoint{1.845670in}{0.773160in}}%
\pgfpathlineto{\pgfqpoint{1.839401in}{0.780802in}}%
\pgfpathlineto{\pgfqpoint{1.830014in}{0.789707in}}%
\pgfpathlineto{\pgfqpoint{1.824131in}{0.794413in}}%
\pgfpathlineto{\pgfqpoint{1.814357in}{0.801335in}}%
\pgfpathlineto{\pgfqpoint{1.801439in}{0.808024in}}%
\pgfpathlineto{\pgfqpoint{1.798700in}{0.809435in}}%
\pgfpathlineto{\pgfqpoint{1.783044in}{0.814304in}}%
\pgfpathlineto{\pgfqpoint{1.767387in}{0.815693in}}%
\pgfpathlineto{\pgfqpoint{1.751731in}{0.813610in}}%
\pgfpathlineto{\pgfqpoint{1.736074in}{0.808043in}}%
\pgfpathlineto{\pgfqpoint{1.736040in}{0.808024in}}%
\pgfpathlineto{\pgfqpoint{1.720418in}{0.799326in}}%
\pgfpathlineto{\pgfqpoint{1.713762in}{0.794413in}}%
\pgfpathlineto{\pgfqpoint{1.704761in}{0.786890in}}%
\pgfpathlineto{\pgfqpoint{1.698479in}{0.780802in}}%
\pgfpathlineto{\pgfqpoint{1.689104in}{0.769108in}}%
\pgfpathlineto{\pgfqpoint{1.687614in}{0.767191in}}%
\pgfpathlineto{\pgfqpoint{1.680520in}{0.753579in}}%
\pgfpathlineto{\pgfqpoint{1.677374in}{0.739968in}}%
\pgfpathlineto{\pgfqpoint{1.678160in}{0.726357in}}%
\pgfpathlineto{\pgfqpoint{1.682883in}{0.712746in}}%
\pgfpathlineto{\pgfqpoint{1.689104in}{0.702877in}}%
\pgfpathlineto{\pgfqpoint{1.691516in}{0.699135in}}%
\pgfpathlineto{\pgfqpoint{1.703869in}{0.685524in}}%
\pgfpathlineto{\pgfqpoint{1.704761in}{0.684711in}}%
\pgfpathlineto{\pgfqpoint{1.720418in}{0.672430in}}%
\pgfpathlineto{\pgfqpoint{1.721285in}{0.671913in}}%
\pgfpathlineto{\pgfqpoint{1.736074in}{0.663678in}}%
\pgfpathlineto{\pgfqpoint{1.751730in}{0.658302in}}%
\pgfpathlineto{\pgfqpoint{1.751731in}{0.658301in}}%
\pgfpathclose%
\pgfpathmoveto{\pgfqpoint{1.741366in}{0.685524in}}%
\pgfpathlineto{\pgfqpoint{1.736074in}{0.687961in}}%
\pgfpathlineto{\pgfqpoint{1.720837in}{0.699135in}}%
\pgfpathlineto{\pgfqpoint{1.720418in}{0.699560in}}%
\pgfpathlineto{\pgfqpoint{1.710293in}{0.712746in}}%
\pgfpathlineto{\pgfqpoint{1.704761in}{0.725950in}}%
\pgfpathlineto{\pgfqpoint{1.704608in}{0.726357in}}%
\pgfpathlineto{\pgfqpoint{1.703739in}{0.739968in}}%
\pgfpathlineto{\pgfqpoint{1.704761in}{0.744009in}}%
\pgfpathlineto{\pgfqpoint{1.707442in}{0.753579in}}%
\pgfpathlineto{\pgfqpoint{1.716000in}{0.767191in}}%
\pgfpathlineto{\pgfqpoint{1.720418in}{0.771857in}}%
\pgfpathlineto{\pgfqpoint{1.731403in}{0.780802in}}%
\pgfpathlineto{\pgfqpoint{1.736074in}{0.783826in}}%
\pgfpathlineto{\pgfqpoint{1.751731in}{0.790266in}}%
\pgfpathlineto{\pgfqpoint{1.767387in}{0.792676in}}%
\pgfpathlineto{\pgfqpoint{1.783044in}{0.791070in}}%
\pgfpathlineto{\pgfqpoint{1.798700in}{0.785437in}}%
\pgfpathlineto{\pgfqpoint{1.806374in}{0.780802in}}%
\pgfpathlineto{\pgfqpoint{1.814357in}{0.774757in}}%
\pgfpathlineto{\pgfqpoint{1.821821in}{0.767191in}}%
\pgfpathlineto{\pgfqpoint{1.830014in}{0.754592in}}%
\pgfpathlineto{\pgfqpoint{1.830585in}{0.753579in}}%
\pgfpathlineto{\pgfqpoint{1.834096in}{0.739968in}}%
\pgfpathlineto{\pgfqpoint{1.833219in}{0.726357in}}%
\pgfpathlineto{\pgfqpoint{1.830014in}{0.718010in}}%
\pgfpathlineto{\pgfqpoint{1.827710in}{0.712746in}}%
\pgfpathlineto{\pgfqpoint{1.816914in}{0.699135in}}%
\pgfpathlineto{\pgfqpoint{1.814357in}{0.696807in}}%
\pgfpathlineto{\pgfqpoint{1.798700in}{0.686191in}}%
\pgfpathlineto{\pgfqpoint{1.797050in}{0.685524in}}%
\pgfpathlineto{\pgfqpoint{1.783044in}{0.680690in}}%
\pgfpathlineto{\pgfqpoint{1.767387in}{0.679145in}}%
\pgfpathlineto{\pgfqpoint{1.751731in}{0.681463in}}%
\pgfpathlineto{\pgfqpoint{1.741366in}{0.685524in}}%
\pgfpathclose%
\pgfpathmoveto{\pgfqpoint{0.514862in}{0.926982in}}%
\pgfpathlineto{\pgfqpoint{0.530519in}{0.925565in}}%
\pgfpathlineto{\pgfqpoint{0.546175in}{0.927691in}}%
\pgfpathlineto{\pgfqpoint{0.554098in}{0.930524in}}%
\pgfpathlineto{\pgfqpoint{0.561832in}{0.933210in}}%
\pgfpathlineto{\pgfqpoint{0.577488in}{0.941920in}}%
\pgfpathlineto{\pgfqpoint{0.580531in}{0.944135in}}%
\pgfpathlineto{\pgfqpoint{0.593145in}{0.954314in}}%
\pgfpathlineto{\pgfqpoint{0.596805in}{0.957746in}}%
\pgfpathlineto{\pgfqpoint{0.608406in}{0.971357in}}%
\pgfpathlineto{\pgfqpoint{0.608801in}{0.972029in}}%
\pgfpathlineto{\pgfqpoint{0.616283in}{0.984968in}}%
\pgfpathlineto{\pgfqpoint{0.620218in}{0.998579in}}%
\pgfpathlineto{\pgfqpoint{0.620218in}{1.012191in}}%
\pgfpathlineto{\pgfqpoint{0.616283in}{1.025802in}}%
\pgfpathlineto{\pgfqpoint{0.608801in}{1.038741in}}%
\pgfpathlineto{\pgfqpoint{0.608406in}{1.039413in}}%
\pgfpathlineto{\pgfqpoint{0.596805in}{1.053024in}}%
\pgfpathlineto{\pgfqpoint{0.593145in}{1.056456in}}%
\pgfpathlineto{\pgfqpoint{0.580531in}{1.066635in}}%
\pgfpathlineto{\pgfqpoint{0.577488in}{1.068850in}}%
\pgfpathlineto{\pgfqpoint{0.561832in}{1.077560in}}%
\pgfpathlineto{\pgfqpoint{0.554098in}{1.080246in}}%
\pgfpathlineto{\pgfqpoint{0.546175in}{1.083079in}}%
\pgfpathlineto{\pgfqpoint{0.530519in}{1.085205in}}%
\pgfpathlineto{\pgfqpoint{0.514862in}{1.083788in}}%
\pgfpathlineto{\pgfqpoint{0.503618in}{1.080246in}}%
\pgfpathlineto{\pgfqpoint{0.499205in}{1.078901in}}%
\pgfpathlineto{\pgfqpoint{0.483549in}{1.070858in}}%
\pgfpathlineto{\pgfqpoint{0.477502in}{1.066635in}}%
\pgfpathlineto{\pgfqpoint{0.467892in}{1.059213in}}%
\pgfpathlineto{\pgfqpoint{0.461149in}{1.053024in}}%
\pgfpathlineto{\pgfqpoint{0.452236in}{1.042785in}}%
\pgfpathlineto{\pgfqpoint{0.449456in}{1.039413in}}%
\pgfpathlineto{\pgfqpoint{0.441662in}{1.025802in}}%
\pgfpathlineto{\pgfqpoint{0.437772in}{1.012191in}}%
\pgfpathlineto{\pgfqpoint{0.437772in}{0.998579in}}%
\pgfpathlineto{\pgfqpoint{0.441662in}{0.984968in}}%
\pgfpathlineto{\pgfqpoint{0.449456in}{0.971357in}}%
\pgfpathlineto{\pgfqpoint{0.452236in}{0.967985in}}%
\pgfpathlineto{\pgfqpoint{0.461149in}{0.957746in}}%
\pgfpathlineto{\pgfqpoint{0.467892in}{0.951557in}}%
\pgfpathlineto{\pgfqpoint{0.477502in}{0.944135in}}%
\pgfpathlineto{\pgfqpoint{0.483549in}{0.939912in}}%
\pgfpathlineto{\pgfqpoint{0.499205in}{0.931869in}}%
\pgfpathlineto{\pgfqpoint{0.503618in}{0.930524in}}%
\pgfpathlineto{\pgfqpoint{0.514862in}{0.926982in}}%
\pgfpathclose%
\pgfpathmoveto{\pgfqpoint{0.495765in}{0.957746in}}%
\pgfpathlineto{\pgfqpoint{0.483549in}{0.966463in}}%
\pgfpathlineto{\pgfqpoint{0.478440in}{0.971357in}}%
\pgfpathlineto{\pgfqpoint{0.468627in}{0.984968in}}%
\pgfpathlineto{\pgfqpoint{0.467892in}{0.986980in}}%
\pgfpathlineto{\pgfqpoint{0.464161in}{0.998579in}}%
\pgfpathlineto{\pgfqpoint{0.464161in}{1.012191in}}%
\pgfpathlineto{\pgfqpoint{0.467892in}{1.023790in}}%
\pgfpathlineto{\pgfqpoint{0.468627in}{1.025802in}}%
\pgfpathlineto{\pgfqpoint{0.478440in}{1.039413in}}%
\pgfpathlineto{\pgfqpoint{0.483549in}{1.044307in}}%
\pgfpathlineto{\pgfqpoint{0.495765in}{1.053024in}}%
\pgfpathlineto{\pgfqpoint{0.499205in}{1.055035in}}%
\pgfpathlineto{\pgfqpoint{0.514862in}{1.060547in}}%
\pgfpathlineto{\pgfqpoint{0.530519in}{1.062119in}}%
\pgfpathlineto{\pgfqpoint{0.546175in}{1.059760in}}%
\pgfpathlineto{\pgfqpoint{0.561832in}{1.053458in}}%
\pgfpathlineto{\pgfqpoint{0.562525in}{1.053024in}}%
\pgfpathlineto{\pgfqpoint{0.577488in}{1.041543in}}%
\pgfpathlineto{\pgfqpoint{0.579622in}{1.039413in}}%
\pgfpathlineto{\pgfqpoint{0.589134in}{1.025802in}}%
\pgfpathlineto{\pgfqpoint{0.593145in}{1.014334in}}%
\pgfpathlineto{\pgfqpoint{0.593820in}{1.012191in}}%
\pgfpathlineto{\pgfqpoint{0.593820in}{0.998579in}}%
\pgfpathlineto{\pgfqpoint{0.593145in}{0.996436in}}%
\pgfpathlineto{\pgfqpoint{0.589134in}{0.984968in}}%
\pgfpathlineto{\pgfqpoint{0.579622in}{0.971357in}}%
\pgfpathlineto{\pgfqpoint{0.577488in}{0.969227in}}%
\pgfpathlineto{\pgfqpoint{0.562525in}{0.957746in}}%
\pgfpathlineto{\pgfqpoint{0.561832in}{0.957312in}}%
\pgfpathlineto{\pgfqpoint{0.546175in}{0.951010in}}%
\pgfpathlineto{\pgfqpoint{0.530519in}{0.948651in}}%
\pgfpathlineto{\pgfqpoint{0.514862in}{0.950223in}}%
\pgfpathlineto{\pgfqpoint{0.499205in}{0.955735in}}%
\pgfpathlineto{\pgfqpoint{0.495765in}{0.957746in}}%
\pgfpathclose%
\pgfpathmoveto{\pgfqpoint{0.827993in}{0.926415in}}%
\pgfpathlineto{\pgfqpoint{0.843650in}{0.925707in}}%
\pgfpathlineto{\pgfqpoint{0.859306in}{0.928542in}}%
\pgfpathlineto{\pgfqpoint{0.864270in}{0.930524in}}%
\pgfpathlineto{\pgfqpoint{0.874963in}{0.934685in}}%
\pgfpathlineto{\pgfqpoint{0.890620in}{0.944058in}}%
\pgfpathlineto{\pgfqpoint{0.890721in}{0.944135in}}%
\pgfpathlineto{\pgfqpoint{0.906276in}{0.957191in}}%
\pgfpathlineto{\pgfqpoint{0.906857in}{0.957746in}}%
\pgfpathlineto{\pgfqpoint{0.918403in}{0.971357in}}%
\pgfpathlineto{\pgfqpoint{0.921933in}{0.977463in}}%
\pgfpathlineto{\pgfqpoint{0.926278in}{0.984968in}}%
\pgfpathlineto{\pgfqpoint{0.930275in}{0.998579in}}%
\pgfpathlineto{\pgfqpoint{0.930275in}{1.012191in}}%
\pgfpathlineto{\pgfqpoint{0.926278in}{1.025802in}}%
\pgfpathlineto{\pgfqpoint{0.921933in}{1.033307in}}%
\pgfpathlineto{\pgfqpoint{0.918403in}{1.039413in}}%
\pgfpathlineto{\pgfqpoint{0.906857in}{1.053024in}}%
\pgfpathlineto{\pgfqpoint{0.906276in}{1.053579in}}%
\pgfpathlineto{\pgfqpoint{0.890721in}{1.066635in}}%
\pgfpathlineto{\pgfqpoint{0.890620in}{1.066712in}}%
\pgfpathlineto{\pgfqpoint{0.874963in}{1.076085in}}%
\pgfpathlineto{\pgfqpoint{0.864270in}{1.080246in}}%
\pgfpathlineto{\pgfqpoint{0.859306in}{1.082228in}}%
\pgfpathlineto{\pgfqpoint{0.843650in}{1.085063in}}%
\pgfpathlineto{\pgfqpoint{0.827993in}{1.084355in}}%
\pgfpathlineto{\pgfqpoint{0.812867in}{1.080246in}}%
\pgfpathlineto{\pgfqpoint{0.812337in}{1.080107in}}%
\pgfpathlineto{\pgfqpoint{0.796680in}{1.072733in}}%
\pgfpathlineto{\pgfqpoint{0.787536in}{1.066635in}}%
\pgfpathlineto{\pgfqpoint{0.781024in}{1.061844in}}%
\pgfpathlineto{\pgfqpoint{0.771168in}{1.053024in}}%
\pgfpathlineto{\pgfqpoint{0.765367in}{1.046537in}}%
\pgfpathlineto{\pgfqpoint{0.759455in}{1.039413in}}%
\pgfpathlineto{\pgfqpoint{0.751719in}{1.025802in}}%
\pgfpathlineto{\pgfqpoint{0.749710in}{1.018784in}}%
\pgfpathlineto{\pgfqpoint{0.747705in}{1.012191in}}%
\pgfpathlineto{\pgfqpoint{0.747705in}{0.998579in}}%
\pgfpathlineto{\pgfqpoint{0.749710in}{0.991986in}}%
\pgfpathlineto{\pgfqpoint{0.751719in}{0.984968in}}%
\pgfpathlineto{\pgfqpoint{0.759455in}{0.971357in}}%
\pgfpathlineto{\pgfqpoint{0.765367in}{0.964233in}}%
\pgfpathlineto{\pgfqpoint{0.771168in}{0.957746in}}%
\pgfpathlineto{\pgfqpoint{0.781024in}{0.948926in}}%
\pgfpathlineto{\pgfqpoint{0.787536in}{0.944135in}}%
\pgfpathlineto{\pgfqpoint{0.796680in}{0.938037in}}%
\pgfpathlineto{\pgfqpoint{0.812337in}{0.930663in}}%
\pgfpathlineto{\pgfqpoint{0.812867in}{0.930524in}}%
\pgfpathlineto{\pgfqpoint{0.827993in}{0.926415in}}%
\pgfpathclose%
\pgfpathmoveto{\pgfqpoint{0.805998in}{0.957746in}}%
\pgfpathlineto{\pgfqpoint{0.796680in}{0.963882in}}%
\pgfpathlineto{\pgfqpoint{0.788509in}{0.971357in}}%
\pgfpathlineto{\pgfqpoint{0.781024in}{0.981274in}}%
\pgfpathlineto{\pgfqpoint{0.778667in}{0.984968in}}%
\pgfpathlineto{\pgfqpoint{0.774219in}{0.998579in}}%
\pgfpathlineto{\pgfqpoint{0.774219in}{1.012191in}}%
\pgfpathlineto{\pgfqpoint{0.778667in}{1.025802in}}%
\pgfpathlineto{\pgfqpoint{0.781024in}{1.029496in}}%
\pgfpathlineto{\pgfqpoint{0.788509in}{1.039413in}}%
\pgfpathlineto{\pgfqpoint{0.796680in}{1.046888in}}%
\pgfpathlineto{\pgfqpoint{0.805998in}{1.053024in}}%
\pgfpathlineto{\pgfqpoint{0.812337in}{1.056453in}}%
\pgfpathlineto{\pgfqpoint{0.827993in}{1.061176in}}%
\pgfpathlineto{\pgfqpoint{0.843650in}{1.061962in}}%
\pgfpathlineto{\pgfqpoint{0.859306in}{1.058816in}}%
\pgfpathlineto{\pgfqpoint{0.872168in}{1.053024in}}%
\pgfpathlineto{\pgfqpoint{0.874963in}{1.051502in}}%
\pgfpathlineto{\pgfqpoint{0.889677in}{1.039413in}}%
\pgfpathlineto{\pgfqpoint{0.890620in}{1.038298in}}%
\pgfpathlineto{\pgfqpoint{0.899101in}{1.025802in}}%
\pgfpathlineto{\pgfqpoint{0.903724in}{1.012191in}}%
\pgfpathlineto{\pgfqpoint{0.903724in}{0.998579in}}%
\pgfpathlineto{\pgfqpoint{0.899101in}{0.984968in}}%
\pgfpathlineto{\pgfqpoint{0.890620in}{0.972472in}}%
\pgfpathlineto{\pgfqpoint{0.889677in}{0.971357in}}%
\pgfpathlineto{\pgfqpoint{0.874963in}{0.959268in}}%
\pgfpathlineto{\pgfqpoint{0.872168in}{0.957746in}}%
\pgfpathlineto{\pgfqpoint{0.859306in}{0.951954in}}%
\pgfpathlineto{\pgfqpoint{0.843650in}{0.948808in}}%
\pgfpathlineto{\pgfqpoint{0.827993in}{0.949594in}}%
\pgfpathlineto{\pgfqpoint{0.812337in}{0.954317in}}%
\pgfpathlineto{\pgfqpoint{0.805998in}{0.957746in}}%
\pgfpathclose%
\pgfpathmoveto{\pgfqpoint{1.125468in}{0.929535in}}%
\pgfpathlineto{\pgfqpoint{1.141125in}{0.925990in}}%
\pgfpathlineto{\pgfqpoint{1.156781in}{0.925990in}}%
\pgfpathlineto{\pgfqpoint{1.172438in}{0.929535in}}%
\pgfpathlineto{\pgfqpoint{1.174684in}{0.930524in}}%
\pgfpathlineto{\pgfqpoint{1.188094in}{0.936294in}}%
\pgfpathlineto{\pgfqpoint{1.200467in}{0.944135in}}%
\pgfpathlineto{\pgfqpoint{1.203751in}{0.946424in}}%
\pgfpathlineto{\pgfqpoint{1.216774in}{0.957746in}}%
\pgfpathlineto{\pgfqpoint{1.219407in}{0.960601in}}%
\pgfpathlineto{\pgfqpoint{1.228427in}{0.971357in}}%
\pgfpathlineto{\pgfqpoint{1.235064in}{0.983015in}}%
\pgfpathlineto{\pgfqpoint{1.236201in}{0.984968in}}%
\pgfpathlineto{\pgfqpoint{1.240279in}{0.998579in}}%
\pgfpathlineto{\pgfqpoint{1.240279in}{1.012191in}}%
\pgfpathlineto{\pgfqpoint{1.236201in}{1.025802in}}%
\pgfpathlineto{\pgfqpoint{1.235064in}{1.027755in}}%
\pgfpathlineto{\pgfqpoint{1.228427in}{1.039413in}}%
\pgfpathlineto{\pgfqpoint{1.219407in}{1.050169in}}%
\pgfpathlineto{\pgfqpoint{1.216774in}{1.053024in}}%
\pgfpathlineto{\pgfqpoint{1.203751in}{1.064346in}}%
\pgfpathlineto{\pgfqpoint{1.200467in}{1.066635in}}%
\pgfpathlineto{\pgfqpoint{1.188094in}{1.074476in}}%
\pgfpathlineto{\pgfqpoint{1.174684in}{1.080246in}}%
\pgfpathlineto{\pgfqpoint{1.172438in}{1.081235in}}%
\pgfpathlineto{\pgfqpoint{1.156781in}{1.084780in}}%
\pgfpathlineto{\pgfqpoint{1.141125in}{1.084780in}}%
\pgfpathlineto{\pgfqpoint{1.125468in}{1.081235in}}%
\pgfpathlineto{\pgfqpoint{1.123221in}{1.080246in}}%
\pgfpathlineto{\pgfqpoint{1.109812in}{1.074476in}}%
\pgfpathlineto{\pgfqpoint{1.097439in}{1.066635in}}%
\pgfpathlineto{\pgfqpoint{1.094155in}{1.064346in}}%
\pgfpathlineto{\pgfqpoint{1.081132in}{1.053024in}}%
\pgfpathlineto{\pgfqpoint{1.078498in}{1.050169in}}%
\pgfpathlineto{\pgfqpoint{1.069479in}{1.039413in}}%
\pgfpathlineto{\pgfqpoint{1.062842in}{1.027755in}}%
\pgfpathlineto{\pgfqpoint{1.061705in}{1.025802in}}%
\pgfpathlineto{\pgfqpoint{1.057627in}{1.012191in}}%
\pgfpathlineto{\pgfqpoint{1.057627in}{0.998579in}}%
\pgfpathlineto{\pgfqpoint{1.061705in}{0.984968in}}%
\pgfpathlineto{\pgfqpoint{1.062842in}{0.983015in}}%
\pgfpathlineto{\pgfqpoint{1.069479in}{0.971357in}}%
\pgfpathlineto{\pgfqpoint{1.078498in}{0.960601in}}%
\pgfpathlineto{\pgfqpoint{1.081132in}{0.957746in}}%
\pgfpathlineto{\pgfqpoint{1.094155in}{0.946424in}}%
\pgfpathlineto{\pgfqpoint{1.097439in}{0.944135in}}%
\pgfpathlineto{\pgfqpoint{1.109812in}{0.936294in}}%
\pgfpathlineto{\pgfqpoint{1.123221in}{0.930524in}}%
\pgfpathlineto{\pgfqpoint{1.125468in}{0.929535in}}%
\pgfpathclose%
\pgfpathmoveto{\pgfqpoint{1.116019in}{0.957746in}}%
\pgfpathlineto{\pgfqpoint{1.109812in}{0.961483in}}%
\pgfpathlineto{\pgfqpoint{1.098453in}{0.971357in}}%
\pgfpathlineto{\pgfqpoint{1.094155in}{0.976753in}}%
\pgfpathlineto{\pgfqpoint{1.088760in}{0.984968in}}%
\pgfpathlineto{\pgfqpoint{1.084236in}{0.998579in}}%
\pgfpathlineto{\pgfqpoint{1.084236in}{1.012191in}}%
\pgfpathlineto{\pgfqpoint{1.088760in}{1.025802in}}%
\pgfpathlineto{\pgfqpoint{1.094155in}{1.034017in}}%
\pgfpathlineto{\pgfqpoint{1.098453in}{1.039413in}}%
\pgfpathlineto{\pgfqpoint{1.109812in}{1.049287in}}%
\pgfpathlineto{\pgfqpoint{1.116019in}{1.053024in}}%
\pgfpathlineto{\pgfqpoint{1.125468in}{1.057714in}}%
\pgfpathlineto{\pgfqpoint{1.141125in}{1.061647in}}%
\pgfpathlineto{\pgfqpoint{1.156781in}{1.061647in}}%
\pgfpathlineto{\pgfqpoint{1.172438in}{1.057714in}}%
\pgfpathlineto{\pgfqpoint{1.181887in}{1.053024in}}%
\pgfpathlineto{\pgfqpoint{1.188094in}{1.049287in}}%
\pgfpathlineto{\pgfqpoint{1.199453in}{1.039413in}}%
\pgfpathlineto{\pgfqpoint{1.203751in}{1.034017in}}%
\pgfpathlineto{\pgfqpoint{1.209145in}{1.025802in}}%
\pgfpathlineto{\pgfqpoint{1.213670in}{1.012191in}}%
\pgfpathlineto{\pgfqpoint{1.213670in}{0.998579in}}%
\pgfpathlineto{\pgfqpoint{1.209145in}{0.984968in}}%
\pgfpathlineto{\pgfqpoint{1.203751in}{0.976753in}}%
\pgfpathlineto{\pgfqpoint{1.199453in}{0.971357in}}%
\pgfpathlineto{\pgfqpoint{1.188094in}{0.961483in}}%
\pgfpathlineto{\pgfqpoint{1.181887in}{0.957746in}}%
\pgfpathlineto{\pgfqpoint{1.172438in}{0.953056in}}%
\pgfpathlineto{\pgfqpoint{1.156781in}{0.949123in}}%
\pgfpathlineto{\pgfqpoint{1.141125in}{0.949123in}}%
\pgfpathlineto{\pgfqpoint{1.125468in}{0.953056in}}%
\pgfpathlineto{\pgfqpoint{1.116019in}{0.957746in}}%
\pgfpathclose%
\pgfpathmoveto{\pgfqpoint{1.438599in}{0.928542in}}%
\pgfpathlineto{\pgfqpoint{1.454256in}{0.925707in}}%
\pgfpathlineto{\pgfqpoint{1.469913in}{0.926415in}}%
\pgfpathlineto{\pgfqpoint{1.485039in}{0.930524in}}%
\pgfpathlineto{\pgfqpoint{1.485569in}{0.930663in}}%
\pgfpathlineto{\pgfqpoint{1.501226in}{0.938037in}}%
\pgfpathlineto{\pgfqpoint{1.510370in}{0.944135in}}%
\pgfpathlineto{\pgfqpoint{1.516882in}{0.948926in}}%
\pgfpathlineto{\pgfqpoint{1.526737in}{0.957746in}}%
\pgfpathlineto{\pgfqpoint{1.532539in}{0.964233in}}%
\pgfpathlineto{\pgfqpoint{1.538451in}{0.971357in}}%
\pgfpathlineto{\pgfqpoint{1.546187in}{0.984968in}}%
\pgfpathlineto{\pgfqpoint{1.548195in}{0.991986in}}%
\pgfpathlineto{\pgfqpoint{1.550201in}{0.998579in}}%
\pgfpathlineto{\pgfqpoint{1.550201in}{1.012191in}}%
\pgfpathlineto{\pgfqpoint{1.548195in}{1.018784in}}%
\pgfpathlineto{\pgfqpoint{1.546187in}{1.025802in}}%
\pgfpathlineto{\pgfqpoint{1.538451in}{1.039413in}}%
\pgfpathlineto{\pgfqpoint{1.532539in}{1.046537in}}%
\pgfpathlineto{\pgfqpoint{1.526737in}{1.053024in}}%
\pgfpathlineto{\pgfqpoint{1.516882in}{1.061844in}}%
\pgfpathlineto{\pgfqpoint{1.510370in}{1.066635in}}%
\pgfpathlineto{\pgfqpoint{1.501226in}{1.072733in}}%
\pgfpathlineto{\pgfqpoint{1.485569in}{1.080107in}}%
\pgfpathlineto{\pgfqpoint{1.485039in}{1.080246in}}%
\pgfpathlineto{\pgfqpoint{1.469913in}{1.084355in}}%
\pgfpathlineto{\pgfqpoint{1.454256in}{1.085063in}}%
\pgfpathlineto{\pgfqpoint{1.438599in}{1.082228in}}%
\pgfpathlineto{\pgfqpoint{1.433636in}{1.080246in}}%
\pgfpathlineto{\pgfqpoint{1.422943in}{1.076085in}}%
\pgfpathlineto{\pgfqpoint{1.407286in}{1.066712in}}%
\pgfpathlineto{\pgfqpoint{1.407185in}{1.066635in}}%
\pgfpathlineto{\pgfqpoint{1.391630in}{1.053579in}}%
\pgfpathlineto{\pgfqpoint{1.391049in}{1.053024in}}%
\pgfpathlineto{\pgfqpoint{1.379502in}{1.039413in}}%
\pgfpathlineto{\pgfqpoint{1.375973in}{1.033307in}}%
\pgfpathlineto{\pgfqpoint{1.371628in}{1.025802in}}%
\pgfpathlineto{\pgfqpoint{1.367631in}{1.012191in}}%
\pgfpathlineto{\pgfqpoint{1.367631in}{0.998579in}}%
\pgfpathlineto{\pgfqpoint{1.371628in}{0.984968in}}%
\pgfpathlineto{\pgfqpoint{1.375973in}{0.977463in}}%
\pgfpathlineto{\pgfqpoint{1.379502in}{0.971357in}}%
\pgfpathlineto{\pgfqpoint{1.391049in}{0.957746in}}%
\pgfpathlineto{\pgfqpoint{1.391630in}{0.957191in}}%
\pgfpathlineto{\pgfqpoint{1.407185in}{0.944135in}}%
\pgfpathlineto{\pgfqpoint{1.407286in}{0.944058in}}%
\pgfpathlineto{\pgfqpoint{1.422943in}{0.934685in}}%
\pgfpathlineto{\pgfqpoint{1.433636in}{0.930524in}}%
\pgfpathlineto{\pgfqpoint{1.438599in}{0.928542in}}%
\pgfpathclose%
\pgfpathmoveto{\pgfqpoint{1.425738in}{0.957746in}}%
\pgfpathlineto{\pgfqpoint{1.422943in}{0.959268in}}%
\pgfpathlineto{\pgfqpoint{1.408229in}{0.971357in}}%
\pgfpathlineto{\pgfqpoint{1.407286in}{0.972472in}}%
\pgfpathlineto{\pgfqpoint{1.398805in}{0.984968in}}%
\pgfpathlineto{\pgfqpoint{1.394181in}{0.998579in}}%
\pgfpathlineto{\pgfqpoint{1.394181in}{1.012191in}}%
\pgfpathlineto{\pgfqpoint{1.398805in}{1.025802in}}%
\pgfpathlineto{\pgfqpoint{1.407286in}{1.038298in}}%
\pgfpathlineto{\pgfqpoint{1.408229in}{1.039413in}}%
\pgfpathlineto{\pgfqpoint{1.422943in}{1.051502in}}%
\pgfpathlineto{\pgfqpoint{1.425738in}{1.053024in}}%
\pgfpathlineto{\pgfqpoint{1.438599in}{1.058816in}}%
\pgfpathlineto{\pgfqpoint{1.454256in}{1.061962in}}%
\pgfpathlineto{\pgfqpoint{1.469913in}{1.061176in}}%
\pgfpathlineto{\pgfqpoint{1.485569in}{1.056453in}}%
\pgfpathlineto{\pgfqpoint{1.491907in}{1.053024in}}%
\pgfpathlineto{\pgfqpoint{1.501226in}{1.046888in}}%
\pgfpathlineto{\pgfqpoint{1.509397in}{1.039413in}}%
\pgfpathlineto{\pgfqpoint{1.516882in}{1.029496in}}%
\pgfpathlineto{\pgfqpoint{1.519239in}{1.025802in}}%
\pgfpathlineto{\pgfqpoint{1.523686in}{1.012191in}}%
\pgfpathlineto{\pgfqpoint{1.523686in}{0.998579in}}%
\pgfpathlineto{\pgfqpoint{1.519239in}{0.984968in}}%
\pgfpathlineto{\pgfqpoint{1.516882in}{0.981274in}}%
\pgfpathlineto{\pgfqpoint{1.509397in}{0.971357in}}%
\pgfpathlineto{\pgfqpoint{1.501226in}{0.963882in}}%
\pgfpathlineto{\pgfqpoint{1.491907in}{0.957746in}}%
\pgfpathlineto{\pgfqpoint{1.485569in}{0.954317in}}%
\pgfpathlineto{\pgfqpoint{1.469913in}{0.949594in}}%
\pgfpathlineto{\pgfqpoint{1.454256in}{0.948808in}}%
\pgfpathlineto{\pgfqpoint{1.438599in}{0.951954in}}%
\pgfpathlineto{\pgfqpoint{1.425738in}{0.957746in}}%
\pgfpathclose%
\pgfpathmoveto{\pgfqpoint{1.751731in}{0.927691in}}%
\pgfpathlineto{\pgfqpoint{1.767387in}{0.925565in}}%
\pgfpathlineto{\pgfqpoint{1.783044in}{0.926982in}}%
\pgfpathlineto{\pgfqpoint{1.794288in}{0.930524in}}%
\pgfpathlineto{\pgfqpoint{1.798700in}{0.931869in}}%
\pgfpathlineto{\pgfqpoint{1.814357in}{0.939912in}}%
\pgfpathlineto{\pgfqpoint{1.820404in}{0.944135in}}%
\pgfpathlineto{\pgfqpoint{1.830014in}{0.951557in}}%
\pgfpathlineto{\pgfqpoint{1.836757in}{0.957746in}}%
\pgfpathlineto{\pgfqpoint{1.845670in}{0.967985in}}%
\pgfpathlineto{\pgfqpoint{1.848450in}{0.971357in}}%
\pgfpathlineto{\pgfqpoint{1.856244in}{0.984968in}}%
\pgfpathlineto{\pgfqpoint{1.860134in}{0.998579in}}%
\pgfpathlineto{\pgfqpoint{1.860134in}{1.012191in}}%
\pgfpathlineto{\pgfqpoint{1.856244in}{1.025802in}}%
\pgfpathlineto{\pgfqpoint{1.848450in}{1.039413in}}%
\pgfpathlineto{\pgfqpoint{1.845670in}{1.042785in}}%
\pgfpathlineto{\pgfqpoint{1.836757in}{1.053024in}}%
\pgfpathlineto{\pgfqpoint{1.830014in}{1.059213in}}%
\pgfpathlineto{\pgfqpoint{1.820404in}{1.066635in}}%
\pgfpathlineto{\pgfqpoint{1.814357in}{1.070858in}}%
\pgfpathlineto{\pgfqpoint{1.798700in}{1.078901in}}%
\pgfpathlineto{\pgfqpoint{1.794288in}{1.080246in}}%
\pgfpathlineto{\pgfqpoint{1.783044in}{1.083788in}}%
\pgfpathlineto{\pgfqpoint{1.767387in}{1.085205in}}%
\pgfpathlineto{\pgfqpoint{1.751731in}{1.083079in}}%
\pgfpathlineto{\pgfqpoint{1.743808in}{1.080246in}}%
\pgfpathlineto{\pgfqpoint{1.736074in}{1.077560in}}%
\pgfpathlineto{\pgfqpoint{1.720418in}{1.068850in}}%
\pgfpathlineto{\pgfqpoint{1.717374in}{1.066635in}}%
\pgfpathlineto{\pgfqpoint{1.704761in}{1.056456in}}%
\pgfpathlineto{\pgfqpoint{1.701101in}{1.053024in}}%
\pgfpathlineto{\pgfqpoint{1.689500in}{1.039413in}}%
\pgfpathlineto{\pgfqpoint{1.689104in}{1.038741in}}%
\pgfpathlineto{\pgfqpoint{1.681623in}{1.025802in}}%
\pgfpathlineto{\pgfqpoint{1.677688in}{1.012191in}}%
\pgfpathlineto{\pgfqpoint{1.677688in}{0.998579in}}%
\pgfpathlineto{\pgfqpoint{1.681623in}{0.984968in}}%
\pgfpathlineto{\pgfqpoint{1.689104in}{0.972029in}}%
\pgfpathlineto{\pgfqpoint{1.689500in}{0.971357in}}%
\pgfpathlineto{\pgfqpoint{1.701101in}{0.957746in}}%
\pgfpathlineto{\pgfqpoint{1.704761in}{0.954314in}}%
\pgfpathlineto{\pgfqpoint{1.717374in}{0.944135in}}%
\pgfpathlineto{\pgfqpoint{1.720418in}{0.941920in}}%
\pgfpathlineto{\pgfqpoint{1.736074in}{0.933210in}}%
\pgfpathlineto{\pgfqpoint{1.743808in}{0.930524in}}%
\pgfpathlineto{\pgfqpoint{1.751731in}{0.927691in}}%
\pgfpathclose%
\pgfpathmoveto{\pgfqpoint{1.735381in}{0.957746in}}%
\pgfpathlineto{\pgfqpoint{1.720418in}{0.969227in}}%
\pgfpathlineto{\pgfqpoint{1.718284in}{0.971357in}}%
\pgfpathlineto{\pgfqpoint{1.708772in}{0.984968in}}%
\pgfpathlineto{\pgfqpoint{1.704761in}{0.996436in}}%
\pgfpathlineto{\pgfqpoint{1.704086in}{0.998579in}}%
\pgfpathlineto{\pgfqpoint{1.704086in}{1.012191in}}%
\pgfpathlineto{\pgfqpoint{1.704761in}{1.014334in}}%
\pgfpathlineto{\pgfqpoint{1.708772in}{1.025802in}}%
\pgfpathlineto{\pgfqpoint{1.718284in}{1.039413in}}%
\pgfpathlineto{\pgfqpoint{1.720418in}{1.041543in}}%
\pgfpathlineto{\pgfqpoint{1.735381in}{1.053024in}}%
\pgfpathlineto{\pgfqpoint{1.736074in}{1.053458in}}%
\pgfpathlineto{\pgfqpoint{1.751731in}{1.059760in}}%
\pgfpathlineto{\pgfqpoint{1.767387in}{1.062119in}}%
\pgfpathlineto{\pgfqpoint{1.783044in}{1.060547in}}%
\pgfpathlineto{\pgfqpoint{1.798700in}{1.055035in}}%
\pgfpathlineto{\pgfqpoint{1.802141in}{1.053024in}}%
\pgfpathlineto{\pgfqpoint{1.814357in}{1.044307in}}%
\pgfpathlineto{\pgfqpoint{1.819466in}{1.039413in}}%
\pgfpathlineto{\pgfqpoint{1.829279in}{1.025802in}}%
\pgfpathlineto{\pgfqpoint{1.830014in}{1.023790in}}%
\pgfpathlineto{\pgfqpoint{1.833745in}{1.012191in}}%
\pgfpathlineto{\pgfqpoint{1.833745in}{0.998579in}}%
\pgfpathlineto{\pgfqpoint{1.830014in}{0.986980in}}%
\pgfpathlineto{\pgfqpoint{1.829279in}{0.984968in}}%
\pgfpathlineto{\pgfqpoint{1.819466in}{0.971357in}}%
\pgfpathlineto{\pgfqpoint{1.814357in}{0.966463in}}%
\pgfpathlineto{\pgfqpoint{1.802141in}{0.957746in}}%
\pgfpathlineto{\pgfqpoint{1.798700in}{0.955735in}}%
\pgfpathlineto{\pgfqpoint{1.783044in}{0.950223in}}%
\pgfpathlineto{\pgfqpoint{1.767387in}{0.948651in}}%
\pgfpathlineto{\pgfqpoint{1.751731in}{0.951010in}}%
\pgfpathlineto{\pgfqpoint{1.736074in}{0.957312in}}%
\pgfpathlineto{\pgfqpoint{1.735381in}{0.957746in}}%
\pgfpathclose%
\pgfpathmoveto{\pgfqpoint{0.499205in}{1.201335in}}%
\pgfpathlineto{\pgfqpoint{0.514862in}{1.196466in}}%
\pgfpathlineto{\pgfqpoint{0.530519in}{1.195077in}}%
\pgfpathlineto{\pgfqpoint{0.546175in}{1.197160in}}%
\pgfpathlineto{\pgfqpoint{0.561832in}{1.202727in}}%
\pgfpathlineto{\pgfqpoint{0.561866in}{1.202746in}}%
\pgfpathlineto{\pgfqpoint{0.577488in}{1.211444in}}%
\pgfpathlineto{\pgfqpoint{0.584144in}{1.216357in}}%
\pgfpathlineto{\pgfqpoint{0.593145in}{1.223880in}}%
\pgfpathlineto{\pgfqpoint{0.599427in}{1.229968in}}%
\pgfpathlineto{\pgfqpoint{0.608801in}{1.241662in}}%
\pgfpathlineto{\pgfqpoint{0.610292in}{1.243579in}}%
\pgfpathlineto{\pgfqpoint{0.617386in}{1.257191in}}%
\pgfpathlineto{\pgfqpoint{0.620532in}{1.270802in}}%
\pgfpathlineto{\pgfqpoint{0.619746in}{1.284413in}}%
\pgfpathlineto{\pgfqpoint{0.615023in}{1.298024in}}%
\pgfpathlineto{\pgfqpoint{0.608801in}{1.307893in}}%
\pgfpathlineto{\pgfqpoint{0.606390in}{1.311635in}}%
\pgfpathlineto{\pgfqpoint{0.594037in}{1.325246in}}%
\pgfpathlineto{\pgfqpoint{0.593145in}{1.326059in}}%
\pgfpathlineto{\pgfqpoint{0.577488in}{1.338340in}}%
\pgfpathlineto{\pgfqpoint{0.576621in}{1.338857in}}%
\pgfpathlineto{\pgfqpoint{0.561832in}{1.347092in}}%
\pgfpathlineto{\pgfqpoint{0.546176in}{1.352468in}}%
\pgfpathlineto{\pgfqpoint{0.546175in}{1.352469in}}%
\pgfpathlineto{\pgfqpoint{0.530519in}{1.354647in}}%
\pgfpathlineto{\pgfqpoint{0.514862in}{1.353195in}}%
\pgfpathlineto{\pgfqpoint{0.512578in}{1.352468in}}%
\pgfpathlineto{\pgfqpoint{0.499205in}{1.348437in}}%
\pgfpathlineto{\pgfqpoint{0.483549in}{1.340366in}}%
\pgfpathlineto{\pgfqpoint{0.481365in}{1.338857in}}%
\pgfpathlineto{\pgfqpoint{0.467892in}{1.328769in}}%
\pgfpathlineto{\pgfqpoint{0.463941in}{1.325246in}}%
\pgfpathlineto{\pgfqpoint{0.452236in}{1.312504in}}%
\pgfpathlineto{\pgfqpoint{0.451482in}{1.311635in}}%
\pgfpathlineto{\pgfqpoint{0.442908in}{1.298024in}}%
\pgfpathlineto{\pgfqpoint{0.438239in}{1.284413in}}%
\pgfpathlineto{\pgfqpoint{0.437462in}{1.270802in}}%
\pgfpathlineto{\pgfqpoint{0.440572in}{1.257191in}}%
\pgfpathlineto{\pgfqpoint{0.447584in}{1.243579in}}%
\pgfpathlineto{\pgfqpoint{0.452236in}{1.237610in}}%
\pgfpathlineto{\pgfqpoint{0.458505in}{1.229968in}}%
\pgfpathlineto{\pgfqpoint{0.467892in}{1.221063in}}%
\pgfpathlineto{\pgfqpoint{0.473775in}{1.216357in}}%
\pgfpathlineto{\pgfqpoint{0.483549in}{1.209435in}}%
\pgfpathlineto{\pgfqpoint{0.496467in}{1.202746in}}%
\pgfpathlineto{\pgfqpoint{0.499205in}{1.201335in}}%
\pgfpathclose%
\pgfpathmoveto{\pgfqpoint{0.491532in}{1.229968in}}%
\pgfpathlineto{\pgfqpoint{0.483549in}{1.236013in}}%
\pgfpathlineto{\pgfqpoint{0.476084in}{1.243579in}}%
\pgfpathlineto{\pgfqpoint{0.467892in}{1.256178in}}%
\pgfpathlineto{\pgfqpoint{0.467321in}{1.257191in}}%
\pgfpathlineto{\pgfqpoint{0.463810in}{1.270802in}}%
\pgfpathlineto{\pgfqpoint{0.464687in}{1.284413in}}%
\pgfpathlineto{\pgfqpoint{0.467892in}{1.292760in}}%
\pgfpathlineto{\pgfqpoint{0.470196in}{1.298024in}}%
\pgfpathlineto{\pgfqpoint{0.480992in}{1.311635in}}%
\pgfpathlineto{\pgfqpoint{0.483549in}{1.313963in}}%
\pgfpathlineto{\pgfqpoint{0.499205in}{1.324579in}}%
\pgfpathlineto{\pgfqpoint{0.500856in}{1.325246in}}%
\pgfpathlineto{\pgfqpoint{0.514862in}{1.330080in}}%
\pgfpathlineto{\pgfqpoint{0.530519in}{1.331625in}}%
\pgfpathlineto{\pgfqpoint{0.546175in}{1.329307in}}%
\pgfpathlineto{\pgfqpoint{0.556540in}{1.325246in}}%
\pgfpathlineto{\pgfqpoint{0.561832in}{1.322809in}}%
\pgfpathlineto{\pgfqpoint{0.577069in}{1.311635in}}%
\pgfpathlineto{\pgfqpoint{0.577488in}{1.311210in}}%
\pgfpathlineto{\pgfqpoint{0.587613in}{1.298024in}}%
\pgfpathlineto{\pgfqpoint{0.593145in}{1.284820in}}%
\pgfpathlineto{\pgfqpoint{0.593298in}{1.284413in}}%
\pgfpathlineto{\pgfqpoint{0.594167in}{1.270802in}}%
\pgfpathlineto{\pgfqpoint{0.593145in}{1.266761in}}%
\pgfpathlineto{\pgfqpoint{0.590464in}{1.257191in}}%
\pgfpathlineto{\pgfqpoint{0.581906in}{1.243579in}}%
\pgfpathlineto{\pgfqpoint{0.577488in}{1.238913in}}%
\pgfpathlineto{\pgfqpoint{0.566503in}{1.229968in}}%
\pgfpathlineto{\pgfqpoint{0.561832in}{1.226944in}}%
\pgfpathlineto{\pgfqpoint{0.546175in}{1.220504in}}%
\pgfpathlineto{\pgfqpoint{0.530519in}{1.218094in}}%
\pgfpathlineto{\pgfqpoint{0.514862in}{1.219700in}}%
\pgfpathlineto{\pgfqpoint{0.499205in}{1.225333in}}%
\pgfpathlineto{\pgfqpoint{0.491532in}{1.229968in}}%
\pgfpathclose%
\pgfpathmoveto{\pgfqpoint{0.812337in}{1.200082in}}%
\pgfpathlineto{\pgfqpoint{0.827993in}{1.195910in}}%
\pgfpathlineto{\pgfqpoint{0.843650in}{1.195216in}}%
\pgfpathlineto{\pgfqpoint{0.859306in}{1.197995in}}%
\pgfpathlineto{\pgfqpoint{0.871272in}{1.202746in}}%
\pgfpathlineto{\pgfqpoint{0.874963in}{1.204204in}}%
\pgfpathlineto{\pgfqpoint{0.890620in}{1.213584in}}%
\pgfpathlineto{\pgfqpoint{0.894240in}{1.216357in}}%
\pgfpathlineto{\pgfqpoint{0.906276in}{1.226821in}}%
\pgfpathlineto{\pgfqpoint{0.909466in}{1.229968in}}%
\pgfpathlineto{\pgfqpoint{0.920256in}{1.243579in}}%
\pgfpathlineto{\pgfqpoint{0.921933in}{1.246788in}}%
\pgfpathlineto{\pgfqpoint{0.927398in}{1.257191in}}%
\pgfpathlineto{\pgfqpoint{0.930595in}{1.270802in}}%
\pgfpathlineto{\pgfqpoint{0.929796in}{1.284413in}}%
\pgfpathlineto{\pgfqpoint{0.924998in}{1.298024in}}%
\pgfpathlineto{\pgfqpoint{0.921933in}{1.302879in}}%
\pgfpathlineto{\pgfqpoint{0.916397in}{1.311635in}}%
\pgfpathlineto{\pgfqpoint{0.906276in}{1.322944in}}%
\pgfpathlineto{\pgfqpoint{0.903956in}{1.325246in}}%
\pgfpathlineto{\pgfqpoint{0.890620in}{1.336143in}}%
\pgfpathlineto{\pgfqpoint{0.886358in}{1.338857in}}%
\pgfpathlineto{\pgfqpoint{0.874963in}{1.345612in}}%
\pgfpathlineto{\pgfqpoint{0.859306in}{1.351663in}}%
\pgfpathlineto{\pgfqpoint{0.854656in}{1.352468in}}%
\pgfpathlineto{\pgfqpoint{0.843650in}{1.354502in}}%
\pgfpathlineto{\pgfqpoint{0.827993in}{1.353776in}}%
\pgfpathlineto{\pgfqpoint{0.823227in}{1.352468in}}%
\pgfpathlineto{\pgfqpoint{0.812337in}{1.349647in}}%
\pgfpathlineto{\pgfqpoint{0.796680in}{1.342248in}}%
\pgfpathlineto{\pgfqpoint{0.791542in}{1.338857in}}%
\pgfpathlineto{\pgfqpoint{0.781024in}{1.331355in}}%
\pgfpathlineto{\pgfqpoint{0.773997in}{1.325246in}}%
\pgfpathlineto{\pgfqpoint{0.765367in}{1.316102in}}%
\pgfpathlineto{\pgfqpoint{0.761466in}{1.311635in}}%
\pgfpathlineto{\pgfqpoint{0.752956in}{1.298024in}}%
\pgfpathlineto{\pgfqpoint{0.749710in}{1.288556in}}%
\pgfpathlineto{\pgfqpoint{0.748206in}{1.284413in}}%
\pgfpathlineto{\pgfqpoint{0.747371in}{1.270802in}}%
\pgfpathlineto{\pgfqpoint{0.749710in}{1.261233in}}%
\pgfpathlineto{\pgfqpoint{0.750637in}{1.257191in}}%
\pgfpathlineto{\pgfqpoint{0.757598in}{1.243579in}}%
\pgfpathlineto{\pgfqpoint{0.765367in}{1.233673in}}%
\pgfpathlineto{\pgfqpoint{0.768489in}{1.229968in}}%
\pgfpathlineto{\pgfqpoint{0.781024in}{1.218375in}}%
\pgfpathlineto{\pgfqpoint{0.783671in}{1.216357in}}%
\pgfpathlineto{\pgfqpoint{0.796680in}{1.207558in}}%
\pgfpathlineto{\pgfqpoint{0.806752in}{1.202746in}}%
\pgfpathlineto{\pgfqpoint{0.812337in}{1.200082in}}%
\pgfpathclose%
\pgfpathmoveto{\pgfqpoint{0.801455in}{1.229968in}}%
\pgfpathlineto{\pgfqpoint{0.796680in}{1.233304in}}%
\pgfpathlineto{\pgfqpoint{0.786066in}{1.243579in}}%
\pgfpathlineto{\pgfqpoint{0.781024in}{1.250966in}}%
\pgfpathlineto{\pgfqpoint{0.777421in}{1.257191in}}%
\pgfpathlineto{\pgfqpoint{0.773864in}{1.270802in}}%
\pgfpathlineto{\pgfqpoint{0.774753in}{1.284413in}}%
\pgfpathlineto{\pgfqpoint{0.780092in}{1.298024in}}%
\pgfpathlineto{\pgfqpoint{0.781024in}{1.299364in}}%
\pgfpathlineto{\pgfqpoint{0.791155in}{1.311635in}}%
\pgfpathlineto{\pgfqpoint{0.796680in}{1.316438in}}%
\pgfpathlineto{\pgfqpoint{0.810796in}{1.325246in}}%
\pgfpathlineto{\pgfqpoint{0.812337in}{1.326056in}}%
\pgfpathlineto{\pgfqpoint{0.827993in}{1.330698in}}%
\pgfpathlineto{\pgfqpoint{0.843650in}{1.331470in}}%
\pgfpathlineto{\pgfqpoint{0.859306in}{1.328378in}}%
\pgfpathlineto{\pgfqpoint{0.866466in}{1.325246in}}%
\pgfpathlineto{\pgfqpoint{0.874963in}{1.320863in}}%
\pgfpathlineto{\pgfqpoint{0.886782in}{1.311635in}}%
\pgfpathlineto{\pgfqpoint{0.890620in}{1.307484in}}%
\pgfpathlineto{\pgfqpoint{0.897619in}{1.298024in}}%
\pgfpathlineto{\pgfqpoint{0.903170in}{1.284413in}}%
\pgfpathlineto{\pgfqpoint{0.904094in}{1.270802in}}%
\pgfpathlineto{\pgfqpoint{0.900396in}{1.257191in}}%
\pgfpathlineto{\pgfqpoint{0.892060in}{1.243579in}}%
\pgfpathlineto{\pgfqpoint{0.890620in}{1.242000in}}%
\pgfpathlineto{\pgfqpoint{0.876779in}{1.229968in}}%
\pgfpathlineto{\pgfqpoint{0.874963in}{1.228716in}}%
\pgfpathlineto{\pgfqpoint{0.859306in}{1.221469in}}%
\pgfpathlineto{\pgfqpoint{0.843650in}{1.218254in}}%
\pgfpathlineto{\pgfqpoint{0.827993in}{1.219057in}}%
\pgfpathlineto{\pgfqpoint{0.812337in}{1.223883in}}%
\pgfpathlineto{\pgfqpoint{0.801455in}{1.229968in}}%
\pgfpathclose%
\pgfpathmoveto{\pgfqpoint{1.125468in}{1.198968in}}%
\pgfpathlineto{\pgfqpoint{1.141125in}{1.195494in}}%
\pgfpathlineto{\pgfqpoint{1.156781in}{1.195494in}}%
\pgfpathlineto{\pgfqpoint{1.172438in}{1.198968in}}%
\pgfpathlineto{\pgfqpoint{1.181071in}{1.202746in}}%
\pgfpathlineto{\pgfqpoint{1.188094in}{1.205814in}}%
\pgfpathlineto{\pgfqpoint{1.203751in}{1.215852in}}%
\pgfpathlineto{\pgfqpoint{1.204389in}{1.216357in}}%
\pgfpathlineto{\pgfqpoint{1.219407in}{1.229880in}}%
\pgfpathlineto{\pgfqpoint{1.219496in}{1.229968in}}%
\pgfpathlineto{\pgfqpoint{1.230278in}{1.243579in}}%
\pgfpathlineto{\pgfqpoint{1.235064in}{1.252875in}}%
\pgfpathlineto{\pgfqpoint{1.237344in}{1.257191in}}%
\pgfpathlineto{\pgfqpoint{1.240605in}{1.270802in}}%
\pgfpathlineto{\pgfqpoint{1.239790in}{1.284413in}}%
\pgfpathlineto{\pgfqpoint{1.235064in}{1.297563in}}%
\pgfpathlineto{\pgfqpoint{1.234904in}{1.298024in}}%
\pgfpathlineto{\pgfqpoint{1.226422in}{1.311635in}}%
\pgfpathlineto{\pgfqpoint{1.219407in}{1.319584in}}%
\pgfpathlineto{\pgfqpoint{1.213896in}{1.325246in}}%
\pgfpathlineto{\pgfqpoint{1.203751in}{1.333814in}}%
\pgfpathlineto{\pgfqpoint{1.196289in}{1.338857in}}%
\pgfpathlineto{\pgfqpoint{1.188094in}{1.343997in}}%
\pgfpathlineto{\pgfqpoint{1.172438in}{1.350722in}}%
\pgfpathlineto{\pgfqpoint{1.164365in}{1.352468in}}%
\pgfpathlineto{\pgfqpoint{1.156781in}{1.354212in}}%
\pgfpathlineto{\pgfqpoint{1.141125in}{1.354212in}}%
\pgfpathlineto{\pgfqpoint{1.133541in}{1.352468in}}%
\pgfpathlineto{\pgfqpoint{1.125468in}{1.350722in}}%
\pgfpathlineto{\pgfqpoint{1.109812in}{1.343997in}}%
\pgfpathlineto{\pgfqpoint{1.101617in}{1.338857in}}%
\pgfpathlineto{\pgfqpoint{1.094155in}{1.333814in}}%
\pgfpathlineto{\pgfqpoint{1.084009in}{1.325246in}}%
\pgfpathlineto{\pgfqpoint{1.078498in}{1.319584in}}%
\pgfpathlineto{\pgfqpoint{1.071484in}{1.311635in}}%
\pgfpathlineto{\pgfqpoint{1.063002in}{1.298024in}}%
\pgfpathlineto{\pgfqpoint{1.062842in}{1.297563in}}%
\pgfpathlineto{\pgfqpoint{1.058115in}{1.284413in}}%
\pgfpathlineto{\pgfqpoint{1.057301in}{1.270802in}}%
\pgfpathlineto{\pgfqpoint{1.060562in}{1.257191in}}%
\pgfpathlineto{\pgfqpoint{1.062842in}{1.252875in}}%
\pgfpathlineto{\pgfqpoint{1.067628in}{1.243579in}}%
\pgfpathlineto{\pgfqpoint{1.078410in}{1.229968in}}%
\pgfpathlineto{\pgfqpoint{1.078498in}{1.229880in}}%
\pgfpathlineto{\pgfqpoint{1.093517in}{1.216357in}}%
\pgfpathlineto{\pgfqpoint{1.094155in}{1.215852in}}%
\pgfpathlineto{\pgfqpoint{1.109812in}{1.205814in}}%
\pgfpathlineto{\pgfqpoint{1.116835in}{1.202746in}}%
\pgfpathlineto{\pgfqpoint{1.125468in}{1.198968in}}%
\pgfpathclose%
\pgfpathmoveto{\pgfqpoint{1.111094in}{1.229968in}}%
\pgfpathlineto{\pgfqpoint{1.109812in}{1.230788in}}%
\pgfpathlineto{\pgfqpoint{1.095905in}{1.243579in}}%
\pgfpathlineto{\pgfqpoint{1.094155in}{1.246010in}}%
\pgfpathlineto{\pgfqpoint{1.087492in}{1.257191in}}%
\pgfpathlineto{\pgfqpoint{1.083874in}{1.270802in}}%
\pgfpathlineto{\pgfqpoint{1.084778in}{1.284413in}}%
\pgfpathlineto{\pgfqpoint{1.090210in}{1.298024in}}%
\pgfpathlineto{\pgfqpoint{1.094155in}{1.303534in}}%
\pgfpathlineto{\pgfqpoint{1.101212in}{1.311635in}}%
\pgfpathlineto{\pgfqpoint{1.109812in}{1.318739in}}%
\pgfpathlineto{\pgfqpoint{1.121218in}{1.325246in}}%
\pgfpathlineto{\pgfqpoint{1.125468in}{1.327295in}}%
\pgfpathlineto{\pgfqpoint{1.141125in}{1.331161in}}%
\pgfpathlineto{\pgfqpoint{1.156781in}{1.331161in}}%
\pgfpathlineto{\pgfqpoint{1.172438in}{1.327295in}}%
\pgfpathlineto{\pgfqpoint{1.176687in}{1.325246in}}%
\pgfpathlineto{\pgfqpoint{1.188094in}{1.318739in}}%
\pgfpathlineto{\pgfqpoint{1.196693in}{1.311635in}}%
\pgfpathlineto{\pgfqpoint{1.203751in}{1.303534in}}%
\pgfpathlineto{\pgfqpoint{1.207696in}{1.298024in}}%
\pgfpathlineto{\pgfqpoint{1.213128in}{1.284413in}}%
\pgfpathlineto{\pgfqpoint{1.214032in}{1.270802in}}%
\pgfpathlineto{\pgfqpoint{1.210413in}{1.257191in}}%
\pgfpathlineto{\pgfqpoint{1.203751in}{1.246010in}}%
\pgfpathlineto{\pgfqpoint{1.202000in}{1.243579in}}%
\pgfpathlineto{\pgfqpoint{1.188094in}{1.230788in}}%
\pgfpathlineto{\pgfqpoint{1.186812in}{1.229968in}}%
\pgfpathlineto{\pgfqpoint{1.172438in}{1.222595in}}%
\pgfpathlineto{\pgfqpoint{1.156781in}{1.218576in}}%
\pgfpathlineto{\pgfqpoint{1.141125in}{1.218576in}}%
\pgfpathlineto{\pgfqpoint{1.125468in}{1.222595in}}%
\pgfpathlineto{\pgfqpoint{1.111094in}{1.229968in}}%
\pgfpathclose%
\pgfpathmoveto{\pgfqpoint{1.438599in}{1.197995in}}%
\pgfpathlineto{\pgfqpoint{1.454256in}{1.195216in}}%
\pgfpathlineto{\pgfqpoint{1.469913in}{1.195910in}}%
\pgfpathlineto{\pgfqpoint{1.485569in}{1.200082in}}%
\pgfpathlineto{\pgfqpoint{1.491154in}{1.202746in}}%
\pgfpathlineto{\pgfqpoint{1.501226in}{1.207558in}}%
\pgfpathlineto{\pgfqpoint{1.514235in}{1.216357in}}%
\pgfpathlineto{\pgfqpoint{1.516882in}{1.218375in}}%
\pgfpathlineto{\pgfqpoint{1.529417in}{1.229968in}}%
\pgfpathlineto{\pgfqpoint{1.532539in}{1.233673in}}%
\pgfpathlineto{\pgfqpoint{1.540308in}{1.243579in}}%
\pgfpathlineto{\pgfqpoint{1.547269in}{1.257191in}}%
\pgfpathlineto{\pgfqpoint{1.548195in}{1.261233in}}%
\pgfpathlineto{\pgfqpoint{1.550535in}{1.270802in}}%
\pgfpathlineto{\pgfqpoint{1.549700in}{1.284413in}}%
\pgfpathlineto{\pgfqpoint{1.548195in}{1.288556in}}%
\pgfpathlineto{\pgfqpoint{1.544950in}{1.298024in}}%
\pgfpathlineto{\pgfqpoint{1.536439in}{1.311635in}}%
\pgfpathlineto{\pgfqpoint{1.532539in}{1.316102in}}%
\pgfpathlineto{\pgfqpoint{1.523909in}{1.325246in}}%
\pgfpathlineto{\pgfqpoint{1.516882in}{1.331355in}}%
\pgfpathlineto{\pgfqpoint{1.506363in}{1.338857in}}%
\pgfpathlineto{\pgfqpoint{1.501226in}{1.342248in}}%
\pgfpathlineto{\pgfqpoint{1.485569in}{1.349647in}}%
\pgfpathlineto{\pgfqpoint{1.474679in}{1.352468in}}%
\pgfpathlineto{\pgfqpoint{1.469913in}{1.353776in}}%
\pgfpathlineto{\pgfqpoint{1.454256in}{1.354502in}}%
\pgfpathlineto{\pgfqpoint{1.443250in}{1.352468in}}%
\pgfpathlineto{\pgfqpoint{1.438599in}{1.351663in}}%
\pgfpathlineto{\pgfqpoint{1.422943in}{1.345612in}}%
\pgfpathlineto{\pgfqpoint{1.411548in}{1.338857in}}%
\pgfpathlineto{\pgfqpoint{1.407286in}{1.336143in}}%
\pgfpathlineto{\pgfqpoint{1.393950in}{1.325246in}}%
\pgfpathlineto{\pgfqpoint{1.391630in}{1.322944in}}%
\pgfpathlineto{\pgfqpoint{1.381508in}{1.311635in}}%
\pgfpathlineto{\pgfqpoint{1.375973in}{1.302879in}}%
\pgfpathlineto{\pgfqpoint{1.372908in}{1.298024in}}%
\pgfpathlineto{\pgfqpoint{1.368110in}{1.284413in}}%
\pgfpathlineto{\pgfqpoint{1.367311in}{1.270802in}}%
\pgfpathlineto{\pgfqpoint{1.370508in}{1.257191in}}%
\pgfpathlineto{\pgfqpoint{1.375973in}{1.246788in}}%
\pgfpathlineto{\pgfqpoint{1.377650in}{1.243579in}}%
\pgfpathlineto{\pgfqpoint{1.388439in}{1.229968in}}%
\pgfpathlineto{\pgfqpoint{1.391630in}{1.226821in}}%
\pgfpathlineto{\pgfqpoint{1.403666in}{1.216357in}}%
\pgfpathlineto{\pgfqpoint{1.407286in}{1.213584in}}%
\pgfpathlineto{\pgfqpoint{1.422943in}{1.204204in}}%
\pgfpathlineto{\pgfqpoint{1.426633in}{1.202746in}}%
\pgfpathlineto{\pgfqpoint{1.438599in}{1.197995in}}%
\pgfpathclose%
\pgfpathmoveto{\pgfqpoint{1.421126in}{1.229968in}}%
\pgfpathlineto{\pgfqpoint{1.407286in}{1.242000in}}%
\pgfpathlineto{\pgfqpoint{1.405846in}{1.243579in}}%
\pgfpathlineto{\pgfqpoint{1.397509in}{1.257191in}}%
\pgfpathlineto{\pgfqpoint{1.393812in}{1.270802in}}%
\pgfpathlineto{\pgfqpoint{1.394736in}{1.284413in}}%
\pgfpathlineto{\pgfqpoint{1.400286in}{1.298024in}}%
\pgfpathlineto{\pgfqpoint{1.407286in}{1.307484in}}%
\pgfpathlineto{\pgfqpoint{1.411124in}{1.311635in}}%
\pgfpathlineto{\pgfqpoint{1.422943in}{1.320863in}}%
\pgfpathlineto{\pgfqpoint{1.431439in}{1.325246in}}%
\pgfpathlineto{\pgfqpoint{1.438599in}{1.328378in}}%
\pgfpathlineto{\pgfqpoint{1.454256in}{1.331470in}}%
\pgfpathlineto{\pgfqpoint{1.469913in}{1.330698in}}%
\pgfpathlineto{\pgfqpoint{1.485569in}{1.326056in}}%
\pgfpathlineto{\pgfqpoint{1.487110in}{1.325246in}}%
\pgfpathlineto{\pgfqpoint{1.501226in}{1.316438in}}%
\pgfpathlineto{\pgfqpoint{1.506751in}{1.311635in}}%
\pgfpathlineto{\pgfqpoint{1.516882in}{1.299364in}}%
\pgfpathlineto{\pgfqpoint{1.517814in}{1.298024in}}%
\pgfpathlineto{\pgfqpoint{1.523153in}{1.284413in}}%
\pgfpathlineto{\pgfqpoint{1.524042in}{1.270802in}}%
\pgfpathlineto{\pgfqpoint{1.520485in}{1.257191in}}%
\pgfpathlineto{\pgfqpoint{1.516882in}{1.250966in}}%
\pgfpathlineto{\pgfqpoint{1.511840in}{1.243579in}}%
\pgfpathlineto{\pgfqpoint{1.501226in}{1.233304in}}%
\pgfpathlineto{\pgfqpoint{1.496451in}{1.229968in}}%
\pgfpathlineto{\pgfqpoint{1.485569in}{1.223883in}}%
\pgfpathlineto{\pgfqpoint{1.469913in}{1.219057in}}%
\pgfpathlineto{\pgfqpoint{1.454256in}{1.218254in}}%
\pgfpathlineto{\pgfqpoint{1.438599in}{1.221469in}}%
\pgfpathlineto{\pgfqpoint{1.422943in}{1.228716in}}%
\pgfpathlineto{\pgfqpoint{1.421126in}{1.229968in}}%
\pgfpathclose%
\pgfpathmoveto{\pgfqpoint{1.736074in}{1.202727in}}%
\pgfpathlineto{\pgfqpoint{1.751731in}{1.197160in}}%
\pgfpathlineto{\pgfqpoint{1.767387in}{1.195077in}}%
\pgfpathlineto{\pgfqpoint{1.783044in}{1.196466in}}%
\pgfpathlineto{\pgfqpoint{1.798700in}{1.201335in}}%
\pgfpathlineto{\pgfqpoint{1.801439in}{1.202746in}}%
\pgfpathlineto{\pgfqpoint{1.814357in}{1.209435in}}%
\pgfpathlineto{\pgfqpoint{1.824131in}{1.216357in}}%
\pgfpathlineto{\pgfqpoint{1.830014in}{1.221063in}}%
\pgfpathlineto{\pgfqpoint{1.839401in}{1.229968in}}%
\pgfpathlineto{\pgfqpoint{1.845670in}{1.237610in}}%
\pgfpathlineto{\pgfqpoint{1.850321in}{1.243579in}}%
\pgfpathlineto{\pgfqpoint{1.857334in}{1.257191in}}%
\pgfpathlineto{\pgfqpoint{1.860444in}{1.270802in}}%
\pgfpathlineto{\pgfqpoint{1.859667in}{1.284413in}}%
\pgfpathlineto{\pgfqpoint{1.854998in}{1.298024in}}%
\pgfpathlineto{\pgfqpoint{1.846424in}{1.311635in}}%
\pgfpathlineto{\pgfqpoint{1.845670in}{1.312504in}}%
\pgfpathlineto{\pgfqpoint{1.833964in}{1.325246in}}%
\pgfpathlineto{\pgfqpoint{1.830014in}{1.328769in}}%
\pgfpathlineto{\pgfqpoint{1.816540in}{1.338857in}}%
\pgfpathlineto{\pgfqpoint{1.814357in}{1.340366in}}%
\pgfpathlineto{\pgfqpoint{1.798700in}{1.348437in}}%
\pgfpathlineto{\pgfqpoint{1.785328in}{1.352468in}}%
\pgfpathlineto{\pgfqpoint{1.783044in}{1.353195in}}%
\pgfpathlineto{\pgfqpoint{1.767387in}{1.354647in}}%
\pgfpathlineto{\pgfqpoint{1.751731in}{1.352469in}}%
\pgfpathlineto{\pgfqpoint{1.751730in}{1.352468in}}%
\pgfpathlineto{\pgfqpoint{1.736074in}{1.347092in}}%
\pgfpathlineto{\pgfqpoint{1.721285in}{1.338857in}}%
\pgfpathlineto{\pgfqpoint{1.720418in}{1.338340in}}%
\pgfpathlineto{\pgfqpoint{1.704761in}{1.326059in}}%
\pgfpathlineto{\pgfqpoint{1.703869in}{1.325246in}}%
\pgfpathlineto{\pgfqpoint{1.691516in}{1.311635in}}%
\pgfpathlineto{\pgfqpoint{1.689104in}{1.307893in}}%
\pgfpathlineto{\pgfqpoint{1.682883in}{1.298024in}}%
\pgfpathlineto{\pgfqpoint{1.678160in}{1.284413in}}%
\pgfpathlineto{\pgfqpoint{1.677374in}{1.270802in}}%
\pgfpathlineto{\pgfqpoint{1.680520in}{1.257191in}}%
\pgfpathlineto{\pgfqpoint{1.687614in}{1.243579in}}%
\pgfpathlineto{\pgfqpoint{1.689104in}{1.241662in}}%
\pgfpathlineto{\pgfqpoint{1.698479in}{1.229968in}}%
\pgfpathlineto{\pgfqpoint{1.704761in}{1.223880in}}%
\pgfpathlineto{\pgfqpoint{1.713762in}{1.216357in}}%
\pgfpathlineto{\pgfqpoint{1.720418in}{1.211444in}}%
\pgfpathlineto{\pgfqpoint{1.736040in}{1.202746in}}%
\pgfpathlineto{\pgfqpoint{1.736074in}{1.202727in}}%
\pgfpathclose%
\pgfpathmoveto{\pgfqpoint{1.731403in}{1.229968in}}%
\pgfpathlineto{\pgfqpoint{1.720418in}{1.238913in}}%
\pgfpathlineto{\pgfqpoint{1.716000in}{1.243579in}}%
\pgfpathlineto{\pgfqpoint{1.707442in}{1.257191in}}%
\pgfpathlineto{\pgfqpoint{1.704761in}{1.266761in}}%
\pgfpathlineto{\pgfqpoint{1.703739in}{1.270802in}}%
\pgfpathlineto{\pgfqpoint{1.704608in}{1.284413in}}%
\pgfpathlineto{\pgfqpoint{1.704761in}{1.284820in}}%
\pgfpathlineto{\pgfqpoint{1.710293in}{1.298024in}}%
\pgfpathlineto{\pgfqpoint{1.720418in}{1.311210in}}%
\pgfpathlineto{\pgfqpoint{1.720837in}{1.311635in}}%
\pgfpathlineto{\pgfqpoint{1.736074in}{1.322809in}}%
\pgfpathlineto{\pgfqpoint{1.741366in}{1.325246in}}%
\pgfpathlineto{\pgfqpoint{1.751731in}{1.329307in}}%
\pgfpathlineto{\pgfqpoint{1.767387in}{1.331625in}}%
\pgfpathlineto{\pgfqpoint{1.783044in}{1.330080in}}%
\pgfpathlineto{\pgfqpoint{1.797050in}{1.325246in}}%
\pgfpathlineto{\pgfqpoint{1.798700in}{1.324579in}}%
\pgfpathlineto{\pgfqpoint{1.814357in}{1.313963in}}%
\pgfpathlineto{\pgfqpoint{1.816914in}{1.311635in}}%
\pgfpathlineto{\pgfqpoint{1.827710in}{1.298024in}}%
\pgfpathlineto{\pgfqpoint{1.830014in}{1.292760in}}%
\pgfpathlineto{\pgfqpoint{1.833219in}{1.284413in}}%
\pgfpathlineto{\pgfqpoint{1.834096in}{1.270802in}}%
\pgfpathlineto{\pgfqpoint{1.830585in}{1.257191in}}%
\pgfpathlineto{\pgfqpoint{1.830014in}{1.256178in}}%
\pgfpathlineto{\pgfqpoint{1.821821in}{1.243579in}}%
\pgfpathlineto{\pgfqpoint{1.814357in}{1.236013in}}%
\pgfpathlineto{\pgfqpoint{1.806374in}{1.229968in}}%
\pgfpathlineto{\pgfqpoint{1.798700in}{1.225333in}}%
\pgfpathlineto{\pgfqpoint{1.783044in}{1.219700in}}%
\pgfpathlineto{\pgfqpoint{1.767387in}{1.218094in}}%
\pgfpathlineto{\pgfqpoint{1.751731in}{1.220504in}}%
\pgfpathlineto{\pgfqpoint{1.736074in}{1.226944in}}%
\pgfpathlineto{\pgfqpoint{1.731403in}{1.229968in}}%
\pgfpathclose%
\pgfpathmoveto{\pgfqpoint{0.499205in}{1.470793in}}%
\pgfpathlineto{\pgfqpoint{0.514862in}{1.466000in}}%
\pgfpathlineto{\pgfqpoint{0.530519in}{1.464634in}}%
\pgfpathlineto{\pgfqpoint{0.546175in}{1.466684in}}%
\pgfpathlineto{\pgfqpoint{0.561832in}{1.472164in}}%
\pgfpathlineto{\pgfqpoint{0.566914in}{1.474968in}}%
\pgfpathlineto{\pgfqpoint{0.577488in}{1.480969in}}%
\pgfpathlineto{\pgfqpoint{0.587617in}{1.488579in}}%
\pgfpathlineto{\pgfqpoint{0.593145in}{1.493385in}}%
\pgfpathlineto{\pgfqpoint{0.601899in}{1.502191in}}%
\pgfpathlineto{\pgfqpoint{0.608801in}{1.511383in}}%
\pgfpathlineto{\pgfqpoint{0.612027in}{1.515802in}}%
\pgfpathlineto{\pgfqpoint{0.618330in}{1.529413in}}%
\pgfpathlineto{\pgfqpoint{0.620689in}{1.543024in}}%
\pgfpathlineto{\pgfqpoint{0.619117in}{1.556635in}}%
\pgfpathlineto{\pgfqpoint{0.613604in}{1.570246in}}%
\pgfpathlineto{\pgfqpoint{0.608801in}{1.577298in}}%
\pgfpathlineto{\pgfqpoint{0.604221in}{1.583857in}}%
\pgfpathlineto{\pgfqpoint{0.593145in}{1.595469in}}%
\pgfpathlineto{\pgfqpoint{0.590943in}{1.597468in}}%
\pgfpathlineto{\pgfqpoint{0.577488in}{1.607798in}}%
\pgfpathlineto{\pgfqpoint{0.571838in}{1.611079in}}%
\pgfpathlineto{\pgfqpoint{0.561832in}{1.616614in}}%
\pgfpathlineto{\pgfqpoint{0.546175in}{1.622031in}}%
\pgfpathlineto{\pgfqpoint{0.530519in}{1.624059in}}%
\pgfpathlineto{\pgfqpoint{0.514862in}{1.622707in}}%
\pgfpathlineto{\pgfqpoint{0.499205in}{1.617969in}}%
\pgfpathlineto{\pgfqpoint{0.485854in}{1.611079in}}%
\pgfpathlineto{\pgfqpoint{0.483549in}{1.609834in}}%
\pgfpathlineto{\pgfqpoint{0.467892in}{1.598350in}}%
\pgfpathlineto{\pgfqpoint{0.466878in}{1.597468in}}%
\pgfpathlineto{\pgfqpoint{0.453668in}{1.583857in}}%
\pgfpathlineto{\pgfqpoint{0.452236in}{1.581853in}}%
\pgfpathlineto{\pgfqpoint{0.444310in}{1.570246in}}%
\pgfpathlineto{\pgfqpoint{0.438860in}{1.556635in}}%
\pgfpathlineto{\pgfqpoint{0.437306in}{1.543024in}}%
\pgfpathlineto{\pgfqpoint{0.439638in}{1.529413in}}%
\pgfpathlineto{\pgfqpoint{0.445869in}{1.515802in}}%
\pgfpathlineto{\pgfqpoint{0.452236in}{1.507103in}}%
\pgfpathlineto{\pgfqpoint{0.456010in}{1.502191in}}%
\pgfpathlineto{\pgfqpoint{0.467892in}{1.490493in}}%
\pgfpathlineto{\pgfqpoint{0.470193in}{1.488579in}}%
\pgfpathlineto{\pgfqpoint{0.483549in}{1.478950in}}%
\pgfpathlineto{\pgfqpoint{0.491094in}{1.474968in}}%
\pgfpathlineto{\pgfqpoint{0.499205in}{1.470793in}}%
\pgfpathclose%
\pgfpathmoveto{\pgfqpoint{0.519724in}{1.488579in}}%
\pgfpathlineto{\pgfqpoint{0.514862in}{1.489094in}}%
\pgfpathlineto{\pgfqpoint{0.499205in}{1.494876in}}%
\pgfpathlineto{\pgfqpoint{0.487539in}{1.502191in}}%
\pgfpathlineto{\pgfqpoint{0.483549in}{1.505417in}}%
\pgfpathlineto{\pgfqpoint{0.473925in}{1.515802in}}%
\pgfpathlineto{\pgfqpoint{0.467892in}{1.526197in}}%
\pgfpathlineto{\pgfqpoint{0.466267in}{1.529413in}}%
\pgfpathlineto{\pgfqpoint{0.463635in}{1.543024in}}%
\pgfpathlineto{\pgfqpoint{0.465389in}{1.556635in}}%
\pgfpathlineto{\pgfqpoint{0.467892in}{1.562257in}}%
\pgfpathlineto{\pgfqpoint{0.471962in}{1.570246in}}%
\pgfpathlineto{\pgfqpoint{0.483549in}{1.583648in}}%
\pgfpathlineto{\pgfqpoint{0.483790in}{1.583857in}}%
\pgfpathlineto{\pgfqpoint{0.499205in}{1.593930in}}%
\pgfpathlineto{\pgfqpoint{0.508395in}{1.597468in}}%
\pgfpathlineto{\pgfqpoint{0.514862in}{1.599644in}}%
\pgfpathlineto{\pgfqpoint{0.530519in}{1.601170in}}%
\pgfpathlineto{\pgfqpoint{0.546175in}{1.598881in}}%
\pgfpathlineto{\pgfqpoint{0.549874in}{1.597468in}}%
\pgfpathlineto{\pgfqpoint{0.561832in}{1.592224in}}%
\pgfpathlineto{\pgfqpoint{0.573777in}{1.583857in}}%
\pgfpathlineto{\pgfqpoint{0.577488in}{1.580389in}}%
\pgfpathlineto{\pgfqpoint{0.585901in}{1.570246in}}%
\pgfpathlineto{\pgfqpoint{0.592553in}{1.556635in}}%
\pgfpathlineto{\pgfqpoint{0.593145in}{1.552409in}}%
\pgfpathlineto{\pgfqpoint{0.594341in}{1.543024in}}%
\pgfpathlineto{\pgfqpoint{0.593145in}{1.536746in}}%
\pgfpathlineto{\pgfqpoint{0.591604in}{1.529413in}}%
\pgfpathlineto{\pgfqpoint{0.583999in}{1.515802in}}%
\pgfpathlineto{\pgfqpoint{0.577488in}{1.508479in}}%
\pgfpathlineto{\pgfqpoint{0.570255in}{1.502191in}}%
\pgfpathlineto{\pgfqpoint{0.561832in}{1.496530in}}%
\pgfpathlineto{\pgfqpoint{0.546175in}{1.489919in}}%
\pgfpathlineto{\pgfqpoint{0.537740in}{1.488579in}}%
\pgfpathlineto{\pgfqpoint{0.530519in}{1.487539in}}%
\pgfpathlineto{\pgfqpoint{0.519724in}{1.488579in}}%
\pgfpathclose%
\pgfpathmoveto{\pgfqpoint{0.812337in}{1.469560in}}%
\pgfpathlineto{\pgfqpoint{0.827993in}{1.465454in}}%
\pgfpathlineto{\pgfqpoint{0.843650in}{1.464770in}}%
\pgfpathlineto{\pgfqpoint{0.859306in}{1.467506in}}%
\pgfpathlineto{\pgfqpoint{0.874963in}{1.473673in}}%
\pgfpathlineto{\pgfqpoint{0.877169in}{1.474968in}}%
\pgfpathlineto{\pgfqpoint{0.890620in}{1.483118in}}%
\pgfpathlineto{\pgfqpoint{0.897623in}{1.488579in}}%
\pgfpathlineto{\pgfqpoint{0.906276in}{1.496404in}}%
\pgfpathlineto{\pgfqpoint{0.911928in}{1.502191in}}%
\pgfpathlineto{\pgfqpoint{0.921933in}{1.515772in}}%
\pgfpathlineto{\pgfqpoint{0.921954in}{1.515802in}}%
\pgfpathlineto{\pgfqpoint{0.928358in}{1.529413in}}%
\pgfpathlineto{\pgfqpoint{0.930754in}{1.543024in}}%
\pgfpathlineto{\pgfqpoint{0.929157in}{1.556635in}}%
\pgfpathlineto{\pgfqpoint{0.923556in}{1.570246in}}%
\pgfpathlineto{\pgfqpoint{0.921933in}{1.572627in}}%
\pgfpathlineto{\pgfqpoint{0.914238in}{1.583857in}}%
\pgfpathlineto{\pgfqpoint{0.906276in}{1.592354in}}%
\pgfpathlineto{\pgfqpoint{0.900863in}{1.597468in}}%
\pgfpathlineto{\pgfqpoint{0.890620in}{1.605630in}}%
\pgfpathlineto{\pgfqpoint{0.881830in}{1.611079in}}%
\pgfpathlineto{\pgfqpoint{0.874963in}{1.615123in}}%
\pgfpathlineto{\pgfqpoint{0.859306in}{1.621220in}}%
\pgfpathlineto{\pgfqpoint{0.843650in}{1.623923in}}%
\pgfpathlineto{\pgfqpoint{0.827993in}{1.623248in}}%
\pgfpathlineto{\pgfqpoint{0.812337in}{1.619189in}}%
\pgfpathlineto{\pgfqpoint{0.796680in}{1.611735in}}%
\pgfpathlineto{\pgfqpoint{0.795681in}{1.611079in}}%
\pgfpathlineto{\pgfqpoint{0.781024in}{1.600903in}}%
\pgfpathlineto{\pgfqpoint{0.776972in}{1.597468in}}%
\pgfpathlineto{\pgfqpoint{0.765367in}{1.585755in}}%
\pgfpathlineto{\pgfqpoint{0.763631in}{1.583857in}}%
\pgfpathlineto{\pgfqpoint{0.754348in}{1.570246in}}%
\pgfpathlineto{\pgfqpoint{0.749710in}{1.558621in}}%
\pgfpathlineto{\pgfqpoint{0.748874in}{1.556635in}}%
\pgfpathlineto{\pgfqpoint{0.747204in}{1.543024in}}%
\pgfpathlineto{\pgfqpoint{0.749710in}{1.529413in}}%
\pgfpathlineto{\pgfqpoint{0.749710in}{1.529412in}}%
\pgfpathlineto{\pgfqpoint{0.755895in}{1.515802in}}%
\pgfpathlineto{\pgfqpoint{0.765367in}{1.502945in}}%
\pgfpathlineto{\pgfqpoint{0.765962in}{1.502191in}}%
\pgfpathlineto{\pgfqpoint{0.780089in}{1.488579in}}%
\pgfpathlineto{\pgfqpoint{0.781024in}{1.487804in}}%
\pgfpathlineto{\pgfqpoint{0.796680in}{1.477065in}}%
\pgfpathlineto{\pgfqpoint{0.800985in}{1.474968in}}%
\pgfpathlineto{\pgfqpoint{0.812337in}{1.469560in}}%
\pgfpathclose%
\pgfpathmoveto{\pgfqpoint{0.827525in}{1.488579in}}%
\pgfpathlineto{\pgfqpoint{0.812337in}{1.493388in}}%
\pgfpathlineto{\pgfqpoint{0.797169in}{1.502191in}}%
\pgfpathlineto{\pgfqpoint{0.796680in}{1.502555in}}%
\pgfpathlineto{\pgfqpoint{0.783826in}{1.515802in}}%
\pgfpathlineto{\pgfqpoint{0.781024in}{1.520402in}}%
\pgfpathlineto{\pgfqpoint{0.776353in}{1.529413in}}%
\pgfpathlineto{\pgfqpoint{0.773686in}{1.543024in}}%
\pgfpathlineto{\pgfqpoint{0.775464in}{1.556635in}}%
\pgfpathlineto{\pgfqpoint{0.781024in}{1.568811in}}%
\pgfpathlineto{\pgfqpoint{0.781791in}{1.570246in}}%
\pgfpathlineto{\pgfqpoint{0.794003in}{1.583857in}}%
\pgfpathlineto{\pgfqpoint{0.796680in}{1.586080in}}%
\pgfpathlineto{\pgfqpoint{0.812337in}{1.595465in}}%
\pgfpathlineto{\pgfqpoint{0.818391in}{1.597468in}}%
\pgfpathlineto{\pgfqpoint{0.827993in}{1.600255in}}%
\pgfpathlineto{\pgfqpoint{0.843650in}{1.601017in}}%
\pgfpathlineto{\pgfqpoint{0.859306in}{1.597965in}}%
\pgfpathlineto{\pgfqpoint{0.860471in}{1.597468in}}%
\pgfpathlineto{\pgfqpoint{0.874963in}{1.590346in}}%
\pgfpathlineto{\pgfqpoint{0.883666in}{1.583857in}}%
\pgfpathlineto{\pgfqpoint{0.890620in}{1.576917in}}%
\pgfpathlineto{\pgfqpoint{0.895952in}{1.570246in}}%
\pgfpathlineto{\pgfqpoint{0.902431in}{1.556635in}}%
\pgfpathlineto{\pgfqpoint{0.904279in}{1.543024in}}%
\pgfpathlineto{\pgfqpoint{0.901506in}{1.529413in}}%
\pgfpathlineto{\pgfqpoint{0.894099in}{1.515802in}}%
\pgfpathlineto{\pgfqpoint{0.890620in}{1.511741in}}%
\pgfpathlineto{\pgfqpoint{0.880331in}{1.502191in}}%
\pgfpathlineto{\pgfqpoint{0.874963in}{1.498350in}}%
\pgfpathlineto{\pgfqpoint{0.859306in}{1.490910in}}%
\pgfpathlineto{\pgfqpoint{0.848298in}{1.488579in}}%
\pgfpathlineto{\pgfqpoint{0.843650in}{1.487691in}}%
\pgfpathlineto{\pgfqpoint{0.827993in}{1.488446in}}%
\pgfpathlineto{\pgfqpoint{0.827525in}{1.488579in}}%
\pgfpathclose%
\pgfpathmoveto{\pgfqpoint{1.125468in}{1.468464in}}%
\pgfpathlineto{\pgfqpoint{1.141125in}{1.465043in}}%
\pgfpathlineto{\pgfqpoint{1.156781in}{1.465043in}}%
\pgfpathlineto{\pgfqpoint{1.172438in}{1.468464in}}%
\pgfpathlineto{\pgfqpoint{1.187322in}{1.474968in}}%
\pgfpathlineto{\pgfqpoint{1.188094in}{1.475312in}}%
\pgfpathlineto{\pgfqpoint{1.203751in}{1.485397in}}%
\pgfpathlineto{\pgfqpoint{1.207699in}{1.488579in}}%
\pgfpathlineto{\pgfqpoint{1.219407in}{1.499545in}}%
\pgfpathlineto{\pgfqpoint{1.221955in}{1.502191in}}%
\pgfpathlineto{\pgfqpoint{1.231975in}{1.515802in}}%
\pgfpathlineto{\pgfqpoint{1.235064in}{1.522525in}}%
\pgfpathlineto{\pgfqpoint{1.238323in}{1.529413in}}%
\pgfpathlineto{\pgfqpoint{1.240768in}{1.543024in}}%
\pgfpathlineto{\pgfqpoint{1.239138in}{1.556635in}}%
\pgfpathlineto{\pgfqpoint{1.235064in}{1.566410in}}%
\pgfpathlineto{\pgfqpoint{1.233517in}{1.570246in}}%
\pgfpathlineto{\pgfqpoint{1.224265in}{1.583857in}}%
\pgfpathlineto{\pgfqpoint{1.219407in}{1.589114in}}%
\pgfpathlineto{\pgfqpoint{1.210870in}{1.597468in}}%
\pgfpathlineto{\pgfqpoint{1.203751in}{1.603330in}}%
\pgfpathlineto{\pgfqpoint{1.191973in}{1.611079in}}%
\pgfpathlineto{\pgfqpoint{1.188094in}{1.613496in}}%
\pgfpathlineto{\pgfqpoint{1.172438in}{1.620272in}}%
\pgfpathlineto{\pgfqpoint{1.156781in}{1.623653in}}%
\pgfpathlineto{\pgfqpoint{1.141125in}{1.623653in}}%
\pgfpathlineto{\pgfqpoint{1.125468in}{1.620272in}}%
\pgfpathlineto{\pgfqpoint{1.109812in}{1.613496in}}%
\pgfpathlineto{\pgfqpoint{1.105932in}{1.611079in}}%
\pgfpathlineto{\pgfqpoint{1.094155in}{1.603330in}}%
\pgfpathlineto{\pgfqpoint{1.087036in}{1.597468in}}%
\pgfpathlineto{\pgfqpoint{1.078498in}{1.589114in}}%
\pgfpathlineto{\pgfqpoint{1.073641in}{1.583857in}}%
\pgfpathlineto{\pgfqpoint{1.064389in}{1.570246in}}%
\pgfpathlineto{\pgfqpoint{1.062842in}{1.566410in}}%
\pgfpathlineto{\pgfqpoint{1.058767in}{1.556635in}}%
\pgfpathlineto{\pgfqpoint{1.057138in}{1.543024in}}%
\pgfpathlineto{\pgfqpoint{1.059583in}{1.529413in}}%
\pgfpathlineto{\pgfqpoint{1.062842in}{1.522525in}}%
\pgfpathlineto{\pgfqpoint{1.065931in}{1.515802in}}%
\pgfpathlineto{\pgfqpoint{1.075951in}{1.502191in}}%
\pgfpathlineto{\pgfqpoint{1.078498in}{1.499545in}}%
\pgfpathlineto{\pgfqpoint{1.090207in}{1.488579in}}%
\pgfpathlineto{\pgfqpoint{1.094155in}{1.485397in}}%
\pgfpathlineto{\pgfqpoint{1.109812in}{1.475312in}}%
\pgfpathlineto{\pgfqpoint{1.110584in}{1.474968in}}%
\pgfpathlineto{\pgfqpoint{1.125468in}{1.468464in}}%
\pgfpathclose%
\pgfpathmoveto{\pgfqpoint{1.138659in}{1.488579in}}%
\pgfpathlineto{\pgfqpoint{1.125468in}{1.492066in}}%
\pgfpathlineto{\pgfqpoint{1.109812in}{1.500335in}}%
\pgfpathlineto{\pgfqpoint{1.107361in}{1.502191in}}%
\pgfpathlineto{\pgfqpoint{1.094155in}{1.515199in}}%
\pgfpathlineto{\pgfqpoint{1.093655in}{1.515802in}}%
\pgfpathlineto{\pgfqpoint{1.086406in}{1.529413in}}%
\pgfpathlineto{\pgfqpoint{1.083693in}{1.543024in}}%
\pgfpathlineto{\pgfqpoint{1.085502in}{1.556635in}}%
\pgfpathlineto{\pgfqpoint{1.091842in}{1.570246in}}%
\pgfpathlineto{\pgfqpoint{1.094155in}{1.573237in}}%
\pgfpathlineto{\pgfqpoint{1.104182in}{1.583857in}}%
\pgfpathlineto{\pgfqpoint{1.109812in}{1.588298in}}%
\pgfpathlineto{\pgfqpoint{1.125468in}{1.596829in}}%
\pgfpathlineto{\pgfqpoint{1.127782in}{1.597468in}}%
\pgfpathlineto{\pgfqpoint{1.141125in}{1.600712in}}%
\pgfpathlineto{\pgfqpoint{1.156781in}{1.600712in}}%
\pgfpathlineto{\pgfqpoint{1.170124in}{1.597468in}}%
\pgfpathlineto{\pgfqpoint{1.172438in}{1.596829in}}%
\pgfpathlineto{\pgfqpoint{1.188094in}{1.588298in}}%
\pgfpathlineto{\pgfqpoint{1.193724in}{1.583857in}}%
\pgfpathlineto{\pgfqpoint{1.203751in}{1.573237in}}%
\pgfpathlineto{\pgfqpoint{1.206064in}{1.570246in}}%
\pgfpathlineto{\pgfqpoint{1.212404in}{1.556635in}}%
\pgfpathlineto{\pgfqpoint{1.214212in}{1.543024in}}%
\pgfpathlineto{\pgfqpoint{1.211500in}{1.529413in}}%
\pgfpathlineto{\pgfqpoint{1.204251in}{1.515802in}}%
\pgfpathlineto{\pgfqpoint{1.203751in}{1.515199in}}%
\pgfpathlineto{\pgfqpoint{1.190545in}{1.502191in}}%
\pgfpathlineto{\pgfqpoint{1.188094in}{1.500335in}}%
\pgfpathlineto{\pgfqpoint{1.172438in}{1.492066in}}%
\pgfpathlineto{\pgfqpoint{1.159247in}{1.488579in}}%
\pgfpathlineto{\pgfqpoint{1.156781in}{1.487993in}}%
\pgfpathlineto{\pgfqpoint{1.141125in}{1.487993in}}%
\pgfpathlineto{\pgfqpoint{1.138659in}{1.488579in}}%
\pgfpathclose%
\pgfpathmoveto{\pgfqpoint{1.422943in}{1.473673in}}%
\pgfpathlineto{\pgfqpoint{1.438599in}{1.467506in}}%
\pgfpathlineto{\pgfqpoint{1.454256in}{1.464770in}}%
\pgfpathlineto{\pgfqpoint{1.469913in}{1.465454in}}%
\pgfpathlineto{\pgfqpoint{1.485569in}{1.469560in}}%
\pgfpathlineto{\pgfqpoint{1.496921in}{1.474968in}}%
\pgfpathlineto{\pgfqpoint{1.501226in}{1.477065in}}%
\pgfpathlineto{\pgfqpoint{1.516882in}{1.487804in}}%
\pgfpathlineto{\pgfqpoint{1.517817in}{1.488579in}}%
\pgfpathlineto{\pgfqpoint{1.531944in}{1.502191in}}%
\pgfpathlineto{\pgfqpoint{1.532539in}{1.502945in}}%
\pgfpathlineto{\pgfqpoint{1.542011in}{1.515802in}}%
\pgfpathlineto{\pgfqpoint{1.548195in}{1.529412in}}%
\pgfpathlineto{\pgfqpoint{1.548196in}{1.529413in}}%
\pgfpathlineto{\pgfqpoint{1.550702in}{1.543024in}}%
\pgfpathlineto{\pgfqpoint{1.549031in}{1.556635in}}%
\pgfpathlineto{\pgfqpoint{1.548195in}{1.558621in}}%
\pgfpathlineto{\pgfqpoint{1.543558in}{1.570246in}}%
\pgfpathlineto{\pgfqpoint{1.534275in}{1.583857in}}%
\pgfpathlineto{\pgfqpoint{1.532539in}{1.585755in}}%
\pgfpathlineto{\pgfqpoint{1.520934in}{1.597468in}}%
\pgfpathlineto{\pgfqpoint{1.516882in}{1.600903in}}%
\pgfpathlineto{\pgfqpoint{1.502225in}{1.611079in}}%
\pgfpathlineto{\pgfqpoint{1.501226in}{1.611735in}}%
\pgfpathlineto{\pgfqpoint{1.485569in}{1.619189in}}%
\pgfpathlineto{\pgfqpoint{1.469913in}{1.623248in}}%
\pgfpathlineto{\pgfqpoint{1.454256in}{1.623923in}}%
\pgfpathlineto{\pgfqpoint{1.438599in}{1.621220in}}%
\pgfpathlineto{\pgfqpoint{1.422943in}{1.615123in}}%
\pgfpathlineto{\pgfqpoint{1.416076in}{1.611079in}}%
\pgfpathlineto{\pgfqpoint{1.407286in}{1.605630in}}%
\pgfpathlineto{\pgfqpoint{1.397043in}{1.597468in}}%
\pgfpathlineto{\pgfqpoint{1.391630in}{1.592354in}}%
\pgfpathlineto{\pgfqpoint{1.383667in}{1.583857in}}%
\pgfpathlineto{\pgfqpoint{1.375973in}{1.572627in}}%
\pgfpathlineto{\pgfqpoint{1.374350in}{1.570246in}}%
\pgfpathlineto{\pgfqpoint{1.368749in}{1.556635in}}%
\pgfpathlineto{\pgfqpoint{1.367152in}{1.543024in}}%
\pgfpathlineto{\pgfqpoint{1.369548in}{1.529413in}}%
\pgfpathlineto{\pgfqpoint{1.375952in}{1.515802in}}%
\pgfpathlineto{\pgfqpoint{1.375973in}{1.515772in}}%
\pgfpathlineto{\pgfqpoint{1.385978in}{1.502191in}}%
\pgfpathlineto{\pgfqpoint{1.391630in}{1.496404in}}%
\pgfpathlineto{\pgfqpoint{1.400283in}{1.488579in}}%
\pgfpathlineto{\pgfqpoint{1.407286in}{1.483118in}}%
\pgfpathlineto{\pgfqpoint{1.420737in}{1.474968in}}%
\pgfpathlineto{\pgfqpoint{1.422943in}{1.473673in}}%
\pgfpathclose%
\pgfpathmoveto{\pgfqpoint{1.449608in}{1.488579in}}%
\pgfpathlineto{\pgfqpoint{1.438599in}{1.490910in}}%
\pgfpathlineto{\pgfqpoint{1.422943in}{1.498350in}}%
\pgfpathlineto{\pgfqpoint{1.417575in}{1.502191in}}%
\pgfpathlineto{\pgfqpoint{1.407286in}{1.511741in}}%
\pgfpathlineto{\pgfqpoint{1.403807in}{1.515802in}}%
\pgfpathlineto{\pgfqpoint{1.396399in}{1.529413in}}%
\pgfpathlineto{\pgfqpoint{1.393627in}{1.543024in}}%
\pgfpathlineto{\pgfqpoint{1.395475in}{1.556635in}}%
\pgfpathlineto{\pgfqpoint{1.401954in}{1.570246in}}%
\pgfpathlineto{\pgfqpoint{1.407286in}{1.576917in}}%
\pgfpathlineto{\pgfqpoint{1.414240in}{1.583857in}}%
\pgfpathlineto{\pgfqpoint{1.422943in}{1.590346in}}%
\pgfpathlineto{\pgfqpoint{1.437435in}{1.597468in}}%
\pgfpathlineto{\pgfqpoint{1.438599in}{1.597965in}}%
\pgfpathlineto{\pgfqpoint{1.454256in}{1.601017in}}%
\pgfpathlineto{\pgfqpoint{1.469913in}{1.600255in}}%
\pgfpathlineto{\pgfqpoint{1.479515in}{1.597468in}}%
\pgfpathlineto{\pgfqpoint{1.485569in}{1.595465in}}%
\pgfpathlineto{\pgfqpoint{1.501226in}{1.586080in}}%
\pgfpathlineto{\pgfqpoint{1.503903in}{1.583857in}}%
\pgfpathlineto{\pgfqpoint{1.516115in}{1.570246in}}%
\pgfpathlineto{\pgfqpoint{1.516882in}{1.568811in}}%
\pgfpathlineto{\pgfqpoint{1.522442in}{1.556635in}}%
\pgfpathlineto{\pgfqpoint{1.524219in}{1.543024in}}%
\pgfpathlineto{\pgfqpoint{1.521553in}{1.529413in}}%
\pgfpathlineto{\pgfqpoint{1.516882in}{1.520402in}}%
\pgfpathlineto{\pgfqpoint{1.514079in}{1.515802in}}%
\pgfpathlineto{\pgfqpoint{1.501226in}{1.502555in}}%
\pgfpathlineto{\pgfqpoint{1.500737in}{1.502191in}}%
\pgfpathlineto{\pgfqpoint{1.485569in}{1.493388in}}%
\pgfpathlineto{\pgfqpoint{1.470381in}{1.488579in}}%
\pgfpathlineto{\pgfqpoint{1.469913in}{1.488446in}}%
\pgfpathlineto{\pgfqpoint{1.454256in}{1.487691in}}%
\pgfpathlineto{\pgfqpoint{1.449608in}{1.488579in}}%
\pgfpathclose%
\pgfpathmoveto{\pgfqpoint{1.736074in}{1.472164in}}%
\pgfpathlineto{\pgfqpoint{1.751731in}{1.466684in}}%
\pgfpathlineto{\pgfqpoint{1.767387in}{1.464634in}}%
\pgfpathlineto{\pgfqpoint{1.783044in}{1.466000in}}%
\pgfpathlineto{\pgfqpoint{1.798700in}{1.470793in}}%
\pgfpathlineto{\pgfqpoint{1.806812in}{1.474968in}}%
\pgfpathlineto{\pgfqpoint{1.814357in}{1.478950in}}%
\pgfpathlineto{\pgfqpoint{1.827713in}{1.488579in}}%
\pgfpathlineto{\pgfqpoint{1.830014in}{1.490493in}}%
\pgfpathlineto{\pgfqpoint{1.841896in}{1.502191in}}%
\pgfpathlineto{\pgfqpoint{1.845670in}{1.507103in}}%
\pgfpathlineto{\pgfqpoint{1.852037in}{1.515802in}}%
\pgfpathlineto{\pgfqpoint{1.858268in}{1.529413in}}%
\pgfpathlineto{\pgfqpoint{1.860600in}{1.543024in}}%
\pgfpathlineto{\pgfqpoint{1.859045in}{1.556635in}}%
\pgfpathlineto{\pgfqpoint{1.853595in}{1.570246in}}%
\pgfpathlineto{\pgfqpoint{1.845670in}{1.581853in}}%
\pgfpathlineto{\pgfqpoint{1.844238in}{1.583857in}}%
\pgfpathlineto{\pgfqpoint{1.831028in}{1.597468in}}%
\pgfpathlineto{\pgfqpoint{1.830014in}{1.598350in}}%
\pgfpathlineto{\pgfqpoint{1.814357in}{1.609834in}}%
\pgfpathlineto{\pgfqpoint{1.812052in}{1.611079in}}%
\pgfpathlineto{\pgfqpoint{1.798700in}{1.617969in}}%
\pgfpathlineto{\pgfqpoint{1.783044in}{1.622707in}}%
\pgfpathlineto{\pgfqpoint{1.767387in}{1.624059in}}%
\pgfpathlineto{\pgfqpoint{1.751731in}{1.622031in}}%
\pgfpathlineto{\pgfqpoint{1.736074in}{1.616614in}}%
\pgfpathlineto{\pgfqpoint{1.726068in}{1.611079in}}%
\pgfpathlineto{\pgfqpoint{1.720418in}{1.607798in}}%
\pgfpathlineto{\pgfqpoint{1.706963in}{1.597468in}}%
\pgfpathlineto{\pgfqpoint{1.704761in}{1.595469in}}%
\pgfpathlineto{\pgfqpoint{1.693685in}{1.583857in}}%
\pgfpathlineto{\pgfqpoint{1.689104in}{1.577298in}}%
\pgfpathlineto{\pgfqpoint{1.684302in}{1.570246in}}%
\pgfpathlineto{\pgfqpoint{1.678789in}{1.556635in}}%
\pgfpathlineto{\pgfqpoint{1.677217in}{1.543024in}}%
\pgfpathlineto{\pgfqpoint{1.679576in}{1.529413in}}%
\pgfpathlineto{\pgfqpoint{1.685879in}{1.515802in}}%
\pgfpathlineto{\pgfqpoint{1.689104in}{1.511383in}}%
\pgfpathlineto{\pgfqpoint{1.696007in}{1.502191in}}%
\pgfpathlineto{\pgfqpoint{1.704761in}{1.493385in}}%
\pgfpathlineto{\pgfqpoint{1.710289in}{1.488579in}}%
\pgfpathlineto{\pgfqpoint{1.720418in}{1.480969in}}%
\pgfpathlineto{\pgfqpoint{1.730991in}{1.474968in}}%
\pgfpathlineto{\pgfqpoint{1.736074in}{1.472164in}}%
\pgfpathclose%
\pgfpathmoveto{\pgfqpoint{1.760166in}{1.488579in}}%
\pgfpathlineto{\pgfqpoint{1.751731in}{1.489919in}}%
\pgfpathlineto{\pgfqpoint{1.736074in}{1.496530in}}%
\pgfpathlineto{\pgfqpoint{1.727651in}{1.502191in}}%
\pgfpathlineto{\pgfqpoint{1.720418in}{1.508479in}}%
\pgfpathlineto{\pgfqpoint{1.713907in}{1.515802in}}%
\pgfpathlineto{\pgfqpoint{1.706302in}{1.529413in}}%
\pgfpathlineto{\pgfqpoint{1.704761in}{1.536746in}}%
\pgfpathlineto{\pgfqpoint{1.703565in}{1.543024in}}%
\pgfpathlineto{\pgfqpoint{1.704761in}{1.552409in}}%
\pgfpathlineto{\pgfqpoint{1.705353in}{1.556635in}}%
\pgfpathlineto{\pgfqpoint{1.712004in}{1.570246in}}%
\pgfpathlineto{\pgfqpoint{1.720418in}{1.580389in}}%
\pgfpathlineto{\pgfqpoint{1.724128in}{1.583857in}}%
\pgfpathlineto{\pgfqpoint{1.736074in}{1.592224in}}%
\pgfpathlineto{\pgfqpoint{1.748032in}{1.597468in}}%
\pgfpathlineto{\pgfqpoint{1.751731in}{1.598881in}}%
\pgfpathlineto{\pgfqpoint{1.767387in}{1.601170in}}%
\pgfpathlineto{\pgfqpoint{1.783044in}{1.599644in}}%
\pgfpathlineto{\pgfqpoint{1.789510in}{1.597468in}}%
\pgfpathlineto{\pgfqpoint{1.798700in}{1.593930in}}%
\pgfpathlineto{\pgfqpoint{1.814116in}{1.583857in}}%
\pgfpathlineto{\pgfqpoint{1.814357in}{1.583648in}}%
\pgfpathlineto{\pgfqpoint{1.825944in}{1.570246in}}%
\pgfpathlineto{\pgfqpoint{1.830014in}{1.562257in}}%
\pgfpathlineto{\pgfqpoint{1.832517in}{1.556635in}}%
\pgfpathlineto{\pgfqpoint{1.834271in}{1.543024in}}%
\pgfpathlineto{\pgfqpoint{1.831639in}{1.529413in}}%
\pgfpathlineto{\pgfqpoint{1.830014in}{1.526197in}}%
\pgfpathlineto{\pgfqpoint{1.823981in}{1.515802in}}%
\pgfpathlineto{\pgfqpoint{1.814357in}{1.505417in}}%
\pgfpathlineto{\pgfqpoint{1.810367in}{1.502191in}}%
\pgfpathlineto{\pgfqpoint{1.798700in}{1.494876in}}%
\pgfpathlineto{\pgfqpoint{1.783044in}{1.489094in}}%
\pgfpathlineto{\pgfqpoint{1.778182in}{1.488579in}}%
\pgfpathlineto{\pgfqpoint{1.767387in}{1.487539in}}%
\pgfpathlineto{\pgfqpoint{1.760166in}{1.488579in}}%
\pgfpathclose%
\pgfusepath{fill}%
\end{pgfscope}%
\begin{pgfscope}%
\pgfpathrectangle{\pgfqpoint{0.373953in}{0.331635in}}{\pgfqpoint{1.550000in}{1.347500in}}%
\pgfusepath{clip}%
\pgfsetbuttcap%
\pgfsetroundjoin%
\definecolor{currentfill}{rgb}{0.709962,0.212797,0.477201}%
\pgfsetfillcolor{currentfill}%
\pgfsetlinewidth{0.000000pt}%
\definecolor{currentstroke}{rgb}{0.000000,0.000000,0.000000}%
\pgfsetstrokecolor{currentstroke}%
\pgfsetdash{}{0pt}%
\pgfpathmoveto{\pgfqpoint{0.530519in}{0.358525in}}%
\pgfpathlineto{\pgfqpoint{0.531883in}{0.358857in}}%
\pgfpathlineto{\pgfqpoint{0.546175in}{0.361115in}}%
\pgfpathlineto{\pgfqpoint{0.561832in}{0.367719in}}%
\pgfpathlineto{\pgfqpoint{0.568936in}{0.372468in}}%
\pgfpathlineto{\pgfqpoint{0.577488in}{0.377221in}}%
\pgfpathlineto{\pgfqpoint{0.589511in}{0.386079in}}%
\pgfpathlineto{\pgfqpoint{0.593145in}{0.388646in}}%
\pgfpathlineto{\pgfqpoint{0.606485in}{0.399691in}}%
\pgfpathlineto{\pgfqpoint{0.608801in}{0.401751in}}%
\pgfpathlineto{\pgfqpoint{0.621153in}{0.413302in}}%
\pgfpathlineto{\pgfqpoint{0.624458in}{0.417008in}}%
\pgfpathlineto{\pgfqpoint{0.633909in}{0.426913in}}%
\pgfpathlineto{\pgfqpoint{0.640115in}{0.435796in}}%
\pgfpathlineto{\pgfqpoint{0.644233in}{0.440524in}}%
\pgfpathlineto{\pgfqpoint{0.651335in}{0.454135in}}%
\pgfpathlineto{\pgfqpoint{0.653361in}{0.467746in}}%
\pgfpathlineto{\pgfqpoint{0.650322in}{0.481357in}}%
\pgfpathlineto{\pgfqpoint{0.642201in}{0.494968in}}%
\pgfpathlineto{\pgfqpoint{0.640115in}{0.497203in}}%
\pgfpathlineto{\pgfqpoint{0.631570in}{0.508579in}}%
\pgfpathlineto{\pgfqpoint{0.624458in}{0.515731in}}%
\pgfpathlineto{\pgfqpoint{0.618438in}{0.522191in}}%
\pgfpathlineto{\pgfqpoint{0.608801in}{0.531011in}}%
\pgfpathlineto{\pgfqpoint{0.603291in}{0.535802in}}%
\pgfpathlineto{\pgfqpoint{0.593145in}{0.544180in}}%
\pgfpathlineto{\pgfqpoint{0.585714in}{0.549413in}}%
\pgfpathlineto{\pgfqpoint{0.577488in}{0.555596in}}%
\pgfpathlineto{\pgfqpoint{0.564402in}{0.563024in}}%
\pgfpathlineto{\pgfqpoint{0.561832in}{0.564838in}}%
\pgfpathlineto{\pgfqpoint{0.546175in}{0.571898in}}%
\pgfpathlineto{\pgfqpoint{0.530519in}{0.574540in}}%
\pgfpathlineto{\pgfqpoint{0.514862in}{0.572779in}}%
\pgfpathlineto{\pgfqpoint{0.499205in}{0.566604in}}%
\pgfpathlineto{\pgfqpoint{0.493767in}{0.563024in}}%
\pgfpathlineto{\pgfqpoint{0.483549in}{0.557629in}}%
\pgfpathlineto{\pgfqpoint{0.472156in}{0.549413in}}%
\pgfpathlineto{\pgfqpoint{0.467892in}{0.546539in}}%
\pgfpathlineto{\pgfqpoint{0.454606in}{0.535802in}}%
\pgfpathlineto{\pgfqpoint{0.452236in}{0.533788in}}%
\pgfpathlineto{\pgfqpoint{0.439531in}{0.522191in}}%
\pgfpathlineto{\pgfqpoint{0.436579in}{0.519032in}}%
\pgfpathlineto{\pgfqpoint{0.426389in}{0.508579in}}%
\pgfpathlineto{\pgfqpoint{0.420923in}{0.501144in}}%
\pgfpathlineto{\pgfqpoint{0.415459in}{0.494968in}}%
\pgfpathlineto{\pgfqpoint{0.407863in}{0.481357in}}%
\pgfpathlineto{\pgfqpoint{0.405266in}{0.468933in}}%
\pgfpathlineto{\pgfqpoint{0.404884in}{0.467746in}}%
\pgfpathlineto{\pgfqpoint{0.405266in}{0.465972in}}%
\pgfpathlineto{\pgfqpoint{0.406915in}{0.454135in}}%
\pgfpathlineto{\pgfqpoint{0.413559in}{0.440524in}}%
\pgfpathlineto{\pgfqpoint{0.420923in}{0.431602in}}%
\pgfpathlineto{\pgfqpoint{0.424129in}{0.426913in}}%
\pgfpathlineto{\pgfqpoint{0.436579in}{0.413602in}}%
\pgfpathlineto{\pgfqpoint{0.436848in}{0.413302in}}%
\pgfpathlineto{\pgfqpoint{0.451386in}{0.399691in}}%
\pgfpathlineto{\pgfqpoint{0.452236in}{0.398952in}}%
\pgfpathlineto{\pgfqpoint{0.467892in}{0.386313in}}%
\pgfpathlineto{\pgfqpoint{0.468238in}{0.386079in}}%
\pgfpathlineto{\pgfqpoint{0.483549in}{0.375256in}}%
\pgfpathlineto{\pgfqpoint{0.488943in}{0.372468in}}%
\pgfpathlineto{\pgfqpoint{0.499205in}{0.366066in}}%
\pgfpathlineto{\pgfqpoint{0.514862in}{0.360291in}}%
\pgfpathlineto{\pgfqpoint{0.528478in}{0.358857in}}%
\pgfpathlineto{\pgfqpoint{0.530519in}{0.358525in}}%
\pgfpathclose%
\pgfpathmoveto{\pgfqpoint{0.485854in}{0.399691in}}%
\pgfpathlineto{\pgfqpoint{0.483549in}{0.400936in}}%
\pgfpathlineto{\pgfqpoint{0.467892in}{0.412420in}}%
\pgfpathlineto{\pgfqpoint{0.466878in}{0.413302in}}%
\pgfpathlineto{\pgfqpoint{0.453668in}{0.426913in}}%
\pgfpathlineto{\pgfqpoint{0.452236in}{0.428917in}}%
\pgfpathlineto{\pgfqpoint{0.444310in}{0.440524in}}%
\pgfpathlineto{\pgfqpoint{0.438860in}{0.454135in}}%
\pgfpathlineto{\pgfqpoint{0.437306in}{0.467746in}}%
\pgfpathlineto{\pgfqpoint{0.439638in}{0.481357in}}%
\pgfpathlineto{\pgfqpoint{0.445869in}{0.494968in}}%
\pgfpathlineto{\pgfqpoint{0.452236in}{0.503667in}}%
\pgfpathlineto{\pgfqpoint{0.456010in}{0.508579in}}%
\pgfpathlineto{\pgfqpoint{0.467892in}{0.520277in}}%
\pgfpathlineto{\pgfqpoint{0.470193in}{0.522191in}}%
\pgfpathlineto{\pgfqpoint{0.483549in}{0.531820in}}%
\pgfpathlineto{\pgfqpoint{0.491094in}{0.535802in}}%
\pgfpathlineto{\pgfqpoint{0.499205in}{0.539977in}}%
\pgfpathlineto{\pgfqpoint{0.514862in}{0.544770in}}%
\pgfpathlineto{\pgfqpoint{0.530519in}{0.546136in}}%
\pgfpathlineto{\pgfqpoint{0.546175in}{0.544086in}}%
\pgfpathlineto{\pgfqpoint{0.561832in}{0.538606in}}%
\pgfpathlineto{\pgfqpoint{0.566914in}{0.535802in}}%
\pgfpathlineto{\pgfqpoint{0.577488in}{0.529801in}}%
\pgfpathlineto{\pgfqpoint{0.587617in}{0.522191in}}%
\pgfpathlineto{\pgfqpoint{0.593145in}{0.517385in}}%
\pgfpathlineto{\pgfqpoint{0.601899in}{0.508579in}}%
\pgfpathlineto{\pgfqpoint{0.608801in}{0.499387in}}%
\pgfpathlineto{\pgfqpoint{0.612027in}{0.494968in}}%
\pgfpathlineto{\pgfqpoint{0.618330in}{0.481357in}}%
\pgfpathlineto{\pgfqpoint{0.620689in}{0.467746in}}%
\pgfpathlineto{\pgfqpoint{0.619117in}{0.454135in}}%
\pgfpathlineto{\pgfqpoint{0.613604in}{0.440524in}}%
\pgfpathlineto{\pgfqpoint{0.608801in}{0.433472in}}%
\pgfpathlineto{\pgfqpoint{0.604221in}{0.426913in}}%
\pgfpathlineto{\pgfqpoint{0.593145in}{0.415301in}}%
\pgfpathlineto{\pgfqpoint{0.590943in}{0.413302in}}%
\pgfpathlineto{\pgfqpoint{0.577488in}{0.402972in}}%
\pgfpathlineto{\pgfqpoint{0.571838in}{0.399691in}}%
\pgfpathlineto{\pgfqpoint{0.561832in}{0.394156in}}%
\pgfpathlineto{\pgfqpoint{0.546175in}{0.388739in}}%
\pgfpathlineto{\pgfqpoint{0.530519in}{0.386711in}}%
\pgfpathlineto{\pgfqpoint{0.514862in}{0.388063in}}%
\pgfpathlineto{\pgfqpoint{0.499205in}{0.392801in}}%
\pgfpathlineto{\pgfqpoint{0.485854in}{0.399691in}}%
\pgfpathclose%
\pgfpathmoveto{\pgfqpoint{0.843650in}{0.358781in}}%
\pgfpathlineto{\pgfqpoint{0.843884in}{0.358857in}}%
\pgfpathlineto{\pgfqpoint{0.859306in}{0.362105in}}%
\pgfpathlineto{\pgfqpoint{0.874963in}{0.369536in}}%
\pgfpathlineto{\pgfqpoint{0.879082in}{0.372468in}}%
\pgfpathlineto{\pgfqpoint{0.890620in}{0.379313in}}%
\pgfpathlineto{\pgfqpoint{0.899468in}{0.386079in}}%
\pgfpathlineto{\pgfqpoint{0.906276in}{0.391081in}}%
\pgfpathlineto{\pgfqpoint{0.916492in}{0.399691in}}%
\pgfpathlineto{\pgfqpoint{0.921933in}{0.404624in}}%
\pgfpathlineto{\pgfqpoint{0.931225in}{0.413302in}}%
\pgfpathlineto{\pgfqpoint{0.937589in}{0.420427in}}%
\pgfpathlineto{\pgfqpoint{0.943937in}{0.426913in}}%
\pgfpathlineto{\pgfqpoint{0.953246in}{0.439881in}}%
\pgfpathlineto{\pgfqpoint{0.953842in}{0.440524in}}%
\pgfpathlineto{\pgfqpoint{0.961505in}{0.454135in}}%
\pgfpathlineto{\pgfqpoint{0.963691in}{0.467746in}}%
\pgfpathlineto{\pgfqpoint{0.960412in}{0.481357in}}%
\pgfpathlineto{\pgfqpoint{0.953246in}{0.492548in}}%
\pgfpathlineto{\pgfqpoint{0.952063in}{0.494968in}}%
\pgfpathlineto{\pgfqpoint{0.941503in}{0.508579in}}%
\pgfpathlineto{\pgfqpoint{0.937589in}{0.512416in}}%
\pgfpathlineto{\pgfqpoint{0.928468in}{0.522191in}}%
\pgfpathlineto{\pgfqpoint{0.921933in}{0.528164in}}%
\pgfpathlineto{\pgfqpoint{0.913313in}{0.535802in}}%
\pgfpathlineto{\pgfqpoint{0.906276in}{0.541716in}}%
\pgfpathlineto{\pgfqpoint{0.895769in}{0.549413in}}%
\pgfpathlineto{\pgfqpoint{0.890620in}{0.553430in}}%
\pgfpathlineto{\pgfqpoint{0.874963in}{0.562924in}}%
\pgfpathlineto{\pgfqpoint{0.874716in}{0.563024in}}%
\pgfpathlineto{\pgfqpoint{0.859306in}{0.570840in}}%
\pgfpathlineto{\pgfqpoint{0.843650in}{0.574364in}}%
\pgfpathlineto{\pgfqpoint{0.827993in}{0.573483in}}%
\pgfpathlineto{\pgfqpoint{0.812337in}{0.568193in}}%
\pgfpathlineto{\pgfqpoint{0.803854in}{0.563024in}}%
\pgfpathlineto{\pgfqpoint{0.796680in}{0.559529in}}%
\pgfpathlineto{\pgfqpoint{0.781992in}{0.549413in}}%
\pgfpathlineto{\pgfqpoint{0.781024in}{0.548791in}}%
\pgfpathlineto{\pgfqpoint{0.765367in}{0.536498in}}%
\pgfpathlineto{\pgfqpoint{0.764559in}{0.535802in}}%
\pgfpathlineto{\pgfqpoint{0.749710in}{0.522294in}}%
\pgfpathlineto{\pgfqpoint{0.749595in}{0.522191in}}%
\pgfpathlineto{\pgfqpoint{0.736527in}{0.508579in}}%
\pgfpathlineto{\pgfqpoint{0.734054in}{0.505161in}}%
\pgfpathlineto{\pgfqpoint{0.725438in}{0.494968in}}%
\pgfpathlineto{\pgfqpoint{0.718397in}{0.481594in}}%
\pgfpathlineto{\pgfqpoint{0.718219in}{0.481357in}}%
\pgfpathlineto{\pgfqpoint{0.714272in}{0.467746in}}%
\pgfpathlineto{\pgfqpoint{0.716903in}{0.454135in}}%
\pgfpathlineto{\pgfqpoint{0.718397in}{0.451883in}}%
\pgfpathlineto{\pgfqpoint{0.723646in}{0.440524in}}%
\pgfpathlineto{\pgfqpoint{0.734054in}{0.427327in}}%
\pgfpathlineto{\pgfqpoint{0.734332in}{0.426913in}}%
\pgfpathlineto{\pgfqpoint{0.746712in}{0.413302in}}%
\pgfpathlineto{\pgfqpoint{0.749710in}{0.410545in}}%
\pgfpathlineto{\pgfqpoint{0.761372in}{0.399691in}}%
\pgfpathlineto{\pgfqpoint{0.765367in}{0.396240in}}%
\pgfpathlineto{\pgfqpoint{0.778314in}{0.386079in}}%
\pgfpathlineto{\pgfqpoint{0.781024in}{0.383875in}}%
\pgfpathlineto{\pgfqpoint{0.796680in}{0.373420in}}%
\pgfpathlineto{\pgfqpoint{0.798676in}{0.372468in}}%
\pgfpathlineto{\pgfqpoint{0.812337in}{0.364580in}}%
\pgfpathlineto{\pgfqpoint{0.827993in}{0.359632in}}%
\pgfpathlineto{\pgfqpoint{0.842722in}{0.358857in}}%
\pgfpathlineto{\pgfqpoint{0.843650in}{0.358781in}}%
\pgfpathclose%
\pgfpathmoveto{\pgfqpoint{0.795681in}{0.399691in}}%
\pgfpathlineto{\pgfqpoint{0.781024in}{0.409867in}}%
\pgfpathlineto{\pgfqpoint{0.776972in}{0.413302in}}%
\pgfpathlineto{\pgfqpoint{0.765367in}{0.425015in}}%
\pgfpathlineto{\pgfqpoint{0.763631in}{0.426913in}}%
\pgfpathlineto{\pgfqpoint{0.754348in}{0.440524in}}%
\pgfpathlineto{\pgfqpoint{0.749710in}{0.452149in}}%
\pgfpathlineto{\pgfqpoint{0.748874in}{0.454135in}}%
\pgfpathlineto{\pgfqpoint{0.747204in}{0.467746in}}%
\pgfpathlineto{\pgfqpoint{0.749710in}{0.481357in}}%
\pgfpathlineto{\pgfqpoint{0.749710in}{0.481358in}}%
\pgfpathlineto{\pgfqpoint{0.755895in}{0.494968in}}%
\pgfpathlineto{\pgfqpoint{0.765367in}{0.507825in}}%
\pgfpathlineto{\pgfqpoint{0.765962in}{0.508579in}}%
\pgfpathlineto{\pgfqpoint{0.780089in}{0.522191in}}%
\pgfpathlineto{\pgfqpoint{0.781024in}{0.522966in}}%
\pgfpathlineto{\pgfqpoint{0.796680in}{0.533705in}}%
\pgfpathlineto{\pgfqpoint{0.800985in}{0.535802in}}%
\pgfpathlineto{\pgfqpoint{0.812337in}{0.541210in}}%
\pgfpathlineto{\pgfqpoint{0.827993in}{0.545316in}}%
\pgfpathlineto{\pgfqpoint{0.843650in}{0.546000in}}%
\pgfpathlineto{\pgfqpoint{0.859306in}{0.543264in}}%
\pgfpathlineto{\pgfqpoint{0.874963in}{0.537097in}}%
\pgfpathlineto{\pgfqpoint{0.877169in}{0.535802in}}%
\pgfpathlineto{\pgfqpoint{0.890620in}{0.527652in}}%
\pgfpathlineto{\pgfqpoint{0.897623in}{0.522191in}}%
\pgfpathlineto{\pgfqpoint{0.906276in}{0.514366in}}%
\pgfpathlineto{\pgfqpoint{0.911928in}{0.508579in}}%
\pgfpathlineto{\pgfqpoint{0.921933in}{0.494998in}}%
\pgfpathlineto{\pgfqpoint{0.921954in}{0.494968in}}%
\pgfpathlineto{\pgfqpoint{0.928358in}{0.481357in}}%
\pgfpathlineto{\pgfqpoint{0.930754in}{0.467746in}}%
\pgfpathlineto{\pgfqpoint{0.929157in}{0.454135in}}%
\pgfpathlineto{\pgfqpoint{0.923556in}{0.440524in}}%
\pgfpathlineto{\pgfqpoint{0.921933in}{0.438143in}}%
\pgfpathlineto{\pgfqpoint{0.914238in}{0.426913in}}%
\pgfpathlineto{\pgfqpoint{0.906276in}{0.418416in}}%
\pgfpathlineto{\pgfqpoint{0.900863in}{0.413302in}}%
\pgfpathlineto{\pgfqpoint{0.890620in}{0.405140in}}%
\pgfpathlineto{\pgfqpoint{0.881830in}{0.399691in}}%
\pgfpathlineto{\pgfqpoint{0.874963in}{0.395647in}}%
\pgfpathlineto{\pgfqpoint{0.859306in}{0.389550in}}%
\pgfpathlineto{\pgfqpoint{0.843650in}{0.386847in}}%
\pgfpathlineto{\pgfqpoint{0.827993in}{0.387522in}}%
\pgfpathlineto{\pgfqpoint{0.812337in}{0.391581in}}%
\pgfpathlineto{\pgfqpoint{0.796680in}{0.399035in}}%
\pgfpathlineto{\pgfqpoint{0.795681in}{0.399691in}}%
\pgfpathclose%
\pgfpathmoveto{\pgfqpoint{1.454256in}{0.358781in}}%
\pgfpathlineto{\pgfqpoint{1.455184in}{0.358857in}}%
\pgfpathlineto{\pgfqpoint{1.469913in}{0.359632in}}%
\pgfpathlineto{\pgfqpoint{1.485569in}{0.364580in}}%
\pgfpathlineto{\pgfqpoint{1.499230in}{0.372468in}}%
\pgfpathlineto{\pgfqpoint{1.501226in}{0.373420in}}%
\pgfpathlineto{\pgfqpoint{1.516882in}{0.383875in}}%
\pgfpathlineto{\pgfqpoint{1.519592in}{0.386079in}}%
\pgfpathlineto{\pgfqpoint{1.532539in}{0.396240in}}%
\pgfpathlineto{\pgfqpoint{1.536534in}{0.399691in}}%
\pgfpathlineto{\pgfqpoint{1.548195in}{0.410545in}}%
\pgfpathlineto{\pgfqpoint{1.551194in}{0.413302in}}%
\pgfpathlineto{\pgfqpoint{1.563573in}{0.426913in}}%
\pgfpathlineto{\pgfqpoint{1.563852in}{0.427327in}}%
\pgfpathlineto{\pgfqpoint{1.574260in}{0.440524in}}%
\pgfpathlineto{\pgfqpoint{1.579508in}{0.451883in}}%
\pgfpathlineto{\pgfqpoint{1.581003in}{0.454135in}}%
\pgfpathlineto{\pgfqpoint{1.583634in}{0.467746in}}%
\pgfpathlineto{\pgfqpoint{1.579687in}{0.481357in}}%
\pgfpathlineto{\pgfqpoint{1.579508in}{0.481594in}}%
\pgfpathlineto{\pgfqpoint{1.572468in}{0.494968in}}%
\pgfpathlineto{\pgfqpoint{1.563852in}{0.505161in}}%
\pgfpathlineto{\pgfqpoint{1.561378in}{0.508579in}}%
\pgfpathlineto{\pgfqpoint{1.548311in}{0.522191in}}%
\pgfpathlineto{\pgfqpoint{1.548195in}{0.522294in}}%
\pgfpathlineto{\pgfqpoint{1.533347in}{0.535802in}}%
\pgfpathlineto{\pgfqpoint{1.532539in}{0.536498in}}%
\pgfpathlineto{\pgfqpoint{1.516882in}{0.548791in}}%
\pgfpathlineto{\pgfqpoint{1.515914in}{0.549413in}}%
\pgfpathlineto{\pgfqpoint{1.501226in}{0.559529in}}%
\pgfpathlineto{\pgfqpoint{1.494052in}{0.563024in}}%
\pgfpathlineto{\pgfqpoint{1.485569in}{0.568193in}}%
\pgfpathlineto{\pgfqpoint{1.469913in}{0.573483in}}%
\pgfpathlineto{\pgfqpoint{1.454256in}{0.574364in}}%
\pgfpathlineto{\pgfqpoint{1.438599in}{0.570840in}}%
\pgfpathlineto{\pgfqpoint{1.423190in}{0.563024in}}%
\pgfpathlineto{\pgfqpoint{1.422943in}{0.562924in}}%
\pgfpathlineto{\pgfqpoint{1.407286in}{0.553430in}}%
\pgfpathlineto{\pgfqpoint{1.402136in}{0.549413in}}%
\pgfpathlineto{\pgfqpoint{1.391630in}{0.541716in}}%
\pgfpathlineto{\pgfqpoint{1.384593in}{0.535802in}}%
\pgfpathlineto{\pgfqpoint{1.375973in}{0.528164in}}%
\pgfpathlineto{\pgfqpoint{1.369438in}{0.522191in}}%
\pgfpathlineto{\pgfqpoint{1.360317in}{0.512416in}}%
\pgfpathlineto{\pgfqpoint{1.356403in}{0.508579in}}%
\pgfpathlineto{\pgfqpoint{1.345843in}{0.494968in}}%
\pgfpathlineto{\pgfqpoint{1.344660in}{0.492548in}}%
\pgfpathlineto{\pgfqpoint{1.337494in}{0.481357in}}%
\pgfpathlineto{\pgfqpoint{1.334215in}{0.467746in}}%
\pgfpathlineto{\pgfqpoint{1.336401in}{0.454135in}}%
\pgfpathlineto{\pgfqpoint{1.344063in}{0.440524in}}%
\pgfpathlineto{\pgfqpoint{1.344660in}{0.439881in}}%
\pgfpathlineto{\pgfqpoint{1.353969in}{0.426913in}}%
\pgfpathlineto{\pgfqpoint{1.360317in}{0.420427in}}%
\pgfpathlineto{\pgfqpoint{1.366681in}{0.413302in}}%
\pgfpathlineto{\pgfqpoint{1.375973in}{0.404624in}}%
\pgfpathlineto{\pgfqpoint{1.381414in}{0.399691in}}%
\pgfpathlineto{\pgfqpoint{1.391630in}{0.391081in}}%
\pgfpathlineto{\pgfqpoint{1.398438in}{0.386079in}}%
\pgfpathlineto{\pgfqpoint{1.407286in}{0.379313in}}%
\pgfpathlineto{\pgfqpoint{1.418823in}{0.372468in}}%
\pgfpathlineto{\pgfqpoint{1.422943in}{0.369536in}}%
\pgfpathlineto{\pgfqpoint{1.438599in}{0.362105in}}%
\pgfpathlineto{\pgfqpoint{1.454022in}{0.358857in}}%
\pgfpathlineto{\pgfqpoint{1.454256in}{0.358781in}}%
\pgfpathclose%
\pgfpathmoveto{\pgfqpoint{1.416076in}{0.399691in}}%
\pgfpathlineto{\pgfqpoint{1.407286in}{0.405140in}}%
\pgfpathlineto{\pgfqpoint{1.397043in}{0.413302in}}%
\pgfpathlineto{\pgfqpoint{1.391630in}{0.418416in}}%
\pgfpathlineto{\pgfqpoint{1.383667in}{0.426913in}}%
\pgfpathlineto{\pgfqpoint{1.375973in}{0.438143in}}%
\pgfpathlineto{\pgfqpoint{1.374350in}{0.440524in}}%
\pgfpathlineto{\pgfqpoint{1.368749in}{0.454135in}}%
\pgfpathlineto{\pgfqpoint{1.367152in}{0.467746in}}%
\pgfpathlineto{\pgfqpoint{1.369548in}{0.481357in}}%
\pgfpathlineto{\pgfqpoint{1.375952in}{0.494968in}}%
\pgfpathlineto{\pgfqpoint{1.375973in}{0.494998in}}%
\pgfpathlineto{\pgfqpoint{1.385978in}{0.508579in}}%
\pgfpathlineto{\pgfqpoint{1.391630in}{0.514366in}}%
\pgfpathlineto{\pgfqpoint{1.400283in}{0.522191in}}%
\pgfpathlineto{\pgfqpoint{1.407286in}{0.527652in}}%
\pgfpathlineto{\pgfqpoint{1.420737in}{0.535802in}}%
\pgfpathlineto{\pgfqpoint{1.422943in}{0.537097in}}%
\pgfpathlineto{\pgfqpoint{1.438599in}{0.543264in}}%
\pgfpathlineto{\pgfqpoint{1.454256in}{0.546000in}}%
\pgfpathlineto{\pgfqpoint{1.469913in}{0.545316in}}%
\pgfpathlineto{\pgfqpoint{1.485569in}{0.541210in}}%
\pgfpathlineto{\pgfqpoint{1.496921in}{0.535802in}}%
\pgfpathlineto{\pgfqpoint{1.501226in}{0.533705in}}%
\pgfpathlineto{\pgfqpoint{1.516882in}{0.522966in}}%
\pgfpathlineto{\pgfqpoint{1.517817in}{0.522191in}}%
\pgfpathlineto{\pgfqpoint{1.531944in}{0.508579in}}%
\pgfpathlineto{\pgfqpoint{1.532539in}{0.507825in}}%
\pgfpathlineto{\pgfqpoint{1.542011in}{0.494968in}}%
\pgfpathlineto{\pgfqpoint{1.548195in}{0.481358in}}%
\pgfpathlineto{\pgfqpoint{1.548196in}{0.481357in}}%
\pgfpathlineto{\pgfqpoint{1.550702in}{0.467746in}}%
\pgfpathlineto{\pgfqpoint{1.549031in}{0.454135in}}%
\pgfpathlineto{\pgfqpoint{1.548195in}{0.452149in}}%
\pgfpathlineto{\pgfqpoint{1.543558in}{0.440524in}}%
\pgfpathlineto{\pgfqpoint{1.534275in}{0.426913in}}%
\pgfpathlineto{\pgfqpoint{1.532539in}{0.425015in}}%
\pgfpathlineto{\pgfqpoint{1.520934in}{0.413302in}}%
\pgfpathlineto{\pgfqpoint{1.516882in}{0.409867in}}%
\pgfpathlineto{\pgfqpoint{1.502225in}{0.399691in}}%
\pgfpathlineto{\pgfqpoint{1.501226in}{0.399035in}}%
\pgfpathlineto{\pgfqpoint{1.485569in}{0.391581in}}%
\pgfpathlineto{\pgfqpoint{1.469913in}{0.387522in}}%
\pgfpathlineto{\pgfqpoint{1.454256in}{0.386847in}}%
\pgfpathlineto{\pgfqpoint{1.438599in}{0.389550in}}%
\pgfpathlineto{\pgfqpoint{1.422943in}{0.395647in}}%
\pgfpathlineto{\pgfqpoint{1.416076in}{0.399691in}}%
\pgfpathclose%
\pgfpathmoveto{\pgfqpoint{1.767387in}{0.358525in}}%
\pgfpathlineto{\pgfqpoint{1.769428in}{0.358857in}}%
\pgfpathlineto{\pgfqpoint{1.783044in}{0.360291in}}%
\pgfpathlineto{\pgfqpoint{1.798700in}{0.366066in}}%
\pgfpathlineto{\pgfqpoint{1.808963in}{0.372468in}}%
\pgfpathlineto{\pgfqpoint{1.814357in}{0.375256in}}%
\pgfpathlineto{\pgfqpoint{1.829668in}{0.386079in}}%
\pgfpathlineto{\pgfqpoint{1.830014in}{0.386313in}}%
\pgfpathlineto{\pgfqpoint{1.845670in}{0.398952in}}%
\pgfpathlineto{\pgfqpoint{1.846519in}{0.399691in}}%
\pgfpathlineto{\pgfqpoint{1.861058in}{0.413302in}}%
\pgfpathlineto{\pgfqpoint{1.861327in}{0.413602in}}%
\pgfpathlineto{\pgfqpoint{1.873777in}{0.426913in}}%
\pgfpathlineto{\pgfqpoint{1.876983in}{0.431602in}}%
\pgfpathlineto{\pgfqpoint{1.884347in}{0.440524in}}%
\pgfpathlineto{\pgfqpoint{1.890991in}{0.454135in}}%
\pgfpathlineto{\pgfqpoint{1.892640in}{0.465972in}}%
\pgfpathlineto{\pgfqpoint{1.893022in}{0.467746in}}%
\pgfpathlineto{\pgfqpoint{1.892640in}{0.468933in}}%
\pgfpathlineto{\pgfqpoint{1.890042in}{0.481357in}}%
\pgfpathlineto{\pgfqpoint{1.882447in}{0.494968in}}%
\pgfpathlineto{\pgfqpoint{1.876983in}{0.501144in}}%
\pgfpathlineto{\pgfqpoint{1.871517in}{0.508579in}}%
\pgfpathlineto{\pgfqpoint{1.861327in}{0.519032in}}%
\pgfpathlineto{\pgfqpoint{1.858375in}{0.522191in}}%
\pgfpathlineto{\pgfqpoint{1.845670in}{0.533788in}}%
\pgfpathlineto{\pgfqpoint{1.843300in}{0.535802in}}%
\pgfpathlineto{\pgfqpoint{1.830014in}{0.546539in}}%
\pgfpathlineto{\pgfqpoint{1.825750in}{0.549413in}}%
\pgfpathlineto{\pgfqpoint{1.814357in}{0.557629in}}%
\pgfpathlineto{\pgfqpoint{1.804139in}{0.563024in}}%
\pgfpathlineto{\pgfqpoint{1.798700in}{0.566604in}}%
\pgfpathlineto{\pgfqpoint{1.783044in}{0.572779in}}%
\pgfpathlineto{\pgfqpoint{1.767387in}{0.574540in}}%
\pgfpathlineto{\pgfqpoint{1.751731in}{0.571898in}}%
\pgfpathlineto{\pgfqpoint{1.736074in}{0.564838in}}%
\pgfpathlineto{\pgfqpoint{1.733504in}{0.563024in}}%
\pgfpathlineto{\pgfqpoint{1.720418in}{0.555596in}}%
\pgfpathlineto{\pgfqpoint{1.712192in}{0.549413in}}%
\pgfpathlineto{\pgfqpoint{1.704761in}{0.544180in}}%
\pgfpathlineto{\pgfqpoint{1.694615in}{0.535802in}}%
\pgfpathlineto{\pgfqpoint{1.689104in}{0.531011in}}%
\pgfpathlineto{\pgfqpoint{1.679467in}{0.522191in}}%
\pgfpathlineto{\pgfqpoint{1.673448in}{0.515731in}}%
\pgfpathlineto{\pgfqpoint{1.666336in}{0.508579in}}%
\pgfpathlineto{\pgfqpoint{1.657791in}{0.497203in}}%
\pgfpathlineto{\pgfqpoint{1.655705in}{0.494968in}}%
\pgfpathlineto{\pgfqpoint{1.647584in}{0.481357in}}%
\pgfpathlineto{\pgfqpoint{1.644545in}{0.467746in}}%
\pgfpathlineto{\pgfqpoint{1.646571in}{0.454135in}}%
\pgfpathlineto{\pgfqpoint{1.653673in}{0.440524in}}%
\pgfpathlineto{\pgfqpoint{1.657791in}{0.435796in}}%
\pgfpathlineto{\pgfqpoint{1.663997in}{0.426913in}}%
\pgfpathlineto{\pgfqpoint{1.673448in}{0.417008in}}%
\pgfpathlineto{\pgfqpoint{1.676753in}{0.413302in}}%
\pgfpathlineto{\pgfqpoint{1.689104in}{0.401751in}}%
\pgfpathlineto{\pgfqpoint{1.691421in}{0.399691in}}%
\pgfpathlineto{\pgfqpoint{1.704761in}{0.388646in}}%
\pgfpathlineto{\pgfqpoint{1.708395in}{0.386079in}}%
\pgfpathlineto{\pgfqpoint{1.720418in}{0.377221in}}%
\pgfpathlineto{\pgfqpoint{1.728970in}{0.372468in}}%
\pgfpathlineto{\pgfqpoint{1.736074in}{0.367719in}}%
\pgfpathlineto{\pgfqpoint{1.751731in}{0.361115in}}%
\pgfpathlineto{\pgfqpoint{1.766022in}{0.358857in}}%
\pgfpathlineto{\pgfqpoint{1.767387in}{0.358525in}}%
\pgfpathclose%
\pgfpathmoveto{\pgfqpoint{1.726068in}{0.399691in}}%
\pgfpathlineto{\pgfqpoint{1.720418in}{0.402972in}}%
\pgfpathlineto{\pgfqpoint{1.706963in}{0.413302in}}%
\pgfpathlineto{\pgfqpoint{1.704761in}{0.415301in}}%
\pgfpathlineto{\pgfqpoint{1.693685in}{0.426913in}}%
\pgfpathlineto{\pgfqpoint{1.689104in}{0.433472in}}%
\pgfpathlineto{\pgfqpoint{1.684302in}{0.440524in}}%
\pgfpathlineto{\pgfqpoint{1.678789in}{0.454135in}}%
\pgfpathlineto{\pgfqpoint{1.677217in}{0.467746in}}%
\pgfpathlineto{\pgfqpoint{1.679576in}{0.481357in}}%
\pgfpathlineto{\pgfqpoint{1.685879in}{0.494968in}}%
\pgfpathlineto{\pgfqpoint{1.689104in}{0.499387in}}%
\pgfpathlineto{\pgfqpoint{1.696007in}{0.508579in}}%
\pgfpathlineto{\pgfqpoint{1.704761in}{0.517385in}}%
\pgfpathlineto{\pgfqpoint{1.710289in}{0.522191in}}%
\pgfpathlineto{\pgfqpoint{1.720418in}{0.529801in}}%
\pgfpathlineto{\pgfqpoint{1.730991in}{0.535802in}}%
\pgfpathlineto{\pgfqpoint{1.736074in}{0.538606in}}%
\pgfpathlineto{\pgfqpoint{1.751731in}{0.544086in}}%
\pgfpathlineto{\pgfqpoint{1.767387in}{0.546136in}}%
\pgfpathlineto{\pgfqpoint{1.783044in}{0.544770in}}%
\pgfpathlineto{\pgfqpoint{1.798700in}{0.539977in}}%
\pgfpathlineto{\pgfqpoint{1.806812in}{0.535802in}}%
\pgfpathlineto{\pgfqpoint{1.814357in}{0.531820in}}%
\pgfpathlineto{\pgfqpoint{1.827713in}{0.522191in}}%
\pgfpathlineto{\pgfqpoint{1.830014in}{0.520277in}}%
\pgfpathlineto{\pgfqpoint{1.841896in}{0.508579in}}%
\pgfpathlineto{\pgfqpoint{1.845670in}{0.503667in}}%
\pgfpathlineto{\pgfqpoint{1.852037in}{0.494968in}}%
\pgfpathlineto{\pgfqpoint{1.858268in}{0.481357in}}%
\pgfpathlineto{\pgfqpoint{1.860600in}{0.467746in}}%
\pgfpathlineto{\pgfqpoint{1.859045in}{0.454135in}}%
\pgfpathlineto{\pgfqpoint{1.853595in}{0.440524in}}%
\pgfpathlineto{\pgfqpoint{1.845670in}{0.428917in}}%
\pgfpathlineto{\pgfqpoint{1.844238in}{0.426913in}}%
\pgfpathlineto{\pgfqpoint{1.831028in}{0.413302in}}%
\pgfpathlineto{\pgfqpoint{1.830014in}{0.412420in}}%
\pgfpathlineto{\pgfqpoint{1.814357in}{0.400936in}}%
\pgfpathlineto{\pgfqpoint{1.812052in}{0.399691in}}%
\pgfpathlineto{\pgfqpoint{1.798700in}{0.392801in}}%
\pgfpathlineto{\pgfqpoint{1.783044in}{0.388063in}}%
\pgfpathlineto{\pgfqpoint{1.767387in}{0.386711in}}%
\pgfpathlineto{\pgfqpoint{1.751731in}{0.388739in}}%
\pgfpathlineto{\pgfqpoint{1.736074in}{0.394156in}}%
\pgfpathlineto{\pgfqpoint{1.726068in}{0.399691in}}%
\pgfpathclose%
\pgfpathmoveto{\pgfqpoint{1.109812in}{0.371519in}}%
\pgfpathlineto{\pgfqpoint{1.125468in}{0.363260in}}%
\pgfpathlineto{\pgfqpoint{1.141125in}{0.359138in}}%
\pgfpathlineto{\pgfqpoint{1.156781in}{0.359138in}}%
\pgfpathlineto{\pgfqpoint{1.172438in}{0.363260in}}%
\pgfpathlineto{\pgfqpoint{1.188094in}{0.371519in}}%
\pgfpathlineto{\pgfqpoint{1.189355in}{0.372468in}}%
\pgfpathlineto{\pgfqpoint{1.203751in}{0.381532in}}%
\pgfpathlineto{\pgfqpoint{1.209505in}{0.386079in}}%
\pgfpathlineto{\pgfqpoint{1.219407in}{0.393614in}}%
\pgfpathlineto{\pgfqpoint{1.226517in}{0.399691in}}%
\pgfpathlineto{\pgfqpoint{1.235064in}{0.407559in}}%
\pgfpathlineto{\pgfqpoint{1.241248in}{0.413302in}}%
\pgfpathlineto{\pgfqpoint{1.250721in}{0.423843in}}%
\pgfpathlineto{\pgfqpoint{1.253821in}{0.426913in}}%
\pgfpathlineto{\pgfqpoint{1.264026in}{0.440524in}}%
\pgfpathlineto{\pgfqpoint{1.266377in}{0.445823in}}%
\pgfpathlineto{\pgfqpoint{1.271429in}{0.454135in}}%
\pgfpathlineto{\pgfqpoint{1.273812in}{0.467746in}}%
\pgfpathlineto{\pgfqpoint{1.270237in}{0.481357in}}%
\pgfpathlineto{\pgfqpoint{1.266377in}{0.486951in}}%
\pgfpathlineto{\pgfqpoint{1.262325in}{0.494968in}}%
\pgfpathlineto{\pgfqpoint{1.251274in}{0.508579in}}%
\pgfpathlineto{\pgfqpoint{1.250721in}{0.509105in}}%
\pgfpathlineto{\pgfqpoint{1.238435in}{0.522191in}}%
\pgfpathlineto{\pgfqpoint{1.235064in}{0.525255in}}%
\pgfpathlineto{\pgfqpoint{1.223340in}{0.535802in}}%
\pgfpathlineto{\pgfqpoint{1.219407in}{0.539154in}}%
\pgfpathlineto{\pgfqpoint{1.205885in}{0.549413in}}%
\pgfpathlineto{\pgfqpoint{1.203751in}{0.551134in}}%
\pgfpathlineto{\pgfqpoint{1.188094in}{0.561294in}}%
\pgfpathlineto{\pgfqpoint{1.184211in}{0.563024in}}%
\pgfpathlineto{\pgfqpoint{1.172438in}{0.569605in}}%
\pgfpathlineto{\pgfqpoint{1.156781in}{0.574012in}}%
\pgfpathlineto{\pgfqpoint{1.141125in}{0.574012in}}%
\pgfpathlineto{\pgfqpoint{1.125468in}{0.569605in}}%
\pgfpathlineto{\pgfqpoint{1.113695in}{0.563024in}}%
\pgfpathlineto{\pgfqpoint{1.109812in}{0.561294in}}%
\pgfpathlineto{\pgfqpoint{1.094155in}{0.551134in}}%
\pgfpathlineto{\pgfqpoint{1.092020in}{0.549413in}}%
\pgfpathlineto{\pgfqpoint{1.078498in}{0.539154in}}%
\pgfpathlineto{\pgfqpoint{1.074566in}{0.535802in}}%
\pgfpathlineto{\pgfqpoint{1.062842in}{0.525255in}}%
\pgfpathlineto{\pgfqpoint{1.059471in}{0.522191in}}%
\pgfpathlineto{\pgfqpoint{1.047185in}{0.509105in}}%
\pgfpathlineto{\pgfqpoint{1.046632in}{0.508579in}}%
\pgfpathlineto{\pgfqpoint{1.035581in}{0.494968in}}%
\pgfpathlineto{\pgfqpoint{1.031529in}{0.486951in}}%
\pgfpathlineto{\pgfqpoint{1.027669in}{0.481357in}}%
\pgfpathlineto{\pgfqpoint{1.024094in}{0.467746in}}%
\pgfpathlineto{\pgfqpoint{1.026477in}{0.454135in}}%
\pgfpathlineto{\pgfqpoint{1.031529in}{0.445823in}}%
\pgfpathlineto{\pgfqpoint{1.033880in}{0.440524in}}%
\pgfpathlineto{\pgfqpoint{1.044085in}{0.426913in}}%
\pgfpathlineto{\pgfqpoint{1.047185in}{0.423843in}}%
\pgfpathlineto{\pgfqpoint{1.056658in}{0.413302in}}%
\pgfpathlineto{\pgfqpoint{1.062842in}{0.407559in}}%
\pgfpathlineto{\pgfqpoint{1.071389in}{0.399691in}}%
\pgfpathlineto{\pgfqpoint{1.078498in}{0.393614in}}%
\pgfpathlineto{\pgfqpoint{1.088401in}{0.386079in}}%
\pgfpathlineto{\pgfqpoint{1.094155in}{0.381532in}}%
\pgfpathlineto{\pgfqpoint{1.108551in}{0.372468in}}%
\pgfpathlineto{\pgfqpoint{1.109812in}{0.371519in}}%
\pgfpathclose%
\pgfpathmoveto{\pgfqpoint{1.105932in}{0.399691in}}%
\pgfpathlineto{\pgfqpoint{1.094155in}{0.407440in}}%
\pgfpathlineto{\pgfqpoint{1.087036in}{0.413302in}}%
\pgfpathlineto{\pgfqpoint{1.078498in}{0.421656in}}%
\pgfpathlineto{\pgfqpoint{1.073641in}{0.426913in}}%
\pgfpathlineto{\pgfqpoint{1.064389in}{0.440524in}}%
\pgfpathlineto{\pgfqpoint{1.062842in}{0.444360in}}%
\pgfpathlineto{\pgfqpoint{1.058767in}{0.454135in}}%
\pgfpathlineto{\pgfqpoint{1.057138in}{0.467746in}}%
\pgfpathlineto{\pgfqpoint{1.059583in}{0.481357in}}%
\pgfpathlineto{\pgfqpoint{1.062842in}{0.488245in}}%
\pgfpathlineto{\pgfqpoint{1.065931in}{0.494968in}}%
\pgfpathlineto{\pgfqpoint{1.075951in}{0.508579in}}%
\pgfpathlineto{\pgfqpoint{1.078498in}{0.511225in}}%
\pgfpathlineto{\pgfqpoint{1.090207in}{0.522191in}}%
\pgfpathlineto{\pgfqpoint{1.094155in}{0.525373in}}%
\pgfpathlineto{\pgfqpoint{1.109812in}{0.535458in}}%
\pgfpathlineto{\pgfqpoint{1.110584in}{0.535802in}}%
\pgfpathlineto{\pgfqpoint{1.125468in}{0.542306in}}%
\pgfpathlineto{\pgfqpoint{1.141125in}{0.545727in}}%
\pgfpathlineto{\pgfqpoint{1.156781in}{0.545727in}}%
\pgfpathlineto{\pgfqpoint{1.172438in}{0.542306in}}%
\pgfpathlineto{\pgfqpoint{1.187322in}{0.535802in}}%
\pgfpathlineto{\pgfqpoint{1.188094in}{0.535458in}}%
\pgfpathlineto{\pgfqpoint{1.203751in}{0.525373in}}%
\pgfpathlineto{\pgfqpoint{1.207699in}{0.522191in}}%
\pgfpathlineto{\pgfqpoint{1.219407in}{0.511225in}}%
\pgfpathlineto{\pgfqpoint{1.221955in}{0.508579in}}%
\pgfpathlineto{\pgfqpoint{1.231975in}{0.494968in}}%
\pgfpathlineto{\pgfqpoint{1.235064in}{0.488245in}}%
\pgfpathlineto{\pgfqpoint{1.238323in}{0.481357in}}%
\pgfpathlineto{\pgfqpoint{1.240768in}{0.467746in}}%
\pgfpathlineto{\pgfqpoint{1.239138in}{0.454135in}}%
\pgfpathlineto{\pgfqpoint{1.235064in}{0.444360in}}%
\pgfpathlineto{\pgfqpoint{1.233517in}{0.440524in}}%
\pgfpathlineto{\pgfqpoint{1.224265in}{0.426913in}}%
\pgfpathlineto{\pgfqpoint{1.219407in}{0.421656in}}%
\pgfpathlineto{\pgfqpoint{1.210870in}{0.413302in}}%
\pgfpathlineto{\pgfqpoint{1.203751in}{0.407440in}}%
\pgfpathlineto{\pgfqpoint{1.191973in}{0.399691in}}%
\pgfpathlineto{\pgfqpoint{1.188094in}{0.397274in}}%
\pgfpathlineto{\pgfqpoint{1.172438in}{0.390498in}}%
\pgfpathlineto{\pgfqpoint{1.156781in}{0.387117in}}%
\pgfpathlineto{\pgfqpoint{1.141125in}{0.387117in}}%
\pgfpathlineto{\pgfqpoint{1.125468in}{0.390498in}}%
\pgfpathlineto{\pgfqpoint{1.109812in}{0.397274in}}%
\pgfpathlineto{\pgfqpoint{1.105932in}{0.399691in}}%
\pgfpathclose%
\pgfpathmoveto{\pgfqpoint{0.514862in}{0.629780in}}%
\pgfpathlineto{\pgfqpoint{0.530519in}{0.627493in}}%
\pgfpathlineto{\pgfqpoint{0.546175in}{0.630925in}}%
\pgfpathlineto{\pgfqpoint{0.546447in}{0.631079in}}%
\pgfpathlineto{\pgfqpoint{0.561832in}{0.637200in}}%
\pgfpathlineto{\pgfqpoint{0.573556in}{0.644691in}}%
\pgfpathlineto{\pgfqpoint{0.577488in}{0.646841in}}%
\pgfpathlineto{\pgfqpoint{0.593145in}{0.658201in}}%
\pgfpathlineto{\pgfqpoint{0.593264in}{0.658302in}}%
\pgfpathlineto{\pgfqpoint{0.608801in}{0.671210in}}%
\pgfpathlineto{\pgfqpoint{0.609602in}{0.671913in}}%
\pgfpathlineto{\pgfqpoint{0.623743in}{0.685524in}}%
\pgfpathlineto{\pgfqpoint{0.624458in}{0.686365in}}%
\pgfpathlineto{\pgfqpoint{0.636095in}{0.699135in}}%
\pgfpathlineto{\pgfqpoint{0.640115in}{0.705372in}}%
\pgfpathlineto{\pgfqpoint{0.646060in}{0.712746in}}%
\pgfpathlineto{\pgfqpoint{0.652146in}{0.726357in}}%
\pgfpathlineto{\pgfqpoint{0.653158in}{0.739968in}}%
\pgfpathlineto{\pgfqpoint{0.649105in}{0.753579in}}%
\pgfpathlineto{\pgfqpoint{0.640115in}{0.766976in}}%
\pgfpathlineto{\pgfqpoint{0.640000in}{0.767191in}}%
\pgfpathlineto{\pgfqpoint{0.629079in}{0.780802in}}%
\pgfpathlineto{\pgfqpoint{0.624458in}{0.785279in}}%
\pgfpathlineto{\pgfqpoint{0.615605in}{0.794413in}}%
\pgfpathlineto{\pgfqpoint{0.608801in}{0.800530in}}%
\pgfpathlineto{\pgfqpoint{0.600016in}{0.808024in}}%
\pgfpathlineto{\pgfqpoint{0.593145in}{0.813705in}}%
\pgfpathlineto{\pgfqpoint{0.581901in}{0.821635in}}%
\pgfpathlineto{\pgfqpoint{0.577488in}{0.825037in}}%
\pgfpathlineto{\pgfqpoint{0.561832in}{0.834218in}}%
\pgfpathlineto{\pgfqpoint{0.559047in}{0.835246in}}%
\pgfpathlineto{\pgfqpoint{0.546175in}{0.841476in}}%
\pgfpathlineto{\pgfqpoint{0.530519in}{0.844326in}}%
\pgfpathlineto{\pgfqpoint{0.514862in}{0.842427in}}%
\pgfpathlineto{\pgfqpoint{0.499205in}{0.835765in}}%
\pgfpathlineto{\pgfqpoint{0.498466in}{0.835246in}}%
\pgfpathlineto{\pgfqpoint{0.483549in}{0.827153in}}%
\pgfpathlineto{\pgfqpoint{0.476089in}{0.821635in}}%
\pgfpathlineto{\pgfqpoint{0.467892in}{0.816102in}}%
\pgfpathlineto{\pgfqpoint{0.457910in}{0.808024in}}%
\pgfpathlineto{\pgfqpoint{0.452236in}{0.803294in}}%
\pgfpathlineto{\pgfqpoint{0.442332in}{0.794413in}}%
\pgfpathlineto{\pgfqpoint{0.436579in}{0.788494in}}%
\pgfpathlineto{\pgfqpoint{0.428796in}{0.780802in}}%
\pgfpathlineto{\pgfqpoint{0.420923in}{0.770772in}}%
\pgfpathlineto{\pgfqpoint{0.417550in}{0.767191in}}%
\pgfpathlineto{\pgfqpoint{0.409001in}{0.753579in}}%
\pgfpathlineto{\pgfqpoint{0.405266in}{0.740172in}}%
\pgfpathlineto{\pgfqpoint{0.405179in}{0.739968in}}%
\pgfpathlineto{\pgfqpoint{0.405266in}{0.739161in}}%
\pgfpathlineto{\pgfqpoint{0.406157in}{0.726357in}}%
\pgfpathlineto{\pgfqpoint{0.411849in}{0.712746in}}%
\pgfpathlineto{\pgfqpoint{0.420923in}{0.700870in}}%
\pgfpathlineto{\pgfqpoint{0.422017in}{0.699135in}}%
\pgfpathlineto{\pgfqpoint{0.434043in}{0.685524in}}%
\pgfpathlineto{\pgfqpoint{0.436579in}{0.683168in}}%
\pgfpathlineto{\pgfqpoint{0.448266in}{0.671913in}}%
\pgfpathlineto{\pgfqpoint{0.452236in}{0.668439in}}%
\pgfpathlineto{\pgfqpoint{0.464721in}{0.658302in}}%
\pgfpathlineto{\pgfqpoint{0.467892in}{0.655695in}}%
\pgfpathlineto{\pgfqpoint{0.483549in}{0.644933in}}%
\pgfpathlineto{\pgfqpoint{0.484025in}{0.644691in}}%
\pgfpathlineto{\pgfqpoint{0.499205in}{0.635642in}}%
\pgfpathlineto{\pgfqpoint{0.512271in}{0.631079in}}%
\pgfpathlineto{\pgfqpoint{0.514862in}{0.629780in}}%
\pgfpathclose%
\pgfpathmoveto{\pgfqpoint{0.512578in}{0.658302in}}%
\pgfpathlineto{\pgfqpoint{0.499205in}{0.662333in}}%
\pgfpathlineto{\pgfqpoint{0.483549in}{0.670404in}}%
\pgfpathlineto{\pgfqpoint{0.481365in}{0.671913in}}%
\pgfpathlineto{\pgfqpoint{0.467892in}{0.682001in}}%
\pgfpathlineto{\pgfqpoint{0.463941in}{0.685524in}}%
\pgfpathlineto{\pgfqpoint{0.452236in}{0.698266in}}%
\pgfpathlineto{\pgfqpoint{0.451482in}{0.699135in}}%
\pgfpathlineto{\pgfqpoint{0.442908in}{0.712746in}}%
\pgfpathlineto{\pgfqpoint{0.438239in}{0.726357in}}%
\pgfpathlineto{\pgfqpoint{0.437462in}{0.739968in}}%
\pgfpathlineto{\pgfqpoint{0.440572in}{0.753579in}}%
\pgfpathlineto{\pgfqpoint{0.447584in}{0.767191in}}%
\pgfpathlineto{\pgfqpoint{0.452236in}{0.773160in}}%
\pgfpathlineto{\pgfqpoint{0.458505in}{0.780802in}}%
\pgfpathlineto{\pgfqpoint{0.467892in}{0.789707in}}%
\pgfpathlineto{\pgfqpoint{0.473775in}{0.794413in}}%
\pgfpathlineto{\pgfqpoint{0.483549in}{0.801335in}}%
\pgfpathlineto{\pgfqpoint{0.496467in}{0.808024in}}%
\pgfpathlineto{\pgfqpoint{0.499205in}{0.809435in}}%
\pgfpathlineto{\pgfqpoint{0.514862in}{0.814304in}}%
\pgfpathlineto{\pgfqpoint{0.530519in}{0.815693in}}%
\pgfpathlineto{\pgfqpoint{0.546175in}{0.813610in}}%
\pgfpathlineto{\pgfqpoint{0.561832in}{0.808043in}}%
\pgfpathlineto{\pgfqpoint{0.561866in}{0.808024in}}%
\pgfpathlineto{\pgfqpoint{0.577488in}{0.799326in}}%
\pgfpathlineto{\pgfqpoint{0.584144in}{0.794413in}}%
\pgfpathlineto{\pgfqpoint{0.593145in}{0.786890in}}%
\pgfpathlineto{\pgfqpoint{0.599427in}{0.780802in}}%
\pgfpathlineto{\pgfqpoint{0.608801in}{0.769108in}}%
\pgfpathlineto{\pgfqpoint{0.610292in}{0.767191in}}%
\pgfpathlineto{\pgfqpoint{0.617386in}{0.753579in}}%
\pgfpathlineto{\pgfqpoint{0.620532in}{0.739968in}}%
\pgfpathlineto{\pgfqpoint{0.619746in}{0.726357in}}%
\pgfpathlineto{\pgfqpoint{0.615023in}{0.712746in}}%
\pgfpathlineto{\pgfqpoint{0.608801in}{0.702877in}}%
\pgfpathlineto{\pgfqpoint{0.606390in}{0.699135in}}%
\pgfpathlineto{\pgfqpoint{0.594037in}{0.685524in}}%
\pgfpathlineto{\pgfqpoint{0.593145in}{0.684711in}}%
\pgfpathlineto{\pgfqpoint{0.577488in}{0.672430in}}%
\pgfpathlineto{\pgfqpoint{0.576621in}{0.671913in}}%
\pgfpathlineto{\pgfqpoint{0.561832in}{0.663678in}}%
\pgfpathlineto{\pgfqpoint{0.546176in}{0.658302in}}%
\pgfpathlineto{\pgfqpoint{0.546175in}{0.658301in}}%
\pgfpathlineto{\pgfqpoint{0.530519in}{0.656123in}}%
\pgfpathlineto{\pgfqpoint{0.514862in}{0.657575in}}%
\pgfpathlineto{\pgfqpoint{0.512578in}{0.658302in}}%
\pgfpathclose%
\pgfpathmoveto{\pgfqpoint{0.827993in}{0.628865in}}%
\pgfpathlineto{\pgfqpoint{0.843650in}{0.627722in}}%
\pgfpathlineto{\pgfqpoint{0.855181in}{0.631079in}}%
\pgfpathlineto{\pgfqpoint{0.859306in}{0.631907in}}%
\pgfpathlineto{\pgfqpoint{0.874963in}{0.638913in}}%
\pgfpathlineto{\pgfqpoint{0.883457in}{0.644691in}}%
\pgfpathlineto{\pgfqpoint{0.890620in}{0.648873in}}%
\pgfpathlineto{\pgfqpoint{0.903134in}{0.658302in}}%
\pgfpathlineto{\pgfqpoint{0.906276in}{0.660626in}}%
\pgfpathlineto{\pgfqpoint{0.919581in}{0.671913in}}%
\pgfpathlineto{\pgfqpoint{0.921933in}{0.674104in}}%
\pgfpathlineto{\pgfqpoint{0.933857in}{0.685524in}}%
\pgfpathlineto{\pgfqpoint{0.937589in}{0.689911in}}%
\pgfpathlineto{\pgfqpoint{0.946211in}{0.699135in}}%
\pgfpathlineto{\pgfqpoint{0.953246in}{0.709756in}}%
\pgfpathlineto{\pgfqpoint{0.955814in}{0.712746in}}%
\pgfpathlineto{\pgfqpoint{0.962380in}{0.726357in}}%
\pgfpathlineto{\pgfqpoint{0.963472in}{0.739968in}}%
\pgfpathlineto{\pgfqpoint{0.959099in}{0.753579in}}%
\pgfpathlineto{\pgfqpoint{0.953246in}{0.761765in}}%
\pgfpathlineto{\pgfqpoint{0.950275in}{0.767191in}}%
\pgfpathlineto{\pgfqpoint{0.938911in}{0.780802in}}%
\pgfpathlineto{\pgfqpoint{0.937589in}{0.782050in}}%
\pgfpathlineto{\pgfqpoint{0.925589in}{0.794413in}}%
\pgfpathlineto{\pgfqpoint{0.921933in}{0.797696in}}%
\pgfpathlineto{\pgfqpoint{0.910053in}{0.808024in}}%
\pgfpathlineto{\pgfqpoint{0.906276in}{0.811203in}}%
\pgfpathlineto{\pgfqpoint{0.892056in}{0.821635in}}%
\pgfpathlineto{\pgfqpoint{0.890620in}{0.822784in}}%
\pgfpathlineto{\pgfqpoint{0.874963in}{0.832663in}}%
\pgfpathlineto{\pgfqpoint{0.868722in}{0.835246in}}%
\pgfpathlineto{\pgfqpoint{0.859306in}{0.840335in}}%
\pgfpathlineto{\pgfqpoint{0.843650in}{0.844136in}}%
\pgfpathlineto{\pgfqpoint{0.827993in}{0.843187in}}%
\pgfpathlineto{\pgfqpoint{0.812337in}{0.837479in}}%
\pgfpathlineto{\pgfqpoint{0.808898in}{0.835246in}}%
\pgfpathlineto{\pgfqpoint{0.796680in}{0.829130in}}%
\pgfpathlineto{\pgfqpoint{0.786071in}{0.821635in}}%
\pgfpathlineto{\pgfqpoint{0.781024in}{0.818390in}}%
\pgfpathlineto{\pgfqpoint{0.767887in}{0.808024in}}%
\pgfpathlineto{\pgfqpoint{0.765367in}{0.805979in}}%
\pgfpathlineto{\pgfqpoint{0.752384in}{0.794413in}}%
\pgfpathlineto{\pgfqpoint{0.749710in}{0.791681in}}%
\pgfpathlineto{\pgfqpoint{0.738865in}{0.780802in}}%
\pgfpathlineto{\pgfqpoint{0.734054in}{0.774575in}}%
\pgfpathlineto{\pgfqpoint{0.727409in}{0.767191in}}%
\pgfpathlineto{\pgfqpoint{0.719350in}{0.753579in}}%
\pgfpathlineto{\pgfqpoint{0.718397in}{0.749993in}}%
\pgfpathlineto{\pgfqpoint{0.714535in}{0.739968in}}%
\pgfpathlineto{\pgfqpoint{0.715850in}{0.726357in}}%
\pgfpathlineto{\pgfqpoint{0.718397in}{0.721905in}}%
\pgfpathlineto{\pgfqpoint{0.722034in}{0.712746in}}%
\pgfpathlineto{\pgfqpoint{0.731888in}{0.699135in}}%
\pgfpathlineto{\pgfqpoint{0.734054in}{0.696973in}}%
\pgfpathlineto{\pgfqpoint{0.743960in}{0.685524in}}%
\pgfpathlineto{\pgfqpoint{0.749710in}{0.680102in}}%
\pgfpathlineto{\pgfqpoint{0.758275in}{0.671913in}}%
\pgfpathlineto{\pgfqpoint{0.765367in}{0.665747in}}%
\pgfpathlineto{\pgfqpoint{0.774787in}{0.658302in}}%
\pgfpathlineto{\pgfqpoint{0.781024in}{0.653303in}}%
\pgfpathlineto{\pgfqpoint{0.794194in}{0.644691in}}%
\pgfpathlineto{\pgfqpoint{0.796680in}{0.642808in}}%
\pgfpathlineto{\pgfqpoint{0.812337in}{0.634241in}}%
\pgfpathlineto{\pgfqpoint{0.822873in}{0.631079in}}%
\pgfpathlineto{\pgfqpoint{0.827993in}{0.628865in}}%
\pgfpathclose%
\pgfpathmoveto{\pgfqpoint{0.823227in}{0.658302in}}%
\pgfpathlineto{\pgfqpoint{0.812337in}{0.661123in}}%
\pgfpathlineto{\pgfqpoint{0.796680in}{0.668522in}}%
\pgfpathlineto{\pgfqpoint{0.791542in}{0.671913in}}%
\pgfpathlineto{\pgfqpoint{0.781024in}{0.679415in}}%
\pgfpathlineto{\pgfqpoint{0.773997in}{0.685524in}}%
\pgfpathlineto{\pgfqpoint{0.765367in}{0.694668in}}%
\pgfpathlineto{\pgfqpoint{0.761466in}{0.699135in}}%
\pgfpathlineto{\pgfqpoint{0.752956in}{0.712746in}}%
\pgfpathlineto{\pgfqpoint{0.749710in}{0.722214in}}%
\pgfpathlineto{\pgfqpoint{0.748206in}{0.726357in}}%
\pgfpathlineto{\pgfqpoint{0.747371in}{0.739968in}}%
\pgfpathlineto{\pgfqpoint{0.749710in}{0.749537in}}%
\pgfpathlineto{\pgfqpoint{0.750637in}{0.753579in}}%
\pgfpathlineto{\pgfqpoint{0.757598in}{0.767191in}}%
\pgfpathlineto{\pgfqpoint{0.765367in}{0.777097in}}%
\pgfpathlineto{\pgfqpoint{0.768489in}{0.780802in}}%
\pgfpathlineto{\pgfqpoint{0.781024in}{0.792395in}}%
\pgfpathlineto{\pgfqpoint{0.783671in}{0.794413in}}%
\pgfpathlineto{\pgfqpoint{0.796680in}{0.803212in}}%
\pgfpathlineto{\pgfqpoint{0.806752in}{0.808024in}}%
\pgfpathlineto{\pgfqpoint{0.812337in}{0.810688in}}%
\pgfpathlineto{\pgfqpoint{0.827993in}{0.814860in}}%
\pgfpathlineto{\pgfqpoint{0.843650in}{0.815554in}}%
\pgfpathlineto{\pgfqpoint{0.859306in}{0.812775in}}%
\pgfpathlineto{\pgfqpoint{0.871272in}{0.808024in}}%
\pgfpathlineto{\pgfqpoint{0.874963in}{0.806566in}}%
\pgfpathlineto{\pgfqpoint{0.890620in}{0.797186in}}%
\pgfpathlineto{\pgfqpoint{0.894240in}{0.794413in}}%
\pgfpathlineto{\pgfqpoint{0.906276in}{0.783949in}}%
\pgfpathlineto{\pgfqpoint{0.909466in}{0.780802in}}%
\pgfpathlineto{\pgfqpoint{0.920256in}{0.767191in}}%
\pgfpathlineto{\pgfqpoint{0.921933in}{0.763982in}}%
\pgfpathlineto{\pgfqpoint{0.927398in}{0.753579in}}%
\pgfpathlineto{\pgfqpoint{0.930595in}{0.739968in}}%
\pgfpathlineto{\pgfqpoint{0.929796in}{0.726357in}}%
\pgfpathlineto{\pgfqpoint{0.924998in}{0.712746in}}%
\pgfpathlineto{\pgfqpoint{0.921933in}{0.707891in}}%
\pgfpathlineto{\pgfqpoint{0.916397in}{0.699135in}}%
\pgfpathlineto{\pgfqpoint{0.906276in}{0.687826in}}%
\pgfpathlineto{\pgfqpoint{0.903956in}{0.685524in}}%
\pgfpathlineto{\pgfqpoint{0.890620in}{0.674627in}}%
\pgfpathlineto{\pgfqpoint{0.886358in}{0.671913in}}%
\pgfpathlineto{\pgfqpoint{0.874963in}{0.665158in}}%
\pgfpathlineto{\pgfqpoint{0.859306in}{0.659107in}}%
\pgfpathlineto{\pgfqpoint{0.854656in}{0.658302in}}%
\pgfpathlineto{\pgfqpoint{0.843650in}{0.656268in}}%
\pgfpathlineto{\pgfqpoint{0.827993in}{0.656994in}}%
\pgfpathlineto{\pgfqpoint{0.823227in}{0.658302in}}%
\pgfpathclose%
\pgfpathmoveto{\pgfqpoint{1.141125in}{0.628179in}}%
\pgfpathlineto{\pgfqpoint{1.156781in}{0.628179in}}%
\pgfpathlineto{\pgfqpoint{1.164787in}{0.631079in}}%
\pgfpathlineto{\pgfqpoint{1.172438in}{0.632996in}}%
\pgfpathlineto{\pgfqpoint{1.188094in}{0.640783in}}%
\pgfpathlineto{\pgfqpoint{1.193524in}{0.644691in}}%
\pgfpathlineto{\pgfqpoint{1.203751in}{0.651028in}}%
\pgfpathlineto{\pgfqpoint{1.213093in}{0.658302in}}%
\pgfpathlineto{\pgfqpoint{1.219407in}{0.663140in}}%
\pgfpathlineto{\pgfqpoint{1.229603in}{0.671913in}}%
\pgfpathlineto{\pgfqpoint{1.235064in}{0.677077in}}%
\pgfpathlineto{\pgfqpoint{1.243933in}{0.685524in}}%
\pgfpathlineto{\pgfqpoint{1.250721in}{0.693454in}}%
\pgfpathlineto{\pgfqpoint{1.256201in}{0.699135in}}%
\pgfpathlineto{\pgfqpoint{1.265556in}{0.712746in}}%
\pgfpathlineto{\pgfqpoint{1.266377in}{0.714899in}}%
\pgfpathlineto{\pgfqpoint{1.272382in}{0.726357in}}%
\pgfpathlineto{\pgfqpoint{1.273574in}{0.739968in}}%
\pgfpathlineto{\pgfqpoint{1.268805in}{0.753579in}}%
\pgfpathlineto{\pgfqpoint{1.266377in}{0.756731in}}%
\pgfpathlineto{\pgfqpoint{1.260453in}{0.767191in}}%
\pgfpathlineto{\pgfqpoint{1.250721in}{0.778426in}}%
\pgfpathlineto{\pgfqpoint{1.248905in}{0.780802in}}%
\pgfpathlineto{\pgfqpoint{1.235498in}{0.794413in}}%
\pgfpathlineto{\pgfqpoint{1.235064in}{0.794801in}}%
\pgfpathlineto{\pgfqpoint{1.220082in}{0.808024in}}%
\pgfpathlineto{\pgfqpoint{1.219407in}{0.808600in}}%
\pgfpathlineto{\pgfqpoint{1.203751in}{0.820565in}}%
\pgfpathlineto{\pgfqpoint{1.201996in}{0.821635in}}%
\pgfpathlineto{\pgfqpoint{1.188094in}{0.830967in}}%
\pgfpathlineto{\pgfqpoint{1.178745in}{0.835246in}}%
\pgfpathlineto{\pgfqpoint{1.172438in}{0.839002in}}%
\pgfpathlineto{\pgfqpoint{1.156781in}{0.843757in}}%
\pgfpathlineto{\pgfqpoint{1.141125in}{0.843757in}}%
\pgfpathlineto{\pgfqpoint{1.125468in}{0.839002in}}%
\pgfpathlineto{\pgfqpoint{1.119161in}{0.835246in}}%
\pgfpathlineto{\pgfqpoint{1.109812in}{0.830967in}}%
\pgfpathlineto{\pgfqpoint{1.095910in}{0.821635in}}%
\pgfpathlineto{\pgfqpoint{1.094155in}{0.820565in}}%
\pgfpathlineto{\pgfqpoint{1.078498in}{0.808600in}}%
\pgfpathlineto{\pgfqpoint{1.077824in}{0.808024in}}%
\pgfpathlineto{\pgfqpoint{1.062842in}{0.794801in}}%
\pgfpathlineto{\pgfqpoint{1.062408in}{0.794413in}}%
\pgfpathlineto{\pgfqpoint{1.049001in}{0.780802in}}%
\pgfpathlineto{\pgfqpoint{1.047185in}{0.778426in}}%
\pgfpathlineto{\pgfqpoint{1.037452in}{0.767191in}}%
\pgfpathlineto{\pgfqpoint{1.031529in}{0.756731in}}%
\pgfpathlineto{\pgfqpoint{1.029101in}{0.753579in}}%
\pgfpathlineto{\pgfqpoint{1.024332in}{0.739968in}}%
\pgfpathlineto{\pgfqpoint{1.025524in}{0.726357in}}%
\pgfpathlineto{\pgfqpoint{1.031529in}{0.714899in}}%
\pgfpathlineto{\pgfqpoint{1.032349in}{0.712746in}}%
\pgfpathlineto{\pgfqpoint{1.041705in}{0.699135in}}%
\pgfpathlineto{\pgfqpoint{1.047185in}{0.693454in}}%
\pgfpathlineto{\pgfqpoint{1.053973in}{0.685524in}}%
\pgfpathlineto{\pgfqpoint{1.062842in}{0.677077in}}%
\pgfpathlineto{\pgfqpoint{1.068303in}{0.671913in}}%
\pgfpathlineto{\pgfqpoint{1.078498in}{0.663140in}}%
\pgfpathlineto{\pgfqpoint{1.084813in}{0.658302in}}%
\pgfpathlineto{\pgfqpoint{1.094155in}{0.651028in}}%
\pgfpathlineto{\pgfqpoint{1.104382in}{0.644691in}}%
\pgfpathlineto{\pgfqpoint{1.109812in}{0.640783in}}%
\pgfpathlineto{\pgfqpoint{1.125468in}{0.632996in}}%
\pgfpathlineto{\pgfqpoint{1.133119in}{0.631079in}}%
\pgfpathlineto{\pgfqpoint{1.141125in}{0.628179in}}%
\pgfpathclose%
\pgfpathmoveto{\pgfqpoint{1.133541in}{0.658302in}}%
\pgfpathlineto{\pgfqpoint{1.125468in}{0.660048in}}%
\pgfpathlineto{\pgfqpoint{1.109812in}{0.666773in}}%
\pgfpathlineto{\pgfqpoint{1.101617in}{0.671913in}}%
\pgfpathlineto{\pgfqpoint{1.094155in}{0.676956in}}%
\pgfpathlineto{\pgfqpoint{1.084009in}{0.685524in}}%
\pgfpathlineto{\pgfqpoint{1.078498in}{0.691186in}}%
\pgfpathlineto{\pgfqpoint{1.071484in}{0.699135in}}%
\pgfpathlineto{\pgfqpoint{1.063002in}{0.712746in}}%
\pgfpathlineto{\pgfqpoint{1.062842in}{0.713207in}}%
\pgfpathlineto{\pgfqpoint{1.058115in}{0.726357in}}%
\pgfpathlineto{\pgfqpoint{1.057301in}{0.739968in}}%
\pgfpathlineto{\pgfqpoint{1.060562in}{0.753579in}}%
\pgfpathlineto{\pgfqpoint{1.062842in}{0.757895in}}%
\pgfpathlineto{\pgfqpoint{1.067628in}{0.767191in}}%
\pgfpathlineto{\pgfqpoint{1.078410in}{0.780802in}}%
\pgfpathlineto{\pgfqpoint{1.078498in}{0.780890in}}%
\pgfpathlineto{\pgfqpoint{1.093517in}{0.794413in}}%
\pgfpathlineto{\pgfqpoint{1.094155in}{0.794918in}}%
\pgfpathlineto{\pgfqpoint{1.109812in}{0.804956in}}%
\pgfpathlineto{\pgfqpoint{1.116835in}{0.808024in}}%
\pgfpathlineto{\pgfqpoint{1.125468in}{0.811802in}}%
\pgfpathlineto{\pgfqpoint{1.141125in}{0.815276in}}%
\pgfpathlineto{\pgfqpoint{1.156781in}{0.815276in}}%
\pgfpathlineto{\pgfqpoint{1.172438in}{0.811802in}}%
\pgfpathlineto{\pgfqpoint{1.181071in}{0.808024in}}%
\pgfpathlineto{\pgfqpoint{1.188094in}{0.804956in}}%
\pgfpathlineto{\pgfqpoint{1.203751in}{0.794918in}}%
\pgfpathlineto{\pgfqpoint{1.204389in}{0.794413in}}%
\pgfpathlineto{\pgfqpoint{1.219407in}{0.780890in}}%
\pgfpathlineto{\pgfqpoint{1.219496in}{0.780802in}}%
\pgfpathlineto{\pgfqpoint{1.230278in}{0.767191in}}%
\pgfpathlineto{\pgfqpoint{1.235064in}{0.757895in}}%
\pgfpathlineto{\pgfqpoint{1.237344in}{0.753579in}}%
\pgfpathlineto{\pgfqpoint{1.240605in}{0.739968in}}%
\pgfpathlineto{\pgfqpoint{1.239790in}{0.726357in}}%
\pgfpathlineto{\pgfqpoint{1.235064in}{0.713207in}}%
\pgfpathlineto{\pgfqpoint{1.234904in}{0.712746in}}%
\pgfpathlineto{\pgfqpoint{1.226422in}{0.699135in}}%
\pgfpathlineto{\pgfqpoint{1.219407in}{0.691186in}}%
\pgfpathlineto{\pgfqpoint{1.213896in}{0.685524in}}%
\pgfpathlineto{\pgfqpoint{1.203751in}{0.676956in}}%
\pgfpathlineto{\pgfqpoint{1.196289in}{0.671913in}}%
\pgfpathlineto{\pgfqpoint{1.188094in}{0.666773in}}%
\pgfpathlineto{\pgfqpoint{1.172438in}{0.660048in}}%
\pgfpathlineto{\pgfqpoint{1.164365in}{0.658302in}}%
\pgfpathlineto{\pgfqpoint{1.156781in}{0.656558in}}%
\pgfpathlineto{\pgfqpoint{1.141125in}{0.656558in}}%
\pgfpathlineto{\pgfqpoint{1.133541in}{0.658302in}}%
\pgfpathclose%
\pgfpathmoveto{\pgfqpoint{1.454256in}{0.627722in}}%
\pgfpathlineto{\pgfqpoint{1.469913in}{0.628865in}}%
\pgfpathlineto{\pgfqpoint{1.475033in}{0.631079in}}%
\pgfpathlineto{\pgfqpoint{1.485569in}{0.634241in}}%
\pgfpathlineto{\pgfqpoint{1.501226in}{0.642808in}}%
\pgfpathlineto{\pgfqpoint{1.503712in}{0.644691in}}%
\pgfpathlineto{\pgfqpoint{1.516882in}{0.653303in}}%
\pgfpathlineto{\pgfqpoint{1.523119in}{0.658302in}}%
\pgfpathlineto{\pgfqpoint{1.532539in}{0.665747in}}%
\pgfpathlineto{\pgfqpoint{1.539631in}{0.671913in}}%
\pgfpathlineto{\pgfqpoint{1.548195in}{0.680102in}}%
\pgfpathlineto{\pgfqpoint{1.553946in}{0.685524in}}%
\pgfpathlineto{\pgfqpoint{1.563852in}{0.696973in}}%
\pgfpathlineto{\pgfqpoint{1.566018in}{0.699135in}}%
\pgfpathlineto{\pgfqpoint{1.575872in}{0.712746in}}%
\pgfpathlineto{\pgfqpoint{1.579508in}{0.721905in}}%
\pgfpathlineto{\pgfqpoint{1.582056in}{0.726357in}}%
\pgfpathlineto{\pgfqpoint{1.583371in}{0.739968in}}%
\pgfpathlineto{\pgfqpoint{1.579508in}{0.749993in}}%
\pgfpathlineto{\pgfqpoint{1.578556in}{0.753579in}}%
\pgfpathlineto{\pgfqpoint{1.570497in}{0.767191in}}%
\pgfpathlineto{\pgfqpoint{1.563852in}{0.774575in}}%
\pgfpathlineto{\pgfqpoint{1.559041in}{0.780802in}}%
\pgfpathlineto{\pgfqpoint{1.548195in}{0.791681in}}%
\pgfpathlineto{\pgfqpoint{1.545522in}{0.794413in}}%
\pgfpathlineto{\pgfqpoint{1.532539in}{0.805979in}}%
\pgfpathlineto{\pgfqpoint{1.530019in}{0.808024in}}%
\pgfpathlineto{\pgfqpoint{1.516882in}{0.818390in}}%
\pgfpathlineto{\pgfqpoint{1.511835in}{0.821635in}}%
\pgfpathlineto{\pgfqpoint{1.501226in}{0.829130in}}%
\pgfpathlineto{\pgfqpoint{1.489008in}{0.835246in}}%
\pgfpathlineto{\pgfqpoint{1.485569in}{0.837479in}}%
\pgfpathlineto{\pgfqpoint{1.469913in}{0.843187in}}%
\pgfpathlineto{\pgfqpoint{1.454256in}{0.844136in}}%
\pgfpathlineto{\pgfqpoint{1.438599in}{0.840335in}}%
\pgfpathlineto{\pgfqpoint{1.429184in}{0.835246in}}%
\pgfpathlineto{\pgfqpoint{1.422943in}{0.832663in}}%
\pgfpathlineto{\pgfqpoint{1.407286in}{0.822784in}}%
\pgfpathlineto{\pgfqpoint{1.405850in}{0.821635in}}%
\pgfpathlineto{\pgfqpoint{1.391630in}{0.811203in}}%
\pgfpathlineto{\pgfqpoint{1.387853in}{0.808024in}}%
\pgfpathlineto{\pgfqpoint{1.375973in}{0.797696in}}%
\pgfpathlineto{\pgfqpoint{1.372317in}{0.794413in}}%
\pgfpathlineto{\pgfqpoint{1.360317in}{0.782050in}}%
\pgfpathlineto{\pgfqpoint{1.358995in}{0.780802in}}%
\pgfpathlineto{\pgfqpoint{1.347631in}{0.767191in}}%
\pgfpathlineto{\pgfqpoint{1.344660in}{0.761765in}}%
\pgfpathlineto{\pgfqpoint{1.338807in}{0.753579in}}%
\pgfpathlineto{\pgfqpoint{1.334434in}{0.739968in}}%
\pgfpathlineto{\pgfqpoint{1.335526in}{0.726357in}}%
\pgfpathlineto{\pgfqpoint{1.342091in}{0.712746in}}%
\pgfpathlineto{\pgfqpoint{1.344660in}{0.709756in}}%
\pgfpathlineto{\pgfqpoint{1.351695in}{0.699135in}}%
\pgfpathlineto{\pgfqpoint{1.360317in}{0.689911in}}%
\pgfpathlineto{\pgfqpoint{1.364049in}{0.685524in}}%
\pgfpathlineto{\pgfqpoint{1.375973in}{0.674104in}}%
\pgfpathlineto{\pgfqpoint{1.378325in}{0.671913in}}%
\pgfpathlineto{\pgfqpoint{1.391630in}{0.660626in}}%
\pgfpathlineto{\pgfqpoint{1.394772in}{0.658302in}}%
\pgfpathlineto{\pgfqpoint{1.407286in}{0.648873in}}%
\pgfpathlineto{\pgfqpoint{1.414449in}{0.644691in}}%
\pgfpathlineto{\pgfqpoint{1.422943in}{0.638913in}}%
\pgfpathlineto{\pgfqpoint{1.438599in}{0.631907in}}%
\pgfpathlineto{\pgfqpoint{1.442725in}{0.631079in}}%
\pgfpathlineto{\pgfqpoint{1.454256in}{0.627722in}}%
\pgfpathclose%
\pgfpathmoveto{\pgfqpoint{1.443250in}{0.658302in}}%
\pgfpathlineto{\pgfqpoint{1.438599in}{0.659107in}}%
\pgfpathlineto{\pgfqpoint{1.422943in}{0.665158in}}%
\pgfpathlineto{\pgfqpoint{1.411548in}{0.671913in}}%
\pgfpathlineto{\pgfqpoint{1.407286in}{0.674627in}}%
\pgfpathlineto{\pgfqpoint{1.393950in}{0.685524in}}%
\pgfpathlineto{\pgfqpoint{1.391630in}{0.687826in}}%
\pgfpathlineto{\pgfqpoint{1.381508in}{0.699135in}}%
\pgfpathlineto{\pgfqpoint{1.375973in}{0.707891in}}%
\pgfpathlineto{\pgfqpoint{1.372908in}{0.712746in}}%
\pgfpathlineto{\pgfqpoint{1.368110in}{0.726357in}}%
\pgfpathlineto{\pgfqpoint{1.367311in}{0.739968in}}%
\pgfpathlineto{\pgfqpoint{1.370508in}{0.753579in}}%
\pgfpathlineto{\pgfqpoint{1.375973in}{0.763982in}}%
\pgfpathlineto{\pgfqpoint{1.377650in}{0.767191in}}%
\pgfpathlineto{\pgfqpoint{1.388439in}{0.780802in}}%
\pgfpathlineto{\pgfqpoint{1.391630in}{0.783949in}}%
\pgfpathlineto{\pgfqpoint{1.403666in}{0.794413in}}%
\pgfpathlineto{\pgfqpoint{1.407286in}{0.797186in}}%
\pgfpathlineto{\pgfqpoint{1.422943in}{0.806566in}}%
\pgfpathlineto{\pgfqpoint{1.426633in}{0.808024in}}%
\pgfpathlineto{\pgfqpoint{1.438599in}{0.812775in}}%
\pgfpathlineto{\pgfqpoint{1.454256in}{0.815554in}}%
\pgfpathlineto{\pgfqpoint{1.469913in}{0.814860in}}%
\pgfpathlineto{\pgfqpoint{1.485569in}{0.810688in}}%
\pgfpathlineto{\pgfqpoint{1.491154in}{0.808024in}}%
\pgfpathlineto{\pgfqpoint{1.501226in}{0.803212in}}%
\pgfpathlineto{\pgfqpoint{1.514235in}{0.794413in}}%
\pgfpathlineto{\pgfqpoint{1.516882in}{0.792395in}}%
\pgfpathlineto{\pgfqpoint{1.529417in}{0.780802in}}%
\pgfpathlineto{\pgfqpoint{1.532539in}{0.777097in}}%
\pgfpathlineto{\pgfqpoint{1.540308in}{0.767191in}}%
\pgfpathlineto{\pgfqpoint{1.547269in}{0.753579in}}%
\pgfpathlineto{\pgfqpoint{1.548195in}{0.749537in}}%
\pgfpathlineto{\pgfqpoint{1.550535in}{0.739968in}}%
\pgfpathlineto{\pgfqpoint{1.549700in}{0.726357in}}%
\pgfpathlineto{\pgfqpoint{1.548195in}{0.722214in}}%
\pgfpathlineto{\pgfqpoint{1.544950in}{0.712746in}}%
\pgfpathlineto{\pgfqpoint{1.536439in}{0.699135in}}%
\pgfpathlineto{\pgfqpoint{1.532539in}{0.694668in}}%
\pgfpathlineto{\pgfqpoint{1.523909in}{0.685524in}}%
\pgfpathlineto{\pgfqpoint{1.516882in}{0.679415in}}%
\pgfpathlineto{\pgfqpoint{1.506363in}{0.671913in}}%
\pgfpathlineto{\pgfqpoint{1.501226in}{0.668522in}}%
\pgfpathlineto{\pgfqpoint{1.485569in}{0.661123in}}%
\pgfpathlineto{\pgfqpoint{1.474679in}{0.658302in}}%
\pgfpathlineto{\pgfqpoint{1.469913in}{0.656994in}}%
\pgfpathlineto{\pgfqpoint{1.454256in}{0.656268in}}%
\pgfpathlineto{\pgfqpoint{1.443250in}{0.658302in}}%
\pgfpathclose%
\pgfpathmoveto{\pgfqpoint{1.751731in}{0.630925in}}%
\pgfpathlineto{\pgfqpoint{1.767387in}{0.627493in}}%
\pgfpathlineto{\pgfqpoint{1.783044in}{0.629780in}}%
\pgfpathlineto{\pgfqpoint{1.785635in}{0.631079in}}%
\pgfpathlineto{\pgfqpoint{1.798700in}{0.635642in}}%
\pgfpathlineto{\pgfqpoint{1.813881in}{0.644691in}}%
\pgfpathlineto{\pgfqpoint{1.814357in}{0.644933in}}%
\pgfpathlineto{\pgfqpoint{1.830014in}{0.655695in}}%
\pgfpathlineto{\pgfqpoint{1.833185in}{0.658302in}}%
\pgfpathlineto{\pgfqpoint{1.845670in}{0.668439in}}%
\pgfpathlineto{\pgfqpoint{1.849639in}{0.671913in}}%
\pgfpathlineto{\pgfqpoint{1.861327in}{0.683168in}}%
\pgfpathlineto{\pgfqpoint{1.863863in}{0.685524in}}%
\pgfpathlineto{\pgfqpoint{1.875889in}{0.699135in}}%
\pgfpathlineto{\pgfqpoint{1.876983in}{0.700870in}}%
\pgfpathlineto{\pgfqpoint{1.886057in}{0.712746in}}%
\pgfpathlineto{\pgfqpoint{1.891749in}{0.726357in}}%
\pgfpathlineto{\pgfqpoint{1.892640in}{0.739161in}}%
\pgfpathlineto{\pgfqpoint{1.892727in}{0.739968in}}%
\pgfpathlineto{\pgfqpoint{1.892640in}{0.740172in}}%
\pgfpathlineto{\pgfqpoint{1.888904in}{0.753579in}}%
\pgfpathlineto{\pgfqpoint{1.880356in}{0.767191in}}%
\pgfpathlineto{\pgfqpoint{1.876983in}{0.770772in}}%
\pgfpathlineto{\pgfqpoint{1.869109in}{0.780802in}}%
\pgfpathlineto{\pgfqpoint{1.861327in}{0.788494in}}%
\pgfpathlineto{\pgfqpoint{1.855574in}{0.794413in}}%
\pgfpathlineto{\pgfqpoint{1.845670in}{0.803294in}}%
\pgfpathlineto{\pgfqpoint{1.839996in}{0.808024in}}%
\pgfpathlineto{\pgfqpoint{1.830014in}{0.816102in}}%
\pgfpathlineto{\pgfqpoint{1.821817in}{0.821635in}}%
\pgfpathlineto{\pgfqpoint{1.814357in}{0.827153in}}%
\pgfpathlineto{\pgfqpoint{1.799440in}{0.835246in}}%
\pgfpathlineto{\pgfqpoint{1.798700in}{0.835765in}}%
\pgfpathlineto{\pgfqpoint{1.783044in}{0.842427in}}%
\pgfpathlineto{\pgfqpoint{1.767387in}{0.844326in}}%
\pgfpathlineto{\pgfqpoint{1.751731in}{0.841476in}}%
\pgfpathlineto{\pgfqpoint{1.738858in}{0.835246in}}%
\pgfpathlineto{\pgfqpoint{1.736074in}{0.834218in}}%
\pgfpathlineto{\pgfqpoint{1.720418in}{0.825037in}}%
\pgfpathlineto{\pgfqpoint{1.716004in}{0.821635in}}%
\pgfpathlineto{\pgfqpoint{1.704761in}{0.813705in}}%
\pgfpathlineto{\pgfqpoint{1.697890in}{0.808024in}}%
\pgfpathlineto{\pgfqpoint{1.689104in}{0.800530in}}%
\pgfpathlineto{\pgfqpoint{1.682301in}{0.794413in}}%
\pgfpathlineto{\pgfqpoint{1.673448in}{0.785279in}}%
\pgfpathlineto{\pgfqpoint{1.668827in}{0.780802in}}%
\pgfpathlineto{\pgfqpoint{1.657906in}{0.767191in}}%
\pgfpathlineto{\pgfqpoint{1.657791in}{0.766976in}}%
\pgfpathlineto{\pgfqpoint{1.648801in}{0.753579in}}%
\pgfpathlineto{\pgfqpoint{1.644747in}{0.739968in}}%
\pgfpathlineto{\pgfqpoint{1.645760in}{0.726357in}}%
\pgfpathlineto{\pgfqpoint{1.651845in}{0.712746in}}%
\pgfpathlineto{\pgfqpoint{1.657791in}{0.705372in}}%
\pgfpathlineto{\pgfqpoint{1.661811in}{0.699135in}}%
\pgfpathlineto{\pgfqpoint{1.673448in}{0.686365in}}%
\pgfpathlineto{\pgfqpoint{1.674163in}{0.685524in}}%
\pgfpathlineto{\pgfqpoint{1.688304in}{0.671913in}}%
\pgfpathlineto{\pgfqpoint{1.689104in}{0.671210in}}%
\pgfpathlineto{\pgfqpoint{1.704642in}{0.658302in}}%
\pgfpathlineto{\pgfqpoint{1.704761in}{0.658201in}}%
\pgfpathlineto{\pgfqpoint{1.720418in}{0.646841in}}%
\pgfpathlineto{\pgfqpoint{1.724349in}{0.644691in}}%
\pgfpathlineto{\pgfqpoint{1.736074in}{0.637200in}}%
\pgfpathlineto{\pgfqpoint{1.751459in}{0.631079in}}%
\pgfpathlineto{\pgfqpoint{1.751731in}{0.630925in}}%
\pgfpathclose%
\pgfpathmoveto{\pgfqpoint{1.751730in}{0.658302in}}%
\pgfpathlineto{\pgfqpoint{1.736074in}{0.663678in}}%
\pgfpathlineto{\pgfqpoint{1.721285in}{0.671913in}}%
\pgfpathlineto{\pgfqpoint{1.720418in}{0.672430in}}%
\pgfpathlineto{\pgfqpoint{1.704761in}{0.684711in}}%
\pgfpathlineto{\pgfqpoint{1.703869in}{0.685524in}}%
\pgfpathlineto{\pgfqpoint{1.691516in}{0.699135in}}%
\pgfpathlineto{\pgfqpoint{1.689104in}{0.702877in}}%
\pgfpathlineto{\pgfqpoint{1.682883in}{0.712746in}}%
\pgfpathlineto{\pgfqpoint{1.678160in}{0.726357in}}%
\pgfpathlineto{\pgfqpoint{1.677374in}{0.739968in}}%
\pgfpathlineto{\pgfqpoint{1.680520in}{0.753579in}}%
\pgfpathlineto{\pgfqpoint{1.687614in}{0.767191in}}%
\pgfpathlineto{\pgfqpoint{1.689104in}{0.769108in}}%
\pgfpathlineto{\pgfqpoint{1.698479in}{0.780802in}}%
\pgfpathlineto{\pgfqpoint{1.704761in}{0.786890in}}%
\pgfpathlineto{\pgfqpoint{1.713762in}{0.794413in}}%
\pgfpathlineto{\pgfqpoint{1.720418in}{0.799326in}}%
\pgfpathlineto{\pgfqpoint{1.736040in}{0.808024in}}%
\pgfpathlineto{\pgfqpoint{1.736074in}{0.808043in}}%
\pgfpathlineto{\pgfqpoint{1.751731in}{0.813610in}}%
\pgfpathlineto{\pgfqpoint{1.767387in}{0.815693in}}%
\pgfpathlineto{\pgfqpoint{1.783044in}{0.814304in}}%
\pgfpathlineto{\pgfqpoint{1.798700in}{0.809435in}}%
\pgfpathlineto{\pgfqpoint{1.801439in}{0.808024in}}%
\pgfpathlineto{\pgfqpoint{1.814357in}{0.801335in}}%
\pgfpathlineto{\pgfqpoint{1.824131in}{0.794413in}}%
\pgfpathlineto{\pgfqpoint{1.830014in}{0.789707in}}%
\pgfpathlineto{\pgfqpoint{1.839401in}{0.780802in}}%
\pgfpathlineto{\pgfqpoint{1.845670in}{0.773160in}}%
\pgfpathlineto{\pgfqpoint{1.850321in}{0.767191in}}%
\pgfpathlineto{\pgfqpoint{1.857334in}{0.753579in}}%
\pgfpathlineto{\pgfqpoint{1.860444in}{0.739968in}}%
\pgfpathlineto{\pgfqpoint{1.859667in}{0.726357in}}%
\pgfpathlineto{\pgfqpoint{1.854998in}{0.712746in}}%
\pgfpathlineto{\pgfqpoint{1.846424in}{0.699135in}}%
\pgfpathlineto{\pgfqpoint{1.845670in}{0.698266in}}%
\pgfpathlineto{\pgfqpoint{1.833964in}{0.685524in}}%
\pgfpathlineto{\pgfqpoint{1.830014in}{0.682001in}}%
\pgfpathlineto{\pgfqpoint{1.816540in}{0.671913in}}%
\pgfpathlineto{\pgfqpoint{1.814357in}{0.670404in}}%
\pgfpathlineto{\pgfqpoint{1.798700in}{0.662333in}}%
\pgfpathlineto{\pgfqpoint{1.785328in}{0.658302in}}%
\pgfpathlineto{\pgfqpoint{1.783044in}{0.657575in}}%
\pgfpathlineto{\pgfqpoint{1.767387in}{0.656123in}}%
\pgfpathlineto{\pgfqpoint{1.751731in}{0.658301in}}%
\pgfpathlineto{\pgfqpoint{1.751730in}{0.658302in}}%
\pgfpathclose%
\pgfpathmoveto{\pgfqpoint{0.514862in}{0.898910in}}%
\pgfpathlineto{\pgfqpoint{0.530519in}{0.896838in}}%
\pgfpathlineto{\pgfqpoint{0.546175in}{0.899946in}}%
\pgfpathlineto{\pgfqpoint{0.552610in}{0.903302in}}%
\pgfpathlineto{\pgfqpoint{0.561832in}{0.906824in}}%
\pgfpathlineto{\pgfqpoint{0.577488in}{0.916432in}}%
\pgfpathlineto{\pgfqpoint{0.578093in}{0.916913in}}%
\pgfpathlineto{\pgfqpoint{0.593145in}{0.927593in}}%
\pgfpathlineto{\pgfqpoint{0.596669in}{0.930524in}}%
\pgfpathlineto{\pgfqpoint{0.608801in}{0.940716in}}%
\pgfpathlineto{\pgfqpoint{0.612657in}{0.944135in}}%
\pgfpathlineto{\pgfqpoint{0.624458in}{0.955890in}}%
\pgfpathlineto{\pgfqpoint{0.626437in}{0.957746in}}%
\pgfpathlineto{\pgfqpoint{0.638125in}{0.971357in}}%
\pgfpathlineto{\pgfqpoint{0.640115in}{0.974733in}}%
\pgfpathlineto{\pgfqpoint{0.647685in}{0.984968in}}%
\pgfpathlineto{\pgfqpoint{0.652753in}{0.998579in}}%
\pgfpathlineto{\pgfqpoint{0.652753in}{1.012191in}}%
\pgfpathlineto{\pgfqpoint{0.647685in}{1.025802in}}%
\pgfpathlineto{\pgfqpoint{0.640115in}{1.036037in}}%
\pgfpathlineto{\pgfqpoint{0.638125in}{1.039413in}}%
\pgfpathlineto{\pgfqpoint{0.626437in}{1.053024in}}%
\pgfpathlineto{\pgfqpoint{0.624458in}{1.054880in}}%
\pgfpathlineto{\pgfqpoint{0.612657in}{1.066635in}}%
\pgfpathlineto{\pgfqpoint{0.608801in}{1.070054in}}%
\pgfpathlineto{\pgfqpoint{0.596669in}{1.080246in}}%
\pgfpathlineto{\pgfqpoint{0.593145in}{1.083177in}}%
\pgfpathlineto{\pgfqpoint{0.578093in}{1.093857in}}%
\pgfpathlineto{\pgfqpoint{0.577488in}{1.094338in}}%
\pgfpathlineto{\pgfqpoint{0.561832in}{1.103946in}}%
\pgfpathlineto{\pgfqpoint{0.552610in}{1.107468in}}%
\pgfpathlineto{\pgfqpoint{0.546175in}{1.110824in}}%
\pgfpathlineto{\pgfqpoint{0.530519in}{1.113932in}}%
\pgfpathlineto{\pgfqpoint{0.514862in}{1.111860in}}%
\pgfpathlineto{\pgfqpoint{0.505301in}{1.107468in}}%
\pgfpathlineto{\pgfqpoint{0.499205in}{1.105424in}}%
\pgfpathlineto{\pgfqpoint{0.483549in}{1.096552in}}%
\pgfpathlineto{\pgfqpoint{0.480018in}{1.093857in}}%
\pgfpathlineto{\pgfqpoint{0.467892in}{1.085622in}}%
\pgfpathlineto{\pgfqpoint{0.461286in}{1.080246in}}%
\pgfpathlineto{\pgfqpoint{0.452236in}{1.072815in}}%
\pgfpathlineto{\pgfqpoint{0.445246in}{1.066635in}}%
\pgfpathlineto{\pgfqpoint{0.436579in}{1.058026in}}%
\pgfpathlineto{\pgfqpoint{0.431349in}{1.053024in}}%
\pgfpathlineto{\pgfqpoint{0.420923in}{1.040509in}}%
\pgfpathlineto{\pgfqpoint{0.419831in}{1.039413in}}%
\pgfpathlineto{\pgfqpoint{0.410330in}{1.025802in}}%
\pgfpathlineto{\pgfqpoint{0.405589in}{1.012191in}}%
\pgfpathlineto{\pgfqpoint{0.405589in}{0.998579in}}%
\pgfpathlineto{\pgfqpoint{0.410330in}{0.984968in}}%
\pgfpathlineto{\pgfqpoint{0.419831in}{0.971357in}}%
\pgfpathlineto{\pgfqpoint{0.420923in}{0.970261in}}%
\pgfpathlineto{\pgfqpoint{0.431349in}{0.957746in}}%
\pgfpathlineto{\pgfqpoint{0.436579in}{0.952744in}}%
\pgfpathlineto{\pgfqpoint{0.445246in}{0.944135in}}%
\pgfpathlineto{\pgfqpoint{0.452236in}{0.937955in}}%
\pgfpathlineto{\pgfqpoint{0.461286in}{0.930524in}}%
\pgfpathlineto{\pgfqpoint{0.467892in}{0.925148in}}%
\pgfpathlineto{\pgfqpoint{0.480018in}{0.916913in}}%
\pgfpathlineto{\pgfqpoint{0.483549in}{0.914218in}}%
\pgfpathlineto{\pgfqpoint{0.499205in}{0.905346in}}%
\pgfpathlineto{\pgfqpoint{0.505301in}{0.903302in}}%
\pgfpathlineto{\pgfqpoint{0.514862in}{0.898910in}}%
\pgfpathclose%
\pgfpathmoveto{\pgfqpoint{0.503618in}{0.930524in}}%
\pgfpathlineto{\pgfqpoint{0.499205in}{0.931869in}}%
\pgfpathlineto{\pgfqpoint{0.483549in}{0.939912in}}%
\pgfpathlineto{\pgfqpoint{0.477502in}{0.944135in}}%
\pgfpathlineto{\pgfqpoint{0.467892in}{0.951557in}}%
\pgfpathlineto{\pgfqpoint{0.461149in}{0.957746in}}%
\pgfpathlineto{\pgfqpoint{0.452236in}{0.967985in}}%
\pgfpathlineto{\pgfqpoint{0.449456in}{0.971357in}}%
\pgfpathlineto{\pgfqpoint{0.441662in}{0.984968in}}%
\pgfpathlineto{\pgfqpoint{0.437772in}{0.998579in}}%
\pgfpathlineto{\pgfqpoint{0.437772in}{1.012191in}}%
\pgfpathlineto{\pgfqpoint{0.441662in}{1.025802in}}%
\pgfpathlineto{\pgfqpoint{0.449456in}{1.039413in}}%
\pgfpathlineto{\pgfqpoint{0.452236in}{1.042785in}}%
\pgfpathlineto{\pgfqpoint{0.461149in}{1.053024in}}%
\pgfpathlineto{\pgfqpoint{0.467892in}{1.059213in}}%
\pgfpathlineto{\pgfqpoint{0.477502in}{1.066635in}}%
\pgfpathlineto{\pgfqpoint{0.483549in}{1.070858in}}%
\pgfpathlineto{\pgfqpoint{0.499205in}{1.078901in}}%
\pgfpathlineto{\pgfqpoint{0.503618in}{1.080246in}}%
\pgfpathlineto{\pgfqpoint{0.514862in}{1.083788in}}%
\pgfpathlineto{\pgfqpoint{0.530519in}{1.085205in}}%
\pgfpathlineto{\pgfqpoint{0.546175in}{1.083079in}}%
\pgfpathlineto{\pgfqpoint{0.554098in}{1.080246in}}%
\pgfpathlineto{\pgfqpoint{0.561832in}{1.077560in}}%
\pgfpathlineto{\pgfqpoint{0.577488in}{1.068850in}}%
\pgfpathlineto{\pgfqpoint{0.580531in}{1.066635in}}%
\pgfpathlineto{\pgfqpoint{0.593145in}{1.056456in}}%
\pgfpathlineto{\pgfqpoint{0.596805in}{1.053024in}}%
\pgfpathlineto{\pgfqpoint{0.608406in}{1.039413in}}%
\pgfpathlineto{\pgfqpoint{0.608801in}{1.038741in}}%
\pgfpathlineto{\pgfqpoint{0.616283in}{1.025802in}}%
\pgfpathlineto{\pgfqpoint{0.620218in}{1.012191in}}%
\pgfpathlineto{\pgfqpoint{0.620218in}{0.998579in}}%
\pgfpathlineto{\pgfqpoint{0.616283in}{0.984968in}}%
\pgfpathlineto{\pgfqpoint{0.608801in}{0.972029in}}%
\pgfpathlineto{\pgfqpoint{0.608406in}{0.971357in}}%
\pgfpathlineto{\pgfqpoint{0.596805in}{0.957746in}}%
\pgfpathlineto{\pgfqpoint{0.593145in}{0.954314in}}%
\pgfpathlineto{\pgfqpoint{0.580531in}{0.944135in}}%
\pgfpathlineto{\pgfqpoint{0.577488in}{0.941920in}}%
\pgfpathlineto{\pgfqpoint{0.561832in}{0.933210in}}%
\pgfpathlineto{\pgfqpoint{0.554098in}{0.930524in}}%
\pgfpathlineto{\pgfqpoint{0.546175in}{0.927691in}}%
\pgfpathlineto{\pgfqpoint{0.530519in}{0.925565in}}%
\pgfpathlineto{\pgfqpoint{0.514862in}{0.926982in}}%
\pgfpathlineto{\pgfqpoint{0.503618in}{0.930524in}}%
\pgfpathclose%
\pgfpathmoveto{\pgfqpoint{0.827993in}{0.898081in}}%
\pgfpathlineto{\pgfqpoint{0.843650in}{0.897045in}}%
\pgfpathlineto{\pgfqpoint{0.859306in}{0.901191in}}%
\pgfpathlineto{\pgfqpoint{0.862932in}{0.903302in}}%
\pgfpathlineto{\pgfqpoint{0.874963in}{0.908452in}}%
\pgfpathlineto{\pgfqpoint{0.887887in}{0.916913in}}%
\pgfpathlineto{\pgfqpoint{0.890620in}{0.918491in}}%
\pgfpathlineto{\pgfqpoint{0.906276in}{0.930146in}}%
\pgfpathlineto{\pgfqpoint{0.906722in}{0.930524in}}%
\pgfpathlineto{\pgfqpoint{0.921933in}{0.943549in}}%
\pgfpathlineto{\pgfqpoint{0.922595in}{0.944135in}}%
\pgfpathlineto{\pgfqpoint{0.936359in}{0.957746in}}%
\pgfpathlineto{\pgfqpoint{0.937589in}{0.959272in}}%
\pgfpathlineto{\pgfqpoint{0.948324in}{0.971357in}}%
\pgfpathlineto{\pgfqpoint{0.953246in}{0.979485in}}%
\pgfpathlineto{\pgfqpoint{0.957567in}{0.984968in}}%
\pgfpathlineto{\pgfqpoint{0.963035in}{0.998579in}}%
\pgfpathlineto{\pgfqpoint{0.963035in}{1.012191in}}%
\pgfpathlineto{\pgfqpoint{0.957567in}{1.025802in}}%
\pgfpathlineto{\pgfqpoint{0.953246in}{1.031285in}}%
\pgfpathlineto{\pgfqpoint{0.948324in}{1.039413in}}%
\pgfpathlineto{\pgfqpoint{0.937589in}{1.051498in}}%
\pgfpathlineto{\pgfqpoint{0.936359in}{1.053024in}}%
\pgfpathlineto{\pgfqpoint{0.922595in}{1.066635in}}%
\pgfpathlineto{\pgfqpoint{0.921933in}{1.067221in}}%
\pgfpathlineto{\pgfqpoint{0.906722in}{1.080246in}}%
\pgfpathlineto{\pgfqpoint{0.906276in}{1.080624in}}%
\pgfpathlineto{\pgfqpoint{0.890620in}{1.092279in}}%
\pgfpathlineto{\pgfqpoint{0.887887in}{1.093857in}}%
\pgfpathlineto{\pgfqpoint{0.874963in}{1.102318in}}%
\pgfpathlineto{\pgfqpoint{0.862932in}{1.107468in}}%
\pgfpathlineto{\pgfqpoint{0.859306in}{1.109579in}}%
\pgfpathlineto{\pgfqpoint{0.843650in}{1.113725in}}%
\pgfpathlineto{\pgfqpoint{0.827993in}{1.112689in}}%
\pgfpathlineto{\pgfqpoint{0.814813in}{1.107468in}}%
\pgfpathlineto{\pgfqpoint{0.812337in}{1.106755in}}%
\pgfpathlineto{\pgfqpoint{0.796680in}{1.098621in}}%
\pgfpathlineto{\pgfqpoint{0.790145in}{1.093857in}}%
\pgfpathlineto{\pgfqpoint{0.781024in}{1.087957in}}%
\pgfpathlineto{\pgfqpoint{0.771307in}{1.080246in}}%
\pgfpathlineto{\pgfqpoint{0.765367in}{1.075499in}}%
\pgfpathlineto{\pgfqpoint{0.755276in}{1.066635in}}%
\pgfpathlineto{\pgfqpoint{0.749710in}{1.061145in}}%
\pgfpathlineto{\pgfqpoint{0.741343in}{1.053024in}}%
\pgfpathlineto{\pgfqpoint{0.734054in}{1.044133in}}%
\pgfpathlineto{\pgfqpoint{0.729559in}{1.039413in}}%
\pgfpathlineto{\pgfqpoint{0.720602in}{1.025802in}}%
\pgfpathlineto{\pgfqpoint{0.718397in}{1.019151in}}%
\pgfpathlineto{\pgfqpoint{0.715061in}{1.012191in}}%
\pgfpathlineto{\pgfqpoint{0.715061in}{0.998579in}}%
\pgfpathlineto{\pgfqpoint{0.718397in}{0.991619in}}%
\pgfpathlineto{\pgfqpoint{0.720602in}{0.984968in}}%
\pgfpathlineto{\pgfqpoint{0.729559in}{0.971357in}}%
\pgfpathlineto{\pgfqpoint{0.734054in}{0.966637in}}%
\pgfpathlineto{\pgfqpoint{0.741343in}{0.957746in}}%
\pgfpathlineto{\pgfqpoint{0.749710in}{0.949625in}}%
\pgfpathlineto{\pgfqpoint{0.755276in}{0.944135in}}%
\pgfpathlineto{\pgfqpoint{0.765367in}{0.935271in}}%
\pgfpathlineto{\pgfqpoint{0.771307in}{0.930524in}}%
\pgfpathlineto{\pgfqpoint{0.781024in}{0.922813in}}%
\pgfpathlineto{\pgfqpoint{0.790145in}{0.916913in}}%
\pgfpathlineto{\pgfqpoint{0.796680in}{0.912149in}}%
\pgfpathlineto{\pgfqpoint{0.812337in}{0.904015in}}%
\pgfpathlineto{\pgfqpoint{0.814813in}{0.903302in}}%
\pgfpathlineto{\pgfqpoint{0.827993in}{0.898081in}}%
\pgfpathclose%
\pgfpathmoveto{\pgfqpoint{0.812867in}{0.930524in}}%
\pgfpathlineto{\pgfqpoint{0.812337in}{0.930663in}}%
\pgfpathlineto{\pgfqpoint{0.796680in}{0.938037in}}%
\pgfpathlineto{\pgfqpoint{0.787536in}{0.944135in}}%
\pgfpathlineto{\pgfqpoint{0.781024in}{0.948926in}}%
\pgfpathlineto{\pgfqpoint{0.771168in}{0.957746in}}%
\pgfpathlineto{\pgfqpoint{0.765367in}{0.964233in}}%
\pgfpathlineto{\pgfqpoint{0.759455in}{0.971357in}}%
\pgfpathlineto{\pgfqpoint{0.751719in}{0.984968in}}%
\pgfpathlineto{\pgfqpoint{0.749710in}{0.991986in}}%
\pgfpathlineto{\pgfqpoint{0.747705in}{0.998579in}}%
\pgfpathlineto{\pgfqpoint{0.747705in}{1.012191in}}%
\pgfpathlineto{\pgfqpoint{0.749710in}{1.018784in}}%
\pgfpathlineto{\pgfqpoint{0.751719in}{1.025802in}}%
\pgfpathlineto{\pgfqpoint{0.759455in}{1.039413in}}%
\pgfpathlineto{\pgfqpoint{0.765367in}{1.046537in}}%
\pgfpathlineto{\pgfqpoint{0.771168in}{1.053024in}}%
\pgfpathlineto{\pgfqpoint{0.781024in}{1.061844in}}%
\pgfpathlineto{\pgfqpoint{0.787536in}{1.066635in}}%
\pgfpathlineto{\pgfqpoint{0.796680in}{1.072733in}}%
\pgfpathlineto{\pgfqpoint{0.812337in}{1.080107in}}%
\pgfpathlineto{\pgfqpoint{0.812867in}{1.080246in}}%
\pgfpathlineto{\pgfqpoint{0.827993in}{1.084355in}}%
\pgfpathlineto{\pgfqpoint{0.843650in}{1.085063in}}%
\pgfpathlineto{\pgfqpoint{0.859306in}{1.082228in}}%
\pgfpathlineto{\pgfqpoint{0.864270in}{1.080246in}}%
\pgfpathlineto{\pgfqpoint{0.874963in}{1.076085in}}%
\pgfpathlineto{\pgfqpoint{0.890620in}{1.066712in}}%
\pgfpathlineto{\pgfqpoint{0.890721in}{1.066635in}}%
\pgfpathlineto{\pgfqpoint{0.906276in}{1.053579in}}%
\pgfpathlineto{\pgfqpoint{0.906857in}{1.053024in}}%
\pgfpathlineto{\pgfqpoint{0.918403in}{1.039413in}}%
\pgfpathlineto{\pgfqpoint{0.921933in}{1.033307in}}%
\pgfpathlineto{\pgfqpoint{0.926278in}{1.025802in}}%
\pgfpathlineto{\pgfqpoint{0.930275in}{1.012191in}}%
\pgfpathlineto{\pgfqpoint{0.930275in}{0.998579in}}%
\pgfpathlineto{\pgfqpoint{0.926278in}{0.984968in}}%
\pgfpathlineto{\pgfqpoint{0.921933in}{0.977463in}}%
\pgfpathlineto{\pgfqpoint{0.918403in}{0.971357in}}%
\pgfpathlineto{\pgfqpoint{0.906857in}{0.957746in}}%
\pgfpathlineto{\pgfqpoint{0.906276in}{0.957191in}}%
\pgfpathlineto{\pgfqpoint{0.890721in}{0.944135in}}%
\pgfpathlineto{\pgfqpoint{0.890620in}{0.944058in}}%
\pgfpathlineto{\pgfqpoint{0.874963in}{0.934685in}}%
\pgfpathlineto{\pgfqpoint{0.864270in}{0.930524in}}%
\pgfpathlineto{\pgfqpoint{0.859306in}{0.928542in}}%
\pgfpathlineto{\pgfqpoint{0.843650in}{0.925707in}}%
\pgfpathlineto{\pgfqpoint{0.827993in}{0.926415in}}%
\pgfpathlineto{\pgfqpoint{0.812867in}{0.930524in}}%
\pgfpathclose%
\pgfpathmoveto{\pgfqpoint{1.125468in}{0.902643in}}%
\pgfpathlineto{\pgfqpoint{1.141125in}{0.897460in}}%
\pgfpathlineto{\pgfqpoint{1.156781in}{0.897460in}}%
\pgfpathlineto{\pgfqpoint{1.172438in}{0.902643in}}%
\pgfpathlineto{\pgfqpoint{1.173464in}{0.903302in}}%
\pgfpathlineto{\pgfqpoint{1.188094in}{0.910226in}}%
\pgfpathlineto{\pgfqpoint{1.197746in}{0.916913in}}%
\pgfpathlineto{\pgfqpoint{1.203751in}{0.920594in}}%
\pgfpathlineto{\pgfqpoint{1.216633in}{0.930524in}}%
\pgfpathlineto{\pgfqpoint{1.219407in}{0.932674in}}%
\pgfpathlineto{\pgfqpoint{1.232591in}{0.944135in}}%
\pgfpathlineto{\pgfqpoint{1.235064in}{0.946547in}}%
\pgfpathlineto{\pgfqpoint{1.246486in}{0.957746in}}%
\pgfpathlineto{\pgfqpoint{1.250721in}{0.962966in}}%
\pgfpathlineto{\pgfqpoint{1.258412in}{0.971357in}}%
\pgfpathlineto{\pgfqpoint{1.266377in}{0.984076in}}%
\pgfpathlineto{\pgfqpoint{1.267134in}{0.984968in}}%
\pgfpathlineto{\pgfqpoint{1.273097in}{0.998579in}}%
\pgfpathlineto{\pgfqpoint{1.273097in}{1.012191in}}%
\pgfpathlineto{\pgfqpoint{1.267134in}{1.025802in}}%
\pgfpathlineto{\pgfqpoint{1.266377in}{1.026694in}}%
\pgfpathlineto{\pgfqpoint{1.258412in}{1.039413in}}%
\pgfpathlineto{\pgfqpoint{1.250721in}{1.047804in}}%
\pgfpathlineto{\pgfqpoint{1.246486in}{1.053024in}}%
\pgfpathlineto{\pgfqpoint{1.235064in}{1.064223in}}%
\pgfpathlineto{\pgfqpoint{1.232591in}{1.066635in}}%
\pgfpathlineto{\pgfqpoint{1.219407in}{1.078096in}}%
\pgfpathlineto{\pgfqpoint{1.216633in}{1.080246in}}%
\pgfpathlineto{\pgfqpoint{1.203751in}{1.090176in}}%
\pgfpathlineto{\pgfqpoint{1.197746in}{1.093857in}}%
\pgfpathlineto{\pgfqpoint{1.188094in}{1.100544in}}%
\pgfpathlineto{\pgfqpoint{1.173464in}{1.107468in}}%
\pgfpathlineto{\pgfqpoint{1.172438in}{1.108127in}}%
\pgfpathlineto{\pgfqpoint{1.156781in}{1.113310in}}%
\pgfpathlineto{\pgfqpoint{1.141125in}{1.113310in}}%
\pgfpathlineto{\pgfqpoint{1.125468in}{1.108127in}}%
\pgfpathlineto{\pgfqpoint{1.124442in}{1.107468in}}%
\pgfpathlineto{\pgfqpoint{1.109812in}{1.100544in}}%
\pgfpathlineto{\pgfqpoint{1.100160in}{1.093857in}}%
\pgfpathlineto{\pgfqpoint{1.094155in}{1.090176in}}%
\pgfpathlineto{\pgfqpoint{1.081273in}{1.080246in}}%
\pgfpathlineto{\pgfqpoint{1.078498in}{1.078096in}}%
\pgfpathlineto{\pgfqpoint{1.065315in}{1.066635in}}%
\pgfpathlineto{\pgfqpoint{1.062842in}{1.064223in}}%
\pgfpathlineto{\pgfqpoint{1.051419in}{1.053024in}}%
\pgfpathlineto{\pgfqpoint{1.047185in}{1.047804in}}%
\pgfpathlineto{\pgfqpoint{1.039494in}{1.039413in}}%
\pgfpathlineto{\pgfqpoint{1.031529in}{1.026694in}}%
\pgfpathlineto{\pgfqpoint{1.030771in}{1.025802in}}%
\pgfpathlineto{\pgfqpoint{1.024809in}{1.012191in}}%
\pgfpathlineto{\pgfqpoint{1.024809in}{0.998579in}}%
\pgfpathlineto{\pgfqpoint{1.030771in}{0.984968in}}%
\pgfpathlineto{\pgfqpoint{1.031529in}{0.984076in}}%
\pgfpathlineto{\pgfqpoint{1.039494in}{0.971357in}}%
\pgfpathlineto{\pgfqpoint{1.047185in}{0.962966in}}%
\pgfpathlineto{\pgfqpoint{1.051419in}{0.957746in}}%
\pgfpathlineto{\pgfqpoint{1.062842in}{0.946547in}}%
\pgfpathlineto{\pgfqpoint{1.065315in}{0.944135in}}%
\pgfpathlineto{\pgfqpoint{1.078498in}{0.932674in}}%
\pgfpathlineto{\pgfqpoint{1.081273in}{0.930524in}}%
\pgfpathlineto{\pgfqpoint{1.094155in}{0.920594in}}%
\pgfpathlineto{\pgfqpoint{1.100160in}{0.916913in}}%
\pgfpathlineto{\pgfqpoint{1.109812in}{0.910226in}}%
\pgfpathlineto{\pgfqpoint{1.124442in}{0.903302in}}%
\pgfpathlineto{\pgfqpoint{1.125468in}{0.902643in}}%
\pgfpathclose%
\pgfpathmoveto{\pgfqpoint{1.123221in}{0.930524in}}%
\pgfpathlineto{\pgfqpoint{1.109812in}{0.936294in}}%
\pgfpathlineto{\pgfqpoint{1.097439in}{0.944135in}}%
\pgfpathlineto{\pgfqpoint{1.094155in}{0.946424in}}%
\pgfpathlineto{\pgfqpoint{1.081132in}{0.957746in}}%
\pgfpathlineto{\pgfqpoint{1.078498in}{0.960601in}}%
\pgfpathlineto{\pgfqpoint{1.069479in}{0.971357in}}%
\pgfpathlineto{\pgfqpoint{1.062842in}{0.983015in}}%
\pgfpathlineto{\pgfqpoint{1.061705in}{0.984968in}}%
\pgfpathlineto{\pgfqpoint{1.057627in}{0.998579in}}%
\pgfpathlineto{\pgfqpoint{1.057627in}{1.012191in}}%
\pgfpathlineto{\pgfqpoint{1.061705in}{1.025802in}}%
\pgfpathlineto{\pgfqpoint{1.062842in}{1.027755in}}%
\pgfpathlineto{\pgfqpoint{1.069479in}{1.039413in}}%
\pgfpathlineto{\pgfqpoint{1.078498in}{1.050169in}}%
\pgfpathlineto{\pgfqpoint{1.081132in}{1.053024in}}%
\pgfpathlineto{\pgfqpoint{1.094155in}{1.064346in}}%
\pgfpathlineto{\pgfqpoint{1.097439in}{1.066635in}}%
\pgfpathlineto{\pgfqpoint{1.109812in}{1.074476in}}%
\pgfpathlineto{\pgfqpoint{1.123221in}{1.080246in}}%
\pgfpathlineto{\pgfqpoint{1.125468in}{1.081235in}}%
\pgfpathlineto{\pgfqpoint{1.141125in}{1.084780in}}%
\pgfpathlineto{\pgfqpoint{1.156781in}{1.084780in}}%
\pgfpathlineto{\pgfqpoint{1.172438in}{1.081235in}}%
\pgfpathlineto{\pgfqpoint{1.174684in}{1.080246in}}%
\pgfpathlineto{\pgfqpoint{1.188094in}{1.074476in}}%
\pgfpathlineto{\pgfqpoint{1.200467in}{1.066635in}}%
\pgfpathlineto{\pgfqpoint{1.203751in}{1.064346in}}%
\pgfpathlineto{\pgfqpoint{1.216774in}{1.053024in}}%
\pgfpathlineto{\pgfqpoint{1.219407in}{1.050169in}}%
\pgfpathlineto{\pgfqpoint{1.228427in}{1.039413in}}%
\pgfpathlineto{\pgfqpoint{1.235064in}{1.027755in}}%
\pgfpathlineto{\pgfqpoint{1.236201in}{1.025802in}}%
\pgfpathlineto{\pgfqpoint{1.240279in}{1.012191in}}%
\pgfpathlineto{\pgfqpoint{1.240279in}{0.998579in}}%
\pgfpathlineto{\pgfqpoint{1.236201in}{0.984968in}}%
\pgfpathlineto{\pgfqpoint{1.235064in}{0.983015in}}%
\pgfpathlineto{\pgfqpoint{1.228427in}{0.971357in}}%
\pgfpathlineto{\pgfqpoint{1.219407in}{0.960601in}}%
\pgfpathlineto{\pgfqpoint{1.216774in}{0.957746in}}%
\pgfpathlineto{\pgfqpoint{1.203751in}{0.946424in}}%
\pgfpathlineto{\pgfqpoint{1.200467in}{0.944135in}}%
\pgfpathlineto{\pgfqpoint{1.188094in}{0.936294in}}%
\pgfpathlineto{\pgfqpoint{1.174684in}{0.930524in}}%
\pgfpathlineto{\pgfqpoint{1.172438in}{0.929535in}}%
\pgfpathlineto{\pgfqpoint{1.156781in}{0.925990in}}%
\pgfpathlineto{\pgfqpoint{1.141125in}{0.925990in}}%
\pgfpathlineto{\pgfqpoint{1.125468in}{0.929535in}}%
\pgfpathlineto{\pgfqpoint{1.123221in}{0.930524in}}%
\pgfpathclose%
\pgfpathmoveto{\pgfqpoint{1.438599in}{0.901191in}}%
\pgfpathlineto{\pgfqpoint{1.454256in}{0.897045in}}%
\pgfpathlineto{\pgfqpoint{1.469913in}{0.898081in}}%
\pgfpathlineto{\pgfqpoint{1.483093in}{0.903302in}}%
\pgfpathlineto{\pgfqpoint{1.485569in}{0.904015in}}%
\pgfpathlineto{\pgfqpoint{1.501226in}{0.912149in}}%
\pgfpathlineto{\pgfqpoint{1.507760in}{0.916913in}}%
\pgfpathlineto{\pgfqpoint{1.516882in}{0.922813in}}%
\pgfpathlineto{\pgfqpoint{1.526599in}{0.930524in}}%
\pgfpathlineto{\pgfqpoint{1.532539in}{0.935271in}}%
\pgfpathlineto{\pgfqpoint{1.542629in}{0.944135in}}%
\pgfpathlineto{\pgfqpoint{1.548195in}{0.949625in}}%
\pgfpathlineto{\pgfqpoint{1.556562in}{0.957746in}}%
\pgfpathlineto{\pgfqpoint{1.563852in}{0.966637in}}%
\pgfpathlineto{\pgfqpoint{1.568347in}{0.971357in}}%
\pgfpathlineto{\pgfqpoint{1.577304in}{0.984968in}}%
\pgfpathlineto{\pgfqpoint{1.579508in}{0.991619in}}%
\pgfpathlineto{\pgfqpoint{1.582845in}{0.998579in}}%
\pgfpathlineto{\pgfqpoint{1.582845in}{1.012191in}}%
\pgfpathlineto{\pgfqpoint{1.579508in}{1.019151in}}%
\pgfpathlineto{\pgfqpoint{1.577304in}{1.025802in}}%
\pgfpathlineto{\pgfqpoint{1.568347in}{1.039413in}}%
\pgfpathlineto{\pgfqpoint{1.563852in}{1.044133in}}%
\pgfpathlineto{\pgfqpoint{1.556562in}{1.053024in}}%
\pgfpathlineto{\pgfqpoint{1.548195in}{1.061145in}}%
\pgfpathlineto{\pgfqpoint{1.542629in}{1.066635in}}%
\pgfpathlineto{\pgfqpoint{1.532539in}{1.075499in}}%
\pgfpathlineto{\pgfqpoint{1.526599in}{1.080246in}}%
\pgfpathlineto{\pgfqpoint{1.516882in}{1.087957in}}%
\pgfpathlineto{\pgfqpoint{1.507760in}{1.093857in}}%
\pgfpathlineto{\pgfqpoint{1.501226in}{1.098621in}}%
\pgfpathlineto{\pgfqpoint{1.485569in}{1.106755in}}%
\pgfpathlineto{\pgfqpoint{1.483093in}{1.107468in}}%
\pgfpathlineto{\pgfqpoint{1.469913in}{1.112689in}}%
\pgfpathlineto{\pgfqpoint{1.454256in}{1.113725in}}%
\pgfpathlineto{\pgfqpoint{1.438599in}{1.109579in}}%
\pgfpathlineto{\pgfqpoint{1.434974in}{1.107468in}}%
\pgfpathlineto{\pgfqpoint{1.422943in}{1.102318in}}%
\pgfpathlineto{\pgfqpoint{1.410019in}{1.093857in}}%
\pgfpathlineto{\pgfqpoint{1.407286in}{1.092279in}}%
\pgfpathlineto{\pgfqpoint{1.391630in}{1.080624in}}%
\pgfpathlineto{\pgfqpoint{1.391184in}{1.080246in}}%
\pgfpathlineto{\pgfqpoint{1.375973in}{1.067221in}}%
\pgfpathlineto{\pgfqpoint{1.375311in}{1.066635in}}%
\pgfpathlineto{\pgfqpoint{1.361547in}{1.053024in}}%
\pgfpathlineto{\pgfqpoint{1.360317in}{1.051498in}}%
\pgfpathlineto{\pgfqpoint{1.349582in}{1.039413in}}%
\pgfpathlineto{\pgfqpoint{1.344660in}{1.031285in}}%
\pgfpathlineto{\pgfqpoint{1.340339in}{1.025802in}}%
\pgfpathlineto{\pgfqpoint{1.334871in}{1.012191in}}%
\pgfpathlineto{\pgfqpoint{1.334871in}{0.998579in}}%
\pgfpathlineto{\pgfqpoint{1.340339in}{0.984968in}}%
\pgfpathlineto{\pgfqpoint{1.344660in}{0.979485in}}%
\pgfpathlineto{\pgfqpoint{1.349582in}{0.971357in}}%
\pgfpathlineto{\pgfqpoint{1.360317in}{0.959272in}}%
\pgfpathlineto{\pgfqpoint{1.361547in}{0.957746in}}%
\pgfpathlineto{\pgfqpoint{1.375311in}{0.944135in}}%
\pgfpathlineto{\pgfqpoint{1.375973in}{0.943549in}}%
\pgfpathlineto{\pgfqpoint{1.391184in}{0.930524in}}%
\pgfpathlineto{\pgfqpoint{1.391630in}{0.930146in}}%
\pgfpathlineto{\pgfqpoint{1.407286in}{0.918491in}}%
\pgfpathlineto{\pgfqpoint{1.410019in}{0.916913in}}%
\pgfpathlineto{\pgfqpoint{1.422943in}{0.908452in}}%
\pgfpathlineto{\pgfqpoint{1.434974in}{0.903302in}}%
\pgfpathlineto{\pgfqpoint{1.438599in}{0.901191in}}%
\pgfpathclose%
\pgfpathmoveto{\pgfqpoint{1.433636in}{0.930524in}}%
\pgfpathlineto{\pgfqpoint{1.422943in}{0.934685in}}%
\pgfpathlineto{\pgfqpoint{1.407286in}{0.944058in}}%
\pgfpathlineto{\pgfqpoint{1.407185in}{0.944135in}}%
\pgfpathlineto{\pgfqpoint{1.391630in}{0.957191in}}%
\pgfpathlineto{\pgfqpoint{1.391049in}{0.957746in}}%
\pgfpathlineto{\pgfqpoint{1.379502in}{0.971357in}}%
\pgfpathlineto{\pgfqpoint{1.375973in}{0.977463in}}%
\pgfpathlineto{\pgfqpoint{1.371628in}{0.984968in}}%
\pgfpathlineto{\pgfqpoint{1.367631in}{0.998579in}}%
\pgfpathlineto{\pgfqpoint{1.367631in}{1.012191in}}%
\pgfpathlineto{\pgfqpoint{1.371628in}{1.025802in}}%
\pgfpathlineto{\pgfqpoint{1.375973in}{1.033307in}}%
\pgfpathlineto{\pgfqpoint{1.379502in}{1.039413in}}%
\pgfpathlineto{\pgfqpoint{1.391049in}{1.053024in}}%
\pgfpathlineto{\pgfqpoint{1.391630in}{1.053579in}}%
\pgfpathlineto{\pgfqpoint{1.407185in}{1.066635in}}%
\pgfpathlineto{\pgfqpoint{1.407286in}{1.066712in}}%
\pgfpathlineto{\pgfqpoint{1.422943in}{1.076085in}}%
\pgfpathlineto{\pgfqpoint{1.433636in}{1.080246in}}%
\pgfpathlineto{\pgfqpoint{1.438599in}{1.082228in}}%
\pgfpathlineto{\pgfqpoint{1.454256in}{1.085063in}}%
\pgfpathlineto{\pgfqpoint{1.469913in}{1.084355in}}%
\pgfpathlineto{\pgfqpoint{1.485039in}{1.080246in}}%
\pgfpathlineto{\pgfqpoint{1.485569in}{1.080107in}}%
\pgfpathlineto{\pgfqpoint{1.501226in}{1.072733in}}%
\pgfpathlineto{\pgfqpoint{1.510370in}{1.066635in}}%
\pgfpathlineto{\pgfqpoint{1.516882in}{1.061844in}}%
\pgfpathlineto{\pgfqpoint{1.526737in}{1.053024in}}%
\pgfpathlineto{\pgfqpoint{1.532539in}{1.046537in}}%
\pgfpathlineto{\pgfqpoint{1.538451in}{1.039413in}}%
\pgfpathlineto{\pgfqpoint{1.546187in}{1.025802in}}%
\pgfpathlineto{\pgfqpoint{1.548195in}{1.018784in}}%
\pgfpathlineto{\pgfqpoint{1.550201in}{1.012191in}}%
\pgfpathlineto{\pgfqpoint{1.550201in}{0.998579in}}%
\pgfpathlineto{\pgfqpoint{1.548195in}{0.991986in}}%
\pgfpathlineto{\pgfqpoint{1.546187in}{0.984968in}}%
\pgfpathlineto{\pgfqpoint{1.538451in}{0.971357in}}%
\pgfpathlineto{\pgfqpoint{1.532539in}{0.964233in}}%
\pgfpathlineto{\pgfqpoint{1.526737in}{0.957746in}}%
\pgfpathlineto{\pgfqpoint{1.516882in}{0.948926in}}%
\pgfpathlineto{\pgfqpoint{1.510370in}{0.944135in}}%
\pgfpathlineto{\pgfqpoint{1.501226in}{0.938037in}}%
\pgfpathlineto{\pgfqpoint{1.485569in}{0.930663in}}%
\pgfpathlineto{\pgfqpoint{1.485039in}{0.930524in}}%
\pgfpathlineto{\pgfqpoint{1.469913in}{0.926415in}}%
\pgfpathlineto{\pgfqpoint{1.454256in}{0.925707in}}%
\pgfpathlineto{\pgfqpoint{1.438599in}{0.928542in}}%
\pgfpathlineto{\pgfqpoint{1.433636in}{0.930524in}}%
\pgfpathclose%
\pgfpathmoveto{\pgfqpoint{1.751731in}{0.899946in}}%
\pgfpathlineto{\pgfqpoint{1.767387in}{0.896838in}}%
\pgfpathlineto{\pgfqpoint{1.783044in}{0.898910in}}%
\pgfpathlineto{\pgfqpoint{1.792605in}{0.903302in}}%
\pgfpathlineto{\pgfqpoint{1.798700in}{0.905346in}}%
\pgfpathlineto{\pgfqpoint{1.814357in}{0.914218in}}%
\pgfpathlineto{\pgfqpoint{1.817888in}{0.916913in}}%
\pgfpathlineto{\pgfqpoint{1.830014in}{0.925148in}}%
\pgfpathlineto{\pgfqpoint{1.836620in}{0.930524in}}%
\pgfpathlineto{\pgfqpoint{1.845670in}{0.937955in}}%
\pgfpathlineto{\pgfqpoint{1.852660in}{0.944135in}}%
\pgfpathlineto{\pgfqpoint{1.861327in}{0.952744in}}%
\pgfpathlineto{\pgfqpoint{1.866557in}{0.957746in}}%
\pgfpathlineto{\pgfqpoint{1.876983in}{0.970261in}}%
\pgfpathlineto{\pgfqpoint{1.878075in}{0.971357in}}%
\pgfpathlineto{\pgfqpoint{1.887576in}{0.984968in}}%
\pgfpathlineto{\pgfqpoint{1.892317in}{0.998579in}}%
\pgfpathlineto{\pgfqpoint{1.892317in}{1.012191in}}%
\pgfpathlineto{\pgfqpoint{1.887576in}{1.025802in}}%
\pgfpathlineto{\pgfqpoint{1.878075in}{1.039413in}}%
\pgfpathlineto{\pgfqpoint{1.876983in}{1.040509in}}%
\pgfpathlineto{\pgfqpoint{1.866557in}{1.053024in}}%
\pgfpathlineto{\pgfqpoint{1.861327in}{1.058026in}}%
\pgfpathlineto{\pgfqpoint{1.852660in}{1.066635in}}%
\pgfpathlineto{\pgfqpoint{1.845670in}{1.072815in}}%
\pgfpathlineto{\pgfqpoint{1.836620in}{1.080246in}}%
\pgfpathlineto{\pgfqpoint{1.830014in}{1.085622in}}%
\pgfpathlineto{\pgfqpoint{1.817888in}{1.093857in}}%
\pgfpathlineto{\pgfqpoint{1.814357in}{1.096552in}}%
\pgfpathlineto{\pgfqpoint{1.798700in}{1.105424in}}%
\pgfpathlineto{\pgfqpoint{1.792605in}{1.107468in}}%
\pgfpathlineto{\pgfqpoint{1.783044in}{1.111860in}}%
\pgfpathlineto{\pgfqpoint{1.767387in}{1.113932in}}%
\pgfpathlineto{\pgfqpoint{1.751731in}{1.110824in}}%
\pgfpathlineto{\pgfqpoint{1.745296in}{1.107468in}}%
\pgfpathlineto{\pgfqpoint{1.736074in}{1.103946in}}%
\pgfpathlineto{\pgfqpoint{1.720418in}{1.094338in}}%
\pgfpathlineto{\pgfqpoint{1.719813in}{1.093857in}}%
\pgfpathlineto{\pgfqpoint{1.704761in}{1.083177in}}%
\pgfpathlineto{\pgfqpoint{1.701236in}{1.080246in}}%
\pgfpathlineto{\pgfqpoint{1.689104in}{1.070054in}}%
\pgfpathlineto{\pgfqpoint{1.685248in}{1.066635in}}%
\pgfpathlineto{\pgfqpoint{1.673448in}{1.054880in}}%
\pgfpathlineto{\pgfqpoint{1.671468in}{1.053024in}}%
\pgfpathlineto{\pgfqpoint{1.659781in}{1.039413in}}%
\pgfpathlineto{\pgfqpoint{1.657791in}{1.036037in}}%
\pgfpathlineto{\pgfqpoint{1.650221in}{1.025802in}}%
\pgfpathlineto{\pgfqpoint{1.645152in}{1.012191in}}%
\pgfpathlineto{\pgfqpoint{1.645152in}{0.998579in}}%
\pgfpathlineto{\pgfqpoint{1.650221in}{0.984968in}}%
\pgfpathlineto{\pgfqpoint{1.657791in}{0.974733in}}%
\pgfpathlineto{\pgfqpoint{1.659781in}{0.971357in}}%
\pgfpathlineto{\pgfqpoint{1.671468in}{0.957746in}}%
\pgfpathlineto{\pgfqpoint{1.673448in}{0.955890in}}%
\pgfpathlineto{\pgfqpoint{1.685248in}{0.944135in}}%
\pgfpathlineto{\pgfqpoint{1.689104in}{0.940716in}}%
\pgfpathlineto{\pgfqpoint{1.701236in}{0.930524in}}%
\pgfpathlineto{\pgfqpoint{1.704761in}{0.927593in}}%
\pgfpathlineto{\pgfqpoint{1.719813in}{0.916913in}}%
\pgfpathlineto{\pgfqpoint{1.720418in}{0.916432in}}%
\pgfpathlineto{\pgfqpoint{1.736074in}{0.906824in}}%
\pgfpathlineto{\pgfqpoint{1.745296in}{0.903302in}}%
\pgfpathlineto{\pgfqpoint{1.751731in}{0.899946in}}%
\pgfpathclose%
\pgfpathmoveto{\pgfqpoint{1.743808in}{0.930524in}}%
\pgfpathlineto{\pgfqpoint{1.736074in}{0.933210in}}%
\pgfpathlineto{\pgfqpoint{1.720418in}{0.941920in}}%
\pgfpathlineto{\pgfqpoint{1.717374in}{0.944135in}}%
\pgfpathlineto{\pgfqpoint{1.704761in}{0.954314in}}%
\pgfpathlineto{\pgfqpoint{1.701101in}{0.957746in}}%
\pgfpathlineto{\pgfqpoint{1.689500in}{0.971357in}}%
\pgfpathlineto{\pgfqpoint{1.689104in}{0.972029in}}%
\pgfpathlineto{\pgfqpoint{1.681623in}{0.984968in}}%
\pgfpathlineto{\pgfqpoint{1.677688in}{0.998579in}}%
\pgfpathlineto{\pgfqpoint{1.677688in}{1.012191in}}%
\pgfpathlineto{\pgfqpoint{1.681623in}{1.025802in}}%
\pgfpathlineto{\pgfqpoint{1.689104in}{1.038741in}}%
\pgfpathlineto{\pgfqpoint{1.689500in}{1.039413in}}%
\pgfpathlineto{\pgfqpoint{1.701101in}{1.053024in}}%
\pgfpathlineto{\pgfqpoint{1.704761in}{1.056456in}}%
\pgfpathlineto{\pgfqpoint{1.717374in}{1.066635in}}%
\pgfpathlineto{\pgfqpoint{1.720418in}{1.068850in}}%
\pgfpathlineto{\pgfqpoint{1.736074in}{1.077560in}}%
\pgfpathlineto{\pgfqpoint{1.743808in}{1.080246in}}%
\pgfpathlineto{\pgfqpoint{1.751731in}{1.083079in}}%
\pgfpathlineto{\pgfqpoint{1.767387in}{1.085205in}}%
\pgfpathlineto{\pgfqpoint{1.783044in}{1.083788in}}%
\pgfpathlineto{\pgfqpoint{1.794288in}{1.080246in}}%
\pgfpathlineto{\pgfqpoint{1.798700in}{1.078901in}}%
\pgfpathlineto{\pgfqpoint{1.814357in}{1.070858in}}%
\pgfpathlineto{\pgfqpoint{1.820404in}{1.066635in}}%
\pgfpathlineto{\pgfqpoint{1.830014in}{1.059213in}}%
\pgfpathlineto{\pgfqpoint{1.836757in}{1.053024in}}%
\pgfpathlineto{\pgfqpoint{1.845670in}{1.042785in}}%
\pgfpathlineto{\pgfqpoint{1.848450in}{1.039413in}}%
\pgfpathlineto{\pgfqpoint{1.856244in}{1.025802in}}%
\pgfpathlineto{\pgfqpoint{1.860134in}{1.012191in}}%
\pgfpathlineto{\pgfqpoint{1.860134in}{0.998579in}}%
\pgfpathlineto{\pgfqpoint{1.856244in}{0.984968in}}%
\pgfpathlineto{\pgfqpoint{1.848450in}{0.971357in}}%
\pgfpathlineto{\pgfqpoint{1.845670in}{0.967985in}}%
\pgfpathlineto{\pgfqpoint{1.836757in}{0.957746in}}%
\pgfpathlineto{\pgfqpoint{1.830014in}{0.951557in}}%
\pgfpathlineto{\pgfqpoint{1.820404in}{0.944135in}}%
\pgfpathlineto{\pgfqpoint{1.814357in}{0.939912in}}%
\pgfpathlineto{\pgfqpoint{1.798700in}{0.931869in}}%
\pgfpathlineto{\pgfqpoint{1.794288in}{0.930524in}}%
\pgfpathlineto{\pgfqpoint{1.783044in}{0.926982in}}%
\pgfpathlineto{\pgfqpoint{1.767387in}{0.925565in}}%
\pgfpathlineto{\pgfqpoint{1.751731in}{0.927691in}}%
\pgfpathlineto{\pgfqpoint{1.743808in}{0.930524in}}%
\pgfpathclose%
\pgfpathmoveto{\pgfqpoint{0.499205in}{1.175005in}}%
\pgfpathlineto{\pgfqpoint{0.514862in}{1.168343in}}%
\pgfpathlineto{\pgfqpoint{0.530519in}{1.166444in}}%
\pgfpathlineto{\pgfqpoint{0.546175in}{1.169294in}}%
\pgfpathlineto{\pgfqpoint{0.559047in}{1.175524in}}%
\pgfpathlineto{\pgfqpoint{0.561832in}{1.176552in}}%
\pgfpathlineto{\pgfqpoint{0.577488in}{1.185733in}}%
\pgfpathlineto{\pgfqpoint{0.581901in}{1.189135in}}%
\pgfpathlineto{\pgfqpoint{0.593145in}{1.197065in}}%
\pgfpathlineto{\pgfqpoint{0.600016in}{1.202746in}}%
\pgfpathlineto{\pgfqpoint{0.608801in}{1.210240in}}%
\pgfpathlineto{\pgfqpoint{0.615605in}{1.216357in}}%
\pgfpathlineto{\pgfqpoint{0.624458in}{1.225491in}}%
\pgfpathlineto{\pgfqpoint{0.629079in}{1.229968in}}%
\pgfpathlineto{\pgfqpoint{0.640000in}{1.243579in}}%
\pgfpathlineto{\pgfqpoint{0.640115in}{1.243794in}}%
\pgfpathlineto{\pgfqpoint{0.649105in}{1.257191in}}%
\pgfpathlineto{\pgfqpoint{0.653158in}{1.270802in}}%
\pgfpathlineto{\pgfqpoint{0.652146in}{1.284413in}}%
\pgfpathlineto{\pgfqpoint{0.646060in}{1.298024in}}%
\pgfpathlineto{\pgfqpoint{0.640115in}{1.305398in}}%
\pgfpathlineto{\pgfqpoint{0.636095in}{1.311635in}}%
\pgfpathlineto{\pgfqpoint{0.624458in}{1.324405in}}%
\pgfpathlineto{\pgfqpoint{0.623743in}{1.325246in}}%
\pgfpathlineto{\pgfqpoint{0.609602in}{1.338857in}}%
\pgfpathlineto{\pgfqpoint{0.608801in}{1.339560in}}%
\pgfpathlineto{\pgfqpoint{0.593264in}{1.352468in}}%
\pgfpathlineto{\pgfqpoint{0.593145in}{1.352569in}}%
\pgfpathlineto{\pgfqpoint{0.577488in}{1.363929in}}%
\pgfpathlineto{\pgfqpoint{0.573556in}{1.366079in}}%
\pgfpathlineto{\pgfqpoint{0.561832in}{1.373570in}}%
\pgfpathlineto{\pgfqpoint{0.546447in}{1.379691in}}%
\pgfpathlineto{\pgfqpoint{0.546175in}{1.379845in}}%
\pgfpathlineto{\pgfqpoint{0.530519in}{1.383277in}}%
\pgfpathlineto{\pgfqpoint{0.514862in}{1.380990in}}%
\pgfpathlineto{\pgfqpoint{0.512271in}{1.379691in}}%
\pgfpathlineto{\pgfqpoint{0.499205in}{1.375128in}}%
\pgfpathlineto{\pgfqpoint{0.484025in}{1.366079in}}%
\pgfpathlineto{\pgfqpoint{0.483549in}{1.365837in}}%
\pgfpathlineto{\pgfqpoint{0.467892in}{1.355075in}}%
\pgfpathlineto{\pgfqpoint{0.464721in}{1.352468in}}%
\pgfpathlineto{\pgfqpoint{0.452236in}{1.342331in}}%
\pgfpathlineto{\pgfqpoint{0.448266in}{1.338857in}}%
\pgfpathlineto{\pgfqpoint{0.436579in}{1.327602in}}%
\pgfpathlineto{\pgfqpoint{0.434043in}{1.325246in}}%
\pgfpathlineto{\pgfqpoint{0.422017in}{1.311635in}}%
\pgfpathlineto{\pgfqpoint{0.420923in}{1.309900in}}%
\pgfpathlineto{\pgfqpoint{0.411849in}{1.298024in}}%
\pgfpathlineto{\pgfqpoint{0.406157in}{1.284413in}}%
\pgfpathlineto{\pgfqpoint{0.405266in}{1.271609in}}%
\pgfpathlineto{\pgfqpoint{0.405179in}{1.270802in}}%
\pgfpathlineto{\pgfqpoint{0.405266in}{1.270598in}}%
\pgfpathlineto{\pgfqpoint{0.409001in}{1.257191in}}%
\pgfpathlineto{\pgfqpoint{0.417550in}{1.243579in}}%
\pgfpathlineto{\pgfqpoint{0.420923in}{1.239998in}}%
\pgfpathlineto{\pgfqpoint{0.428796in}{1.229968in}}%
\pgfpathlineto{\pgfqpoint{0.436579in}{1.222276in}}%
\pgfpathlineto{\pgfqpoint{0.442332in}{1.216357in}}%
\pgfpathlineto{\pgfqpoint{0.452236in}{1.207476in}}%
\pgfpathlineto{\pgfqpoint{0.457910in}{1.202746in}}%
\pgfpathlineto{\pgfqpoint{0.467892in}{1.194668in}}%
\pgfpathlineto{\pgfqpoint{0.476089in}{1.189135in}}%
\pgfpathlineto{\pgfqpoint{0.483549in}{1.183617in}}%
\pgfpathlineto{\pgfqpoint{0.498466in}{1.175524in}}%
\pgfpathlineto{\pgfqpoint{0.499205in}{1.175005in}}%
\pgfpathclose%
\pgfpathmoveto{\pgfqpoint{0.496467in}{1.202746in}}%
\pgfpathlineto{\pgfqpoint{0.483549in}{1.209435in}}%
\pgfpathlineto{\pgfqpoint{0.473775in}{1.216357in}}%
\pgfpathlineto{\pgfqpoint{0.467892in}{1.221063in}}%
\pgfpathlineto{\pgfqpoint{0.458505in}{1.229968in}}%
\pgfpathlineto{\pgfqpoint{0.452236in}{1.237610in}}%
\pgfpathlineto{\pgfqpoint{0.447584in}{1.243579in}}%
\pgfpathlineto{\pgfqpoint{0.440572in}{1.257191in}}%
\pgfpathlineto{\pgfqpoint{0.437462in}{1.270802in}}%
\pgfpathlineto{\pgfqpoint{0.438239in}{1.284413in}}%
\pgfpathlineto{\pgfqpoint{0.442908in}{1.298024in}}%
\pgfpathlineto{\pgfqpoint{0.451482in}{1.311635in}}%
\pgfpathlineto{\pgfqpoint{0.452236in}{1.312504in}}%
\pgfpathlineto{\pgfqpoint{0.463941in}{1.325246in}}%
\pgfpathlineto{\pgfqpoint{0.467892in}{1.328769in}}%
\pgfpathlineto{\pgfqpoint{0.481365in}{1.338857in}}%
\pgfpathlineto{\pgfqpoint{0.483549in}{1.340366in}}%
\pgfpathlineto{\pgfqpoint{0.499205in}{1.348437in}}%
\pgfpathlineto{\pgfqpoint{0.512578in}{1.352468in}}%
\pgfpathlineto{\pgfqpoint{0.514862in}{1.353195in}}%
\pgfpathlineto{\pgfqpoint{0.530519in}{1.354647in}}%
\pgfpathlineto{\pgfqpoint{0.546175in}{1.352469in}}%
\pgfpathlineto{\pgfqpoint{0.546176in}{1.352468in}}%
\pgfpathlineto{\pgfqpoint{0.561832in}{1.347092in}}%
\pgfpathlineto{\pgfqpoint{0.576621in}{1.338857in}}%
\pgfpathlineto{\pgfqpoint{0.577488in}{1.338340in}}%
\pgfpathlineto{\pgfqpoint{0.593145in}{1.326059in}}%
\pgfpathlineto{\pgfqpoint{0.594037in}{1.325246in}}%
\pgfpathlineto{\pgfqpoint{0.606390in}{1.311635in}}%
\pgfpathlineto{\pgfqpoint{0.608801in}{1.307893in}}%
\pgfpathlineto{\pgfqpoint{0.615023in}{1.298024in}}%
\pgfpathlineto{\pgfqpoint{0.619746in}{1.284413in}}%
\pgfpathlineto{\pgfqpoint{0.620532in}{1.270802in}}%
\pgfpathlineto{\pgfqpoint{0.617386in}{1.257191in}}%
\pgfpathlineto{\pgfqpoint{0.610292in}{1.243579in}}%
\pgfpathlineto{\pgfqpoint{0.608801in}{1.241662in}}%
\pgfpathlineto{\pgfqpoint{0.599427in}{1.229968in}}%
\pgfpathlineto{\pgfqpoint{0.593145in}{1.223880in}}%
\pgfpathlineto{\pgfqpoint{0.584144in}{1.216357in}}%
\pgfpathlineto{\pgfqpoint{0.577488in}{1.211444in}}%
\pgfpathlineto{\pgfqpoint{0.561866in}{1.202746in}}%
\pgfpathlineto{\pgfqpoint{0.561832in}{1.202727in}}%
\pgfpathlineto{\pgfqpoint{0.546175in}{1.197160in}}%
\pgfpathlineto{\pgfqpoint{0.530519in}{1.195077in}}%
\pgfpathlineto{\pgfqpoint{0.514862in}{1.196466in}}%
\pgfpathlineto{\pgfqpoint{0.499205in}{1.201335in}}%
\pgfpathlineto{\pgfqpoint{0.496467in}{1.202746in}}%
\pgfpathclose%
\pgfpathmoveto{\pgfqpoint{0.812337in}{1.173291in}}%
\pgfpathlineto{\pgfqpoint{0.827993in}{1.167583in}}%
\pgfpathlineto{\pgfqpoint{0.843650in}{1.166634in}}%
\pgfpathlineto{\pgfqpoint{0.859306in}{1.170435in}}%
\pgfpathlineto{\pgfqpoint{0.868722in}{1.175524in}}%
\pgfpathlineto{\pgfqpoint{0.874963in}{1.178107in}}%
\pgfpathlineto{\pgfqpoint{0.890620in}{1.187986in}}%
\pgfpathlineto{\pgfqpoint{0.892056in}{1.189135in}}%
\pgfpathlineto{\pgfqpoint{0.906276in}{1.199567in}}%
\pgfpathlineto{\pgfqpoint{0.910053in}{1.202746in}}%
\pgfpathlineto{\pgfqpoint{0.921933in}{1.213074in}}%
\pgfpathlineto{\pgfqpoint{0.925589in}{1.216357in}}%
\pgfpathlineto{\pgfqpoint{0.937589in}{1.228720in}}%
\pgfpathlineto{\pgfqpoint{0.938911in}{1.229968in}}%
\pgfpathlineto{\pgfqpoint{0.950275in}{1.243579in}}%
\pgfpathlineto{\pgfqpoint{0.953246in}{1.249005in}}%
\pgfpathlineto{\pgfqpoint{0.959099in}{1.257191in}}%
\pgfpathlineto{\pgfqpoint{0.963472in}{1.270802in}}%
\pgfpathlineto{\pgfqpoint{0.962380in}{1.284413in}}%
\pgfpathlineto{\pgfqpoint{0.955814in}{1.298024in}}%
\pgfpathlineto{\pgfqpoint{0.953246in}{1.301014in}}%
\pgfpathlineto{\pgfqpoint{0.946211in}{1.311635in}}%
\pgfpathlineto{\pgfqpoint{0.937589in}{1.320859in}}%
\pgfpathlineto{\pgfqpoint{0.933857in}{1.325246in}}%
\pgfpathlineto{\pgfqpoint{0.921933in}{1.336666in}}%
\pgfpathlineto{\pgfqpoint{0.919581in}{1.338857in}}%
\pgfpathlineto{\pgfqpoint{0.906276in}{1.350144in}}%
\pgfpathlineto{\pgfqpoint{0.903134in}{1.352468in}}%
\pgfpathlineto{\pgfqpoint{0.890620in}{1.361897in}}%
\pgfpathlineto{\pgfqpoint{0.883457in}{1.366079in}}%
\pgfpathlineto{\pgfqpoint{0.874963in}{1.371857in}}%
\pgfpathlineto{\pgfqpoint{0.859306in}{1.378863in}}%
\pgfpathlineto{\pgfqpoint{0.855181in}{1.379691in}}%
\pgfpathlineto{\pgfqpoint{0.843650in}{1.383048in}}%
\pgfpathlineto{\pgfqpoint{0.827993in}{1.381905in}}%
\pgfpathlineto{\pgfqpoint{0.822873in}{1.379691in}}%
\pgfpathlineto{\pgfqpoint{0.812337in}{1.376529in}}%
\pgfpathlineto{\pgfqpoint{0.796680in}{1.367962in}}%
\pgfpathlineto{\pgfqpoint{0.794194in}{1.366079in}}%
\pgfpathlineto{\pgfqpoint{0.781024in}{1.357467in}}%
\pgfpathlineto{\pgfqpoint{0.774787in}{1.352468in}}%
\pgfpathlineto{\pgfqpoint{0.765367in}{1.345023in}}%
\pgfpathlineto{\pgfqpoint{0.758275in}{1.338857in}}%
\pgfpathlineto{\pgfqpoint{0.749710in}{1.330668in}}%
\pgfpathlineto{\pgfqpoint{0.743960in}{1.325246in}}%
\pgfpathlineto{\pgfqpoint{0.734054in}{1.313797in}}%
\pgfpathlineto{\pgfqpoint{0.731888in}{1.311635in}}%
\pgfpathlineto{\pgfqpoint{0.722034in}{1.298024in}}%
\pgfpathlineto{\pgfqpoint{0.718397in}{1.288865in}}%
\pgfpathlineto{\pgfqpoint{0.715850in}{1.284413in}}%
\pgfpathlineto{\pgfqpoint{0.714535in}{1.270802in}}%
\pgfpathlineto{\pgfqpoint{0.718397in}{1.260777in}}%
\pgfpathlineto{\pgfqpoint{0.719350in}{1.257191in}}%
\pgfpathlineto{\pgfqpoint{0.727409in}{1.243579in}}%
\pgfpathlineto{\pgfqpoint{0.734054in}{1.236195in}}%
\pgfpathlineto{\pgfqpoint{0.738865in}{1.229968in}}%
\pgfpathlineto{\pgfqpoint{0.749710in}{1.219089in}}%
\pgfpathlineto{\pgfqpoint{0.752384in}{1.216357in}}%
\pgfpathlineto{\pgfqpoint{0.765367in}{1.204791in}}%
\pgfpathlineto{\pgfqpoint{0.767887in}{1.202746in}}%
\pgfpathlineto{\pgfqpoint{0.781024in}{1.192380in}}%
\pgfpathlineto{\pgfqpoint{0.786071in}{1.189135in}}%
\pgfpathlineto{\pgfqpoint{0.796680in}{1.181640in}}%
\pgfpathlineto{\pgfqpoint{0.808898in}{1.175524in}}%
\pgfpathlineto{\pgfqpoint{0.812337in}{1.173291in}}%
\pgfpathclose%
\pgfpathmoveto{\pgfqpoint{0.806752in}{1.202746in}}%
\pgfpathlineto{\pgfqpoint{0.796680in}{1.207558in}}%
\pgfpathlineto{\pgfqpoint{0.783671in}{1.216357in}}%
\pgfpathlineto{\pgfqpoint{0.781024in}{1.218375in}}%
\pgfpathlineto{\pgfqpoint{0.768489in}{1.229968in}}%
\pgfpathlineto{\pgfqpoint{0.765367in}{1.233673in}}%
\pgfpathlineto{\pgfqpoint{0.757598in}{1.243579in}}%
\pgfpathlineto{\pgfqpoint{0.750637in}{1.257191in}}%
\pgfpathlineto{\pgfqpoint{0.749710in}{1.261233in}}%
\pgfpathlineto{\pgfqpoint{0.747371in}{1.270802in}}%
\pgfpathlineto{\pgfqpoint{0.748206in}{1.284413in}}%
\pgfpathlineto{\pgfqpoint{0.749710in}{1.288556in}}%
\pgfpathlineto{\pgfqpoint{0.752956in}{1.298024in}}%
\pgfpathlineto{\pgfqpoint{0.761466in}{1.311635in}}%
\pgfpathlineto{\pgfqpoint{0.765367in}{1.316102in}}%
\pgfpathlineto{\pgfqpoint{0.773997in}{1.325246in}}%
\pgfpathlineto{\pgfqpoint{0.781024in}{1.331355in}}%
\pgfpathlineto{\pgfqpoint{0.791542in}{1.338857in}}%
\pgfpathlineto{\pgfqpoint{0.796680in}{1.342248in}}%
\pgfpathlineto{\pgfqpoint{0.812337in}{1.349647in}}%
\pgfpathlineto{\pgfqpoint{0.823227in}{1.352468in}}%
\pgfpathlineto{\pgfqpoint{0.827993in}{1.353776in}}%
\pgfpathlineto{\pgfqpoint{0.843650in}{1.354502in}}%
\pgfpathlineto{\pgfqpoint{0.854656in}{1.352468in}}%
\pgfpathlineto{\pgfqpoint{0.859306in}{1.351663in}}%
\pgfpathlineto{\pgfqpoint{0.874963in}{1.345612in}}%
\pgfpathlineto{\pgfqpoint{0.886358in}{1.338857in}}%
\pgfpathlineto{\pgfqpoint{0.890620in}{1.336143in}}%
\pgfpathlineto{\pgfqpoint{0.903956in}{1.325246in}}%
\pgfpathlineto{\pgfqpoint{0.906276in}{1.322944in}}%
\pgfpathlineto{\pgfqpoint{0.916397in}{1.311635in}}%
\pgfpathlineto{\pgfqpoint{0.921933in}{1.302879in}}%
\pgfpathlineto{\pgfqpoint{0.924998in}{1.298024in}}%
\pgfpathlineto{\pgfqpoint{0.929796in}{1.284413in}}%
\pgfpathlineto{\pgfqpoint{0.930595in}{1.270802in}}%
\pgfpathlineto{\pgfqpoint{0.927398in}{1.257191in}}%
\pgfpathlineto{\pgfqpoint{0.921933in}{1.246788in}}%
\pgfpathlineto{\pgfqpoint{0.920256in}{1.243579in}}%
\pgfpathlineto{\pgfqpoint{0.909466in}{1.229968in}}%
\pgfpathlineto{\pgfqpoint{0.906276in}{1.226821in}}%
\pgfpathlineto{\pgfqpoint{0.894240in}{1.216357in}}%
\pgfpathlineto{\pgfqpoint{0.890620in}{1.213584in}}%
\pgfpathlineto{\pgfqpoint{0.874963in}{1.204204in}}%
\pgfpathlineto{\pgfqpoint{0.871272in}{1.202746in}}%
\pgfpathlineto{\pgfqpoint{0.859306in}{1.197995in}}%
\pgfpathlineto{\pgfqpoint{0.843650in}{1.195216in}}%
\pgfpathlineto{\pgfqpoint{0.827993in}{1.195910in}}%
\pgfpathlineto{\pgfqpoint{0.812337in}{1.200082in}}%
\pgfpathlineto{\pgfqpoint{0.806752in}{1.202746in}}%
\pgfpathclose%
\pgfpathmoveto{\pgfqpoint{1.125468in}{1.171768in}}%
\pgfpathlineto{\pgfqpoint{1.141125in}{1.167013in}}%
\pgfpathlineto{\pgfqpoint{1.156781in}{1.167013in}}%
\pgfpathlineto{\pgfqpoint{1.172438in}{1.171768in}}%
\pgfpathlineto{\pgfqpoint{1.178745in}{1.175524in}}%
\pgfpathlineto{\pgfqpoint{1.188094in}{1.179803in}}%
\pgfpathlineto{\pgfqpoint{1.201996in}{1.189135in}}%
\pgfpathlineto{\pgfqpoint{1.203751in}{1.190205in}}%
\pgfpathlineto{\pgfqpoint{1.219407in}{1.202170in}}%
\pgfpathlineto{\pgfqpoint{1.220082in}{1.202746in}}%
\pgfpathlineto{\pgfqpoint{1.235064in}{1.215969in}}%
\pgfpathlineto{\pgfqpoint{1.235498in}{1.216357in}}%
\pgfpathlineto{\pgfqpoint{1.248905in}{1.229968in}}%
\pgfpathlineto{\pgfqpoint{1.250721in}{1.232344in}}%
\pgfpathlineto{\pgfqpoint{1.260453in}{1.243579in}}%
\pgfpathlineto{\pgfqpoint{1.266377in}{1.254039in}}%
\pgfpathlineto{\pgfqpoint{1.268805in}{1.257191in}}%
\pgfpathlineto{\pgfqpoint{1.273574in}{1.270802in}}%
\pgfpathlineto{\pgfqpoint{1.272382in}{1.284413in}}%
\pgfpathlineto{\pgfqpoint{1.266377in}{1.295871in}}%
\pgfpathlineto{\pgfqpoint{1.265556in}{1.298024in}}%
\pgfpathlineto{\pgfqpoint{1.256201in}{1.311635in}}%
\pgfpathlineto{\pgfqpoint{1.250721in}{1.317316in}}%
\pgfpathlineto{\pgfqpoint{1.243933in}{1.325246in}}%
\pgfpathlineto{\pgfqpoint{1.235064in}{1.333693in}}%
\pgfpathlineto{\pgfqpoint{1.229603in}{1.338857in}}%
\pgfpathlineto{\pgfqpoint{1.219407in}{1.347630in}}%
\pgfpathlineto{\pgfqpoint{1.213093in}{1.352468in}}%
\pgfpathlineto{\pgfqpoint{1.203751in}{1.359742in}}%
\pgfpathlineto{\pgfqpoint{1.193524in}{1.366079in}}%
\pgfpathlineto{\pgfqpoint{1.188094in}{1.369987in}}%
\pgfpathlineto{\pgfqpoint{1.172438in}{1.377774in}}%
\pgfpathlineto{\pgfqpoint{1.164787in}{1.379691in}}%
\pgfpathlineto{\pgfqpoint{1.156781in}{1.382591in}}%
\pgfpathlineto{\pgfqpoint{1.141125in}{1.382591in}}%
\pgfpathlineto{\pgfqpoint{1.133119in}{1.379691in}}%
\pgfpathlineto{\pgfqpoint{1.125468in}{1.377774in}}%
\pgfpathlineto{\pgfqpoint{1.109812in}{1.369987in}}%
\pgfpathlineto{\pgfqpoint{1.104382in}{1.366079in}}%
\pgfpathlineto{\pgfqpoint{1.094155in}{1.359742in}}%
\pgfpathlineto{\pgfqpoint{1.084813in}{1.352468in}}%
\pgfpathlineto{\pgfqpoint{1.078498in}{1.347630in}}%
\pgfpathlineto{\pgfqpoint{1.068303in}{1.338857in}}%
\pgfpathlineto{\pgfqpoint{1.062842in}{1.333693in}}%
\pgfpathlineto{\pgfqpoint{1.053973in}{1.325246in}}%
\pgfpathlineto{\pgfqpoint{1.047185in}{1.317316in}}%
\pgfpathlineto{\pgfqpoint{1.041705in}{1.311635in}}%
\pgfpathlineto{\pgfqpoint{1.032349in}{1.298024in}}%
\pgfpathlineto{\pgfqpoint{1.031529in}{1.295871in}}%
\pgfpathlineto{\pgfqpoint{1.025524in}{1.284413in}}%
\pgfpathlineto{\pgfqpoint{1.024332in}{1.270802in}}%
\pgfpathlineto{\pgfqpoint{1.029101in}{1.257191in}}%
\pgfpathlineto{\pgfqpoint{1.031529in}{1.254039in}}%
\pgfpathlineto{\pgfqpoint{1.037452in}{1.243579in}}%
\pgfpathlineto{\pgfqpoint{1.047185in}{1.232344in}}%
\pgfpathlineto{\pgfqpoint{1.049001in}{1.229968in}}%
\pgfpathlineto{\pgfqpoint{1.062408in}{1.216357in}}%
\pgfpathlineto{\pgfqpoint{1.062842in}{1.215969in}}%
\pgfpathlineto{\pgfqpoint{1.077824in}{1.202746in}}%
\pgfpathlineto{\pgfqpoint{1.078498in}{1.202170in}}%
\pgfpathlineto{\pgfqpoint{1.094155in}{1.190205in}}%
\pgfpathlineto{\pgfqpoint{1.095910in}{1.189135in}}%
\pgfpathlineto{\pgfqpoint{1.109812in}{1.179803in}}%
\pgfpathlineto{\pgfqpoint{1.119161in}{1.175524in}}%
\pgfpathlineto{\pgfqpoint{1.125468in}{1.171768in}}%
\pgfpathclose%
\pgfpathmoveto{\pgfqpoint{1.116835in}{1.202746in}}%
\pgfpathlineto{\pgfqpoint{1.109812in}{1.205814in}}%
\pgfpathlineto{\pgfqpoint{1.094155in}{1.215852in}}%
\pgfpathlineto{\pgfqpoint{1.093517in}{1.216357in}}%
\pgfpathlineto{\pgfqpoint{1.078498in}{1.229880in}}%
\pgfpathlineto{\pgfqpoint{1.078410in}{1.229968in}}%
\pgfpathlineto{\pgfqpoint{1.067628in}{1.243579in}}%
\pgfpathlineto{\pgfqpoint{1.062842in}{1.252875in}}%
\pgfpathlineto{\pgfqpoint{1.060562in}{1.257191in}}%
\pgfpathlineto{\pgfqpoint{1.057301in}{1.270802in}}%
\pgfpathlineto{\pgfqpoint{1.058115in}{1.284413in}}%
\pgfpathlineto{\pgfqpoint{1.062842in}{1.297563in}}%
\pgfpathlineto{\pgfqpoint{1.063002in}{1.298024in}}%
\pgfpathlineto{\pgfqpoint{1.071484in}{1.311635in}}%
\pgfpathlineto{\pgfqpoint{1.078498in}{1.319584in}}%
\pgfpathlineto{\pgfqpoint{1.084009in}{1.325246in}}%
\pgfpathlineto{\pgfqpoint{1.094155in}{1.333814in}}%
\pgfpathlineto{\pgfqpoint{1.101617in}{1.338857in}}%
\pgfpathlineto{\pgfqpoint{1.109812in}{1.343997in}}%
\pgfpathlineto{\pgfqpoint{1.125468in}{1.350722in}}%
\pgfpathlineto{\pgfqpoint{1.133541in}{1.352468in}}%
\pgfpathlineto{\pgfqpoint{1.141125in}{1.354212in}}%
\pgfpathlineto{\pgfqpoint{1.156781in}{1.354212in}}%
\pgfpathlineto{\pgfqpoint{1.164365in}{1.352468in}}%
\pgfpathlineto{\pgfqpoint{1.172438in}{1.350722in}}%
\pgfpathlineto{\pgfqpoint{1.188094in}{1.343997in}}%
\pgfpathlineto{\pgfqpoint{1.196289in}{1.338857in}}%
\pgfpathlineto{\pgfqpoint{1.203751in}{1.333814in}}%
\pgfpathlineto{\pgfqpoint{1.213896in}{1.325246in}}%
\pgfpathlineto{\pgfqpoint{1.219407in}{1.319584in}}%
\pgfpathlineto{\pgfqpoint{1.226422in}{1.311635in}}%
\pgfpathlineto{\pgfqpoint{1.234904in}{1.298024in}}%
\pgfpathlineto{\pgfqpoint{1.235064in}{1.297563in}}%
\pgfpathlineto{\pgfqpoint{1.239790in}{1.284413in}}%
\pgfpathlineto{\pgfqpoint{1.240605in}{1.270802in}}%
\pgfpathlineto{\pgfqpoint{1.237344in}{1.257191in}}%
\pgfpathlineto{\pgfqpoint{1.235064in}{1.252875in}}%
\pgfpathlineto{\pgfqpoint{1.230278in}{1.243579in}}%
\pgfpathlineto{\pgfqpoint{1.219496in}{1.229968in}}%
\pgfpathlineto{\pgfqpoint{1.219407in}{1.229880in}}%
\pgfpathlineto{\pgfqpoint{1.204389in}{1.216357in}}%
\pgfpathlineto{\pgfqpoint{1.203751in}{1.215852in}}%
\pgfpathlineto{\pgfqpoint{1.188094in}{1.205814in}}%
\pgfpathlineto{\pgfqpoint{1.181071in}{1.202746in}}%
\pgfpathlineto{\pgfqpoint{1.172438in}{1.198968in}}%
\pgfpathlineto{\pgfqpoint{1.156781in}{1.195494in}}%
\pgfpathlineto{\pgfqpoint{1.141125in}{1.195494in}}%
\pgfpathlineto{\pgfqpoint{1.125468in}{1.198968in}}%
\pgfpathlineto{\pgfqpoint{1.116835in}{1.202746in}}%
\pgfpathclose%
\pgfpathmoveto{\pgfqpoint{1.438599in}{1.170435in}}%
\pgfpathlineto{\pgfqpoint{1.454256in}{1.166634in}}%
\pgfpathlineto{\pgfqpoint{1.469913in}{1.167583in}}%
\pgfpathlineto{\pgfqpoint{1.485569in}{1.173291in}}%
\pgfpathlineto{\pgfqpoint{1.489008in}{1.175524in}}%
\pgfpathlineto{\pgfqpoint{1.501226in}{1.181640in}}%
\pgfpathlineto{\pgfqpoint{1.511835in}{1.189135in}}%
\pgfpathlineto{\pgfqpoint{1.516882in}{1.192380in}}%
\pgfpathlineto{\pgfqpoint{1.530019in}{1.202746in}}%
\pgfpathlineto{\pgfqpoint{1.532539in}{1.204791in}}%
\pgfpathlineto{\pgfqpoint{1.545522in}{1.216357in}}%
\pgfpathlineto{\pgfqpoint{1.548195in}{1.219089in}}%
\pgfpathlineto{\pgfqpoint{1.559041in}{1.229968in}}%
\pgfpathlineto{\pgfqpoint{1.563852in}{1.236195in}}%
\pgfpathlineto{\pgfqpoint{1.570497in}{1.243579in}}%
\pgfpathlineto{\pgfqpoint{1.578556in}{1.257191in}}%
\pgfpathlineto{\pgfqpoint{1.579508in}{1.260777in}}%
\pgfpathlineto{\pgfqpoint{1.583371in}{1.270802in}}%
\pgfpathlineto{\pgfqpoint{1.582056in}{1.284413in}}%
\pgfpathlineto{\pgfqpoint{1.579508in}{1.288865in}}%
\pgfpathlineto{\pgfqpoint{1.575872in}{1.298024in}}%
\pgfpathlineto{\pgfqpoint{1.566018in}{1.311635in}}%
\pgfpathlineto{\pgfqpoint{1.563852in}{1.313797in}}%
\pgfpathlineto{\pgfqpoint{1.553946in}{1.325246in}}%
\pgfpathlineto{\pgfqpoint{1.548195in}{1.330668in}}%
\pgfpathlineto{\pgfqpoint{1.539631in}{1.338857in}}%
\pgfpathlineto{\pgfqpoint{1.532539in}{1.345023in}}%
\pgfpathlineto{\pgfqpoint{1.523119in}{1.352468in}}%
\pgfpathlineto{\pgfqpoint{1.516882in}{1.357467in}}%
\pgfpathlineto{\pgfqpoint{1.503712in}{1.366079in}}%
\pgfpathlineto{\pgfqpoint{1.501226in}{1.367962in}}%
\pgfpathlineto{\pgfqpoint{1.485569in}{1.376529in}}%
\pgfpathlineto{\pgfqpoint{1.475033in}{1.379691in}}%
\pgfpathlineto{\pgfqpoint{1.469913in}{1.381905in}}%
\pgfpathlineto{\pgfqpoint{1.454256in}{1.383048in}}%
\pgfpathlineto{\pgfqpoint{1.442725in}{1.379691in}}%
\pgfpathlineto{\pgfqpoint{1.438599in}{1.378863in}}%
\pgfpathlineto{\pgfqpoint{1.422943in}{1.371857in}}%
\pgfpathlineto{\pgfqpoint{1.414449in}{1.366079in}}%
\pgfpathlineto{\pgfqpoint{1.407286in}{1.361897in}}%
\pgfpathlineto{\pgfqpoint{1.394772in}{1.352468in}}%
\pgfpathlineto{\pgfqpoint{1.391630in}{1.350144in}}%
\pgfpathlineto{\pgfqpoint{1.378325in}{1.338857in}}%
\pgfpathlineto{\pgfqpoint{1.375973in}{1.336666in}}%
\pgfpathlineto{\pgfqpoint{1.364049in}{1.325246in}}%
\pgfpathlineto{\pgfqpoint{1.360317in}{1.320859in}}%
\pgfpathlineto{\pgfqpoint{1.351695in}{1.311635in}}%
\pgfpathlineto{\pgfqpoint{1.344660in}{1.301014in}}%
\pgfpathlineto{\pgfqpoint{1.342091in}{1.298024in}}%
\pgfpathlineto{\pgfqpoint{1.335526in}{1.284413in}}%
\pgfpathlineto{\pgfqpoint{1.334434in}{1.270802in}}%
\pgfpathlineto{\pgfqpoint{1.338807in}{1.257191in}}%
\pgfpathlineto{\pgfqpoint{1.344660in}{1.249005in}}%
\pgfpathlineto{\pgfqpoint{1.347631in}{1.243579in}}%
\pgfpathlineto{\pgfqpoint{1.358995in}{1.229968in}}%
\pgfpathlineto{\pgfqpoint{1.360317in}{1.228720in}}%
\pgfpathlineto{\pgfqpoint{1.372317in}{1.216357in}}%
\pgfpathlineto{\pgfqpoint{1.375973in}{1.213074in}}%
\pgfpathlineto{\pgfqpoint{1.387853in}{1.202746in}}%
\pgfpathlineto{\pgfqpoint{1.391630in}{1.199567in}}%
\pgfpathlineto{\pgfqpoint{1.405850in}{1.189135in}}%
\pgfpathlineto{\pgfqpoint{1.407286in}{1.187986in}}%
\pgfpathlineto{\pgfqpoint{1.422943in}{1.178107in}}%
\pgfpathlineto{\pgfqpoint{1.429184in}{1.175524in}}%
\pgfpathlineto{\pgfqpoint{1.438599in}{1.170435in}}%
\pgfpathclose%
\pgfpathmoveto{\pgfqpoint{1.426633in}{1.202746in}}%
\pgfpathlineto{\pgfqpoint{1.422943in}{1.204204in}}%
\pgfpathlineto{\pgfqpoint{1.407286in}{1.213584in}}%
\pgfpathlineto{\pgfqpoint{1.403666in}{1.216357in}}%
\pgfpathlineto{\pgfqpoint{1.391630in}{1.226821in}}%
\pgfpathlineto{\pgfqpoint{1.388439in}{1.229968in}}%
\pgfpathlineto{\pgfqpoint{1.377650in}{1.243579in}}%
\pgfpathlineto{\pgfqpoint{1.375973in}{1.246788in}}%
\pgfpathlineto{\pgfqpoint{1.370508in}{1.257191in}}%
\pgfpathlineto{\pgfqpoint{1.367311in}{1.270802in}}%
\pgfpathlineto{\pgfqpoint{1.368110in}{1.284413in}}%
\pgfpathlineto{\pgfqpoint{1.372908in}{1.298024in}}%
\pgfpathlineto{\pgfqpoint{1.375973in}{1.302879in}}%
\pgfpathlineto{\pgfqpoint{1.381508in}{1.311635in}}%
\pgfpathlineto{\pgfqpoint{1.391630in}{1.322944in}}%
\pgfpathlineto{\pgfqpoint{1.393950in}{1.325246in}}%
\pgfpathlineto{\pgfqpoint{1.407286in}{1.336143in}}%
\pgfpathlineto{\pgfqpoint{1.411548in}{1.338857in}}%
\pgfpathlineto{\pgfqpoint{1.422943in}{1.345612in}}%
\pgfpathlineto{\pgfqpoint{1.438599in}{1.351663in}}%
\pgfpathlineto{\pgfqpoint{1.443250in}{1.352468in}}%
\pgfpathlineto{\pgfqpoint{1.454256in}{1.354502in}}%
\pgfpathlineto{\pgfqpoint{1.469913in}{1.353776in}}%
\pgfpathlineto{\pgfqpoint{1.474679in}{1.352468in}}%
\pgfpathlineto{\pgfqpoint{1.485569in}{1.349647in}}%
\pgfpathlineto{\pgfqpoint{1.501226in}{1.342248in}}%
\pgfpathlineto{\pgfqpoint{1.506363in}{1.338857in}}%
\pgfpathlineto{\pgfqpoint{1.516882in}{1.331355in}}%
\pgfpathlineto{\pgfqpoint{1.523909in}{1.325246in}}%
\pgfpathlineto{\pgfqpoint{1.532539in}{1.316102in}}%
\pgfpathlineto{\pgfqpoint{1.536439in}{1.311635in}}%
\pgfpathlineto{\pgfqpoint{1.544950in}{1.298024in}}%
\pgfpathlineto{\pgfqpoint{1.548195in}{1.288556in}}%
\pgfpathlineto{\pgfqpoint{1.549700in}{1.284413in}}%
\pgfpathlineto{\pgfqpoint{1.550535in}{1.270802in}}%
\pgfpathlineto{\pgfqpoint{1.548195in}{1.261233in}}%
\pgfpathlineto{\pgfqpoint{1.547269in}{1.257191in}}%
\pgfpathlineto{\pgfqpoint{1.540308in}{1.243579in}}%
\pgfpathlineto{\pgfqpoint{1.532539in}{1.233673in}}%
\pgfpathlineto{\pgfqpoint{1.529417in}{1.229968in}}%
\pgfpathlineto{\pgfqpoint{1.516882in}{1.218375in}}%
\pgfpathlineto{\pgfqpoint{1.514235in}{1.216357in}}%
\pgfpathlineto{\pgfqpoint{1.501226in}{1.207558in}}%
\pgfpathlineto{\pgfqpoint{1.491154in}{1.202746in}}%
\pgfpathlineto{\pgfqpoint{1.485569in}{1.200082in}}%
\pgfpathlineto{\pgfqpoint{1.469913in}{1.195910in}}%
\pgfpathlineto{\pgfqpoint{1.454256in}{1.195216in}}%
\pgfpathlineto{\pgfqpoint{1.438599in}{1.197995in}}%
\pgfpathlineto{\pgfqpoint{1.426633in}{1.202746in}}%
\pgfpathclose%
\pgfpathmoveto{\pgfqpoint{1.751731in}{1.169294in}}%
\pgfpathlineto{\pgfqpoint{1.767387in}{1.166444in}}%
\pgfpathlineto{\pgfqpoint{1.783044in}{1.168343in}}%
\pgfpathlineto{\pgfqpoint{1.798700in}{1.175005in}}%
\pgfpathlineto{\pgfqpoint{1.799440in}{1.175524in}}%
\pgfpathlineto{\pgfqpoint{1.814357in}{1.183617in}}%
\pgfpathlineto{\pgfqpoint{1.821817in}{1.189135in}}%
\pgfpathlineto{\pgfqpoint{1.830014in}{1.194668in}}%
\pgfpathlineto{\pgfqpoint{1.839996in}{1.202746in}}%
\pgfpathlineto{\pgfqpoint{1.845670in}{1.207476in}}%
\pgfpathlineto{\pgfqpoint{1.855574in}{1.216357in}}%
\pgfpathlineto{\pgfqpoint{1.861327in}{1.222276in}}%
\pgfpathlineto{\pgfqpoint{1.869109in}{1.229968in}}%
\pgfpathlineto{\pgfqpoint{1.876983in}{1.239998in}}%
\pgfpathlineto{\pgfqpoint{1.880356in}{1.243579in}}%
\pgfpathlineto{\pgfqpoint{1.888904in}{1.257191in}}%
\pgfpathlineto{\pgfqpoint{1.892640in}{1.270598in}}%
\pgfpathlineto{\pgfqpoint{1.892727in}{1.270802in}}%
\pgfpathlineto{\pgfqpoint{1.892640in}{1.271609in}}%
\pgfpathlineto{\pgfqpoint{1.891749in}{1.284413in}}%
\pgfpathlineto{\pgfqpoint{1.886057in}{1.298024in}}%
\pgfpathlineto{\pgfqpoint{1.876983in}{1.309900in}}%
\pgfpathlineto{\pgfqpoint{1.875889in}{1.311635in}}%
\pgfpathlineto{\pgfqpoint{1.863863in}{1.325246in}}%
\pgfpathlineto{\pgfqpoint{1.861327in}{1.327602in}}%
\pgfpathlineto{\pgfqpoint{1.849639in}{1.338857in}}%
\pgfpathlineto{\pgfqpoint{1.845670in}{1.342331in}}%
\pgfpathlineto{\pgfqpoint{1.833185in}{1.352468in}}%
\pgfpathlineto{\pgfqpoint{1.830014in}{1.355075in}}%
\pgfpathlineto{\pgfqpoint{1.814357in}{1.365837in}}%
\pgfpathlineto{\pgfqpoint{1.813881in}{1.366079in}}%
\pgfpathlineto{\pgfqpoint{1.798700in}{1.375128in}}%
\pgfpathlineto{\pgfqpoint{1.785635in}{1.379691in}}%
\pgfpathlineto{\pgfqpoint{1.783044in}{1.380990in}}%
\pgfpathlineto{\pgfqpoint{1.767387in}{1.383277in}}%
\pgfpathlineto{\pgfqpoint{1.751731in}{1.379845in}}%
\pgfpathlineto{\pgfqpoint{1.751459in}{1.379691in}}%
\pgfpathlineto{\pgfqpoint{1.736074in}{1.373570in}}%
\pgfpathlineto{\pgfqpoint{1.724349in}{1.366079in}}%
\pgfpathlineto{\pgfqpoint{1.720418in}{1.363929in}}%
\pgfpathlineto{\pgfqpoint{1.704761in}{1.352569in}}%
\pgfpathlineto{\pgfqpoint{1.704642in}{1.352468in}}%
\pgfpathlineto{\pgfqpoint{1.689104in}{1.339560in}}%
\pgfpathlineto{\pgfqpoint{1.688304in}{1.338857in}}%
\pgfpathlineto{\pgfqpoint{1.674163in}{1.325246in}}%
\pgfpathlineto{\pgfqpoint{1.673448in}{1.324405in}}%
\pgfpathlineto{\pgfqpoint{1.661811in}{1.311635in}}%
\pgfpathlineto{\pgfqpoint{1.657791in}{1.305398in}}%
\pgfpathlineto{\pgfqpoint{1.651845in}{1.298024in}}%
\pgfpathlineto{\pgfqpoint{1.645760in}{1.284413in}}%
\pgfpathlineto{\pgfqpoint{1.644747in}{1.270802in}}%
\pgfpathlineto{\pgfqpoint{1.648801in}{1.257191in}}%
\pgfpathlineto{\pgfqpoint{1.657791in}{1.243794in}}%
\pgfpathlineto{\pgfqpoint{1.657906in}{1.243579in}}%
\pgfpathlineto{\pgfqpoint{1.668827in}{1.229968in}}%
\pgfpathlineto{\pgfqpoint{1.673448in}{1.225491in}}%
\pgfpathlineto{\pgfqpoint{1.682301in}{1.216357in}}%
\pgfpathlineto{\pgfqpoint{1.689104in}{1.210240in}}%
\pgfpathlineto{\pgfqpoint{1.697890in}{1.202746in}}%
\pgfpathlineto{\pgfqpoint{1.704761in}{1.197065in}}%
\pgfpathlineto{\pgfqpoint{1.716004in}{1.189135in}}%
\pgfpathlineto{\pgfqpoint{1.720418in}{1.185733in}}%
\pgfpathlineto{\pgfqpoint{1.736074in}{1.176552in}}%
\pgfpathlineto{\pgfqpoint{1.738858in}{1.175524in}}%
\pgfpathlineto{\pgfqpoint{1.751731in}{1.169294in}}%
\pgfpathclose%
\pgfpathmoveto{\pgfqpoint{1.736040in}{1.202746in}}%
\pgfpathlineto{\pgfqpoint{1.720418in}{1.211444in}}%
\pgfpathlineto{\pgfqpoint{1.713762in}{1.216357in}}%
\pgfpathlineto{\pgfqpoint{1.704761in}{1.223880in}}%
\pgfpathlineto{\pgfqpoint{1.698479in}{1.229968in}}%
\pgfpathlineto{\pgfqpoint{1.689104in}{1.241662in}}%
\pgfpathlineto{\pgfqpoint{1.687614in}{1.243579in}}%
\pgfpathlineto{\pgfqpoint{1.680520in}{1.257191in}}%
\pgfpathlineto{\pgfqpoint{1.677374in}{1.270802in}}%
\pgfpathlineto{\pgfqpoint{1.678160in}{1.284413in}}%
\pgfpathlineto{\pgfqpoint{1.682883in}{1.298024in}}%
\pgfpathlineto{\pgfqpoint{1.689104in}{1.307893in}}%
\pgfpathlineto{\pgfqpoint{1.691516in}{1.311635in}}%
\pgfpathlineto{\pgfqpoint{1.703869in}{1.325246in}}%
\pgfpathlineto{\pgfqpoint{1.704761in}{1.326059in}}%
\pgfpathlineto{\pgfqpoint{1.720418in}{1.338340in}}%
\pgfpathlineto{\pgfqpoint{1.721285in}{1.338857in}}%
\pgfpathlineto{\pgfqpoint{1.736074in}{1.347092in}}%
\pgfpathlineto{\pgfqpoint{1.751730in}{1.352468in}}%
\pgfpathlineto{\pgfqpoint{1.751731in}{1.352469in}}%
\pgfpathlineto{\pgfqpoint{1.767387in}{1.354647in}}%
\pgfpathlineto{\pgfqpoint{1.783044in}{1.353195in}}%
\pgfpathlineto{\pgfqpoint{1.785328in}{1.352468in}}%
\pgfpathlineto{\pgfqpoint{1.798700in}{1.348437in}}%
\pgfpathlineto{\pgfqpoint{1.814357in}{1.340366in}}%
\pgfpathlineto{\pgfqpoint{1.816540in}{1.338857in}}%
\pgfpathlineto{\pgfqpoint{1.830014in}{1.328769in}}%
\pgfpathlineto{\pgfqpoint{1.833964in}{1.325246in}}%
\pgfpathlineto{\pgfqpoint{1.845670in}{1.312504in}}%
\pgfpathlineto{\pgfqpoint{1.846424in}{1.311635in}}%
\pgfpathlineto{\pgfqpoint{1.854998in}{1.298024in}}%
\pgfpathlineto{\pgfqpoint{1.859667in}{1.284413in}}%
\pgfpathlineto{\pgfqpoint{1.860444in}{1.270802in}}%
\pgfpathlineto{\pgfqpoint{1.857334in}{1.257191in}}%
\pgfpathlineto{\pgfqpoint{1.850321in}{1.243579in}}%
\pgfpathlineto{\pgfqpoint{1.845670in}{1.237610in}}%
\pgfpathlineto{\pgfqpoint{1.839401in}{1.229968in}}%
\pgfpathlineto{\pgfqpoint{1.830014in}{1.221063in}}%
\pgfpathlineto{\pgfqpoint{1.824131in}{1.216357in}}%
\pgfpathlineto{\pgfqpoint{1.814357in}{1.209435in}}%
\pgfpathlineto{\pgfqpoint{1.801439in}{1.202746in}}%
\pgfpathlineto{\pgfqpoint{1.798700in}{1.201335in}}%
\pgfpathlineto{\pgfqpoint{1.783044in}{1.196466in}}%
\pgfpathlineto{\pgfqpoint{1.767387in}{1.195077in}}%
\pgfpathlineto{\pgfqpoint{1.751731in}{1.197160in}}%
\pgfpathlineto{\pgfqpoint{1.736074in}{1.202727in}}%
\pgfpathlineto{\pgfqpoint{1.736040in}{1.202746in}}%
\pgfpathclose%
\pgfpathmoveto{\pgfqpoint{0.499205in}{1.444166in}}%
\pgfpathlineto{\pgfqpoint{0.514862in}{1.437991in}}%
\pgfpathlineto{\pgfqpoint{0.530519in}{1.436230in}}%
\pgfpathlineto{\pgfqpoint{0.546175in}{1.438872in}}%
\pgfpathlineto{\pgfqpoint{0.561832in}{1.445932in}}%
\pgfpathlineto{\pgfqpoint{0.564402in}{1.447746in}}%
\pgfpathlineto{\pgfqpoint{0.577488in}{1.455174in}}%
\pgfpathlineto{\pgfqpoint{0.585714in}{1.461357in}}%
\pgfpathlineto{\pgfqpoint{0.593145in}{1.466590in}}%
\pgfpathlineto{\pgfqpoint{0.603291in}{1.474968in}}%
\pgfpathlineto{\pgfqpoint{0.608801in}{1.479759in}}%
\pgfpathlineto{\pgfqpoint{0.618438in}{1.488579in}}%
\pgfpathlineto{\pgfqpoint{0.624458in}{1.495039in}}%
\pgfpathlineto{\pgfqpoint{0.631570in}{1.502191in}}%
\pgfpathlineto{\pgfqpoint{0.640115in}{1.513567in}}%
\pgfpathlineto{\pgfqpoint{0.642201in}{1.515802in}}%
\pgfpathlineto{\pgfqpoint{0.650322in}{1.529413in}}%
\pgfpathlineto{\pgfqpoint{0.653361in}{1.543024in}}%
\pgfpathlineto{\pgfqpoint{0.651335in}{1.556635in}}%
\pgfpathlineto{\pgfqpoint{0.644233in}{1.570246in}}%
\pgfpathlineto{\pgfqpoint{0.640115in}{1.574974in}}%
\pgfpathlineto{\pgfqpoint{0.633909in}{1.583857in}}%
\pgfpathlineto{\pgfqpoint{0.624458in}{1.593762in}}%
\pgfpathlineto{\pgfqpoint{0.621153in}{1.597468in}}%
\pgfpathlineto{\pgfqpoint{0.608801in}{1.609019in}}%
\pgfpathlineto{\pgfqpoint{0.606485in}{1.611079in}}%
\pgfpathlineto{\pgfqpoint{0.593145in}{1.622124in}}%
\pgfpathlineto{\pgfqpoint{0.589511in}{1.624691in}}%
\pgfpathlineto{\pgfqpoint{0.577488in}{1.633549in}}%
\pgfpathlineto{\pgfqpoint{0.568936in}{1.638302in}}%
\pgfpathlineto{\pgfqpoint{0.561832in}{1.643051in}}%
\pgfpathlineto{\pgfqpoint{0.546175in}{1.649655in}}%
\pgfpathlineto{\pgfqpoint{0.531883in}{1.651913in}}%
\pgfpathlineto{\pgfqpoint{0.530519in}{1.652245in}}%
\pgfpathlineto{\pgfqpoint{0.528478in}{1.651913in}}%
\pgfpathlineto{\pgfqpoint{0.514862in}{1.650479in}}%
\pgfpathlineto{\pgfqpoint{0.499205in}{1.644704in}}%
\pgfpathlineto{\pgfqpoint{0.488943in}{1.638302in}}%
\pgfpathlineto{\pgfqpoint{0.483549in}{1.635514in}}%
\pgfpathlineto{\pgfqpoint{0.468238in}{1.624691in}}%
\pgfpathlineto{\pgfqpoint{0.467892in}{1.624457in}}%
\pgfpathlineto{\pgfqpoint{0.452236in}{1.611818in}}%
\pgfpathlineto{\pgfqpoint{0.451386in}{1.611079in}}%
\pgfpathlineto{\pgfqpoint{0.436848in}{1.597468in}}%
\pgfpathlineto{\pgfqpoint{0.436579in}{1.597168in}}%
\pgfpathlineto{\pgfqpoint{0.424129in}{1.583857in}}%
\pgfpathlineto{\pgfqpoint{0.420923in}{1.579168in}}%
\pgfpathlineto{\pgfqpoint{0.413559in}{1.570246in}}%
\pgfpathlineto{\pgfqpoint{0.406915in}{1.556635in}}%
\pgfpathlineto{\pgfqpoint{0.405266in}{1.544798in}}%
\pgfpathlineto{\pgfqpoint{0.404884in}{1.543024in}}%
\pgfpathlineto{\pgfqpoint{0.405266in}{1.541837in}}%
\pgfpathlineto{\pgfqpoint{0.407863in}{1.529413in}}%
\pgfpathlineto{\pgfqpoint{0.415459in}{1.515802in}}%
\pgfpathlineto{\pgfqpoint{0.420923in}{1.509626in}}%
\pgfpathlineto{\pgfqpoint{0.426389in}{1.502191in}}%
\pgfpathlineto{\pgfqpoint{0.436579in}{1.491738in}}%
\pgfpathlineto{\pgfqpoint{0.439531in}{1.488579in}}%
\pgfpathlineto{\pgfqpoint{0.452236in}{1.476982in}}%
\pgfpathlineto{\pgfqpoint{0.454606in}{1.474968in}}%
\pgfpathlineto{\pgfqpoint{0.467892in}{1.464231in}}%
\pgfpathlineto{\pgfqpoint{0.472156in}{1.461357in}}%
\pgfpathlineto{\pgfqpoint{0.483549in}{1.453141in}}%
\pgfpathlineto{\pgfqpoint{0.493767in}{1.447746in}}%
\pgfpathlineto{\pgfqpoint{0.499205in}{1.444166in}}%
\pgfpathclose%
\pgfpathmoveto{\pgfqpoint{0.491094in}{1.474968in}}%
\pgfpathlineto{\pgfqpoint{0.483549in}{1.478950in}}%
\pgfpathlineto{\pgfqpoint{0.470193in}{1.488579in}}%
\pgfpathlineto{\pgfqpoint{0.467892in}{1.490493in}}%
\pgfpathlineto{\pgfqpoint{0.456010in}{1.502191in}}%
\pgfpathlineto{\pgfqpoint{0.452236in}{1.507103in}}%
\pgfpathlineto{\pgfqpoint{0.445869in}{1.515802in}}%
\pgfpathlineto{\pgfqpoint{0.439638in}{1.529413in}}%
\pgfpathlineto{\pgfqpoint{0.437306in}{1.543024in}}%
\pgfpathlineto{\pgfqpoint{0.438860in}{1.556635in}}%
\pgfpathlineto{\pgfqpoint{0.444310in}{1.570246in}}%
\pgfpathlineto{\pgfqpoint{0.452236in}{1.581853in}}%
\pgfpathlineto{\pgfqpoint{0.453668in}{1.583857in}}%
\pgfpathlineto{\pgfqpoint{0.466878in}{1.597468in}}%
\pgfpathlineto{\pgfqpoint{0.467892in}{1.598350in}}%
\pgfpathlineto{\pgfqpoint{0.483549in}{1.609834in}}%
\pgfpathlineto{\pgfqpoint{0.485854in}{1.611079in}}%
\pgfpathlineto{\pgfqpoint{0.499205in}{1.617969in}}%
\pgfpathlineto{\pgfqpoint{0.514862in}{1.622707in}}%
\pgfpathlineto{\pgfqpoint{0.530519in}{1.624059in}}%
\pgfpathlineto{\pgfqpoint{0.546175in}{1.622031in}}%
\pgfpathlineto{\pgfqpoint{0.561832in}{1.616614in}}%
\pgfpathlineto{\pgfqpoint{0.571838in}{1.611079in}}%
\pgfpathlineto{\pgfqpoint{0.577488in}{1.607798in}}%
\pgfpathlineto{\pgfqpoint{0.590943in}{1.597468in}}%
\pgfpathlineto{\pgfqpoint{0.593145in}{1.595469in}}%
\pgfpathlineto{\pgfqpoint{0.604221in}{1.583857in}}%
\pgfpathlineto{\pgfqpoint{0.608801in}{1.577298in}}%
\pgfpathlineto{\pgfqpoint{0.613604in}{1.570246in}}%
\pgfpathlineto{\pgfqpoint{0.619117in}{1.556635in}}%
\pgfpathlineto{\pgfqpoint{0.620689in}{1.543024in}}%
\pgfpathlineto{\pgfqpoint{0.618330in}{1.529413in}}%
\pgfpathlineto{\pgfqpoint{0.612027in}{1.515802in}}%
\pgfpathlineto{\pgfqpoint{0.608801in}{1.511383in}}%
\pgfpathlineto{\pgfqpoint{0.601899in}{1.502191in}}%
\pgfpathlineto{\pgfqpoint{0.593145in}{1.493385in}}%
\pgfpathlineto{\pgfqpoint{0.587617in}{1.488579in}}%
\pgfpathlineto{\pgfqpoint{0.577488in}{1.480969in}}%
\pgfpathlineto{\pgfqpoint{0.566914in}{1.474968in}}%
\pgfpathlineto{\pgfqpoint{0.561832in}{1.472164in}}%
\pgfpathlineto{\pgfqpoint{0.546175in}{1.466684in}}%
\pgfpathlineto{\pgfqpoint{0.530519in}{1.464634in}}%
\pgfpathlineto{\pgfqpoint{0.514862in}{1.466000in}}%
\pgfpathlineto{\pgfqpoint{0.499205in}{1.470793in}}%
\pgfpathlineto{\pgfqpoint{0.491094in}{1.474968in}}%
\pgfpathclose%
\pgfpathmoveto{\pgfqpoint{0.812337in}{1.442577in}}%
\pgfpathlineto{\pgfqpoint{0.827993in}{1.437287in}}%
\pgfpathlineto{\pgfqpoint{0.843650in}{1.436406in}}%
\pgfpathlineto{\pgfqpoint{0.859306in}{1.439930in}}%
\pgfpathlineto{\pgfqpoint{0.874716in}{1.447746in}}%
\pgfpathlineto{\pgfqpoint{0.874963in}{1.447846in}}%
\pgfpathlineto{\pgfqpoint{0.890620in}{1.457340in}}%
\pgfpathlineto{\pgfqpoint{0.895769in}{1.461357in}}%
\pgfpathlineto{\pgfqpoint{0.906276in}{1.469054in}}%
\pgfpathlineto{\pgfqpoint{0.913313in}{1.474968in}}%
\pgfpathlineto{\pgfqpoint{0.921933in}{1.482606in}}%
\pgfpathlineto{\pgfqpoint{0.928468in}{1.488579in}}%
\pgfpathlineto{\pgfqpoint{0.937589in}{1.498354in}}%
\pgfpathlineto{\pgfqpoint{0.941503in}{1.502191in}}%
\pgfpathlineto{\pgfqpoint{0.952063in}{1.515802in}}%
\pgfpathlineto{\pgfqpoint{0.953246in}{1.518222in}}%
\pgfpathlineto{\pgfqpoint{0.960412in}{1.529413in}}%
\pgfpathlineto{\pgfqpoint{0.963691in}{1.543024in}}%
\pgfpathlineto{\pgfqpoint{0.961505in}{1.556635in}}%
\pgfpathlineto{\pgfqpoint{0.953842in}{1.570246in}}%
\pgfpathlineto{\pgfqpoint{0.953246in}{1.570889in}}%
\pgfpathlineto{\pgfqpoint{0.943937in}{1.583857in}}%
\pgfpathlineto{\pgfqpoint{0.937589in}{1.590343in}}%
\pgfpathlineto{\pgfqpoint{0.931225in}{1.597468in}}%
\pgfpathlineto{\pgfqpoint{0.921933in}{1.606146in}}%
\pgfpathlineto{\pgfqpoint{0.916492in}{1.611079in}}%
\pgfpathlineto{\pgfqpoint{0.906276in}{1.619689in}}%
\pgfpathlineto{\pgfqpoint{0.899468in}{1.624691in}}%
\pgfpathlineto{\pgfqpoint{0.890620in}{1.631457in}}%
\pgfpathlineto{\pgfqpoint{0.879082in}{1.638302in}}%
\pgfpathlineto{\pgfqpoint{0.874963in}{1.641234in}}%
\pgfpathlineto{\pgfqpoint{0.859306in}{1.648665in}}%
\pgfpathlineto{\pgfqpoint{0.843884in}{1.651913in}}%
\pgfpathlineto{\pgfqpoint{0.843650in}{1.651989in}}%
\pgfpathlineto{\pgfqpoint{0.842722in}{1.651913in}}%
\pgfpathlineto{\pgfqpoint{0.827993in}{1.651138in}}%
\pgfpathlineto{\pgfqpoint{0.812337in}{1.646190in}}%
\pgfpathlineto{\pgfqpoint{0.798676in}{1.638302in}}%
\pgfpathlineto{\pgfqpoint{0.796680in}{1.637350in}}%
\pgfpathlineto{\pgfqpoint{0.781024in}{1.626895in}}%
\pgfpathlineto{\pgfqpoint{0.778314in}{1.624691in}}%
\pgfpathlineto{\pgfqpoint{0.765367in}{1.614530in}}%
\pgfpathlineto{\pgfqpoint{0.761372in}{1.611079in}}%
\pgfpathlineto{\pgfqpoint{0.749710in}{1.600225in}}%
\pgfpathlineto{\pgfqpoint{0.746712in}{1.597468in}}%
\pgfpathlineto{\pgfqpoint{0.734332in}{1.583857in}}%
\pgfpathlineto{\pgfqpoint{0.734054in}{1.583443in}}%
\pgfpathlineto{\pgfqpoint{0.723646in}{1.570246in}}%
\pgfpathlineto{\pgfqpoint{0.718397in}{1.558887in}}%
\pgfpathlineto{\pgfqpoint{0.716903in}{1.556635in}}%
\pgfpathlineto{\pgfqpoint{0.714272in}{1.543024in}}%
\pgfpathlineto{\pgfqpoint{0.718219in}{1.529413in}}%
\pgfpathlineto{\pgfqpoint{0.718397in}{1.529176in}}%
\pgfpathlineto{\pgfqpoint{0.725438in}{1.515802in}}%
\pgfpathlineto{\pgfqpoint{0.734054in}{1.505609in}}%
\pgfpathlineto{\pgfqpoint{0.736527in}{1.502191in}}%
\pgfpathlineto{\pgfqpoint{0.749595in}{1.488579in}}%
\pgfpathlineto{\pgfqpoint{0.749710in}{1.488476in}}%
\pgfpathlineto{\pgfqpoint{0.764559in}{1.474968in}}%
\pgfpathlineto{\pgfqpoint{0.765367in}{1.474272in}}%
\pgfpathlineto{\pgfqpoint{0.781024in}{1.461979in}}%
\pgfpathlineto{\pgfqpoint{0.781992in}{1.461357in}}%
\pgfpathlineto{\pgfqpoint{0.796680in}{1.451241in}}%
\pgfpathlineto{\pgfqpoint{0.803854in}{1.447746in}}%
\pgfpathlineto{\pgfqpoint{0.812337in}{1.442577in}}%
\pgfpathclose%
\pgfpathmoveto{\pgfqpoint{0.800985in}{1.474968in}}%
\pgfpathlineto{\pgfqpoint{0.796680in}{1.477065in}}%
\pgfpathlineto{\pgfqpoint{0.781024in}{1.487804in}}%
\pgfpathlineto{\pgfqpoint{0.780089in}{1.488579in}}%
\pgfpathlineto{\pgfqpoint{0.765962in}{1.502191in}}%
\pgfpathlineto{\pgfqpoint{0.765367in}{1.502945in}}%
\pgfpathlineto{\pgfqpoint{0.755895in}{1.515802in}}%
\pgfpathlineto{\pgfqpoint{0.749710in}{1.529412in}}%
\pgfpathlineto{\pgfqpoint{0.749710in}{1.529413in}}%
\pgfpathlineto{\pgfqpoint{0.747204in}{1.543024in}}%
\pgfpathlineto{\pgfqpoint{0.748874in}{1.556635in}}%
\pgfpathlineto{\pgfqpoint{0.749710in}{1.558621in}}%
\pgfpathlineto{\pgfqpoint{0.754348in}{1.570246in}}%
\pgfpathlineto{\pgfqpoint{0.763631in}{1.583857in}}%
\pgfpathlineto{\pgfqpoint{0.765367in}{1.585755in}}%
\pgfpathlineto{\pgfqpoint{0.776972in}{1.597468in}}%
\pgfpathlineto{\pgfqpoint{0.781024in}{1.600903in}}%
\pgfpathlineto{\pgfqpoint{0.795681in}{1.611079in}}%
\pgfpathlineto{\pgfqpoint{0.796680in}{1.611735in}}%
\pgfpathlineto{\pgfqpoint{0.812337in}{1.619189in}}%
\pgfpathlineto{\pgfqpoint{0.827993in}{1.623248in}}%
\pgfpathlineto{\pgfqpoint{0.843650in}{1.623923in}}%
\pgfpathlineto{\pgfqpoint{0.859306in}{1.621220in}}%
\pgfpathlineto{\pgfqpoint{0.874963in}{1.615123in}}%
\pgfpathlineto{\pgfqpoint{0.881830in}{1.611079in}}%
\pgfpathlineto{\pgfqpoint{0.890620in}{1.605630in}}%
\pgfpathlineto{\pgfqpoint{0.900863in}{1.597468in}}%
\pgfpathlineto{\pgfqpoint{0.906276in}{1.592354in}}%
\pgfpathlineto{\pgfqpoint{0.914238in}{1.583857in}}%
\pgfpathlineto{\pgfqpoint{0.921933in}{1.572627in}}%
\pgfpathlineto{\pgfqpoint{0.923556in}{1.570246in}}%
\pgfpathlineto{\pgfqpoint{0.929157in}{1.556635in}}%
\pgfpathlineto{\pgfqpoint{0.930754in}{1.543024in}}%
\pgfpathlineto{\pgfqpoint{0.928358in}{1.529413in}}%
\pgfpathlineto{\pgfqpoint{0.921954in}{1.515802in}}%
\pgfpathlineto{\pgfqpoint{0.921933in}{1.515772in}}%
\pgfpathlineto{\pgfqpoint{0.911928in}{1.502191in}}%
\pgfpathlineto{\pgfqpoint{0.906276in}{1.496404in}}%
\pgfpathlineto{\pgfqpoint{0.897623in}{1.488579in}}%
\pgfpathlineto{\pgfqpoint{0.890620in}{1.483118in}}%
\pgfpathlineto{\pgfqpoint{0.877169in}{1.474968in}}%
\pgfpathlineto{\pgfqpoint{0.874963in}{1.473673in}}%
\pgfpathlineto{\pgfqpoint{0.859306in}{1.467506in}}%
\pgfpathlineto{\pgfqpoint{0.843650in}{1.464770in}}%
\pgfpathlineto{\pgfqpoint{0.827993in}{1.465454in}}%
\pgfpathlineto{\pgfqpoint{0.812337in}{1.469560in}}%
\pgfpathlineto{\pgfqpoint{0.800985in}{1.474968in}}%
\pgfpathclose%
\pgfpathmoveto{\pgfqpoint{1.125468in}{1.441165in}}%
\pgfpathlineto{\pgfqpoint{1.141125in}{1.436758in}}%
\pgfpathlineto{\pgfqpoint{1.156781in}{1.436758in}}%
\pgfpathlineto{\pgfqpoint{1.172438in}{1.441165in}}%
\pgfpathlineto{\pgfqpoint{1.184211in}{1.447746in}}%
\pgfpathlineto{\pgfqpoint{1.188094in}{1.449476in}}%
\pgfpathlineto{\pgfqpoint{1.203751in}{1.459636in}}%
\pgfpathlineto{\pgfqpoint{1.205885in}{1.461357in}}%
\pgfpathlineto{\pgfqpoint{1.219407in}{1.471616in}}%
\pgfpathlineto{\pgfqpoint{1.223340in}{1.474968in}}%
\pgfpathlineto{\pgfqpoint{1.235064in}{1.485515in}}%
\pgfpathlineto{\pgfqpoint{1.238435in}{1.488579in}}%
\pgfpathlineto{\pgfqpoint{1.250721in}{1.501665in}}%
\pgfpathlineto{\pgfqpoint{1.251274in}{1.502191in}}%
\pgfpathlineto{\pgfqpoint{1.262325in}{1.515802in}}%
\pgfpathlineto{\pgfqpoint{1.266377in}{1.523819in}}%
\pgfpathlineto{\pgfqpoint{1.270237in}{1.529413in}}%
\pgfpathlineto{\pgfqpoint{1.273812in}{1.543024in}}%
\pgfpathlineto{\pgfqpoint{1.271429in}{1.556635in}}%
\pgfpathlineto{\pgfqpoint{1.266377in}{1.564947in}}%
\pgfpathlineto{\pgfqpoint{1.264026in}{1.570246in}}%
\pgfpathlineto{\pgfqpoint{1.253821in}{1.583857in}}%
\pgfpathlineto{\pgfqpoint{1.250721in}{1.586927in}}%
\pgfpathlineto{\pgfqpoint{1.241248in}{1.597468in}}%
\pgfpathlineto{\pgfqpoint{1.235064in}{1.603211in}}%
\pgfpathlineto{\pgfqpoint{1.226517in}{1.611079in}}%
\pgfpathlineto{\pgfqpoint{1.219407in}{1.617156in}}%
\pgfpathlineto{\pgfqpoint{1.209505in}{1.624691in}}%
\pgfpathlineto{\pgfqpoint{1.203751in}{1.629238in}}%
\pgfpathlineto{\pgfqpoint{1.189355in}{1.638302in}}%
\pgfpathlineto{\pgfqpoint{1.188094in}{1.639251in}}%
\pgfpathlineto{\pgfqpoint{1.172438in}{1.647510in}}%
\pgfpathlineto{\pgfqpoint{1.156781in}{1.651632in}}%
\pgfpathlineto{\pgfqpoint{1.141125in}{1.651632in}}%
\pgfpathlineto{\pgfqpoint{1.125468in}{1.647510in}}%
\pgfpathlineto{\pgfqpoint{1.109812in}{1.639251in}}%
\pgfpathlineto{\pgfqpoint{1.108551in}{1.638302in}}%
\pgfpathlineto{\pgfqpoint{1.094155in}{1.629238in}}%
\pgfpathlineto{\pgfqpoint{1.088401in}{1.624691in}}%
\pgfpathlineto{\pgfqpoint{1.078498in}{1.617156in}}%
\pgfpathlineto{\pgfqpoint{1.071389in}{1.611079in}}%
\pgfpathlineto{\pgfqpoint{1.062842in}{1.603211in}}%
\pgfpathlineto{\pgfqpoint{1.056658in}{1.597468in}}%
\pgfpathlineto{\pgfqpoint{1.047185in}{1.586927in}}%
\pgfpathlineto{\pgfqpoint{1.044085in}{1.583857in}}%
\pgfpathlineto{\pgfqpoint{1.033880in}{1.570246in}}%
\pgfpathlineto{\pgfqpoint{1.031529in}{1.564947in}}%
\pgfpathlineto{\pgfqpoint{1.026477in}{1.556635in}}%
\pgfpathlineto{\pgfqpoint{1.024094in}{1.543024in}}%
\pgfpathlineto{\pgfqpoint{1.027669in}{1.529413in}}%
\pgfpathlineto{\pgfqpoint{1.031529in}{1.523819in}}%
\pgfpathlineto{\pgfqpoint{1.035581in}{1.515802in}}%
\pgfpathlineto{\pgfqpoint{1.046632in}{1.502191in}}%
\pgfpathlineto{\pgfqpoint{1.047185in}{1.501665in}}%
\pgfpathlineto{\pgfqpoint{1.059471in}{1.488579in}}%
\pgfpathlineto{\pgfqpoint{1.062842in}{1.485515in}}%
\pgfpathlineto{\pgfqpoint{1.074566in}{1.474968in}}%
\pgfpathlineto{\pgfqpoint{1.078498in}{1.471616in}}%
\pgfpathlineto{\pgfqpoint{1.092020in}{1.461357in}}%
\pgfpathlineto{\pgfqpoint{1.094155in}{1.459636in}}%
\pgfpathlineto{\pgfqpoint{1.109812in}{1.449476in}}%
\pgfpathlineto{\pgfqpoint{1.113695in}{1.447746in}}%
\pgfpathlineto{\pgfqpoint{1.125468in}{1.441165in}}%
\pgfpathclose%
\pgfpathmoveto{\pgfqpoint{1.110584in}{1.474968in}}%
\pgfpathlineto{\pgfqpoint{1.109812in}{1.475312in}}%
\pgfpathlineto{\pgfqpoint{1.094155in}{1.485397in}}%
\pgfpathlineto{\pgfqpoint{1.090207in}{1.488579in}}%
\pgfpathlineto{\pgfqpoint{1.078498in}{1.499545in}}%
\pgfpathlineto{\pgfqpoint{1.075951in}{1.502191in}}%
\pgfpathlineto{\pgfqpoint{1.065931in}{1.515802in}}%
\pgfpathlineto{\pgfqpoint{1.062842in}{1.522525in}}%
\pgfpathlineto{\pgfqpoint{1.059583in}{1.529413in}}%
\pgfpathlineto{\pgfqpoint{1.057138in}{1.543024in}}%
\pgfpathlineto{\pgfqpoint{1.058767in}{1.556635in}}%
\pgfpathlineto{\pgfqpoint{1.062842in}{1.566410in}}%
\pgfpathlineto{\pgfqpoint{1.064389in}{1.570246in}}%
\pgfpathlineto{\pgfqpoint{1.073641in}{1.583857in}}%
\pgfpathlineto{\pgfqpoint{1.078498in}{1.589114in}}%
\pgfpathlineto{\pgfqpoint{1.087036in}{1.597468in}}%
\pgfpathlineto{\pgfqpoint{1.094155in}{1.603330in}}%
\pgfpathlineto{\pgfqpoint{1.105932in}{1.611079in}}%
\pgfpathlineto{\pgfqpoint{1.109812in}{1.613496in}}%
\pgfpathlineto{\pgfqpoint{1.125468in}{1.620272in}}%
\pgfpathlineto{\pgfqpoint{1.141125in}{1.623653in}}%
\pgfpathlineto{\pgfqpoint{1.156781in}{1.623653in}}%
\pgfpathlineto{\pgfqpoint{1.172438in}{1.620272in}}%
\pgfpathlineto{\pgfqpoint{1.188094in}{1.613496in}}%
\pgfpathlineto{\pgfqpoint{1.191973in}{1.611079in}}%
\pgfpathlineto{\pgfqpoint{1.203751in}{1.603330in}}%
\pgfpathlineto{\pgfqpoint{1.210870in}{1.597468in}}%
\pgfpathlineto{\pgfqpoint{1.219407in}{1.589114in}}%
\pgfpathlineto{\pgfqpoint{1.224265in}{1.583857in}}%
\pgfpathlineto{\pgfqpoint{1.233517in}{1.570246in}}%
\pgfpathlineto{\pgfqpoint{1.235064in}{1.566410in}}%
\pgfpathlineto{\pgfqpoint{1.239138in}{1.556635in}}%
\pgfpathlineto{\pgfqpoint{1.240768in}{1.543024in}}%
\pgfpathlineto{\pgfqpoint{1.238323in}{1.529413in}}%
\pgfpathlineto{\pgfqpoint{1.235064in}{1.522525in}}%
\pgfpathlineto{\pgfqpoint{1.231975in}{1.515802in}}%
\pgfpathlineto{\pgfqpoint{1.221955in}{1.502191in}}%
\pgfpathlineto{\pgfqpoint{1.219407in}{1.499545in}}%
\pgfpathlineto{\pgfqpoint{1.207699in}{1.488579in}}%
\pgfpathlineto{\pgfqpoint{1.203751in}{1.485397in}}%
\pgfpathlineto{\pgfqpoint{1.188094in}{1.475312in}}%
\pgfpathlineto{\pgfqpoint{1.187322in}{1.474968in}}%
\pgfpathlineto{\pgfqpoint{1.172438in}{1.468464in}}%
\pgfpathlineto{\pgfqpoint{1.156781in}{1.465043in}}%
\pgfpathlineto{\pgfqpoint{1.141125in}{1.465043in}}%
\pgfpathlineto{\pgfqpoint{1.125468in}{1.468464in}}%
\pgfpathlineto{\pgfqpoint{1.110584in}{1.474968in}}%
\pgfpathclose%
\pgfpathmoveto{\pgfqpoint{1.438599in}{1.439930in}}%
\pgfpathlineto{\pgfqpoint{1.454256in}{1.436406in}}%
\pgfpathlineto{\pgfqpoint{1.469913in}{1.437287in}}%
\pgfpathlineto{\pgfqpoint{1.485569in}{1.442577in}}%
\pgfpathlineto{\pgfqpoint{1.494052in}{1.447746in}}%
\pgfpathlineto{\pgfqpoint{1.501226in}{1.451241in}}%
\pgfpathlineto{\pgfqpoint{1.515914in}{1.461357in}}%
\pgfpathlineto{\pgfqpoint{1.516882in}{1.461979in}}%
\pgfpathlineto{\pgfqpoint{1.532539in}{1.474272in}}%
\pgfpathlineto{\pgfqpoint{1.533347in}{1.474968in}}%
\pgfpathlineto{\pgfqpoint{1.548195in}{1.488476in}}%
\pgfpathlineto{\pgfqpoint{1.548311in}{1.488579in}}%
\pgfpathlineto{\pgfqpoint{1.561378in}{1.502191in}}%
\pgfpathlineto{\pgfqpoint{1.563852in}{1.505609in}}%
\pgfpathlineto{\pgfqpoint{1.572468in}{1.515802in}}%
\pgfpathlineto{\pgfqpoint{1.579508in}{1.529176in}}%
\pgfpathlineto{\pgfqpoint{1.579687in}{1.529413in}}%
\pgfpathlineto{\pgfqpoint{1.583634in}{1.543024in}}%
\pgfpathlineto{\pgfqpoint{1.581003in}{1.556635in}}%
\pgfpathlineto{\pgfqpoint{1.579508in}{1.558887in}}%
\pgfpathlineto{\pgfqpoint{1.574260in}{1.570246in}}%
\pgfpathlineto{\pgfqpoint{1.563852in}{1.583443in}}%
\pgfpathlineto{\pgfqpoint{1.563573in}{1.583857in}}%
\pgfpathlineto{\pgfqpoint{1.551194in}{1.597468in}}%
\pgfpathlineto{\pgfqpoint{1.548195in}{1.600225in}}%
\pgfpathlineto{\pgfqpoint{1.536534in}{1.611079in}}%
\pgfpathlineto{\pgfqpoint{1.532539in}{1.614530in}}%
\pgfpathlineto{\pgfqpoint{1.519592in}{1.624691in}}%
\pgfpathlineto{\pgfqpoint{1.516882in}{1.626895in}}%
\pgfpathlineto{\pgfqpoint{1.501226in}{1.637350in}}%
\pgfpathlineto{\pgfqpoint{1.499230in}{1.638302in}}%
\pgfpathlineto{\pgfqpoint{1.485569in}{1.646190in}}%
\pgfpathlineto{\pgfqpoint{1.469913in}{1.651138in}}%
\pgfpathlineto{\pgfqpoint{1.455184in}{1.651913in}}%
\pgfpathlineto{\pgfqpoint{1.454256in}{1.651989in}}%
\pgfpathlineto{\pgfqpoint{1.454022in}{1.651913in}}%
\pgfpathlineto{\pgfqpoint{1.438599in}{1.648665in}}%
\pgfpathlineto{\pgfqpoint{1.422943in}{1.641234in}}%
\pgfpathlineto{\pgfqpoint{1.418823in}{1.638302in}}%
\pgfpathlineto{\pgfqpoint{1.407286in}{1.631457in}}%
\pgfpathlineto{\pgfqpoint{1.398438in}{1.624691in}}%
\pgfpathlineto{\pgfqpoint{1.391630in}{1.619689in}}%
\pgfpathlineto{\pgfqpoint{1.381414in}{1.611079in}}%
\pgfpathlineto{\pgfqpoint{1.375973in}{1.606146in}}%
\pgfpathlineto{\pgfqpoint{1.366681in}{1.597468in}}%
\pgfpathlineto{\pgfqpoint{1.360317in}{1.590343in}}%
\pgfpathlineto{\pgfqpoint{1.353969in}{1.583857in}}%
\pgfpathlineto{\pgfqpoint{1.344660in}{1.570889in}}%
\pgfpathlineto{\pgfqpoint{1.344063in}{1.570246in}}%
\pgfpathlineto{\pgfqpoint{1.336401in}{1.556635in}}%
\pgfpathlineto{\pgfqpoint{1.334215in}{1.543024in}}%
\pgfpathlineto{\pgfqpoint{1.337494in}{1.529413in}}%
\pgfpathlineto{\pgfqpoint{1.344660in}{1.518222in}}%
\pgfpathlineto{\pgfqpoint{1.345843in}{1.515802in}}%
\pgfpathlineto{\pgfqpoint{1.356403in}{1.502191in}}%
\pgfpathlineto{\pgfqpoint{1.360317in}{1.498354in}}%
\pgfpathlineto{\pgfqpoint{1.369438in}{1.488579in}}%
\pgfpathlineto{\pgfqpoint{1.375973in}{1.482606in}}%
\pgfpathlineto{\pgfqpoint{1.384593in}{1.474968in}}%
\pgfpathlineto{\pgfqpoint{1.391630in}{1.469054in}}%
\pgfpathlineto{\pgfqpoint{1.402136in}{1.461357in}}%
\pgfpathlineto{\pgfqpoint{1.407286in}{1.457340in}}%
\pgfpathlineto{\pgfqpoint{1.422943in}{1.447846in}}%
\pgfpathlineto{\pgfqpoint{1.423190in}{1.447746in}}%
\pgfpathlineto{\pgfqpoint{1.438599in}{1.439930in}}%
\pgfpathclose%
\pgfpathmoveto{\pgfqpoint{1.420737in}{1.474968in}}%
\pgfpathlineto{\pgfqpoint{1.407286in}{1.483118in}}%
\pgfpathlineto{\pgfqpoint{1.400283in}{1.488579in}}%
\pgfpathlineto{\pgfqpoint{1.391630in}{1.496404in}}%
\pgfpathlineto{\pgfqpoint{1.385978in}{1.502191in}}%
\pgfpathlineto{\pgfqpoint{1.375973in}{1.515772in}}%
\pgfpathlineto{\pgfqpoint{1.375952in}{1.515802in}}%
\pgfpathlineto{\pgfqpoint{1.369548in}{1.529413in}}%
\pgfpathlineto{\pgfqpoint{1.367152in}{1.543024in}}%
\pgfpathlineto{\pgfqpoint{1.368749in}{1.556635in}}%
\pgfpathlineto{\pgfqpoint{1.374350in}{1.570246in}}%
\pgfpathlineto{\pgfqpoint{1.375973in}{1.572627in}}%
\pgfpathlineto{\pgfqpoint{1.383667in}{1.583857in}}%
\pgfpathlineto{\pgfqpoint{1.391630in}{1.592354in}}%
\pgfpathlineto{\pgfqpoint{1.397043in}{1.597468in}}%
\pgfpathlineto{\pgfqpoint{1.407286in}{1.605630in}}%
\pgfpathlineto{\pgfqpoint{1.416076in}{1.611079in}}%
\pgfpathlineto{\pgfqpoint{1.422943in}{1.615123in}}%
\pgfpathlineto{\pgfqpoint{1.438599in}{1.621220in}}%
\pgfpathlineto{\pgfqpoint{1.454256in}{1.623923in}}%
\pgfpathlineto{\pgfqpoint{1.469913in}{1.623248in}}%
\pgfpathlineto{\pgfqpoint{1.485569in}{1.619189in}}%
\pgfpathlineto{\pgfqpoint{1.501226in}{1.611735in}}%
\pgfpathlineto{\pgfqpoint{1.502225in}{1.611079in}}%
\pgfpathlineto{\pgfqpoint{1.516882in}{1.600903in}}%
\pgfpathlineto{\pgfqpoint{1.520934in}{1.597468in}}%
\pgfpathlineto{\pgfqpoint{1.532539in}{1.585755in}}%
\pgfpathlineto{\pgfqpoint{1.534275in}{1.583857in}}%
\pgfpathlineto{\pgfqpoint{1.543558in}{1.570246in}}%
\pgfpathlineto{\pgfqpoint{1.548195in}{1.558621in}}%
\pgfpathlineto{\pgfqpoint{1.549031in}{1.556635in}}%
\pgfpathlineto{\pgfqpoint{1.550702in}{1.543024in}}%
\pgfpathlineto{\pgfqpoint{1.548196in}{1.529413in}}%
\pgfpathlineto{\pgfqpoint{1.548195in}{1.529412in}}%
\pgfpathlineto{\pgfqpoint{1.542011in}{1.515802in}}%
\pgfpathlineto{\pgfqpoint{1.532539in}{1.502945in}}%
\pgfpathlineto{\pgfqpoint{1.531944in}{1.502191in}}%
\pgfpathlineto{\pgfqpoint{1.517817in}{1.488579in}}%
\pgfpathlineto{\pgfqpoint{1.516882in}{1.487804in}}%
\pgfpathlineto{\pgfqpoint{1.501226in}{1.477065in}}%
\pgfpathlineto{\pgfqpoint{1.496921in}{1.474968in}}%
\pgfpathlineto{\pgfqpoint{1.485569in}{1.469560in}}%
\pgfpathlineto{\pgfqpoint{1.469913in}{1.465454in}}%
\pgfpathlineto{\pgfqpoint{1.454256in}{1.464770in}}%
\pgfpathlineto{\pgfqpoint{1.438599in}{1.467506in}}%
\pgfpathlineto{\pgfqpoint{1.422943in}{1.473673in}}%
\pgfpathlineto{\pgfqpoint{1.420737in}{1.474968in}}%
\pgfpathclose%
\pgfpathmoveto{\pgfqpoint{1.736074in}{1.445932in}}%
\pgfpathlineto{\pgfqpoint{1.751731in}{1.438872in}}%
\pgfpathlineto{\pgfqpoint{1.767387in}{1.436230in}}%
\pgfpathlineto{\pgfqpoint{1.783044in}{1.437991in}}%
\pgfpathlineto{\pgfqpoint{1.798700in}{1.444166in}}%
\pgfpathlineto{\pgfqpoint{1.804139in}{1.447746in}}%
\pgfpathlineto{\pgfqpoint{1.814357in}{1.453141in}}%
\pgfpathlineto{\pgfqpoint{1.825750in}{1.461357in}}%
\pgfpathlineto{\pgfqpoint{1.830014in}{1.464231in}}%
\pgfpathlineto{\pgfqpoint{1.843300in}{1.474968in}}%
\pgfpathlineto{\pgfqpoint{1.845670in}{1.476982in}}%
\pgfpathlineto{\pgfqpoint{1.858375in}{1.488579in}}%
\pgfpathlineto{\pgfqpoint{1.861327in}{1.491738in}}%
\pgfpathlineto{\pgfqpoint{1.871517in}{1.502191in}}%
\pgfpathlineto{\pgfqpoint{1.876983in}{1.509626in}}%
\pgfpathlineto{\pgfqpoint{1.882447in}{1.515802in}}%
\pgfpathlineto{\pgfqpoint{1.890042in}{1.529413in}}%
\pgfpathlineto{\pgfqpoint{1.892640in}{1.541837in}}%
\pgfpathlineto{\pgfqpoint{1.893022in}{1.543024in}}%
\pgfpathlineto{\pgfqpoint{1.892640in}{1.544798in}}%
\pgfpathlineto{\pgfqpoint{1.890991in}{1.556635in}}%
\pgfpathlineto{\pgfqpoint{1.884347in}{1.570246in}}%
\pgfpathlineto{\pgfqpoint{1.876983in}{1.579168in}}%
\pgfpathlineto{\pgfqpoint{1.873777in}{1.583857in}}%
\pgfpathlineto{\pgfqpoint{1.861327in}{1.597168in}}%
\pgfpathlineto{\pgfqpoint{1.861058in}{1.597468in}}%
\pgfpathlineto{\pgfqpoint{1.846519in}{1.611079in}}%
\pgfpathlineto{\pgfqpoint{1.845670in}{1.611818in}}%
\pgfpathlineto{\pgfqpoint{1.830014in}{1.624457in}}%
\pgfpathlineto{\pgfqpoint{1.829668in}{1.624691in}}%
\pgfpathlineto{\pgfqpoint{1.814357in}{1.635514in}}%
\pgfpathlineto{\pgfqpoint{1.808963in}{1.638302in}}%
\pgfpathlineto{\pgfqpoint{1.798700in}{1.644704in}}%
\pgfpathlineto{\pgfqpoint{1.783044in}{1.650479in}}%
\pgfpathlineto{\pgfqpoint{1.769428in}{1.651913in}}%
\pgfpathlineto{\pgfqpoint{1.767387in}{1.652245in}}%
\pgfpathlineto{\pgfqpoint{1.766022in}{1.651913in}}%
\pgfpathlineto{\pgfqpoint{1.751731in}{1.649655in}}%
\pgfpathlineto{\pgfqpoint{1.736074in}{1.643051in}}%
\pgfpathlineto{\pgfqpoint{1.728970in}{1.638302in}}%
\pgfpathlineto{\pgfqpoint{1.720418in}{1.633549in}}%
\pgfpathlineto{\pgfqpoint{1.708395in}{1.624691in}}%
\pgfpathlineto{\pgfqpoint{1.704761in}{1.622124in}}%
\pgfpathlineto{\pgfqpoint{1.691421in}{1.611079in}}%
\pgfpathlineto{\pgfqpoint{1.689104in}{1.609019in}}%
\pgfpathlineto{\pgfqpoint{1.676753in}{1.597468in}}%
\pgfpathlineto{\pgfqpoint{1.673448in}{1.593762in}}%
\pgfpathlineto{\pgfqpoint{1.663997in}{1.583857in}}%
\pgfpathlineto{\pgfqpoint{1.657791in}{1.574974in}}%
\pgfpathlineto{\pgfqpoint{1.653673in}{1.570246in}}%
\pgfpathlineto{\pgfqpoint{1.646571in}{1.556635in}}%
\pgfpathlineto{\pgfqpoint{1.644545in}{1.543024in}}%
\pgfpathlineto{\pgfqpoint{1.647584in}{1.529413in}}%
\pgfpathlineto{\pgfqpoint{1.655705in}{1.515802in}}%
\pgfpathlineto{\pgfqpoint{1.657791in}{1.513567in}}%
\pgfpathlineto{\pgfqpoint{1.666336in}{1.502191in}}%
\pgfpathlineto{\pgfqpoint{1.673448in}{1.495039in}}%
\pgfpathlineto{\pgfqpoint{1.679467in}{1.488579in}}%
\pgfpathlineto{\pgfqpoint{1.689104in}{1.479759in}}%
\pgfpathlineto{\pgfqpoint{1.694615in}{1.474968in}}%
\pgfpathlineto{\pgfqpoint{1.704761in}{1.466590in}}%
\pgfpathlineto{\pgfqpoint{1.712192in}{1.461357in}}%
\pgfpathlineto{\pgfqpoint{1.720418in}{1.455174in}}%
\pgfpathlineto{\pgfqpoint{1.733504in}{1.447746in}}%
\pgfpathlineto{\pgfqpoint{1.736074in}{1.445932in}}%
\pgfpathclose%
\pgfpathmoveto{\pgfqpoint{1.730991in}{1.474968in}}%
\pgfpathlineto{\pgfqpoint{1.720418in}{1.480969in}}%
\pgfpathlineto{\pgfqpoint{1.710289in}{1.488579in}}%
\pgfpathlineto{\pgfqpoint{1.704761in}{1.493385in}}%
\pgfpathlineto{\pgfqpoint{1.696007in}{1.502191in}}%
\pgfpathlineto{\pgfqpoint{1.689104in}{1.511383in}}%
\pgfpathlineto{\pgfqpoint{1.685879in}{1.515802in}}%
\pgfpathlineto{\pgfqpoint{1.679576in}{1.529413in}}%
\pgfpathlineto{\pgfqpoint{1.677217in}{1.543024in}}%
\pgfpathlineto{\pgfqpoint{1.678789in}{1.556635in}}%
\pgfpathlineto{\pgfqpoint{1.684302in}{1.570246in}}%
\pgfpathlineto{\pgfqpoint{1.689104in}{1.577298in}}%
\pgfpathlineto{\pgfqpoint{1.693685in}{1.583857in}}%
\pgfpathlineto{\pgfqpoint{1.704761in}{1.595469in}}%
\pgfpathlineto{\pgfqpoint{1.706963in}{1.597468in}}%
\pgfpathlineto{\pgfqpoint{1.720418in}{1.607798in}}%
\pgfpathlineto{\pgfqpoint{1.726068in}{1.611079in}}%
\pgfpathlineto{\pgfqpoint{1.736074in}{1.616614in}}%
\pgfpathlineto{\pgfqpoint{1.751731in}{1.622031in}}%
\pgfpathlineto{\pgfqpoint{1.767387in}{1.624059in}}%
\pgfpathlineto{\pgfqpoint{1.783044in}{1.622707in}}%
\pgfpathlineto{\pgfqpoint{1.798700in}{1.617969in}}%
\pgfpathlineto{\pgfqpoint{1.812052in}{1.611079in}}%
\pgfpathlineto{\pgfqpoint{1.814357in}{1.609834in}}%
\pgfpathlineto{\pgfqpoint{1.830014in}{1.598350in}}%
\pgfpathlineto{\pgfqpoint{1.831028in}{1.597468in}}%
\pgfpathlineto{\pgfqpoint{1.844238in}{1.583857in}}%
\pgfpathlineto{\pgfqpoint{1.845670in}{1.581853in}}%
\pgfpathlineto{\pgfqpoint{1.853595in}{1.570246in}}%
\pgfpathlineto{\pgfqpoint{1.859045in}{1.556635in}}%
\pgfpathlineto{\pgfqpoint{1.860600in}{1.543024in}}%
\pgfpathlineto{\pgfqpoint{1.858268in}{1.529413in}}%
\pgfpathlineto{\pgfqpoint{1.852037in}{1.515802in}}%
\pgfpathlineto{\pgfqpoint{1.845670in}{1.507103in}}%
\pgfpathlineto{\pgfqpoint{1.841896in}{1.502191in}}%
\pgfpathlineto{\pgfqpoint{1.830014in}{1.490493in}}%
\pgfpathlineto{\pgfqpoint{1.827713in}{1.488579in}}%
\pgfpathlineto{\pgfqpoint{1.814357in}{1.478950in}}%
\pgfpathlineto{\pgfqpoint{1.806812in}{1.474968in}}%
\pgfpathlineto{\pgfqpoint{1.798700in}{1.470793in}}%
\pgfpathlineto{\pgfqpoint{1.783044in}{1.466000in}}%
\pgfpathlineto{\pgfqpoint{1.767387in}{1.464634in}}%
\pgfpathlineto{\pgfqpoint{1.751731in}{1.466684in}}%
\pgfpathlineto{\pgfqpoint{1.736074in}{1.472164in}}%
\pgfpathlineto{\pgfqpoint{1.730991in}{1.474968in}}%
\pgfpathclose%
\pgfusepath{fill}%
\end{pgfscope}%
\begin{pgfscope}%
\pgfpathrectangle{\pgfqpoint{0.373953in}{0.331635in}}{\pgfqpoint{1.550000in}{1.347500in}}%
\pgfusepath{clip}%
\pgfsetbuttcap%
\pgfsetroundjoin%
\definecolor{currentfill}{rgb}{0.481929,0.136891,0.507989}%
\pgfsetfillcolor{currentfill}%
\pgfsetlinewidth{0.000000pt}%
\definecolor{currentstroke}{rgb}{0.000000,0.000000,0.000000}%
\pgfsetstrokecolor{currentstroke}%
\pgfsetdash{}{0pt}%
\pgfpathmoveto{\pgfqpoint{0.483549in}{0.331635in}}%
\pgfpathlineto{\pgfqpoint{0.499205in}{0.331635in}}%
\pgfpathlineto{\pgfqpoint{0.514862in}{0.331635in}}%
\pgfpathlineto{\pgfqpoint{0.530519in}{0.331635in}}%
\pgfpathlineto{\pgfqpoint{0.546175in}{0.331635in}}%
\pgfpathlineto{\pgfqpoint{0.561832in}{0.331635in}}%
\pgfpathlineto{\pgfqpoint{0.577488in}{0.331635in}}%
\pgfpathlineto{\pgfqpoint{0.581789in}{0.331635in}}%
\pgfpathlineto{\pgfqpoint{0.585574in}{0.345246in}}%
\pgfpathlineto{\pgfqpoint{0.593145in}{0.355039in}}%
\pgfpathlineto{\pgfqpoint{0.595745in}{0.358857in}}%
\pgfpathlineto{\pgfqpoint{0.608801in}{0.372046in}}%
\pgfpathlineto{\pgfqpoint{0.609202in}{0.372468in}}%
\pgfpathlineto{\pgfqpoint{0.624341in}{0.386079in}}%
\pgfpathlineto{\pgfqpoint{0.624458in}{0.386182in}}%
\pgfpathlineto{\pgfqpoint{0.640115in}{0.399101in}}%
\pgfpathlineto{\pgfqpoint{0.640979in}{0.399691in}}%
\pgfpathlineto{\pgfqpoint{0.655771in}{0.410388in}}%
\pgfpathlineto{\pgfqpoint{0.662044in}{0.413302in}}%
\pgfpathlineto{\pgfqpoint{0.671428in}{0.418338in}}%
\pgfpathlineto{\pgfqpoint{0.687084in}{0.420390in}}%
\pgfpathlineto{\pgfqpoint{0.702741in}{0.415687in}}%
\pgfpathlineto{\pgfqpoint{0.706247in}{0.413302in}}%
\pgfpathlineto{\pgfqpoint{0.718397in}{0.406214in}}%
\pgfpathlineto{\pgfqpoint{0.726515in}{0.399691in}}%
\pgfpathlineto{\pgfqpoint{0.734054in}{0.394016in}}%
\pgfpathlineto{\pgfqpoint{0.743325in}{0.386079in}}%
\pgfpathlineto{\pgfqpoint{0.749710in}{0.380446in}}%
\pgfpathlineto{\pgfqpoint{0.758666in}{0.372468in}}%
\pgfpathlineto{\pgfqpoint{0.765367in}{0.365473in}}%
\pgfpathlineto{\pgfqpoint{0.772252in}{0.358857in}}%
\pgfpathlineto{\pgfqpoint{0.781024in}{0.346564in}}%
\pgfpathlineto{\pgfqpoint{0.782141in}{0.345246in}}%
\pgfpathlineto{\pgfqpoint{0.786191in}{0.331635in}}%
\pgfpathlineto{\pgfqpoint{0.796680in}{0.331635in}}%
\pgfpathlineto{\pgfqpoint{0.812337in}{0.331635in}}%
\pgfpathlineto{\pgfqpoint{0.827993in}{0.331635in}}%
\pgfpathlineto{\pgfqpoint{0.843650in}{0.331635in}}%
\pgfpathlineto{\pgfqpoint{0.859306in}{0.331635in}}%
\pgfpathlineto{\pgfqpoint{0.874963in}{0.331635in}}%
\pgfpathlineto{\pgfqpoint{0.890620in}{0.331635in}}%
\pgfpathlineto{\pgfqpoint{0.891946in}{0.331635in}}%
\pgfpathlineto{\pgfqpoint{0.895633in}{0.345246in}}%
\pgfpathlineto{\pgfqpoint{0.905770in}{0.358857in}}%
\pgfpathlineto{\pgfqpoint{0.906276in}{0.359313in}}%
\pgfpathlineto{\pgfqpoint{0.919190in}{0.372468in}}%
\pgfpathlineto{\pgfqpoint{0.921933in}{0.374848in}}%
\pgfpathlineto{\pgfqpoint{0.934464in}{0.386079in}}%
\pgfpathlineto{\pgfqpoint{0.937589in}{0.388801in}}%
\pgfpathlineto{\pgfqpoint{0.951085in}{0.399691in}}%
\pgfpathlineto{\pgfqpoint{0.953246in}{0.401560in}}%
\pgfpathlineto{\pgfqpoint{0.968902in}{0.412250in}}%
\pgfpathlineto{\pgfqpoint{0.971533in}{0.413302in}}%
\pgfpathlineto{\pgfqpoint{0.984559in}{0.419289in}}%
\pgfpathlineto{\pgfqpoint{1.000216in}{0.419975in}}%
\pgfpathlineto{\pgfqpoint{1.015872in}{0.414013in}}%
\pgfpathlineto{\pgfqpoint{1.016826in}{0.413302in}}%
\pgfpathlineto{\pgfqpoint{1.031529in}{0.403938in}}%
\pgfpathlineto{\pgfqpoint{1.036604in}{0.399691in}}%
\pgfpathlineto{\pgfqpoint{1.047185in}{0.391417in}}%
\pgfpathlineto{\pgfqpoint{1.053353in}{0.386079in}}%
\pgfpathlineto{\pgfqpoint{1.062842in}{0.377623in}}%
\pgfpathlineto{\pgfqpoint{1.068693in}{0.372468in}}%
\pgfpathlineto{\pgfqpoint{1.078498in}{0.362338in}}%
\pgfpathlineto{\pgfqpoint{1.082234in}{0.358857in}}%
\pgfpathlineto{\pgfqpoint{1.092154in}{0.345246in}}%
\pgfpathlineto{\pgfqpoint{1.094155in}{0.337649in}}%
\pgfpathlineto{\pgfqpoint{1.096036in}{0.331635in}}%
\pgfpathlineto{\pgfqpoint{1.109812in}{0.331635in}}%
\pgfpathlineto{\pgfqpoint{1.125468in}{0.331635in}}%
\pgfpathlineto{\pgfqpoint{1.141125in}{0.331635in}}%
\pgfpathlineto{\pgfqpoint{1.156781in}{0.331635in}}%
\pgfpathlineto{\pgfqpoint{1.172438in}{0.331635in}}%
\pgfpathlineto{\pgfqpoint{1.188094in}{0.331635in}}%
\pgfpathlineto{\pgfqpoint{1.201870in}{0.331635in}}%
\pgfpathlineto{\pgfqpoint{1.203751in}{0.337649in}}%
\pgfpathlineto{\pgfqpoint{1.205752in}{0.345246in}}%
\pgfpathlineto{\pgfqpoint{1.215672in}{0.358857in}}%
\pgfpathlineto{\pgfqpoint{1.219407in}{0.362338in}}%
\pgfpathlineto{\pgfqpoint{1.229213in}{0.372468in}}%
\pgfpathlineto{\pgfqpoint{1.235064in}{0.377623in}}%
\pgfpathlineto{\pgfqpoint{1.244553in}{0.386079in}}%
\pgfpathlineto{\pgfqpoint{1.250721in}{0.391417in}}%
\pgfpathlineto{\pgfqpoint{1.261302in}{0.399691in}}%
\pgfpathlineto{\pgfqpoint{1.266377in}{0.403938in}}%
\pgfpathlineto{\pgfqpoint{1.281080in}{0.413302in}}%
\pgfpathlineto{\pgfqpoint{1.282034in}{0.414013in}}%
\pgfpathlineto{\pgfqpoint{1.297690in}{0.419975in}}%
\pgfpathlineto{\pgfqpoint{1.313347in}{0.419289in}}%
\pgfpathlineto{\pgfqpoint{1.326373in}{0.413302in}}%
\pgfpathlineto{\pgfqpoint{1.329003in}{0.412250in}}%
\pgfpathlineto{\pgfqpoint{1.344660in}{0.401560in}}%
\pgfpathlineto{\pgfqpoint{1.346820in}{0.399691in}}%
\pgfpathlineto{\pgfqpoint{1.360317in}{0.388801in}}%
\pgfpathlineto{\pgfqpoint{1.363442in}{0.386079in}}%
\pgfpathlineto{\pgfqpoint{1.375973in}{0.374848in}}%
\pgfpathlineto{\pgfqpoint{1.378716in}{0.372468in}}%
\pgfpathlineto{\pgfqpoint{1.391630in}{0.359313in}}%
\pgfpathlineto{\pgfqpoint{1.392136in}{0.358857in}}%
\pgfpathlineto{\pgfqpoint{1.402272in}{0.345246in}}%
\pgfpathlineto{\pgfqpoint{1.405960in}{0.331635in}}%
\pgfpathlineto{\pgfqpoint{1.407286in}{0.331635in}}%
\pgfpathlineto{\pgfqpoint{1.422943in}{0.331635in}}%
\pgfpathlineto{\pgfqpoint{1.438599in}{0.331635in}}%
\pgfpathlineto{\pgfqpoint{1.454256in}{0.331635in}}%
\pgfpathlineto{\pgfqpoint{1.469913in}{0.331635in}}%
\pgfpathlineto{\pgfqpoint{1.485569in}{0.331635in}}%
\pgfpathlineto{\pgfqpoint{1.501226in}{0.331635in}}%
\pgfpathlineto{\pgfqpoint{1.511715in}{0.331635in}}%
\pgfpathlineto{\pgfqpoint{1.515765in}{0.345246in}}%
\pgfpathlineto{\pgfqpoint{1.516882in}{0.346564in}}%
\pgfpathlineto{\pgfqpoint{1.525654in}{0.358857in}}%
\pgfpathlineto{\pgfqpoint{1.532539in}{0.365473in}}%
\pgfpathlineto{\pgfqpoint{1.539239in}{0.372468in}}%
\pgfpathlineto{\pgfqpoint{1.548195in}{0.380446in}}%
\pgfpathlineto{\pgfqpoint{1.554580in}{0.386079in}}%
\pgfpathlineto{\pgfqpoint{1.563852in}{0.394016in}}%
\pgfpathlineto{\pgfqpoint{1.571391in}{0.399691in}}%
\pgfpathlineto{\pgfqpoint{1.579508in}{0.406214in}}%
\pgfpathlineto{\pgfqpoint{1.591659in}{0.413302in}}%
\pgfpathlineto{\pgfqpoint{1.595165in}{0.415687in}}%
\pgfpathlineto{\pgfqpoint{1.610822in}{0.420390in}}%
\pgfpathlineto{\pgfqpoint{1.626478in}{0.418338in}}%
\pgfpathlineto{\pgfqpoint{1.635862in}{0.413302in}}%
\pgfpathlineto{\pgfqpoint{1.642135in}{0.410388in}}%
\pgfpathlineto{\pgfqpoint{1.656927in}{0.399691in}}%
\pgfpathlineto{\pgfqpoint{1.657791in}{0.399101in}}%
\pgfpathlineto{\pgfqpoint{1.673448in}{0.386182in}}%
\pgfpathlineto{\pgfqpoint{1.673565in}{0.386079in}}%
\pgfpathlineto{\pgfqpoint{1.688703in}{0.372468in}}%
\pgfpathlineto{\pgfqpoint{1.689104in}{0.372046in}}%
\pgfpathlineto{\pgfqpoint{1.702161in}{0.358857in}}%
\pgfpathlineto{\pgfqpoint{1.704761in}{0.355039in}}%
\pgfpathlineto{\pgfqpoint{1.712332in}{0.345246in}}%
\pgfpathlineto{\pgfqpoint{1.716117in}{0.331635in}}%
\pgfpathlineto{\pgfqpoint{1.720418in}{0.331635in}}%
\pgfpathlineto{\pgfqpoint{1.736074in}{0.331635in}}%
\pgfpathlineto{\pgfqpoint{1.751731in}{0.331635in}}%
\pgfpathlineto{\pgfqpoint{1.767387in}{0.331635in}}%
\pgfpathlineto{\pgfqpoint{1.783044in}{0.331635in}}%
\pgfpathlineto{\pgfqpoint{1.798700in}{0.331635in}}%
\pgfpathlineto{\pgfqpoint{1.814357in}{0.331635in}}%
\pgfpathlineto{\pgfqpoint{1.821701in}{0.331635in}}%
\pgfpathlineto{\pgfqpoint{1.825606in}{0.345246in}}%
\pgfpathlineto{\pgfqpoint{1.830014in}{0.350703in}}%
\pgfpathlineto{\pgfqpoint{1.835687in}{0.358857in}}%
\pgfpathlineto{\pgfqpoint{1.845670in}{0.368712in}}%
\pgfpathlineto{\pgfqpoint{1.849245in}{0.372468in}}%
\pgfpathlineto{\pgfqpoint{1.861327in}{0.383306in}}%
\pgfpathlineto{\pgfqpoint{1.864516in}{0.386079in}}%
\pgfpathlineto{\pgfqpoint{1.876983in}{0.396583in}}%
\pgfpathlineto{\pgfqpoint{1.881304in}{0.399691in}}%
\pgfpathlineto{\pgfqpoint{1.892640in}{0.408370in}}%
\pgfpathlineto{\pgfqpoint{1.902020in}{0.413302in}}%
\pgfpathlineto{\pgfqpoint{1.908296in}{0.417133in}}%
\pgfpathlineto{\pgfqpoint{1.923953in}{0.420528in}}%
\pgfpathlineto{\pgfqpoint{1.923953in}{0.426913in}}%
\pgfpathlineto{\pgfqpoint{1.923953in}{0.440524in}}%
\pgfpathlineto{\pgfqpoint{1.923953in}{0.454135in}}%
\pgfpathlineto{\pgfqpoint{1.923953in}{0.467746in}}%
\pgfpathlineto{\pgfqpoint{1.923953in}{0.481357in}}%
\pgfpathlineto{\pgfqpoint{1.923953in}{0.494968in}}%
\pgfpathlineto{\pgfqpoint{1.923953in}{0.508579in}}%
\pgfpathlineto{\pgfqpoint{1.923953in}{0.512318in}}%
\pgfpathlineto{\pgfqpoint{1.908296in}{0.515609in}}%
\pgfpathlineto{\pgfqpoint{1.897031in}{0.522191in}}%
\pgfpathlineto{\pgfqpoint{1.892640in}{0.524451in}}%
\pgfpathlineto{\pgfqpoint{1.877469in}{0.535802in}}%
\pgfpathlineto{\pgfqpoint{1.876983in}{0.536150in}}%
\pgfpathlineto{\pgfqpoint{1.861327in}{0.549311in}}%
\pgfpathlineto{\pgfqpoint{1.861209in}{0.549413in}}%
\pgfpathlineto{\pgfqpoint{1.846348in}{0.563024in}}%
\pgfpathlineto{\pgfqpoint{1.845670in}{0.563776in}}%
\pgfpathlineto{\pgfqpoint{1.833366in}{0.576635in}}%
\pgfpathlineto{\pgfqpoint{1.830014in}{0.582089in}}%
\pgfpathlineto{\pgfqpoint{1.824220in}{0.590246in}}%
\pgfpathlineto{\pgfqpoint{1.821860in}{0.603857in}}%
\pgfpathlineto{\pgfqpoint{1.827270in}{0.617468in}}%
\pgfpathlineto{\pgfqpoint{1.830014in}{0.620516in}}%
\pgfpathlineto{\pgfqpoint{1.838167in}{0.631079in}}%
\pgfpathlineto{\pgfqpoint{1.845670in}{0.638136in}}%
\pgfpathlineto{\pgfqpoint{1.852197in}{0.644691in}}%
\pgfpathlineto{\pgfqpoint{1.861327in}{0.652751in}}%
\pgfpathlineto{\pgfqpoint{1.867807in}{0.658302in}}%
\pgfpathlineto{\pgfqpoint{1.876983in}{0.666088in}}%
\pgfpathlineto{\pgfqpoint{1.885030in}{0.671913in}}%
\pgfpathlineto{\pgfqpoint{1.892640in}{0.677898in}}%
\pgfpathlineto{\pgfqpoint{1.906781in}{0.685524in}}%
\pgfpathlineto{\pgfqpoint{1.908296in}{0.686495in}}%
\pgfpathlineto{\pgfqpoint{1.923953in}{0.690016in}}%
\pgfpathlineto{\pgfqpoint{1.923953in}{0.699135in}}%
\pgfpathlineto{\pgfqpoint{1.923953in}{0.712746in}}%
\pgfpathlineto{\pgfqpoint{1.923953in}{0.726357in}}%
\pgfpathlineto{\pgfqpoint{1.923953in}{0.739968in}}%
\pgfpathlineto{\pgfqpoint{1.923953in}{0.753579in}}%
\pgfpathlineto{\pgfqpoint{1.923953in}{0.767191in}}%
\pgfpathlineto{\pgfqpoint{1.923953in}{0.780802in}}%
\pgfpathlineto{\pgfqpoint{1.923953in}{0.781955in}}%
\pgfpathlineto{\pgfqpoint{1.908296in}{0.785160in}}%
\pgfpathlineto{\pgfqpoint{1.892640in}{0.793973in}}%
\pgfpathlineto{\pgfqpoint{1.892116in}{0.794413in}}%
\pgfpathlineto{\pgfqpoint{1.876983in}{0.805639in}}%
\pgfpathlineto{\pgfqpoint{1.874246in}{0.808024in}}%
\pgfpathlineto{\pgfqpoint{1.861327in}{0.818918in}}%
\pgfpathlineto{\pgfqpoint{1.858197in}{0.821635in}}%
\pgfpathlineto{\pgfqpoint{1.845670in}{0.833368in}}%
\pgfpathlineto{\pgfqpoint{1.843519in}{0.835246in}}%
\pgfpathlineto{\pgfqpoint{1.831223in}{0.848857in}}%
\pgfpathlineto{\pgfqpoint{1.830014in}{0.851144in}}%
\pgfpathlineto{\pgfqpoint{1.823127in}{0.862468in}}%
\pgfpathlineto{\pgfqpoint{1.822337in}{0.876079in}}%
\pgfpathlineto{\pgfqpoint{1.829195in}{0.889691in}}%
\pgfpathlineto{\pgfqpoint{1.830014in}{0.890520in}}%
\pgfpathlineto{\pgfqpoint{1.840785in}{0.903302in}}%
\pgfpathlineto{\pgfqpoint{1.845670in}{0.907714in}}%
\pgfpathlineto{\pgfqpoint{1.855187in}{0.916913in}}%
\pgfpathlineto{\pgfqpoint{1.861327in}{0.922275in}}%
\pgfpathlineto{\pgfqpoint{1.871054in}{0.930524in}}%
\pgfpathlineto{\pgfqpoint{1.876983in}{0.935611in}}%
\pgfpathlineto{\pgfqpoint{1.888636in}{0.944135in}}%
\pgfpathlineto{\pgfqpoint{1.892640in}{0.947383in}}%
\pgfpathlineto{\pgfqpoint{1.908296in}{0.956006in}}%
\pgfpathlineto{\pgfqpoint{1.917035in}{0.957746in}}%
\pgfpathlineto{\pgfqpoint{1.923953in}{0.959381in}}%
\pgfpathlineto{\pgfqpoint{1.923953in}{0.971357in}}%
\pgfpathlineto{\pgfqpoint{1.923953in}{0.984968in}}%
\pgfpathlineto{\pgfqpoint{1.923953in}{0.998579in}}%
\pgfpathlineto{\pgfqpoint{1.923953in}{1.012191in}}%
\pgfpathlineto{\pgfqpoint{1.923953in}{1.025802in}}%
\pgfpathlineto{\pgfqpoint{1.923953in}{1.039413in}}%
\pgfpathlineto{\pgfqpoint{1.923953in}{1.051389in}}%
\pgfpathlineto{\pgfqpoint{1.917035in}{1.053024in}}%
\pgfpathlineto{\pgfqpoint{1.908296in}{1.054764in}}%
\pgfpathlineto{\pgfqpoint{1.892640in}{1.063387in}}%
\pgfpathlineto{\pgfqpoint{1.888636in}{1.066635in}}%
\pgfpathlineto{\pgfqpoint{1.876983in}{1.075159in}}%
\pgfpathlineto{\pgfqpoint{1.871054in}{1.080246in}}%
\pgfpathlineto{\pgfqpoint{1.861327in}{1.088495in}}%
\pgfpathlineto{\pgfqpoint{1.855187in}{1.093857in}}%
\pgfpathlineto{\pgfqpoint{1.845670in}{1.103056in}}%
\pgfpathlineto{\pgfqpoint{1.840785in}{1.107468in}}%
\pgfpathlineto{\pgfqpoint{1.830014in}{1.120250in}}%
\pgfpathlineto{\pgfqpoint{1.829195in}{1.121079in}}%
\pgfpathlineto{\pgfqpoint{1.822337in}{1.134691in}}%
\pgfpathlineto{\pgfqpoint{1.823127in}{1.148302in}}%
\pgfpathlineto{\pgfqpoint{1.830014in}{1.159626in}}%
\pgfpathlineto{\pgfqpoint{1.831223in}{1.161913in}}%
\pgfpathlineto{\pgfqpoint{1.843519in}{1.175524in}}%
\pgfpathlineto{\pgfqpoint{1.845670in}{1.177402in}}%
\pgfpathlineto{\pgfqpoint{1.858197in}{1.189135in}}%
\pgfpathlineto{\pgfqpoint{1.861327in}{1.191852in}}%
\pgfpathlineto{\pgfqpoint{1.874246in}{1.202746in}}%
\pgfpathlineto{\pgfqpoint{1.876983in}{1.205131in}}%
\pgfpathlineto{\pgfqpoint{1.892116in}{1.216357in}}%
\pgfpathlineto{\pgfqpoint{1.892640in}{1.216797in}}%
\pgfpathlineto{\pgfqpoint{1.908296in}{1.225610in}}%
\pgfpathlineto{\pgfqpoint{1.923953in}{1.228815in}}%
\pgfpathlineto{\pgfqpoint{1.923953in}{1.229968in}}%
\pgfpathlineto{\pgfqpoint{1.923953in}{1.243579in}}%
\pgfpathlineto{\pgfqpoint{1.923953in}{1.257191in}}%
\pgfpathlineto{\pgfqpoint{1.923953in}{1.270802in}}%
\pgfpathlineto{\pgfqpoint{1.923953in}{1.284413in}}%
\pgfpathlineto{\pgfqpoint{1.923953in}{1.298024in}}%
\pgfpathlineto{\pgfqpoint{1.923953in}{1.311635in}}%
\pgfpathlineto{\pgfqpoint{1.923953in}{1.320754in}}%
\pgfpathlineto{\pgfqpoint{1.908296in}{1.324275in}}%
\pgfpathlineto{\pgfqpoint{1.906781in}{1.325246in}}%
\pgfpathlineto{\pgfqpoint{1.892640in}{1.332872in}}%
\pgfpathlineto{\pgfqpoint{1.885030in}{1.338857in}}%
\pgfpathlineto{\pgfqpoint{1.876983in}{1.344682in}}%
\pgfpathlineto{\pgfqpoint{1.867807in}{1.352468in}}%
\pgfpathlineto{\pgfqpoint{1.861327in}{1.358019in}}%
\pgfpathlineto{\pgfqpoint{1.852197in}{1.366079in}}%
\pgfpathlineto{\pgfqpoint{1.845670in}{1.372634in}}%
\pgfpathlineto{\pgfqpoint{1.838167in}{1.379691in}}%
\pgfpathlineto{\pgfqpoint{1.830014in}{1.390254in}}%
\pgfpathlineto{\pgfqpoint{1.827270in}{1.393302in}}%
\pgfpathlineto{\pgfqpoint{1.821860in}{1.406913in}}%
\pgfpathlineto{\pgfqpoint{1.824220in}{1.420524in}}%
\pgfpathlineto{\pgfqpoint{1.830014in}{1.428681in}}%
\pgfpathlineto{\pgfqpoint{1.833366in}{1.434135in}}%
\pgfpathlineto{\pgfqpoint{1.845670in}{1.446994in}}%
\pgfpathlineto{\pgfqpoint{1.846348in}{1.447746in}}%
\pgfpathlineto{\pgfqpoint{1.861209in}{1.461357in}}%
\pgfpathlineto{\pgfqpoint{1.861327in}{1.461459in}}%
\pgfpathlineto{\pgfqpoint{1.876983in}{1.474620in}}%
\pgfpathlineto{\pgfqpoint{1.877469in}{1.474968in}}%
\pgfpathlineto{\pgfqpoint{1.892640in}{1.486319in}}%
\pgfpathlineto{\pgfqpoint{1.897031in}{1.488579in}}%
\pgfpathlineto{\pgfqpoint{1.908296in}{1.495161in}}%
\pgfpathlineto{\pgfqpoint{1.923953in}{1.498452in}}%
\pgfpathlineto{\pgfqpoint{1.923953in}{1.502191in}}%
\pgfpathlineto{\pgfqpoint{1.923953in}{1.515802in}}%
\pgfpathlineto{\pgfqpoint{1.923953in}{1.529413in}}%
\pgfpathlineto{\pgfqpoint{1.923953in}{1.543024in}}%
\pgfpathlineto{\pgfqpoint{1.923953in}{1.556635in}}%
\pgfpathlineto{\pgfqpoint{1.923953in}{1.570246in}}%
\pgfpathlineto{\pgfqpoint{1.923953in}{1.583857in}}%
\pgfpathlineto{\pgfqpoint{1.923953in}{1.590242in}}%
\pgfpathlineto{\pgfqpoint{1.908296in}{1.593637in}}%
\pgfpathlineto{\pgfqpoint{1.902020in}{1.597468in}}%
\pgfpathlineto{\pgfqpoint{1.892640in}{1.602400in}}%
\pgfpathlineto{\pgfqpoint{1.881304in}{1.611079in}}%
\pgfpathlineto{\pgfqpoint{1.876983in}{1.614187in}}%
\pgfpathlineto{\pgfqpoint{1.864516in}{1.624691in}}%
\pgfpathlineto{\pgfqpoint{1.861327in}{1.627464in}}%
\pgfpathlineto{\pgfqpoint{1.849245in}{1.638302in}}%
\pgfpathlineto{\pgfqpoint{1.845670in}{1.642058in}}%
\pgfpathlineto{\pgfqpoint{1.835687in}{1.651913in}}%
\pgfpathlineto{\pgfqpoint{1.830014in}{1.660067in}}%
\pgfpathlineto{\pgfqpoint{1.825606in}{1.665524in}}%
\pgfpathlineto{\pgfqpoint{1.821701in}{1.679135in}}%
\pgfpathlineto{\pgfqpoint{1.814357in}{1.679135in}}%
\pgfpathlineto{\pgfqpoint{1.798700in}{1.679135in}}%
\pgfpathlineto{\pgfqpoint{1.783044in}{1.679135in}}%
\pgfpathlineto{\pgfqpoint{1.767387in}{1.679135in}}%
\pgfpathlineto{\pgfqpoint{1.751731in}{1.679135in}}%
\pgfpathlineto{\pgfqpoint{1.736074in}{1.679135in}}%
\pgfpathlineto{\pgfqpoint{1.720418in}{1.679135in}}%
\pgfpathlineto{\pgfqpoint{1.716117in}{1.679135in}}%
\pgfpathlineto{\pgfqpoint{1.712332in}{1.665524in}}%
\pgfpathlineto{\pgfqpoint{1.704761in}{1.655731in}}%
\pgfpathlineto{\pgfqpoint{1.702161in}{1.651913in}}%
\pgfpathlineto{\pgfqpoint{1.689104in}{1.638724in}}%
\pgfpathlineto{\pgfqpoint{1.688703in}{1.638302in}}%
\pgfpathlineto{\pgfqpoint{1.673565in}{1.624691in}}%
\pgfpathlineto{\pgfqpoint{1.673448in}{1.624588in}}%
\pgfpathlineto{\pgfqpoint{1.657791in}{1.611669in}}%
\pgfpathlineto{\pgfqpoint{1.656927in}{1.611079in}}%
\pgfpathlineto{\pgfqpoint{1.642135in}{1.600382in}}%
\pgfpathlineto{\pgfqpoint{1.635862in}{1.597468in}}%
\pgfpathlineto{\pgfqpoint{1.626478in}{1.592432in}}%
\pgfpathlineto{\pgfqpoint{1.610822in}{1.590380in}}%
\pgfpathlineto{\pgfqpoint{1.595165in}{1.595083in}}%
\pgfpathlineto{\pgfqpoint{1.591659in}{1.597468in}}%
\pgfpathlineto{\pgfqpoint{1.579508in}{1.604556in}}%
\pgfpathlineto{\pgfqpoint{1.571391in}{1.611079in}}%
\pgfpathlineto{\pgfqpoint{1.563852in}{1.616754in}}%
\pgfpathlineto{\pgfqpoint{1.554580in}{1.624691in}}%
\pgfpathlineto{\pgfqpoint{1.548195in}{1.630324in}}%
\pgfpathlineto{\pgfqpoint{1.539239in}{1.638302in}}%
\pgfpathlineto{\pgfqpoint{1.532539in}{1.645297in}}%
\pgfpathlineto{\pgfqpoint{1.525654in}{1.651913in}}%
\pgfpathlineto{\pgfqpoint{1.516882in}{1.664206in}}%
\pgfpathlineto{\pgfqpoint{1.515765in}{1.665524in}}%
\pgfpathlineto{\pgfqpoint{1.511715in}{1.679135in}}%
\pgfpathlineto{\pgfqpoint{1.501226in}{1.679135in}}%
\pgfpathlineto{\pgfqpoint{1.485569in}{1.679135in}}%
\pgfpathlineto{\pgfqpoint{1.469913in}{1.679135in}}%
\pgfpathlineto{\pgfqpoint{1.454256in}{1.679135in}}%
\pgfpathlineto{\pgfqpoint{1.438599in}{1.679135in}}%
\pgfpathlineto{\pgfqpoint{1.422943in}{1.679135in}}%
\pgfpathlineto{\pgfqpoint{1.407286in}{1.679135in}}%
\pgfpathlineto{\pgfqpoint{1.405960in}{1.679135in}}%
\pgfpathlineto{\pgfqpoint{1.402272in}{1.665524in}}%
\pgfpathlineto{\pgfqpoint{1.392136in}{1.651913in}}%
\pgfpathlineto{\pgfqpoint{1.391630in}{1.651457in}}%
\pgfpathlineto{\pgfqpoint{1.378716in}{1.638302in}}%
\pgfpathlineto{\pgfqpoint{1.375973in}{1.635922in}}%
\pgfpathlineto{\pgfqpoint{1.363442in}{1.624691in}}%
\pgfpathlineto{\pgfqpoint{1.360317in}{1.621969in}}%
\pgfpathlineto{\pgfqpoint{1.346820in}{1.611079in}}%
\pgfpathlineto{\pgfqpoint{1.344660in}{1.609210in}}%
\pgfpathlineto{\pgfqpoint{1.329003in}{1.598520in}}%
\pgfpathlineto{\pgfqpoint{1.326373in}{1.597468in}}%
\pgfpathlineto{\pgfqpoint{1.313347in}{1.591481in}}%
\pgfpathlineto{\pgfqpoint{1.297690in}{1.590795in}}%
\pgfpathlineto{\pgfqpoint{1.282034in}{1.596757in}}%
\pgfpathlineto{\pgfqpoint{1.281080in}{1.597468in}}%
\pgfpathlineto{\pgfqpoint{1.266377in}{1.606832in}}%
\pgfpathlineto{\pgfqpoint{1.261302in}{1.611079in}}%
\pgfpathlineto{\pgfqpoint{1.250721in}{1.619353in}}%
\pgfpathlineto{\pgfqpoint{1.244553in}{1.624691in}}%
\pgfpathlineto{\pgfqpoint{1.235064in}{1.633147in}}%
\pgfpathlineto{\pgfqpoint{1.229213in}{1.638302in}}%
\pgfpathlineto{\pgfqpoint{1.219407in}{1.648432in}}%
\pgfpathlineto{\pgfqpoint{1.215672in}{1.651913in}}%
\pgfpathlineto{\pgfqpoint{1.205752in}{1.665524in}}%
\pgfpathlineto{\pgfqpoint{1.203751in}{1.673121in}}%
\pgfpathlineto{\pgfqpoint{1.201870in}{1.679135in}}%
\pgfpathlineto{\pgfqpoint{1.188094in}{1.679135in}}%
\pgfpathlineto{\pgfqpoint{1.172438in}{1.679135in}}%
\pgfpathlineto{\pgfqpoint{1.156781in}{1.679135in}}%
\pgfpathlineto{\pgfqpoint{1.141125in}{1.679135in}}%
\pgfpathlineto{\pgfqpoint{1.125468in}{1.679135in}}%
\pgfpathlineto{\pgfqpoint{1.109812in}{1.679135in}}%
\pgfpathlineto{\pgfqpoint{1.096036in}{1.679135in}}%
\pgfpathlineto{\pgfqpoint{1.094155in}{1.673121in}}%
\pgfpathlineto{\pgfqpoint{1.092154in}{1.665524in}}%
\pgfpathlineto{\pgfqpoint{1.082234in}{1.651913in}}%
\pgfpathlineto{\pgfqpoint{1.078498in}{1.648432in}}%
\pgfpathlineto{\pgfqpoint{1.068693in}{1.638302in}}%
\pgfpathlineto{\pgfqpoint{1.062842in}{1.633147in}}%
\pgfpathlineto{\pgfqpoint{1.053353in}{1.624691in}}%
\pgfpathlineto{\pgfqpoint{1.047185in}{1.619353in}}%
\pgfpathlineto{\pgfqpoint{1.036604in}{1.611079in}}%
\pgfpathlineto{\pgfqpoint{1.031529in}{1.606832in}}%
\pgfpathlineto{\pgfqpoint{1.016826in}{1.597468in}}%
\pgfpathlineto{\pgfqpoint{1.015872in}{1.596757in}}%
\pgfpathlineto{\pgfqpoint{1.000216in}{1.590795in}}%
\pgfpathlineto{\pgfqpoint{0.984559in}{1.591481in}}%
\pgfpathlineto{\pgfqpoint{0.971533in}{1.597468in}}%
\pgfpathlineto{\pgfqpoint{0.968902in}{1.598520in}}%
\pgfpathlineto{\pgfqpoint{0.953246in}{1.609210in}}%
\pgfpathlineto{\pgfqpoint{0.951085in}{1.611079in}}%
\pgfpathlineto{\pgfqpoint{0.937589in}{1.621969in}}%
\pgfpathlineto{\pgfqpoint{0.934464in}{1.624691in}}%
\pgfpathlineto{\pgfqpoint{0.921933in}{1.635922in}}%
\pgfpathlineto{\pgfqpoint{0.919190in}{1.638302in}}%
\pgfpathlineto{\pgfqpoint{0.906276in}{1.651457in}}%
\pgfpathlineto{\pgfqpoint{0.905770in}{1.651913in}}%
\pgfpathlineto{\pgfqpoint{0.895633in}{1.665524in}}%
\pgfpathlineto{\pgfqpoint{0.891946in}{1.679135in}}%
\pgfpathlineto{\pgfqpoint{0.890620in}{1.679135in}}%
\pgfpathlineto{\pgfqpoint{0.874963in}{1.679135in}}%
\pgfpathlineto{\pgfqpoint{0.859306in}{1.679135in}}%
\pgfpathlineto{\pgfqpoint{0.843650in}{1.679135in}}%
\pgfpathlineto{\pgfqpoint{0.827993in}{1.679135in}}%
\pgfpathlineto{\pgfqpoint{0.812337in}{1.679135in}}%
\pgfpathlineto{\pgfqpoint{0.796680in}{1.679135in}}%
\pgfpathlineto{\pgfqpoint{0.786191in}{1.679135in}}%
\pgfpathlineto{\pgfqpoint{0.782141in}{1.665524in}}%
\pgfpathlineto{\pgfqpoint{0.781024in}{1.664206in}}%
\pgfpathlineto{\pgfqpoint{0.772252in}{1.651913in}}%
\pgfpathlineto{\pgfqpoint{0.765367in}{1.645297in}}%
\pgfpathlineto{\pgfqpoint{0.758666in}{1.638302in}}%
\pgfpathlineto{\pgfqpoint{0.749710in}{1.630324in}}%
\pgfpathlineto{\pgfqpoint{0.743325in}{1.624691in}}%
\pgfpathlineto{\pgfqpoint{0.734054in}{1.616754in}}%
\pgfpathlineto{\pgfqpoint{0.726515in}{1.611079in}}%
\pgfpathlineto{\pgfqpoint{0.718397in}{1.604556in}}%
\pgfpathlineto{\pgfqpoint{0.706247in}{1.597468in}}%
\pgfpathlineto{\pgfqpoint{0.702741in}{1.595083in}}%
\pgfpathlineto{\pgfqpoint{0.687084in}{1.590380in}}%
\pgfpathlineto{\pgfqpoint{0.671428in}{1.592432in}}%
\pgfpathlineto{\pgfqpoint{0.662044in}{1.597468in}}%
\pgfpathlineto{\pgfqpoint{0.655771in}{1.600382in}}%
\pgfpathlineto{\pgfqpoint{0.640979in}{1.611079in}}%
\pgfpathlineto{\pgfqpoint{0.640115in}{1.611669in}}%
\pgfpathlineto{\pgfqpoint{0.624458in}{1.624588in}}%
\pgfpathlineto{\pgfqpoint{0.624341in}{1.624691in}}%
\pgfpathlineto{\pgfqpoint{0.609202in}{1.638302in}}%
\pgfpathlineto{\pgfqpoint{0.608801in}{1.638724in}}%
\pgfpathlineto{\pgfqpoint{0.595745in}{1.651913in}}%
\pgfpathlineto{\pgfqpoint{0.593145in}{1.655731in}}%
\pgfpathlineto{\pgfqpoint{0.585574in}{1.665524in}}%
\pgfpathlineto{\pgfqpoint{0.581789in}{1.679135in}}%
\pgfpathlineto{\pgfqpoint{0.577488in}{1.679135in}}%
\pgfpathlineto{\pgfqpoint{0.561832in}{1.679135in}}%
\pgfpathlineto{\pgfqpoint{0.546175in}{1.679135in}}%
\pgfpathlineto{\pgfqpoint{0.530519in}{1.679135in}}%
\pgfpathlineto{\pgfqpoint{0.514862in}{1.679135in}}%
\pgfpathlineto{\pgfqpoint{0.499205in}{1.679135in}}%
\pgfpathlineto{\pgfqpoint{0.483549in}{1.679135in}}%
\pgfpathlineto{\pgfqpoint{0.476205in}{1.679135in}}%
\pgfpathlineto{\pgfqpoint{0.472300in}{1.665524in}}%
\pgfpathlineto{\pgfqpoint{0.467892in}{1.660067in}}%
\pgfpathlineto{\pgfqpoint{0.462219in}{1.651913in}}%
\pgfpathlineto{\pgfqpoint{0.452236in}{1.642058in}}%
\pgfpathlineto{\pgfqpoint{0.448661in}{1.638302in}}%
\pgfpathlineto{\pgfqpoint{0.436579in}{1.627464in}}%
\pgfpathlineto{\pgfqpoint{0.433389in}{1.624691in}}%
\pgfpathlineto{\pgfqpoint{0.420923in}{1.614187in}}%
\pgfpathlineto{\pgfqpoint{0.416602in}{1.611079in}}%
\pgfpathlineto{\pgfqpoint{0.405266in}{1.602400in}}%
\pgfpathlineto{\pgfqpoint{0.395886in}{1.597468in}}%
\pgfpathlineto{\pgfqpoint{0.389609in}{1.593637in}}%
\pgfpathlineto{\pgfqpoint{0.373953in}{1.590242in}}%
\pgfpathlineto{\pgfqpoint{0.373953in}{1.583857in}}%
\pgfpathlineto{\pgfqpoint{0.373953in}{1.570246in}}%
\pgfpathlineto{\pgfqpoint{0.373953in}{1.556635in}}%
\pgfpathlineto{\pgfqpoint{0.373953in}{1.543024in}}%
\pgfpathlineto{\pgfqpoint{0.373953in}{1.529413in}}%
\pgfpathlineto{\pgfqpoint{0.373953in}{1.515802in}}%
\pgfpathlineto{\pgfqpoint{0.373953in}{1.502191in}}%
\pgfpathlineto{\pgfqpoint{0.373953in}{1.498452in}}%
\pgfpathlineto{\pgfqpoint{0.389609in}{1.495161in}}%
\pgfpathlineto{\pgfqpoint{0.400874in}{1.488579in}}%
\pgfpathlineto{\pgfqpoint{0.405266in}{1.486319in}}%
\pgfpathlineto{\pgfqpoint{0.420436in}{1.474968in}}%
\pgfpathlineto{\pgfqpoint{0.420923in}{1.474620in}}%
\pgfpathlineto{\pgfqpoint{0.436579in}{1.461459in}}%
\pgfpathlineto{\pgfqpoint{0.436697in}{1.461357in}}%
\pgfpathlineto{\pgfqpoint{0.451558in}{1.447746in}}%
\pgfpathlineto{\pgfqpoint{0.452236in}{1.446994in}}%
\pgfpathlineto{\pgfqpoint{0.464540in}{1.434135in}}%
\pgfpathlineto{\pgfqpoint{0.467892in}{1.428681in}}%
\pgfpathlineto{\pgfqpoint{0.473686in}{1.420524in}}%
\pgfpathlineto{\pgfqpoint{0.476046in}{1.406913in}}%
\pgfpathlineto{\pgfqpoint{0.470636in}{1.393302in}}%
\pgfpathlineto{\pgfqpoint{0.467892in}{1.390254in}}%
\pgfpathlineto{\pgfqpoint{0.459739in}{1.379691in}}%
\pgfpathlineto{\pgfqpoint{0.452236in}{1.372634in}}%
\pgfpathlineto{\pgfqpoint{0.445709in}{1.366079in}}%
\pgfpathlineto{\pgfqpoint{0.436579in}{1.358019in}}%
\pgfpathlineto{\pgfqpoint{0.430099in}{1.352468in}}%
\pgfpathlineto{\pgfqpoint{0.420923in}{1.344682in}}%
\pgfpathlineto{\pgfqpoint{0.412876in}{1.338857in}}%
\pgfpathlineto{\pgfqpoint{0.405266in}{1.332872in}}%
\pgfpathlineto{\pgfqpoint{0.391125in}{1.325246in}}%
\pgfpathlineto{\pgfqpoint{0.389609in}{1.324275in}}%
\pgfpathlineto{\pgfqpoint{0.373953in}{1.320754in}}%
\pgfpathlineto{\pgfqpoint{0.373953in}{1.311635in}}%
\pgfpathlineto{\pgfqpoint{0.373953in}{1.298024in}}%
\pgfpathlineto{\pgfqpoint{0.373953in}{1.284413in}}%
\pgfpathlineto{\pgfqpoint{0.373953in}{1.270802in}}%
\pgfpathlineto{\pgfqpoint{0.373953in}{1.257191in}}%
\pgfpathlineto{\pgfqpoint{0.373953in}{1.243579in}}%
\pgfpathlineto{\pgfqpoint{0.373953in}{1.229968in}}%
\pgfpathlineto{\pgfqpoint{0.373953in}{1.228815in}}%
\pgfpathlineto{\pgfqpoint{0.389609in}{1.225610in}}%
\pgfpathlineto{\pgfqpoint{0.405266in}{1.216797in}}%
\pgfpathlineto{\pgfqpoint{0.405790in}{1.216357in}}%
\pgfpathlineto{\pgfqpoint{0.420923in}{1.205131in}}%
\pgfpathlineto{\pgfqpoint{0.423660in}{1.202746in}}%
\pgfpathlineto{\pgfqpoint{0.436579in}{1.191852in}}%
\pgfpathlineto{\pgfqpoint{0.439709in}{1.189135in}}%
\pgfpathlineto{\pgfqpoint{0.452236in}{1.177402in}}%
\pgfpathlineto{\pgfqpoint{0.454386in}{1.175524in}}%
\pgfpathlineto{\pgfqpoint{0.466682in}{1.161913in}}%
\pgfpathlineto{\pgfqpoint{0.467892in}{1.159626in}}%
\pgfpathlineto{\pgfqpoint{0.474779in}{1.148302in}}%
\pgfpathlineto{\pgfqpoint{0.475568in}{1.134691in}}%
\pgfpathlineto{\pgfqpoint{0.468710in}{1.121079in}}%
\pgfpathlineto{\pgfqpoint{0.467892in}{1.120250in}}%
\pgfpathlineto{\pgfqpoint{0.457121in}{1.107468in}}%
\pgfpathlineto{\pgfqpoint{0.452236in}{1.103056in}}%
\pgfpathlineto{\pgfqpoint{0.442719in}{1.093857in}}%
\pgfpathlineto{\pgfqpoint{0.436579in}{1.088495in}}%
\pgfpathlineto{\pgfqpoint{0.426851in}{1.080246in}}%
\pgfpathlineto{\pgfqpoint{0.420923in}{1.075159in}}%
\pgfpathlineto{\pgfqpoint{0.409269in}{1.066635in}}%
\pgfpathlineto{\pgfqpoint{0.405266in}{1.063387in}}%
\pgfpathlineto{\pgfqpoint{0.389609in}{1.054764in}}%
\pgfpathlineto{\pgfqpoint{0.380871in}{1.053024in}}%
\pgfpathlineto{\pgfqpoint{0.373953in}{1.051389in}}%
\pgfpathlineto{\pgfqpoint{0.373953in}{1.039413in}}%
\pgfpathlineto{\pgfqpoint{0.373953in}{1.025802in}}%
\pgfpathlineto{\pgfqpoint{0.373953in}{1.012191in}}%
\pgfpathlineto{\pgfqpoint{0.373953in}{0.998579in}}%
\pgfpathlineto{\pgfqpoint{0.373953in}{0.984968in}}%
\pgfpathlineto{\pgfqpoint{0.373953in}{0.971357in}}%
\pgfpathlineto{\pgfqpoint{0.373953in}{0.959381in}}%
\pgfpathlineto{\pgfqpoint{0.380871in}{0.957746in}}%
\pgfpathlineto{\pgfqpoint{0.389609in}{0.956006in}}%
\pgfpathlineto{\pgfqpoint{0.405266in}{0.947383in}}%
\pgfpathlineto{\pgfqpoint{0.409269in}{0.944135in}}%
\pgfpathlineto{\pgfqpoint{0.420923in}{0.935611in}}%
\pgfpathlineto{\pgfqpoint{0.426851in}{0.930524in}}%
\pgfpathlineto{\pgfqpoint{0.436579in}{0.922275in}}%
\pgfpathlineto{\pgfqpoint{0.442719in}{0.916913in}}%
\pgfpathlineto{\pgfqpoint{0.452236in}{0.907714in}}%
\pgfpathlineto{\pgfqpoint{0.457121in}{0.903302in}}%
\pgfpathlineto{\pgfqpoint{0.467892in}{0.890520in}}%
\pgfpathlineto{\pgfqpoint{0.468710in}{0.889691in}}%
\pgfpathlineto{\pgfqpoint{0.475568in}{0.876079in}}%
\pgfpathlineto{\pgfqpoint{0.474779in}{0.862468in}}%
\pgfpathlineto{\pgfqpoint{0.467892in}{0.851144in}}%
\pgfpathlineto{\pgfqpoint{0.466682in}{0.848857in}}%
\pgfpathlineto{\pgfqpoint{0.454386in}{0.835246in}}%
\pgfpathlineto{\pgfqpoint{0.452236in}{0.833368in}}%
\pgfpathlineto{\pgfqpoint{0.439709in}{0.821635in}}%
\pgfpathlineto{\pgfqpoint{0.436579in}{0.818918in}}%
\pgfpathlineto{\pgfqpoint{0.423660in}{0.808024in}}%
\pgfpathlineto{\pgfqpoint{0.420923in}{0.805639in}}%
\pgfpathlineto{\pgfqpoint{0.405790in}{0.794413in}}%
\pgfpathlineto{\pgfqpoint{0.405266in}{0.793973in}}%
\pgfpathlineto{\pgfqpoint{0.389609in}{0.785160in}}%
\pgfpathlineto{\pgfqpoint{0.373953in}{0.781955in}}%
\pgfpathlineto{\pgfqpoint{0.373953in}{0.780802in}}%
\pgfpathlineto{\pgfqpoint{0.373953in}{0.767191in}}%
\pgfpathlineto{\pgfqpoint{0.373953in}{0.753579in}}%
\pgfpathlineto{\pgfqpoint{0.373953in}{0.739968in}}%
\pgfpathlineto{\pgfqpoint{0.373953in}{0.726357in}}%
\pgfpathlineto{\pgfqpoint{0.373953in}{0.712746in}}%
\pgfpathlineto{\pgfqpoint{0.373953in}{0.699135in}}%
\pgfpathlineto{\pgfqpoint{0.373953in}{0.690016in}}%
\pgfpathlineto{\pgfqpoint{0.389609in}{0.686495in}}%
\pgfpathlineto{\pgfqpoint{0.391125in}{0.685524in}}%
\pgfpathlineto{\pgfqpoint{0.405266in}{0.677898in}}%
\pgfpathlineto{\pgfqpoint{0.412876in}{0.671913in}}%
\pgfpathlineto{\pgfqpoint{0.420923in}{0.666088in}}%
\pgfpathlineto{\pgfqpoint{0.430099in}{0.658302in}}%
\pgfpathlineto{\pgfqpoint{0.436579in}{0.652751in}}%
\pgfpathlineto{\pgfqpoint{0.445709in}{0.644691in}}%
\pgfpathlineto{\pgfqpoint{0.452236in}{0.638136in}}%
\pgfpathlineto{\pgfqpoint{0.459739in}{0.631079in}}%
\pgfpathlineto{\pgfqpoint{0.467892in}{0.620516in}}%
\pgfpathlineto{\pgfqpoint{0.470636in}{0.617468in}}%
\pgfpathlineto{\pgfqpoint{0.476046in}{0.603857in}}%
\pgfpathlineto{\pgfqpoint{0.473686in}{0.590246in}}%
\pgfpathlineto{\pgfqpoint{0.467892in}{0.582089in}}%
\pgfpathlineto{\pgfqpoint{0.464540in}{0.576635in}}%
\pgfpathlineto{\pgfqpoint{0.452236in}{0.563776in}}%
\pgfpathlineto{\pgfqpoint{0.451558in}{0.563024in}}%
\pgfpathlineto{\pgfqpoint{0.436697in}{0.549413in}}%
\pgfpathlineto{\pgfqpoint{0.436579in}{0.549311in}}%
\pgfpathlineto{\pgfqpoint{0.420923in}{0.536150in}}%
\pgfpathlineto{\pgfqpoint{0.420436in}{0.535802in}}%
\pgfpathlineto{\pgfqpoint{0.405266in}{0.524451in}}%
\pgfpathlineto{\pgfqpoint{0.400874in}{0.522191in}}%
\pgfpathlineto{\pgfqpoint{0.389609in}{0.515609in}}%
\pgfpathlineto{\pgfqpoint{0.373953in}{0.512318in}}%
\pgfpathlineto{\pgfqpoint{0.373953in}{0.508579in}}%
\pgfpathlineto{\pgfqpoint{0.373953in}{0.494968in}}%
\pgfpathlineto{\pgfqpoint{0.373953in}{0.481357in}}%
\pgfpathlineto{\pgfqpoint{0.373953in}{0.467746in}}%
\pgfpathlineto{\pgfqpoint{0.373953in}{0.454135in}}%
\pgfpathlineto{\pgfqpoint{0.373953in}{0.440524in}}%
\pgfpathlineto{\pgfqpoint{0.373953in}{0.426913in}}%
\pgfpathlineto{\pgfqpoint{0.373953in}{0.420528in}}%
\pgfpathlineto{\pgfqpoint{0.389609in}{0.417133in}}%
\pgfpathlineto{\pgfqpoint{0.395886in}{0.413302in}}%
\pgfpathlineto{\pgfqpoint{0.405266in}{0.408370in}}%
\pgfpathlineto{\pgfqpoint{0.416602in}{0.399691in}}%
\pgfpathlineto{\pgfqpoint{0.420923in}{0.396583in}}%
\pgfpathlineto{\pgfqpoint{0.433389in}{0.386079in}}%
\pgfpathlineto{\pgfqpoint{0.436579in}{0.383306in}}%
\pgfpathlineto{\pgfqpoint{0.448661in}{0.372468in}}%
\pgfpathlineto{\pgfqpoint{0.452236in}{0.368712in}}%
\pgfpathlineto{\pgfqpoint{0.462219in}{0.358857in}}%
\pgfpathlineto{\pgfqpoint{0.467892in}{0.350703in}}%
\pgfpathlineto{\pgfqpoint{0.472300in}{0.345246in}}%
\pgfpathlineto{\pgfqpoint{0.476205in}{0.331635in}}%
\pgfpathlineto{\pgfqpoint{0.483549in}{0.331635in}}%
\pgfpathclose%
\pgfpathmoveto{\pgfqpoint{0.528478in}{0.358857in}}%
\pgfpathlineto{\pgfqpoint{0.514862in}{0.360291in}}%
\pgfpathlineto{\pgfqpoint{0.499205in}{0.366066in}}%
\pgfpathlineto{\pgfqpoint{0.488943in}{0.372468in}}%
\pgfpathlineto{\pgfqpoint{0.483549in}{0.375256in}}%
\pgfpathlineto{\pgfqpoint{0.468238in}{0.386079in}}%
\pgfpathlineto{\pgfqpoint{0.467892in}{0.386313in}}%
\pgfpathlineto{\pgfqpoint{0.452236in}{0.398952in}}%
\pgfpathlineto{\pgfqpoint{0.451386in}{0.399691in}}%
\pgfpathlineto{\pgfqpoint{0.436848in}{0.413302in}}%
\pgfpathlineto{\pgfqpoint{0.436579in}{0.413602in}}%
\pgfpathlineto{\pgfqpoint{0.424129in}{0.426913in}}%
\pgfpathlineto{\pgfqpoint{0.420923in}{0.431602in}}%
\pgfpathlineto{\pgfqpoint{0.413559in}{0.440524in}}%
\pgfpathlineto{\pgfqpoint{0.406915in}{0.454135in}}%
\pgfpathlineto{\pgfqpoint{0.405266in}{0.465972in}}%
\pgfpathlineto{\pgfqpoint{0.404884in}{0.467746in}}%
\pgfpathlineto{\pgfqpoint{0.405266in}{0.468933in}}%
\pgfpathlineto{\pgfqpoint{0.407863in}{0.481357in}}%
\pgfpathlineto{\pgfqpoint{0.415459in}{0.494968in}}%
\pgfpathlineto{\pgfqpoint{0.420923in}{0.501144in}}%
\pgfpathlineto{\pgfqpoint{0.426389in}{0.508579in}}%
\pgfpathlineto{\pgfqpoint{0.436579in}{0.519032in}}%
\pgfpathlineto{\pgfqpoint{0.439531in}{0.522191in}}%
\pgfpathlineto{\pgfqpoint{0.452236in}{0.533788in}}%
\pgfpathlineto{\pgfqpoint{0.454606in}{0.535802in}}%
\pgfpathlineto{\pgfqpoint{0.467892in}{0.546539in}}%
\pgfpathlineto{\pgfqpoint{0.472156in}{0.549413in}}%
\pgfpathlineto{\pgfqpoint{0.483549in}{0.557629in}}%
\pgfpathlineto{\pgfqpoint{0.493767in}{0.563024in}}%
\pgfpathlineto{\pgfqpoint{0.499205in}{0.566604in}}%
\pgfpathlineto{\pgfqpoint{0.514862in}{0.572779in}}%
\pgfpathlineto{\pgfqpoint{0.530519in}{0.574540in}}%
\pgfpathlineto{\pgfqpoint{0.546175in}{0.571898in}}%
\pgfpathlineto{\pgfqpoint{0.561832in}{0.564838in}}%
\pgfpathlineto{\pgfqpoint{0.564402in}{0.563024in}}%
\pgfpathlineto{\pgfqpoint{0.577488in}{0.555596in}}%
\pgfpathlineto{\pgfqpoint{0.585714in}{0.549413in}}%
\pgfpathlineto{\pgfqpoint{0.593145in}{0.544180in}}%
\pgfpathlineto{\pgfqpoint{0.603291in}{0.535802in}}%
\pgfpathlineto{\pgfqpoint{0.608801in}{0.531011in}}%
\pgfpathlineto{\pgfqpoint{0.618438in}{0.522191in}}%
\pgfpathlineto{\pgfqpoint{0.624458in}{0.515731in}}%
\pgfpathlineto{\pgfqpoint{0.631570in}{0.508579in}}%
\pgfpathlineto{\pgfqpoint{0.640115in}{0.497203in}}%
\pgfpathlineto{\pgfqpoint{0.642201in}{0.494968in}}%
\pgfpathlineto{\pgfqpoint{0.650322in}{0.481357in}}%
\pgfpathlineto{\pgfqpoint{0.653361in}{0.467746in}}%
\pgfpathlineto{\pgfqpoint{0.651335in}{0.454135in}}%
\pgfpathlineto{\pgfqpoint{0.644233in}{0.440524in}}%
\pgfpathlineto{\pgfqpoint{0.640115in}{0.435796in}}%
\pgfpathlineto{\pgfqpoint{0.633909in}{0.426913in}}%
\pgfpathlineto{\pgfqpoint{0.624458in}{0.417008in}}%
\pgfpathlineto{\pgfqpoint{0.621153in}{0.413302in}}%
\pgfpathlineto{\pgfqpoint{0.608801in}{0.401751in}}%
\pgfpathlineto{\pgfqpoint{0.606485in}{0.399691in}}%
\pgfpathlineto{\pgfqpoint{0.593145in}{0.388646in}}%
\pgfpathlineto{\pgfqpoint{0.589511in}{0.386079in}}%
\pgfpathlineto{\pgfqpoint{0.577488in}{0.377221in}}%
\pgfpathlineto{\pgfqpoint{0.568936in}{0.372468in}}%
\pgfpathlineto{\pgfqpoint{0.561832in}{0.367719in}}%
\pgfpathlineto{\pgfqpoint{0.546175in}{0.361115in}}%
\pgfpathlineto{\pgfqpoint{0.531883in}{0.358857in}}%
\pgfpathlineto{\pgfqpoint{0.530519in}{0.358525in}}%
\pgfpathlineto{\pgfqpoint{0.528478in}{0.358857in}}%
\pgfpathclose%
\pgfpathmoveto{\pgfqpoint{0.842722in}{0.358857in}}%
\pgfpathlineto{\pgfqpoint{0.827993in}{0.359632in}}%
\pgfpathlineto{\pgfqpoint{0.812337in}{0.364580in}}%
\pgfpathlineto{\pgfqpoint{0.798676in}{0.372468in}}%
\pgfpathlineto{\pgfqpoint{0.796680in}{0.373420in}}%
\pgfpathlineto{\pgfqpoint{0.781024in}{0.383875in}}%
\pgfpathlineto{\pgfqpoint{0.778314in}{0.386079in}}%
\pgfpathlineto{\pgfqpoint{0.765367in}{0.396240in}}%
\pgfpathlineto{\pgfqpoint{0.761372in}{0.399691in}}%
\pgfpathlineto{\pgfqpoint{0.749710in}{0.410545in}}%
\pgfpathlineto{\pgfqpoint{0.746712in}{0.413302in}}%
\pgfpathlineto{\pgfqpoint{0.734332in}{0.426913in}}%
\pgfpathlineto{\pgfqpoint{0.734054in}{0.427327in}}%
\pgfpathlineto{\pgfqpoint{0.723646in}{0.440524in}}%
\pgfpathlineto{\pgfqpoint{0.718397in}{0.451883in}}%
\pgfpathlineto{\pgfqpoint{0.716903in}{0.454135in}}%
\pgfpathlineto{\pgfqpoint{0.714272in}{0.467746in}}%
\pgfpathlineto{\pgfqpoint{0.718219in}{0.481357in}}%
\pgfpathlineto{\pgfqpoint{0.718397in}{0.481594in}}%
\pgfpathlineto{\pgfqpoint{0.725438in}{0.494968in}}%
\pgfpathlineto{\pgfqpoint{0.734054in}{0.505161in}}%
\pgfpathlineto{\pgfqpoint{0.736527in}{0.508579in}}%
\pgfpathlineto{\pgfqpoint{0.749595in}{0.522191in}}%
\pgfpathlineto{\pgfqpoint{0.749710in}{0.522294in}}%
\pgfpathlineto{\pgfqpoint{0.764559in}{0.535802in}}%
\pgfpathlineto{\pgfqpoint{0.765367in}{0.536498in}}%
\pgfpathlineto{\pgfqpoint{0.781024in}{0.548791in}}%
\pgfpathlineto{\pgfqpoint{0.781992in}{0.549413in}}%
\pgfpathlineto{\pgfqpoint{0.796680in}{0.559529in}}%
\pgfpathlineto{\pgfqpoint{0.803854in}{0.563024in}}%
\pgfpathlineto{\pgfqpoint{0.812337in}{0.568193in}}%
\pgfpathlineto{\pgfqpoint{0.827993in}{0.573483in}}%
\pgfpathlineto{\pgfqpoint{0.843650in}{0.574364in}}%
\pgfpathlineto{\pgfqpoint{0.859306in}{0.570840in}}%
\pgfpathlineto{\pgfqpoint{0.874716in}{0.563024in}}%
\pgfpathlineto{\pgfqpoint{0.874963in}{0.562924in}}%
\pgfpathlineto{\pgfqpoint{0.890620in}{0.553430in}}%
\pgfpathlineto{\pgfqpoint{0.895769in}{0.549413in}}%
\pgfpathlineto{\pgfqpoint{0.906276in}{0.541716in}}%
\pgfpathlineto{\pgfqpoint{0.913313in}{0.535802in}}%
\pgfpathlineto{\pgfqpoint{0.921933in}{0.528164in}}%
\pgfpathlineto{\pgfqpoint{0.928468in}{0.522191in}}%
\pgfpathlineto{\pgfqpoint{0.937589in}{0.512416in}}%
\pgfpathlineto{\pgfqpoint{0.941503in}{0.508579in}}%
\pgfpathlineto{\pgfqpoint{0.952063in}{0.494968in}}%
\pgfpathlineto{\pgfqpoint{0.953246in}{0.492548in}}%
\pgfpathlineto{\pgfqpoint{0.960412in}{0.481357in}}%
\pgfpathlineto{\pgfqpoint{0.963691in}{0.467746in}}%
\pgfpathlineto{\pgfqpoint{0.961505in}{0.454135in}}%
\pgfpathlineto{\pgfqpoint{0.953842in}{0.440524in}}%
\pgfpathlineto{\pgfqpoint{0.953246in}{0.439881in}}%
\pgfpathlineto{\pgfqpoint{0.943937in}{0.426913in}}%
\pgfpathlineto{\pgfqpoint{0.937589in}{0.420427in}}%
\pgfpathlineto{\pgfqpoint{0.931225in}{0.413302in}}%
\pgfpathlineto{\pgfqpoint{0.921933in}{0.404624in}}%
\pgfpathlineto{\pgfqpoint{0.916492in}{0.399691in}}%
\pgfpathlineto{\pgfqpoint{0.906276in}{0.391081in}}%
\pgfpathlineto{\pgfqpoint{0.899468in}{0.386079in}}%
\pgfpathlineto{\pgfqpoint{0.890620in}{0.379313in}}%
\pgfpathlineto{\pgfqpoint{0.879082in}{0.372468in}}%
\pgfpathlineto{\pgfqpoint{0.874963in}{0.369536in}}%
\pgfpathlineto{\pgfqpoint{0.859306in}{0.362105in}}%
\pgfpathlineto{\pgfqpoint{0.843884in}{0.358857in}}%
\pgfpathlineto{\pgfqpoint{0.843650in}{0.358781in}}%
\pgfpathlineto{\pgfqpoint{0.842722in}{0.358857in}}%
\pgfpathclose%
\pgfpathmoveto{\pgfqpoint{1.454022in}{0.358857in}}%
\pgfpathlineto{\pgfqpoint{1.438599in}{0.362105in}}%
\pgfpathlineto{\pgfqpoint{1.422943in}{0.369536in}}%
\pgfpathlineto{\pgfqpoint{1.418823in}{0.372468in}}%
\pgfpathlineto{\pgfqpoint{1.407286in}{0.379313in}}%
\pgfpathlineto{\pgfqpoint{1.398438in}{0.386079in}}%
\pgfpathlineto{\pgfqpoint{1.391630in}{0.391081in}}%
\pgfpathlineto{\pgfqpoint{1.381414in}{0.399691in}}%
\pgfpathlineto{\pgfqpoint{1.375973in}{0.404624in}}%
\pgfpathlineto{\pgfqpoint{1.366681in}{0.413302in}}%
\pgfpathlineto{\pgfqpoint{1.360317in}{0.420427in}}%
\pgfpathlineto{\pgfqpoint{1.353969in}{0.426913in}}%
\pgfpathlineto{\pgfqpoint{1.344660in}{0.439881in}}%
\pgfpathlineto{\pgfqpoint{1.344063in}{0.440524in}}%
\pgfpathlineto{\pgfqpoint{1.336401in}{0.454135in}}%
\pgfpathlineto{\pgfqpoint{1.334215in}{0.467746in}}%
\pgfpathlineto{\pgfqpoint{1.337494in}{0.481357in}}%
\pgfpathlineto{\pgfqpoint{1.344660in}{0.492548in}}%
\pgfpathlineto{\pgfqpoint{1.345843in}{0.494968in}}%
\pgfpathlineto{\pgfqpoint{1.356403in}{0.508579in}}%
\pgfpathlineto{\pgfqpoint{1.360317in}{0.512416in}}%
\pgfpathlineto{\pgfqpoint{1.369438in}{0.522191in}}%
\pgfpathlineto{\pgfqpoint{1.375973in}{0.528164in}}%
\pgfpathlineto{\pgfqpoint{1.384593in}{0.535802in}}%
\pgfpathlineto{\pgfqpoint{1.391630in}{0.541716in}}%
\pgfpathlineto{\pgfqpoint{1.402136in}{0.549413in}}%
\pgfpathlineto{\pgfqpoint{1.407286in}{0.553430in}}%
\pgfpathlineto{\pgfqpoint{1.422943in}{0.562924in}}%
\pgfpathlineto{\pgfqpoint{1.423190in}{0.563024in}}%
\pgfpathlineto{\pgfqpoint{1.438599in}{0.570840in}}%
\pgfpathlineto{\pgfqpoint{1.454256in}{0.574364in}}%
\pgfpathlineto{\pgfqpoint{1.469913in}{0.573483in}}%
\pgfpathlineto{\pgfqpoint{1.485569in}{0.568193in}}%
\pgfpathlineto{\pgfqpoint{1.494052in}{0.563024in}}%
\pgfpathlineto{\pgfqpoint{1.501226in}{0.559529in}}%
\pgfpathlineto{\pgfqpoint{1.515914in}{0.549413in}}%
\pgfpathlineto{\pgfqpoint{1.516882in}{0.548791in}}%
\pgfpathlineto{\pgfqpoint{1.532539in}{0.536498in}}%
\pgfpathlineto{\pgfqpoint{1.533347in}{0.535802in}}%
\pgfpathlineto{\pgfqpoint{1.548195in}{0.522294in}}%
\pgfpathlineto{\pgfqpoint{1.548311in}{0.522191in}}%
\pgfpathlineto{\pgfqpoint{1.561378in}{0.508579in}}%
\pgfpathlineto{\pgfqpoint{1.563852in}{0.505161in}}%
\pgfpathlineto{\pgfqpoint{1.572468in}{0.494968in}}%
\pgfpathlineto{\pgfqpoint{1.579508in}{0.481594in}}%
\pgfpathlineto{\pgfqpoint{1.579687in}{0.481357in}}%
\pgfpathlineto{\pgfqpoint{1.583634in}{0.467746in}}%
\pgfpathlineto{\pgfqpoint{1.581003in}{0.454135in}}%
\pgfpathlineto{\pgfqpoint{1.579508in}{0.451883in}}%
\pgfpathlineto{\pgfqpoint{1.574260in}{0.440524in}}%
\pgfpathlineto{\pgfqpoint{1.563852in}{0.427327in}}%
\pgfpathlineto{\pgfqpoint{1.563573in}{0.426913in}}%
\pgfpathlineto{\pgfqpoint{1.551194in}{0.413302in}}%
\pgfpathlineto{\pgfqpoint{1.548195in}{0.410545in}}%
\pgfpathlineto{\pgfqpoint{1.536534in}{0.399691in}}%
\pgfpathlineto{\pgfqpoint{1.532539in}{0.396240in}}%
\pgfpathlineto{\pgfqpoint{1.519592in}{0.386079in}}%
\pgfpathlineto{\pgfqpoint{1.516882in}{0.383875in}}%
\pgfpathlineto{\pgfqpoint{1.501226in}{0.373420in}}%
\pgfpathlineto{\pgfqpoint{1.499230in}{0.372468in}}%
\pgfpathlineto{\pgfqpoint{1.485569in}{0.364580in}}%
\pgfpathlineto{\pgfqpoint{1.469913in}{0.359632in}}%
\pgfpathlineto{\pgfqpoint{1.455184in}{0.358857in}}%
\pgfpathlineto{\pgfqpoint{1.454256in}{0.358781in}}%
\pgfpathlineto{\pgfqpoint{1.454022in}{0.358857in}}%
\pgfpathclose%
\pgfpathmoveto{\pgfqpoint{1.766022in}{0.358857in}}%
\pgfpathlineto{\pgfqpoint{1.751731in}{0.361115in}}%
\pgfpathlineto{\pgfqpoint{1.736074in}{0.367719in}}%
\pgfpathlineto{\pgfqpoint{1.728970in}{0.372468in}}%
\pgfpathlineto{\pgfqpoint{1.720418in}{0.377221in}}%
\pgfpathlineto{\pgfqpoint{1.708395in}{0.386079in}}%
\pgfpathlineto{\pgfqpoint{1.704761in}{0.388646in}}%
\pgfpathlineto{\pgfqpoint{1.691421in}{0.399691in}}%
\pgfpathlineto{\pgfqpoint{1.689104in}{0.401751in}}%
\pgfpathlineto{\pgfqpoint{1.676753in}{0.413302in}}%
\pgfpathlineto{\pgfqpoint{1.673448in}{0.417008in}}%
\pgfpathlineto{\pgfqpoint{1.663997in}{0.426913in}}%
\pgfpathlineto{\pgfqpoint{1.657791in}{0.435796in}}%
\pgfpathlineto{\pgfqpoint{1.653673in}{0.440524in}}%
\pgfpathlineto{\pgfqpoint{1.646571in}{0.454135in}}%
\pgfpathlineto{\pgfqpoint{1.644545in}{0.467746in}}%
\pgfpathlineto{\pgfqpoint{1.647584in}{0.481357in}}%
\pgfpathlineto{\pgfqpoint{1.655705in}{0.494968in}}%
\pgfpathlineto{\pgfqpoint{1.657791in}{0.497203in}}%
\pgfpathlineto{\pgfqpoint{1.666336in}{0.508579in}}%
\pgfpathlineto{\pgfqpoint{1.673448in}{0.515731in}}%
\pgfpathlineto{\pgfqpoint{1.679467in}{0.522191in}}%
\pgfpathlineto{\pgfqpoint{1.689104in}{0.531011in}}%
\pgfpathlineto{\pgfqpoint{1.694615in}{0.535802in}}%
\pgfpathlineto{\pgfqpoint{1.704761in}{0.544180in}}%
\pgfpathlineto{\pgfqpoint{1.712192in}{0.549413in}}%
\pgfpathlineto{\pgfqpoint{1.720418in}{0.555596in}}%
\pgfpathlineto{\pgfqpoint{1.733504in}{0.563024in}}%
\pgfpathlineto{\pgfqpoint{1.736074in}{0.564838in}}%
\pgfpathlineto{\pgfqpoint{1.751731in}{0.571898in}}%
\pgfpathlineto{\pgfqpoint{1.767387in}{0.574540in}}%
\pgfpathlineto{\pgfqpoint{1.783044in}{0.572779in}}%
\pgfpathlineto{\pgfqpoint{1.798700in}{0.566604in}}%
\pgfpathlineto{\pgfqpoint{1.804139in}{0.563024in}}%
\pgfpathlineto{\pgfqpoint{1.814357in}{0.557629in}}%
\pgfpathlineto{\pgfqpoint{1.825750in}{0.549413in}}%
\pgfpathlineto{\pgfqpoint{1.830014in}{0.546539in}}%
\pgfpathlineto{\pgfqpoint{1.843300in}{0.535802in}}%
\pgfpathlineto{\pgfqpoint{1.845670in}{0.533788in}}%
\pgfpathlineto{\pgfqpoint{1.858375in}{0.522191in}}%
\pgfpathlineto{\pgfqpoint{1.861327in}{0.519032in}}%
\pgfpathlineto{\pgfqpoint{1.871517in}{0.508579in}}%
\pgfpathlineto{\pgfqpoint{1.876983in}{0.501144in}}%
\pgfpathlineto{\pgfqpoint{1.882447in}{0.494968in}}%
\pgfpathlineto{\pgfqpoint{1.890042in}{0.481357in}}%
\pgfpathlineto{\pgfqpoint{1.892640in}{0.468933in}}%
\pgfpathlineto{\pgfqpoint{1.893022in}{0.467746in}}%
\pgfpathlineto{\pgfqpoint{1.892640in}{0.465972in}}%
\pgfpathlineto{\pgfqpoint{1.890991in}{0.454135in}}%
\pgfpathlineto{\pgfqpoint{1.884347in}{0.440524in}}%
\pgfpathlineto{\pgfqpoint{1.876983in}{0.431602in}}%
\pgfpathlineto{\pgfqpoint{1.873777in}{0.426913in}}%
\pgfpathlineto{\pgfqpoint{1.861327in}{0.413602in}}%
\pgfpathlineto{\pgfqpoint{1.861058in}{0.413302in}}%
\pgfpathlineto{\pgfqpoint{1.846519in}{0.399691in}}%
\pgfpathlineto{\pgfqpoint{1.845670in}{0.398952in}}%
\pgfpathlineto{\pgfqpoint{1.830014in}{0.386313in}}%
\pgfpathlineto{\pgfqpoint{1.829668in}{0.386079in}}%
\pgfpathlineto{\pgfqpoint{1.814357in}{0.375256in}}%
\pgfpathlineto{\pgfqpoint{1.808963in}{0.372468in}}%
\pgfpathlineto{\pgfqpoint{1.798700in}{0.366066in}}%
\pgfpathlineto{\pgfqpoint{1.783044in}{0.360291in}}%
\pgfpathlineto{\pgfqpoint{1.769428in}{0.358857in}}%
\pgfpathlineto{\pgfqpoint{1.767387in}{0.358525in}}%
\pgfpathlineto{\pgfqpoint{1.766022in}{0.358857in}}%
\pgfpathclose%
\pgfpathmoveto{\pgfqpoint{1.108551in}{0.372468in}}%
\pgfpathlineto{\pgfqpoint{1.094155in}{0.381532in}}%
\pgfpathlineto{\pgfqpoint{1.088401in}{0.386079in}}%
\pgfpathlineto{\pgfqpoint{1.078498in}{0.393614in}}%
\pgfpathlineto{\pgfqpoint{1.071389in}{0.399691in}}%
\pgfpathlineto{\pgfqpoint{1.062842in}{0.407559in}}%
\pgfpathlineto{\pgfqpoint{1.056658in}{0.413302in}}%
\pgfpathlineto{\pgfqpoint{1.047185in}{0.423843in}}%
\pgfpathlineto{\pgfqpoint{1.044085in}{0.426913in}}%
\pgfpathlineto{\pgfqpoint{1.033880in}{0.440524in}}%
\pgfpathlineto{\pgfqpoint{1.031529in}{0.445823in}}%
\pgfpathlineto{\pgfqpoint{1.026477in}{0.454135in}}%
\pgfpathlineto{\pgfqpoint{1.024094in}{0.467746in}}%
\pgfpathlineto{\pgfqpoint{1.027669in}{0.481357in}}%
\pgfpathlineto{\pgfqpoint{1.031529in}{0.486951in}}%
\pgfpathlineto{\pgfqpoint{1.035581in}{0.494968in}}%
\pgfpathlineto{\pgfqpoint{1.046632in}{0.508579in}}%
\pgfpathlineto{\pgfqpoint{1.047185in}{0.509105in}}%
\pgfpathlineto{\pgfqpoint{1.059471in}{0.522191in}}%
\pgfpathlineto{\pgfqpoint{1.062842in}{0.525255in}}%
\pgfpathlineto{\pgfqpoint{1.074566in}{0.535802in}}%
\pgfpathlineto{\pgfqpoint{1.078498in}{0.539154in}}%
\pgfpathlineto{\pgfqpoint{1.092020in}{0.549413in}}%
\pgfpathlineto{\pgfqpoint{1.094155in}{0.551134in}}%
\pgfpathlineto{\pgfqpoint{1.109812in}{0.561294in}}%
\pgfpathlineto{\pgfqpoint{1.113695in}{0.563024in}}%
\pgfpathlineto{\pgfqpoint{1.125468in}{0.569605in}}%
\pgfpathlineto{\pgfqpoint{1.141125in}{0.574012in}}%
\pgfpathlineto{\pgfqpoint{1.156781in}{0.574012in}}%
\pgfpathlineto{\pgfqpoint{1.172438in}{0.569605in}}%
\pgfpathlineto{\pgfqpoint{1.184211in}{0.563024in}}%
\pgfpathlineto{\pgfqpoint{1.188094in}{0.561294in}}%
\pgfpathlineto{\pgfqpoint{1.203751in}{0.551134in}}%
\pgfpathlineto{\pgfqpoint{1.205885in}{0.549413in}}%
\pgfpathlineto{\pgfqpoint{1.219407in}{0.539154in}}%
\pgfpathlineto{\pgfqpoint{1.223340in}{0.535802in}}%
\pgfpathlineto{\pgfqpoint{1.235064in}{0.525255in}}%
\pgfpathlineto{\pgfqpoint{1.238435in}{0.522191in}}%
\pgfpathlineto{\pgfqpoint{1.250721in}{0.509105in}}%
\pgfpathlineto{\pgfqpoint{1.251274in}{0.508579in}}%
\pgfpathlineto{\pgfqpoint{1.262325in}{0.494968in}}%
\pgfpathlineto{\pgfqpoint{1.266377in}{0.486951in}}%
\pgfpathlineto{\pgfqpoint{1.270237in}{0.481357in}}%
\pgfpathlineto{\pgfqpoint{1.273812in}{0.467746in}}%
\pgfpathlineto{\pgfqpoint{1.271429in}{0.454135in}}%
\pgfpathlineto{\pgfqpoint{1.266377in}{0.445823in}}%
\pgfpathlineto{\pgfqpoint{1.264026in}{0.440524in}}%
\pgfpathlineto{\pgfqpoint{1.253821in}{0.426913in}}%
\pgfpathlineto{\pgfqpoint{1.250721in}{0.423843in}}%
\pgfpathlineto{\pgfqpoint{1.241248in}{0.413302in}}%
\pgfpathlineto{\pgfqpoint{1.235064in}{0.407559in}}%
\pgfpathlineto{\pgfqpoint{1.226517in}{0.399691in}}%
\pgfpathlineto{\pgfqpoint{1.219407in}{0.393614in}}%
\pgfpathlineto{\pgfqpoint{1.209505in}{0.386079in}}%
\pgfpathlineto{\pgfqpoint{1.203751in}{0.381532in}}%
\pgfpathlineto{\pgfqpoint{1.189355in}{0.372468in}}%
\pgfpathlineto{\pgfqpoint{1.188094in}{0.371519in}}%
\pgfpathlineto{\pgfqpoint{1.172438in}{0.363260in}}%
\pgfpathlineto{\pgfqpoint{1.156781in}{0.359138in}}%
\pgfpathlineto{\pgfqpoint{1.141125in}{0.359138in}}%
\pgfpathlineto{\pgfqpoint{1.125468in}{0.363260in}}%
\pgfpathlineto{\pgfqpoint{1.109812in}{0.371519in}}%
\pgfpathlineto{\pgfqpoint{1.108551in}{0.372468in}}%
\pgfpathclose%
\pgfpathmoveto{\pgfqpoint{0.656342in}{0.522191in}}%
\pgfpathlineto{\pgfqpoint{0.655771in}{0.522450in}}%
\pgfpathlineto{\pgfqpoint{0.640115in}{0.533640in}}%
\pgfpathlineto{\pgfqpoint{0.637627in}{0.535802in}}%
\pgfpathlineto{\pgfqpoint{0.624458in}{0.546672in}}%
\pgfpathlineto{\pgfqpoint{0.621306in}{0.549413in}}%
\pgfpathlineto{\pgfqpoint{0.608801in}{0.560861in}}%
\pgfpathlineto{\pgfqpoint{0.606315in}{0.563024in}}%
\pgfpathlineto{\pgfqpoint{0.593444in}{0.576635in}}%
\pgfpathlineto{\pgfqpoint{0.593145in}{0.577132in}}%
\pgfpathlineto{\pgfqpoint{0.584231in}{0.590246in}}%
\pgfpathlineto{\pgfqpoint{0.581943in}{0.603857in}}%
\pgfpathlineto{\pgfqpoint{0.587187in}{0.617468in}}%
\pgfpathlineto{\pgfqpoint{0.593145in}{0.624384in}}%
\pgfpathlineto{\pgfqpoint{0.598203in}{0.631079in}}%
\pgfpathlineto{\pgfqpoint{0.608801in}{0.641280in}}%
\pgfpathlineto{\pgfqpoint{0.612189in}{0.644691in}}%
\pgfpathlineto{\pgfqpoint{0.624458in}{0.655554in}}%
\pgfpathlineto{\pgfqpoint{0.627731in}{0.658302in}}%
\pgfpathlineto{\pgfqpoint{0.640115in}{0.668587in}}%
\pgfpathlineto{\pgfqpoint{0.644962in}{0.671913in}}%
\pgfpathlineto{\pgfqpoint{0.655771in}{0.679943in}}%
\pgfpathlineto{\pgfqpoint{0.667486in}{0.685524in}}%
\pgfpathlineto{\pgfqpoint{0.671428in}{0.687745in}}%
\pgfpathlineto{\pgfqpoint{0.687084in}{0.689872in}}%
\pgfpathlineto{\pgfqpoint{0.701013in}{0.685524in}}%
\pgfpathlineto{\pgfqpoint{0.702741in}{0.685061in}}%
\pgfpathlineto{\pgfqpoint{0.718397in}{0.675714in}}%
\pgfpathlineto{\pgfqpoint{0.723002in}{0.671913in}}%
\pgfpathlineto{\pgfqpoint{0.734054in}{0.663540in}}%
\pgfpathlineto{\pgfqpoint{0.740130in}{0.658302in}}%
\pgfpathlineto{\pgfqpoint{0.749710in}{0.649973in}}%
\pgfpathlineto{\pgfqpoint{0.755736in}{0.644691in}}%
\pgfpathlineto{\pgfqpoint{0.765367in}{0.635083in}}%
\pgfpathlineto{\pgfqpoint{0.769740in}{0.631079in}}%
\pgfpathlineto{\pgfqpoint{0.780491in}{0.617468in}}%
\pgfpathlineto{\pgfqpoint{0.781024in}{0.615967in}}%
\pgfpathlineto{\pgfqpoint{0.786026in}{0.603857in}}%
\pgfpathlineto{\pgfqpoint{0.783578in}{0.590246in}}%
\pgfpathlineto{\pgfqpoint{0.781024in}{0.586820in}}%
\pgfpathlineto{\pgfqpoint{0.774604in}{0.576635in}}%
\pgfpathlineto{\pgfqpoint{0.765367in}{0.567238in}}%
\pgfpathlineto{\pgfqpoint{0.761542in}{0.563024in}}%
\pgfpathlineto{\pgfqpoint{0.749710in}{0.552258in}}%
\pgfpathlineto{\pgfqpoint{0.746549in}{0.549413in}}%
\pgfpathlineto{\pgfqpoint{0.734054in}{0.538747in}}%
\pgfpathlineto{\pgfqpoint{0.730130in}{0.535802in}}%
\pgfpathlineto{\pgfqpoint{0.718397in}{0.526588in}}%
\pgfpathlineto{\pgfqpoint{0.710696in}{0.522191in}}%
\pgfpathlineto{\pgfqpoint{0.702741in}{0.517011in}}%
\pgfpathlineto{\pgfqpoint{0.687084in}{0.512452in}}%
\pgfpathlineto{\pgfqpoint{0.671428in}{0.514441in}}%
\pgfpathlineto{\pgfqpoint{0.656342in}{0.522191in}}%
\pgfpathclose%
\pgfpathmoveto{\pgfqpoint{0.966661in}{0.522191in}}%
\pgfpathlineto{\pgfqpoint{0.953246in}{0.531200in}}%
\pgfpathlineto{\pgfqpoint{0.947805in}{0.535802in}}%
\pgfpathlineto{\pgfqpoint{0.937589in}{0.544023in}}%
\pgfpathlineto{\pgfqpoint{0.931380in}{0.549413in}}%
\pgfpathlineto{\pgfqpoint{0.921933in}{0.558051in}}%
\pgfpathlineto{\pgfqpoint{0.916322in}{0.563024in}}%
\pgfpathlineto{\pgfqpoint{0.906276in}{0.573825in}}%
\pgfpathlineto{\pgfqpoint{0.903325in}{0.576635in}}%
\pgfpathlineto{\pgfqpoint{0.894325in}{0.590246in}}%
\pgfpathlineto{\pgfqpoint{0.892097in}{0.603857in}}%
\pgfpathlineto{\pgfqpoint{0.897204in}{0.617468in}}%
\pgfpathlineto{\pgfqpoint{0.906276in}{0.628422in}}%
\pgfpathlineto{\pgfqpoint{0.908249in}{0.631079in}}%
\pgfpathlineto{\pgfqpoint{0.921933in}{0.644503in}}%
\pgfpathlineto{\pgfqpoint{0.922120in}{0.644691in}}%
\pgfpathlineto{\pgfqpoint{0.937520in}{0.658302in}}%
\pgfpathlineto{\pgfqpoint{0.937589in}{0.658363in}}%
\pgfpathlineto{\pgfqpoint{0.953246in}{0.671022in}}%
\pgfpathlineto{\pgfqpoint{0.954630in}{0.671913in}}%
\pgfpathlineto{\pgfqpoint{0.968902in}{0.681829in}}%
\pgfpathlineto{\pgfqpoint{0.977914in}{0.685524in}}%
\pgfpathlineto{\pgfqpoint{0.984559in}{0.688730in}}%
\pgfpathlineto{\pgfqpoint{1.000216in}{0.689442in}}%
\pgfpathlineto{\pgfqpoint{1.010047in}{0.685524in}}%
\pgfpathlineto{\pgfqpoint{1.015872in}{0.683541in}}%
\pgfpathlineto{\pgfqpoint{1.031529in}{0.673409in}}%
\pgfpathlineto{\pgfqpoint{1.033269in}{0.671913in}}%
\pgfpathlineto{\pgfqpoint{1.047185in}{0.660960in}}%
\pgfpathlineto{\pgfqpoint{1.050235in}{0.658302in}}%
\pgfpathlineto{\pgfqpoint{1.062842in}{0.647231in}}%
\pgfpathlineto{\pgfqpoint{1.065772in}{0.644691in}}%
\pgfpathlineto{\pgfqpoint{1.078498in}{0.632127in}}%
\pgfpathlineto{\pgfqpoint{1.079678in}{0.631079in}}%
\pgfpathlineto{\pgfqpoint{1.090616in}{0.617468in}}%
\pgfpathlineto{\pgfqpoint{1.094155in}{0.607777in}}%
\pgfpathlineto{\pgfqpoint{1.095863in}{0.603857in}}%
\pgfpathlineto{\pgfqpoint{1.094155in}{0.594772in}}%
\pgfpathlineto{\pgfqpoint{1.093434in}{0.590246in}}%
\pgfpathlineto{\pgfqpoint{1.084627in}{0.576635in}}%
\pgfpathlineto{\pgfqpoint{1.078498in}{0.570591in}}%
\pgfpathlineto{\pgfqpoint{1.071559in}{0.563024in}}%
\pgfpathlineto{\pgfqpoint{1.062842in}{0.555180in}}%
\pgfpathlineto{\pgfqpoint{1.056499in}{0.549413in}}%
\pgfpathlineto{\pgfqpoint{1.047185in}{0.541376in}}%
\pgfpathlineto{\pgfqpoint{1.040036in}{0.535802in}}%
\pgfpathlineto{\pgfqpoint{1.031529in}{0.528844in}}%
\pgfpathlineto{\pgfqpoint{1.020855in}{0.522191in}}%
\pgfpathlineto{\pgfqpoint{1.015872in}{0.518634in}}%
\pgfpathlineto{\pgfqpoint{1.000216in}{0.512855in}}%
\pgfpathlineto{\pgfqpoint{0.984559in}{0.513520in}}%
\pgfpathlineto{\pgfqpoint{0.968902in}{0.520460in}}%
\pgfpathlineto{\pgfqpoint{0.966661in}{0.522191in}}%
\pgfpathclose%
\pgfpathmoveto{\pgfqpoint{1.277051in}{0.522191in}}%
\pgfpathlineto{\pgfqpoint{1.266377in}{0.528844in}}%
\pgfpathlineto{\pgfqpoint{1.257869in}{0.535802in}}%
\pgfpathlineto{\pgfqpoint{1.250721in}{0.541376in}}%
\pgfpathlineto{\pgfqpoint{1.241407in}{0.549413in}}%
\pgfpathlineto{\pgfqpoint{1.235064in}{0.555180in}}%
\pgfpathlineto{\pgfqpoint{1.226347in}{0.563024in}}%
\pgfpathlineto{\pgfqpoint{1.219407in}{0.570591in}}%
\pgfpathlineto{\pgfqpoint{1.213279in}{0.576635in}}%
\pgfpathlineto{\pgfqpoint{1.204472in}{0.590246in}}%
\pgfpathlineto{\pgfqpoint{1.203751in}{0.594772in}}%
\pgfpathlineto{\pgfqpoint{1.202042in}{0.603857in}}%
\pgfpathlineto{\pgfqpoint{1.203751in}{0.607777in}}%
\pgfpathlineto{\pgfqpoint{1.207290in}{0.617468in}}%
\pgfpathlineto{\pgfqpoint{1.218228in}{0.631079in}}%
\pgfpathlineto{\pgfqpoint{1.219407in}{0.632127in}}%
\pgfpathlineto{\pgfqpoint{1.232134in}{0.644691in}}%
\pgfpathlineto{\pgfqpoint{1.235064in}{0.647231in}}%
\pgfpathlineto{\pgfqpoint{1.247671in}{0.658302in}}%
\pgfpathlineto{\pgfqpoint{1.250721in}{0.660960in}}%
\pgfpathlineto{\pgfqpoint{1.264637in}{0.671913in}}%
\pgfpathlineto{\pgfqpoint{1.266377in}{0.673409in}}%
\pgfpathlineto{\pgfqpoint{1.282034in}{0.683541in}}%
\pgfpathlineto{\pgfqpoint{1.287859in}{0.685524in}}%
\pgfpathlineto{\pgfqpoint{1.297690in}{0.689442in}}%
\pgfpathlineto{\pgfqpoint{1.313347in}{0.688730in}}%
\pgfpathlineto{\pgfqpoint{1.319992in}{0.685524in}}%
\pgfpathlineto{\pgfqpoint{1.329003in}{0.681829in}}%
\pgfpathlineto{\pgfqpoint{1.343276in}{0.671913in}}%
\pgfpathlineto{\pgfqpoint{1.344660in}{0.671022in}}%
\pgfpathlineto{\pgfqpoint{1.360317in}{0.658363in}}%
\pgfpathlineto{\pgfqpoint{1.360386in}{0.658302in}}%
\pgfpathlineto{\pgfqpoint{1.375786in}{0.644691in}}%
\pgfpathlineto{\pgfqpoint{1.375973in}{0.644503in}}%
\pgfpathlineto{\pgfqpoint{1.389657in}{0.631079in}}%
\pgfpathlineto{\pgfqpoint{1.391630in}{0.628422in}}%
\pgfpathlineto{\pgfqpoint{1.400701in}{0.617468in}}%
\pgfpathlineto{\pgfqpoint{1.405809in}{0.603857in}}%
\pgfpathlineto{\pgfqpoint{1.403581in}{0.590246in}}%
\pgfpathlineto{\pgfqpoint{1.394581in}{0.576635in}}%
\pgfpathlineto{\pgfqpoint{1.391630in}{0.573825in}}%
\pgfpathlineto{\pgfqpoint{1.381584in}{0.563024in}}%
\pgfpathlineto{\pgfqpoint{1.375973in}{0.558051in}}%
\pgfpathlineto{\pgfqpoint{1.366525in}{0.549413in}}%
\pgfpathlineto{\pgfqpoint{1.360317in}{0.544023in}}%
\pgfpathlineto{\pgfqpoint{1.350100in}{0.535802in}}%
\pgfpathlineto{\pgfqpoint{1.344660in}{0.531200in}}%
\pgfpathlineto{\pgfqpoint{1.331245in}{0.522191in}}%
\pgfpathlineto{\pgfqpoint{1.329003in}{0.520460in}}%
\pgfpathlineto{\pgfqpoint{1.313347in}{0.513520in}}%
\pgfpathlineto{\pgfqpoint{1.297690in}{0.512855in}}%
\pgfpathlineto{\pgfqpoint{1.282034in}{0.518634in}}%
\pgfpathlineto{\pgfqpoint{1.277051in}{0.522191in}}%
\pgfpathclose%
\pgfpathmoveto{\pgfqpoint{1.587210in}{0.522191in}}%
\pgfpathlineto{\pgfqpoint{1.579508in}{0.526588in}}%
\pgfpathlineto{\pgfqpoint{1.567776in}{0.535802in}}%
\pgfpathlineto{\pgfqpoint{1.563852in}{0.538747in}}%
\pgfpathlineto{\pgfqpoint{1.551356in}{0.549413in}}%
\pgfpathlineto{\pgfqpoint{1.548195in}{0.552258in}}%
\pgfpathlineto{\pgfqpoint{1.536364in}{0.563024in}}%
\pgfpathlineto{\pgfqpoint{1.532539in}{0.567238in}}%
\pgfpathlineto{\pgfqpoint{1.523302in}{0.576635in}}%
\pgfpathlineto{\pgfqpoint{1.516882in}{0.586820in}}%
\pgfpathlineto{\pgfqpoint{1.514328in}{0.590246in}}%
\pgfpathlineto{\pgfqpoint{1.511880in}{0.603857in}}%
\pgfpathlineto{\pgfqpoint{1.516882in}{0.615967in}}%
\pgfpathlineto{\pgfqpoint{1.517415in}{0.617468in}}%
\pgfpathlineto{\pgfqpoint{1.528166in}{0.631079in}}%
\pgfpathlineto{\pgfqpoint{1.532539in}{0.635083in}}%
\pgfpathlineto{\pgfqpoint{1.542170in}{0.644691in}}%
\pgfpathlineto{\pgfqpoint{1.548195in}{0.649973in}}%
\pgfpathlineto{\pgfqpoint{1.557776in}{0.658302in}}%
\pgfpathlineto{\pgfqpoint{1.563852in}{0.663540in}}%
\pgfpathlineto{\pgfqpoint{1.574903in}{0.671913in}}%
\pgfpathlineto{\pgfqpoint{1.579508in}{0.675714in}}%
\pgfpathlineto{\pgfqpoint{1.595165in}{0.685061in}}%
\pgfpathlineto{\pgfqpoint{1.596892in}{0.685524in}}%
\pgfpathlineto{\pgfqpoint{1.610822in}{0.689872in}}%
\pgfpathlineto{\pgfqpoint{1.626478in}{0.687745in}}%
\pgfpathlineto{\pgfqpoint{1.630419in}{0.685524in}}%
\pgfpathlineto{\pgfqpoint{1.642135in}{0.679943in}}%
\pgfpathlineto{\pgfqpoint{1.652943in}{0.671913in}}%
\pgfpathlineto{\pgfqpoint{1.657791in}{0.668587in}}%
\pgfpathlineto{\pgfqpoint{1.670175in}{0.658302in}}%
\pgfpathlineto{\pgfqpoint{1.673448in}{0.655554in}}%
\pgfpathlineto{\pgfqpoint{1.685716in}{0.644691in}}%
\pgfpathlineto{\pgfqpoint{1.689104in}{0.641280in}}%
\pgfpathlineto{\pgfqpoint{1.699703in}{0.631079in}}%
\pgfpathlineto{\pgfqpoint{1.704761in}{0.624384in}}%
\pgfpathlineto{\pgfqpoint{1.710719in}{0.617468in}}%
\pgfpathlineto{\pgfqpoint{1.715962in}{0.603857in}}%
\pgfpathlineto{\pgfqpoint{1.713675in}{0.590246in}}%
\pgfpathlineto{\pgfqpoint{1.704761in}{0.577132in}}%
\pgfpathlineto{\pgfqpoint{1.704462in}{0.576635in}}%
\pgfpathlineto{\pgfqpoint{1.691591in}{0.563024in}}%
\pgfpathlineto{\pgfqpoint{1.689104in}{0.560861in}}%
\pgfpathlineto{\pgfqpoint{1.676600in}{0.549413in}}%
\pgfpathlineto{\pgfqpoint{1.673448in}{0.546672in}}%
\pgfpathlineto{\pgfqpoint{1.660279in}{0.535802in}}%
\pgfpathlineto{\pgfqpoint{1.657791in}{0.533640in}}%
\pgfpathlineto{\pgfqpoint{1.642135in}{0.522450in}}%
\pgfpathlineto{\pgfqpoint{1.641563in}{0.522191in}}%
\pgfpathlineto{\pgfqpoint{1.626478in}{0.514441in}}%
\pgfpathlineto{\pgfqpoint{1.610822in}{0.512452in}}%
\pgfpathlineto{\pgfqpoint{1.595165in}{0.517011in}}%
\pgfpathlineto{\pgfqpoint{1.587210in}{0.522191in}}%
\pgfpathclose%
\pgfpathmoveto{\pgfqpoint{0.512271in}{0.631079in}}%
\pgfpathlineto{\pgfqpoint{0.499205in}{0.635642in}}%
\pgfpathlineto{\pgfqpoint{0.484025in}{0.644691in}}%
\pgfpathlineto{\pgfqpoint{0.483549in}{0.644933in}}%
\pgfpathlineto{\pgfqpoint{0.467892in}{0.655695in}}%
\pgfpathlineto{\pgfqpoint{0.464721in}{0.658302in}}%
\pgfpathlineto{\pgfqpoint{0.452236in}{0.668439in}}%
\pgfpathlineto{\pgfqpoint{0.448266in}{0.671913in}}%
\pgfpathlineto{\pgfqpoint{0.436579in}{0.683168in}}%
\pgfpathlineto{\pgfqpoint{0.434043in}{0.685524in}}%
\pgfpathlineto{\pgfqpoint{0.422017in}{0.699135in}}%
\pgfpathlineto{\pgfqpoint{0.420923in}{0.700870in}}%
\pgfpathlineto{\pgfqpoint{0.411849in}{0.712746in}}%
\pgfpathlineto{\pgfqpoint{0.406157in}{0.726357in}}%
\pgfpathlineto{\pgfqpoint{0.405266in}{0.739161in}}%
\pgfpathlineto{\pgfqpoint{0.405179in}{0.739968in}}%
\pgfpathlineto{\pgfqpoint{0.405266in}{0.740172in}}%
\pgfpathlineto{\pgfqpoint{0.409001in}{0.753579in}}%
\pgfpathlineto{\pgfqpoint{0.417550in}{0.767191in}}%
\pgfpathlineto{\pgfqpoint{0.420923in}{0.770772in}}%
\pgfpathlineto{\pgfqpoint{0.428796in}{0.780802in}}%
\pgfpathlineto{\pgfqpoint{0.436579in}{0.788494in}}%
\pgfpathlineto{\pgfqpoint{0.442332in}{0.794413in}}%
\pgfpathlineto{\pgfqpoint{0.452236in}{0.803294in}}%
\pgfpathlineto{\pgfqpoint{0.457910in}{0.808024in}}%
\pgfpathlineto{\pgfqpoint{0.467892in}{0.816102in}}%
\pgfpathlineto{\pgfqpoint{0.476089in}{0.821635in}}%
\pgfpathlineto{\pgfqpoint{0.483549in}{0.827153in}}%
\pgfpathlineto{\pgfqpoint{0.498466in}{0.835246in}}%
\pgfpathlineto{\pgfqpoint{0.499205in}{0.835765in}}%
\pgfpathlineto{\pgfqpoint{0.514862in}{0.842427in}}%
\pgfpathlineto{\pgfqpoint{0.530519in}{0.844326in}}%
\pgfpathlineto{\pgfqpoint{0.546175in}{0.841476in}}%
\pgfpathlineto{\pgfqpoint{0.559047in}{0.835246in}}%
\pgfpathlineto{\pgfqpoint{0.561832in}{0.834218in}}%
\pgfpathlineto{\pgfqpoint{0.577488in}{0.825037in}}%
\pgfpathlineto{\pgfqpoint{0.581901in}{0.821635in}}%
\pgfpathlineto{\pgfqpoint{0.593145in}{0.813705in}}%
\pgfpathlineto{\pgfqpoint{0.600016in}{0.808024in}}%
\pgfpathlineto{\pgfqpoint{0.608801in}{0.800530in}}%
\pgfpathlineto{\pgfqpoint{0.615605in}{0.794413in}}%
\pgfpathlineto{\pgfqpoint{0.624458in}{0.785279in}}%
\pgfpathlineto{\pgfqpoint{0.629079in}{0.780802in}}%
\pgfpathlineto{\pgfqpoint{0.640000in}{0.767191in}}%
\pgfpathlineto{\pgfqpoint{0.640115in}{0.766976in}}%
\pgfpathlineto{\pgfqpoint{0.649105in}{0.753579in}}%
\pgfpathlineto{\pgfqpoint{0.653158in}{0.739968in}}%
\pgfpathlineto{\pgfqpoint{0.652146in}{0.726357in}}%
\pgfpathlineto{\pgfqpoint{0.646060in}{0.712746in}}%
\pgfpathlineto{\pgfqpoint{0.640115in}{0.705372in}}%
\pgfpathlineto{\pgfqpoint{0.636095in}{0.699135in}}%
\pgfpathlineto{\pgfqpoint{0.624458in}{0.686365in}}%
\pgfpathlineto{\pgfqpoint{0.623743in}{0.685524in}}%
\pgfpathlineto{\pgfqpoint{0.609602in}{0.671913in}}%
\pgfpathlineto{\pgfqpoint{0.608801in}{0.671210in}}%
\pgfpathlineto{\pgfqpoint{0.593264in}{0.658302in}}%
\pgfpathlineto{\pgfqpoint{0.593145in}{0.658201in}}%
\pgfpathlineto{\pgfqpoint{0.577488in}{0.646841in}}%
\pgfpathlineto{\pgfqpoint{0.573556in}{0.644691in}}%
\pgfpathlineto{\pgfqpoint{0.561832in}{0.637200in}}%
\pgfpathlineto{\pgfqpoint{0.546447in}{0.631079in}}%
\pgfpathlineto{\pgfqpoint{0.546175in}{0.630925in}}%
\pgfpathlineto{\pgfqpoint{0.530519in}{0.627493in}}%
\pgfpathlineto{\pgfqpoint{0.514862in}{0.629780in}}%
\pgfpathlineto{\pgfqpoint{0.512271in}{0.631079in}}%
\pgfpathclose%
\pgfpathmoveto{\pgfqpoint{0.822873in}{0.631079in}}%
\pgfpathlineto{\pgfqpoint{0.812337in}{0.634241in}}%
\pgfpathlineto{\pgfqpoint{0.796680in}{0.642808in}}%
\pgfpathlineto{\pgfqpoint{0.794194in}{0.644691in}}%
\pgfpathlineto{\pgfqpoint{0.781024in}{0.653303in}}%
\pgfpathlineto{\pgfqpoint{0.774787in}{0.658302in}}%
\pgfpathlineto{\pgfqpoint{0.765367in}{0.665747in}}%
\pgfpathlineto{\pgfqpoint{0.758275in}{0.671913in}}%
\pgfpathlineto{\pgfqpoint{0.749710in}{0.680102in}}%
\pgfpathlineto{\pgfqpoint{0.743960in}{0.685524in}}%
\pgfpathlineto{\pgfqpoint{0.734054in}{0.696973in}}%
\pgfpathlineto{\pgfqpoint{0.731888in}{0.699135in}}%
\pgfpathlineto{\pgfqpoint{0.722034in}{0.712746in}}%
\pgfpathlineto{\pgfqpoint{0.718397in}{0.721905in}}%
\pgfpathlineto{\pgfqpoint{0.715850in}{0.726357in}}%
\pgfpathlineto{\pgfqpoint{0.714535in}{0.739968in}}%
\pgfpathlineto{\pgfqpoint{0.718397in}{0.749993in}}%
\pgfpathlineto{\pgfqpoint{0.719350in}{0.753579in}}%
\pgfpathlineto{\pgfqpoint{0.727409in}{0.767191in}}%
\pgfpathlineto{\pgfqpoint{0.734054in}{0.774575in}}%
\pgfpathlineto{\pgfqpoint{0.738865in}{0.780802in}}%
\pgfpathlineto{\pgfqpoint{0.749710in}{0.791681in}}%
\pgfpathlineto{\pgfqpoint{0.752384in}{0.794413in}}%
\pgfpathlineto{\pgfqpoint{0.765367in}{0.805979in}}%
\pgfpathlineto{\pgfqpoint{0.767887in}{0.808024in}}%
\pgfpathlineto{\pgfqpoint{0.781024in}{0.818390in}}%
\pgfpathlineto{\pgfqpoint{0.786071in}{0.821635in}}%
\pgfpathlineto{\pgfqpoint{0.796680in}{0.829130in}}%
\pgfpathlineto{\pgfqpoint{0.808898in}{0.835246in}}%
\pgfpathlineto{\pgfqpoint{0.812337in}{0.837479in}}%
\pgfpathlineto{\pgfqpoint{0.827993in}{0.843187in}}%
\pgfpathlineto{\pgfqpoint{0.843650in}{0.844136in}}%
\pgfpathlineto{\pgfqpoint{0.859306in}{0.840335in}}%
\pgfpathlineto{\pgfqpoint{0.868722in}{0.835246in}}%
\pgfpathlineto{\pgfqpoint{0.874963in}{0.832663in}}%
\pgfpathlineto{\pgfqpoint{0.890620in}{0.822784in}}%
\pgfpathlineto{\pgfqpoint{0.892056in}{0.821635in}}%
\pgfpathlineto{\pgfqpoint{0.906276in}{0.811203in}}%
\pgfpathlineto{\pgfqpoint{0.910053in}{0.808024in}}%
\pgfpathlineto{\pgfqpoint{0.921933in}{0.797696in}}%
\pgfpathlineto{\pgfqpoint{0.925589in}{0.794413in}}%
\pgfpathlineto{\pgfqpoint{0.937589in}{0.782050in}}%
\pgfpathlineto{\pgfqpoint{0.938911in}{0.780802in}}%
\pgfpathlineto{\pgfqpoint{0.950275in}{0.767191in}}%
\pgfpathlineto{\pgfqpoint{0.953246in}{0.761765in}}%
\pgfpathlineto{\pgfqpoint{0.959099in}{0.753579in}}%
\pgfpathlineto{\pgfqpoint{0.963472in}{0.739968in}}%
\pgfpathlineto{\pgfqpoint{0.962380in}{0.726357in}}%
\pgfpathlineto{\pgfqpoint{0.955814in}{0.712746in}}%
\pgfpathlineto{\pgfqpoint{0.953246in}{0.709756in}}%
\pgfpathlineto{\pgfqpoint{0.946211in}{0.699135in}}%
\pgfpathlineto{\pgfqpoint{0.937589in}{0.689911in}}%
\pgfpathlineto{\pgfqpoint{0.933857in}{0.685524in}}%
\pgfpathlineto{\pgfqpoint{0.921933in}{0.674104in}}%
\pgfpathlineto{\pgfqpoint{0.919581in}{0.671913in}}%
\pgfpathlineto{\pgfqpoint{0.906276in}{0.660626in}}%
\pgfpathlineto{\pgfqpoint{0.903134in}{0.658302in}}%
\pgfpathlineto{\pgfqpoint{0.890620in}{0.648873in}}%
\pgfpathlineto{\pgfqpoint{0.883457in}{0.644691in}}%
\pgfpathlineto{\pgfqpoint{0.874963in}{0.638913in}}%
\pgfpathlineto{\pgfqpoint{0.859306in}{0.631907in}}%
\pgfpathlineto{\pgfqpoint{0.855181in}{0.631079in}}%
\pgfpathlineto{\pgfqpoint{0.843650in}{0.627722in}}%
\pgfpathlineto{\pgfqpoint{0.827993in}{0.628865in}}%
\pgfpathlineto{\pgfqpoint{0.822873in}{0.631079in}}%
\pgfpathclose%
\pgfpathmoveto{\pgfqpoint{1.133119in}{0.631079in}}%
\pgfpathlineto{\pgfqpoint{1.125468in}{0.632996in}}%
\pgfpathlineto{\pgfqpoint{1.109812in}{0.640783in}}%
\pgfpathlineto{\pgfqpoint{1.104382in}{0.644691in}}%
\pgfpathlineto{\pgfqpoint{1.094155in}{0.651028in}}%
\pgfpathlineto{\pgfqpoint{1.084813in}{0.658302in}}%
\pgfpathlineto{\pgfqpoint{1.078498in}{0.663140in}}%
\pgfpathlineto{\pgfqpoint{1.068303in}{0.671913in}}%
\pgfpathlineto{\pgfqpoint{1.062842in}{0.677077in}}%
\pgfpathlineto{\pgfqpoint{1.053973in}{0.685524in}}%
\pgfpathlineto{\pgfqpoint{1.047185in}{0.693454in}}%
\pgfpathlineto{\pgfqpoint{1.041705in}{0.699135in}}%
\pgfpathlineto{\pgfqpoint{1.032349in}{0.712746in}}%
\pgfpathlineto{\pgfqpoint{1.031529in}{0.714899in}}%
\pgfpathlineto{\pgfqpoint{1.025524in}{0.726357in}}%
\pgfpathlineto{\pgfqpoint{1.024332in}{0.739968in}}%
\pgfpathlineto{\pgfqpoint{1.029101in}{0.753579in}}%
\pgfpathlineto{\pgfqpoint{1.031529in}{0.756731in}}%
\pgfpathlineto{\pgfqpoint{1.037452in}{0.767191in}}%
\pgfpathlineto{\pgfqpoint{1.047185in}{0.778426in}}%
\pgfpathlineto{\pgfqpoint{1.049001in}{0.780802in}}%
\pgfpathlineto{\pgfqpoint{1.062408in}{0.794413in}}%
\pgfpathlineto{\pgfqpoint{1.062842in}{0.794801in}}%
\pgfpathlineto{\pgfqpoint{1.077824in}{0.808024in}}%
\pgfpathlineto{\pgfqpoint{1.078498in}{0.808600in}}%
\pgfpathlineto{\pgfqpoint{1.094155in}{0.820565in}}%
\pgfpathlineto{\pgfqpoint{1.095910in}{0.821635in}}%
\pgfpathlineto{\pgfqpoint{1.109812in}{0.830967in}}%
\pgfpathlineto{\pgfqpoint{1.119161in}{0.835246in}}%
\pgfpathlineto{\pgfqpoint{1.125468in}{0.839002in}}%
\pgfpathlineto{\pgfqpoint{1.141125in}{0.843757in}}%
\pgfpathlineto{\pgfqpoint{1.156781in}{0.843757in}}%
\pgfpathlineto{\pgfqpoint{1.172438in}{0.839002in}}%
\pgfpathlineto{\pgfqpoint{1.178745in}{0.835246in}}%
\pgfpathlineto{\pgfqpoint{1.188094in}{0.830967in}}%
\pgfpathlineto{\pgfqpoint{1.201996in}{0.821635in}}%
\pgfpathlineto{\pgfqpoint{1.203751in}{0.820565in}}%
\pgfpathlineto{\pgfqpoint{1.219407in}{0.808600in}}%
\pgfpathlineto{\pgfqpoint{1.220082in}{0.808024in}}%
\pgfpathlineto{\pgfqpoint{1.235064in}{0.794801in}}%
\pgfpathlineto{\pgfqpoint{1.235498in}{0.794413in}}%
\pgfpathlineto{\pgfqpoint{1.248905in}{0.780802in}}%
\pgfpathlineto{\pgfqpoint{1.250721in}{0.778426in}}%
\pgfpathlineto{\pgfqpoint{1.260453in}{0.767191in}}%
\pgfpathlineto{\pgfqpoint{1.266377in}{0.756731in}}%
\pgfpathlineto{\pgfqpoint{1.268805in}{0.753579in}}%
\pgfpathlineto{\pgfqpoint{1.273574in}{0.739968in}}%
\pgfpathlineto{\pgfqpoint{1.272382in}{0.726357in}}%
\pgfpathlineto{\pgfqpoint{1.266377in}{0.714899in}}%
\pgfpathlineto{\pgfqpoint{1.265556in}{0.712746in}}%
\pgfpathlineto{\pgfqpoint{1.256201in}{0.699135in}}%
\pgfpathlineto{\pgfqpoint{1.250721in}{0.693454in}}%
\pgfpathlineto{\pgfqpoint{1.243933in}{0.685524in}}%
\pgfpathlineto{\pgfqpoint{1.235064in}{0.677077in}}%
\pgfpathlineto{\pgfqpoint{1.229603in}{0.671913in}}%
\pgfpathlineto{\pgfqpoint{1.219407in}{0.663140in}}%
\pgfpathlineto{\pgfqpoint{1.213093in}{0.658302in}}%
\pgfpathlineto{\pgfqpoint{1.203751in}{0.651028in}}%
\pgfpathlineto{\pgfqpoint{1.193524in}{0.644691in}}%
\pgfpathlineto{\pgfqpoint{1.188094in}{0.640783in}}%
\pgfpathlineto{\pgfqpoint{1.172438in}{0.632996in}}%
\pgfpathlineto{\pgfqpoint{1.164787in}{0.631079in}}%
\pgfpathlineto{\pgfqpoint{1.156781in}{0.628179in}}%
\pgfpathlineto{\pgfqpoint{1.141125in}{0.628179in}}%
\pgfpathlineto{\pgfqpoint{1.133119in}{0.631079in}}%
\pgfpathclose%
\pgfpathmoveto{\pgfqpoint{1.442725in}{0.631079in}}%
\pgfpathlineto{\pgfqpoint{1.438599in}{0.631907in}}%
\pgfpathlineto{\pgfqpoint{1.422943in}{0.638913in}}%
\pgfpathlineto{\pgfqpoint{1.414449in}{0.644691in}}%
\pgfpathlineto{\pgfqpoint{1.407286in}{0.648873in}}%
\pgfpathlineto{\pgfqpoint{1.394772in}{0.658302in}}%
\pgfpathlineto{\pgfqpoint{1.391630in}{0.660626in}}%
\pgfpathlineto{\pgfqpoint{1.378325in}{0.671913in}}%
\pgfpathlineto{\pgfqpoint{1.375973in}{0.674104in}}%
\pgfpathlineto{\pgfqpoint{1.364049in}{0.685524in}}%
\pgfpathlineto{\pgfqpoint{1.360317in}{0.689911in}}%
\pgfpathlineto{\pgfqpoint{1.351695in}{0.699135in}}%
\pgfpathlineto{\pgfqpoint{1.344660in}{0.709756in}}%
\pgfpathlineto{\pgfqpoint{1.342091in}{0.712746in}}%
\pgfpathlineto{\pgfqpoint{1.335526in}{0.726357in}}%
\pgfpathlineto{\pgfqpoint{1.334434in}{0.739968in}}%
\pgfpathlineto{\pgfqpoint{1.338807in}{0.753579in}}%
\pgfpathlineto{\pgfqpoint{1.344660in}{0.761765in}}%
\pgfpathlineto{\pgfqpoint{1.347631in}{0.767191in}}%
\pgfpathlineto{\pgfqpoint{1.358995in}{0.780802in}}%
\pgfpathlineto{\pgfqpoint{1.360317in}{0.782050in}}%
\pgfpathlineto{\pgfqpoint{1.372317in}{0.794413in}}%
\pgfpathlineto{\pgfqpoint{1.375973in}{0.797696in}}%
\pgfpathlineto{\pgfqpoint{1.387853in}{0.808024in}}%
\pgfpathlineto{\pgfqpoint{1.391630in}{0.811203in}}%
\pgfpathlineto{\pgfqpoint{1.405850in}{0.821635in}}%
\pgfpathlineto{\pgfqpoint{1.407286in}{0.822784in}}%
\pgfpathlineto{\pgfqpoint{1.422943in}{0.832663in}}%
\pgfpathlineto{\pgfqpoint{1.429184in}{0.835246in}}%
\pgfpathlineto{\pgfqpoint{1.438599in}{0.840335in}}%
\pgfpathlineto{\pgfqpoint{1.454256in}{0.844136in}}%
\pgfpathlineto{\pgfqpoint{1.469913in}{0.843187in}}%
\pgfpathlineto{\pgfqpoint{1.485569in}{0.837479in}}%
\pgfpathlineto{\pgfqpoint{1.489008in}{0.835246in}}%
\pgfpathlineto{\pgfqpoint{1.501226in}{0.829130in}}%
\pgfpathlineto{\pgfqpoint{1.511835in}{0.821635in}}%
\pgfpathlineto{\pgfqpoint{1.516882in}{0.818390in}}%
\pgfpathlineto{\pgfqpoint{1.530019in}{0.808024in}}%
\pgfpathlineto{\pgfqpoint{1.532539in}{0.805979in}}%
\pgfpathlineto{\pgfqpoint{1.545522in}{0.794413in}}%
\pgfpathlineto{\pgfqpoint{1.548195in}{0.791681in}}%
\pgfpathlineto{\pgfqpoint{1.559041in}{0.780802in}}%
\pgfpathlineto{\pgfqpoint{1.563852in}{0.774575in}}%
\pgfpathlineto{\pgfqpoint{1.570497in}{0.767191in}}%
\pgfpathlineto{\pgfqpoint{1.578556in}{0.753579in}}%
\pgfpathlineto{\pgfqpoint{1.579508in}{0.749993in}}%
\pgfpathlineto{\pgfqpoint{1.583371in}{0.739968in}}%
\pgfpathlineto{\pgfqpoint{1.582056in}{0.726357in}}%
\pgfpathlineto{\pgfqpoint{1.579508in}{0.721905in}}%
\pgfpathlineto{\pgfqpoint{1.575872in}{0.712746in}}%
\pgfpathlineto{\pgfqpoint{1.566018in}{0.699135in}}%
\pgfpathlineto{\pgfqpoint{1.563852in}{0.696973in}}%
\pgfpathlineto{\pgfqpoint{1.553946in}{0.685524in}}%
\pgfpathlineto{\pgfqpoint{1.548195in}{0.680102in}}%
\pgfpathlineto{\pgfqpoint{1.539631in}{0.671913in}}%
\pgfpathlineto{\pgfqpoint{1.532539in}{0.665747in}}%
\pgfpathlineto{\pgfqpoint{1.523119in}{0.658302in}}%
\pgfpathlineto{\pgfqpoint{1.516882in}{0.653303in}}%
\pgfpathlineto{\pgfqpoint{1.503712in}{0.644691in}}%
\pgfpathlineto{\pgfqpoint{1.501226in}{0.642808in}}%
\pgfpathlineto{\pgfqpoint{1.485569in}{0.634241in}}%
\pgfpathlineto{\pgfqpoint{1.475033in}{0.631079in}}%
\pgfpathlineto{\pgfqpoint{1.469913in}{0.628865in}}%
\pgfpathlineto{\pgfqpoint{1.454256in}{0.627722in}}%
\pgfpathlineto{\pgfqpoint{1.442725in}{0.631079in}}%
\pgfpathclose%
\pgfpathmoveto{\pgfqpoint{1.751459in}{0.631079in}}%
\pgfpathlineto{\pgfqpoint{1.736074in}{0.637200in}}%
\pgfpathlineto{\pgfqpoint{1.724349in}{0.644691in}}%
\pgfpathlineto{\pgfqpoint{1.720418in}{0.646841in}}%
\pgfpathlineto{\pgfqpoint{1.704761in}{0.658201in}}%
\pgfpathlineto{\pgfqpoint{1.704642in}{0.658302in}}%
\pgfpathlineto{\pgfqpoint{1.689104in}{0.671210in}}%
\pgfpathlineto{\pgfqpoint{1.688304in}{0.671913in}}%
\pgfpathlineto{\pgfqpoint{1.674163in}{0.685524in}}%
\pgfpathlineto{\pgfqpoint{1.673448in}{0.686365in}}%
\pgfpathlineto{\pgfqpoint{1.661811in}{0.699135in}}%
\pgfpathlineto{\pgfqpoint{1.657791in}{0.705372in}}%
\pgfpathlineto{\pgfqpoint{1.651845in}{0.712746in}}%
\pgfpathlineto{\pgfqpoint{1.645760in}{0.726357in}}%
\pgfpathlineto{\pgfqpoint{1.644747in}{0.739968in}}%
\pgfpathlineto{\pgfqpoint{1.648801in}{0.753579in}}%
\pgfpathlineto{\pgfqpoint{1.657791in}{0.766976in}}%
\pgfpathlineto{\pgfqpoint{1.657906in}{0.767191in}}%
\pgfpathlineto{\pgfqpoint{1.668827in}{0.780802in}}%
\pgfpathlineto{\pgfqpoint{1.673448in}{0.785279in}}%
\pgfpathlineto{\pgfqpoint{1.682301in}{0.794413in}}%
\pgfpathlineto{\pgfqpoint{1.689104in}{0.800530in}}%
\pgfpathlineto{\pgfqpoint{1.697890in}{0.808024in}}%
\pgfpathlineto{\pgfqpoint{1.704761in}{0.813705in}}%
\pgfpathlineto{\pgfqpoint{1.716004in}{0.821635in}}%
\pgfpathlineto{\pgfqpoint{1.720418in}{0.825037in}}%
\pgfpathlineto{\pgfqpoint{1.736074in}{0.834218in}}%
\pgfpathlineto{\pgfqpoint{1.738858in}{0.835246in}}%
\pgfpathlineto{\pgfqpoint{1.751731in}{0.841476in}}%
\pgfpathlineto{\pgfqpoint{1.767387in}{0.844326in}}%
\pgfpathlineto{\pgfqpoint{1.783044in}{0.842427in}}%
\pgfpathlineto{\pgfqpoint{1.798700in}{0.835765in}}%
\pgfpathlineto{\pgfqpoint{1.799440in}{0.835246in}}%
\pgfpathlineto{\pgfqpoint{1.814357in}{0.827153in}}%
\pgfpathlineto{\pgfqpoint{1.821817in}{0.821635in}}%
\pgfpathlineto{\pgfqpoint{1.830014in}{0.816102in}}%
\pgfpathlineto{\pgfqpoint{1.839996in}{0.808024in}}%
\pgfpathlineto{\pgfqpoint{1.845670in}{0.803294in}}%
\pgfpathlineto{\pgfqpoint{1.855574in}{0.794413in}}%
\pgfpathlineto{\pgfqpoint{1.861327in}{0.788494in}}%
\pgfpathlineto{\pgfqpoint{1.869109in}{0.780802in}}%
\pgfpathlineto{\pgfqpoint{1.876983in}{0.770772in}}%
\pgfpathlineto{\pgfqpoint{1.880356in}{0.767191in}}%
\pgfpathlineto{\pgfqpoint{1.888904in}{0.753579in}}%
\pgfpathlineto{\pgfqpoint{1.892640in}{0.740172in}}%
\pgfpathlineto{\pgfqpoint{1.892727in}{0.739968in}}%
\pgfpathlineto{\pgfqpoint{1.892640in}{0.739161in}}%
\pgfpathlineto{\pgfqpoint{1.891749in}{0.726357in}}%
\pgfpathlineto{\pgfqpoint{1.886057in}{0.712746in}}%
\pgfpathlineto{\pgfqpoint{1.876983in}{0.700870in}}%
\pgfpathlineto{\pgfqpoint{1.875889in}{0.699135in}}%
\pgfpathlineto{\pgfqpoint{1.863863in}{0.685524in}}%
\pgfpathlineto{\pgfqpoint{1.861327in}{0.683168in}}%
\pgfpathlineto{\pgfqpoint{1.849639in}{0.671913in}}%
\pgfpathlineto{\pgfqpoint{1.845670in}{0.668439in}}%
\pgfpathlineto{\pgfqpoint{1.833185in}{0.658302in}}%
\pgfpathlineto{\pgfqpoint{1.830014in}{0.655695in}}%
\pgfpathlineto{\pgfqpoint{1.814357in}{0.644933in}}%
\pgfpathlineto{\pgfqpoint{1.813881in}{0.644691in}}%
\pgfpathlineto{\pgfqpoint{1.798700in}{0.635642in}}%
\pgfpathlineto{\pgfqpoint{1.785635in}{0.631079in}}%
\pgfpathlineto{\pgfqpoint{1.783044in}{0.629780in}}%
\pgfpathlineto{\pgfqpoint{1.767387in}{0.627493in}}%
\pgfpathlineto{\pgfqpoint{1.751731in}{0.630925in}}%
\pgfpathlineto{\pgfqpoint{1.751459in}{0.631079in}}%
\pgfpathclose%
\pgfpathmoveto{\pgfqpoint{0.652538in}{0.794413in}}%
\pgfpathlineto{\pgfqpoint{0.640115in}{0.803146in}}%
\pgfpathlineto{\pgfqpoint{0.634394in}{0.808024in}}%
\pgfpathlineto{\pgfqpoint{0.624458in}{0.816237in}}%
\pgfpathlineto{\pgfqpoint{0.618258in}{0.821635in}}%
\pgfpathlineto{\pgfqpoint{0.608801in}{0.830517in}}%
\pgfpathlineto{\pgfqpoint{0.603509in}{0.835246in}}%
\pgfpathlineto{\pgfqpoint{0.593145in}{0.846909in}}%
\pgfpathlineto{\pgfqpoint{0.591155in}{0.848857in}}%
\pgfpathlineto{\pgfqpoint{0.583171in}{0.862468in}}%
\pgfpathlineto{\pgfqpoint{0.582406in}{0.876079in}}%
\pgfpathlineto{\pgfqpoint{0.589053in}{0.889691in}}%
\pgfpathlineto{\pgfqpoint{0.593145in}{0.894023in}}%
\pgfpathlineto{\pgfqpoint{0.600798in}{0.903302in}}%
\pgfpathlineto{\pgfqpoint{0.608801in}{0.910698in}}%
\pgfpathlineto{\pgfqpoint{0.615214in}{0.916913in}}%
\pgfpathlineto{\pgfqpoint{0.624458in}{0.925010in}}%
\pgfpathlineto{\pgfqpoint{0.631092in}{0.930524in}}%
\pgfpathlineto{\pgfqpoint{0.640115in}{0.938102in}}%
\pgfpathlineto{\pgfqpoint{0.648819in}{0.944135in}}%
\pgfpathlineto{\pgfqpoint{0.655771in}{0.949463in}}%
\pgfpathlineto{\pgfqpoint{0.671428in}{0.957119in}}%
\pgfpathlineto{\pgfqpoint{0.676633in}{0.957746in}}%
\pgfpathlineto{\pgfqpoint{0.687084in}{0.959231in}}%
\pgfpathlineto{\pgfqpoint{0.691593in}{0.957746in}}%
\pgfpathlineto{\pgfqpoint{0.702741in}{0.954670in}}%
\pgfpathlineto{\pgfqpoint{0.718397in}{0.945161in}}%
\pgfpathlineto{\pgfqpoint{0.719602in}{0.944135in}}%
\pgfpathlineto{\pgfqpoint{0.734054in}{0.933071in}}%
\pgfpathlineto{\pgfqpoint{0.736976in}{0.930524in}}%
\pgfpathlineto{\pgfqpoint{0.749710in}{0.919564in}}%
\pgfpathlineto{\pgfqpoint{0.752768in}{0.916913in}}%
\pgfpathlineto{\pgfqpoint{0.765367in}{0.904814in}}%
\pgfpathlineto{\pgfqpoint{0.767088in}{0.903302in}}%
\pgfpathlineto{\pgfqpoint{0.778742in}{0.889691in}}%
\pgfpathlineto{\pgfqpoint{0.781024in}{0.884626in}}%
\pgfpathlineto{\pgfqpoint{0.785531in}{0.876079in}}%
\pgfpathlineto{\pgfqpoint{0.784712in}{0.862468in}}%
\pgfpathlineto{\pgfqpoint{0.781024in}{0.856691in}}%
\pgfpathlineto{\pgfqpoint{0.776774in}{0.848857in}}%
\pgfpathlineto{\pgfqpoint{0.765367in}{0.836449in}}%
\pgfpathlineto{\pgfqpoint{0.764342in}{0.835246in}}%
\pgfpathlineto{\pgfqpoint{0.749781in}{0.821635in}}%
\pgfpathlineto{\pgfqpoint{0.749710in}{0.821574in}}%
\pgfpathlineto{\pgfqpoint{0.734054in}{0.808186in}}%
\pgfpathlineto{\pgfqpoint{0.733838in}{0.808024in}}%
\pgfpathlineto{\pgfqpoint{0.718397in}{0.796127in}}%
\pgfpathlineto{\pgfqpoint{0.715340in}{0.794413in}}%
\pgfpathlineto{\pgfqpoint{0.702741in}{0.786526in}}%
\pgfpathlineto{\pgfqpoint{0.687084in}{0.782086in}}%
\pgfpathlineto{\pgfqpoint{0.671428in}{0.784023in}}%
\pgfpathlineto{\pgfqpoint{0.655771in}{0.791847in}}%
\pgfpathlineto{\pgfqpoint{0.652538in}{0.794413in}}%
\pgfpathclose%
\pgfpathmoveto{\pgfqpoint{0.962803in}{0.794413in}}%
\pgfpathlineto{\pgfqpoint{0.953246in}{0.800718in}}%
\pgfpathlineto{\pgfqpoint{0.944442in}{0.808024in}}%
\pgfpathlineto{\pgfqpoint{0.937589in}{0.813546in}}%
\pgfpathlineto{\pgfqpoint{0.928285in}{0.821635in}}%
\pgfpathlineto{\pgfqpoint{0.921933in}{0.827592in}}%
\pgfpathlineto{\pgfqpoint{0.913529in}{0.835246in}}%
\pgfpathlineto{\pgfqpoint{0.906276in}{0.843555in}}%
\pgfpathlineto{\pgfqpoint{0.901069in}{0.848857in}}%
\pgfpathlineto{\pgfqpoint{0.893292in}{0.862468in}}%
\pgfpathlineto{\pgfqpoint{0.892547in}{0.876079in}}%
\pgfpathlineto{\pgfqpoint{0.899022in}{0.889691in}}%
\pgfpathlineto{\pgfqpoint{0.906276in}{0.897680in}}%
\pgfpathlineto{\pgfqpoint{0.910831in}{0.903302in}}%
\pgfpathlineto{\pgfqpoint{0.921933in}{0.913758in}}%
\pgfpathlineto{\pgfqpoint{0.925192in}{0.916913in}}%
\pgfpathlineto{\pgfqpoint{0.937589in}{0.927756in}}%
\pgfpathlineto{\pgfqpoint{0.941005in}{0.930524in}}%
\pgfpathlineto{\pgfqpoint{0.953246in}{0.940528in}}%
\pgfpathlineto{\pgfqpoint{0.958790in}{0.944135in}}%
\pgfpathlineto{\pgfqpoint{0.968902in}{0.951382in}}%
\pgfpathlineto{\pgfqpoint{0.983981in}{0.957746in}}%
\pgfpathlineto{\pgfqpoint{0.984559in}{0.958041in}}%
\pgfpathlineto{\pgfqpoint{1.000216in}{0.958783in}}%
\pgfpathlineto{\pgfqpoint{1.002680in}{0.957746in}}%
\pgfpathlineto{\pgfqpoint{1.015872in}{0.953123in}}%
\pgfpathlineto{\pgfqpoint{1.029438in}{0.944135in}}%
\pgfpathlineto{\pgfqpoint{1.031529in}{0.942872in}}%
\pgfpathlineto{\pgfqpoint{1.047153in}{0.930524in}}%
\pgfpathlineto{\pgfqpoint{1.047185in}{0.930499in}}%
\pgfpathlineto{\pgfqpoint{1.062813in}{0.916913in}}%
\pgfpathlineto{\pgfqpoint{1.062842in}{0.916885in}}%
\pgfpathlineto{\pgfqpoint{1.077046in}{0.903302in}}%
\pgfpathlineto{\pgfqpoint{1.078498in}{0.901484in}}%
\pgfpathlineto{\pgfqpoint{1.088837in}{0.889691in}}%
\pgfpathlineto{\pgfqpoint{1.094155in}{0.878222in}}%
\pgfpathlineto{\pgfqpoint{1.095347in}{0.876079in}}%
\pgfpathlineto{\pgfqpoint{1.094494in}{0.862468in}}%
\pgfpathlineto{\pgfqpoint{1.094155in}{0.861966in}}%
\pgfpathlineto{\pgfqpoint{1.086834in}{0.848857in}}%
\pgfpathlineto{\pgfqpoint{1.078498in}{0.840066in}}%
\pgfpathlineto{\pgfqpoint{1.074350in}{0.835246in}}%
\pgfpathlineto{\pgfqpoint{1.062842in}{0.824605in}}%
\pgfpathlineto{\pgfqpoint{1.059657in}{0.821635in}}%
\pgfpathlineto{\pgfqpoint{1.047185in}{0.810857in}}%
\pgfpathlineto{\pgfqpoint{1.043557in}{0.808024in}}%
\pgfpathlineto{\pgfqpoint{1.031529in}{0.798373in}}%
\pgfpathlineto{\pgfqpoint{1.025062in}{0.794413in}}%
\pgfpathlineto{\pgfqpoint{1.015872in}{0.788107in}}%
\pgfpathlineto{\pgfqpoint{1.000216in}{0.782477in}}%
\pgfpathlineto{\pgfqpoint{0.984559in}{0.783125in}}%
\pgfpathlineto{\pgfqpoint{0.968902in}{0.789886in}}%
\pgfpathlineto{\pgfqpoint{0.962803in}{0.794413in}}%
\pgfpathclose%
\pgfpathmoveto{\pgfqpoint{1.272844in}{0.794413in}}%
\pgfpathlineto{\pgfqpoint{1.266377in}{0.798373in}}%
\pgfpathlineto{\pgfqpoint{1.254349in}{0.808024in}}%
\pgfpathlineto{\pgfqpoint{1.250721in}{0.810857in}}%
\pgfpathlineto{\pgfqpoint{1.238248in}{0.821635in}}%
\pgfpathlineto{\pgfqpoint{1.235064in}{0.824605in}}%
\pgfpathlineto{\pgfqpoint{1.223556in}{0.835246in}}%
\pgfpathlineto{\pgfqpoint{1.219407in}{0.840066in}}%
\pgfpathlineto{\pgfqpoint{1.211071in}{0.848857in}}%
\pgfpathlineto{\pgfqpoint{1.203751in}{0.861966in}}%
\pgfpathlineto{\pgfqpoint{1.203412in}{0.862468in}}%
\pgfpathlineto{\pgfqpoint{1.202558in}{0.876079in}}%
\pgfpathlineto{\pgfqpoint{1.203751in}{0.878222in}}%
\pgfpathlineto{\pgfqpoint{1.209069in}{0.889691in}}%
\pgfpathlineto{\pgfqpoint{1.219407in}{0.901484in}}%
\pgfpathlineto{\pgfqpoint{1.220860in}{0.903302in}}%
\pgfpathlineto{\pgfqpoint{1.235064in}{0.916885in}}%
\pgfpathlineto{\pgfqpoint{1.235093in}{0.916913in}}%
\pgfpathlineto{\pgfqpoint{1.250721in}{0.930499in}}%
\pgfpathlineto{\pgfqpoint{1.250753in}{0.930524in}}%
\pgfpathlineto{\pgfqpoint{1.266377in}{0.942872in}}%
\pgfpathlineto{\pgfqpoint{1.268468in}{0.944135in}}%
\pgfpathlineto{\pgfqpoint{1.282034in}{0.953123in}}%
\pgfpathlineto{\pgfqpoint{1.295225in}{0.957746in}}%
\pgfpathlineto{\pgfqpoint{1.297690in}{0.958783in}}%
\pgfpathlineto{\pgfqpoint{1.313347in}{0.958041in}}%
\pgfpathlineto{\pgfqpoint{1.313925in}{0.957746in}}%
\pgfpathlineto{\pgfqpoint{1.329003in}{0.951382in}}%
\pgfpathlineto{\pgfqpoint{1.339116in}{0.944135in}}%
\pgfpathlineto{\pgfqpoint{1.344660in}{0.940528in}}%
\pgfpathlineto{\pgfqpoint{1.356900in}{0.930524in}}%
\pgfpathlineto{\pgfqpoint{1.360317in}{0.927756in}}%
\pgfpathlineto{\pgfqpoint{1.372714in}{0.916913in}}%
\pgfpathlineto{\pgfqpoint{1.375973in}{0.913758in}}%
\pgfpathlineto{\pgfqpoint{1.387074in}{0.903302in}}%
\pgfpathlineto{\pgfqpoint{1.391630in}{0.897680in}}%
\pgfpathlineto{\pgfqpoint{1.398884in}{0.889691in}}%
\pgfpathlineto{\pgfqpoint{1.405359in}{0.876079in}}%
\pgfpathlineto{\pgfqpoint{1.404613in}{0.862468in}}%
\pgfpathlineto{\pgfqpoint{1.396837in}{0.848857in}}%
\pgfpathlineto{\pgfqpoint{1.391630in}{0.843555in}}%
\pgfpathlineto{\pgfqpoint{1.384376in}{0.835246in}}%
\pgfpathlineto{\pgfqpoint{1.375973in}{0.827592in}}%
\pgfpathlineto{\pgfqpoint{1.369621in}{0.821635in}}%
\pgfpathlineto{\pgfqpoint{1.360317in}{0.813546in}}%
\pgfpathlineto{\pgfqpoint{1.353464in}{0.808024in}}%
\pgfpathlineto{\pgfqpoint{1.344660in}{0.800718in}}%
\pgfpathlineto{\pgfqpoint{1.335103in}{0.794413in}}%
\pgfpathlineto{\pgfqpoint{1.329003in}{0.789886in}}%
\pgfpathlineto{\pgfqpoint{1.313347in}{0.783125in}}%
\pgfpathlineto{\pgfqpoint{1.297690in}{0.782477in}}%
\pgfpathlineto{\pgfqpoint{1.282034in}{0.788107in}}%
\pgfpathlineto{\pgfqpoint{1.272844in}{0.794413in}}%
\pgfpathclose%
\pgfpathmoveto{\pgfqpoint{1.582566in}{0.794413in}}%
\pgfpathlineto{\pgfqpoint{1.579508in}{0.796127in}}%
\pgfpathlineto{\pgfqpoint{1.564068in}{0.808024in}}%
\pgfpathlineto{\pgfqpoint{1.563852in}{0.808186in}}%
\pgfpathlineto{\pgfqpoint{1.548195in}{0.821574in}}%
\pgfpathlineto{\pgfqpoint{1.548125in}{0.821635in}}%
\pgfpathlineto{\pgfqpoint{1.533564in}{0.835246in}}%
\pgfpathlineto{\pgfqpoint{1.532539in}{0.836449in}}%
\pgfpathlineto{\pgfqpoint{1.521132in}{0.848857in}}%
\pgfpathlineto{\pgfqpoint{1.516882in}{0.856691in}}%
\pgfpathlineto{\pgfqpoint{1.513194in}{0.862468in}}%
\pgfpathlineto{\pgfqpoint{1.512375in}{0.876079in}}%
\pgfpathlineto{\pgfqpoint{1.516882in}{0.884626in}}%
\pgfpathlineto{\pgfqpoint{1.519163in}{0.889691in}}%
\pgfpathlineto{\pgfqpoint{1.530818in}{0.903302in}}%
\pgfpathlineto{\pgfqpoint{1.532539in}{0.904814in}}%
\pgfpathlineto{\pgfqpoint{1.545138in}{0.916913in}}%
\pgfpathlineto{\pgfqpoint{1.548195in}{0.919564in}}%
\pgfpathlineto{\pgfqpoint{1.560930in}{0.930524in}}%
\pgfpathlineto{\pgfqpoint{1.563852in}{0.933071in}}%
\pgfpathlineto{\pgfqpoint{1.578304in}{0.944135in}}%
\pgfpathlineto{\pgfqpoint{1.579508in}{0.945161in}}%
\pgfpathlineto{\pgfqpoint{1.595165in}{0.954670in}}%
\pgfpathlineto{\pgfqpoint{1.606313in}{0.957746in}}%
\pgfpathlineto{\pgfqpoint{1.610822in}{0.959231in}}%
\pgfpathlineto{\pgfqpoint{1.621272in}{0.957746in}}%
\pgfpathlineto{\pgfqpoint{1.626478in}{0.957119in}}%
\pgfpathlineto{\pgfqpoint{1.642135in}{0.949463in}}%
\pgfpathlineto{\pgfqpoint{1.649087in}{0.944135in}}%
\pgfpathlineto{\pgfqpoint{1.657791in}{0.938102in}}%
\pgfpathlineto{\pgfqpoint{1.666814in}{0.930524in}}%
\pgfpathlineto{\pgfqpoint{1.673448in}{0.925010in}}%
\pgfpathlineto{\pgfqpoint{1.682692in}{0.916913in}}%
\pgfpathlineto{\pgfqpoint{1.689104in}{0.910698in}}%
\pgfpathlineto{\pgfqpoint{1.697108in}{0.903302in}}%
\pgfpathlineto{\pgfqpoint{1.704761in}{0.894023in}}%
\pgfpathlineto{\pgfqpoint{1.708852in}{0.889691in}}%
\pgfpathlineto{\pgfqpoint{1.715500in}{0.876079in}}%
\pgfpathlineto{\pgfqpoint{1.714735in}{0.862468in}}%
\pgfpathlineto{\pgfqpoint{1.706751in}{0.848857in}}%
\pgfpathlineto{\pgfqpoint{1.704761in}{0.846909in}}%
\pgfpathlineto{\pgfqpoint{1.694397in}{0.835246in}}%
\pgfpathlineto{\pgfqpoint{1.689104in}{0.830517in}}%
\pgfpathlineto{\pgfqpoint{1.679648in}{0.821635in}}%
\pgfpathlineto{\pgfqpoint{1.673448in}{0.816237in}}%
\pgfpathlineto{\pgfqpoint{1.663512in}{0.808024in}}%
\pgfpathlineto{\pgfqpoint{1.657791in}{0.803146in}}%
\pgfpathlineto{\pgfqpoint{1.645367in}{0.794413in}}%
\pgfpathlineto{\pgfqpoint{1.642135in}{0.791847in}}%
\pgfpathlineto{\pgfqpoint{1.626478in}{0.784023in}}%
\pgfpathlineto{\pgfqpoint{1.610822in}{0.782086in}}%
\pgfpathlineto{\pgfqpoint{1.595165in}{0.786526in}}%
\pgfpathlineto{\pgfqpoint{1.582566in}{0.794413in}}%
\pgfpathclose%
\pgfpathmoveto{\pgfqpoint{0.505301in}{0.903302in}}%
\pgfpathlineto{\pgfqpoint{0.499205in}{0.905346in}}%
\pgfpathlineto{\pgfqpoint{0.483549in}{0.914218in}}%
\pgfpathlineto{\pgfqpoint{0.480018in}{0.916913in}}%
\pgfpathlineto{\pgfqpoint{0.467892in}{0.925148in}}%
\pgfpathlineto{\pgfqpoint{0.461286in}{0.930524in}}%
\pgfpathlineto{\pgfqpoint{0.452236in}{0.937955in}}%
\pgfpathlineto{\pgfqpoint{0.445246in}{0.944135in}}%
\pgfpathlineto{\pgfqpoint{0.436579in}{0.952744in}}%
\pgfpathlineto{\pgfqpoint{0.431349in}{0.957746in}}%
\pgfpathlineto{\pgfqpoint{0.420923in}{0.970261in}}%
\pgfpathlineto{\pgfqpoint{0.419831in}{0.971357in}}%
\pgfpathlineto{\pgfqpoint{0.410330in}{0.984968in}}%
\pgfpathlineto{\pgfqpoint{0.405589in}{0.998579in}}%
\pgfpathlineto{\pgfqpoint{0.405589in}{1.012191in}}%
\pgfpathlineto{\pgfqpoint{0.410330in}{1.025802in}}%
\pgfpathlineto{\pgfqpoint{0.419831in}{1.039413in}}%
\pgfpathlineto{\pgfqpoint{0.420923in}{1.040509in}}%
\pgfpathlineto{\pgfqpoint{0.431349in}{1.053024in}}%
\pgfpathlineto{\pgfqpoint{0.436579in}{1.058026in}}%
\pgfpathlineto{\pgfqpoint{0.445246in}{1.066635in}}%
\pgfpathlineto{\pgfqpoint{0.452236in}{1.072815in}}%
\pgfpathlineto{\pgfqpoint{0.461286in}{1.080246in}}%
\pgfpathlineto{\pgfqpoint{0.467892in}{1.085622in}}%
\pgfpathlineto{\pgfqpoint{0.480018in}{1.093857in}}%
\pgfpathlineto{\pgfqpoint{0.483549in}{1.096552in}}%
\pgfpathlineto{\pgfqpoint{0.499205in}{1.105424in}}%
\pgfpathlineto{\pgfqpoint{0.505301in}{1.107468in}}%
\pgfpathlineto{\pgfqpoint{0.514862in}{1.111860in}}%
\pgfpathlineto{\pgfqpoint{0.530519in}{1.113932in}}%
\pgfpathlineto{\pgfqpoint{0.546175in}{1.110824in}}%
\pgfpathlineto{\pgfqpoint{0.552610in}{1.107468in}}%
\pgfpathlineto{\pgfqpoint{0.561832in}{1.103946in}}%
\pgfpathlineto{\pgfqpoint{0.577488in}{1.094338in}}%
\pgfpathlineto{\pgfqpoint{0.578093in}{1.093857in}}%
\pgfpathlineto{\pgfqpoint{0.593145in}{1.083177in}}%
\pgfpathlineto{\pgfqpoint{0.596669in}{1.080246in}}%
\pgfpathlineto{\pgfqpoint{0.608801in}{1.070054in}}%
\pgfpathlineto{\pgfqpoint{0.612657in}{1.066635in}}%
\pgfpathlineto{\pgfqpoint{0.624458in}{1.054880in}}%
\pgfpathlineto{\pgfqpoint{0.626437in}{1.053024in}}%
\pgfpathlineto{\pgfqpoint{0.638125in}{1.039413in}}%
\pgfpathlineto{\pgfqpoint{0.640115in}{1.036037in}}%
\pgfpathlineto{\pgfqpoint{0.647685in}{1.025802in}}%
\pgfpathlineto{\pgfqpoint{0.652753in}{1.012191in}}%
\pgfpathlineto{\pgfqpoint{0.652753in}{0.998579in}}%
\pgfpathlineto{\pgfqpoint{0.647685in}{0.984968in}}%
\pgfpathlineto{\pgfqpoint{0.640115in}{0.974733in}}%
\pgfpathlineto{\pgfqpoint{0.638125in}{0.971357in}}%
\pgfpathlineto{\pgfqpoint{0.626437in}{0.957746in}}%
\pgfpathlineto{\pgfqpoint{0.624458in}{0.955890in}}%
\pgfpathlineto{\pgfqpoint{0.612657in}{0.944135in}}%
\pgfpathlineto{\pgfqpoint{0.608801in}{0.940716in}}%
\pgfpathlineto{\pgfqpoint{0.596669in}{0.930524in}}%
\pgfpathlineto{\pgfqpoint{0.593145in}{0.927593in}}%
\pgfpathlineto{\pgfqpoint{0.578093in}{0.916913in}}%
\pgfpathlineto{\pgfqpoint{0.577488in}{0.916432in}}%
\pgfpathlineto{\pgfqpoint{0.561832in}{0.906824in}}%
\pgfpathlineto{\pgfqpoint{0.552610in}{0.903302in}}%
\pgfpathlineto{\pgfqpoint{0.546175in}{0.899946in}}%
\pgfpathlineto{\pgfqpoint{0.530519in}{0.896838in}}%
\pgfpathlineto{\pgfqpoint{0.514862in}{0.898910in}}%
\pgfpathlineto{\pgfqpoint{0.505301in}{0.903302in}}%
\pgfpathclose%
\pgfpathmoveto{\pgfqpoint{0.814813in}{0.903302in}}%
\pgfpathlineto{\pgfqpoint{0.812337in}{0.904015in}}%
\pgfpathlineto{\pgfqpoint{0.796680in}{0.912149in}}%
\pgfpathlineto{\pgfqpoint{0.790145in}{0.916913in}}%
\pgfpathlineto{\pgfqpoint{0.781024in}{0.922813in}}%
\pgfpathlineto{\pgfqpoint{0.771307in}{0.930524in}}%
\pgfpathlineto{\pgfqpoint{0.765367in}{0.935271in}}%
\pgfpathlineto{\pgfqpoint{0.755276in}{0.944135in}}%
\pgfpathlineto{\pgfqpoint{0.749710in}{0.949625in}}%
\pgfpathlineto{\pgfqpoint{0.741343in}{0.957746in}}%
\pgfpathlineto{\pgfqpoint{0.734054in}{0.966637in}}%
\pgfpathlineto{\pgfqpoint{0.729559in}{0.971357in}}%
\pgfpathlineto{\pgfqpoint{0.720602in}{0.984968in}}%
\pgfpathlineto{\pgfqpoint{0.718397in}{0.991619in}}%
\pgfpathlineto{\pgfqpoint{0.715061in}{0.998579in}}%
\pgfpathlineto{\pgfqpoint{0.715061in}{1.012191in}}%
\pgfpathlineto{\pgfqpoint{0.718397in}{1.019151in}}%
\pgfpathlineto{\pgfqpoint{0.720602in}{1.025802in}}%
\pgfpathlineto{\pgfqpoint{0.729559in}{1.039413in}}%
\pgfpathlineto{\pgfqpoint{0.734054in}{1.044133in}}%
\pgfpathlineto{\pgfqpoint{0.741343in}{1.053024in}}%
\pgfpathlineto{\pgfqpoint{0.749710in}{1.061145in}}%
\pgfpathlineto{\pgfqpoint{0.755276in}{1.066635in}}%
\pgfpathlineto{\pgfqpoint{0.765367in}{1.075499in}}%
\pgfpathlineto{\pgfqpoint{0.771307in}{1.080246in}}%
\pgfpathlineto{\pgfqpoint{0.781024in}{1.087957in}}%
\pgfpathlineto{\pgfqpoint{0.790145in}{1.093857in}}%
\pgfpathlineto{\pgfqpoint{0.796680in}{1.098621in}}%
\pgfpathlineto{\pgfqpoint{0.812337in}{1.106755in}}%
\pgfpathlineto{\pgfqpoint{0.814813in}{1.107468in}}%
\pgfpathlineto{\pgfqpoint{0.827993in}{1.112689in}}%
\pgfpathlineto{\pgfqpoint{0.843650in}{1.113725in}}%
\pgfpathlineto{\pgfqpoint{0.859306in}{1.109579in}}%
\pgfpathlineto{\pgfqpoint{0.862932in}{1.107468in}}%
\pgfpathlineto{\pgfqpoint{0.874963in}{1.102318in}}%
\pgfpathlineto{\pgfqpoint{0.887887in}{1.093857in}}%
\pgfpathlineto{\pgfqpoint{0.890620in}{1.092279in}}%
\pgfpathlineto{\pgfqpoint{0.906276in}{1.080624in}}%
\pgfpathlineto{\pgfqpoint{0.906722in}{1.080246in}}%
\pgfpathlineto{\pgfqpoint{0.921933in}{1.067221in}}%
\pgfpathlineto{\pgfqpoint{0.922595in}{1.066635in}}%
\pgfpathlineto{\pgfqpoint{0.936359in}{1.053024in}}%
\pgfpathlineto{\pgfqpoint{0.937589in}{1.051498in}}%
\pgfpathlineto{\pgfqpoint{0.948324in}{1.039413in}}%
\pgfpathlineto{\pgfqpoint{0.953246in}{1.031285in}}%
\pgfpathlineto{\pgfqpoint{0.957567in}{1.025802in}}%
\pgfpathlineto{\pgfqpoint{0.963035in}{1.012191in}}%
\pgfpathlineto{\pgfqpoint{0.963035in}{0.998579in}}%
\pgfpathlineto{\pgfqpoint{0.957567in}{0.984968in}}%
\pgfpathlineto{\pgfqpoint{0.953246in}{0.979485in}}%
\pgfpathlineto{\pgfqpoint{0.948324in}{0.971357in}}%
\pgfpathlineto{\pgfqpoint{0.937589in}{0.959272in}}%
\pgfpathlineto{\pgfqpoint{0.936359in}{0.957746in}}%
\pgfpathlineto{\pgfqpoint{0.922595in}{0.944135in}}%
\pgfpathlineto{\pgfqpoint{0.921933in}{0.943549in}}%
\pgfpathlineto{\pgfqpoint{0.906722in}{0.930524in}}%
\pgfpathlineto{\pgfqpoint{0.906276in}{0.930146in}}%
\pgfpathlineto{\pgfqpoint{0.890620in}{0.918491in}}%
\pgfpathlineto{\pgfqpoint{0.887887in}{0.916913in}}%
\pgfpathlineto{\pgfqpoint{0.874963in}{0.908452in}}%
\pgfpathlineto{\pgfqpoint{0.862932in}{0.903302in}}%
\pgfpathlineto{\pgfqpoint{0.859306in}{0.901191in}}%
\pgfpathlineto{\pgfqpoint{0.843650in}{0.897045in}}%
\pgfpathlineto{\pgfqpoint{0.827993in}{0.898081in}}%
\pgfpathlineto{\pgfqpoint{0.814813in}{0.903302in}}%
\pgfpathclose%
\pgfpathmoveto{\pgfqpoint{1.124442in}{0.903302in}}%
\pgfpathlineto{\pgfqpoint{1.109812in}{0.910226in}}%
\pgfpathlineto{\pgfqpoint{1.100160in}{0.916913in}}%
\pgfpathlineto{\pgfqpoint{1.094155in}{0.920594in}}%
\pgfpathlineto{\pgfqpoint{1.081273in}{0.930524in}}%
\pgfpathlineto{\pgfqpoint{1.078498in}{0.932674in}}%
\pgfpathlineto{\pgfqpoint{1.065315in}{0.944135in}}%
\pgfpathlineto{\pgfqpoint{1.062842in}{0.946547in}}%
\pgfpathlineto{\pgfqpoint{1.051419in}{0.957746in}}%
\pgfpathlineto{\pgfqpoint{1.047185in}{0.962966in}}%
\pgfpathlineto{\pgfqpoint{1.039494in}{0.971357in}}%
\pgfpathlineto{\pgfqpoint{1.031529in}{0.984076in}}%
\pgfpathlineto{\pgfqpoint{1.030771in}{0.984968in}}%
\pgfpathlineto{\pgfqpoint{1.024809in}{0.998579in}}%
\pgfpathlineto{\pgfqpoint{1.024809in}{1.012191in}}%
\pgfpathlineto{\pgfqpoint{1.030771in}{1.025802in}}%
\pgfpathlineto{\pgfqpoint{1.031529in}{1.026694in}}%
\pgfpathlineto{\pgfqpoint{1.039494in}{1.039413in}}%
\pgfpathlineto{\pgfqpoint{1.047185in}{1.047804in}}%
\pgfpathlineto{\pgfqpoint{1.051419in}{1.053024in}}%
\pgfpathlineto{\pgfqpoint{1.062842in}{1.064223in}}%
\pgfpathlineto{\pgfqpoint{1.065315in}{1.066635in}}%
\pgfpathlineto{\pgfqpoint{1.078498in}{1.078096in}}%
\pgfpathlineto{\pgfqpoint{1.081273in}{1.080246in}}%
\pgfpathlineto{\pgfqpoint{1.094155in}{1.090176in}}%
\pgfpathlineto{\pgfqpoint{1.100160in}{1.093857in}}%
\pgfpathlineto{\pgfqpoint{1.109812in}{1.100544in}}%
\pgfpathlineto{\pgfqpoint{1.124442in}{1.107468in}}%
\pgfpathlineto{\pgfqpoint{1.125468in}{1.108127in}}%
\pgfpathlineto{\pgfqpoint{1.141125in}{1.113310in}}%
\pgfpathlineto{\pgfqpoint{1.156781in}{1.113310in}}%
\pgfpathlineto{\pgfqpoint{1.172438in}{1.108127in}}%
\pgfpathlineto{\pgfqpoint{1.173464in}{1.107468in}}%
\pgfpathlineto{\pgfqpoint{1.188094in}{1.100544in}}%
\pgfpathlineto{\pgfqpoint{1.197746in}{1.093857in}}%
\pgfpathlineto{\pgfqpoint{1.203751in}{1.090176in}}%
\pgfpathlineto{\pgfqpoint{1.216633in}{1.080246in}}%
\pgfpathlineto{\pgfqpoint{1.219407in}{1.078096in}}%
\pgfpathlineto{\pgfqpoint{1.232591in}{1.066635in}}%
\pgfpathlineto{\pgfqpoint{1.235064in}{1.064223in}}%
\pgfpathlineto{\pgfqpoint{1.246486in}{1.053024in}}%
\pgfpathlineto{\pgfqpoint{1.250721in}{1.047804in}}%
\pgfpathlineto{\pgfqpoint{1.258412in}{1.039413in}}%
\pgfpathlineto{\pgfqpoint{1.266377in}{1.026694in}}%
\pgfpathlineto{\pgfqpoint{1.267134in}{1.025802in}}%
\pgfpathlineto{\pgfqpoint{1.273097in}{1.012191in}}%
\pgfpathlineto{\pgfqpoint{1.273097in}{0.998579in}}%
\pgfpathlineto{\pgfqpoint{1.267134in}{0.984968in}}%
\pgfpathlineto{\pgfqpoint{1.266377in}{0.984076in}}%
\pgfpathlineto{\pgfqpoint{1.258412in}{0.971357in}}%
\pgfpathlineto{\pgfqpoint{1.250721in}{0.962966in}}%
\pgfpathlineto{\pgfqpoint{1.246486in}{0.957746in}}%
\pgfpathlineto{\pgfqpoint{1.235064in}{0.946547in}}%
\pgfpathlineto{\pgfqpoint{1.232591in}{0.944135in}}%
\pgfpathlineto{\pgfqpoint{1.219407in}{0.932674in}}%
\pgfpathlineto{\pgfqpoint{1.216633in}{0.930524in}}%
\pgfpathlineto{\pgfqpoint{1.203751in}{0.920594in}}%
\pgfpathlineto{\pgfqpoint{1.197746in}{0.916913in}}%
\pgfpathlineto{\pgfqpoint{1.188094in}{0.910226in}}%
\pgfpathlineto{\pgfqpoint{1.173464in}{0.903302in}}%
\pgfpathlineto{\pgfqpoint{1.172438in}{0.902643in}}%
\pgfpathlineto{\pgfqpoint{1.156781in}{0.897460in}}%
\pgfpathlineto{\pgfqpoint{1.141125in}{0.897460in}}%
\pgfpathlineto{\pgfqpoint{1.125468in}{0.902643in}}%
\pgfpathlineto{\pgfqpoint{1.124442in}{0.903302in}}%
\pgfpathclose%
\pgfpathmoveto{\pgfqpoint{1.434974in}{0.903302in}}%
\pgfpathlineto{\pgfqpoint{1.422943in}{0.908452in}}%
\pgfpathlineto{\pgfqpoint{1.410019in}{0.916913in}}%
\pgfpathlineto{\pgfqpoint{1.407286in}{0.918491in}}%
\pgfpathlineto{\pgfqpoint{1.391630in}{0.930146in}}%
\pgfpathlineto{\pgfqpoint{1.391184in}{0.930524in}}%
\pgfpathlineto{\pgfqpoint{1.375973in}{0.943549in}}%
\pgfpathlineto{\pgfqpoint{1.375311in}{0.944135in}}%
\pgfpathlineto{\pgfqpoint{1.361547in}{0.957746in}}%
\pgfpathlineto{\pgfqpoint{1.360317in}{0.959272in}}%
\pgfpathlineto{\pgfqpoint{1.349582in}{0.971357in}}%
\pgfpathlineto{\pgfqpoint{1.344660in}{0.979485in}}%
\pgfpathlineto{\pgfqpoint{1.340339in}{0.984968in}}%
\pgfpathlineto{\pgfqpoint{1.334871in}{0.998579in}}%
\pgfpathlineto{\pgfqpoint{1.334871in}{1.012191in}}%
\pgfpathlineto{\pgfqpoint{1.340339in}{1.025802in}}%
\pgfpathlineto{\pgfqpoint{1.344660in}{1.031285in}}%
\pgfpathlineto{\pgfqpoint{1.349582in}{1.039413in}}%
\pgfpathlineto{\pgfqpoint{1.360317in}{1.051498in}}%
\pgfpathlineto{\pgfqpoint{1.361547in}{1.053024in}}%
\pgfpathlineto{\pgfqpoint{1.375311in}{1.066635in}}%
\pgfpathlineto{\pgfqpoint{1.375973in}{1.067221in}}%
\pgfpathlineto{\pgfqpoint{1.391184in}{1.080246in}}%
\pgfpathlineto{\pgfqpoint{1.391630in}{1.080624in}}%
\pgfpathlineto{\pgfqpoint{1.407286in}{1.092279in}}%
\pgfpathlineto{\pgfqpoint{1.410019in}{1.093857in}}%
\pgfpathlineto{\pgfqpoint{1.422943in}{1.102318in}}%
\pgfpathlineto{\pgfqpoint{1.434974in}{1.107468in}}%
\pgfpathlineto{\pgfqpoint{1.438599in}{1.109579in}}%
\pgfpathlineto{\pgfqpoint{1.454256in}{1.113725in}}%
\pgfpathlineto{\pgfqpoint{1.469913in}{1.112689in}}%
\pgfpathlineto{\pgfqpoint{1.483093in}{1.107468in}}%
\pgfpathlineto{\pgfqpoint{1.485569in}{1.106755in}}%
\pgfpathlineto{\pgfqpoint{1.501226in}{1.098621in}}%
\pgfpathlineto{\pgfqpoint{1.507760in}{1.093857in}}%
\pgfpathlineto{\pgfqpoint{1.516882in}{1.087957in}}%
\pgfpathlineto{\pgfqpoint{1.526599in}{1.080246in}}%
\pgfpathlineto{\pgfqpoint{1.532539in}{1.075499in}}%
\pgfpathlineto{\pgfqpoint{1.542629in}{1.066635in}}%
\pgfpathlineto{\pgfqpoint{1.548195in}{1.061145in}}%
\pgfpathlineto{\pgfqpoint{1.556562in}{1.053024in}}%
\pgfpathlineto{\pgfqpoint{1.563852in}{1.044133in}}%
\pgfpathlineto{\pgfqpoint{1.568347in}{1.039413in}}%
\pgfpathlineto{\pgfqpoint{1.577304in}{1.025802in}}%
\pgfpathlineto{\pgfqpoint{1.579508in}{1.019151in}}%
\pgfpathlineto{\pgfqpoint{1.582845in}{1.012191in}}%
\pgfpathlineto{\pgfqpoint{1.582845in}{0.998579in}}%
\pgfpathlineto{\pgfqpoint{1.579508in}{0.991619in}}%
\pgfpathlineto{\pgfqpoint{1.577304in}{0.984968in}}%
\pgfpathlineto{\pgfqpoint{1.568347in}{0.971357in}}%
\pgfpathlineto{\pgfqpoint{1.563852in}{0.966637in}}%
\pgfpathlineto{\pgfqpoint{1.556562in}{0.957746in}}%
\pgfpathlineto{\pgfqpoint{1.548195in}{0.949625in}}%
\pgfpathlineto{\pgfqpoint{1.542629in}{0.944135in}}%
\pgfpathlineto{\pgfqpoint{1.532539in}{0.935271in}}%
\pgfpathlineto{\pgfqpoint{1.526599in}{0.930524in}}%
\pgfpathlineto{\pgfqpoint{1.516882in}{0.922813in}}%
\pgfpathlineto{\pgfqpoint{1.507760in}{0.916913in}}%
\pgfpathlineto{\pgfqpoint{1.501226in}{0.912149in}}%
\pgfpathlineto{\pgfqpoint{1.485569in}{0.904015in}}%
\pgfpathlineto{\pgfqpoint{1.483093in}{0.903302in}}%
\pgfpathlineto{\pgfqpoint{1.469913in}{0.898081in}}%
\pgfpathlineto{\pgfqpoint{1.454256in}{0.897045in}}%
\pgfpathlineto{\pgfqpoint{1.438599in}{0.901191in}}%
\pgfpathlineto{\pgfqpoint{1.434974in}{0.903302in}}%
\pgfpathclose%
\pgfpathmoveto{\pgfqpoint{1.745296in}{0.903302in}}%
\pgfpathlineto{\pgfqpoint{1.736074in}{0.906824in}}%
\pgfpathlineto{\pgfqpoint{1.720418in}{0.916432in}}%
\pgfpathlineto{\pgfqpoint{1.719813in}{0.916913in}}%
\pgfpathlineto{\pgfqpoint{1.704761in}{0.927593in}}%
\pgfpathlineto{\pgfqpoint{1.701236in}{0.930524in}}%
\pgfpathlineto{\pgfqpoint{1.689104in}{0.940716in}}%
\pgfpathlineto{\pgfqpoint{1.685248in}{0.944135in}}%
\pgfpathlineto{\pgfqpoint{1.673448in}{0.955890in}}%
\pgfpathlineto{\pgfqpoint{1.671468in}{0.957746in}}%
\pgfpathlineto{\pgfqpoint{1.659781in}{0.971357in}}%
\pgfpathlineto{\pgfqpoint{1.657791in}{0.974733in}}%
\pgfpathlineto{\pgfqpoint{1.650221in}{0.984968in}}%
\pgfpathlineto{\pgfqpoint{1.645152in}{0.998579in}}%
\pgfpathlineto{\pgfqpoint{1.645152in}{1.012191in}}%
\pgfpathlineto{\pgfqpoint{1.650221in}{1.025802in}}%
\pgfpathlineto{\pgfqpoint{1.657791in}{1.036037in}}%
\pgfpathlineto{\pgfqpoint{1.659781in}{1.039413in}}%
\pgfpathlineto{\pgfqpoint{1.671468in}{1.053024in}}%
\pgfpathlineto{\pgfqpoint{1.673448in}{1.054880in}}%
\pgfpathlineto{\pgfqpoint{1.685248in}{1.066635in}}%
\pgfpathlineto{\pgfqpoint{1.689104in}{1.070054in}}%
\pgfpathlineto{\pgfqpoint{1.701236in}{1.080246in}}%
\pgfpathlineto{\pgfqpoint{1.704761in}{1.083177in}}%
\pgfpathlineto{\pgfqpoint{1.719813in}{1.093857in}}%
\pgfpathlineto{\pgfqpoint{1.720418in}{1.094338in}}%
\pgfpathlineto{\pgfqpoint{1.736074in}{1.103946in}}%
\pgfpathlineto{\pgfqpoint{1.745296in}{1.107468in}}%
\pgfpathlineto{\pgfqpoint{1.751731in}{1.110824in}}%
\pgfpathlineto{\pgfqpoint{1.767387in}{1.113932in}}%
\pgfpathlineto{\pgfqpoint{1.783044in}{1.111860in}}%
\pgfpathlineto{\pgfqpoint{1.792605in}{1.107468in}}%
\pgfpathlineto{\pgfqpoint{1.798700in}{1.105424in}}%
\pgfpathlineto{\pgfqpoint{1.814357in}{1.096552in}}%
\pgfpathlineto{\pgfqpoint{1.817888in}{1.093857in}}%
\pgfpathlineto{\pgfqpoint{1.830014in}{1.085622in}}%
\pgfpathlineto{\pgfqpoint{1.836620in}{1.080246in}}%
\pgfpathlineto{\pgfqpoint{1.845670in}{1.072815in}}%
\pgfpathlineto{\pgfqpoint{1.852660in}{1.066635in}}%
\pgfpathlineto{\pgfqpoint{1.861327in}{1.058026in}}%
\pgfpathlineto{\pgfqpoint{1.866557in}{1.053024in}}%
\pgfpathlineto{\pgfqpoint{1.876983in}{1.040509in}}%
\pgfpathlineto{\pgfqpoint{1.878075in}{1.039413in}}%
\pgfpathlineto{\pgfqpoint{1.887576in}{1.025802in}}%
\pgfpathlineto{\pgfqpoint{1.892317in}{1.012191in}}%
\pgfpathlineto{\pgfqpoint{1.892317in}{0.998579in}}%
\pgfpathlineto{\pgfqpoint{1.887576in}{0.984968in}}%
\pgfpathlineto{\pgfqpoint{1.878075in}{0.971357in}}%
\pgfpathlineto{\pgfqpoint{1.876983in}{0.970261in}}%
\pgfpathlineto{\pgfqpoint{1.866557in}{0.957746in}}%
\pgfpathlineto{\pgfqpoint{1.861327in}{0.952744in}}%
\pgfpathlineto{\pgfqpoint{1.852660in}{0.944135in}}%
\pgfpathlineto{\pgfqpoint{1.845670in}{0.937955in}}%
\pgfpathlineto{\pgfqpoint{1.836620in}{0.930524in}}%
\pgfpathlineto{\pgfqpoint{1.830014in}{0.925148in}}%
\pgfpathlineto{\pgfqpoint{1.817888in}{0.916913in}}%
\pgfpathlineto{\pgfqpoint{1.814357in}{0.914218in}}%
\pgfpathlineto{\pgfqpoint{1.798700in}{0.905346in}}%
\pgfpathlineto{\pgfqpoint{1.792605in}{0.903302in}}%
\pgfpathlineto{\pgfqpoint{1.783044in}{0.898910in}}%
\pgfpathlineto{\pgfqpoint{1.767387in}{0.896838in}}%
\pgfpathlineto{\pgfqpoint{1.751731in}{0.899946in}}%
\pgfpathlineto{\pgfqpoint{1.745296in}{0.903302in}}%
\pgfpathclose%
\pgfpathmoveto{\pgfqpoint{0.676633in}{1.053024in}}%
\pgfpathlineto{\pgfqpoint{0.671428in}{1.053651in}}%
\pgfpathlineto{\pgfqpoint{0.655771in}{1.061307in}}%
\pgfpathlineto{\pgfqpoint{0.648819in}{1.066635in}}%
\pgfpathlineto{\pgfqpoint{0.640115in}{1.072668in}}%
\pgfpathlineto{\pgfqpoint{0.631092in}{1.080246in}}%
\pgfpathlineto{\pgfqpoint{0.624458in}{1.085760in}}%
\pgfpathlineto{\pgfqpoint{0.615214in}{1.093857in}}%
\pgfpathlineto{\pgfqpoint{0.608801in}{1.100072in}}%
\pgfpathlineto{\pgfqpoint{0.600798in}{1.107468in}}%
\pgfpathlineto{\pgfqpoint{0.593145in}{1.116747in}}%
\pgfpathlineto{\pgfqpoint{0.589053in}{1.121079in}}%
\pgfpathlineto{\pgfqpoint{0.582406in}{1.134691in}}%
\pgfpathlineto{\pgfqpoint{0.583171in}{1.148302in}}%
\pgfpathlineto{\pgfqpoint{0.591155in}{1.161913in}}%
\pgfpathlineto{\pgfqpoint{0.593145in}{1.163861in}}%
\pgfpathlineto{\pgfqpoint{0.603509in}{1.175524in}}%
\pgfpathlineto{\pgfqpoint{0.608801in}{1.180253in}}%
\pgfpathlineto{\pgfqpoint{0.618258in}{1.189135in}}%
\pgfpathlineto{\pgfqpoint{0.624458in}{1.194533in}}%
\pgfpathlineto{\pgfqpoint{0.634394in}{1.202746in}}%
\pgfpathlineto{\pgfqpoint{0.640115in}{1.207624in}}%
\pgfpathlineto{\pgfqpoint{0.652538in}{1.216357in}}%
\pgfpathlineto{\pgfqpoint{0.655771in}{1.218923in}}%
\pgfpathlineto{\pgfqpoint{0.671428in}{1.226747in}}%
\pgfpathlineto{\pgfqpoint{0.687084in}{1.228684in}}%
\pgfpathlineto{\pgfqpoint{0.702741in}{1.224244in}}%
\pgfpathlineto{\pgfqpoint{0.715340in}{1.216357in}}%
\pgfpathlineto{\pgfqpoint{0.718397in}{1.214643in}}%
\pgfpathlineto{\pgfqpoint{0.733838in}{1.202746in}}%
\pgfpathlineto{\pgfqpoint{0.734054in}{1.202584in}}%
\pgfpathlineto{\pgfqpoint{0.749710in}{1.189196in}}%
\pgfpathlineto{\pgfqpoint{0.749781in}{1.189135in}}%
\pgfpathlineto{\pgfqpoint{0.764342in}{1.175524in}}%
\pgfpathlineto{\pgfqpoint{0.765367in}{1.174321in}}%
\pgfpathlineto{\pgfqpoint{0.776774in}{1.161913in}}%
\pgfpathlineto{\pgfqpoint{0.781024in}{1.154079in}}%
\pgfpathlineto{\pgfqpoint{0.784712in}{1.148302in}}%
\pgfpathlineto{\pgfqpoint{0.785531in}{1.134691in}}%
\pgfpathlineto{\pgfqpoint{0.781024in}{1.126144in}}%
\pgfpathlineto{\pgfqpoint{0.778742in}{1.121079in}}%
\pgfpathlineto{\pgfqpoint{0.767088in}{1.107468in}}%
\pgfpathlineto{\pgfqpoint{0.765367in}{1.105956in}}%
\pgfpathlineto{\pgfqpoint{0.752768in}{1.093857in}}%
\pgfpathlineto{\pgfqpoint{0.749710in}{1.091206in}}%
\pgfpathlineto{\pgfqpoint{0.736976in}{1.080246in}}%
\pgfpathlineto{\pgfqpoint{0.734054in}{1.077699in}}%
\pgfpathlineto{\pgfqpoint{0.719602in}{1.066635in}}%
\pgfpathlineto{\pgfqpoint{0.718397in}{1.065609in}}%
\pgfpathlineto{\pgfqpoint{0.702741in}{1.056100in}}%
\pgfpathlineto{\pgfqpoint{0.691593in}{1.053024in}}%
\pgfpathlineto{\pgfqpoint{0.687084in}{1.051539in}}%
\pgfpathlineto{\pgfqpoint{0.676633in}{1.053024in}}%
\pgfpathclose%
\pgfpathmoveto{\pgfqpoint{0.983981in}{1.053024in}}%
\pgfpathlineto{\pgfqpoint{0.968902in}{1.059388in}}%
\pgfpathlineto{\pgfqpoint{0.958790in}{1.066635in}}%
\pgfpathlineto{\pgfqpoint{0.953246in}{1.070242in}}%
\pgfpathlineto{\pgfqpoint{0.941005in}{1.080246in}}%
\pgfpathlineto{\pgfqpoint{0.937589in}{1.083014in}}%
\pgfpathlineto{\pgfqpoint{0.925192in}{1.093857in}}%
\pgfpathlineto{\pgfqpoint{0.921933in}{1.097012in}}%
\pgfpathlineto{\pgfqpoint{0.910831in}{1.107468in}}%
\pgfpathlineto{\pgfqpoint{0.906276in}{1.113090in}}%
\pgfpathlineto{\pgfqpoint{0.899022in}{1.121079in}}%
\pgfpathlineto{\pgfqpoint{0.892547in}{1.134691in}}%
\pgfpathlineto{\pgfqpoint{0.893292in}{1.148302in}}%
\pgfpathlineto{\pgfqpoint{0.901069in}{1.161913in}}%
\pgfpathlineto{\pgfqpoint{0.906276in}{1.167215in}}%
\pgfpathlineto{\pgfqpoint{0.913529in}{1.175524in}}%
\pgfpathlineto{\pgfqpoint{0.921933in}{1.183178in}}%
\pgfpathlineto{\pgfqpoint{0.928285in}{1.189135in}}%
\pgfpathlineto{\pgfqpoint{0.937589in}{1.197224in}}%
\pgfpathlineto{\pgfqpoint{0.944442in}{1.202746in}}%
\pgfpathlineto{\pgfqpoint{0.953246in}{1.210052in}}%
\pgfpathlineto{\pgfqpoint{0.962803in}{1.216357in}}%
\pgfpathlineto{\pgfqpoint{0.968902in}{1.220884in}}%
\pgfpathlineto{\pgfqpoint{0.984559in}{1.227645in}}%
\pgfpathlineto{\pgfqpoint{1.000216in}{1.228293in}}%
\pgfpathlineto{\pgfqpoint{1.015872in}{1.222663in}}%
\pgfpathlineto{\pgfqpoint{1.025062in}{1.216357in}}%
\pgfpathlineto{\pgfqpoint{1.031529in}{1.212397in}}%
\pgfpathlineto{\pgfqpoint{1.043557in}{1.202746in}}%
\pgfpathlineto{\pgfqpoint{1.047185in}{1.199913in}}%
\pgfpathlineto{\pgfqpoint{1.059657in}{1.189135in}}%
\pgfpathlineto{\pgfqpoint{1.062842in}{1.186165in}}%
\pgfpathlineto{\pgfqpoint{1.074350in}{1.175524in}}%
\pgfpathlineto{\pgfqpoint{1.078498in}{1.170704in}}%
\pgfpathlineto{\pgfqpoint{1.086834in}{1.161913in}}%
\pgfpathlineto{\pgfqpoint{1.094155in}{1.148804in}}%
\pgfpathlineto{\pgfqpoint{1.094494in}{1.148302in}}%
\pgfpathlineto{\pgfqpoint{1.095347in}{1.134691in}}%
\pgfpathlineto{\pgfqpoint{1.094155in}{1.132548in}}%
\pgfpathlineto{\pgfqpoint{1.088837in}{1.121079in}}%
\pgfpathlineto{\pgfqpoint{1.078498in}{1.109286in}}%
\pgfpathlineto{\pgfqpoint{1.077046in}{1.107468in}}%
\pgfpathlineto{\pgfqpoint{1.062842in}{1.093885in}}%
\pgfpathlineto{\pgfqpoint{1.062813in}{1.093857in}}%
\pgfpathlineto{\pgfqpoint{1.047185in}{1.080271in}}%
\pgfpathlineto{\pgfqpoint{1.047153in}{1.080246in}}%
\pgfpathlineto{\pgfqpoint{1.031529in}{1.067898in}}%
\pgfpathlineto{\pgfqpoint{1.029438in}{1.066635in}}%
\pgfpathlineto{\pgfqpoint{1.015872in}{1.057647in}}%
\pgfpathlineto{\pgfqpoint{1.002680in}{1.053024in}}%
\pgfpathlineto{\pgfqpoint{1.000216in}{1.051987in}}%
\pgfpathlineto{\pgfqpoint{0.984559in}{1.052729in}}%
\pgfpathlineto{\pgfqpoint{0.983981in}{1.053024in}}%
\pgfpathclose%
\pgfpathmoveto{\pgfqpoint{1.295225in}{1.053024in}}%
\pgfpathlineto{\pgfqpoint{1.282034in}{1.057647in}}%
\pgfpathlineto{\pgfqpoint{1.268468in}{1.066635in}}%
\pgfpathlineto{\pgfqpoint{1.266377in}{1.067898in}}%
\pgfpathlineto{\pgfqpoint{1.250753in}{1.080246in}}%
\pgfpathlineto{\pgfqpoint{1.250721in}{1.080271in}}%
\pgfpathlineto{\pgfqpoint{1.235093in}{1.093857in}}%
\pgfpathlineto{\pgfqpoint{1.235064in}{1.093885in}}%
\pgfpathlineto{\pgfqpoint{1.220860in}{1.107468in}}%
\pgfpathlineto{\pgfqpoint{1.219407in}{1.109286in}}%
\pgfpathlineto{\pgfqpoint{1.209069in}{1.121079in}}%
\pgfpathlineto{\pgfqpoint{1.203751in}{1.132548in}}%
\pgfpathlineto{\pgfqpoint{1.202558in}{1.134691in}}%
\pgfpathlineto{\pgfqpoint{1.203412in}{1.148302in}}%
\pgfpathlineto{\pgfqpoint{1.203751in}{1.148804in}}%
\pgfpathlineto{\pgfqpoint{1.211071in}{1.161913in}}%
\pgfpathlineto{\pgfqpoint{1.219407in}{1.170704in}}%
\pgfpathlineto{\pgfqpoint{1.223556in}{1.175524in}}%
\pgfpathlineto{\pgfqpoint{1.235064in}{1.186165in}}%
\pgfpathlineto{\pgfqpoint{1.238248in}{1.189135in}}%
\pgfpathlineto{\pgfqpoint{1.250721in}{1.199913in}}%
\pgfpathlineto{\pgfqpoint{1.254349in}{1.202746in}}%
\pgfpathlineto{\pgfqpoint{1.266377in}{1.212397in}}%
\pgfpathlineto{\pgfqpoint{1.272844in}{1.216357in}}%
\pgfpathlineto{\pgfqpoint{1.282034in}{1.222663in}}%
\pgfpathlineto{\pgfqpoint{1.297690in}{1.228293in}}%
\pgfpathlineto{\pgfqpoint{1.313347in}{1.227645in}}%
\pgfpathlineto{\pgfqpoint{1.329003in}{1.220884in}}%
\pgfpathlineto{\pgfqpoint{1.335103in}{1.216357in}}%
\pgfpathlineto{\pgfqpoint{1.344660in}{1.210052in}}%
\pgfpathlineto{\pgfqpoint{1.353464in}{1.202746in}}%
\pgfpathlineto{\pgfqpoint{1.360317in}{1.197224in}}%
\pgfpathlineto{\pgfqpoint{1.369621in}{1.189135in}}%
\pgfpathlineto{\pgfqpoint{1.375973in}{1.183178in}}%
\pgfpathlineto{\pgfqpoint{1.384376in}{1.175524in}}%
\pgfpathlineto{\pgfqpoint{1.391630in}{1.167215in}}%
\pgfpathlineto{\pgfqpoint{1.396837in}{1.161913in}}%
\pgfpathlineto{\pgfqpoint{1.404613in}{1.148302in}}%
\pgfpathlineto{\pgfqpoint{1.405359in}{1.134691in}}%
\pgfpathlineto{\pgfqpoint{1.398884in}{1.121079in}}%
\pgfpathlineto{\pgfqpoint{1.391630in}{1.113090in}}%
\pgfpathlineto{\pgfqpoint{1.387074in}{1.107468in}}%
\pgfpathlineto{\pgfqpoint{1.375973in}{1.097012in}}%
\pgfpathlineto{\pgfqpoint{1.372714in}{1.093857in}}%
\pgfpathlineto{\pgfqpoint{1.360317in}{1.083014in}}%
\pgfpathlineto{\pgfqpoint{1.356900in}{1.080246in}}%
\pgfpathlineto{\pgfqpoint{1.344660in}{1.070242in}}%
\pgfpathlineto{\pgfqpoint{1.339116in}{1.066635in}}%
\pgfpathlineto{\pgfqpoint{1.329003in}{1.059388in}}%
\pgfpathlineto{\pgfqpoint{1.313925in}{1.053024in}}%
\pgfpathlineto{\pgfqpoint{1.313347in}{1.052729in}}%
\pgfpathlineto{\pgfqpoint{1.297690in}{1.051987in}}%
\pgfpathlineto{\pgfqpoint{1.295225in}{1.053024in}}%
\pgfpathclose%
\pgfpathmoveto{\pgfqpoint{1.606313in}{1.053024in}}%
\pgfpathlineto{\pgfqpoint{1.595165in}{1.056100in}}%
\pgfpathlineto{\pgfqpoint{1.579508in}{1.065609in}}%
\pgfpathlineto{\pgfqpoint{1.578304in}{1.066635in}}%
\pgfpathlineto{\pgfqpoint{1.563852in}{1.077699in}}%
\pgfpathlineto{\pgfqpoint{1.560930in}{1.080246in}}%
\pgfpathlineto{\pgfqpoint{1.548195in}{1.091206in}}%
\pgfpathlineto{\pgfqpoint{1.545138in}{1.093857in}}%
\pgfpathlineto{\pgfqpoint{1.532539in}{1.105956in}}%
\pgfpathlineto{\pgfqpoint{1.530818in}{1.107468in}}%
\pgfpathlineto{\pgfqpoint{1.519163in}{1.121079in}}%
\pgfpathlineto{\pgfqpoint{1.516882in}{1.126144in}}%
\pgfpathlineto{\pgfqpoint{1.512375in}{1.134691in}}%
\pgfpathlineto{\pgfqpoint{1.513194in}{1.148302in}}%
\pgfpathlineto{\pgfqpoint{1.516882in}{1.154079in}}%
\pgfpathlineto{\pgfqpoint{1.521132in}{1.161913in}}%
\pgfpathlineto{\pgfqpoint{1.532539in}{1.174321in}}%
\pgfpathlineto{\pgfqpoint{1.533564in}{1.175524in}}%
\pgfpathlineto{\pgfqpoint{1.548125in}{1.189135in}}%
\pgfpathlineto{\pgfqpoint{1.548195in}{1.189196in}}%
\pgfpathlineto{\pgfqpoint{1.563852in}{1.202584in}}%
\pgfpathlineto{\pgfqpoint{1.564068in}{1.202746in}}%
\pgfpathlineto{\pgfqpoint{1.579508in}{1.214643in}}%
\pgfpathlineto{\pgfqpoint{1.582566in}{1.216357in}}%
\pgfpathlineto{\pgfqpoint{1.595165in}{1.224244in}}%
\pgfpathlineto{\pgfqpoint{1.610822in}{1.228684in}}%
\pgfpathlineto{\pgfqpoint{1.626478in}{1.226747in}}%
\pgfpathlineto{\pgfqpoint{1.642135in}{1.218923in}}%
\pgfpathlineto{\pgfqpoint{1.645367in}{1.216357in}}%
\pgfpathlineto{\pgfqpoint{1.657791in}{1.207624in}}%
\pgfpathlineto{\pgfqpoint{1.663512in}{1.202746in}}%
\pgfpathlineto{\pgfqpoint{1.673448in}{1.194533in}}%
\pgfpathlineto{\pgfqpoint{1.679648in}{1.189135in}}%
\pgfpathlineto{\pgfqpoint{1.689104in}{1.180253in}}%
\pgfpathlineto{\pgfqpoint{1.694397in}{1.175524in}}%
\pgfpathlineto{\pgfqpoint{1.704761in}{1.163861in}}%
\pgfpathlineto{\pgfqpoint{1.706751in}{1.161913in}}%
\pgfpathlineto{\pgfqpoint{1.714735in}{1.148302in}}%
\pgfpathlineto{\pgfqpoint{1.715500in}{1.134691in}}%
\pgfpathlineto{\pgfqpoint{1.708852in}{1.121079in}}%
\pgfpathlineto{\pgfqpoint{1.704761in}{1.116747in}}%
\pgfpathlineto{\pgfqpoint{1.697108in}{1.107468in}}%
\pgfpathlineto{\pgfqpoint{1.689104in}{1.100072in}}%
\pgfpathlineto{\pgfqpoint{1.682692in}{1.093857in}}%
\pgfpathlineto{\pgfqpoint{1.673448in}{1.085760in}}%
\pgfpathlineto{\pgfqpoint{1.666814in}{1.080246in}}%
\pgfpathlineto{\pgfqpoint{1.657791in}{1.072668in}}%
\pgfpathlineto{\pgfqpoint{1.649087in}{1.066635in}}%
\pgfpathlineto{\pgfqpoint{1.642135in}{1.061307in}}%
\pgfpathlineto{\pgfqpoint{1.626478in}{1.053651in}}%
\pgfpathlineto{\pgfqpoint{1.621272in}{1.053024in}}%
\pgfpathlineto{\pgfqpoint{1.610822in}{1.051539in}}%
\pgfpathlineto{\pgfqpoint{1.606313in}{1.053024in}}%
\pgfpathclose%
\pgfpathmoveto{\pgfqpoint{0.498466in}{1.175524in}}%
\pgfpathlineto{\pgfqpoint{0.483549in}{1.183617in}}%
\pgfpathlineto{\pgfqpoint{0.476089in}{1.189135in}}%
\pgfpathlineto{\pgfqpoint{0.467892in}{1.194668in}}%
\pgfpathlineto{\pgfqpoint{0.457910in}{1.202746in}}%
\pgfpathlineto{\pgfqpoint{0.452236in}{1.207476in}}%
\pgfpathlineto{\pgfqpoint{0.442332in}{1.216357in}}%
\pgfpathlineto{\pgfqpoint{0.436579in}{1.222276in}}%
\pgfpathlineto{\pgfqpoint{0.428796in}{1.229968in}}%
\pgfpathlineto{\pgfqpoint{0.420923in}{1.239998in}}%
\pgfpathlineto{\pgfqpoint{0.417550in}{1.243579in}}%
\pgfpathlineto{\pgfqpoint{0.409001in}{1.257191in}}%
\pgfpathlineto{\pgfqpoint{0.405266in}{1.270598in}}%
\pgfpathlineto{\pgfqpoint{0.405179in}{1.270802in}}%
\pgfpathlineto{\pgfqpoint{0.405266in}{1.271609in}}%
\pgfpathlineto{\pgfqpoint{0.406157in}{1.284413in}}%
\pgfpathlineto{\pgfqpoint{0.411849in}{1.298024in}}%
\pgfpathlineto{\pgfqpoint{0.420923in}{1.309900in}}%
\pgfpathlineto{\pgfqpoint{0.422017in}{1.311635in}}%
\pgfpathlineto{\pgfqpoint{0.434043in}{1.325246in}}%
\pgfpathlineto{\pgfqpoint{0.436579in}{1.327602in}}%
\pgfpathlineto{\pgfqpoint{0.448266in}{1.338857in}}%
\pgfpathlineto{\pgfqpoint{0.452236in}{1.342331in}}%
\pgfpathlineto{\pgfqpoint{0.464721in}{1.352468in}}%
\pgfpathlineto{\pgfqpoint{0.467892in}{1.355075in}}%
\pgfpathlineto{\pgfqpoint{0.483549in}{1.365837in}}%
\pgfpathlineto{\pgfqpoint{0.484025in}{1.366079in}}%
\pgfpathlineto{\pgfqpoint{0.499205in}{1.375128in}}%
\pgfpathlineto{\pgfqpoint{0.512271in}{1.379691in}}%
\pgfpathlineto{\pgfqpoint{0.514862in}{1.380990in}}%
\pgfpathlineto{\pgfqpoint{0.530519in}{1.383277in}}%
\pgfpathlineto{\pgfqpoint{0.546175in}{1.379845in}}%
\pgfpathlineto{\pgfqpoint{0.546447in}{1.379691in}}%
\pgfpathlineto{\pgfqpoint{0.561832in}{1.373570in}}%
\pgfpathlineto{\pgfqpoint{0.573556in}{1.366079in}}%
\pgfpathlineto{\pgfqpoint{0.577488in}{1.363929in}}%
\pgfpathlineto{\pgfqpoint{0.593145in}{1.352569in}}%
\pgfpathlineto{\pgfqpoint{0.593264in}{1.352468in}}%
\pgfpathlineto{\pgfqpoint{0.608801in}{1.339560in}}%
\pgfpathlineto{\pgfqpoint{0.609602in}{1.338857in}}%
\pgfpathlineto{\pgfqpoint{0.623743in}{1.325246in}}%
\pgfpathlineto{\pgfqpoint{0.624458in}{1.324405in}}%
\pgfpathlineto{\pgfqpoint{0.636095in}{1.311635in}}%
\pgfpathlineto{\pgfqpoint{0.640115in}{1.305398in}}%
\pgfpathlineto{\pgfqpoint{0.646060in}{1.298024in}}%
\pgfpathlineto{\pgfqpoint{0.652146in}{1.284413in}}%
\pgfpathlineto{\pgfqpoint{0.653158in}{1.270802in}}%
\pgfpathlineto{\pgfqpoint{0.649105in}{1.257191in}}%
\pgfpathlineto{\pgfqpoint{0.640115in}{1.243794in}}%
\pgfpathlineto{\pgfqpoint{0.640000in}{1.243579in}}%
\pgfpathlineto{\pgfqpoint{0.629079in}{1.229968in}}%
\pgfpathlineto{\pgfqpoint{0.624458in}{1.225491in}}%
\pgfpathlineto{\pgfqpoint{0.615605in}{1.216357in}}%
\pgfpathlineto{\pgfqpoint{0.608801in}{1.210240in}}%
\pgfpathlineto{\pgfqpoint{0.600016in}{1.202746in}}%
\pgfpathlineto{\pgfqpoint{0.593145in}{1.197065in}}%
\pgfpathlineto{\pgfqpoint{0.581901in}{1.189135in}}%
\pgfpathlineto{\pgfqpoint{0.577488in}{1.185733in}}%
\pgfpathlineto{\pgfqpoint{0.561832in}{1.176552in}}%
\pgfpathlineto{\pgfqpoint{0.559047in}{1.175524in}}%
\pgfpathlineto{\pgfqpoint{0.546175in}{1.169294in}}%
\pgfpathlineto{\pgfqpoint{0.530519in}{1.166444in}}%
\pgfpathlineto{\pgfqpoint{0.514862in}{1.168343in}}%
\pgfpathlineto{\pgfqpoint{0.499205in}{1.175005in}}%
\pgfpathlineto{\pgfqpoint{0.498466in}{1.175524in}}%
\pgfpathclose%
\pgfpathmoveto{\pgfqpoint{0.808898in}{1.175524in}}%
\pgfpathlineto{\pgfqpoint{0.796680in}{1.181640in}}%
\pgfpathlineto{\pgfqpoint{0.786071in}{1.189135in}}%
\pgfpathlineto{\pgfqpoint{0.781024in}{1.192380in}}%
\pgfpathlineto{\pgfqpoint{0.767887in}{1.202746in}}%
\pgfpathlineto{\pgfqpoint{0.765367in}{1.204791in}}%
\pgfpathlineto{\pgfqpoint{0.752384in}{1.216357in}}%
\pgfpathlineto{\pgfqpoint{0.749710in}{1.219089in}}%
\pgfpathlineto{\pgfqpoint{0.738865in}{1.229968in}}%
\pgfpathlineto{\pgfqpoint{0.734054in}{1.236195in}}%
\pgfpathlineto{\pgfqpoint{0.727409in}{1.243579in}}%
\pgfpathlineto{\pgfqpoint{0.719350in}{1.257191in}}%
\pgfpathlineto{\pgfqpoint{0.718397in}{1.260777in}}%
\pgfpathlineto{\pgfqpoint{0.714535in}{1.270802in}}%
\pgfpathlineto{\pgfqpoint{0.715850in}{1.284413in}}%
\pgfpathlineto{\pgfqpoint{0.718397in}{1.288865in}}%
\pgfpathlineto{\pgfqpoint{0.722034in}{1.298024in}}%
\pgfpathlineto{\pgfqpoint{0.731888in}{1.311635in}}%
\pgfpathlineto{\pgfqpoint{0.734054in}{1.313797in}}%
\pgfpathlineto{\pgfqpoint{0.743960in}{1.325246in}}%
\pgfpathlineto{\pgfqpoint{0.749710in}{1.330668in}}%
\pgfpathlineto{\pgfqpoint{0.758275in}{1.338857in}}%
\pgfpathlineto{\pgfqpoint{0.765367in}{1.345023in}}%
\pgfpathlineto{\pgfqpoint{0.774787in}{1.352468in}}%
\pgfpathlineto{\pgfqpoint{0.781024in}{1.357467in}}%
\pgfpathlineto{\pgfqpoint{0.794194in}{1.366079in}}%
\pgfpathlineto{\pgfqpoint{0.796680in}{1.367962in}}%
\pgfpathlineto{\pgfqpoint{0.812337in}{1.376529in}}%
\pgfpathlineto{\pgfqpoint{0.822873in}{1.379691in}}%
\pgfpathlineto{\pgfqpoint{0.827993in}{1.381905in}}%
\pgfpathlineto{\pgfqpoint{0.843650in}{1.383048in}}%
\pgfpathlineto{\pgfqpoint{0.855181in}{1.379691in}}%
\pgfpathlineto{\pgfqpoint{0.859306in}{1.378863in}}%
\pgfpathlineto{\pgfqpoint{0.874963in}{1.371857in}}%
\pgfpathlineto{\pgfqpoint{0.883457in}{1.366079in}}%
\pgfpathlineto{\pgfqpoint{0.890620in}{1.361897in}}%
\pgfpathlineto{\pgfqpoint{0.903134in}{1.352468in}}%
\pgfpathlineto{\pgfqpoint{0.906276in}{1.350144in}}%
\pgfpathlineto{\pgfqpoint{0.919581in}{1.338857in}}%
\pgfpathlineto{\pgfqpoint{0.921933in}{1.336666in}}%
\pgfpathlineto{\pgfqpoint{0.933857in}{1.325246in}}%
\pgfpathlineto{\pgfqpoint{0.937589in}{1.320859in}}%
\pgfpathlineto{\pgfqpoint{0.946211in}{1.311635in}}%
\pgfpathlineto{\pgfqpoint{0.953246in}{1.301014in}}%
\pgfpathlineto{\pgfqpoint{0.955814in}{1.298024in}}%
\pgfpathlineto{\pgfqpoint{0.962380in}{1.284413in}}%
\pgfpathlineto{\pgfqpoint{0.963472in}{1.270802in}}%
\pgfpathlineto{\pgfqpoint{0.959099in}{1.257191in}}%
\pgfpathlineto{\pgfqpoint{0.953246in}{1.249005in}}%
\pgfpathlineto{\pgfqpoint{0.950275in}{1.243579in}}%
\pgfpathlineto{\pgfqpoint{0.938911in}{1.229968in}}%
\pgfpathlineto{\pgfqpoint{0.937589in}{1.228720in}}%
\pgfpathlineto{\pgfqpoint{0.925589in}{1.216357in}}%
\pgfpathlineto{\pgfqpoint{0.921933in}{1.213074in}}%
\pgfpathlineto{\pgfqpoint{0.910053in}{1.202746in}}%
\pgfpathlineto{\pgfqpoint{0.906276in}{1.199567in}}%
\pgfpathlineto{\pgfqpoint{0.892056in}{1.189135in}}%
\pgfpathlineto{\pgfqpoint{0.890620in}{1.187986in}}%
\pgfpathlineto{\pgfqpoint{0.874963in}{1.178107in}}%
\pgfpathlineto{\pgfqpoint{0.868722in}{1.175524in}}%
\pgfpathlineto{\pgfqpoint{0.859306in}{1.170435in}}%
\pgfpathlineto{\pgfqpoint{0.843650in}{1.166634in}}%
\pgfpathlineto{\pgfqpoint{0.827993in}{1.167583in}}%
\pgfpathlineto{\pgfqpoint{0.812337in}{1.173291in}}%
\pgfpathlineto{\pgfqpoint{0.808898in}{1.175524in}}%
\pgfpathclose%
\pgfpathmoveto{\pgfqpoint{1.119161in}{1.175524in}}%
\pgfpathlineto{\pgfqpoint{1.109812in}{1.179803in}}%
\pgfpathlineto{\pgfqpoint{1.095910in}{1.189135in}}%
\pgfpathlineto{\pgfqpoint{1.094155in}{1.190205in}}%
\pgfpathlineto{\pgfqpoint{1.078498in}{1.202170in}}%
\pgfpathlineto{\pgfqpoint{1.077824in}{1.202746in}}%
\pgfpathlineto{\pgfqpoint{1.062842in}{1.215969in}}%
\pgfpathlineto{\pgfqpoint{1.062408in}{1.216357in}}%
\pgfpathlineto{\pgfqpoint{1.049001in}{1.229968in}}%
\pgfpathlineto{\pgfqpoint{1.047185in}{1.232344in}}%
\pgfpathlineto{\pgfqpoint{1.037452in}{1.243579in}}%
\pgfpathlineto{\pgfqpoint{1.031529in}{1.254039in}}%
\pgfpathlineto{\pgfqpoint{1.029101in}{1.257191in}}%
\pgfpathlineto{\pgfqpoint{1.024332in}{1.270802in}}%
\pgfpathlineto{\pgfqpoint{1.025524in}{1.284413in}}%
\pgfpathlineto{\pgfqpoint{1.031529in}{1.295871in}}%
\pgfpathlineto{\pgfqpoint{1.032349in}{1.298024in}}%
\pgfpathlineto{\pgfqpoint{1.041705in}{1.311635in}}%
\pgfpathlineto{\pgfqpoint{1.047185in}{1.317316in}}%
\pgfpathlineto{\pgfqpoint{1.053973in}{1.325246in}}%
\pgfpathlineto{\pgfqpoint{1.062842in}{1.333693in}}%
\pgfpathlineto{\pgfqpoint{1.068303in}{1.338857in}}%
\pgfpathlineto{\pgfqpoint{1.078498in}{1.347630in}}%
\pgfpathlineto{\pgfqpoint{1.084813in}{1.352468in}}%
\pgfpathlineto{\pgfqpoint{1.094155in}{1.359742in}}%
\pgfpathlineto{\pgfqpoint{1.104382in}{1.366079in}}%
\pgfpathlineto{\pgfqpoint{1.109812in}{1.369987in}}%
\pgfpathlineto{\pgfqpoint{1.125468in}{1.377774in}}%
\pgfpathlineto{\pgfqpoint{1.133119in}{1.379691in}}%
\pgfpathlineto{\pgfqpoint{1.141125in}{1.382591in}}%
\pgfpathlineto{\pgfqpoint{1.156781in}{1.382591in}}%
\pgfpathlineto{\pgfqpoint{1.164787in}{1.379691in}}%
\pgfpathlineto{\pgfqpoint{1.172438in}{1.377774in}}%
\pgfpathlineto{\pgfqpoint{1.188094in}{1.369987in}}%
\pgfpathlineto{\pgfqpoint{1.193524in}{1.366079in}}%
\pgfpathlineto{\pgfqpoint{1.203751in}{1.359742in}}%
\pgfpathlineto{\pgfqpoint{1.213093in}{1.352468in}}%
\pgfpathlineto{\pgfqpoint{1.219407in}{1.347630in}}%
\pgfpathlineto{\pgfqpoint{1.229603in}{1.338857in}}%
\pgfpathlineto{\pgfqpoint{1.235064in}{1.333693in}}%
\pgfpathlineto{\pgfqpoint{1.243933in}{1.325246in}}%
\pgfpathlineto{\pgfqpoint{1.250721in}{1.317316in}}%
\pgfpathlineto{\pgfqpoint{1.256201in}{1.311635in}}%
\pgfpathlineto{\pgfqpoint{1.265556in}{1.298024in}}%
\pgfpathlineto{\pgfqpoint{1.266377in}{1.295871in}}%
\pgfpathlineto{\pgfqpoint{1.272382in}{1.284413in}}%
\pgfpathlineto{\pgfqpoint{1.273574in}{1.270802in}}%
\pgfpathlineto{\pgfqpoint{1.268805in}{1.257191in}}%
\pgfpathlineto{\pgfqpoint{1.266377in}{1.254039in}}%
\pgfpathlineto{\pgfqpoint{1.260453in}{1.243579in}}%
\pgfpathlineto{\pgfqpoint{1.250721in}{1.232344in}}%
\pgfpathlineto{\pgfqpoint{1.248905in}{1.229968in}}%
\pgfpathlineto{\pgfqpoint{1.235498in}{1.216357in}}%
\pgfpathlineto{\pgfqpoint{1.235064in}{1.215969in}}%
\pgfpathlineto{\pgfqpoint{1.220082in}{1.202746in}}%
\pgfpathlineto{\pgfqpoint{1.219407in}{1.202170in}}%
\pgfpathlineto{\pgfqpoint{1.203751in}{1.190205in}}%
\pgfpathlineto{\pgfqpoint{1.201996in}{1.189135in}}%
\pgfpathlineto{\pgfqpoint{1.188094in}{1.179803in}}%
\pgfpathlineto{\pgfqpoint{1.178745in}{1.175524in}}%
\pgfpathlineto{\pgfqpoint{1.172438in}{1.171768in}}%
\pgfpathlineto{\pgfqpoint{1.156781in}{1.167013in}}%
\pgfpathlineto{\pgfqpoint{1.141125in}{1.167013in}}%
\pgfpathlineto{\pgfqpoint{1.125468in}{1.171768in}}%
\pgfpathlineto{\pgfqpoint{1.119161in}{1.175524in}}%
\pgfpathclose%
\pgfpathmoveto{\pgfqpoint{1.429184in}{1.175524in}}%
\pgfpathlineto{\pgfqpoint{1.422943in}{1.178107in}}%
\pgfpathlineto{\pgfqpoint{1.407286in}{1.187986in}}%
\pgfpathlineto{\pgfqpoint{1.405850in}{1.189135in}}%
\pgfpathlineto{\pgfqpoint{1.391630in}{1.199567in}}%
\pgfpathlineto{\pgfqpoint{1.387853in}{1.202746in}}%
\pgfpathlineto{\pgfqpoint{1.375973in}{1.213074in}}%
\pgfpathlineto{\pgfqpoint{1.372317in}{1.216357in}}%
\pgfpathlineto{\pgfqpoint{1.360317in}{1.228720in}}%
\pgfpathlineto{\pgfqpoint{1.358995in}{1.229968in}}%
\pgfpathlineto{\pgfqpoint{1.347631in}{1.243579in}}%
\pgfpathlineto{\pgfqpoint{1.344660in}{1.249005in}}%
\pgfpathlineto{\pgfqpoint{1.338807in}{1.257191in}}%
\pgfpathlineto{\pgfqpoint{1.334434in}{1.270802in}}%
\pgfpathlineto{\pgfqpoint{1.335526in}{1.284413in}}%
\pgfpathlineto{\pgfqpoint{1.342091in}{1.298024in}}%
\pgfpathlineto{\pgfqpoint{1.344660in}{1.301014in}}%
\pgfpathlineto{\pgfqpoint{1.351695in}{1.311635in}}%
\pgfpathlineto{\pgfqpoint{1.360317in}{1.320859in}}%
\pgfpathlineto{\pgfqpoint{1.364049in}{1.325246in}}%
\pgfpathlineto{\pgfqpoint{1.375973in}{1.336666in}}%
\pgfpathlineto{\pgfqpoint{1.378325in}{1.338857in}}%
\pgfpathlineto{\pgfqpoint{1.391630in}{1.350144in}}%
\pgfpathlineto{\pgfqpoint{1.394772in}{1.352468in}}%
\pgfpathlineto{\pgfqpoint{1.407286in}{1.361897in}}%
\pgfpathlineto{\pgfqpoint{1.414449in}{1.366079in}}%
\pgfpathlineto{\pgfqpoint{1.422943in}{1.371857in}}%
\pgfpathlineto{\pgfqpoint{1.438599in}{1.378863in}}%
\pgfpathlineto{\pgfqpoint{1.442725in}{1.379691in}}%
\pgfpathlineto{\pgfqpoint{1.454256in}{1.383048in}}%
\pgfpathlineto{\pgfqpoint{1.469913in}{1.381905in}}%
\pgfpathlineto{\pgfqpoint{1.475033in}{1.379691in}}%
\pgfpathlineto{\pgfqpoint{1.485569in}{1.376529in}}%
\pgfpathlineto{\pgfqpoint{1.501226in}{1.367962in}}%
\pgfpathlineto{\pgfqpoint{1.503712in}{1.366079in}}%
\pgfpathlineto{\pgfqpoint{1.516882in}{1.357467in}}%
\pgfpathlineto{\pgfqpoint{1.523119in}{1.352468in}}%
\pgfpathlineto{\pgfqpoint{1.532539in}{1.345023in}}%
\pgfpathlineto{\pgfqpoint{1.539631in}{1.338857in}}%
\pgfpathlineto{\pgfqpoint{1.548195in}{1.330668in}}%
\pgfpathlineto{\pgfqpoint{1.553946in}{1.325246in}}%
\pgfpathlineto{\pgfqpoint{1.563852in}{1.313797in}}%
\pgfpathlineto{\pgfqpoint{1.566018in}{1.311635in}}%
\pgfpathlineto{\pgfqpoint{1.575872in}{1.298024in}}%
\pgfpathlineto{\pgfqpoint{1.579508in}{1.288865in}}%
\pgfpathlineto{\pgfqpoint{1.582056in}{1.284413in}}%
\pgfpathlineto{\pgfqpoint{1.583371in}{1.270802in}}%
\pgfpathlineto{\pgfqpoint{1.579508in}{1.260777in}}%
\pgfpathlineto{\pgfqpoint{1.578556in}{1.257191in}}%
\pgfpathlineto{\pgfqpoint{1.570497in}{1.243579in}}%
\pgfpathlineto{\pgfqpoint{1.563852in}{1.236195in}}%
\pgfpathlineto{\pgfqpoint{1.559041in}{1.229968in}}%
\pgfpathlineto{\pgfqpoint{1.548195in}{1.219089in}}%
\pgfpathlineto{\pgfqpoint{1.545522in}{1.216357in}}%
\pgfpathlineto{\pgfqpoint{1.532539in}{1.204791in}}%
\pgfpathlineto{\pgfqpoint{1.530019in}{1.202746in}}%
\pgfpathlineto{\pgfqpoint{1.516882in}{1.192380in}}%
\pgfpathlineto{\pgfqpoint{1.511835in}{1.189135in}}%
\pgfpathlineto{\pgfqpoint{1.501226in}{1.181640in}}%
\pgfpathlineto{\pgfqpoint{1.489008in}{1.175524in}}%
\pgfpathlineto{\pgfqpoint{1.485569in}{1.173291in}}%
\pgfpathlineto{\pgfqpoint{1.469913in}{1.167583in}}%
\pgfpathlineto{\pgfqpoint{1.454256in}{1.166634in}}%
\pgfpathlineto{\pgfqpoint{1.438599in}{1.170435in}}%
\pgfpathlineto{\pgfqpoint{1.429184in}{1.175524in}}%
\pgfpathclose%
\pgfpathmoveto{\pgfqpoint{1.738858in}{1.175524in}}%
\pgfpathlineto{\pgfqpoint{1.736074in}{1.176552in}}%
\pgfpathlineto{\pgfqpoint{1.720418in}{1.185733in}}%
\pgfpathlineto{\pgfqpoint{1.716004in}{1.189135in}}%
\pgfpathlineto{\pgfqpoint{1.704761in}{1.197065in}}%
\pgfpathlineto{\pgfqpoint{1.697890in}{1.202746in}}%
\pgfpathlineto{\pgfqpoint{1.689104in}{1.210240in}}%
\pgfpathlineto{\pgfqpoint{1.682301in}{1.216357in}}%
\pgfpathlineto{\pgfqpoint{1.673448in}{1.225491in}}%
\pgfpathlineto{\pgfqpoint{1.668827in}{1.229968in}}%
\pgfpathlineto{\pgfqpoint{1.657906in}{1.243579in}}%
\pgfpathlineto{\pgfqpoint{1.657791in}{1.243794in}}%
\pgfpathlineto{\pgfqpoint{1.648801in}{1.257191in}}%
\pgfpathlineto{\pgfqpoint{1.644747in}{1.270802in}}%
\pgfpathlineto{\pgfqpoint{1.645760in}{1.284413in}}%
\pgfpathlineto{\pgfqpoint{1.651845in}{1.298024in}}%
\pgfpathlineto{\pgfqpoint{1.657791in}{1.305398in}}%
\pgfpathlineto{\pgfqpoint{1.661811in}{1.311635in}}%
\pgfpathlineto{\pgfqpoint{1.673448in}{1.324405in}}%
\pgfpathlineto{\pgfqpoint{1.674163in}{1.325246in}}%
\pgfpathlineto{\pgfqpoint{1.688304in}{1.338857in}}%
\pgfpathlineto{\pgfqpoint{1.689104in}{1.339560in}}%
\pgfpathlineto{\pgfqpoint{1.704642in}{1.352468in}}%
\pgfpathlineto{\pgfqpoint{1.704761in}{1.352569in}}%
\pgfpathlineto{\pgfqpoint{1.720418in}{1.363929in}}%
\pgfpathlineto{\pgfqpoint{1.724349in}{1.366079in}}%
\pgfpathlineto{\pgfqpoint{1.736074in}{1.373570in}}%
\pgfpathlineto{\pgfqpoint{1.751459in}{1.379691in}}%
\pgfpathlineto{\pgfqpoint{1.751731in}{1.379845in}}%
\pgfpathlineto{\pgfqpoint{1.767387in}{1.383277in}}%
\pgfpathlineto{\pgfqpoint{1.783044in}{1.380990in}}%
\pgfpathlineto{\pgfqpoint{1.785635in}{1.379691in}}%
\pgfpathlineto{\pgfqpoint{1.798700in}{1.375128in}}%
\pgfpathlineto{\pgfqpoint{1.813881in}{1.366079in}}%
\pgfpathlineto{\pgfqpoint{1.814357in}{1.365837in}}%
\pgfpathlineto{\pgfqpoint{1.830014in}{1.355075in}}%
\pgfpathlineto{\pgfqpoint{1.833185in}{1.352468in}}%
\pgfpathlineto{\pgfqpoint{1.845670in}{1.342331in}}%
\pgfpathlineto{\pgfqpoint{1.849639in}{1.338857in}}%
\pgfpathlineto{\pgfqpoint{1.861327in}{1.327602in}}%
\pgfpathlineto{\pgfqpoint{1.863863in}{1.325246in}}%
\pgfpathlineto{\pgfqpoint{1.875889in}{1.311635in}}%
\pgfpathlineto{\pgfqpoint{1.876983in}{1.309900in}}%
\pgfpathlineto{\pgfqpoint{1.886057in}{1.298024in}}%
\pgfpathlineto{\pgfqpoint{1.891749in}{1.284413in}}%
\pgfpathlineto{\pgfqpoint{1.892640in}{1.271609in}}%
\pgfpathlineto{\pgfqpoint{1.892727in}{1.270802in}}%
\pgfpathlineto{\pgfqpoint{1.892640in}{1.270598in}}%
\pgfpathlineto{\pgfqpoint{1.888904in}{1.257191in}}%
\pgfpathlineto{\pgfqpoint{1.880356in}{1.243579in}}%
\pgfpathlineto{\pgfqpoint{1.876983in}{1.239998in}}%
\pgfpathlineto{\pgfqpoint{1.869109in}{1.229968in}}%
\pgfpathlineto{\pgfqpoint{1.861327in}{1.222276in}}%
\pgfpathlineto{\pgfqpoint{1.855574in}{1.216357in}}%
\pgfpathlineto{\pgfqpoint{1.845670in}{1.207476in}}%
\pgfpathlineto{\pgfqpoint{1.839996in}{1.202746in}}%
\pgfpathlineto{\pgfqpoint{1.830014in}{1.194668in}}%
\pgfpathlineto{\pgfqpoint{1.821817in}{1.189135in}}%
\pgfpathlineto{\pgfqpoint{1.814357in}{1.183617in}}%
\pgfpathlineto{\pgfqpoint{1.799440in}{1.175524in}}%
\pgfpathlineto{\pgfqpoint{1.798700in}{1.175005in}}%
\pgfpathlineto{\pgfqpoint{1.783044in}{1.168343in}}%
\pgfpathlineto{\pgfqpoint{1.767387in}{1.166444in}}%
\pgfpathlineto{\pgfqpoint{1.751731in}{1.169294in}}%
\pgfpathlineto{\pgfqpoint{1.738858in}{1.175524in}}%
\pgfpathclose%
\pgfpathmoveto{\pgfqpoint{0.667486in}{1.325246in}}%
\pgfpathlineto{\pgfqpoint{0.655771in}{1.330827in}}%
\pgfpathlineto{\pgfqpoint{0.644962in}{1.338857in}}%
\pgfpathlineto{\pgfqpoint{0.640115in}{1.342183in}}%
\pgfpathlineto{\pgfqpoint{0.627731in}{1.352468in}}%
\pgfpathlineto{\pgfqpoint{0.624458in}{1.355216in}}%
\pgfpathlineto{\pgfqpoint{0.612189in}{1.366079in}}%
\pgfpathlineto{\pgfqpoint{0.608801in}{1.369490in}}%
\pgfpathlineto{\pgfqpoint{0.598203in}{1.379691in}}%
\pgfpathlineto{\pgfqpoint{0.593145in}{1.386386in}}%
\pgfpathlineto{\pgfqpoint{0.587187in}{1.393302in}}%
\pgfpathlineto{\pgfqpoint{0.581943in}{1.406913in}}%
\pgfpathlineto{\pgfqpoint{0.584231in}{1.420524in}}%
\pgfpathlineto{\pgfqpoint{0.593145in}{1.433638in}}%
\pgfpathlineto{\pgfqpoint{0.593444in}{1.434135in}}%
\pgfpathlineto{\pgfqpoint{0.606315in}{1.447746in}}%
\pgfpathlineto{\pgfqpoint{0.608801in}{1.449909in}}%
\pgfpathlineto{\pgfqpoint{0.621306in}{1.461357in}}%
\pgfpathlineto{\pgfqpoint{0.624458in}{1.464098in}}%
\pgfpathlineto{\pgfqpoint{0.637627in}{1.474968in}}%
\pgfpathlineto{\pgfqpoint{0.640115in}{1.477130in}}%
\pgfpathlineto{\pgfqpoint{0.655771in}{1.488320in}}%
\pgfpathlineto{\pgfqpoint{0.656342in}{1.488579in}}%
\pgfpathlineto{\pgfqpoint{0.671428in}{1.496329in}}%
\pgfpathlineto{\pgfqpoint{0.687084in}{1.498318in}}%
\pgfpathlineto{\pgfqpoint{0.702741in}{1.493759in}}%
\pgfpathlineto{\pgfqpoint{0.710696in}{1.488579in}}%
\pgfpathlineto{\pgfqpoint{0.718397in}{1.484182in}}%
\pgfpathlineto{\pgfqpoint{0.730130in}{1.474968in}}%
\pgfpathlineto{\pgfqpoint{0.734054in}{1.472023in}}%
\pgfpathlineto{\pgfqpoint{0.746549in}{1.461357in}}%
\pgfpathlineto{\pgfqpoint{0.749710in}{1.458512in}}%
\pgfpathlineto{\pgfqpoint{0.761542in}{1.447746in}}%
\pgfpathlineto{\pgfqpoint{0.765367in}{1.443532in}}%
\pgfpathlineto{\pgfqpoint{0.774604in}{1.434135in}}%
\pgfpathlineto{\pgfqpoint{0.781024in}{1.423950in}}%
\pgfpathlineto{\pgfqpoint{0.783578in}{1.420524in}}%
\pgfpathlineto{\pgfqpoint{0.786026in}{1.406913in}}%
\pgfpathlineto{\pgfqpoint{0.781024in}{1.394803in}}%
\pgfpathlineto{\pgfqpoint{0.780491in}{1.393302in}}%
\pgfpathlineto{\pgfqpoint{0.769740in}{1.379691in}}%
\pgfpathlineto{\pgfqpoint{0.765367in}{1.375687in}}%
\pgfpathlineto{\pgfqpoint{0.755736in}{1.366079in}}%
\pgfpathlineto{\pgfqpoint{0.749710in}{1.360797in}}%
\pgfpathlineto{\pgfqpoint{0.740130in}{1.352468in}}%
\pgfpathlineto{\pgfqpoint{0.734054in}{1.347230in}}%
\pgfpathlineto{\pgfqpoint{0.723002in}{1.338857in}}%
\pgfpathlineto{\pgfqpoint{0.718397in}{1.335056in}}%
\pgfpathlineto{\pgfqpoint{0.702741in}{1.325709in}}%
\pgfpathlineto{\pgfqpoint{0.701013in}{1.325246in}}%
\pgfpathlineto{\pgfqpoint{0.687084in}{1.320898in}}%
\pgfpathlineto{\pgfqpoint{0.671428in}{1.323025in}}%
\pgfpathlineto{\pgfqpoint{0.667486in}{1.325246in}}%
\pgfpathclose%
\pgfpathmoveto{\pgfqpoint{0.977914in}{1.325246in}}%
\pgfpathlineto{\pgfqpoint{0.968902in}{1.328941in}}%
\pgfpathlineto{\pgfqpoint{0.954630in}{1.338857in}}%
\pgfpathlineto{\pgfqpoint{0.953246in}{1.339748in}}%
\pgfpathlineto{\pgfqpoint{0.937589in}{1.352407in}}%
\pgfpathlineto{\pgfqpoint{0.937520in}{1.352468in}}%
\pgfpathlineto{\pgfqpoint{0.922120in}{1.366079in}}%
\pgfpathlineto{\pgfqpoint{0.921933in}{1.366267in}}%
\pgfpathlineto{\pgfqpoint{0.908249in}{1.379691in}}%
\pgfpathlineto{\pgfqpoint{0.906276in}{1.382348in}}%
\pgfpathlineto{\pgfqpoint{0.897204in}{1.393302in}}%
\pgfpathlineto{\pgfqpoint{0.892097in}{1.406913in}}%
\pgfpathlineto{\pgfqpoint{0.894325in}{1.420524in}}%
\pgfpathlineto{\pgfqpoint{0.903325in}{1.434135in}}%
\pgfpathlineto{\pgfqpoint{0.906276in}{1.436945in}}%
\pgfpathlineto{\pgfqpoint{0.916322in}{1.447746in}}%
\pgfpathlineto{\pgfqpoint{0.921933in}{1.452719in}}%
\pgfpathlineto{\pgfqpoint{0.931380in}{1.461357in}}%
\pgfpathlineto{\pgfqpoint{0.937589in}{1.466747in}}%
\pgfpathlineto{\pgfqpoint{0.947805in}{1.474968in}}%
\pgfpathlineto{\pgfqpoint{0.953246in}{1.479570in}}%
\pgfpathlineto{\pgfqpoint{0.966661in}{1.488579in}}%
\pgfpathlineto{\pgfqpoint{0.968902in}{1.490310in}}%
\pgfpathlineto{\pgfqpoint{0.984559in}{1.497250in}}%
\pgfpathlineto{\pgfqpoint{1.000216in}{1.497915in}}%
\pgfpathlineto{\pgfqpoint{1.015872in}{1.492136in}}%
\pgfpathlineto{\pgfqpoint{1.020855in}{1.488579in}}%
\pgfpathlineto{\pgfqpoint{1.031529in}{1.481926in}}%
\pgfpathlineto{\pgfqpoint{1.040036in}{1.474968in}}%
\pgfpathlineto{\pgfqpoint{1.047185in}{1.469394in}}%
\pgfpathlineto{\pgfqpoint{1.056499in}{1.461357in}}%
\pgfpathlineto{\pgfqpoint{1.062842in}{1.455590in}}%
\pgfpathlineto{\pgfqpoint{1.071559in}{1.447746in}}%
\pgfpathlineto{\pgfqpoint{1.078498in}{1.440179in}}%
\pgfpathlineto{\pgfqpoint{1.084627in}{1.434135in}}%
\pgfpathlineto{\pgfqpoint{1.093434in}{1.420524in}}%
\pgfpathlineto{\pgfqpoint{1.094155in}{1.415998in}}%
\pgfpathlineto{\pgfqpoint{1.095863in}{1.406913in}}%
\pgfpathlineto{\pgfqpoint{1.094155in}{1.402993in}}%
\pgfpathlineto{\pgfqpoint{1.090616in}{1.393302in}}%
\pgfpathlineto{\pgfqpoint{1.079678in}{1.379691in}}%
\pgfpathlineto{\pgfqpoint{1.078498in}{1.378643in}}%
\pgfpathlineto{\pgfqpoint{1.065772in}{1.366079in}}%
\pgfpathlineto{\pgfqpoint{1.062842in}{1.363539in}}%
\pgfpathlineto{\pgfqpoint{1.050235in}{1.352468in}}%
\pgfpathlineto{\pgfqpoint{1.047185in}{1.349810in}}%
\pgfpathlineto{\pgfqpoint{1.033269in}{1.338857in}}%
\pgfpathlineto{\pgfqpoint{1.031529in}{1.337361in}}%
\pgfpathlineto{\pgfqpoint{1.015872in}{1.327229in}}%
\pgfpathlineto{\pgfqpoint{1.010047in}{1.325246in}}%
\pgfpathlineto{\pgfqpoint{1.000216in}{1.321328in}}%
\pgfpathlineto{\pgfqpoint{0.984559in}{1.322040in}}%
\pgfpathlineto{\pgfqpoint{0.977914in}{1.325246in}}%
\pgfpathclose%
\pgfpathmoveto{\pgfqpoint{1.287859in}{1.325246in}}%
\pgfpathlineto{\pgfqpoint{1.282034in}{1.327229in}}%
\pgfpathlineto{\pgfqpoint{1.266377in}{1.337361in}}%
\pgfpathlineto{\pgfqpoint{1.264637in}{1.338857in}}%
\pgfpathlineto{\pgfqpoint{1.250721in}{1.349810in}}%
\pgfpathlineto{\pgfqpoint{1.247671in}{1.352468in}}%
\pgfpathlineto{\pgfqpoint{1.235064in}{1.363539in}}%
\pgfpathlineto{\pgfqpoint{1.232134in}{1.366079in}}%
\pgfpathlineto{\pgfqpoint{1.219407in}{1.378643in}}%
\pgfpathlineto{\pgfqpoint{1.218228in}{1.379691in}}%
\pgfpathlineto{\pgfqpoint{1.207290in}{1.393302in}}%
\pgfpathlineto{\pgfqpoint{1.203751in}{1.402993in}}%
\pgfpathlineto{\pgfqpoint{1.202042in}{1.406913in}}%
\pgfpathlineto{\pgfqpoint{1.203751in}{1.415998in}}%
\pgfpathlineto{\pgfqpoint{1.204472in}{1.420524in}}%
\pgfpathlineto{\pgfqpoint{1.213279in}{1.434135in}}%
\pgfpathlineto{\pgfqpoint{1.219407in}{1.440179in}}%
\pgfpathlineto{\pgfqpoint{1.226347in}{1.447746in}}%
\pgfpathlineto{\pgfqpoint{1.235064in}{1.455590in}}%
\pgfpathlineto{\pgfqpoint{1.241407in}{1.461357in}}%
\pgfpathlineto{\pgfqpoint{1.250721in}{1.469394in}}%
\pgfpathlineto{\pgfqpoint{1.257869in}{1.474968in}}%
\pgfpathlineto{\pgfqpoint{1.266377in}{1.481926in}}%
\pgfpathlineto{\pgfqpoint{1.277051in}{1.488579in}}%
\pgfpathlineto{\pgfqpoint{1.282034in}{1.492136in}}%
\pgfpathlineto{\pgfqpoint{1.297690in}{1.497915in}}%
\pgfpathlineto{\pgfqpoint{1.313347in}{1.497250in}}%
\pgfpathlineto{\pgfqpoint{1.329003in}{1.490310in}}%
\pgfpathlineto{\pgfqpoint{1.331245in}{1.488579in}}%
\pgfpathlineto{\pgfqpoint{1.344660in}{1.479570in}}%
\pgfpathlineto{\pgfqpoint{1.350100in}{1.474968in}}%
\pgfpathlineto{\pgfqpoint{1.360317in}{1.466747in}}%
\pgfpathlineto{\pgfqpoint{1.366525in}{1.461357in}}%
\pgfpathlineto{\pgfqpoint{1.375973in}{1.452719in}}%
\pgfpathlineto{\pgfqpoint{1.381584in}{1.447746in}}%
\pgfpathlineto{\pgfqpoint{1.391630in}{1.436945in}}%
\pgfpathlineto{\pgfqpoint{1.394581in}{1.434135in}}%
\pgfpathlineto{\pgfqpoint{1.403581in}{1.420524in}}%
\pgfpathlineto{\pgfqpoint{1.405809in}{1.406913in}}%
\pgfpathlineto{\pgfqpoint{1.400701in}{1.393302in}}%
\pgfpathlineto{\pgfqpoint{1.391630in}{1.382348in}}%
\pgfpathlineto{\pgfqpoint{1.389657in}{1.379691in}}%
\pgfpathlineto{\pgfqpoint{1.375973in}{1.366267in}}%
\pgfpathlineto{\pgfqpoint{1.375786in}{1.366079in}}%
\pgfpathlineto{\pgfqpoint{1.360386in}{1.352468in}}%
\pgfpathlineto{\pgfqpoint{1.360317in}{1.352407in}}%
\pgfpathlineto{\pgfqpoint{1.344660in}{1.339748in}}%
\pgfpathlineto{\pgfqpoint{1.343276in}{1.338857in}}%
\pgfpathlineto{\pgfqpoint{1.329003in}{1.328941in}}%
\pgfpathlineto{\pgfqpoint{1.319992in}{1.325246in}}%
\pgfpathlineto{\pgfqpoint{1.313347in}{1.322040in}}%
\pgfpathlineto{\pgfqpoint{1.297690in}{1.321328in}}%
\pgfpathlineto{\pgfqpoint{1.287859in}{1.325246in}}%
\pgfpathclose%
\pgfpathmoveto{\pgfqpoint{1.596892in}{1.325246in}}%
\pgfpathlineto{\pgfqpoint{1.595165in}{1.325709in}}%
\pgfpathlineto{\pgfqpoint{1.579508in}{1.335056in}}%
\pgfpathlineto{\pgfqpoint{1.574903in}{1.338857in}}%
\pgfpathlineto{\pgfqpoint{1.563852in}{1.347230in}}%
\pgfpathlineto{\pgfqpoint{1.557776in}{1.352468in}}%
\pgfpathlineto{\pgfqpoint{1.548195in}{1.360797in}}%
\pgfpathlineto{\pgfqpoint{1.542170in}{1.366079in}}%
\pgfpathlineto{\pgfqpoint{1.532539in}{1.375687in}}%
\pgfpathlineto{\pgfqpoint{1.528166in}{1.379691in}}%
\pgfpathlineto{\pgfqpoint{1.517415in}{1.393302in}}%
\pgfpathlineto{\pgfqpoint{1.516882in}{1.394803in}}%
\pgfpathlineto{\pgfqpoint{1.511880in}{1.406913in}}%
\pgfpathlineto{\pgfqpoint{1.514328in}{1.420524in}}%
\pgfpathlineto{\pgfqpoint{1.516882in}{1.423950in}}%
\pgfpathlineto{\pgfqpoint{1.523302in}{1.434135in}}%
\pgfpathlineto{\pgfqpoint{1.532539in}{1.443532in}}%
\pgfpathlineto{\pgfqpoint{1.536364in}{1.447746in}}%
\pgfpathlineto{\pgfqpoint{1.548195in}{1.458512in}}%
\pgfpathlineto{\pgfqpoint{1.551356in}{1.461357in}}%
\pgfpathlineto{\pgfqpoint{1.563852in}{1.472023in}}%
\pgfpathlineto{\pgfqpoint{1.567776in}{1.474968in}}%
\pgfpathlineto{\pgfqpoint{1.579508in}{1.484182in}}%
\pgfpathlineto{\pgfqpoint{1.587210in}{1.488579in}}%
\pgfpathlineto{\pgfqpoint{1.595165in}{1.493759in}}%
\pgfpathlineto{\pgfqpoint{1.610822in}{1.498318in}}%
\pgfpathlineto{\pgfqpoint{1.626478in}{1.496329in}}%
\pgfpathlineto{\pgfqpoint{1.641563in}{1.488579in}}%
\pgfpathlineto{\pgfqpoint{1.642135in}{1.488320in}}%
\pgfpathlineto{\pgfqpoint{1.657791in}{1.477130in}}%
\pgfpathlineto{\pgfqpoint{1.660279in}{1.474968in}}%
\pgfpathlineto{\pgfqpoint{1.673448in}{1.464098in}}%
\pgfpathlineto{\pgfqpoint{1.676600in}{1.461357in}}%
\pgfpathlineto{\pgfqpoint{1.689104in}{1.449909in}}%
\pgfpathlineto{\pgfqpoint{1.691591in}{1.447746in}}%
\pgfpathlineto{\pgfqpoint{1.704462in}{1.434135in}}%
\pgfpathlineto{\pgfqpoint{1.704761in}{1.433638in}}%
\pgfpathlineto{\pgfqpoint{1.713675in}{1.420524in}}%
\pgfpathlineto{\pgfqpoint{1.715962in}{1.406913in}}%
\pgfpathlineto{\pgfqpoint{1.710719in}{1.393302in}}%
\pgfpathlineto{\pgfqpoint{1.704761in}{1.386386in}}%
\pgfpathlineto{\pgfqpoint{1.699703in}{1.379691in}}%
\pgfpathlineto{\pgfqpoint{1.689104in}{1.369490in}}%
\pgfpathlineto{\pgfqpoint{1.685716in}{1.366079in}}%
\pgfpathlineto{\pgfqpoint{1.673448in}{1.355216in}}%
\pgfpathlineto{\pgfqpoint{1.670175in}{1.352468in}}%
\pgfpathlineto{\pgfqpoint{1.657791in}{1.342183in}}%
\pgfpathlineto{\pgfqpoint{1.652943in}{1.338857in}}%
\pgfpathlineto{\pgfqpoint{1.642135in}{1.330827in}}%
\pgfpathlineto{\pgfqpoint{1.630419in}{1.325246in}}%
\pgfpathlineto{\pgfqpoint{1.626478in}{1.323025in}}%
\pgfpathlineto{\pgfqpoint{1.610822in}{1.320898in}}%
\pgfpathlineto{\pgfqpoint{1.596892in}{1.325246in}}%
\pgfpathclose%
\pgfpathmoveto{\pgfqpoint{0.493767in}{1.447746in}}%
\pgfpathlineto{\pgfqpoint{0.483549in}{1.453141in}}%
\pgfpathlineto{\pgfqpoint{0.472156in}{1.461357in}}%
\pgfpathlineto{\pgfqpoint{0.467892in}{1.464231in}}%
\pgfpathlineto{\pgfqpoint{0.454606in}{1.474968in}}%
\pgfpathlineto{\pgfqpoint{0.452236in}{1.476982in}}%
\pgfpathlineto{\pgfqpoint{0.439531in}{1.488579in}}%
\pgfpathlineto{\pgfqpoint{0.436579in}{1.491738in}}%
\pgfpathlineto{\pgfqpoint{0.426389in}{1.502191in}}%
\pgfpathlineto{\pgfqpoint{0.420923in}{1.509626in}}%
\pgfpathlineto{\pgfqpoint{0.415459in}{1.515802in}}%
\pgfpathlineto{\pgfqpoint{0.407863in}{1.529413in}}%
\pgfpathlineto{\pgfqpoint{0.405266in}{1.541837in}}%
\pgfpathlineto{\pgfqpoint{0.404884in}{1.543024in}}%
\pgfpathlineto{\pgfqpoint{0.405266in}{1.544798in}}%
\pgfpathlineto{\pgfqpoint{0.406915in}{1.556635in}}%
\pgfpathlineto{\pgfqpoint{0.413559in}{1.570246in}}%
\pgfpathlineto{\pgfqpoint{0.420923in}{1.579168in}}%
\pgfpathlineto{\pgfqpoint{0.424129in}{1.583857in}}%
\pgfpathlineto{\pgfqpoint{0.436579in}{1.597168in}}%
\pgfpathlineto{\pgfqpoint{0.436848in}{1.597468in}}%
\pgfpathlineto{\pgfqpoint{0.451386in}{1.611079in}}%
\pgfpathlineto{\pgfqpoint{0.452236in}{1.611818in}}%
\pgfpathlineto{\pgfqpoint{0.467892in}{1.624457in}}%
\pgfpathlineto{\pgfqpoint{0.468238in}{1.624691in}}%
\pgfpathlineto{\pgfqpoint{0.483549in}{1.635514in}}%
\pgfpathlineto{\pgfqpoint{0.488943in}{1.638302in}}%
\pgfpathlineto{\pgfqpoint{0.499205in}{1.644704in}}%
\pgfpathlineto{\pgfqpoint{0.514862in}{1.650479in}}%
\pgfpathlineto{\pgfqpoint{0.528478in}{1.651913in}}%
\pgfpathlineto{\pgfqpoint{0.530519in}{1.652245in}}%
\pgfpathlineto{\pgfqpoint{0.531883in}{1.651913in}}%
\pgfpathlineto{\pgfqpoint{0.546175in}{1.649655in}}%
\pgfpathlineto{\pgfqpoint{0.561832in}{1.643051in}}%
\pgfpathlineto{\pgfqpoint{0.568936in}{1.638302in}}%
\pgfpathlineto{\pgfqpoint{0.577488in}{1.633549in}}%
\pgfpathlineto{\pgfqpoint{0.589511in}{1.624691in}}%
\pgfpathlineto{\pgfqpoint{0.593145in}{1.622124in}}%
\pgfpathlineto{\pgfqpoint{0.606485in}{1.611079in}}%
\pgfpathlineto{\pgfqpoint{0.608801in}{1.609019in}}%
\pgfpathlineto{\pgfqpoint{0.621153in}{1.597468in}}%
\pgfpathlineto{\pgfqpoint{0.624458in}{1.593762in}}%
\pgfpathlineto{\pgfqpoint{0.633909in}{1.583857in}}%
\pgfpathlineto{\pgfqpoint{0.640115in}{1.574974in}}%
\pgfpathlineto{\pgfqpoint{0.644233in}{1.570246in}}%
\pgfpathlineto{\pgfqpoint{0.651335in}{1.556635in}}%
\pgfpathlineto{\pgfqpoint{0.653361in}{1.543024in}}%
\pgfpathlineto{\pgfqpoint{0.650322in}{1.529413in}}%
\pgfpathlineto{\pgfqpoint{0.642201in}{1.515802in}}%
\pgfpathlineto{\pgfqpoint{0.640115in}{1.513567in}}%
\pgfpathlineto{\pgfqpoint{0.631570in}{1.502191in}}%
\pgfpathlineto{\pgfqpoint{0.624458in}{1.495039in}}%
\pgfpathlineto{\pgfqpoint{0.618438in}{1.488579in}}%
\pgfpathlineto{\pgfqpoint{0.608801in}{1.479759in}}%
\pgfpathlineto{\pgfqpoint{0.603291in}{1.474968in}}%
\pgfpathlineto{\pgfqpoint{0.593145in}{1.466590in}}%
\pgfpathlineto{\pgfqpoint{0.585714in}{1.461357in}}%
\pgfpathlineto{\pgfqpoint{0.577488in}{1.455174in}}%
\pgfpathlineto{\pgfqpoint{0.564402in}{1.447746in}}%
\pgfpathlineto{\pgfqpoint{0.561832in}{1.445932in}}%
\pgfpathlineto{\pgfqpoint{0.546175in}{1.438872in}}%
\pgfpathlineto{\pgfqpoint{0.530519in}{1.436230in}}%
\pgfpathlineto{\pgfqpoint{0.514862in}{1.437991in}}%
\pgfpathlineto{\pgfqpoint{0.499205in}{1.444166in}}%
\pgfpathlineto{\pgfqpoint{0.493767in}{1.447746in}}%
\pgfpathclose%
\pgfpathmoveto{\pgfqpoint{0.803854in}{1.447746in}}%
\pgfpathlineto{\pgfqpoint{0.796680in}{1.451241in}}%
\pgfpathlineto{\pgfqpoint{0.781992in}{1.461357in}}%
\pgfpathlineto{\pgfqpoint{0.781024in}{1.461979in}}%
\pgfpathlineto{\pgfqpoint{0.765367in}{1.474272in}}%
\pgfpathlineto{\pgfqpoint{0.764559in}{1.474968in}}%
\pgfpathlineto{\pgfqpoint{0.749710in}{1.488476in}}%
\pgfpathlineto{\pgfqpoint{0.749595in}{1.488579in}}%
\pgfpathlineto{\pgfqpoint{0.736527in}{1.502191in}}%
\pgfpathlineto{\pgfqpoint{0.734054in}{1.505609in}}%
\pgfpathlineto{\pgfqpoint{0.725438in}{1.515802in}}%
\pgfpathlineto{\pgfqpoint{0.718397in}{1.529176in}}%
\pgfpathlineto{\pgfqpoint{0.718219in}{1.529413in}}%
\pgfpathlineto{\pgfqpoint{0.714272in}{1.543024in}}%
\pgfpathlineto{\pgfqpoint{0.716903in}{1.556635in}}%
\pgfpathlineto{\pgfqpoint{0.718397in}{1.558887in}}%
\pgfpathlineto{\pgfqpoint{0.723646in}{1.570246in}}%
\pgfpathlineto{\pgfqpoint{0.734054in}{1.583443in}}%
\pgfpathlineto{\pgfqpoint{0.734332in}{1.583857in}}%
\pgfpathlineto{\pgfqpoint{0.746712in}{1.597468in}}%
\pgfpathlineto{\pgfqpoint{0.749710in}{1.600225in}}%
\pgfpathlineto{\pgfqpoint{0.761372in}{1.611079in}}%
\pgfpathlineto{\pgfqpoint{0.765367in}{1.614530in}}%
\pgfpathlineto{\pgfqpoint{0.778314in}{1.624691in}}%
\pgfpathlineto{\pgfqpoint{0.781024in}{1.626895in}}%
\pgfpathlineto{\pgfqpoint{0.796680in}{1.637350in}}%
\pgfpathlineto{\pgfqpoint{0.798676in}{1.638302in}}%
\pgfpathlineto{\pgfqpoint{0.812337in}{1.646190in}}%
\pgfpathlineto{\pgfqpoint{0.827993in}{1.651138in}}%
\pgfpathlineto{\pgfqpoint{0.842722in}{1.651913in}}%
\pgfpathlineto{\pgfqpoint{0.843650in}{1.651989in}}%
\pgfpathlineto{\pgfqpoint{0.843884in}{1.651913in}}%
\pgfpathlineto{\pgfqpoint{0.859306in}{1.648665in}}%
\pgfpathlineto{\pgfqpoint{0.874963in}{1.641234in}}%
\pgfpathlineto{\pgfqpoint{0.879082in}{1.638302in}}%
\pgfpathlineto{\pgfqpoint{0.890620in}{1.631457in}}%
\pgfpathlineto{\pgfqpoint{0.899468in}{1.624691in}}%
\pgfpathlineto{\pgfqpoint{0.906276in}{1.619689in}}%
\pgfpathlineto{\pgfqpoint{0.916492in}{1.611079in}}%
\pgfpathlineto{\pgfqpoint{0.921933in}{1.606146in}}%
\pgfpathlineto{\pgfqpoint{0.931225in}{1.597468in}}%
\pgfpathlineto{\pgfqpoint{0.937589in}{1.590343in}}%
\pgfpathlineto{\pgfqpoint{0.943937in}{1.583857in}}%
\pgfpathlineto{\pgfqpoint{0.953246in}{1.570889in}}%
\pgfpathlineto{\pgfqpoint{0.953842in}{1.570246in}}%
\pgfpathlineto{\pgfqpoint{0.961505in}{1.556635in}}%
\pgfpathlineto{\pgfqpoint{0.963691in}{1.543024in}}%
\pgfpathlineto{\pgfqpoint{0.960412in}{1.529413in}}%
\pgfpathlineto{\pgfqpoint{0.953246in}{1.518222in}}%
\pgfpathlineto{\pgfqpoint{0.952063in}{1.515802in}}%
\pgfpathlineto{\pgfqpoint{0.941503in}{1.502191in}}%
\pgfpathlineto{\pgfqpoint{0.937589in}{1.498354in}}%
\pgfpathlineto{\pgfqpoint{0.928468in}{1.488579in}}%
\pgfpathlineto{\pgfqpoint{0.921933in}{1.482606in}}%
\pgfpathlineto{\pgfqpoint{0.913313in}{1.474968in}}%
\pgfpathlineto{\pgfqpoint{0.906276in}{1.469054in}}%
\pgfpathlineto{\pgfqpoint{0.895769in}{1.461357in}}%
\pgfpathlineto{\pgfqpoint{0.890620in}{1.457340in}}%
\pgfpathlineto{\pgfqpoint{0.874963in}{1.447846in}}%
\pgfpathlineto{\pgfqpoint{0.874716in}{1.447746in}}%
\pgfpathlineto{\pgfqpoint{0.859306in}{1.439930in}}%
\pgfpathlineto{\pgfqpoint{0.843650in}{1.436406in}}%
\pgfpathlineto{\pgfqpoint{0.827993in}{1.437287in}}%
\pgfpathlineto{\pgfqpoint{0.812337in}{1.442577in}}%
\pgfpathlineto{\pgfqpoint{0.803854in}{1.447746in}}%
\pgfpathclose%
\pgfpathmoveto{\pgfqpoint{1.113695in}{1.447746in}}%
\pgfpathlineto{\pgfqpoint{1.109812in}{1.449476in}}%
\pgfpathlineto{\pgfqpoint{1.094155in}{1.459636in}}%
\pgfpathlineto{\pgfqpoint{1.092020in}{1.461357in}}%
\pgfpathlineto{\pgfqpoint{1.078498in}{1.471616in}}%
\pgfpathlineto{\pgfqpoint{1.074566in}{1.474968in}}%
\pgfpathlineto{\pgfqpoint{1.062842in}{1.485515in}}%
\pgfpathlineto{\pgfqpoint{1.059471in}{1.488579in}}%
\pgfpathlineto{\pgfqpoint{1.047185in}{1.501665in}}%
\pgfpathlineto{\pgfqpoint{1.046632in}{1.502191in}}%
\pgfpathlineto{\pgfqpoint{1.035581in}{1.515802in}}%
\pgfpathlineto{\pgfqpoint{1.031529in}{1.523819in}}%
\pgfpathlineto{\pgfqpoint{1.027669in}{1.529413in}}%
\pgfpathlineto{\pgfqpoint{1.024094in}{1.543024in}}%
\pgfpathlineto{\pgfqpoint{1.026477in}{1.556635in}}%
\pgfpathlineto{\pgfqpoint{1.031529in}{1.564947in}}%
\pgfpathlineto{\pgfqpoint{1.033880in}{1.570246in}}%
\pgfpathlineto{\pgfqpoint{1.044085in}{1.583857in}}%
\pgfpathlineto{\pgfqpoint{1.047185in}{1.586927in}}%
\pgfpathlineto{\pgfqpoint{1.056658in}{1.597468in}}%
\pgfpathlineto{\pgfqpoint{1.062842in}{1.603211in}}%
\pgfpathlineto{\pgfqpoint{1.071389in}{1.611079in}}%
\pgfpathlineto{\pgfqpoint{1.078498in}{1.617156in}}%
\pgfpathlineto{\pgfqpoint{1.088401in}{1.624691in}}%
\pgfpathlineto{\pgfqpoint{1.094155in}{1.629238in}}%
\pgfpathlineto{\pgfqpoint{1.108551in}{1.638302in}}%
\pgfpathlineto{\pgfqpoint{1.109812in}{1.639251in}}%
\pgfpathlineto{\pgfqpoint{1.125468in}{1.647510in}}%
\pgfpathlineto{\pgfqpoint{1.141125in}{1.651632in}}%
\pgfpathlineto{\pgfqpoint{1.156781in}{1.651632in}}%
\pgfpathlineto{\pgfqpoint{1.172438in}{1.647510in}}%
\pgfpathlineto{\pgfqpoint{1.188094in}{1.639251in}}%
\pgfpathlineto{\pgfqpoint{1.189355in}{1.638302in}}%
\pgfpathlineto{\pgfqpoint{1.203751in}{1.629238in}}%
\pgfpathlineto{\pgfqpoint{1.209505in}{1.624691in}}%
\pgfpathlineto{\pgfqpoint{1.219407in}{1.617156in}}%
\pgfpathlineto{\pgfqpoint{1.226517in}{1.611079in}}%
\pgfpathlineto{\pgfqpoint{1.235064in}{1.603211in}}%
\pgfpathlineto{\pgfqpoint{1.241248in}{1.597468in}}%
\pgfpathlineto{\pgfqpoint{1.250721in}{1.586927in}}%
\pgfpathlineto{\pgfqpoint{1.253821in}{1.583857in}}%
\pgfpathlineto{\pgfqpoint{1.264026in}{1.570246in}}%
\pgfpathlineto{\pgfqpoint{1.266377in}{1.564947in}}%
\pgfpathlineto{\pgfqpoint{1.271429in}{1.556635in}}%
\pgfpathlineto{\pgfqpoint{1.273812in}{1.543024in}}%
\pgfpathlineto{\pgfqpoint{1.270237in}{1.529413in}}%
\pgfpathlineto{\pgfqpoint{1.266377in}{1.523819in}}%
\pgfpathlineto{\pgfqpoint{1.262325in}{1.515802in}}%
\pgfpathlineto{\pgfqpoint{1.251274in}{1.502191in}}%
\pgfpathlineto{\pgfqpoint{1.250721in}{1.501665in}}%
\pgfpathlineto{\pgfqpoint{1.238435in}{1.488579in}}%
\pgfpathlineto{\pgfqpoint{1.235064in}{1.485515in}}%
\pgfpathlineto{\pgfqpoint{1.223340in}{1.474968in}}%
\pgfpathlineto{\pgfqpoint{1.219407in}{1.471616in}}%
\pgfpathlineto{\pgfqpoint{1.205885in}{1.461357in}}%
\pgfpathlineto{\pgfqpoint{1.203751in}{1.459636in}}%
\pgfpathlineto{\pgfqpoint{1.188094in}{1.449476in}}%
\pgfpathlineto{\pgfqpoint{1.184211in}{1.447746in}}%
\pgfpathlineto{\pgfqpoint{1.172438in}{1.441165in}}%
\pgfpathlineto{\pgfqpoint{1.156781in}{1.436758in}}%
\pgfpathlineto{\pgfqpoint{1.141125in}{1.436758in}}%
\pgfpathlineto{\pgfqpoint{1.125468in}{1.441165in}}%
\pgfpathlineto{\pgfqpoint{1.113695in}{1.447746in}}%
\pgfpathclose%
\pgfpathmoveto{\pgfqpoint{1.423190in}{1.447746in}}%
\pgfpathlineto{\pgfqpoint{1.422943in}{1.447846in}}%
\pgfpathlineto{\pgfqpoint{1.407286in}{1.457340in}}%
\pgfpathlineto{\pgfqpoint{1.402136in}{1.461357in}}%
\pgfpathlineto{\pgfqpoint{1.391630in}{1.469054in}}%
\pgfpathlineto{\pgfqpoint{1.384593in}{1.474968in}}%
\pgfpathlineto{\pgfqpoint{1.375973in}{1.482606in}}%
\pgfpathlineto{\pgfqpoint{1.369438in}{1.488579in}}%
\pgfpathlineto{\pgfqpoint{1.360317in}{1.498354in}}%
\pgfpathlineto{\pgfqpoint{1.356403in}{1.502191in}}%
\pgfpathlineto{\pgfqpoint{1.345843in}{1.515802in}}%
\pgfpathlineto{\pgfqpoint{1.344660in}{1.518222in}}%
\pgfpathlineto{\pgfqpoint{1.337494in}{1.529413in}}%
\pgfpathlineto{\pgfqpoint{1.334215in}{1.543024in}}%
\pgfpathlineto{\pgfqpoint{1.336401in}{1.556635in}}%
\pgfpathlineto{\pgfqpoint{1.344063in}{1.570246in}}%
\pgfpathlineto{\pgfqpoint{1.344660in}{1.570889in}}%
\pgfpathlineto{\pgfqpoint{1.353969in}{1.583857in}}%
\pgfpathlineto{\pgfqpoint{1.360317in}{1.590343in}}%
\pgfpathlineto{\pgfqpoint{1.366681in}{1.597468in}}%
\pgfpathlineto{\pgfqpoint{1.375973in}{1.606146in}}%
\pgfpathlineto{\pgfqpoint{1.381414in}{1.611079in}}%
\pgfpathlineto{\pgfqpoint{1.391630in}{1.619689in}}%
\pgfpathlineto{\pgfqpoint{1.398438in}{1.624691in}}%
\pgfpathlineto{\pgfqpoint{1.407286in}{1.631457in}}%
\pgfpathlineto{\pgfqpoint{1.418823in}{1.638302in}}%
\pgfpathlineto{\pgfqpoint{1.422943in}{1.641234in}}%
\pgfpathlineto{\pgfqpoint{1.438599in}{1.648665in}}%
\pgfpathlineto{\pgfqpoint{1.454022in}{1.651913in}}%
\pgfpathlineto{\pgfqpoint{1.454256in}{1.651989in}}%
\pgfpathlineto{\pgfqpoint{1.455184in}{1.651913in}}%
\pgfpathlineto{\pgfqpoint{1.469913in}{1.651138in}}%
\pgfpathlineto{\pgfqpoint{1.485569in}{1.646190in}}%
\pgfpathlineto{\pgfqpoint{1.499230in}{1.638302in}}%
\pgfpathlineto{\pgfqpoint{1.501226in}{1.637350in}}%
\pgfpathlineto{\pgfqpoint{1.516882in}{1.626895in}}%
\pgfpathlineto{\pgfqpoint{1.519592in}{1.624691in}}%
\pgfpathlineto{\pgfqpoint{1.532539in}{1.614530in}}%
\pgfpathlineto{\pgfqpoint{1.536534in}{1.611079in}}%
\pgfpathlineto{\pgfqpoint{1.548195in}{1.600225in}}%
\pgfpathlineto{\pgfqpoint{1.551194in}{1.597468in}}%
\pgfpathlineto{\pgfqpoint{1.563573in}{1.583857in}}%
\pgfpathlineto{\pgfqpoint{1.563852in}{1.583443in}}%
\pgfpathlineto{\pgfqpoint{1.574260in}{1.570246in}}%
\pgfpathlineto{\pgfqpoint{1.579508in}{1.558887in}}%
\pgfpathlineto{\pgfqpoint{1.581003in}{1.556635in}}%
\pgfpathlineto{\pgfqpoint{1.583634in}{1.543024in}}%
\pgfpathlineto{\pgfqpoint{1.579687in}{1.529413in}}%
\pgfpathlineto{\pgfqpoint{1.579508in}{1.529176in}}%
\pgfpathlineto{\pgfqpoint{1.572468in}{1.515802in}}%
\pgfpathlineto{\pgfqpoint{1.563852in}{1.505609in}}%
\pgfpathlineto{\pgfqpoint{1.561378in}{1.502191in}}%
\pgfpathlineto{\pgfqpoint{1.548311in}{1.488579in}}%
\pgfpathlineto{\pgfqpoint{1.548195in}{1.488476in}}%
\pgfpathlineto{\pgfqpoint{1.533347in}{1.474968in}}%
\pgfpathlineto{\pgfqpoint{1.532539in}{1.474272in}}%
\pgfpathlineto{\pgfqpoint{1.516882in}{1.461979in}}%
\pgfpathlineto{\pgfqpoint{1.515914in}{1.461357in}}%
\pgfpathlineto{\pgfqpoint{1.501226in}{1.451241in}}%
\pgfpathlineto{\pgfqpoint{1.494052in}{1.447746in}}%
\pgfpathlineto{\pgfqpoint{1.485569in}{1.442577in}}%
\pgfpathlineto{\pgfqpoint{1.469913in}{1.437287in}}%
\pgfpathlineto{\pgfqpoint{1.454256in}{1.436406in}}%
\pgfpathlineto{\pgfqpoint{1.438599in}{1.439930in}}%
\pgfpathlineto{\pgfqpoint{1.423190in}{1.447746in}}%
\pgfpathclose%
\pgfpathmoveto{\pgfqpoint{1.733504in}{1.447746in}}%
\pgfpathlineto{\pgfqpoint{1.720418in}{1.455174in}}%
\pgfpathlineto{\pgfqpoint{1.712192in}{1.461357in}}%
\pgfpathlineto{\pgfqpoint{1.704761in}{1.466590in}}%
\pgfpathlineto{\pgfqpoint{1.694615in}{1.474968in}}%
\pgfpathlineto{\pgfqpoint{1.689104in}{1.479759in}}%
\pgfpathlineto{\pgfqpoint{1.679467in}{1.488579in}}%
\pgfpathlineto{\pgfqpoint{1.673448in}{1.495039in}}%
\pgfpathlineto{\pgfqpoint{1.666336in}{1.502191in}}%
\pgfpathlineto{\pgfqpoint{1.657791in}{1.513567in}}%
\pgfpathlineto{\pgfqpoint{1.655705in}{1.515802in}}%
\pgfpathlineto{\pgfqpoint{1.647584in}{1.529413in}}%
\pgfpathlineto{\pgfqpoint{1.644545in}{1.543024in}}%
\pgfpathlineto{\pgfqpoint{1.646571in}{1.556635in}}%
\pgfpathlineto{\pgfqpoint{1.653673in}{1.570246in}}%
\pgfpathlineto{\pgfqpoint{1.657791in}{1.574974in}}%
\pgfpathlineto{\pgfqpoint{1.663997in}{1.583857in}}%
\pgfpathlineto{\pgfqpoint{1.673448in}{1.593762in}}%
\pgfpathlineto{\pgfqpoint{1.676753in}{1.597468in}}%
\pgfpathlineto{\pgfqpoint{1.689104in}{1.609019in}}%
\pgfpathlineto{\pgfqpoint{1.691421in}{1.611079in}}%
\pgfpathlineto{\pgfqpoint{1.704761in}{1.622124in}}%
\pgfpathlineto{\pgfqpoint{1.708395in}{1.624691in}}%
\pgfpathlineto{\pgfqpoint{1.720418in}{1.633549in}}%
\pgfpathlineto{\pgfqpoint{1.728970in}{1.638302in}}%
\pgfpathlineto{\pgfqpoint{1.736074in}{1.643051in}}%
\pgfpathlineto{\pgfqpoint{1.751731in}{1.649655in}}%
\pgfpathlineto{\pgfqpoint{1.766022in}{1.651913in}}%
\pgfpathlineto{\pgfqpoint{1.767387in}{1.652245in}}%
\pgfpathlineto{\pgfqpoint{1.769428in}{1.651913in}}%
\pgfpathlineto{\pgfqpoint{1.783044in}{1.650479in}}%
\pgfpathlineto{\pgfqpoint{1.798700in}{1.644704in}}%
\pgfpathlineto{\pgfqpoint{1.808963in}{1.638302in}}%
\pgfpathlineto{\pgfqpoint{1.814357in}{1.635514in}}%
\pgfpathlineto{\pgfqpoint{1.829668in}{1.624691in}}%
\pgfpathlineto{\pgfqpoint{1.830014in}{1.624457in}}%
\pgfpathlineto{\pgfqpoint{1.845670in}{1.611818in}}%
\pgfpathlineto{\pgfqpoint{1.846519in}{1.611079in}}%
\pgfpathlineto{\pgfqpoint{1.861058in}{1.597468in}}%
\pgfpathlineto{\pgfqpoint{1.861327in}{1.597168in}}%
\pgfpathlineto{\pgfqpoint{1.873777in}{1.583857in}}%
\pgfpathlineto{\pgfqpoint{1.876983in}{1.579168in}}%
\pgfpathlineto{\pgfqpoint{1.884347in}{1.570246in}}%
\pgfpathlineto{\pgfqpoint{1.890991in}{1.556635in}}%
\pgfpathlineto{\pgfqpoint{1.892640in}{1.544798in}}%
\pgfpathlineto{\pgfqpoint{1.893022in}{1.543024in}}%
\pgfpathlineto{\pgfqpoint{1.892640in}{1.541837in}}%
\pgfpathlineto{\pgfqpoint{1.890042in}{1.529413in}}%
\pgfpathlineto{\pgfqpoint{1.882447in}{1.515802in}}%
\pgfpathlineto{\pgfqpoint{1.876983in}{1.509626in}}%
\pgfpathlineto{\pgfqpoint{1.871517in}{1.502191in}}%
\pgfpathlineto{\pgfqpoint{1.861327in}{1.491738in}}%
\pgfpathlineto{\pgfqpoint{1.858375in}{1.488579in}}%
\pgfpathlineto{\pgfqpoint{1.845670in}{1.476982in}}%
\pgfpathlineto{\pgfqpoint{1.843300in}{1.474968in}}%
\pgfpathlineto{\pgfqpoint{1.830014in}{1.464231in}}%
\pgfpathlineto{\pgfqpoint{1.825750in}{1.461357in}}%
\pgfpathlineto{\pgfqpoint{1.814357in}{1.453141in}}%
\pgfpathlineto{\pgfqpoint{1.804139in}{1.447746in}}%
\pgfpathlineto{\pgfqpoint{1.798700in}{1.444166in}}%
\pgfpathlineto{\pgfqpoint{1.783044in}{1.437991in}}%
\pgfpathlineto{\pgfqpoint{1.767387in}{1.436230in}}%
\pgfpathlineto{\pgfqpoint{1.751731in}{1.438872in}}%
\pgfpathlineto{\pgfqpoint{1.736074in}{1.445932in}}%
\pgfpathlineto{\pgfqpoint{1.733504in}{1.447746in}}%
\pgfpathclose%
\pgfusepath{fill}%
\end{pgfscope}%
\begin{pgfscope}%
\pgfpathrectangle{\pgfqpoint{0.373953in}{0.331635in}}{\pgfqpoint{1.550000in}{1.347500in}}%
\pgfusepath{clip}%
\pgfsetbuttcap%
\pgfsetroundjoin%
\definecolor{currentfill}{rgb}{0.252220,0.059415,0.453248}%
\pgfsetfillcolor{currentfill}%
\pgfsetlinewidth{0.000000pt}%
\definecolor{currentstroke}{rgb}{0.000000,0.000000,0.000000}%
\pgfsetstrokecolor{currentstroke}%
\pgfsetdash{}{0pt}%
\pgfpathmoveto{\pgfqpoint{0.436579in}{0.331635in}}%
\pgfpathlineto{\pgfqpoint{0.452236in}{0.331635in}}%
\pgfpathlineto{\pgfqpoint{0.467892in}{0.331635in}}%
\pgfpathlineto{\pgfqpoint{0.476205in}{0.331635in}}%
\pgfpathlineto{\pgfqpoint{0.472300in}{0.345246in}}%
\pgfpathlineto{\pgfqpoint{0.467892in}{0.350703in}}%
\pgfpathlineto{\pgfqpoint{0.462219in}{0.358857in}}%
\pgfpathlineto{\pgfqpoint{0.452236in}{0.368712in}}%
\pgfpathlineto{\pgfqpoint{0.448661in}{0.372468in}}%
\pgfpathlineto{\pgfqpoint{0.436579in}{0.383306in}}%
\pgfpathlineto{\pgfqpoint{0.433389in}{0.386079in}}%
\pgfpathlineto{\pgfqpoint{0.420923in}{0.396583in}}%
\pgfpathlineto{\pgfqpoint{0.416602in}{0.399691in}}%
\pgfpathlineto{\pgfqpoint{0.405266in}{0.408370in}}%
\pgfpathlineto{\pgfqpoint{0.395886in}{0.413302in}}%
\pgfpathlineto{\pgfqpoint{0.389609in}{0.417133in}}%
\pgfpathlineto{\pgfqpoint{0.373953in}{0.420528in}}%
\pgfpathlineto{\pgfqpoint{0.373953in}{0.413302in}}%
\pgfpathlineto{\pgfqpoint{0.373953in}{0.399691in}}%
\pgfpathlineto{\pgfqpoint{0.373953in}{0.386079in}}%
\pgfpathlineto{\pgfqpoint{0.373953in}{0.383489in}}%
\pgfpathlineto{\pgfqpoint{0.389609in}{0.380692in}}%
\pgfpathlineto{\pgfqpoint{0.405266in}{0.373002in}}%
\pgfpathlineto{\pgfqpoint{0.405994in}{0.372468in}}%
\pgfpathlineto{\pgfqpoint{0.420923in}{0.359490in}}%
\pgfpathlineto{\pgfqpoint{0.421536in}{0.358857in}}%
\pgfpathlineto{\pgfqpoint{0.430382in}{0.345246in}}%
\pgfpathlineto{\pgfqpoint{0.433600in}{0.331635in}}%
\pgfpathlineto{\pgfqpoint{0.436579in}{0.331635in}}%
\pgfpathclose%
\pgfpathmoveto{\pgfqpoint{0.593145in}{0.331635in}}%
\pgfpathlineto{\pgfqpoint{0.608801in}{0.331635in}}%
\pgfpathlineto{\pgfqpoint{0.624148in}{0.331635in}}%
\pgfpathlineto{\pgfqpoint{0.624458in}{0.333049in}}%
\pgfpathlineto{\pgfqpoint{0.627438in}{0.345246in}}%
\pgfpathlineto{\pgfqpoint{0.636592in}{0.358857in}}%
\pgfpathlineto{\pgfqpoint{0.640115in}{0.362414in}}%
\pgfpathlineto{\pgfqpoint{0.652320in}{0.372468in}}%
\pgfpathlineto{\pgfqpoint{0.655771in}{0.374857in}}%
\pgfpathlineto{\pgfqpoint{0.671428in}{0.381685in}}%
\pgfpathlineto{\pgfqpoint{0.687084in}{0.383375in}}%
\pgfpathlineto{\pgfqpoint{0.702741in}{0.379500in}}%
\pgfpathlineto{\pgfqpoint{0.715624in}{0.372468in}}%
\pgfpathlineto{\pgfqpoint{0.718397in}{0.370646in}}%
\pgfpathlineto{\pgfqpoint{0.731317in}{0.358857in}}%
\pgfpathlineto{\pgfqpoint{0.734054in}{0.355203in}}%
\pgfpathlineto{\pgfqpoint{0.740405in}{0.345246in}}%
\pgfpathlineto{\pgfqpoint{0.743530in}{0.331635in}}%
\pgfpathlineto{\pgfqpoint{0.749710in}{0.331635in}}%
\pgfpathlineto{\pgfqpoint{0.765367in}{0.331635in}}%
\pgfpathlineto{\pgfqpoint{0.781024in}{0.331635in}}%
\pgfpathlineto{\pgfqpoint{0.786191in}{0.331635in}}%
\pgfpathlineto{\pgfqpoint{0.782141in}{0.345246in}}%
\pgfpathlineto{\pgfqpoint{0.781024in}{0.346564in}}%
\pgfpathlineto{\pgfqpoint{0.772252in}{0.358857in}}%
\pgfpathlineto{\pgfqpoint{0.765367in}{0.365473in}}%
\pgfpathlineto{\pgfqpoint{0.758666in}{0.372468in}}%
\pgfpathlineto{\pgfqpoint{0.749710in}{0.380446in}}%
\pgfpathlineto{\pgfqpoint{0.743325in}{0.386079in}}%
\pgfpathlineto{\pgfqpoint{0.734054in}{0.394016in}}%
\pgfpathlineto{\pgfqpoint{0.726515in}{0.399691in}}%
\pgfpathlineto{\pgfqpoint{0.718397in}{0.406214in}}%
\pgfpathlineto{\pgfqpoint{0.706247in}{0.413302in}}%
\pgfpathlineto{\pgfqpoint{0.702741in}{0.415687in}}%
\pgfpathlineto{\pgfqpoint{0.687084in}{0.420390in}}%
\pgfpathlineto{\pgfqpoint{0.671428in}{0.418338in}}%
\pgfpathlineto{\pgfqpoint{0.662044in}{0.413302in}}%
\pgfpathlineto{\pgfqpoint{0.655771in}{0.410388in}}%
\pgfpathlineto{\pgfqpoint{0.640979in}{0.399691in}}%
\pgfpathlineto{\pgfqpoint{0.640115in}{0.399101in}}%
\pgfpathlineto{\pgfqpoint{0.624458in}{0.386182in}}%
\pgfpathlineto{\pgfqpoint{0.624341in}{0.386079in}}%
\pgfpathlineto{\pgfqpoint{0.609202in}{0.372468in}}%
\pgfpathlineto{\pgfqpoint{0.608801in}{0.372046in}}%
\pgfpathlineto{\pgfqpoint{0.595745in}{0.358857in}}%
\pgfpathlineto{\pgfqpoint{0.593145in}{0.355039in}}%
\pgfpathlineto{\pgfqpoint{0.585574in}{0.345246in}}%
\pgfpathlineto{\pgfqpoint{0.581789in}{0.331635in}}%
\pgfpathlineto{\pgfqpoint{0.593145in}{0.331635in}}%
\pgfpathclose%
\pgfpathmoveto{\pgfqpoint{0.906276in}{0.331635in}}%
\pgfpathlineto{\pgfqpoint{0.921933in}{0.331635in}}%
\pgfpathlineto{\pgfqpoint{0.934268in}{0.331635in}}%
\pgfpathlineto{\pgfqpoint{0.937257in}{0.345246in}}%
\pgfpathlineto{\pgfqpoint{0.937589in}{0.345776in}}%
\pgfpathlineto{\pgfqpoint{0.946729in}{0.358857in}}%
\pgfpathlineto{\pgfqpoint{0.953246in}{0.365261in}}%
\pgfpathlineto{\pgfqpoint{0.962568in}{0.372468in}}%
\pgfpathlineto{\pgfqpoint{0.968902in}{0.376568in}}%
\pgfpathlineto{\pgfqpoint{0.984559in}{0.382468in}}%
\pgfpathlineto{\pgfqpoint{1.000216in}{0.383033in}}%
\pgfpathlineto{\pgfqpoint{1.015872in}{0.378121in}}%
\pgfpathlineto{\pgfqpoint{1.025319in}{0.372468in}}%
\pgfpathlineto{\pgfqpoint{1.031529in}{0.368012in}}%
\pgfpathlineto{\pgfqpoint{1.041163in}{0.358857in}}%
\pgfpathlineto{\pgfqpoint{1.047185in}{0.350505in}}%
\pgfpathlineto{\pgfqpoint{1.050503in}{0.345246in}}%
\pgfpathlineto{\pgfqpoint{1.053553in}{0.331635in}}%
\pgfpathlineto{\pgfqpoint{1.062842in}{0.331635in}}%
\pgfpathlineto{\pgfqpoint{1.078498in}{0.331635in}}%
\pgfpathlineto{\pgfqpoint{1.094155in}{0.331635in}}%
\pgfpathlineto{\pgfqpoint{1.096036in}{0.331635in}}%
\pgfpathlineto{\pgfqpoint{1.094155in}{0.337649in}}%
\pgfpathlineto{\pgfqpoint{1.092154in}{0.345246in}}%
\pgfpathlineto{\pgfqpoint{1.082234in}{0.358857in}}%
\pgfpathlineto{\pgfqpoint{1.078498in}{0.362338in}}%
\pgfpathlineto{\pgfqpoint{1.068693in}{0.372468in}}%
\pgfpathlineto{\pgfqpoint{1.062842in}{0.377623in}}%
\pgfpathlineto{\pgfqpoint{1.053353in}{0.386079in}}%
\pgfpathlineto{\pgfqpoint{1.047185in}{0.391417in}}%
\pgfpathlineto{\pgfqpoint{1.036604in}{0.399691in}}%
\pgfpathlineto{\pgfqpoint{1.031529in}{0.403938in}}%
\pgfpathlineto{\pgfqpoint{1.016826in}{0.413302in}}%
\pgfpathlineto{\pgfqpoint{1.015872in}{0.414013in}}%
\pgfpathlineto{\pgfqpoint{1.000216in}{0.419975in}}%
\pgfpathlineto{\pgfqpoint{0.984559in}{0.419289in}}%
\pgfpathlineto{\pgfqpoint{0.971533in}{0.413302in}}%
\pgfpathlineto{\pgfqpoint{0.968902in}{0.412250in}}%
\pgfpathlineto{\pgfqpoint{0.953246in}{0.401560in}}%
\pgfpathlineto{\pgfqpoint{0.951085in}{0.399691in}}%
\pgfpathlineto{\pgfqpoint{0.937589in}{0.388801in}}%
\pgfpathlineto{\pgfqpoint{0.934464in}{0.386079in}}%
\pgfpathlineto{\pgfqpoint{0.921933in}{0.374848in}}%
\pgfpathlineto{\pgfqpoint{0.919190in}{0.372468in}}%
\pgfpathlineto{\pgfqpoint{0.906276in}{0.359313in}}%
\pgfpathlineto{\pgfqpoint{0.905770in}{0.358857in}}%
\pgfpathlineto{\pgfqpoint{0.895633in}{0.345246in}}%
\pgfpathlineto{\pgfqpoint{0.891946in}{0.331635in}}%
\pgfpathlineto{\pgfqpoint{0.906276in}{0.331635in}}%
\pgfpathclose%
\pgfpathmoveto{\pgfqpoint{1.203751in}{0.331635in}}%
\pgfpathlineto{\pgfqpoint{1.219407in}{0.331635in}}%
\pgfpathlineto{\pgfqpoint{1.235064in}{0.331635in}}%
\pgfpathlineto{\pgfqpoint{1.244353in}{0.331635in}}%
\pgfpathlineto{\pgfqpoint{1.247402in}{0.345246in}}%
\pgfpathlineto{\pgfqpoint{1.250721in}{0.350505in}}%
\pgfpathlineto{\pgfqpoint{1.256742in}{0.358857in}}%
\pgfpathlineto{\pgfqpoint{1.266377in}{0.368012in}}%
\pgfpathlineto{\pgfqpoint{1.272587in}{0.372468in}}%
\pgfpathlineto{\pgfqpoint{1.282034in}{0.378121in}}%
\pgfpathlineto{\pgfqpoint{1.297690in}{0.383033in}}%
\pgfpathlineto{\pgfqpoint{1.313347in}{0.382468in}}%
\pgfpathlineto{\pgfqpoint{1.329003in}{0.376568in}}%
\pgfpathlineto{\pgfqpoint{1.335338in}{0.372468in}}%
\pgfpathlineto{\pgfqpoint{1.344660in}{0.365261in}}%
\pgfpathlineto{\pgfqpoint{1.351177in}{0.358857in}}%
\pgfpathlineto{\pgfqpoint{1.360317in}{0.345776in}}%
\pgfpathlineto{\pgfqpoint{1.360649in}{0.345246in}}%
\pgfpathlineto{\pgfqpoint{1.363638in}{0.331635in}}%
\pgfpathlineto{\pgfqpoint{1.375973in}{0.331635in}}%
\pgfpathlineto{\pgfqpoint{1.391630in}{0.331635in}}%
\pgfpathlineto{\pgfqpoint{1.405960in}{0.331635in}}%
\pgfpathlineto{\pgfqpoint{1.402272in}{0.345246in}}%
\pgfpathlineto{\pgfqpoint{1.392136in}{0.358857in}}%
\pgfpathlineto{\pgfqpoint{1.391630in}{0.359313in}}%
\pgfpathlineto{\pgfqpoint{1.378716in}{0.372468in}}%
\pgfpathlineto{\pgfqpoint{1.375973in}{0.374848in}}%
\pgfpathlineto{\pgfqpoint{1.363442in}{0.386079in}}%
\pgfpathlineto{\pgfqpoint{1.360317in}{0.388801in}}%
\pgfpathlineto{\pgfqpoint{1.346820in}{0.399691in}}%
\pgfpathlineto{\pgfqpoint{1.344660in}{0.401560in}}%
\pgfpathlineto{\pgfqpoint{1.329003in}{0.412250in}}%
\pgfpathlineto{\pgfqpoint{1.326373in}{0.413302in}}%
\pgfpathlineto{\pgfqpoint{1.313347in}{0.419289in}}%
\pgfpathlineto{\pgfqpoint{1.297690in}{0.419975in}}%
\pgfpathlineto{\pgfqpoint{1.282034in}{0.414013in}}%
\pgfpathlineto{\pgfqpoint{1.281080in}{0.413302in}}%
\pgfpathlineto{\pgfqpoint{1.266377in}{0.403938in}}%
\pgfpathlineto{\pgfqpoint{1.261302in}{0.399691in}}%
\pgfpathlineto{\pgfqpoint{1.250721in}{0.391417in}}%
\pgfpathlineto{\pgfqpoint{1.244553in}{0.386079in}}%
\pgfpathlineto{\pgfqpoint{1.235064in}{0.377623in}}%
\pgfpathlineto{\pgfqpoint{1.229213in}{0.372468in}}%
\pgfpathlineto{\pgfqpoint{1.219407in}{0.362338in}}%
\pgfpathlineto{\pgfqpoint{1.215672in}{0.358857in}}%
\pgfpathlineto{\pgfqpoint{1.205752in}{0.345246in}}%
\pgfpathlineto{\pgfqpoint{1.203751in}{0.337649in}}%
\pgfpathlineto{\pgfqpoint{1.201870in}{0.331635in}}%
\pgfpathlineto{\pgfqpoint{1.203751in}{0.331635in}}%
\pgfpathclose%
\pgfpathmoveto{\pgfqpoint{1.516882in}{0.331635in}}%
\pgfpathlineto{\pgfqpoint{1.532539in}{0.331635in}}%
\pgfpathlineto{\pgfqpoint{1.548195in}{0.331635in}}%
\pgfpathlineto{\pgfqpoint{1.554376in}{0.331635in}}%
\pgfpathlineto{\pgfqpoint{1.557501in}{0.345246in}}%
\pgfpathlineto{\pgfqpoint{1.563852in}{0.355203in}}%
\pgfpathlineto{\pgfqpoint{1.566589in}{0.358857in}}%
\pgfpathlineto{\pgfqpoint{1.579508in}{0.370646in}}%
\pgfpathlineto{\pgfqpoint{1.582282in}{0.372468in}}%
\pgfpathlineto{\pgfqpoint{1.595165in}{0.379500in}}%
\pgfpathlineto{\pgfqpoint{1.610822in}{0.383375in}}%
\pgfpathlineto{\pgfqpoint{1.626478in}{0.381685in}}%
\pgfpathlineto{\pgfqpoint{1.642135in}{0.374857in}}%
\pgfpathlineto{\pgfqpoint{1.645586in}{0.372468in}}%
\pgfpathlineto{\pgfqpoint{1.657791in}{0.362414in}}%
\pgfpathlineto{\pgfqpoint{1.661314in}{0.358857in}}%
\pgfpathlineto{\pgfqpoint{1.670468in}{0.345246in}}%
\pgfpathlineto{\pgfqpoint{1.673448in}{0.333049in}}%
\pgfpathlineto{\pgfqpoint{1.673757in}{0.331635in}}%
\pgfpathlineto{\pgfqpoint{1.689104in}{0.331635in}}%
\pgfpathlineto{\pgfqpoint{1.704761in}{0.331635in}}%
\pgfpathlineto{\pgfqpoint{1.716117in}{0.331635in}}%
\pgfpathlineto{\pgfqpoint{1.712332in}{0.345246in}}%
\pgfpathlineto{\pgfqpoint{1.704761in}{0.355039in}}%
\pgfpathlineto{\pgfqpoint{1.702161in}{0.358857in}}%
\pgfpathlineto{\pgfqpoint{1.689104in}{0.372046in}}%
\pgfpathlineto{\pgfqpoint{1.688703in}{0.372468in}}%
\pgfpathlineto{\pgfqpoint{1.673565in}{0.386079in}}%
\pgfpathlineto{\pgfqpoint{1.673448in}{0.386182in}}%
\pgfpathlineto{\pgfqpoint{1.657791in}{0.399101in}}%
\pgfpathlineto{\pgfqpoint{1.656927in}{0.399691in}}%
\pgfpathlineto{\pgfqpoint{1.642135in}{0.410388in}}%
\pgfpathlineto{\pgfqpoint{1.635862in}{0.413302in}}%
\pgfpathlineto{\pgfqpoint{1.626478in}{0.418338in}}%
\pgfpathlineto{\pgfqpoint{1.610822in}{0.420390in}}%
\pgfpathlineto{\pgfqpoint{1.595165in}{0.415687in}}%
\pgfpathlineto{\pgfqpoint{1.591659in}{0.413302in}}%
\pgfpathlineto{\pgfqpoint{1.579508in}{0.406214in}}%
\pgfpathlineto{\pgfqpoint{1.571391in}{0.399691in}}%
\pgfpathlineto{\pgfqpoint{1.563852in}{0.394016in}}%
\pgfpathlineto{\pgfqpoint{1.554580in}{0.386079in}}%
\pgfpathlineto{\pgfqpoint{1.548195in}{0.380446in}}%
\pgfpathlineto{\pgfqpoint{1.539239in}{0.372468in}}%
\pgfpathlineto{\pgfqpoint{1.532539in}{0.365473in}}%
\pgfpathlineto{\pgfqpoint{1.525654in}{0.358857in}}%
\pgfpathlineto{\pgfqpoint{1.516882in}{0.346564in}}%
\pgfpathlineto{\pgfqpoint{1.515765in}{0.345246in}}%
\pgfpathlineto{\pgfqpoint{1.511715in}{0.331635in}}%
\pgfpathlineto{\pgfqpoint{1.516882in}{0.331635in}}%
\pgfpathclose%
\pgfpathmoveto{\pgfqpoint{1.830014in}{0.331635in}}%
\pgfpathlineto{\pgfqpoint{1.845670in}{0.331635in}}%
\pgfpathlineto{\pgfqpoint{1.861327in}{0.331635in}}%
\pgfpathlineto{\pgfqpoint{1.864306in}{0.331635in}}%
\pgfpathlineto{\pgfqpoint{1.867524in}{0.345246in}}%
\pgfpathlineto{\pgfqpoint{1.876369in}{0.358857in}}%
\pgfpathlineto{\pgfqpoint{1.876983in}{0.359490in}}%
\pgfpathlineto{\pgfqpoint{1.891911in}{0.372468in}}%
\pgfpathlineto{\pgfqpoint{1.892640in}{0.373002in}}%
\pgfpathlineto{\pgfqpoint{1.908296in}{0.380692in}}%
\pgfpathlineto{\pgfqpoint{1.923953in}{0.383489in}}%
\pgfpathlineto{\pgfqpoint{1.923953in}{0.386079in}}%
\pgfpathlineto{\pgfqpoint{1.923953in}{0.399691in}}%
\pgfpathlineto{\pgfqpoint{1.923953in}{0.413302in}}%
\pgfpathlineto{\pgfqpoint{1.923953in}{0.420528in}}%
\pgfpathlineto{\pgfqpoint{1.908296in}{0.417133in}}%
\pgfpathlineto{\pgfqpoint{1.902020in}{0.413302in}}%
\pgfpathlineto{\pgfqpoint{1.892640in}{0.408370in}}%
\pgfpathlineto{\pgfqpoint{1.881304in}{0.399691in}}%
\pgfpathlineto{\pgfqpoint{1.876983in}{0.396583in}}%
\pgfpathlineto{\pgfqpoint{1.864516in}{0.386079in}}%
\pgfpathlineto{\pgfqpoint{1.861327in}{0.383306in}}%
\pgfpathlineto{\pgfqpoint{1.849245in}{0.372468in}}%
\pgfpathlineto{\pgfqpoint{1.845670in}{0.368712in}}%
\pgfpathlineto{\pgfqpoint{1.835687in}{0.358857in}}%
\pgfpathlineto{\pgfqpoint{1.830014in}{0.350703in}}%
\pgfpathlineto{\pgfqpoint{1.825606in}{0.345246in}}%
\pgfpathlineto{\pgfqpoint{1.821701in}{0.331635in}}%
\pgfpathlineto{\pgfqpoint{1.830014in}{0.331635in}}%
\pgfpathclose%
\pgfpathmoveto{\pgfqpoint{0.389609in}{0.515609in}}%
\pgfpathlineto{\pgfqpoint{0.400874in}{0.522191in}}%
\pgfpathlineto{\pgfqpoint{0.405266in}{0.524451in}}%
\pgfpathlineto{\pgfqpoint{0.420436in}{0.535802in}}%
\pgfpathlineto{\pgfqpoint{0.420923in}{0.536150in}}%
\pgfpathlineto{\pgfqpoint{0.436579in}{0.549311in}}%
\pgfpathlineto{\pgfqpoint{0.436697in}{0.549413in}}%
\pgfpathlineto{\pgfqpoint{0.451558in}{0.563024in}}%
\pgfpathlineto{\pgfqpoint{0.452236in}{0.563776in}}%
\pgfpathlineto{\pgfqpoint{0.464540in}{0.576635in}}%
\pgfpathlineto{\pgfqpoint{0.467892in}{0.582089in}}%
\pgfpathlineto{\pgfqpoint{0.473686in}{0.590246in}}%
\pgfpathlineto{\pgfqpoint{0.476046in}{0.603857in}}%
\pgfpathlineto{\pgfqpoint{0.470636in}{0.617468in}}%
\pgfpathlineto{\pgfqpoint{0.467892in}{0.620516in}}%
\pgfpathlineto{\pgfqpoint{0.459739in}{0.631079in}}%
\pgfpathlineto{\pgfqpoint{0.452236in}{0.638136in}}%
\pgfpathlineto{\pgfqpoint{0.445709in}{0.644691in}}%
\pgfpathlineto{\pgfqpoint{0.436579in}{0.652751in}}%
\pgfpathlineto{\pgfqpoint{0.430099in}{0.658302in}}%
\pgfpathlineto{\pgfqpoint{0.420923in}{0.666088in}}%
\pgfpathlineto{\pgfqpoint{0.412876in}{0.671913in}}%
\pgfpathlineto{\pgfqpoint{0.405266in}{0.677898in}}%
\pgfpathlineto{\pgfqpoint{0.391125in}{0.685524in}}%
\pgfpathlineto{\pgfqpoint{0.389609in}{0.686495in}}%
\pgfpathlineto{\pgfqpoint{0.373953in}{0.690016in}}%
\pgfpathlineto{\pgfqpoint{0.373953in}{0.685524in}}%
\pgfpathlineto{\pgfqpoint{0.373953in}{0.671913in}}%
\pgfpathlineto{\pgfqpoint{0.373953in}{0.658302in}}%
\pgfpathlineto{\pgfqpoint{0.373953in}{0.652928in}}%
\pgfpathlineto{\pgfqpoint{0.389609in}{0.650212in}}%
\pgfpathlineto{\pgfqpoint{0.401063in}{0.644691in}}%
\pgfpathlineto{\pgfqpoint{0.405266in}{0.642311in}}%
\pgfpathlineto{\pgfqpoint{0.418826in}{0.631079in}}%
\pgfpathlineto{\pgfqpoint{0.420923in}{0.628669in}}%
\pgfpathlineto{\pgfqpoint{0.429011in}{0.617468in}}%
\pgfpathlineto{\pgfqpoint{0.433468in}{0.603857in}}%
\pgfpathlineto{\pgfqpoint{0.431524in}{0.590246in}}%
\pgfpathlineto{\pgfqpoint{0.423670in}{0.576635in}}%
\pgfpathlineto{\pgfqpoint{0.420923in}{0.573635in}}%
\pgfpathlineto{\pgfqpoint{0.409357in}{0.563024in}}%
\pgfpathlineto{\pgfqpoint{0.405266in}{0.559962in}}%
\pgfpathlineto{\pgfqpoint{0.389609in}{0.552003in}}%
\pgfpathlineto{\pgfqpoint{0.375580in}{0.549413in}}%
\pgfpathlineto{\pgfqpoint{0.373953in}{0.549144in}}%
\pgfpathlineto{\pgfqpoint{0.373953in}{0.535802in}}%
\pgfpathlineto{\pgfqpoint{0.373953in}{0.522191in}}%
\pgfpathlineto{\pgfqpoint{0.373953in}{0.512318in}}%
\pgfpathlineto{\pgfqpoint{0.389609in}{0.515609in}}%
\pgfpathclose%
\pgfpathmoveto{\pgfqpoint{0.671428in}{0.514441in}}%
\pgfpathlineto{\pgfqpoint{0.687084in}{0.512452in}}%
\pgfpathlineto{\pgfqpoint{0.702741in}{0.517011in}}%
\pgfpathlineto{\pgfqpoint{0.710696in}{0.522191in}}%
\pgfpathlineto{\pgfqpoint{0.718397in}{0.526588in}}%
\pgfpathlineto{\pgfqpoint{0.730130in}{0.535802in}}%
\pgfpathlineto{\pgfqpoint{0.734054in}{0.538747in}}%
\pgfpathlineto{\pgfqpoint{0.746549in}{0.549413in}}%
\pgfpathlineto{\pgfqpoint{0.749710in}{0.552258in}}%
\pgfpathlineto{\pgfqpoint{0.761542in}{0.563024in}}%
\pgfpathlineto{\pgfqpoint{0.765367in}{0.567238in}}%
\pgfpathlineto{\pgfqpoint{0.774604in}{0.576635in}}%
\pgfpathlineto{\pgfqpoint{0.781024in}{0.586820in}}%
\pgfpathlineto{\pgfqpoint{0.783578in}{0.590246in}}%
\pgfpathlineto{\pgfqpoint{0.786026in}{0.603857in}}%
\pgfpathlineto{\pgfqpoint{0.781024in}{0.615967in}}%
\pgfpathlineto{\pgfqpoint{0.780491in}{0.617468in}}%
\pgfpathlineto{\pgfqpoint{0.769740in}{0.631079in}}%
\pgfpathlineto{\pgfqpoint{0.765367in}{0.635083in}}%
\pgfpathlineto{\pgfqpoint{0.755736in}{0.644691in}}%
\pgfpathlineto{\pgfqpoint{0.749710in}{0.649973in}}%
\pgfpathlineto{\pgfqpoint{0.740130in}{0.658302in}}%
\pgfpathlineto{\pgfqpoint{0.734054in}{0.663540in}}%
\pgfpathlineto{\pgfqpoint{0.723002in}{0.671913in}}%
\pgfpathlineto{\pgfqpoint{0.718397in}{0.675714in}}%
\pgfpathlineto{\pgfqpoint{0.702741in}{0.685061in}}%
\pgfpathlineto{\pgfqpoint{0.701013in}{0.685524in}}%
\pgfpathlineto{\pgfqpoint{0.687084in}{0.689872in}}%
\pgfpathlineto{\pgfqpoint{0.671428in}{0.687745in}}%
\pgfpathlineto{\pgfqpoint{0.667486in}{0.685524in}}%
\pgfpathlineto{\pgfqpoint{0.655771in}{0.679943in}}%
\pgfpathlineto{\pgfqpoint{0.644962in}{0.671913in}}%
\pgfpathlineto{\pgfqpoint{0.640115in}{0.668587in}}%
\pgfpathlineto{\pgfqpoint{0.627731in}{0.658302in}}%
\pgfpathlineto{\pgfqpoint{0.624458in}{0.655554in}}%
\pgfpathlineto{\pgfqpoint{0.612189in}{0.644691in}}%
\pgfpathlineto{\pgfqpoint{0.608801in}{0.641280in}}%
\pgfpathlineto{\pgfqpoint{0.598203in}{0.631079in}}%
\pgfpathlineto{\pgfqpoint{0.593145in}{0.624384in}}%
\pgfpathlineto{\pgfqpoint{0.587187in}{0.617468in}}%
\pgfpathlineto{\pgfqpoint{0.581943in}{0.603857in}}%
\pgfpathlineto{\pgfqpoint{0.584231in}{0.590246in}}%
\pgfpathlineto{\pgfqpoint{0.593145in}{0.577132in}}%
\pgfpathlineto{\pgfqpoint{0.593444in}{0.576635in}}%
\pgfpathlineto{\pgfqpoint{0.606315in}{0.563024in}}%
\pgfpathlineto{\pgfqpoint{0.608801in}{0.560861in}}%
\pgfpathlineto{\pgfqpoint{0.621306in}{0.549413in}}%
\pgfpathlineto{\pgfqpoint{0.624458in}{0.546672in}}%
\pgfpathlineto{\pgfqpoint{0.637627in}{0.535802in}}%
\pgfpathlineto{\pgfqpoint{0.640115in}{0.533640in}}%
\pgfpathlineto{\pgfqpoint{0.655771in}{0.522450in}}%
\pgfpathlineto{\pgfqpoint{0.656342in}{0.522191in}}%
\pgfpathlineto{\pgfqpoint{0.671428in}{0.514441in}}%
\pgfpathclose%
\pgfpathmoveto{\pgfqpoint{0.685429in}{0.549413in}}%
\pgfpathlineto{\pgfqpoint{0.671428in}{0.550976in}}%
\pgfpathlineto{\pgfqpoint{0.655771in}{0.558042in}}%
\pgfpathlineto{\pgfqpoint{0.648724in}{0.563024in}}%
\pgfpathlineto{\pgfqpoint{0.640115in}{0.570509in}}%
\pgfpathlineto{\pgfqpoint{0.634384in}{0.576635in}}%
\pgfpathlineto{\pgfqpoint{0.626256in}{0.590246in}}%
\pgfpathlineto{\pgfqpoint{0.624458in}{0.602419in}}%
\pgfpathlineto{\pgfqpoint{0.624269in}{0.603857in}}%
\pgfpathlineto{\pgfqpoint{0.624458in}{0.604478in}}%
\pgfpathlineto{\pgfqpoint{0.628857in}{0.617468in}}%
\pgfpathlineto{\pgfqpoint{0.638951in}{0.631079in}}%
\pgfpathlineto{\pgfqpoint{0.640115in}{0.632199in}}%
\pgfpathlineto{\pgfqpoint{0.655771in}{0.644513in}}%
\pgfpathlineto{\pgfqpoint{0.656127in}{0.644691in}}%
\pgfpathlineto{\pgfqpoint{0.671428in}{0.651176in}}%
\pgfpathlineto{\pgfqpoint{0.687084in}{0.652817in}}%
\pgfpathlineto{\pgfqpoint{0.702741in}{0.649054in}}%
\pgfpathlineto{\pgfqpoint{0.710864in}{0.644691in}}%
\pgfpathlineto{\pgfqpoint{0.718397in}{0.639959in}}%
\pgfpathlineto{\pgfqpoint{0.728612in}{0.631079in}}%
\pgfpathlineto{\pgfqpoint{0.734054in}{0.624530in}}%
\pgfpathlineto{\pgfqpoint{0.739073in}{0.617468in}}%
\pgfpathlineto{\pgfqpoint{0.743402in}{0.603857in}}%
\pgfpathlineto{\pgfqpoint{0.741514in}{0.590246in}}%
\pgfpathlineto{\pgfqpoint{0.734054in}{0.576944in}}%
\pgfpathlineto{\pgfqpoint{0.733849in}{0.576635in}}%
\pgfpathlineto{\pgfqpoint{0.719685in}{0.563024in}}%
\pgfpathlineto{\pgfqpoint{0.718397in}{0.562012in}}%
\pgfpathlineto{\pgfqpoint{0.702741in}{0.553237in}}%
\pgfpathlineto{\pgfqpoint{0.687798in}{0.549413in}}%
\pgfpathlineto{\pgfqpoint{0.687084in}{0.549248in}}%
\pgfpathlineto{\pgfqpoint{0.685429in}{0.549413in}}%
\pgfpathclose%
\pgfpathmoveto{\pgfqpoint{0.968902in}{0.520460in}}%
\pgfpathlineto{\pgfqpoint{0.984559in}{0.513520in}}%
\pgfpathlineto{\pgfqpoint{1.000216in}{0.512855in}}%
\pgfpathlineto{\pgfqpoint{1.015872in}{0.518634in}}%
\pgfpathlineto{\pgfqpoint{1.020855in}{0.522191in}}%
\pgfpathlineto{\pgfqpoint{1.031529in}{0.528844in}}%
\pgfpathlineto{\pgfqpoint{1.040036in}{0.535802in}}%
\pgfpathlineto{\pgfqpoint{1.047185in}{0.541376in}}%
\pgfpathlineto{\pgfqpoint{1.056499in}{0.549413in}}%
\pgfpathlineto{\pgfqpoint{1.062842in}{0.555180in}}%
\pgfpathlineto{\pgfqpoint{1.071559in}{0.563024in}}%
\pgfpathlineto{\pgfqpoint{1.078498in}{0.570591in}}%
\pgfpathlineto{\pgfqpoint{1.084627in}{0.576635in}}%
\pgfpathlineto{\pgfqpoint{1.093434in}{0.590246in}}%
\pgfpathlineto{\pgfqpoint{1.094155in}{0.594772in}}%
\pgfpathlineto{\pgfqpoint{1.095863in}{0.603857in}}%
\pgfpathlineto{\pgfqpoint{1.094155in}{0.607777in}}%
\pgfpathlineto{\pgfqpoint{1.090616in}{0.617468in}}%
\pgfpathlineto{\pgfqpoint{1.079678in}{0.631079in}}%
\pgfpathlineto{\pgfqpoint{1.078498in}{0.632127in}}%
\pgfpathlineto{\pgfqpoint{1.065772in}{0.644691in}}%
\pgfpathlineto{\pgfqpoint{1.062842in}{0.647231in}}%
\pgfpathlineto{\pgfqpoint{1.050235in}{0.658302in}}%
\pgfpathlineto{\pgfqpoint{1.047185in}{0.660960in}}%
\pgfpathlineto{\pgfqpoint{1.033269in}{0.671913in}}%
\pgfpathlineto{\pgfqpoint{1.031529in}{0.673409in}}%
\pgfpathlineto{\pgfqpoint{1.015872in}{0.683541in}}%
\pgfpathlineto{\pgfqpoint{1.010047in}{0.685524in}}%
\pgfpathlineto{\pgfqpoint{1.000216in}{0.689442in}}%
\pgfpathlineto{\pgfqpoint{0.984559in}{0.688730in}}%
\pgfpathlineto{\pgfqpoint{0.977914in}{0.685524in}}%
\pgfpathlineto{\pgfqpoint{0.968902in}{0.681829in}}%
\pgfpathlineto{\pgfqpoint{0.954630in}{0.671913in}}%
\pgfpathlineto{\pgfqpoint{0.953246in}{0.671022in}}%
\pgfpathlineto{\pgfqpoint{0.937589in}{0.658363in}}%
\pgfpathlineto{\pgfqpoint{0.937520in}{0.658302in}}%
\pgfpathlineto{\pgfqpoint{0.922120in}{0.644691in}}%
\pgfpathlineto{\pgfqpoint{0.921933in}{0.644503in}}%
\pgfpathlineto{\pgfqpoint{0.908249in}{0.631079in}}%
\pgfpathlineto{\pgfqpoint{0.906276in}{0.628422in}}%
\pgfpathlineto{\pgfqpoint{0.897204in}{0.617468in}}%
\pgfpathlineto{\pgfqpoint{0.892097in}{0.603857in}}%
\pgfpathlineto{\pgfqpoint{0.894325in}{0.590246in}}%
\pgfpathlineto{\pgfqpoint{0.903325in}{0.576635in}}%
\pgfpathlineto{\pgfqpoint{0.906276in}{0.573825in}}%
\pgfpathlineto{\pgfqpoint{0.916322in}{0.563024in}}%
\pgfpathlineto{\pgfqpoint{0.921933in}{0.558051in}}%
\pgfpathlineto{\pgfqpoint{0.931380in}{0.549413in}}%
\pgfpathlineto{\pgfqpoint{0.937589in}{0.544023in}}%
\pgfpathlineto{\pgfqpoint{0.947805in}{0.535802in}}%
\pgfpathlineto{\pgfqpoint{0.953246in}{0.531200in}}%
\pgfpathlineto{\pgfqpoint{0.966661in}{0.522191in}}%
\pgfpathlineto{\pgfqpoint{0.968902in}{0.520460in}}%
\pgfpathclose%
\pgfpathmoveto{\pgfqpoint{0.958688in}{0.563024in}}%
\pgfpathlineto{\pgfqpoint{0.953246in}{0.567465in}}%
\pgfpathlineto{\pgfqpoint{0.944431in}{0.576635in}}%
\pgfpathlineto{\pgfqpoint{0.937589in}{0.587720in}}%
\pgfpathlineto{\pgfqpoint{0.936196in}{0.590246in}}%
\pgfpathlineto{\pgfqpoint{0.934390in}{0.603857in}}%
\pgfpathlineto{\pgfqpoint{0.937589in}{0.614328in}}%
\pgfpathlineto{\pgfqpoint{0.938680in}{0.617468in}}%
\pgfpathlineto{\pgfqpoint{0.949183in}{0.631079in}}%
\pgfpathlineto{\pgfqpoint{0.953246in}{0.634883in}}%
\pgfpathlineto{\pgfqpoint{0.966521in}{0.644691in}}%
\pgfpathlineto{\pgfqpoint{0.968902in}{0.646207in}}%
\pgfpathlineto{\pgfqpoint{0.984559in}{0.651936in}}%
\pgfpathlineto{\pgfqpoint{1.000216in}{0.652485in}}%
\pgfpathlineto{\pgfqpoint{1.015872in}{0.647715in}}%
\pgfpathlineto{\pgfqpoint{1.021008in}{0.644691in}}%
\pgfpathlineto{\pgfqpoint{1.031529in}{0.637477in}}%
\pgfpathlineto{\pgfqpoint{1.038595in}{0.631079in}}%
\pgfpathlineto{\pgfqpoint{1.047185in}{0.620340in}}%
\pgfpathlineto{\pgfqpoint{1.049204in}{0.617468in}}%
\pgfpathlineto{\pgfqpoint{1.053428in}{0.603857in}}%
\pgfpathlineto{\pgfqpoint{1.051585in}{0.590246in}}%
\pgfpathlineto{\pgfqpoint{1.047185in}{0.582314in}}%
\pgfpathlineto{\pgfqpoint{1.043568in}{0.576635in}}%
\pgfpathlineto{\pgfqpoint{1.031529in}{0.564524in}}%
\pgfpathlineto{\pgfqpoint{1.029548in}{0.563024in}}%
\pgfpathlineto{\pgfqpoint{1.015872in}{0.554664in}}%
\pgfpathlineto{\pgfqpoint{1.000216in}{0.549580in}}%
\pgfpathlineto{\pgfqpoint{0.984559in}{0.550165in}}%
\pgfpathlineto{\pgfqpoint{0.968902in}{0.556271in}}%
\pgfpathlineto{\pgfqpoint{0.958688in}{0.563024in}}%
\pgfpathclose%
\pgfpathmoveto{\pgfqpoint{1.282034in}{0.518634in}}%
\pgfpathlineto{\pgfqpoint{1.297690in}{0.512855in}}%
\pgfpathlineto{\pgfqpoint{1.313347in}{0.513520in}}%
\pgfpathlineto{\pgfqpoint{1.329003in}{0.520460in}}%
\pgfpathlineto{\pgfqpoint{1.331245in}{0.522191in}}%
\pgfpathlineto{\pgfqpoint{1.344660in}{0.531200in}}%
\pgfpathlineto{\pgfqpoint{1.350100in}{0.535802in}}%
\pgfpathlineto{\pgfqpoint{1.360317in}{0.544023in}}%
\pgfpathlineto{\pgfqpoint{1.366525in}{0.549413in}}%
\pgfpathlineto{\pgfqpoint{1.375973in}{0.558051in}}%
\pgfpathlineto{\pgfqpoint{1.381584in}{0.563024in}}%
\pgfpathlineto{\pgfqpoint{1.391630in}{0.573825in}}%
\pgfpathlineto{\pgfqpoint{1.394581in}{0.576635in}}%
\pgfpathlineto{\pgfqpoint{1.403581in}{0.590246in}}%
\pgfpathlineto{\pgfqpoint{1.405809in}{0.603857in}}%
\pgfpathlineto{\pgfqpoint{1.400701in}{0.617468in}}%
\pgfpathlineto{\pgfqpoint{1.391630in}{0.628422in}}%
\pgfpathlineto{\pgfqpoint{1.389657in}{0.631079in}}%
\pgfpathlineto{\pgfqpoint{1.375973in}{0.644503in}}%
\pgfpathlineto{\pgfqpoint{1.375786in}{0.644691in}}%
\pgfpathlineto{\pgfqpoint{1.360386in}{0.658302in}}%
\pgfpathlineto{\pgfqpoint{1.360317in}{0.658363in}}%
\pgfpathlineto{\pgfqpoint{1.344660in}{0.671022in}}%
\pgfpathlineto{\pgfqpoint{1.343276in}{0.671913in}}%
\pgfpathlineto{\pgfqpoint{1.329003in}{0.681829in}}%
\pgfpathlineto{\pgfqpoint{1.319992in}{0.685524in}}%
\pgfpathlineto{\pgfqpoint{1.313347in}{0.688730in}}%
\pgfpathlineto{\pgfqpoint{1.297690in}{0.689442in}}%
\pgfpathlineto{\pgfqpoint{1.287859in}{0.685524in}}%
\pgfpathlineto{\pgfqpoint{1.282034in}{0.683541in}}%
\pgfpathlineto{\pgfqpoint{1.266377in}{0.673409in}}%
\pgfpathlineto{\pgfqpoint{1.264637in}{0.671913in}}%
\pgfpathlineto{\pgfqpoint{1.250721in}{0.660960in}}%
\pgfpathlineto{\pgfqpoint{1.247671in}{0.658302in}}%
\pgfpathlineto{\pgfqpoint{1.235064in}{0.647231in}}%
\pgfpathlineto{\pgfqpoint{1.232134in}{0.644691in}}%
\pgfpathlineto{\pgfqpoint{1.219407in}{0.632127in}}%
\pgfpathlineto{\pgfqpoint{1.218228in}{0.631079in}}%
\pgfpathlineto{\pgfqpoint{1.207290in}{0.617468in}}%
\pgfpathlineto{\pgfqpoint{1.203751in}{0.607777in}}%
\pgfpathlineto{\pgfqpoint{1.202042in}{0.603857in}}%
\pgfpathlineto{\pgfqpoint{1.203751in}{0.594772in}}%
\pgfpathlineto{\pgfqpoint{1.204472in}{0.590246in}}%
\pgfpathlineto{\pgfqpoint{1.213279in}{0.576635in}}%
\pgfpathlineto{\pgfqpoint{1.219407in}{0.570591in}}%
\pgfpathlineto{\pgfqpoint{1.226347in}{0.563024in}}%
\pgfpathlineto{\pgfqpoint{1.235064in}{0.555180in}}%
\pgfpathlineto{\pgfqpoint{1.241407in}{0.549413in}}%
\pgfpathlineto{\pgfqpoint{1.250721in}{0.541376in}}%
\pgfpathlineto{\pgfqpoint{1.257869in}{0.535802in}}%
\pgfpathlineto{\pgfqpoint{1.266377in}{0.528844in}}%
\pgfpathlineto{\pgfqpoint{1.277051in}{0.522191in}}%
\pgfpathlineto{\pgfqpoint{1.282034in}{0.518634in}}%
\pgfpathclose%
\pgfpathmoveto{\pgfqpoint{1.268358in}{0.563024in}}%
\pgfpathlineto{\pgfqpoint{1.266377in}{0.564524in}}%
\pgfpathlineto{\pgfqpoint{1.254338in}{0.576635in}}%
\pgfpathlineto{\pgfqpoint{1.250721in}{0.582314in}}%
\pgfpathlineto{\pgfqpoint{1.246320in}{0.590246in}}%
\pgfpathlineto{\pgfqpoint{1.244478in}{0.603857in}}%
\pgfpathlineto{\pgfqpoint{1.248701in}{0.617468in}}%
\pgfpathlineto{\pgfqpoint{1.250721in}{0.620340in}}%
\pgfpathlineto{\pgfqpoint{1.259311in}{0.631079in}}%
\pgfpathlineto{\pgfqpoint{1.266377in}{0.637477in}}%
\pgfpathlineto{\pgfqpoint{1.276898in}{0.644691in}}%
\pgfpathlineto{\pgfqpoint{1.282034in}{0.647715in}}%
\pgfpathlineto{\pgfqpoint{1.297690in}{0.652485in}}%
\pgfpathlineto{\pgfqpoint{1.313347in}{0.651936in}}%
\pgfpathlineto{\pgfqpoint{1.329003in}{0.646207in}}%
\pgfpathlineto{\pgfqpoint{1.331384in}{0.644691in}}%
\pgfpathlineto{\pgfqpoint{1.344660in}{0.634883in}}%
\pgfpathlineto{\pgfqpoint{1.348723in}{0.631079in}}%
\pgfpathlineto{\pgfqpoint{1.359226in}{0.617468in}}%
\pgfpathlineto{\pgfqpoint{1.360317in}{0.614328in}}%
\pgfpathlineto{\pgfqpoint{1.363515in}{0.603857in}}%
\pgfpathlineto{\pgfqpoint{1.361710in}{0.590246in}}%
\pgfpathlineto{\pgfqpoint{1.360317in}{0.587720in}}%
\pgfpathlineto{\pgfqpoint{1.353474in}{0.576635in}}%
\pgfpathlineto{\pgfqpoint{1.344660in}{0.567465in}}%
\pgfpathlineto{\pgfqpoint{1.339217in}{0.563024in}}%
\pgfpathlineto{\pgfqpoint{1.329003in}{0.556271in}}%
\pgfpathlineto{\pgfqpoint{1.313347in}{0.550165in}}%
\pgfpathlineto{\pgfqpoint{1.297690in}{0.549580in}}%
\pgfpathlineto{\pgfqpoint{1.282034in}{0.554664in}}%
\pgfpathlineto{\pgfqpoint{1.268358in}{0.563024in}}%
\pgfpathclose%
\pgfpathmoveto{\pgfqpoint{1.595165in}{0.517011in}}%
\pgfpathlineto{\pgfqpoint{1.610822in}{0.512452in}}%
\pgfpathlineto{\pgfqpoint{1.626478in}{0.514441in}}%
\pgfpathlineto{\pgfqpoint{1.641563in}{0.522191in}}%
\pgfpathlineto{\pgfqpoint{1.642135in}{0.522450in}}%
\pgfpathlineto{\pgfqpoint{1.657791in}{0.533640in}}%
\pgfpathlineto{\pgfqpoint{1.660279in}{0.535802in}}%
\pgfpathlineto{\pgfqpoint{1.673448in}{0.546672in}}%
\pgfpathlineto{\pgfqpoint{1.676600in}{0.549413in}}%
\pgfpathlineto{\pgfqpoint{1.689104in}{0.560861in}}%
\pgfpathlineto{\pgfqpoint{1.691591in}{0.563024in}}%
\pgfpathlineto{\pgfqpoint{1.704462in}{0.576635in}}%
\pgfpathlineto{\pgfqpoint{1.704761in}{0.577132in}}%
\pgfpathlineto{\pgfqpoint{1.713675in}{0.590246in}}%
\pgfpathlineto{\pgfqpoint{1.715962in}{0.603857in}}%
\pgfpathlineto{\pgfqpoint{1.710719in}{0.617468in}}%
\pgfpathlineto{\pgfqpoint{1.704761in}{0.624384in}}%
\pgfpathlineto{\pgfqpoint{1.699703in}{0.631079in}}%
\pgfpathlineto{\pgfqpoint{1.689104in}{0.641280in}}%
\pgfpathlineto{\pgfqpoint{1.685716in}{0.644691in}}%
\pgfpathlineto{\pgfqpoint{1.673448in}{0.655554in}}%
\pgfpathlineto{\pgfqpoint{1.670175in}{0.658302in}}%
\pgfpathlineto{\pgfqpoint{1.657791in}{0.668587in}}%
\pgfpathlineto{\pgfqpoint{1.652943in}{0.671913in}}%
\pgfpathlineto{\pgfqpoint{1.642135in}{0.679943in}}%
\pgfpathlineto{\pgfqpoint{1.630419in}{0.685524in}}%
\pgfpathlineto{\pgfqpoint{1.626478in}{0.687745in}}%
\pgfpathlineto{\pgfqpoint{1.610822in}{0.689872in}}%
\pgfpathlineto{\pgfqpoint{1.596892in}{0.685524in}}%
\pgfpathlineto{\pgfqpoint{1.595165in}{0.685061in}}%
\pgfpathlineto{\pgfqpoint{1.579508in}{0.675714in}}%
\pgfpathlineto{\pgfqpoint{1.574903in}{0.671913in}}%
\pgfpathlineto{\pgfqpoint{1.563852in}{0.663540in}}%
\pgfpathlineto{\pgfqpoint{1.557776in}{0.658302in}}%
\pgfpathlineto{\pgfqpoint{1.548195in}{0.649973in}}%
\pgfpathlineto{\pgfqpoint{1.542170in}{0.644691in}}%
\pgfpathlineto{\pgfqpoint{1.532539in}{0.635083in}}%
\pgfpathlineto{\pgfqpoint{1.528166in}{0.631079in}}%
\pgfpathlineto{\pgfqpoint{1.517415in}{0.617468in}}%
\pgfpathlineto{\pgfqpoint{1.516882in}{0.615967in}}%
\pgfpathlineto{\pgfqpoint{1.511880in}{0.603857in}}%
\pgfpathlineto{\pgfqpoint{1.514328in}{0.590246in}}%
\pgfpathlineto{\pgfqpoint{1.516882in}{0.586820in}}%
\pgfpathlineto{\pgfqpoint{1.523302in}{0.576635in}}%
\pgfpathlineto{\pgfqpoint{1.532539in}{0.567238in}}%
\pgfpathlineto{\pgfqpoint{1.536364in}{0.563024in}}%
\pgfpathlineto{\pgfqpoint{1.548195in}{0.552258in}}%
\pgfpathlineto{\pgfqpoint{1.551356in}{0.549413in}}%
\pgfpathlineto{\pgfqpoint{1.563852in}{0.538747in}}%
\pgfpathlineto{\pgfqpoint{1.567776in}{0.535802in}}%
\pgfpathlineto{\pgfqpoint{1.579508in}{0.526588in}}%
\pgfpathlineto{\pgfqpoint{1.587210in}{0.522191in}}%
\pgfpathlineto{\pgfqpoint{1.595165in}{0.517011in}}%
\pgfpathclose%
\pgfpathmoveto{\pgfqpoint{1.610108in}{0.549413in}}%
\pgfpathlineto{\pgfqpoint{1.595165in}{0.553237in}}%
\pgfpathlineto{\pgfqpoint{1.579508in}{0.562012in}}%
\pgfpathlineto{\pgfqpoint{1.578221in}{0.563024in}}%
\pgfpathlineto{\pgfqpoint{1.564056in}{0.576635in}}%
\pgfpathlineto{\pgfqpoint{1.563852in}{0.576944in}}%
\pgfpathlineto{\pgfqpoint{1.556392in}{0.590246in}}%
\pgfpathlineto{\pgfqpoint{1.554504in}{0.603857in}}%
\pgfpathlineto{\pgfqpoint{1.558832in}{0.617468in}}%
\pgfpathlineto{\pgfqpoint{1.563852in}{0.624530in}}%
\pgfpathlineto{\pgfqpoint{1.569294in}{0.631079in}}%
\pgfpathlineto{\pgfqpoint{1.579508in}{0.639959in}}%
\pgfpathlineto{\pgfqpoint{1.587042in}{0.644691in}}%
\pgfpathlineto{\pgfqpoint{1.595165in}{0.649054in}}%
\pgfpathlineto{\pgfqpoint{1.610822in}{0.652817in}}%
\pgfpathlineto{\pgfqpoint{1.626478in}{0.651176in}}%
\pgfpathlineto{\pgfqpoint{1.641779in}{0.644691in}}%
\pgfpathlineto{\pgfqpoint{1.642135in}{0.644513in}}%
\pgfpathlineto{\pgfqpoint{1.657791in}{0.632199in}}%
\pgfpathlineto{\pgfqpoint{1.658955in}{0.631079in}}%
\pgfpathlineto{\pgfqpoint{1.669049in}{0.617468in}}%
\pgfpathlineto{\pgfqpoint{1.673448in}{0.604478in}}%
\pgfpathlineto{\pgfqpoint{1.673637in}{0.603857in}}%
\pgfpathlineto{\pgfqpoint{1.673448in}{0.602419in}}%
\pgfpathlineto{\pgfqpoint{1.671650in}{0.590246in}}%
\pgfpathlineto{\pgfqpoint{1.663522in}{0.576635in}}%
\pgfpathlineto{\pgfqpoint{1.657791in}{0.570509in}}%
\pgfpathlineto{\pgfqpoint{1.649181in}{0.563024in}}%
\pgfpathlineto{\pgfqpoint{1.642135in}{0.558042in}}%
\pgfpathlineto{\pgfqpoint{1.626478in}{0.550976in}}%
\pgfpathlineto{\pgfqpoint{1.612476in}{0.549413in}}%
\pgfpathlineto{\pgfqpoint{1.610822in}{0.549248in}}%
\pgfpathlineto{\pgfqpoint{1.610108in}{0.549413in}}%
\pgfpathclose%
\pgfpathmoveto{\pgfqpoint{1.908296in}{0.515609in}}%
\pgfpathlineto{\pgfqpoint{1.923953in}{0.512318in}}%
\pgfpathlineto{\pgfqpoint{1.923953in}{0.522191in}}%
\pgfpathlineto{\pgfqpoint{1.923953in}{0.535802in}}%
\pgfpathlineto{\pgfqpoint{1.923953in}{0.549144in}}%
\pgfpathlineto{\pgfqpoint{1.922326in}{0.549413in}}%
\pgfpathlineto{\pgfqpoint{1.908296in}{0.552003in}}%
\pgfpathlineto{\pgfqpoint{1.892640in}{0.559962in}}%
\pgfpathlineto{\pgfqpoint{1.888548in}{0.563024in}}%
\pgfpathlineto{\pgfqpoint{1.876983in}{0.573635in}}%
\pgfpathlineto{\pgfqpoint{1.874236in}{0.576635in}}%
\pgfpathlineto{\pgfqpoint{1.866382in}{0.590246in}}%
\pgfpathlineto{\pgfqpoint{1.864438in}{0.603857in}}%
\pgfpathlineto{\pgfqpoint{1.868895in}{0.617468in}}%
\pgfpathlineto{\pgfqpoint{1.876983in}{0.628669in}}%
\pgfpathlineto{\pgfqpoint{1.879080in}{0.631079in}}%
\pgfpathlineto{\pgfqpoint{1.892640in}{0.642311in}}%
\pgfpathlineto{\pgfqpoint{1.896843in}{0.644691in}}%
\pgfpathlineto{\pgfqpoint{1.908296in}{0.650212in}}%
\pgfpathlineto{\pgfqpoint{1.923953in}{0.652928in}}%
\pgfpathlineto{\pgfqpoint{1.923953in}{0.658302in}}%
\pgfpathlineto{\pgfqpoint{1.923953in}{0.671913in}}%
\pgfpathlineto{\pgfqpoint{1.923953in}{0.685524in}}%
\pgfpathlineto{\pgfqpoint{1.923953in}{0.690016in}}%
\pgfpathlineto{\pgfqpoint{1.908296in}{0.686495in}}%
\pgfpathlineto{\pgfqpoint{1.906781in}{0.685524in}}%
\pgfpathlineto{\pgfqpoint{1.892640in}{0.677898in}}%
\pgfpathlineto{\pgfqpoint{1.885030in}{0.671913in}}%
\pgfpathlineto{\pgfqpoint{1.876983in}{0.666088in}}%
\pgfpathlineto{\pgfqpoint{1.867807in}{0.658302in}}%
\pgfpathlineto{\pgfqpoint{1.861327in}{0.652751in}}%
\pgfpathlineto{\pgfqpoint{1.852197in}{0.644691in}}%
\pgfpathlineto{\pgfqpoint{1.845670in}{0.638136in}}%
\pgfpathlineto{\pgfqpoint{1.838167in}{0.631079in}}%
\pgfpathlineto{\pgfqpoint{1.830014in}{0.620516in}}%
\pgfpathlineto{\pgfqpoint{1.827270in}{0.617468in}}%
\pgfpathlineto{\pgfqpoint{1.821860in}{0.603857in}}%
\pgfpathlineto{\pgfqpoint{1.824220in}{0.590246in}}%
\pgfpathlineto{\pgfqpoint{1.830014in}{0.582089in}}%
\pgfpathlineto{\pgfqpoint{1.833366in}{0.576635in}}%
\pgfpathlineto{\pgfqpoint{1.845670in}{0.563776in}}%
\pgfpathlineto{\pgfqpoint{1.846348in}{0.563024in}}%
\pgfpathlineto{\pgfqpoint{1.861209in}{0.549413in}}%
\pgfpathlineto{\pgfqpoint{1.861327in}{0.549311in}}%
\pgfpathlineto{\pgfqpoint{1.876983in}{0.536150in}}%
\pgfpathlineto{\pgfqpoint{1.877469in}{0.535802in}}%
\pgfpathlineto{\pgfqpoint{1.892640in}{0.524451in}}%
\pgfpathlineto{\pgfqpoint{1.897031in}{0.522191in}}%
\pgfpathlineto{\pgfqpoint{1.908296in}{0.515609in}}%
\pgfpathclose%
\pgfpathmoveto{\pgfqpoint{0.389609in}{0.785160in}}%
\pgfpathlineto{\pgfqpoint{0.405266in}{0.793973in}}%
\pgfpathlineto{\pgfqpoint{0.405790in}{0.794413in}}%
\pgfpathlineto{\pgfqpoint{0.420923in}{0.805639in}}%
\pgfpathlineto{\pgfqpoint{0.423660in}{0.808024in}}%
\pgfpathlineto{\pgfqpoint{0.436579in}{0.818918in}}%
\pgfpathlineto{\pgfqpoint{0.439709in}{0.821635in}}%
\pgfpathlineto{\pgfqpoint{0.452236in}{0.833368in}}%
\pgfpathlineto{\pgfqpoint{0.454386in}{0.835246in}}%
\pgfpathlineto{\pgfqpoint{0.466682in}{0.848857in}}%
\pgfpathlineto{\pgfqpoint{0.467892in}{0.851144in}}%
\pgfpathlineto{\pgfqpoint{0.474779in}{0.862468in}}%
\pgfpathlineto{\pgfqpoint{0.475568in}{0.876079in}}%
\pgfpathlineto{\pgfqpoint{0.468710in}{0.889691in}}%
\pgfpathlineto{\pgfqpoint{0.467892in}{0.890520in}}%
\pgfpathlineto{\pgfqpoint{0.457121in}{0.903302in}}%
\pgfpathlineto{\pgfqpoint{0.452236in}{0.907714in}}%
\pgfpathlineto{\pgfqpoint{0.442719in}{0.916913in}}%
\pgfpathlineto{\pgfqpoint{0.436579in}{0.922275in}}%
\pgfpathlineto{\pgfqpoint{0.426851in}{0.930524in}}%
\pgfpathlineto{\pgfqpoint{0.420923in}{0.935611in}}%
\pgfpathlineto{\pgfqpoint{0.409269in}{0.944135in}}%
\pgfpathlineto{\pgfqpoint{0.405266in}{0.947383in}}%
\pgfpathlineto{\pgfqpoint{0.389609in}{0.956006in}}%
\pgfpathlineto{\pgfqpoint{0.380871in}{0.957746in}}%
\pgfpathlineto{\pgfqpoint{0.373953in}{0.959381in}}%
\pgfpathlineto{\pgfqpoint{0.373953in}{0.957746in}}%
\pgfpathlineto{\pgfqpoint{0.373953in}{0.944135in}}%
\pgfpathlineto{\pgfqpoint{0.373953in}{0.930524in}}%
\pgfpathlineto{\pgfqpoint{0.373953in}{0.922448in}}%
\pgfpathlineto{\pgfqpoint{0.389609in}{0.919798in}}%
\pgfpathlineto{\pgfqpoint{0.395659in}{0.916913in}}%
\pgfpathlineto{\pgfqpoint{0.405266in}{0.911678in}}%
\pgfpathlineto{\pgfqpoint{0.415797in}{0.903302in}}%
\pgfpathlineto{\pgfqpoint{0.420923in}{0.897903in}}%
\pgfpathlineto{\pgfqpoint{0.427425in}{0.889691in}}%
\pgfpathlineto{\pgfqpoint{0.433075in}{0.876079in}}%
\pgfpathlineto{\pgfqpoint{0.432425in}{0.862468in}}%
\pgfpathlineto{\pgfqpoint{0.425639in}{0.848857in}}%
\pgfpathlineto{\pgfqpoint{0.420923in}{0.843350in}}%
\pgfpathlineto{\pgfqpoint{0.412633in}{0.835246in}}%
\pgfpathlineto{\pgfqpoint{0.405266in}{0.829580in}}%
\pgfpathlineto{\pgfqpoint{0.390220in}{0.821635in}}%
\pgfpathlineto{\pgfqpoint{0.389609in}{0.821346in}}%
\pgfpathlineto{\pgfqpoint{0.373953in}{0.818748in}}%
\pgfpathlineto{\pgfqpoint{0.373953in}{0.808024in}}%
\pgfpathlineto{\pgfqpoint{0.373953in}{0.794413in}}%
\pgfpathlineto{\pgfqpoint{0.373953in}{0.781955in}}%
\pgfpathlineto{\pgfqpoint{0.389609in}{0.785160in}}%
\pgfpathclose%
\pgfpathmoveto{\pgfqpoint{0.655771in}{0.791847in}}%
\pgfpathlineto{\pgfqpoint{0.671428in}{0.784023in}}%
\pgfpathlineto{\pgfqpoint{0.687084in}{0.782086in}}%
\pgfpathlineto{\pgfqpoint{0.702741in}{0.786526in}}%
\pgfpathlineto{\pgfqpoint{0.715340in}{0.794413in}}%
\pgfpathlineto{\pgfqpoint{0.718397in}{0.796127in}}%
\pgfpathlineto{\pgfqpoint{0.733838in}{0.808024in}}%
\pgfpathlineto{\pgfqpoint{0.734054in}{0.808186in}}%
\pgfpathlineto{\pgfqpoint{0.749710in}{0.821574in}}%
\pgfpathlineto{\pgfqpoint{0.749781in}{0.821635in}}%
\pgfpathlineto{\pgfqpoint{0.764342in}{0.835246in}}%
\pgfpathlineto{\pgfqpoint{0.765367in}{0.836449in}}%
\pgfpathlineto{\pgfqpoint{0.776774in}{0.848857in}}%
\pgfpathlineto{\pgfqpoint{0.781024in}{0.856691in}}%
\pgfpathlineto{\pgfqpoint{0.784712in}{0.862468in}}%
\pgfpathlineto{\pgfqpoint{0.785531in}{0.876079in}}%
\pgfpathlineto{\pgfqpoint{0.781024in}{0.884626in}}%
\pgfpathlineto{\pgfqpoint{0.778742in}{0.889691in}}%
\pgfpathlineto{\pgfqpoint{0.767088in}{0.903302in}}%
\pgfpathlineto{\pgfqpoint{0.765367in}{0.904814in}}%
\pgfpathlineto{\pgfqpoint{0.752768in}{0.916913in}}%
\pgfpathlineto{\pgfqpoint{0.749710in}{0.919564in}}%
\pgfpathlineto{\pgfqpoint{0.736976in}{0.930524in}}%
\pgfpathlineto{\pgfqpoint{0.734054in}{0.933071in}}%
\pgfpathlineto{\pgfqpoint{0.719602in}{0.944135in}}%
\pgfpathlineto{\pgfqpoint{0.718397in}{0.945161in}}%
\pgfpathlineto{\pgfqpoint{0.702741in}{0.954670in}}%
\pgfpathlineto{\pgfqpoint{0.691593in}{0.957746in}}%
\pgfpathlineto{\pgfqpoint{0.687084in}{0.959231in}}%
\pgfpathlineto{\pgfqpoint{0.676633in}{0.957746in}}%
\pgfpathlineto{\pgfqpoint{0.671428in}{0.957119in}}%
\pgfpathlineto{\pgfqpoint{0.655771in}{0.949463in}}%
\pgfpathlineto{\pgfqpoint{0.648819in}{0.944135in}}%
\pgfpathlineto{\pgfqpoint{0.640115in}{0.938102in}}%
\pgfpathlineto{\pgfqpoint{0.631092in}{0.930524in}}%
\pgfpathlineto{\pgfqpoint{0.624458in}{0.925010in}}%
\pgfpathlineto{\pgfqpoint{0.615214in}{0.916913in}}%
\pgfpathlineto{\pgfqpoint{0.608801in}{0.910698in}}%
\pgfpathlineto{\pgfqpoint{0.600798in}{0.903302in}}%
\pgfpathlineto{\pgfqpoint{0.593145in}{0.894023in}}%
\pgfpathlineto{\pgfqpoint{0.589053in}{0.889691in}}%
\pgfpathlineto{\pgfqpoint{0.582406in}{0.876079in}}%
\pgfpathlineto{\pgfqpoint{0.583171in}{0.862468in}}%
\pgfpathlineto{\pgfqpoint{0.591155in}{0.848857in}}%
\pgfpathlineto{\pgfqpoint{0.593145in}{0.846909in}}%
\pgfpathlineto{\pgfqpoint{0.603509in}{0.835246in}}%
\pgfpathlineto{\pgfqpoint{0.608801in}{0.830517in}}%
\pgfpathlineto{\pgfqpoint{0.618258in}{0.821635in}}%
\pgfpathlineto{\pgfqpoint{0.624458in}{0.816237in}}%
\pgfpathlineto{\pgfqpoint{0.634394in}{0.808024in}}%
\pgfpathlineto{\pgfqpoint{0.640115in}{0.803146in}}%
\pgfpathlineto{\pgfqpoint{0.652538in}{0.794413in}}%
\pgfpathlineto{\pgfqpoint{0.655771in}{0.791847in}}%
\pgfpathclose%
\pgfpathmoveto{\pgfqpoint{0.668522in}{0.821635in}}%
\pgfpathlineto{\pgfqpoint{0.655771in}{0.827583in}}%
\pgfpathlineto{\pgfqpoint{0.645223in}{0.835246in}}%
\pgfpathlineto{\pgfqpoint{0.640115in}{0.839978in}}%
\pgfpathlineto{\pgfqpoint{0.632347in}{0.848857in}}%
\pgfpathlineto{\pgfqpoint{0.625324in}{0.862468in}}%
\pgfpathlineto{\pgfqpoint{0.624650in}{0.876079in}}%
\pgfpathlineto{\pgfqpoint{0.630498in}{0.889691in}}%
\pgfpathlineto{\pgfqpoint{0.640115in}{0.901580in}}%
\pgfpathlineto{\pgfqpoint{0.641840in}{0.903302in}}%
\pgfpathlineto{\pgfqpoint{0.655771in}{0.913768in}}%
\pgfpathlineto{\pgfqpoint{0.662304in}{0.916913in}}%
\pgfpathlineto{\pgfqpoint{0.671428in}{0.920738in}}%
\pgfpathlineto{\pgfqpoint{0.687084in}{0.922340in}}%
\pgfpathlineto{\pgfqpoint{0.702741in}{0.918668in}}%
\pgfpathlineto{\pgfqpoint{0.706044in}{0.916913in}}%
\pgfpathlineto{\pgfqpoint{0.718397in}{0.909445in}}%
\pgfpathlineto{\pgfqpoint{0.725756in}{0.903302in}}%
\pgfpathlineto{\pgfqpoint{0.734054in}{0.894155in}}%
\pgfpathlineto{\pgfqpoint{0.737533in}{0.889691in}}%
\pgfpathlineto{\pgfqpoint{0.743020in}{0.876079in}}%
\pgfpathlineto{\pgfqpoint{0.742389in}{0.862468in}}%
\pgfpathlineto{\pgfqpoint{0.735798in}{0.848857in}}%
\pgfpathlineto{\pgfqpoint{0.734054in}{0.846787in}}%
\pgfpathlineto{\pgfqpoint{0.722773in}{0.835246in}}%
\pgfpathlineto{\pgfqpoint{0.718397in}{0.831714in}}%
\pgfpathlineto{\pgfqpoint{0.702741in}{0.822583in}}%
\pgfpathlineto{\pgfqpoint{0.699129in}{0.821635in}}%
\pgfpathlineto{\pgfqpoint{0.687084in}{0.818854in}}%
\pgfpathlineto{\pgfqpoint{0.671428in}{0.820424in}}%
\pgfpathlineto{\pgfqpoint{0.668522in}{0.821635in}}%
\pgfpathclose%
\pgfpathmoveto{\pgfqpoint{0.968902in}{0.789886in}}%
\pgfpathlineto{\pgfqpoint{0.984559in}{0.783125in}}%
\pgfpathlineto{\pgfqpoint{1.000216in}{0.782477in}}%
\pgfpathlineto{\pgfqpoint{1.015872in}{0.788107in}}%
\pgfpathlineto{\pgfqpoint{1.025062in}{0.794413in}}%
\pgfpathlineto{\pgfqpoint{1.031529in}{0.798373in}}%
\pgfpathlineto{\pgfqpoint{1.043557in}{0.808024in}}%
\pgfpathlineto{\pgfqpoint{1.047185in}{0.810857in}}%
\pgfpathlineto{\pgfqpoint{1.059657in}{0.821635in}}%
\pgfpathlineto{\pgfqpoint{1.062842in}{0.824605in}}%
\pgfpathlineto{\pgfqpoint{1.074350in}{0.835246in}}%
\pgfpathlineto{\pgfqpoint{1.078498in}{0.840066in}}%
\pgfpathlineto{\pgfqpoint{1.086834in}{0.848857in}}%
\pgfpathlineto{\pgfqpoint{1.094155in}{0.861966in}}%
\pgfpathlineto{\pgfqpoint{1.094494in}{0.862468in}}%
\pgfpathlineto{\pgfqpoint{1.095347in}{0.876079in}}%
\pgfpathlineto{\pgfqpoint{1.094155in}{0.878222in}}%
\pgfpathlineto{\pgfqpoint{1.088837in}{0.889691in}}%
\pgfpathlineto{\pgfqpoint{1.078498in}{0.901484in}}%
\pgfpathlineto{\pgfqpoint{1.077046in}{0.903302in}}%
\pgfpathlineto{\pgfqpoint{1.062842in}{0.916885in}}%
\pgfpathlineto{\pgfqpoint{1.062813in}{0.916913in}}%
\pgfpathlineto{\pgfqpoint{1.047185in}{0.930499in}}%
\pgfpathlineto{\pgfqpoint{1.047153in}{0.930524in}}%
\pgfpathlineto{\pgfqpoint{1.031529in}{0.942872in}}%
\pgfpathlineto{\pgfqpoint{1.029438in}{0.944135in}}%
\pgfpathlineto{\pgfqpoint{1.015872in}{0.953123in}}%
\pgfpathlineto{\pgfqpoint{1.002680in}{0.957746in}}%
\pgfpathlineto{\pgfqpoint{1.000216in}{0.958783in}}%
\pgfpathlineto{\pgfqpoint{0.984559in}{0.958041in}}%
\pgfpathlineto{\pgfqpoint{0.983981in}{0.957746in}}%
\pgfpathlineto{\pgfqpoint{0.968902in}{0.951382in}}%
\pgfpathlineto{\pgfqpoint{0.958790in}{0.944135in}}%
\pgfpathlineto{\pgfqpoint{0.953246in}{0.940528in}}%
\pgfpathlineto{\pgfqpoint{0.941005in}{0.930524in}}%
\pgfpathlineto{\pgfqpoint{0.937589in}{0.927756in}}%
\pgfpathlineto{\pgfqpoint{0.925192in}{0.916913in}}%
\pgfpathlineto{\pgfqpoint{0.921933in}{0.913758in}}%
\pgfpathlineto{\pgfqpoint{0.910831in}{0.903302in}}%
\pgfpathlineto{\pgfqpoint{0.906276in}{0.897680in}}%
\pgfpathlineto{\pgfqpoint{0.899022in}{0.889691in}}%
\pgfpathlineto{\pgfqpoint{0.892547in}{0.876079in}}%
\pgfpathlineto{\pgfqpoint{0.893292in}{0.862468in}}%
\pgfpathlineto{\pgfqpoint{0.901069in}{0.848857in}}%
\pgfpathlineto{\pgfqpoint{0.906276in}{0.843555in}}%
\pgfpathlineto{\pgfqpoint{0.913529in}{0.835246in}}%
\pgfpathlineto{\pgfqpoint{0.921933in}{0.827592in}}%
\pgfpathlineto{\pgfqpoint{0.928285in}{0.821635in}}%
\pgfpathlineto{\pgfqpoint{0.937589in}{0.813546in}}%
\pgfpathlineto{\pgfqpoint{0.944442in}{0.808024in}}%
\pgfpathlineto{\pgfqpoint{0.953246in}{0.800718in}}%
\pgfpathlineto{\pgfqpoint{0.962803in}{0.794413in}}%
\pgfpathlineto{\pgfqpoint{0.968902in}{0.789886in}}%
\pgfpathclose%
\pgfpathmoveto{\pgfqpoint{0.979128in}{0.821635in}}%
\pgfpathlineto{\pgfqpoint{0.968902in}{0.825740in}}%
\pgfpathlineto{\pgfqpoint{0.954911in}{0.835246in}}%
\pgfpathlineto{\pgfqpoint{0.953246in}{0.836693in}}%
\pgfpathlineto{\pgfqpoint{0.942311in}{0.848857in}}%
\pgfpathlineto{\pgfqpoint{0.937589in}{0.857746in}}%
\pgfpathlineto{\pgfqpoint{0.935360in}{0.862468in}}%
\pgfpathlineto{\pgfqpoint{0.934755in}{0.876079in}}%
\pgfpathlineto{\pgfqpoint{0.937589in}{0.883345in}}%
\pgfpathlineto{\pgfqpoint{0.940388in}{0.889691in}}%
\pgfpathlineto{\pgfqpoint{0.951774in}{0.903302in}}%
\pgfpathlineto{\pgfqpoint{0.953246in}{0.904625in}}%
\pgfpathlineto{\pgfqpoint{0.968902in}{0.915697in}}%
\pgfpathlineto{\pgfqpoint{0.971838in}{0.916913in}}%
\pgfpathlineto{\pgfqpoint{0.984559in}{0.921480in}}%
\pgfpathlineto{\pgfqpoint{1.000216in}{0.922016in}}%
\pgfpathlineto{\pgfqpoint{1.015872in}{0.917361in}}%
\pgfpathlineto{\pgfqpoint{1.016642in}{0.916913in}}%
\pgfpathlineto{\pgfqpoint{1.031529in}{0.907087in}}%
\pgfpathlineto{\pgfqpoint{1.035883in}{0.903302in}}%
\pgfpathlineto{\pgfqpoint{1.047185in}{0.890360in}}%
\pgfpathlineto{\pgfqpoint{1.047701in}{0.889691in}}%
\pgfpathlineto{\pgfqpoint{1.053055in}{0.876079in}}%
\pgfpathlineto{\pgfqpoint{1.052439in}{0.862468in}}%
\pgfpathlineto{\pgfqpoint{1.047185in}{0.851409in}}%
\pgfpathlineto{\pgfqpoint{1.045786in}{0.848857in}}%
\pgfpathlineto{\pgfqpoint{1.033051in}{0.835246in}}%
\pgfpathlineto{\pgfqpoint{1.031529in}{0.833966in}}%
\pgfpathlineto{\pgfqpoint{1.015872in}{0.824068in}}%
\pgfpathlineto{\pgfqpoint{1.008573in}{0.821635in}}%
\pgfpathlineto{\pgfqpoint{1.000216in}{0.819171in}}%
\pgfpathlineto{\pgfqpoint{0.984559in}{0.819697in}}%
\pgfpathlineto{\pgfqpoint{0.979128in}{0.821635in}}%
\pgfpathclose%
\pgfpathmoveto{\pgfqpoint{1.282034in}{0.788107in}}%
\pgfpathlineto{\pgfqpoint{1.297690in}{0.782477in}}%
\pgfpathlineto{\pgfqpoint{1.313347in}{0.783125in}}%
\pgfpathlineto{\pgfqpoint{1.329003in}{0.789886in}}%
\pgfpathlineto{\pgfqpoint{1.335103in}{0.794413in}}%
\pgfpathlineto{\pgfqpoint{1.344660in}{0.800718in}}%
\pgfpathlineto{\pgfqpoint{1.353464in}{0.808024in}}%
\pgfpathlineto{\pgfqpoint{1.360317in}{0.813546in}}%
\pgfpathlineto{\pgfqpoint{1.369621in}{0.821635in}}%
\pgfpathlineto{\pgfqpoint{1.375973in}{0.827592in}}%
\pgfpathlineto{\pgfqpoint{1.384376in}{0.835246in}}%
\pgfpathlineto{\pgfqpoint{1.391630in}{0.843555in}}%
\pgfpathlineto{\pgfqpoint{1.396837in}{0.848857in}}%
\pgfpathlineto{\pgfqpoint{1.404613in}{0.862468in}}%
\pgfpathlineto{\pgfqpoint{1.405359in}{0.876079in}}%
\pgfpathlineto{\pgfqpoint{1.398884in}{0.889691in}}%
\pgfpathlineto{\pgfqpoint{1.391630in}{0.897680in}}%
\pgfpathlineto{\pgfqpoint{1.387074in}{0.903302in}}%
\pgfpathlineto{\pgfqpoint{1.375973in}{0.913758in}}%
\pgfpathlineto{\pgfqpoint{1.372714in}{0.916913in}}%
\pgfpathlineto{\pgfqpoint{1.360317in}{0.927756in}}%
\pgfpathlineto{\pgfqpoint{1.356900in}{0.930524in}}%
\pgfpathlineto{\pgfqpoint{1.344660in}{0.940528in}}%
\pgfpathlineto{\pgfqpoint{1.339116in}{0.944135in}}%
\pgfpathlineto{\pgfqpoint{1.329003in}{0.951382in}}%
\pgfpathlineto{\pgfqpoint{1.313925in}{0.957746in}}%
\pgfpathlineto{\pgfqpoint{1.313347in}{0.958041in}}%
\pgfpathlineto{\pgfqpoint{1.297690in}{0.958783in}}%
\pgfpathlineto{\pgfqpoint{1.295225in}{0.957746in}}%
\pgfpathlineto{\pgfqpoint{1.282034in}{0.953123in}}%
\pgfpathlineto{\pgfqpoint{1.268468in}{0.944135in}}%
\pgfpathlineto{\pgfqpoint{1.266377in}{0.942872in}}%
\pgfpathlineto{\pgfqpoint{1.250753in}{0.930524in}}%
\pgfpathlineto{\pgfqpoint{1.250721in}{0.930499in}}%
\pgfpathlineto{\pgfqpoint{1.235093in}{0.916913in}}%
\pgfpathlineto{\pgfqpoint{1.235064in}{0.916885in}}%
\pgfpathlineto{\pgfqpoint{1.220860in}{0.903302in}}%
\pgfpathlineto{\pgfqpoint{1.219407in}{0.901484in}}%
\pgfpathlineto{\pgfqpoint{1.209069in}{0.889691in}}%
\pgfpathlineto{\pgfqpoint{1.203751in}{0.878222in}}%
\pgfpathlineto{\pgfqpoint{1.202558in}{0.876079in}}%
\pgfpathlineto{\pgfqpoint{1.203412in}{0.862468in}}%
\pgfpathlineto{\pgfqpoint{1.203751in}{0.861966in}}%
\pgfpathlineto{\pgfqpoint{1.211071in}{0.848857in}}%
\pgfpathlineto{\pgfqpoint{1.219407in}{0.840066in}}%
\pgfpathlineto{\pgfqpoint{1.223556in}{0.835246in}}%
\pgfpathlineto{\pgfqpoint{1.235064in}{0.824605in}}%
\pgfpathlineto{\pgfqpoint{1.238248in}{0.821635in}}%
\pgfpathlineto{\pgfqpoint{1.250721in}{0.810857in}}%
\pgfpathlineto{\pgfqpoint{1.254349in}{0.808024in}}%
\pgfpathlineto{\pgfqpoint{1.266377in}{0.798373in}}%
\pgfpathlineto{\pgfqpoint{1.272844in}{0.794413in}}%
\pgfpathlineto{\pgfqpoint{1.282034in}{0.788107in}}%
\pgfpathclose%
\pgfpathmoveto{\pgfqpoint{1.289333in}{0.821635in}}%
\pgfpathlineto{\pgfqpoint{1.282034in}{0.824068in}}%
\pgfpathlineto{\pgfqpoint{1.266377in}{0.833966in}}%
\pgfpathlineto{\pgfqpoint{1.264855in}{0.835246in}}%
\pgfpathlineto{\pgfqpoint{1.252119in}{0.848857in}}%
\pgfpathlineto{\pgfqpoint{1.250721in}{0.851409in}}%
\pgfpathlineto{\pgfqpoint{1.245467in}{0.862468in}}%
\pgfpathlineto{\pgfqpoint{1.244850in}{0.876079in}}%
\pgfpathlineto{\pgfqpoint{1.250204in}{0.889691in}}%
\pgfpathlineto{\pgfqpoint{1.250721in}{0.890360in}}%
\pgfpathlineto{\pgfqpoint{1.262022in}{0.903302in}}%
\pgfpathlineto{\pgfqpoint{1.266377in}{0.907087in}}%
\pgfpathlineto{\pgfqpoint{1.281264in}{0.916913in}}%
\pgfpathlineto{\pgfqpoint{1.282034in}{0.917361in}}%
\pgfpathlineto{\pgfqpoint{1.297690in}{0.922016in}}%
\pgfpathlineto{\pgfqpoint{1.313347in}{0.921480in}}%
\pgfpathlineto{\pgfqpoint{1.326068in}{0.916913in}}%
\pgfpathlineto{\pgfqpoint{1.329003in}{0.915697in}}%
\pgfpathlineto{\pgfqpoint{1.344660in}{0.904625in}}%
\pgfpathlineto{\pgfqpoint{1.346132in}{0.903302in}}%
\pgfpathlineto{\pgfqpoint{1.357518in}{0.889691in}}%
\pgfpathlineto{\pgfqpoint{1.360317in}{0.883345in}}%
\pgfpathlineto{\pgfqpoint{1.363150in}{0.876079in}}%
\pgfpathlineto{\pgfqpoint{1.362546in}{0.862468in}}%
\pgfpathlineto{\pgfqpoint{1.360317in}{0.857746in}}%
\pgfpathlineto{\pgfqpoint{1.355594in}{0.848857in}}%
\pgfpathlineto{\pgfqpoint{1.344660in}{0.836693in}}%
\pgfpathlineto{\pgfqpoint{1.342995in}{0.835246in}}%
\pgfpathlineto{\pgfqpoint{1.329003in}{0.825740in}}%
\pgfpathlineto{\pgfqpoint{1.318778in}{0.821635in}}%
\pgfpathlineto{\pgfqpoint{1.313347in}{0.819697in}}%
\pgfpathlineto{\pgfqpoint{1.297690in}{0.819171in}}%
\pgfpathlineto{\pgfqpoint{1.289333in}{0.821635in}}%
\pgfpathclose%
\pgfpathmoveto{\pgfqpoint{1.595165in}{0.786526in}}%
\pgfpathlineto{\pgfqpoint{1.610822in}{0.782086in}}%
\pgfpathlineto{\pgfqpoint{1.626478in}{0.784023in}}%
\pgfpathlineto{\pgfqpoint{1.642135in}{0.791847in}}%
\pgfpathlineto{\pgfqpoint{1.645367in}{0.794413in}}%
\pgfpathlineto{\pgfqpoint{1.657791in}{0.803146in}}%
\pgfpathlineto{\pgfqpoint{1.663512in}{0.808024in}}%
\pgfpathlineto{\pgfqpoint{1.673448in}{0.816237in}}%
\pgfpathlineto{\pgfqpoint{1.679648in}{0.821635in}}%
\pgfpathlineto{\pgfqpoint{1.689104in}{0.830517in}}%
\pgfpathlineto{\pgfqpoint{1.694397in}{0.835246in}}%
\pgfpathlineto{\pgfqpoint{1.704761in}{0.846909in}}%
\pgfpathlineto{\pgfqpoint{1.706751in}{0.848857in}}%
\pgfpathlineto{\pgfqpoint{1.714735in}{0.862468in}}%
\pgfpathlineto{\pgfqpoint{1.715500in}{0.876079in}}%
\pgfpathlineto{\pgfqpoint{1.708852in}{0.889691in}}%
\pgfpathlineto{\pgfqpoint{1.704761in}{0.894023in}}%
\pgfpathlineto{\pgfqpoint{1.697108in}{0.903302in}}%
\pgfpathlineto{\pgfqpoint{1.689104in}{0.910698in}}%
\pgfpathlineto{\pgfqpoint{1.682692in}{0.916913in}}%
\pgfpathlineto{\pgfqpoint{1.673448in}{0.925010in}}%
\pgfpathlineto{\pgfqpoint{1.666814in}{0.930524in}}%
\pgfpathlineto{\pgfqpoint{1.657791in}{0.938102in}}%
\pgfpathlineto{\pgfqpoint{1.649087in}{0.944135in}}%
\pgfpathlineto{\pgfqpoint{1.642135in}{0.949463in}}%
\pgfpathlineto{\pgfqpoint{1.626478in}{0.957119in}}%
\pgfpathlineto{\pgfqpoint{1.621272in}{0.957746in}}%
\pgfpathlineto{\pgfqpoint{1.610822in}{0.959231in}}%
\pgfpathlineto{\pgfqpoint{1.606313in}{0.957746in}}%
\pgfpathlineto{\pgfqpoint{1.595165in}{0.954670in}}%
\pgfpathlineto{\pgfqpoint{1.579508in}{0.945161in}}%
\pgfpathlineto{\pgfqpoint{1.578304in}{0.944135in}}%
\pgfpathlineto{\pgfqpoint{1.563852in}{0.933071in}}%
\pgfpathlineto{\pgfqpoint{1.560930in}{0.930524in}}%
\pgfpathlineto{\pgfqpoint{1.548195in}{0.919564in}}%
\pgfpathlineto{\pgfqpoint{1.545138in}{0.916913in}}%
\pgfpathlineto{\pgfqpoint{1.532539in}{0.904814in}}%
\pgfpathlineto{\pgfqpoint{1.530818in}{0.903302in}}%
\pgfpathlineto{\pgfqpoint{1.519163in}{0.889691in}}%
\pgfpathlineto{\pgfqpoint{1.516882in}{0.884626in}}%
\pgfpathlineto{\pgfqpoint{1.512375in}{0.876079in}}%
\pgfpathlineto{\pgfqpoint{1.513194in}{0.862468in}}%
\pgfpathlineto{\pgfqpoint{1.516882in}{0.856691in}}%
\pgfpathlineto{\pgfqpoint{1.521132in}{0.848857in}}%
\pgfpathlineto{\pgfqpoint{1.532539in}{0.836449in}}%
\pgfpathlineto{\pgfqpoint{1.533564in}{0.835246in}}%
\pgfpathlineto{\pgfqpoint{1.548125in}{0.821635in}}%
\pgfpathlineto{\pgfqpoint{1.548195in}{0.821574in}}%
\pgfpathlineto{\pgfqpoint{1.563852in}{0.808186in}}%
\pgfpathlineto{\pgfqpoint{1.564068in}{0.808024in}}%
\pgfpathlineto{\pgfqpoint{1.579508in}{0.796127in}}%
\pgfpathlineto{\pgfqpoint{1.582566in}{0.794413in}}%
\pgfpathlineto{\pgfqpoint{1.595165in}{0.786526in}}%
\pgfpathclose%
\pgfpathmoveto{\pgfqpoint{1.598777in}{0.821635in}}%
\pgfpathlineto{\pgfqpoint{1.595165in}{0.822583in}}%
\pgfpathlineto{\pgfqpoint{1.579508in}{0.831714in}}%
\pgfpathlineto{\pgfqpoint{1.575133in}{0.835246in}}%
\pgfpathlineto{\pgfqpoint{1.563852in}{0.846787in}}%
\pgfpathlineto{\pgfqpoint{1.562107in}{0.848857in}}%
\pgfpathlineto{\pgfqpoint{1.555517in}{0.862468in}}%
\pgfpathlineto{\pgfqpoint{1.554886in}{0.876079in}}%
\pgfpathlineto{\pgfqpoint{1.560373in}{0.889691in}}%
\pgfpathlineto{\pgfqpoint{1.563852in}{0.894155in}}%
\pgfpathlineto{\pgfqpoint{1.572150in}{0.903302in}}%
\pgfpathlineto{\pgfqpoint{1.579508in}{0.909445in}}%
\pgfpathlineto{\pgfqpoint{1.591862in}{0.916913in}}%
\pgfpathlineto{\pgfqpoint{1.595165in}{0.918668in}}%
\pgfpathlineto{\pgfqpoint{1.610822in}{0.922340in}}%
\pgfpathlineto{\pgfqpoint{1.626478in}{0.920738in}}%
\pgfpathlineto{\pgfqpoint{1.635602in}{0.916913in}}%
\pgfpathlineto{\pgfqpoint{1.642135in}{0.913768in}}%
\pgfpathlineto{\pgfqpoint{1.656066in}{0.903302in}}%
\pgfpathlineto{\pgfqpoint{1.657791in}{0.901580in}}%
\pgfpathlineto{\pgfqpoint{1.667407in}{0.889691in}}%
\pgfpathlineto{\pgfqpoint{1.673255in}{0.876079in}}%
\pgfpathlineto{\pgfqpoint{1.672582in}{0.862468in}}%
\pgfpathlineto{\pgfqpoint{1.665559in}{0.848857in}}%
\pgfpathlineto{\pgfqpoint{1.657791in}{0.839978in}}%
\pgfpathlineto{\pgfqpoint{1.652683in}{0.835246in}}%
\pgfpathlineto{\pgfqpoint{1.642135in}{0.827583in}}%
\pgfpathlineto{\pgfqpoint{1.629384in}{0.821635in}}%
\pgfpathlineto{\pgfqpoint{1.626478in}{0.820424in}}%
\pgfpathlineto{\pgfqpoint{1.610822in}{0.818854in}}%
\pgfpathlineto{\pgfqpoint{1.598777in}{0.821635in}}%
\pgfpathclose%
\pgfpathmoveto{\pgfqpoint{1.892640in}{0.793973in}}%
\pgfpathlineto{\pgfqpoint{1.908296in}{0.785160in}}%
\pgfpathlineto{\pgfqpoint{1.923953in}{0.781955in}}%
\pgfpathlineto{\pgfqpoint{1.923953in}{0.794413in}}%
\pgfpathlineto{\pgfqpoint{1.923953in}{0.808024in}}%
\pgfpathlineto{\pgfqpoint{1.923953in}{0.818748in}}%
\pgfpathlineto{\pgfqpoint{1.908296in}{0.821346in}}%
\pgfpathlineto{\pgfqpoint{1.907686in}{0.821635in}}%
\pgfpathlineto{\pgfqpoint{1.892640in}{0.829580in}}%
\pgfpathlineto{\pgfqpoint{1.885273in}{0.835246in}}%
\pgfpathlineto{\pgfqpoint{1.876983in}{0.843350in}}%
\pgfpathlineto{\pgfqpoint{1.872267in}{0.848857in}}%
\pgfpathlineto{\pgfqpoint{1.865481in}{0.862468in}}%
\pgfpathlineto{\pgfqpoint{1.864831in}{0.876079in}}%
\pgfpathlineto{\pgfqpoint{1.870481in}{0.889691in}}%
\pgfpathlineto{\pgfqpoint{1.876983in}{0.897903in}}%
\pgfpathlineto{\pgfqpoint{1.882109in}{0.903302in}}%
\pgfpathlineto{\pgfqpoint{1.892640in}{0.911678in}}%
\pgfpathlineto{\pgfqpoint{1.902247in}{0.916913in}}%
\pgfpathlineto{\pgfqpoint{1.908296in}{0.919798in}}%
\pgfpathlineto{\pgfqpoint{1.923953in}{0.922448in}}%
\pgfpathlineto{\pgfqpoint{1.923953in}{0.930524in}}%
\pgfpathlineto{\pgfqpoint{1.923953in}{0.944135in}}%
\pgfpathlineto{\pgfqpoint{1.923953in}{0.957746in}}%
\pgfpathlineto{\pgfqpoint{1.923953in}{0.959381in}}%
\pgfpathlineto{\pgfqpoint{1.917035in}{0.957746in}}%
\pgfpathlineto{\pgfqpoint{1.908296in}{0.956006in}}%
\pgfpathlineto{\pgfqpoint{1.892640in}{0.947383in}}%
\pgfpathlineto{\pgfqpoint{1.888636in}{0.944135in}}%
\pgfpathlineto{\pgfqpoint{1.876983in}{0.935611in}}%
\pgfpathlineto{\pgfqpoint{1.871054in}{0.930524in}}%
\pgfpathlineto{\pgfqpoint{1.861327in}{0.922275in}}%
\pgfpathlineto{\pgfqpoint{1.855187in}{0.916913in}}%
\pgfpathlineto{\pgfqpoint{1.845670in}{0.907714in}}%
\pgfpathlineto{\pgfqpoint{1.840785in}{0.903302in}}%
\pgfpathlineto{\pgfqpoint{1.830014in}{0.890520in}}%
\pgfpathlineto{\pgfqpoint{1.829195in}{0.889691in}}%
\pgfpathlineto{\pgfqpoint{1.822337in}{0.876079in}}%
\pgfpathlineto{\pgfqpoint{1.823127in}{0.862468in}}%
\pgfpathlineto{\pgfqpoint{1.830014in}{0.851144in}}%
\pgfpathlineto{\pgfqpoint{1.831223in}{0.848857in}}%
\pgfpathlineto{\pgfqpoint{1.843519in}{0.835246in}}%
\pgfpathlineto{\pgfqpoint{1.845670in}{0.833368in}}%
\pgfpathlineto{\pgfqpoint{1.858197in}{0.821635in}}%
\pgfpathlineto{\pgfqpoint{1.861327in}{0.818918in}}%
\pgfpathlineto{\pgfqpoint{1.874246in}{0.808024in}}%
\pgfpathlineto{\pgfqpoint{1.876983in}{0.805639in}}%
\pgfpathlineto{\pgfqpoint{1.892116in}{0.794413in}}%
\pgfpathlineto{\pgfqpoint{1.892640in}{0.793973in}}%
\pgfpathclose%
\pgfpathmoveto{\pgfqpoint{0.380871in}{1.053024in}}%
\pgfpathlineto{\pgfqpoint{0.389609in}{1.054764in}}%
\pgfpathlineto{\pgfqpoint{0.405266in}{1.063387in}}%
\pgfpathlineto{\pgfqpoint{0.409269in}{1.066635in}}%
\pgfpathlineto{\pgfqpoint{0.420923in}{1.075159in}}%
\pgfpathlineto{\pgfqpoint{0.426851in}{1.080246in}}%
\pgfpathlineto{\pgfqpoint{0.436579in}{1.088495in}}%
\pgfpathlineto{\pgfqpoint{0.442719in}{1.093857in}}%
\pgfpathlineto{\pgfqpoint{0.452236in}{1.103056in}}%
\pgfpathlineto{\pgfqpoint{0.457121in}{1.107468in}}%
\pgfpathlineto{\pgfqpoint{0.467892in}{1.120250in}}%
\pgfpathlineto{\pgfqpoint{0.468710in}{1.121079in}}%
\pgfpathlineto{\pgfqpoint{0.475568in}{1.134691in}}%
\pgfpathlineto{\pgfqpoint{0.474779in}{1.148302in}}%
\pgfpathlineto{\pgfqpoint{0.467892in}{1.159626in}}%
\pgfpathlineto{\pgfqpoint{0.466682in}{1.161913in}}%
\pgfpathlineto{\pgfqpoint{0.454386in}{1.175524in}}%
\pgfpathlineto{\pgfqpoint{0.452236in}{1.177402in}}%
\pgfpathlineto{\pgfqpoint{0.439709in}{1.189135in}}%
\pgfpathlineto{\pgfqpoint{0.436579in}{1.191852in}}%
\pgfpathlineto{\pgfqpoint{0.423660in}{1.202746in}}%
\pgfpathlineto{\pgfqpoint{0.420923in}{1.205131in}}%
\pgfpathlineto{\pgfqpoint{0.405790in}{1.216357in}}%
\pgfpathlineto{\pgfqpoint{0.405266in}{1.216797in}}%
\pgfpathlineto{\pgfqpoint{0.389609in}{1.225610in}}%
\pgfpathlineto{\pgfqpoint{0.373953in}{1.228815in}}%
\pgfpathlineto{\pgfqpoint{0.373953in}{1.216357in}}%
\pgfpathlineto{\pgfqpoint{0.373953in}{1.202746in}}%
\pgfpathlineto{\pgfqpoint{0.373953in}{1.192022in}}%
\pgfpathlineto{\pgfqpoint{0.389609in}{1.189424in}}%
\pgfpathlineto{\pgfqpoint{0.390220in}{1.189135in}}%
\pgfpathlineto{\pgfqpoint{0.405266in}{1.181190in}}%
\pgfpathlineto{\pgfqpoint{0.412633in}{1.175524in}}%
\pgfpathlineto{\pgfqpoint{0.420923in}{1.167420in}}%
\pgfpathlineto{\pgfqpoint{0.425639in}{1.161913in}}%
\pgfpathlineto{\pgfqpoint{0.432425in}{1.148302in}}%
\pgfpathlineto{\pgfqpoint{0.433075in}{1.134691in}}%
\pgfpathlineto{\pgfqpoint{0.427425in}{1.121079in}}%
\pgfpathlineto{\pgfqpoint{0.420923in}{1.112867in}}%
\pgfpathlineto{\pgfqpoint{0.415797in}{1.107468in}}%
\pgfpathlineto{\pgfqpoint{0.405266in}{1.099092in}}%
\pgfpathlineto{\pgfqpoint{0.395659in}{1.093857in}}%
\pgfpathlineto{\pgfqpoint{0.389609in}{1.090972in}}%
\pgfpathlineto{\pgfqpoint{0.373953in}{1.088322in}}%
\pgfpathlineto{\pgfqpoint{0.373953in}{1.080246in}}%
\pgfpathlineto{\pgfqpoint{0.373953in}{1.066635in}}%
\pgfpathlineto{\pgfqpoint{0.373953in}{1.053024in}}%
\pgfpathlineto{\pgfqpoint{0.373953in}{1.051389in}}%
\pgfpathlineto{\pgfqpoint{0.380871in}{1.053024in}}%
\pgfpathclose%
\pgfpathmoveto{\pgfqpoint{0.687084in}{1.051539in}}%
\pgfpathlineto{\pgfqpoint{0.691593in}{1.053024in}}%
\pgfpathlineto{\pgfqpoint{0.702741in}{1.056100in}}%
\pgfpathlineto{\pgfqpoint{0.718397in}{1.065609in}}%
\pgfpathlineto{\pgfqpoint{0.719602in}{1.066635in}}%
\pgfpathlineto{\pgfqpoint{0.734054in}{1.077699in}}%
\pgfpathlineto{\pgfqpoint{0.736976in}{1.080246in}}%
\pgfpathlineto{\pgfqpoint{0.749710in}{1.091206in}}%
\pgfpathlineto{\pgfqpoint{0.752768in}{1.093857in}}%
\pgfpathlineto{\pgfqpoint{0.765367in}{1.105956in}}%
\pgfpathlineto{\pgfqpoint{0.767088in}{1.107468in}}%
\pgfpathlineto{\pgfqpoint{0.778742in}{1.121079in}}%
\pgfpathlineto{\pgfqpoint{0.781024in}{1.126144in}}%
\pgfpathlineto{\pgfqpoint{0.785531in}{1.134691in}}%
\pgfpathlineto{\pgfqpoint{0.784712in}{1.148302in}}%
\pgfpathlineto{\pgfqpoint{0.781024in}{1.154079in}}%
\pgfpathlineto{\pgfqpoint{0.776774in}{1.161913in}}%
\pgfpathlineto{\pgfqpoint{0.765367in}{1.174321in}}%
\pgfpathlineto{\pgfqpoint{0.764342in}{1.175524in}}%
\pgfpathlineto{\pgfqpoint{0.749781in}{1.189135in}}%
\pgfpathlineto{\pgfqpoint{0.749710in}{1.189196in}}%
\pgfpathlineto{\pgfqpoint{0.734054in}{1.202584in}}%
\pgfpathlineto{\pgfqpoint{0.733838in}{1.202746in}}%
\pgfpathlineto{\pgfqpoint{0.718397in}{1.214643in}}%
\pgfpathlineto{\pgfqpoint{0.715340in}{1.216357in}}%
\pgfpathlineto{\pgfqpoint{0.702741in}{1.224244in}}%
\pgfpathlineto{\pgfqpoint{0.687084in}{1.228684in}}%
\pgfpathlineto{\pgfqpoint{0.671428in}{1.226747in}}%
\pgfpathlineto{\pgfqpoint{0.655771in}{1.218923in}}%
\pgfpathlineto{\pgfqpoint{0.652538in}{1.216357in}}%
\pgfpathlineto{\pgfqpoint{0.640115in}{1.207624in}}%
\pgfpathlineto{\pgfqpoint{0.634394in}{1.202746in}}%
\pgfpathlineto{\pgfqpoint{0.624458in}{1.194533in}}%
\pgfpathlineto{\pgfqpoint{0.618258in}{1.189135in}}%
\pgfpathlineto{\pgfqpoint{0.608801in}{1.180253in}}%
\pgfpathlineto{\pgfqpoint{0.603509in}{1.175524in}}%
\pgfpathlineto{\pgfqpoint{0.593145in}{1.163861in}}%
\pgfpathlineto{\pgfqpoint{0.591155in}{1.161913in}}%
\pgfpathlineto{\pgfqpoint{0.583171in}{1.148302in}}%
\pgfpathlineto{\pgfqpoint{0.582406in}{1.134691in}}%
\pgfpathlineto{\pgfqpoint{0.589053in}{1.121079in}}%
\pgfpathlineto{\pgfqpoint{0.593145in}{1.116747in}}%
\pgfpathlineto{\pgfqpoint{0.600798in}{1.107468in}}%
\pgfpathlineto{\pgfqpoint{0.608801in}{1.100072in}}%
\pgfpathlineto{\pgfqpoint{0.615214in}{1.093857in}}%
\pgfpathlineto{\pgfqpoint{0.624458in}{1.085760in}}%
\pgfpathlineto{\pgfqpoint{0.631092in}{1.080246in}}%
\pgfpathlineto{\pgfqpoint{0.640115in}{1.072668in}}%
\pgfpathlineto{\pgfqpoint{0.648819in}{1.066635in}}%
\pgfpathlineto{\pgfqpoint{0.655771in}{1.061307in}}%
\pgfpathlineto{\pgfqpoint{0.671428in}{1.053651in}}%
\pgfpathlineto{\pgfqpoint{0.676633in}{1.053024in}}%
\pgfpathlineto{\pgfqpoint{0.687084in}{1.051539in}}%
\pgfpathclose%
\pgfpathmoveto{\pgfqpoint{0.662304in}{1.093857in}}%
\pgfpathlineto{\pgfqpoint{0.655771in}{1.097002in}}%
\pgfpathlineto{\pgfqpoint{0.641840in}{1.107468in}}%
\pgfpathlineto{\pgfqpoint{0.640115in}{1.109190in}}%
\pgfpathlineto{\pgfqpoint{0.630498in}{1.121079in}}%
\pgfpathlineto{\pgfqpoint{0.624650in}{1.134691in}}%
\pgfpathlineto{\pgfqpoint{0.625324in}{1.148302in}}%
\pgfpathlineto{\pgfqpoint{0.632347in}{1.161913in}}%
\pgfpathlineto{\pgfqpoint{0.640115in}{1.170792in}}%
\pgfpathlineto{\pgfqpoint{0.645223in}{1.175524in}}%
\pgfpathlineto{\pgfqpoint{0.655771in}{1.183187in}}%
\pgfpathlineto{\pgfqpoint{0.668522in}{1.189135in}}%
\pgfpathlineto{\pgfqpoint{0.671428in}{1.190346in}}%
\pgfpathlineto{\pgfqpoint{0.687084in}{1.191916in}}%
\pgfpathlineto{\pgfqpoint{0.699129in}{1.189135in}}%
\pgfpathlineto{\pgfqpoint{0.702741in}{1.188187in}}%
\pgfpathlineto{\pgfqpoint{0.718397in}{1.179056in}}%
\pgfpathlineto{\pgfqpoint{0.722773in}{1.175524in}}%
\pgfpathlineto{\pgfqpoint{0.734054in}{1.163983in}}%
\pgfpathlineto{\pgfqpoint{0.735798in}{1.161913in}}%
\pgfpathlineto{\pgfqpoint{0.742389in}{1.148302in}}%
\pgfpathlineto{\pgfqpoint{0.743020in}{1.134691in}}%
\pgfpathlineto{\pgfqpoint{0.737533in}{1.121079in}}%
\pgfpathlineto{\pgfqpoint{0.734054in}{1.116615in}}%
\pgfpathlineto{\pgfqpoint{0.725756in}{1.107468in}}%
\pgfpathlineto{\pgfqpoint{0.718397in}{1.101325in}}%
\pgfpathlineto{\pgfqpoint{0.706044in}{1.093857in}}%
\pgfpathlineto{\pgfqpoint{0.702741in}{1.092102in}}%
\pgfpathlineto{\pgfqpoint{0.687084in}{1.088430in}}%
\pgfpathlineto{\pgfqpoint{0.671428in}{1.090032in}}%
\pgfpathlineto{\pgfqpoint{0.662304in}{1.093857in}}%
\pgfpathclose%
\pgfpathmoveto{\pgfqpoint{0.984559in}{1.052729in}}%
\pgfpathlineto{\pgfqpoint{1.000216in}{1.051987in}}%
\pgfpathlineto{\pgfqpoint{1.002680in}{1.053024in}}%
\pgfpathlineto{\pgfqpoint{1.015872in}{1.057647in}}%
\pgfpathlineto{\pgfqpoint{1.029438in}{1.066635in}}%
\pgfpathlineto{\pgfqpoint{1.031529in}{1.067898in}}%
\pgfpathlineto{\pgfqpoint{1.047153in}{1.080246in}}%
\pgfpathlineto{\pgfqpoint{1.047185in}{1.080271in}}%
\pgfpathlineto{\pgfqpoint{1.062813in}{1.093857in}}%
\pgfpathlineto{\pgfqpoint{1.062842in}{1.093885in}}%
\pgfpathlineto{\pgfqpoint{1.077046in}{1.107468in}}%
\pgfpathlineto{\pgfqpoint{1.078498in}{1.109286in}}%
\pgfpathlineto{\pgfqpoint{1.088837in}{1.121079in}}%
\pgfpathlineto{\pgfqpoint{1.094155in}{1.132548in}}%
\pgfpathlineto{\pgfqpoint{1.095347in}{1.134691in}}%
\pgfpathlineto{\pgfqpoint{1.094494in}{1.148302in}}%
\pgfpathlineto{\pgfqpoint{1.094155in}{1.148804in}}%
\pgfpathlineto{\pgfqpoint{1.086834in}{1.161913in}}%
\pgfpathlineto{\pgfqpoint{1.078498in}{1.170704in}}%
\pgfpathlineto{\pgfqpoint{1.074350in}{1.175524in}}%
\pgfpathlineto{\pgfqpoint{1.062842in}{1.186165in}}%
\pgfpathlineto{\pgfqpoint{1.059657in}{1.189135in}}%
\pgfpathlineto{\pgfqpoint{1.047185in}{1.199913in}}%
\pgfpathlineto{\pgfqpoint{1.043557in}{1.202746in}}%
\pgfpathlineto{\pgfqpoint{1.031529in}{1.212397in}}%
\pgfpathlineto{\pgfqpoint{1.025062in}{1.216357in}}%
\pgfpathlineto{\pgfqpoint{1.015872in}{1.222663in}}%
\pgfpathlineto{\pgfqpoint{1.000216in}{1.228293in}}%
\pgfpathlineto{\pgfqpoint{0.984559in}{1.227645in}}%
\pgfpathlineto{\pgfqpoint{0.968902in}{1.220884in}}%
\pgfpathlineto{\pgfqpoint{0.962803in}{1.216357in}}%
\pgfpathlineto{\pgfqpoint{0.953246in}{1.210052in}}%
\pgfpathlineto{\pgfqpoint{0.944442in}{1.202746in}}%
\pgfpathlineto{\pgfqpoint{0.937589in}{1.197224in}}%
\pgfpathlineto{\pgfqpoint{0.928285in}{1.189135in}}%
\pgfpathlineto{\pgfqpoint{0.921933in}{1.183178in}}%
\pgfpathlineto{\pgfqpoint{0.913529in}{1.175524in}}%
\pgfpathlineto{\pgfqpoint{0.906276in}{1.167215in}}%
\pgfpathlineto{\pgfqpoint{0.901069in}{1.161913in}}%
\pgfpathlineto{\pgfqpoint{0.893292in}{1.148302in}}%
\pgfpathlineto{\pgfqpoint{0.892547in}{1.134691in}}%
\pgfpathlineto{\pgfqpoint{0.899022in}{1.121079in}}%
\pgfpathlineto{\pgfqpoint{0.906276in}{1.113090in}}%
\pgfpathlineto{\pgfqpoint{0.910831in}{1.107468in}}%
\pgfpathlineto{\pgfqpoint{0.921933in}{1.097012in}}%
\pgfpathlineto{\pgfqpoint{0.925192in}{1.093857in}}%
\pgfpathlineto{\pgfqpoint{0.937589in}{1.083014in}}%
\pgfpathlineto{\pgfqpoint{0.941005in}{1.080246in}}%
\pgfpathlineto{\pgfqpoint{0.953246in}{1.070242in}}%
\pgfpathlineto{\pgfqpoint{0.958790in}{1.066635in}}%
\pgfpathlineto{\pgfqpoint{0.968902in}{1.059388in}}%
\pgfpathlineto{\pgfqpoint{0.983981in}{1.053024in}}%
\pgfpathlineto{\pgfqpoint{0.984559in}{1.052729in}}%
\pgfpathclose%
\pgfpathmoveto{\pgfqpoint{0.971838in}{1.093857in}}%
\pgfpathlineto{\pgfqpoint{0.968902in}{1.095073in}}%
\pgfpathlineto{\pgfqpoint{0.953246in}{1.106145in}}%
\pgfpathlineto{\pgfqpoint{0.951774in}{1.107468in}}%
\pgfpathlineto{\pgfqpoint{0.940388in}{1.121079in}}%
\pgfpathlineto{\pgfqpoint{0.937589in}{1.127425in}}%
\pgfpathlineto{\pgfqpoint{0.934755in}{1.134691in}}%
\pgfpathlineto{\pgfqpoint{0.935360in}{1.148302in}}%
\pgfpathlineto{\pgfqpoint{0.937589in}{1.153024in}}%
\pgfpathlineto{\pgfqpoint{0.942311in}{1.161913in}}%
\pgfpathlineto{\pgfqpoint{0.953246in}{1.174077in}}%
\pgfpathlineto{\pgfqpoint{0.954911in}{1.175524in}}%
\pgfpathlineto{\pgfqpoint{0.968902in}{1.185030in}}%
\pgfpathlineto{\pgfqpoint{0.979128in}{1.189135in}}%
\pgfpathlineto{\pgfqpoint{0.984559in}{1.191073in}}%
\pgfpathlineto{\pgfqpoint{1.000216in}{1.191599in}}%
\pgfpathlineto{\pgfqpoint{1.008573in}{1.189135in}}%
\pgfpathlineto{\pgfqpoint{1.015872in}{1.186702in}}%
\pgfpathlineto{\pgfqpoint{1.031529in}{1.176804in}}%
\pgfpathlineto{\pgfqpoint{1.033051in}{1.175524in}}%
\pgfpathlineto{\pgfqpoint{1.045786in}{1.161913in}}%
\pgfpathlineto{\pgfqpoint{1.047185in}{1.159361in}}%
\pgfpathlineto{\pgfqpoint{1.052439in}{1.148302in}}%
\pgfpathlineto{\pgfqpoint{1.053055in}{1.134691in}}%
\pgfpathlineto{\pgfqpoint{1.047701in}{1.121079in}}%
\pgfpathlineto{\pgfqpoint{1.047185in}{1.120410in}}%
\pgfpathlineto{\pgfqpoint{1.035883in}{1.107468in}}%
\pgfpathlineto{\pgfqpoint{1.031529in}{1.103683in}}%
\pgfpathlineto{\pgfqpoint{1.016642in}{1.093857in}}%
\pgfpathlineto{\pgfqpoint{1.015872in}{1.093409in}}%
\pgfpathlineto{\pgfqpoint{1.000216in}{1.088754in}}%
\pgfpathlineto{\pgfqpoint{0.984559in}{1.089290in}}%
\pgfpathlineto{\pgfqpoint{0.971838in}{1.093857in}}%
\pgfpathclose%
\pgfpathmoveto{\pgfqpoint{1.297690in}{1.051987in}}%
\pgfpathlineto{\pgfqpoint{1.313347in}{1.052729in}}%
\pgfpathlineto{\pgfqpoint{1.313925in}{1.053024in}}%
\pgfpathlineto{\pgfqpoint{1.329003in}{1.059388in}}%
\pgfpathlineto{\pgfqpoint{1.339116in}{1.066635in}}%
\pgfpathlineto{\pgfqpoint{1.344660in}{1.070242in}}%
\pgfpathlineto{\pgfqpoint{1.356900in}{1.080246in}}%
\pgfpathlineto{\pgfqpoint{1.360317in}{1.083014in}}%
\pgfpathlineto{\pgfqpoint{1.372714in}{1.093857in}}%
\pgfpathlineto{\pgfqpoint{1.375973in}{1.097012in}}%
\pgfpathlineto{\pgfqpoint{1.387074in}{1.107468in}}%
\pgfpathlineto{\pgfqpoint{1.391630in}{1.113090in}}%
\pgfpathlineto{\pgfqpoint{1.398884in}{1.121079in}}%
\pgfpathlineto{\pgfqpoint{1.405359in}{1.134691in}}%
\pgfpathlineto{\pgfqpoint{1.404613in}{1.148302in}}%
\pgfpathlineto{\pgfqpoint{1.396837in}{1.161913in}}%
\pgfpathlineto{\pgfqpoint{1.391630in}{1.167215in}}%
\pgfpathlineto{\pgfqpoint{1.384376in}{1.175524in}}%
\pgfpathlineto{\pgfqpoint{1.375973in}{1.183178in}}%
\pgfpathlineto{\pgfqpoint{1.369621in}{1.189135in}}%
\pgfpathlineto{\pgfqpoint{1.360317in}{1.197224in}}%
\pgfpathlineto{\pgfqpoint{1.353464in}{1.202746in}}%
\pgfpathlineto{\pgfqpoint{1.344660in}{1.210052in}}%
\pgfpathlineto{\pgfqpoint{1.335103in}{1.216357in}}%
\pgfpathlineto{\pgfqpoint{1.329003in}{1.220884in}}%
\pgfpathlineto{\pgfqpoint{1.313347in}{1.227645in}}%
\pgfpathlineto{\pgfqpoint{1.297690in}{1.228293in}}%
\pgfpathlineto{\pgfqpoint{1.282034in}{1.222663in}}%
\pgfpathlineto{\pgfqpoint{1.272844in}{1.216357in}}%
\pgfpathlineto{\pgfqpoint{1.266377in}{1.212397in}}%
\pgfpathlineto{\pgfqpoint{1.254349in}{1.202746in}}%
\pgfpathlineto{\pgfqpoint{1.250721in}{1.199913in}}%
\pgfpathlineto{\pgfqpoint{1.238248in}{1.189135in}}%
\pgfpathlineto{\pgfqpoint{1.235064in}{1.186165in}}%
\pgfpathlineto{\pgfqpoint{1.223556in}{1.175524in}}%
\pgfpathlineto{\pgfqpoint{1.219407in}{1.170704in}}%
\pgfpathlineto{\pgfqpoint{1.211071in}{1.161913in}}%
\pgfpathlineto{\pgfqpoint{1.203751in}{1.148804in}}%
\pgfpathlineto{\pgfqpoint{1.203412in}{1.148302in}}%
\pgfpathlineto{\pgfqpoint{1.202558in}{1.134691in}}%
\pgfpathlineto{\pgfqpoint{1.203751in}{1.132548in}}%
\pgfpathlineto{\pgfqpoint{1.209069in}{1.121079in}}%
\pgfpathlineto{\pgfqpoint{1.219407in}{1.109286in}}%
\pgfpathlineto{\pgfqpoint{1.220860in}{1.107468in}}%
\pgfpathlineto{\pgfqpoint{1.235064in}{1.093885in}}%
\pgfpathlineto{\pgfqpoint{1.235093in}{1.093857in}}%
\pgfpathlineto{\pgfqpoint{1.250721in}{1.080271in}}%
\pgfpathlineto{\pgfqpoint{1.250753in}{1.080246in}}%
\pgfpathlineto{\pgfqpoint{1.266377in}{1.067898in}}%
\pgfpathlineto{\pgfqpoint{1.268468in}{1.066635in}}%
\pgfpathlineto{\pgfqpoint{1.282034in}{1.057647in}}%
\pgfpathlineto{\pgfqpoint{1.295225in}{1.053024in}}%
\pgfpathlineto{\pgfqpoint{1.297690in}{1.051987in}}%
\pgfpathclose%
\pgfpathmoveto{\pgfqpoint{1.281264in}{1.093857in}}%
\pgfpathlineto{\pgfqpoint{1.266377in}{1.103683in}}%
\pgfpathlineto{\pgfqpoint{1.262022in}{1.107468in}}%
\pgfpathlineto{\pgfqpoint{1.250721in}{1.120410in}}%
\pgfpathlineto{\pgfqpoint{1.250204in}{1.121079in}}%
\pgfpathlineto{\pgfqpoint{1.244850in}{1.134691in}}%
\pgfpathlineto{\pgfqpoint{1.245467in}{1.148302in}}%
\pgfpathlineto{\pgfqpoint{1.250721in}{1.159361in}}%
\pgfpathlineto{\pgfqpoint{1.252119in}{1.161913in}}%
\pgfpathlineto{\pgfqpoint{1.264855in}{1.175524in}}%
\pgfpathlineto{\pgfqpoint{1.266377in}{1.176804in}}%
\pgfpathlineto{\pgfqpoint{1.282034in}{1.186702in}}%
\pgfpathlineto{\pgfqpoint{1.289333in}{1.189135in}}%
\pgfpathlineto{\pgfqpoint{1.297690in}{1.191599in}}%
\pgfpathlineto{\pgfqpoint{1.313347in}{1.191073in}}%
\pgfpathlineto{\pgfqpoint{1.318778in}{1.189135in}}%
\pgfpathlineto{\pgfqpoint{1.329003in}{1.185030in}}%
\pgfpathlineto{\pgfqpoint{1.342995in}{1.175524in}}%
\pgfpathlineto{\pgfqpoint{1.344660in}{1.174077in}}%
\pgfpathlineto{\pgfqpoint{1.355594in}{1.161913in}}%
\pgfpathlineto{\pgfqpoint{1.360317in}{1.153024in}}%
\pgfpathlineto{\pgfqpoint{1.362546in}{1.148302in}}%
\pgfpathlineto{\pgfqpoint{1.363150in}{1.134691in}}%
\pgfpathlineto{\pgfqpoint{1.360317in}{1.127425in}}%
\pgfpathlineto{\pgfqpoint{1.357518in}{1.121079in}}%
\pgfpathlineto{\pgfqpoint{1.346132in}{1.107468in}}%
\pgfpathlineto{\pgfqpoint{1.344660in}{1.106145in}}%
\pgfpathlineto{\pgfqpoint{1.329003in}{1.095073in}}%
\pgfpathlineto{\pgfqpoint{1.326068in}{1.093857in}}%
\pgfpathlineto{\pgfqpoint{1.313347in}{1.089290in}}%
\pgfpathlineto{\pgfqpoint{1.297690in}{1.088754in}}%
\pgfpathlineto{\pgfqpoint{1.282034in}{1.093409in}}%
\pgfpathlineto{\pgfqpoint{1.281264in}{1.093857in}}%
\pgfpathclose%
\pgfpathmoveto{\pgfqpoint{1.610822in}{1.051539in}}%
\pgfpathlineto{\pgfqpoint{1.621272in}{1.053024in}}%
\pgfpathlineto{\pgfqpoint{1.626478in}{1.053651in}}%
\pgfpathlineto{\pgfqpoint{1.642135in}{1.061307in}}%
\pgfpathlineto{\pgfqpoint{1.649087in}{1.066635in}}%
\pgfpathlineto{\pgfqpoint{1.657791in}{1.072668in}}%
\pgfpathlineto{\pgfqpoint{1.666814in}{1.080246in}}%
\pgfpathlineto{\pgfqpoint{1.673448in}{1.085760in}}%
\pgfpathlineto{\pgfqpoint{1.682692in}{1.093857in}}%
\pgfpathlineto{\pgfqpoint{1.689104in}{1.100072in}}%
\pgfpathlineto{\pgfqpoint{1.697108in}{1.107468in}}%
\pgfpathlineto{\pgfqpoint{1.704761in}{1.116747in}}%
\pgfpathlineto{\pgfqpoint{1.708852in}{1.121079in}}%
\pgfpathlineto{\pgfqpoint{1.715500in}{1.134691in}}%
\pgfpathlineto{\pgfqpoint{1.714735in}{1.148302in}}%
\pgfpathlineto{\pgfqpoint{1.706751in}{1.161913in}}%
\pgfpathlineto{\pgfqpoint{1.704761in}{1.163861in}}%
\pgfpathlineto{\pgfqpoint{1.694397in}{1.175524in}}%
\pgfpathlineto{\pgfqpoint{1.689104in}{1.180253in}}%
\pgfpathlineto{\pgfqpoint{1.679648in}{1.189135in}}%
\pgfpathlineto{\pgfqpoint{1.673448in}{1.194533in}}%
\pgfpathlineto{\pgfqpoint{1.663512in}{1.202746in}}%
\pgfpathlineto{\pgfqpoint{1.657791in}{1.207624in}}%
\pgfpathlineto{\pgfqpoint{1.645367in}{1.216357in}}%
\pgfpathlineto{\pgfqpoint{1.642135in}{1.218923in}}%
\pgfpathlineto{\pgfqpoint{1.626478in}{1.226747in}}%
\pgfpathlineto{\pgfqpoint{1.610822in}{1.228684in}}%
\pgfpathlineto{\pgfqpoint{1.595165in}{1.224244in}}%
\pgfpathlineto{\pgfqpoint{1.582566in}{1.216357in}}%
\pgfpathlineto{\pgfqpoint{1.579508in}{1.214643in}}%
\pgfpathlineto{\pgfqpoint{1.564068in}{1.202746in}}%
\pgfpathlineto{\pgfqpoint{1.563852in}{1.202584in}}%
\pgfpathlineto{\pgfqpoint{1.548195in}{1.189196in}}%
\pgfpathlineto{\pgfqpoint{1.548125in}{1.189135in}}%
\pgfpathlineto{\pgfqpoint{1.533564in}{1.175524in}}%
\pgfpathlineto{\pgfqpoint{1.532539in}{1.174321in}}%
\pgfpathlineto{\pgfqpoint{1.521132in}{1.161913in}}%
\pgfpathlineto{\pgfqpoint{1.516882in}{1.154079in}}%
\pgfpathlineto{\pgfqpoint{1.513194in}{1.148302in}}%
\pgfpathlineto{\pgfqpoint{1.512375in}{1.134691in}}%
\pgfpathlineto{\pgfqpoint{1.516882in}{1.126144in}}%
\pgfpathlineto{\pgfqpoint{1.519163in}{1.121079in}}%
\pgfpathlineto{\pgfqpoint{1.530818in}{1.107468in}}%
\pgfpathlineto{\pgfqpoint{1.532539in}{1.105956in}}%
\pgfpathlineto{\pgfqpoint{1.545138in}{1.093857in}}%
\pgfpathlineto{\pgfqpoint{1.548195in}{1.091206in}}%
\pgfpathlineto{\pgfqpoint{1.560930in}{1.080246in}}%
\pgfpathlineto{\pgfqpoint{1.563852in}{1.077699in}}%
\pgfpathlineto{\pgfqpoint{1.578304in}{1.066635in}}%
\pgfpathlineto{\pgfqpoint{1.579508in}{1.065609in}}%
\pgfpathlineto{\pgfqpoint{1.595165in}{1.056100in}}%
\pgfpathlineto{\pgfqpoint{1.606313in}{1.053024in}}%
\pgfpathlineto{\pgfqpoint{1.610822in}{1.051539in}}%
\pgfpathclose%
\pgfpathmoveto{\pgfqpoint{1.591862in}{1.093857in}}%
\pgfpathlineto{\pgfqpoint{1.579508in}{1.101325in}}%
\pgfpathlineto{\pgfqpoint{1.572150in}{1.107468in}}%
\pgfpathlineto{\pgfqpoint{1.563852in}{1.116615in}}%
\pgfpathlineto{\pgfqpoint{1.560373in}{1.121079in}}%
\pgfpathlineto{\pgfqpoint{1.554886in}{1.134691in}}%
\pgfpathlineto{\pgfqpoint{1.555517in}{1.148302in}}%
\pgfpathlineto{\pgfqpoint{1.562107in}{1.161913in}}%
\pgfpathlineto{\pgfqpoint{1.563852in}{1.163983in}}%
\pgfpathlineto{\pgfqpoint{1.575133in}{1.175524in}}%
\pgfpathlineto{\pgfqpoint{1.579508in}{1.179056in}}%
\pgfpathlineto{\pgfqpoint{1.595165in}{1.188187in}}%
\pgfpathlineto{\pgfqpoint{1.598777in}{1.189135in}}%
\pgfpathlineto{\pgfqpoint{1.610822in}{1.191916in}}%
\pgfpathlineto{\pgfqpoint{1.626478in}{1.190346in}}%
\pgfpathlineto{\pgfqpoint{1.629384in}{1.189135in}}%
\pgfpathlineto{\pgfqpoint{1.642135in}{1.183187in}}%
\pgfpathlineto{\pgfqpoint{1.652683in}{1.175524in}}%
\pgfpathlineto{\pgfqpoint{1.657791in}{1.170792in}}%
\pgfpathlineto{\pgfqpoint{1.665559in}{1.161913in}}%
\pgfpathlineto{\pgfqpoint{1.672582in}{1.148302in}}%
\pgfpathlineto{\pgfqpoint{1.673255in}{1.134691in}}%
\pgfpathlineto{\pgfqpoint{1.667407in}{1.121079in}}%
\pgfpathlineto{\pgfqpoint{1.657791in}{1.109190in}}%
\pgfpathlineto{\pgfqpoint{1.656066in}{1.107468in}}%
\pgfpathlineto{\pgfqpoint{1.642135in}{1.097002in}}%
\pgfpathlineto{\pgfqpoint{1.635602in}{1.093857in}}%
\pgfpathlineto{\pgfqpoint{1.626478in}{1.090032in}}%
\pgfpathlineto{\pgfqpoint{1.610822in}{1.088430in}}%
\pgfpathlineto{\pgfqpoint{1.595165in}{1.092102in}}%
\pgfpathlineto{\pgfqpoint{1.591862in}{1.093857in}}%
\pgfpathclose%
\pgfpathmoveto{\pgfqpoint{1.923953in}{1.051389in}}%
\pgfpathlineto{\pgfqpoint{1.923953in}{1.053024in}}%
\pgfpathlineto{\pgfqpoint{1.923953in}{1.066635in}}%
\pgfpathlineto{\pgfqpoint{1.923953in}{1.080246in}}%
\pgfpathlineto{\pgfqpoint{1.923953in}{1.088322in}}%
\pgfpathlineto{\pgfqpoint{1.908296in}{1.090972in}}%
\pgfpathlineto{\pgfqpoint{1.902247in}{1.093857in}}%
\pgfpathlineto{\pgfqpoint{1.892640in}{1.099092in}}%
\pgfpathlineto{\pgfqpoint{1.882109in}{1.107468in}}%
\pgfpathlineto{\pgfqpoint{1.876983in}{1.112867in}}%
\pgfpathlineto{\pgfqpoint{1.870481in}{1.121079in}}%
\pgfpathlineto{\pgfqpoint{1.864831in}{1.134691in}}%
\pgfpathlineto{\pgfqpoint{1.865481in}{1.148302in}}%
\pgfpathlineto{\pgfqpoint{1.872267in}{1.161913in}}%
\pgfpathlineto{\pgfqpoint{1.876983in}{1.167420in}}%
\pgfpathlineto{\pgfqpoint{1.885273in}{1.175524in}}%
\pgfpathlineto{\pgfqpoint{1.892640in}{1.181190in}}%
\pgfpathlineto{\pgfqpoint{1.907686in}{1.189135in}}%
\pgfpathlineto{\pgfqpoint{1.908296in}{1.189424in}}%
\pgfpathlineto{\pgfqpoint{1.923953in}{1.192022in}}%
\pgfpathlineto{\pgfqpoint{1.923953in}{1.202746in}}%
\pgfpathlineto{\pgfqpoint{1.923953in}{1.216357in}}%
\pgfpathlineto{\pgfqpoint{1.923953in}{1.228815in}}%
\pgfpathlineto{\pgfqpoint{1.908296in}{1.225610in}}%
\pgfpathlineto{\pgfqpoint{1.892640in}{1.216797in}}%
\pgfpathlineto{\pgfqpoint{1.892116in}{1.216357in}}%
\pgfpathlineto{\pgfqpoint{1.876983in}{1.205131in}}%
\pgfpathlineto{\pgfqpoint{1.874246in}{1.202746in}}%
\pgfpathlineto{\pgfqpoint{1.861327in}{1.191852in}}%
\pgfpathlineto{\pgfqpoint{1.858197in}{1.189135in}}%
\pgfpathlineto{\pgfqpoint{1.845670in}{1.177402in}}%
\pgfpathlineto{\pgfqpoint{1.843519in}{1.175524in}}%
\pgfpathlineto{\pgfqpoint{1.831223in}{1.161913in}}%
\pgfpathlineto{\pgfqpoint{1.830014in}{1.159626in}}%
\pgfpathlineto{\pgfqpoint{1.823127in}{1.148302in}}%
\pgfpathlineto{\pgfqpoint{1.822337in}{1.134691in}}%
\pgfpathlineto{\pgfqpoint{1.829195in}{1.121079in}}%
\pgfpathlineto{\pgfqpoint{1.830014in}{1.120250in}}%
\pgfpathlineto{\pgfqpoint{1.840785in}{1.107468in}}%
\pgfpathlineto{\pgfqpoint{1.845670in}{1.103056in}}%
\pgfpathlineto{\pgfqpoint{1.855187in}{1.093857in}}%
\pgfpathlineto{\pgfqpoint{1.861327in}{1.088495in}}%
\pgfpathlineto{\pgfqpoint{1.871054in}{1.080246in}}%
\pgfpathlineto{\pgfqpoint{1.876983in}{1.075159in}}%
\pgfpathlineto{\pgfqpoint{1.888636in}{1.066635in}}%
\pgfpathlineto{\pgfqpoint{1.892640in}{1.063387in}}%
\pgfpathlineto{\pgfqpoint{1.908296in}{1.054764in}}%
\pgfpathlineto{\pgfqpoint{1.917035in}{1.053024in}}%
\pgfpathlineto{\pgfqpoint{1.923953in}{1.051389in}}%
\pgfpathclose%
\pgfpathmoveto{\pgfqpoint{0.389609in}{1.324275in}}%
\pgfpathlineto{\pgfqpoint{0.391125in}{1.325246in}}%
\pgfpathlineto{\pgfqpoint{0.405266in}{1.332872in}}%
\pgfpathlineto{\pgfqpoint{0.412876in}{1.338857in}}%
\pgfpathlineto{\pgfqpoint{0.420923in}{1.344682in}}%
\pgfpathlineto{\pgfqpoint{0.430099in}{1.352468in}}%
\pgfpathlineto{\pgfqpoint{0.436579in}{1.358019in}}%
\pgfpathlineto{\pgfqpoint{0.445709in}{1.366079in}}%
\pgfpathlineto{\pgfqpoint{0.452236in}{1.372634in}}%
\pgfpathlineto{\pgfqpoint{0.459739in}{1.379691in}}%
\pgfpathlineto{\pgfqpoint{0.467892in}{1.390254in}}%
\pgfpathlineto{\pgfqpoint{0.470636in}{1.393302in}}%
\pgfpathlineto{\pgfqpoint{0.476046in}{1.406913in}}%
\pgfpathlineto{\pgfqpoint{0.473686in}{1.420524in}}%
\pgfpathlineto{\pgfqpoint{0.467892in}{1.428681in}}%
\pgfpathlineto{\pgfqpoint{0.464540in}{1.434135in}}%
\pgfpathlineto{\pgfqpoint{0.452236in}{1.446994in}}%
\pgfpathlineto{\pgfqpoint{0.451558in}{1.447746in}}%
\pgfpathlineto{\pgfqpoint{0.436697in}{1.461357in}}%
\pgfpathlineto{\pgfqpoint{0.436579in}{1.461459in}}%
\pgfpathlineto{\pgfqpoint{0.420923in}{1.474620in}}%
\pgfpathlineto{\pgfqpoint{0.420436in}{1.474968in}}%
\pgfpathlineto{\pgfqpoint{0.405266in}{1.486319in}}%
\pgfpathlineto{\pgfqpoint{0.400874in}{1.488579in}}%
\pgfpathlineto{\pgfqpoint{0.389609in}{1.495161in}}%
\pgfpathlineto{\pgfqpoint{0.373953in}{1.498452in}}%
\pgfpathlineto{\pgfqpoint{0.373953in}{1.488579in}}%
\pgfpathlineto{\pgfqpoint{0.373953in}{1.474968in}}%
\pgfpathlineto{\pgfqpoint{0.373953in}{1.461626in}}%
\pgfpathlineto{\pgfqpoint{0.375580in}{1.461357in}}%
\pgfpathlineto{\pgfqpoint{0.389609in}{1.458767in}}%
\pgfpathlineto{\pgfqpoint{0.405266in}{1.450808in}}%
\pgfpathlineto{\pgfqpoint{0.409357in}{1.447746in}}%
\pgfpathlineto{\pgfqpoint{0.420923in}{1.437135in}}%
\pgfpathlineto{\pgfqpoint{0.423670in}{1.434135in}}%
\pgfpathlineto{\pgfqpoint{0.431524in}{1.420524in}}%
\pgfpathlineto{\pgfqpoint{0.433468in}{1.406913in}}%
\pgfpathlineto{\pgfqpoint{0.429011in}{1.393302in}}%
\pgfpathlineto{\pgfqpoint{0.420923in}{1.382101in}}%
\pgfpathlineto{\pgfqpoint{0.418826in}{1.379691in}}%
\pgfpathlineto{\pgfqpoint{0.405266in}{1.368459in}}%
\pgfpathlineto{\pgfqpoint{0.401063in}{1.366079in}}%
\pgfpathlineto{\pgfqpoint{0.389609in}{1.360558in}}%
\pgfpathlineto{\pgfqpoint{0.373953in}{1.357842in}}%
\pgfpathlineto{\pgfqpoint{0.373953in}{1.352468in}}%
\pgfpathlineto{\pgfqpoint{0.373953in}{1.338857in}}%
\pgfpathlineto{\pgfqpoint{0.373953in}{1.325246in}}%
\pgfpathlineto{\pgfqpoint{0.373953in}{1.320754in}}%
\pgfpathlineto{\pgfqpoint{0.389609in}{1.324275in}}%
\pgfpathclose%
\pgfpathmoveto{\pgfqpoint{0.671428in}{1.323025in}}%
\pgfpathlineto{\pgfqpoint{0.687084in}{1.320898in}}%
\pgfpathlineto{\pgfqpoint{0.701013in}{1.325246in}}%
\pgfpathlineto{\pgfqpoint{0.702741in}{1.325709in}}%
\pgfpathlineto{\pgfqpoint{0.718397in}{1.335056in}}%
\pgfpathlineto{\pgfqpoint{0.723002in}{1.338857in}}%
\pgfpathlineto{\pgfqpoint{0.734054in}{1.347230in}}%
\pgfpathlineto{\pgfqpoint{0.740130in}{1.352468in}}%
\pgfpathlineto{\pgfqpoint{0.749710in}{1.360797in}}%
\pgfpathlineto{\pgfqpoint{0.755736in}{1.366079in}}%
\pgfpathlineto{\pgfqpoint{0.765367in}{1.375687in}}%
\pgfpathlineto{\pgfqpoint{0.769740in}{1.379691in}}%
\pgfpathlineto{\pgfqpoint{0.780491in}{1.393302in}}%
\pgfpathlineto{\pgfqpoint{0.781024in}{1.394803in}}%
\pgfpathlineto{\pgfqpoint{0.786026in}{1.406913in}}%
\pgfpathlineto{\pgfqpoint{0.783578in}{1.420524in}}%
\pgfpathlineto{\pgfqpoint{0.781024in}{1.423950in}}%
\pgfpathlineto{\pgfqpoint{0.774604in}{1.434135in}}%
\pgfpathlineto{\pgfqpoint{0.765367in}{1.443532in}}%
\pgfpathlineto{\pgfqpoint{0.761542in}{1.447746in}}%
\pgfpathlineto{\pgfqpoint{0.749710in}{1.458512in}}%
\pgfpathlineto{\pgfqpoint{0.746549in}{1.461357in}}%
\pgfpathlineto{\pgfqpoint{0.734054in}{1.472023in}}%
\pgfpathlineto{\pgfqpoint{0.730130in}{1.474968in}}%
\pgfpathlineto{\pgfqpoint{0.718397in}{1.484182in}}%
\pgfpathlineto{\pgfqpoint{0.710696in}{1.488579in}}%
\pgfpathlineto{\pgfqpoint{0.702741in}{1.493759in}}%
\pgfpathlineto{\pgfqpoint{0.687084in}{1.498318in}}%
\pgfpathlineto{\pgfqpoint{0.671428in}{1.496329in}}%
\pgfpathlineto{\pgfqpoint{0.656342in}{1.488579in}}%
\pgfpathlineto{\pgfqpoint{0.655771in}{1.488320in}}%
\pgfpathlineto{\pgfqpoint{0.640115in}{1.477130in}}%
\pgfpathlineto{\pgfqpoint{0.637627in}{1.474968in}}%
\pgfpathlineto{\pgfqpoint{0.624458in}{1.464098in}}%
\pgfpathlineto{\pgfqpoint{0.621306in}{1.461357in}}%
\pgfpathlineto{\pgfqpoint{0.608801in}{1.449909in}}%
\pgfpathlineto{\pgfqpoint{0.606315in}{1.447746in}}%
\pgfpathlineto{\pgfqpoint{0.593444in}{1.434135in}}%
\pgfpathlineto{\pgfqpoint{0.593145in}{1.433638in}}%
\pgfpathlineto{\pgfqpoint{0.584231in}{1.420524in}}%
\pgfpathlineto{\pgfqpoint{0.581943in}{1.406913in}}%
\pgfpathlineto{\pgfqpoint{0.587187in}{1.393302in}}%
\pgfpathlineto{\pgfqpoint{0.593145in}{1.386386in}}%
\pgfpathlineto{\pgfqpoint{0.598203in}{1.379691in}}%
\pgfpathlineto{\pgfqpoint{0.608801in}{1.369490in}}%
\pgfpathlineto{\pgfqpoint{0.612189in}{1.366079in}}%
\pgfpathlineto{\pgfqpoint{0.624458in}{1.355216in}}%
\pgfpathlineto{\pgfqpoint{0.627731in}{1.352468in}}%
\pgfpathlineto{\pgfqpoint{0.640115in}{1.342183in}}%
\pgfpathlineto{\pgfqpoint{0.644962in}{1.338857in}}%
\pgfpathlineto{\pgfqpoint{0.655771in}{1.330827in}}%
\pgfpathlineto{\pgfqpoint{0.667486in}{1.325246in}}%
\pgfpathlineto{\pgfqpoint{0.671428in}{1.323025in}}%
\pgfpathclose%
\pgfpathmoveto{\pgfqpoint{0.656127in}{1.366079in}}%
\pgfpathlineto{\pgfqpoint{0.655771in}{1.366257in}}%
\pgfpathlineto{\pgfqpoint{0.640115in}{1.378571in}}%
\pgfpathlineto{\pgfqpoint{0.638951in}{1.379691in}}%
\pgfpathlineto{\pgfqpoint{0.628857in}{1.393302in}}%
\pgfpathlineto{\pgfqpoint{0.624458in}{1.406292in}}%
\pgfpathlineto{\pgfqpoint{0.624269in}{1.406913in}}%
\pgfpathlineto{\pgfqpoint{0.624458in}{1.408351in}}%
\pgfpathlineto{\pgfqpoint{0.626256in}{1.420524in}}%
\pgfpathlineto{\pgfqpoint{0.634384in}{1.434135in}}%
\pgfpathlineto{\pgfqpoint{0.640115in}{1.440261in}}%
\pgfpathlineto{\pgfqpoint{0.648724in}{1.447746in}}%
\pgfpathlineto{\pgfqpoint{0.655771in}{1.452728in}}%
\pgfpathlineto{\pgfqpoint{0.671428in}{1.459794in}}%
\pgfpathlineto{\pgfqpoint{0.685429in}{1.461357in}}%
\pgfpathlineto{\pgfqpoint{0.687084in}{1.461522in}}%
\pgfpathlineto{\pgfqpoint{0.687798in}{1.461357in}}%
\pgfpathlineto{\pgfqpoint{0.702741in}{1.457533in}}%
\pgfpathlineto{\pgfqpoint{0.718397in}{1.448758in}}%
\pgfpathlineto{\pgfqpoint{0.719685in}{1.447746in}}%
\pgfpathlineto{\pgfqpoint{0.733849in}{1.434135in}}%
\pgfpathlineto{\pgfqpoint{0.734054in}{1.433826in}}%
\pgfpathlineto{\pgfqpoint{0.741514in}{1.420524in}}%
\pgfpathlineto{\pgfqpoint{0.743402in}{1.406913in}}%
\pgfpathlineto{\pgfqpoint{0.739073in}{1.393302in}}%
\pgfpathlineto{\pgfqpoint{0.734054in}{1.386240in}}%
\pgfpathlineto{\pgfqpoint{0.728612in}{1.379691in}}%
\pgfpathlineto{\pgfqpoint{0.718397in}{1.370811in}}%
\pgfpathlineto{\pgfqpoint{0.710864in}{1.366079in}}%
\pgfpathlineto{\pgfqpoint{0.702741in}{1.361716in}}%
\pgfpathlineto{\pgfqpoint{0.687084in}{1.357953in}}%
\pgfpathlineto{\pgfqpoint{0.671428in}{1.359594in}}%
\pgfpathlineto{\pgfqpoint{0.656127in}{1.366079in}}%
\pgfpathclose%
\pgfpathmoveto{\pgfqpoint{0.984559in}{1.322040in}}%
\pgfpathlineto{\pgfqpoint{1.000216in}{1.321328in}}%
\pgfpathlineto{\pgfqpoint{1.010047in}{1.325246in}}%
\pgfpathlineto{\pgfqpoint{1.015872in}{1.327229in}}%
\pgfpathlineto{\pgfqpoint{1.031529in}{1.337361in}}%
\pgfpathlineto{\pgfqpoint{1.033269in}{1.338857in}}%
\pgfpathlineto{\pgfqpoint{1.047185in}{1.349810in}}%
\pgfpathlineto{\pgfqpoint{1.050235in}{1.352468in}}%
\pgfpathlineto{\pgfqpoint{1.062842in}{1.363539in}}%
\pgfpathlineto{\pgfqpoint{1.065772in}{1.366079in}}%
\pgfpathlineto{\pgfqpoint{1.078498in}{1.378643in}}%
\pgfpathlineto{\pgfqpoint{1.079678in}{1.379691in}}%
\pgfpathlineto{\pgfqpoint{1.090616in}{1.393302in}}%
\pgfpathlineto{\pgfqpoint{1.094155in}{1.402993in}}%
\pgfpathlineto{\pgfqpoint{1.095863in}{1.406913in}}%
\pgfpathlineto{\pgfqpoint{1.094155in}{1.415998in}}%
\pgfpathlineto{\pgfqpoint{1.093434in}{1.420524in}}%
\pgfpathlineto{\pgfqpoint{1.084627in}{1.434135in}}%
\pgfpathlineto{\pgfqpoint{1.078498in}{1.440179in}}%
\pgfpathlineto{\pgfqpoint{1.071559in}{1.447746in}}%
\pgfpathlineto{\pgfqpoint{1.062842in}{1.455590in}}%
\pgfpathlineto{\pgfqpoint{1.056499in}{1.461357in}}%
\pgfpathlineto{\pgfqpoint{1.047185in}{1.469394in}}%
\pgfpathlineto{\pgfqpoint{1.040036in}{1.474968in}}%
\pgfpathlineto{\pgfqpoint{1.031529in}{1.481926in}}%
\pgfpathlineto{\pgfqpoint{1.020855in}{1.488579in}}%
\pgfpathlineto{\pgfqpoint{1.015872in}{1.492136in}}%
\pgfpathlineto{\pgfqpoint{1.000216in}{1.497915in}}%
\pgfpathlineto{\pgfqpoint{0.984559in}{1.497250in}}%
\pgfpathlineto{\pgfqpoint{0.968902in}{1.490310in}}%
\pgfpathlineto{\pgfqpoint{0.966661in}{1.488579in}}%
\pgfpathlineto{\pgfqpoint{0.953246in}{1.479570in}}%
\pgfpathlineto{\pgfqpoint{0.947805in}{1.474968in}}%
\pgfpathlineto{\pgfqpoint{0.937589in}{1.466747in}}%
\pgfpathlineto{\pgfqpoint{0.931380in}{1.461357in}}%
\pgfpathlineto{\pgfqpoint{0.921933in}{1.452719in}}%
\pgfpathlineto{\pgfqpoint{0.916322in}{1.447746in}}%
\pgfpathlineto{\pgfqpoint{0.906276in}{1.436945in}}%
\pgfpathlineto{\pgfqpoint{0.903325in}{1.434135in}}%
\pgfpathlineto{\pgfqpoint{0.894325in}{1.420524in}}%
\pgfpathlineto{\pgfqpoint{0.892097in}{1.406913in}}%
\pgfpathlineto{\pgfqpoint{0.897204in}{1.393302in}}%
\pgfpathlineto{\pgfqpoint{0.906276in}{1.382348in}}%
\pgfpathlineto{\pgfqpoint{0.908249in}{1.379691in}}%
\pgfpathlineto{\pgfqpoint{0.921933in}{1.366267in}}%
\pgfpathlineto{\pgfqpoint{0.922120in}{1.366079in}}%
\pgfpathlineto{\pgfqpoint{0.937520in}{1.352468in}}%
\pgfpathlineto{\pgfqpoint{0.937589in}{1.352407in}}%
\pgfpathlineto{\pgfqpoint{0.953246in}{1.339748in}}%
\pgfpathlineto{\pgfqpoint{0.954630in}{1.338857in}}%
\pgfpathlineto{\pgfqpoint{0.968902in}{1.328941in}}%
\pgfpathlineto{\pgfqpoint{0.977914in}{1.325246in}}%
\pgfpathlineto{\pgfqpoint{0.984559in}{1.322040in}}%
\pgfpathclose%
\pgfpathmoveto{\pgfqpoint{0.966521in}{1.366079in}}%
\pgfpathlineto{\pgfqpoint{0.953246in}{1.375887in}}%
\pgfpathlineto{\pgfqpoint{0.949183in}{1.379691in}}%
\pgfpathlineto{\pgfqpoint{0.938680in}{1.393302in}}%
\pgfpathlineto{\pgfqpoint{0.937589in}{1.396442in}}%
\pgfpathlineto{\pgfqpoint{0.934390in}{1.406913in}}%
\pgfpathlineto{\pgfqpoint{0.936196in}{1.420524in}}%
\pgfpathlineto{\pgfqpoint{0.937589in}{1.423050in}}%
\pgfpathlineto{\pgfqpoint{0.944431in}{1.434135in}}%
\pgfpathlineto{\pgfqpoint{0.953246in}{1.443305in}}%
\pgfpathlineto{\pgfqpoint{0.958688in}{1.447746in}}%
\pgfpathlineto{\pgfqpoint{0.968902in}{1.454499in}}%
\pgfpathlineto{\pgfqpoint{0.984559in}{1.460605in}}%
\pgfpathlineto{\pgfqpoint{1.000216in}{1.461190in}}%
\pgfpathlineto{\pgfqpoint{1.015872in}{1.456106in}}%
\pgfpathlineto{\pgfqpoint{1.029548in}{1.447746in}}%
\pgfpathlineto{\pgfqpoint{1.031529in}{1.446246in}}%
\pgfpathlineto{\pgfqpoint{1.043568in}{1.434135in}}%
\pgfpathlineto{\pgfqpoint{1.047185in}{1.428456in}}%
\pgfpathlineto{\pgfqpoint{1.051585in}{1.420524in}}%
\pgfpathlineto{\pgfqpoint{1.053428in}{1.406913in}}%
\pgfpathlineto{\pgfqpoint{1.049204in}{1.393302in}}%
\pgfpathlineto{\pgfqpoint{1.047185in}{1.390430in}}%
\pgfpathlineto{\pgfqpoint{1.038595in}{1.379691in}}%
\pgfpathlineto{\pgfqpoint{1.031529in}{1.373293in}}%
\pgfpathlineto{\pgfqpoint{1.021008in}{1.366079in}}%
\pgfpathlineto{\pgfqpoint{1.015872in}{1.363055in}}%
\pgfpathlineto{\pgfqpoint{1.000216in}{1.358285in}}%
\pgfpathlineto{\pgfqpoint{0.984559in}{1.358834in}}%
\pgfpathlineto{\pgfqpoint{0.968902in}{1.364563in}}%
\pgfpathlineto{\pgfqpoint{0.966521in}{1.366079in}}%
\pgfpathclose%
\pgfpathmoveto{\pgfqpoint{1.297690in}{1.321328in}}%
\pgfpathlineto{\pgfqpoint{1.313347in}{1.322040in}}%
\pgfpathlineto{\pgfqpoint{1.319992in}{1.325246in}}%
\pgfpathlineto{\pgfqpoint{1.329003in}{1.328941in}}%
\pgfpathlineto{\pgfqpoint{1.343276in}{1.338857in}}%
\pgfpathlineto{\pgfqpoint{1.344660in}{1.339748in}}%
\pgfpathlineto{\pgfqpoint{1.360317in}{1.352407in}}%
\pgfpathlineto{\pgfqpoint{1.360386in}{1.352468in}}%
\pgfpathlineto{\pgfqpoint{1.375786in}{1.366079in}}%
\pgfpathlineto{\pgfqpoint{1.375973in}{1.366267in}}%
\pgfpathlineto{\pgfqpoint{1.389657in}{1.379691in}}%
\pgfpathlineto{\pgfqpoint{1.391630in}{1.382348in}}%
\pgfpathlineto{\pgfqpoint{1.400701in}{1.393302in}}%
\pgfpathlineto{\pgfqpoint{1.405809in}{1.406913in}}%
\pgfpathlineto{\pgfqpoint{1.403581in}{1.420524in}}%
\pgfpathlineto{\pgfqpoint{1.394581in}{1.434135in}}%
\pgfpathlineto{\pgfqpoint{1.391630in}{1.436945in}}%
\pgfpathlineto{\pgfqpoint{1.381584in}{1.447746in}}%
\pgfpathlineto{\pgfqpoint{1.375973in}{1.452719in}}%
\pgfpathlineto{\pgfqpoint{1.366525in}{1.461357in}}%
\pgfpathlineto{\pgfqpoint{1.360317in}{1.466747in}}%
\pgfpathlineto{\pgfqpoint{1.350100in}{1.474968in}}%
\pgfpathlineto{\pgfqpoint{1.344660in}{1.479570in}}%
\pgfpathlineto{\pgfqpoint{1.331245in}{1.488579in}}%
\pgfpathlineto{\pgfqpoint{1.329003in}{1.490310in}}%
\pgfpathlineto{\pgfqpoint{1.313347in}{1.497250in}}%
\pgfpathlineto{\pgfqpoint{1.297690in}{1.497915in}}%
\pgfpathlineto{\pgfqpoint{1.282034in}{1.492136in}}%
\pgfpathlineto{\pgfqpoint{1.277051in}{1.488579in}}%
\pgfpathlineto{\pgfqpoint{1.266377in}{1.481926in}}%
\pgfpathlineto{\pgfqpoint{1.257869in}{1.474968in}}%
\pgfpathlineto{\pgfqpoint{1.250721in}{1.469394in}}%
\pgfpathlineto{\pgfqpoint{1.241407in}{1.461357in}}%
\pgfpathlineto{\pgfqpoint{1.235064in}{1.455590in}}%
\pgfpathlineto{\pgfqpoint{1.226347in}{1.447746in}}%
\pgfpathlineto{\pgfqpoint{1.219407in}{1.440179in}}%
\pgfpathlineto{\pgfqpoint{1.213279in}{1.434135in}}%
\pgfpathlineto{\pgfqpoint{1.204472in}{1.420524in}}%
\pgfpathlineto{\pgfqpoint{1.203751in}{1.415998in}}%
\pgfpathlineto{\pgfqpoint{1.202042in}{1.406913in}}%
\pgfpathlineto{\pgfqpoint{1.203751in}{1.402993in}}%
\pgfpathlineto{\pgfqpoint{1.207290in}{1.393302in}}%
\pgfpathlineto{\pgfqpoint{1.218228in}{1.379691in}}%
\pgfpathlineto{\pgfqpoint{1.219407in}{1.378643in}}%
\pgfpathlineto{\pgfqpoint{1.232134in}{1.366079in}}%
\pgfpathlineto{\pgfqpoint{1.235064in}{1.363539in}}%
\pgfpathlineto{\pgfqpoint{1.247671in}{1.352468in}}%
\pgfpathlineto{\pgfqpoint{1.250721in}{1.349810in}}%
\pgfpathlineto{\pgfqpoint{1.264637in}{1.338857in}}%
\pgfpathlineto{\pgfqpoint{1.266377in}{1.337361in}}%
\pgfpathlineto{\pgfqpoint{1.282034in}{1.327229in}}%
\pgfpathlineto{\pgfqpoint{1.287859in}{1.325246in}}%
\pgfpathlineto{\pgfqpoint{1.297690in}{1.321328in}}%
\pgfpathclose%
\pgfpathmoveto{\pgfqpoint{1.276898in}{1.366079in}}%
\pgfpathlineto{\pgfqpoint{1.266377in}{1.373293in}}%
\pgfpathlineto{\pgfqpoint{1.259311in}{1.379691in}}%
\pgfpathlineto{\pgfqpoint{1.250721in}{1.390430in}}%
\pgfpathlineto{\pgfqpoint{1.248701in}{1.393302in}}%
\pgfpathlineto{\pgfqpoint{1.244478in}{1.406913in}}%
\pgfpathlineto{\pgfqpoint{1.246320in}{1.420524in}}%
\pgfpathlineto{\pgfqpoint{1.250721in}{1.428456in}}%
\pgfpathlineto{\pgfqpoint{1.254338in}{1.434135in}}%
\pgfpathlineto{\pgfqpoint{1.266377in}{1.446246in}}%
\pgfpathlineto{\pgfqpoint{1.268358in}{1.447746in}}%
\pgfpathlineto{\pgfqpoint{1.282034in}{1.456106in}}%
\pgfpathlineto{\pgfqpoint{1.297690in}{1.461190in}}%
\pgfpathlineto{\pgfqpoint{1.313347in}{1.460605in}}%
\pgfpathlineto{\pgfqpoint{1.329003in}{1.454499in}}%
\pgfpathlineto{\pgfqpoint{1.339217in}{1.447746in}}%
\pgfpathlineto{\pgfqpoint{1.344660in}{1.443305in}}%
\pgfpathlineto{\pgfqpoint{1.353474in}{1.434135in}}%
\pgfpathlineto{\pgfqpoint{1.360317in}{1.423050in}}%
\pgfpathlineto{\pgfqpoint{1.361710in}{1.420524in}}%
\pgfpathlineto{\pgfqpoint{1.363515in}{1.406913in}}%
\pgfpathlineto{\pgfqpoint{1.360317in}{1.396442in}}%
\pgfpathlineto{\pgfqpoint{1.359226in}{1.393302in}}%
\pgfpathlineto{\pgfqpoint{1.348723in}{1.379691in}}%
\pgfpathlineto{\pgfqpoint{1.344660in}{1.375887in}}%
\pgfpathlineto{\pgfqpoint{1.331384in}{1.366079in}}%
\pgfpathlineto{\pgfqpoint{1.329003in}{1.364563in}}%
\pgfpathlineto{\pgfqpoint{1.313347in}{1.358834in}}%
\pgfpathlineto{\pgfqpoint{1.297690in}{1.358285in}}%
\pgfpathlineto{\pgfqpoint{1.282034in}{1.363055in}}%
\pgfpathlineto{\pgfqpoint{1.276898in}{1.366079in}}%
\pgfpathclose%
\pgfpathmoveto{\pgfqpoint{1.610822in}{1.320898in}}%
\pgfpathlineto{\pgfqpoint{1.626478in}{1.323025in}}%
\pgfpathlineto{\pgfqpoint{1.630419in}{1.325246in}}%
\pgfpathlineto{\pgfqpoint{1.642135in}{1.330827in}}%
\pgfpathlineto{\pgfqpoint{1.652943in}{1.338857in}}%
\pgfpathlineto{\pgfqpoint{1.657791in}{1.342183in}}%
\pgfpathlineto{\pgfqpoint{1.670175in}{1.352468in}}%
\pgfpathlineto{\pgfqpoint{1.673448in}{1.355216in}}%
\pgfpathlineto{\pgfqpoint{1.685716in}{1.366079in}}%
\pgfpathlineto{\pgfqpoint{1.689104in}{1.369490in}}%
\pgfpathlineto{\pgfqpoint{1.699703in}{1.379691in}}%
\pgfpathlineto{\pgfqpoint{1.704761in}{1.386386in}}%
\pgfpathlineto{\pgfqpoint{1.710719in}{1.393302in}}%
\pgfpathlineto{\pgfqpoint{1.715962in}{1.406913in}}%
\pgfpathlineto{\pgfqpoint{1.713675in}{1.420524in}}%
\pgfpathlineto{\pgfqpoint{1.704761in}{1.433638in}}%
\pgfpathlineto{\pgfqpoint{1.704462in}{1.434135in}}%
\pgfpathlineto{\pgfqpoint{1.691591in}{1.447746in}}%
\pgfpathlineto{\pgfqpoint{1.689104in}{1.449909in}}%
\pgfpathlineto{\pgfqpoint{1.676600in}{1.461357in}}%
\pgfpathlineto{\pgfqpoint{1.673448in}{1.464098in}}%
\pgfpathlineto{\pgfqpoint{1.660279in}{1.474968in}}%
\pgfpathlineto{\pgfqpoint{1.657791in}{1.477130in}}%
\pgfpathlineto{\pgfqpoint{1.642135in}{1.488320in}}%
\pgfpathlineto{\pgfqpoint{1.641563in}{1.488579in}}%
\pgfpathlineto{\pgfqpoint{1.626478in}{1.496329in}}%
\pgfpathlineto{\pgfqpoint{1.610822in}{1.498318in}}%
\pgfpathlineto{\pgfqpoint{1.595165in}{1.493759in}}%
\pgfpathlineto{\pgfqpoint{1.587210in}{1.488579in}}%
\pgfpathlineto{\pgfqpoint{1.579508in}{1.484182in}}%
\pgfpathlineto{\pgfqpoint{1.567776in}{1.474968in}}%
\pgfpathlineto{\pgfqpoint{1.563852in}{1.472023in}}%
\pgfpathlineto{\pgfqpoint{1.551356in}{1.461357in}}%
\pgfpathlineto{\pgfqpoint{1.548195in}{1.458512in}}%
\pgfpathlineto{\pgfqpoint{1.536364in}{1.447746in}}%
\pgfpathlineto{\pgfqpoint{1.532539in}{1.443532in}}%
\pgfpathlineto{\pgfqpoint{1.523302in}{1.434135in}}%
\pgfpathlineto{\pgfqpoint{1.516882in}{1.423950in}}%
\pgfpathlineto{\pgfqpoint{1.514328in}{1.420524in}}%
\pgfpathlineto{\pgfqpoint{1.511880in}{1.406913in}}%
\pgfpathlineto{\pgfqpoint{1.516882in}{1.394803in}}%
\pgfpathlineto{\pgfqpoint{1.517415in}{1.393302in}}%
\pgfpathlineto{\pgfqpoint{1.528166in}{1.379691in}}%
\pgfpathlineto{\pgfqpoint{1.532539in}{1.375687in}}%
\pgfpathlineto{\pgfqpoint{1.542170in}{1.366079in}}%
\pgfpathlineto{\pgfqpoint{1.548195in}{1.360797in}}%
\pgfpathlineto{\pgfqpoint{1.557776in}{1.352468in}}%
\pgfpathlineto{\pgfqpoint{1.563852in}{1.347230in}}%
\pgfpathlineto{\pgfqpoint{1.574903in}{1.338857in}}%
\pgfpathlineto{\pgfqpoint{1.579508in}{1.335056in}}%
\pgfpathlineto{\pgfqpoint{1.595165in}{1.325709in}}%
\pgfpathlineto{\pgfqpoint{1.596892in}{1.325246in}}%
\pgfpathlineto{\pgfqpoint{1.610822in}{1.320898in}}%
\pgfpathclose%
\pgfpathmoveto{\pgfqpoint{1.587042in}{1.366079in}}%
\pgfpathlineto{\pgfqpoint{1.579508in}{1.370811in}}%
\pgfpathlineto{\pgfqpoint{1.569294in}{1.379691in}}%
\pgfpathlineto{\pgfqpoint{1.563852in}{1.386240in}}%
\pgfpathlineto{\pgfqpoint{1.558832in}{1.393302in}}%
\pgfpathlineto{\pgfqpoint{1.554504in}{1.406913in}}%
\pgfpathlineto{\pgfqpoint{1.556392in}{1.420524in}}%
\pgfpathlineto{\pgfqpoint{1.563852in}{1.433826in}}%
\pgfpathlineto{\pgfqpoint{1.564056in}{1.434135in}}%
\pgfpathlineto{\pgfqpoint{1.578221in}{1.447746in}}%
\pgfpathlineto{\pgfqpoint{1.579508in}{1.448758in}}%
\pgfpathlineto{\pgfqpoint{1.595165in}{1.457533in}}%
\pgfpathlineto{\pgfqpoint{1.610108in}{1.461357in}}%
\pgfpathlineto{\pgfqpoint{1.610822in}{1.461522in}}%
\pgfpathlineto{\pgfqpoint{1.612476in}{1.461357in}}%
\pgfpathlineto{\pgfqpoint{1.626478in}{1.459794in}}%
\pgfpathlineto{\pgfqpoint{1.642135in}{1.452728in}}%
\pgfpathlineto{\pgfqpoint{1.649181in}{1.447746in}}%
\pgfpathlineto{\pgfqpoint{1.657791in}{1.440261in}}%
\pgfpathlineto{\pgfqpoint{1.663522in}{1.434135in}}%
\pgfpathlineto{\pgfqpoint{1.671650in}{1.420524in}}%
\pgfpathlineto{\pgfqpoint{1.673448in}{1.408351in}}%
\pgfpathlineto{\pgfqpoint{1.673637in}{1.406913in}}%
\pgfpathlineto{\pgfqpoint{1.673448in}{1.406292in}}%
\pgfpathlineto{\pgfqpoint{1.669049in}{1.393302in}}%
\pgfpathlineto{\pgfqpoint{1.658955in}{1.379691in}}%
\pgfpathlineto{\pgfqpoint{1.657791in}{1.378571in}}%
\pgfpathlineto{\pgfqpoint{1.642135in}{1.366257in}}%
\pgfpathlineto{\pgfqpoint{1.641779in}{1.366079in}}%
\pgfpathlineto{\pgfqpoint{1.626478in}{1.359594in}}%
\pgfpathlineto{\pgfqpoint{1.610822in}{1.357953in}}%
\pgfpathlineto{\pgfqpoint{1.595165in}{1.361716in}}%
\pgfpathlineto{\pgfqpoint{1.587042in}{1.366079in}}%
\pgfpathclose%
\pgfpathmoveto{\pgfqpoint{1.908296in}{1.324275in}}%
\pgfpathlineto{\pgfqpoint{1.923953in}{1.320754in}}%
\pgfpathlineto{\pgfqpoint{1.923953in}{1.325246in}}%
\pgfpathlineto{\pgfqpoint{1.923953in}{1.338857in}}%
\pgfpathlineto{\pgfqpoint{1.923953in}{1.352468in}}%
\pgfpathlineto{\pgfqpoint{1.923953in}{1.357842in}}%
\pgfpathlineto{\pgfqpoint{1.908296in}{1.360558in}}%
\pgfpathlineto{\pgfqpoint{1.896843in}{1.366079in}}%
\pgfpathlineto{\pgfqpoint{1.892640in}{1.368459in}}%
\pgfpathlineto{\pgfqpoint{1.879080in}{1.379691in}}%
\pgfpathlineto{\pgfqpoint{1.876983in}{1.382101in}}%
\pgfpathlineto{\pgfqpoint{1.868895in}{1.393302in}}%
\pgfpathlineto{\pgfqpoint{1.864438in}{1.406913in}}%
\pgfpathlineto{\pgfqpoint{1.866382in}{1.420524in}}%
\pgfpathlineto{\pgfqpoint{1.874236in}{1.434135in}}%
\pgfpathlineto{\pgfqpoint{1.876983in}{1.437135in}}%
\pgfpathlineto{\pgfqpoint{1.888548in}{1.447746in}}%
\pgfpathlineto{\pgfqpoint{1.892640in}{1.450808in}}%
\pgfpathlineto{\pgfqpoint{1.908296in}{1.458767in}}%
\pgfpathlineto{\pgfqpoint{1.922326in}{1.461357in}}%
\pgfpathlineto{\pgfqpoint{1.923953in}{1.461626in}}%
\pgfpathlineto{\pgfqpoint{1.923953in}{1.474968in}}%
\pgfpathlineto{\pgfqpoint{1.923953in}{1.488579in}}%
\pgfpathlineto{\pgfqpoint{1.923953in}{1.498452in}}%
\pgfpathlineto{\pgfqpoint{1.908296in}{1.495161in}}%
\pgfpathlineto{\pgfqpoint{1.897031in}{1.488579in}}%
\pgfpathlineto{\pgfqpoint{1.892640in}{1.486319in}}%
\pgfpathlineto{\pgfqpoint{1.877469in}{1.474968in}}%
\pgfpathlineto{\pgfqpoint{1.876983in}{1.474620in}}%
\pgfpathlineto{\pgfqpoint{1.861327in}{1.461459in}}%
\pgfpathlineto{\pgfqpoint{1.861209in}{1.461357in}}%
\pgfpathlineto{\pgfqpoint{1.846348in}{1.447746in}}%
\pgfpathlineto{\pgfqpoint{1.845670in}{1.446994in}}%
\pgfpathlineto{\pgfqpoint{1.833366in}{1.434135in}}%
\pgfpathlineto{\pgfqpoint{1.830014in}{1.428681in}}%
\pgfpathlineto{\pgfqpoint{1.824220in}{1.420524in}}%
\pgfpathlineto{\pgfqpoint{1.821860in}{1.406913in}}%
\pgfpathlineto{\pgfqpoint{1.827270in}{1.393302in}}%
\pgfpathlineto{\pgfqpoint{1.830014in}{1.390254in}}%
\pgfpathlineto{\pgfqpoint{1.838167in}{1.379691in}}%
\pgfpathlineto{\pgfqpoint{1.845670in}{1.372634in}}%
\pgfpathlineto{\pgfqpoint{1.852197in}{1.366079in}}%
\pgfpathlineto{\pgfqpoint{1.861327in}{1.358019in}}%
\pgfpathlineto{\pgfqpoint{1.867807in}{1.352468in}}%
\pgfpathlineto{\pgfqpoint{1.876983in}{1.344682in}}%
\pgfpathlineto{\pgfqpoint{1.885030in}{1.338857in}}%
\pgfpathlineto{\pgfqpoint{1.892640in}{1.332872in}}%
\pgfpathlineto{\pgfqpoint{1.906781in}{1.325246in}}%
\pgfpathlineto{\pgfqpoint{1.908296in}{1.324275in}}%
\pgfpathclose%
\pgfpathmoveto{\pgfqpoint{0.389609in}{1.593637in}}%
\pgfpathlineto{\pgfqpoint{0.395886in}{1.597468in}}%
\pgfpathlineto{\pgfqpoint{0.405266in}{1.602400in}}%
\pgfpathlineto{\pgfqpoint{0.416602in}{1.611079in}}%
\pgfpathlineto{\pgfqpoint{0.420923in}{1.614187in}}%
\pgfpathlineto{\pgfqpoint{0.433389in}{1.624691in}}%
\pgfpathlineto{\pgfqpoint{0.436579in}{1.627464in}}%
\pgfpathlineto{\pgfqpoint{0.448661in}{1.638302in}}%
\pgfpathlineto{\pgfqpoint{0.452236in}{1.642058in}}%
\pgfpathlineto{\pgfqpoint{0.462219in}{1.651913in}}%
\pgfpathlineto{\pgfqpoint{0.467892in}{1.660067in}}%
\pgfpathlineto{\pgfqpoint{0.472300in}{1.665524in}}%
\pgfpathlineto{\pgfqpoint{0.476205in}{1.679135in}}%
\pgfpathlineto{\pgfqpoint{0.467892in}{1.679135in}}%
\pgfpathlineto{\pgfqpoint{0.452236in}{1.679135in}}%
\pgfpathlineto{\pgfqpoint{0.436579in}{1.679135in}}%
\pgfpathlineto{\pgfqpoint{0.433600in}{1.679135in}}%
\pgfpathlineto{\pgfqpoint{0.430382in}{1.665524in}}%
\pgfpathlineto{\pgfqpoint{0.421536in}{1.651913in}}%
\pgfpathlineto{\pgfqpoint{0.420923in}{1.651280in}}%
\pgfpathlineto{\pgfqpoint{0.405994in}{1.638302in}}%
\pgfpathlineto{\pgfqpoint{0.405266in}{1.637768in}}%
\pgfpathlineto{\pgfqpoint{0.389609in}{1.630078in}}%
\pgfpathlineto{\pgfqpoint{0.373953in}{1.627281in}}%
\pgfpathlineto{\pgfqpoint{0.373953in}{1.624691in}}%
\pgfpathlineto{\pgfqpoint{0.373953in}{1.611079in}}%
\pgfpathlineto{\pgfqpoint{0.373953in}{1.597468in}}%
\pgfpathlineto{\pgfqpoint{0.373953in}{1.590242in}}%
\pgfpathlineto{\pgfqpoint{0.389609in}{1.593637in}}%
\pgfpathclose%
\pgfpathmoveto{\pgfqpoint{0.671428in}{1.592432in}}%
\pgfpathlineto{\pgfqpoint{0.687084in}{1.590380in}}%
\pgfpathlineto{\pgfqpoint{0.702741in}{1.595083in}}%
\pgfpathlineto{\pgfqpoint{0.706247in}{1.597468in}}%
\pgfpathlineto{\pgfqpoint{0.718397in}{1.604556in}}%
\pgfpathlineto{\pgfqpoint{0.726515in}{1.611079in}}%
\pgfpathlineto{\pgfqpoint{0.734054in}{1.616754in}}%
\pgfpathlineto{\pgfqpoint{0.743325in}{1.624691in}}%
\pgfpathlineto{\pgfqpoint{0.749710in}{1.630324in}}%
\pgfpathlineto{\pgfqpoint{0.758666in}{1.638302in}}%
\pgfpathlineto{\pgfqpoint{0.765367in}{1.645297in}}%
\pgfpathlineto{\pgfqpoint{0.772252in}{1.651913in}}%
\pgfpathlineto{\pgfqpoint{0.781024in}{1.664206in}}%
\pgfpathlineto{\pgfqpoint{0.782141in}{1.665524in}}%
\pgfpathlineto{\pgfqpoint{0.786191in}{1.679135in}}%
\pgfpathlineto{\pgfqpoint{0.781024in}{1.679135in}}%
\pgfpathlineto{\pgfqpoint{0.765367in}{1.679135in}}%
\pgfpathlineto{\pgfqpoint{0.749710in}{1.679135in}}%
\pgfpathlineto{\pgfqpoint{0.743530in}{1.679135in}}%
\pgfpathlineto{\pgfqpoint{0.740405in}{1.665524in}}%
\pgfpathlineto{\pgfqpoint{0.734054in}{1.655567in}}%
\pgfpathlineto{\pgfqpoint{0.731317in}{1.651913in}}%
\pgfpathlineto{\pgfqpoint{0.718397in}{1.640124in}}%
\pgfpathlineto{\pgfqpoint{0.715624in}{1.638302in}}%
\pgfpathlineto{\pgfqpoint{0.702741in}{1.631270in}}%
\pgfpathlineto{\pgfqpoint{0.687084in}{1.627395in}}%
\pgfpathlineto{\pgfqpoint{0.671428in}{1.629085in}}%
\pgfpathlineto{\pgfqpoint{0.655771in}{1.635913in}}%
\pgfpathlineto{\pgfqpoint{0.652320in}{1.638302in}}%
\pgfpathlineto{\pgfqpoint{0.640115in}{1.648356in}}%
\pgfpathlineto{\pgfqpoint{0.636592in}{1.651913in}}%
\pgfpathlineto{\pgfqpoint{0.627438in}{1.665524in}}%
\pgfpathlineto{\pgfqpoint{0.624458in}{1.677721in}}%
\pgfpathlineto{\pgfqpoint{0.624148in}{1.679135in}}%
\pgfpathlineto{\pgfqpoint{0.608801in}{1.679135in}}%
\pgfpathlineto{\pgfqpoint{0.593145in}{1.679135in}}%
\pgfpathlineto{\pgfqpoint{0.581789in}{1.679135in}}%
\pgfpathlineto{\pgfqpoint{0.585574in}{1.665524in}}%
\pgfpathlineto{\pgfqpoint{0.593145in}{1.655731in}}%
\pgfpathlineto{\pgfqpoint{0.595745in}{1.651913in}}%
\pgfpathlineto{\pgfqpoint{0.608801in}{1.638724in}}%
\pgfpathlineto{\pgfqpoint{0.609202in}{1.638302in}}%
\pgfpathlineto{\pgfqpoint{0.624341in}{1.624691in}}%
\pgfpathlineto{\pgfqpoint{0.624458in}{1.624588in}}%
\pgfpathlineto{\pgfqpoint{0.640115in}{1.611669in}}%
\pgfpathlineto{\pgfqpoint{0.640979in}{1.611079in}}%
\pgfpathlineto{\pgfqpoint{0.655771in}{1.600382in}}%
\pgfpathlineto{\pgfqpoint{0.662044in}{1.597468in}}%
\pgfpathlineto{\pgfqpoint{0.671428in}{1.592432in}}%
\pgfpathclose%
\pgfpathmoveto{\pgfqpoint{0.984559in}{1.591481in}}%
\pgfpathlineto{\pgfqpoint{1.000216in}{1.590795in}}%
\pgfpathlineto{\pgfqpoint{1.015872in}{1.596757in}}%
\pgfpathlineto{\pgfqpoint{1.016826in}{1.597468in}}%
\pgfpathlineto{\pgfqpoint{1.031529in}{1.606832in}}%
\pgfpathlineto{\pgfqpoint{1.036604in}{1.611079in}}%
\pgfpathlineto{\pgfqpoint{1.047185in}{1.619353in}}%
\pgfpathlineto{\pgfqpoint{1.053353in}{1.624691in}}%
\pgfpathlineto{\pgfqpoint{1.062842in}{1.633147in}}%
\pgfpathlineto{\pgfqpoint{1.068693in}{1.638302in}}%
\pgfpathlineto{\pgfqpoint{1.078498in}{1.648432in}}%
\pgfpathlineto{\pgfqpoint{1.082234in}{1.651913in}}%
\pgfpathlineto{\pgfqpoint{1.092154in}{1.665524in}}%
\pgfpathlineto{\pgfqpoint{1.094155in}{1.673121in}}%
\pgfpathlineto{\pgfqpoint{1.096036in}{1.679135in}}%
\pgfpathlineto{\pgfqpoint{1.094155in}{1.679135in}}%
\pgfpathlineto{\pgfqpoint{1.078498in}{1.679135in}}%
\pgfpathlineto{\pgfqpoint{1.062842in}{1.679135in}}%
\pgfpathlineto{\pgfqpoint{1.053553in}{1.679135in}}%
\pgfpathlineto{\pgfqpoint{1.050503in}{1.665524in}}%
\pgfpathlineto{\pgfqpoint{1.047185in}{1.660265in}}%
\pgfpathlineto{\pgfqpoint{1.041163in}{1.651913in}}%
\pgfpathlineto{\pgfqpoint{1.031529in}{1.642758in}}%
\pgfpathlineto{\pgfqpoint{1.025319in}{1.638302in}}%
\pgfpathlineto{\pgfqpoint{1.015872in}{1.632649in}}%
\pgfpathlineto{\pgfqpoint{1.000216in}{1.627737in}}%
\pgfpathlineto{\pgfqpoint{0.984559in}{1.628302in}}%
\pgfpathlineto{\pgfqpoint{0.968902in}{1.634202in}}%
\pgfpathlineto{\pgfqpoint{0.962568in}{1.638302in}}%
\pgfpathlineto{\pgfqpoint{0.953246in}{1.645509in}}%
\pgfpathlineto{\pgfqpoint{0.946729in}{1.651913in}}%
\pgfpathlineto{\pgfqpoint{0.937589in}{1.664994in}}%
\pgfpathlineto{\pgfqpoint{0.937257in}{1.665524in}}%
\pgfpathlineto{\pgfqpoint{0.934268in}{1.679135in}}%
\pgfpathlineto{\pgfqpoint{0.921933in}{1.679135in}}%
\pgfpathlineto{\pgfqpoint{0.906276in}{1.679135in}}%
\pgfpathlineto{\pgfqpoint{0.891946in}{1.679135in}}%
\pgfpathlineto{\pgfqpoint{0.895633in}{1.665524in}}%
\pgfpathlineto{\pgfqpoint{0.905770in}{1.651913in}}%
\pgfpathlineto{\pgfqpoint{0.906276in}{1.651457in}}%
\pgfpathlineto{\pgfqpoint{0.919190in}{1.638302in}}%
\pgfpathlineto{\pgfqpoint{0.921933in}{1.635922in}}%
\pgfpathlineto{\pgfqpoint{0.934464in}{1.624691in}}%
\pgfpathlineto{\pgfqpoint{0.937589in}{1.621969in}}%
\pgfpathlineto{\pgfqpoint{0.951085in}{1.611079in}}%
\pgfpathlineto{\pgfqpoint{0.953246in}{1.609210in}}%
\pgfpathlineto{\pgfqpoint{0.968902in}{1.598520in}}%
\pgfpathlineto{\pgfqpoint{0.971533in}{1.597468in}}%
\pgfpathlineto{\pgfqpoint{0.984559in}{1.591481in}}%
\pgfpathclose%
\pgfpathmoveto{\pgfqpoint{1.282034in}{1.596757in}}%
\pgfpathlineto{\pgfqpoint{1.297690in}{1.590795in}}%
\pgfpathlineto{\pgfqpoint{1.313347in}{1.591481in}}%
\pgfpathlineto{\pgfqpoint{1.326373in}{1.597468in}}%
\pgfpathlineto{\pgfqpoint{1.329003in}{1.598520in}}%
\pgfpathlineto{\pgfqpoint{1.344660in}{1.609210in}}%
\pgfpathlineto{\pgfqpoint{1.346820in}{1.611079in}}%
\pgfpathlineto{\pgfqpoint{1.360317in}{1.621969in}}%
\pgfpathlineto{\pgfqpoint{1.363442in}{1.624691in}}%
\pgfpathlineto{\pgfqpoint{1.375973in}{1.635922in}}%
\pgfpathlineto{\pgfqpoint{1.378716in}{1.638302in}}%
\pgfpathlineto{\pgfqpoint{1.391630in}{1.651457in}}%
\pgfpathlineto{\pgfqpoint{1.392136in}{1.651913in}}%
\pgfpathlineto{\pgfqpoint{1.402272in}{1.665524in}}%
\pgfpathlineto{\pgfqpoint{1.405960in}{1.679135in}}%
\pgfpathlineto{\pgfqpoint{1.391630in}{1.679135in}}%
\pgfpathlineto{\pgfqpoint{1.375973in}{1.679135in}}%
\pgfpathlineto{\pgfqpoint{1.363638in}{1.679135in}}%
\pgfpathlineto{\pgfqpoint{1.360649in}{1.665524in}}%
\pgfpathlineto{\pgfqpoint{1.360317in}{1.664994in}}%
\pgfpathlineto{\pgfqpoint{1.351177in}{1.651913in}}%
\pgfpathlineto{\pgfqpoint{1.344660in}{1.645509in}}%
\pgfpathlineto{\pgfqpoint{1.335338in}{1.638302in}}%
\pgfpathlineto{\pgfqpoint{1.329003in}{1.634202in}}%
\pgfpathlineto{\pgfqpoint{1.313347in}{1.628302in}}%
\pgfpathlineto{\pgfqpoint{1.297690in}{1.627737in}}%
\pgfpathlineto{\pgfqpoint{1.282034in}{1.632649in}}%
\pgfpathlineto{\pgfqpoint{1.272587in}{1.638302in}}%
\pgfpathlineto{\pgfqpoint{1.266377in}{1.642758in}}%
\pgfpathlineto{\pgfqpoint{1.256742in}{1.651913in}}%
\pgfpathlineto{\pgfqpoint{1.250721in}{1.660265in}}%
\pgfpathlineto{\pgfqpoint{1.247402in}{1.665524in}}%
\pgfpathlineto{\pgfqpoint{1.244353in}{1.679135in}}%
\pgfpathlineto{\pgfqpoint{1.235064in}{1.679135in}}%
\pgfpathlineto{\pgfqpoint{1.219407in}{1.679135in}}%
\pgfpathlineto{\pgfqpoint{1.203751in}{1.679135in}}%
\pgfpathlineto{\pgfqpoint{1.201870in}{1.679135in}}%
\pgfpathlineto{\pgfqpoint{1.203751in}{1.673121in}}%
\pgfpathlineto{\pgfqpoint{1.205752in}{1.665524in}}%
\pgfpathlineto{\pgfqpoint{1.215672in}{1.651913in}}%
\pgfpathlineto{\pgfqpoint{1.219407in}{1.648432in}}%
\pgfpathlineto{\pgfqpoint{1.229213in}{1.638302in}}%
\pgfpathlineto{\pgfqpoint{1.235064in}{1.633147in}}%
\pgfpathlineto{\pgfqpoint{1.244553in}{1.624691in}}%
\pgfpathlineto{\pgfqpoint{1.250721in}{1.619353in}}%
\pgfpathlineto{\pgfqpoint{1.261302in}{1.611079in}}%
\pgfpathlineto{\pgfqpoint{1.266377in}{1.606832in}}%
\pgfpathlineto{\pgfqpoint{1.281080in}{1.597468in}}%
\pgfpathlineto{\pgfqpoint{1.282034in}{1.596757in}}%
\pgfpathclose%
\pgfpathmoveto{\pgfqpoint{1.595165in}{1.595083in}}%
\pgfpathlineto{\pgfqpoint{1.610822in}{1.590380in}}%
\pgfpathlineto{\pgfqpoint{1.626478in}{1.592432in}}%
\pgfpathlineto{\pgfqpoint{1.635862in}{1.597468in}}%
\pgfpathlineto{\pgfqpoint{1.642135in}{1.600382in}}%
\pgfpathlineto{\pgfqpoint{1.656927in}{1.611079in}}%
\pgfpathlineto{\pgfqpoint{1.657791in}{1.611669in}}%
\pgfpathlineto{\pgfqpoint{1.673448in}{1.624588in}}%
\pgfpathlineto{\pgfqpoint{1.673565in}{1.624691in}}%
\pgfpathlineto{\pgfqpoint{1.688703in}{1.638302in}}%
\pgfpathlineto{\pgfqpoint{1.689104in}{1.638724in}}%
\pgfpathlineto{\pgfqpoint{1.702161in}{1.651913in}}%
\pgfpathlineto{\pgfqpoint{1.704761in}{1.655731in}}%
\pgfpathlineto{\pgfqpoint{1.712332in}{1.665524in}}%
\pgfpathlineto{\pgfqpoint{1.716117in}{1.679135in}}%
\pgfpathlineto{\pgfqpoint{1.704761in}{1.679135in}}%
\pgfpathlineto{\pgfqpoint{1.689104in}{1.679135in}}%
\pgfpathlineto{\pgfqpoint{1.673757in}{1.679135in}}%
\pgfpathlineto{\pgfqpoint{1.673448in}{1.677721in}}%
\pgfpathlineto{\pgfqpoint{1.670468in}{1.665524in}}%
\pgfpathlineto{\pgfqpoint{1.661314in}{1.651913in}}%
\pgfpathlineto{\pgfqpoint{1.657791in}{1.648356in}}%
\pgfpathlineto{\pgfqpoint{1.645586in}{1.638302in}}%
\pgfpathlineto{\pgfqpoint{1.642135in}{1.635913in}}%
\pgfpathlineto{\pgfqpoint{1.626478in}{1.629085in}}%
\pgfpathlineto{\pgfqpoint{1.610822in}{1.627395in}}%
\pgfpathlineto{\pgfqpoint{1.595165in}{1.631270in}}%
\pgfpathlineto{\pgfqpoint{1.582282in}{1.638302in}}%
\pgfpathlineto{\pgfqpoint{1.579508in}{1.640124in}}%
\pgfpathlineto{\pgfqpoint{1.566589in}{1.651913in}}%
\pgfpathlineto{\pgfqpoint{1.563852in}{1.655567in}}%
\pgfpathlineto{\pgfqpoint{1.557501in}{1.665524in}}%
\pgfpathlineto{\pgfqpoint{1.554376in}{1.679135in}}%
\pgfpathlineto{\pgfqpoint{1.548195in}{1.679135in}}%
\pgfpathlineto{\pgfqpoint{1.532539in}{1.679135in}}%
\pgfpathlineto{\pgfqpoint{1.516882in}{1.679135in}}%
\pgfpathlineto{\pgfqpoint{1.511715in}{1.679135in}}%
\pgfpathlineto{\pgfqpoint{1.515765in}{1.665524in}}%
\pgfpathlineto{\pgfqpoint{1.516882in}{1.664206in}}%
\pgfpathlineto{\pgfqpoint{1.525654in}{1.651913in}}%
\pgfpathlineto{\pgfqpoint{1.532539in}{1.645297in}}%
\pgfpathlineto{\pgfqpoint{1.539239in}{1.638302in}}%
\pgfpathlineto{\pgfqpoint{1.548195in}{1.630324in}}%
\pgfpathlineto{\pgfqpoint{1.554580in}{1.624691in}}%
\pgfpathlineto{\pgfqpoint{1.563852in}{1.616754in}}%
\pgfpathlineto{\pgfqpoint{1.571391in}{1.611079in}}%
\pgfpathlineto{\pgfqpoint{1.579508in}{1.604556in}}%
\pgfpathlineto{\pgfqpoint{1.591659in}{1.597468in}}%
\pgfpathlineto{\pgfqpoint{1.595165in}{1.595083in}}%
\pgfpathclose%
\pgfpathmoveto{\pgfqpoint{1.908296in}{1.593637in}}%
\pgfpathlineto{\pgfqpoint{1.923953in}{1.590242in}}%
\pgfpathlineto{\pgfqpoint{1.923953in}{1.597468in}}%
\pgfpathlineto{\pgfqpoint{1.923953in}{1.611079in}}%
\pgfpathlineto{\pgfqpoint{1.923953in}{1.624691in}}%
\pgfpathlineto{\pgfqpoint{1.923953in}{1.627281in}}%
\pgfpathlineto{\pgfqpoint{1.908296in}{1.630078in}}%
\pgfpathlineto{\pgfqpoint{1.892640in}{1.637768in}}%
\pgfpathlineto{\pgfqpoint{1.891911in}{1.638302in}}%
\pgfpathlineto{\pgfqpoint{1.876983in}{1.651280in}}%
\pgfpathlineto{\pgfqpoint{1.876369in}{1.651913in}}%
\pgfpathlineto{\pgfqpoint{1.867524in}{1.665524in}}%
\pgfpathlineto{\pgfqpoint{1.864306in}{1.679135in}}%
\pgfpathlineto{\pgfqpoint{1.861327in}{1.679135in}}%
\pgfpathlineto{\pgfqpoint{1.845670in}{1.679135in}}%
\pgfpathlineto{\pgfqpoint{1.830014in}{1.679135in}}%
\pgfpathlineto{\pgfqpoint{1.821701in}{1.679135in}}%
\pgfpathlineto{\pgfqpoint{1.825606in}{1.665524in}}%
\pgfpathlineto{\pgfqpoint{1.830014in}{1.660067in}}%
\pgfpathlineto{\pgfqpoint{1.835687in}{1.651913in}}%
\pgfpathlineto{\pgfqpoint{1.845670in}{1.642058in}}%
\pgfpathlineto{\pgfqpoint{1.849245in}{1.638302in}}%
\pgfpathlineto{\pgfqpoint{1.861327in}{1.627464in}}%
\pgfpathlineto{\pgfqpoint{1.864516in}{1.624691in}}%
\pgfpathlineto{\pgfqpoint{1.876983in}{1.614187in}}%
\pgfpathlineto{\pgfqpoint{1.881304in}{1.611079in}}%
\pgfpathlineto{\pgfqpoint{1.892640in}{1.602400in}}%
\pgfpathlineto{\pgfqpoint{1.902020in}{1.597468in}}%
\pgfpathlineto{\pgfqpoint{1.908296in}{1.593637in}}%
\pgfpathclose%
\pgfusepath{fill}%
\end{pgfscope}%
\begin{pgfscope}%
\pgfpathrectangle{\pgfqpoint{0.373953in}{0.331635in}}{\pgfqpoint{1.550000in}{1.347500in}}%
\pgfusepath{clip}%
\pgfsetbuttcap%
\pgfsetroundjoin%
\definecolor{currentfill}{rgb}{0.048062,0.036607,0.150327}%
\pgfsetfillcolor{currentfill}%
\pgfsetlinewidth{0.000000pt}%
\definecolor{currentstroke}{rgb}{0.000000,0.000000,0.000000}%
\pgfsetstrokecolor{currentstroke}%
\pgfsetdash{}{0pt}%
\pgfpathmoveto{\pgfqpoint{0.389609in}{0.331635in}}%
\pgfpathlineto{\pgfqpoint{0.405266in}{0.331635in}}%
\pgfpathlineto{\pgfqpoint{0.420923in}{0.331635in}}%
\pgfpathlineto{\pgfqpoint{0.433600in}{0.331635in}}%
\pgfpathlineto{\pgfqpoint{0.430382in}{0.345246in}}%
\pgfpathlineto{\pgfqpoint{0.421536in}{0.358857in}}%
\pgfpathlineto{\pgfqpoint{0.420923in}{0.359490in}}%
\pgfpathlineto{\pgfqpoint{0.405994in}{0.372468in}}%
\pgfpathlineto{\pgfqpoint{0.405266in}{0.373002in}}%
\pgfpathlineto{\pgfqpoint{0.389609in}{0.380692in}}%
\pgfpathlineto{\pgfqpoint{0.373953in}{0.383489in}}%
\pgfpathlineto{\pgfqpoint{0.373953in}{0.372468in}}%
\pgfpathlineto{\pgfqpoint{0.373953in}{0.358857in}}%
\pgfpathlineto{\pgfqpoint{0.373953in}{0.345246in}}%
\pgfpathlineto{\pgfqpoint{0.373953in}{0.331635in}}%
\pgfpathlineto{\pgfqpoint{0.389609in}{0.331635in}}%
\pgfpathclose%
\pgfpathmoveto{\pgfqpoint{0.624458in}{0.331635in}}%
\pgfpathlineto{\pgfqpoint{0.640115in}{0.331635in}}%
\pgfpathlineto{\pgfqpoint{0.655771in}{0.331635in}}%
\pgfpathlineto{\pgfqpoint{0.671428in}{0.331635in}}%
\pgfpathlineto{\pgfqpoint{0.687084in}{0.331635in}}%
\pgfpathlineto{\pgfqpoint{0.702741in}{0.331635in}}%
\pgfpathlineto{\pgfqpoint{0.718397in}{0.331635in}}%
\pgfpathlineto{\pgfqpoint{0.734054in}{0.331635in}}%
\pgfpathlineto{\pgfqpoint{0.743530in}{0.331635in}}%
\pgfpathlineto{\pgfqpoint{0.740405in}{0.345246in}}%
\pgfpathlineto{\pgfqpoint{0.734054in}{0.355203in}}%
\pgfpathlineto{\pgfqpoint{0.731317in}{0.358857in}}%
\pgfpathlineto{\pgfqpoint{0.718397in}{0.370646in}}%
\pgfpathlineto{\pgfqpoint{0.715624in}{0.372468in}}%
\pgfpathlineto{\pgfqpoint{0.702741in}{0.379500in}}%
\pgfpathlineto{\pgfqpoint{0.687084in}{0.383375in}}%
\pgfpathlineto{\pgfqpoint{0.671428in}{0.381685in}}%
\pgfpathlineto{\pgfqpoint{0.655771in}{0.374857in}}%
\pgfpathlineto{\pgfqpoint{0.652320in}{0.372468in}}%
\pgfpathlineto{\pgfqpoint{0.640115in}{0.362414in}}%
\pgfpathlineto{\pgfqpoint{0.636592in}{0.358857in}}%
\pgfpathlineto{\pgfqpoint{0.627438in}{0.345246in}}%
\pgfpathlineto{\pgfqpoint{0.624458in}{0.333049in}}%
\pgfpathlineto{\pgfqpoint{0.624148in}{0.331635in}}%
\pgfpathlineto{\pgfqpoint{0.624458in}{0.331635in}}%
\pgfpathclose%
\pgfpathmoveto{\pgfqpoint{0.937589in}{0.331635in}}%
\pgfpathlineto{\pgfqpoint{0.953246in}{0.331635in}}%
\pgfpathlineto{\pgfqpoint{0.968902in}{0.331635in}}%
\pgfpathlineto{\pgfqpoint{0.984559in}{0.331635in}}%
\pgfpathlineto{\pgfqpoint{1.000216in}{0.331635in}}%
\pgfpathlineto{\pgfqpoint{1.015872in}{0.331635in}}%
\pgfpathlineto{\pgfqpoint{1.031529in}{0.331635in}}%
\pgfpathlineto{\pgfqpoint{1.047185in}{0.331635in}}%
\pgfpathlineto{\pgfqpoint{1.053553in}{0.331635in}}%
\pgfpathlineto{\pgfqpoint{1.050503in}{0.345246in}}%
\pgfpathlineto{\pgfqpoint{1.047185in}{0.350505in}}%
\pgfpathlineto{\pgfqpoint{1.041163in}{0.358857in}}%
\pgfpathlineto{\pgfqpoint{1.031529in}{0.368012in}}%
\pgfpathlineto{\pgfqpoint{1.025319in}{0.372468in}}%
\pgfpathlineto{\pgfqpoint{1.015872in}{0.378121in}}%
\pgfpathlineto{\pgfqpoint{1.000216in}{0.383033in}}%
\pgfpathlineto{\pgfqpoint{0.984559in}{0.382468in}}%
\pgfpathlineto{\pgfqpoint{0.968902in}{0.376568in}}%
\pgfpathlineto{\pgfqpoint{0.962568in}{0.372468in}}%
\pgfpathlineto{\pgfqpoint{0.953246in}{0.365261in}}%
\pgfpathlineto{\pgfqpoint{0.946729in}{0.358857in}}%
\pgfpathlineto{\pgfqpoint{0.937589in}{0.345776in}}%
\pgfpathlineto{\pgfqpoint{0.937257in}{0.345246in}}%
\pgfpathlineto{\pgfqpoint{0.934268in}{0.331635in}}%
\pgfpathlineto{\pgfqpoint{0.937589in}{0.331635in}}%
\pgfpathclose%
\pgfpathmoveto{\pgfqpoint{1.250721in}{0.331635in}}%
\pgfpathlineto{\pgfqpoint{1.266377in}{0.331635in}}%
\pgfpathlineto{\pgfqpoint{1.282034in}{0.331635in}}%
\pgfpathlineto{\pgfqpoint{1.297690in}{0.331635in}}%
\pgfpathlineto{\pgfqpoint{1.313347in}{0.331635in}}%
\pgfpathlineto{\pgfqpoint{1.329003in}{0.331635in}}%
\pgfpathlineto{\pgfqpoint{1.344660in}{0.331635in}}%
\pgfpathlineto{\pgfqpoint{1.360317in}{0.331635in}}%
\pgfpathlineto{\pgfqpoint{1.363638in}{0.331635in}}%
\pgfpathlineto{\pgfqpoint{1.360649in}{0.345246in}}%
\pgfpathlineto{\pgfqpoint{1.360317in}{0.345776in}}%
\pgfpathlineto{\pgfqpoint{1.351177in}{0.358857in}}%
\pgfpathlineto{\pgfqpoint{1.344660in}{0.365261in}}%
\pgfpathlineto{\pgfqpoint{1.335338in}{0.372468in}}%
\pgfpathlineto{\pgfqpoint{1.329003in}{0.376568in}}%
\pgfpathlineto{\pgfqpoint{1.313347in}{0.382468in}}%
\pgfpathlineto{\pgfqpoint{1.297690in}{0.383033in}}%
\pgfpathlineto{\pgfqpoint{1.282034in}{0.378121in}}%
\pgfpathlineto{\pgfqpoint{1.272587in}{0.372468in}}%
\pgfpathlineto{\pgfqpoint{1.266377in}{0.368012in}}%
\pgfpathlineto{\pgfqpoint{1.256742in}{0.358857in}}%
\pgfpathlineto{\pgfqpoint{1.250721in}{0.350505in}}%
\pgfpathlineto{\pgfqpoint{1.247402in}{0.345246in}}%
\pgfpathlineto{\pgfqpoint{1.244353in}{0.331635in}}%
\pgfpathlineto{\pgfqpoint{1.250721in}{0.331635in}}%
\pgfpathclose%
\pgfpathmoveto{\pgfqpoint{1.563852in}{0.331635in}}%
\pgfpathlineto{\pgfqpoint{1.579508in}{0.331635in}}%
\pgfpathlineto{\pgfqpoint{1.595165in}{0.331635in}}%
\pgfpathlineto{\pgfqpoint{1.610822in}{0.331635in}}%
\pgfpathlineto{\pgfqpoint{1.626478in}{0.331635in}}%
\pgfpathlineto{\pgfqpoint{1.642135in}{0.331635in}}%
\pgfpathlineto{\pgfqpoint{1.657791in}{0.331635in}}%
\pgfpathlineto{\pgfqpoint{1.673448in}{0.331635in}}%
\pgfpathlineto{\pgfqpoint{1.673757in}{0.331635in}}%
\pgfpathlineto{\pgfqpoint{1.673448in}{0.333049in}}%
\pgfpathlineto{\pgfqpoint{1.670468in}{0.345246in}}%
\pgfpathlineto{\pgfqpoint{1.661314in}{0.358857in}}%
\pgfpathlineto{\pgfqpoint{1.657791in}{0.362414in}}%
\pgfpathlineto{\pgfqpoint{1.645586in}{0.372468in}}%
\pgfpathlineto{\pgfqpoint{1.642135in}{0.374857in}}%
\pgfpathlineto{\pgfqpoint{1.626478in}{0.381685in}}%
\pgfpathlineto{\pgfqpoint{1.610822in}{0.383375in}}%
\pgfpathlineto{\pgfqpoint{1.595165in}{0.379500in}}%
\pgfpathlineto{\pgfqpoint{1.582282in}{0.372468in}}%
\pgfpathlineto{\pgfqpoint{1.579508in}{0.370646in}}%
\pgfpathlineto{\pgfqpoint{1.566589in}{0.358857in}}%
\pgfpathlineto{\pgfqpoint{1.563852in}{0.355203in}}%
\pgfpathlineto{\pgfqpoint{1.557501in}{0.345246in}}%
\pgfpathlineto{\pgfqpoint{1.554376in}{0.331635in}}%
\pgfpathlineto{\pgfqpoint{1.563852in}{0.331635in}}%
\pgfpathclose%
\pgfpathmoveto{\pgfqpoint{1.876983in}{0.331635in}}%
\pgfpathlineto{\pgfqpoint{1.892640in}{0.331635in}}%
\pgfpathlineto{\pgfqpoint{1.908296in}{0.331635in}}%
\pgfpathlineto{\pgfqpoint{1.923953in}{0.331635in}}%
\pgfpathlineto{\pgfqpoint{1.923953in}{0.345246in}}%
\pgfpathlineto{\pgfqpoint{1.923953in}{0.358857in}}%
\pgfpathlineto{\pgfqpoint{1.923953in}{0.372468in}}%
\pgfpathlineto{\pgfqpoint{1.923953in}{0.383489in}}%
\pgfpathlineto{\pgfqpoint{1.908296in}{0.380692in}}%
\pgfpathlineto{\pgfqpoint{1.892640in}{0.373002in}}%
\pgfpathlineto{\pgfqpoint{1.891911in}{0.372468in}}%
\pgfpathlineto{\pgfqpoint{1.876983in}{0.359490in}}%
\pgfpathlineto{\pgfqpoint{1.876369in}{0.358857in}}%
\pgfpathlineto{\pgfqpoint{1.867524in}{0.345246in}}%
\pgfpathlineto{\pgfqpoint{1.864306in}{0.331635in}}%
\pgfpathlineto{\pgfqpoint{1.876983in}{0.331635in}}%
\pgfpathclose%
\pgfpathmoveto{\pgfqpoint{0.375580in}{0.549413in}}%
\pgfpathlineto{\pgfqpoint{0.389609in}{0.552003in}}%
\pgfpathlineto{\pgfqpoint{0.405266in}{0.559962in}}%
\pgfpathlineto{\pgfqpoint{0.409357in}{0.563024in}}%
\pgfpathlineto{\pgfqpoint{0.420923in}{0.573635in}}%
\pgfpathlineto{\pgfqpoint{0.423670in}{0.576635in}}%
\pgfpathlineto{\pgfqpoint{0.431524in}{0.590246in}}%
\pgfpathlineto{\pgfqpoint{0.433468in}{0.603857in}}%
\pgfpathlineto{\pgfqpoint{0.429011in}{0.617468in}}%
\pgfpathlineto{\pgfqpoint{0.420923in}{0.628669in}}%
\pgfpathlineto{\pgfqpoint{0.418826in}{0.631079in}}%
\pgfpathlineto{\pgfqpoint{0.405266in}{0.642311in}}%
\pgfpathlineto{\pgfqpoint{0.401063in}{0.644691in}}%
\pgfpathlineto{\pgfqpoint{0.389609in}{0.650212in}}%
\pgfpathlineto{\pgfqpoint{0.373953in}{0.652928in}}%
\pgfpathlineto{\pgfqpoint{0.373953in}{0.644691in}}%
\pgfpathlineto{\pgfqpoint{0.373953in}{0.631079in}}%
\pgfpathlineto{\pgfqpoint{0.373953in}{0.617468in}}%
\pgfpathlineto{\pgfqpoint{0.373953in}{0.603857in}}%
\pgfpathlineto{\pgfqpoint{0.373953in}{0.590246in}}%
\pgfpathlineto{\pgfqpoint{0.373953in}{0.576635in}}%
\pgfpathlineto{\pgfqpoint{0.373953in}{0.563024in}}%
\pgfpathlineto{\pgfqpoint{0.373953in}{0.549413in}}%
\pgfpathlineto{\pgfqpoint{0.373953in}{0.549144in}}%
\pgfpathlineto{\pgfqpoint{0.375580in}{0.549413in}}%
\pgfpathclose%
\pgfpathmoveto{\pgfqpoint{0.687084in}{0.549248in}}%
\pgfpathlineto{\pgfqpoint{0.687798in}{0.549413in}}%
\pgfpathlineto{\pgfqpoint{0.702741in}{0.553237in}}%
\pgfpathlineto{\pgfqpoint{0.718397in}{0.562012in}}%
\pgfpathlineto{\pgfqpoint{0.719685in}{0.563024in}}%
\pgfpathlineto{\pgfqpoint{0.733849in}{0.576635in}}%
\pgfpathlineto{\pgfqpoint{0.734054in}{0.576944in}}%
\pgfpathlineto{\pgfqpoint{0.741514in}{0.590246in}}%
\pgfpathlineto{\pgfqpoint{0.743402in}{0.603857in}}%
\pgfpathlineto{\pgfqpoint{0.739073in}{0.617468in}}%
\pgfpathlineto{\pgfqpoint{0.734054in}{0.624530in}}%
\pgfpathlineto{\pgfqpoint{0.728612in}{0.631079in}}%
\pgfpathlineto{\pgfqpoint{0.718397in}{0.639959in}}%
\pgfpathlineto{\pgfqpoint{0.710864in}{0.644691in}}%
\pgfpathlineto{\pgfqpoint{0.702741in}{0.649054in}}%
\pgfpathlineto{\pgfqpoint{0.687084in}{0.652817in}}%
\pgfpathlineto{\pgfqpoint{0.671428in}{0.651176in}}%
\pgfpathlineto{\pgfqpoint{0.656127in}{0.644691in}}%
\pgfpathlineto{\pgfqpoint{0.655771in}{0.644513in}}%
\pgfpathlineto{\pgfqpoint{0.640115in}{0.632199in}}%
\pgfpathlineto{\pgfqpoint{0.638951in}{0.631079in}}%
\pgfpathlineto{\pgfqpoint{0.628857in}{0.617468in}}%
\pgfpathlineto{\pgfqpoint{0.624458in}{0.604478in}}%
\pgfpathlineto{\pgfqpoint{0.624269in}{0.603857in}}%
\pgfpathlineto{\pgfqpoint{0.624458in}{0.602419in}}%
\pgfpathlineto{\pgfqpoint{0.626256in}{0.590246in}}%
\pgfpathlineto{\pgfqpoint{0.634384in}{0.576635in}}%
\pgfpathlineto{\pgfqpoint{0.640115in}{0.570509in}}%
\pgfpathlineto{\pgfqpoint{0.648724in}{0.563024in}}%
\pgfpathlineto{\pgfqpoint{0.655771in}{0.558042in}}%
\pgfpathlineto{\pgfqpoint{0.671428in}{0.550976in}}%
\pgfpathlineto{\pgfqpoint{0.685429in}{0.549413in}}%
\pgfpathlineto{\pgfqpoint{0.687084in}{0.549248in}}%
\pgfpathclose%
\pgfpathmoveto{\pgfqpoint{1.610822in}{0.549248in}}%
\pgfpathlineto{\pgfqpoint{1.612476in}{0.549413in}}%
\pgfpathlineto{\pgfqpoint{1.626478in}{0.550976in}}%
\pgfpathlineto{\pgfqpoint{1.642135in}{0.558042in}}%
\pgfpathlineto{\pgfqpoint{1.649181in}{0.563024in}}%
\pgfpathlineto{\pgfqpoint{1.657791in}{0.570509in}}%
\pgfpathlineto{\pgfqpoint{1.663522in}{0.576635in}}%
\pgfpathlineto{\pgfqpoint{1.671650in}{0.590246in}}%
\pgfpathlineto{\pgfqpoint{1.673448in}{0.602419in}}%
\pgfpathlineto{\pgfqpoint{1.673637in}{0.603857in}}%
\pgfpathlineto{\pgfqpoint{1.673448in}{0.604478in}}%
\pgfpathlineto{\pgfqpoint{1.669049in}{0.617468in}}%
\pgfpathlineto{\pgfqpoint{1.658955in}{0.631079in}}%
\pgfpathlineto{\pgfqpoint{1.657791in}{0.632199in}}%
\pgfpathlineto{\pgfqpoint{1.642135in}{0.644513in}}%
\pgfpathlineto{\pgfqpoint{1.641779in}{0.644691in}}%
\pgfpathlineto{\pgfqpoint{1.626478in}{0.651176in}}%
\pgfpathlineto{\pgfqpoint{1.610822in}{0.652817in}}%
\pgfpathlineto{\pgfqpoint{1.595165in}{0.649054in}}%
\pgfpathlineto{\pgfqpoint{1.587042in}{0.644691in}}%
\pgfpathlineto{\pgfqpoint{1.579508in}{0.639959in}}%
\pgfpathlineto{\pgfqpoint{1.569294in}{0.631079in}}%
\pgfpathlineto{\pgfqpoint{1.563852in}{0.624530in}}%
\pgfpathlineto{\pgfqpoint{1.558832in}{0.617468in}}%
\pgfpathlineto{\pgfqpoint{1.554504in}{0.603857in}}%
\pgfpathlineto{\pgfqpoint{1.556392in}{0.590246in}}%
\pgfpathlineto{\pgfqpoint{1.563852in}{0.576944in}}%
\pgfpathlineto{\pgfqpoint{1.564056in}{0.576635in}}%
\pgfpathlineto{\pgfqpoint{1.578221in}{0.563024in}}%
\pgfpathlineto{\pgfqpoint{1.579508in}{0.562012in}}%
\pgfpathlineto{\pgfqpoint{1.595165in}{0.553237in}}%
\pgfpathlineto{\pgfqpoint{1.610108in}{0.549413in}}%
\pgfpathlineto{\pgfqpoint{1.610822in}{0.549248in}}%
\pgfpathclose%
\pgfpathmoveto{\pgfqpoint{1.923953in}{0.549144in}}%
\pgfpathlineto{\pgfqpoint{1.923953in}{0.549413in}}%
\pgfpathlineto{\pgfqpoint{1.923953in}{0.563024in}}%
\pgfpathlineto{\pgfqpoint{1.923953in}{0.576635in}}%
\pgfpathlineto{\pgfqpoint{1.923953in}{0.590246in}}%
\pgfpathlineto{\pgfqpoint{1.923953in}{0.603857in}}%
\pgfpathlineto{\pgfqpoint{1.923953in}{0.617468in}}%
\pgfpathlineto{\pgfqpoint{1.923953in}{0.631079in}}%
\pgfpathlineto{\pgfqpoint{1.923953in}{0.644691in}}%
\pgfpathlineto{\pgfqpoint{1.923953in}{0.652928in}}%
\pgfpathlineto{\pgfqpoint{1.908296in}{0.650212in}}%
\pgfpathlineto{\pgfqpoint{1.896843in}{0.644691in}}%
\pgfpathlineto{\pgfqpoint{1.892640in}{0.642311in}}%
\pgfpathlineto{\pgfqpoint{1.879080in}{0.631079in}}%
\pgfpathlineto{\pgfqpoint{1.876983in}{0.628669in}}%
\pgfpathlineto{\pgfqpoint{1.868895in}{0.617468in}}%
\pgfpathlineto{\pgfqpoint{1.864438in}{0.603857in}}%
\pgfpathlineto{\pgfqpoint{1.866382in}{0.590246in}}%
\pgfpathlineto{\pgfqpoint{1.874236in}{0.576635in}}%
\pgfpathlineto{\pgfqpoint{1.876983in}{0.573635in}}%
\pgfpathlineto{\pgfqpoint{1.888548in}{0.563024in}}%
\pgfpathlineto{\pgfqpoint{1.892640in}{0.559962in}}%
\pgfpathlineto{\pgfqpoint{1.908296in}{0.552003in}}%
\pgfpathlineto{\pgfqpoint{1.922326in}{0.549413in}}%
\pgfpathlineto{\pgfqpoint{1.923953in}{0.549144in}}%
\pgfpathclose%
\pgfpathmoveto{\pgfqpoint{0.968902in}{0.556271in}}%
\pgfpathlineto{\pgfqpoint{0.984559in}{0.550165in}}%
\pgfpathlineto{\pgfqpoint{1.000216in}{0.549580in}}%
\pgfpathlineto{\pgfqpoint{1.015872in}{0.554664in}}%
\pgfpathlineto{\pgfqpoint{1.029548in}{0.563024in}}%
\pgfpathlineto{\pgfqpoint{1.031529in}{0.564524in}}%
\pgfpathlineto{\pgfqpoint{1.043568in}{0.576635in}}%
\pgfpathlineto{\pgfqpoint{1.047185in}{0.582314in}}%
\pgfpathlineto{\pgfqpoint{1.051585in}{0.590246in}}%
\pgfpathlineto{\pgfqpoint{1.053428in}{0.603857in}}%
\pgfpathlineto{\pgfqpoint{1.049204in}{0.617468in}}%
\pgfpathlineto{\pgfqpoint{1.047185in}{0.620340in}}%
\pgfpathlineto{\pgfqpoint{1.038595in}{0.631079in}}%
\pgfpathlineto{\pgfqpoint{1.031529in}{0.637477in}}%
\pgfpathlineto{\pgfqpoint{1.021008in}{0.644691in}}%
\pgfpathlineto{\pgfqpoint{1.015872in}{0.647715in}}%
\pgfpathlineto{\pgfqpoint{1.000216in}{0.652485in}}%
\pgfpathlineto{\pgfqpoint{0.984559in}{0.651936in}}%
\pgfpathlineto{\pgfqpoint{0.968902in}{0.646207in}}%
\pgfpathlineto{\pgfqpoint{0.966521in}{0.644691in}}%
\pgfpathlineto{\pgfqpoint{0.953246in}{0.634883in}}%
\pgfpathlineto{\pgfqpoint{0.949183in}{0.631079in}}%
\pgfpathlineto{\pgfqpoint{0.938680in}{0.617468in}}%
\pgfpathlineto{\pgfqpoint{0.937589in}{0.614328in}}%
\pgfpathlineto{\pgfqpoint{0.934390in}{0.603857in}}%
\pgfpathlineto{\pgfqpoint{0.936196in}{0.590246in}}%
\pgfpathlineto{\pgfqpoint{0.937589in}{0.587720in}}%
\pgfpathlineto{\pgfqpoint{0.944431in}{0.576635in}}%
\pgfpathlineto{\pgfqpoint{0.953246in}{0.567465in}}%
\pgfpathlineto{\pgfqpoint{0.958688in}{0.563024in}}%
\pgfpathlineto{\pgfqpoint{0.968902in}{0.556271in}}%
\pgfpathclose%
\pgfpathmoveto{\pgfqpoint{1.282034in}{0.554664in}}%
\pgfpathlineto{\pgfqpoint{1.297690in}{0.549580in}}%
\pgfpathlineto{\pgfqpoint{1.313347in}{0.550165in}}%
\pgfpathlineto{\pgfqpoint{1.329003in}{0.556271in}}%
\pgfpathlineto{\pgfqpoint{1.339217in}{0.563024in}}%
\pgfpathlineto{\pgfqpoint{1.344660in}{0.567465in}}%
\pgfpathlineto{\pgfqpoint{1.353474in}{0.576635in}}%
\pgfpathlineto{\pgfqpoint{1.360317in}{0.587720in}}%
\pgfpathlineto{\pgfqpoint{1.361710in}{0.590246in}}%
\pgfpathlineto{\pgfqpoint{1.363515in}{0.603857in}}%
\pgfpathlineto{\pgfqpoint{1.360317in}{0.614328in}}%
\pgfpathlineto{\pgfqpoint{1.359226in}{0.617468in}}%
\pgfpathlineto{\pgfqpoint{1.348723in}{0.631079in}}%
\pgfpathlineto{\pgfqpoint{1.344660in}{0.634883in}}%
\pgfpathlineto{\pgfqpoint{1.331384in}{0.644691in}}%
\pgfpathlineto{\pgfqpoint{1.329003in}{0.646207in}}%
\pgfpathlineto{\pgfqpoint{1.313347in}{0.651936in}}%
\pgfpathlineto{\pgfqpoint{1.297690in}{0.652485in}}%
\pgfpathlineto{\pgfqpoint{1.282034in}{0.647715in}}%
\pgfpathlineto{\pgfqpoint{1.276898in}{0.644691in}}%
\pgfpathlineto{\pgfqpoint{1.266377in}{0.637477in}}%
\pgfpathlineto{\pgfqpoint{1.259311in}{0.631079in}}%
\pgfpathlineto{\pgfqpoint{1.250721in}{0.620340in}}%
\pgfpathlineto{\pgfqpoint{1.248701in}{0.617468in}}%
\pgfpathlineto{\pgfqpoint{1.244478in}{0.603857in}}%
\pgfpathlineto{\pgfqpoint{1.246320in}{0.590246in}}%
\pgfpathlineto{\pgfqpoint{1.250721in}{0.582314in}}%
\pgfpathlineto{\pgfqpoint{1.254338in}{0.576635in}}%
\pgfpathlineto{\pgfqpoint{1.266377in}{0.564524in}}%
\pgfpathlineto{\pgfqpoint{1.268358in}{0.563024in}}%
\pgfpathlineto{\pgfqpoint{1.282034in}{0.554664in}}%
\pgfpathclose%
\pgfpathmoveto{\pgfqpoint{0.389609in}{0.821346in}}%
\pgfpathlineto{\pgfqpoint{0.390220in}{0.821635in}}%
\pgfpathlineto{\pgfqpoint{0.405266in}{0.829580in}}%
\pgfpathlineto{\pgfqpoint{0.412633in}{0.835246in}}%
\pgfpathlineto{\pgfqpoint{0.420923in}{0.843350in}}%
\pgfpathlineto{\pgfqpoint{0.425639in}{0.848857in}}%
\pgfpathlineto{\pgfqpoint{0.432425in}{0.862468in}}%
\pgfpathlineto{\pgfqpoint{0.433075in}{0.876079in}}%
\pgfpathlineto{\pgfqpoint{0.427425in}{0.889691in}}%
\pgfpathlineto{\pgfqpoint{0.420923in}{0.897903in}}%
\pgfpathlineto{\pgfqpoint{0.415797in}{0.903302in}}%
\pgfpathlineto{\pgfqpoint{0.405266in}{0.911678in}}%
\pgfpathlineto{\pgfqpoint{0.395659in}{0.916913in}}%
\pgfpathlineto{\pgfqpoint{0.389609in}{0.919798in}}%
\pgfpathlineto{\pgfqpoint{0.373953in}{0.922448in}}%
\pgfpathlineto{\pgfqpoint{0.373953in}{0.916913in}}%
\pgfpathlineto{\pgfqpoint{0.373953in}{0.903302in}}%
\pgfpathlineto{\pgfqpoint{0.373953in}{0.889691in}}%
\pgfpathlineto{\pgfqpoint{0.373953in}{0.876079in}}%
\pgfpathlineto{\pgfqpoint{0.373953in}{0.862468in}}%
\pgfpathlineto{\pgfqpoint{0.373953in}{0.848857in}}%
\pgfpathlineto{\pgfqpoint{0.373953in}{0.835246in}}%
\pgfpathlineto{\pgfqpoint{0.373953in}{0.821635in}}%
\pgfpathlineto{\pgfqpoint{0.373953in}{0.818748in}}%
\pgfpathlineto{\pgfqpoint{0.389609in}{0.821346in}}%
\pgfpathclose%
\pgfpathmoveto{\pgfqpoint{0.671428in}{0.820424in}}%
\pgfpathlineto{\pgfqpoint{0.687084in}{0.818854in}}%
\pgfpathlineto{\pgfqpoint{0.699129in}{0.821635in}}%
\pgfpathlineto{\pgfqpoint{0.702741in}{0.822583in}}%
\pgfpathlineto{\pgfqpoint{0.718397in}{0.831714in}}%
\pgfpathlineto{\pgfqpoint{0.722773in}{0.835246in}}%
\pgfpathlineto{\pgfqpoint{0.734054in}{0.846787in}}%
\pgfpathlineto{\pgfqpoint{0.735798in}{0.848857in}}%
\pgfpathlineto{\pgfqpoint{0.742389in}{0.862468in}}%
\pgfpathlineto{\pgfqpoint{0.743020in}{0.876079in}}%
\pgfpathlineto{\pgfqpoint{0.737533in}{0.889691in}}%
\pgfpathlineto{\pgfqpoint{0.734054in}{0.894155in}}%
\pgfpathlineto{\pgfqpoint{0.725756in}{0.903302in}}%
\pgfpathlineto{\pgfqpoint{0.718397in}{0.909445in}}%
\pgfpathlineto{\pgfqpoint{0.706044in}{0.916913in}}%
\pgfpathlineto{\pgfqpoint{0.702741in}{0.918668in}}%
\pgfpathlineto{\pgfqpoint{0.687084in}{0.922340in}}%
\pgfpathlineto{\pgfqpoint{0.671428in}{0.920738in}}%
\pgfpathlineto{\pgfqpoint{0.662304in}{0.916913in}}%
\pgfpathlineto{\pgfqpoint{0.655771in}{0.913768in}}%
\pgfpathlineto{\pgfqpoint{0.641840in}{0.903302in}}%
\pgfpathlineto{\pgfqpoint{0.640115in}{0.901580in}}%
\pgfpathlineto{\pgfqpoint{0.630498in}{0.889691in}}%
\pgfpathlineto{\pgfqpoint{0.624650in}{0.876079in}}%
\pgfpathlineto{\pgfqpoint{0.625324in}{0.862468in}}%
\pgfpathlineto{\pgfqpoint{0.632347in}{0.848857in}}%
\pgfpathlineto{\pgfqpoint{0.640115in}{0.839978in}}%
\pgfpathlineto{\pgfqpoint{0.645223in}{0.835246in}}%
\pgfpathlineto{\pgfqpoint{0.655771in}{0.827583in}}%
\pgfpathlineto{\pgfqpoint{0.668522in}{0.821635in}}%
\pgfpathlineto{\pgfqpoint{0.671428in}{0.820424in}}%
\pgfpathclose%
\pgfpathmoveto{\pgfqpoint{0.984559in}{0.819697in}}%
\pgfpathlineto{\pgfqpoint{1.000216in}{0.819171in}}%
\pgfpathlineto{\pgfqpoint{1.008573in}{0.821635in}}%
\pgfpathlineto{\pgfqpoint{1.015872in}{0.824068in}}%
\pgfpathlineto{\pgfqpoint{1.031529in}{0.833966in}}%
\pgfpathlineto{\pgfqpoint{1.033051in}{0.835246in}}%
\pgfpathlineto{\pgfqpoint{1.045786in}{0.848857in}}%
\pgfpathlineto{\pgfqpoint{1.047185in}{0.851409in}}%
\pgfpathlineto{\pgfqpoint{1.052439in}{0.862468in}}%
\pgfpathlineto{\pgfqpoint{1.053055in}{0.876079in}}%
\pgfpathlineto{\pgfqpoint{1.047701in}{0.889691in}}%
\pgfpathlineto{\pgfqpoint{1.047185in}{0.890360in}}%
\pgfpathlineto{\pgfqpoint{1.035883in}{0.903302in}}%
\pgfpathlineto{\pgfqpoint{1.031529in}{0.907087in}}%
\pgfpathlineto{\pgfqpoint{1.016642in}{0.916913in}}%
\pgfpathlineto{\pgfqpoint{1.015872in}{0.917361in}}%
\pgfpathlineto{\pgfqpoint{1.000216in}{0.922016in}}%
\pgfpathlineto{\pgfqpoint{0.984559in}{0.921480in}}%
\pgfpathlineto{\pgfqpoint{0.971838in}{0.916913in}}%
\pgfpathlineto{\pgfqpoint{0.968902in}{0.915697in}}%
\pgfpathlineto{\pgfqpoint{0.953246in}{0.904625in}}%
\pgfpathlineto{\pgfqpoint{0.951774in}{0.903302in}}%
\pgfpathlineto{\pgfqpoint{0.940388in}{0.889691in}}%
\pgfpathlineto{\pgfqpoint{0.937589in}{0.883345in}}%
\pgfpathlineto{\pgfqpoint{0.934755in}{0.876079in}}%
\pgfpathlineto{\pgfqpoint{0.935360in}{0.862468in}}%
\pgfpathlineto{\pgfqpoint{0.937589in}{0.857746in}}%
\pgfpathlineto{\pgfqpoint{0.942311in}{0.848857in}}%
\pgfpathlineto{\pgfqpoint{0.953246in}{0.836693in}}%
\pgfpathlineto{\pgfqpoint{0.954911in}{0.835246in}}%
\pgfpathlineto{\pgfqpoint{0.968902in}{0.825740in}}%
\pgfpathlineto{\pgfqpoint{0.979128in}{0.821635in}}%
\pgfpathlineto{\pgfqpoint{0.984559in}{0.819697in}}%
\pgfpathclose%
\pgfpathmoveto{\pgfqpoint{1.297690in}{0.819171in}}%
\pgfpathlineto{\pgfqpoint{1.313347in}{0.819697in}}%
\pgfpathlineto{\pgfqpoint{1.318778in}{0.821635in}}%
\pgfpathlineto{\pgfqpoint{1.329003in}{0.825740in}}%
\pgfpathlineto{\pgfqpoint{1.342995in}{0.835246in}}%
\pgfpathlineto{\pgfqpoint{1.344660in}{0.836693in}}%
\pgfpathlineto{\pgfqpoint{1.355594in}{0.848857in}}%
\pgfpathlineto{\pgfqpoint{1.360317in}{0.857746in}}%
\pgfpathlineto{\pgfqpoint{1.362546in}{0.862468in}}%
\pgfpathlineto{\pgfqpoint{1.363150in}{0.876079in}}%
\pgfpathlineto{\pgfqpoint{1.360317in}{0.883345in}}%
\pgfpathlineto{\pgfqpoint{1.357518in}{0.889691in}}%
\pgfpathlineto{\pgfqpoint{1.346132in}{0.903302in}}%
\pgfpathlineto{\pgfqpoint{1.344660in}{0.904625in}}%
\pgfpathlineto{\pgfqpoint{1.329003in}{0.915697in}}%
\pgfpathlineto{\pgfqpoint{1.326068in}{0.916913in}}%
\pgfpathlineto{\pgfqpoint{1.313347in}{0.921480in}}%
\pgfpathlineto{\pgfqpoint{1.297690in}{0.922016in}}%
\pgfpathlineto{\pgfqpoint{1.282034in}{0.917361in}}%
\pgfpathlineto{\pgfqpoint{1.281264in}{0.916913in}}%
\pgfpathlineto{\pgfqpoint{1.266377in}{0.907087in}}%
\pgfpathlineto{\pgfqpoint{1.262022in}{0.903302in}}%
\pgfpathlineto{\pgfqpoint{1.250721in}{0.890360in}}%
\pgfpathlineto{\pgfqpoint{1.250204in}{0.889691in}}%
\pgfpathlineto{\pgfqpoint{1.244850in}{0.876079in}}%
\pgfpathlineto{\pgfqpoint{1.245467in}{0.862468in}}%
\pgfpathlineto{\pgfqpoint{1.250721in}{0.851409in}}%
\pgfpathlineto{\pgfqpoint{1.252119in}{0.848857in}}%
\pgfpathlineto{\pgfqpoint{1.264855in}{0.835246in}}%
\pgfpathlineto{\pgfqpoint{1.266377in}{0.833966in}}%
\pgfpathlineto{\pgfqpoint{1.282034in}{0.824068in}}%
\pgfpathlineto{\pgfqpoint{1.289333in}{0.821635in}}%
\pgfpathlineto{\pgfqpoint{1.297690in}{0.819171in}}%
\pgfpathclose%
\pgfpathmoveto{\pgfqpoint{1.610822in}{0.818854in}}%
\pgfpathlineto{\pgfqpoint{1.626478in}{0.820424in}}%
\pgfpathlineto{\pgfqpoint{1.629384in}{0.821635in}}%
\pgfpathlineto{\pgfqpoint{1.642135in}{0.827583in}}%
\pgfpathlineto{\pgfqpoint{1.652683in}{0.835246in}}%
\pgfpathlineto{\pgfqpoint{1.657791in}{0.839978in}}%
\pgfpathlineto{\pgfqpoint{1.665559in}{0.848857in}}%
\pgfpathlineto{\pgfqpoint{1.672582in}{0.862468in}}%
\pgfpathlineto{\pgfqpoint{1.673255in}{0.876079in}}%
\pgfpathlineto{\pgfqpoint{1.667407in}{0.889691in}}%
\pgfpathlineto{\pgfqpoint{1.657791in}{0.901580in}}%
\pgfpathlineto{\pgfqpoint{1.656066in}{0.903302in}}%
\pgfpathlineto{\pgfqpoint{1.642135in}{0.913768in}}%
\pgfpathlineto{\pgfqpoint{1.635602in}{0.916913in}}%
\pgfpathlineto{\pgfqpoint{1.626478in}{0.920738in}}%
\pgfpathlineto{\pgfqpoint{1.610822in}{0.922340in}}%
\pgfpathlineto{\pgfqpoint{1.595165in}{0.918668in}}%
\pgfpathlineto{\pgfqpoint{1.591862in}{0.916913in}}%
\pgfpathlineto{\pgfqpoint{1.579508in}{0.909445in}}%
\pgfpathlineto{\pgfqpoint{1.572150in}{0.903302in}}%
\pgfpathlineto{\pgfqpoint{1.563852in}{0.894155in}}%
\pgfpathlineto{\pgfqpoint{1.560373in}{0.889691in}}%
\pgfpathlineto{\pgfqpoint{1.554886in}{0.876079in}}%
\pgfpathlineto{\pgfqpoint{1.555517in}{0.862468in}}%
\pgfpathlineto{\pgfqpoint{1.562107in}{0.848857in}}%
\pgfpathlineto{\pgfqpoint{1.563852in}{0.846787in}}%
\pgfpathlineto{\pgfqpoint{1.575133in}{0.835246in}}%
\pgfpathlineto{\pgfqpoint{1.579508in}{0.831714in}}%
\pgfpathlineto{\pgfqpoint{1.595165in}{0.822583in}}%
\pgfpathlineto{\pgfqpoint{1.598777in}{0.821635in}}%
\pgfpathlineto{\pgfqpoint{1.610822in}{0.818854in}}%
\pgfpathclose%
\pgfpathmoveto{\pgfqpoint{1.908296in}{0.821346in}}%
\pgfpathlineto{\pgfqpoint{1.923953in}{0.818748in}}%
\pgfpathlineto{\pgfqpoint{1.923953in}{0.821635in}}%
\pgfpathlineto{\pgfqpoint{1.923953in}{0.835246in}}%
\pgfpathlineto{\pgfqpoint{1.923953in}{0.848857in}}%
\pgfpathlineto{\pgfqpoint{1.923953in}{0.862468in}}%
\pgfpathlineto{\pgfqpoint{1.923953in}{0.876079in}}%
\pgfpathlineto{\pgfqpoint{1.923953in}{0.889691in}}%
\pgfpathlineto{\pgfqpoint{1.923953in}{0.903302in}}%
\pgfpathlineto{\pgfqpoint{1.923953in}{0.916913in}}%
\pgfpathlineto{\pgfqpoint{1.923953in}{0.922448in}}%
\pgfpathlineto{\pgfqpoint{1.908296in}{0.919798in}}%
\pgfpathlineto{\pgfqpoint{1.902247in}{0.916913in}}%
\pgfpathlineto{\pgfqpoint{1.892640in}{0.911678in}}%
\pgfpathlineto{\pgfqpoint{1.882109in}{0.903302in}}%
\pgfpathlineto{\pgfqpoint{1.876983in}{0.897903in}}%
\pgfpathlineto{\pgfqpoint{1.870481in}{0.889691in}}%
\pgfpathlineto{\pgfqpoint{1.864831in}{0.876079in}}%
\pgfpathlineto{\pgfqpoint{1.865481in}{0.862468in}}%
\pgfpathlineto{\pgfqpoint{1.872267in}{0.848857in}}%
\pgfpathlineto{\pgfqpoint{1.876983in}{0.843350in}}%
\pgfpathlineto{\pgfqpoint{1.885273in}{0.835246in}}%
\pgfpathlineto{\pgfqpoint{1.892640in}{0.829580in}}%
\pgfpathlineto{\pgfqpoint{1.907686in}{0.821635in}}%
\pgfpathlineto{\pgfqpoint{1.908296in}{0.821346in}}%
\pgfpathclose%
\pgfpathmoveto{\pgfqpoint{0.389609in}{1.090972in}}%
\pgfpathlineto{\pgfqpoint{0.395659in}{1.093857in}}%
\pgfpathlineto{\pgfqpoint{0.405266in}{1.099092in}}%
\pgfpathlineto{\pgfqpoint{0.415797in}{1.107468in}}%
\pgfpathlineto{\pgfqpoint{0.420923in}{1.112867in}}%
\pgfpathlineto{\pgfqpoint{0.427425in}{1.121079in}}%
\pgfpathlineto{\pgfqpoint{0.433075in}{1.134691in}}%
\pgfpathlineto{\pgfqpoint{0.432425in}{1.148302in}}%
\pgfpathlineto{\pgfqpoint{0.425639in}{1.161913in}}%
\pgfpathlineto{\pgfqpoint{0.420923in}{1.167420in}}%
\pgfpathlineto{\pgfqpoint{0.412633in}{1.175524in}}%
\pgfpathlineto{\pgfqpoint{0.405266in}{1.181190in}}%
\pgfpathlineto{\pgfqpoint{0.390220in}{1.189135in}}%
\pgfpathlineto{\pgfqpoint{0.389609in}{1.189424in}}%
\pgfpathlineto{\pgfqpoint{0.373953in}{1.192022in}}%
\pgfpathlineto{\pgfqpoint{0.373953in}{1.189135in}}%
\pgfpathlineto{\pgfqpoint{0.373953in}{1.175524in}}%
\pgfpathlineto{\pgfqpoint{0.373953in}{1.161913in}}%
\pgfpathlineto{\pgfqpoint{0.373953in}{1.148302in}}%
\pgfpathlineto{\pgfqpoint{0.373953in}{1.134691in}}%
\pgfpathlineto{\pgfqpoint{0.373953in}{1.121079in}}%
\pgfpathlineto{\pgfqpoint{0.373953in}{1.107468in}}%
\pgfpathlineto{\pgfqpoint{0.373953in}{1.093857in}}%
\pgfpathlineto{\pgfqpoint{0.373953in}{1.088322in}}%
\pgfpathlineto{\pgfqpoint{0.389609in}{1.090972in}}%
\pgfpathclose%
\pgfpathmoveto{\pgfqpoint{0.671428in}{1.090032in}}%
\pgfpathlineto{\pgfqpoint{0.687084in}{1.088430in}}%
\pgfpathlineto{\pgfqpoint{0.702741in}{1.092102in}}%
\pgfpathlineto{\pgfqpoint{0.706044in}{1.093857in}}%
\pgfpathlineto{\pgfqpoint{0.718397in}{1.101325in}}%
\pgfpathlineto{\pgfqpoint{0.725756in}{1.107468in}}%
\pgfpathlineto{\pgfqpoint{0.734054in}{1.116615in}}%
\pgfpathlineto{\pgfqpoint{0.737533in}{1.121079in}}%
\pgfpathlineto{\pgfqpoint{0.743020in}{1.134691in}}%
\pgfpathlineto{\pgfqpoint{0.742389in}{1.148302in}}%
\pgfpathlineto{\pgfqpoint{0.735798in}{1.161913in}}%
\pgfpathlineto{\pgfqpoint{0.734054in}{1.163983in}}%
\pgfpathlineto{\pgfqpoint{0.722773in}{1.175524in}}%
\pgfpathlineto{\pgfqpoint{0.718397in}{1.179056in}}%
\pgfpathlineto{\pgfqpoint{0.702741in}{1.188187in}}%
\pgfpathlineto{\pgfqpoint{0.699129in}{1.189135in}}%
\pgfpathlineto{\pgfqpoint{0.687084in}{1.191916in}}%
\pgfpathlineto{\pgfqpoint{0.671428in}{1.190346in}}%
\pgfpathlineto{\pgfqpoint{0.668522in}{1.189135in}}%
\pgfpathlineto{\pgfqpoint{0.655771in}{1.183187in}}%
\pgfpathlineto{\pgfqpoint{0.645223in}{1.175524in}}%
\pgfpathlineto{\pgfqpoint{0.640115in}{1.170792in}}%
\pgfpathlineto{\pgfqpoint{0.632347in}{1.161913in}}%
\pgfpathlineto{\pgfqpoint{0.625324in}{1.148302in}}%
\pgfpathlineto{\pgfqpoint{0.624650in}{1.134691in}}%
\pgfpathlineto{\pgfqpoint{0.630498in}{1.121079in}}%
\pgfpathlineto{\pgfqpoint{0.640115in}{1.109190in}}%
\pgfpathlineto{\pgfqpoint{0.641840in}{1.107468in}}%
\pgfpathlineto{\pgfqpoint{0.655771in}{1.097002in}}%
\pgfpathlineto{\pgfqpoint{0.662304in}{1.093857in}}%
\pgfpathlineto{\pgfqpoint{0.671428in}{1.090032in}}%
\pgfpathclose%
\pgfpathmoveto{\pgfqpoint{0.984559in}{1.089290in}}%
\pgfpathlineto{\pgfqpoint{1.000216in}{1.088754in}}%
\pgfpathlineto{\pgfqpoint{1.015872in}{1.093409in}}%
\pgfpathlineto{\pgfqpoint{1.016642in}{1.093857in}}%
\pgfpathlineto{\pgfqpoint{1.031529in}{1.103683in}}%
\pgfpathlineto{\pgfqpoint{1.035883in}{1.107468in}}%
\pgfpathlineto{\pgfqpoint{1.047185in}{1.120410in}}%
\pgfpathlineto{\pgfqpoint{1.047701in}{1.121079in}}%
\pgfpathlineto{\pgfqpoint{1.053055in}{1.134691in}}%
\pgfpathlineto{\pgfqpoint{1.052439in}{1.148302in}}%
\pgfpathlineto{\pgfqpoint{1.047185in}{1.159361in}}%
\pgfpathlineto{\pgfqpoint{1.045786in}{1.161913in}}%
\pgfpathlineto{\pgfqpoint{1.033051in}{1.175524in}}%
\pgfpathlineto{\pgfqpoint{1.031529in}{1.176804in}}%
\pgfpathlineto{\pgfqpoint{1.015872in}{1.186702in}}%
\pgfpathlineto{\pgfqpoint{1.008573in}{1.189135in}}%
\pgfpathlineto{\pgfqpoint{1.000216in}{1.191599in}}%
\pgfpathlineto{\pgfqpoint{0.984559in}{1.191073in}}%
\pgfpathlineto{\pgfqpoint{0.979128in}{1.189135in}}%
\pgfpathlineto{\pgfqpoint{0.968902in}{1.185030in}}%
\pgfpathlineto{\pgfqpoint{0.954911in}{1.175524in}}%
\pgfpathlineto{\pgfqpoint{0.953246in}{1.174077in}}%
\pgfpathlineto{\pgfqpoint{0.942311in}{1.161913in}}%
\pgfpathlineto{\pgfqpoint{0.937589in}{1.153024in}}%
\pgfpathlineto{\pgfqpoint{0.935360in}{1.148302in}}%
\pgfpathlineto{\pgfqpoint{0.934755in}{1.134691in}}%
\pgfpathlineto{\pgfqpoint{0.937589in}{1.127425in}}%
\pgfpathlineto{\pgfqpoint{0.940388in}{1.121079in}}%
\pgfpathlineto{\pgfqpoint{0.951774in}{1.107468in}}%
\pgfpathlineto{\pgfqpoint{0.953246in}{1.106145in}}%
\pgfpathlineto{\pgfqpoint{0.968902in}{1.095073in}}%
\pgfpathlineto{\pgfqpoint{0.971838in}{1.093857in}}%
\pgfpathlineto{\pgfqpoint{0.984559in}{1.089290in}}%
\pgfpathclose%
\pgfpathmoveto{\pgfqpoint{1.282034in}{1.093409in}}%
\pgfpathlineto{\pgfqpoint{1.297690in}{1.088754in}}%
\pgfpathlineto{\pgfqpoint{1.313347in}{1.089290in}}%
\pgfpathlineto{\pgfqpoint{1.326068in}{1.093857in}}%
\pgfpathlineto{\pgfqpoint{1.329003in}{1.095073in}}%
\pgfpathlineto{\pgfqpoint{1.344660in}{1.106145in}}%
\pgfpathlineto{\pgfqpoint{1.346132in}{1.107468in}}%
\pgfpathlineto{\pgfqpoint{1.357518in}{1.121079in}}%
\pgfpathlineto{\pgfqpoint{1.360317in}{1.127425in}}%
\pgfpathlineto{\pgfqpoint{1.363150in}{1.134691in}}%
\pgfpathlineto{\pgfqpoint{1.362546in}{1.148302in}}%
\pgfpathlineto{\pgfqpoint{1.360317in}{1.153024in}}%
\pgfpathlineto{\pgfqpoint{1.355594in}{1.161913in}}%
\pgfpathlineto{\pgfqpoint{1.344660in}{1.174077in}}%
\pgfpathlineto{\pgfqpoint{1.342995in}{1.175524in}}%
\pgfpathlineto{\pgfqpoint{1.329003in}{1.185030in}}%
\pgfpathlineto{\pgfqpoint{1.318778in}{1.189135in}}%
\pgfpathlineto{\pgfqpoint{1.313347in}{1.191073in}}%
\pgfpathlineto{\pgfqpoint{1.297690in}{1.191599in}}%
\pgfpathlineto{\pgfqpoint{1.289333in}{1.189135in}}%
\pgfpathlineto{\pgfqpoint{1.282034in}{1.186702in}}%
\pgfpathlineto{\pgfqpoint{1.266377in}{1.176804in}}%
\pgfpathlineto{\pgfqpoint{1.264855in}{1.175524in}}%
\pgfpathlineto{\pgfqpoint{1.252119in}{1.161913in}}%
\pgfpathlineto{\pgfqpoint{1.250721in}{1.159361in}}%
\pgfpathlineto{\pgfqpoint{1.245467in}{1.148302in}}%
\pgfpathlineto{\pgfqpoint{1.244850in}{1.134691in}}%
\pgfpathlineto{\pgfqpoint{1.250204in}{1.121079in}}%
\pgfpathlineto{\pgfqpoint{1.250721in}{1.120410in}}%
\pgfpathlineto{\pgfqpoint{1.262022in}{1.107468in}}%
\pgfpathlineto{\pgfqpoint{1.266377in}{1.103683in}}%
\pgfpathlineto{\pgfqpoint{1.281264in}{1.093857in}}%
\pgfpathlineto{\pgfqpoint{1.282034in}{1.093409in}}%
\pgfpathclose%
\pgfpathmoveto{\pgfqpoint{1.595165in}{1.092102in}}%
\pgfpathlineto{\pgfqpoint{1.610822in}{1.088430in}}%
\pgfpathlineto{\pgfqpoint{1.626478in}{1.090032in}}%
\pgfpathlineto{\pgfqpoint{1.635602in}{1.093857in}}%
\pgfpathlineto{\pgfqpoint{1.642135in}{1.097002in}}%
\pgfpathlineto{\pgfqpoint{1.656066in}{1.107468in}}%
\pgfpathlineto{\pgfqpoint{1.657791in}{1.109190in}}%
\pgfpathlineto{\pgfqpoint{1.667407in}{1.121079in}}%
\pgfpathlineto{\pgfqpoint{1.673255in}{1.134691in}}%
\pgfpathlineto{\pgfqpoint{1.672582in}{1.148302in}}%
\pgfpathlineto{\pgfqpoint{1.665559in}{1.161913in}}%
\pgfpathlineto{\pgfqpoint{1.657791in}{1.170792in}}%
\pgfpathlineto{\pgfqpoint{1.652683in}{1.175524in}}%
\pgfpathlineto{\pgfqpoint{1.642135in}{1.183187in}}%
\pgfpathlineto{\pgfqpoint{1.629384in}{1.189135in}}%
\pgfpathlineto{\pgfqpoint{1.626478in}{1.190346in}}%
\pgfpathlineto{\pgfqpoint{1.610822in}{1.191916in}}%
\pgfpathlineto{\pgfqpoint{1.598777in}{1.189135in}}%
\pgfpathlineto{\pgfqpoint{1.595165in}{1.188187in}}%
\pgfpathlineto{\pgfqpoint{1.579508in}{1.179056in}}%
\pgfpathlineto{\pgfqpoint{1.575133in}{1.175524in}}%
\pgfpathlineto{\pgfqpoint{1.563852in}{1.163983in}}%
\pgfpathlineto{\pgfqpoint{1.562107in}{1.161913in}}%
\pgfpathlineto{\pgfqpoint{1.555517in}{1.148302in}}%
\pgfpathlineto{\pgfqpoint{1.554886in}{1.134691in}}%
\pgfpathlineto{\pgfqpoint{1.560373in}{1.121079in}}%
\pgfpathlineto{\pgfqpoint{1.563852in}{1.116615in}}%
\pgfpathlineto{\pgfqpoint{1.572150in}{1.107468in}}%
\pgfpathlineto{\pgfqpoint{1.579508in}{1.101325in}}%
\pgfpathlineto{\pgfqpoint{1.591862in}{1.093857in}}%
\pgfpathlineto{\pgfqpoint{1.595165in}{1.092102in}}%
\pgfpathclose%
\pgfpathmoveto{\pgfqpoint{1.908296in}{1.090972in}}%
\pgfpathlineto{\pgfqpoint{1.923953in}{1.088322in}}%
\pgfpathlineto{\pgfqpoint{1.923953in}{1.093857in}}%
\pgfpathlineto{\pgfqpoint{1.923953in}{1.107468in}}%
\pgfpathlineto{\pgfqpoint{1.923953in}{1.121079in}}%
\pgfpathlineto{\pgfqpoint{1.923953in}{1.134691in}}%
\pgfpathlineto{\pgfqpoint{1.923953in}{1.148302in}}%
\pgfpathlineto{\pgfqpoint{1.923953in}{1.161913in}}%
\pgfpathlineto{\pgfqpoint{1.923953in}{1.175524in}}%
\pgfpathlineto{\pgfqpoint{1.923953in}{1.189135in}}%
\pgfpathlineto{\pgfqpoint{1.923953in}{1.192022in}}%
\pgfpathlineto{\pgfqpoint{1.908296in}{1.189424in}}%
\pgfpathlineto{\pgfqpoint{1.907686in}{1.189135in}}%
\pgfpathlineto{\pgfqpoint{1.892640in}{1.181190in}}%
\pgfpathlineto{\pgfqpoint{1.885273in}{1.175524in}}%
\pgfpathlineto{\pgfqpoint{1.876983in}{1.167420in}}%
\pgfpathlineto{\pgfqpoint{1.872267in}{1.161913in}}%
\pgfpathlineto{\pgfqpoint{1.865481in}{1.148302in}}%
\pgfpathlineto{\pgfqpoint{1.864831in}{1.134691in}}%
\pgfpathlineto{\pgfqpoint{1.870481in}{1.121079in}}%
\pgfpathlineto{\pgfqpoint{1.876983in}{1.112867in}}%
\pgfpathlineto{\pgfqpoint{1.882109in}{1.107468in}}%
\pgfpathlineto{\pgfqpoint{1.892640in}{1.099092in}}%
\pgfpathlineto{\pgfqpoint{1.902247in}{1.093857in}}%
\pgfpathlineto{\pgfqpoint{1.908296in}{1.090972in}}%
\pgfpathclose%
\pgfpathmoveto{\pgfqpoint{0.389609in}{1.360558in}}%
\pgfpathlineto{\pgfqpoint{0.401063in}{1.366079in}}%
\pgfpathlineto{\pgfqpoint{0.405266in}{1.368459in}}%
\pgfpathlineto{\pgfqpoint{0.418826in}{1.379691in}}%
\pgfpathlineto{\pgfqpoint{0.420923in}{1.382101in}}%
\pgfpathlineto{\pgfqpoint{0.429011in}{1.393302in}}%
\pgfpathlineto{\pgfqpoint{0.433468in}{1.406913in}}%
\pgfpathlineto{\pgfqpoint{0.431524in}{1.420524in}}%
\pgfpathlineto{\pgfqpoint{0.423670in}{1.434135in}}%
\pgfpathlineto{\pgfqpoint{0.420923in}{1.437135in}}%
\pgfpathlineto{\pgfqpoint{0.409357in}{1.447746in}}%
\pgfpathlineto{\pgfqpoint{0.405266in}{1.450808in}}%
\pgfpathlineto{\pgfqpoint{0.389609in}{1.458767in}}%
\pgfpathlineto{\pgfqpoint{0.375580in}{1.461357in}}%
\pgfpathlineto{\pgfqpoint{0.373953in}{1.461626in}}%
\pgfpathlineto{\pgfqpoint{0.373953in}{1.461357in}}%
\pgfpathlineto{\pgfqpoint{0.373953in}{1.447746in}}%
\pgfpathlineto{\pgfqpoint{0.373953in}{1.434135in}}%
\pgfpathlineto{\pgfqpoint{0.373953in}{1.420524in}}%
\pgfpathlineto{\pgfqpoint{0.373953in}{1.406913in}}%
\pgfpathlineto{\pgfqpoint{0.373953in}{1.393302in}}%
\pgfpathlineto{\pgfqpoint{0.373953in}{1.379691in}}%
\pgfpathlineto{\pgfqpoint{0.373953in}{1.366079in}}%
\pgfpathlineto{\pgfqpoint{0.373953in}{1.357842in}}%
\pgfpathlineto{\pgfqpoint{0.389609in}{1.360558in}}%
\pgfpathclose%
\pgfpathmoveto{\pgfqpoint{0.671428in}{1.359594in}}%
\pgfpathlineto{\pgfqpoint{0.687084in}{1.357953in}}%
\pgfpathlineto{\pgfqpoint{0.702741in}{1.361716in}}%
\pgfpathlineto{\pgfqpoint{0.710864in}{1.366079in}}%
\pgfpathlineto{\pgfqpoint{0.718397in}{1.370811in}}%
\pgfpathlineto{\pgfqpoint{0.728612in}{1.379691in}}%
\pgfpathlineto{\pgfqpoint{0.734054in}{1.386240in}}%
\pgfpathlineto{\pgfqpoint{0.739073in}{1.393302in}}%
\pgfpathlineto{\pgfqpoint{0.743402in}{1.406913in}}%
\pgfpathlineto{\pgfqpoint{0.741514in}{1.420524in}}%
\pgfpathlineto{\pgfqpoint{0.734054in}{1.433826in}}%
\pgfpathlineto{\pgfqpoint{0.733849in}{1.434135in}}%
\pgfpathlineto{\pgfqpoint{0.719685in}{1.447746in}}%
\pgfpathlineto{\pgfqpoint{0.718397in}{1.448758in}}%
\pgfpathlineto{\pgfqpoint{0.702741in}{1.457533in}}%
\pgfpathlineto{\pgfqpoint{0.687798in}{1.461357in}}%
\pgfpathlineto{\pgfqpoint{0.687084in}{1.461522in}}%
\pgfpathlineto{\pgfqpoint{0.685429in}{1.461357in}}%
\pgfpathlineto{\pgfqpoint{0.671428in}{1.459794in}}%
\pgfpathlineto{\pgfqpoint{0.655771in}{1.452728in}}%
\pgfpathlineto{\pgfqpoint{0.648724in}{1.447746in}}%
\pgfpathlineto{\pgfqpoint{0.640115in}{1.440261in}}%
\pgfpathlineto{\pgfqpoint{0.634384in}{1.434135in}}%
\pgfpathlineto{\pgfqpoint{0.626256in}{1.420524in}}%
\pgfpathlineto{\pgfqpoint{0.624458in}{1.408351in}}%
\pgfpathlineto{\pgfqpoint{0.624269in}{1.406913in}}%
\pgfpathlineto{\pgfqpoint{0.624458in}{1.406292in}}%
\pgfpathlineto{\pgfqpoint{0.628857in}{1.393302in}}%
\pgfpathlineto{\pgfqpoint{0.638951in}{1.379691in}}%
\pgfpathlineto{\pgfqpoint{0.640115in}{1.378571in}}%
\pgfpathlineto{\pgfqpoint{0.655771in}{1.366257in}}%
\pgfpathlineto{\pgfqpoint{0.656127in}{1.366079in}}%
\pgfpathlineto{\pgfqpoint{0.671428in}{1.359594in}}%
\pgfpathclose%
\pgfpathmoveto{\pgfqpoint{0.968902in}{1.364563in}}%
\pgfpathlineto{\pgfqpoint{0.984559in}{1.358834in}}%
\pgfpathlineto{\pgfqpoint{1.000216in}{1.358285in}}%
\pgfpathlineto{\pgfqpoint{1.015872in}{1.363055in}}%
\pgfpathlineto{\pgfqpoint{1.021008in}{1.366079in}}%
\pgfpathlineto{\pgfqpoint{1.031529in}{1.373293in}}%
\pgfpathlineto{\pgfqpoint{1.038595in}{1.379691in}}%
\pgfpathlineto{\pgfqpoint{1.047185in}{1.390430in}}%
\pgfpathlineto{\pgfqpoint{1.049204in}{1.393302in}}%
\pgfpathlineto{\pgfqpoint{1.053428in}{1.406913in}}%
\pgfpathlineto{\pgfqpoint{1.051585in}{1.420524in}}%
\pgfpathlineto{\pgfqpoint{1.047185in}{1.428456in}}%
\pgfpathlineto{\pgfqpoint{1.043568in}{1.434135in}}%
\pgfpathlineto{\pgfqpoint{1.031529in}{1.446246in}}%
\pgfpathlineto{\pgfqpoint{1.029548in}{1.447746in}}%
\pgfpathlineto{\pgfqpoint{1.015872in}{1.456106in}}%
\pgfpathlineto{\pgfqpoint{1.000216in}{1.461190in}}%
\pgfpathlineto{\pgfqpoint{0.984559in}{1.460605in}}%
\pgfpathlineto{\pgfqpoint{0.968902in}{1.454499in}}%
\pgfpathlineto{\pgfqpoint{0.958688in}{1.447746in}}%
\pgfpathlineto{\pgfqpoint{0.953246in}{1.443305in}}%
\pgfpathlineto{\pgfqpoint{0.944431in}{1.434135in}}%
\pgfpathlineto{\pgfqpoint{0.937589in}{1.423050in}}%
\pgfpathlineto{\pgfqpoint{0.936196in}{1.420524in}}%
\pgfpathlineto{\pgfqpoint{0.934390in}{1.406913in}}%
\pgfpathlineto{\pgfqpoint{0.937589in}{1.396442in}}%
\pgfpathlineto{\pgfqpoint{0.938680in}{1.393302in}}%
\pgfpathlineto{\pgfqpoint{0.949183in}{1.379691in}}%
\pgfpathlineto{\pgfqpoint{0.953246in}{1.375887in}}%
\pgfpathlineto{\pgfqpoint{0.966521in}{1.366079in}}%
\pgfpathlineto{\pgfqpoint{0.968902in}{1.364563in}}%
\pgfpathclose%
\pgfpathmoveto{\pgfqpoint{1.282034in}{1.363055in}}%
\pgfpathlineto{\pgfqpoint{1.297690in}{1.358285in}}%
\pgfpathlineto{\pgfqpoint{1.313347in}{1.358834in}}%
\pgfpathlineto{\pgfqpoint{1.329003in}{1.364563in}}%
\pgfpathlineto{\pgfqpoint{1.331384in}{1.366079in}}%
\pgfpathlineto{\pgfqpoint{1.344660in}{1.375887in}}%
\pgfpathlineto{\pgfqpoint{1.348723in}{1.379691in}}%
\pgfpathlineto{\pgfqpoint{1.359226in}{1.393302in}}%
\pgfpathlineto{\pgfqpoint{1.360317in}{1.396442in}}%
\pgfpathlineto{\pgfqpoint{1.363515in}{1.406913in}}%
\pgfpathlineto{\pgfqpoint{1.361710in}{1.420524in}}%
\pgfpathlineto{\pgfqpoint{1.360317in}{1.423050in}}%
\pgfpathlineto{\pgfqpoint{1.353474in}{1.434135in}}%
\pgfpathlineto{\pgfqpoint{1.344660in}{1.443305in}}%
\pgfpathlineto{\pgfqpoint{1.339217in}{1.447746in}}%
\pgfpathlineto{\pgfqpoint{1.329003in}{1.454499in}}%
\pgfpathlineto{\pgfqpoint{1.313347in}{1.460605in}}%
\pgfpathlineto{\pgfqpoint{1.297690in}{1.461190in}}%
\pgfpathlineto{\pgfqpoint{1.282034in}{1.456106in}}%
\pgfpathlineto{\pgfqpoint{1.268358in}{1.447746in}}%
\pgfpathlineto{\pgfqpoint{1.266377in}{1.446246in}}%
\pgfpathlineto{\pgfqpoint{1.254338in}{1.434135in}}%
\pgfpathlineto{\pgfqpoint{1.250721in}{1.428456in}}%
\pgfpathlineto{\pgfqpoint{1.246320in}{1.420524in}}%
\pgfpathlineto{\pgfqpoint{1.244478in}{1.406913in}}%
\pgfpathlineto{\pgfqpoint{1.248701in}{1.393302in}}%
\pgfpathlineto{\pgfqpoint{1.250721in}{1.390430in}}%
\pgfpathlineto{\pgfqpoint{1.259311in}{1.379691in}}%
\pgfpathlineto{\pgfqpoint{1.266377in}{1.373293in}}%
\pgfpathlineto{\pgfqpoint{1.276898in}{1.366079in}}%
\pgfpathlineto{\pgfqpoint{1.282034in}{1.363055in}}%
\pgfpathclose%
\pgfpathmoveto{\pgfqpoint{1.595165in}{1.361716in}}%
\pgfpathlineto{\pgfqpoint{1.610822in}{1.357953in}}%
\pgfpathlineto{\pgfqpoint{1.626478in}{1.359594in}}%
\pgfpathlineto{\pgfqpoint{1.641779in}{1.366079in}}%
\pgfpathlineto{\pgfqpoint{1.642135in}{1.366257in}}%
\pgfpathlineto{\pgfqpoint{1.657791in}{1.378571in}}%
\pgfpathlineto{\pgfqpoint{1.658955in}{1.379691in}}%
\pgfpathlineto{\pgfqpoint{1.669049in}{1.393302in}}%
\pgfpathlineto{\pgfqpoint{1.673448in}{1.406292in}}%
\pgfpathlineto{\pgfqpoint{1.673637in}{1.406913in}}%
\pgfpathlineto{\pgfqpoint{1.673448in}{1.408351in}}%
\pgfpathlineto{\pgfqpoint{1.671650in}{1.420524in}}%
\pgfpathlineto{\pgfqpoint{1.663522in}{1.434135in}}%
\pgfpathlineto{\pgfqpoint{1.657791in}{1.440261in}}%
\pgfpathlineto{\pgfqpoint{1.649181in}{1.447746in}}%
\pgfpathlineto{\pgfqpoint{1.642135in}{1.452728in}}%
\pgfpathlineto{\pgfqpoint{1.626478in}{1.459794in}}%
\pgfpathlineto{\pgfqpoint{1.612476in}{1.461357in}}%
\pgfpathlineto{\pgfqpoint{1.610822in}{1.461522in}}%
\pgfpathlineto{\pgfqpoint{1.610108in}{1.461357in}}%
\pgfpathlineto{\pgfqpoint{1.595165in}{1.457533in}}%
\pgfpathlineto{\pgfqpoint{1.579508in}{1.448758in}}%
\pgfpathlineto{\pgfqpoint{1.578221in}{1.447746in}}%
\pgfpathlineto{\pgfqpoint{1.564056in}{1.434135in}}%
\pgfpathlineto{\pgfqpoint{1.563852in}{1.433826in}}%
\pgfpathlineto{\pgfqpoint{1.556392in}{1.420524in}}%
\pgfpathlineto{\pgfqpoint{1.554504in}{1.406913in}}%
\pgfpathlineto{\pgfqpoint{1.558832in}{1.393302in}}%
\pgfpathlineto{\pgfqpoint{1.563852in}{1.386240in}}%
\pgfpathlineto{\pgfqpoint{1.569294in}{1.379691in}}%
\pgfpathlineto{\pgfqpoint{1.579508in}{1.370811in}}%
\pgfpathlineto{\pgfqpoint{1.587042in}{1.366079in}}%
\pgfpathlineto{\pgfqpoint{1.595165in}{1.361716in}}%
\pgfpathclose%
\pgfpathmoveto{\pgfqpoint{1.908296in}{1.360558in}}%
\pgfpathlineto{\pgfqpoint{1.923953in}{1.357842in}}%
\pgfpathlineto{\pgfqpoint{1.923953in}{1.366079in}}%
\pgfpathlineto{\pgfqpoint{1.923953in}{1.379691in}}%
\pgfpathlineto{\pgfqpoint{1.923953in}{1.393302in}}%
\pgfpathlineto{\pgfqpoint{1.923953in}{1.406913in}}%
\pgfpathlineto{\pgfqpoint{1.923953in}{1.420524in}}%
\pgfpathlineto{\pgfqpoint{1.923953in}{1.434135in}}%
\pgfpathlineto{\pgfqpoint{1.923953in}{1.447746in}}%
\pgfpathlineto{\pgfqpoint{1.923953in}{1.461357in}}%
\pgfpathlineto{\pgfqpoint{1.923953in}{1.461626in}}%
\pgfpathlineto{\pgfqpoint{1.922326in}{1.461357in}}%
\pgfpathlineto{\pgfqpoint{1.908296in}{1.458767in}}%
\pgfpathlineto{\pgfqpoint{1.892640in}{1.450808in}}%
\pgfpathlineto{\pgfqpoint{1.888548in}{1.447746in}}%
\pgfpathlineto{\pgfqpoint{1.876983in}{1.437135in}}%
\pgfpathlineto{\pgfqpoint{1.874236in}{1.434135in}}%
\pgfpathlineto{\pgfqpoint{1.866382in}{1.420524in}}%
\pgfpathlineto{\pgfqpoint{1.864438in}{1.406913in}}%
\pgfpathlineto{\pgfqpoint{1.868895in}{1.393302in}}%
\pgfpathlineto{\pgfqpoint{1.876983in}{1.382101in}}%
\pgfpathlineto{\pgfqpoint{1.879080in}{1.379691in}}%
\pgfpathlineto{\pgfqpoint{1.892640in}{1.368459in}}%
\pgfpathlineto{\pgfqpoint{1.896843in}{1.366079in}}%
\pgfpathlineto{\pgfqpoint{1.908296in}{1.360558in}}%
\pgfpathclose%
\pgfpathmoveto{\pgfqpoint{0.389609in}{1.630078in}}%
\pgfpathlineto{\pgfqpoint{0.405266in}{1.637768in}}%
\pgfpathlineto{\pgfqpoint{0.405994in}{1.638302in}}%
\pgfpathlineto{\pgfqpoint{0.420923in}{1.651280in}}%
\pgfpathlineto{\pgfqpoint{0.421536in}{1.651913in}}%
\pgfpathlineto{\pgfqpoint{0.430382in}{1.665524in}}%
\pgfpathlineto{\pgfqpoint{0.433600in}{1.679135in}}%
\pgfpathlineto{\pgfqpoint{0.420923in}{1.679135in}}%
\pgfpathlineto{\pgfqpoint{0.405266in}{1.679135in}}%
\pgfpathlineto{\pgfqpoint{0.389609in}{1.679135in}}%
\pgfpathlineto{\pgfqpoint{0.373953in}{1.679135in}}%
\pgfpathlineto{\pgfqpoint{0.373953in}{1.665524in}}%
\pgfpathlineto{\pgfqpoint{0.373953in}{1.651913in}}%
\pgfpathlineto{\pgfqpoint{0.373953in}{1.638302in}}%
\pgfpathlineto{\pgfqpoint{0.373953in}{1.627281in}}%
\pgfpathlineto{\pgfqpoint{0.389609in}{1.630078in}}%
\pgfpathclose%
\pgfpathmoveto{\pgfqpoint{0.655771in}{1.635913in}}%
\pgfpathlineto{\pgfqpoint{0.671428in}{1.629085in}}%
\pgfpathlineto{\pgfqpoint{0.687084in}{1.627395in}}%
\pgfpathlineto{\pgfqpoint{0.702741in}{1.631270in}}%
\pgfpathlineto{\pgfqpoint{0.715624in}{1.638302in}}%
\pgfpathlineto{\pgfqpoint{0.718397in}{1.640124in}}%
\pgfpathlineto{\pgfqpoint{0.731317in}{1.651913in}}%
\pgfpathlineto{\pgfqpoint{0.734054in}{1.655567in}}%
\pgfpathlineto{\pgfqpoint{0.740405in}{1.665524in}}%
\pgfpathlineto{\pgfqpoint{0.743530in}{1.679135in}}%
\pgfpathlineto{\pgfqpoint{0.734054in}{1.679135in}}%
\pgfpathlineto{\pgfqpoint{0.718397in}{1.679135in}}%
\pgfpathlineto{\pgfqpoint{0.702741in}{1.679135in}}%
\pgfpathlineto{\pgfqpoint{0.687084in}{1.679135in}}%
\pgfpathlineto{\pgfqpoint{0.671428in}{1.679135in}}%
\pgfpathlineto{\pgfqpoint{0.655771in}{1.679135in}}%
\pgfpathlineto{\pgfqpoint{0.640115in}{1.679135in}}%
\pgfpathlineto{\pgfqpoint{0.624458in}{1.679135in}}%
\pgfpathlineto{\pgfqpoint{0.624148in}{1.679135in}}%
\pgfpathlineto{\pgfqpoint{0.624458in}{1.677721in}}%
\pgfpathlineto{\pgfqpoint{0.627438in}{1.665524in}}%
\pgfpathlineto{\pgfqpoint{0.636592in}{1.651913in}}%
\pgfpathlineto{\pgfqpoint{0.640115in}{1.648356in}}%
\pgfpathlineto{\pgfqpoint{0.652320in}{1.638302in}}%
\pgfpathlineto{\pgfqpoint{0.655771in}{1.635913in}}%
\pgfpathclose%
\pgfpathmoveto{\pgfqpoint{0.968902in}{1.634202in}}%
\pgfpathlineto{\pgfqpoint{0.984559in}{1.628302in}}%
\pgfpathlineto{\pgfqpoint{1.000216in}{1.627737in}}%
\pgfpathlineto{\pgfqpoint{1.015872in}{1.632649in}}%
\pgfpathlineto{\pgfqpoint{1.025319in}{1.638302in}}%
\pgfpathlineto{\pgfqpoint{1.031529in}{1.642758in}}%
\pgfpathlineto{\pgfqpoint{1.041163in}{1.651913in}}%
\pgfpathlineto{\pgfqpoint{1.047185in}{1.660265in}}%
\pgfpathlineto{\pgfqpoint{1.050503in}{1.665524in}}%
\pgfpathlineto{\pgfqpoint{1.053553in}{1.679135in}}%
\pgfpathlineto{\pgfqpoint{1.047185in}{1.679135in}}%
\pgfpathlineto{\pgfqpoint{1.031529in}{1.679135in}}%
\pgfpathlineto{\pgfqpoint{1.015872in}{1.679135in}}%
\pgfpathlineto{\pgfqpoint{1.000216in}{1.679135in}}%
\pgfpathlineto{\pgfqpoint{0.984559in}{1.679135in}}%
\pgfpathlineto{\pgfqpoint{0.968902in}{1.679135in}}%
\pgfpathlineto{\pgfqpoint{0.953246in}{1.679135in}}%
\pgfpathlineto{\pgfqpoint{0.937589in}{1.679135in}}%
\pgfpathlineto{\pgfqpoint{0.934268in}{1.679135in}}%
\pgfpathlineto{\pgfqpoint{0.937257in}{1.665524in}}%
\pgfpathlineto{\pgfqpoint{0.937589in}{1.664994in}}%
\pgfpathlineto{\pgfqpoint{0.946729in}{1.651913in}}%
\pgfpathlineto{\pgfqpoint{0.953246in}{1.645509in}}%
\pgfpathlineto{\pgfqpoint{0.962568in}{1.638302in}}%
\pgfpathlineto{\pgfqpoint{0.968902in}{1.634202in}}%
\pgfpathclose%
\pgfpathmoveto{\pgfqpoint{1.282034in}{1.632649in}}%
\pgfpathlineto{\pgfqpoint{1.297690in}{1.627737in}}%
\pgfpathlineto{\pgfqpoint{1.313347in}{1.628302in}}%
\pgfpathlineto{\pgfqpoint{1.329003in}{1.634202in}}%
\pgfpathlineto{\pgfqpoint{1.335338in}{1.638302in}}%
\pgfpathlineto{\pgfqpoint{1.344660in}{1.645509in}}%
\pgfpathlineto{\pgfqpoint{1.351177in}{1.651913in}}%
\pgfpathlineto{\pgfqpoint{1.360317in}{1.664994in}}%
\pgfpathlineto{\pgfqpoint{1.360649in}{1.665524in}}%
\pgfpathlineto{\pgfqpoint{1.363638in}{1.679135in}}%
\pgfpathlineto{\pgfqpoint{1.360317in}{1.679135in}}%
\pgfpathlineto{\pgfqpoint{1.344660in}{1.679135in}}%
\pgfpathlineto{\pgfqpoint{1.329003in}{1.679135in}}%
\pgfpathlineto{\pgfqpoint{1.313347in}{1.679135in}}%
\pgfpathlineto{\pgfqpoint{1.297690in}{1.679135in}}%
\pgfpathlineto{\pgfqpoint{1.282034in}{1.679135in}}%
\pgfpathlineto{\pgfqpoint{1.266377in}{1.679135in}}%
\pgfpathlineto{\pgfqpoint{1.250721in}{1.679135in}}%
\pgfpathlineto{\pgfqpoint{1.244353in}{1.679135in}}%
\pgfpathlineto{\pgfqpoint{1.247402in}{1.665524in}}%
\pgfpathlineto{\pgfqpoint{1.250721in}{1.660265in}}%
\pgfpathlineto{\pgfqpoint{1.256742in}{1.651913in}}%
\pgfpathlineto{\pgfqpoint{1.266377in}{1.642758in}}%
\pgfpathlineto{\pgfqpoint{1.272587in}{1.638302in}}%
\pgfpathlineto{\pgfqpoint{1.282034in}{1.632649in}}%
\pgfpathclose%
\pgfpathmoveto{\pgfqpoint{1.595165in}{1.631270in}}%
\pgfpathlineto{\pgfqpoint{1.610822in}{1.627395in}}%
\pgfpathlineto{\pgfqpoint{1.626478in}{1.629085in}}%
\pgfpathlineto{\pgfqpoint{1.642135in}{1.635913in}}%
\pgfpathlineto{\pgfqpoint{1.645586in}{1.638302in}}%
\pgfpathlineto{\pgfqpoint{1.657791in}{1.648356in}}%
\pgfpathlineto{\pgfqpoint{1.661314in}{1.651913in}}%
\pgfpathlineto{\pgfqpoint{1.670468in}{1.665524in}}%
\pgfpathlineto{\pgfqpoint{1.673448in}{1.677721in}}%
\pgfpathlineto{\pgfqpoint{1.673757in}{1.679135in}}%
\pgfpathlineto{\pgfqpoint{1.673448in}{1.679135in}}%
\pgfpathlineto{\pgfqpoint{1.657791in}{1.679135in}}%
\pgfpathlineto{\pgfqpoint{1.642135in}{1.679135in}}%
\pgfpathlineto{\pgfqpoint{1.626478in}{1.679135in}}%
\pgfpathlineto{\pgfqpoint{1.610822in}{1.679135in}}%
\pgfpathlineto{\pgfqpoint{1.595165in}{1.679135in}}%
\pgfpathlineto{\pgfqpoint{1.579508in}{1.679135in}}%
\pgfpathlineto{\pgfqpoint{1.563852in}{1.679135in}}%
\pgfpathlineto{\pgfqpoint{1.554376in}{1.679135in}}%
\pgfpathlineto{\pgfqpoint{1.557501in}{1.665524in}}%
\pgfpathlineto{\pgfqpoint{1.563852in}{1.655567in}}%
\pgfpathlineto{\pgfqpoint{1.566589in}{1.651913in}}%
\pgfpathlineto{\pgfqpoint{1.579508in}{1.640124in}}%
\pgfpathlineto{\pgfqpoint{1.582282in}{1.638302in}}%
\pgfpathlineto{\pgfqpoint{1.595165in}{1.631270in}}%
\pgfpathclose%
\pgfpathmoveto{\pgfqpoint{1.892640in}{1.637768in}}%
\pgfpathlineto{\pgfqpoint{1.908296in}{1.630078in}}%
\pgfpathlineto{\pgfqpoint{1.923953in}{1.627281in}}%
\pgfpathlineto{\pgfqpoint{1.923953in}{1.638302in}}%
\pgfpathlineto{\pgfqpoint{1.923953in}{1.651913in}}%
\pgfpathlineto{\pgfqpoint{1.923953in}{1.665524in}}%
\pgfpathlineto{\pgfqpoint{1.923953in}{1.679135in}}%
\pgfpathlineto{\pgfqpoint{1.908296in}{1.679135in}}%
\pgfpathlineto{\pgfqpoint{1.892640in}{1.679135in}}%
\pgfpathlineto{\pgfqpoint{1.876983in}{1.679135in}}%
\pgfpathlineto{\pgfqpoint{1.864306in}{1.679135in}}%
\pgfpathlineto{\pgfqpoint{1.867524in}{1.665524in}}%
\pgfpathlineto{\pgfqpoint{1.876369in}{1.651913in}}%
\pgfpathlineto{\pgfqpoint{1.876983in}{1.651280in}}%
\pgfpathlineto{\pgfqpoint{1.891911in}{1.638302in}}%
\pgfpathlineto{\pgfqpoint{1.892640in}{1.637768in}}%
\pgfpathclose%
\pgfusepath{fill}%
\end{pgfscope}%
\begin{pgfscope}%
\pgfsetbuttcap%
\pgfsetroundjoin%
\definecolor{currentfill}{rgb}{0.000000,0.000000,0.000000}%
\pgfsetfillcolor{currentfill}%
\pgfsetlinewidth{0.803000pt}%
\definecolor{currentstroke}{rgb}{0.000000,0.000000,0.000000}%
\pgfsetstrokecolor{currentstroke}%
\pgfsetdash{}{0pt}%
\pgfsys@defobject{currentmarker}{\pgfqpoint{0.000000in}{-0.048611in}}{\pgfqpoint{0.000000in}{0.000000in}}{%
\pgfpathmoveto{\pgfqpoint{0.000000in}{0.000000in}}%
\pgfpathlineto{\pgfqpoint{0.000000in}{-0.048611in}}%
\pgfusepath{stroke,fill}%
}%
\begin{pgfscope}%
\pgfsys@transformshift{0.373953in}{0.331635in}%
\pgfsys@useobject{currentmarker}{}%
\end{pgfscope}%
\end{pgfscope}%
\begin{pgfscope}%
\definecolor{textcolor}{rgb}{0.000000,0.000000,0.000000}%
\pgfsetstrokecolor{textcolor}%
\pgfsetfillcolor{textcolor}%
\pgftext[x=0.373953in,y=0.234413in,,top]{\color{textcolor}{\sffamily\fontsize{10.000000}{12.000000}\selectfont\catcode`\^=\active\def^{\ifmmode\sp\else\^{}\fi}\catcode`\%=\active\def%{\%}0}}%
\end{pgfscope}%
\begin{pgfscope}%
\pgfsetbuttcap%
\pgfsetroundjoin%
\definecolor{currentfill}{rgb}{0.000000,0.000000,0.000000}%
\pgfsetfillcolor{currentfill}%
\pgfsetlinewidth{0.803000pt}%
\definecolor{currentstroke}{rgb}{0.000000,0.000000,0.000000}%
\pgfsetstrokecolor{currentstroke}%
\pgfsetdash{}{0pt}%
\pgfsys@defobject{currentmarker}{\pgfqpoint{0.000000in}{-0.048611in}}{\pgfqpoint{0.000000in}{0.000000in}}{%
\pgfpathmoveto{\pgfqpoint{0.000000in}{0.000000in}}%
\pgfpathlineto{\pgfqpoint{0.000000in}{-0.048611in}}%
\pgfusepath{stroke,fill}%
}%
\begin{pgfscope}%
\pgfsys@transformshift{0.890620in}{0.331635in}%
\pgfsys@useobject{currentmarker}{}%
\end{pgfscope}%
\end{pgfscope}%
\begin{pgfscope}%
\definecolor{textcolor}{rgb}{0.000000,0.000000,0.000000}%
\pgfsetstrokecolor{textcolor}%
\pgfsetfillcolor{textcolor}%
\pgftext[x=0.890620in,y=0.234413in,,top]{\color{textcolor}{\sffamily\fontsize{10.000000}{12.000000}\selectfont\catcode`\^=\active\def^{\ifmmode\sp\else\^{}\fi}\catcode`\%=\active\def%{\%}10}}%
\end{pgfscope}%
\begin{pgfscope}%
\pgfsetbuttcap%
\pgfsetroundjoin%
\definecolor{currentfill}{rgb}{0.000000,0.000000,0.000000}%
\pgfsetfillcolor{currentfill}%
\pgfsetlinewidth{0.803000pt}%
\definecolor{currentstroke}{rgb}{0.000000,0.000000,0.000000}%
\pgfsetstrokecolor{currentstroke}%
\pgfsetdash{}{0pt}%
\pgfsys@defobject{currentmarker}{\pgfqpoint{0.000000in}{-0.048611in}}{\pgfqpoint{0.000000in}{0.000000in}}{%
\pgfpathmoveto{\pgfqpoint{0.000000in}{0.000000in}}%
\pgfpathlineto{\pgfqpoint{0.000000in}{-0.048611in}}%
\pgfusepath{stroke,fill}%
}%
\begin{pgfscope}%
\pgfsys@transformshift{1.407286in}{0.331635in}%
\pgfsys@useobject{currentmarker}{}%
\end{pgfscope}%
\end{pgfscope}%
\begin{pgfscope}%
\definecolor{textcolor}{rgb}{0.000000,0.000000,0.000000}%
\pgfsetstrokecolor{textcolor}%
\pgfsetfillcolor{textcolor}%
\pgftext[x=1.407286in,y=0.234413in,,top]{\color{textcolor}{\sffamily\fontsize{10.000000}{12.000000}\selectfont\catcode`\^=\active\def^{\ifmmode\sp\else\^{}\fi}\catcode`\%=\active\def%{\%}20}}%
\end{pgfscope}%
\begin{pgfscope}%
\pgfsetbuttcap%
\pgfsetroundjoin%
\definecolor{currentfill}{rgb}{0.000000,0.000000,0.000000}%
\pgfsetfillcolor{currentfill}%
\pgfsetlinewidth{0.803000pt}%
\definecolor{currentstroke}{rgb}{0.000000,0.000000,0.000000}%
\pgfsetstrokecolor{currentstroke}%
\pgfsetdash{}{0pt}%
\pgfsys@defobject{currentmarker}{\pgfqpoint{0.000000in}{-0.048611in}}{\pgfqpoint{0.000000in}{0.000000in}}{%
\pgfpathmoveto{\pgfqpoint{0.000000in}{0.000000in}}%
\pgfpathlineto{\pgfqpoint{0.000000in}{-0.048611in}}%
\pgfusepath{stroke,fill}%
}%
\begin{pgfscope}%
\pgfsys@transformshift{1.923953in}{0.331635in}%
\pgfsys@useobject{currentmarker}{}%
\end{pgfscope}%
\end{pgfscope}%
\begin{pgfscope}%
\definecolor{textcolor}{rgb}{0.000000,0.000000,0.000000}%
\pgfsetstrokecolor{textcolor}%
\pgfsetfillcolor{textcolor}%
\pgftext[x=1.923953in,y=0.234413in,,top]{\color{textcolor}{\sffamily\fontsize{10.000000}{12.000000}\selectfont\catcode`\^=\active\def^{\ifmmode\sp\else\^{}\fi}\catcode`\%=\active\def%{\%}30}}%
\end{pgfscope}%
\begin{pgfscope}%
\pgfsetbuttcap%
\pgfsetroundjoin%
\definecolor{currentfill}{rgb}{0.000000,0.000000,0.000000}%
\pgfsetfillcolor{currentfill}%
\pgfsetlinewidth{0.803000pt}%
\definecolor{currentstroke}{rgb}{0.000000,0.000000,0.000000}%
\pgfsetstrokecolor{currentstroke}%
\pgfsetdash{}{0pt}%
\pgfsys@defobject{currentmarker}{\pgfqpoint{-0.048611in}{0.000000in}}{\pgfqpoint{-0.000000in}{0.000000in}}{%
\pgfpathmoveto{\pgfqpoint{-0.000000in}{0.000000in}}%
\pgfpathlineto{\pgfqpoint{-0.048611in}{0.000000in}}%
\pgfusepath{stroke,fill}%
}%
\begin{pgfscope}%
\pgfsys@transformshift{0.373953in}{0.331635in}%
\pgfsys@useobject{currentmarker}{}%
\end{pgfscope}%
\end{pgfscope}%
\begin{pgfscope}%
\definecolor{textcolor}{rgb}{0.000000,0.000000,0.000000}%
\pgfsetstrokecolor{textcolor}%
\pgfsetfillcolor{textcolor}%
\pgftext[x=0.188365in, y=0.278873in, left, base]{\color{textcolor}{\sffamily\fontsize{10.000000}{12.000000}\selectfont\catcode`\^=\active\def^{\ifmmode\sp\else\^{}\fi}\catcode`\%=\active\def%{\%}0}}%
\end{pgfscope}%
\begin{pgfscope}%
\pgfsetbuttcap%
\pgfsetroundjoin%
\definecolor{currentfill}{rgb}{0.000000,0.000000,0.000000}%
\pgfsetfillcolor{currentfill}%
\pgfsetlinewidth{0.803000pt}%
\definecolor{currentstroke}{rgb}{0.000000,0.000000,0.000000}%
\pgfsetstrokecolor{currentstroke}%
\pgfsetdash{}{0pt}%
\pgfsys@defobject{currentmarker}{\pgfqpoint{-0.048611in}{0.000000in}}{\pgfqpoint{-0.000000in}{0.000000in}}{%
\pgfpathmoveto{\pgfqpoint{-0.000000in}{0.000000in}}%
\pgfpathlineto{\pgfqpoint{-0.048611in}{0.000000in}}%
\pgfusepath{stroke,fill}%
}%
\begin{pgfscope}%
\pgfsys@transformshift{0.373953in}{0.780802in}%
\pgfsys@useobject{currentmarker}{}%
\end{pgfscope}%
\end{pgfscope}%
\begin{pgfscope}%
\definecolor{textcolor}{rgb}{0.000000,0.000000,0.000000}%
\pgfsetstrokecolor{textcolor}%
\pgfsetfillcolor{textcolor}%
\pgftext[x=0.100000in, y=0.728040in, left, base]{\color{textcolor}{\sffamily\fontsize{10.000000}{12.000000}\selectfont\catcode`\^=\active\def^{\ifmmode\sp\else\^{}\fi}\catcode`\%=\active\def%{\%}10}}%
\end{pgfscope}%
\begin{pgfscope}%
\pgfsetbuttcap%
\pgfsetroundjoin%
\definecolor{currentfill}{rgb}{0.000000,0.000000,0.000000}%
\pgfsetfillcolor{currentfill}%
\pgfsetlinewidth{0.803000pt}%
\definecolor{currentstroke}{rgb}{0.000000,0.000000,0.000000}%
\pgfsetstrokecolor{currentstroke}%
\pgfsetdash{}{0pt}%
\pgfsys@defobject{currentmarker}{\pgfqpoint{-0.048611in}{0.000000in}}{\pgfqpoint{-0.000000in}{0.000000in}}{%
\pgfpathmoveto{\pgfqpoint{-0.000000in}{0.000000in}}%
\pgfpathlineto{\pgfqpoint{-0.048611in}{0.000000in}}%
\pgfusepath{stroke,fill}%
}%
\begin{pgfscope}%
\pgfsys@transformshift{0.373953in}{1.229968in}%
\pgfsys@useobject{currentmarker}{}%
\end{pgfscope}%
\end{pgfscope}%
\begin{pgfscope}%
\definecolor{textcolor}{rgb}{0.000000,0.000000,0.000000}%
\pgfsetstrokecolor{textcolor}%
\pgfsetfillcolor{textcolor}%
\pgftext[x=0.100000in, y=1.177207in, left, base]{\color{textcolor}{\sffamily\fontsize{10.000000}{12.000000}\selectfont\catcode`\^=\active\def^{\ifmmode\sp\else\^{}\fi}\catcode`\%=\active\def%{\%}20}}%
\end{pgfscope}%
\begin{pgfscope}%
\pgfsetbuttcap%
\pgfsetroundjoin%
\definecolor{currentfill}{rgb}{0.000000,0.000000,0.000000}%
\pgfsetfillcolor{currentfill}%
\pgfsetlinewidth{0.803000pt}%
\definecolor{currentstroke}{rgb}{0.000000,0.000000,0.000000}%
\pgfsetstrokecolor{currentstroke}%
\pgfsetdash{}{0pt}%
\pgfsys@defobject{currentmarker}{\pgfqpoint{-0.048611in}{0.000000in}}{\pgfqpoint{-0.000000in}{0.000000in}}{%
\pgfpathmoveto{\pgfqpoint{-0.000000in}{0.000000in}}%
\pgfpathlineto{\pgfqpoint{-0.048611in}{0.000000in}}%
\pgfusepath{stroke,fill}%
}%
\begin{pgfscope}%
\pgfsys@transformshift{0.373953in}{1.679135in}%
\pgfsys@useobject{currentmarker}{}%
\end{pgfscope}%
\end{pgfscope}%
\begin{pgfscope}%
\definecolor{textcolor}{rgb}{0.000000,0.000000,0.000000}%
\pgfsetstrokecolor{textcolor}%
\pgfsetfillcolor{textcolor}%
\pgftext[x=0.100000in, y=1.626373in, left, base]{\color{textcolor}{\sffamily\fontsize{10.000000}{12.000000}\selectfont\catcode`\^=\active\def^{\ifmmode\sp\else\^{}\fi}\catcode`\%=\active\def%{\%}30}}%
\end{pgfscope}%
\begin{pgfscope}%
\pgfsetrectcap%
\pgfsetmiterjoin%
\pgfsetlinewidth{0.803000pt}%
\definecolor{currentstroke}{rgb}{0.000000,0.000000,0.000000}%
\pgfsetstrokecolor{currentstroke}%
\pgfsetdash{}{0pt}%
\pgfpathmoveto{\pgfqpoint{0.373953in}{0.331635in}}%
\pgfpathlineto{\pgfqpoint{0.373953in}{1.679135in}}%
\pgfusepath{stroke}%
\end{pgfscope}%
\begin{pgfscope}%
\pgfsetrectcap%
\pgfsetmiterjoin%
\pgfsetlinewidth{0.803000pt}%
\definecolor{currentstroke}{rgb}{0.000000,0.000000,0.000000}%
\pgfsetstrokecolor{currentstroke}%
\pgfsetdash{}{0pt}%
\pgfpathmoveto{\pgfqpoint{1.923953in}{0.331635in}}%
\pgfpathlineto{\pgfqpoint{1.923953in}{1.679135in}}%
\pgfusepath{stroke}%
\end{pgfscope}%
\begin{pgfscope}%
\pgfsetrectcap%
\pgfsetmiterjoin%
\pgfsetlinewidth{0.803000pt}%
\definecolor{currentstroke}{rgb}{0.000000,0.000000,0.000000}%
\pgfsetstrokecolor{currentstroke}%
\pgfsetdash{}{0pt}%
\pgfpathmoveto{\pgfqpoint{0.373953in}{0.331635in}}%
\pgfpathlineto{\pgfqpoint{1.923953in}{0.331635in}}%
\pgfusepath{stroke}%
\end{pgfscope}%
\begin{pgfscope}%
\pgfsetrectcap%
\pgfsetmiterjoin%
\pgfsetlinewidth{0.803000pt}%
\definecolor{currentstroke}{rgb}{0.000000,0.000000,0.000000}%
\pgfsetstrokecolor{currentstroke}%
\pgfsetdash{}{0pt}%
\pgfpathmoveto{\pgfqpoint{0.373953in}{1.679135in}}%
\pgfpathlineto{\pgfqpoint{1.923953in}{1.679135in}}%
\pgfusepath{stroke}%
\end{pgfscope}%
\end{pgfpicture}%
\makeatother%
\endgroup%

        \caption{$c=5$}
        \label{fig:5-experiments-periodic-gaussian-well-5}
    \end{subfigure}
    \caption{Two dimensional periodic potential $V$ for different sizes $c$ of the computational domain.}
    \label{fig:5-experiments-periodic-gaussian-well}
\end{figure}

For Gaussian \gls{smoothing-kernel} \refequ{equ:1-introduction-def-gaussian-kernel}
with \gls{smoothing-parameter} $=0.05$ we plot for two choices of \gls{chebyshev-degree}
the convergence of the error with \gls{sketch-size} in \reffig{fig:5-experiments-electronic-structure-convergence-nv}
and equally for two choices of \gls{sketch-size} the convergence of the
error with \gls{chebyshev-degree} in \reffig{fig:5-experiments-electronic-structure-convergence-nv}.

\begin{figure}[ht]
    \begin{subfigure}[b]{0.49\columnwidth}
        %% Creator: Matplotlib, PGF backend
%%
%% To include the figure in your LaTeX document, write
%%   \input{<filename>.pgf}
%%
%% Make sure the required packages are loaded in your preamble
%%   \usepackage{pgf}
%%
%% Also ensure that all the required font packages are loaded; for instance,
%% the lmodern package is sometimes necessary when using math font.
%%   \usepackage{lmodern}
%%
%% Figures using additional raster images can only be included by \input if
%% they are in the same directory as the main LaTeX file. For loading figures
%% from other directories you can use the `import` package
%%   \usepackage{import}
%%
%% and then include the figures with
%%   \import{<path to file>}{<filename>.pgf}
%%
%% Matplotlib used the following preamble
%%   \def\mathdefault#1{#1}
%%   \everymath=\expandafter{\the\everymath\displaystyle}
%%   
%%   \usepackage{fontspec}
%%   \setmainfont{DejaVuSerif.ttf}[Path=\detokenize{C:/Users/fabio/Documents/Work/MasterThesis/Rand-SD/.venv/Lib/site-packages/matplotlib/mpl-data/fonts/ttf/}]
%%   \setsansfont{DejaVuSans.ttf}[Path=\detokenize{C:/Users/fabio/Documents/Work/MasterThesis/Rand-SD/.venv/Lib/site-packages/matplotlib/mpl-data/fonts/ttf/}]
%%   \setmonofont{DejaVuSansMono.ttf}[Path=\detokenize{C:/Users/fabio/Documents/Work/MasterThesis/Rand-SD/.venv/Lib/site-packages/matplotlib/mpl-data/fonts/ttf/}]
%%   \makeatletter\@ifpackageloaded{underscore}{}{\usepackage[strings]{underscore}}\makeatother
%%
\begingroup%
\makeatletter%
\begin{pgfpicture}%
\pgfpathrectangle{\pgfpointorigin}{\pgfqpoint{2.712693in}{2.546603in}}%
\pgfusepath{use as bounding box, clip}%
\begin{pgfscope}%
\pgfsetbuttcap%
\pgfsetmiterjoin%
\definecolor{currentfill}{rgb}{1.000000,1.000000,1.000000}%
\pgfsetfillcolor{currentfill}%
\pgfsetlinewidth{0.000000pt}%
\definecolor{currentstroke}{rgb}{1.000000,1.000000,1.000000}%
\pgfsetstrokecolor{currentstroke}%
\pgfsetdash{}{0pt}%
\pgfpathmoveto{\pgfqpoint{0.000000in}{0.000000in}}%
\pgfpathlineto{\pgfqpoint{2.712693in}{0.000000in}}%
\pgfpathlineto{\pgfqpoint{2.712693in}{2.546603in}}%
\pgfpathlineto{\pgfqpoint{0.000000in}{2.546603in}}%
\pgfpathlineto{\pgfqpoint{0.000000in}{0.000000in}}%
\pgfpathclose%
\pgfusepath{fill}%
\end{pgfscope}%
\begin{pgfscope}%
\pgfsetbuttcap%
\pgfsetmiterjoin%
\definecolor{currentfill}{rgb}{1.000000,1.000000,1.000000}%
\pgfsetfillcolor{currentfill}%
\pgfsetlinewidth{0.000000pt}%
\definecolor{currentstroke}{rgb}{0.000000,0.000000,0.000000}%
\pgfsetstrokecolor{currentstroke}%
\pgfsetstrokeopacity{0.000000}%
\pgfsetdash{}{0pt}%
\pgfpathmoveto{\pgfqpoint{0.675193in}{0.521603in}}%
\pgfpathlineto{\pgfqpoint{2.612693in}{0.521603in}}%
\pgfpathlineto{\pgfqpoint{2.612693in}{2.446603in}}%
\pgfpathlineto{\pgfqpoint{0.675193in}{2.446603in}}%
\pgfpathlineto{\pgfqpoint{0.675193in}{0.521603in}}%
\pgfpathclose%
\pgfusepath{fill}%
\end{pgfscope}%
\begin{pgfscope}%
\pgfsetbuttcap%
\pgfsetroundjoin%
\definecolor{currentfill}{rgb}{0.000000,0.000000,0.000000}%
\pgfsetfillcolor{currentfill}%
\pgfsetlinewidth{0.803000pt}%
\definecolor{currentstroke}{rgb}{0.000000,0.000000,0.000000}%
\pgfsetstrokecolor{currentstroke}%
\pgfsetdash{}{0pt}%
\pgfsys@defobject{currentmarker}{\pgfqpoint{0.000000in}{-0.048611in}}{\pgfqpoint{0.000000in}{0.000000in}}{%
\pgfpathmoveto{\pgfqpoint{0.000000in}{0.000000in}}%
\pgfpathlineto{\pgfqpoint{0.000000in}{-0.048611in}}%
\pgfusepath{stroke,fill}%
}%
\begin{pgfscope}%
\pgfsys@transformshift{1.724845in}{0.521603in}%
\pgfsys@useobject{currentmarker}{}%
\end{pgfscope}%
\end{pgfscope}%
\begin{pgfscope}%
\definecolor{textcolor}{rgb}{0.000000,0.000000,0.000000}%
\pgfsetstrokecolor{textcolor}%
\pgfsetfillcolor{textcolor}%
\pgftext[x=1.724845in,y=0.424381in,,top]{\color{textcolor}{\sffamily\fontsize{10.000000}{12.000000}\selectfont\catcode`\^=\active\def^{\ifmmode\sp\else\^{}\fi}\catcode`\%=\active\def%{\%}$\mathdefault{10^{2}}$}}%
\end{pgfscope}%
\begin{pgfscope}%
\pgfsetbuttcap%
\pgfsetroundjoin%
\definecolor{currentfill}{rgb}{0.000000,0.000000,0.000000}%
\pgfsetfillcolor{currentfill}%
\pgfsetlinewidth{0.602250pt}%
\definecolor{currentstroke}{rgb}{0.000000,0.000000,0.000000}%
\pgfsetstrokecolor{currentstroke}%
\pgfsetdash{}{0pt}%
\pgfsys@defobject{currentmarker}{\pgfqpoint{0.000000in}{-0.027778in}}{\pgfqpoint{0.000000in}{0.000000in}}{%
\pgfpathmoveto{\pgfqpoint{0.000000in}{0.000000in}}%
\pgfpathlineto{\pgfqpoint{0.000000in}{-0.027778in}}%
\pgfusepath{stroke,fill}%
}%
\begin{pgfscope}%
\pgfsys@transformshift{0.792961in}{0.521603in}%
\pgfsys@useobject{currentmarker}{}%
\end{pgfscope}%
\end{pgfscope}%
\begin{pgfscope}%
\pgfsetbuttcap%
\pgfsetroundjoin%
\definecolor{currentfill}{rgb}{0.000000,0.000000,0.000000}%
\pgfsetfillcolor{currentfill}%
\pgfsetlinewidth{0.602250pt}%
\definecolor{currentstroke}{rgb}{0.000000,0.000000,0.000000}%
\pgfsetstrokecolor{currentstroke}%
\pgfsetdash{}{0pt}%
\pgfsys@defobject{currentmarker}{\pgfqpoint{0.000000in}{-0.027778in}}{\pgfqpoint{0.000000in}{0.000000in}}{%
\pgfpathmoveto{\pgfqpoint{0.000000in}{0.000000in}}%
\pgfpathlineto{\pgfqpoint{0.000000in}{-0.027778in}}%
\pgfusepath{stroke,fill}%
}%
\begin{pgfscope}%
\pgfsys@transformshift{1.027730in}{0.521603in}%
\pgfsys@useobject{currentmarker}{}%
\end{pgfscope}%
\end{pgfscope}%
\begin{pgfscope}%
\pgfsetbuttcap%
\pgfsetroundjoin%
\definecolor{currentfill}{rgb}{0.000000,0.000000,0.000000}%
\pgfsetfillcolor{currentfill}%
\pgfsetlinewidth{0.602250pt}%
\definecolor{currentstroke}{rgb}{0.000000,0.000000,0.000000}%
\pgfsetstrokecolor{currentstroke}%
\pgfsetdash{}{0pt}%
\pgfsys@defobject{currentmarker}{\pgfqpoint{0.000000in}{-0.027778in}}{\pgfqpoint{0.000000in}{0.000000in}}{%
\pgfpathmoveto{\pgfqpoint{0.000000in}{0.000000in}}%
\pgfpathlineto{\pgfqpoint{0.000000in}{-0.027778in}}%
\pgfusepath{stroke,fill}%
}%
\begin{pgfscope}%
\pgfsys@transformshift{1.194302in}{0.521603in}%
\pgfsys@useobject{currentmarker}{}%
\end{pgfscope}%
\end{pgfscope}%
\begin{pgfscope}%
\pgfsetbuttcap%
\pgfsetroundjoin%
\definecolor{currentfill}{rgb}{0.000000,0.000000,0.000000}%
\pgfsetfillcolor{currentfill}%
\pgfsetlinewidth{0.602250pt}%
\definecolor{currentstroke}{rgb}{0.000000,0.000000,0.000000}%
\pgfsetstrokecolor{currentstroke}%
\pgfsetdash{}{0pt}%
\pgfsys@defobject{currentmarker}{\pgfqpoint{0.000000in}{-0.027778in}}{\pgfqpoint{0.000000in}{0.000000in}}{%
\pgfpathmoveto{\pgfqpoint{0.000000in}{0.000000in}}%
\pgfpathlineto{\pgfqpoint{0.000000in}{-0.027778in}}%
\pgfusepath{stroke,fill}%
}%
\begin{pgfscope}%
\pgfsys@transformshift{1.323505in}{0.521603in}%
\pgfsys@useobject{currentmarker}{}%
\end{pgfscope}%
\end{pgfscope}%
\begin{pgfscope}%
\pgfsetbuttcap%
\pgfsetroundjoin%
\definecolor{currentfill}{rgb}{0.000000,0.000000,0.000000}%
\pgfsetfillcolor{currentfill}%
\pgfsetlinewidth{0.602250pt}%
\definecolor{currentstroke}{rgb}{0.000000,0.000000,0.000000}%
\pgfsetstrokecolor{currentstroke}%
\pgfsetdash{}{0pt}%
\pgfsys@defobject{currentmarker}{\pgfqpoint{0.000000in}{-0.027778in}}{\pgfqpoint{0.000000in}{0.000000in}}{%
\pgfpathmoveto{\pgfqpoint{0.000000in}{0.000000in}}%
\pgfpathlineto{\pgfqpoint{0.000000in}{-0.027778in}}%
\pgfusepath{stroke,fill}%
}%
\begin{pgfscope}%
\pgfsys@transformshift{1.429071in}{0.521603in}%
\pgfsys@useobject{currentmarker}{}%
\end{pgfscope}%
\end{pgfscope}%
\begin{pgfscope}%
\pgfsetbuttcap%
\pgfsetroundjoin%
\definecolor{currentfill}{rgb}{0.000000,0.000000,0.000000}%
\pgfsetfillcolor{currentfill}%
\pgfsetlinewidth{0.602250pt}%
\definecolor{currentstroke}{rgb}{0.000000,0.000000,0.000000}%
\pgfsetstrokecolor{currentstroke}%
\pgfsetdash{}{0pt}%
\pgfsys@defobject{currentmarker}{\pgfqpoint{0.000000in}{-0.027778in}}{\pgfqpoint{0.000000in}{0.000000in}}{%
\pgfpathmoveto{\pgfqpoint{0.000000in}{0.000000in}}%
\pgfpathlineto{\pgfqpoint{0.000000in}{-0.027778in}}%
\pgfusepath{stroke,fill}%
}%
\begin{pgfscope}%
\pgfsys@transformshift{1.518326in}{0.521603in}%
\pgfsys@useobject{currentmarker}{}%
\end{pgfscope}%
\end{pgfscope}%
\begin{pgfscope}%
\pgfsetbuttcap%
\pgfsetroundjoin%
\definecolor{currentfill}{rgb}{0.000000,0.000000,0.000000}%
\pgfsetfillcolor{currentfill}%
\pgfsetlinewidth{0.602250pt}%
\definecolor{currentstroke}{rgb}{0.000000,0.000000,0.000000}%
\pgfsetstrokecolor{currentstroke}%
\pgfsetdash{}{0pt}%
\pgfsys@defobject{currentmarker}{\pgfqpoint{0.000000in}{-0.027778in}}{\pgfqpoint{0.000000in}{0.000000in}}{%
\pgfpathmoveto{\pgfqpoint{0.000000in}{0.000000in}}%
\pgfpathlineto{\pgfqpoint{0.000000in}{-0.027778in}}%
\pgfusepath{stroke,fill}%
}%
\begin{pgfscope}%
\pgfsys@transformshift{1.595642in}{0.521603in}%
\pgfsys@useobject{currentmarker}{}%
\end{pgfscope}%
\end{pgfscope}%
\begin{pgfscope}%
\pgfsetbuttcap%
\pgfsetroundjoin%
\definecolor{currentfill}{rgb}{0.000000,0.000000,0.000000}%
\pgfsetfillcolor{currentfill}%
\pgfsetlinewidth{0.602250pt}%
\definecolor{currentstroke}{rgb}{0.000000,0.000000,0.000000}%
\pgfsetstrokecolor{currentstroke}%
\pgfsetdash{}{0pt}%
\pgfsys@defobject{currentmarker}{\pgfqpoint{0.000000in}{-0.027778in}}{\pgfqpoint{0.000000in}{0.000000in}}{%
\pgfpathmoveto{\pgfqpoint{0.000000in}{0.000000in}}%
\pgfpathlineto{\pgfqpoint{0.000000in}{-0.027778in}}%
\pgfusepath{stroke,fill}%
}%
\begin{pgfscope}%
\pgfsys@transformshift{1.663840in}{0.521603in}%
\pgfsys@useobject{currentmarker}{}%
\end{pgfscope}%
\end{pgfscope}%
\begin{pgfscope}%
\pgfsetbuttcap%
\pgfsetroundjoin%
\definecolor{currentfill}{rgb}{0.000000,0.000000,0.000000}%
\pgfsetfillcolor{currentfill}%
\pgfsetlinewidth{0.602250pt}%
\definecolor{currentstroke}{rgb}{0.000000,0.000000,0.000000}%
\pgfsetstrokecolor{currentstroke}%
\pgfsetdash{}{0pt}%
\pgfsys@defobject{currentmarker}{\pgfqpoint{0.000000in}{-0.027778in}}{\pgfqpoint{0.000000in}{0.000000in}}{%
\pgfpathmoveto{\pgfqpoint{0.000000in}{0.000000in}}%
\pgfpathlineto{\pgfqpoint{0.000000in}{-0.027778in}}%
\pgfusepath{stroke,fill}%
}%
\begin{pgfscope}%
\pgfsys@transformshift{2.126186in}{0.521603in}%
\pgfsys@useobject{currentmarker}{}%
\end{pgfscope}%
\end{pgfscope}%
\begin{pgfscope}%
\pgfsetbuttcap%
\pgfsetroundjoin%
\definecolor{currentfill}{rgb}{0.000000,0.000000,0.000000}%
\pgfsetfillcolor{currentfill}%
\pgfsetlinewidth{0.602250pt}%
\definecolor{currentstroke}{rgb}{0.000000,0.000000,0.000000}%
\pgfsetstrokecolor{currentstroke}%
\pgfsetdash{}{0pt}%
\pgfsys@defobject{currentmarker}{\pgfqpoint{0.000000in}{-0.027778in}}{\pgfqpoint{0.000000in}{0.000000in}}{%
\pgfpathmoveto{\pgfqpoint{0.000000in}{0.000000in}}%
\pgfpathlineto{\pgfqpoint{0.000000in}{-0.027778in}}%
\pgfusepath{stroke,fill}%
}%
\begin{pgfscope}%
\pgfsys@transformshift{2.360956in}{0.521603in}%
\pgfsys@useobject{currentmarker}{}%
\end{pgfscope}%
\end{pgfscope}%
\begin{pgfscope}%
\pgfsetbuttcap%
\pgfsetroundjoin%
\definecolor{currentfill}{rgb}{0.000000,0.000000,0.000000}%
\pgfsetfillcolor{currentfill}%
\pgfsetlinewidth{0.602250pt}%
\definecolor{currentstroke}{rgb}{0.000000,0.000000,0.000000}%
\pgfsetstrokecolor{currentstroke}%
\pgfsetdash{}{0pt}%
\pgfsys@defobject{currentmarker}{\pgfqpoint{0.000000in}{-0.027778in}}{\pgfqpoint{0.000000in}{0.000000in}}{%
\pgfpathmoveto{\pgfqpoint{0.000000in}{0.000000in}}%
\pgfpathlineto{\pgfqpoint{0.000000in}{-0.027778in}}%
\pgfusepath{stroke,fill}%
}%
\begin{pgfscope}%
\pgfsys@transformshift{2.527527in}{0.521603in}%
\pgfsys@useobject{currentmarker}{}%
\end{pgfscope}%
\end{pgfscope}%
\begin{pgfscope}%
\definecolor{textcolor}{rgb}{0.000000,0.000000,0.000000}%
\pgfsetstrokecolor{textcolor}%
\pgfsetfillcolor{textcolor}%
\pgftext[x=1.643943in,y=0.234413in,,top]{\color{textcolor}{\sffamily\fontsize{10.000000}{12.000000}\selectfont\catcode`\^=\active\def^{\ifmmode\sp\else\^{}\fi}\catcode`\%=\active\def%{\%}$n_v$}}%
\end{pgfscope}%
\begin{pgfscope}%
\pgfsetbuttcap%
\pgfsetroundjoin%
\definecolor{currentfill}{rgb}{0.000000,0.000000,0.000000}%
\pgfsetfillcolor{currentfill}%
\pgfsetlinewidth{0.803000pt}%
\definecolor{currentstroke}{rgb}{0.000000,0.000000,0.000000}%
\pgfsetstrokecolor{currentstroke}%
\pgfsetdash{}{0pt}%
\pgfsys@defobject{currentmarker}{\pgfqpoint{-0.048611in}{0.000000in}}{\pgfqpoint{-0.000000in}{0.000000in}}{%
\pgfpathmoveto{\pgfqpoint{-0.000000in}{0.000000in}}%
\pgfpathlineto{\pgfqpoint{-0.048611in}{0.000000in}}%
\pgfusepath{stroke,fill}%
}%
\begin{pgfscope}%
\pgfsys@transformshift{0.675193in}{1.728051in}%
\pgfsys@useobject{currentmarker}{}%
\end{pgfscope}%
\end{pgfscope}%
\begin{pgfscope}%
\definecolor{textcolor}{rgb}{0.000000,0.000000,0.000000}%
\pgfsetstrokecolor{textcolor}%
\pgfsetfillcolor{textcolor}%
\pgftext[x=0.289968in, y=1.675289in, left, base]{\color{textcolor}{\sffamily\fontsize{10.000000}{12.000000}\selectfont\catcode`\^=\active\def^{\ifmmode\sp\else\^{}\fi}\catcode`\%=\active\def%{\%}$\mathdefault{10^{-1}}$}}%
\end{pgfscope}%
\begin{pgfscope}%
\pgfsetbuttcap%
\pgfsetroundjoin%
\definecolor{currentfill}{rgb}{0.000000,0.000000,0.000000}%
\pgfsetfillcolor{currentfill}%
\pgfsetlinewidth{0.602250pt}%
\definecolor{currentstroke}{rgb}{0.000000,0.000000,0.000000}%
\pgfsetstrokecolor{currentstroke}%
\pgfsetdash{}{0pt}%
\pgfsys@defobject{currentmarker}{\pgfqpoint{-0.027778in}{0.000000in}}{\pgfqpoint{-0.000000in}{0.000000in}}{%
\pgfpathmoveto{\pgfqpoint{-0.000000in}{0.000000in}}%
\pgfpathlineto{\pgfqpoint{-0.027778in}{0.000000in}}%
\pgfusepath{stroke,fill}%
}%
\begin{pgfscope}%
\pgfsys@transformshift{0.675193in}{0.584383in}%
\pgfsys@useobject{currentmarker}{}%
\end{pgfscope}%
\end{pgfscope}%
\begin{pgfscope}%
\pgfsetbuttcap%
\pgfsetroundjoin%
\definecolor{currentfill}{rgb}{0.000000,0.000000,0.000000}%
\pgfsetfillcolor{currentfill}%
\pgfsetlinewidth{0.602250pt}%
\definecolor{currentstroke}{rgb}{0.000000,0.000000,0.000000}%
\pgfsetstrokecolor{currentstroke}%
\pgfsetdash{}{0pt}%
\pgfsys@defobject{currentmarker}{\pgfqpoint{-0.027778in}{0.000000in}}{\pgfqpoint{-0.000000in}{0.000000in}}{%
\pgfpathmoveto{\pgfqpoint{-0.000000in}{0.000000in}}%
\pgfpathlineto{\pgfqpoint{-0.027778in}{0.000000in}}%
\pgfusepath{stroke,fill}%
}%
\begin{pgfscope}%
\pgfsys@transformshift{0.675193in}{0.872506in}%
\pgfsys@useobject{currentmarker}{}%
\end{pgfscope}%
\end{pgfscope}%
\begin{pgfscope}%
\pgfsetbuttcap%
\pgfsetroundjoin%
\definecolor{currentfill}{rgb}{0.000000,0.000000,0.000000}%
\pgfsetfillcolor{currentfill}%
\pgfsetlinewidth{0.602250pt}%
\definecolor{currentstroke}{rgb}{0.000000,0.000000,0.000000}%
\pgfsetstrokecolor{currentstroke}%
\pgfsetdash{}{0pt}%
\pgfsys@defobject{currentmarker}{\pgfqpoint{-0.027778in}{0.000000in}}{\pgfqpoint{-0.000000in}{0.000000in}}{%
\pgfpathmoveto{\pgfqpoint{-0.000000in}{0.000000in}}%
\pgfpathlineto{\pgfqpoint{-0.027778in}{0.000000in}}%
\pgfusepath{stroke,fill}%
}%
\begin{pgfscope}%
\pgfsys@transformshift{0.675193in}{1.076934in}%
\pgfsys@useobject{currentmarker}{}%
\end{pgfscope}%
\end{pgfscope}%
\begin{pgfscope}%
\pgfsetbuttcap%
\pgfsetroundjoin%
\definecolor{currentfill}{rgb}{0.000000,0.000000,0.000000}%
\pgfsetfillcolor{currentfill}%
\pgfsetlinewidth{0.602250pt}%
\definecolor{currentstroke}{rgb}{0.000000,0.000000,0.000000}%
\pgfsetstrokecolor{currentstroke}%
\pgfsetdash{}{0pt}%
\pgfsys@defobject{currentmarker}{\pgfqpoint{-0.027778in}{0.000000in}}{\pgfqpoint{-0.000000in}{0.000000in}}{%
\pgfpathmoveto{\pgfqpoint{-0.000000in}{0.000000in}}%
\pgfpathlineto{\pgfqpoint{-0.027778in}{0.000000in}}%
\pgfusepath{stroke,fill}%
}%
\begin{pgfscope}%
\pgfsys@transformshift{0.675193in}{1.235500in}%
\pgfsys@useobject{currentmarker}{}%
\end{pgfscope}%
\end{pgfscope}%
\begin{pgfscope}%
\pgfsetbuttcap%
\pgfsetroundjoin%
\definecolor{currentfill}{rgb}{0.000000,0.000000,0.000000}%
\pgfsetfillcolor{currentfill}%
\pgfsetlinewidth{0.602250pt}%
\definecolor{currentstroke}{rgb}{0.000000,0.000000,0.000000}%
\pgfsetstrokecolor{currentstroke}%
\pgfsetdash{}{0pt}%
\pgfsys@defobject{currentmarker}{\pgfqpoint{-0.027778in}{0.000000in}}{\pgfqpoint{-0.000000in}{0.000000in}}{%
\pgfpathmoveto{\pgfqpoint{-0.000000in}{0.000000in}}%
\pgfpathlineto{\pgfqpoint{-0.027778in}{0.000000in}}%
\pgfusepath{stroke,fill}%
}%
\begin{pgfscope}%
\pgfsys@transformshift{0.675193in}{1.365057in}%
\pgfsys@useobject{currentmarker}{}%
\end{pgfscope}%
\end{pgfscope}%
\begin{pgfscope}%
\pgfsetbuttcap%
\pgfsetroundjoin%
\definecolor{currentfill}{rgb}{0.000000,0.000000,0.000000}%
\pgfsetfillcolor{currentfill}%
\pgfsetlinewidth{0.602250pt}%
\definecolor{currentstroke}{rgb}{0.000000,0.000000,0.000000}%
\pgfsetstrokecolor{currentstroke}%
\pgfsetdash{}{0pt}%
\pgfsys@defobject{currentmarker}{\pgfqpoint{-0.027778in}{0.000000in}}{\pgfqpoint{-0.000000in}{0.000000in}}{%
\pgfpathmoveto{\pgfqpoint{-0.000000in}{0.000000in}}%
\pgfpathlineto{\pgfqpoint{-0.027778in}{0.000000in}}%
\pgfusepath{stroke,fill}%
}%
\begin{pgfscope}%
\pgfsys@transformshift{0.675193in}{1.474597in}%
\pgfsys@useobject{currentmarker}{}%
\end{pgfscope}%
\end{pgfscope}%
\begin{pgfscope}%
\pgfsetbuttcap%
\pgfsetroundjoin%
\definecolor{currentfill}{rgb}{0.000000,0.000000,0.000000}%
\pgfsetfillcolor{currentfill}%
\pgfsetlinewidth{0.602250pt}%
\definecolor{currentstroke}{rgb}{0.000000,0.000000,0.000000}%
\pgfsetstrokecolor{currentstroke}%
\pgfsetdash{}{0pt}%
\pgfsys@defobject{currentmarker}{\pgfqpoint{-0.027778in}{0.000000in}}{\pgfqpoint{-0.000000in}{0.000000in}}{%
\pgfpathmoveto{\pgfqpoint{-0.000000in}{0.000000in}}%
\pgfpathlineto{\pgfqpoint{-0.027778in}{0.000000in}}%
\pgfusepath{stroke,fill}%
}%
\begin{pgfscope}%
\pgfsys@transformshift{0.675193in}{1.569485in}%
\pgfsys@useobject{currentmarker}{}%
\end{pgfscope}%
\end{pgfscope}%
\begin{pgfscope}%
\pgfsetbuttcap%
\pgfsetroundjoin%
\definecolor{currentfill}{rgb}{0.000000,0.000000,0.000000}%
\pgfsetfillcolor{currentfill}%
\pgfsetlinewidth{0.602250pt}%
\definecolor{currentstroke}{rgb}{0.000000,0.000000,0.000000}%
\pgfsetstrokecolor{currentstroke}%
\pgfsetdash{}{0pt}%
\pgfsys@defobject{currentmarker}{\pgfqpoint{-0.027778in}{0.000000in}}{\pgfqpoint{-0.000000in}{0.000000in}}{%
\pgfpathmoveto{\pgfqpoint{-0.000000in}{0.000000in}}%
\pgfpathlineto{\pgfqpoint{-0.027778in}{0.000000in}}%
\pgfusepath{stroke,fill}%
}%
\begin{pgfscope}%
\pgfsys@transformshift{0.675193in}{1.653181in}%
\pgfsys@useobject{currentmarker}{}%
\end{pgfscope}%
\end{pgfscope}%
\begin{pgfscope}%
\pgfsetbuttcap%
\pgfsetroundjoin%
\definecolor{currentfill}{rgb}{0.000000,0.000000,0.000000}%
\pgfsetfillcolor{currentfill}%
\pgfsetlinewidth{0.602250pt}%
\definecolor{currentstroke}{rgb}{0.000000,0.000000,0.000000}%
\pgfsetstrokecolor{currentstroke}%
\pgfsetdash{}{0pt}%
\pgfsys@defobject{currentmarker}{\pgfqpoint{-0.027778in}{0.000000in}}{\pgfqpoint{-0.000000in}{0.000000in}}{%
\pgfpathmoveto{\pgfqpoint{-0.000000in}{0.000000in}}%
\pgfpathlineto{\pgfqpoint{-0.027778in}{0.000000in}}%
\pgfusepath{stroke,fill}%
}%
\begin{pgfscope}%
\pgfsys@transformshift{0.675193in}{2.220601in}%
\pgfsys@useobject{currentmarker}{}%
\end{pgfscope}%
\end{pgfscope}%
\begin{pgfscope}%
\definecolor{textcolor}{rgb}{0.000000,0.000000,0.000000}%
\pgfsetstrokecolor{textcolor}%
\pgfsetfillcolor{textcolor}%
\pgftext[x=0.234413in,y=1.484103in,,bottom,rotate=90.000000]{\color{textcolor}{\sffamily\fontsize{10.000000}{12.000000}\selectfont\catcode`\^=\active\def^{\ifmmode\sp\else\^{}\fi}\catcode`\%=\active\def%{\%}$L^1$ error}}%
\end{pgfscope}%
\begin{pgfscope}%
\pgfpathrectangle{\pgfqpoint{0.675193in}{0.521603in}}{\pgfqpoint{1.937500in}{1.925000in}}%
\pgfusepath{clip}%
\pgfsetrectcap%
\pgfsetroundjoin%
\pgfsetlinewidth{1.003750pt}%
\definecolor{currentstroke}{rgb}{0.001462,0.000466,0.013866}%
\pgfsetstrokecolor{currentstroke}%
\pgfsetdash{}{0pt}%
\pgfpathmoveto{\pgfqpoint{0.763261in}{1.425969in}}%
\pgfpathlineto{\pgfqpoint{1.133297in}{1.423297in}}%
\pgfpathlineto{\pgfqpoint{1.484257in}{1.265698in}}%
\pgfpathlineto{\pgfqpoint{1.830412in}{1.075945in}}%
\pgfpathlineto{\pgfqpoint{2.176084in}{0.962804in}}%
\pgfpathlineto{\pgfqpoint{2.524625in}{0.841212in}}%
\pgfusepath{stroke}%
\end{pgfscope}%
\begin{pgfscope}%
\pgfpathrectangle{\pgfqpoint{0.675193in}{0.521603in}}{\pgfqpoint{1.937500in}{1.925000in}}%
\pgfusepath{clip}%
\pgfsetbuttcap%
\pgfsetroundjoin%
\definecolor{currentfill}{rgb}{0.001462,0.000466,0.013866}%
\pgfsetfillcolor{currentfill}%
\pgfsetlinewidth{1.003750pt}%
\definecolor{currentstroke}{rgb}{0.001462,0.000466,0.013866}%
\pgfsetstrokecolor{currentstroke}%
\pgfsetdash{}{0pt}%
\pgfsys@defobject{currentmarker}{\pgfqpoint{-0.020833in}{-0.020833in}}{\pgfqpoint{0.020833in}{0.020833in}}{%
\pgfpathmoveto{\pgfqpoint{0.000000in}{-0.020833in}}%
\pgfpathcurveto{\pgfqpoint{0.005525in}{-0.020833in}}{\pgfqpoint{0.010825in}{-0.018638in}}{\pgfqpoint{0.014731in}{-0.014731in}}%
\pgfpathcurveto{\pgfqpoint{0.018638in}{-0.010825in}}{\pgfqpoint{0.020833in}{-0.005525in}}{\pgfqpoint{0.020833in}{0.000000in}}%
\pgfpathcurveto{\pgfqpoint{0.020833in}{0.005525in}}{\pgfqpoint{0.018638in}{0.010825in}}{\pgfqpoint{0.014731in}{0.014731in}}%
\pgfpathcurveto{\pgfqpoint{0.010825in}{0.018638in}}{\pgfqpoint{0.005525in}{0.020833in}}{\pgfqpoint{0.000000in}{0.020833in}}%
\pgfpathcurveto{\pgfqpoint{-0.005525in}{0.020833in}}{\pgfqpoint{-0.010825in}{0.018638in}}{\pgfqpoint{-0.014731in}{0.014731in}}%
\pgfpathcurveto{\pgfqpoint{-0.018638in}{0.010825in}}{\pgfqpoint{-0.020833in}{0.005525in}}{\pgfqpoint{-0.020833in}{0.000000in}}%
\pgfpathcurveto{\pgfqpoint{-0.020833in}{-0.005525in}}{\pgfqpoint{-0.018638in}{-0.010825in}}{\pgfqpoint{-0.014731in}{-0.014731in}}%
\pgfpathcurveto{\pgfqpoint{-0.010825in}{-0.018638in}}{\pgfqpoint{-0.005525in}{-0.020833in}}{\pgfqpoint{0.000000in}{-0.020833in}}%
\pgfpathlineto{\pgfqpoint{0.000000in}{-0.020833in}}%
\pgfpathclose%
\pgfusepath{stroke,fill}%
}%
\begin{pgfscope}%
\pgfsys@transformshift{0.763261in}{1.425969in}%
\pgfsys@useobject{currentmarker}{}%
\end{pgfscope}%
\begin{pgfscope}%
\pgfsys@transformshift{1.133297in}{1.423297in}%
\pgfsys@useobject{currentmarker}{}%
\end{pgfscope}%
\begin{pgfscope}%
\pgfsys@transformshift{1.484257in}{1.265698in}%
\pgfsys@useobject{currentmarker}{}%
\end{pgfscope}%
\begin{pgfscope}%
\pgfsys@transformshift{1.830412in}{1.075945in}%
\pgfsys@useobject{currentmarker}{}%
\end{pgfscope}%
\begin{pgfscope}%
\pgfsys@transformshift{2.176084in}{0.962804in}%
\pgfsys@useobject{currentmarker}{}%
\end{pgfscope}%
\begin{pgfscope}%
\pgfsys@transformshift{2.524625in}{0.841212in}%
\pgfsys@useobject{currentmarker}{}%
\end{pgfscope}%
\end{pgfscope}%
\begin{pgfscope}%
\pgfpathrectangle{\pgfqpoint{0.675193in}{0.521603in}}{\pgfqpoint{1.937500in}{1.925000in}}%
\pgfusepath{clip}%
\pgfsetrectcap%
\pgfsetroundjoin%
\pgfsetlinewidth{1.003750pt}%
\definecolor{currentstroke}{rgb}{0.445163,0.122724,0.506901}%
\pgfsetstrokecolor{currentstroke}%
\pgfsetdash{}{0pt}%
\pgfpathmoveto{\pgfqpoint{0.763261in}{2.359103in}}%
\pgfpathlineto{\pgfqpoint{1.133297in}{1.795103in}}%
\pgfpathlineto{\pgfqpoint{1.484257in}{1.351405in}}%
\pgfpathlineto{\pgfqpoint{1.830412in}{1.419822in}}%
\pgfpathlineto{\pgfqpoint{2.176084in}{1.563555in}}%
\pgfpathlineto{\pgfqpoint{2.524625in}{1.644265in}}%
\pgfusepath{stroke}%
\end{pgfscope}%
\begin{pgfscope}%
\pgfpathrectangle{\pgfqpoint{0.675193in}{0.521603in}}{\pgfqpoint{1.937500in}{1.925000in}}%
\pgfusepath{clip}%
\pgfsetbuttcap%
\pgfsetroundjoin%
\definecolor{currentfill}{rgb}{0.445163,0.122724,0.506901}%
\pgfsetfillcolor{currentfill}%
\pgfsetlinewidth{1.003750pt}%
\definecolor{currentstroke}{rgb}{0.445163,0.122724,0.506901}%
\pgfsetstrokecolor{currentstroke}%
\pgfsetdash{}{0pt}%
\pgfsys@defobject{currentmarker}{\pgfqpoint{-0.020833in}{-0.020833in}}{\pgfqpoint{0.020833in}{0.020833in}}{%
\pgfpathmoveto{\pgfqpoint{0.000000in}{-0.020833in}}%
\pgfpathcurveto{\pgfqpoint{0.005525in}{-0.020833in}}{\pgfqpoint{0.010825in}{-0.018638in}}{\pgfqpoint{0.014731in}{-0.014731in}}%
\pgfpathcurveto{\pgfqpoint{0.018638in}{-0.010825in}}{\pgfqpoint{0.020833in}{-0.005525in}}{\pgfqpoint{0.020833in}{0.000000in}}%
\pgfpathcurveto{\pgfqpoint{0.020833in}{0.005525in}}{\pgfqpoint{0.018638in}{0.010825in}}{\pgfqpoint{0.014731in}{0.014731in}}%
\pgfpathcurveto{\pgfqpoint{0.010825in}{0.018638in}}{\pgfqpoint{0.005525in}{0.020833in}}{\pgfqpoint{0.000000in}{0.020833in}}%
\pgfpathcurveto{\pgfqpoint{-0.005525in}{0.020833in}}{\pgfqpoint{-0.010825in}{0.018638in}}{\pgfqpoint{-0.014731in}{0.014731in}}%
\pgfpathcurveto{\pgfqpoint{-0.018638in}{0.010825in}}{\pgfqpoint{-0.020833in}{0.005525in}}{\pgfqpoint{-0.020833in}{0.000000in}}%
\pgfpathcurveto{\pgfqpoint{-0.020833in}{-0.005525in}}{\pgfqpoint{-0.018638in}{-0.010825in}}{\pgfqpoint{-0.014731in}{-0.014731in}}%
\pgfpathcurveto{\pgfqpoint{-0.010825in}{-0.018638in}}{\pgfqpoint{-0.005525in}{-0.020833in}}{\pgfqpoint{0.000000in}{-0.020833in}}%
\pgfpathlineto{\pgfqpoint{0.000000in}{-0.020833in}}%
\pgfpathclose%
\pgfusepath{stroke,fill}%
}%
\begin{pgfscope}%
\pgfsys@transformshift{0.763261in}{2.359103in}%
\pgfsys@useobject{currentmarker}{}%
\end{pgfscope}%
\begin{pgfscope}%
\pgfsys@transformshift{1.133297in}{1.795103in}%
\pgfsys@useobject{currentmarker}{}%
\end{pgfscope}%
\begin{pgfscope}%
\pgfsys@transformshift{1.484257in}{1.351405in}%
\pgfsys@useobject{currentmarker}{}%
\end{pgfscope}%
\begin{pgfscope}%
\pgfsys@transformshift{1.830412in}{1.419822in}%
\pgfsys@useobject{currentmarker}{}%
\end{pgfscope}%
\begin{pgfscope}%
\pgfsys@transformshift{2.176084in}{1.563555in}%
\pgfsys@useobject{currentmarker}{}%
\end{pgfscope}%
\begin{pgfscope}%
\pgfsys@transformshift{2.524625in}{1.644265in}%
\pgfsys@useobject{currentmarker}{}%
\end{pgfscope}%
\end{pgfscope}%
\begin{pgfscope}%
\pgfpathrectangle{\pgfqpoint{0.675193in}{0.521603in}}{\pgfqpoint{1.937500in}{1.925000in}}%
\pgfusepath{clip}%
\pgfsetrectcap%
\pgfsetroundjoin%
\pgfsetlinewidth{1.003750pt}%
\definecolor{currentstroke}{rgb}{0.944006,0.377643,0.365136}%
\pgfsetstrokecolor{currentstroke}%
\pgfsetdash{}{0pt}%
\pgfpathmoveto{\pgfqpoint{0.763261in}{1.453881in}}%
\pgfpathlineto{\pgfqpoint{1.133297in}{0.719228in}}%
\pgfpathlineto{\pgfqpoint{1.484257in}{0.772685in}}%
\pgfpathlineto{\pgfqpoint{1.830412in}{0.651025in}}%
\pgfpathlineto{\pgfqpoint{2.176084in}{0.662282in}}%
\pgfpathlineto{\pgfqpoint{2.524625in}{0.609103in}}%
\pgfusepath{stroke}%
\end{pgfscope}%
\begin{pgfscope}%
\pgfpathrectangle{\pgfqpoint{0.675193in}{0.521603in}}{\pgfqpoint{1.937500in}{1.925000in}}%
\pgfusepath{clip}%
\pgfsetbuttcap%
\pgfsetroundjoin%
\definecolor{currentfill}{rgb}{0.944006,0.377643,0.365136}%
\pgfsetfillcolor{currentfill}%
\pgfsetlinewidth{1.003750pt}%
\definecolor{currentstroke}{rgb}{0.944006,0.377643,0.365136}%
\pgfsetstrokecolor{currentstroke}%
\pgfsetdash{}{0pt}%
\pgfsys@defobject{currentmarker}{\pgfqpoint{-0.020833in}{-0.020833in}}{\pgfqpoint{0.020833in}{0.020833in}}{%
\pgfpathmoveto{\pgfqpoint{0.000000in}{-0.020833in}}%
\pgfpathcurveto{\pgfqpoint{0.005525in}{-0.020833in}}{\pgfqpoint{0.010825in}{-0.018638in}}{\pgfqpoint{0.014731in}{-0.014731in}}%
\pgfpathcurveto{\pgfqpoint{0.018638in}{-0.010825in}}{\pgfqpoint{0.020833in}{-0.005525in}}{\pgfqpoint{0.020833in}{0.000000in}}%
\pgfpathcurveto{\pgfqpoint{0.020833in}{0.005525in}}{\pgfqpoint{0.018638in}{0.010825in}}{\pgfqpoint{0.014731in}{0.014731in}}%
\pgfpathcurveto{\pgfqpoint{0.010825in}{0.018638in}}{\pgfqpoint{0.005525in}{0.020833in}}{\pgfqpoint{0.000000in}{0.020833in}}%
\pgfpathcurveto{\pgfqpoint{-0.005525in}{0.020833in}}{\pgfqpoint{-0.010825in}{0.018638in}}{\pgfqpoint{-0.014731in}{0.014731in}}%
\pgfpathcurveto{\pgfqpoint{-0.018638in}{0.010825in}}{\pgfqpoint{-0.020833in}{0.005525in}}{\pgfqpoint{-0.020833in}{0.000000in}}%
\pgfpathcurveto{\pgfqpoint{-0.020833in}{-0.005525in}}{\pgfqpoint{-0.018638in}{-0.010825in}}{\pgfqpoint{-0.014731in}{-0.014731in}}%
\pgfpathcurveto{\pgfqpoint{-0.010825in}{-0.018638in}}{\pgfqpoint{-0.005525in}{-0.020833in}}{\pgfqpoint{0.000000in}{-0.020833in}}%
\pgfpathlineto{\pgfqpoint{0.000000in}{-0.020833in}}%
\pgfpathclose%
\pgfusepath{stroke,fill}%
}%
\begin{pgfscope}%
\pgfsys@transformshift{0.763261in}{1.453881in}%
\pgfsys@useobject{currentmarker}{}%
\end{pgfscope}%
\begin{pgfscope}%
\pgfsys@transformshift{1.133297in}{0.719228in}%
\pgfsys@useobject{currentmarker}{}%
\end{pgfscope}%
\begin{pgfscope}%
\pgfsys@transformshift{1.484257in}{0.772685in}%
\pgfsys@useobject{currentmarker}{}%
\end{pgfscope}%
\begin{pgfscope}%
\pgfsys@transformshift{1.830412in}{0.651025in}%
\pgfsys@useobject{currentmarker}{}%
\end{pgfscope}%
\begin{pgfscope}%
\pgfsys@transformshift{2.176084in}{0.662282in}%
\pgfsys@useobject{currentmarker}{}%
\end{pgfscope}%
\begin{pgfscope}%
\pgfsys@transformshift{2.524625in}{0.609103in}%
\pgfsys@useobject{currentmarker}{}%
\end{pgfscope}%
\end{pgfscope}%
\begin{pgfscope}%
\pgfsetrectcap%
\pgfsetmiterjoin%
\pgfsetlinewidth{0.803000pt}%
\definecolor{currentstroke}{rgb}{0.000000,0.000000,0.000000}%
\pgfsetstrokecolor{currentstroke}%
\pgfsetdash{}{0pt}%
\pgfpathmoveto{\pgfqpoint{0.675193in}{0.521603in}}%
\pgfpathlineto{\pgfqpoint{0.675193in}{2.446603in}}%
\pgfusepath{stroke}%
\end{pgfscope}%
\begin{pgfscope}%
\pgfsetrectcap%
\pgfsetmiterjoin%
\pgfsetlinewidth{0.803000pt}%
\definecolor{currentstroke}{rgb}{0.000000,0.000000,0.000000}%
\pgfsetstrokecolor{currentstroke}%
\pgfsetdash{}{0pt}%
\pgfpathmoveto{\pgfqpoint{2.612693in}{0.521603in}}%
\pgfpathlineto{\pgfqpoint{2.612693in}{2.446603in}}%
\pgfusepath{stroke}%
\end{pgfscope}%
\begin{pgfscope}%
\pgfsetrectcap%
\pgfsetmiterjoin%
\pgfsetlinewidth{0.803000pt}%
\definecolor{currentstroke}{rgb}{0.000000,0.000000,0.000000}%
\pgfsetstrokecolor{currentstroke}%
\pgfsetdash{}{0pt}%
\pgfpathmoveto{\pgfqpoint{0.675193in}{0.521603in}}%
\pgfpathlineto{\pgfqpoint{2.612693in}{0.521603in}}%
\pgfusepath{stroke}%
\end{pgfscope}%
\begin{pgfscope}%
\pgfsetrectcap%
\pgfsetmiterjoin%
\pgfsetlinewidth{0.803000pt}%
\definecolor{currentstroke}{rgb}{0.000000,0.000000,0.000000}%
\pgfsetstrokecolor{currentstroke}%
\pgfsetdash{}{0pt}%
\pgfpathmoveto{\pgfqpoint{0.675193in}{2.446603in}}%
\pgfpathlineto{\pgfqpoint{2.612693in}{2.446603in}}%
\pgfusepath{stroke}%
\end{pgfscope}%
\begin{pgfscope}%
\pgfsetbuttcap%
\pgfsetmiterjoin%
\definecolor{currentfill}{rgb}{1.000000,1.000000,1.000000}%
\pgfsetfillcolor{currentfill}%
\pgfsetfillopacity{0.800000}%
\pgfsetlinewidth{1.003750pt}%
\definecolor{currentstroke}{rgb}{0.800000,0.800000,0.800000}%
\pgfsetstrokecolor{currentstroke}%
\pgfsetstrokeopacity{0.800000}%
\pgfsetdash{}{0pt}%
\pgfpathmoveto{\pgfqpoint{1.637406in}{1.723921in}}%
\pgfpathlineto{\pgfqpoint{2.515471in}{1.723921in}}%
\pgfpathquadraticcurveto{\pgfqpoint{2.543249in}{1.723921in}}{\pgfqpoint{2.543249in}{1.751698in}}%
\pgfpathlineto{\pgfqpoint{2.543249in}{2.349381in}}%
\pgfpathquadraticcurveto{\pgfqpoint{2.543249in}{2.377159in}}{\pgfqpoint{2.515471in}{2.377159in}}%
\pgfpathlineto{\pgfqpoint{1.637406in}{2.377159in}}%
\pgfpathquadraticcurveto{\pgfqpoint{1.609628in}{2.377159in}}{\pgfqpoint{1.609628in}{2.349381in}}%
\pgfpathlineto{\pgfqpoint{1.609628in}{1.751698in}}%
\pgfpathquadraticcurveto{\pgfqpoint{1.609628in}{1.723921in}}{\pgfqpoint{1.637406in}{1.723921in}}%
\pgfpathlineto{\pgfqpoint{1.637406in}{1.723921in}}%
\pgfpathclose%
\pgfusepath{stroke,fill}%
\end{pgfscope}%
\begin{pgfscope}%
\pgfsetrectcap%
\pgfsetroundjoin%
\pgfsetlinewidth{1.003750pt}%
\definecolor{currentstroke}{rgb}{0.001462,0.000466,0.013866}%
\pgfsetstrokecolor{currentstroke}%
\pgfsetdash{}{0pt}%
\pgfpathmoveto{\pgfqpoint{1.665183in}{2.264691in}}%
\pgfpathlineto{\pgfqpoint{1.804072in}{2.264691in}}%
\pgfpathlineto{\pgfqpoint{1.942961in}{2.264691in}}%
\pgfusepath{stroke}%
\end{pgfscope}%
\begin{pgfscope}%
\pgfsetbuttcap%
\pgfsetroundjoin%
\definecolor{currentfill}{rgb}{0.001462,0.000466,0.013866}%
\pgfsetfillcolor{currentfill}%
\pgfsetlinewidth{1.003750pt}%
\definecolor{currentstroke}{rgb}{0.001462,0.000466,0.013866}%
\pgfsetstrokecolor{currentstroke}%
\pgfsetdash{}{0pt}%
\pgfsys@defobject{currentmarker}{\pgfqpoint{-0.020833in}{-0.020833in}}{\pgfqpoint{0.020833in}{0.020833in}}{%
\pgfpathmoveto{\pgfqpoint{0.000000in}{-0.020833in}}%
\pgfpathcurveto{\pgfqpoint{0.005525in}{-0.020833in}}{\pgfqpoint{0.010825in}{-0.018638in}}{\pgfqpoint{0.014731in}{-0.014731in}}%
\pgfpathcurveto{\pgfqpoint{0.018638in}{-0.010825in}}{\pgfqpoint{0.020833in}{-0.005525in}}{\pgfqpoint{0.020833in}{0.000000in}}%
\pgfpathcurveto{\pgfqpoint{0.020833in}{0.005525in}}{\pgfqpoint{0.018638in}{0.010825in}}{\pgfqpoint{0.014731in}{0.014731in}}%
\pgfpathcurveto{\pgfqpoint{0.010825in}{0.018638in}}{\pgfqpoint{0.005525in}{0.020833in}}{\pgfqpoint{0.000000in}{0.020833in}}%
\pgfpathcurveto{\pgfqpoint{-0.005525in}{0.020833in}}{\pgfqpoint{-0.010825in}{0.018638in}}{\pgfqpoint{-0.014731in}{0.014731in}}%
\pgfpathcurveto{\pgfqpoint{-0.018638in}{0.010825in}}{\pgfqpoint{-0.020833in}{0.005525in}}{\pgfqpoint{-0.020833in}{0.000000in}}%
\pgfpathcurveto{\pgfqpoint{-0.020833in}{-0.005525in}}{\pgfqpoint{-0.018638in}{-0.010825in}}{\pgfqpoint{-0.014731in}{-0.014731in}}%
\pgfpathcurveto{\pgfqpoint{-0.010825in}{-0.018638in}}{\pgfqpoint{-0.005525in}{-0.020833in}}{\pgfqpoint{0.000000in}{-0.020833in}}%
\pgfpathlineto{\pgfqpoint{0.000000in}{-0.020833in}}%
\pgfpathclose%
\pgfusepath{stroke,fill}%
}%
\begin{pgfscope}%
\pgfsys@transformshift{1.804072in}{2.264691in}%
\pgfsys@useobject{currentmarker}{}%
\end{pgfscope}%
\end{pgfscope}%
\begin{pgfscope}%
\definecolor{textcolor}{rgb}{0.000000,0.000000,0.000000}%
\pgfsetstrokecolor{textcolor}%
\pgfsetfillcolor{textcolor}%
\pgftext[x=2.054072in,y=2.216080in,left,base]{\color{textcolor}{\sffamily\fontsize{10.000000}{12.000000}\selectfont\catcode`\^=\active\def^{\ifmmode\sp\else\^{}\fi}\catcode`\%=\active\def%{\%}DGC}}%
\end{pgfscope}%
\begin{pgfscope}%
\pgfsetrectcap%
\pgfsetroundjoin%
\pgfsetlinewidth{1.003750pt}%
\definecolor{currentstroke}{rgb}{0.445163,0.122724,0.506901}%
\pgfsetstrokecolor{currentstroke}%
\pgfsetdash{}{0pt}%
\pgfpathmoveto{\pgfqpoint{1.665183in}{2.060834in}}%
\pgfpathlineto{\pgfqpoint{1.804072in}{2.060834in}}%
\pgfpathlineto{\pgfqpoint{1.942961in}{2.060834in}}%
\pgfusepath{stroke}%
\end{pgfscope}%
\begin{pgfscope}%
\pgfsetbuttcap%
\pgfsetroundjoin%
\definecolor{currentfill}{rgb}{0.445163,0.122724,0.506901}%
\pgfsetfillcolor{currentfill}%
\pgfsetlinewidth{1.003750pt}%
\definecolor{currentstroke}{rgb}{0.445163,0.122724,0.506901}%
\pgfsetstrokecolor{currentstroke}%
\pgfsetdash{}{0pt}%
\pgfsys@defobject{currentmarker}{\pgfqpoint{-0.020833in}{-0.020833in}}{\pgfqpoint{0.020833in}{0.020833in}}{%
\pgfpathmoveto{\pgfqpoint{0.000000in}{-0.020833in}}%
\pgfpathcurveto{\pgfqpoint{0.005525in}{-0.020833in}}{\pgfqpoint{0.010825in}{-0.018638in}}{\pgfqpoint{0.014731in}{-0.014731in}}%
\pgfpathcurveto{\pgfqpoint{0.018638in}{-0.010825in}}{\pgfqpoint{0.020833in}{-0.005525in}}{\pgfqpoint{0.020833in}{0.000000in}}%
\pgfpathcurveto{\pgfqpoint{0.020833in}{0.005525in}}{\pgfqpoint{0.018638in}{0.010825in}}{\pgfqpoint{0.014731in}{0.014731in}}%
\pgfpathcurveto{\pgfqpoint{0.010825in}{0.018638in}}{\pgfqpoint{0.005525in}{0.020833in}}{\pgfqpoint{0.000000in}{0.020833in}}%
\pgfpathcurveto{\pgfqpoint{-0.005525in}{0.020833in}}{\pgfqpoint{-0.010825in}{0.018638in}}{\pgfqpoint{-0.014731in}{0.014731in}}%
\pgfpathcurveto{\pgfqpoint{-0.018638in}{0.010825in}}{\pgfqpoint{-0.020833in}{0.005525in}}{\pgfqpoint{-0.020833in}{0.000000in}}%
\pgfpathcurveto{\pgfqpoint{-0.020833in}{-0.005525in}}{\pgfqpoint{-0.018638in}{-0.010825in}}{\pgfqpoint{-0.014731in}{-0.014731in}}%
\pgfpathcurveto{\pgfqpoint{-0.010825in}{-0.018638in}}{\pgfqpoint{-0.005525in}{-0.020833in}}{\pgfqpoint{0.000000in}{-0.020833in}}%
\pgfpathlineto{\pgfqpoint{0.000000in}{-0.020833in}}%
\pgfpathclose%
\pgfusepath{stroke,fill}%
}%
\begin{pgfscope}%
\pgfsys@transformshift{1.804072in}{2.060834in}%
\pgfsys@useobject{currentmarker}{}%
\end{pgfscope}%
\end{pgfscope}%
\begin{pgfscope}%
\definecolor{textcolor}{rgb}{0.000000,0.000000,0.000000}%
\pgfsetstrokecolor{textcolor}%
\pgfsetfillcolor{textcolor}%
\pgftext[x=2.054072in,y=2.012223in,left,base]{\color{textcolor}{\sffamily\fontsize{10.000000}{12.000000}\selectfont\catcode`\^=\active\def^{\ifmmode\sp\else\^{}\fi}\catcode`\%=\active\def%{\%}NC}}%
\end{pgfscope}%
\begin{pgfscope}%
\pgfsetrectcap%
\pgfsetroundjoin%
\pgfsetlinewidth{1.003750pt}%
\definecolor{currentstroke}{rgb}{0.944006,0.377643,0.365136}%
\pgfsetstrokecolor{currentstroke}%
\pgfsetdash{}{0pt}%
\pgfpathmoveto{\pgfqpoint{1.665183in}{1.856977in}}%
\pgfpathlineto{\pgfqpoint{1.804072in}{1.856977in}}%
\pgfpathlineto{\pgfqpoint{1.942961in}{1.856977in}}%
\pgfusepath{stroke}%
\end{pgfscope}%
\begin{pgfscope}%
\pgfsetbuttcap%
\pgfsetroundjoin%
\definecolor{currentfill}{rgb}{0.944006,0.377643,0.365136}%
\pgfsetfillcolor{currentfill}%
\pgfsetlinewidth{1.003750pt}%
\definecolor{currentstroke}{rgb}{0.944006,0.377643,0.365136}%
\pgfsetstrokecolor{currentstroke}%
\pgfsetdash{}{0pt}%
\pgfsys@defobject{currentmarker}{\pgfqpoint{-0.020833in}{-0.020833in}}{\pgfqpoint{0.020833in}{0.020833in}}{%
\pgfpathmoveto{\pgfqpoint{0.000000in}{-0.020833in}}%
\pgfpathcurveto{\pgfqpoint{0.005525in}{-0.020833in}}{\pgfqpoint{0.010825in}{-0.018638in}}{\pgfqpoint{0.014731in}{-0.014731in}}%
\pgfpathcurveto{\pgfqpoint{0.018638in}{-0.010825in}}{\pgfqpoint{0.020833in}{-0.005525in}}{\pgfqpoint{0.020833in}{0.000000in}}%
\pgfpathcurveto{\pgfqpoint{0.020833in}{0.005525in}}{\pgfqpoint{0.018638in}{0.010825in}}{\pgfqpoint{0.014731in}{0.014731in}}%
\pgfpathcurveto{\pgfqpoint{0.010825in}{0.018638in}}{\pgfqpoint{0.005525in}{0.020833in}}{\pgfqpoint{0.000000in}{0.020833in}}%
\pgfpathcurveto{\pgfqpoint{-0.005525in}{0.020833in}}{\pgfqpoint{-0.010825in}{0.018638in}}{\pgfqpoint{-0.014731in}{0.014731in}}%
\pgfpathcurveto{\pgfqpoint{-0.018638in}{0.010825in}}{\pgfqpoint{-0.020833in}{0.005525in}}{\pgfqpoint{-0.020833in}{0.000000in}}%
\pgfpathcurveto{\pgfqpoint{-0.020833in}{-0.005525in}}{\pgfqpoint{-0.018638in}{-0.010825in}}{\pgfqpoint{-0.014731in}{-0.014731in}}%
\pgfpathcurveto{\pgfqpoint{-0.010825in}{-0.018638in}}{\pgfqpoint{-0.005525in}{-0.020833in}}{\pgfqpoint{0.000000in}{-0.020833in}}%
\pgfpathlineto{\pgfqpoint{0.000000in}{-0.020833in}}%
\pgfpathclose%
\pgfusepath{stroke,fill}%
}%
\begin{pgfscope}%
\pgfsys@transformshift{1.804072in}{1.856977in}%
\pgfsys@useobject{currentmarker}{}%
\end{pgfscope}%
\end{pgfscope}%
\begin{pgfscope}%
\definecolor{textcolor}{rgb}{0.000000,0.000000,0.000000}%
\pgfsetstrokecolor{textcolor}%
\pgfsetfillcolor{textcolor}%
\pgftext[x=2.054072in,y=1.808366in,left,base]{\color{textcolor}{\sffamily\fontsize{10.000000}{12.000000}\selectfont\catcode`\^=\active\def^{\ifmmode\sp\else\^{}\fi}\catcode`\%=\active\def%{\%}NC++}}%
\end{pgfscope}%
\end{pgfpicture}%
\makeatother%
\endgroup%

        \caption{$m=800$}
        \label{fig:5-experiments-electronic-structure-convergence-nv-m800}
    \end{subfigure}
    \begin{subfigure}[b]{0.49\columnwidth}
        %% Creator: Matplotlib, PGF backend
%%
%% To include the figure in your LaTeX document, write
%%   \input{<filename>.pgf}
%%
%% Make sure the required packages are loaded in your preamble
%%   \usepackage{pgf}
%%
%% Also ensure that all the required font packages are loaded; for instance,
%% the lmodern package is sometimes necessary when using math font.
%%   \usepackage{lmodern}
%%
%% Figures using additional raster images can only be included by \input if
%% they are in the same directory as the main LaTeX file. For loading figures
%% from other directories you can use the `import` package
%%   \usepackage{import}
%%
%% and then include the figures with
%%   \import{<path to file>}{<filename>.pgf}
%%
%% Matplotlib used the following preamble
%%   \def\mathdefault#1{#1}
%%   \everymath=\expandafter{\the\everymath\displaystyle}
%%   
%%   \usepackage{fontspec}
%%   \setmainfont{DejaVuSans.ttf}[Path=\detokenize{C:/Users/fabio/AppData/Local/Programs/Python/Python311/Lib/site-packages/matplotlib/mpl-data/fonts/ttf/}]
%%   \setsansfont{DejaVuSans.ttf}[Path=\detokenize{C:/Users/fabio/AppData/Local/Programs/Python/Python311/Lib/site-packages/matplotlib/mpl-data/fonts/ttf/}]
%%   \setmonofont{DejaVuSansMono.ttf}[Path=\detokenize{C:/Users/fabio/AppData/Local/Programs/Python/Python311/Lib/site-packages/matplotlib/mpl-data/fonts/ttf/}]
%%   \makeatletter\@ifpackageloaded{underscore}{}{\usepackage[strings]{underscore}}\makeatother
%%
\begingroup%
\makeatletter%
\begin{pgfpicture}%
\pgfpathrectangle{\pgfpointorigin}{\pgfqpoint{2.772561in}{2.600369in}}%
\pgfusepath{use as bounding box, clip}%
\begin{pgfscope}%
\pgfsetbuttcap%
\pgfsetmiterjoin%
\definecolor{currentfill}{rgb}{1.000000,1.000000,1.000000}%
\pgfsetfillcolor{currentfill}%
\pgfsetlinewidth{0.000000pt}%
\definecolor{currentstroke}{rgb}{1.000000,1.000000,1.000000}%
\pgfsetstrokecolor{currentstroke}%
\pgfsetdash{}{0pt}%
\pgfpathmoveto{\pgfqpoint{0.000000in}{-0.000000in}}%
\pgfpathlineto{\pgfqpoint{2.772561in}{-0.000000in}}%
\pgfpathlineto{\pgfqpoint{2.772561in}{2.600369in}}%
\pgfpathlineto{\pgfqpoint{0.000000in}{2.600369in}}%
\pgfpathlineto{\pgfqpoint{0.000000in}{-0.000000in}}%
\pgfpathclose%
\pgfusepath{fill}%
\end{pgfscope}%
\begin{pgfscope}%
\pgfsetbuttcap%
\pgfsetmiterjoin%
\definecolor{currentfill}{rgb}{1.000000,1.000000,1.000000}%
\pgfsetfillcolor{currentfill}%
\pgfsetlinewidth{0.000000pt}%
\definecolor{currentstroke}{rgb}{0.000000,0.000000,0.000000}%
\pgfsetstrokecolor{currentstroke}%
\pgfsetstrokeopacity{0.000000}%
\pgfsetdash{}{0pt}%
\pgfpathmoveto{\pgfqpoint{0.735061in}{0.575369in}}%
\pgfpathlineto{\pgfqpoint{2.672561in}{0.575369in}}%
\pgfpathlineto{\pgfqpoint{2.672561in}{2.500369in}}%
\pgfpathlineto{\pgfqpoint{0.735061in}{2.500369in}}%
\pgfpathlineto{\pgfqpoint{0.735061in}{0.575369in}}%
\pgfpathclose%
\pgfusepath{fill}%
\end{pgfscope}%
\begin{pgfscope}%
\pgfsetbuttcap%
\pgfsetroundjoin%
\definecolor{currentfill}{rgb}{0.000000,0.000000,0.000000}%
\pgfsetfillcolor{currentfill}%
\pgfsetlinewidth{0.803000pt}%
\definecolor{currentstroke}{rgb}{0.000000,0.000000,0.000000}%
\pgfsetstrokecolor{currentstroke}%
\pgfsetdash{}{0pt}%
\pgfsys@defobject{currentmarker}{\pgfqpoint{0.000000in}{-0.048611in}}{\pgfqpoint{0.000000in}{0.000000in}}{%
\pgfpathmoveto{\pgfqpoint{0.000000in}{0.000000in}}%
\pgfpathlineto{\pgfqpoint{0.000000in}{-0.048611in}}%
\pgfusepath{stroke,fill}%
}%
\begin{pgfscope}%
\pgfsys@transformshift{1.784714in}{0.575369in}%
\pgfsys@useobject{currentmarker}{}%
\end{pgfscope}%
\end{pgfscope}%
\begin{pgfscope}%
\definecolor{textcolor}{rgb}{0.000000,0.000000,0.000000}%
\pgfsetstrokecolor{textcolor}%
\pgfsetfillcolor{textcolor}%
\pgftext[x=1.784714in,y=0.478146in,,top]{\color{textcolor}{\rmfamily\fontsize{12.000000}{14.400000}\selectfont\catcode`\^=\active\def^{\ifmmode\sp\else\^{}\fi}\catcode`\%=\active\def%{\%}$\mathdefault{10^{2}}$}}%
\end{pgfscope}%
\begin{pgfscope}%
\pgfsetbuttcap%
\pgfsetroundjoin%
\definecolor{currentfill}{rgb}{0.000000,0.000000,0.000000}%
\pgfsetfillcolor{currentfill}%
\pgfsetlinewidth{0.602250pt}%
\definecolor{currentstroke}{rgb}{0.000000,0.000000,0.000000}%
\pgfsetstrokecolor{currentstroke}%
\pgfsetdash{}{0pt}%
\pgfsys@defobject{currentmarker}{\pgfqpoint{0.000000in}{-0.027778in}}{\pgfqpoint{0.000000in}{0.000000in}}{%
\pgfpathmoveto{\pgfqpoint{0.000000in}{0.000000in}}%
\pgfpathlineto{\pgfqpoint{0.000000in}{-0.027778in}}%
\pgfusepath{stroke,fill}%
}%
\begin{pgfscope}%
\pgfsys@transformshift{0.852829in}{0.575369in}%
\pgfsys@useobject{currentmarker}{}%
\end{pgfscope}%
\end{pgfscope}%
\begin{pgfscope}%
\pgfsetbuttcap%
\pgfsetroundjoin%
\definecolor{currentfill}{rgb}{0.000000,0.000000,0.000000}%
\pgfsetfillcolor{currentfill}%
\pgfsetlinewidth{0.602250pt}%
\definecolor{currentstroke}{rgb}{0.000000,0.000000,0.000000}%
\pgfsetstrokecolor{currentstroke}%
\pgfsetdash{}{0pt}%
\pgfsys@defobject{currentmarker}{\pgfqpoint{0.000000in}{-0.027778in}}{\pgfqpoint{0.000000in}{0.000000in}}{%
\pgfpathmoveto{\pgfqpoint{0.000000in}{0.000000in}}%
\pgfpathlineto{\pgfqpoint{0.000000in}{-0.027778in}}%
\pgfusepath{stroke,fill}%
}%
\begin{pgfscope}%
\pgfsys@transformshift{1.087598in}{0.575369in}%
\pgfsys@useobject{currentmarker}{}%
\end{pgfscope}%
\end{pgfscope}%
\begin{pgfscope}%
\pgfsetbuttcap%
\pgfsetroundjoin%
\definecolor{currentfill}{rgb}{0.000000,0.000000,0.000000}%
\pgfsetfillcolor{currentfill}%
\pgfsetlinewidth{0.602250pt}%
\definecolor{currentstroke}{rgb}{0.000000,0.000000,0.000000}%
\pgfsetstrokecolor{currentstroke}%
\pgfsetdash{}{0pt}%
\pgfsys@defobject{currentmarker}{\pgfqpoint{0.000000in}{-0.027778in}}{\pgfqpoint{0.000000in}{0.000000in}}{%
\pgfpathmoveto{\pgfqpoint{0.000000in}{0.000000in}}%
\pgfpathlineto{\pgfqpoint{0.000000in}{-0.027778in}}%
\pgfusepath{stroke,fill}%
}%
\begin{pgfscope}%
\pgfsys@transformshift{1.254170in}{0.575369in}%
\pgfsys@useobject{currentmarker}{}%
\end{pgfscope}%
\end{pgfscope}%
\begin{pgfscope}%
\pgfsetbuttcap%
\pgfsetroundjoin%
\definecolor{currentfill}{rgb}{0.000000,0.000000,0.000000}%
\pgfsetfillcolor{currentfill}%
\pgfsetlinewidth{0.602250pt}%
\definecolor{currentstroke}{rgb}{0.000000,0.000000,0.000000}%
\pgfsetstrokecolor{currentstroke}%
\pgfsetdash{}{0pt}%
\pgfsys@defobject{currentmarker}{\pgfqpoint{0.000000in}{-0.027778in}}{\pgfqpoint{0.000000in}{0.000000in}}{%
\pgfpathmoveto{\pgfqpoint{0.000000in}{0.000000in}}%
\pgfpathlineto{\pgfqpoint{0.000000in}{-0.027778in}}%
\pgfusepath{stroke,fill}%
}%
\begin{pgfscope}%
\pgfsys@transformshift{1.383373in}{0.575369in}%
\pgfsys@useobject{currentmarker}{}%
\end{pgfscope}%
\end{pgfscope}%
\begin{pgfscope}%
\pgfsetbuttcap%
\pgfsetroundjoin%
\definecolor{currentfill}{rgb}{0.000000,0.000000,0.000000}%
\pgfsetfillcolor{currentfill}%
\pgfsetlinewidth{0.602250pt}%
\definecolor{currentstroke}{rgb}{0.000000,0.000000,0.000000}%
\pgfsetstrokecolor{currentstroke}%
\pgfsetdash{}{0pt}%
\pgfsys@defobject{currentmarker}{\pgfqpoint{0.000000in}{-0.027778in}}{\pgfqpoint{0.000000in}{0.000000in}}{%
\pgfpathmoveto{\pgfqpoint{0.000000in}{0.000000in}}%
\pgfpathlineto{\pgfqpoint{0.000000in}{-0.027778in}}%
\pgfusepath{stroke,fill}%
}%
\begin{pgfscope}%
\pgfsys@transformshift{1.488939in}{0.575369in}%
\pgfsys@useobject{currentmarker}{}%
\end{pgfscope}%
\end{pgfscope}%
\begin{pgfscope}%
\pgfsetbuttcap%
\pgfsetroundjoin%
\definecolor{currentfill}{rgb}{0.000000,0.000000,0.000000}%
\pgfsetfillcolor{currentfill}%
\pgfsetlinewidth{0.602250pt}%
\definecolor{currentstroke}{rgb}{0.000000,0.000000,0.000000}%
\pgfsetstrokecolor{currentstroke}%
\pgfsetdash{}{0pt}%
\pgfsys@defobject{currentmarker}{\pgfqpoint{0.000000in}{-0.027778in}}{\pgfqpoint{0.000000in}{0.000000in}}{%
\pgfpathmoveto{\pgfqpoint{0.000000in}{0.000000in}}%
\pgfpathlineto{\pgfqpoint{0.000000in}{-0.027778in}}%
\pgfusepath{stroke,fill}%
}%
\begin{pgfscope}%
\pgfsys@transformshift{1.578194in}{0.575369in}%
\pgfsys@useobject{currentmarker}{}%
\end{pgfscope}%
\end{pgfscope}%
\begin{pgfscope}%
\pgfsetbuttcap%
\pgfsetroundjoin%
\definecolor{currentfill}{rgb}{0.000000,0.000000,0.000000}%
\pgfsetfillcolor{currentfill}%
\pgfsetlinewidth{0.602250pt}%
\definecolor{currentstroke}{rgb}{0.000000,0.000000,0.000000}%
\pgfsetstrokecolor{currentstroke}%
\pgfsetdash{}{0pt}%
\pgfsys@defobject{currentmarker}{\pgfqpoint{0.000000in}{-0.027778in}}{\pgfqpoint{0.000000in}{0.000000in}}{%
\pgfpathmoveto{\pgfqpoint{0.000000in}{0.000000in}}%
\pgfpathlineto{\pgfqpoint{0.000000in}{-0.027778in}}%
\pgfusepath{stroke,fill}%
}%
\begin{pgfscope}%
\pgfsys@transformshift{1.655511in}{0.575369in}%
\pgfsys@useobject{currentmarker}{}%
\end{pgfscope}%
\end{pgfscope}%
\begin{pgfscope}%
\pgfsetbuttcap%
\pgfsetroundjoin%
\definecolor{currentfill}{rgb}{0.000000,0.000000,0.000000}%
\pgfsetfillcolor{currentfill}%
\pgfsetlinewidth{0.602250pt}%
\definecolor{currentstroke}{rgb}{0.000000,0.000000,0.000000}%
\pgfsetstrokecolor{currentstroke}%
\pgfsetdash{}{0pt}%
\pgfsys@defobject{currentmarker}{\pgfqpoint{0.000000in}{-0.027778in}}{\pgfqpoint{0.000000in}{0.000000in}}{%
\pgfpathmoveto{\pgfqpoint{0.000000in}{0.000000in}}%
\pgfpathlineto{\pgfqpoint{0.000000in}{-0.027778in}}%
\pgfusepath{stroke,fill}%
}%
\begin{pgfscope}%
\pgfsys@transformshift{1.723709in}{0.575369in}%
\pgfsys@useobject{currentmarker}{}%
\end{pgfscope}%
\end{pgfscope}%
\begin{pgfscope}%
\pgfsetbuttcap%
\pgfsetroundjoin%
\definecolor{currentfill}{rgb}{0.000000,0.000000,0.000000}%
\pgfsetfillcolor{currentfill}%
\pgfsetlinewidth{0.602250pt}%
\definecolor{currentstroke}{rgb}{0.000000,0.000000,0.000000}%
\pgfsetstrokecolor{currentstroke}%
\pgfsetdash{}{0pt}%
\pgfsys@defobject{currentmarker}{\pgfqpoint{0.000000in}{-0.027778in}}{\pgfqpoint{0.000000in}{0.000000in}}{%
\pgfpathmoveto{\pgfqpoint{0.000000in}{0.000000in}}%
\pgfpathlineto{\pgfqpoint{0.000000in}{-0.027778in}}%
\pgfusepath{stroke,fill}%
}%
\begin{pgfscope}%
\pgfsys@transformshift{2.186055in}{0.575369in}%
\pgfsys@useobject{currentmarker}{}%
\end{pgfscope}%
\end{pgfscope}%
\begin{pgfscope}%
\pgfsetbuttcap%
\pgfsetroundjoin%
\definecolor{currentfill}{rgb}{0.000000,0.000000,0.000000}%
\pgfsetfillcolor{currentfill}%
\pgfsetlinewidth{0.602250pt}%
\definecolor{currentstroke}{rgb}{0.000000,0.000000,0.000000}%
\pgfsetstrokecolor{currentstroke}%
\pgfsetdash{}{0pt}%
\pgfsys@defobject{currentmarker}{\pgfqpoint{0.000000in}{-0.027778in}}{\pgfqpoint{0.000000in}{0.000000in}}{%
\pgfpathmoveto{\pgfqpoint{0.000000in}{0.000000in}}%
\pgfpathlineto{\pgfqpoint{0.000000in}{-0.027778in}}%
\pgfusepath{stroke,fill}%
}%
\begin{pgfscope}%
\pgfsys@transformshift{2.420824in}{0.575369in}%
\pgfsys@useobject{currentmarker}{}%
\end{pgfscope}%
\end{pgfscope}%
\begin{pgfscope}%
\pgfsetbuttcap%
\pgfsetroundjoin%
\definecolor{currentfill}{rgb}{0.000000,0.000000,0.000000}%
\pgfsetfillcolor{currentfill}%
\pgfsetlinewidth{0.602250pt}%
\definecolor{currentstroke}{rgb}{0.000000,0.000000,0.000000}%
\pgfsetstrokecolor{currentstroke}%
\pgfsetdash{}{0pt}%
\pgfsys@defobject{currentmarker}{\pgfqpoint{0.000000in}{-0.027778in}}{\pgfqpoint{0.000000in}{0.000000in}}{%
\pgfpathmoveto{\pgfqpoint{0.000000in}{0.000000in}}%
\pgfpathlineto{\pgfqpoint{0.000000in}{-0.027778in}}%
\pgfusepath{stroke,fill}%
}%
\begin{pgfscope}%
\pgfsys@transformshift{2.587396in}{0.575369in}%
\pgfsys@useobject{currentmarker}{}%
\end{pgfscope}%
\end{pgfscope}%
\begin{pgfscope}%
\definecolor{textcolor}{rgb}{0.000000,0.000000,0.000000}%
\pgfsetstrokecolor{textcolor}%
\pgfsetfillcolor{textcolor}%
\pgftext[x=1.703811in,y=0.261295in,,top]{\color{textcolor}{\rmfamily\fontsize{12.000000}{14.400000}\selectfont\catcode`\^=\active\def^{\ifmmode\sp\else\^{}\fi}\catcode`\%=\active\def%{\%}$n_{\Omega}$}}%
\end{pgfscope}%
\begin{pgfscope}%
\pgfsetbuttcap%
\pgfsetroundjoin%
\definecolor{currentfill}{rgb}{0.000000,0.000000,0.000000}%
\pgfsetfillcolor{currentfill}%
\pgfsetlinewidth{0.803000pt}%
\definecolor{currentstroke}{rgb}{0.000000,0.000000,0.000000}%
\pgfsetstrokecolor{currentstroke}%
\pgfsetdash{}{0pt}%
\pgfsys@defobject{currentmarker}{\pgfqpoint{-0.048611in}{0.000000in}}{\pgfqpoint{-0.000000in}{0.000000in}}{%
\pgfpathmoveto{\pgfqpoint{-0.000000in}{0.000000in}}%
\pgfpathlineto{\pgfqpoint{-0.048611in}{0.000000in}}%
\pgfusepath{stroke,fill}%
}%
\begin{pgfscope}%
\pgfsys@transformshift{0.735061in}{0.957169in}%
\pgfsys@useobject{currentmarker}{}%
\end{pgfscope}%
\end{pgfscope}%
\begin{pgfscope}%
\definecolor{textcolor}{rgb}{0.000000,0.000000,0.000000}%
\pgfsetstrokecolor{textcolor}%
\pgfsetfillcolor{textcolor}%
\pgftext[x=0.316851in, y=0.893855in, left, base]{\color{textcolor}{\rmfamily\fontsize{12.000000}{14.400000}\selectfont\catcode`\^=\active\def^{\ifmmode\sp\else\^{}\fi}\catcode`\%=\active\def%{\%}$\mathdefault{10^{-6}}$}}%
\end{pgfscope}%
\begin{pgfscope}%
\pgfsetbuttcap%
\pgfsetroundjoin%
\definecolor{currentfill}{rgb}{0.000000,0.000000,0.000000}%
\pgfsetfillcolor{currentfill}%
\pgfsetlinewidth{0.803000pt}%
\definecolor{currentstroke}{rgb}{0.000000,0.000000,0.000000}%
\pgfsetstrokecolor{currentstroke}%
\pgfsetdash{}{0pt}%
\pgfsys@defobject{currentmarker}{\pgfqpoint{-0.048611in}{0.000000in}}{\pgfqpoint{-0.000000in}{0.000000in}}{%
\pgfpathmoveto{\pgfqpoint{-0.000000in}{0.000000in}}%
\pgfpathlineto{\pgfqpoint{-0.048611in}{0.000000in}}%
\pgfusepath{stroke,fill}%
}%
\begin{pgfscope}%
\pgfsys@transformshift{0.735061in}{1.502659in}%
\pgfsys@useobject{currentmarker}{}%
\end{pgfscope}%
\end{pgfscope}%
\begin{pgfscope}%
\definecolor{textcolor}{rgb}{0.000000,0.000000,0.000000}%
\pgfsetstrokecolor{textcolor}%
\pgfsetfillcolor{textcolor}%
\pgftext[x=0.316851in, y=1.439346in, left, base]{\color{textcolor}{\rmfamily\fontsize{12.000000}{14.400000}\selectfont\catcode`\^=\active\def^{\ifmmode\sp\else\^{}\fi}\catcode`\%=\active\def%{\%}$\mathdefault{10^{-4}}$}}%
\end{pgfscope}%
\begin{pgfscope}%
\pgfsetbuttcap%
\pgfsetroundjoin%
\definecolor{currentfill}{rgb}{0.000000,0.000000,0.000000}%
\pgfsetfillcolor{currentfill}%
\pgfsetlinewidth{0.803000pt}%
\definecolor{currentstroke}{rgb}{0.000000,0.000000,0.000000}%
\pgfsetstrokecolor{currentstroke}%
\pgfsetdash{}{0pt}%
\pgfsys@defobject{currentmarker}{\pgfqpoint{-0.048611in}{0.000000in}}{\pgfqpoint{-0.000000in}{0.000000in}}{%
\pgfpathmoveto{\pgfqpoint{-0.000000in}{0.000000in}}%
\pgfpathlineto{\pgfqpoint{-0.048611in}{0.000000in}}%
\pgfusepath{stroke,fill}%
}%
\begin{pgfscope}%
\pgfsys@transformshift{0.735061in}{2.048150in}%
\pgfsys@useobject{currentmarker}{}%
\end{pgfscope}%
\end{pgfscope}%
\begin{pgfscope}%
\definecolor{textcolor}{rgb}{0.000000,0.000000,0.000000}%
\pgfsetstrokecolor{textcolor}%
\pgfsetfillcolor{textcolor}%
\pgftext[x=0.316851in, y=1.984837in, left, base]{\color{textcolor}{\rmfamily\fontsize{12.000000}{14.400000}\selectfont\catcode`\^=\active\def^{\ifmmode\sp\else\^{}\fi}\catcode`\%=\active\def%{\%}$\mathdefault{10^{-2}}$}}%
\end{pgfscope}%
\begin{pgfscope}%
\definecolor{textcolor}{rgb}{0.000000,0.000000,0.000000}%
\pgfsetstrokecolor{textcolor}%
\pgfsetfillcolor{textcolor}%
\pgftext[x=0.261295in,y=1.537869in,,bottom,rotate=90.000000]{\color{textcolor}{\rmfamily\fontsize{12.000000}{14.400000}\selectfont\catcode`\^=\active\def^{\ifmmode\sp\else\^{}\fi}\catcode`\%=\active\def%{\%}$L^1$ relative error}}%
\end{pgfscope}%
\begin{pgfscope}%
\pgfpathrectangle{\pgfqpoint{0.735061in}{0.575369in}}{\pgfqpoint{1.937500in}{1.925000in}}%
\pgfusepath{clip}%
\pgfsetrectcap%
\pgfsetroundjoin%
\pgfsetlinewidth{1.003750pt}%
\definecolor{currentstroke}{rgb}{0.001462,0.000466,0.013866}%
\pgfsetstrokecolor{currentstroke}%
\pgfsetdash{}{0pt}%
\pgfpathmoveto{\pgfqpoint{0.823130in}{2.258007in}}%
\pgfpathlineto{\pgfqpoint{1.193165in}{2.250922in}}%
\pgfpathlineto{\pgfqpoint{1.544125in}{2.208397in}}%
\pgfpathlineto{\pgfqpoint{1.890280in}{2.151888in}}%
\pgfpathlineto{\pgfqpoint{2.235953in}{2.132875in}}%
\pgfpathlineto{\pgfqpoint{2.584493in}{2.061775in}}%
\pgfusepath{stroke}%
\end{pgfscope}%
\begin{pgfscope}%
\pgfpathrectangle{\pgfqpoint{0.735061in}{0.575369in}}{\pgfqpoint{1.937500in}{1.925000in}}%
\pgfusepath{clip}%
\pgfsetbuttcap%
\pgfsetroundjoin%
\definecolor{currentfill}{rgb}{0.001462,0.000466,0.013866}%
\pgfsetfillcolor{currentfill}%
\pgfsetlinewidth{1.003750pt}%
\definecolor{currentstroke}{rgb}{0.001462,0.000466,0.013866}%
\pgfsetstrokecolor{currentstroke}%
\pgfsetdash{}{0pt}%
\pgfsys@defobject{currentmarker}{\pgfqpoint{-0.020833in}{-0.020833in}}{\pgfqpoint{0.020833in}{0.020833in}}{%
\pgfpathmoveto{\pgfqpoint{0.000000in}{-0.020833in}}%
\pgfpathcurveto{\pgfqpoint{0.005525in}{-0.020833in}}{\pgfqpoint{0.010825in}{-0.018638in}}{\pgfqpoint{0.014731in}{-0.014731in}}%
\pgfpathcurveto{\pgfqpoint{0.018638in}{-0.010825in}}{\pgfqpoint{0.020833in}{-0.005525in}}{\pgfqpoint{0.020833in}{0.000000in}}%
\pgfpathcurveto{\pgfqpoint{0.020833in}{0.005525in}}{\pgfqpoint{0.018638in}{0.010825in}}{\pgfqpoint{0.014731in}{0.014731in}}%
\pgfpathcurveto{\pgfqpoint{0.010825in}{0.018638in}}{\pgfqpoint{0.005525in}{0.020833in}}{\pgfqpoint{0.000000in}{0.020833in}}%
\pgfpathcurveto{\pgfqpoint{-0.005525in}{0.020833in}}{\pgfqpoint{-0.010825in}{0.018638in}}{\pgfqpoint{-0.014731in}{0.014731in}}%
\pgfpathcurveto{\pgfqpoint{-0.018638in}{0.010825in}}{\pgfqpoint{-0.020833in}{0.005525in}}{\pgfqpoint{-0.020833in}{0.000000in}}%
\pgfpathcurveto{\pgfqpoint{-0.020833in}{-0.005525in}}{\pgfqpoint{-0.018638in}{-0.010825in}}{\pgfqpoint{-0.014731in}{-0.014731in}}%
\pgfpathcurveto{\pgfqpoint{-0.010825in}{-0.018638in}}{\pgfqpoint{-0.005525in}{-0.020833in}}{\pgfqpoint{0.000000in}{-0.020833in}}%
\pgfpathlineto{\pgfqpoint{0.000000in}{-0.020833in}}%
\pgfpathclose%
\pgfusepath{stroke,fill}%
}%
\begin{pgfscope}%
\pgfsys@transformshift{0.823130in}{2.258007in}%
\pgfsys@useobject{currentmarker}{}%
\end{pgfscope}%
\begin{pgfscope}%
\pgfsys@transformshift{1.193165in}{2.250922in}%
\pgfsys@useobject{currentmarker}{}%
\end{pgfscope}%
\begin{pgfscope}%
\pgfsys@transformshift{1.544125in}{2.208397in}%
\pgfsys@useobject{currentmarker}{}%
\end{pgfscope}%
\begin{pgfscope}%
\pgfsys@transformshift{1.890280in}{2.151888in}%
\pgfsys@useobject{currentmarker}{}%
\end{pgfscope}%
\begin{pgfscope}%
\pgfsys@transformshift{2.235953in}{2.132875in}%
\pgfsys@useobject{currentmarker}{}%
\end{pgfscope}%
\begin{pgfscope}%
\pgfsys@transformshift{2.584493in}{2.061775in}%
\pgfsys@useobject{currentmarker}{}%
\end{pgfscope}%
\end{pgfscope}%
\begin{pgfscope}%
\pgfpathrectangle{\pgfqpoint{0.735061in}{0.575369in}}{\pgfqpoint{1.937500in}{1.925000in}}%
\pgfusepath{clip}%
\pgfsetrectcap%
\pgfsetroundjoin%
\pgfsetlinewidth{1.003750pt}%
\definecolor{currentstroke}{rgb}{0.445163,0.122724,0.506901}%
\pgfsetstrokecolor{currentstroke}%
\pgfsetdash{}{0pt}%
\pgfpathmoveto{\pgfqpoint{0.823130in}{2.412869in}}%
\pgfpathlineto{\pgfqpoint{1.193165in}{2.284532in}}%
\pgfpathlineto{\pgfqpoint{1.544125in}{1.798824in}}%
\pgfpathlineto{\pgfqpoint{1.890280in}{0.693032in}}%
\pgfpathlineto{\pgfqpoint{2.235953in}{0.679559in}}%
\pgfpathlineto{\pgfqpoint{2.584493in}{0.684647in}}%
\pgfusepath{stroke}%
\end{pgfscope}%
\begin{pgfscope}%
\pgfpathrectangle{\pgfqpoint{0.735061in}{0.575369in}}{\pgfqpoint{1.937500in}{1.925000in}}%
\pgfusepath{clip}%
\pgfsetbuttcap%
\pgfsetroundjoin%
\definecolor{currentfill}{rgb}{0.445163,0.122724,0.506901}%
\pgfsetfillcolor{currentfill}%
\pgfsetlinewidth{1.003750pt}%
\definecolor{currentstroke}{rgb}{0.445163,0.122724,0.506901}%
\pgfsetstrokecolor{currentstroke}%
\pgfsetdash{}{0pt}%
\pgfsys@defobject{currentmarker}{\pgfqpoint{-0.020833in}{-0.020833in}}{\pgfqpoint{0.020833in}{0.020833in}}{%
\pgfpathmoveto{\pgfqpoint{0.000000in}{-0.020833in}}%
\pgfpathcurveto{\pgfqpoint{0.005525in}{-0.020833in}}{\pgfqpoint{0.010825in}{-0.018638in}}{\pgfqpoint{0.014731in}{-0.014731in}}%
\pgfpathcurveto{\pgfqpoint{0.018638in}{-0.010825in}}{\pgfqpoint{0.020833in}{-0.005525in}}{\pgfqpoint{0.020833in}{0.000000in}}%
\pgfpathcurveto{\pgfqpoint{0.020833in}{0.005525in}}{\pgfqpoint{0.018638in}{0.010825in}}{\pgfqpoint{0.014731in}{0.014731in}}%
\pgfpathcurveto{\pgfqpoint{0.010825in}{0.018638in}}{\pgfqpoint{0.005525in}{0.020833in}}{\pgfqpoint{0.000000in}{0.020833in}}%
\pgfpathcurveto{\pgfqpoint{-0.005525in}{0.020833in}}{\pgfqpoint{-0.010825in}{0.018638in}}{\pgfqpoint{-0.014731in}{0.014731in}}%
\pgfpathcurveto{\pgfqpoint{-0.018638in}{0.010825in}}{\pgfqpoint{-0.020833in}{0.005525in}}{\pgfqpoint{-0.020833in}{0.000000in}}%
\pgfpathcurveto{\pgfqpoint{-0.020833in}{-0.005525in}}{\pgfqpoint{-0.018638in}{-0.010825in}}{\pgfqpoint{-0.014731in}{-0.014731in}}%
\pgfpathcurveto{\pgfqpoint{-0.010825in}{-0.018638in}}{\pgfqpoint{-0.005525in}{-0.020833in}}{\pgfqpoint{0.000000in}{-0.020833in}}%
\pgfpathlineto{\pgfqpoint{0.000000in}{-0.020833in}}%
\pgfpathclose%
\pgfusepath{stroke,fill}%
}%
\begin{pgfscope}%
\pgfsys@transformshift{0.823130in}{2.412869in}%
\pgfsys@useobject{currentmarker}{}%
\end{pgfscope}%
\begin{pgfscope}%
\pgfsys@transformshift{1.193165in}{2.284532in}%
\pgfsys@useobject{currentmarker}{}%
\end{pgfscope}%
\begin{pgfscope}%
\pgfsys@transformshift{1.544125in}{1.798824in}%
\pgfsys@useobject{currentmarker}{}%
\end{pgfscope}%
\begin{pgfscope}%
\pgfsys@transformshift{1.890280in}{0.693032in}%
\pgfsys@useobject{currentmarker}{}%
\end{pgfscope}%
\begin{pgfscope}%
\pgfsys@transformshift{2.235953in}{0.679559in}%
\pgfsys@useobject{currentmarker}{}%
\end{pgfscope}%
\begin{pgfscope}%
\pgfsys@transformshift{2.584493in}{0.684647in}%
\pgfsys@useobject{currentmarker}{}%
\end{pgfscope}%
\end{pgfscope}%
\begin{pgfscope}%
\pgfpathrectangle{\pgfqpoint{0.735061in}{0.575369in}}{\pgfqpoint{1.937500in}{1.925000in}}%
\pgfusepath{clip}%
\pgfsetrectcap%
\pgfsetroundjoin%
\pgfsetlinewidth{1.003750pt}%
\definecolor{currentstroke}{rgb}{0.944006,0.377643,0.365136}%
\pgfsetstrokecolor{currentstroke}%
\pgfsetdash{}{0pt}%
\pgfpathmoveto{\pgfqpoint{0.823130in}{2.250839in}}%
\pgfpathlineto{\pgfqpoint{1.193165in}{2.002936in}}%
\pgfpathlineto{\pgfqpoint{1.544125in}{1.922920in}}%
\pgfpathlineto{\pgfqpoint{1.890280in}{1.551407in}}%
\pgfpathlineto{\pgfqpoint{2.235953in}{0.663193in}}%
\pgfpathlineto{\pgfqpoint{2.584493in}{0.662869in}}%
\pgfusepath{stroke}%
\end{pgfscope}%
\begin{pgfscope}%
\pgfpathrectangle{\pgfqpoint{0.735061in}{0.575369in}}{\pgfqpoint{1.937500in}{1.925000in}}%
\pgfusepath{clip}%
\pgfsetbuttcap%
\pgfsetroundjoin%
\definecolor{currentfill}{rgb}{0.944006,0.377643,0.365136}%
\pgfsetfillcolor{currentfill}%
\pgfsetlinewidth{1.003750pt}%
\definecolor{currentstroke}{rgb}{0.944006,0.377643,0.365136}%
\pgfsetstrokecolor{currentstroke}%
\pgfsetdash{}{0pt}%
\pgfsys@defobject{currentmarker}{\pgfqpoint{-0.020833in}{-0.020833in}}{\pgfqpoint{0.020833in}{0.020833in}}{%
\pgfpathmoveto{\pgfqpoint{0.000000in}{-0.020833in}}%
\pgfpathcurveto{\pgfqpoint{0.005525in}{-0.020833in}}{\pgfqpoint{0.010825in}{-0.018638in}}{\pgfqpoint{0.014731in}{-0.014731in}}%
\pgfpathcurveto{\pgfqpoint{0.018638in}{-0.010825in}}{\pgfqpoint{0.020833in}{-0.005525in}}{\pgfqpoint{0.020833in}{0.000000in}}%
\pgfpathcurveto{\pgfqpoint{0.020833in}{0.005525in}}{\pgfqpoint{0.018638in}{0.010825in}}{\pgfqpoint{0.014731in}{0.014731in}}%
\pgfpathcurveto{\pgfqpoint{0.010825in}{0.018638in}}{\pgfqpoint{0.005525in}{0.020833in}}{\pgfqpoint{0.000000in}{0.020833in}}%
\pgfpathcurveto{\pgfqpoint{-0.005525in}{0.020833in}}{\pgfqpoint{-0.010825in}{0.018638in}}{\pgfqpoint{-0.014731in}{0.014731in}}%
\pgfpathcurveto{\pgfqpoint{-0.018638in}{0.010825in}}{\pgfqpoint{-0.020833in}{0.005525in}}{\pgfqpoint{-0.020833in}{0.000000in}}%
\pgfpathcurveto{\pgfqpoint{-0.020833in}{-0.005525in}}{\pgfqpoint{-0.018638in}{-0.010825in}}{\pgfqpoint{-0.014731in}{-0.014731in}}%
\pgfpathcurveto{\pgfqpoint{-0.010825in}{-0.018638in}}{\pgfqpoint{-0.005525in}{-0.020833in}}{\pgfqpoint{0.000000in}{-0.020833in}}%
\pgfpathlineto{\pgfqpoint{0.000000in}{-0.020833in}}%
\pgfpathclose%
\pgfusepath{stroke,fill}%
}%
\begin{pgfscope}%
\pgfsys@transformshift{0.823130in}{2.250839in}%
\pgfsys@useobject{currentmarker}{}%
\end{pgfscope}%
\begin{pgfscope}%
\pgfsys@transformshift{1.193165in}{2.002936in}%
\pgfsys@useobject{currentmarker}{}%
\end{pgfscope}%
\begin{pgfscope}%
\pgfsys@transformshift{1.544125in}{1.922920in}%
\pgfsys@useobject{currentmarker}{}%
\end{pgfscope}%
\begin{pgfscope}%
\pgfsys@transformshift{1.890280in}{1.551407in}%
\pgfsys@useobject{currentmarker}{}%
\end{pgfscope}%
\begin{pgfscope}%
\pgfsys@transformshift{2.235953in}{0.663193in}%
\pgfsys@useobject{currentmarker}{}%
\end{pgfscope}%
\begin{pgfscope}%
\pgfsys@transformshift{2.584493in}{0.662869in}%
\pgfsys@useobject{currentmarker}{}%
\end{pgfscope}%
\end{pgfscope}%
\begin{pgfscope}%
\pgfsetrectcap%
\pgfsetmiterjoin%
\pgfsetlinewidth{0.803000pt}%
\definecolor{currentstroke}{rgb}{0.000000,0.000000,0.000000}%
\pgfsetstrokecolor{currentstroke}%
\pgfsetdash{}{0pt}%
\pgfpathmoveto{\pgfqpoint{0.735061in}{0.575369in}}%
\pgfpathlineto{\pgfqpoint{0.735061in}{2.500369in}}%
\pgfusepath{stroke}%
\end{pgfscope}%
\begin{pgfscope}%
\pgfsetrectcap%
\pgfsetmiterjoin%
\pgfsetlinewidth{0.803000pt}%
\definecolor{currentstroke}{rgb}{0.000000,0.000000,0.000000}%
\pgfsetstrokecolor{currentstroke}%
\pgfsetdash{}{0pt}%
\pgfpathmoveto{\pgfqpoint{2.672561in}{0.575369in}}%
\pgfpathlineto{\pgfqpoint{2.672561in}{2.500369in}}%
\pgfusepath{stroke}%
\end{pgfscope}%
\begin{pgfscope}%
\pgfsetrectcap%
\pgfsetmiterjoin%
\pgfsetlinewidth{0.803000pt}%
\definecolor{currentstroke}{rgb}{0.000000,0.000000,0.000000}%
\pgfsetstrokecolor{currentstroke}%
\pgfsetdash{}{0pt}%
\pgfpathmoveto{\pgfqpoint{0.735061in}{0.575369in}}%
\pgfpathlineto{\pgfqpoint{2.672561in}{0.575369in}}%
\pgfusepath{stroke}%
\end{pgfscope}%
\begin{pgfscope}%
\pgfsetrectcap%
\pgfsetmiterjoin%
\pgfsetlinewidth{0.803000pt}%
\definecolor{currentstroke}{rgb}{0.000000,0.000000,0.000000}%
\pgfsetstrokecolor{currentstroke}%
\pgfsetdash{}{0pt}%
\pgfpathmoveto{\pgfqpoint{0.735061in}{2.500369in}}%
\pgfpathlineto{\pgfqpoint{2.672561in}{2.500369in}}%
\pgfusepath{stroke}%
\end{pgfscope}%
\begin{pgfscope}%
\pgfsetbuttcap%
\pgfsetmiterjoin%
\definecolor{currentfill}{rgb}{1.000000,1.000000,1.000000}%
\pgfsetfillcolor{currentfill}%
\pgfsetfillopacity{0.800000}%
\pgfsetlinewidth{1.003750pt}%
\definecolor{currentstroke}{rgb}{0.800000,0.800000,0.800000}%
\pgfsetstrokecolor{currentstroke}%
\pgfsetstrokeopacity{0.800000}%
\pgfsetdash{}{0pt}%
\pgfpathmoveto{\pgfqpoint{0.851728in}{0.658702in}}%
\pgfpathlineto{\pgfqpoint{1.905406in}{0.658702in}}%
\pgfpathquadraticcurveto{\pgfqpoint{1.938740in}{0.658702in}}{\pgfqpoint{1.938740in}{0.692035in}}%
\pgfpathlineto{\pgfqpoint{1.938740in}{1.409255in}}%
\pgfpathquadraticcurveto{\pgfqpoint{1.938740in}{1.442588in}}{\pgfqpoint{1.905406in}{1.442588in}}%
\pgfpathlineto{\pgfqpoint{0.851728in}{1.442588in}}%
\pgfpathquadraticcurveto{\pgfqpoint{0.818395in}{1.442588in}}{\pgfqpoint{0.818395in}{1.409255in}}%
\pgfpathlineto{\pgfqpoint{0.818395in}{0.692035in}}%
\pgfpathquadraticcurveto{\pgfqpoint{0.818395in}{0.658702in}}{\pgfqpoint{0.851728in}{0.658702in}}%
\pgfpathlineto{\pgfqpoint{0.851728in}{0.658702in}}%
\pgfpathclose%
\pgfusepath{stroke,fill}%
\end{pgfscope}%
\begin{pgfscope}%
\pgfsetrectcap%
\pgfsetroundjoin%
\pgfsetlinewidth{1.003750pt}%
\definecolor{currentstroke}{rgb}{0.001462,0.000466,0.013866}%
\pgfsetstrokecolor{currentstroke}%
\pgfsetdash{}{0pt}%
\pgfpathmoveto{\pgfqpoint{0.885061in}{1.307627in}}%
\pgfpathlineto{\pgfqpoint{1.051728in}{1.307627in}}%
\pgfpathlineto{\pgfqpoint{1.218395in}{1.307627in}}%
\pgfusepath{stroke}%
\end{pgfscope}%
\begin{pgfscope}%
\pgfsetbuttcap%
\pgfsetroundjoin%
\definecolor{currentfill}{rgb}{0.001462,0.000466,0.013866}%
\pgfsetfillcolor{currentfill}%
\pgfsetlinewidth{1.003750pt}%
\definecolor{currentstroke}{rgb}{0.001462,0.000466,0.013866}%
\pgfsetstrokecolor{currentstroke}%
\pgfsetdash{}{0pt}%
\pgfsys@defobject{currentmarker}{\pgfqpoint{-0.020833in}{-0.020833in}}{\pgfqpoint{0.020833in}{0.020833in}}{%
\pgfpathmoveto{\pgfqpoint{0.000000in}{-0.020833in}}%
\pgfpathcurveto{\pgfqpoint{0.005525in}{-0.020833in}}{\pgfqpoint{0.010825in}{-0.018638in}}{\pgfqpoint{0.014731in}{-0.014731in}}%
\pgfpathcurveto{\pgfqpoint{0.018638in}{-0.010825in}}{\pgfqpoint{0.020833in}{-0.005525in}}{\pgfqpoint{0.020833in}{0.000000in}}%
\pgfpathcurveto{\pgfqpoint{0.020833in}{0.005525in}}{\pgfqpoint{0.018638in}{0.010825in}}{\pgfqpoint{0.014731in}{0.014731in}}%
\pgfpathcurveto{\pgfqpoint{0.010825in}{0.018638in}}{\pgfqpoint{0.005525in}{0.020833in}}{\pgfqpoint{0.000000in}{0.020833in}}%
\pgfpathcurveto{\pgfqpoint{-0.005525in}{0.020833in}}{\pgfqpoint{-0.010825in}{0.018638in}}{\pgfqpoint{-0.014731in}{0.014731in}}%
\pgfpathcurveto{\pgfqpoint{-0.018638in}{0.010825in}}{\pgfqpoint{-0.020833in}{0.005525in}}{\pgfqpoint{-0.020833in}{0.000000in}}%
\pgfpathcurveto{\pgfqpoint{-0.020833in}{-0.005525in}}{\pgfqpoint{-0.018638in}{-0.010825in}}{\pgfqpoint{-0.014731in}{-0.014731in}}%
\pgfpathcurveto{\pgfqpoint{-0.010825in}{-0.018638in}}{\pgfqpoint{-0.005525in}{-0.020833in}}{\pgfqpoint{0.000000in}{-0.020833in}}%
\pgfpathlineto{\pgfqpoint{0.000000in}{-0.020833in}}%
\pgfpathclose%
\pgfusepath{stroke,fill}%
}%
\begin{pgfscope}%
\pgfsys@transformshift{1.051728in}{1.307627in}%
\pgfsys@useobject{currentmarker}{}%
\end{pgfscope}%
\end{pgfscope}%
\begin{pgfscope}%
\definecolor{textcolor}{rgb}{0.000000,0.000000,0.000000}%
\pgfsetstrokecolor{textcolor}%
\pgfsetfillcolor{textcolor}%
\pgftext[x=1.351728in,y=1.249294in,left,base]{\color{textcolor}{\rmfamily\fontsize{12.000000}{14.400000}\selectfont\catcode`\^=\active\def^{\ifmmode\sp\else\^{}\fi}\catcode`\%=\active\def%{\%}DGC}}%
\end{pgfscope}%
\begin{pgfscope}%
\pgfsetrectcap%
\pgfsetroundjoin%
\pgfsetlinewidth{1.003750pt}%
\definecolor{currentstroke}{rgb}{0.445163,0.122724,0.506901}%
\pgfsetstrokecolor{currentstroke}%
\pgfsetdash{}{0pt}%
\pgfpathmoveto{\pgfqpoint{0.885061in}{1.062998in}}%
\pgfpathlineto{\pgfqpoint{1.051728in}{1.062998in}}%
\pgfpathlineto{\pgfqpoint{1.218395in}{1.062998in}}%
\pgfusepath{stroke}%
\end{pgfscope}%
\begin{pgfscope}%
\pgfsetbuttcap%
\pgfsetroundjoin%
\definecolor{currentfill}{rgb}{0.445163,0.122724,0.506901}%
\pgfsetfillcolor{currentfill}%
\pgfsetlinewidth{1.003750pt}%
\definecolor{currentstroke}{rgb}{0.445163,0.122724,0.506901}%
\pgfsetstrokecolor{currentstroke}%
\pgfsetdash{}{0pt}%
\pgfsys@defobject{currentmarker}{\pgfqpoint{-0.020833in}{-0.020833in}}{\pgfqpoint{0.020833in}{0.020833in}}{%
\pgfpathmoveto{\pgfqpoint{0.000000in}{-0.020833in}}%
\pgfpathcurveto{\pgfqpoint{0.005525in}{-0.020833in}}{\pgfqpoint{0.010825in}{-0.018638in}}{\pgfqpoint{0.014731in}{-0.014731in}}%
\pgfpathcurveto{\pgfqpoint{0.018638in}{-0.010825in}}{\pgfqpoint{0.020833in}{-0.005525in}}{\pgfqpoint{0.020833in}{0.000000in}}%
\pgfpathcurveto{\pgfqpoint{0.020833in}{0.005525in}}{\pgfqpoint{0.018638in}{0.010825in}}{\pgfqpoint{0.014731in}{0.014731in}}%
\pgfpathcurveto{\pgfqpoint{0.010825in}{0.018638in}}{\pgfqpoint{0.005525in}{0.020833in}}{\pgfqpoint{0.000000in}{0.020833in}}%
\pgfpathcurveto{\pgfqpoint{-0.005525in}{0.020833in}}{\pgfqpoint{-0.010825in}{0.018638in}}{\pgfqpoint{-0.014731in}{0.014731in}}%
\pgfpathcurveto{\pgfqpoint{-0.018638in}{0.010825in}}{\pgfqpoint{-0.020833in}{0.005525in}}{\pgfqpoint{-0.020833in}{0.000000in}}%
\pgfpathcurveto{\pgfqpoint{-0.020833in}{-0.005525in}}{\pgfqpoint{-0.018638in}{-0.010825in}}{\pgfqpoint{-0.014731in}{-0.014731in}}%
\pgfpathcurveto{\pgfqpoint{-0.010825in}{-0.018638in}}{\pgfqpoint{-0.005525in}{-0.020833in}}{\pgfqpoint{0.000000in}{-0.020833in}}%
\pgfpathlineto{\pgfqpoint{0.000000in}{-0.020833in}}%
\pgfpathclose%
\pgfusepath{stroke,fill}%
}%
\begin{pgfscope}%
\pgfsys@transformshift{1.051728in}{1.062998in}%
\pgfsys@useobject{currentmarker}{}%
\end{pgfscope}%
\end{pgfscope}%
\begin{pgfscope}%
\definecolor{textcolor}{rgb}{0.000000,0.000000,0.000000}%
\pgfsetstrokecolor{textcolor}%
\pgfsetfillcolor{textcolor}%
\pgftext[x=1.351728in,y=1.004665in,left,base]{\color{textcolor}{\rmfamily\fontsize{12.000000}{14.400000}\selectfont\catcode`\^=\active\def^{\ifmmode\sp\else\^{}\fi}\catcode`\%=\active\def%{\%}NC}}%
\end{pgfscope}%
\begin{pgfscope}%
\pgfsetrectcap%
\pgfsetroundjoin%
\pgfsetlinewidth{1.003750pt}%
\definecolor{currentstroke}{rgb}{0.944006,0.377643,0.365136}%
\pgfsetstrokecolor{currentstroke}%
\pgfsetdash{}{0pt}%
\pgfpathmoveto{\pgfqpoint{0.885061in}{0.818370in}}%
\pgfpathlineto{\pgfqpoint{1.051728in}{0.818370in}}%
\pgfpathlineto{\pgfqpoint{1.218395in}{0.818370in}}%
\pgfusepath{stroke}%
\end{pgfscope}%
\begin{pgfscope}%
\pgfsetbuttcap%
\pgfsetroundjoin%
\definecolor{currentfill}{rgb}{0.944006,0.377643,0.365136}%
\pgfsetfillcolor{currentfill}%
\pgfsetlinewidth{1.003750pt}%
\definecolor{currentstroke}{rgb}{0.944006,0.377643,0.365136}%
\pgfsetstrokecolor{currentstroke}%
\pgfsetdash{}{0pt}%
\pgfsys@defobject{currentmarker}{\pgfqpoint{-0.020833in}{-0.020833in}}{\pgfqpoint{0.020833in}{0.020833in}}{%
\pgfpathmoveto{\pgfqpoint{0.000000in}{-0.020833in}}%
\pgfpathcurveto{\pgfqpoint{0.005525in}{-0.020833in}}{\pgfqpoint{0.010825in}{-0.018638in}}{\pgfqpoint{0.014731in}{-0.014731in}}%
\pgfpathcurveto{\pgfqpoint{0.018638in}{-0.010825in}}{\pgfqpoint{0.020833in}{-0.005525in}}{\pgfqpoint{0.020833in}{0.000000in}}%
\pgfpathcurveto{\pgfqpoint{0.020833in}{0.005525in}}{\pgfqpoint{0.018638in}{0.010825in}}{\pgfqpoint{0.014731in}{0.014731in}}%
\pgfpathcurveto{\pgfqpoint{0.010825in}{0.018638in}}{\pgfqpoint{0.005525in}{0.020833in}}{\pgfqpoint{0.000000in}{0.020833in}}%
\pgfpathcurveto{\pgfqpoint{-0.005525in}{0.020833in}}{\pgfqpoint{-0.010825in}{0.018638in}}{\pgfqpoint{-0.014731in}{0.014731in}}%
\pgfpathcurveto{\pgfqpoint{-0.018638in}{0.010825in}}{\pgfqpoint{-0.020833in}{0.005525in}}{\pgfqpoint{-0.020833in}{0.000000in}}%
\pgfpathcurveto{\pgfqpoint{-0.020833in}{-0.005525in}}{\pgfqpoint{-0.018638in}{-0.010825in}}{\pgfqpoint{-0.014731in}{-0.014731in}}%
\pgfpathcurveto{\pgfqpoint{-0.010825in}{-0.018638in}}{\pgfqpoint{-0.005525in}{-0.020833in}}{\pgfqpoint{0.000000in}{-0.020833in}}%
\pgfpathlineto{\pgfqpoint{0.000000in}{-0.020833in}}%
\pgfpathclose%
\pgfusepath{stroke,fill}%
}%
\begin{pgfscope}%
\pgfsys@transformshift{1.051728in}{0.818370in}%
\pgfsys@useobject{currentmarker}{}%
\end{pgfscope}%
\end{pgfscope}%
\begin{pgfscope}%
\definecolor{textcolor}{rgb}{0.000000,0.000000,0.000000}%
\pgfsetstrokecolor{textcolor}%
\pgfsetfillcolor{textcolor}%
\pgftext[x=1.351728in,y=0.760036in,left,base]{\color{textcolor}{\rmfamily\fontsize{12.000000}{14.400000}\selectfont\catcode`\^=\active\def^{\ifmmode\sp\else\^{}\fi}\catcode`\%=\active\def%{\%}NC++}}%
\end{pgfscope}%
\end{pgfpicture}%
\makeatother%
\endgroup%

        \caption{$m=2400$}
        \label{fig:5-experiments-electronic-structure-convergence-nv-m2400}
    \end{subfigure}
    \caption{Behavior with $n_{\Omega}$ for $\sigma=0.05$}
    \label{fig:5-experiments-electronic-structure-convergence-nv}
\end{figure}

\begin{figure}[ht]
    \centering
    \begin{subfigure}[b]{0.49\columnwidth}
        %% Creator: Matplotlib, PGF backend
%%
%% To include the figure in your LaTeX document, write
%%   \input{<filename>.pgf}
%%
%% Make sure the required packages are loaded in your preamble
%%   \usepackage{pgf}
%%
%% Also ensure that all the required font packages are loaded; for instance,
%% the lmodern package is sometimes necessary when using math font.
%%   \usepackage{lmodern}
%%
%% Figures using additional raster images can only be included by \input if
%% they are in the same directory as the main LaTeX file. For loading figures
%% from other directories you can use the `import` package
%%   \usepackage{import}
%%
%% and then include the figures with
%%   \import{<path to file>}{<filename>.pgf}
%%
%% Matplotlib used the following preamble
%%   \def\mathdefault#1{#1}
%%   \everymath=\expandafter{\the\everymath\displaystyle}
%%   
%%   \usepackage{fontspec}
%%   \setmainfont{DejaVuSans.ttf}[Path=\detokenize{C:/Users/fabio/AppData/Local/Programs/Python/Python311/Lib/site-packages/matplotlib/mpl-data/fonts/ttf/}]
%%   \setsansfont{DejaVuSans.ttf}[Path=\detokenize{C:/Users/fabio/AppData/Local/Programs/Python/Python311/Lib/site-packages/matplotlib/mpl-data/fonts/ttf/}]
%%   \setmonofont{DejaVuSansMono.ttf}[Path=\detokenize{C:/Users/fabio/AppData/Local/Programs/Python/Python311/Lib/site-packages/matplotlib/mpl-data/fonts/ttf/}]
%%   \makeatletter\@ifpackageloaded{underscore}{}{\usepackage[strings]{underscore}}\makeatother
%%
\begingroup%
\makeatletter%
\begin{pgfpicture}%
\pgfpathrectangle{\pgfpointorigin}{\pgfqpoint{2.772561in}{2.600369in}}%
\pgfusepath{use as bounding box, clip}%
\begin{pgfscope}%
\pgfsetbuttcap%
\pgfsetmiterjoin%
\definecolor{currentfill}{rgb}{1.000000,1.000000,1.000000}%
\pgfsetfillcolor{currentfill}%
\pgfsetlinewidth{0.000000pt}%
\definecolor{currentstroke}{rgb}{1.000000,1.000000,1.000000}%
\pgfsetstrokecolor{currentstroke}%
\pgfsetdash{}{0pt}%
\pgfpathmoveto{\pgfqpoint{0.000000in}{-0.000000in}}%
\pgfpathlineto{\pgfqpoint{2.772561in}{-0.000000in}}%
\pgfpathlineto{\pgfqpoint{2.772561in}{2.600369in}}%
\pgfpathlineto{\pgfqpoint{0.000000in}{2.600369in}}%
\pgfpathlineto{\pgfqpoint{0.000000in}{-0.000000in}}%
\pgfpathclose%
\pgfusepath{fill}%
\end{pgfscope}%
\begin{pgfscope}%
\pgfsetbuttcap%
\pgfsetmiterjoin%
\definecolor{currentfill}{rgb}{1.000000,1.000000,1.000000}%
\pgfsetfillcolor{currentfill}%
\pgfsetlinewidth{0.000000pt}%
\definecolor{currentstroke}{rgb}{0.000000,0.000000,0.000000}%
\pgfsetstrokecolor{currentstroke}%
\pgfsetstrokeopacity{0.000000}%
\pgfsetdash{}{0pt}%
\pgfpathmoveto{\pgfqpoint{0.735061in}{0.575369in}}%
\pgfpathlineto{\pgfqpoint{2.672561in}{0.575369in}}%
\pgfpathlineto{\pgfqpoint{2.672561in}{2.500369in}}%
\pgfpathlineto{\pgfqpoint{0.735061in}{2.500369in}}%
\pgfpathlineto{\pgfqpoint{0.735061in}{0.575369in}}%
\pgfpathclose%
\pgfusepath{fill}%
\end{pgfscope}%
\begin{pgfscope}%
\pgfsetbuttcap%
\pgfsetroundjoin%
\definecolor{currentfill}{rgb}{0.000000,0.000000,0.000000}%
\pgfsetfillcolor{currentfill}%
\pgfsetlinewidth{0.803000pt}%
\definecolor{currentstroke}{rgb}{0.000000,0.000000,0.000000}%
\pgfsetstrokecolor{currentstroke}%
\pgfsetdash{}{0pt}%
\pgfsys@defobject{currentmarker}{\pgfqpoint{0.000000in}{-0.048611in}}{\pgfqpoint{0.000000in}{0.000000in}}{%
\pgfpathmoveto{\pgfqpoint{0.000000in}{0.000000in}}%
\pgfpathlineto{\pgfqpoint{0.000000in}{-0.048611in}}%
\pgfusepath{stroke,fill}%
}%
\begin{pgfscope}%
\pgfsys@transformshift{1.773721in}{0.575369in}%
\pgfsys@useobject{currentmarker}{}%
\end{pgfscope}%
\end{pgfscope}%
\begin{pgfscope}%
\definecolor{textcolor}{rgb}{0.000000,0.000000,0.000000}%
\pgfsetstrokecolor{textcolor}%
\pgfsetfillcolor{textcolor}%
\pgftext[x=1.773721in,y=0.478146in,,top]{\color{textcolor}{\rmfamily\fontsize{12.000000}{14.400000}\selectfont\catcode`\^=\active\def^{\ifmmode\sp\else\^{}\fi}\catcode`\%=\active\def%{\%}$\mathdefault{10^{3}}$}}%
\end{pgfscope}%
\begin{pgfscope}%
\pgfsetbuttcap%
\pgfsetroundjoin%
\definecolor{currentfill}{rgb}{0.000000,0.000000,0.000000}%
\pgfsetfillcolor{currentfill}%
\pgfsetlinewidth{0.602250pt}%
\definecolor{currentstroke}{rgb}{0.000000,0.000000,0.000000}%
\pgfsetstrokecolor{currentstroke}%
\pgfsetdash{}{0pt}%
\pgfsys@defobject{currentmarker}{\pgfqpoint{0.000000in}{-0.027778in}}{\pgfqpoint{0.000000in}{0.000000in}}{%
\pgfpathmoveto{\pgfqpoint{0.000000in}{0.000000in}}%
\pgfpathlineto{\pgfqpoint{0.000000in}{-0.027778in}}%
\pgfusepath{stroke,fill}%
}%
\begin{pgfscope}%
\pgfsys@transformshift{0.829029in}{0.575369in}%
\pgfsys@useobject{currentmarker}{}%
\end{pgfscope}%
\end{pgfscope}%
\begin{pgfscope}%
\pgfsetbuttcap%
\pgfsetroundjoin%
\definecolor{currentfill}{rgb}{0.000000,0.000000,0.000000}%
\pgfsetfillcolor{currentfill}%
\pgfsetlinewidth{0.602250pt}%
\definecolor{currentstroke}{rgb}{0.000000,0.000000,0.000000}%
\pgfsetstrokecolor{currentstroke}%
\pgfsetdash{}{0pt}%
\pgfsys@defobject{currentmarker}{\pgfqpoint{0.000000in}{-0.027778in}}{\pgfqpoint{0.000000in}{0.000000in}}{%
\pgfpathmoveto{\pgfqpoint{0.000000in}{0.000000in}}%
\pgfpathlineto{\pgfqpoint{0.000000in}{-0.027778in}}%
\pgfusepath{stroke,fill}%
}%
\begin{pgfscope}%
\pgfsys@transformshift{1.067025in}{0.575369in}%
\pgfsys@useobject{currentmarker}{}%
\end{pgfscope}%
\end{pgfscope}%
\begin{pgfscope}%
\pgfsetbuttcap%
\pgfsetroundjoin%
\definecolor{currentfill}{rgb}{0.000000,0.000000,0.000000}%
\pgfsetfillcolor{currentfill}%
\pgfsetlinewidth{0.602250pt}%
\definecolor{currentstroke}{rgb}{0.000000,0.000000,0.000000}%
\pgfsetstrokecolor{currentstroke}%
\pgfsetdash{}{0pt}%
\pgfsys@defobject{currentmarker}{\pgfqpoint{0.000000in}{-0.027778in}}{\pgfqpoint{0.000000in}{0.000000in}}{%
\pgfpathmoveto{\pgfqpoint{0.000000in}{0.000000in}}%
\pgfpathlineto{\pgfqpoint{0.000000in}{-0.027778in}}%
\pgfusepath{stroke,fill}%
}%
\begin{pgfscope}%
\pgfsys@transformshift{1.235886in}{0.575369in}%
\pgfsys@useobject{currentmarker}{}%
\end{pgfscope}%
\end{pgfscope}%
\begin{pgfscope}%
\pgfsetbuttcap%
\pgfsetroundjoin%
\definecolor{currentfill}{rgb}{0.000000,0.000000,0.000000}%
\pgfsetfillcolor{currentfill}%
\pgfsetlinewidth{0.602250pt}%
\definecolor{currentstroke}{rgb}{0.000000,0.000000,0.000000}%
\pgfsetstrokecolor{currentstroke}%
\pgfsetdash{}{0pt}%
\pgfsys@defobject{currentmarker}{\pgfqpoint{0.000000in}{-0.027778in}}{\pgfqpoint{0.000000in}{0.000000in}}{%
\pgfpathmoveto{\pgfqpoint{0.000000in}{0.000000in}}%
\pgfpathlineto{\pgfqpoint{0.000000in}{-0.027778in}}%
\pgfusepath{stroke,fill}%
}%
\begin{pgfscope}%
\pgfsys@transformshift{1.366864in}{0.575369in}%
\pgfsys@useobject{currentmarker}{}%
\end{pgfscope}%
\end{pgfscope}%
\begin{pgfscope}%
\pgfsetbuttcap%
\pgfsetroundjoin%
\definecolor{currentfill}{rgb}{0.000000,0.000000,0.000000}%
\pgfsetfillcolor{currentfill}%
\pgfsetlinewidth{0.602250pt}%
\definecolor{currentstroke}{rgb}{0.000000,0.000000,0.000000}%
\pgfsetstrokecolor{currentstroke}%
\pgfsetdash{}{0pt}%
\pgfsys@defobject{currentmarker}{\pgfqpoint{0.000000in}{-0.027778in}}{\pgfqpoint{0.000000in}{0.000000in}}{%
\pgfpathmoveto{\pgfqpoint{0.000000in}{0.000000in}}%
\pgfpathlineto{\pgfqpoint{0.000000in}{-0.027778in}}%
\pgfusepath{stroke,fill}%
}%
\begin{pgfscope}%
\pgfsys@transformshift{1.473882in}{0.575369in}%
\pgfsys@useobject{currentmarker}{}%
\end{pgfscope}%
\end{pgfscope}%
\begin{pgfscope}%
\pgfsetbuttcap%
\pgfsetroundjoin%
\definecolor{currentfill}{rgb}{0.000000,0.000000,0.000000}%
\pgfsetfillcolor{currentfill}%
\pgfsetlinewidth{0.602250pt}%
\definecolor{currentstroke}{rgb}{0.000000,0.000000,0.000000}%
\pgfsetstrokecolor{currentstroke}%
\pgfsetdash{}{0pt}%
\pgfsys@defobject{currentmarker}{\pgfqpoint{0.000000in}{-0.027778in}}{\pgfqpoint{0.000000in}{0.000000in}}{%
\pgfpathmoveto{\pgfqpoint{0.000000in}{0.000000in}}%
\pgfpathlineto{\pgfqpoint{0.000000in}{-0.027778in}}%
\pgfusepath{stroke,fill}%
}%
\begin{pgfscope}%
\pgfsys@transformshift{1.564364in}{0.575369in}%
\pgfsys@useobject{currentmarker}{}%
\end{pgfscope}%
\end{pgfscope}%
\begin{pgfscope}%
\pgfsetbuttcap%
\pgfsetroundjoin%
\definecolor{currentfill}{rgb}{0.000000,0.000000,0.000000}%
\pgfsetfillcolor{currentfill}%
\pgfsetlinewidth{0.602250pt}%
\definecolor{currentstroke}{rgb}{0.000000,0.000000,0.000000}%
\pgfsetstrokecolor{currentstroke}%
\pgfsetdash{}{0pt}%
\pgfsys@defobject{currentmarker}{\pgfqpoint{0.000000in}{-0.027778in}}{\pgfqpoint{0.000000in}{0.000000in}}{%
\pgfpathmoveto{\pgfqpoint{0.000000in}{0.000000in}}%
\pgfpathlineto{\pgfqpoint{0.000000in}{-0.027778in}}%
\pgfusepath{stroke,fill}%
}%
\begin{pgfscope}%
\pgfsys@transformshift{1.642743in}{0.575369in}%
\pgfsys@useobject{currentmarker}{}%
\end{pgfscope}%
\end{pgfscope}%
\begin{pgfscope}%
\pgfsetbuttcap%
\pgfsetroundjoin%
\definecolor{currentfill}{rgb}{0.000000,0.000000,0.000000}%
\pgfsetfillcolor{currentfill}%
\pgfsetlinewidth{0.602250pt}%
\definecolor{currentstroke}{rgb}{0.000000,0.000000,0.000000}%
\pgfsetstrokecolor{currentstroke}%
\pgfsetdash{}{0pt}%
\pgfsys@defobject{currentmarker}{\pgfqpoint{0.000000in}{-0.027778in}}{\pgfqpoint{0.000000in}{0.000000in}}{%
\pgfpathmoveto{\pgfqpoint{0.000000in}{0.000000in}}%
\pgfpathlineto{\pgfqpoint{0.000000in}{-0.027778in}}%
\pgfusepath{stroke,fill}%
}%
\begin{pgfscope}%
\pgfsys@transformshift{1.711878in}{0.575369in}%
\pgfsys@useobject{currentmarker}{}%
\end{pgfscope}%
\end{pgfscope}%
\begin{pgfscope}%
\pgfsetbuttcap%
\pgfsetroundjoin%
\definecolor{currentfill}{rgb}{0.000000,0.000000,0.000000}%
\pgfsetfillcolor{currentfill}%
\pgfsetlinewidth{0.602250pt}%
\definecolor{currentstroke}{rgb}{0.000000,0.000000,0.000000}%
\pgfsetstrokecolor{currentstroke}%
\pgfsetdash{}{0pt}%
\pgfsys@defobject{currentmarker}{\pgfqpoint{0.000000in}{-0.027778in}}{\pgfqpoint{0.000000in}{0.000000in}}{%
\pgfpathmoveto{\pgfqpoint{0.000000in}{0.000000in}}%
\pgfpathlineto{\pgfqpoint{0.000000in}{-0.027778in}}%
\pgfusepath{stroke,fill}%
}%
\begin{pgfscope}%
\pgfsys@transformshift{2.180578in}{0.575369in}%
\pgfsys@useobject{currentmarker}{}%
\end{pgfscope}%
\end{pgfscope}%
\begin{pgfscope}%
\pgfsetbuttcap%
\pgfsetroundjoin%
\definecolor{currentfill}{rgb}{0.000000,0.000000,0.000000}%
\pgfsetfillcolor{currentfill}%
\pgfsetlinewidth{0.602250pt}%
\definecolor{currentstroke}{rgb}{0.000000,0.000000,0.000000}%
\pgfsetstrokecolor{currentstroke}%
\pgfsetdash{}{0pt}%
\pgfsys@defobject{currentmarker}{\pgfqpoint{0.000000in}{-0.027778in}}{\pgfqpoint{0.000000in}{0.000000in}}{%
\pgfpathmoveto{\pgfqpoint{0.000000in}{0.000000in}}%
\pgfpathlineto{\pgfqpoint{0.000000in}{-0.027778in}}%
\pgfusepath{stroke,fill}%
}%
\begin{pgfscope}%
\pgfsys@transformshift{2.418575in}{0.575369in}%
\pgfsys@useobject{currentmarker}{}%
\end{pgfscope}%
\end{pgfscope}%
\begin{pgfscope}%
\pgfsetbuttcap%
\pgfsetroundjoin%
\definecolor{currentfill}{rgb}{0.000000,0.000000,0.000000}%
\pgfsetfillcolor{currentfill}%
\pgfsetlinewidth{0.602250pt}%
\definecolor{currentstroke}{rgb}{0.000000,0.000000,0.000000}%
\pgfsetstrokecolor{currentstroke}%
\pgfsetdash{}{0pt}%
\pgfsys@defobject{currentmarker}{\pgfqpoint{0.000000in}{-0.027778in}}{\pgfqpoint{0.000000in}{0.000000in}}{%
\pgfpathmoveto{\pgfqpoint{0.000000in}{0.000000in}}%
\pgfpathlineto{\pgfqpoint{0.000000in}{-0.027778in}}%
\pgfusepath{stroke,fill}%
}%
\begin{pgfscope}%
\pgfsys@transformshift{2.587435in}{0.575369in}%
\pgfsys@useobject{currentmarker}{}%
\end{pgfscope}%
\end{pgfscope}%
\begin{pgfscope}%
\definecolor{textcolor}{rgb}{0.000000,0.000000,0.000000}%
\pgfsetstrokecolor{textcolor}%
\pgfsetfillcolor{textcolor}%
\pgftext[x=1.703811in,y=0.261295in,,top]{\color{textcolor}{\rmfamily\fontsize{12.000000}{14.400000}\selectfont\catcode`\^=\active\def^{\ifmmode\sp\else\^{}\fi}\catcode`\%=\active\def%{\%}$m$}}%
\end{pgfscope}%
\begin{pgfscope}%
\pgfsetbuttcap%
\pgfsetroundjoin%
\definecolor{currentfill}{rgb}{0.000000,0.000000,0.000000}%
\pgfsetfillcolor{currentfill}%
\pgfsetlinewidth{0.803000pt}%
\definecolor{currentstroke}{rgb}{0.000000,0.000000,0.000000}%
\pgfsetstrokecolor{currentstroke}%
\pgfsetdash{}{0pt}%
\pgfsys@defobject{currentmarker}{\pgfqpoint{-0.048611in}{0.000000in}}{\pgfqpoint{-0.000000in}{0.000000in}}{%
\pgfpathmoveto{\pgfqpoint{-0.000000in}{0.000000in}}%
\pgfpathlineto{\pgfqpoint{-0.048611in}{0.000000in}}%
\pgfusepath{stroke,fill}%
}%
\begin{pgfscope}%
\pgfsys@transformshift{0.735061in}{0.957169in}%
\pgfsys@useobject{currentmarker}{}%
\end{pgfscope}%
\end{pgfscope}%
\begin{pgfscope}%
\definecolor{textcolor}{rgb}{0.000000,0.000000,0.000000}%
\pgfsetstrokecolor{textcolor}%
\pgfsetfillcolor{textcolor}%
\pgftext[x=0.316851in, y=0.893855in, left, base]{\color{textcolor}{\rmfamily\fontsize{12.000000}{14.400000}\selectfont\catcode`\^=\active\def^{\ifmmode\sp\else\^{}\fi}\catcode`\%=\active\def%{\%}$\mathdefault{10^{-6}}$}}%
\end{pgfscope}%
\begin{pgfscope}%
\pgfsetbuttcap%
\pgfsetroundjoin%
\definecolor{currentfill}{rgb}{0.000000,0.000000,0.000000}%
\pgfsetfillcolor{currentfill}%
\pgfsetlinewidth{0.803000pt}%
\definecolor{currentstroke}{rgb}{0.000000,0.000000,0.000000}%
\pgfsetstrokecolor{currentstroke}%
\pgfsetdash{}{0pt}%
\pgfsys@defobject{currentmarker}{\pgfqpoint{-0.048611in}{0.000000in}}{\pgfqpoint{-0.000000in}{0.000000in}}{%
\pgfpathmoveto{\pgfqpoint{-0.000000in}{0.000000in}}%
\pgfpathlineto{\pgfqpoint{-0.048611in}{0.000000in}}%
\pgfusepath{stroke,fill}%
}%
\begin{pgfscope}%
\pgfsys@transformshift{0.735061in}{1.502659in}%
\pgfsys@useobject{currentmarker}{}%
\end{pgfscope}%
\end{pgfscope}%
\begin{pgfscope}%
\definecolor{textcolor}{rgb}{0.000000,0.000000,0.000000}%
\pgfsetstrokecolor{textcolor}%
\pgfsetfillcolor{textcolor}%
\pgftext[x=0.316851in, y=1.439346in, left, base]{\color{textcolor}{\rmfamily\fontsize{12.000000}{14.400000}\selectfont\catcode`\^=\active\def^{\ifmmode\sp\else\^{}\fi}\catcode`\%=\active\def%{\%}$\mathdefault{10^{-4}}$}}%
\end{pgfscope}%
\begin{pgfscope}%
\pgfsetbuttcap%
\pgfsetroundjoin%
\definecolor{currentfill}{rgb}{0.000000,0.000000,0.000000}%
\pgfsetfillcolor{currentfill}%
\pgfsetlinewidth{0.803000pt}%
\definecolor{currentstroke}{rgb}{0.000000,0.000000,0.000000}%
\pgfsetstrokecolor{currentstroke}%
\pgfsetdash{}{0pt}%
\pgfsys@defobject{currentmarker}{\pgfqpoint{-0.048611in}{0.000000in}}{\pgfqpoint{-0.000000in}{0.000000in}}{%
\pgfpathmoveto{\pgfqpoint{-0.000000in}{0.000000in}}%
\pgfpathlineto{\pgfqpoint{-0.048611in}{0.000000in}}%
\pgfusepath{stroke,fill}%
}%
\begin{pgfscope}%
\pgfsys@transformshift{0.735061in}{2.048150in}%
\pgfsys@useobject{currentmarker}{}%
\end{pgfscope}%
\end{pgfscope}%
\begin{pgfscope}%
\definecolor{textcolor}{rgb}{0.000000,0.000000,0.000000}%
\pgfsetstrokecolor{textcolor}%
\pgfsetfillcolor{textcolor}%
\pgftext[x=0.316851in, y=1.984837in, left, base]{\color{textcolor}{\rmfamily\fontsize{12.000000}{14.400000}\selectfont\catcode`\^=\active\def^{\ifmmode\sp\else\^{}\fi}\catcode`\%=\active\def%{\%}$\mathdefault{10^{-2}}$}}%
\end{pgfscope}%
\begin{pgfscope}%
\definecolor{textcolor}{rgb}{0.000000,0.000000,0.000000}%
\pgfsetstrokecolor{textcolor}%
\pgfsetfillcolor{textcolor}%
\pgftext[x=0.261295in,y=1.537869in,,bottom,rotate=90.000000]{\color{textcolor}{\rmfamily\fontsize{12.000000}{14.400000}\selectfont\catcode`\^=\active\def^{\ifmmode\sp\else\^{}\fi}\catcode`\%=\active\def%{\%}$L^1$ relative error}}%
\end{pgfscope}%
\begin{pgfscope}%
\pgfpathrectangle{\pgfqpoint{0.735061in}{0.575369in}}{\pgfqpoint{1.937500in}{1.925000in}}%
\pgfusepath{clip}%
\pgfsetrectcap%
\pgfsetroundjoin%
\pgfsetlinewidth{1.003750pt}%
\definecolor{currentstroke}{rgb}{0.001462,0.000466,0.013866}%
\pgfsetstrokecolor{currentstroke}%
\pgfsetdash{}{0pt}%
\pgfpathmoveto{\pgfqpoint{0.823130in}{2.258007in}}%
\pgfpathlineto{\pgfqpoint{1.177294in}{2.250922in}}%
\pgfpathlineto{\pgfqpoint{1.529826in}{2.208397in}}%
\pgfpathlineto{\pgfqpoint{1.881716in}{2.151888in}}%
\pgfpathlineto{\pgfqpoint{2.232776in}{2.132875in}}%
\pgfpathlineto{\pgfqpoint{2.584493in}{2.061775in}}%
\pgfusepath{stroke}%
\end{pgfscope}%
\begin{pgfscope}%
\pgfpathrectangle{\pgfqpoint{0.735061in}{0.575369in}}{\pgfqpoint{1.937500in}{1.925000in}}%
\pgfusepath{clip}%
\pgfsetbuttcap%
\pgfsetroundjoin%
\definecolor{currentfill}{rgb}{0.001462,0.000466,0.013866}%
\pgfsetfillcolor{currentfill}%
\pgfsetlinewidth{1.003750pt}%
\definecolor{currentstroke}{rgb}{0.001462,0.000466,0.013866}%
\pgfsetstrokecolor{currentstroke}%
\pgfsetdash{}{0pt}%
\pgfsys@defobject{currentmarker}{\pgfqpoint{-0.020833in}{-0.020833in}}{\pgfqpoint{0.020833in}{0.020833in}}{%
\pgfpathmoveto{\pgfqpoint{0.000000in}{-0.020833in}}%
\pgfpathcurveto{\pgfqpoint{0.005525in}{-0.020833in}}{\pgfqpoint{0.010825in}{-0.018638in}}{\pgfqpoint{0.014731in}{-0.014731in}}%
\pgfpathcurveto{\pgfqpoint{0.018638in}{-0.010825in}}{\pgfqpoint{0.020833in}{-0.005525in}}{\pgfqpoint{0.020833in}{0.000000in}}%
\pgfpathcurveto{\pgfqpoint{0.020833in}{0.005525in}}{\pgfqpoint{0.018638in}{0.010825in}}{\pgfqpoint{0.014731in}{0.014731in}}%
\pgfpathcurveto{\pgfqpoint{0.010825in}{0.018638in}}{\pgfqpoint{0.005525in}{0.020833in}}{\pgfqpoint{0.000000in}{0.020833in}}%
\pgfpathcurveto{\pgfqpoint{-0.005525in}{0.020833in}}{\pgfqpoint{-0.010825in}{0.018638in}}{\pgfqpoint{-0.014731in}{0.014731in}}%
\pgfpathcurveto{\pgfqpoint{-0.018638in}{0.010825in}}{\pgfqpoint{-0.020833in}{0.005525in}}{\pgfqpoint{-0.020833in}{0.000000in}}%
\pgfpathcurveto{\pgfqpoint{-0.020833in}{-0.005525in}}{\pgfqpoint{-0.018638in}{-0.010825in}}{\pgfqpoint{-0.014731in}{-0.014731in}}%
\pgfpathcurveto{\pgfqpoint{-0.010825in}{-0.018638in}}{\pgfqpoint{-0.005525in}{-0.020833in}}{\pgfqpoint{0.000000in}{-0.020833in}}%
\pgfpathlineto{\pgfqpoint{0.000000in}{-0.020833in}}%
\pgfpathclose%
\pgfusepath{stroke,fill}%
}%
\begin{pgfscope}%
\pgfsys@transformshift{0.823130in}{2.258007in}%
\pgfsys@useobject{currentmarker}{}%
\end{pgfscope}%
\begin{pgfscope}%
\pgfsys@transformshift{1.177294in}{2.250922in}%
\pgfsys@useobject{currentmarker}{}%
\end{pgfscope}%
\begin{pgfscope}%
\pgfsys@transformshift{1.529826in}{2.208397in}%
\pgfsys@useobject{currentmarker}{}%
\end{pgfscope}%
\begin{pgfscope}%
\pgfsys@transformshift{1.881716in}{2.151888in}%
\pgfsys@useobject{currentmarker}{}%
\end{pgfscope}%
\begin{pgfscope}%
\pgfsys@transformshift{2.232776in}{2.132875in}%
\pgfsys@useobject{currentmarker}{}%
\end{pgfscope}%
\begin{pgfscope}%
\pgfsys@transformshift{2.584493in}{2.061775in}%
\pgfsys@useobject{currentmarker}{}%
\end{pgfscope}%
\end{pgfscope}%
\begin{pgfscope}%
\pgfpathrectangle{\pgfqpoint{0.735061in}{0.575369in}}{\pgfqpoint{1.937500in}{1.925000in}}%
\pgfusepath{clip}%
\pgfsetrectcap%
\pgfsetroundjoin%
\pgfsetlinewidth{1.003750pt}%
\definecolor{currentstroke}{rgb}{0.445163,0.122724,0.506901}%
\pgfsetstrokecolor{currentstroke}%
\pgfsetdash{}{0pt}%
\pgfpathmoveto{\pgfqpoint{0.823130in}{2.412869in}}%
\pgfpathlineto{\pgfqpoint{1.177294in}{2.284532in}}%
\pgfpathlineto{\pgfqpoint{1.529826in}{1.798824in}}%
\pgfpathlineto{\pgfqpoint{1.881716in}{0.693032in}}%
\pgfpathlineto{\pgfqpoint{2.232776in}{0.679559in}}%
\pgfpathlineto{\pgfqpoint{2.584493in}{0.684647in}}%
\pgfusepath{stroke}%
\end{pgfscope}%
\begin{pgfscope}%
\pgfpathrectangle{\pgfqpoint{0.735061in}{0.575369in}}{\pgfqpoint{1.937500in}{1.925000in}}%
\pgfusepath{clip}%
\pgfsetbuttcap%
\pgfsetroundjoin%
\definecolor{currentfill}{rgb}{0.445163,0.122724,0.506901}%
\pgfsetfillcolor{currentfill}%
\pgfsetlinewidth{1.003750pt}%
\definecolor{currentstroke}{rgb}{0.445163,0.122724,0.506901}%
\pgfsetstrokecolor{currentstroke}%
\pgfsetdash{}{0pt}%
\pgfsys@defobject{currentmarker}{\pgfqpoint{-0.020833in}{-0.020833in}}{\pgfqpoint{0.020833in}{0.020833in}}{%
\pgfpathmoveto{\pgfqpoint{0.000000in}{-0.020833in}}%
\pgfpathcurveto{\pgfqpoint{0.005525in}{-0.020833in}}{\pgfqpoint{0.010825in}{-0.018638in}}{\pgfqpoint{0.014731in}{-0.014731in}}%
\pgfpathcurveto{\pgfqpoint{0.018638in}{-0.010825in}}{\pgfqpoint{0.020833in}{-0.005525in}}{\pgfqpoint{0.020833in}{0.000000in}}%
\pgfpathcurveto{\pgfqpoint{0.020833in}{0.005525in}}{\pgfqpoint{0.018638in}{0.010825in}}{\pgfqpoint{0.014731in}{0.014731in}}%
\pgfpathcurveto{\pgfqpoint{0.010825in}{0.018638in}}{\pgfqpoint{0.005525in}{0.020833in}}{\pgfqpoint{0.000000in}{0.020833in}}%
\pgfpathcurveto{\pgfqpoint{-0.005525in}{0.020833in}}{\pgfqpoint{-0.010825in}{0.018638in}}{\pgfqpoint{-0.014731in}{0.014731in}}%
\pgfpathcurveto{\pgfqpoint{-0.018638in}{0.010825in}}{\pgfqpoint{-0.020833in}{0.005525in}}{\pgfqpoint{-0.020833in}{0.000000in}}%
\pgfpathcurveto{\pgfqpoint{-0.020833in}{-0.005525in}}{\pgfqpoint{-0.018638in}{-0.010825in}}{\pgfqpoint{-0.014731in}{-0.014731in}}%
\pgfpathcurveto{\pgfqpoint{-0.010825in}{-0.018638in}}{\pgfqpoint{-0.005525in}{-0.020833in}}{\pgfqpoint{0.000000in}{-0.020833in}}%
\pgfpathlineto{\pgfqpoint{0.000000in}{-0.020833in}}%
\pgfpathclose%
\pgfusepath{stroke,fill}%
}%
\begin{pgfscope}%
\pgfsys@transformshift{0.823130in}{2.412869in}%
\pgfsys@useobject{currentmarker}{}%
\end{pgfscope}%
\begin{pgfscope}%
\pgfsys@transformshift{1.177294in}{2.284532in}%
\pgfsys@useobject{currentmarker}{}%
\end{pgfscope}%
\begin{pgfscope}%
\pgfsys@transformshift{1.529826in}{1.798824in}%
\pgfsys@useobject{currentmarker}{}%
\end{pgfscope}%
\begin{pgfscope}%
\pgfsys@transformshift{1.881716in}{0.693032in}%
\pgfsys@useobject{currentmarker}{}%
\end{pgfscope}%
\begin{pgfscope}%
\pgfsys@transformshift{2.232776in}{0.679559in}%
\pgfsys@useobject{currentmarker}{}%
\end{pgfscope}%
\begin{pgfscope}%
\pgfsys@transformshift{2.584493in}{0.684647in}%
\pgfsys@useobject{currentmarker}{}%
\end{pgfscope}%
\end{pgfscope}%
\begin{pgfscope}%
\pgfpathrectangle{\pgfqpoint{0.735061in}{0.575369in}}{\pgfqpoint{1.937500in}{1.925000in}}%
\pgfusepath{clip}%
\pgfsetrectcap%
\pgfsetroundjoin%
\pgfsetlinewidth{1.003750pt}%
\definecolor{currentstroke}{rgb}{0.944006,0.377643,0.365136}%
\pgfsetstrokecolor{currentstroke}%
\pgfsetdash{}{0pt}%
\pgfpathmoveto{\pgfqpoint{0.823130in}{2.250839in}}%
\pgfpathlineto{\pgfqpoint{1.177294in}{2.002936in}}%
\pgfpathlineto{\pgfqpoint{1.529826in}{1.922920in}}%
\pgfpathlineto{\pgfqpoint{1.881716in}{1.551407in}}%
\pgfpathlineto{\pgfqpoint{2.232776in}{0.663193in}}%
\pgfpathlineto{\pgfqpoint{2.584493in}{0.662869in}}%
\pgfusepath{stroke}%
\end{pgfscope}%
\begin{pgfscope}%
\pgfpathrectangle{\pgfqpoint{0.735061in}{0.575369in}}{\pgfqpoint{1.937500in}{1.925000in}}%
\pgfusepath{clip}%
\pgfsetbuttcap%
\pgfsetroundjoin%
\definecolor{currentfill}{rgb}{0.944006,0.377643,0.365136}%
\pgfsetfillcolor{currentfill}%
\pgfsetlinewidth{1.003750pt}%
\definecolor{currentstroke}{rgb}{0.944006,0.377643,0.365136}%
\pgfsetstrokecolor{currentstroke}%
\pgfsetdash{}{0pt}%
\pgfsys@defobject{currentmarker}{\pgfqpoint{-0.020833in}{-0.020833in}}{\pgfqpoint{0.020833in}{0.020833in}}{%
\pgfpathmoveto{\pgfqpoint{0.000000in}{-0.020833in}}%
\pgfpathcurveto{\pgfqpoint{0.005525in}{-0.020833in}}{\pgfqpoint{0.010825in}{-0.018638in}}{\pgfqpoint{0.014731in}{-0.014731in}}%
\pgfpathcurveto{\pgfqpoint{0.018638in}{-0.010825in}}{\pgfqpoint{0.020833in}{-0.005525in}}{\pgfqpoint{0.020833in}{0.000000in}}%
\pgfpathcurveto{\pgfqpoint{0.020833in}{0.005525in}}{\pgfqpoint{0.018638in}{0.010825in}}{\pgfqpoint{0.014731in}{0.014731in}}%
\pgfpathcurveto{\pgfqpoint{0.010825in}{0.018638in}}{\pgfqpoint{0.005525in}{0.020833in}}{\pgfqpoint{0.000000in}{0.020833in}}%
\pgfpathcurveto{\pgfqpoint{-0.005525in}{0.020833in}}{\pgfqpoint{-0.010825in}{0.018638in}}{\pgfqpoint{-0.014731in}{0.014731in}}%
\pgfpathcurveto{\pgfqpoint{-0.018638in}{0.010825in}}{\pgfqpoint{-0.020833in}{0.005525in}}{\pgfqpoint{-0.020833in}{0.000000in}}%
\pgfpathcurveto{\pgfqpoint{-0.020833in}{-0.005525in}}{\pgfqpoint{-0.018638in}{-0.010825in}}{\pgfqpoint{-0.014731in}{-0.014731in}}%
\pgfpathcurveto{\pgfqpoint{-0.010825in}{-0.018638in}}{\pgfqpoint{-0.005525in}{-0.020833in}}{\pgfqpoint{0.000000in}{-0.020833in}}%
\pgfpathlineto{\pgfqpoint{0.000000in}{-0.020833in}}%
\pgfpathclose%
\pgfusepath{stroke,fill}%
}%
\begin{pgfscope}%
\pgfsys@transformshift{0.823130in}{2.250839in}%
\pgfsys@useobject{currentmarker}{}%
\end{pgfscope}%
\begin{pgfscope}%
\pgfsys@transformshift{1.177294in}{2.002936in}%
\pgfsys@useobject{currentmarker}{}%
\end{pgfscope}%
\begin{pgfscope}%
\pgfsys@transformshift{1.529826in}{1.922920in}%
\pgfsys@useobject{currentmarker}{}%
\end{pgfscope}%
\begin{pgfscope}%
\pgfsys@transformshift{1.881716in}{1.551407in}%
\pgfsys@useobject{currentmarker}{}%
\end{pgfscope}%
\begin{pgfscope}%
\pgfsys@transformshift{2.232776in}{0.663193in}%
\pgfsys@useobject{currentmarker}{}%
\end{pgfscope}%
\begin{pgfscope}%
\pgfsys@transformshift{2.584493in}{0.662869in}%
\pgfsys@useobject{currentmarker}{}%
\end{pgfscope}%
\end{pgfscope}%
\begin{pgfscope}%
\pgfsetrectcap%
\pgfsetmiterjoin%
\pgfsetlinewidth{0.803000pt}%
\definecolor{currentstroke}{rgb}{0.000000,0.000000,0.000000}%
\pgfsetstrokecolor{currentstroke}%
\pgfsetdash{}{0pt}%
\pgfpathmoveto{\pgfqpoint{0.735061in}{0.575369in}}%
\pgfpathlineto{\pgfqpoint{0.735061in}{2.500369in}}%
\pgfusepath{stroke}%
\end{pgfscope}%
\begin{pgfscope}%
\pgfsetrectcap%
\pgfsetmiterjoin%
\pgfsetlinewidth{0.803000pt}%
\definecolor{currentstroke}{rgb}{0.000000,0.000000,0.000000}%
\pgfsetstrokecolor{currentstroke}%
\pgfsetdash{}{0pt}%
\pgfpathmoveto{\pgfqpoint{2.672561in}{0.575369in}}%
\pgfpathlineto{\pgfqpoint{2.672561in}{2.500369in}}%
\pgfusepath{stroke}%
\end{pgfscope}%
\begin{pgfscope}%
\pgfsetrectcap%
\pgfsetmiterjoin%
\pgfsetlinewidth{0.803000pt}%
\definecolor{currentstroke}{rgb}{0.000000,0.000000,0.000000}%
\pgfsetstrokecolor{currentstroke}%
\pgfsetdash{}{0pt}%
\pgfpathmoveto{\pgfqpoint{0.735061in}{0.575369in}}%
\pgfpathlineto{\pgfqpoint{2.672561in}{0.575369in}}%
\pgfusepath{stroke}%
\end{pgfscope}%
\begin{pgfscope}%
\pgfsetrectcap%
\pgfsetmiterjoin%
\pgfsetlinewidth{0.803000pt}%
\definecolor{currentstroke}{rgb}{0.000000,0.000000,0.000000}%
\pgfsetstrokecolor{currentstroke}%
\pgfsetdash{}{0pt}%
\pgfpathmoveto{\pgfqpoint{0.735061in}{2.500369in}}%
\pgfpathlineto{\pgfqpoint{2.672561in}{2.500369in}}%
\pgfusepath{stroke}%
\end{pgfscope}%
\begin{pgfscope}%
\pgfsetbuttcap%
\pgfsetmiterjoin%
\definecolor{currentfill}{rgb}{1.000000,1.000000,1.000000}%
\pgfsetfillcolor{currentfill}%
\pgfsetfillopacity{0.800000}%
\pgfsetlinewidth{1.003750pt}%
\definecolor{currentstroke}{rgb}{0.800000,0.800000,0.800000}%
\pgfsetstrokecolor{currentstroke}%
\pgfsetstrokeopacity{0.800000}%
\pgfsetdash{}{0pt}%
\pgfpathmoveto{\pgfqpoint{0.851728in}{0.658702in}}%
\pgfpathlineto{\pgfqpoint{1.905406in}{0.658702in}}%
\pgfpathquadraticcurveto{\pgfqpoint{1.938740in}{0.658702in}}{\pgfqpoint{1.938740in}{0.692035in}}%
\pgfpathlineto{\pgfqpoint{1.938740in}{1.409255in}}%
\pgfpathquadraticcurveto{\pgfqpoint{1.938740in}{1.442588in}}{\pgfqpoint{1.905406in}{1.442588in}}%
\pgfpathlineto{\pgfqpoint{0.851728in}{1.442588in}}%
\pgfpathquadraticcurveto{\pgfqpoint{0.818395in}{1.442588in}}{\pgfqpoint{0.818395in}{1.409255in}}%
\pgfpathlineto{\pgfqpoint{0.818395in}{0.692035in}}%
\pgfpathquadraticcurveto{\pgfqpoint{0.818395in}{0.658702in}}{\pgfqpoint{0.851728in}{0.658702in}}%
\pgfpathlineto{\pgfqpoint{0.851728in}{0.658702in}}%
\pgfpathclose%
\pgfusepath{stroke,fill}%
\end{pgfscope}%
\begin{pgfscope}%
\pgfsetrectcap%
\pgfsetroundjoin%
\pgfsetlinewidth{1.003750pt}%
\definecolor{currentstroke}{rgb}{0.001462,0.000466,0.013866}%
\pgfsetstrokecolor{currentstroke}%
\pgfsetdash{}{0pt}%
\pgfpathmoveto{\pgfqpoint{0.885061in}{1.307627in}}%
\pgfpathlineto{\pgfqpoint{1.051728in}{1.307627in}}%
\pgfpathlineto{\pgfqpoint{1.218395in}{1.307627in}}%
\pgfusepath{stroke}%
\end{pgfscope}%
\begin{pgfscope}%
\pgfsetbuttcap%
\pgfsetroundjoin%
\definecolor{currentfill}{rgb}{0.001462,0.000466,0.013866}%
\pgfsetfillcolor{currentfill}%
\pgfsetlinewidth{1.003750pt}%
\definecolor{currentstroke}{rgb}{0.001462,0.000466,0.013866}%
\pgfsetstrokecolor{currentstroke}%
\pgfsetdash{}{0pt}%
\pgfsys@defobject{currentmarker}{\pgfqpoint{-0.020833in}{-0.020833in}}{\pgfqpoint{0.020833in}{0.020833in}}{%
\pgfpathmoveto{\pgfqpoint{0.000000in}{-0.020833in}}%
\pgfpathcurveto{\pgfqpoint{0.005525in}{-0.020833in}}{\pgfqpoint{0.010825in}{-0.018638in}}{\pgfqpoint{0.014731in}{-0.014731in}}%
\pgfpathcurveto{\pgfqpoint{0.018638in}{-0.010825in}}{\pgfqpoint{0.020833in}{-0.005525in}}{\pgfqpoint{0.020833in}{0.000000in}}%
\pgfpathcurveto{\pgfqpoint{0.020833in}{0.005525in}}{\pgfqpoint{0.018638in}{0.010825in}}{\pgfqpoint{0.014731in}{0.014731in}}%
\pgfpathcurveto{\pgfqpoint{0.010825in}{0.018638in}}{\pgfqpoint{0.005525in}{0.020833in}}{\pgfqpoint{0.000000in}{0.020833in}}%
\pgfpathcurveto{\pgfqpoint{-0.005525in}{0.020833in}}{\pgfqpoint{-0.010825in}{0.018638in}}{\pgfqpoint{-0.014731in}{0.014731in}}%
\pgfpathcurveto{\pgfqpoint{-0.018638in}{0.010825in}}{\pgfqpoint{-0.020833in}{0.005525in}}{\pgfqpoint{-0.020833in}{0.000000in}}%
\pgfpathcurveto{\pgfqpoint{-0.020833in}{-0.005525in}}{\pgfqpoint{-0.018638in}{-0.010825in}}{\pgfqpoint{-0.014731in}{-0.014731in}}%
\pgfpathcurveto{\pgfqpoint{-0.010825in}{-0.018638in}}{\pgfqpoint{-0.005525in}{-0.020833in}}{\pgfqpoint{0.000000in}{-0.020833in}}%
\pgfpathlineto{\pgfqpoint{0.000000in}{-0.020833in}}%
\pgfpathclose%
\pgfusepath{stroke,fill}%
}%
\begin{pgfscope}%
\pgfsys@transformshift{1.051728in}{1.307627in}%
\pgfsys@useobject{currentmarker}{}%
\end{pgfscope}%
\end{pgfscope}%
\begin{pgfscope}%
\definecolor{textcolor}{rgb}{0.000000,0.000000,0.000000}%
\pgfsetstrokecolor{textcolor}%
\pgfsetfillcolor{textcolor}%
\pgftext[x=1.351728in,y=1.249294in,left,base]{\color{textcolor}{\rmfamily\fontsize{12.000000}{14.400000}\selectfont\catcode`\^=\active\def^{\ifmmode\sp\else\^{}\fi}\catcode`\%=\active\def%{\%}DGC}}%
\end{pgfscope}%
\begin{pgfscope}%
\pgfsetrectcap%
\pgfsetroundjoin%
\pgfsetlinewidth{1.003750pt}%
\definecolor{currentstroke}{rgb}{0.445163,0.122724,0.506901}%
\pgfsetstrokecolor{currentstroke}%
\pgfsetdash{}{0pt}%
\pgfpathmoveto{\pgfqpoint{0.885061in}{1.062998in}}%
\pgfpathlineto{\pgfqpoint{1.051728in}{1.062998in}}%
\pgfpathlineto{\pgfqpoint{1.218395in}{1.062998in}}%
\pgfusepath{stroke}%
\end{pgfscope}%
\begin{pgfscope}%
\pgfsetbuttcap%
\pgfsetroundjoin%
\definecolor{currentfill}{rgb}{0.445163,0.122724,0.506901}%
\pgfsetfillcolor{currentfill}%
\pgfsetlinewidth{1.003750pt}%
\definecolor{currentstroke}{rgb}{0.445163,0.122724,0.506901}%
\pgfsetstrokecolor{currentstroke}%
\pgfsetdash{}{0pt}%
\pgfsys@defobject{currentmarker}{\pgfqpoint{-0.020833in}{-0.020833in}}{\pgfqpoint{0.020833in}{0.020833in}}{%
\pgfpathmoveto{\pgfqpoint{0.000000in}{-0.020833in}}%
\pgfpathcurveto{\pgfqpoint{0.005525in}{-0.020833in}}{\pgfqpoint{0.010825in}{-0.018638in}}{\pgfqpoint{0.014731in}{-0.014731in}}%
\pgfpathcurveto{\pgfqpoint{0.018638in}{-0.010825in}}{\pgfqpoint{0.020833in}{-0.005525in}}{\pgfqpoint{0.020833in}{0.000000in}}%
\pgfpathcurveto{\pgfqpoint{0.020833in}{0.005525in}}{\pgfqpoint{0.018638in}{0.010825in}}{\pgfqpoint{0.014731in}{0.014731in}}%
\pgfpathcurveto{\pgfqpoint{0.010825in}{0.018638in}}{\pgfqpoint{0.005525in}{0.020833in}}{\pgfqpoint{0.000000in}{0.020833in}}%
\pgfpathcurveto{\pgfqpoint{-0.005525in}{0.020833in}}{\pgfqpoint{-0.010825in}{0.018638in}}{\pgfqpoint{-0.014731in}{0.014731in}}%
\pgfpathcurveto{\pgfqpoint{-0.018638in}{0.010825in}}{\pgfqpoint{-0.020833in}{0.005525in}}{\pgfqpoint{-0.020833in}{0.000000in}}%
\pgfpathcurveto{\pgfqpoint{-0.020833in}{-0.005525in}}{\pgfqpoint{-0.018638in}{-0.010825in}}{\pgfqpoint{-0.014731in}{-0.014731in}}%
\pgfpathcurveto{\pgfqpoint{-0.010825in}{-0.018638in}}{\pgfqpoint{-0.005525in}{-0.020833in}}{\pgfqpoint{0.000000in}{-0.020833in}}%
\pgfpathlineto{\pgfqpoint{0.000000in}{-0.020833in}}%
\pgfpathclose%
\pgfusepath{stroke,fill}%
}%
\begin{pgfscope}%
\pgfsys@transformshift{1.051728in}{1.062998in}%
\pgfsys@useobject{currentmarker}{}%
\end{pgfscope}%
\end{pgfscope}%
\begin{pgfscope}%
\definecolor{textcolor}{rgb}{0.000000,0.000000,0.000000}%
\pgfsetstrokecolor{textcolor}%
\pgfsetfillcolor{textcolor}%
\pgftext[x=1.351728in,y=1.004665in,left,base]{\color{textcolor}{\rmfamily\fontsize{12.000000}{14.400000}\selectfont\catcode`\^=\active\def^{\ifmmode\sp\else\^{}\fi}\catcode`\%=\active\def%{\%}NC}}%
\end{pgfscope}%
\begin{pgfscope}%
\pgfsetrectcap%
\pgfsetroundjoin%
\pgfsetlinewidth{1.003750pt}%
\definecolor{currentstroke}{rgb}{0.944006,0.377643,0.365136}%
\pgfsetstrokecolor{currentstroke}%
\pgfsetdash{}{0pt}%
\pgfpathmoveto{\pgfqpoint{0.885061in}{0.818370in}}%
\pgfpathlineto{\pgfqpoint{1.051728in}{0.818370in}}%
\pgfpathlineto{\pgfqpoint{1.218395in}{0.818370in}}%
\pgfusepath{stroke}%
\end{pgfscope}%
\begin{pgfscope}%
\pgfsetbuttcap%
\pgfsetroundjoin%
\definecolor{currentfill}{rgb}{0.944006,0.377643,0.365136}%
\pgfsetfillcolor{currentfill}%
\pgfsetlinewidth{1.003750pt}%
\definecolor{currentstroke}{rgb}{0.944006,0.377643,0.365136}%
\pgfsetstrokecolor{currentstroke}%
\pgfsetdash{}{0pt}%
\pgfsys@defobject{currentmarker}{\pgfqpoint{-0.020833in}{-0.020833in}}{\pgfqpoint{0.020833in}{0.020833in}}{%
\pgfpathmoveto{\pgfqpoint{0.000000in}{-0.020833in}}%
\pgfpathcurveto{\pgfqpoint{0.005525in}{-0.020833in}}{\pgfqpoint{0.010825in}{-0.018638in}}{\pgfqpoint{0.014731in}{-0.014731in}}%
\pgfpathcurveto{\pgfqpoint{0.018638in}{-0.010825in}}{\pgfqpoint{0.020833in}{-0.005525in}}{\pgfqpoint{0.020833in}{0.000000in}}%
\pgfpathcurveto{\pgfqpoint{0.020833in}{0.005525in}}{\pgfqpoint{0.018638in}{0.010825in}}{\pgfqpoint{0.014731in}{0.014731in}}%
\pgfpathcurveto{\pgfqpoint{0.010825in}{0.018638in}}{\pgfqpoint{0.005525in}{0.020833in}}{\pgfqpoint{0.000000in}{0.020833in}}%
\pgfpathcurveto{\pgfqpoint{-0.005525in}{0.020833in}}{\pgfqpoint{-0.010825in}{0.018638in}}{\pgfqpoint{-0.014731in}{0.014731in}}%
\pgfpathcurveto{\pgfqpoint{-0.018638in}{0.010825in}}{\pgfqpoint{-0.020833in}{0.005525in}}{\pgfqpoint{-0.020833in}{0.000000in}}%
\pgfpathcurveto{\pgfqpoint{-0.020833in}{-0.005525in}}{\pgfqpoint{-0.018638in}{-0.010825in}}{\pgfqpoint{-0.014731in}{-0.014731in}}%
\pgfpathcurveto{\pgfqpoint{-0.010825in}{-0.018638in}}{\pgfqpoint{-0.005525in}{-0.020833in}}{\pgfqpoint{0.000000in}{-0.020833in}}%
\pgfpathlineto{\pgfqpoint{0.000000in}{-0.020833in}}%
\pgfpathclose%
\pgfusepath{stroke,fill}%
}%
\begin{pgfscope}%
\pgfsys@transformshift{1.051728in}{0.818370in}%
\pgfsys@useobject{currentmarker}{}%
\end{pgfscope}%
\end{pgfscope}%
\begin{pgfscope}%
\definecolor{textcolor}{rgb}{0.000000,0.000000,0.000000}%
\pgfsetstrokecolor{textcolor}%
\pgfsetfillcolor{textcolor}%
\pgftext[x=1.351728in,y=0.760036in,left,base]{\color{textcolor}{\rmfamily\fontsize{12.000000}{14.400000}\selectfont\catcode`\^=\active\def^{\ifmmode\sp\else\^{}\fi}\catcode`\%=\active\def%{\%}NC++}}%
\end{pgfscope}%
\end{pgfpicture}%
\makeatother%
\endgroup%

        \caption{$n_{\Omega}=40$}
        \label{fig:5-experiments-electronic-structure-convergence-m-nv40}
    \end{subfigure}
    \begin{subfigure}[b]{0.49\columnwidth}
        %% Creator: Matplotlib, PGF backend
%%
%% To include the figure in your LaTeX document, write
%%   \input{<filename>.pgf}
%%
%% Make sure the required packages are loaded in your preamble
%%   \usepackage{pgf}
%%
%% Also ensure that all the required font packages are loaded; for instance,
%% the lmodern package is sometimes necessary when using math font.
%%   \usepackage{lmodern}
%%
%% Figures using additional raster images can only be included by \input if
%% they are in the same directory as the main LaTeX file. For loading figures
%% from other directories you can use the `import` package
%%   \usepackage{import}
%%
%% and then include the figures with
%%   \import{<path to file>}{<filename>.pgf}
%%
%% Matplotlib used the following preamble
%%   \def\mathdefault#1{#1}
%%   \everymath=\expandafter{\the\everymath\displaystyle}
%%   
%%   \makeatletter\@ifpackageloaded{underscore}{}{\usepackage[strings]{underscore}}\makeatother
%%
\begingroup%
\makeatletter%
\begin{pgfpicture}%
\pgfpathrectangle{\pgfpointorigin}{\pgfqpoint{2.759413in}{2.574073in}}%
\pgfusepath{use as bounding box, clip}%
\begin{pgfscope}%
\pgfsetbuttcap%
\pgfsetmiterjoin%
\definecolor{currentfill}{rgb}{1.000000,1.000000,1.000000}%
\pgfsetfillcolor{currentfill}%
\pgfsetlinewidth{0.000000pt}%
\definecolor{currentstroke}{rgb}{1.000000,1.000000,1.000000}%
\pgfsetstrokecolor{currentstroke}%
\pgfsetdash{}{0pt}%
\pgfpathmoveto{\pgfqpoint{0.000000in}{0.000000in}}%
\pgfpathlineto{\pgfqpoint{2.759413in}{0.000000in}}%
\pgfpathlineto{\pgfqpoint{2.759413in}{2.574073in}}%
\pgfpathlineto{\pgfqpoint{0.000000in}{2.574073in}}%
\pgfpathlineto{\pgfqpoint{0.000000in}{0.000000in}}%
\pgfpathclose%
\pgfusepath{fill}%
\end{pgfscope}%
\begin{pgfscope}%
\pgfsetbuttcap%
\pgfsetmiterjoin%
\definecolor{currentfill}{rgb}{1.000000,1.000000,1.000000}%
\pgfsetfillcolor{currentfill}%
\pgfsetlinewidth{0.000000pt}%
\definecolor{currentstroke}{rgb}{0.000000,0.000000,0.000000}%
\pgfsetstrokecolor{currentstroke}%
\pgfsetstrokeopacity{0.000000}%
\pgfsetdash{}{0pt}%
\pgfpathmoveto{\pgfqpoint{0.721913in}{0.549073in}}%
\pgfpathlineto{\pgfqpoint{2.659413in}{0.549073in}}%
\pgfpathlineto{\pgfqpoint{2.659413in}{2.474073in}}%
\pgfpathlineto{\pgfqpoint{0.721913in}{2.474073in}}%
\pgfpathlineto{\pgfqpoint{0.721913in}{0.549073in}}%
\pgfpathclose%
\pgfusepath{fill}%
\end{pgfscope}%
\begin{pgfscope}%
\pgfsetbuttcap%
\pgfsetroundjoin%
\definecolor{currentfill}{rgb}{0.000000,0.000000,0.000000}%
\pgfsetfillcolor{currentfill}%
\pgfsetlinewidth{0.803000pt}%
\definecolor{currentstroke}{rgb}{0.000000,0.000000,0.000000}%
\pgfsetstrokecolor{currentstroke}%
\pgfsetdash{}{0pt}%
\pgfsys@defobject{currentmarker}{\pgfqpoint{0.000000in}{-0.048611in}}{\pgfqpoint{0.000000in}{0.000000in}}{%
\pgfpathmoveto{\pgfqpoint{0.000000in}{0.000000in}}%
\pgfpathlineto{\pgfqpoint{0.000000in}{-0.048611in}}%
\pgfusepath{stroke,fill}%
}%
\begin{pgfscope}%
\pgfsys@transformshift{1.760574in}{0.549073in}%
\pgfsys@useobject{currentmarker}{}%
\end{pgfscope}%
\end{pgfscope}%
\begin{pgfscope}%
\definecolor{textcolor}{rgb}{0.000000,0.000000,0.000000}%
\pgfsetstrokecolor{textcolor}%
\pgfsetfillcolor{textcolor}%
\pgftext[x=1.760574in,y=0.451851in,,top]{\color{textcolor}{\rmfamily\fontsize{12.000000}{14.400000}\selectfont\catcode`\^=\active\def^{\ifmmode\sp\else\^{}\fi}\catcode`\%=\active\def%{\%}$\mathdefault{10^{3}}$}}%
\end{pgfscope}%
\begin{pgfscope}%
\pgfsetbuttcap%
\pgfsetroundjoin%
\definecolor{currentfill}{rgb}{0.000000,0.000000,0.000000}%
\pgfsetfillcolor{currentfill}%
\pgfsetlinewidth{0.602250pt}%
\definecolor{currentstroke}{rgb}{0.000000,0.000000,0.000000}%
\pgfsetstrokecolor{currentstroke}%
\pgfsetdash{}{0pt}%
\pgfsys@defobject{currentmarker}{\pgfqpoint{0.000000in}{-0.027778in}}{\pgfqpoint{0.000000in}{0.000000in}}{%
\pgfpathmoveto{\pgfqpoint{0.000000in}{0.000000in}}%
\pgfpathlineto{\pgfqpoint{0.000000in}{-0.027778in}}%
\pgfusepath{stroke,fill}%
}%
\begin{pgfscope}%
\pgfsys@transformshift{0.815881in}{0.549073in}%
\pgfsys@useobject{currentmarker}{}%
\end{pgfscope}%
\end{pgfscope}%
\begin{pgfscope}%
\pgfsetbuttcap%
\pgfsetroundjoin%
\definecolor{currentfill}{rgb}{0.000000,0.000000,0.000000}%
\pgfsetfillcolor{currentfill}%
\pgfsetlinewidth{0.602250pt}%
\definecolor{currentstroke}{rgb}{0.000000,0.000000,0.000000}%
\pgfsetstrokecolor{currentstroke}%
\pgfsetdash{}{0pt}%
\pgfsys@defobject{currentmarker}{\pgfqpoint{0.000000in}{-0.027778in}}{\pgfqpoint{0.000000in}{0.000000in}}{%
\pgfpathmoveto{\pgfqpoint{0.000000in}{0.000000in}}%
\pgfpathlineto{\pgfqpoint{0.000000in}{-0.027778in}}%
\pgfusepath{stroke,fill}%
}%
\begin{pgfscope}%
\pgfsys@transformshift{1.053877in}{0.549073in}%
\pgfsys@useobject{currentmarker}{}%
\end{pgfscope}%
\end{pgfscope}%
\begin{pgfscope}%
\pgfsetbuttcap%
\pgfsetroundjoin%
\definecolor{currentfill}{rgb}{0.000000,0.000000,0.000000}%
\pgfsetfillcolor{currentfill}%
\pgfsetlinewidth{0.602250pt}%
\definecolor{currentstroke}{rgb}{0.000000,0.000000,0.000000}%
\pgfsetstrokecolor{currentstroke}%
\pgfsetdash{}{0pt}%
\pgfsys@defobject{currentmarker}{\pgfqpoint{0.000000in}{-0.027778in}}{\pgfqpoint{0.000000in}{0.000000in}}{%
\pgfpathmoveto{\pgfqpoint{0.000000in}{0.000000in}}%
\pgfpathlineto{\pgfqpoint{0.000000in}{-0.027778in}}%
\pgfusepath{stroke,fill}%
}%
\begin{pgfscope}%
\pgfsys@transformshift{1.222738in}{0.549073in}%
\pgfsys@useobject{currentmarker}{}%
\end{pgfscope}%
\end{pgfscope}%
\begin{pgfscope}%
\pgfsetbuttcap%
\pgfsetroundjoin%
\definecolor{currentfill}{rgb}{0.000000,0.000000,0.000000}%
\pgfsetfillcolor{currentfill}%
\pgfsetlinewidth{0.602250pt}%
\definecolor{currentstroke}{rgb}{0.000000,0.000000,0.000000}%
\pgfsetstrokecolor{currentstroke}%
\pgfsetdash{}{0pt}%
\pgfsys@defobject{currentmarker}{\pgfqpoint{0.000000in}{-0.027778in}}{\pgfqpoint{0.000000in}{0.000000in}}{%
\pgfpathmoveto{\pgfqpoint{0.000000in}{0.000000in}}%
\pgfpathlineto{\pgfqpoint{0.000000in}{-0.027778in}}%
\pgfusepath{stroke,fill}%
}%
\begin{pgfscope}%
\pgfsys@transformshift{1.353717in}{0.549073in}%
\pgfsys@useobject{currentmarker}{}%
\end{pgfscope}%
\end{pgfscope}%
\begin{pgfscope}%
\pgfsetbuttcap%
\pgfsetroundjoin%
\definecolor{currentfill}{rgb}{0.000000,0.000000,0.000000}%
\pgfsetfillcolor{currentfill}%
\pgfsetlinewidth{0.602250pt}%
\definecolor{currentstroke}{rgb}{0.000000,0.000000,0.000000}%
\pgfsetstrokecolor{currentstroke}%
\pgfsetdash{}{0pt}%
\pgfsys@defobject{currentmarker}{\pgfqpoint{0.000000in}{-0.027778in}}{\pgfqpoint{0.000000in}{0.000000in}}{%
\pgfpathmoveto{\pgfqpoint{0.000000in}{0.000000in}}%
\pgfpathlineto{\pgfqpoint{0.000000in}{-0.027778in}}%
\pgfusepath{stroke,fill}%
}%
\begin{pgfscope}%
\pgfsys@transformshift{1.460734in}{0.549073in}%
\pgfsys@useobject{currentmarker}{}%
\end{pgfscope}%
\end{pgfscope}%
\begin{pgfscope}%
\pgfsetbuttcap%
\pgfsetroundjoin%
\definecolor{currentfill}{rgb}{0.000000,0.000000,0.000000}%
\pgfsetfillcolor{currentfill}%
\pgfsetlinewidth{0.602250pt}%
\definecolor{currentstroke}{rgb}{0.000000,0.000000,0.000000}%
\pgfsetstrokecolor{currentstroke}%
\pgfsetdash{}{0pt}%
\pgfsys@defobject{currentmarker}{\pgfqpoint{0.000000in}{-0.027778in}}{\pgfqpoint{0.000000in}{0.000000in}}{%
\pgfpathmoveto{\pgfqpoint{0.000000in}{0.000000in}}%
\pgfpathlineto{\pgfqpoint{0.000000in}{-0.027778in}}%
\pgfusepath{stroke,fill}%
}%
\begin{pgfscope}%
\pgfsys@transformshift{1.551216in}{0.549073in}%
\pgfsys@useobject{currentmarker}{}%
\end{pgfscope}%
\end{pgfscope}%
\begin{pgfscope}%
\pgfsetbuttcap%
\pgfsetroundjoin%
\definecolor{currentfill}{rgb}{0.000000,0.000000,0.000000}%
\pgfsetfillcolor{currentfill}%
\pgfsetlinewidth{0.602250pt}%
\definecolor{currentstroke}{rgb}{0.000000,0.000000,0.000000}%
\pgfsetstrokecolor{currentstroke}%
\pgfsetdash{}{0pt}%
\pgfsys@defobject{currentmarker}{\pgfqpoint{0.000000in}{-0.027778in}}{\pgfqpoint{0.000000in}{0.000000in}}{%
\pgfpathmoveto{\pgfqpoint{0.000000in}{0.000000in}}%
\pgfpathlineto{\pgfqpoint{0.000000in}{-0.027778in}}%
\pgfusepath{stroke,fill}%
}%
\begin{pgfscope}%
\pgfsys@transformshift{1.629595in}{0.549073in}%
\pgfsys@useobject{currentmarker}{}%
\end{pgfscope}%
\end{pgfscope}%
\begin{pgfscope}%
\pgfsetbuttcap%
\pgfsetroundjoin%
\definecolor{currentfill}{rgb}{0.000000,0.000000,0.000000}%
\pgfsetfillcolor{currentfill}%
\pgfsetlinewidth{0.602250pt}%
\definecolor{currentstroke}{rgb}{0.000000,0.000000,0.000000}%
\pgfsetstrokecolor{currentstroke}%
\pgfsetdash{}{0pt}%
\pgfsys@defobject{currentmarker}{\pgfqpoint{0.000000in}{-0.027778in}}{\pgfqpoint{0.000000in}{0.000000in}}{%
\pgfpathmoveto{\pgfqpoint{0.000000in}{0.000000in}}%
\pgfpathlineto{\pgfqpoint{0.000000in}{-0.027778in}}%
\pgfusepath{stroke,fill}%
}%
\begin{pgfscope}%
\pgfsys@transformshift{1.698730in}{0.549073in}%
\pgfsys@useobject{currentmarker}{}%
\end{pgfscope}%
\end{pgfscope}%
\begin{pgfscope}%
\pgfsetbuttcap%
\pgfsetroundjoin%
\definecolor{currentfill}{rgb}{0.000000,0.000000,0.000000}%
\pgfsetfillcolor{currentfill}%
\pgfsetlinewidth{0.602250pt}%
\definecolor{currentstroke}{rgb}{0.000000,0.000000,0.000000}%
\pgfsetstrokecolor{currentstroke}%
\pgfsetdash{}{0pt}%
\pgfsys@defobject{currentmarker}{\pgfqpoint{0.000000in}{-0.027778in}}{\pgfqpoint{0.000000in}{0.000000in}}{%
\pgfpathmoveto{\pgfqpoint{0.000000in}{0.000000in}}%
\pgfpathlineto{\pgfqpoint{0.000000in}{-0.027778in}}%
\pgfusepath{stroke,fill}%
}%
\begin{pgfscope}%
\pgfsys@transformshift{2.167431in}{0.549073in}%
\pgfsys@useobject{currentmarker}{}%
\end{pgfscope}%
\end{pgfscope}%
\begin{pgfscope}%
\pgfsetbuttcap%
\pgfsetroundjoin%
\definecolor{currentfill}{rgb}{0.000000,0.000000,0.000000}%
\pgfsetfillcolor{currentfill}%
\pgfsetlinewidth{0.602250pt}%
\definecolor{currentstroke}{rgb}{0.000000,0.000000,0.000000}%
\pgfsetstrokecolor{currentstroke}%
\pgfsetdash{}{0pt}%
\pgfsys@defobject{currentmarker}{\pgfqpoint{0.000000in}{-0.027778in}}{\pgfqpoint{0.000000in}{0.000000in}}{%
\pgfpathmoveto{\pgfqpoint{0.000000in}{0.000000in}}%
\pgfpathlineto{\pgfqpoint{0.000000in}{-0.027778in}}%
\pgfusepath{stroke,fill}%
}%
\begin{pgfscope}%
\pgfsys@transformshift{2.405427in}{0.549073in}%
\pgfsys@useobject{currentmarker}{}%
\end{pgfscope}%
\end{pgfscope}%
\begin{pgfscope}%
\pgfsetbuttcap%
\pgfsetroundjoin%
\definecolor{currentfill}{rgb}{0.000000,0.000000,0.000000}%
\pgfsetfillcolor{currentfill}%
\pgfsetlinewidth{0.602250pt}%
\definecolor{currentstroke}{rgb}{0.000000,0.000000,0.000000}%
\pgfsetstrokecolor{currentstroke}%
\pgfsetdash{}{0pt}%
\pgfsys@defobject{currentmarker}{\pgfqpoint{0.000000in}{-0.027778in}}{\pgfqpoint{0.000000in}{0.000000in}}{%
\pgfpathmoveto{\pgfqpoint{0.000000in}{0.000000in}}%
\pgfpathlineto{\pgfqpoint{0.000000in}{-0.027778in}}%
\pgfusepath{stroke,fill}%
}%
\begin{pgfscope}%
\pgfsys@transformshift{2.574288in}{0.549073in}%
\pgfsys@useobject{currentmarker}{}%
\end{pgfscope}%
\end{pgfscope}%
\begin{pgfscope}%
\definecolor{textcolor}{rgb}{0.000000,0.000000,0.000000}%
\pgfsetstrokecolor{textcolor}%
\pgfsetfillcolor{textcolor}%
\pgftext[x=1.690663in,y=0.248148in,,top]{\color{textcolor}{\rmfamily\fontsize{12.000000}{14.400000}\selectfont\catcode`\^=\active\def^{\ifmmode\sp\else\^{}\fi}\catcode`\%=\active\def%{\%}$m$}}%
\end{pgfscope}%
\begin{pgfscope}%
\pgfsetbuttcap%
\pgfsetroundjoin%
\definecolor{currentfill}{rgb}{0.000000,0.000000,0.000000}%
\pgfsetfillcolor{currentfill}%
\pgfsetlinewidth{0.803000pt}%
\definecolor{currentstroke}{rgb}{0.000000,0.000000,0.000000}%
\pgfsetstrokecolor{currentstroke}%
\pgfsetdash{}{0pt}%
\pgfsys@defobject{currentmarker}{\pgfqpoint{-0.048611in}{0.000000in}}{\pgfqpoint{-0.000000in}{0.000000in}}{%
\pgfpathmoveto{\pgfqpoint{-0.000000in}{0.000000in}}%
\pgfpathlineto{\pgfqpoint{-0.048611in}{0.000000in}}%
\pgfusepath{stroke,fill}%
}%
\begin{pgfscope}%
\pgfsys@transformshift{0.721913in}{0.891761in}%
\pgfsys@useobject{currentmarker}{}%
\end{pgfscope}%
\end{pgfscope}%
\begin{pgfscope}%
\definecolor{textcolor}{rgb}{0.000000,0.000000,0.000000}%
\pgfsetstrokecolor{textcolor}%
\pgfsetfillcolor{textcolor}%
\pgftext[x=0.303703in, y=0.833891in, left, base]{\color{textcolor}{\rmfamily\fontsize{12.000000}{14.400000}\selectfont\catcode`\^=\active\def^{\ifmmode\sp\else\^{}\fi}\catcode`\%=\active\def%{\%}$\mathdefault{10^{-6}}$}}%
\end{pgfscope}%
\begin{pgfscope}%
\pgfsetbuttcap%
\pgfsetroundjoin%
\definecolor{currentfill}{rgb}{0.000000,0.000000,0.000000}%
\pgfsetfillcolor{currentfill}%
\pgfsetlinewidth{0.803000pt}%
\definecolor{currentstroke}{rgb}{0.000000,0.000000,0.000000}%
\pgfsetstrokecolor{currentstroke}%
\pgfsetdash{}{0pt}%
\pgfsys@defobject{currentmarker}{\pgfqpoint{-0.048611in}{0.000000in}}{\pgfqpoint{-0.000000in}{0.000000in}}{%
\pgfpathmoveto{\pgfqpoint{-0.000000in}{0.000000in}}%
\pgfpathlineto{\pgfqpoint{-0.048611in}{0.000000in}}%
\pgfusepath{stroke,fill}%
}%
\begin{pgfscope}%
\pgfsys@transformshift{0.721913in}{1.380149in}%
\pgfsys@useobject{currentmarker}{}%
\end{pgfscope}%
\end{pgfscope}%
\begin{pgfscope}%
\definecolor{textcolor}{rgb}{0.000000,0.000000,0.000000}%
\pgfsetstrokecolor{textcolor}%
\pgfsetfillcolor{textcolor}%
\pgftext[x=0.303703in, y=1.322279in, left, base]{\color{textcolor}{\rmfamily\fontsize{12.000000}{14.400000}\selectfont\catcode`\^=\active\def^{\ifmmode\sp\else\^{}\fi}\catcode`\%=\active\def%{\%}$\mathdefault{10^{-4}}$}}%
\end{pgfscope}%
\begin{pgfscope}%
\pgfsetbuttcap%
\pgfsetroundjoin%
\definecolor{currentfill}{rgb}{0.000000,0.000000,0.000000}%
\pgfsetfillcolor{currentfill}%
\pgfsetlinewidth{0.803000pt}%
\definecolor{currentstroke}{rgb}{0.000000,0.000000,0.000000}%
\pgfsetstrokecolor{currentstroke}%
\pgfsetdash{}{0pt}%
\pgfsys@defobject{currentmarker}{\pgfqpoint{-0.048611in}{0.000000in}}{\pgfqpoint{-0.000000in}{0.000000in}}{%
\pgfpathmoveto{\pgfqpoint{-0.000000in}{0.000000in}}%
\pgfpathlineto{\pgfqpoint{-0.048611in}{0.000000in}}%
\pgfusepath{stroke,fill}%
}%
\begin{pgfscope}%
\pgfsys@transformshift{0.721913in}{1.868538in}%
\pgfsys@useobject{currentmarker}{}%
\end{pgfscope}%
\end{pgfscope}%
\begin{pgfscope}%
\definecolor{textcolor}{rgb}{0.000000,0.000000,0.000000}%
\pgfsetstrokecolor{textcolor}%
\pgfsetfillcolor{textcolor}%
\pgftext[x=0.303703in, y=1.810668in, left, base]{\color{textcolor}{\rmfamily\fontsize{12.000000}{14.400000}\selectfont\catcode`\^=\active\def^{\ifmmode\sp\else\^{}\fi}\catcode`\%=\active\def%{\%}$\mathdefault{10^{-2}}$}}%
\end{pgfscope}%
\begin{pgfscope}%
\pgfsetbuttcap%
\pgfsetroundjoin%
\definecolor{currentfill}{rgb}{0.000000,0.000000,0.000000}%
\pgfsetfillcolor{currentfill}%
\pgfsetlinewidth{0.803000pt}%
\definecolor{currentstroke}{rgb}{0.000000,0.000000,0.000000}%
\pgfsetstrokecolor{currentstroke}%
\pgfsetdash{}{0pt}%
\pgfsys@defobject{currentmarker}{\pgfqpoint{-0.048611in}{0.000000in}}{\pgfqpoint{-0.000000in}{0.000000in}}{%
\pgfpathmoveto{\pgfqpoint{-0.000000in}{0.000000in}}%
\pgfpathlineto{\pgfqpoint{-0.048611in}{0.000000in}}%
\pgfusepath{stroke,fill}%
}%
\begin{pgfscope}%
\pgfsys@transformshift{0.721913in}{2.356926in}%
\pgfsys@useobject{currentmarker}{}%
\end{pgfscope}%
\end{pgfscope}%
\begin{pgfscope}%
\definecolor{textcolor}{rgb}{0.000000,0.000000,0.000000}%
\pgfsetstrokecolor{textcolor}%
\pgfsetfillcolor{textcolor}%
\pgftext[x=0.395525in, y=2.299056in, left, base]{\color{textcolor}{\rmfamily\fontsize{12.000000}{14.400000}\selectfont\catcode`\^=\active\def^{\ifmmode\sp\else\^{}\fi}\catcode`\%=\active\def%{\%}$\mathdefault{10^{0}}$}}%
\end{pgfscope}%
\begin{pgfscope}%
\definecolor{textcolor}{rgb}{0.000000,0.000000,0.000000}%
\pgfsetstrokecolor{textcolor}%
\pgfsetfillcolor{textcolor}%
\pgftext[x=0.248148in,y=1.511573in,,bottom,rotate=90.000000]{\color{textcolor}{\rmfamily\fontsize{12.000000}{14.400000}\selectfont\catcode`\^=\active\def^{\ifmmode\sp\else\^{}\fi}\catcode`\%=\active\def%{\%}$L^1$ relative error}}%
\end{pgfscope}%
\begin{pgfscope}%
\pgfpathrectangle{\pgfqpoint{0.721913in}{0.549073in}}{\pgfqpoint{1.937500in}{1.925000in}}%
\pgfusepath{clip}%
\pgfsetrectcap%
\pgfsetroundjoin%
\pgfsetlinewidth{1.003750pt}%
\definecolor{currentstroke}{rgb}{0.537255,0.647059,0.760784}%
\pgfsetstrokecolor{currentstroke}%
\pgfsetdash{}{0pt}%
\pgfpathmoveto{\pgfqpoint{0.809982in}{2.334824in}}%
\pgfpathlineto{\pgfqpoint{1.164146in}{2.222692in}}%
\pgfpathlineto{\pgfqpoint{1.516678in}{2.055636in}}%
\pgfpathlineto{\pgfqpoint{1.868568in}{1.948700in}}%
\pgfpathlineto{\pgfqpoint{2.219628in}{1.948372in}}%
\pgfpathlineto{\pgfqpoint{2.571345in}{1.948372in}}%
\pgfusepath{stroke}%
\end{pgfscope}%
\begin{pgfscope}%
\pgfpathrectangle{\pgfqpoint{0.721913in}{0.549073in}}{\pgfqpoint{1.937500in}{1.925000in}}%
\pgfusepath{clip}%
\pgfsetbuttcap%
\pgfsetroundjoin%
\definecolor{currentfill}{rgb}{0.537255,0.647059,0.760784}%
\pgfsetfillcolor{currentfill}%
\pgfsetlinewidth{1.003750pt}%
\definecolor{currentstroke}{rgb}{0.537255,0.647059,0.760784}%
\pgfsetstrokecolor{currentstroke}%
\pgfsetdash{}{0pt}%
\pgfsys@defobject{currentmarker}{\pgfqpoint{-0.020833in}{-0.020833in}}{\pgfqpoint{0.020833in}{0.020833in}}{%
\pgfpathmoveto{\pgfqpoint{0.000000in}{-0.020833in}}%
\pgfpathcurveto{\pgfqpoint{0.005525in}{-0.020833in}}{\pgfqpoint{0.010825in}{-0.018638in}}{\pgfqpoint{0.014731in}{-0.014731in}}%
\pgfpathcurveto{\pgfqpoint{0.018638in}{-0.010825in}}{\pgfqpoint{0.020833in}{-0.005525in}}{\pgfqpoint{0.020833in}{0.000000in}}%
\pgfpathcurveto{\pgfqpoint{0.020833in}{0.005525in}}{\pgfqpoint{0.018638in}{0.010825in}}{\pgfqpoint{0.014731in}{0.014731in}}%
\pgfpathcurveto{\pgfqpoint{0.010825in}{0.018638in}}{\pgfqpoint{0.005525in}{0.020833in}}{\pgfqpoint{0.000000in}{0.020833in}}%
\pgfpathcurveto{\pgfqpoint{-0.005525in}{0.020833in}}{\pgfqpoint{-0.010825in}{0.018638in}}{\pgfqpoint{-0.014731in}{0.014731in}}%
\pgfpathcurveto{\pgfqpoint{-0.018638in}{0.010825in}}{\pgfqpoint{-0.020833in}{0.005525in}}{\pgfqpoint{-0.020833in}{0.000000in}}%
\pgfpathcurveto{\pgfqpoint{-0.020833in}{-0.005525in}}{\pgfqpoint{-0.018638in}{-0.010825in}}{\pgfqpoint{-0.014731in}{-0.014731in}}%
\pgfpathcurveto{\pgfqpoint{-0.010825in}{-0.018638in}}{\pgfqpoint{-0.005525in}{-0.020833in}}{\pgfqpoint{0.000000in}{-0.020833in}}%
\pgfpathlineto{\pgfqpoint{0.000000in}{-0.020833in}}%
\pgfpathclose%
\pgfusepath{stroke,fill}%
}%
\begin{pgfscope}%
\pgfsys@transformshift{0.809982in}{2.334824in}%
\pgfsys@useobject{currentmarker}{}%
\end{pgfscope}%
\begin{pgfscope}%
\pgfsys@transformshift{1.164146in}{2.222692in}%
\pgfsys@useobject{currentmarker}{}%
\end{pgfscope}%
\begin{pgfscope}%
\pgfsys@transformshift{1.516678in}{2.055636in}%
\pgfsys@useobject{currentmarker}{}%
\end{pgfscope}%
\begin{pgfscope}%
\pgfsys@transformshift{1.868568in}{1.948700in}%
\pgfsys@useobject{currentmarker}{}%
\end{pgfscope}%
\begin{pgfscope}%
\pgfsys@transformshift{2.219628in}{1.948372in}%
\pgfsys@useobject{currentmarker}{}%
\end{pgfscope}%
\begin{pgfscope}%
\pgfsys@transformshift{2.571345in}{1.948372in}%
\pgfsys@useobject{currentmarker}{}%
\end{pgfscope}%
\end{pgfscope}%
\begin{pgfscope}%
\pgfpathrectangle{\pgfqpoint{0.721913in}{0.549073in}}{\pgfqpoint{1.937500in}{1.925000in}}%
\pgfusepath{clip}%
\pgfsetrectcap%
\pgfsetroundjoin%
\pgfsetlinewidth{1.003750pt}%
\definecolor{currentstroke}{rgb}{0.184314,0.270588,0.360784}%
\pgfsetstrokecolor{currentstroke}%
\pgfsetdash{}{0pt}%
\pgfpathmoveto{\pgfqpoint{0.809982in}{2.386573in}}%
\pgfpathlineto{\pgfqpoint{1.164146in}{2.316626in}}%
\pgfpathlineto{\pgfqpoint{1.516678in}{2.183606in}}%
\pgfpathlineto{\pgfqpoint{1.868568in}{1.687104in}}%
\pgfpathlineto{\pgfqpoint{2.219628in}{0.777079in}}%
\pgfpathlineto{\pgfqpoint{2.571345in}{0.636573in}}%
\pgfusepath{stroke}%
\end{pgfscope}%
\begin{pgfscope}%
\pgfpathrectangle{\pgfqpoint{0.721913in}{0.549073in}}{\pgfqpoint{1.937500in}{1.925000in}}%
\pgfusepath{clip}%
\pgfsetbuttcap%
\pgfsetroundjoin%
\definecolor{currentfill}{rgb}{0.184314,0.270588,0.360784}%
\pgfsetfillcolor{currentfill}%
\pgfsetlinewidth{1.003750pt}%
\definecolor{currentstroke}{rgb}{0.184314,0.270588,0.360784}%
\pgfsetstrokecolor{currentstroke}%
\pgfsetdash{}{0pt}%
\pgfsys@defobject{currentmarker}{\pgfqpoint{-0.020833in}{-0.020833in}}{\pgfqpoint{0.020833in}{0.020833in}}{%
\pgfpathmoveto{\pgfqpoint{0.000000in}{-0.020833in}}%
\pgfpathcurveto{\pgfqpoint{0.005525in}{-0.020833in}}{\pgfqpoint{0.010825in}{-0.018638in}}{\pgfqpoint{0.014731in}{-0.014731in}}%
\pgfpathcurveto{\pgfqpoint{0.018638in}{-0.010825in}}{\pgfqpoint{0.020833in}{-0.005525in}}{\pgfqpoint{0.020833in}{0.000000in}}%
\pgfpathcurveto{\pgfqpoint{0.020833in}{0.005525in}}{\pgfqpoint{0.018638in}{0.010825in}}{\pgfqpoint{0.014731in}{0.014731in}}%
\pgfpathcurveto{\pgfqpoint{0.010825in}{0.018638in}}{\pgfqpoint{0.005525in}{0.020833in}}{\pgfqpoint{0.000000in}{0.020833in}}%
\pgfpathcurveto{\pgfqpoint{-0.005525in}{0.020833in}}{\pgfqpoint{-0.010825in}{0.018638in}}{\pgfqpoint{-0.014731in}{0.014731in}}%
\pgfpathcurveto{\pgfqpoint{-0.018638in}{0.010825in}}{\pgfqpoint{-0.020833in}{0.005525in}}{\pgfqpoint{-0.020833in}{0.000000in}}%
\pgfpathcurveto{\pgfqpoint{-0.020833in}{-0.005525in}}{\pgfqpoint{-0.018638in}{-0.010825in}}{\pgfqpoint{-0.014731in}{-0.014731in}}%
\pgfpathcurveto{\pgfqpoint{-0.010825in}{-0.018638in}}{\pgfqpoint{-0.005525in}{-0.020833in}}{\pgfqpoint{0.000000in}{-0.020833in}}%
\pgfpathlineto{\pgfqpoint{0.000000in}{-0.020833in}}%
\pgfpathclose%
\pgfusepath{stroke,fill}%
}%
\begin{pgfscope}%
\pgfsys@transformshift{0.809982in}{2.386573in}%
\pgfsys@useobject{currentmarker}{}%
\end{pgfscope}%
\begin{pgfscope}%
\pgfsys@transformshift{1.164146in}{2.316626in}%
\pgfsys@useobject{currentmarker}{}%
\end{pgfscope}%
\begin{pgfscope}%
\pgfsys@transformshift{1.516678in}{2.183606in}%
\pgfsys@useobject{currentmarker}{}%
\end{pgfscope}%
\begin{pgfscope}%
\pgfsys@transformshift{1.868568in}{1.687104in}%
\pgfsys@useobject{currentmarker}{}%
\end{pgfscope}%
\begin{pgfscope}%
\pgfsys@transformshift{2.219628in}{0.777079in}%
\pgfsys@useobject{currentmarker}{}%
\end{pgfscope}%
\begin{pgfscope}%
\pgfsys@transformshift{2.571345in}{0.636573in}%
\pgfsys@useobject{currentmarker}{}%
\end{pgfscope}%
\end{pgfscope}%
\begin{pgfscope}%
\pgfpathrectangle{\pgfqpoint{0.721913in}{0.549073in}}{\pgfqpoint{1.937500in}{1.925000in}}%
\pgfusepath{clip}%
\pgfsetrectcap%
\pgfsetroundjoin%
\pgfsetlinewidth{1.003750pt}%
\definecolor{currentstroke}{rgb}{0.976471,0.505882,0.145098}%
\pgfsetstrokecolor{currentstroke}%
\pgfsetdash{}{0pt}%
\pgfpathmoveto{\pgfqpoint{0.809982in}{2.326427in}}%
\pgfpathlineto{\pgfqpoint{1.164146in}{2.217542in}}%
\pgfpathlineto{\pgfqpoint{1.516678in}{2.027230in}}%
\pgfpathlineto{\pgfqpoint{1.868568in}{1.505345in}}%
\pgfpathlineto{\pgfqpoint{2.219628in}{0.856828in}}%
\pgfpathlineto{\pgfqpoint{2.571345in}{0.830200in}}%
\pgfusepath{stroke}%
\end{pgfscope}%
\begin{pgfscope}%
\pgfpathrectangle{\pgfqpoint{0.721913in}{0.549073in}}{\pgfqpoint{1.937500in}{1.925000in}}%
\pgfusepath{clip}%
\pgfsetbuttcap%
\pgfsetroundjoin%
\definecolor{currentfill}{rgb}{0.976471,0.505882,0.145098}%
\pgfsetfillcolor{currentfill}%
\pgfsetlinewidth{1.003750pt}%
\definecolor{currentstroke}{rgb}{0.976471,0.505882,0.145098}%
\pgfsetstrokecolor{currentstroke}%
\pgfsetdash{}{0pt}%
\pgfsys@defobject{currentmarker}{\pgfqpoint{-0.020833in}{-0.020833in}}{\pgfqpoint{0.020833in}{0.020833in}}{%
\pgfpathmoveto{\pgfqpoint{0.000000in}{-0.020833in}}%
\pgfpathcurveto{\pgfqpoint{0.005525in}{-0.020833in}}{\pgfqpoint{0.010825in}{-0.018638in}}{\pgfqpoint{0.014731in}{-0.014731in}}%
\pgfpathcurveto{\pgfqpoint{0.018638in}{-0.010825in}}{\pgfqpoint{0.020833in}{-0.005525in}}{\pgfqpoint{0.020833in}{0.000000in}}%
\pgfpathcurveto{\pgfqpoint{0.020833in}{0.005525in}}{\pgfqpoint{0.018638in}{0.010825in}}{\pgfqpoint{0.014731in}{0.014731in}}%
\pgfpathcurveto{\pgfqpoint{0.010825in}{0.018638in}}{\pgfqpoint{0.005525in}{0.020833in}}{\pgfqpoint{0.000000in}{0.020833in}}%
\pgfpathcurveto{\pgfqpoint{-0.005525in}{0.020833in}}{\pgfqpoint{-0.010825in}{0.018638in}}{\pgfqpoint{-0.014731in}{0.014731in}}%
\pgfpathcurveto{\pgfqpoint{-0.018638in}{0.010825in}}{\pgfqpoint{-0.020833in}{0.005525in}}{\pgfqpoint{-0.020833in}{0.000000in}}%
\pgfpathcurveto{\pgfqpoint{-0.020833in}{-0.005525in}}{\pgfqpoint{-0.018638in}{-0.010825in}}{\pgfqpoint{-0.014731in}{-0.014731in}}%
\pgfpathcurveto{\pgfqpoint{-0.010825in}{-0.018638in}}{\pgfqpoint{-0.005525in}{-0.020833in}}{\pgfqpoint{0.000000in}{-0.020833in}}%
\pgfpathlineto{\pgfqpoint{0.000000in}{-0.020833in}}%
\pgfpathclose%
\pgfusepath{stroke,fill}%
}%
\begin{pgfscope}%
\pgfsys@transformshift{0.809982in}{2.326427in}%
\pgfsys@useobject{currentmarker}{}%
\end{pgfscope}%
\begin{pgfscope}%
\pgfsys@transformshift{1.164146in}{2.217542in}%
\pgfsys@useobject{currentmarker}{}%
\end{pgfscope}%
\begin{pgfscope}%
\pgfsys@transformshift{1.516678in}{2.027230in}%
\pgfsys@useobject{currentmarker}{}%
\end{pgfscope}%
\begin{pgfscope}%
\pgfsys@transformshift{1.868568in}{1.505345in}%
\pgfsys@useobject{currentmarker}{}%
\end{pgfscope}%
\begin{pgfscope}%
\pgfsys@transformshift{2.219628in}{0.856828in}%
\pgfsys@useobject{currentmarker}{}%
\end{pgfscope}%
\begin{pgfscope}%
\pgfsys@transformshift{2.571345in}{0.830200in}%
\pgfsys@useobject{currentmarker}{}%
\end{pgfscope}%
\end{pgfscope}%
\begin{pgfscope}%
\pgfsetrectcap%
\pgfsetmiterjoin%
\pgfsetlinewidth{0.803000pt}%
\definecolor{currentstroke}{rgb}{0.000000,0.000000,0.000000}%
\pgfsetstrokecolor{currentstroke}%
\pgfsetdash{}{0pt}%
\pgfpathmoveto{\pgfqpoint{0.721913in}{0.549073in}}%
\pgfpathlineto{\pgfqpoint{0.721913in}{2.474073in}}%
\pgfusepath{stroke}%
\end{pgfscope}%
\begin{pgfscope}%
\pgfsetrectcap%
\pgfsetmiterjoin%
\pgfsetlinewidth{0.803000pt}%
\definecolor{currentstroke}{rgb}{0.000000,0.000000,0.000000}%
\pgfsetstrokecolor{currentstroke}%
\pgfsetdash{}{0pt}%
\pgfpathmoveto{\pgfqpoint{2.659413in}{0.549073in}}%
\pgfpathlineto{\pgfqpoint{2.659413in}{2.474073in}}%
\pgfusepath{stroke}%
\end{pgfscope}%
\begin{pgfscope}%
\pgfsetrectcap%
\pgfsetmiterjoin%
\pgfsetlinewidth{0.803000pt}%
\definecolor{currentstroke}{rgb}{0.000000,0.000000,0.000000}%
\pgfsetstrokecolor{currentstroke}%
\pgfsetdash{}{0pt}%
\pgfpathmoveto{\pgfqpoint{0.721913in}{0.549073in}}%
\pgfpathlineto{\pgfqpoint{2.659413in}{0.549073in}}%
\pgfusepath{stroke}%
\end{pgfscope}%
\begin{pgfscope}%
\pgfsetrectcap%
\pgfsetmiterjoin%
\pgfsetlinewidth{0.803000pt}%
\definecolor{currentstroke}{rgb}{0.000000,0.000000,0.000000}%
\pgfsetstrokecolor{currentstroke}%
\pgfsetdash{}{0pt}%
\pgfpathmoveto{\pgfqpoint{0.721913in}{2.474073in}}%
\pgfpathlineto{\pgfqpoint{2.659413in}{2.474073in}}%
\pgfusepath{stroke}%
\end{pgfscope}%
\begin{pgfscope}%
\pgfsetbuttcap%
\pgfsetmiterjoin%
\definecolor{currentfill}{rgb}{1.000000,1.000000,1.000000}%
\pgfsetfillcolor{currentfill}%
\pgfsetfillopacity{0.800000}%
\pgfsetlinewidth{1.003750pt}%
\definecolor{currentstroke}{rgb}{0.800000,0.800000,0.800000}%
\pgfsetstrokecolor{currentstroke}%
\pgfsetstrokeopacity{0.800000}%
\pgfsetdash{}{0pt}%
\pgfpathmoveto{\pgfqpoint{0.838580in}{0.632406in}}%
\pgfpathlineto{\pgfqpoint{1.865967in}{0.632406in}}%
\pgfpathquadraticcurveto{\pgfqpoint{1.899300in}{0.632406in}}{\pgfqpoint{1.899300in}{0.665739in}}%
\pgfpathlineto{\pgfqpoint{1.899300in}{1.346294in}}%
\pgfpathquadraticcurveto{\pgfqpoint{1.899300in}{1.379627in}}{\pgfqpoint{1.865967in}{1.379627in}}%
\pgfpathlineto{\pgfqpoint{0.838580in}{1.379627in}}%
\pgfpathquadraticcurveto{\pgfqpoint{0.805247in}{1.379627in}}{\pgfqpoint{0.805247in}{1.346294in}}%
\pgfpathlineto{\pgfqpoint{0.805247in}{0.665739in}}%
\pgfpathquadraticcurveto{\pgfqpoint{0.805247in}{0.632406in}}{\pgfqpoint{0.838580in}{0.632406in}}%
\pgfpathlineto{\pgfqpoint{0.838580in}{0.632406in}}%
\pgfpathclose%
\pgfusepath{stroke,fill}%
\end{pgfscope}%
\begin{pgfscope}%
\pgfsetrectcap%
\pgfsetroundjoin%
\pgfsetlinewidth{1.003750pt}%
\definecolor{currentstroke}{rgb}{0.537255,0.647059,0.760784}%
\pgfsetstrokecolor{currentstroke}%
\pgfsetdash{}{0pt}%
\pgfpathmoveto{\pgfqpoint{0.871913in}{1.254627in}}%
\pgfpathlineto{\pgfqpoint{1.038580in}{1.254627in}}%
\pgfpathlineto{\pgfqpoint{1.205247in}{1.254627in}}%
\pgfusepath{stroke}%
\end{pgfscope}%
\begin{pgfscope}%
\pgfsetbuttcap%
\pgfsetroundjoin%
\definecolor{currentfill}{rgb}{0.537255,0.647059,0.760784}%
\pgfsetfillcolor{currentfill}%
\pgfsetlinewidth{1.003750pt}%
\definecolor{currentstroke}{rgb}{0.537255,0.647059,0.760784}%
\pgfsetstrokecolor{currentstroke}%
\pgfsetdash{}{0pt}%
\pgfsys@defobject{currentmarker}{\pgfqpoint{-0.020833in}{-0.020833in}}{\pgfqpoint{0.020833in}{0.020833in}}{%
\pgfpathmoveto{\pgfqpoint{0.000000in}{-0.020833in}}%
\pgfpathcurveto{\pgfqpoint{0.005525in}{-0.020833in}}{\pgfqpoint{0.010825in}{-0.018638in}}{\pgfqpoint{0.014731in}{-0.014731in}}%
\pgfpathcurveto{\pgfqpoint{0.018638in}{-0.010825in}}{\pgfqpoint{0.020833in}{-0.005525in}}{\pgfqpoint{0.020833in}{0.000000in}}%
\pgfpathcurveto{\pgfqpoint{0.020833in}{0.005525in}}{\pgfqpoint{0.018638in}{0.010825in}}{\pgfqpoint{0.014731in}{0.014731in}}%
\pgfpathcurveto{\pgfqpoint{0.010825in}{0.018638in}}{\pgfqpoint{0.005525in}{0.020833in}}{\pgfqpoint{0.000000in}{0.020833in}}%
\pgfpathcurveto{\pgfqpoint{-0.005525in}{0.020833in}}{\pgfqpoint{-0.010825in}{0.018638in}}{\pgfqpoint{-0.014731in}{0.014731in}}%
\pgfpathcurveto{\pgfqpoint{-0.018638in}{0.010825in}}{\pgfqpoint{-0.020833in}{0.005525in}}{\pgfqpoint{-0.020833in}{0.000000in}}%
\pgfpathcurveto{\pgfqpoint{-0.020833in}{-0.005525in}}{\pgfqpoint{-0.018638in}{-0.010825in}}{\pgfqpoint{-0.014731in}{-0.014731in}}%
\pgfpathcurveto{\pgfqpoint{-0.010825in}{-0.018638in}}{\pgfqpoint{-0.005525in}{-0.020833in}}{\pgfqpoint{0.000000in}{-0.020833in}}%
\pgfpathlineto{\pgfqpoint{0.000000in}{-0.020833in}}%
\pgfpathclose%
\pgfusepath{stroke,fill}%
}%
\begin{pgfscope}%
\pgfsys@transformshift{1.038580in}{1.254627in}%
\pgfsys@useobject{currentmarker}{}%
\end{pgfscope}%
\end{pgfscope}%
\begin{pgfscope}%
\definecolor{textcolor}{rgb}{0.000000,0.000000,0.000000}%
\pgfsetstrokecolor{textcolor}%
\pgfsetfillcolor{textcolor}%
\pgftext[x=1.338580in,y=1.196294in,left,base]{\color{textcolor}{\rmfamily\fontsize{12.000000}{14.400000}\selectfont\catcode`\^=\active\def^{\ifmmode\sp\else\^{}\fi}\catcode`\%=\active\def%{\%}DGC}}%
\end{pgfscope}%
\begin{pgfscope}%
\pgfsetrectcap%
\pgfsetroundjoin%
\pgfsetlinewidth{1.003750pt}%
\definecolor{currentstroke}{rgb}{0.184314,0.270588,0.360784}%
\pgfsetstrokecolor{currentstroke}%
\pgfsetdash{}{0pt}%
\pgfpathmoveto{\pgfqpoint{0.871913in}{1.022220in}}%
\pgfpathlineto{\pgfqpoint{1.038580in}{1.022220in}}%
\pgfpathlineto{\pgfqpoint{1.205247in}{1.022220in}}%
\pgfusepath{stroke}%
\end{pgfscope}%
\begin{pgfscope}%
\pgfsetbuttcap%
\pgfsetroundjoin%
\definecolor{currentfill}{rgb}{0.184314,0.270588,0.360784}%
\pgfsetfillcolor{currentfill}%
\pgfsetlinewidth{1.003750pt}%
\definecolor{currentstroke}{rgb}{0.184314,0.270588,0.360784}%
\pgfsetstrokecolor{currentstroke}%
\pgfsetdash{}{0pt}%
\pgfsys@defobject{currentmarker}{\pgfqpoint{-0.020833in}{-0.020833in}}{\pgfqpoint{0.020833in}{0.020833in}}{%
\pgfpathmoveto{\pgfqpoint{0.000000in}{-0.020833in}}%
\pgfpathcurveto{\pgfqpoint{0.005525in}{-0.020833in}}{\pgfqpoint{0.010825in}{-0.018638in}}{\pgfqpoint{0.014731in}{-0.014731in}}%
\pgfpathcurveto{\pgfqpoint{0.018638in}{-0.010825in}}{\pgfqpoint{0.020833in}{-0.005525in}}{\pgfqpoint{0.020833in}{0.000000in}}%
\pgfpathcurveto{\pgfqpoint{0.020833in}{0.005525in}}{\pgfqpoint{0.018638in}{0.010825in}}{\pgfqpoint{0.014731in}{0.014731in}}%
\pgfpathcurveto{\pgfqpoint{0.010825in}{0.018638in}}{\pgfqpoint{0.005525in}{0.020833in}}{\pgfqpoint{0.000000in}{0.020833in}}%
\pgfpathcurveto{\pgfqpoint{-0.005525in}{0.020833in}}{\pgfqpoint{-0.010825in}{0.018638in}}{\pgfqpoint{-0.014731in}{0.014731in}}%
\pgfpathcurveto{\pgfqpoint{-0.018638in}{0.010825in}}{\pgfqpoint{-0.020833in}{0.005525in}}{\pgfqpoint{-0.020833in}{0.000000in}}%
\pgfpathcurveto{\pgfqpoint{-0.020833in}{-0.005525in}}{\pgfqpoint{-0.018638in}{-0.010825in}}{\pgfqpoint{-0.014731in}{-0.014731in}}%
\pgfpathcurveto{\pgfqpoint{-0.010825in}{-0.018638in}}{\pgfqpoint{-0.005525in}{-0.020833in}}{\pgfqpoint{0.000000in}{-0.020833in}}%
\pgfpathlineto{\pgfqpoint{0.000000in}{-0.020833in}}%
\pgfpathclose%
\pgfusepath{stroke,fill}%
}%
\begin{pgfscope}%
\pgfsys@transformshift{1.038580in}{1.022220in}%
\pgfsys@useobject{currentmarker}{}%
\end{pgfscope}%
\end{pgfscope}%
\begin{pgfscope}%
\definecolor{textcolor}{rgb}{0.000000,0.000000,0.000000}%
\pgfsetstrokecolor{textcolor}%
\pgfsetfillcolor{textcolor}%
\pgftext[x=1.338580in,y=0.963887in,left,base]{\color{textcolor}{\rmfamily\fontsize{12.000000}{14.400000}\selectfont\catcode`\^=\active\def^{\ifmmode\sp\else\^{}\fi}\catcode`\%=\active\def%{\%}NC}}%
\end{pgfscope}%
\begin{pgfscope}%
\pgfsetrectcap%
\pgfsetroundjoin%
\pgfsetlinewidth{1.003750pt}%
\definecolor{currentstroke}{rgb}{0.976471,0.505882,0.145098}%
\pgfsetstrokecolor{currentstroke}%
\pgfsetdash{}{0pt}%
\pgfpathmoveto{\pgfqpoint{0.871913in}{0.789813in}}%
\pgfpathlineto{\pgfqpoint{1.038580in}{0.789813in}}%
\pgfpathlineto{\pgfqpoint{1.205247in}{0.789813in}}%
\pgfusepath{stroke}%
\end{pgfscope}%
\begin{pgfscope}%
\pgfsetbuttcap%
\pgfsetroundjoin%
\definecolor{currentfill}{rgb}{0.976471,0.505882,0.145098}%
\pgfsetfillcolor{currentfill}%
\pgfsetlinewidth{1.003750pt}%
\definecolor{currentstroke}{rgb}{0.976471,0.505882,0.145098}%
\pgfsetstrokecolor{currentstroke}%
\pgfsetdash{}{0pt}%
\pgfsys@defobject{currentmarker}{\pgfqpoint{-0.020833in}{-0.020833in}}{\pgfqpoint{0.020833in}{0.020833in}}{%
\pgfpathmoveto{\pgfqpoint{0.000000in}{-0.020833in}}%
\pgfpathcurveto{\pgfqpoint{0.005525in}{-0.020833in}}{\pgfqpoint{0.010825in}{-0.018638in}}{\pgfqpoint{0.014731in}{-0.014731in}}%
\pgfpathcurveto{\pgfqpoint{0.018638in}{-0.010825in}}{\pgfqpoint{0.020833in}{-0.005525in}}{\pgfqpoint{0.020833in}{0.000000in}}%
\pgfpathcurveto{\pgfqpoint{0.020833in}{0.005525in}}{\pgfqpoint{0.018638in}{0.010825in}}{\pgfqpoint{0.014731in}{0.014731in}}%
\pgfpathcurveto{\pgfqpoint{0.010825in}{0.018638in}}{\pgfqpoint{0.005525in}{0.020833in}}{\pgfqpoint{0.000000in}{0.020833in}}%
\pgfpathcurveto{\pgfqpoint{-0.005525in}{0.020833in}}{\pgfqpoint{-0.010825in}{0.018638in}}{\pgfqpoint{-0.014731in}{0.014731in}}%
\pgfpathcurveto{\pgfqpoint{-0.018638in}{0.010825in}}{\pgfqpoint{-0.020833in}{0.005525in}}{\pgfqpoint{-0.020833in}{0.000000in}}%
\pgfpathcurveto{\pgfqpoint{-0.020833in}{-0.005525in}}{\pgfqpoint{-0.018638in}{-0.010825in}}{\pgfqpoint{-0.014731in}{-0.014731in}}%
\pgfpathcurveto{\pgfqpoint{-0.010825in}{-0.018638in}}{\pgfqpoint{-0.005525in}{-0.020833in}}{\pgfqpoint{0.000000in}{-0.020833in}}%
\pgfpathlineto{\pgfqpoint{0.000000in}{-0.020833in}}%
\pgfpathclose%
\pgfusepath{stroke,fill}%
}%
\begin{pgfscope}%
\pgfsys@transformshift{1.038580in}{0.789813in}%
\pgfsys@useobject{currentmarker}{}%
\end{pgfscope}%
\end{pgfscope}%
\begin{pgfscope}%
\definecolor{textcolor}{rgb}{0.000000,0.000000,0.000000}%
\pgfsetstrokecolor{textcolor}%
\pgfsetfillcolor{textcolor}%
\pgftext[x=1.338580in,y=0.731480in,left,base]{\color{textcolor}{\rmfamily\fontsize{12.000000}{14.400000}\selectfont\catcode`\^=\active\def^{\ifmmode\sp\else\^{}\fi}\catcode`\%=\active\def%{\%}NC++}}%
\end{pgfscope}%
\end{pgfpicture}%
\makeatother%
\endgroup%

        \caption{$n_{\Omega}=160$}
        \label{fig:5-experiments-electronic-structure-convergence-m-nv160}
    \end{subfigure}
    \caption{Behavior with $m$ for $\sigma=0.05$}
    \label{fig:5-experiments-electronic-structure-convergence-m}
\end{figure}

In \reftab{tab:5-experiments-timing-DGC} we list the wall time each method
takes to compute an approximate \gls{spectral-density} at $n_t=100$ points
for different values of \gls{sketch-size} and \gls{chebyshev-degree}.

\begin{table}[ht]
    \caption{Runtime comparison}
    \label{tab:5-experiments-timing-DGC}
    \centering
\renewcommand{\arraystretch}{1.2}
\begin{tabular}{@{}lcccc@{}}
\toprule
 & \shortstack[c]{$m=800$ \\ $n_{\Omega} + n_{\Psi}=40$} & \shortstack[c]{$m=2400$ \\ $n_{\Omega} + n_{\Psi}=40$} & \shortstack[c]{$m=800$ \\ $n_{\Omega} + n_{\Psi}=160$} & \shortstack[c]{$m=2400$ \\ $n_{\Omega} + n_{\Psi}=160$}\\
\midrule
DGC & $0.262$ $\pm$ $0.005$ & $0.776$ $\pm$ $0.001$ & $0.956$ $\pm$ $0.001$ & $2.853$ $\pm$ $0.001$ \\
NC & $0.941$ $\pm$ $0.003$ & $2.819$ $\pm$ $0.004$ & $4.270$ $\pm$ $0.021$ & $12.845$ $\pm$ $0.011$ \\
NC++ & $0.826$ $\pm$ $0.002$ & $2.472$ $\pm$ $0.004$ & $2.944$ $\pm$ $0.006$ & $8.699$ $\pm$ $0.068$ \\
\bottomrule
\end{tabular}

\end{table}

Back in \refchp{chp:3-nystrom} we already saw that for small values of \gls{smoothing-parameter}
the Nystr\"om approximation will only need a small \gls{sketching-matrix} in order
to achieve an accurate approximation. On the other hand, for large choices of
\gls{smoothing-parameter} the low-rank approximation will by itself not suffice.
The interplay between the two parts which make up the \gls{NCPP},
on one hand the low-rank approximation and on the other hand the
trace estimation on the residual, is illustrated well in
\reffig{fig:5-experiments-electronic-structure-matvec-mixture}.
For various values of \gls{smoothing-parameter} and a simultaneously changing
\gls{chebyshev-degree} $=120 / \sigma$ to keep an approximately equal interpolation
accuracy, the behavior of the error for fixed $n_{\Omega} + n_{\Psi} = 80$ is plotted.

\begin{figure}[ht]
    \centering
    %% Creator: Matplotlib, PGF backend
%%
%% To include the figure in your LaTeX document, write
%%   \input{<filename>.pgf}
%%
%% Make sure the required packages are loaded in your preamble
%%   \usepackage{pgf}
%%
%% Also ensure that all the required font packages are loaded; for instance,
%% the lmodern package is sometimes necessary when using math font.
%%   \usepackage{lmodern}
%%
%% Figures using additional raster images can only be included by \input if
%% they are in the same directory as the main LaTeX file. For loading figures
%% from other directories you can use the `import` package
%%   \usepackage{import}
%%
%% and then include the figures with
%%   \import{<path to file>}{<filename>.pgf}
%%
%% Matplotlib used the following preamble
%%   \def\mathdefault#1{#1}
%%   \everymath=\expandafter{\the\everymath\displaystyle}
%%   
%%   \makeatletter\@ifpackageloaded{underscore}{}{\usepackage[strings]{underscore}}\makeatother
%%
\begingroup%
\makeatletter%
\begin{pgfpicture}%
\pgfpathrectangle{\pgfpointorigin}{\pgfqpoint{5.471913in}{2.959073in}}%
\pgfusepath{use as bounding box, clip}%
\begin{pgfscope}%
\pgfsetbuttcap%
\pgfsetmiterjoin%
\definecolor{currentfill}{rgb}{1.000000,1.000000,1.000000}%
\pgfsetfillcolor{currentfill}%
\pgfsetlinewidth{0.000000pt}%
\definecolor{currentstroke}{rgb}{1.000000,1.000000,1.000000}%
\pgfsetstrokecolor{currentstroke}%
\pgfsetdash{}{0pt}%
\pgfpathmoveto{\pgfqpoint{0.000000in}{-0.000000in}}%
\pgfpathlineto{\pgfqpoint{5.471913in}{-0.000000in}}%
\pgfpathlineto{\pgfqpoint{5.471913in}{2.959073in}}%
\pgfpathlineto{\pgfqpoint{0.000000in}{2.959073in}}%
\pgfpathlineto{\pgfqpoint{0.000000in}{-0.000000in}}%
\pgfpathclose%
\pgfusepath{fill}%
\end{pgfscope}%
\begin{pgfscope}%
\pgfsetbuttcap%
\pgfsetmiterjoin%
\definecolor{currentfill}{rgb}{1.000000,1.000000,1.000000}%
\pgfsetfillcolor{currentfill}%
\pgfsetlinewidth{0.000000pt}%
\definecolor{currentstroke}{rgb}{0.000000,0.000000,0.000000}%
\pgfsetstrokecolor{currentstroke}%
\pgfsetstrokeopacity{0.000000}%
\pgfsetdash{}{0pt}%
\pgfpathmoveto{\pgfqpoint{0.721913in}{0.549073in}}%
\pgfpathlineto{\pgfqpoint{5.371913in}{0.549073in}}%
\pgfpathlineto{\pgfqpoint{5.371913in}{2.859073in}}%
\pgfpathlineto{\pgfqpoint{0.721913in}{2.859073in}}%
\pgfpathlineto{\pgfqpoint{0.721913in}{0.549073in}}%
\pgfpathclose%
\pgfusepath{fill}%
\end{pgfscope}%
\begin{pgfscope}%
\pgfsetbuttcap%
\pgfsetroundjoin%
\definecolor{currentfill}{rgb}{0.000000,0.000000,0.000000}%
\pgfsetfillcolor{currentfill}%
\pgfsetlinewidth{0.803000pt}%
\definecolor{currentstroke}{rgb}{0.000000,0.000000,0.000000}%
\pgfsetstrokecolor{currentstroke}%
\pgfsetdash{}{0pt}%
\pgfsys@defobject{currentmarker}{\pgfqpoint{0.000000in}{-0.048611in}}{\pgfqpoint{0.000000in}{0.000000in}}{%
\pgfpathmoveto{\pgfqpoint{0.000000in}{0.000000in}}%
\pgfpathlineto{\pgfqpoint{0.000000in}{-0.048611in}}%
\pgfusepath{stroke,fill}%
}%
\begin{pgfscope}%
\pgfsys@transformshift{0.933277in}{0.549073in}%
\pgfsys@useobject{currentmarker}{}%
\end{pgfscope}%
\end{pgfscope}%
\begin{pgfscope}%
\definecolor{textcolor}{rgb}{0.000000,0.000000,0.000000}%
\pgfsetstrokecolor{textcolor}%
\pgfsetfillcolor{textcolor}%
\pgftext[x=0.933277in,y=0.451851in,,top]{\color{textcolor}{\rmfamily\fontsize{12.000000}{14.400000}\selectfont\catcode`\^=\active\def^{\ifmmode\sp\else\^{}\fi}\catcode`\%=\active\def%{\%}$\mathdefault{10^{-2}}$}}%
\end{pgfscope}%
\begin{pgfscope}%
\pgfsetbuttcap%
\pgfsetroundjoin%
\definecolor{currentfill}{rgb}{0.000000,0.000000,0.000000}%
\pgfsetfillcolor{currentfill}%
\pgfsetlinewidth{0.803000pt}%
\definecolor{currentstroke}{rgb}{0.000000,0.000000,0.000000}%
\pgfsetstrokecolor{currentstroke}%
\pgfsetdash{}{0pt}%
\pgfsys@defobject{currentmarker}{\pgfqpoint{0.000000in}{-0.048611in}}{\pgfqpoint{0.000000in}{0.000000in}}{%
\pgfpathmoveto{\pgfqpoint{0.000000in}{0.000000in}}%
\pgfpathlineto{\pgfqpoint{0.000000in}{-0.048611in}}%
\pgfusepath{stroke,fill}%
}%
\begin{pgfscope}%
\pgfsys@transformshift{3.046913in}{0.549073in}%
\pgfsys@useobject{currentmarker}{}%
\end{pgfscope}%
\end{pgfscope}%
\begin{pgfscope}%
\definecolor{textcolor}{rgb}{0.000000,0.000000,0.000000}%
\pgfsetstrokecolor{textcolor}%
\pgfsetfillcolor{textcolor}%
\pgftext[x=3.046913in,y=0.451851in,,top]{\color{textcolor}{\rmfamily\fontsize{12.000000}{14.400000}\selectfont\catcode`\^=\active\def^{\ifmmode\sp\else\^{}\fi}\catcode`\%=\active\def%{\%}$\mathdefault{10^{-1}}$}}%
\end{pgfscope}%
\begin{pgfscope}%
\pgfsetbuttcap%
\pgfsetroundjoin%
\definecolor{currentfill}{rgb}{0.000000,0.000000,0.000000}%
\pgfsetfillcolor{currentfill}%
\pgfsetlinewidth{0.803000pt}%
\definecolor{currentstroke}{rgb}{0.000000,0.000000,0.000000}%
\pgfsetstrokecolor{currentstroke}%
\pgfsetdash{}{0pt}%
\pgfsys@defobject{currentmarker}{\pgfqpoint{0.000000in}{-0.048611in}}{\pgfqpoint{0.000000in}{0.000000in}}{%
\pgfpathmoveto{\pgfqpoint{0.000000in}{0.000000in}}%
\pgfpathlineto{\pgfqpoint{0.000000in}{-0.048611in}}%
\pgfusepath{stroke,fill}%
}%
\begin{pgfscope}%
\pgfsys@transformshift{5.160550in}{0.549073in}%
\pgfsys@useobject{currentmarker}{}%
\end{pgfscope}%
\end{pgfscope}%
\begin{pgfscope}%
\definecolor{textcolor}{rgb}{0.000000,0.000000,0.000000}%
\pgfsetstrokecolor{textcolor}%
\pgfsetfillcolor{textcolor}%
\pgftext[x=5.160550in,y=0.451851in,,top]{\color{textcolor}{\rmfamily\fontsize{12.000000}{14.400000}\selectfont\catcode`\^=\active\def^{\ifmmode\sp\else\^{}\fi}\catcode`\%=\active\def%{\%}$\mathdefault{10^{0}}$}}%
\end{pgfscope}%
\begin{pgfscope}%
\pgfsetbuttcap%
\pgfsetroundjoin%
\definecolor{currentfill}{rgb}{0.000000,0.000000,0.000000}%
\pgfsetfillcolor{currentfill}%
\pgfsetlinewidth{0.602250pt}%
\definecolor{currentstroke}{rgb}{0.000000,0.000000,0.000000}%
\pgfsetstrokecolor{currentstroke}%
\pgfsetdash{}{0pt}%
\pgfsys@defobject{currentmarker}{\pgfqpoint{0.000000in}{-0.027778in}}{\pgfqpoint{0.000000in}{0.000000in}}{%
\pgfpathmoveto{\pgfqpoint{0.000000in}{0.000000in}}%
\pgfpathlineto{\pgfqpoint{0.000000in}{-0.027778in}}%
\pgfusepath{stroke,fill}%
}%
\begin{pgfscope}%
\pgfsys@transformshift{0.728445in}{0.549073in}%
\pgfsys@useobject{currentmarker}{}%
\end{pgfscope}%
\end{pgfscope}%
\begin{pgfscope}%
\pgfsetbuttcap%
\pgfsetroundjoin%
\definecolor{currentfill}{rgb}{0.000000,0.000000,0.000000}%
\pgfsetfillcolor{currentfill}%
\pgfsetlinewidth{0.602250pt}%
\definecolor{currentstroke}{rgb}{0.000000,0.000000,0.000000}%
\pgfsetstrokecolor{currentstroke}%
\pgfsetdash{}{0pt}%
\pgfsys@defobject{currentmarker}{\pgfqpoint{0.000000in}{-0.027778in}}{\pgfqpoint{0.000000in}{0.000000in}}{%
\pgfpathmoveto{\pgfqpoint{0.000000in}{0.000000in}}%
\pgfpathlineto{\pgfqpoint{0.000000in}{-0.027778in}}%
\pgfusepath{stroke,fill}%
}%
\begin{pgfscope}%
\pgfsys@transformshift{0.836562in}{0.549073in}%
\pgfsys@useobject{currentmarker}{}%
\end{pgfscope}%
\end{pgfscope}%
\begin{pgfscope}%
\pgfsetbuttcap%
\pgfsetroundjoin%
\definecolor{currentfill}{rgb}{0.000000,0.000000,0.000000}%
\pgfsetfillcolor{currentfill}%
\pgfsetlinewidth{0.602250pt}%
\definecolor{currentstroke}{rgb}{0.000000,0.000000,0.000000}%
\pgfsetstrokecolor{currentstroke}%
\pgfsetdash{}{0pt}%
\pgfsys@defobject{currentmarker}{\pgfqpoint{0.000000in}{-0.027778in}}{\pgfqpoint{0.000000in}{0.000000in}}{%
\pgfpathmoveto{\pgfqpoint{0.000000in}{0.000000in}}%
\pgfpathlineto{\pgfqpoint{0.000000in}{-0.027778in}}%
\pgfusepath{stroke,fill}%
}%
\begin{pgfscope}%
\pgfsys@transformshift{1.569545in}{0.549073in}%
\pgfsys@useobject{currentmarker}{}%
\end{pgfscope}%
\end{pgfscope}%
\begin{pgfscope}%
\pgfsetbuttcap%
\pgfsetroundjoin%
\definecolor{currentfill}{rgb}{0.000000,0.000000,0.000000}%
\pgfsetfillcolor{currentfill}%
\pgfsetlinewidth{0.602250pt}%
\definecolor{currentstroke}{rgb}{0.000000,0.000000,0.000000}%
\pgfsetstrokecolor{currentstroke}%
\pgfsetdash{}{0pt}%
\pgfsys@defobject{currentmarker}{\pgfqpoint{0.000000in}{-0.027778in}}{\pgfqpoint{0.000000in}{0.000000in}}{%
\pgfpathmoveto{\pgfqpoint{0.000000in}{0.000000in}}%
\pgfpathlineto{\pgfqpoint{0.000000in}{-0.027778in}}%
\pgfusepath{stroke,fill}%
}%
\begin{pgfscope}%
\pgfsys@transformshift{1.941738in}{0.549073in}%
\pgfsys@useobject{currentmarker}{}%
\end{pgfscope}%
\end{pgfscope}%
\begin{pgfscope}%
\pgfsetbuttcap%
\pgfsetroundjoin%
\definecolor{currentfill}{rgb}{0.000000,0.000000,0.000000}%
\pgfsetfillcolor{currentfill}%
\pgfsetlinewidth{0.602250pt}%
\definecolor{currentstroke}{rgb}{0.000000,0.000000,0.000000}%
\pgfsetstrokecolor{currentstroke}%
\pgfsetdash{}{0pt}%
\pgfsys@defobject{currentmarker}{\pgfqpoint{0.000000in}{-0.027778in}}{\pgfqpoint{0.000000in}{0.000000in}}{%
\pgfpathmoveto{\pgfqpoint{0.000000in}{0.000000in}}%
\pgfpathlineto{\pgfqpoint{0.000000in}{-0.027778in}}%
\pgfusepath{stroke,fill}%
}%
\begin{pgfscope}%
\pgfsys@transformshift{2.205813in}{0.549073in}%
\pgfsys@useobject{currentmarker}{}%
\end{pgfscope}%
\end{pgfscope}%
\begin{pgfscope}%
\pgfsetbuttcap%
\pgfsetroundjoin%
\definecolor{currentfill}{rgb}{0.000000,0.000000,0.000000}%
\pgfsetfillcolor{currentfill}%
\pgfsetlinewidth{0.602250pt}%
\definecolor{currentstroke}{rgb}{0.000000,0.000000,0.000000}%
\pgfsetstrokecolor{currentstroke}%
\pgfsetdash{}{0pt}%
\pgfsys@defobject{currentmarker}{\pgfqpoint{0.000000in}{-0.027778in}}{\pgfqpoint{0.000000in}{0.000000in}}{%
\pgfpathmoveto{\pgfqpoint{0.000000in}{0.000000in}}%
\pgfpathlineto{\pgfqpoint{0.000000in}{-0.027778in}}%
\pgfusepath{stroke,fill}%
}%
\begin{pgfscope}%
\pgfsys@transformshift{2.410646in}{0.549073in}%
\pgfsys@useobject{currentmarker}{}%
\end{pgfscope}%
\end{pgfscope}%
\begin{pgfscope}%
\pgfsetbuttcap%
\pgfsetroundjoin%
\definecolor{currentfill}{rgb}{0.000000,0.000000,0.000000}%
\pgfsetfillcolor{currentfill}%
\pgfsetlinewidth{0.602250pt}%
\definecolor{currentstroke}{rgb}{0.000000,0.000000,0.000000}%
\pgfsetstrokecolor{currentstroke}%
\pgfsetdash{}{0pt}%
\pgfsys@defobject{currentmarker}{\pgfqpoint{0.000000in}{-0.027778in}}{\pgfqpoint{0.000000in}{0.000000in}}{%
\pgfpathmoveto{\pgfqpoint{0.000000in}{0.000000in}}%
\pgfpathlineto{\pgfqpoint{0.000000in}{-0.027778in}}%
\pgfusepath{stroke,fill}%
}%
\begin{pgfscope}%
\pgfsys@transformshift{2.578006in}{0.549073in}%
\pgfsys@useobject{currentmarker}{}%
\end{pgfscope}%
\end{pgfscope}%
\begin{pgfscope}%
\pgfsetbuttcap%
\pgfsetroundjoin%
\definecolor{currentfill}{rgb}{0.000000,0.000000,0.000000}%
\pgfsetfillcolor{currentfill}%
\pgfsetlinewidth{0.602250pt}%
\definecolor{currentstroke}{rgb}{0.000000,0.000000,0.000000}%
\pgfsetstrokecolor{currentstroke}%
\pgfsetdash{}{0pt}%
\pgfsys@defobject{currentmarker}{\pgfqpoint{0.000000in}{-0.027778in}}{\pgfqpoint{0.000000in}{0.000000in}}{%
\pgfpathmoveto{\pgfqpoint{0.000000in}{0.000000in}}%
\pgfpathlineto{\pgfqpoint{0.000000in}{-0.027778in}}%
\pgfusepath{stroke,fill}%
}%
\begin{pgfscope}%
\pgfsys@transformshift{2.719507in}{0.549073in}%
\pgfsys@useobject{currentmarker}{}%
\end{pgfscope}%
\end{pgfscope}%
\begin{pgfscope}%
\pgfsetbuttcap%
\pgfsetroundjoin%
\definecolor{currentfill}{rgb}{0.000000,0.000000,0.000000}%
\pgfsetfillcolor{currentfill}%
\pgfsetlinewidth{0.602250pt}%
\definecolor{currentstroke}{rgb}{0.000000,0.000000,0.000000}%
\pgfsetstrokecolor{currentstroke}%
\pgfsetdash{}{0pt}%
\pgfsys@defobject{currentmarker}{\pgfqpoint{0.000000in}{-0.027778in}}{\pgfqpoint{0.000000in}{0.000000in}}{%
\pgfpathmoveto{\pgfqpoint{0.000000in}{0.000000in}}%
\pgfpathlineto{\pgfqpoint{0.000000in}{-0.027778in}}%
\pgfusepath{stroke,fill}%
}%
\begin{pgfscope}%
\pgfsys@transformshift{2.842081in}{0.549073in}%
\pgfsys@useobject{currentmarker}{}%
\end{pgfscope}%
\end{pgfscope}%
\begin{pgfscope}%
\pgfsetbuttcap%
\pgfsetroundjoin%
\definecolor{currentfill}{rgb}{0.000000,0.000000,0.000000}%
\pgfsetfillcolor{currentfill}%
\pgfsetlinewidth{0.602250pt}%
\definecolor{currentstroke}{rgb}{0.000000,0.000000,0.000000}%
\pgfsetstrokecolor{currentstroke}%
\pgfsetdash{}{0pt}%
\pgfsys@defobject{currentmarker}{\pgfqpoint{0.000000in}{-0.027778in}}{\pgfqpoint{0.000000in}{0.000000in}}{%
\pgfpathmoveto{\pgfqpoint{0.000000in}{0.000000in}}%
\pgfpathlineto{\pgfqpoint{0.000000in}{-0.027778in}}%
\pgfusepath{stroke,fill}%
}%
\begin{pgfscope}%
\pgfsys@transformshift{2.950199in}{0.549073in}%
\pgfsys@useobject{currentmarker}{}%
\end{pgfscope}%
\end{pgfscope}%
\begin{pgfscope}%
\pgfsetbuttcap%
\pgfsetroundjoin%
\definecolor{currentfill}{rgb}{0.000000,0.000000,0.000000}%
\pgfsetfillcolor{currentfill}%
\pgfsetlinewidth{0.602250pt}%
\definecolor{currentstroke}{rgb}{0.000000,0.000000,0.000000}%
\pgfsetstrokecolor{currentstroke}%
\pgfsetdash{}{0pt}%
\pgfsys@defobject{currentmarker}{\pgfqpoint{0.000000in}{-0.027778in}}{\pgfqpoint{0.000000in}{0.000000in}}{%
\pgfpathmoveto{\pgfqpoint{0.000000in}{0.000000in}}%
\pgfpathlineto{\pgfqpoint{0.000000in}{-0.027778in}}%
\pgfusepath{stroke,fill}%
}%
\begin{pgfscope}%
\pgfsys@transformshift{3.683181in}{0.549073in}%
\pgfsys@useobject{currentmarker}{}%
\end{pgfscope}%
\end{pgfscope}%
\begin{pgfscope}%
\pgfsetbuttcap%
\pgfsetroundjoin%
\definecolor{currentfill}{rgb}{0.000000,0.000000,0.000000}%
\pgfsetfillcolor{currentfill}%
\pgfsetlinewidth{0.602250pt}%
\definecolor{currentstroke}{rgb}{0.000000,0.000000,0.000000}%
\pgfsetstrokecolor{currentstroke}%
\pgfsetdash{}{0pt}%
\pgfsys@defobject{currentmarker}{\pgfqpoint{0.000000in}{-0.027778in}}{\pgfqpoint{0.000000in}{0.000000in}}{%
\pgfpathmoveto{\pgfqpoint{0.000000in}{0.000000in}}%
\pgfpathlineto{\pgfqpoint{0.000000in}{-0.027778in}}%
\pgfusepath{stroke,fill}%
}%
\begin{pgfscope}%
\pgfsys@transformshift{4.055374in}{0.549073in}%
\pgfsys@useobject{currentmarker}{}%
\end{pgfscope}%
\end{pgfscope}%
\begin{pgfscope}%
\pgfsetbuttcap%
\pgfsetroundjoin%
\definecolor{currentfill}{rgb}{0.000000,0.000000,0.000000}%
\pgfsetfillcolor{currentfill}%
\pgfsetlinewidth{0.602250pt}%
\definecolor{currentstroke}{rgb}{0.000000,0.000000,0.000000}%
\pgfsetstrokecolor{currentstroke}%
\pgfsetdash{}{0pt}%
\pgfsys@defobject{currentmarker}{\pgfqpoint{0.000000in}{-0.027778in}}{\pgfqpoint{0.000000in}{0.000000in}}{%
\pgfpathmoveto{\pgfqpoint{0.000000in}{0.000000in}}%
\pgfpathlineto{\pgfqpoint{0.000000in}{-0.027778in}}%
\pgfusepath{stroke,fill}%
}%
\begin{pgfscope}%
\pgfsys@transformshift{4.319449in}{0.549073in}%
\pgfsys@useobject{currentmarker}{}%
\end{pgfscope}%
\end{pgfscope}%
\begin{pgfscope}%
\pgfsetbuttcap%
\pgfsetroundjoin%
\definecolor{currentfill}{rgb}{0.000000,0.000000,0.000000}%
\pgfsetfillcolor{currentfill}%
\pgfsetlinewidth{0.602250pt}%
\definecolor{currentstroke}{rgb}{0.000000,0.000000,0.000000}%
\pgfsetstrokecolor{currentstroke}%
\pgfsetdash{}{0pt}%
\pgfsys@defobject{currentmarker}{\pgfqpoint{0.000000in}{-0.027778in}}{\pgfqpoint{0.000000in}{0.000000in}}{%
\pgfpathmoveto{\pgfqpoint{0.000000in}{0.000000in}}%
\pgfpathlineto{\pgfqpoint{0.000000in}{-0.027778in}}%
\pgfusepath{stroke,fill}%
}%
\begin{pgfscope}%
\pgfsys@transformshift{4.524282in}{0.549073in}%
\pgfsys@useobject{currentmarker}{}%
\end{pgfscope}%
\end{pgfscope}%
\begin{pgfscope}%
\pgfsetbuttcap%
\pgfsetroundjoin%
\definecolor{currentfill}{rgb}{0.000000,0.000000,0.000000}%
\pgfsetfillcolor{currentfill}%
\pgfsetlinewidth{0.602250pt}%
\definecolor{currentstroke}{rgb}{0.000000,0.000000,0.000000}%
\pgfsetstrokecolor{currentstroke}%
\pgfsetdash{}{0pt}%
\pgfsys@defobject{currentmarker}{\pgfqpoint{0.000000in}{-0.027778in}}{\pgfqpoint{0.000000in}{0.000000in}}{%
\pgfpathmoveto{\pgfqpoint{0.000000in}{0.000000in}}%
\pgfpathlineto{\pgfqpoint{0.000000in}{-0.027778in}}%
\pgfusepath{stroke,fill}%
}%
\begin{pgfscope}%
\pgfsys@transformshift{4.691642in}{0.549073in}%
\pgfsys@useobject{currentmarker}{}%
\end{pgfscope}%
\end{pgfscope}%
\begin{pgfscope}%
\pgfsetbuttcap%
\pgfsetroundjoin%
\definecolor{currentfill}{rgb}{0.000000,0.000000,0.000000}%
\pgfsetfillcolor{currentfill}%
\pgfsetlinewidth{0.602250pt}%
\definecolor{currentstroke}{rgb}{0.000000,0.000000,0.000000}%
\pgfsetstrokecolor{currentstroke}%
\pgfsetdash{}{0pt}%
\pgfsys@defobject{currentmarker}{\pgfqpoint{0.000000in}{-0.027778in}}{\pgfqpoint{0.000000in}{0.000000in}}{%
\pgfpathmoveto{\pgfqpoint{0.000000in}{0.000000in}}%
\pgfpathlineto{\pgfqpoint{0.000000in}{-0.027778in}}%
\pgfusepath{stroke,fill}%
}%
\begin{pgfscope}%
\pgfsys@transformshift{4.833143in}{0.549073in}%
\pgfsys@useobject{currentmarker}{}%
\end{pgfscope}%
\end{pgfscope}%
\begin{pgfscope}%
\pgfsetbuttcap%
\pgfsetroundjoin%
\definecolor{currentfill}{rgb}{0.000000,0.000000,0.000000}%
\pgfsetfillcolor{currentfill}%
\pgfsetlinewidth{0.602250pt}%
\definecolor{currentstroke}{rgb}{0.000000,0.000000,0.000000}%
\pgfsetstrokecolor{currentstroke}%
\pgfsetdash{}{0pt}%
\pgfsys@defobject{currentmarker}{\pgfqpoint{0.000000in}{-0.027778in}}{\pgfqpoint{0.000000in}{0.000000in}}{%
\pgfpathmoveto{\pgfqpoint{0.000000in}{0.000000in}}%
\pgfpathlineto{\pgfqpoint{0.000000in}{-0.027778in}}%
\pgfusepath{stroke,fill}%
}%
\begin{pgfscope}%
\pgfsys@transformshift{4.955717in}{0.549073in}%
\pgfsys@useobject{currentmarker}{}%
\end{pgfscope}%
\end{pgfscope}%
\begin{pgfscope}%
\pgfsetbuttcap%
\pgfsetroundjoin%
\definecolor{currentfill}{rgb}{0.000000,0.000000,0.000000}%
\pgfsetfillcolor{currentfill}%
\pgfsetlinewidth{0.602250pt}%
\definecolor{currentstroke}{rgb}{0.000000,0.000000,0.000000}%
\pgfsetstrokecolor{currentstroke}%
\pgfsetdash{}{0pt}%
\pgfsys@defobject{currentmarker}{\pgfqpoint{0.000000in}{-0.027778in}}{\pgfqpoint{0.000000in}{0.000000in}}{%
\pgfpathmoveto{\pgfqpoint{0.000000in}{0.000000in}}%
\pgfpathlineto{\pgfqpoint{0.000000in}{-0.027778in}}%
\pgfusepath{stroke,fill}%
}%
\begin{pgfscope}%
\pgfsys@transformshift{5.063835in}{0.549073in}%
\pgfsys@useobject{currentmarker}{}%
\end{pgfscope}%
\end{pgfscope}%
\begin{pgfscope}%
\definecolor{textcolor}{rgb}{0.000000,0.000000,0.000000}%
\pgfsetstrokecolor{textcolor}%
\pgfsetfillcolor{textcolor}%
\pgftext[x=3.046913in,y=0.248148in,,top]{\color{textcolor}{\rmfamily\fontsize{12.000000}{14.400000}\selectfont\catcode`\^=\active\def^{\ifmmode\sp\else\^{}\fi}\catcode`\%=\active\def%{\%}$\sigma$}}%
\end{pgfscope}%
\begin{pgfscope}%
\pgfsetbuttcap%
\pgfsetroundjoin%
\definecolor{currentfill}{rgb}{0.000000,0.000000,0.000000}%
\pgfsetfillcolor{currentfill}%
\pgfsetlinewidth{0.803000pt}%
\definecolor{currentstroke}{rgb}{0.000000,0.000000,0.000000}%
\pgfsetstrokecolor{currentstroke}%
\pgfsetdash{}{0pt}%
\pgfsys@defobject{currentmarker}{\pgfqpoint{-0.048611in}{0.000000in}}{\pgfqpoint{-0.000000in}{0.000000in}}{%
\pgfpathmoveto{\pgfqpoint{-0.000000in}{0.000000in}}%
\pgfpathlineto{\pgfqpoint{-0.048611in}{0.000000in}}%
\pgfusepath{stroke,fill}%
}%
\begin{pgfscope}%
\pgfsys@transformshift{0.721913in}{0.986593in}%
\pgfsys@useobject{currentmarker}{}%
\end{pgfscope}%
\end{pgfscope}%
\begin{pgfscope}%
\definecolor{textcolor}{rgb}{0.000000,0.000000,0.000000}%
\pgfsetstrokecolor{textcolor}%
\pgfsetfillcolor{textcolor}%
\pgftext[x=0.303703in, y=0.928723in, left, base]{\color{textcolor}{\rmfamily\fontsize{12.000000}{14.400000}\selectfont\catcode`\^=\active\def^{\ifmmode\sp\else\^{}\fi}\catcode`\%=\active\def%{\%}$\mathdefault{10^{-6}}$}}%
\end{pgfscope}%
\begin{pgfscope}%
\pgfsetbuttcap%
\pgfsetroundjoin%
\definecolor{currentfill}{rgb}{0.000000,0.000000,0.000000}%
\pgfsetfillcolor{currentfill}%
\pgfsetlinewidth{0.803000pt}%
\definecolor{currentstroke}{rgb}{0.000000,0.000000,0.000000}%
\pgfsetstrokecolor{currentstroke}%
\pgfsetdash{}{0pt}%
\pgfsys@defobject{currentmarker}{\pgfqpoint{-0.048611in}{0.000000in}}{\pgfqpoint{-0.000000in}{0.000000in}}{%
\pgfpathmoveto{\pgfqpoint{-0.000000in}{0.000000in}}%
\pgfpathlineto{\pgfqpoint{-0.048611in}{0.000000in}}%
\pgfusepath{stroke,fill}%
}%
\begin{pgfscope}%
\pgfsys@transformshift{0.721913in}{1.611802in}%
\pgfsys@useobject{currentmarker}{}%
\end{pgfscope}%
\end{pgfscope}%
\begin{pgfscope}%
\definecolor{textcolor}{rgb}{0.000000,0.000000,0.000000}%
\pgfsetstrokecolor{textcolor}%
\pgfsetfillcolor{textcolor}%
\pgftext[x=0.303703in, y=1.553932in, left, base]{\color{textcolor}{\rmfamily\fontsize{12.000000}{14.400000}\selectfont\catcode`\^=\active\def^{\ifmmode\sp\else\^{}\fi}\catcode`\%=\active\def%{\%}$\mathdefault{10^{-4}}$}}%
\end{pgfscope}%
\begin{pgfscope}%
\pgfsetbuttcap%
\pgfsetroundjoin%
\definecolor{currentfill}{rgb}{0.000000,0.000000,0.000000}%
\pgfsetfillcolor{currentfill}%
\pgfsetlinewidth{0.803000pt}%
\definecolor{currentstroke}{rgb}{0.000000,0.000000,0.000000}%
\pgfsetstrokecolor{currentstroke}%
\pgfsetdash{}{0pt}%
\pgfsys@defobject{currentmarker}{\pgfqpoint{-0.048611in}{0.000000in}}{\pgfqpoint{-0.000000in}{0.000000in}}{%
\pgfpathmoveto{\pgfqpoint{-0.000000in}{0.000000in}}%
\pgfpathlineto{\pgfqpoint{-0.048611in}{0.000000in}}%
\pgfusepath{stroke,fill}%
}%
\begin{pgfscope}%
\pgfsys@transformshift{0.721913in}{2.237011in}%
\pgfsys@useobject{currentmarker}{}%
\end{pgfscope}%
\end{pgfscope}%
\begin{pgfscope}%
\definecolor{textcolor}{rgb}{0.000000,0.000000,0.000000}%
\pgfsetstrokecolor{textcolor}%
\pgfsetfillcolor{textcolor}%
\pgftext[x=0.303703in, y=2.179141in, left, base]{\color{textcolor}{\rmfamily\fontsize{12.000000}{14.400000}\selectfont\catcode`\^=\active\def^{\ifmmode\sp\else\^{}\fi}\catcode`\%=\active\def%{\%}$\mathdefault{10^{-2}}$}}%
\end{pgfscope}%
\begin{pgfscope}%
\definecolor{textcolor}{rgb}{0.000000,0.000000,0.000000}%
\pgfsetstrokecolor{textcolor}%
\pgfsetfillcolor{textcolor}%
\pgftext[x=0.248148in,y=1.704073in,,bottom,rotate=90.000000]{\color{textcolor}{\rmfamily\fontsize{12.000000}{14.400000}\selectfont\catcode`\^=\active\def^{\ifmmode\sp\else\^{}\fi}\catcode`\%=\active\def%{\%}$L^1$ relative error}}%
\end{pgfscope}%
\begin{pgfscope}%
\pgfpathrectangle{\pgfqpoint{0.721913in}{0.549073in}}{\pgfqpoint{4.650000in}{2.310000in}}%
\pgfusepath{clip}%
\pgfsetrectcap%
\pgfsetroundjoin%
\pgfsetlinewidth{1.003750pt}%
\definecolor{currentstroke}{rgb}{0.001462,0.000466,0.013866}%
\pgfsetstrokecolor{currentstroke}%
\pgfsetdash{}{0pt}%
\pgfpathmoveto{\pgfqpoint{0.933277in}{2.479576in}}%
\pgfpathlineto{\pgfqpoint{1.402974in}{2.462425in}}%
\pgfpathlineto{\pgfqpoint{1.872671in}{2.439931in}}%
\pgfpathlineto{\pgfqpoint{2.342368in}{2.416488in}}%
\pgfpathlineto{\pgfqpoint{2.812065in}{2.384914in}}%
\pgfpathlineto{\pgfqpoint{3.281762in}{2.368537in}}%
\pgfpathlineto{\pgfqpoint{3.751459in}{2.357517in}}%
\pgfpathlineto{\pgfqpoint{4.221156in}{2.341240in}}%
\pgfpathlineto{\pgfqpoint{4.690853in}{2.315218in}}%
\pgfpathlineto{\pgfqpoint{5.160550in}{2.277226in}}%
\pgfusepath{stroke}%
\end{pgfscope}%
\begin{pgfscope}%
\pgfpathrectangle{\pgfqpoint{0.721913in}{0.549073in}}{\pgfqpoint{4.650000in}{2.310000in}}%
\pgfusepath{clip}%
\pgfsetbuttcap%
\pgfsetroundjoin%
\definecolor{currentfill}{rgb}{0.001462,0.000466,0.013866}%
\pgfsetfillcolor{currentfill}%
\pgfsetlinewidth{1.003750pt}%
\definecolor{currentstroke}{rgb}{0.001462,0.000466,0.013866}%
\pgfsetstrokecolor{currentstroke}%
\pgfsetdash{}{0pt}%
\pgfsys@defobject{currentmarker}{\pgfqpoint{-0.020833in}{-0.020833in}}{\pgfqpoint{0.020833in}{0.020833in}}{%
\pgfpathmoveto{\pgfqpoint{0.000000in}{-0.020833in}}%
\pgfpathcurveto{\pgfqpoint{0.005525in}{-0.020833in}}{\pgfqpoint{0.010825in}{-0.018638in}}{\pgfqpoint{0.014731in}{-0.014731in}}%
\pgfpathcurveto{\pgfqpoint{0.018638in}{-0.010825in}}{\pgfqpoint{0.020833in}{-0.005525in}}{\pgfqpoint{0.020833in}{0.000000in}}%
\pgfpathcurveto{\pgfqpoint{0.020833in}{0.005525in}}{\pgfqpoint{0.018638in}{0.010825in}}{\pgfqpoint{0.014731in}{0.014731in}}%
\pgfpathcurveto{\pgfqpoint{0.010825in}{0.018638in}}{\pgfqpoint{0.005525in}{0.020833in}}{\pgfqpoint{0.000000in}{0.020833in}}%
\pgfpathcurveto{\pgfqpoint{-0.005525in}{0.020833in}}{\pgfqpoint{-0.010825in}{0.018638in}}{\pgfqpoint{-0.014731in}{0.014731in}}%
\pgfpathcurveto{\pgfqpoint{-0.018638in}{0.010825in}}{\pgfqpoint{-0.020833in}{0.005525in}}{\pgfqpoint{-0.020833in}{0.000000in}}%
\pgfpathcurveto{\pgfqpoint{-0.020833in}{-0.005525in}}{\pgfqpoint{-0.018638in}{-0.010825in}}{\pgfqpoint{-0.014731in}{-0.014731in}}%
\pgfpathcurveto{\pgfqpoint{-0.010825in}{-0.018638in}}{\pgfqpoint{-0.005525in}{-0.020833in}}{\pgfqpoint{0.000000in}{-0.020833in}}%
\pgfpathlineto{\pgfqpoint{0.000000in}{-0.020833in}}%
\pgfpathclose%
\pgfusepath{stroke,fill}%
}%
\begin{pgfscope}%
\pgfsys@transformshift{0.933277in}{2.479576in}%
\pgfsys@useobject{currentmarker}{}%
\end{pgfscope}%
\begin{pgfscope}%
\pgfsys@transformshift{1.402974in}{2.462425in}%
\pgfsys@useobject{currentmarker}{}%
\end{pgfscope}%
\begin{pgfscope}%
\pgfsys@transformshift{1.872671in}{2.439931in}%
\pgfsys@useobject{currentmarker}{}%
\end{pgfscope}%
\begin{pgfscope}%
\pgfsys@transformshift{2.342368in}{2.416488in}%
\pgfsys@useobject{currentmarker}{}%
\end{pgfscope}%
\begin{pgfscope}%
\pgfsys@transformshift{2.812065in}{2.384914in}%
\pgfsys@useobject{currentmarker}{}%
\end{pgfscope}%
\begin{pgfscope}%
\pgfsys@transformshift{3.281762in}{2.368537in}%
\pgfsys@useobject{currentmarker}{}%
\end{pgfscope}%
\begin{pgfscope}%
\pgfsys@transformshift{3.751459in}{2.357517in}%
\pgfsys@useobject{currentmarker}{}%
\end{pgfscope}%
\begin{pgfscope}%
\pgfsys@transformshift{4.221156in}{2.341240in}%
\pgfsys@useobject{currentmarker}{}%
\end{pgfscope}%
\begin{pgfscope}%
\pgfsys@transformshift{4.690853in}{2.315218in}%
\pgfsys@useobject{currentmarker}{}%
\end{pgfscope}%
\begin{pgfscope}%
\pgfsys@transformshift{5.160550in}{2.277226in}%
\pgfsys@useobject{currentmarker}{}%
\end{pgfscope}%
\end{pgfscope}%
\begin{pgfscope}%
\pgfpathrectangle{\pgfqpoint{0.721913in}{0.549073in}}{\pgfqpoint{4.650000in}{2.310000in}}%
\pgfusepath{clip}%
\pgfsetrectcap%
\pgfsetroundjoin%
\pgfsetlinewidth{1.003750pt}%
\definecolor{currentstroke}{rgb}{0.232077,0.059889,0.437695}%
\pgfsetstrokecolor{currentstroke}%
\pgfsetdash{}{0pt}%
\pgfpathmoveto{\pgfqpoint{0.933277in}{1.755723in}}%
\pgfpathlineto{\pgfqpoint{1.402974in}{2.007690in}}%
\pgfpathlineto{\pgfqpoint{1.872671in}{2.063318in}}%
\pgfpathlineto{\pgfqpoint{2.342368in}{2.137295in}}%
\pgfpathlineto{\pgfqpoint{2.812065in}{2.217068in}}%
\pgfpathlineto{\pgfqpoint{3.281762in}{2.251388in}}%
\pgfpathlineto{\pgfqpoint{3.751459in}{2.291969in}}%
\pgfpathlineto{\pgfqpoint{4.221156in}{2.298214in}}%
\pgfpathlineto{\pgfqpoint{4.690853in}{2.289794in}}%
\pgfpathlineto{\pgfqpoint{5.160550in}{2.265671in}}%
\pgfusepath{stroke}%
\end{pgfscope}%
\begin{pgfscope}%
\pgfpathrectangle{\pgfqpoint{0.721913in}{0.549073in}}{\pgfqpoint{4.650000in}{2.310000in}}%
\pgfusepath{clip}%
\pgfsetbuttcap%
\pgfsetroundjoin%
\definecolor{currentfill}{rgb}{0.232077,0.059889,0.437695}%
\pgfsetfillcolor{currentfill}%
\pgfsetlinewidth{1.003750pt}%
\definecolor{currentstroke}{rgb}{0.232077,0.059889,0.437695}%
\pgfsetstrokecolor{currentstroke}%
\pgfsetdash{}{0pt}%
\pgfsys@defobject{currentmarker}{\pgfqpoint{-0.020833in}{-0.020833in}}{\pgfqpoint{0.020833in}{0.020833in}}{%
\pgfpathmoveto{\pgfqpoint{0.000000in}{-0.020833in}}%
\pgfpathcurveto{\pgfqpoint{0.005525in}{-0.020833in}}{\pgfqpoint{0.010825in}{-0.018638in}}{\pgfqpoint{0.014731in}{-0.014731in}}%
\pgfpathcurveto{\pgfqpoint{0.018638in}{-0.010825in}}{\pgfqpoint{0.020833in}{-0.005525in}}{\pgfqpoint{0.020833in}{0.000000in}}%
\pgfpathcurveto{\pgfqpoint{0.020833in}{0.005525in}}{\pgfqpoint{0.018638in}{0.010825in}}{\pgfqpoint{0.014731in}{0.014731in}}%
\pgfpathcurveto{\pgfqpoint{0.010825in}{0.018638in}}{\pgfqpoint{0.005525in}{0.020833in}}{\pgfqpoint{0.000000in}{0.020833in}}%
\pgfpathcurveto{\pgfqpoint{-0.005525in}{0.020833in}}{\pgfqpoint{-0.010825in}{0.018638in}}{\pgfqpoint{-0.014731in}{0.014731in}}%
\pgfpathcurveto{\pgfqpoint{-0.018638in}{0.010825in}}{\pgfqpoint{-0.020833in}{0.005525in}}{\pgfqpoint{-0.020833in}{0.000000in}}%
\pgfpathcurveto{\pgfqpoint{-0.020833in}{-0.005525in}}{\pgfqpoint{-0.018638in}{-0.010825in}}{\pgfqpoint{-0.014731in}{-0.014731in}}%
\pgfpathcurveto{\pgfqpoint{-0.010825in}{-0.018638in}}{\pgfqpoint{-0.005525in}{-0.020833in}}{\pgfqpoint{0.000000in}{-0.020833in}}%
\pgfpathlineto{\pgfqpoint{0.000000in}{-0.020833in}}%
\pgfpathclose%
\pgfusepath{stroke,fill}%
}%
\begin{pgfscope}%
\pgfsys@transformshift{0.933277in}{1.755723in}%
\pgfsys@useobject{currentmarker}{}%
\end{pgfscope}%
\begin{pgfscope}%
\pgfsys@transformshift{1.402974in}{2.007690in}%
\pgfsys@useobject{currentmarker}{}%
\end{pgfscope}%
\begin{pgfscope}%
\pgfsys@transformshift{1.872671in}{2.063318in}%
\pgfsys@useobject{currentmarker}{}%
\end{pgfscope}%
\begin{pgfscope}%
\pgfsys@transformshift{2.342368in}{2.137295in}%
\pgfsys@useobject{currentmarker}{}%
\end{pgfscope}%
\begin{pgfscope}%
\pgfsys@transformshift{2.812065in}{2.217068in}%
\pgfsys@useobject{currentmarker}{}%
\end{pgfscope}%
\begin{pgfscope}%
\pgfsys@transformshift{3.281762in}{2.251388in}%
\pgfsys@useobject{currentmarker}{}%
\end{pgfscope}%
\begin{pgfscope}%
\pgfsys@transformshift{3.751459in}{2.291969in}%
\pgfsys@useobject{currentmarker}{}%
\end{pgfscope}%
\begin{pgfscope}%
\pgfsys@transformshift{4.221156in}{2.298214in}%
\pgfsys@useobject{currentmarker}{}%
\end{pgfscope}%
\begin{pgfscope}%
\pgfsys@transformshift{4.690853in}{2.289794in}%
\pgfsys@useobject{currentmarker}{}%
\end{pgfscope}%
\begin{pgfscope}%
\pgfsys@transformshift{5.160550in}{2.265671in}%
\pgfsys@useobject{currentmarker}{}%
\end{pgfscope}%
\end{pgfscope}%
\begin{pgfscope}%
\pgfpathrectangle{\pgfqpoint{0.721913in}{0.549073in}}{\pgfqpoint{4.650000in}{2.310000in}}%
\pgfusepath{clip}%
\pgfsetrectcap%
\pgfsetroundjoin%
\pgfsetlinewidth{1.003750pt}%
\definecolor{currentstroke}{rgb}{0.550287,0.161158,0.505719}%
\pgfsetstrokecolor{currentstroke}%
\pgfsetdash{}{0pt}%
\pgfpathmoveto{\pgfqpoint{0.933277in}{0.788915in}}%
\pgfpathlineto{\pgfqpoint{1.402974in}{0.895829in}}%
\pgfpathlineto{\pgfqpoint{1.872671in}{1.662245in}}%
\pgfpathlineto{\pgfqpoint{2.342368in}{2.085820in}}%
\pgfpathlineto{\pgfqpoint{2.812065in}{2.187431in}}%
\pgfpathlineto{\pgfqpoint{3.281762in}{2.171611in}}%
\pgfpathlineto{\pgfqpoint{3.751459in}{2.189289in}}%
\pgfpathlineto{\pgfqpoint{4.221156in}{2.214277in}}%
\pgfpathlineto{\pgfqpoint{4.690853in}{2.238580in}}%
\pgfpathlineto{\pgfqpoint{5.160550in}{2.256783in}}%
\pgfusepath{stroke}%
\end{pgfscope}%
\begin{pgfscope}%
\pgfpathrectangle{\pgfqpoint{0.721913in}{0.549073in}}{\pgfqpoint{4.650000in}{2.310000in}}%
\pgfusepath{clip}%
\pgfsetbuttcap%
\pgfsetroundjoin%
\definecolor{currentfill}{rgb}{0.550287,0.161158,0.505719}%
\pgfsetfillcolor{currentfill}%
\pgfsetlinewidth{1.003750pt}%
\definecolor{currentstroke}{rgb}{0.550287,0.161158,0.505719}%
\pgfsetstrokecolor{currentstroke}%
\pgfsetdash{}{0pt}%
\pgfsys@defobject{currentmarker}{\pgfqpoint{-0.020833in}{-0.020833in}}{\pgfqpoint{0.020833in}{0.020833in}}{%
\pgfpathmoveto{\pgfqpoint{0.000000in}{-0.020833in}}%
\pgfpathcurveto{\pgfqpoint{0.005525in}{-0.020833in}}{\pgfqpoint{0.010825in}{-0.018638in}}{\pgfqpoint{0.014731in}{-0.014731in}}%
\pgfpathcurveto{\pgfqpoint{0.018638in}{-0.010825in}}{\pgfqpoint{0.020833in}{-0.005525in}}{\pgfqpoint{0.020833in}{0.000000in}}%
\pgfpathcurveto{\pgfqpoint{0.020833in}{0.005525in}}{\pgfqpoint{0.018638in}{0.010825in}}{\pgfqpoint{0.014731in}{0.014731in}}%
\pgfpathcurveto{\pgfqpoint{0.010825in}{0.018638in}}{\pgfqpoint{0.005525in}{0.020833in}}{\pgfqpoint{0.000000in}{0.020833in}}%
\pgfpathcurveto{\pgfqpoint{-0.005525in}{0.020833in}}{\pgfqpoint{-0.010825in}{0.018638in}}{\pgfqpoint{-0.014731in}{0.014731in}}%
\pgfpathcurveto{\pgfqpoint{-0.018638in}{0.010825in}}{\pgfqpoint{-0.020833in}{0.005525in}}{\pgfqpoint{-0.020833in}{0.000000in}}%
\pgfpathcurveto{\pgfqpoint{-0.020833in}{-0.005525in}}{\pgfqpoint{-0.018638in}{-0.010825in}}{\pgfqpoint{-0.014731in}{-0.014731in}}%
\pgfpathcurveto{\pgfqpoint{-0.010825in}{-0.018638in}}{\pgfqpoint{-0.005525in}{-0.020833in}}{\pgfqpoint{0.000000in}{-0.020833in}}%
\pgfpathlineto{\pgfqpoint{0.000000in}{-0.020833in}}%
\pgfpathclose%
\pgfusepath{stroke,fill}%
}%
\begin{pgfscope}%
\pgfsys@transformshift{0.933277in}{0.788915in}%
\pgfsys@useobject{currentmarker}{}%
\end{pgfscope}%
\begin{pgfscope}%
\pgfsys@transformshift{1.402974in}{0.895829in}%
\pgfsys@useobject{currentmarker}{}%
\end{pgfscope}%
\begin{pgfscope}%
\pgfsys@transformshift{1.872671in}{1.662245in}%
\pgfsys@useobject{currentmarker}{}%
\end{pgfscope}%
\begin{pgfscope}%
\pgfsys@transformshift{2.342368in}{2.085820in}%
\pgfsys@useobject{currentmarker}{}%
\end{pgfscope}%
\begin{pgfscope}%
\pgfsys@transformshift{2.812065in}{2.187431in}%
\pgfsys@useobject{currentmarker}{}%
\end{pgfscope}%
\begin{pgfscope}%
\pgfsys@transformshift{3.281762in}{2.171611in}%
\pgfsys@useobject{currentmarker}{}%
\end{pgfscope}%
\begin{pgfscope}%
\pgfsys@transformshift{3.751459in}{2.189289in}%
\pgfsys@useobject{currentmarker}{}%
\end{pgfscope}%
\begin{pgfscope}%
\pgfsys@transformshift{4.221156in}{2.214277in}%
\pgfsys@useobject{currentmarker}{}%
\end{pgfscope}%
\begin{pgfscope}%
\pgfsys@transformshift{4.690853in}{2.238580in}%
\pgfsys@useobject{currentmarker}{}%
\end{pgfscope}%
\begin{pgfscope}%
\pgfsys@transformshift{5.160550in}{2.256783in}%
\pgfsys@useobject{currentmarker}{}%
\end{pgfscope}%
\end{pgfscope}%
\begin{pgfscope}%
\pgfpathrectangle{\pgfqpoint{0.721913in}{0.549073in}}{\pgfqpoint{4.650000in}{2.310000in}}%
\pgfusepath{clip}%
\pgfsetrectcap%
\pgfsetroundjoin%
\pgfsetlinewidth{1.003750pt}%
\definecolor{currentstroke}{rgb}{0.868793,0.287728,0.409303}%
\pgfsetstrokecolor{currentstroke}%
\pgfsetdash{}{0pt}%
\pgfpathmoveto{\pgfqpoint{0.933277in}{0.786265in}}%
\pgfpathlineto{\pgfqpoint{1.402974in}{0.654073in}}%
\pgfpathlineto{\pgfqpoint{1.872671in}{0.878990in}}%
\pgfpathlineto{\pgfqpoint{2.342368in}{1.669917in}}%
\pgfpathlineto{\pgfqpoint{2.812065in}{1.987922in}}%
\pgfpathlineto{\pgfqpoint{3.281762in}{2.048616in}}%
\pgfpathlineto{\pgfqpoint{3.751459in}{2.083031in}}%
\pgfpathlineto{\pgfqpoint{4.221156in}{2.165306in}}%
\pgfpathlineto{\pgfqpoint{4.690853in}{2.233814in}}%
\pgfpathlineto{\pgfqpoint{5.160550in}{2.251381in}}%
\pgfusepath{stroke}%
\end{pgfscope}%
\begin{pgfscope}%
\pgfpathrectangle{\pgfqpoint{0.721913in}{0.549073in}}{\pgfqpoint{4.650000in}{2.310000in}}%
\pgfusepath{clip}%
\pgfsetbuttcap%
\pgfsetroundjoin%
\definecolor{currentfill}{rgb}{0.868793,0.287728,0.409303}%
\pgfsetfillcolor{currentfill}%
\pgfsetlinewidth{1.003750pt}%
\definecolor{currentstroke}{rgb}{0.868793,0.287728,0.409303}%
\pgfsetstrokecolor{currentstroke}%
\pgfsetdash{}{0pt}%
\pgfsys@defobject{currentmarker}{\pgfqpoint{-0.020833in}{-0.020833in}}{\pgfqpoint{0.020833in}{0.020833in}}{%
\pgfpathmoveto{\pgfqpoint{0.000000in}{-0.020833in}}%
\pgfpathcurveto{\pgfqpoint{0.005525in}{-0.020833in}}{\pgfqpoint{0.010825in}{-0.018638in}}{\pgfqpoint{0.014731in}{-0.014731in}}%
\pgfpathcurveto{\pgfqpoint{0.018638in}{-0.010825in}}{\pgfqpoint{0.020833in}{-0.005525in}}{\pgfqpoint{0.020833in}{0.000000in}}%
\pgfpathcurveto{\pgfqpoint{0.020833in}{0.005525in}}{\pgfqpoint{0.018638in}{0.010825in}}{\pgfqpoint{0.014731in}{0.014731in}}%
\pgfpathcurveto{\pgfqpoint{0.010825in}{0.018638in}}{\pgfqpoint{0.005525in}{0.020833in}}{\pgfqpoint{0.000000in}{0.020833in}}%
\pgfpathcurveto{\pgfqpoint{-0.005525in}{0.020833in}}{\pgfqpoint{-0.010825in}{0.018638in}}{\pgfqpoint{-0.014731in}{0.014731in}}%
\pgfpathcurveto{\pgfqpoint{-0.018638in}{0.010825in}}{\pgfqpoint{-0.020833in}{0.005525in}}{\pgfqpoint{-0.020833in}{0.000000in}}%
\pgfpathcurveto{\pgfqpoint{-0.020833in}{-0.005525in}}{\pgfqpoint{-0.018638in}{-0.010825in}}{\pgfqpoint{-0.014731in}{-0.014731in}}%
\pgfpathcurveto{\pgfqpoint{-0.010825in}{-0.018638in}}{\pgfqpoint{-0.005525in}{-0.020833in}}{\pgfqpoint{0.000000in}{-0.020833in}}%
\pgfpathlineto{\pgfqpoint{0.000000in}{-0.020833in}}%
\pgfpathclose%
\pgfusepath{stroke,fill}%
}%
\begin{pgfscope}%
\pgfsys@transformshift{0.933277in}{0.786265in}%
\pgfsys@useobject{currentmarker}{}%
\end{pgfscope}%
\begin{pgfscope}%
\pgfsys@transformshift{1.402974in}{0.654073in}%
\pgfsys@useobject{currentmarker}{}%
\end{pgfscope}%
\begin{pgfscope}%
\pgfsys@transformshift{1.872671in}{0.878990in}%
\pgfsys@useobject{currentmarker}{}%
\end{pgfscope}%
\begin{pgfscope}%
\pgfsys@transformshift{2.342368in}{1.669917in}%
\pgfsys@useobject{currentmarker}{}%
\end{pgfscope}%
\begin{pgfscope}%
\pgfsys@transformshift{2.812065in}{1.987922in}%
\pgfsys@useobject{currentmarker}{}%
\end{pgfscope}%
\begin{pgfscope}%
\pgfsys@transformshift{3.281762in}{2.048616in}%
\pgfsys@useobject{currentmarker}{}%
\end{pgfscope}%
\begin{pgfscope}%
\pgfsys@transformshift{3.751459in}{2.083031in}%
\pgfsys@useobject{currentmarker}{}%
\end{pgfscope}%
\begin{pgfscope}%
\pgfsys@transformshift{4.221156in}{2.165306in}%
\pgfsys@useobject{currentmarker}{}%
\end{pgfscope}%
\begin{pgfscope}%
\pgfsys@transformshift{4.690853in}{2.233814in}%
\pgfsys@useobject{currentmarker}{}%
\end{pgfscope}%
\begin{pgfscope}%
\pgfsys@transformshift{5.160550in}{2.251381in}%
\pgfsys@useobject{currentmarker}{}%
\end{pgfscope}%
\end{pgfscope}%
\begin{pgfscope}%
\pgfpathrectangle{\pgfqpoint{0.721913in}{0.549073in}}{\pgfqpoint{4.650000in}{2.310000in}}%
\pgfusepath{clip}%
\pgfsetrectcap%
\pgfsetroundjoin%
\pgfsetlinewidth{1.003750pt}%
\definecolor{currentstroke}{rgb}{0.994738,0.624350,0.427397}%
\pgfsetstrokecolor{currentstroke}%
\pgfsetdash{}{0pt}%
\pgfpathmoveto{\pgfqpoint{0.933277in}{0.787561in}}%
\pgfpathlineto{\pgfqpoint{1.402974in}{0.678593in}}%
\pgfpathlineto{\pgfqpoint{1.872671in}{0.689462in}}%
\pgfpathlineto{\pgfqpoint{2.342368in}{1.059988in}}%
\pgfpathlineto{\pgfqpoint{2.812065in}{1.933710in}}%
\pgfpathlineto{\pgfqpoint{3.281762in}{2.271899in}}%
\pgfpathlineto{\pgfqpoint{3.751459in}{2.428254in}}%
\pgfpathlineto{\pgfqpoint{4.221156in}{2.569872in}}%
\pgfpathlineto{\pgfqpoint{4.690853in}{2.682477in}}%
\pgfpathlineto{\pgfqpoint{5.160550in}{2.754073in}}%
\pgfusepath{stroke}%
\end{pgfscope}%
\begin{pgfscope}%
\pgfpathrectangle{\pgfqpoint{0.721913in}{0.549073in}}{\pgfqpoint{4.650000in}{2.310000in}}%
\pgfusepath{clip}%
\pgfsetbuttcap%
\pgfsetroundjoin%
\definecolor{currentfill}{rgb}{0.994738,0.624350,0.427397}%
\pgfsetfillcolor{currentfill}%
\pgfsetlinewidth{1.003750pt}%
\definecolor{currentstroke}{rgb}{0.994738,0.624350,0.427397}%
\pgfsetstrokecolor{currentstroke}%
\pgfsetdash{}{0pt}%
\pgfsys@defobject{currentmarker}{\pgfqpoint{-0.020833in}{-0.020833in}}{\pgfqpoint{0.020833in}{0.020833in}}{%
\pgfpathmoveto{\pgfqpoint{0.000000in}{-0.020833in}}%
\pgfpathcurveto{\pgfqpoint{0.005525in}{-0.020833in}}{\pgfqpoint{0.010825in}{-0.018638in}}{\pgfqpoint{0.014731in}{-0.014731in}}%
\pgfpathcurveto{\pgfqpoint{0.018638in}{-0.010825in}}{\pgfqpoint{0.020833in}{-0.005525in}}{\pgfqpoint{0.020833in}{0.000000in}}%
\pgfpathcurveto{\pgfqpoint{0.020833in}{0.005525in}}{\pgfqpoint{0.018638in}{0.010825in}}{\pgfqpoint{0.014731in}{0.014731in}}%
\pgfpathcurveto{\pgfqpoint{0.010825in}{0.018638in}}{\pgfqpoint{0.005525in}{0.020833in}}{\pgfqpoint{0.000000in}{0.020833in}}%
\pgfpathcurveto{\pgfqpoint{-0.005525in}{0.020833in}}{\pgfqpoint{-0.010825in}{0.018638in}}{\pgfqpoint{-0.014731in}{0.014731in}}%
\pgfpathcurveto{\pgfqpoint{-0.018638in}{0.010825in}}{\pgfqpoint{-0.020833in}{0.005525in}}{\pgfqpoint{-0.020833in}{0.000000in}}%
\pgfpathcurveto{\pgfqpoint{-0.020833in}{-0.005525in}}{\pgfqpoint{-0.018638in}{-0.010825in}}{\pgfqpoint{-0.014731in}{-0.014731in}}%
\pgfpathcurveto{\pgfqpoint{-0.010825in}{-0.018638in}}{\pgfqpoint{-0.005525in}{-0.020833in}}{\pgfqpoint{0.000000in}{-0.020833in}}%
\pgfpathlineto{\pgfqpoint{0.000000in}{-0.020833in}}%
\pgfpathclose%
\pgfusepath{stroke,fill}%
}%
\begin{pgfscope}%
\pgfsys@transformshift{0.933277in}{0.787561in}%
\pgfsys@useobject{currentmarker}{}%
\end{pgfscope}%
\begin{pgfscope}%
\pgfsys@transformshift{1.402974in}{0.678593in}%
\pgfsys@useobject{currentmarker}{}%
\end{pgfscope}%
\begin{pgfscope}%
\pgfsys@transformshift{1.872671in}{0.689462in}%
\pgfsys@useobject{currentmarker}{}%
\end{pgfscope}%
\begin{pgfscope}%
\pgfsys@transformshift{2.342368in}{1.059988in}%
\pgfsys@useobject{currentmarker}{}%
\end{pgfscope}%
\begin{pgfscope}%
\pgfsys@transformshift{2.812065in}{1.933710in}%
\pgfsys@useobject{currentmarker}{}%
\end{pgfscope}%
\begin{pgfscope}%
\pgfsys@transformshift{3.281762in}{2.271899in}%
\pgfsys@useobject{currentmarker}{}%
\end{pgfscope}%
\begin{pgfscope}%
\pgfsys@transformshift{3.751459in}{2.428254in}%
\pgfsys@useobject{currentmarker}{}%
\end{pgfscope}%
\begin{pgfscope}%
\pgfsys@transformshift{4.221156in}{2.569872in}%
\pgfsys@useobject{currentmarker}{}%
\end{pgfscope}%
\begin{pgfscope}%
\pgfsys@transformshift{4.690853in}{2.682477in}%
\pgfsys@useobject{currentmarker}{}%
\end{pgfscope}%
\begin{pgfscope}%
\pgfsys@transformshift{5.160550in}{2.754073in}%
\pgfsys@useobject{currentmarker}{}%
\end{pgfscope}%
\end{pgfscope}%
\begin{pgfscope}%
\pgfsetrectcap%
\pgfsetmiterjoin%
\pgfsetlinewidth{0.803000pt}%
\definecolor{currentstroke}{rgb}{0.000000,0.000000,0.000000}%
\pgfsetstrokecolor{currentstroke}%
\pgfsetdash{}{0pt}%
\pgfpathmoveto{\pgfqpoint{0.721913in}{0.549073in}}%
\pgfpathlineto{\pgfqpoint{0.721913in}{2.859073in}}%
\pgfusepath{stroke}%
\end{pgfscope}%
\begin{pgfscope}%
\pgfsetrectcap%
\pgfsetmiterjoin%
\pgfsetlinewidth{0.803000pt}%
\definecolor{currentstroke}{rgb}{0.000000,0.000000,0.000000}%
\pgfsetstrokecolor{currentstroke}%
\pgfsetdash{}{0pt}%
\pgfpathmoveto{\pgfqpoint{5.371913in}{0.549073in}}%
\pgfpathlineto{\pgfqpoint{5.371913in}{2.859073in}}%
\pgfusepath{stroke}%
\end{pgfscope}%
\begin{pgfscope}%
\pgfsetrectcap%
\pgfsetmiterjoin%
\pgfsetlinewidth{0.803000pt}%
\definecolor{currentstroke}{rgb}{0.000000,0.000000,0.000000}%
\pgfsetstrokecolor{currentstroke}%
\pgfsetdash{}{0pt}%
\pgfpathmoveto{\pgfqpoint{0.721913in}{0.549073in}}%
\pgfpathlineto{\pgfqpoint{5.371913in}{0.549073in}}%
\pgfusepath{stroke}%
\end{pgfscope}%
\begin{pgfscope}%
\pgfsetrectcap%
\pgfsetmiterjoin%
\pgfsetlinewidth{0.803000pt}%
\definecolor{currentstroke}{rgb}{0.000000,0.000000,0.000000}%
\pgfsetstrokecolor{currentstroke}%
\pgfsetdash{}{0pt}%
\pgfpathmoveto{\pgfqpoint{0.721913in}{2.859073in}}%
\pgfpathlineto{\pgfqpoint{5.371913in}{2.859073in}}%
\pgfusepath{stroke}%
\end{pgfscope}%
\begin{pgfscope}%
\pgfsetbuttcap%
\pgfsetmiterjoin%
\definecolor{currentfill}{rgb}{1.000000,1.000000,1.000000}%
\pgfsetfillcolor{currentfill}%
\pgfsetfillopacity{0.800000}%
\pgfsetlinewidth{1.003750pt}%
\definecolor{currentstroke}{rgb}{0.800000,0.800000,0.800000}%
\pgfsetstrokecolor{currentstroke}%
\pgfsetstrokeopacity{0.800000}%
\pgfsetdash{}{0pt}%
\pgfpathmoveto{\pgfqpoint{2.795580in}{0.632406in}}%
\pgfpathlineto{\pgfqpoint{5.255247in}{0.632406in}}%
\pgfpathquadraticcurveto{\pgfqpoint{5.288580in}{0.632406in}}{\pgfqpoint{5.288580in}{0.665739in}}%
\pgfpathlineto{\pgfqpoint{5.288580in}{1.899073in}}%
\pgfpathquadraticcurveto{\pgfqpoint{5.288580in}{1.932406in}}{\pgfqpoint{5.255247in}{1.932406in}}%
\pgfpathlineto{\pgfqpoint{2.795580in}{1.932406in}}%
\pgfpathquadraticcurveto{\pgfqpoint{2.762246in}{1.932406in}}{\pgfqpoint{2.762246in}{1.899073in}}%
\pgfpathlineto{\pgfqpoint{2.762246in}{0.665739in}}%
\pgfpathquadraticcurveto{\pgfqpoint{2.762246in}{0.632406in}}{\pgfqpoint{2.795580in}{0.632406in}}%
\pgfpathlineto{\pgfqpoint{2.795580in}{0.632406in}}%
\pgfpathclose%
\pgfusepath{stroke,fill}%
\end{pgfscope}%
\begin{pgfscope}%
\pgfsetrectcap%
\pgfsetroundjoin%
\pgfsetlinewidth{1.003750pt}%
\definecolor{currentstroke}{rgb}{0.001462,0.000466,0.013866}%
\pgfsetstrokecolor{currentstroke}%
\pgfsetdash{}{0pt}%
\pgfpathmoveto{\pgfqpoint{2.828913in}{1.799073in}}%
\pgfpathlineto{\pgfqpoint{2.995580in}{1.799073in}}%
\pgfpathlineto{\pgfqpoint{3.162246in}{1.799073in}}%
\pgfusepath{stroke}%
\end{pgfscope}%
\begin{pgfscope}%
\pgfsetbuttcap%
\pgfsetroundjoin%
\definecolor{currentfill}{rgb}{0.001462,0.000466,0.013866}%
\pgfsetfillcolor{currentfill}%
\pgfsetlinewidth{1.003750pt}%
\definecolor{currentstroke}{rgb}{0.001462,0.000466,0.013866}%
\pgfsetstrokecolor{currentstroke}%
\pgfsetdash{}{0pt}%
\pgfsys@defobject{currentmarker}{\pgfqpoint{-0.020833in}{-0.020833in}}{\pgfqpoint{0.020833in}{0.020833in}}{%
\pgfpathmoveto{\pgfqpoint{0.000000in}{-0.020833in}}%
\pgfpathcurveto{\pgfqpoint{0.005525in}{-0.020833in}}{\pgfqpoint{0.010825in}{-0.018638in}}{\pgfqpoint{0.014731in}{-0.014731in}}%
\pgfpathcurveto{\pgfqpoint{0.018638in}{-0.010825in}}{\pgfqpoint{0.020833in}{-0.005525in}}{\pgfqpoint{0.020833in}{0.000000in}}%
\pgfpathcurveto{\pgfqpoint{0.020833in}{0.005525in}}{\pgfqpoint{0.018638in}{0.010825in}}{\pgfqpoint{0.014731in}{0.014731in}}%
\pgfpathcurveto{\pgfqpoint{0.010825in}{0.018638in}}{\pgfqpoint{0.005525in}{0.020833in}}{\pgfqpoint{0.000000in}{0.020833in}}%
\pgfpathcurveto{\pgfqpoint{-0.005525in}{0.020833in}}{\pgfqpoint{-0.010825in}{0.018638in}}{\pgfqpoint{-0.014731in}{0.014731in}}%
\pgfpathcurveto{\pgfqpoint{-0.018638in}{0.010825in}}{\pgfqpoint{-0.020833in}{0.005525in}}{\pgfqpoint{-0.020833in}{0.000000in}}%
\pgfpathcurveto{\pgfqpoint{-0.020833in}{-0.005525in}}{\pgfqpoint{-0.018638in}{-0.010825in}}{\pgfqpoint{-0.014731in}{-0.014731in}}%
\pgfpathcurveto{\pgfqpoint{-0.010825in}{-0.018638in}}{\pgfqpoint{-0.005525in}{-0.020833in}}{\pgfqpoint{0.000000in}{-0.020833in}}%
\pgfpathlineto{\pgfqpoint{0.000000in}{-0.020833in}}%
\pgfpathclose%
\pgfusepath{stroke,fill}%
}%
\begin{pgfscope}%
\pgfsys@transformshift{2.995580in}{1.799073in}%
\pgfsys@useobject{currentmarker}{}%
\end{pgfscope}%
\end{pgfscope}%
\begin{pgfscope}%
\definecolor{textcolor}{rgb}{0.000000,0.000000,0.000000}%
\pgfsetstrokecolor{textcolor}%
\pgfsetfillcolor{textcolor}%
\pgftext[x=3.295580in,y=1.740739in,left,base]{\color{textcolor}{\rmfamily\fontsize{12.000000}{14.400000}\selectfont\catcode`\^=\active\def^{\ifmmode\sp\else\^{}\fi}\catcode`\%=\active\def%{\%}$n_{\Omega}=0$, $n_{\Psi}=80$ (DGC)}}%
\end{pgfscope}%
\begin{pgfscope}%
\pgfsetrectcap%
\pgfsetroundjoin%
\pgfsetlinewidth{1.003750pt}%
\definecolor{currentstroke}{rgb}{0.232077,0.059889,0.437695}%
\pgfsetstrokecolor{currentstroke}%
\pgfsetdash{}{0pt}%
\pgfpathmoveto{\pgfqpoint{2.828913in}{1.549073in}}%
\pgfpathlineto{\pgfqpoint{2.995580in}{1.549073in}}%
\pgfpathlineto{\pgfqpoint{3.162246in}{1.549073in}}%
\pgfusepath{stroke}%
\end{pgfscope}%
\begin{pgfscope}%
\pgfsetbuttcap%
\pgfsetroundjoin%
\definecolor{currentfill}{rgb}{0.232077,0.059889,0.437695}%
\pgfsetfillcolor{currentfill}%
\pgfsetlinewidth{1.003750pt}%
\definecolor{currentstroke}{rgb}{0.232077,0.059889,0.437695}%
\pgfsetstrokecolor{currentstroke}%
\pgfsetdash{}{0pt}%
\pgfsys@defobject{currentmarker}{\pgfqpoint{-0.020833in}{-0.020833in}}{\pgfqpoint{0.020833in}{0.020833in}}{%
\pgfpathmoveto{\pgfqpoint{0.000000in}{-0.020833in}}%
\pgfpathcurveto{\pgfqpoint{0.005525in}{-0.020833in}}{\pgfqpoint{0.010825in}{-0.018638in}}{\pgfqpoint{0.014731in}{-0.014731in}}%
\pgfpathcurveto{\pgfqpoint{0.018638in}{-0.010825in}}{\pgfqpoint{0.020833in}{-0.005525in}}{\pgfqpoint{0.020833in}{0.000000in}}%
\pgfpathcurveto{\pgfqpoint{0.020833in}{0.005525in}}{\pgfqpoint{0.018638in}{0.010825in}}{\pgfqpoint{0.014731in}{0.014731in}}%
\pgfpathcurveto{\pgfqpoint{0.010825in}{0.018638in}}{\pgfqpoint{0.005525in}{0.020833in}}{\pgfqpoint{0.000000in}{0.020833in}}%
\pgfpathcurveto{\pgfqpoint{-0.005525in}{0.020833in}}{\pgfqpoint{-0.010825in}{0.018638in}}{\pgfqpoint{-0.014731in}{0.014731in}}%
\pgfpathcurveto{\pgfqpoint{-0.018638in}{0.010825in}}{\pgfqpoint{-0.020833in}{0.005525in}}{\pgfqpoint{-0.020833in}{0.000000in}}%
\pgfpathcurveto{\pgfqpoint{-0.020833in}{-0.005525in}}{\pgfqpoint{-0.018638in}{-0.010825in}}{\pgfqpoint{-0.014731in}{-0.014731in}}%
\pgfpathcurveto{\pgfqpoint{-0.010825in}{-0.018638in}}{\pgfqpoint{-0.005525in}{-0.020833in}}{\pgfqpoint{0.000000in}{-0.020833in}}%
\pgfpathlineto{\pgfqpoint{0.000000in}{-0.020833in}}%
\pgfpathclose%
\pgfusepath{stroke,fill}%
}%
\begin{pgfscope}%
\pgfsys@transformshift{2.995580in}{1.549073in}%
\pgfsys@useobject{currentmarker}{}%
\end{pgfscope}%
\end{pgfscope}%
\begin{pgfscope}%
\definecolor{textcolor}{rgb}{0.000000,0.000000,0.000000}%
\pgfsetstrokecolor{textcolor}%
\pgfsetfillcolor{textcolor}%
\pgftext[x=3.295580in,y=1.490739in,left,base]{\color{textcolor}{\rmfamily\fontsize{12.000000}{14.400000}\selectfont\catcode`\^=\active\def^{\ifmmode\sp\else\^{}\fi}\catcode`\%=\active\def%{\%}$n_{\Omega}=20$, $n_{\Psi}=60$ (NC++)}}%
\end{pgfscope}%
\begin{pgfscope}%
\pgfsetrectcap%
\pgfsetroundjoin%
\pgfsetlinewidth{1.003750pt}%
\definecolor{currentstroke}{rgb}{0.550287,0.161158,0.505719}%
\pgfsetstrokecolor{currentstroke}%
\pgfsetdash{}{0pt}%
\pgfpathmoveto{\pgfqpoint{2.828913in}{1.299073in}}%
\pgfpathlineto{\pgfqpoint{2.995580in}{1.299073in}}%
\pgfpathlineto{\pgfqpoint{3.162246in}{1.299073in}}%
\pgfusepath{stroke}%
\end{pgfscope}%
\begin{pgfscope}%
\pgfsetbuttcap%
\pgfsetroundjoin%
\definecolor{currentfill}{rgb}{0.550287,0.161158,0.505719}%
\pgfsetfillcolor{currentfill}%
\pgfsetlinewidth{1.003750pt}%
\definecolor{currentstroke}{rgb}{0.550287,0.161158,0.505719}%
\pgfsetstrokecolor{currentstroke}%
\pgfsetdash{}{0pt}%
\pgfsys@defobject{currentmarker}{\pgfqpoint{-0.020833in}{-0.020833in}}{\pgfqpoint{0.020833in}{0.020833in}}{%
\pgfpathmoveto{\pgfqpoint{0.000000in}{-0.020833in}}%
\pgfpathcurveto{\pgfqpoint{0.005525in}{-0.020833in}}{\pgfqpoint{0.010825in}{-0.018638in}}{\pgfqpoint{0.014731in}{-0.014731in}}%
\pgfpathcurveto{\pgfqpoint{0.018638in}{-0.010825in}}{\pgfqpoint{0.020833in}{-0.005525in}}{\pgfqpoint{0.020833in}{0.000000in}}%
\pgfpathcurveto{\pgfqpoint{0.020833in}{0.005525in}}{\pgfqpoint{0.018638in}{0.010825in}}{\pgfqpoint{0.014731in}{0.014731in}}%
\pgfpathcurveto{\pgfqpoint{0.010825in}{0.018638in}}{\pgfqpoint{0.005525in}{0.020833in}}{\pgfqpoint{0.000000in}{0.020833in}}%
\pgfpathcurveto{\pgfqpoint{-0.005525in}{0.020833in}}{\pgfqpoint{-0.010825in}{0.018638in}}{\pgfqpoint{-0.014731in}{0.014731in}}%
\pgfpathcurveto{\pgfqpoint{-0.018638in}{0.010825in}}{\pgfqpoint{-0.020833in}{0.005525in}}{\pgfqpoint{-0.020833in}{0.000000in}}%
\pgfpathcurveto{\pgfqpoint{-0.020833in}{-0.005525in}}{\pgfqpoint{-0.018638in}{-0.010825in}}{\pgfqpoint{-0.014731in}{-0.014731in}}%
\pgfpathcurveto{\pgfqpoint{-0.010825in}{-0.018638in}}{\pgfqpoint{-0.005525in}{-0.020833in}}{\pgfqpoint{0.000000in}{-0.020833in}}%
\pgfpathlineto{\pgfqpoint{0.000000in}{-0.020833in}}%
\pgfpathclose%
\pgfusepath{stroke,fill}%
}%
\begin{pgfscope}%
\pgfsys@transformshift{2.995580in}{1.299073in}%
\pgfsys@useobject{currentmarker}{}%
\end{pgfscope}%
\end{pgfscope}%
\begin{pgfscope}%
\definecolor{textcolor}{rgb}{0.000000,0.000000,0.000000}%
\pgfsetstrokecolor{textcolor}%
\pgfsetfillcolor{textcolor}%
\pgftext[x=3.295580in,y=1.240739in,left,base]{\color{textcolor}{\rmfamily\fontsize{12.000000}{14.400000}\selectfont\catcode`\^=\active\def^{\ifmmode\sp\else\^{}\fi}\catcode`\%=\active\def%{\%}$n_{\Omega}=40$, $n_{\Psi}=40$ (NC++)}}%
\end{pgfscope}%
\begin{pgfscope}%
\pgfsetrectcap%
\pgfsetroundjoin%
\pgfsetlinewidth{1.003750pt}%
\definecolor{currentstroke}{rgb}{0.868793,0.287728,0.409303}%
\pgfsetstrokecolor{currentstroke}%
\pgfsetdash{}{0pt}%
\pgfpathmoveto{\pgfqpoint{2.828913in}{1.049073in}}%
\pgfpathlineto{\pgfqpoint{2.995580in}{1.049073in}}%
\pgfpathlineto{\pgfqpoint{3.162246in}{1.049073in}}%
\pgfusepath{stroke}%
\end{pgfscope}%
\begin{pgfscope}%
\pgfsetbuttcap%
\pgfsetroundjoin%
\definecolor{currentfill}{rgb}{0.868793,0.287728,0.409303}%
\pgfsetfillcolor{currentfill}%
\pgfsetlinewidth{1.003750pt}%
\definecolor{currentstroke}{rgb}{0.868793,0.287728,0.409303}%
\pgfsetstrokecolor{currentstroke}%
\pgfsetdash{}{0pt}%
\pgfsys@defobject{currentmarker}{\pgfqpoint{-0.020833in}{-0.020833in}}{\pgfqpoint{0.020833in}{0.020833in}}{%
\pgfpathmoveto{\pgfqpoint{0.000000in}{-0.020833in}}%
\pgfpathcurveto{\pgfqpoint{0.005525in}{-0.020833in}}{\pgfqpoint{0.010825in}{-0.018638in}}{\pgfqpoint{0.014731in}{-0.014731in}}%
\pgfpathcurveto{\pgfqpoint{0.018638in}{-0.010825in}}{\pgfqpoint{0.020833in}{-0.005525in}}{\pgfqpoint{0.020833in}{0.000000in}}%
\pgfpathcurveto{\pgfqpoint{0.020833in}{0.005525in}}{\pgfqpoint{0.018638in}{0.010825in}}{\pgfqpoint{0.014731in}{0.014731in}}%
\pgfpathcurveto{\pgfqpoint{0.010825in}{0.018638in}}{\pgfqpoint{0.005525in}{0.020833in}}{\pgfqpoint{0.000000in}{0.020833in}}%
\pgfpathcurveto{\pgfqpoint{-0.005525in}{0.020833in}}{\pgfqpoint{-0.010825in}{0.018638in}}{\pgfqpoint{-0.014731in}{0.014731in}}%
\pgfpathcurveto{\pgfqpoint{-0.018638in}{0.010825in}}{\pgfqpoint{-0.020833in}{0.005525in}}{\pgfqpoint{-0.020833in}{0.000000in}}%
\pgfpathcurveto{\pgfqpoint{-0.020833in}{-0.005525in}}{\pgfqpoint{-0.018638in}{-0.010825in}}{\pgfqpoint{-0.014731in}{-0.014731in}}%
\pgfpathcurveto{\pgfqpoint{-0.010825in}{-0.018638in}}{\pgfqpoint{-0.005525in}{-0.020833in}}{\pgfqpoint{0.000000in}{-0.020833in}}%
\pgfpathlineto{\pgfqpoint{0.000000in}{-0.020833in}}%
\pgfpathclose%
\pgfusepath{stroke,fill}%
}%
\begin{pgfscope}%
\pgfsys@transformshift{2.995580in}{1.049073in}%
\pgfsys@useobject{currentmarker}{}%
\end{pgfscope}%
\end{pgfscope}%
\begin{pgfscope}%
\definecolor{textcolor}{rgb}{0.000000,0.000000,0.000000}%
\pgfsetstrokecolor{textcolor}%
\pgfsetfillcolor{textcolor}%
\pgftext[x=3.295580in,y=0.990739in,left,base]{\color{textcolor}{\rmfamily\fontsize{12.000000}{14.400000}\selectfont\catcode`\^=\active\def^{\ifmmode\sp\else\^{}\fi}\catcode`\%=\active\def%{\%}$n_{\Omega}=60$, $n_{\Psi}=20$ (NC++)}}%
\end{pgfscope}%
\begin{pgfscope}%
\pgfsetrectcap%
\pgfsetroundjoin%
\pgfsetlinewidth{1.003750pt}%
\definecolor{currentstroke}{rgb}{0.994738,0.624350,0.427397}%
\pgfsetstrokecolor{currentstroke}%
\pgfsetdash{}{0pt}%
\pgfpathmoveto{\pgfqpoint{2.828913in}{0.799073in}}%
\pgfpathlineto{\pgfqpoint{2.995580in}{0.799073in}}%
\pgfpathlineto{\pgfqpoint{3.162246in}{0.799073in}}%
\pgfusepath{stroke}%
\end{pgfscope}%
\begin{pgfscope}%
\pgfsetbuttcap%
\pgfsetroundjoin%
\definecolor{currentfill}{rgb}{0.994738,0.624350,0.427397}%
\pgfsetfillcolor{currentfill}%
\pgfsetlinewidth{1.003750pt}%
\definecolor{currentstroke}{rgb}{0.994738,0.624350,0.427397}%
\pgfsetstrokecolor{currentstroke}%
\pgfsetdash{}{0pt}%
\pgfsys@defobject{currentmarker}{\pgfqpoint{-0.020833in}{-0.020833in}}{\pgfqpoint{0.020833in}{0.020833in}}{%
\pgfpathmoveto{\pgfqpoint{0.000000in}{-0.020833in}}%
\pgfpathcurveto{\pgfqpoint{0.005525in}{-0.020833in}}{\pgfqpoint{0.010825in}{-0.018638in}}{\pgfqpoint{0.014731in}{-0.014731in}}%
\pgfpathcurveto{\pgfqpoint{0.018638in}{-0.010825in}}{\pgfqpoint{0.020833in}{-0.005525in}}{\pgfqpoint{0.020833in}{0.000000in}}%
\pgfpathcurveto{\pgfqpoint{0.020833in}{0.005525in}}{\pgfqpoint{0.018638in}{0.010825in}}{\pgfqpoint{0.014731in}{0.014731in}}%
\pgfpathcurveto{\pgfqpoint{0.010825in}{0.018638in}}{\pgfqpoint{0.005525in}{0.020833in}}{\pgfqpoint{0.000000in}{0.020833in}}%
\pgfpathcurveto{\pgfqpoint{-0.005525in}{0.020833in}}{\pgfqpoint{-0.010825in}{0.018638in}}{\pgfqpoint{-0.014731in}{0.014731in}}%
\pgfpathcurveto{\pgfqpoint{-0.018638in}{0.010825in}}{\pgfqpoint{-0.020833in}{0.005525in}}{\pgfqpoint{-0.020833in}{0.000000in}}%
\pgfpathcurveto{\pgfqpoint{-0.020833in}{-0.005525in}}{\pgfqpoint{-0.018638in}{-0.010825in}}{\pgfqpoint{-0.014731in}{-0.014731in}}%
\pgfpathcurveto{\pgfqpoint{-0.010825in}{-0.018638in}}{\pgfqpoint{-0.005525in}{-0.020833in}}{\pgfqpoint{0.000000in}{-0.020833in}}%
\pgfpathlineto{\pgfqpoint{0.000000in}{-0.020833in}}%
\pgfpathclose%
\pgfusepath{stroke,fill}%
}%
\begin{pgfscope}%
\pgfsys@transformshift{2.995580in}{0.799073in}%
\pgfsys@useobject{currentmarker}{}%
\end{pgfscope}%
\end{pgfscope}%
\begin{pgfscope}%
\definecolor{textcolor}{rgb}{0.000000,0.000000,0.000000}%
\pgfsetstrokecolor{textcolor}%
\pgfsetfillcolor{textcolor}%
\pgftext[x=3.295580in,y=0.740739in,left,base]{\color{textcolor}{\rmfamily\fontsize{12.000000}{14.400000}\selectfont\catcode`\^=\active\def^{\ifmmode\sp\else\^{}\fi}\catcode`\%=\active\def%{\%}$n_{\Omega}=80$, $n_{\Psi}=0$ (NC)}}%
\end{pgfscope}%
\end{pgfpicture}%
\makeatother%
\endgroup%

    \caption{Allocation of mat-vecs to low-rank and trace approximation.}
    \label{fig:5-experiments-electronic-structure-matvec-mixture}
\end{figure}

%%%%%%%%%%%%%%%%%%%%%%%%%%%%%%%%%%%%%%%%%%%%%%%%%%%%%%%%%%%%%%%%%%%%%%%%%%%%%%%%

\section{Benchmark against Haydock's method}
\label{sec:5-experiments-haydock-method}

Haydock's method \cite{haydock1972electronic,lin2016review} is a specialized technique for approximating \gls{smooth-spectral-density}
in the case where a Lorentzian smoothing kernel
\begin{equation}
    g_{\sigma}(t) = \frac{1}{\pi} \frac{\sigma}{t^2 + \sigma^2} = -\frac{1}{\pi} \Im\left\{ \frac{1}{t + i\sigma} \right\}
    \label{equ:5-experiments-cauchy-kernel}
\end{equation}
is used.
%Estimating \gls{smooth-spectral-density} then becomes the
%trace estimation problem
%\begin{equation}
%    \phi_{\sigma}(t) = \Tr(g_{\sigma}(t\mtx{I} - \mtx{A})) = - \frac{1}{n \pi} \Im \left\{ \Tr\left[((t + i\sigma)I - A)^{-1}\right]  \right\}.
%    \label{equ:5-experiments-haydock-trace}
%\end{equation}
%Similarly to the \gls{DGC} method (see \refsec{sec:2-chebyshev-delta-gauss-chebyshev}),
%the Hutchinson's trace estimator with standard Gaussian random vectors
%$\vct{\psi} \in \mathbb{R}^n$ is used to
%approximate the trace
%\begin{equation}
%    \Tr\left[((t + i\sigma)I - A)^{-1}\right] \approx \frac{1}{n_{\Omega}} \sum_{j=1}^{n_{\Omega}} \left( \vct{\psi}_j \right)^{\top} ((t - i\sigma)\mtx{I} - \mtx{A})^{-1} \vct{\psi}_j.
%    \label{equ:5-experiments-haydock-hutchinson}
%\end{equation}
%It turns out that each summand in \refequ{5-experiments-haydock-hutchinson} can
%be efficiently evaluated for multiple $t$ by running Lanczos
%on $\mtx{A}$ with starting vector $\vct{\psi}_j$
%\begin{equation}
%    \vct{\psi}_j^{\top} ((t - i\sigma)\mtx{I} - \mtx{A})^{-1} \vct{\psi}_j% &\approx \vct{e}_1^{\top} ((t - i\sigma)\mtx{I} - \mtx{H}_k)^{-1} \vct{e}_1 \notag \\
%    \approx \cfrac{1}{(t - i\sigma) - \alpha_1 - \cfrac{\beta_2^2}{(t - i\sigma) - \alpha_2 - \dots}}
%    \label{equ:5-experiments-haydock-recursion}
%\end{equation}

We repeat the exact same experiments from \refsec{sec:5-experiments-density-function},
plot the results in \reffig{fig:5-experiments-haydock-convergence-nv} and
\reffig{fig:5-experiments-haydock-convergence-m}, and compare the wall time
between the methods in \reftab{tab:5-experiments-timing-haydock}.

\begin{figure}[ht]
    \begin{subfigure}[b]{0.49\columnwidth}
        %% Creator: Matplotlib, PGF backend
%%
%% To include the figure in your LaTeX document, write
%%   \input{<filename>.pgf}
%%
%% Make sure the required packages are loaded in your preamble
%%   \usepackage{pgf}
%%
%% Also ensure that all the required font packages are loaded; for instance,
%% the lmodern package is sometimes necessary when using math font.
%%   \usepackage{lmodern}
%%
%% Figures using additional raster images can only be included by \input if
%% they are in the same directory as the main LaTeX file. For loading figures
%% from other directories you can use the `import` package
%%   \usepackage{import}
%%
%% and then include the figures with
%%   \import{<path to file>}{<filename>.pgf}
%%
%% Matplotlib used the following preamble
%%   \def\mathdefault#1{#1}
%%   \everymath=\expandafter{\the\everymath\displaystyle}
%%   
%%   \makeatletter\@ifpackageloaded{underscore}{}{\usepackage[strings]{underscore}}\makeatother
%%
\begingroup%
\makeatletter%
\begin{pgfpicture}%
\pgfpathrectangle{\pgfpointorigin}{\pgfqpoint{2.759413in}{2.574073in}}%
\pgfusepath{use as bounding box, clip}%
\begin{pgfscope}%
\pgfsetbuttcap%
\pgfsetmiterjoin%
\definecolor{currentfill}{rgb}{1.000000,1.000000,1.000000}%
\pgfsetfillcolor{currentfill}%
\pgfsetlinewidth{0.000000pt}%
\definecolor{currentstroke}{rgb}{1.000000,1.000000,1.000000}%
\pgfsetstrokecolor{currentstroke}%
\pgfsetdash{}{0pt}%
\pgfpathmoveto{\pgfqpoint{0.000000in}{0.000000in}}%
\pgfpathlineto{\pgfqpoint{2.759413in}{0.000000in}}%
\pgfpathlineto{\pgfqpoint{2.759413in}{2.574073in}}%
\pgfpathlineto{\pgfqpoint{0.000000in}{2.574073in}}%
\pgfpathlineto{\pgfqpoint{0.000000in}{0.000000in}}%
\pgfpathclose%
\pgfusepath{fill}%
\end{pgfscope}%
\begin{pgfscope}%
\pgfsetbuttcap%
\pgfsetmiterjoin%
\definecolor{currentfill}{rgb}{1.000000,1.000000,1.000000}%
\pgfsetfillcolor{currentfill}%
\pgfsetlinewidth{0.000000pt}%
\definecolor{currentstroke}{rgb}{0.000000,0.000000,0.000000}%
\pgfsetstrokecolor{currentstroke}%
\pgfsetstrokeopacity{0.000000}%
\pgfsetdash{}{0pt}%
\pgfpathmoveto{\pgfqpoint{0.721913in}{0.549073in}}%
\pgfpathlineto{\pgfqpoint{2.659413in}{0.549073in}}%
\pgfpathlineto{\pgfqpoint{2.659413in}{2.474073in}}%
\pgfpathlineto{\pgfqpoint{0.721913in}{2.474073in}}%
\pgfpathlineto{\pgfqpoint{0.721913in}{0.549073in}}%
\pgfpathclose%
\pgfusepath{fill}%
\end{pgfscope}%
\begin{pgfscope}%
\pgfsetbuttcap%
\pgfsetroundjoin%
\definecolor{currentfill}{rgb}{0.000000,0.000000,0.000000}%
\pgfsetfillcolor{currentfill}%
\pgfsetlinewidth{0.803000pt}%
\definecolor{currentstroke}{rgb}{0.000000,0.000000,0.000000}%
\pgfsetstrokecolor{currentstroke}%
\pgfsetdash{}{0pt}%
\pgfsys@defobject{currentmarker}{\pgfqpoint{0.000000in}{-0.048611in}}{\pgfqpoint{0.000000in}{0.000000in}}{%
\pgfpathmoveto{\pgfqpoint{0.000000in}{0.000000in}}%
\pgfpathlineto{\pgfqpoint{0.000000in}{-0.048611in}}%
\pgfusepath{stroke,fill}%
}%
\begin{pgfscope}%
\pgfsys@transformshift{0.910580in}{0.549073in}%
\pgfsys@useobject{currentmarker}{}%
\end{pgfscope}%
\end{pgfscope}%
\begin{pgfscope}%
\definecolor{textcolor}{rgb}{0.000000,0.000000,0.000000}%
\pgfsetstrokecolor{textcolor}%
\pgfsetfillcolor{textcolor}%
\pgftext[x=0.910580in,y=0.451851in,,top]{\color{textcolor}{\rmfamily\fontsize{12.000000}{14.400000}\selectfont\catcode`\^=\active\def^{\ifmmode\sp\else\^{}\fi}\catcode`\%=\active\def%{\%}$\mathdefault{10^{1}}$}}%
\end{pgfscope}%
\begin{pgfscope}%
\pgfsetbuttcap%
\pgfsetroundjoin%
\definecolor{currentfill}{rgb}{0.000000,0.000000,0.000000}%
\pgfsetfillcolor{currentfill}%
\pgfsetlinewidth{0.803000pt}%
\definecolor{currentstroke}{rgb}{0.000000,0.000000,0.000000}%
\pgfsetstrokecolor{currentstroke}%
\pgfsetdash{}{0pt}%
\pgfsys@defobject{currentmarker}{\pgfqpoint{0.000000in}{-0.048611in}}{\pgfqpoint{0.000000in}{0.000000in}}{%
\pgfpathmoveto{\pgfqpoint{0.000000in}{0.000000in}}%
\pgfpathlineto{\pgfqpoint{0.000000in}{-0.048611in}}%
\pgfusepath{stroke,fill}%
}%
\begin{pgfscope}%
\pgfsys@transformshift{1.948634in}{0.549073in}%
\pgfsys@useobject{currentmarker}{}%
\end{pgfscope}%
\end{pgfscope}%
\begin{pgfscope}%
\definecolor{textcolor}{rgb}{0.000000,0.000000,0.000000}%
\pgfsetstrokecolor{textcolor}%
\pgfsetfillcolor{textcolor}%
\pgftext[x=1.948634in,y=0.451851in,,top]{\color{textcolor}{\rmfamily\fontsize{12.000000}{14.400000}\selectfont\catcode`\^=\active\def^{\ifmmode\sp\else\^{}\fi}\catcode`\%=\active\def%{\%}$\mathdefault{10^{2}}$}}%
\end{pgfscope}%
\begin{pgfscope}%
\pgfsetbuttcap%
\pgfsetroundjoin%
\definecolor{currentfill}{rgb}{0.000000,0.000000,0.000000}%
\pgfsetfillcolor{currentfill}%
\pgfsetlinewidth{0.602250pt}%
\definecolor{currentstroke}{rgb}{0.000000,0.000000,0.000000}%
\pgfsetstrokecolor{currentstroke}%
\pgfsetdash{}{0pt}%
\pgfsys@defobject{currentmarker}{\pgfqpoint{0.000000in}{-0.027778in}}{\pgfqpoint{0.000000in}{0.000000in}}{%
\pgfpathmoveto{\pgfqpoint{0.000000in}{0.000000in}}%
\pgfpathlineto{\pgfqpoint{0.000000in}{-0.027778in}}%
\pgfusepath{stroke,fill}%
}%
\begin{pgfscope}%
\pgfsys@transformshift{0.749783in}{0.549073in}%
\pgfsys@useobject{currentmarker}{}%
\end{pgfscope}%
\end{pgfscope}%
\begin{pgfscope}%
\pgfsetbuttcap%
\pgfsetroundjoin%
\definecolor{currentfill}{rgb}{0.000000,0.000000,0.000000}%
\pgfsetfillcolor{currentfill}%
\pgfsetlinewidth{0.602250pt}%
\definecolor{currentstroke}{rgb}{0.000000,0.000000,0.000000}%
\pgfsetstrokecolor{currentstroke}%
\pgfsetdash{}{0pt}%
\pgfsys@defobject{currentmarker}{\pgfqpoint{0.000000in}{-0.027778in}}{\pgfqpoint{0.000000in}{0.000000in}}{%
\pgfpathmoveto{\pgfqpoint{0.000000in}{0.000000in}}%
\pgfpathlineto{\pgfqpoint{0.000000in}{-0.027778in}}%
\pgfusepath{stroke,fill}%
}%
\begin{pgfscope}%
\pgfsys@transformshift{0.809982in}{0.549073in}%
\pgfsys@useobject{currentmarker}{}%
\end{pgfscope}%
\end{pgfscope}%
\begin{pgfscope}%
\pgfsetbuttcap%
\pgfsetroundjoin%
\definecolor{currentfill}{rgb}{0.000000,0.000000,0.000000}%
\pgfsetfillcolor{currentfill}%
\pgfsetlinewidth{0.602250pt}%
\definecolor{currentstroke}{rgb}{0.000000,0.000000,0.000000}%
\pgfsetstrokecolor{currentstroke}%
\pgfsetdash{}{0pt}%
\pgfsys@defobject{currentmarker}{\pgfqpoint{0.000000in}{-0.027778in}}{\pgfqpoint{0.000000in}{0.000000in}}{%
\pgfpathmoveto{\pgfqpoint{0.000000in}{0.000000in}}%
\pgfpathlineto{\pgfqpoint{0.000000in}{-0.027778in}}%
\pgfusepath{stroke,fill}%
}%
\begin{pgfscope}%
\pgfsys@transformshift{0.863081in}{0.549073in}%
\pgfsys@useobject{currentmarker}{}%
\end{pgfscope}%
\end{pgfscope}%
\begin{pgfscope}%
\pgfsetbuttcap%
\pgfsetroundjoin%
\definecolor{currentfill}{rgb}{0.000000,0.000000,0.000000}%
\pgfsetfillcolor{currentfill}%
\pgfsetlinewidth{0.602250pt}%
\definecolor{currentstroke}{rgb}{0.000000,0.000000,0.000000}%
\pgfsetstrokecolor{currentstroke}%
\pgfsetdash{}{0pt}%
\pgfsys@defobject{currentmarker}{\pgfqpoint{0.000000in}{-0.027778in}}{\pgfqpoint{0.000000in}{0.000000in}}{%
\pgfpathmoveto{\pgfqpoint{0.000000in}{0.000000in}}%
\pgfpathlineto{\pgfqpoint{0.000000in}{-0.027778in}}%
\pgfusepath{stroke,fill}%
}%
\begin{pgfscope}%
\pgfsys@transformshift{1.223065in}{0.549073in}%
\pgfsys@useobject{currentmarker}{}%
\end{pgfscope}%
\end{pgfscope}%
\begin{pgfscope}%
\pgfsetbuttcap%
\pgfsetroundjoin%
\definecolor{currentfill}{rgb}{0.000000,0.000000,0.000000}%
\pgfsetfillcolor{currentfill}%
\pgfsetlinewidth{0.602250pt}%
\definecolor{currentstroke}{rgb}{0.000000,0.000000,0.000000}%
\pgfsetstrokecolor{currentstroke}%
\pgfsetdash{}{0pt}%
\pgfsys@defobject{currentmarker}{\pgfqpoint{0.000000in}{-0.027778in}}{\pgfqpoint{0.000000in}{0.000000in}}{%
\pgfpathmoveto{\pgfqpoint{0.000000in}{0.000000in}}%
\pgfpathlineto{\pgfqpoint{0.000000in}{-0.027778in}}%
\pgfusepath{stroke,fill}%
}%
\begin{pgfscope}%
\pgfsys@transformshift{1.405857in}{0.549073in}%
\pgfsys@useobject{currentmarker}{}%
\end{pgfscope}%
\end{pgfscope}%
\begin{pgfscope}%
\pgfsetbuttcap%
\pgfsetroundjoin%
\definecolor{currentfill}{rgb}{0.000000,0.000000,0.000000}%
\pgfsetfillcolor{currentfill}%
\pgfsetlinewidth{0.602250pt}%
\definecolor{currentstroke}{rgb}{0.000000,0.000000,0.000000}%
\pgfsetstrokecolor{currentstroke}%
\pgfsetdash{}{0pt}%
\pgfsys@defobject{currentmarker}{\pgfqpoint{0.000000in}{-0.027778in}}{\pgfqpoint{0.000000in}{0.000000in}}{%
\pgfpathmoveto{\pgfqpoint{0.000000in}{0.000000in}}%
\pgfpathlineto{\pgfqpoint{0.000000in}{-0.027778in}}%
\pgfusepath{stroke,fill}%
}%
\begin{pgfscope}%
\pgfsys@transformshift{1.535551in}{0.549073in}%
\pgfsys@useobject{currentmarker}{}%
\end{pgfscope}%
\end{pgfscope}%
\begin{pgfscope}%
\pgfsetbuttcap%
\pgfsetroundjoin%
\definecolor{currentfill}{rgb}{0.000000,0.000000,0.000000}%
\pgfsetfillcolor{currentfill}%
\pgfsetlinewidth{0.602250pt}%
\definecolor{currentstroke}{rgb}{0.000000,0.000000,0.000000}%
\pgfsetstrokecolor{currentstroke}%
\pgfsetdash{}{0pt}%
\pgfsys@defobject{currentmarker}{\pgfqpoint{0.000000in}{-0.027778in}}{\pgfqpoint{0.000000in}{0.000000in}}{%
\pgfpathmoveto{\pgfqpoint{0.000000in}{0.000000in}}%
\pgfpathlineto{\pgfqpoint{0.000000in}{-0.027778in}}%
\pgfusepath{stroke,fill}%
}%
\begin{pgfscope}%
\pgfsys@transformshift{1.636148in}{0.549073in}%
\pgfsys@useobject{currentmarker}{}%
\end{pgfscope}%
\end{pgfscope}%
\begin{pgfscope}%
\pgfsetbuttcap%
\pgfsetroundjoin%
\definecolor{currentfill}{rgb}{0.000000,0.000000,0.000000}%
\pgfsetfillcolor{currentfill}%
\pgfsetlinewidth{0.602250pt}%
\definecolor{currentstroke}{rgb}{0.000000,0.000000,0.000000}%
\pgfsetstrokecolor{currentstroke}%
\pgfsetdash{}{0pt}%
\pgfsys@defobject{currentmarker}{\pgfqpoint{0.000000in}{-0.027778in}}{\pgfqpoint{0.000000in}{0.000000in}}{%
\pgfpathmoveto{\pgfqpoint{0.000000in}{0.000000in}}%
\pgfpathlineto{\pgfqpoint{0.000000in}{-0.027778in}}%
\pgfusepath{stroke,fill}%
}%
\begin{pgfscope}%
\pgfsys@transformshift{1.718343in}{0.549073in}%
\pgfsys@useobject{currentmarker}{}%
\end{pgfscope}%
\end{pgfscope}%
\begin{pgfscope}%
\pgfsetbuttcap%
\pgfsetroundjoin%
\definecolor{currentfill}{rgb}{0.000000,0.000000,0.000000}%
\pgfsetfillcolor{currentfill}%
\pgfsetlinewidth{0.602250pt}%
\definecolor{currentstroke}{rgb}{0.000000,0.000000,0.000000}%
\pgfsetstrokecolor{currentstroke}%
\pgfsetdash{}{0pt}%
\pgfsys@defobject{currentmarker}{\pgfqpoint{0.000000in}{-0.027778in}}{\pgfqpoint{0.000000in}{0.000000in}}{%
\pgfpathmoveto{\pgfqpoint{0.000000in}{0.000000in}}%
\pgfpathlineto{\pgfqpoint{0.000000in}{-0.027778in}}%
\pgfusepath{stroke,fill}%
}%
\begin{pgfscope}%
\pgfsys@transformshift{1.787837in}{0.549073in}%
\pgfsys@useobject{currentmarker}{}%
\end{pgfscope}%
\end{pgfscope}%
\begin{pgfscope}%
\pgfsetbuttcap%
\pgfsetroundjoin%
\definecolor{currentfill}{rgb}{0.000000,0.000000,0.000000}%
\pgfsetfillcolor{currentfill}%
\pgfsetlinewidth{0.602250pt}%
\definecolor{currentstroke}{rgb}{0.000000,0.000000,0.000000}%
\pgfsetstrokecolor{currentstroke}%
\pgfsetdash{}{0pt}%
\pgfsys@defobject{currentmarker}{\pgfqpoint{0.000000in}{-0.027778in}}{\pgfqpoint{0.000000in}{0.000000in}}{%
\pgfpathmoveto{\pgfqpoint{0.000000in}{0.000000in}}%
\pgfpathlineto{\pgfqpoint{0.000000in}{-0.027778in}}%
\pgfusepath{stroke,fill}%
}%
\begin{pgfscope}%
\pgfsys@transformshift{1.848036in}{0.549073in}%
\pgfsys@useobject{currentmarker}{}%
\end{pgfscope}%
\end{pgfscope}%
\begin{pgfscope}%
\pgfsetbuttcap%
\pgfsetroundjoin%
\definecolor{currentfill}{rgb}{0.000000,0.000000,0.000000}%
\pgfsetfillcolor{currentfill}%
\pgfsetlinewidth{0.602250pt}%
\definecolor{currentstroke}{rgb}{0.000000,0.000000,0.000000}%
\pgfsetstrokecolor{currentstroke}%
\pgfsetdash{}{0pt}%
\pgfsys@defobject{currentmarker}{\pgfqpoint{0.000000in}{-0.027778in}}{\pgfqpoint{0.000000in}{0.000000in}}{%
\pgfpathmoveto{\pgfqpoint{0.000000in}{0.000000in}}%
\pgfpathlineto{\pgfqpoint{0.000000in}{-0.027778in}}%
\pgfusepath{stroke,fill}%
}%
\begin{pgfscope}%
\pgfsys@transformshift{1.901135in}{0.549073in}%
\pgfsys@useobject{currentmarker}{}%
\end{pgfscope}%
\end{pgfscope}%
\begin{pgfscope}%
\pgfsetbuttcap%
\pgfsetroundjoin%
\definecolor{currentfill}{rgb}{0.000000,0.000000,0.000000}%
\pgfsetfillcolor{currentfill}%
\pgfsetlinewidth{0.602250pt}%
\definecolor{currentstroke}{rgb}{0.000000,0.000000,0.000000}%
\pgfsetstrokecolor{currentstroke}%
\pgfsetdash{}{0pt}%
\pgfsys@defobject{currentmarker}{\pgfqpoint{0.000000in}{-0.027778in}}{\pgfqpoint{0.000000in}{0.000000in}}{%
\pgfpathmoveto{\pgfqpoint{0.000000in}{0.000000in}}%
\pgfpathlineto{\pgfqpoint{0.000000in}{-0.027778in}}%
\pgfusepath{stroke,fill}%
}%
\begin{pgfscope}%
\pgfsys@transformshift{2.261120in}{0.549073in}%
\pgfsys@useobject{currentmarker}{}%
\end{pgfscope}%
\end{pgfscope}%
\begin{pgfscope}%
\pgfsetbuttcap%
\pgfsetroundjoin%
\definecolor{currentfill}{rgb}{0.000000,0.000000,0.000000}%
\pgfsetfillcolor{currentfill}%
\pgfsetlinewidth{0.602250pt}%
\definecolor{currentstroke}{rgb}{0.000000,0.000000,0.000000}%
\pgfsetstrokecolor{currentstroke}%
\pgfsetdash{}{0pt}%
\pgfsys@defobject{currentmarker}{\pgfqpoint{0.000000in}{-0.027778in}}{\pgfqpoint{0.000000in}{0.000000in}}{%
\pgfpathmoveto{\pgfqpoint{0.000000in}{0.000000in}}%
\pgfpathlineto{\pgfqpoint{0.000000in}{-0.027778in}}%
\pgfusepath{stroke,fill}%
}%
\begin{pgfscope}%
\pgfsys@transformshift{2.443912in}{0.549073in}%
\pgfsys@useobject{currentmarker}{}%
\end{pgfscope}%
\end{pgfscope}%
\begin{pgfscope}%
\pgfsetbuttcap%
\pgfsetroundjoin%
\definecolor{currentfill}{rgb}{0.000000,0.000000,0.000000}%
\pgfsetfillcolor{currentfill}%
\pgfsetlinewidth{0.602250pt}%
\definecolor{currentstroke}{rgb}{0.000000,0.000000,0.000000}%
\pgfsetstrokecolor{currentstroke}%
\pgfsetdash{}{0pt}%
\pgfsys@defobject{currentmarker}{\pgfqpoint{0.000000in}{-0.027778in}}{\pgfqpoint{0.000000in}{0.000000in}}{%
\pgfpathmoveto{\pgfqpoint{0.000000in}{0.000000in}}%
\pgfpathlineto{\pgfqpoint{0.000000in}{-0.027778in}}%
\pgfusepath{stroke,fill}%
}%
\begin{pgfscope}%
\pgfsys@transformshift{2.573605in}{0.549073in}%
\pgfsys@useobject{currentmarker}{}%
\end{pgfscope}%
\end{pgfscope}%
\begin{pgfscope}%
\definecolor{textcolor}{rgb}{0.000000,0.000000,0.000000}%
\pgfsetstrokecolor{textcolor}%
\pgfsetfillcolor{textcolor}%
\pgftext[x=1.690663in,y=0.248148in,,top]{\color{textcolor}{\rmfamily\fontsize{12.000000}{14.400000}\selectfont\catcode`\^=\active\def^{\ifmmode\sp\else\^{}\fi}\catcode`\%=\active\def%{\%}$n_{\Omega} + n_{\Psi}$}}%
\end{pgfscope}%
\begin{pgfscope}%
\pgfsetbuttcap%
\pgfsetroundjoin%
\definecolor{currentfill}{rgb}{0.000000,0.000000,0.000000}%
\pgfsetfillcolor{currentfill}%
\pgfsetlinewidth{0.803000pt}%
\definecolor{currentstroke}{rgb}{0.000000,0.000000,0.000000}%
\pgfsetstrokecolor{currentstroke}%
\pgfsetdash{}{0pt}%
\pgfsys@defobject{currentmarker}{\pgfqpoint{-0.048611in}{0.000000in}}{\pgfqpoint{-0.000000in}{0.000000in}}{%
\pgfpathmoveto{\pgfqpoint{-0.000000in}{0.000000in}}%
\pgfpathlineto{\pgfqpoint{-0.048611in}{0.000000in}}%
\pgfusepath{stroke,fill}%
}%
\begin{pgfscope}%
\pgfsys@transformshift{0.721913in}{1.545069in}%
\pgfsys@useobject{currentmarker}{}%
\end{pgfscope}%
\end{pgfscope}%
\begin{pgfscope}%
\definecolor{textcolor}{rgb}{0.000000,0.000000,0.000000}%
\pgfsetstrokecolor{textcolor}%
\pgfsetfillcolor{textcolor}%
\pgftext[x=0.303703in, y=1.487199in, left, base]{\color{textcolor}{\rmfamily\fontsize{12.000000}{14.400000}\selectfont\catcode`\^=\active\def^{\ifmmode\sp\else\^{}\fi}\catcode`\%=\active\def%{\%}$\mathdefault{10^{-1}}$}}%
\end{pgfscope}%
\begin{pgfscope}%
\pgfsetbuttcap%
\pgfsetroundjoin%
\definecolor{currentfill}{rgb}{0.000000,0.000000,0.000000}%
\pgfsetfillcolor{currentfill}%
\pgfsetlinewidth{0.602250pt}%
\definecolor{currentstroke}{rgb}{0.000000,0.000000,0.000000}%
\pgfsetstrokecolor{currentstroke}%
\pgfsetdash{}{0pt}%
\pgfsys@defobject{currentmarker}{\pgfqpoint{-0.027778in}{0.000000in}}{\pgfqpoint{-0.000000in}{0.000000in}}{%
\pgfpathmoveto{\pgfqpoint{-0.000000in}{0.000000in}}%
\pgfpathlineto{\pgfqpoint{-0.027778in}{0.000000in}}%
\pgfusepath{stroke,fill}%
}%
\begin{pgfscope}%
\pgfsys@transformshift{0.721913in}{0.830964in}%
\pgfsys@useobject{currentmarker}{}%
\end{pgfscope}%
\end{pgfscope}%
\begin{pgfscope}%
\pgfsetbuttcap%
\pgfsetroundjoin%
\definecolor{currentfill}{rgb}{0.000000,0.000000,0.000000}%
\pgfsetfillcolor{currentfill}%
\pgfsetlinewidth{0.602250pt}%
\definecolor{currentstroke}{rgb}{0.000000,0.000000,0.000000}%
\pgfsetstrokecolor{currentstroke}%
\pgfsetdash{}{0pt}%
\pgfsys@defobject{currentmarker}{\pgfqpoint{-0.027778in}{0.000000in}}{\pgfqpoint{-0.000000in}{0.000000in}}{%
\pgfpathmoveto{\pgfqpoint{-0.000000in}{0.000000in}}%
\pgfpathlineto{\pgfqpoint{-0.027778in}{0.000000in}}%
\pgfusepath{stroke,fill}%
}%
\begin{pgfscope}%
\pgfsys@transformshift{0.721913in}{1.010868in}%
\pgfsys@useobject{currentmarker}{}%
\end{pgfscope}%
\end{pgfscope}%
\begin{pgfscope}%
\pgfsetbuttcap%
\pgfsetroundjoin%
\definecolor{currentfill}{rgb}{0.000000,0.000000,0.000000}%
\pgfsetfillcolor{currentfill}%
\pgfsetlinewidth{0.602250pt}%
\definecolor{currentstroke}{rgb}{0.000000,0.000000,0.000000}%
\pgfsetstrokecolor{currentstroke}%
\pgfsetdash{}{0pt}%
\pgfsys@defobject{currentmarker}{\pgfqpoint{-0.027778in}{0.000000in}}{\pgfqpoint{-0.000000in}{0.000000in}}{%
\pgfpathmoveto{\pgfqpoint{-0.000000in}{0.000000in}}%
\pgfpathlineto{\pgfqpoint{-0.027778in}{0.000000in}}%
\pgfusepath{stroke,fill}%
}%
\begin{pgfscope}%
\pgfsys@transformshift{0.721913in}{1.138512in}%
\pgfsys@useobject{currentmarker}{}%
\end{pgfscope}%
\end{pgfscope}%
\begin{pgfscope}%
\pgfsetbuttcap%
\pgfsetroundjoin%
\definecolor{currentfill}{rgb}{0.000000,0.000000,0.000000}%
\pgfsetfillcolor{currentfill}%
\pgfsetlinewidth{0.602250pt}%
\definecolor{currentstroke}{rgb}{0.000000,0.000000,0.000000}%
\pgfsetstrokecolor{currentstroke}%
\pgfsetdash{}{0pt}%
\pgfsys@defobject{currentmarker}{\pgfqpoint{-0.027778in}{0.000000in}}{\pgfqpoint{-0.000000in}{0.000000in}}{%
\pgfpathmoveto{\pgfqpoint{-0.000000in}{0.000000in}}%
\pgfpathlineto{\pgfqpoint{-0.027778in}{0.000000in}}%
\pgfusepath{stroke,fill}%
}%
\begin{pgfscope}%
\pgfsys@transformshift{0.721913in}{1.237521in}%
\pgfsys@useobject{currentmarker}{}%
\end{pgfscope}%
\end{pgfscope}%
\begin{pgfscope}%
\pgfsetbuttcap%
\pgfsetroundjoin%
\definecolor{currentfill}{rgb}{0.000000,0.000000,0.000000}%
\pgfsetfillcolor{currentfill}%
\pgfsetlinewidth{0.602250pt}%
\definecolor{currentstroke}{rgb}{0.000000,0.000000,0.000000}%
\pgfsetstrokecolor{currentstroke}%
\pgfsetdash{}{0pt}%
\pgfsys@defobject{currentmarker}{\pgfqpoint{-0.027778in}{0.000000in}}{\pgfqpoint{-0.000000in}{0.000000in}}{%
\pgfpathmoveto{\pgfqpoint{-0.000000in}{0.000000in}}%
\pgfpathlineto{\pgfqpoint{-0.027778in}{0.000000in}}%
\pgfusepath{stroke,fill}%
}%
\begin{pgfscope}%
\pgfsys@transformshift{0.721913in}{1.318417in}%
\pgfsys@useobject{currentmarker}{}%
\end{pgfscope}%
\end{pgfscope}%
\begin{pgfscope}%
\pgfsetbuttcap%
\pgfsetroundjoin%
\definecolor{currentfill}{rgb}{0.000000,0.000000,0.000000}%
\pgfsetfillcolor{currentfill}%
\pgfsetlinewidth{0.602250pt}%
\definecolor{currentstroke}{rgb}{0.000000,0.000000,0.000000}%
\pgfsetstrokecolor{currentstroke}%
\pgfsetdash{}{0pt}%
\pgfsys@defobject{currentmarker}{\pgfqpoint{-0.027778in}{0.000000in}}{\pgfqpoint{-0.000000in}{0.000000in}}{%
\pgfpathmoveto{\pgfqpoint{-0.000000in}{0.000000in}}%
\pgfpathlineto{\pgfqpoint{-0.027778in}{0.000000in}}%
\pgfusepath{stroke,fill}%
}%
\begin{pgfscope}%
\pgfsys@transformshift{0.721913in}{1.386813in}%
\pgfsys@useobject{currentmarker}{}%
\end{pgfscope}%
\end{pgfscope}%
\begin{pgfscope}%
\pgfsetbuttcap%
\pgfsetroundjoin%
\definecolor{currentfill}{rgb}{0.000000,0.000000,0.000000}%
\pgfsetfillcolor{currentfill}%
\pgfsetlinewidth{0.602250pt}%
\definecolor{currentstroke}{rgb}{0.000000,0.000000,0.000000}%
\pgfsetstrokecolor{currentstroke}%
\pgfsetdash{}{0pt}%
\pgfsys@defobject{currentmarker}{\pgfqpoint{-0.027778in}{0.000000in}}{\pgfqpoint{-0.000000in}{0.000000in}}{%
\pgfpathmoveto{\pgfqpoint{-0.000000in}{0.000000in}}%
\pgfpathlineto{\pgfqpoint{-0.027778in}{0.000000in}}%
\pgfusepath{stroke,fill}%
}%
\begin{pgfscope}%
\pgfsys@transformshift{0.721913in}{1.446061in}%
\pgfsys@useobject{currentmarker}{}%
\end{pgfscope}%
\end{pgfscope}%
\begin{pgfscope}%
\pgfsetbuttcap%
\pgfsetroundjoin%
\definecolor{currentfill}{rgb}{0.000000,0.000000,0.000000}%
\pgfsetfillcolor{currentfill}%
\pgfsetlinewidth{0.602250pt}%
\definecolor{currentstroke}{rgb}{0.000000,0.000000,0.000000}%
\pgfsetstrokecolor{currentstroke}%
\pgfsetdash{}{0pt}%
\pgfsys@defobject{currentmarker}{\pgfqpoint{-0.027778in}{0.000000in}}{\pgfqpoint{-0.000000in}{0.000000in}}{%
\pgfpathmoveto{\pgfqpoint{-0.000000in}{0.000000in}}%
\pgfpathlineto{\pgfqpoint{-0.027778in}{0.000000in}}%
\pgfusepath{stroke,fill}%
}%
\begin{pgfscope}%
\pgfsys@transformshift{0.721913in}{1.498321in}%
\pgfsys@useobject{currentmarker}{}%
\end{pgfscope}%
\end{pgfscope}%
\begin{pgfscope}%
\pgfsetbuttcap%
\pgfsetroundjoin%
\definecolor{currentfill}{rgb}{0.000000,0.000000,0.000000}%
\pgfsetfillcolor{currentfill}%
\pgfsetlinewidth{0.602250pt}%
\definecolor{currentstroke}{rgb}{0.000000,0.000000,0.000000}%
\pgfsetstrokecolor{currentstroke}%
\pgfsetdash{}{0pt}%
\pgfsys@defobject{currentmarker}{\pgfqpoint{-0.027778in}{0.000000in}}{\pgfqpoint{-0.000000in}{0.000000in}}{%
\pgfpathmoveto{\pgfqpoint{-0.000000in}{0.000000in}}%
\pgfpathlineto{\pgfqpoint{-0.027778in}{0.000000in}}%
\pgfusepath{stroke,fill}%
}%
\begin{pgfscope}%
\pgfsys@transformshift{0.721913in}{1.852618in}%
\pgfsys@useobject{currentmarker}{}%
\end{pgfscope}%
\end{pgfscope}%
\begin{pgfscope}%
\pgfsetbuttcap%
\pgfsetroundjoin%
\definecolor{currentfill}{rgb}{0.000000,0.000000,0.000000}%
\pgfsetfillcolor{currentfill}%
\pgfsetlinewidth{0.602250pt}%
\definecolor{currentstroke}{rgb}{0.000000,0.000000,0.000000}%
\pgfsetstrokecolor{currentstroke}%
\pgfsetdash{}{0pt}%
\pgfsys@defobject{currentmarker}{\pgfqpoint{-0.027778in}{0.000000in}}{\pgfqpoint{-0.000000in}{0.000000in}}{%
\pgfpathmoveto{\pgfqpoint{-0.000000in}{0.000000in}}%
\pgfpathlineto{\pgfqpoint{-0.027778in}{0.000000in}}%
\pgfusepath{stroke,fill}%
}%
\begin{pgfscope}%
\pgfsys@transformshift{0.721913in}{2.032523in}%
\pgfsys@useobject{currentmarker}{}%
\end{pgfscope}%
\end{pgfscope}%
\begin{pgfscope}%
\pgfsetbuttcap%
\pgfsetroundjoin%
\definecolor{currentfill}{rgb}{0.000000,0.000000,0.000000}%
\pgfsetfillcolor{currentfill}%
\pgfsetlinewidth{0.602250pt}%
\definecolor{currentstroke}{rgb}{0.000000,0.000000,0.000000}%
\pgfsetstrokecolor{currentstroke}%
\pgfsetdash{}{0pt}%
\pgfsys@defobject{currentmarker}{\pgfqpoint{-0.027778in}{0.000000in}}{\pgfqpoint{-0.000000in}{0.000000in}}{%
\pgfpathmoveto{\pgfqpoint{-0.000000in}{0.000000in}}%
\pgfpathlineto{\pgfqpoint{-0.027778in}{0.000000in}}%
\pgfusepath{stroke,fill}%
}%
\begin{pgfscope}%
\pgfsys@transformshift{0.721913in}{2.160167in}%
\pgfsys@useobject{currentmarker}{}%
\end{pgfscope}%
\end{pgfscope}%
\begin{pgfscope}%
\pgfsetbuttcap%
\pgfsetroundjoin%
\definecolor{currentfill}{rgb}{0.000000,0.000000,0.000000}%
\pgfsetfillcolor{currentfill}%
\pgfsetlinewidth{0.602250pt}%
\definecolor{currentstroke}{rgb}{0.000000,0.000000,0.000000}%
\pgfsetstrokecolor{currentstroke}%
\pgfsetdash{}{0pt}%
\pgfsys@defobject{currentmarker}{\pgfqpoint{-0.027778in}{0.000000in}}{\pgfqpoint{-0.000000in}{0.000000in}}{%
\pgfpathmoveto{\pgfqpoint{-0.000000in}{0.000000in}}%
\pgfpathlineto{\pgfqpoint{-0.027778in}{0.000000in}}%
\pgfusepath{stroke,fill}%
}%
\begin{pgfscope}%
\pgfsys@transformshift{0.721913in}{2.259175in}%
\pgfsys@useobject{currentmarker}{}%
\end{pgfscope}%
\end{pgfscope}%
\begin{pgfscope}%
\pgfsetbuttcap%
\pgfsetroundjoin%
\definecolor{currentfill}{rgb}{0.000000,0.000000,0.000000}%
\pgfsetfillcolor{currentfill}%
\pgfsetlinewidth{0.602250pt}%
\definecolor{currentstroke}{rgb}{0.000000,0.000000,0.000000}%
\pgfsetstrokecolor{currentstroke}%
\pgfsetdash{}{0pt}%
\pgfsys@defobject{currentmarker}{\pgfqpoint{-0.027778in}{0.000000in}}{\pgfqpoint{-0.000000in}{0.000000in}}{%
\pgfpathmoveto{\pgfqpoint{-0.000000in}{0.000000in}}%
\pgfpathlineto{\pgfqpoint{-0.027778in}{0.000000in}}%
\pgfusepath{stroke,fill}%
}%
\begin{pgfscope}%
\pgfsys@transformshift{0.721913in}{2.340071in}%
\pgfsys@useobject{currentmarker}{}%
\end{pgfscope}%
\end{pgfscope}%
\begin{pgfscope}%
\pgfsetbuttcap%
\pgfsetroundjoin%
\definecolor{currentfill}{rgb}{0.000000,0.000000,0.000000}%
\pgfsetfillcolor{currentfill}%
\pgfsetlinewidth{0.602250pt}%
\definecolor{currentstroke}{rgb}{0.000000,0.000000,0.000000}%
\pgfsetstrokecolor{currentstroke}%
\pgfsetdash{}{0pt}%
\pgfsys@defobject{currentmarker}{\pgfqpoint{-0.027778in}{0.000000in}}{\pgfqpoint{-0.000000in}{0.000000in}}{%
\pgfpathmoveto{\pgfqpoint{-0.000000in}{0.000000in}}%
\pgfpathlineto{\pgfqpoint{-0.027778in}{0.000000in}}%
\pgfusepath{stroke,fill}%
}%
\begin{pgfscope}%
\pgfsys@transformshift{0.721913in}{2.408468in}%
\pgfsys@useobject{currentmarker}{}%
\end{pgfscope}%
\end{pgfscope}%
\begin{pgfscope}%
\pgfsetbuttcap%
\pgfsetroundjoin%
\definecolor{currentfill}{rgb}{0.000000,0.000000,0.000000}%
\pgfsetfillcolor{currentfill}%
\pgfsetlinewidth{0.602250pt}%
\definecolor{currentstroke}{rgb}{0.000000,0.000000,0.000000}%
\pgfsetstrokecolor{currentstroke}%
\pgfsetdash{}{0pt}%
\pgfsys@defobject{currentmarker}{\pgfqpoint{-0.027778in}{0.000000in}}{\pgfqpoint{-0.000000in}{0.000000in}}{%
\pgfpathmoveto{\pgfqpoint{-0.000000in}{0.000000in}}%
\pgfpathlineto{\pgfqpoint{-0.027778in}{0.000000in}}%
\pgfusepath{stroke,fill}%
}%
\begin{pgfscope}%
\pgfsys@transformshift{0.721913in}{2.467716in}%
\pgfsys@useobject{currentmarker}{}%
\end{pgfscope}%
\end{pgfscope}%
\begin{pgfscope}%
\definecolor{textcolor}{rgb}{0.000000,0.000000,0.000000}%
\pgfsetstrokecolor{textcolor}%
\pgfsetfillcolor{textcolor}%
\pgftext[x=0.248148in,y=1.511573in,,bottom,rotate=90.000000]{\color{textcolor}{\rmfamily\fontsize{12.000000}{14.400000}\selectfont\catcode`\^=\active\def^{\ifmmode\sp\else\^{}\fi}\catcode`\%=\active\def%{\%}$L^1$ relative error}}%
\end{pgfscope}%
\begin{pgfscope}%
\pgfpathrectangle{\pgfqpoint{0.721913in}{0.549073in}}{\pgfqpoint{1.937500in}{1.925000in}}%
\pgfusepath{clip}%
\pgfsetrectcap%
\pgfsetroundjoin%
\pgfsetlinewidth{1.003750pt}%
\definecolor{currentstroke}{rgb}{0.537255,0.647059,0.760784}%
\pgfsetstrokecolor{currentstroke}%
\pgfsetdash{}{0pt}%
\pgfpathmoveto{\pgfqpoint{0.809982in}{1.381236in}}%
\pgfpathlineto{\pgfqpoint{1.122467in}{1.327322in}}%
\pgfpathlineto{\pgfqpoint{1.420640in}{1.189284in}}%
\pgfpathlineto{\pgfqpoint{1.710766in}{1.056198in}}%
\pgfpathlineto{\pgfqpoint{1.999725in}{0.914362in}}%
\pgfpathlineto{\pgfqpoint{2.285257in}{0.751521in}}%
\pgfpathlineto{\pgfqpoint{2.571345in}{0.636573in}}%
\pgfusepath{stroke}%
\end{pgfscope}%
\begin{pgfscope}%
\pgfpathrectangle{\pgfqpoint{0.721913in}{0.549073in}}{\pgfqpoint{1.937500in}{1.925000in}}%
\pgfusepath{clip}%
\pgfsetbuttcap%
\pgfsetroundjoin%
\definecolor{currentfill}{rgb}{0.537255,0.647059,0.760784}%
\pgfsetfillcolor{currentfill}%
\pgfsetlinewidth{1.003750pt}%
\definecolor{currentstroke}{rgb}{0.537255,0.647059,0.760784}%
\pgfsetstrokecolor{currentstroke}%
\pgfsetdash{}{0pt}%
\pgfsys@defobject{currentmarker}{\pgfqpoint{-0.020833in}{-0.020833in}}{\pgfqpoint{0.020833in}{0.020833in}}{%
\pgfpathmoveto{\pgfqpoint{0.000000in}{-0.020833in}}%
\pgfpathcurveto{\pgfqpoint{0.005525in}{-0.020833in}}{\pgfqpoint{0.010825in}{-0.018638in}}{\pgfqpoint{0.014731in}{-0.014731in}}%
\pgfpathcurveto{\pgfqpoint{0.018638in}{-0.010825in}}{\pgfqpoint{0.020833in}{-0.005525in}}{\pgfqpoint{0.020833in}{0.000000in}}%
\pgfpathcurveto{\pgfqpoint{0.020833in}{0.005525in}}{\pgfqpoint{0.018638in}{0.010825in}}{\pgfqpoint{0.014731in}{0.014731in}}%
\pgfpathcurveto{\pgfqpoint{0.010825in}{0.018638in}}{\pgfqpoint{0.005525in}{0.020833in}}{\pgfqpoint{0.000000in}{0.020833in}}%
\pgfpathcurveto{\pgfqpoint{-0.005525in}{0.020833in}}{\pgfqpoint{-0.010825in}{0.018638in}}{\pgfqpoint{-0.014731in}{0.014731in}}%
\pgfpathcurveto{\pgfqpoint{-0.018638in}{0.010825in}}{\pgfqpoint{-0.020833in}{0.005525in}}{\pgfqpoint{-0.020833in}{0.000000in}}%
\pgfpathcurveto{\pgfqpoint{-0.020833in}{-0.005525in}}{\pgfqpoint{-0.018638in}{-0.010825in}}{\pgfqpoint{-0.014731in}{-0.014731in}}%
\pgfpathcurveto{\pgfqpoint{-0.010825in}{-0.018638in}}{\pgfqpoint{-0.005525in}{-0.020833in}}{\pgfqpoint{0.000000in}{-0.020833in}}%
\pgfpathlineto{\pgfqpoint{0.000000in}{-0.020833in}}%
\pgfpathclose%
\pgfusepath{stroke,fill}%
}%
\begin{pgfscope}%
\pgfsys@transformshift{0.809982in}{1.381236in}%
\pgfsys@useobject{currentmarker}{}%
\end{pgfscope}%
\begin{pgfscope}%
\pgfsys@transformshift{1.122467in}{1.327322in}%
\pgfsys@useobject{currentmarker}{}%
\end{pgfscope}%
\begin{pgfscope}%
\pgfsys@transformshift{1.420640in}{1.189284in}%
\pgfsys@useobject{currentmarker}{}%
\end{pgfscope}%
\begin{pgfscope}%
\pgfsys@transformshift{1.710766in}{1.056198in}%
\pgfsys@useobject{currentmarker}{}%
\end{pgfscope}%
\begin{pgfscope}%
\pgfsys@transformshift{1.999725in}{0.914362in}%
\pgfsys@useobject{currentmarker}{}%
\end{pgfscope}%
\begin{pgfscope}%
\pgfsys@transformshift{2.285257in}{0.751521in}%
\pgfsys@useobject{currentmarker}{}%
\end{pgfscope}%
\begin{pgfscope}%
\pgfsys@transformshift{2.571345in}{0.636573in}%
\pgfsys@useobject{currentmarker}{}%
\end{pgfscope}%
\end{pgfscope}%
\begin{pgfscope}%
\pgfpathrectangle{\pgfqpoint{0.721913in}{0.549073in}}{\pgfqpoint{1.937500in}{1.925000in}}%
\pgfusepath{clip}%
\pgfsetrectcap%
\pgfsetroundjoin%
\pgfsetlinewidth{1.003750pt}%
\definecolor{currentstroke}{rgb}{0.184314,0.270588,0.360784}%
\pgfsetstrokecolor{currentstroke}%
\pgfsetdash{}{0pt}%
\pgfpathmoveto{\pgfqpoint{0.809982in}{2.386573in}}%
\pgfpathlineto{\pgfqpoint{1.122467in}{2.224603in}}%
\pgfpathlineto{\pgfqpoint{1.420640in}{1.973236in}}%
\pgfpathlineto{\pgfqpoint{1.710766in}{1.688538in}}%
\pgfpathlineto{\pgfqpoint{1.999725in}{1.820119in}}%
\pgfpathlineto{\pgfqpoint{2.285257in}{1.813566in}}%
\pgfpathlineto{\pgfqpoint{2.571345in}{1.830887in}}%
\pgfusepath{stroke}%
\end{pgfscope}%
\begin{pgfscope}%
\pgfpathrectangle{\pgfqpoint{0.721913in}{0.549073in}}{\pgfqpoint{1.937500in}{1.925000in}}%
\pgfusepath{clip}%
\pgfsetbuttcap%
\pgfsetroundjoin%
\definecolor{currentfill}{rgb}{0.184314,0.270588,0.360784}%
\pgfsetfillcolor{currentfill}%
\pgfsetlinewidth{1.003750pt}%
\definecolor{currentstroke}{rgb}{0.184314,0.270588,0.360784}%
\pgfsetstrokecolor{currentstroke}%
\pgfsetdash{}{0pt}%
\pgfsys@defobject{currentmarker}{\pgfqpoint{-0.020833in}{-0.020833in}}{\pgfqpoint{0.020833in}{0.020833in}}{%
\pgfpathmoveto{\pgfqpoint{0.000000in}{-0.020833in}}%
\pgfpathcurveto{\pgfqpoint{0.005525in}{-0.020833in}}{\pgfqpoint{0.010825in}{-0.018638in}}{\pgfqpoint{0.014731in}{-0.014731in}}%
\pgfpathcurveto{\pgfqpoint{0.018638in}{-0.010825in}}{\pgfqpoint{0.020833in}{-0.005525in}}{\pgfqpoint{0.020833in}{0.000000in}}%
\pgfpathcurveto{\pgfqpoint{0.020833in}{0.005525in}}{\pgfqpoint{0.018638in}{0.010825in}}{\pgfqpoint{0.014731in}{0.014731in}}%
\pgfpathcurveto{\pgfqpoint{0.010825in}{0.018638in}}{\pgfqpoint{0.005525in}{0.020833in}}{\pgfqpoint{0.000000in}{0.020833in}}%
\pgfpathcurveto{\pgfqpoint{-0.005525in}{0.020833in}}{\pgfqpoint{-0.010825in}{0.018638in}}{\pgfqpoint{-0.014731in}{0.014731in}}%
\pgfpathcurveto{\pgfqpoint{-0.018638in}{0.010825in}}{\pgfqpoint{-0.020833in}{0.005525in}}{\pgfqpoint{-0.020833in}{0.000000in}}%
\pgfpathcurveto{\pgfqpoint{-0.020833in}{-0.005525in}}{\pgfqpoint{-0.018638in}{-0.010825in}}{\pgfqpoint{-0.014731in}{-0.014731in}}%
\pgfpathcurveto{\pgfqpoint{-0.010825in}{-0.018638in}}{\pgfqpoint{-0.005525in}{-0.020833in}}{\pgfqpoint{0.000000in}{-0.020833in}}%
\pgfpathlineto{\pgfqpoint{0.000000in}{-0.020833in}}%
\pgfpathclose%
\pgfusepath{stroke,fill}%
}%
\begin{pgfscope}%
\pgfsys@transformshift{0.809982in}{2.386573in}%
\pgfsys@useobject{currentmarker}{}%
\end{pgfscope}%
\begin{pgfscope}%
\pgfsys@transformshift{1.122467in}{2.224603in}%
\pgfsys@useobject{currentmarker}{}%
\end{pgfscope}%
\begin{pgfscope}%
\pgfsys@transformshift{1.420640in}{1.973236in}%
\pgfsys@useobject{currentmarker}{}%
\end{pgfscope}%
\begin{pgfscope}%
\pgfsys@transformshift{1.710766in}{1.688538in}%
\pgfsys@useobject{currentmarker}{}%
\end{pgfscope}%
\begin{pgfscope}%
\pgfsys@transformshift{1.999725in}{1.820119in}%
\pgfsys@useobject{currentmarker}{}%
\end{pgfscope}%
\begin{pgfscope}%
\pgfsys@transformshift{2.285257in}{1.813566in}%
\pgfsys@useobject{currentmarker}{}%
\end{pgfscope}%
\begin{pgfscope}%
\pgfsys@transformshift{2.571345in}{1.830887in}%
\pgfsys@useobject{currentmarker}{}%
\end{pgfscope}%
\end{pgfscope}%
\begin{pgfscope}%
\pgfpathrectangle{\pgfqpoint{0.721913in}{0.549073in}}{\pgfqpoint{1.937500in}{1.925000in}}%
\pgfusepath{clip}%
\pgfsetrectcap%
\pgfsetroundjoin%
\pgfsetlinewidth{1.003750pt}%
\definecolor{currentstroke}{rgb}{0.976471,0.505882,0.145098}%
\pgfsetstrokecolor{currentstroke}%
\pgfsetdash{}{0pt}%
\pgfpathmoveto{\pgfqpoint{0.809982in}{1.659833in}}%
\pgfpathlineto{\pgfqpoint{1.122467in}{1.398499in}}%
\pgfpathlineto{\pgfqpoint{1.420640in}{1.385974in}}%
\pgfpathlineto{\pgfqpoint{1.710766in}{1.341183in}}%
\pgfpathlineto{\pgfqpoint{1.999725in}{1.328314in}}%
\pgfpathlineto{\pgfqpoint{2.285257in}{1.336252in}}%
\pgfpathlineto{\pgfqpoint{2.571345in}{1.347791in}}%
\pgfusepath{stroke}%
\end{pgfscope}%
\begin{pgfscope}%
\pgfpathrectangle{\pgfqpoint{0.721913in}{0.549073in}}{\pgfqpoint{1.937500in}{1.925000in}}%
\pgfusepath{clip}%
\pgfsetbuttcap%
\pgfsetroundjoin%
\definecolor{currentfill}{rgb}{0.976471,0.505882,0.145098}%
\pgfsetfillcolor{currentfill}%
\pgfsetlinewidth{1.003750pt}%
\definecolor{currentstroke}{rgb}{0.976471,0.505882,0.145098}%
\pgfsetstrokecolor{currentstroke}%
\pgfsetdash{}{0pt}%
\pgfsys@defobject{currentmarker}{\pgfqpoint{-0.020833in}{-0.020833in}}{\pgfqpoint{0.020833in}{0.020833in}}{%
\pgfpathmoveto{\pgfqpoint{0.000000in}{-0.020833in}}%
\pgfpathcurveto{\pgfqpoint{0.005525in}{-0.020833in}}{\pgfqpoint{0.010825in}{-0.018638in}}{\pgfqpoint{0.014731in}{-0.014731in}}%
\pgfpathcurveto{\pgfqpoint{0.018638in}{-0.010825in}}{\pgfqpoint{0.020833in}{-0.005525in}}{\pgfqpoint{0.020833in}{0.000000in}}%
\pgfpathcurveto{\pgfqpoint{0.020833in}{0.005525in}}{\pgfqpoint{0.018638in}{0.010825in}}{\pgfqpoint{0.014731in}{0.014731in}}%
\pgfpathcurveto{\pgfqpoint{0.010825in}{0.018638in}}{\pgfqpoint{0.005525in}{0.020833in}}{\pgfqpoint{0.000000in}{0.020833in}}%
\pgfpathcurveto{\pgfqpoint{-0.005525in}{0.020833in}}{\pgfqpoint{-0.010825in}{0.018638in}}{\pgfqpoint{-0.014731in}{0.014731in}}%
\pgfpathcurveto{\pgfqpoint{-0.018638in}{0.010825in}}{\pgfqpoint{-0.020833in}{0.005525in}}{\pgfqpoint{-0.020833in}{0.000000in}}%
\pgfpathcurveto{\pgfqpoint{-0.020833in}{-0.005525in}}{\pgfqpoint{-0.018638in}{-0.010825in}}{\pgfqpoint{-0.014731in}{-0.014731in}}%
\pgfpathcurveto{\pgfqpoint{-0.010825in}{-0.018638in}}{\pgfqpoint{-0.005525in}{-0.020833in}}{\pgfqpoint{0.000000in}{-0.020833in}}%
\pgfpathlineto{\pgfqpoint{0.000000in}{-0.020833in}}%
\pgfpathclose%
\pgfusepath{stroke,fill}%
}%
\begin{pgfscope}%
\pgfsys@transformshift{0.809982in}{1.659833in}%
\pgfsys@useobject{currentmarker}{}%
\end{pgfscope}%
\begin{pgfscope}%
\pgfsys@transformshift{1.122467in}{1.398499in}%
\pgfsys@useobject{currentmarker}{}%
\end{pgfscope}%
\begin{pgfscope}%
\pgfsys@transformshift{1.420640in}{1.385974in}%
\pgfsys@useobject{currentmarker}{}%
\end{pgfscope}%
\begin{pgfscope}%
\pgfsys@transformshift{1.710766in}{1.341183in}%
\pgfsys@useobject{currentmarker}{}%
\end{pgfscope}%
\begin{pgfscope}%
\pgfsys@transformshift{1.999725in}{1.328314in}%
\pgfsys@useobject{currentmarker}{}%
\end{pgfscope}%
\begin{pgfscope}%
\pgfsys@transformshift{2.285257in}{1.336252in}%
\pgfsys@useobject{currentmarker}{}%
\end{pgfscope}%
\begin{pgfscope}%
\pgfsys@transformshift{2.571345in}{1.347791in}%
\pgfsys@useobject{currentmarker}{}%
\end{pgfscope}%
\end{pgfscope}%
\begin{pgfscope}%
\pgfsetrectcap%
\pgfsetmiterjoin%
\pgfsetlinewidth{0.803000pt}%
\definecolor{currentstroke}{rgb}{0.000000,0.000000,0.000000}%
\pgfsetstrokecolor{currentstroke}%
\pgfsetdash{}{0pt}%
\pgfpathmoveto{\pgfqpoint{0.721913in}{0.549073in}}%
\pgfpathlineto{\pgfqpoint{0.721913in}{2.474073in}}%
\pgfusepath{stroke}%
\end{pgfscope}%
\begin{pgfscope}%
\pgfsetrectcap%
\pgfsetmiterjoin%
\pgfsetlinewidth{0.803000pt}%
\definecolor{currentstroke}{rgb}{0.000000,0.000000,0.000000}%
\pgfsetstrokecolor{currentstroke}%
\pgfsetdash{}{0pt}%
\pgfpathmoveto{\pgfqpoint{2.659413in}{0.549073in}}%
\pgfpathlineto{\pgfqpoint{2.659413in}{2.474073in}}%
\pgfusepath{stroke}%
\end{pgfscope}%
\begin{pgfscope}%
\pgfsetrectcap%
\pgfsetmiterjoin%
\pgfsetlinewidth{0.803000pt}%
\definecolor{currentstroke}{rgb}{0.000000,0.000000,0.000000}%
\pgfsetstrokecolor{currentstroke}%
\pgfsetdash{}{0pt}%
\pgfpathmoveto{\pgfqpoint{0.721913in}{0.549073in}}%
\pgfpathlineto{\pgfqpoint{2.659413in}{0.549073in}}%
\pgfusepath{stroke}%
\end{pgfscope}%
\begin{pgfscope}%
\pgfsetrectcap%
\pgfsetmiterjoin%
\pgfsetlinewidth{0.803000pt}%
\definecolor{currentstroke}{rgb}{0.000000,0.000000,0.000000}%
\pgfsetstrokecolor{currentstroke}%
\pgfsetdash{}{0pt}%
\pgfpathmoveto{\pgfqpoint{0.721913in}{2.474073in}}%
\pgfpathlineto{\pgfqpoint{2.659413in}{2.474073in}}%
\pgfusepath{stroke}%
\end{pgfscope}%
\begin{pgfscope}%
\pgfsetbuttcap%
\pgfsetmiterjoin%
\definecolor{currentfill}{rgb}{1.000000,1.000000,1.000000}%
\pgfsetfillcolor{currentfill}%
\pgfsetfillopacity{0.800000}%
\pgfsetlinewidth{1.003750pt}%
\definecolor{currentstroke}{rgb}{0.800000,0.800000,0.800000}%
\pgfsetstrokecolor{currentstroke}%
\pgfsetstrokeopacity{0.800000}%
\pgfsetdash{}{0pt}%
\pgfpathmoveto{\pgfqpoint{1.392965in}{1.643518in}}%
\pgfpathlineto{\pgfqpoint{2.542747in}{1.643518in}}%
\pgfpathquadraticcurveto{\pgfqpoint{2.576080in}{1.643518in}}{\pgfqpoint{2.576080in}{1.676852in}}%
\pgfpathlineto{\pgfqpoint{2.576080in}{2.357406in}}%
\pgfpathquadraticcurveto{\pgfqpoint{2.576080in}{2.390739in}}{\pgfqpoint{2.542747in}{2.390739in}}%
\pgfpathlineto{\pgfqpoint{1.392965in}{2.390739in}}%
\pgfpathquadraticcurveto{\pgfqpoint{1.359632in}{2.390739in}}{\pgfqpoint{1.359632in}{2.357406in}}%
\pgfpathlineto{\pgfqpoint{1.359632in}{1.676852in}}%
\pgfpathquadraticcurveto{\pgfqpoint{1.359632in}{1.643518in}}{\pgfqpoint{1.392965in}{1.643518in}}%
\pgfpathlineto{\pgfqpoint{1.392965in}{1.643518in}}%
\pgfpathclose%
\pgfusepath{stroke,fill}%
\end{pgfscope}%
\begin{pgfscope}%
\pgfsetrectcap%
\pgfsetroundjoin%
\pgfsetlinewidth{1.003750pt}%
\definecolor{currentstroke}{rgb}{0.537255,0.647059,0.760784}%
\pgfsetstrokecolor{currentstroke}%
\pgfsetdash{}{0pt}%
\pgfpathmoveto{\pgfqpoint{1.426299in}{2.265739in}}%
\pgfpathlineto{\pgfqpoint{1.592965in}{2.265739in}}%
\pgfpathlineto{\pgfqpoint{1.759632in}{2.265739in}}%
\pgfusepath{stroke}%
\end{pgfscope}%
\begin{pgfscope}%
\pgfsetbuttcap%
\pgfsetroundjoin%
\definecolor{currentfill}{rgb}{0.537255,0.647059,0.760784}%
\pgfsetfillcolor{currentfill}%
\pgfsetlinewidth{1.003750pt}%
\definecolor{currentstroke}{rgb}{0.537255,0.647059,0.760784}%
\pgfsetstrokecolor{currentstroke}%
\pgfsetdash{}{0pt}%
\pgfsys@defobject{currentmarker}{\pgfqpoint{-0.020833in}{-0.020833in}}{\pgfqpoint{0.020833in}{0.020833in}}{%
\pgfpathmoveto{\pgfqpoint{0.000000in}{-0.020833in}}%
\pgfpathcurveto{\pgfqpoint{0.005525in}{-0.020833in}}{\pgfqpoint{0.010825in}{-0.018638in}}{\pgfqpoint{0.014731in}{-0.014731in}}%
\pgfpathcurveto{\pgfqpoint{0.018638in}{-0.010825in}}{\pgfqpoint{0.020833in}{-0.005525in}}{\pgfqpoint{0.020833in}{0.000000in}}%
\pgfpathcurveto{\pgfqpoint{0.020833in}{0.005525in}}{\pgfqpoint{0.018638in}{0.010825in}}{\pgfqpoint{0.014731in}{0.014731in}}%
\pgfpathcurveto{\pgfqpoint{0.010825in}{0.018638in}}{\pgfqpoint{0.005525in}{0.020833in}}{\pgfqpoint{0.000000in}{0.020833in}}%
\pgfpathcurveto{\pgfqpoint{-0.005525in}{0.020833in}}{\pgfqpoint{-0.010825in}{0.018638in}}{\pgfqpoint{-0.014731in}{0.014731in}}%
\pgfpathcurveto{\pgfqpoint{-0.018638in}{0.010825in}}{\pgfqpoint{-0.020833in}{0.005525in}}{\pgfqpoint{-0.020833in}{0.000000in}}%
\pgfpathcurveto{\pgfqpoint{-0.020833in}{-0.005525in}}{\pgfqpoint{-0.018638in}{-0.010825in}}{\pgfqpoint{-0.014731in}{-0.014731in}}%
\pgfpathcurveto{\pgfqpoint{-0.010825in}{-0.018638in}}{\pgfqpoint{-0.005525in}{-0.020833in}}{\pgfqpoint{0.000000in}{-0.020833in}}%
\pgfpathlineto{\pgfqpoint{0.000000in}{-0.020833in}}%
\pgfpathclose%
\pgfusepath{stroke,fill}%
}%
\begin{pgfscope}%
\pgfsys@transformshift{1.592965in}{2.265739in}%
\pgfsys@useobject{currentmarker}{}%
\end{pgfscope}%
\end{pgfscope}%
\begin{pgfscope}%
\definecolor{textcolor}{rgb}{0.000000,0.000000,0.000000}%
\pgfsetstrokecolor{textcolor}%
\pgfsetfillcolor{textcolor}%
\pgftext[x=1.892965in,y=2.207406in,left,base]{\color{textcolor}{\rmfamily\fontsize{12.000000}{14.400000}\selectfont\catcode`\^=\active\def^{\ifmmode\sp\else\^{}\fi}\catcode`\%=\active\def%{\%}Haydock}}%
\end{pgfscope}%
\begin{pgfscope}%
\pgfsetrectcap%
\pgfsetroundjoin%
\pgfsetlinewidth{1.003750pt}%
\definecolor{currentstroke}{rgb}{0.184314,0.270588,0.360784}%
\pgfsetstrokecolor{currentstroke}%
\pgfsetdash{}{0pt}%
\pgfpathmoveto{\pgfqpoint{1.426299in}{2.033332in}}%
\pgfpathlineto{\pgfqpoint{1.592965in}{2.033332in}}%
\pgfpathlineto{\pgfqpoint{1.759632in}{2.033332in}}%
\pgfusepath{stroke}%
\end{pgfscope}%
\begin{pgfscope}%
\pgfsetbuttcap%
\pgfsetroundjoin%
\definecolor{currentfill}{rgb}{0.184314,0.270588,0.360784}%
\pgfsetfillcolor{currentfill}%
\pgfsetlinewidth{1.003750pt}%
\definecolor{currentstroke}{rgb}{0.184314,0.270588,0.360784}%
\pgfsetstrokecolor{currentstroke}%
\pgfsetdash{}{0pt}%
\pgfsys@defobject{currentmarker}{\pgfqpoint{-0.020833in}{-0.020833in}}{\pgfqpoint{0.020833in}{0.020833in}}{%
\pgfpathmoveto{\pgfqpoint{0.000000in}{-0.020833in}}%
\pgfpathcurveto{\pgfqpoint{0.005525in}{-0.020833in}}{\pgfqpoint{0.010825in}{-0.018638in}}{\pgfqpoint{0.014731in}{-0.014731in}}%
\pgfpathcurveto{\pgfqpoint{0.018638in}{-0.010825in}}{\pgfqpoint{0.020833in}{-0.005525in}}{\pgfqpoint{0.020833in}{0.000000in}}%
\pgfpathcurveto{\pgfqpoint{0.020833in}{0.005525in}}{\pgfqpoint{0.018638in}{0.010825in}}{\pgfqpoint{0.014731in}{0.014731in}}%
\pgfpathcurveto{\pgfqpoint{0.010825in}{0.018638in}}{\pgfqpoint{0.005525in}{0.020833in}}{\pgfqpoint{0.000000in}{0.020833in}}%
\pgfpathcurveto{\pgfqpoint{-0.005525in}{0.020833in}}{\pgfqpoint{-0.010825in}{0.018638in}}{\pgfqpoint{-0.014731in}{0.014731in}}%
\pgfpathcurveto{\pgfqpoint{-0.018638in}{0.010825in}}{\pgfqpoint{-0.020833in}{0.005525in}}{\pgfqpoint{-0.020833in}{0.000000in}}%
\pgfpathcurveto{\pgfqpoint{-0.020833in}{-0.005525in}}{\pgfqpoint{-0.018638in}{-0.010825in}}{\pgfqpoint{-0.014731in}{-0.014731in}}%
\pgfpathcurveto{\pgfqpoint{-0.010825in}{-0.018638in}}{\pgfqpoint{-0.005525in}{-0.020833in}}{\pgfqpoint{0.000000in}{-0.020833in}}%
\pgfpathlineto{\pgfqpoint{0.000000in}{-0.020833in}}%
\pgfpathclose%
\pgfusepath{stroke,fill}%
}%
\begin{pgfscope}%
\pgfsys@transformshift{1.592965in}{2.033332in}%
\pgfsys@useobject{currentmarker}{}%
\end{pgfscope}%
\end{pgfscope}%
\begin{pgfscope}%
\definecolor{textcolor}{rgb}{0.000000,0.000000,0.000000}%
\pgfsetstrokecolor{textcolor}%
\pgfsetfillcolor{textcolor}%
\pgftext[x=1.892965in,y=1.974999in,left,base]{\color{textcolor}{\rmfamily\fontsize{12.000000}{14.400000}\selectfont\catcode`\^=\active\def^{\ifmmode\sp\else\^{}\fi}\catcode`\%=\active\def%{\%}NC}}%
\end{pgfscope}%
\begin{pgfscope}%
\pgfsetrectcap%
\pgfsetroundjoin%
\pgfsetlinewidth{1.003750pt}%
\definecolor{currentstroke}{rgb}{0.976471,0.505882,0.145098}%
\pgfsetstrokecolor{currentstroke}%
\pgfsetdash{}{0pt}%
\pgfpathmoveto{\pgfqpoint{1.426299in}{1.800925in}}%
\pgfpathlineto{\pgfqpoint{1.592965in}{1.800925in}}%
\pgfpathlineto{\pgfqpoint{1.759632in}{1.800925in}}%
\pgfusepath{stroke}%
\end{pgfscope}%
\begin{pgfscope}%
\pgfsetbuttcap%
\pgfsetroundjoin%
\definecolor{currentfill}{rgb}{0.976471,0.505882,0.145098}%
\pgfsetfillcolor{currentfill}%
\pgfsetlinewidth{1.003750pt}%
\definecolor{currentstroke}{rgb}{0.976471,0.505882,0.145098}%
\pgfsetstrokecolor{currentstroke}%
\pgfsetdash{}{0pt}%
\pgfsys@defobject{currentmarker}{\pgfqpoint{-0.020833in}{-0.020833in}}{\pgfqpoint{0.020833in}{0.020833in}}{%
\pgfpathmoveto{\pgfqpoint{0.000000in}{-0.020833in}}%
\pgfpathcurveto{\pgfqpoint{0.005525in}{-0.020833in}}{\pgfqpoint{0.010825in}{-0.018638in}}{\pgfqpoint{0.014731in}{-0.014731in}}%
\pgfpathcurveto{\pgfqpoint{0.018638in}{-0.010825in}}{\pgfqpoint{0.020833in}{-0.005525in}}{\pgfqpoint{0.020833in}{0.000000in}}%
\pgfpathcurveto{\pgfqpoint{0.020833in}{0.005525in}}{\pgfqpoint{0.018638in}{0.010825in}}{\pgfqpoint{0.014731in}{0.014731in}}%
\pgfpathcurveto{\pgfqpoint{0.010825in}{0.018638in}}{\pgfqpoint{0.005525in}{0.020833in}}{\pgfqpoint{0.000000in}{0.020833in}}%
\pgfpathcurveto{\pgfqpoint{-0.005525in}{0.020833in}}{\pgfqpoint{-0.010825in}{0.018638in}}{\pgfqpoint{-0.014731in}{0.014731in}}%
\pgfpathcurveto{\pgfqpoint{-0.018638in}{0.010825in}}{\pgfqpoint{-0.020833in}{0.005525in}}{\pgfqpoint{-0.020833in}{0.000000in}}%
\pgfpathcurveto{\pgfqpoint{-0.020833in}{-0.005525in}}{\pgfqpoint{-0.018638in}{-0.010825in}}{\pgfqpoint{-0.014731in}{-0.014731in}}%
\pgfpathcurveto{\pgfqpoint{-0.010825in}{-0.018638in}}{\pgfqpoint{-0.005525in}{-0.020833in}}{\pgfqpoint{0.000000in}{-0.020833in}}%
\pgfpathlineto{\pgfqpoint{0.000000in}{-0.020833in}}%
\pgfpathclose%
\pgfusepath{stroke,fill}%
}%
\begin{pgfscope}%
\pgfsys@transformshift{1.592965in}{1.800925in}%
\pgfsys@useobject{currentmarker}{}%
\end{pgfscope}%
\end{pgfscope}%
\begin{pgfscope}%
\definecolor{textcolor}{rgb}{0.000000,0.000000,0.000000}%
\pgfsetstrokecolor{textcolor}%
\pgfsetfillcolor{textcolor}%
\pgftext[x=1.892965in,y=1.742592in,left,base]{\color{textcolor}{\rmfamily\fontsize{12.000000}{14.400000}\selectfont\catcode`\^=\active\def^{\ifmmode\sp\else\^{}\fi}\catcode`\%=\active\def%{\%}NC++}}%
\end{pgfscope}%
\end{pgfpicture}%
\makeatother%
\endgroup%

        \caption{$m=800$}
        \label{fig:5-experiments-haydock-convergence-nv-m800}
    \end{subfigure}
    \begin{subfigure}[b]{0.49\columnwidth}
        %% Creator: Matplotlib, PGF backend
%%
%% To include the figure in your LaTeX document, write
%%   \input{<filename>.pgf}
%%
%% Make sure the required packages are loaded in your preamble
%%   \usepackage{pgf}
%%
%% Also ensure that all the required font packages are loaded; for instance,
%% the lmodern package is sometimes necessary when using math font.
%%   \usepackage{lmodern}
%%
%% Figures using additional raster images can only be included by \input if
%% they are in the same directory as the main LaTeX file. For loading figures
%% from other directories you can use the `import` package
%%   \usepackage{import}
%%
%% and then include the figures with
%%   \import{<path to file>}{<filename>.pgf}
%%
%% Matplotlib used the following preamble
%%   \def\mathdefault#1{#1}
%%   \everymath=\expandafter{\the\everymath\displaystyle}
%%   
%%   \makeatletter\@ifpackageloaded{underscore}{}{\usepackage[strings]{underscore}}\makeatother
%%
\begingroup%
\makeatletter%
\begin{pgfpicture}%
\pgfpathrectangle{\pgfpointorigin}{\pgfqpoint{2.759413in}{2.574073in}}%
\pgfusepath{use as bounding box, clip}%
\begin{pgfscope}%
\pgfsetbuttcap%
\pgfsetmiterjoin%
\definecolor{currentfill}{rgb}{1.000000,1.000000,1.000000}%
\pgfsetfillcolor{currentfill}%
\pgfsetlinewidth{0.000000pt}%
\definecolor{currentstroke}{rgb}{1.000000,1.000000,1.000000}%
\pgfsetstrokecolor{currentstroke}%
\pgfsetdash{}{0pt}%
\pgfpathmoveto{\pgfqpoint{0.000000in}{0.000000in}}%
\pgfpathlineto{\pgfqpoint{2.759413in}{0.000000in}}%
\pgfpathlineto{\pgfqpoint{2.759413in}{2.574073in}}%
\pgfpathlineto{\pgfqpoint{0.000000in}{2.574073in}}%
\pgfpathlineto{\pgfqpoint{0.000000in}{0.000000in}}%
\pgfpathclose%
\pgfusepath{fill}%
\end{pgfscope}%
\begin{pgfscope}%
\pgfsetbuttcap%
\pgfsetmiterjoin%
\definecolor{currentfill}{rgb}{1.000000,1.000000,1.000000}%
\pgfsetfillcolor{currentfill}%
\pgfsetlinewidth{0.000000pt}%
\definecolor{currentstroke}{rgb}{0.000000,0.000000,0.000000}%
\pgfsetstrokecolor{currentstroke}%
\pgfsetstrokeopacity{0.000000}%
\pgfsetdash{}{0pt}%
\pgfpathmoveto{\pgfqpoint{0.721913in}{0.549073in}}%
\pgfpathlineto{\pgfqpoint{2.659413in}{0.549073in}}%
\pgfpathlineto{\pgfqpoint{2.659413in}{2.474073in}}%
\pgfpathlineto{\pgfqpoint{0.721913in}{2.474073in}}%
\pgfpathlineto{\pgfqpoint{0.721913in}{0.549073in}}%
\pgfpathclose%
\pgfusepath{fill}%
\end{pgfscope}%
\begin{pgfscope}%
\pgfsetbuttcap%
\pgfsetroundjoin%
\definecolor{currentfill}{rgb}{0.000000,0.000000,0.000000}%
\pgfsetfillcolor{currentfill}%
\pgfsetlinewidth{0.803000pt}%
\definecolor{currentstroke}{rgb}{0.000000,0.000000,0.000000}%
\pgfsetstrokecolor{currentstroke}%
\pgfsetdash{}{0pt}%
\pgfsys@defobject{currentmarker}{\pgfqpoint{0.000000in}{-0.048611in}}{\pgfqpoint{0.000000in}{0.000000in}}{%
\pgfpathmoveto{\pgfqpoint{0.000000in}{0.000000in}}%
\pgfpathlineto{\pgfqpoint{0.000000in}{-0.048611in}}%
\pgfusepath{stroke,fill}%
}%
\begin{pgfscope}%
\pgfsys@transformshift{0.910580in}{0.549073in}%
\pgfsys@useobject{currentmarker}{}%
\end{pgfscope}%
\end{pgfscope}%
\begin{pgfscope}%
\definecolor{textcolor}{rgb}{0.000000,0.000000,0.000000}%
\pgfsetstrokecolor{textcolor}%
\pgfsetfillcolor{textcolor}%
\pgftext[x=0.910580in,y=0.451851in,,top]{\color{textcolor}{\rmfamily\fontsize{12.000000}{14.400000}\selectfont\catcode`\^=\active\def^{\ifmmode\sp\else\^{}\fi}\catcode`\%=\active\def%{\%}$\mathdefault{10^{1}}$}}%
\end{pgfscope}%
\begin{pgfscope}%
\pgfsetbuttcap%
\pgfsetroundjoin%
\definecolor{currentfill}{rgb}{0.000000,0.000000,0.000000}%
\pgfsetfillcolor{currentfill}%
\pgfsetlinewidth{0.803000pt}%
\definecolor{currentstroke}{rgb}{0.000000,0.000000,0.000000}%
\pgfsetstrokecolor{currentstroke}%
\pgfsetdash{}{0pt}%
\pgfsys@defobject{currentmarker}{\pgfqpoint{0.000000in}{-0.048611in}}{\pgfqpoint{0.000000in}{0.000000in}}{%
\pgfpathmoveto{\pgfqpoint{0.000000in}{0.000000in}}%
\pgfpathlineto{\pgfqpoint{0.000000in}{-0.048611in}}%
\pgfusepath{stroke,fill}%
}%
\begin{pgfscope}%
\pgfsys@transformshift{1.948634in}{0.549073in}%
\pgfsys@useobject{currentmarker}{}%
\end{pgfscope}%
\end{pgfscope}%
\begin{pgfscope}%
\definecolor{textcolor}{rgb}{0.000000,0.000000,0.000000}%
\pgfsetstrokecolor{textcolor}%
\pgfsetfillcolor{textcolor}%
\pgftext[x=1.948634in,y=0.451851in,,top]{\color{textcolor}{\rmfamily\fontsize{12.000000}{14.400000}\selectfont\catcode`\^=\active\def^{\ifmmode\sp\else\^{}\fi}\catcode`\%=\active\def%{\%}$\mathdefault{10^{2}}$}}%
\end{pgfscope}%
\begin{pgfscope}%
\pgfsetbuttcap%
\pgfsetroundjoin%
\definecolor{currentfill}{rgb}{0.000000,0.000000,0.000000}%
\pgfsetfillcolor{currentfill}%
\pgfsetlinewidth{0.602250pt}%
\definecolor{currentstroke}{rgb}{0.000000,0.000000,0.000000}%
\pgfsetstrokecolor{currentstroke}%
\pgfsetdash{}{0pt}%
\pgfsys@defobject{currentmarker}{\pgfqpoint{0.000000in}{-0.027778in}}{\pgfqpoint{0.000000in}{0.000000in}}{%
\pgfpathmoveto{\pgfqpoint{0.000000in}{0.000000in}}%
\pgfpathlineto{\pgfqpoint{0.000000in}{-0.027778in}}%
\pgfusepath{stroke,fill}%
}%
\begin{pgfscope}%
\pgfsys@transformshift{0.749783in}{0.549073in}%
\pgfsys@useobject{currentmarker}{}%
\end{pgfscope}%
\end{pgfscope}%
\begin{pgfscope}%
\pgfsetbuttcap%
\pgfsetroundjoin%
\definecolor{currentfill}{rgb}{0.000000,0.000000,0.000000}%
\pgfsetfillcolor{currentfill}%
\pgfsetlinewidth{0.602250pt}%
\definecolor{currentstroke}{rgb}{0.000000,0.000000,0.000000}%
\pgfsetstrokecolor{currentstroke}%
\pgfsetdash{}{0pt}%
\pgfsys@defobject{currentmarker}{\pgfqpoint{0.000000in}{-0.027778in}}{\pgfqpoint{0.000000in}{0.000000in}}{%
\pgfpathmoveto{\pgfqpoint{0.000000in}{0.000000in}}%
\pgfpathlineto{\pgfqpoint{0.000000in}{-0.027778in}}%
\pgfusepath{stroke,fill}%
}%
\begin{pgfscope}%
\pgfsys@transformshift{0.809982in}{0.549073in}%
\pgfsys@useobject{currentmarker}{}%
\end{pgfscope}%
\end{pgfscope}%
\begin{pgfscope}%
\pgfsetbuttcap%
\pgfsetroundjoin%
\definecolor{currentfill}{rgb}{0.000000,0.000000,0.000000}%
\pgfsetfillcolor{currentfill}%
\pgfsetlinewidth{0.602250pt}%
\definecolor{currentstroke}{rgb}{0.000000,0.000000,0.000000}%
\pgfsetstrokecolor{currentstroke}%
\pgfsetdash{}{0pt}%
\pgfsys@defobject{currentmarker}{\pgfqpoint{0.000000in}{-0.027778in}}{\pgfqpoint{0.000000in}{0.000000in}}{%
\pgfpathmoveto{\pgfqpoint{0.000000in}{0.000000in}}%
\pgfpathlineto{\pgfqpoint{0.000000in}{-0.027778in}}%
\pgfusepath{stroke,fill}%
}%
\begin{pgfscope}%
\pgfsys@transformshift{0.863081in}{0.549073in}%
\pgfsys@useobject{currentmarker}{}%
\end{pgfscope}%
\end{pgfscope}%
\begin{pgfscope}%
\pgfsetbuttcap%
\pgfsetroundjoin%
\definecolor{currentfill}{rgb}{0.000000,0.000000,0.000000}%
\pgfsetfillcolor{currentfill}%
\pgfsetlinewidth{0.602250pt}%
\definecolor{currentstroke}{rgb}{0.000000,0.000000,0.000000}%
\pgfsetstrokecolor{currentstroke}%
\pgfsetdash{}{0pt}%
\pgfsys@defobject{currentmarker}{\pgfqpoint{0.000000in}{-0.027778in}}{\pgfqpoint{0.000000in}{0.000000in}}{%
\pgfpathmoveto{\pgfqpoint{0.000000in}{0.000000in}}%
\pgfpathlineto{\pgfqpoint{0.000000in}{-0.027778in}}%
\pgfusepath{stroke,fill}%
}%
\begin{pgfscope}%
\pgfsys@transformshift{1.223065in}{0.549073in}%
\pgfsys@useobject{currentmarker}{}%
\end{pgfscope}%
\end{pgfscope}%
\begin{pgfscope}%
\pgfsetbuttcap%
\pgfsetroundjoin%
\definecolor{currentfill}{rgb}{0.000000,0.000000,0.000000}%
\pgfsetfillcolor{currentfill}%
\pgfsetlinewidth{0.602250pt}%
\definecolor{currentstroke}{rgb}{0.000000,0.000000,0.000000}%
\pgfsetstrokecolor{currentstroke}%
\pgfsetdash{}{0pt}%
\pgfsys@defobject{currentmarker}{\pgfqpoint{0.000000in}{-0.027778in}}{\pgfqpoint{0.000000in}{0.000000in}}{%
\pgfpathmoveto{\pgfqpoint{0.000000in}{0.000000in}}%
\pgfpathlineto{\pgfqpoint{0.000000in}{-0.027778in}}%
\pgfusepath{stroke,fill}%
}%
\begin{pgfscope}%
\pgfsys@transformshift{1.405857in}{0.549073in}%
\pgfsys@useobject{currentmarker}{}%
\end{pgfscope}%
\end{pgfscope}%
\begin{pgfscope}%
\pgfsetbuttcap%
\pgfsetroundjoin%
\definecolor{currentfill}{rgb}{0.000000,0.000000,0.000000}%
\pgfsetfillcolor{currentfill}%
\pgfsetlinewidth{0.602250pt}%
\definecolor{currentstroke}{rgb}{0.000000,0.000000,0.000000}%
\pgfsetstrokecolor{currentstroke}%
\pgfsetdash{}{0pt}%
\pgfsys@defobject{currentmarker}{\pgfqpoint{0.000000in}{-0.027778in}}{\pgfqpoint{0.000000in}{0.000000in}}{%
\pgfpathmoveto{\pgfqpoint{0.000000in}{0.000000in}}%
\pgfpathlineto{\pgfqpoint{0.000000in}{-0.027778in}}%
\pgfusepath{stroke,fill}%
}%
\begin{pgfscope}%
\pgfsys@transformshift{1.535551in}{0.549073in}%
\pgfsys@useobject{currentmarker}{}%
\end{pgfscope}%
\end{pgfscope}%
\begin{pgfscope}%
\pgfsetbuttcap%
\pgfsetroundjoin%
\definecolor{currentfill}{rgb}{0.000000,0.000000,0.000000}%
\pgfsetfillcolor{currentfill}%
\pgfsetlinewidth{0.602250pt}%
\definecolor{currentstroke}{rgb}{0.000000,0.000000,0.000000}%
\pgfsetstrokecolor{currentstroke}%
\pgfsetdash{}{0pt}%
\pgfsys@defobject{currentmarker}{\pgfqpoint{0.000000in}{-0.027778in}}{\pgfqpoint{0.000000in}{0.000000in}}{%
\pgfpathmoveto{\pgfqpoint{0.000000in}{0.000000in}}%
\pgfpathlineto{\pgfqpoint{0.000000in}{-0.027778in}}%
\pgfusepath{stroke,fill}%
}%
\begin{pgfscope}%
\pgfsys@transformshift{1.636148in}{0.549073in}%
\pgfsys@useobject{currentmarker}{}%
\end{pgfscope}%
\end{pgfscope}%
\begin{pgfscope}%
\pgfsetbuttcap%
\pgfsetroundjoin%
\definecolor{currentfill}{rgb}{0.000000,0.000000,0.000000}%
\pgfsetfillcolor{currentfill}%
\pgfsetlinewidth{0.602250pt}%
\definecolor{currentstroke}{rgb}{0.000000,0.000000,0.000000}%
\pgfsetstrokecolor{currentstroke}%
\pgfsetdash{}{0pt}%
\pgfsys@defobject{currentmarker}{\pgfqpoint{0.000000in}{-0.027778in}}{\pgfqpoint{0.000000in}{0.000000in}}{%
\pgfpathmoveto{\pgfqpoint{0.000000in}{0.000000in}}%
\pgfpathlineto{\pgfqpoint{0.000000in}{-0.027778in}}%
\pgfusepath{stroke,fill}%
}%
\begin{pgfscope}%
\pgfsys@transformshift{1.718343in}{0.549073in}%
\pgfsys@useobject{currentmarker}{}%
\end{pgfscope}%
\end{pgfscope}%
\begin{pgfscope}%
\pgfsetbuttcap%
\pgfsetroundjoin%
\definecolor{currentfill}{rgb}{0.000000,0.000000,0.000000}%
\pgfsetfillcolor{currentfill}%
\pgfsetlinewidth{0.602250pt}%
\definecolor{currentstroke}{rgb}{0.000000,0.000000,0.000000}%
\pgfsetstrokecolor{currentstroke}%
\pgfsetdash{}{0pt}%
\pgfsys@defobject{currentmarker}{\pgfqpoint{0.000000in}{-0.027778in}}{\pgfqpoint{0.000000in}{0.000000in}}{%
\pgfpathmoveto{\pgfqpoint{0.000000in}{0.000000in}}%
\pgfpathlineto{\pgfqpoint{0.000000in}{-0.027778in}}%
\pgfusepath{stroke,fill}%
}%
\begin{pgfscope}%
\pgfsys@transformshift{1.787837in}{0.549073in}%
\pgfsys@useobject{currentmarker}{}%
\end{pgfscope}%
\end{pgfscope}%
\begin{pgfscope}%
\pgfsetbuttcap%
\pgfsetroundjoin%
\definecolor{currentfill}{rgb}{0.000000,0.000000,0.000000}%
\pgfsetfillcolor{currentfill}%
\pgfsetlinewidth{0.602250pt}%
\definecolor{currentstroke}{rgb}{0.000000,0.000000,0.000000}%
\pgfsetstrokecolor{currentstroke}%
\pgfsetdash{}{0pt}%
\pgfsys@defobject{currentmarker}{\pgfqpoint{0.000000in}{-0.027778in}}{\pgfqpoint{0.000000in}{0.000000in}}{%
\pgfpathmoveto{\pgfqpoint{0.000000in}{0.000000in}}%
\pgfpathlineto{\pgfqpoint{0.000000in}{-0.027778in}}%
\pgfusepath{stroke,fill}%
}%
\begin{pgfscope}%
\pgfsys@transformshift{1.848036in}{0.549073in}%
\pgfsys@useobject{currentmarker}{}%
\end{pgfscope}%
\end{pgfscope}%
\begin{pgfscope}%
\pgfsetbuttcap%
\pgfsetroundjoin%
\definecolor{currentfill}{rgb}{0.000000,0.000000,0.000000}%
\pgfsetfillcolor{currentfill}%
\pgfsetlinewidth{0.602250pt}%
\definecolor{currentstroke}{rgb}{0.000000,0.000000,0.000000}%
\pgfsetstrokecolor{currentstroke}%
\pgfsetdash{}{0pt}%
\pgfsys@defobject{currentmarker}{\pgfqpoint{0.000000in}{-0.027778in}}{\pgfqpoint{0.000000in}{0.000000in}}{%
\pgfpathmoveto{\pgfqpoint{0.000000in}{0.000000in}}%
\pgfpathlineto{\pgfqpoint{0.000000in}{-0.027778in}}%
\pgfusepath{stroke,fill}%
}%
\begin{pgfscope}%
\pgfsys@transformshift{1.901135in}{0.549073in}%
\pgfsys@useobject{currentmarker}{}%
\end{pgfscope}%
\end{pgfscope}%
\begin{pgfscope}%
\pgfsetbuttcap%
\pgfsetroundjoin%
\definecolor{currentfill}{rgb}{0.000000,0.000000,0.000000}%
\pgfsetfillcolor{currentfill}%
\pgfsetlinewidth{0.602250pt}%
\definecolor{currentstroke}{rgb}{0.000000,0.000000,0.000000}%
\pgfsetstrokecolor{currentstroke}%
\pgfsetdash{}{0pt}%
\pgfsys@defobject{currentmarker}{\pgfqpoint{0.000000in}{-0.027778in}}{\pgfqpoint{0.000000in}{0.000000in}}{%
\pgfpathmoveto{\pgfqpoint{0.000000in}{0.000000in}}%
\pgfpathlineto{\pgfqpoint{0.000000in}{-0.027778in}}%
\pgfusepath{stroke,fill}%
}%
\begin{pgfscope}%
\pgfsys@transformshift{2.261120in}{0.549073in}%
\pgfsys@useobject{currentmarker}{}%
\end{pgfscope}%
\end{pgfscope}%
\begin{pgfscope}%
\pgfsetbuttcap%
\pgfsetroundjoin%
\definecolor{currentfill}{rgb}{0.000000,0.000000,0.000000}%
\pgfsetfillcolor{currentfill}%
\pgfsetlinewidth{0.602250pt}%
\definecolor{currentstroke}{rgb}{0.000000,0.000000,0.000000}%
\pgfsetstrokecolor{currentstroke}%
\pgfsetdash{}{0pt}%
\pgfsys@defobject{currentmarker}{\pgfqpoint{0.000000in}{-0.027778in}}{\pgfqpoint{0.000000in}{0.000000in}}{%
\pgfpathmoveto{\pgfqpoint{0.000000in}{0.000000in}}%
\pgfpathlineto{\pgfqpoint{0.000000in}{-0.027778in}}%
\pgfusepath{stroke,fill}%
}%
\begin{pgfscope}%
\pgfsys@transformshift{2.443912in}{0.549073in}%
\pgfsys@useobject{currentmarker}{}%
\end{pgfscope}%
\end{pgfscope}%
\begin{pgfscope}%
\pgfsetbuttcap%
\pgfsetroundjoin%
\definecolor{currentfill}{rgb}{0.000000,0.000000,0.000000}%
\pgfsetfillcolor{currentfill}%
\pgfsetlinewidth{0.602250pt}%
\definecolor{currentstroke}{rgb}{0.000000,0.000000,0.000000}%
\pgfsetstrokecolor{currentstroke}%
\pgfsetdash{}{0pt}%
\pgfsys@defobject{currentmarker}{\pgfqpoint{0.000000in}{-0.027778in}}{\pgfqpoint{0.000000in}{0.000000in}}{%
\pgfpathmoveto{\pgfqpoint{0.000000in}{0.000000in}}%
\pgfpathlineto{\pgfqpoint{0.000000in}{-0.027778in}}%
\pgfusepath{stroke,fill}%
}%
\begin{pgfscope}%
\pgfsys@transformshift{2.573605in}{0.549073in}%
\pgfsys@useobject{currentmarker}{}%
\end{pgfscope}%
\end{pgfscope}%
\begin{pgfscope}%
\definecolor{textcolor}{rgb}{0.000000,0.000000,0.000000}%
\pgfsetstrokecolor{textcolor}%
\pgfsetfillcolor{textcolor}%
\pgftext[x=1.690663in,y=0.248148in,,top]{\color{textcolor}{\rmfamily\fontsize{12.000000}{14.400000}\selectfont\catcode`\^=\active\def^{\ifmmode\sp\else\^{}\fi}\catcode`\%=\active\def%{\%}$n_{\Omega} + n_{\Psi}$}}%
\end{pgfscope}%
\begin{pgfscope}%
\pgfsetbuttcap%
\pgfsetroundjoin%
\definecolor{currentfill}{rgb}{0.000000,0.000000,0.000000}%
\pgfsetfillcolor{currentfill}%
\pgfsetlinewidth{0.803000pt}%
\definecolor{currentstroke}{rgb}{0.000000,0.000000,0.000000}%
\pgfsetstrokecolor{currentstroke}%
\pgfsetdash{}{0pt}%
\pgfsys@defobject{currentmarker}{\pgfqpoint{-0.048611in}{0.000000in}}{\pgfqpoint{-0.000000in}{0.000000in}}{%
\pgfpathmoveto{\pgfqpoint{-0.000000in}{0.000000in}}%
\pgfpathlineto{\pgfqpoint{-0.048611in}{0.000000in}}%
\pgfusepath{stroke,fill}%
}%
\begin{pgfscope}%
\pgfsys@transformshift{0.721913in}{0.885775in}%
\pgfsys@useobject{currentmarker}{}%
\end{pgfscope}%
\end{pgfscope}%
\begin{pgfscope}%
\definecolor{textcolor}{rgb}{0.000000,0.000000,0.000000}%
\pgfsetstrokecolor{textcolor}%
\pgfsetfillcolor{textcolor}%
\pgftext[x=0.303703in, y=0.827905in, left, base]{\color{textcolor}{\rmfamily\fontsize{12.000000}{14.400000}\selectfont\catcode`\^=\active\def^{\ifmmode\sp\else\^{}\fi}\catcode`\%=\active\def%{\%}$\mathdefault{10^{-3}}$}}%
\end{pgfscope}%
\begin{pgfscope}%
\pgfsetbuttcap%
\pgfsetroundjoin%
\definecolor{currentfill}{rgb}{0.000000,0.000000,0.000000}%
\pgfsetfillcolor{currentfill}%
\pgfsetlinewidth{0.803000pt}%
\definecolor{currentstroke}{rgb}{0.000000,0.000000,0.000000}%
\pgfsetstrokecolor{currentstroke}%
\pgfsetdash{}{0pt}%
\pgfsys@defobject{currentmarker}{\pgfqpoint{-0.048611in}{0.000000in}}{\pgfqpoint{-0.000000in}{0.000000in}}{%
\pgfpathmoveto{\pgfqpoint{-0.000000in}{0.000000in}}%
\pgfpathlineto{\pgfqpoint{-0.048611in}{0.000000in}}%
\pgfusepath{stroke,fill}%
}%
\begin{pgfscope}%
\pgfsys@transformshift{0.721913in}{1.416127in}%
\pgfsys@useobject{currentmarker}{}%
\end{pgfscope}%
\end{pgfscope}%
\begin{pgfscope}%
\definecolor{textcolor}{rgb}{0.000000,0.000000,0.000000}%
\pgfsetstrokecolor{textcolor}%
\pgfsetfillcolor{textcolor}%
\pgftext[x=0.303703in, y=1.358257in, left, base]{\color{textcolor}{\rmfamily\fontsize{12.000000}{14.400000}\selectfont\catcode`\^=\active\def^{\ifmmode\sp\else\^{}\fi}\catcode`\%=\active\def%{\%}$\mathdefault{10^{-2}}$}}%
\end{pgfscope}%
\begin{pgfscope}%
\pgfsetbuttcap%
\pgfsetroundjoin%
\definecolor{currentfill}{rgb}{0.000000,0.000000,0.000000}%
\pgfsetfillcolor{currentfill}%
\pgfsetlinewidth{0.803000pt}%
\definecolor{currentstroke}{rgb}{0.000000,0.000000,0.000000}%
\pgfsetstrokecolor{currentstroke}%
\pgfsetdash{}{0pt}%
\pgfsys@defobject{currentmarker}{\pgfqpoint{-0.048611in}{0.000000in}}{\pgfqpoint{-0.000000in}{0.000000in}}{%
\pgfpathmoveto{\pgfqpoint{-0.000000in}{0.000000in}}%
\pgfpathlineto{\pgfqpoint{-0.048611in}{0.000000in}}%
\pgfusepath{stroke,fill}%
}%
\begin{pgfscope}%
\pgfsys@transformshift{0.721913in}{1.946479in}%
\pgfsys@useobject{currentmarker}{}%
\end{pgfscope}%
\end{pgfscope}%
\begin{pgfscope}%
\definecolor{textcolor}{rgb}{0.000000,0.000000,0.000000}%
\pgfsetstrokecolor{textcolor}%
\pgfsetfillcolor{textcolor}%
\pgftext[x=0.303703in, y=1.888608in, left, base]{\color{textcolor}{\rmfamily\fontsize{12.000000}{14.400000}\selectfont\catcode`\^=\active\def^{\ifmmode\sp\else\^{}\fi}\catcode`\%=\active\def%{\%}$\mathdefault{10^{-1}}$}}%
\end{pgfscope}%
\begin{pgfscope}%
\pgfsetbuttcap%
\pgfsetroundjoin%
\definecolor{currentfill}{rgb}{0.000000,0.000000,0.000000}%
\pgfsetfillcolor{currentfill}%
\pgfsetlinewidth{0.602250pt}%
\definecolor{currentstroke}{rgb}{0.000000,0.000000,0.000000}%
\pgfsetstrokecolor{currentstroke}%
\pgfsetdash{}{0pt}%
\pgfsys@defobject{currentmarker}{\pgfqpoint{-0.027778in}{0.000000in}}{\pgfqpoint{-0.000000in}{0.000000in}}{%
\pgfpathmoveto{\pgfqpoint{-0.000000in}{0.000000in}}%
\pgfpathlineto{\pgfqpoint{-0.027778in}{0.000000in}}%
\pgfusepath{stroke,fill}%
}%
\begin{pgfscope}%
\pgfsys@transformshift{0.721913in}{0.608466in}%
\pgfsys@useobject{currentmarker}{}%
\end{pgfscope}%
\end{pgfscope}%
\begin{pgfscope}%
\pgfsetbuttcap%
\pgfsetroundjoin%
\definecolor{currentfill}{rgb}{0.000000,0.000000,0.000000}%
\pgfsetfillcolor{currentfill}%
\pgfsetlinewidth{0.602250pt}%
\definecolor{currentstroke}{rgb}{0.000000,0.000000,0.000000}%
\pgfsetstrokecolor{currentstroke}%
\pgfsetdash{}{0pt}%
\pgfsys@defobject{currentmarker}{\pgfqpoint{-0.027778in}{0.000000in}}{\pgfqpoint{-0.000000in}{0.000000in}}{%
\pgfpathmoveto{\pgfqpoint{-0.000000in}{0.000000in}}%
\pgfpathlineto{\pgfqpoint{-0.027778in}{0.000000in}}%
\pgfusepath{stroke,fill}%
}%
\begin{pgfscope}%
\pgfsys@transformshift{0.721913in}{0.674727in}%
\pgfsys@useobject{currentmarker}{}%
\end{pgfscope}%
\end{pgfscope}%
\begin{pgfscope}%
\pgfsetbuttcap%
\pgfsetroundjoin%
\definecolor{currentfill}{rgb}{0.000000,0.000000,0.000000}%
\pgfsetfillcolor{currentfill}%
\pgfsetlinewidth{0.602250pt}%
\definecolor{currentstroke}{rgb}{0.000000,0.000000,0.000000}%
\pgfsetstrokecolor{currentstroke}%
\pgfsetdash{}{0pt}%
\pgfsys@defobject{currentmarker}{\pgfqpoint{-0.027778in}{0.000000in}}{\pgfqpoint{-0.000000in}{0.000000in}}{%
\pgfpathmoveto{\pgfqpoint{-0.000000in}{0.000000in}}%
\pgfpathlineto{\pgfqpoint{-0.027778in}{0.000000in}}%
\pgfusepath{stroke,fill}%
}%
\begin{pgfscope}%
\pgfsys@transformshift{0.721913in}{0.726124in}%
\pgfsys@useobject{currentmarker}{}%
\end{pgfscope}%
\end{pgfscope}%
\begin{pgfscope}%
\pgfsetbuttcap%
\pgfsetroundjoin%
\definecolor{currentfill}{rgb}{0.000000,0.000000,0.000000}%
\pgfsetfillcolor{currentfill}%
\pgfsetlinewidth{0.602250pt}%
\definecolor{currentstroke}{rgb}{0.000000,0.000000,0.000000}%
\pgfsetstrokecolor{currentstroke}%
\pgfsetdash{}{0pt}%
\pgfsys@defobject{currentmarker}{\pgfqpoint{-0.027778in}{0.000000in}}{\pgfqpoint{-0.000000in}{0.000000in}}{%
\pgfpathmoveto{\pgfqpoint{-0.000000in}{0.000000in}}%
\pgfpathlineto{\pgfqpoint{-0.027778in}{0.000000in}}%
\pgfusepath{stroke,fill}%
}%
\begin{pgfscope}%
\pgfsys@transformshift{0.721913in}{0.768118in}%
\pgfsys@useobject{currentmarker}{}%
\end{pgfscope}%
\end{pgfscope}%
\begin{pgfscope}%
\pgfsetbuttcap%
\pgfsetroundjoin%
\definecolor{currentfill}{rgb}{0.000000,0.000000,0.000000}%
\pgfsetfillcolor{currentfill}%
\pgfsetlinewidth{0.602250pt}%
\definecolor{currentstroke}{rgb}{0.000000,0.000000,0.000000}%
\pgfsetstrokecolor{currentstroke}%
\pgfsetdash{}{0pt}%
\pgfsys@defobject{currentmarker}{\pgfqpoint{-0.027778in}{0.000000in}}{\pgfqpoint{-0.000000in}{0.000000in}}{%
\pgfpathmoveto{\pgfqpoint{-0.000000in}{0.000000in}}%
\pgfpathlineto{\pgfqpoint{-0.027778in}{0.000000in}}%
\pgfusepath{stroke,fill}%
}%
\begin{pgfscope}%
\pgfsys@transformshift{0.721913in}{0.803623in}%
\pgfsys@useobject{currentmarker}{}%
\end{pgfscope}%
\end{pgfscope}%
\begin{pgfscope}%
\pgfsetbuttcap%
\pgfsetroundjoin%
\definecolor{currentfill}{rgb}{0.000000,0.000000,0.000000}%
\pgfsetfillcolor{currentfill}%
\pgfsetlinewidth{0.602250pt}%
\definecolor{currentstroke}{rgb}{0.000000,0.000000,0.000000}%
\pgfsetstrokecolor{currentstroke}%
\pgfsetdash{}{0pt}%
\pgfsys@defobject{currentmarker}{\pgfqpoint{-0.027778in}{0.000000in}}{\pgfqpoint{-0.000000in}{0.000000in}}{%
\pgfpathmoveto{\pgfqpoint{-0.000000in}{0.000000in}}%
\pgfpathlineto{\pgfqpoint{-0.027778in}{0.000000in}}%
\pgfusepath{stroke,fill}%
}%
\begin{pgfscope}%
\pgfsys@transformshift{0.721913in}{0.834379in}%
\pgfsys@useobject{currentmarker}{}%
\end{pgfscope}%
\end{pgfscope}%
\begin{pgfscope}%
\pgfsetbuttcap%
\pgfsetroundjoin%
\definecolor{currentfill}{rgb}{0.000000,0.000000,0.000000}%
\pgfsetfillcolor{currentfill}%
\pgfsetlinewidth{0.602250pt}%
\definecolor{currentstroke}{rgb}{0.000000,0.000000,0.000000}%
\pgfsetstrokecolor{currentstroke}%
\pgfsetdash{}{0pt}%
\pgfsys@defobject{currentmarker}{\pgfqpoint{-0.027778in}{0.000000in}}{\pgfqpoint{-0.000000in}{0.000000in}}{%
\pgfpathmoveto{\pgfqpoint{-0.000000in}{0.000000in}}%
\pgfpathlineto{\pgfqpoint{-0.027778in}{0.000000in}}%
\pgfusepath{stroke,fill}%
}%
\begin{pgfscope}%
\pgfsys@transformshift{0.721913in}{0.861508in}%
\pgfsys@useobject{currentmarker}{}%
\end{pgfscope}%
\end{pgfscope}%
\begin{pgfscope}%
\pgfsetbuttcap%
\pgfsetroundjoin%
\definecolor{currentfill}{rgb}{0.000000,0.000000,0.000000}%
\pgfsetfillcolor{currentfill}%
\pgfsetlinewidth{0.602250pt}%
\definecolor{currentstroke}{rgb}{0.000000,0.000000,0.000000}%
\pgfsetstrokecolor{currentstroke}%
\pgfsetdash{}{0pt}%
\pgfsys@defobject{currentmarker}{\pgfqpoint{-0.027778in}{0.000000in}}{\pgfqpoint{-0.000000in}{0.000000in}}{%
\pgfpathmoveto{\pgfqpoint{-0.000000in}{0.000000in}}%
\pgfpathlineto{\pgfqpoint{-0.027778in}{0.000000in}}%
\pgfusepath{stroke,fill}%
}%
\begin{pgfscope}%
\pgfsys@transformshift{0.721913in}{1.045427in}%
\pgfsys@useobject{currentmarker}{}%
\end{pgfscope}%
\end{pgfscope}%
\begin{pgfscope}%
\pgfsetbuttcap%
\pgfsetroundjoin%
\definecolor{currentfill}{rgb}{0.000000,0.000000,0.000000}%
\pgfsetfillcolor{currentfill}%
\pgfsetlinewidth{0.602250pt}%
\definecolor{currentstroke}{rgb}{0.000000,0.000000,0.000000}%
\pgfsetstrokecolor{currentstroke}%
\pgfsetdash{}{0pt}%
\pgfsys@defobject{currentmarker}{\pgfqpoint{-0.027778in}{0.000000in}}{\pgfqpoint{-0.000000in}{0.000000in}}{%
\pgfpathmoveto{\pgfqpoint{-0.000000in}{0.000000in}}%
\pgfpathlineto{\pgfqpoint{-0.027778in}{0.000000in}}%
\pgfusepath{stroke,fill}%
}%
\begin{pgfscope}%
\pgfsys@transformshift{0.721913in}{1.138817in}%
\pgfsys@useobject{currentmarker}{}%
\end{pgfscope}%
\end{pgfscope}%
\begin{pgfscope}%
\pgfsetbuttcap%
\pgfsetroundjoin%
\definecolor{currentfill}{rgb}{0.000000,0.000000,0.000000}%
\pgfsetfillcolor{currentfill}%
\pgfsetlinewidth{0.602250pt}%
\definecolor{currentstroke}{rgb}{0.000000,0.000000,0.000000}%
\pgfsetstrokecolor{currentstroke}%
\pgfsetdash{}{0pt}%
\pgfsys@defobject{currentmarker}{\pgfqpoint{-0.027778in}{0.000000in}}{\pgfqpoint{-0.000000in}{0.000000in}}{%
\pgfpathmoveto{\pgfqpoint{-0.000000in}{0.000000in}}%
\pgfpathlineto{\pgfqpoint{-0.027778in}{0.000000in}}%
\pgfusepath{stroke,fill}%
}%
\begin{pgfscope}%
\pgfsys@transformshift{0.721913in}{1.205079in}%
\pgfsys@useobject{currentmarker}{}%
\end{pgfscope}%
\end{pgfscope}%
\begin{pgfscope}%
\pgfsetbuttcap%
\pgfsetroundjoin%
\definecolor{currentfill}{rgb}{0.000000,0.000000,0.000000}%
\pgfsetfillcolor{currentfill}%
\pgfsetlinewidth{0.602250pt}%
\definecolor{currentstroke}{rgb}{0.000000,0.000000,0.000000}%
\pgfsetstrokecolor{currentstroke}%
\pgfsetdash{}{0pt}%
\pgfsys@defobject{currentmarker}{\pgfqpoint{-0.027778in}{0.000000in}}{\pgfqpoint{-0.000000in}{0.000000in}}{%
\pgfpathmoveto{\pgfqpoint{-0.000000in}{0.000000in}}%
\pgfpathlineto{\pgfqpoint{-0.027778in}{0.000000in}}%
\pgfusepath{stroke,fill}%
}%
\begin{pgfscope}%
\pgfsys@transformshift{0.721913in}{1.256475in}%
\pgfsys@useobject{currentmarker}{}%
\end{pgfscope}%
\end{pgfscope}%
\begin{pgfscope}%
\pgfsetbuttcap%
\pgfsetroundjoin%
\definecolor{currentfill}{rgb}{0.000000,0.000000,0.000000}%
\pgfsetfillcolor{currentfill}%
\pgfsetlinewidth{0.602250pt}%
\definecolor{currentstroke}{rgb}{0.000000,0.000000,0.000000}%
\pgfsetstrokecolor{currentstroke}%
\pgfsetdash{}{0pt}%
\pgfsys@defobject{currentmarker}{\pgfqpoint{-0.027778in}{0.000000in}}{\pgfqpoint{-0.000000in}{0.000000in}}{%
\pgfpathmoveto{\pgfqpoint{-0.000000in}{0.000000in}}%
\pgfpathlineto{\pgfqpoint{-0.027778in}{0.000000in}}%
\pgfusepath{stroke,fill}%
}%
\begin{pgfscope}%
\pgfsys@transformshift{0.721913in}{1.298469in}%
\pgfsys@useobject{currentmarker}{}%
\end{pgfscope}%
\end{pgfscope}%
\begin{pgfscope}%
\pgfsetbuttcap%
\pgfsetroundjoin%
\definecolor{currentfill}{rgb}{0.000000,0.000000,0.000000}%
\pgfsetfillcolor{currentfill}%
\pgfsetlinewidth{0.602250pt}%
\definecolor{currentstroke}{rgb}{0.000000,0.000000,0.000000}%
\pgfsetstrokecolor{currentstroke}%
\pgfsetdash{}{0pt}%
\pgfsys@defobject{currentmarker}{\pgfqpoint{-0.027778in}{0.000000in}}{\pgfqpoint{-0.000000in}{0.000000in}}{%
\pgfpathmoveto{\pgfqpoint{-0.000000in}{0.000000in}}%
\pgfpathlineto{\pgfqpoint{-0.027778in}{0.000000in}}%
\pgfusepath{stroke,fill}%
}%
\begin{pgfscope}%
\pgfsys@transformshift{0.721913in}{1.333974in}%
\pgfsys@useobject{currentmarker}{}%
\end{pgfscope}%
\end{pgfscope}%
\begin{pgfscope}%
\pgfsetbuttcap%
\pgfsetroundjoin%
\definecolor{currentfill}{rgb}{0.000000,0.000000,0.000000}%
\pgfsetfillcolor{currentfill}%
\pgfsetlinewidth{0.602250pt}%
\definecolor{currentstroke}{rgb}{0.000000,0.000000,0.000000}%
\pgfsetstrokecolor{currentstroke}%
\pgfsetdash{}{0pt}%
\pgfsys@defobject{currentmarker}{\pgfqpoint{-0.027778in}{0.000000in}}{\pgfqpoint{-0.000000in}{0.000000in}}{%
\pgfpathmoveto{\pgfqpoint{-0.000000in}{0.000000in}}%
\pgfpathlineto{\pgfqpoint{-0.027778in}{0.000000in}}%
\pgfusepath{stroke,fill}%
}%
\begin{pgfscope}%
\pgfsys@transformshift{0.721913in}{1.364731in}%
\pgfsys@useobject{currentmarker}{}%
\end{pgfscope}%
\end{pgfscope}%
\begin{pgfscope}%
\pgfsetbuttcap%
\pgfsetroundjoin%
\definecolor{currentfill}{rgb}{0.000000,0.000000,0.000000}%
\pgfsetfillcolor{currentfill}%
\pgfsetlinewidth{0.602250pt}%
\definecolor{currentstroke}{rgb}{0.000000,0.000000,0.000000}%
\pgfsetstrokecolor{currentstroke}%
\pgfsetdash{}{0pt}%
\pgfsys@defobject{currentmarker}{\pgfqpoint{-0.027778in}{0.000000in}}{\pgfqpoint{-0.000000in}{0.000000in}}{%
\pgfpathmoveto{\pgfqpoint{-0.000000in}{0.000000in}}%
\pgfpathlineto{\pgfqpoint{-0.027778in}{0.000000in}}%
\pgfusepath{stroke,fill}%
}%
\begin{pgfscope}%
\pgfsys@transformshift{0.721913in}{1.391859in}%
\pgfsys@useobject{currentmarker}{}%
\end{pgfscope}%
\end{pgfscope}%
\begin{pgfscope}%
\pgfsetbuttcap%
\pgfsetroundjoin%
\definecolor{currentfill}{rgb}{0.000000,0.000000,0.000000}%
\pgfsetfillcolor{currentfill}%
\pgfsetlinewidth{0.602250pt}%
\definecolor{currentstroke}{rgb}{0.000000,0.000000,0.000000}%
\pgfsetstrokecolor{currentstroke}%
\pgfsetdash{}{0pt}%
\pgfsys@defobject{currentmarker}{\pgfqpoint{-0.027778in}{0.000000in}}{\pgfqpoint{-0.000000in}{0.000000in}}{%
\pgfpathmoveto{\pgfqpoint{-0.000000in}{0.000000in}}%
\pgfpathlineto{\pgfqpoint{-0.027778in}{0.000000in}}%
\pgfusepath{stroke,fill}%
}%
\begin{pgfscope}%
\pgfsys@transformshift{0.721913in}{1.575779in}%
\pgfsys@useobject{currentmarker}{}%
\end{pgfscope}%
\end{pgfscope}%
\begin{pgfscope}%
\pgfsetbuttcap%
\pgfsetroundjoin%
\definecolor{currentfill}{rgb}{0.000000,0.000000,0.000000}%
\pgfsetfillcolor{currentfill}%
\pgfsetlinewidth{0.602250pt}%
\definecolor{currentstroke}{rgb}{0.000000,0.000000,0.000000}%
\pgfsetstrokecolor{currentstroke}%
\pgfsetdash{}{0pt}%
\pgfsys@defobject{currentmarker}{\pgfqpoint{-0.027778in}{0.000000in}}{\pgfqpoint{-0.000000in}{0.000000in}}{%
\pgfpathmoveto{\pgfqpoint{-0.000000in}{0.000000in}}%
\pgfpathlineto{\pgfqpoint{-0.027778in}{0.000000in}}%
\pgfusepath{stroke,fill}%
}%
\begin{pgfscope}%
\pgfsys@transformshift{0.721913in}{1.669169in}%
\pgfsys@useobject{currentmarker}{}%
\end{pgfscope}%
\end{pgfscope}%
\begin{pgfscope}%
\pgfsetbuttcap%
\pgfsetroundjoin%
\definecolor{currentfill}{rgb}{0.000000,0.000000,0.000000}%
\pgfsetfillcolor{currentfill}%
\pgfsetlinewidth{0.602250pt}%
\definecolor{currentstroke}{rgb}{0.000000,0.000000,0.000000}%
\pgfsetstrokecolor{currentstroke}%
\pgfsetdash{}{0pt}%
\pgfsys@defobject{currentmarker}{\pgfqpoint{-0.027778in}{0.000000in}}{\pgfqpoint{-0.000000in}{0.000000in}}{%
\pgfpathmoveto{\pgfqpoint{-0.000000in}{0.000000in}}%
\pgfpathlineto{\pgfqpoint{-0.027778in}{0.000000in}}%
\pgfusepath{stroke,fill}%
}%
\begin{pgfscope}%
\pgfsys@transformshift{0.721913in}{1.735430in}%
\pgfsys@useobject{currentmarker}{}%
\end{pgfscope}%
\end{pgfscope}%
\begin{pgfscope}%
\pgfsetbuttcap%
\pgfsetroundjoin%
\definecolor{currentfill}{rgb}{0.000000,0.000000,0.000000}%
\pgfsetfillcolor{currentfill}%
\pgfsetlinewidth{0.602250pt}%
\definecolor{currentstroke}{rgb}{0.000000,0.000000,0.000000}%
\pgfsetstrokecolor{currentstroke}%
\pgfsetdash{}{0pt}%
\pgfsys@defobject{currentmarker}{\pgfqpoint{-0.027778in}{0.000000in}}{\pgfqpoint{-0.000000in}{0.000000in}}{%
\pgfpathmoveto{\pgfqpoint{-0.000000in}{0.000000in}}%
\pgfpathlineto{\pgfqpoint{-0.027778in}{0.000000in}}%
\pgfusepath{stroke,fill}%
}%
\begin{pgfscope}%
\pgfsys@transformshift{0.721913in}{1.786827in}%
\pgfsys@useobject{currentmarker}{}%
\end{pgfscope}%
\end{pgfscope}%
\begin{pgfscope}%
\pgfsetbuttcap%
\pgfsetroundjoin%
\definecolor{currentfill}{rgb}{0.000000,0.000000,0.000000}%
\pgfsetfillcolor{currentfill}%
\pgfsetlinewidth{0.602250pt}%
\definecolor{currentstroke}{rgb}{0.000000,0.000000,0.000000}%
\pgfsetstrokecolor{currentstroke}%
\pgfsetdash{}{0pt}%
\pgfsys@defobject{currentmarker}{\pgfqpoint{-0.027778in}{0.000000in}}{\pgfqpoint{-0.000000in}{0.000000in}}{%
\pgfpathmoveto{\pgfqpoint{-0.000000in}{0.000000in}}%
\pgfpathlineto{\pgfqpoint{-0.027778in}{0.000000in}}%
\pgfusepath{stroke,fill}%
}%
\begin{pgfscope}%
\pgfsys@transformshift{0.721913in}{1.828821in}%
\pgfsys@useobject{currentmarker}{}%
\end{pgfscope}%
\end{pgfscope}%
\begin{pgfscope}%
\pgfsetbuttcap%
\pgfsetroundjoin%
\definecolor{currentfill}{rgb}{0.000000,0.000000,0.000000}%
\pgfsetfillcolor{currentfill}%
\pgfsetlinewidth{0.602250pt}%
\definecolor{currentstroke}{rgb}{0.000000,0.000000,0.000000}%
\pgfsetstrokecolor{currentstroke}%
\pgfsetdash{}{0pt}%
\pgfsys@defobject{currentmarker}{\pgfqpoint{-0.027778in}{0.000000in}}{\pgfqpoint{-0.000000in}{0.000000in}}{%
\pgfpathmoveto{\pgfqpoint{-0.000000in}{0.000000in}}%
\pgfpathlineto{\pgfqpoint{-0.027778in}{0.000000in}}%
\pgfusepath{stroke,fill}%
}%
\begin{pgfscope}%
\pgfsys@transformshift{0.721913in}{1.864326in}%
\pgfsys@useobject{currentmarker}{}%
\end{pgfscope}%
\end{pgfscope}%
\begin{pgfscope}%
\pgfsetbuttcap%
\pgfsetroundjoin%
\definecolor{currentfill}{rgb}{0.000000,0.000000,0.000000}%
\pgfsetfillcolor{currentfill}%
\pgfsetlinewidth{0.602250pt}%
\definecolor{currentstroke}{rgb}{0.000000,0.000000,0.000000}%
\pgfsetstrokecolor{currentstroke}%
\pgfsetdash{}{0pt}%
\pgfsys@defobject{currentmarker}{\pgfqpoint{-0.027778in}{0.000000in}}{\pgfqpoint{-0.000000in}{0.000000in}}{%
\pgfpathmoveto{\pgfqpoint{-0.000000in}{0.000000in}}%
\pgfpathlineto{\pgfqpoint{-0.027778in}{0.000000in}}%
\pgfusepath{stroke,fill}%
}%
\begin{pgfscope}%
\pgfsys@transformshift{0.721913in}{1.895082in}%
\pgfsys@useobject{currentmarker}{}%
\end{pgfscope}%
\end{pgfscope}%
\begin{pgfscope}%
\pgfsetbuttcap%
\pgfsetroundjoin%
\definecolor{currentfill}{rgb}{0.000000,0.000000,0.000000}%
\pgfsetfillcolor{currentfill}%
\pgfsetlinewidth{0.602250pt}%
\definecolor{currentstroke}{rgb}{0.000000,0.000000,0.000000}%
\pgfsetstrokecolor{currentstroke}%
\pgfsetdash{}{0pt}%
\pgfsys@defobject{currentmarker}{\pgfqpoint{-0.027778in}{0.000000in}}{\pgfqpoint{-0.000000in}{0.000000in}}{%
\pgfpathmoveto{\pgfqpoint{-0.000000in}{0.000000in}}%
\pgfpathlineto{\pgfqpoint{-0.027778in}{0.000000in}}%
\pgfusepath{stroke,fill}%
}%
\begin{pgfscope}%
\pgfsys@transformshift{0.721913in}{1.922211in}%
\pgfsys@useobject{currentmarker}{}%
\end{pgfscope}%
\end{pgfscope}%
\begin{pgfscope}%
\pgfsetbuttcap%
\pgfsetroundjoin%
\definecolor{currentfill}{rgb}{0.000000,0.000000,0.000000}%
\pgfsetfillcolor{currentfill}%
\pgfsetlinewidth{0.602250pt}%
\definecolor{currentstroke}{rgb}{0.000000,0.000000,0.000000}%
\pgfsetstrokecolor{currentstroke}%
\pgfsetdash{}{0pt}%
\pgfsys@defobject{currentmarker}{\pgfqpoint{-0.027778in}{0.000000in}}{\pgfqpoint{-0.000000in}{0.000000in}}{%
\pgfpathmoveto{\pgfqpoint{-0.000000in}{0.000000in}}%
\pgfpathlineto{\pgfqpoint{-0.027778in}{0.000000in}}%
\pgfusepath{stroke,fill}%
}%
\begin{pgfscope}%
\pgfsys@transformshift{0.721913in}{2.106130in}%
\pgfsys@useobject{currentmarker}{}%
\end{pgfscope}%
\end{pgfscope}%
\begin{pgfscope}%
\pgfsetbuttcap%
\pgfsetroundjoin%
\definecolor{currentfill}{rgb}{0.000000,0.000000,0.000000}%
\pgfsetfillcolor{currentfill}%
\pgfsetlinewidth{0.602250pt}%
\definecolor{currentstroke}{rgb}{0.000000,0.000000,0.000000}%
\pgfsetstrokecolor{currentstroke}%
\pgfsetdash{}{0pt}%
\pgfsys@defobject{currentmarker}{\pgfqpoint{-0.027778in}{0.000000in}}{\pgfqpoint{-0.000000in}{0.000000in}}{%
\pgfpathmoveto{\pgfqpoint{-0.000000in}{0.000000in}}%
\pgfpathlineto{\pgfqpoint{-0.027778in}{0.000000in}}%
\pgfusepath{stroke,fill}%
}%
\begin{pgfscope}%
\pgfsys@transformshift{0.721913in}{2.199521in}%
\pgfsys@useobject{currentmarker}{}%
\end{pgfscope}%
\end{pgfscope}%
\begin{pgfscope}%
\pgfsetbuttcap%
\pgfsetroundjoin%
\definecolor{currentfill}{rgb}{0.000000,0.000000,0.000000}%
\pgfsetfillcolor{currentfill}%
\pgfsetlinewidth{0.602250pt}%
\definecolor{currentstroke}{rgb}{0.000000,0.000000,0.000000}%
\pgfsetstrokecolor{currentstroke}%
\pgfsetdash{}{0pt}%
\pgfsys@defobject{currentmarker}{\pgfqpoint{-0.027778in}{0.000000in}}{\pgfqpoint{-0.000000in}{0.000000in}}{%
\pgfpathmoveto{\pgfqpoint{-0.000000in}{0.000000in}}%
\pgfpathlineto{\pgfqpoint{-0.027778in}{0.000000in}}%
\pgfusepath{stroke,fill}%
}%
\begin{pgfscope}%
\pgfsys@transformshift{0.721913in}{2.265782in}%
\pgfsys@useobject{currentmarker}{}%
\end{pgfscope}%
\end{pgfscope}%
\begin{pgfscope}%
\pgfsetbuttcap%
\pgfsetroundjoin%
\definecolor{currentfill}{rgb}{0.000000,0.000000,0.000000}%
\pgfsetfillcolor{currentfill}%
\pgfsetlinewidth{0.602250pt}%
\definecolor{currentstroke}{rgb}{0.000000,0.000000,0.000000}%
\pgfsetstrokecolor{currentstroke}%
\pgfsetdash{}{0pt}%
\pgfsys@defobject{currentmarker}{\pgfqpoint{-0.027778in}{0.000000in}}{\pgfqpoint{-0.000000in}{0.000000in}}{%
\pgfpathmoveto{\pgfqpoint{-0.000000in}{0.000000in}}%
\pgfpathlineto{\pgfqpoint{-0.027778in}{0.000000in}}%
\pgfusepath{stroke,fill}%
}%
\begin{pgfscope}%
\pgfsys@transformshift{0.721913in}{2.317178in}%
\pgfsys@useobject{currentmarker}{}%
\end{pgfscope}%
\end{pgfscope}%
\begin{pgfscope}%
\pgfsetbuttcap%
\pgfsetroundjoin%
\definecolor{currentfill}{rgb}{0.000000,0.000000,0.000000}%
\pgfsetfillcolor{currentfill}%
\pgfsetlinewidth{0.602250pt}%
\definecolor{currentstroke}{rgb}{0.000000,0.000000,0.000000}%
\pgfsetstrokecolor{currentstroke}%
\pgfsetdash{}{0pt}%
\pgfsys@defobject{currentmarker}{\pgfqpoint{-0.027778in}{0.000000in}}{\pgfqpoint{-0.000000in}{0.000000in}}{%
\pgfpathmoveto{\pgfqpoint{-0.000000in}{0.000000in}}%
\pgfpathlineto{\pgfqpoint{-0.027778in}{0.000000in}}%
\pgfusepath{stroke,fill}%
}%
\begin{pgfscope}%
\pgfsys@transformshift{0.721913in}{2.359172in}%
\pgfsys@useobject{currentmarker}{}%
\end{pgfscope}%
\end{pgfscope}%
\begin{pgfscope}%
\pgfsetbuttcap%
\pgfsetroundjoin%
\definecolor{currentfill}{rgb}{0.000000,0.000000,0.000000}%
\pgfsetfillcolor{currentfill}%
\pgfsetlinewidth{0.602250pt}%
\definecolor{currentstroke}{rgb}{0.000000,0.000000,0.000000}%
\pgfsetstrokecolor{currentstroke}%
\pgfsetdash{}{0pt}%
\pgfsys@defobject{currentmarker}{\pgfqpoint{-0.027778in}{0.000000in}}{\pgfqpoint{-0.000000in}{0.000000in}}{%
\pgfpathmoveto{\pgfqpoint{-0.000000in}{0.000000in}}%
\pgfpathlineto{\pgfqpoint{-0.027778in}{0.000000in}}%
\pgfusepath{stroke,fill}%
}%
\begin{pgfscope}%
\pgfsys@transformshift{0.721913in}{2.394678in}%
\pgfsys@useobject{currentmarker}{}%
\end{pgfscope}%
\end{pgfscope}%
\begin{pgfscope}%
\pgfsetbuttcap%
\pgfsetroundjoin%
\definecolor{currentfill}{rgb}{0.000000,0.000000,0.000000}%
\pgfsetfillcolor{currentfill}%
\pgfsetlinewidth{0.602250pt}%
\definecolor{currentstroke}{rgb}{0.000000,0.000000,0.000000}%
\pgfsetstrokecolor{currentstroke}%
\pgfsetdash{}{0pt}%
\pgfsys@defobject{currentmarker}{\pgfqpoint{-0.027778in}{0.000000in}}{\pgfqpoint{-0.000000in}{0.000000in}}{%
\pgfpathmoveto{\pgfqpoint{-0.000000in}{0.000000in}}%
\pgfpathlineto{\pgfqpoint{-0.027778in}{0.000000in}}%
\pgfusepath{stroke,fill}%
}%
\begin{pgfscope}%
\pgfsys@transformshift{0.721913in}{2.425434in}%
\pgfsys@useobject{currentmarker}{}%
\end{pgfscope}%
\end{pgfscope}%
\begin{pgfscope}%
\pgfsetbuttcap%
\pgfsetroundjoin%
\definecolor{currentfill}{rgb}{0.000000,0.000000,0.000000}%
\pgfsetfillcolor{currentfill}%
\pgfsetlinewidth{0.602250pt}%
\definecolor{currentstroke}{rgb}{0.000000,0.000000,0.000000}%
\pgfsetstrokecolor{currentstroke}%
\pgfsetdash{}{0pt}%
\pgfsys@defobject{currentmarker}{\pgfqpoint{-0.027778in}{0.000000in}}{\pgfqpoint{-0.000000in}{0.000000in}}{%
\pgfpathmoveto{\pgfqpoint{-0.000000in}{0.000000in}}%
\pgfpathlineto{\pgfqpoint{-0.027778in}{0.000000in}}%
\pgfusepath{stroke,fill}%
}%
\begin{pgfscope}%
\pgfsys@transformshift{0.721913in}{2.452563in}%
\pgfsys@useobject{currentmarker}{}%
\end{pgfscope}%
\end{pgfscope}%
\begin{pgfscope}%
\definecolor{textcolor}{rgb}{0.000000,0.000000,0.000000}%
\pgfsetstrokecolor{textcolor}%
\pgfsetfillcolor{textcolor}%
\pgftext[x=0.248148in,y=1.511573in,,bottom,rotate=90.000000]{\color{textcolor}{\rmfamily\fontsize{12.000000}{14.400000}\selectfont\catcode`\^=\active\def^{\ifmmode\sp\else\^{}\fi}\catcode`\%=\active\def%{\%}$L^1$ relative error}}%
\end{pgfscope}%
\begin{pgfscope}%
\pgfpathrectangle{\pgfqpoint{0.721913in}{0.549073in}}{\pgfqpoint{1.937500in}{1.925000in}}%
\pgfusepath{clip}%
\pgfsetbuttcap%
\pgfsetroundjoin%
\pgfsetlinewidth{1.505625pt}%
\definecolor{currentstroke}{rgb}{0.478431,0.478431,0.478431}%
\pgfsetstrokecolor{currentstroke}%
\pgfsetstrokeopacity{0.500000}%
\pgfsetdash{{5.550000pt}{2.400000pt}}{0.000000pt}%
\pgfpathmoveto{\pgfqpoint{0.809982in}{1.898653in}}%
\pgfpathlineto{\pgfqpoint{1.122467in}{1.739002in}}%
\pgfpathlineto{\pgfqpoint{1.420640in}{1.586662in}}%
\pgfpathlineto{\pgfqpoint{1.710766in}{1.438434in}}%
\pgfpathlineto{\pgfqpoint{1.999725in}{1.290802in}}%
\pgfpathlineto{\pgfqpoint{2.285257in}{1.144922in}}%
\pgfpathlineto{\pgfqpoint{2.571345in}{0.998756in}}%
\pgfusepath{stroke}%
\end{pgfscope}%
\begin{pgfscope}%
\pgfpathrectangle{\pgfqpoint{0.721913in}{0.549073in}}{\pgfqpoint{1.937500in}{1.925000in}}%
\pgfusepath{clip}%
\pgfsetbuttcap%
\pgfsetroundjoin%
\pgfsetlinewidth{1.505625pt}%
\definecolor{currentstroke}{rgb}{0.478431,0.478431,0.478431}%
\pgfsetstrokecolor{currentstroke}%
\pgfsetstrokeopacity{0.500000}%
\pgfsetdash{{5.550000pt}{2.400000pt}}{0.000000pt}%
\pgfpathmoveto{\pgfqpoint{0.809982in}{1.960043in}}%
\pgfpathlineto{\pgfqpoint{1.122467in}{1.880217in}}%
\pgfpathlineto{\pgfqpoint{1.420640in}{1.804048in}}%
\pgfpathlineto{\pgfqpoint{1.710766in}{1.729934in}}%
\pgfpathlineto{\pgfqpoint{1.999725in}{1.656118in}}%
\pgfpathlineto{\pgfqpoint{2.285257in}{1.583177in}}%
\pgfpathlineto{\pgfqpoint{2.571345in}{1.510095in}}%
\pgfusepath{stroke}%
\end{pgfscope}%
\begin{pgfscope}%
\pgfpathrectangle{\pgfqpoint{0.721913in}{0.549073in}}{\pgfqpoint{1.937500in}{1.925000in}}%
\pgfusepath{clip}%
\pgfsetrectcap%
\pgfsetroundjoin%
\pgfsetlinewidth{1.003750pt}%
\definecolor{currentstroke}{rgb}{0.537255,0.647059,0.760784}%
\pgfsetstrokecolor{currentstroke}%
\pgfsetdash{}{0pt}%
\pgfpathmoveto{\pgfqpoint{0.809982in}{1.861431in}}%
\pgfpathlineto{\pgfqpoint{1.122467in}{1.833444in}}%
\pgfpathlineto{\pgfqpoint{1.420640in}{1.761787in}}%
\pgfpathlineto{\pgfqpoint{1.710766in}{1.692700in}}%
\pgfpathlineto{\pgfqpoint{1.999725in}{1.619072in}}%
\pgfpathlineto{\pgfqpoint{2.285257in}{1.534539in}}%
\pgfpathlineto{\pgfqpoint{2.571345in}{1.474868in}}%
\pgfusepath{stroke}%
\end{pgfscope}%
\begin{pgfscope}%
\pgfpathrectangle{\pgfqpoint{0.721913in}{0.549073in}}{\pgfqpoint{1.937500in}{1.925000in}}%
\pgfusepath{clip}%
\pgfsetbuttcap%
\pgfsetroundjoin%
\definecolor{currentfill}{rgb}{0.537255,0.647059,0.760784}%
\pgfsetfillcolor{currentfill}%
\pgfsetlinewidth{1.003750pt}%
\definecolor{currentstroke}{rgb}{0.537255,0.647059,0.760784}%
\pgfsetstrokecolor{currentstroke}%
\pgfsetdash{}{0pt}%
\pgfsys@defobject{currentmarker}{\pgfqpoint{-0.020833in}{-0.020833in}}{\pgfqpoint{0.020833in}{0.020833in}}{%
\pgfpathmoveto{\pgfqpoint{0.000000in}{-0.020833in}}%
\pgfpathcurveto{\pgfqpoint{0.005525in}{-0.020833in}}{\pgfqpoint{0.010825in}{-0.018638in}}{\pgfqpoint{0.014731in}{-0.014731in}}%
\pgfpathcurveto{\pgfqpoint{0.018638in}{-0.010825in}}{\pgfqpoint{0.020833in}{-0.005525in}}{\pgfqpoint{0.020833in}{0.000000in}}%
\pgfpathcurveto{\pgfqpoint{0.020833in}{0.005525in}}{\pgfqpoint{0.018638in}{0.010825in}}{\pgfqpoint{0.014731in}{0.014731in}}%
\pgfpathcurveto{\pgfqpoint{0.010825in}{0.018638in}}{\pgfqpoint{0.005525in}{0.020833in}}{\pgfqpoint{0.000000in}{0.020833in}}%
\pgfpathcurveto{\pgfqpoint{-0.005525in}{0.020833in}}{\pgfqpoint{-0.010825in}{0.018638in}}{\pgfqpoint{-0.014731in}{0.014731in}}%
\pgfpathcurveto{\pgfqpoint{-0.018638in}{0.010825in}}{\pgfqpoint{-0.020833in}{0.005525in}}{\pgfqpoint{-0.020833in}{0.000000in}}%
\pgfpathcurveto{\pgfqpoint{-0.020833in}{-0.005525in}}{\pgfqpoint{-0.018638in}{-0.010825in}}{\pgfqpoint{-0.014731in}{-0.014731in}}%
\pgfpathcurveto{\pgfqpoint{-0.010825in}{-0.018638in}}{\pgfqpoint{-0.005525in}{-0.020833in}}{\pgfqpoint{0.000000in}{-0.020833in}}%
\pgfpathlineto{\pgfqpoint{0.000000in}{-0.020833in}}%
\pgfpathclose%
\pgfusepath{stroke,fill}%
}%
\begin{pgfscope}%
\pgfsys@transformshift{0.809982in}{1.861431in}%
\pgfsys@useobject{currentmarker}{}%
\end{pgfscope}%
\begin{pgfscope}%
\pgfsys@transformshift{1.122467in}{1.833444in}%
\pgfsys@useobject{currentmarker}{}%
\end{pgfscope}%
\begin{pgfscope}%
\pgfsys@transformshift{1.420640in}{1.761787in}%
\pgfsys@useobject{currentmarker}{}%
\end{pgfscope}%
\begin{pgfscope}%
\pgfsys@transformshift{1.710766in}{1.692700in}%
\pgfsys@useobject{currentmarker}{}%
\end{pgfscope}%
\begin{pgfscope}%
\pgfsys@transformshift{1.999725in}{1.619072in}%
\pgfsys@useobject{currentmarker}{}%
\end{pgfscope}%
\begin{pgfscope}%
\pgfsys@transformshift{2.285257in}{1.534539in}%
\pgfsys@useobject{currentmarker}{}%
\end{pgfscope}%
\begin{pgfscope}%
\pgfsys@transformshift{2.571345in}{1.474868in}%
\pgfsys@useobject{currentmarker}{}%
\end{pgfscope}%
\end{pgfscope}%
\begin{pgfscope}%
\pgfpathrectangle{\pgfqpoint{0.721913in}{0.549073in}}{\pgfqpoint{1.937500in}{1.925000in}}%
\pgfusepath{clip}%
\pgfsetrectcap%
\pgfsetroundjoin%
\pgfsetlinewidth{1.003750pt}%
\definecolor{currentstroke}{rgb}{0.184314,0.270588,0.360784}%
\pgfsetstrokecolor{currentstroke}%
\pgfsetdash{}{0pt}%
\pgfpathmoveto{\pgfqpoint{0.809982in}{2.386573in}}%
\pgfpathlineto{\pgfqpoint{1.122467in}{2.307684in}}%
\pgfpathlineto{\pgfqpoint{1.420640in}{2.188938in}}%
\pgfpathlineto{\pgfqpoint{1.710766in}{2.017643in}}%
\pgfpathlineto{\pgfqpoint{1.999725in}{1.803614in}}%
\pgfpathlineto{\pgfqpoint{2.285257in}{1.613585in}}%
\pgfpathlineto{\pgfqpoint{2.571345in}{1.397561in}}%
\pgfusepath{stroke}%
\end{pgfscope}%
\begin{pgfscope}%
\pgfpathrectangle{\pgfqpoint{0.721913in}{0.549073in}}{\pgfqpoint{1.937500in}{1.925000in}}%
\pgfusepath{clip}%
\pgfsetbuttcap%
\pgfsetroundjoin%
\definecolor{currentfill}{rgb}{0.184314,0.270588,0.360784}%
\pgfsetfillcolor{currentfill}%
\pgfsetlinewidth{1.003750pt}%
\definecolor{currentstroke}{rgb}{0.184314,0.270588,0.360784}%
\pgfsetstrokecolor{currentstroke}%
\pgfsetdash{}{0pt}%
\pgfsys@defobject{currentmarker}{\pgfqpoint{-0.020833in}{-0.020833in}}{\pgfqpoint{0.020833in}{0.020833in}}{%
\pgfpathmoveto{\pgfqpoint{0.000000in}{-0.020833in}}%
\pgfpathcurveto{\pgfqpoint{0.005525in}{-0.020833in}}{\pgfqpoint{0.010825in}{-0.018638in}}{\pgfqpoint{0.014731in}{-0.014731in}}%
\pgfpathcurveto{\pgfqpoint{0.018638in}{-0.010825in}}{\pgfqpoint{0.020833in}{-0.005525in}}{\pgfqpoint{0.020833in}{0.000000in}}%
\pgfpathcurveto{\pgfqpoint{0.020833in}{0.005525in}}{\pgfqpoint{0.018638in}{0.010825in}}{\pgfqpoint{0.014731in}{0.014731in}}%
\pgfpathcurveto{\pgfqpoint{0.010825in}{0.018638in}}{\pgfqpoint{0.005525in}{0.020833in}}{\pgfqpoint{0.000000in}{0.020833in}}%
\pgfpathcurveto{\pgfqpoint{-0.005525in}{0.020833in}}{\pgfqpoint{-0.010825in}{0.018638in}}{\pgfqpoint{-0.014731in}{0.014731in}}%
\pgfpathcurveto{\pgfqpoint{-0.018638in}{0.010825in}}{\pgfqpoint{-0.020833in}{0.005525in}}{\pgfqpoint{-0.020833in}{0.000000in}}%
\pgfpathcurveto{\pgfqpoint{-0.020833in}{-0.005525in}}{\pgfqpoint{-0.018638in}{-0.010825in}}{\pgfqpoint{-0.014731in}{-0.014731in}}%
\pgfpathcurveto{\pgfqpoint{-0.010825in}{-0.018638in}}{\pgfqpoint{-0.005525in}{-0.020833in}}{\pgfqpoint{0.000000in}{-0.020833in}}%
\pgfpathlineto{\pgfqpoint{0.000000in}{-0.020833in}}%
\pgfpathclose%
\pgfusepath{stroke,fill}%
}%
\begin{pgfscope}%
\pgfsys@transformshift{0.809982in}{2.386573in}%
\pgfsys@useobject{currentmarker}{}%
\end{pgfscope}%
\begin{pgfscope}%
\pgfsys@transformshift{1.122467in}{2.307684in}%
\pgfsys@useobject{currentmarker}{}%
\end{pgfscope}%
\begin{pgfscope}%
\pgfsys@transformshift{1.420640in}{2.188938in}%
\pgfsys@useobject{currentmarker}{}%
\end{pgfscope}%
\begin{pgfscope}%
\pgfsys@transformshift{1.710766in}{2.017643in}%
\pgfsys@useobject{currentmarker}{}%
\end{pgfscope}%
\begin{pgfscope}%
\pgfsys@transformshift{1.999725in}{1.803614in}%
\pgfsys@useobject{currentmarker}{}%
\end{pgfscope}%
\begin{pgfscope}%
\pgfsys@transformshift{2.285257in}{1.613585in}%
\pgfsys@useobject{currentmarker}{}%
\end{pgfscope}%
\begin{pgfscope}%
\pgfsys@transformshift{2.571345in}{1.397561in}%
\pgfsys@useobject{currentmarker}{}%
\end{pgfscope}%
\end{pgfscope}%
\begin{pgfscope}%
\pgfpathrectangle{\pgfqpoint{0.721913in}{0.549073in}}{\pgfqpoint{1.937500in}{1.925000in}}%
\pgfusepath{clip}%
\pgfsetrectcap%
\pgfsetroundjoin%
\pgfsetlinewidth{1.003750pt}%
\definecolor{currentstroke}{rgb}{0.976471,0.505882,0.145098}%
\pgfsetstrokecolor{currentstroke}%
\pgfsetdash{}{0pt}%
\pgfpathmoveto{\pgfqpoint{0.809982in}{1.933913in}}%
\pgfpathlineto{\pgfqpoint{1.122467in}{1.758786in}}%
\pgfpathlineto{\pgfqpoint{1.420640in}{1.630095in}}%
\pgfpathlineto{\pgfqpoint{1.710766in}{1.468680in}}%
\pgfpathlineto{\pgfqpoint{1.999725in}{1.157867in}}%
\pgfpathlineto{\pgfqpoint{2.285257in}{0.796399in}}%
\pgfpathlineto{\pgfqpoint{2.571345in}{0.636573in}}%
\pgfusepath{stroke}%
\end{pgfscope}%
\begin{pgfscope}%
\pgfpathrectangle{\pgfqpoint{0.721913in}{0.549073in}}{\pgfqpoint{1.937500in}{1.925000in}}%
\pgfusepath{clip}%
\pgfsetbuttcap%
\pgfsetroundjoin%
\definecolor{currentfill}{rgb}{0.976471,0.505882,0.145098}%
\pgfsetfillcolor{currentfill}%
\pgfsetlinewidth{1.003750pt}%
\definecolor{currentstroke}{rgb}{0.976471,0.505882,0.145098}%
\pgfsetstrokecolor{currentstroke}%
\pgfsetdash{}{0pt}%
\pgfsys@defobject{currentmarker}{\pgfqpoint{-0.020833in}{-0.020833in}}{\pgfqpoint{0.020833in}{0.020833in}}{%
\pgfpathmoveto{\pgfqpoint{0.000000in}{-0.020833in}}%
\pgfpathcurveto{\pgfqpoint{0.005525in}{-0.020833in}}{\pgfqpoint{0.010825in}{-0.018638in}}{\pgfqpoint{0.014731in}{-0.014731in}}%
\pgfpathcurveto{\pgfqpoint{0.018638in}{-0.010825in}}{\pgfqpoint{0.020833in}{-0.005525in}}{\pgfqpoint{0.020833in}{0.000000in}}%
\pgfpathcurveto{\pgfqpoint{0.020833in}{0.005525in}}{\pgfqpoint{0.018638in}{0.010825in}}{\pgfqpoint{0.014731in}{0.014731in}}%
\pgfpathcurveto{\pgfqpoint{0.010825in}{0.018638in}}{\pgfqpoint{0.005525in}{0.020833in}}{\pgfqpoint{0.000000in}{0.020833in}}%
\pgfpathcurveto{\pgfqpoint{-0.005525in}{0.020833in}}{\pgfqpoint{-0.010825in}{0.018638in}}{\pgfqpoint{-0.014731in}{0.014731in}}%
\pgfpathcurveto{\pgfqpoint{-0.018638in}{0.010825in}}{\pgfqpoint{-0.020833in}{0.005525in}}{\pgfqpoint{-0.020833in}{0.000000in}}%
\pgfpathcurveto{\pgfqpoint{-0.020833in}{-0.005525in}}{\pgfqpoint{-0.018638in}{-0.010825in}}{\pgfqpoint{-0.014731in}{-0.014731in}}%
\pgfpathcurveto{\pgfqpoint{-0.010825in}{-0.018638in}}{\pgfqpoint{-0.005525in}{-0.020833in}}{\pgfqpoint{0.000000in}{-0.020833in}}%
\pgfpathlineto{\pgfqpoint{0.000000in}{-0.020833in}}%
\pgfpathclose%
\pgfusepath{stroke,fill}%
}%
\begin{pgfscope}%
\pgfsys@transformshift{0.809982in}{1.933913in}%
\pgfsys@useobject{currentmarker}{}%
\end{pgfscope}%
\begin{pgfscope}%
\pgfsys@transformshift{1.122467in}{1.758786in}%
\pgfsys@useobject{currentmarker}{}%
\end{pgfscope}%
\begin{pgfscope}%
\pgfsys@transformshift{1.420640in}{1.630095in}%
\pgfsys@useobject{currentmarker}{}%
\end{pgfscope}%
\begin{pgfscope}%
\pgfsys@transformshift{1.710766in}{1.468680in}%
\pgfsys@useobject{currentmarker}{}%
\end{pgfscope}%
\begin{pgfscope}%
\pgfsys@transformshift{1.999725in}{1.157867in}%
\pgfsys@useobject{currentmarker}{}%
\end{pgfscope}%
\begin{pgfscope}%
\pgfsys@transformshift{2.285257in}{0.796399in}%
\pgfsys@useobject{currentmarker}{}%
\end{pgfscope}%
\begin{pgfscope}%
\pgfsys@transformshift{2.571345in}{0.636573in}%
\pgfsys@useobject{currentmarker}{}%
\end{pgfscope}%
\end{pgfscope}%
\begin{pgfscope}%
\pgfsetrectcap%
\pgfsetmiterjoin%
\pgfsetlinewidth{0.803000pt}%
\definecolor{currentstroke}{rgb}{0.000000,0.000000,0.000000}%
\pgfsetstrokecolor{currentstroke}%
\pgfsetdash{}{0pt}%
\pgfpathmoveto{\pgfqpoint{0.721913in}{0.549073in}}%
\pgfpathlineto{\pgfqpoint{0.721913in}{2.474073in}}%
\pgfusepath{stroke}%
\end{pgfscope}%
\begin{pgfscope}%
\pgfsetrectcap%
\pgfsetmiterjoin%
\pgfsetlinewidth{0.803000pt}%
\definecolor{currentstroke}{rgb}{0.000000,0.000000,0.000000}%
\pgfsetstrokecolor{currentstroke}%
\pgfsetdash{}{0pt}%
\pgfpathmoveto{\pgfqpoint{2.659413in}{0.549073in}}%
\pgfpathlineto{\pgfqpoint{2.659413in}{2.474073in}}%
\pgfusepath{stroke}%
\end{pgfscope}%
\begin{pgfscope}%
\pgfsetrectcap%
\pgfsetmiterjoin%
\pgfsetlinewidth{0.803000pt}%
\definecolor{currentstroke}{rgb}{0.000000,0.000000,0.000000}%
\pgfsetstrokecolor{currentstroke}%
\pgfsetdash{}{0pt}%
\pgfpathmoveto{\pgfqpoint{0.721913in}{0.549073in}}%
\pgfpathlineto{\pgfqpoint{2.659413in}{0.549073in}}%
\pgfusepath{stroke}%
\end{pgfscope}%
\begin{pgfscope}%
\pgfsetrectcap%
\pgfsetmiterjoin%
\pgfsetlinewidth{0.803000pt}%
\definecolor{currentstroke}{rgb}{0.000000,0.000000,0.000000}%
\pgfsetstrokecolor{currentstroke}%
\pgfsetdash{}{0pt}%
\pgfpathmoveto{\pgfqpoint{0.721913in}{2.474073in}}%
\pgfpathlineto{\pgfqpoint{2.659413in}{2.474073in}}%
\pgfusepath{stroke}%
\end{pgfscope}%
\begin{pgfscope}%
\definecolor{textcolor}{rgb}{0.478431,0.478431,0.478431}%
\pgfsetstrokecolor{textcolor}%
\pgfsetfillcolor{textcolor}%
\pgftext[x=1.848036in,y=1.391859in,left,base]{\color{textcolor}{\rmfamily\fontsize{12.000000}{14.400000}\selectfont\catcode`\^=\active\def^{\ifmmode\sp\else\^{}\fi}\catcode`\%=\active\def%{\%}$\mathcal{O}(\varepsilon^{-1})$}}%
\end{pgfscope}%
\begin{pgfscope}%
\definecolor{textcolor}{rgb}{0.478431,0.478431,0.478431}%
\pgfsetstrokecolor{textcolor}%
\pgfsetfillcolor{textcolor}%
\pgftext[x=1.341344in,y=1.847257in,left,base]{\color{textcolor}{\rmfamily\fontsize{12.000000}{14.400000}\selectfont\catcode`\^=\active\def^{\ifmmode\sp\else\^{}\fi}\catcode`\%=\active\def%{\%}$\mathcal{O}(\varepsilon^{-2})$}}%
\end{pgfscope}%
\begin{pgfscope}%
\pgfsetbuttcap%
\pgfsetmiterjoin%
\definecolor{currentfill}{rgb}{1.000000,1.000000,1.000000}%
\pgfsetfillcolor{currentfill}%
\pgfsetfillopacity{0.800000}%
\pgfsetlinewidth{1.003750pt}%
\definecolor{currentstroke}{rgb}{0.800000,0.800000,0.800000}%
\pgfsetstrokecolor{currentstroke}%
\pgfsetstrokeopacity{0.800000}%
\pgfsetdash{}{0pt}%
\pgfpathmoveto{\pgfqpoint{0.838580in}{0.632406in}}%
\pgfpathlineto{\pgfqpoint{1.988362in}{0.632406in}}%
\pgfpathquadraticcurveto{\pgfqpoint{2.021695in}{0.632406in}}{\pgfqpoint{2.021695in}{0.665739in}}%
\pgfpathlineto{\pgfqpoint{2.021695in}{1.346294in}}%
\pgfpathquadraticcurveto{\pgfqpoint{2.021695in}{1.379627in}}{\pgfqpoint{1.988362in}{1.379627in}}%
\pgfpathlineto{\pgfqpoint{0.838580in}{1.379627in}}%
\pgfpathquadraticcurveto{\pgfqpoint{0.805247in}{1.379627in}}{\pgfqpoint{0.805247in}{1.346294in}}%
\pgfpathlineto{\pgfqpoint{0.805247in}{0.665739in}}%
\pgfpathquadraticcurveto{\pgfqpoint{0.805247in}{0.632406in}}{\pgfqpoint{0.838580in}{0.632406in}}%
\pgfpathlineto{\pgfqpoint{0.838580in}{0.632406in}}%
\pgfpathclose%
\pgfusepath{stroke,fill}%
\end{pgfscope}%
\begin{pgfscope}%
\pgfsetrectcap%
\pgfsetroundjoin%
\pgfsetlinewidth{1.003750pt}%
\definecolor{currentstroke}{rgb}{0.537255,0.647059,0.760784}%
\pgfsetstrokecolor{currentstroke}%
\pgfsetdash{}{0pt}%
\pgfpathmoveto{\pgfqpoint{0.871913in}{1.254627in}}%
\pgfpathlineto{\pgfqpoint{1.038580in}{1.254627in}}%
\pgfpathlineto{\pgfqpoint{1.205247in}{1.254627in}}%
\pgfusepath{stroke}%
\end{pgfscope}%
\begin{pgfscope}%
\pgfsetbuttcap%
\pgfsetroundjoin%
\definecolor{currentfill}{rgb}{0.537255,0.647059,0.760784}%
\pgfsetfillcolor{currentfill}%
\pgfsetlinewidth{1.003750pt}%
\definecolor{currentstroke}{rgb}{0.537255,0.647059,0.760784}%
\pgfsetstrokecolor{currentstroke}%
\pgfsetdash{}{0pt}%
\pgfsys@defobject{currentmarker}{\pgfqpoint{-0.020833in}{-0.020833in}}{\pgfqpoint{0.020833in}{0.020833in}}{%
\pgfpathmoveto{\pgfqpoint{0.000000in}{-0.020833in}}%
\pgfpathcurveto{\pgfqpoint{0.005525in}{-0.020833in}}{\pgfqpoint{0.010825in}{-0.018638in}}{\pgfqpoint{0.014731in}{-0.014731in}}%
\pgfpathcurveto{\pgfqpoint{0.018638in}{-0.010825in}}{\pgfqpoint{0.020833in}{-0.005525in}}{\pgfqpoint{0.020833in}{0.000000in}}%
\pgfpathcurveto{\pgfqpoint{0.020833in}{0.005525in}}{\pgfqpoint{0.018638in}{0.010825in}}{\pgfqpoint{0.014731in}{0.014731in}}%
\pgfpathcurveto{\pgfqpoint{0.010825in}{0.018638in}}{\pgfqpoint{0.005525in}{0.020833in}}{\pgfqpoint{0.000000in}{0.020833in}}%
\pgfpathcurveto{\pgfqpoint{-0.005525in}{0.020833in}}{\pgfqpoint{-0.010825in}{0.018638in}}{\pgfqpoint{-0.014731in}{0.014731in}}%
\pgfpathcurveto{\pgfqpoint{-0.018638in}{0.010825in}}{\pgfqpoint{-0.020833in}{0.005525in}}{\pgfqpoint{-0.020833in}{0.000000in}}%
\pgfpathcurveto{\pgfqpoint{-0.020833in}{-0.005525in}}{\pgfqpoint{-0.018638in}{-0.010825in}}{\pgfqpoint{-0.014731in}{-0.014731in}}%
\pgfpathcurveto{\pgfqpoint{-0.010825in}{-0.018638in}}{\pgfqpoint{-0.005525in}{-0.020833in}}{\pgfqpoint{0.000000in}{-0.020833in}}%
\pgfpathlineto{\pgfqpoint{0.000000in}{-0.020833in}}%
\pgfpathclose%
\pgfusepath{stroke,fill}%
}%
\begin{pgfscope}%
\pgfsys@transformshift{1.038580in}{1.254627in}%
\pgfsys@useobject{currentmarker}{}%
\end{pgfscope}%
\end{pgfscope}%
\begin{pgfscope}%
\definecolor{textcolor}{rgb}{0.000000,0.000000,0.000000}%
\pgfsetstrokecolor{textcolor}%
\pgfsetfillcolor{textcolor}%
\pgftext[x=1.338580in,y=1.196294in,left,base]{\color{textcolor}{\rmfamily\fontsize{12.000000}{14.400000}\selectfont\catcode`\^=\active\def^{\ifmmode\sp\else\^{}\fi}\catcode`\%=\active\def%{\%}Haydock}}%
\end{pgfscope}%
\begin{pgfscope}%
\pgfsetrectcap%
\pgfsetroundjoin%
\pgfsetlinewidth{1.003750pt}%
\definecolor{currentstroke}{rgb}{0.184314,0.270588,0.360784}%
\pgfsetstrokecolor{currentstroke}%
\pgfsetdash{}{0pt}%
\pgfpathmoveto{\pgfqpoint{0.871913in}{1.022220in}}%
\pgfpathlineto{\pgfqpoint{1.038580in}{1.022220in}}%
\pgfpathlineto{\pgfqpoint{1.205247in}{1.022220in}}%
\pgfusepath{stroke}%
\end{pgfscope}%
\begin{pgfscope}%
\pgfsetbuttcap%
\pgfsetroundjoin%
\definecolor{currentfill}{rgb}{0.184314,0.270588,0.360784}%
\pgfsetfillcolor{currentfill}%
\pgfsetlinewidth{1.003750pt}%
\definecolor{currentstroke}{rgb}{0.184314,0.270588,0.360784}%
\pgfsetstrokecolor{currentstroke}%
\pgfsetdash{}{0pt}%
\pgfsys@defobject{currentmarker}{\pgfqpoint{-0.020833in}{-0.020833in}}{\pgfqpoint{0.020833in}{0.020833in}}{%
\pgfpathmoveto{\pgfqpoint{0.000000in}{-0.020833in}}%
\pgfpathcurveto{\pgfqpoint{0.005525in}{-0.020833in}}{\pgfqpoint{0.010825in}{-0.018638in}}{\pgfqpoint{0.014731in}{-0.014731in}}%
\pgfpathcurveto{\pgfqpoint{0.018638in}{-0.010825in}}{\pgfqpoint{0.020833in}{-0.005525in}}{\pgfqpoint{0.020833in}{0.000000in}}%
\pgfpathcurveto{\pgfqpoint{0.020833in}{0.005525in}}{\pgfqpoint{0.018638in}{0.010825in}}{\pgfqpoint{0.014731in}{0.014731in}}%
\pgfpathcurveto{\pgfqpoint{0.010825in}{0.018638in}}{\pgfqpoint{0.005525in}{0.020833in}}{\pgfqpoint{0.000000in}{0.020833in}}%
\pgfpathcurveto{\pgfqpoint{-0.005525in}{0.020833in}}{\pgfqpoint{-0.010825in}{0.018638in}}{\pgfqpoint{-0.014731in}{0.014731in}}%
\pgfpathcurveto{\pgfqpoint{-0.018638in}{0.010825in}}{\pgfqpoint{-0.020833in}{0.005525in}}{\pgfqpoint{-0.020833in}{0.000000in}}%
\pgfpathcurveto{\pgfqpoint{-0.020833in}{-0.005525in}}{\pgfqpoint{-0.018638in}{-0.010825in}}{\pgfqpoint{-0.014731in}{-0.014731in}}%
\pgfpathcurveto{\pgfqpoint{-0.010825in}{-0.018638in}}{\pgfqpoint{-0.005525in}{-0.020833in}}{\pgfqpoint{0.000000in}{-0.020833in}}%
\pgfpathlineto{\pgfqpoint{0.000000in}{-0.020833in}}%
\pgfpathclose%
\pgfusepath{stroke,fill}%
}%
\begin{pgfscope}%
\pgfsys@transformshift{1.038580in}{1.022220in}%
\pgfsys@useobject{currentmarker}{}%
\end{pgfscope}%
\end{pgfscope}%
\begin{pgfscope}%
\definecolor{textcolor}{rgb}{0.000000,0.000000,0.000000}%
\pgfsetstrokecolor{textcolor}%
\pgfsetfillcolor{textcolor}%
\pgftext[x=1.338580in,y=0.963887in,left,base]{\color{textcolor}{\rmfamily\fontsize{12.000000}{14.400000}\selectfont\catcode`\^=\active\def^{\ifmmode\sp\else\^{}\fi}\catcode`\%=\active\def%{\%}NC}}%
\end{pgfscope}%
\begin{pgfscope}%
\pgfsetrectcap%
\pgfsetroundjoin%
\pgfsetlinewidth{1.003750pt}%
\definecolor{currentstroke}{rgb}{0.976471,0.505882,0.145098}%
\pgfsetstrokecolor{currentstroke}%
\pgfsetdash{}{0pt}%
\pgfpathmoveto{\pgfqpoint{0.871913in}{0.789813in}}%
\pgfpathlineto{\pgfqpoint{1.038580in}{0.789813in}}%
\pgfpathlineto{\pgfqpoint{1.205247in}{0.789813in}}%
\pgfusepath{stroke}%
\end{pgfscope}%
\begin{pgfscope}%
\pgfsetbuttcap%
\pgfsetroundjoin%
\definecolor{currentfill}{rgb}{0.976471,0.505882,0.145098}%
\pgfsetfillcolor{currentfill}%
\pgfsetlinewidth{1.003750pt}%
\definecolor{currentstroke}{rgb}{0.976471,0.505882,0.145098}%
\pgfsetstrokecolor{currentstroke}%
\pgfsetdash{}{0pt}%
\pgfsys@defobject{currentmarker}{\pgfqpoint{-0.020833in}{-0.020833in}}{\pgfqpoint{0.020833in}{0.020833in}}{%
\pgfpathmoveto{\pgfqpoint{0.000000in}{-0.020833in}}%
\pgfpathcurveto{\pgfqpoint{0.005525in}{-0.020833in}}{\pgfqpoint{0.010825in}{-0.018638in}}{\pgfqpoint{0.014731in}{-0.014731in}}%
\pgfpathcurveto{\pgfqpoint{0.018638in}{-0.010825in}}{\pgfqpoint{0.020833in}{-0.005525in}}{\pgfqpoint{0.020833in}{0.000000in}}%
\pgfpathcurveto{\pgfqpoint{0.020833in}{0.005525in}}{\pgfqpoint{0.018638in}{0.010825in}}{\pgfqpoint{0.014731in}{0.014731in}}%
\pgfpathcurveto{\pgfqpoint{0.010825in}{0.018638in}}{\pgfqpoint{0.005525in}{0.020833in}}{\pgfqpoint{0.000000in}{0.020833in}}%
\pgfpathcurveto{\pgfqpoint{-0.005525in}{0.020833in}}{\pgfqpoint{-0.010825in}{0.018638in}}{\pgfqpoint{-0.014731in}{0.014731in}}%
\pgfpathcurveto{\pgfqpoint{-0.018638in}{0.010825in}}{\pgfqpoint{-0.020833in}{0.005525in}}{\pgfqpoint{-0.020833in}{0.000000in}}%
\pgfpathcurveto{\pgfqpoint{-0.020833in}{-0.005525in}}{\pgfqpoint{-0.018638in}{-0.010825in}}{\pgfqpoint{-0.014731in}{-0.014731in}}%
\pgfpathcurveto{\pgfqpoint{-0.010825in}{-0.018638in}}{\pgfqpoint{-0.005525in}{-0.020833in}}{\pgfqpoint{0.000000in}{-0.020833in}}%
\pgfpathlineto{\pgfqpoint{0.000000in}{-0.020833in}}%
\pgfpathclose%
\pgfusepath{stroke,fill}%
}%
\begin{pgfscope}%
\pgfsys@transformshift{1.038580in}{0.789813in}%
\pgfsys@useobject{currentmarker}{}%
\end{pgfscope}%
\end{pgfscope}%
\begin{pgfscope}%
\definecolor{textcolor}{rgb}{0.000000,0.000000,0.000000}%
\pgfsetstrokecolor{textcolor}%
\pgfsetfillcolor{textcolor}%
\pgftext[x=1.338580in,y=0.731480in,left,base]{\color{textcolor}{\rmfamily\fontsize{12.000000}{14.400000}\selectfont\catcode`\^=\active\def^{\ifmmode\sp\else\^{}\fi}\catcode`\%=\active\def%{\%}NC++}}%
\end{pgfscope}%
\end{pgfpicture}%
\makeatother%
\endgroup%

        \caption{$m=2400$}
        \label{fig:5-experiments-haydock-convergence-nv-m2400}
    \end{subfigure}
    \caption{Behavior with $n_{\Omega}$ for $\sigma=0.05$}
    \label{fig:5-experiments-haydock-convergence-nv}
\end{figure}

\begin{figure}[ht]
    \centering
    \begin{subfigure}[b]{0.49\columnwidth}
        %% Creator: Matplotlib, PGF backend
%%
%% To include the figure in your LaTeX document, write
%%   \input{<filename>.pgf}
%%
%% Make sure the required packages are loaded in your preamble
%%   \usepackage{pgf}
%%
%% Also ensure that all the required font packages are loaded; for instance,
%% the lmodern package is sometimes necessary when using math font.
%%   \usepackage{lmodern}
%%
%% Figures using additional raster images can only be included by \input if
%% they are in the same directory as the main LaTeX file. For loading figures
%% from other directories you can use the `import` package
%%   \usepackage{import}
%%
%% and then include the figures with
%%   \import{<path to file>}{<filename>.pgf}
%%
%% Matplotlib used the following preamble
%%   \def\mathdefault#1{#1}
%%   \everymath=\expandafter{\the\everymath\displaystyle}
%%   
%%   \usepackage{fontspec}
%%   \setmainfont{DejaVuSerif.ttf}[Path=\detokenize{C:/Users/fabio/Documents/Work/MasterThesis/Rand-SD/.venv/Lib/site-packages/matplotlib/mpl-data/fonts/ttf/}]
%%   \setsansfont{DejaVuSans.ttf}[Path=\detokenize{C:/Users/fabio/Documents/Work/MasterThesis/Rand-SD/.venv/Lib/site-packages/matplotlib/mpl-data/fonts/ttf/}]
%%   \setmonofont{DejaVuSansMono.ttf}[Path=\detokenize{C:/Users/fabio/Documents/Work/MasterThesis/Rand-SD/.venv/Lib/site-packages/matplotlib/mpl-data/fonts/ttf/}]
%%   \makeatletter\@ifpackageloaded{underscore}{}{\usepackage[strings]{underscore}}\makeatother
%%
\begingroup%
\makeatletter%
\begin{pgfpicture}%
\pgfpathrectangle{\pgfpointorigin}{\pgfqpoint{2.712693in}{2.546603in}}%
\pgfusepath{use as bounding box, clip}%
\begin{pgfscope}%
\pgfsetbuttcap%
\pgfsetmiterjoin%
\definecolor{currentfill}{rgb}{1.000000,1.000000,1.000000}%
\pgfsetfillcolor{currentfill}%
\pgfsetlinewidth{0.000000pt}%
\definecolor{currentstroke}{rgb}{1.000000,1.000000,1.000000}%
\pgfsetstrokecolor{currentstroke}%
\pgfsetdash{}{0pt}%
\pgfpathmoveto{\pgfqpoint{0.000000in}{0.000000in}}%
\pgfpathlineto{\pgfqpoint{2.712693in}{0.000000in}}%
\pgfpathlineto{\pgfqpoint{2.712693in}{2.546603in}}%
\pgfpathlineto{\pgfqpoint{0.000000in}{2.546603in}}%
\pgfpathlineto{\pgfqpoint{0.000000in}{0.000000in}}%
\pgfpathclose%
\pgfusepath{fill}%
\end{pgfscope}%
\begin{pgfscope}%
\pgfsetbuttcap%
\pgfsetmiterjoin%
\definecolor{currentfill}{rgb}{1.000000,1.000000,1.000000}%
\pgfsetfillcolor{currentfill}%
\pgfsetlinewidth{0.000000pt}%
\definecolor{currentstroke}{rgb}{0.000000,0.000000,0.000000}%
\pgfsetstrokecolor{currentstroke}%
\pgfsetstrokeopacity{0.000000}%
\pgfsetdash{}{0pt}%
\pgfpathmoveto{\pgfqpoint{0.675193in}{0.521603in}}%
\pgfpathlineto{\pgfqpoint{2.612693in}{0.521603in}}%
\pgfpathlineto{\pgfqpoint{2.612693in}{2.446603in}}%
\pgfpathlineto{\pgfqpoint{0.675193in}{2.446603in}}%
\pgfpathlineto{\pgfqpoint{0.675193in}{0.521603in}}%
\pgfpathclose%
\pgfusepath{fill}%
\end{pgfscope}%
\begin{pgfscope}%
\pgfsetbuttcap%
\pgfsetroundjoin%
\definecolor{currentfill}{rgb}{0.000000,0.000000,0.000000}%
\pgfsetfillcolor{currentfill}%
\pgfsetlinewidth{0.803000pt}%
\definecolor{currentstroke}{rgb}{0.000000,0.000000,0.000000}%
\pgfsetstrokecolor{currentstroke}%
\pgfsetdash{}{0pt}%
\pgfsys@defobject{currentmarker}{\pgfqpoint{0.000000in}{-0.048611in}}{\pgfqpoint{0.000000in}{0.000000in}}{%
\pgfpathmoveto{\pgfqpoint{0.000000in}{0.000000in}}%
\pgfpathlineto{\pgfqpoint{0.000000in}{-0.048611in}}%
\pgfusepath{stroke,fill}%
}%
\begin{pgfscope}%
\pgfsys@transformshift{1.713853in}{0.521603in}%
\pgfsys@useobject{currentmarker}{}%
\end{pgfscope}%
\end{pgfscope}%
\begin{pgfscope}%
\definecolor{textcolor}{rgb}{0.000000,0.000000,0.000000}%
\pgfsetstrokecolor{textcolor}%
\pgfsetfillcolor{textcolor}%
\pgftext[x=1.713853in,y=0.424381in,,top]{\color{textcolor}{\sffamily\fontsize{10.000000}{12.000000}\selectfont\catcode`\^=\active\def^{\ifmmode\sp\else\^{}\fi}\catcode`\%=\active\def%{\%}$\mathdefault{10^{3}}$}}%
\end{pgfscope}%
\begin{pgfscope}%
\pgfsetbuttcap%
\pgfsetroundjoin%
\definecolor{currentfill}{rgb}{0.000000,0.000000,0.000000}%
\pgfsetfillcolor{currentfill}%
\pgfsetlinewidth{0.602250pt}%
\definecolor{currentstroke}{rgb}{0.000000,0.000000,0.000000}%
\pgfsetstrokecolor{currentstroke}%
\pgfsetdash{}{0pt}%
\pgfsys@defobject{currentmarker}{\pgfqpoint{0.000000in}{-0.027778in}}{\pgfqpoint{0.000000in}{0.000000in}}{%
\pgfpathmoveto{\pgfqpoint{0.000000in}{0.000000in}}%
\pgfpathlineto{\pgfqpoint{0.000000in}{-0.027778in}}%
\pgfusepath{stroke,fill}%
}%
\begin{pgfscope}%
\pgfsys@transformshift{0.769160in}{0.521603in}%
\pgfsys@useobject{currentmarker}{}%
\end{pgfscope}%
\end{pgfscope}%
\begin{pgfscope}%
\pgfsetbuttcap%
\pgfsetroundjoin%
\definecolor{currentfill}{rgb}{0.000000,0.000000,0.000000}%
\pgfsetfillcolor{currentfill}%
\pgfsetlinewidth{0.602250pt}%
\definecolor{currentstroke}{rgb}{0.000000,0.000000,0.000000}%
\pgfsetstrokecolor{currentstroke}%
\pgfsetdash{}{0pt}%
\pgfsys@defobject{currentmarker}{\pgfqpoint{0.000000in}{-0.027778in}}{\pgfqpoint{0.000000in}{0.000000in}}{%
\pgfpathmoveto{\pgfqpoint{0.000000in}{0.000000in}}%
\pgfpathlineto{\pgfqpoint{0.000000in}{-0.027778in}}%
\pgfusepath{stroke,fill}%
}%
\begin{pgfscope}%
\pgfsys@transformshift{1.007157in}{0.521603in}%
\pgfsys@useobject{currentmarker}{}%
\end{pgfscope}%
\end{pgfscope}%
\begin{pgfscope}%
\pgfsetbuttcap%
\pgfsetroundjoin%
\definecolor{currentfill}{rgb}{0.000000,0.000000,0.000000}%
\pgfsetfillcolor{currentfill}%
\pgfsetlinewidth{0.602250pt}%
\definecolor{currentstroke}{rgb}{0.000000,0.000000,0.000000}%
\pgfsetstrokecolor{currentstroke}%
\pgfsetdash{}{0pt}%
\pgfsys@defobject{currentmarker}{\pgfqpoint{0.000000in}{-0.027778in}}{\pgfqpoint{0.000000in}{0.000000in}}{%
\pgfpathmoveto{\pgfqpoint{0.000000in}{0.000000in}}%
\pgfpathlineto{\pgfqpoint{0.000000in}{-0.027778in}}%
\pgfusepath{stroke,fill}%
}%
\begin{pgfscope}%
\pgfsys@transformshift{1.176017in}{0.521603in}%
\pgfsys@useobject{currentmarker}{}%
\end{pgfscope}%
\end{pgfscope}%
\begin{pgfscope}%
\pgfsetbuttcap%
\pgfsetroundjoin%
\definecolor{currentfill}{rgb}{0.000000,0.000000,0.000000}%
\pgfsetfillcolor{currentfill}%
\pgfsetlinewidth{0.602250pt}%
\definecolor{currentstroke}{rgb}{0.000000,0.000000,0.000000}%
\pgfsetstrokecolor{currentstroke}%
\pgfsetdash{}{0pt}%
\pgfsys@defobject{currentmarker}{\pgfqpoint{0.000000in}{-0.027778in}}{\pgfqpoint{0.000000in}{0.000000in}}{%
\pgfpathmoveto{\pgfqpoint{0.000000in}{0.000000in}}%
\pgfpathlineto{\pgfqpoint{0.000000in}{-0.027778in}}%
\pgfusepath{stroke,fill}%
}%
\begin{pgfscope}%
\pgfsys@transformshift{1.306996in}{0.521603in}%
\pgfsys@useobject{currentmarker}{}%
\end{pgfscope}%
\end{pgfscope}%
\begin{pgfscope}%
\pgfsetbuttcap%
\pgfsetroundjoin%
\definecolor{currentfill}{rgb}{0.000000,0.000000,0.000000}%
\pgfsetfillcolor{currentfill}%
\pgfsetlinewidth{0.602250pt}%
\definecolor{currentstroke}{rgb}{0.000000,0.000000,0.000000}%
\pgfsetstrokecolor{currentstroke}%
\pgfsetdash{}{0pt}%
\pgfsys@defobject{currentmarker}{\pgfqpoint{0.000000in}{-0.027778in}}{\pgfqpoint{0.000000in}{0.000000in}}{%
\pgfpathmoveto{\pgfqpoint{0.000000in}{0.000000in}}%
\pgfpathlineto{\pgfqpoint{0.000000in}{-0.027778in}}%
\pgfusepath{stroke,fill}%
}%
\begin{pgfscope}%
\pgfsys@transformshift{1.414014in}{0.521603in}%
\pgfsys@useobject{currentmarker}{}%
\end{pgfscope}%
\end{pgfscope}%
\begin{pgfscope}%
\pgfsetbuttcap%
\pgfsetroundjoin%
\definecolor{currentfill}{rgb}{0.000000,0.000000,0.000000}%
\pgfsetfillcolor{currentfill}%
\pgfsetlinewidth{0.602250pt}%
\definecolor{currentstroke}{rgb}{0.000000,0.000000,0.000000}%
\pgfsetstrokecolor{currentstroke}%
\pgfsetdash{}{0pt}%
\pgfsys@defobject{currentmarker}{\pgfqpoint{0.000000in}{-0.027778in}}{\pgfqpoint{0.000000in}{0.000000in}}{%
\pgfpathmoveto{\pgfqpoint{0.000000in}{0.000000in}}%
\pgfpathlineto{\pgfqpoint{0.000000in}{-0.027778in}}%
\pgfusepath{stroke,fill}%
}%
\begin{pgfscope}%
\pgfsys@transformshift{1.504495in}{0.521603in}%
\pgfsys@useobject{currentmarker}{}%
\end{pgfscope}%
\end{pgfscope}%
\begin{pgfscope}%
\pgfsetbuttcap%
\pgfsetroundjoin%
\definecolor{currentfill}{rgb}{0.000000,0.000000,0.000000}%
\pgfsetfillcolor{currentfill}%
\pgfsetlinewidth{0.602250pt}%
\definecolor{currentstroke}{rgb}{0.000000,0.000000,0.000000}%
\pgfsetstrokecolor{currentstroke}%
\pgfsetdash{}{0pt}%
\pgfsys@defobject{currentmarker}{\pgfqpoint{0.000000in}{-0.027778in}}{\pgfqpoint{0.000000in}{0.000000in}}{%
\pgfpathmoveto{\pgfqpoint{0.000000in}{0.000000in}}%
\pgfpathlineto{\pgfqpoint{0.000000in}{-0.027778in}}%
\pgfusepath{stroke,fill}%
}%
\begin{pgfscope}%
\pgfsys@transformshift{1.582874in}{0.521603in}%
\pgfsys@useobject{currentmarker}{}%
\end{pgfscope}%
\end{pgfscope}%
\begin{pgfscope}%
\pgfsetbuttcap%
\pgfsetroundjoin%
\definecolor{currentfill}{rgb}{0.000000,0.000000,0.000000}%
\pgfsetfillcolor{currentfill}%
\pgfsetlinewidth{0.602250pt}%
\definecolor{currentstroke}{rgb}{0.000000,0.000000,0.000000}%
\pgfsetstrokecolor{currentstroke}%
\pgfsetdash{}{0pt}%
\pgfsys@defobject{currentmarker}{\pgfqpoint{0.000000in}{-0.027778in}}{\pgfqpoint{0.000000in}{0.000000in}}{%
\pgfpathmoveto{\pgfqpoint{0.000000in}{0.000000in}}%
\pgfpathlineto{\pgfqpoint{0.000000in}{-0.027778in}}%
\pgfusepath{stroke,fill}%
}%
\begin{pgfscope}%
\pgfsys@transformshift{1.652010in}{0.521603in}%
\pgfsys@useobject{currentmarker}{}%
\end{pgfscope}%
\end{pgfscope}%
\begin{pgfscope}%
\pgfsetbuttcap%
\pgfsetroundjoin%
\definecolor{currentfill}{rgb}{0.000000,0.000000,0.000000}%
\pgfsetfillcolor{currentfill}%
\pgfsetlinewidth{0.602250pt}%
\definecolor{currentstroke}{rgb}{0.000000,0.000000,0.000000}%
\pgfsetstrokecolor{currentstroke}%
\pgfsetdash{}{0pt}%
\pgfsys@defobject{currentmarker}{\pgfqpoint{0.000000in}{-0.027778in}}{\pgfqpoint{0.000000in}{0.000000in}}{%
\pgfpathmoveto{\pgfqpoint{0.000000in}{0.000000in}}%
\pgfpathlineto{\pgfqpoint{0.000000in}{-0.027778in}}%
\pgfusepath{stroke,fill}%
}%
\begin{pgfscope}%
\pgfsys@transformshift{2.120710in}{0.521603in}%
\pgfsys@useobject{currentmarker}{}%
\end{pgfscope}%
\end{pgfscope}%
\begin{pgfscope}%
\pgfsetbuttcap%
\pgfsetroundjoin%
\definecolor{currentfill}{rgb}{0.000000,0.000000,0.000000}%
\pgfsetfillcolor{currentfill}%
\pgfsetlinewidth{0.602250pt}%
\definecolor{currentstroke}{rgb}{0.000000,0.000000,0.000000}%
\pgfsetstrokecolor{currentstroke}%
\pgfsetdash{}{0pt}%
\pgfsys@defobject{currentmarker}{\pgfqpoint{0.000000in}{-0.027778in}}{\pgfqpoint{0.000000in}{0.000000in}}{%
\pgfpathmoveto{\pgfqpoint{0.000000in}{0.000000in}}%
\pgfpathlineto{\pgfqpoint{0.000000in}{-0.027778in}}%
\pgfusepath{stroke,fill}%
}%
\begin{pgfscope}%
\pgfsys@transformshift{2.358706in}{0.521603in}%
\pgfsys@useobject{currentmarker}{}%
\end{pgfscope}%
\end{pgfscope}%
\begin{pgfscope}%
\pgfsetbuttcap%
\pgfsetroundjoin%
\definecolor{currentfill}{rgb}{0.000000,0.000000,0.000000}%
\pgfsetfillcolor{currentfill}%
\pgfsetlinewidth{0.602250pt}%
\definecolor{currentstroke}{rgb}{0.000000,0.000000,0.000000}%
\pgfsetstrokecolor{currentstroke}%
\pgfsetdash{}{0pt}%
\pgfsys@defobject{currentmarker}{\pgfqpoint{0.000000in}{-0.027778in}}{\pgfqpoint{0.000000in}{0.000000in}}{%
\pgfpathmoveto{\pgfqpoint{0.000000in}{0.000000in}}%
\pgfpathlineto{\pgfqpoint{0.000000in}{-0.027778in}}%
\pgfusepath{stroke,fill}%
}%
\begin{pgfscope}%
\pgfsys@transformshift{2.527567in}{0.521603in}%
\pgfsys@useobject{currentmarker}{}%
\end{pgfscope}%
\end{pgfscope}%
\begin{pgfscope}%
\definecolor{textcolor}{rgb}{0.000000,0.000000,0.000000}%
\pgfsetstrokecolor{textcolor}%
\pgfsetfillcolor{textcolor}%
\pgftext[x=1.643943in,y=0.234413in,,top]{\color{textcolor}{\sffamily\fontsize{10.000000}{12.000000}\selectfont\catcode`\^=\active\def^{\ifmmode\sp\else\^{}\fi}\catcode`\%=\active\def%{\%}$m$}}%
\end{pgfscope}%
\begin{pgfscope}%
\pgfsetbuttcap%
\pgfsetroundjoin%
\definecolor{currentfill}{rgb}{0.000000,0.000000,0.000000}%
\pgfsetfillcolor{currentfill}%
\pgfsetlinewidth{0.803000pt}%
\definecolor{currentstroke}{rgb}{0.000000,0.000000,0.000000}%
\pgfsetstrokecolor{currentstroke}%
\pgfsetdash{}{0pt}%
\pgfsys@defobject{currentmarker}{\pgfqpoint{-0.048611in}{0.000000in}}{\pgfqpoint{-0.000000in}{0.000000in}}{%
\pgfpathmoveto{\pgfqpoint{-0.000000in}{0.000000in}}%
\pgfpathlineto{\pgfqpoint{-0.048611in}{0.000000in}}%
\pgfusepath{stroke,fill}%
}%
\begin{pgfscope}%
\pgfsys@transformshift{0.675193in}{1.463418in}%
\pgfsys@useobject{currentmarker}{}%
\end{pgfscope}%
\end{pgfscope}%
\begin{pgfscope}%
\definecolor{textcolor}{rgb}{0.000000,0.000000,0.000000}%
\pgfsetstrokecolor{textcolor}%
\pgfsetfillcolor{textcolor}%
\pgftext[x=0.289968in, y=1.410656in, left, base]{\color{textcolor}{\sffamily\fontsize{10.000000}{12.000000}\selectfont\catcode`\^=\active\def^{\ifmmode\sp\else\^{}\fi}\catcode`\%=\active\def%{\%}$\mathdefault{10^{-1}}$}}%
\end{pgfscope}%
\begin{pgfscope}%
\pgfsetbuttcap%
\pgfsetroundjoin%
\definecolor{currentfill}{rgb}{0.000000,0.000000,0.000000}%
\pgfsetfillcolor{currentfill}%
\pgfsetlinewidth{0.602250pt}%
\definecolor{currentstroke}{rgb}{0.000000,0.000000,0.000000}%
\pgfsetstrokecolor{currentstroke}%
\pgfsetdash{}{0pt}%
\pgfsys@defobject{currentmarker}{\pgfqpoint{-0.027778in}{0.000000in}}{\pgfqpoint{-0.000000in}{0.000000in}}{%
\pgfpathmoveto{\pgfqpoint{-0.000000in}{0.000000in}}%
\pgfpathlineto{\pgfqpoint{-0.027778in}{0.000000in}}%
\pgfusepath{stroke,fill}%
}%
\begin{pgfscope}%
\pgfsys@transformshift{0.675193in}{0.638741in}%
\pgfsys@useobject{currentmarker}{}%
\end{pgfscope}%
\end{pgfscope}%
\begin{pgfscope}%
\pgfsetbuttcap%
\pgfsetroundjoin%
\definecolor{currentfill}{rgb}{0.000000,0.000000,0.000000}%
\pgfsetfillcolor{currentfill}%
\pgfsetlinewidth{0.602250pt}%
\definecolor{currentstroke}{rgb}{0.000000,0.000000,0.000000}%
\pgfsetstrokecolor{currentstroke}%
\pgfsetdash{}{0pt}%
\pgfsys@defobject{currentmarker}{\pgfqpoint{-0.027778in}{0.000000in}}{\pgfqpoint{-0.000000in}{0.000000in}}{%
\pgfpathmoveto{\pgfqpoint{-0.000000in}{0.000000in}}%
\pgfpathlineto{\pgfqpoint{-0.027778in}{0.000000in}}%
\pgfusepath{stroke,fill}%
}%
\begin{pgfscope}%
\pgfsys@transformshift{0.675193in}{0.846501in}%
\pgfsys@useobject{currentmarker}{}%
\end{pgfscope}%
\end{pgfscope}%
\begin{pgfscope}%
\pgfsetbuttcap%
\pgfsetroundjoin%
\definecolor{currentfill}{rgb}{0.000000,0.000000,0.000000}%
\pgfsetfillcolor{currentfill}%
\pgfsetlinewidth{0.602250pt}%
\definecolor{currentstroke}{rgb}{0.000000,0.000000,0.000000}%
\pgfsetstrokecolor{currentstroke}%
\pgfsetdash{}{0pt}%
\pgfsys@defobject{currentmarker}{\pgfqpoint{-0.027778in}{0.000000in}}{\pgfqpoint{-0.000000in}{0.000000in}}{%
\pgfpathmoveto{\pgfqpoint{-0.000000in}{0.000000in}}%
\pgfpathlineto{\pgfqpoint{-0.027778in}{0.000000in}}%
\pgfusepath{stroke,fill}%
}%
\begin{pgfscope}%
\pgfsys@transformshift{0.675193in}{0.993910in}%
\pgfsys@useobject{currentmarker}{}%
\end{pgfscope}%
\end{pgfscope}%
\begin{pgfscope}%
\pgfsetbuttcap%
\pgfsetroundjoin%
\definecolor{currentfill}{rgb}{0.000000,0.000000,0.000000}%
\pgfsetfillcolor{currentfill}%
\pgfsetlinewidth{0.602250pt}%
\definecolor{currentstroke}{rgb}{0.000000,0.000000,0.000000}%
\pgfsetstrokecolor{currentstroke}%
\pgfsetdash{}{0pt}%
\pgfsys@defobject{currentmarker}{\pgfqpoint{-0.027778in}{0.000000in}}{\pgfqpoint{-0.000000in}{0.000000in}}{%
\pgfpathmoveto{\pgfqpoint{-0.000000in}{0.000000in}}%
\pgfpathlineto{\pgfqpoint{-0.027778in}{0.000000in}}%
\pgfusepath{stroke,fill}%
}%
\begin{pgfscope}%
\pgfsys@transformshift{0.675193in}{1.108249in}%
\pgfsys@useobject{currentmarker}{}%
\end{pgfscope}%
\end{pgfscope}%
\begin{pgfscope}%
\pgfsetbuttcap%
\pgfsetroundjoin%
\definecolor{currentfill}{rgb}{0.000000,0.000000,0.000000}%
\pgfsetfillcolor{currentfill}%
\pgfsetlinewidth{0.602250pt}%
\definecolor{currentstroke}{rgb}{0.000000,0.000000,0.000000}%
\pgfsetstrokecolor{currentstroke}%
\pgfsetdash{}{0pt}%
\pgfsys@defobject{currentmarker}{\pgfqpoint{-0.027778in}{0.000000in}}{\pgfqpoint{-0.000000in}{0.000000in}}{%
\pgfpathmoveto{\pgfqpoint{-0.000000in}{0.000000in}}%
\pgfpathlineto{\pgfqpoint{-0.027778in}{0.000000in}}%
\pgfusepath{stroke,fill}%
}%
\begin{pgfscope}%
\pgfsys@transformshift{0.675193in}{1.201671in}%
\pgfsys@useobject{currentmarker}{}%
\end{pgfscope}%
\end{pgfscope}%
\begin{pgfscope}%
\pgfsetbuttcap%
\pgfsetroundjoin%
\definecolor{currentfill}{rgb}{0.000000,0.000000,0.000000}%
\pgfsetfillcolor{currentfill}%
\pgfsetlinewidth{0.602250pt}%
\definecolor{currentstroke}{rgb}{0.000000,0.000000,0.000000}%
\pgfsetstrokecolor{currentstroke}%
\pgfsetdash{}{0pt}%
\pgfsys@defobject{currentmarker}{\pgfqpoint{-0.027778in}{0.000000in}}{\pgfqpoint{-0.000000in}{0.000000in}}{%
\pgfpathmoveto{\pgfqpoint{-0.000000in}{0.000000in}}%
\pgfpathlineto{\pgfqpoint{-0.027778in}{0.000000in}}%
\pgfusepath{stroke,fill}%
}%
\begin{pgfscope}%
\pgfsys@transformshift{0.675193in}{1.280657in}%
\pgfsys@useobject{currentmarker}{}%
\end{pgfscope}%
\end{pgfscope}%
\begin{pgfscope}%
\pgfsetbuttcap%
\pgfsetroundjoin%
\definecolor{currentfill}{rgb}{0.000000,0.000000,0.000000}%
\pgfsetfillcolor{currentfill}%
\pgfsetlinewidth{0.602250pt}%
\definecolor{currentstroke}{rgb}{0.000000,0.000000,0.000000}%
\pgfsetstrokecolor{currentstroke}%
\pgfsetdash{}{0pt}%
\pgfsys@defobject{currentmarker}{\pgfqpoint{-0.027778in}{0.000000in}}{\pgfqpoint{-0.000000in}{0.000000in}}{%
\pgfpathmoveto{\pgfqpoint{-0.000000in}{0.000000in}}%
\pgfpathlineto{\pgfqpoint{-0.027778in}{0.000000in}}%
\pgfusepath{stroke,fill}%
}%
\begin{pgfscope}%
\pgfsys@transformshift{0.675193in}{1.349079in}%
\pgfsys@useobject{currentmarker}{}%
\end{pgfscope}%
\end{pgfscope}%
\begin{pgfscope}%
\pgfsetbuttcap%
\pgfsetroundjoin%
\definecolor{currentfill}{rgb}{0.000000,0.000000,0.000000}%
\pgfsetfillcolor{currentfill}%
\pgfsetlinewidth{0.602250pt}%
\definecolor{currentstroke}{rgb}{0.000000,0.000000,0.000000}%
\pgfsetstrokecolor{currentstroke}%
\pgfsetdash{}{0pt}%
\pgfsys@defobject{currentmarker}{\pgfqpoint{-0.027778in}{0.000000in}}{\pgfqpoint{-0.000000in}{0.000000in}}{%
\pgfpathmoveto{\pgfqpoint{-0.000000in}{0.000000in}}%
\pgfpathlineto{\pgfqpoint{-0.027778in}{0.000000in}}%
\pgfusepath{stroke,fill}%
}%
\begin{pgfscope}%
\pgfsys@transformshift{0.675193in}{1.409431in}%
\pgfsys@useobject{currentmarker}{}%
\end{pgfscope}%
\end{pgfscope}%
\begin{pgfscope}%
\pgfsetbuttcap%
\pgfsetroundjoin%
\definecolor{currentfill}{rgb}{0.000000,0.000000,0.000000}%
\pgfsetfillcolor{currentfill}%
\pgfsetlinewidth{0.602250pt}%
\definecolor{currentstroke}{rgb}{0.000000,0.000000,0.000000}%
\pgfsetstrokecolor{currentstroke}%
\pgfsetdash{}{0pt}%
\pgfsys@defobject{currentmarker}{\pgfqpoint{-0.027778in}{0.000000in}}{\pgfqpoint{-0.000000in}{0.000000in}}{%
\pgfpathmoveto{\pgfqpoint{-0.000000in}{0.000000in}}%
\pgfpathlineto{\pgfqpoint{-0.027778in}{0.000000in}}%
\pgfusepath{stroke,fill}%
}%
\begin{pgfscope}%
\pgfsys@transformshift{0.675193in}{1.818587in}%
\pgfsys@useobject{currentmarker}{}%
\end{pgfscope}%
\end{pgfscope}%
\begin{pgfscope}%
\pgfsetbuttcap%
\pgfsetroundjoin%
\definecolor{currentfill}{rgb}{0.000000,0.000000,0.000000}%
\pgfsetfillcolor{currentfill}%
\pgfsetlinewidth{0.602250pt}%
\definecolor{currentstroke}{rgb}{0.000000,0.000000,0.000000}%
\pgfsetstrokecolor{currentstroke}%
\pgfsetdash{}{0pt}%
\pgfsys@defobject{currentmarker}{\pgfqpoint{-0.027778in}{0.000000in}}{\pgfqpoint{-0.000000in}{0.000000in}}{%
\pgfpathmoveto{\pgfqpoint{-0.000000in}{0.000000in}}%
\pgfpathlineto{\pgfqpoint{-0.027778in}{0.000000in}}%
\pgfusepath{stroke,fill}%
}%
\begin{pgfscope}%
\pgfsys@transformshift{0.675193in}{2.026348in}%
\pgfsys@useobject{currentmarker}{}%
\end{pgfscope}%
\end{pgfscope}%
\begin{pgfscope}%
\pgfsetbuttcap%
\pgfsetroundjoin%
\definecolor{currentfill}{rgb}{0.000000,0.000000,0.000000}%
\pgfsetfillcolor{currentfill}%
\pgfsetlinewidth{0.602250pt}%
\definecolor{currentstroke}{rgb}{0.000000,0.000000,0.000000}%
\pgfsetstrokecolor{currentstroke}%
\pgfsetdash{}{0pt}%
\pgfsys@defobject{currentmarker}{\pgfqpoint{-0.027778in}{0.000000in}}{\pgfqpoint{-0.000000in}{0.000000in}}{%
\pgfpathmoveto{\pgfqpoint{-0.000000in}{0.000000in}}%
\pgfpathlineto{\pgfqpoint{-0.027778in}{0.000000in}}%
\pgfusepath{stroke,fill}%
}%
\begin{pgfscope}%
\pgfsys@transformshift{0.675193in}{2.173756in}%
\pgfsys@useobject{currentmarker}{}%
\end{pgfscope}%
\end{pgfscope}%
\begin{pgfscope}%
\pgfsetbuttcap%
\pgfsetroundjoin%
\definecolor{currentfill}{rgb}{0.000000,0.000000,0.000000}%
\pgfsetfillcolor{currentfill}%
\pgfsetlinewidth{0.602250pt}%
\definecolor{currentstroke}{rgb}{0.000000,0.000000,0.000000}%
\pgfsetstrokecolor{currentstroke}%
\pgfsetdash{}{0pt}%
\pgfsys@defobject{currentmarker}{\pgfqpoint{-0.027778in}{0.000000in}}{\pgfqpoint{-0.000000in}{0.000000in}}{%
\pgfpathmoveto{\pgfqpoint{-0.000000in}{0.000000in}}%
\pgfpathlineto{\pgfqpoint{-0.027778in}{0.000000in}}%
\pgfusepath{stroke,fill}%
}%
\begin{pgfscope}%
\pgfsys@transformshift{0.675193in}{2.288095in}%
\pgfsys@useobject{currentmarker}{}%
\end{pgfscope}%
\end{pgfscope}%
\begin{pgfscope}%
\pgfsetbuttcap%
\pgfsetroundjoin%
\definecolor{currentfill}{rgb}{0.000000,0.000000,0.000000}%
\pgfsetfillcolor{currentfill}%
\pgfsetlinewidth{0.602250pt}%
\definecolor{currentstroke}{rgb}{0.000000,0.000000,0.000000}%
\pgfsetstrokecolor{currentstroke}%
\pgfsetdash{}{0pt}%
\pgfsys@defobject{currentmarker}{\pgfqpoint{-0.027778in}{0.000000in}}{\pgfqpoint{-0.000000in}{0.000000in}}{%
\pgfpathmoveto{\pgfqpoint{-0.000000in}{0.000000in}}%
\pgfpathlineto{\pgfqpoint{-0.027778in}{0.000000in}}%
\pgfusepath{stroke,fill}%
}%
\begin{pgfscope}%
\pgfsys@transformshift{0.675193in}{2.381517in}%
\pgfsys@useobject{currentmarker}{}%
\end{pgfscope}%
\end{pgfscope}%
\begin{pgfscope}%
\definecolor{textcolor}{rgb}{0.000000,0.000000,0.000000}%
\pgfsetstrokecolor{textcolor}%
\pgfsetfillcolor{textcolor}%
\pgftext[x=0.234413in,y=1.484103in,,bottom,rotate=90.000000]{\color{textcolor}{\sffamily\fontsize{10.000000}{12.000000}\selectfont\catcode`\^=\active\def^{\ifmmode\sp\else\^{}\fi}\catcode`\%=\active\def%{\%}$L^1$ error}}%
\end{pgfscope}%
\begin{pgfscope}%
\pgfpathrectangle{\pgfqpoint{0.675193in}{0.521603in}}{\pgfqpoint{1.937500in}{1.925000in}}%
\pgfusepath{clip}%
\pgfsetrectcap%
\pgfsetroundjoin%
\pgfsetlinewidth{1.003750pt}%
\definecolor{currentstroke}{rgb}{0.001462,0.000466,0.013866}%
\pgfsetstrokecolor{currentstroke}%
\pgfsetdash{}{0pt}%
\pgfpathmoveto{\pgfqpoint{0.763261in}{0.966084in}}%
\pgfpathlineto{\pgfqpoint{1.117426in}{0.949331in}}%
\pgfpathlineto{\pgfqpoint{1.469958in}{0.949406in}}%
\pgfpathlineto{\pgfqpoint{1.821848in}{0.949406in}}%
\pgfpathlineto{\pgfqpoint{2.172907in}{0.949406in}}%
\pgfpathlineto{\pgfqpoint{2.524625in}{0.949406in}}%
\pgfusepath{stroke}%
\end{pgfscope}%
\begin{pgfscope}%
\pgfpathrectangle{\pgfqpoint{0.675193in}{0.521603in}}{\pgfqpoint{1.937500in}{1.925000in}}%
\pgfusepath{clip}%
\pgfsetbuttcap%
\pgfsetroundjoin%
\definecolor{currentfill}{rgb}{0.001462,0.000466,0.013866}%
\pgfsetfillcolor{currentfill}%
\pgfsetlinewidth{1.003750pt}%
\definecolor{currentstroke}{rgb}{0.001462,0.000466,0.013866}%
\pgfsetstrokecolor{currentstroke}%
\pgfsetdash{}{0pt}%
\pgfsys@defobject{currentmarker}{\pgfqpoint{-0.020833in}{-0.020833in}}{\pgfqpoint{0.020833in}{0.020833in}}{%
\pgfpathmoveto{\pgfqpoint{0.000000in}{-0.020833in}}%
\pgfpathcurveto{\pgfqpoint{0.005525in}{-0.020833in}}{\pgfqpoint{0.010825in}{-0.018638in}}{\pgfqpoint{0.014731in}{-0.014731in}}%
\pgfpathcurveto{\pgfqpoint{0.018638in}{-0.010825in}}{\pgfqpoint{0.020833in}{-0.005525in}}{\pgfqpoint{0.020833in}{0.000000in}}%
\pgfpathcurveto{\pgfqpoint{0.020833in}{0.005525in}}{\pgfqpoint{0.018638in}{0.010825in}}{\pgfqpoint{0.014731in}{0.014731in}}%
\pgfpathcurveto{\pgfqpoint{0.010825in}{0.018638in}}{\pgfqpoint{0.005525in}{0.020833in}}{\pgfqpoint{0.000000in}{0.020833in}}%
\pgfpathcurveto{\pgfqpoint{-0.005525in}{0.020833in}}{\pgfqpoint{-0.010825in}{0.018638in}}{\pgfqpoint{-0.014731in}{0.014731in}}%
\pgfpathcurveto{\pgfqpoint{-0.018638in}{0.010825in}}{\pgfqpoint{-0.020833in}{0.005525in}}{\pgfqpoint{-0.020833in}{0.000000in}}%
\pgfpathcurveto{\pgfqpoint{-0.020833in}{-0.005525in}}{\pgfqpoint{-0.018638in}{-0.010825in}}{\pgfqpoint{-0.014731in}{-0.014731in}}%
\pgfpathcurveto{\pgfqpoint{-0.010825in}{-0.018638in}}{\pgfqpoint{-0.005525in}{-0.020833in}}{\pgfqpoint{0.000000in}{-0.020833in}}%
\pgfpathlineto{\pgfqpoint{0.000000in}{-0.020833in}}%
\pgfpathclose%
\pgfusepath{stroke,fill}%
}%
\begin{pgfscope}%
\pgfsys@transformshift{0.763261in}{0.966084in}%
\pgfsys@useobject{currentmarker}{}%
\end{pgfscope}%
\begin{pgfscope}%
\pgfsys@transformshift{1.117426in}{0.949331in}%
\pgfsys@useobject{currentmarker}{}%
\end{pgfscope}%
\begin{pgfscope}%
\pgfsys@transformshift{1.469958in}{0.949406in}%
\pgfsys@useobject{currentmarker}{}%
\end{pgfscope}%
\begin{pgfscope}%
\pgfsys@transformshift{1.821848in}{0.949406in}%
\pgfsys@useobject{currentmarker}{}%
\end{pgfscope}%
\begin{pgfscope}%
\pgfsys@transformshift{2.172907in}{0.949406in}%
\pgfsys@useobject{currentmarker}{}%
\end{pgfscope}%
\begin{pgfscope}%
\pgfsys@transformshift{2.524625in}{0.949406in}%
\pgfsys@useobject{currentmarker}{}%
\end{pgfscope}%
\end{pgfscope}%
\begin{pgfscope}%
\pgfpathrectangle{\pgfqpoint{0.675193in}{0.521603in}}{\pgfqpoint{1.937500in}{1.925000in}}%
\pgfusepath{clip}%
\pgfsetrectcap%
\pgfsetroundjoin%
\pgfsetlinewidth{1.003750pt}%
\definecolor{currentstroke}{rgb}{0.445163,0.122724,0.506901}%
\pgfsetstrokecolor{currentstroke}%
\pgfsetdash{}{0pt}%
\pgfpathmoveto{\pgfqpoint{0.763261in}{2.359103in}}%
\pgfpathlineto{\pgfqpoint{1.117426in}{2.117823in}}%
\pgfpathlineto{\pgfqpoint{1.469958in}{1.833395in}}%
\pgfpathlineto{\pgfqpoint{1.821848in}{1.862624in}}%
\pgfpathlineto{\pgfqpoint{2.172907in}{1.871278in}}%
\pgfpathlineto{\pgfqpoint{2.524625in}{1.871491in}}%
\pgfusepath{stroke}%
\end{pgfscope}%
\begin{pgfscope}%
\pgfpathrectangle{\pgfqpoint{0.675193in}{0.521603in}}{\pgfqpoint{1.937500in}{1.925000in}}%
\pgfusepath{clip}%
\pgfsetbuttcap%
\pgfsetroundjoin%
\definecolor{currentfill}{rgb}{0.445163,0.122724,0.506901}%
\pgfsetfillcolor{currentfill}%
\pgfsetlinewidth{1.003750pt}%
\definecolor{currentstroke}{rgb}{0.445163,0.122724,0.506901}%
\pgfsetstrokecolor{currentstroke}%
\pgfsetdash{}{0pt}%
\pgfsys@defobject{currentmarker}{\pgfqpoint{-0.020833in}{-0.020833in}}{\pgfqpoint{0.020833in}{0.020833in}}{%
\pgfpathmoveto{\pgfqpoint{0.000000in}{-0.020833in}}%
\pgfpathcurveto{\pgfqpoint{0.005525in}{-0.020833in}}{\pgfqpoint{0.010825in}{-0.018638in}}{\pgfqpoint{0.014731in}{-0.014731in}}%
\pgfpathcurveto{\pgfqpoint{0.018638in}{-0.010825in}}{\pgfqpoint{0.020833in}{-0.005525in}}{\pgfqpoint{0.020833in}{0.000000in}}%
\pgfpathcurveto{\pgfqpoint{0.020833in}{0.005525in}}{\pgfqpoint{0.018638in}{0.010825in}}{\pgfqpoint{0.014731in}{0.014731in}}%
\pgfpathcurveto{\pgfqpoint{0.010825in}{0.018638in}}{\pgfqpoint{0.005525in}{0.020833in}}{\pgfqpoint{0.000000in}{0.020833in}}%
\pgfpathcurveto{\pgfqpoint{-0.005525in}{0.020833in}}{\pgfqpoint{-0.010825in}{0.018638in}}{\pgfqpoint{-0.014731in}{0.014731in}}%
\pgfpathcurveto{\pgfqpoint{-0.018638in}{0.010825in}}{\pgfqpoint{-0.020833in}{0.005525in}}{\pgfqpoint{-0.020833in}{0.000000in}}%
\pgfpathcurveto{\pgfqpoint{-0.020833in}{-0.005525in}}{\pgfqpoint{-0.018638in}{-0.010825in}}{\pgfqpoint{-0.014731in}{-0.014731in}}%
\pgfpathcurveto{\pgfqpoint{-0.010825in}{-0.018638in}}{\pgfqpoint{-0.005525in}{-0.020833in}}{\pgfqpoint{0.000000in}{-0.020833in}}%
\pgfpathlineto{\pgfqpoint{0.000000in}{-0.020833in}}%
\pgfpathclose%
\pgfusepath{stroke,fill}%
}%
\begin{pgfscope}%
\pgfsys@transformshift{0.763261in}{2.359103in}%
\pgfsys@useobject{currentmarker}{}%
\end{pgfscope}%
\begin{pgfscope}%
\pgfsys@transformshift{1.117426in}{2.117823in}%
\pgfsys@useobject{currentmarker}{}%
\end{pgfscope}%
\begin{pgfscope}%
\pgfsys@transformshift{1.469958in}{1.833395in}%
\pgfsys@useobject{currentmarker}{}%
\end{pgfscope}%
\begin{pgfscope}%
\pgfsys@transformshift{1.821848in}{1.862624in}%
\pgfsys@useobject{currentmarker}{}%
\end{pgfscope}%
\begin{pgfscope}%
\pgfsys@transformshift{2.172907in}{1.871278in}%
\pgfsys@useobject{currentmarker}{}%
\end{pgfscope}%
\begin{pgfscope}%
\pgfsys@transformshift{2.524625in}{1.871491in}%
\pgfsys@useobject{currentmarker}{}%
\end{pgfscope}%
\end{pgfscope}%
\begin{pgfscope}%
\pgfpathrectangle{\pgfqpoint{0.675193in}{0.521603in}}{\pgfqpoint{1.937500in}{1.925000in}}%
\pgfusepath{clip}%
\pgfsetrectcap%
\pgfsetroundjoin%
\pgfsetlinewidth{1.003750pt}%
\definecolor{currentstroke}{rgb}{0.944006,0.377643,0.365136}%
\pgfsetstrokecolor{currentstroke}%
\pgfsetdash{}{0pt}%
\pgfpathmoveto{\pgfqpoint{0.763261in}{2.290986in}}%
\pgfpathlineto{\pgfqpoint{1.117426in}{1.885941in}}%
\pgfpathlineto{\pgfqpoint{1.469958in}{1.343470in}}%
\pgfpathlineto{\pgfqpoint{1.821848in}{0.697695in}}%
\pgfpathlineto{\pgfqpoint{2.172907in}{0.611690in}}%
\pgfpathlineto{\pgfqpoint{2.524625in}{0.609103in}}%
\pgfusepath{stroke}%
\end{pgfscope}%
\begin{pgfscope}%
\pgfpathrectangle{\pgfqpoint{0.675193in}{0.521603in}}{\pgfqpoint{1.937500in}{1.925000in}}%
\pgfusepath{clip}%
\pgfsetbuttcap%
\pgfsetroundjoin%
\definecolor{currentfill}{rgb}{0.944006,0.377643,0.365136}%
\pgfsetfillcolor{currentfill}%
\pgfsetlinewidth{1.003750pt}%
\definecolor{currentstroke}{rgb}{0.944006,0.377643,0.365136}%
\pgfsetstrokecolor{currentstroke}%
\pgfsetdash{}{0pt}%
\pgfsys@defobject{currentmarker}{\pgfqpoint{-0.020833in}{-0.020833in}}{\pgfqpoint{0.020833in}{0.020833in}}{%
\pgfpathmoveto{\pgfqpoint{0.000000in}{-0.020833in}}%
\pgfpathcurveto{\pgfqpoint{0.005525in}{-0.020833in}}{\pgfqpoint{0.010825in}{-0.018638in}}{\pgfqpoint{0.014731in}{-0.014731in}}%
\pgfpathcurveto{\pgfqpoint{0.018638in}{-0.010825in}}{\pgfqpoint{0.020833in}{-0.005525in}}{\pgfqpoint{0.020833in}{0.000000in}}%
\pgfpathcurveto{\pgfqpoint{0.020833in}{0.005525in}}{\pgfqpoint{0.018638in}{0.010825in}}{\pgfqpoint{0.014731in}{0.014731in}}%
\pgfpathcurveto{\pgfqpoint{0.010825in}{0.018638in}}{\pgfqpoint{0.005525in}{0.020833in}}{\pgfqpoint{0.000000in}{0.020833in}}%
\pgfpathcurveto{\pgfqpoint{-0.005525in}{0.020833in}}{\pgfqpoint{-0.010825in}{0.018638in}}{\pgfqpoint{-0.014731in}{0.014731in}}%
\pgfpathcurveto{\pgfqpoint{-0.018638in}{0.010825in}}{\pgfqpoint{-0.020833in}{0.005525in}}{\pgfqpoint{-0.020833in}{0.000000in}}%
\pgfpathcurveto{\pgfqpoint{-0.020833in}{-0.005525in}}{\pgfqpoint{-0.018638in}{-0.010825in}}{\pgfqpoint{-0.014731in}{-0.014731in}}%
\pgfpathcurveto{\pgfqpoint{-0.010825in}{-0.018638in}}{\pgfqpoint{-0.005525in}{-0.020833in}}{\pgfqpoint{0.000000in}{-0.020833in}}%
\pgfpathlineto{\pgfqpoint{0.000000in}{-0.020833in}}%
\pgfpathclose%
\pgfusepath{stroke,fill}%
}%
\begin{pgfscope}%
\pgfsys@transformshift{0.763261in}{2.290986in}%
\pgfsys@useobject{currentmarker}{}%
\end{pgfscope}%
\begin{pgfscope}%
\pgfsys@transformshift{1.117426in}{1.885941in}%
\pgfsys@useobject{currentmarker}{}%
\end{pgfscope}%
\begin{pgfscope}%
\pgfsys@transformshift{1.469958in}{1.343470in}%
\pgfsys@useobject{currentmarker}{}%
\end{pgfscope}%
\begin{pgfscope}%
\pgfsys@transformshift{1.821848in}{0.697695in}%
\pgfsys@useobject{currentmarker}{}%
\end{pgfscope}%
\begin{pgfscope}%
\pgfsys@transformshift{2.172907in}{0.611690in}%
\pgfsys@useobject{currentmarker}{}%
\end{pgfscope}%
\begin{pgfscope}%
\pgfsys@transformshift{2.524625in}{0.609103in}%
\pgfsys@useobject{currentmarker}{}%
\end{pgfscope}%
\end{pgfscope}%
\begin{pgfscope}%
\pgfsetrectcap%
\pgfsetmiterjoin%
\pgfsetlinewidth{0.803000pt}%
\definecolor{currentstroke}{rgb}{0.000000,0.000000,0.000000}%
\pgfsetstrokecolor{currentstroke}%
\pgfsetdash{}{0pt}%
\pgfpathmoveto{\pgfqpoint{0.675193in}{0.521603in}}%
\pgfpathlineto{\pgfqpoint{0.675193in}{2.446603in}}%
\pgfusepath{stroke}%
\end{pgfscope}%
\begin{pgfscope}%
\pgfsetrectcap%
\pgfsetmiterjoin%
\pgfsetlinewidth{0.803000pt}%
\definecolor{currentstroke}{rgb}{0.000000,0.000000,0.000000}%
\pgfsetstrokecolor{currentstroke}%
\pgfsetdash{}{0pt}%
\pgfpathmoveto{\pgfqpoint{2.612693in}{0.521603in}}%
\pgfpathlineto{\pgfqpoint{2.612693in}{2.446603in}}%
\pgfusepath{stroke}%
\end{pgfscope}%
\begin{pgfscope}%
\pgfsetrectcap%
\pgfsetmiterjoin%
\pgfsetlinewidth{0.803000pt}%
\definecolor{currentstroke}{rgb}{0.000000,0.000000,0.000000}%
\pgfsetstrokecolor{currentstroke}%
\pgfsetdash{}{0pt}%
\pgfpathmoveto{\pgfqpoint{0.675193in}{0.521603in}}%
\pgfpathlineto{\pgfqpoint{2.612693in}{0.521603in}}%
\pgfusepath{stroke}%
\end{pgfscope}%
\begin{pgfscope}%
\pgfsetrectcap%
\pgfsetmiterjoin%
\pgfsetlinewidth{0.803000pt}%
\definecolor{currentstroke}{rgb}{0.000000,0.000000,0.000000}%
\pgfsetstrokecolor{currentstroke}%
\pgfsetdash{}{0pt}%
\pgfpathmoveto{\pgfqpoint{0.675193in}{2.446603in}}%
\pgfpathlineto{\pgfqpoint{2.612693in}{2.446603in}}%
\pgfusepath{stroke}%
\end{pgfscope}%
\begin{pgfscope}%
\pgfsetbuttcap%
\pgfsetmiterjoin%
\definecolor{currentfill}{rgb}{1.000000,1.000000,1.000000}%
\pgfsetfillcolor{currentfill}%
\pgfsetfillopacity{0.800000}%
\pgfsetlinewidth{1.003750pt}%
\definecolor{currentstroke}{rgb}{0.800000,0.800000,0.800000}%
\pgfsetstrokecolor{currentstroke}%
\pgfsetstrokeopacity{0.800000}%
\pgfsetdash{}{0pt}%
\pgfpathmoveto{\pgfqpoint{1.469355in}{1.157484in}}%
\pgfpathlineto{\pgfqpoint{2.515471in}{1.157484in}}%
\pgfpathquadraticcurveto{\pgfqpoint{2.543249in}{1.157484in}}{\pgfqpoint{2.543249in}{1.185262in}}%
\pgfpathlineto{\pgfqpoint{2.543249in}{1.782945in}}%
\pgfpathquadraticcurveto{\pgfqpoint{2.543249in}{1.810723in}}{\pgfqpoint{2.515471in}{1.810723in}}%
\pgfpathlineto{\pgfqpoint{1.469355in}{1.810723in}}%
\pgfpathquadraticcurveto{\pgfqpoint{1.441578in}{1.810723in}}{\pgfqpoint{1.441578in}{1.782945in}}%
\pgfpathlineto{\pgfqpoint{1.441578in}{1.185262in}}%
\pgfpathquadraticcurveto{\pgfqpoint{1.441578in}{1.157484in}}{\pgfqpoint{1.469355in}{1.157484in}}%
\pgfpathlineto{\pgfqpoint{1.469355in}{1.157484in}}%
\pgfpathclose%
\pgfusepath{stroke,fill}%
\end{pgfscope}%
\begin{pgfscope}%
\pgfsetrectcap%
\pgfsetroundjoin%
\pgfsetlinewidth{1.003750pt}%
\definecolor{currentstroke}{rgb}{0.001462,0.000466,0.013866}%
\pgfsetstrokecolor{currentstroke}%
\pgfsetdash{}{0pt}%
\pgfpathmoveto{\pgfqpoint{1.497133in}{1.698255in}}%
\pgfpathlineto{\pgfqpoint{1.636022in}{1.698255in}}%
\pgfpathlineto{\pgfqpoint{1.774911in}{1.698255in}}%
\pgfusepath{stroke}%
\end{pgfscope}%
\begin{pgfscope}%
\pgfsetbuttcap%
\pgfsetroundjoin%
\definecolor{currentfill}{rgb}{0.001462,0.000466,0.013866}%
\pgfsetfillcolor{currentfill}%
\pgfsetlinewidth{1.003750pt}%
\definecolor{currentstroke}{rgb}{0.001462,0.000466,0.013866}%
\pgfsetstrokecolor{currentstroke}%
\pgfsetdash{}{0pt}%
\pgfsys@defobject{currentmarker}{\pgfqpoint{-0.020833in}{-0.020833in}}{\pgfqpoint{0.020833in}{0.020833in}}{%
\pgfpathmoveto{\pgfqpoint{0.000000in}{-0.020833in}}%
\pgfpathcurveto{\pgfqpoint{0.005525in}{-0.020833in}}{\pgfqpoint{0.010825in}{-0.018638in}}{\pgfqpoint{0.014731in}{-0.014731in}}%
\pgfpathcurveto{\pgfqpoint{0.018638in}{-0.010825in}}{\pgfqpoint{0.020833in}{-0.005525in}}{\pgfqpoint{0.020833in}{0.000000in}}%
\pgfpathcurveto{\pgfqpoint{0.020833in}{0.005525in}}{\pgfqpoint{0.018638in}{0.010825in}}{\pgfqpoint{0.014731in}{0.014731in}}%
\pgfpathcurveto{\pgfqpoint{0.010825in}{0.018638in}}{\pgfqpoint{0.005525in}{0.020833in}}{\pgfqpoint{0.000000in}{0.020833in}}%
\pgfpathcurveto{\pgfqpoint{-0.005525in}{0.020833in}}{\pgfqpoint{-0.010825in}{0.018638in}}{\pgfqpoint{-0.014731in}{0.014731in}}%
\pgfpathcurveto{\pgfqpoint{-0.018638in}{0.010825in}}{\pgfqpoint{-0.020833in}{0.005525in}}{\pgfqpoint{-0.020833in}{0.000000in}}%
\pgfpathcurveto{\pgfqpoint{-0.020833in}{-0.005525in}}{\pgfqpoint{-0.018638in}{-0.010825in}}{\pgfqpoint{-0.014731in}{-0.014731in}}%
\pgfpathcurveto{\pgfqpoint{-0.010825in}{-0.018638in}}{\pgfqpoint{-0.005525in}{-0.020833in}}{\pgfqpoint{0.000000in}{-0.020833in}}%
\pgfpathlineto{\pgfqpoint{0.000000in}{-0.020833in}}%
\pgfpathclose%
\pgfusepath{stroke,fill}%
}%
\begin{pgfscope}%
\pgfsys@transformshift{1.636022in}{1.698255in}%
\pgfsys@useobject{currentmarker}{}%
\end{pgfscope}%
\end{pgfscope}%
\begin{pgfscope}%
\definecolor{textcolor}{rgb}{0.000000,0.000000,0.000000}%
\pgfsetstrokecolor{textcolor}%
\pgfsetfillcolor{textcolor}%
\pgftext[x=1.886022in,y=1.649644in,left,base]{\color{textcolor}{\sffamily\fontsize{10.000000}{12.000000}\selectfont\catcode`\^=\active\def^{\ifmmode\sp\else\^{}\fi}\catcode`\%=\active\def%{\%}Haydock}}%
\end{pgfscope}%
\begin{pgfscope}%
\pgfsetrectcap%
\pgfsetroundjoin%
\pgfsetlinewidth{1.003750pt}%
\definecolor{currentstroke}{rgb}{0.445163,0.122724,0.506901}%
\pgfsetstrokecolor{currentstroke}%
\pgfsetdash{}{0pt}%
\pgfpathmoveto{\pgfqpoint{1.497133in}{1.494398in}}%
\pgfpathlineto{\pgfqpoint{1.636022in}{1.494398in}}%
\pgfpathlineto{\pgfqpoint{1.774911in}{1.494398in}}%
\pgfusepath{stroke}%
\end{pgfscope}%
\begin{pgfscope}%
\pgfsetbuttcap%
\pgfsetroundjoin%
\definecolor{currentfill}{rgb}{0.445163,0.122724,0.506901}%
\pgfsetfillcolor{currentfill}%
\pgfsetlinewidth{1.003750pt}%
\definecolor{currentstroke}{rgb}{0.445163,0.122724,0.506901}%
\pgfsetstrokecolor{currentstroke}%
\pgfsetdash{}{0pt}%
\pgfsys@defobject{currentmarker}{\pgfqpoint{-0.020833in}{-0.020833in}}{\pgfqpoint{0.020833in}{0.020833in}}{%
\pgfpathmoveto{\pgfqpoint{0.000000in}{-0.020833in}}%
\pgfpathcurveto{\pgfqpoint{0.005525in}{-0.020833in}}{\pgfqpoint{0.010825in}{-0.018638in}}{\pgfqpoint{0.014731in}{-0.014731in}}%
\pgfpathcurveto{\pgfqpoint{0.018638in}{-0.010825in}}{\pgfqpoint{0.020833in}{-0.005525in}}{\pgfqpoint{0.020833in}{0.000000in}}%
\pgfpathcurveto{\pgfqpoint{0.020833in}{0.005525in}}{\pgfqpoint{0.018638in}{0.010825in}}{\pgfqpoint{0.014731in}{0.014731in}}%
\pgfpathcurveto{\pgfqpoint{0.010825in}{0.018638in}}{\pgfqpoint{0.005525in}{0.020833in}}{\pgfqpoint{0.000000in}{0.020833in}}%
\pgfpathcurveto{\pgfqpoint{-0.005525in}{0.020833in}}{\pgfqpoint{-0.010825in}{0.018638in}}{\pgfqpoint{-0.014731in}{0.014731in}}%
\pgfpathcurveto{\pgfqpoint{-0.018638in}{0.010825in}}{\pgfqpoint{-0.020833in}{0.005525in}}{\pgfqpoint{-0.020833in}{0.000000in}}%
\pgfpathcurveto{\pgfqpoint{-0.020833in}{-0.005525in}}{\pgfqpoint{-0.018638in}{-0.010825in}}{\pgfqpoint{-0.014731in}{-0.014731in}}%
\pgfpathcurveto{\pgfqpoint{-0.010825in}{-0.018638in}}{\pgfqpoint{-0.005525in}{-0.020833in}}{\pgfqpoint{0.000000in}{-0.020833in}}%
\pgfpathlineto{\pgfqpoint{0.000000in}{-0.020833in}}%
\pgfpathclose%
\pgfusepath{stroke,fill}%
}%
\begin{pgfscope}%
\pgfsys@transformshift{1.636022in}{1.494398in}%
\pgfsys@useobject{currentmarker}{}%
\end{pgfscope}%
\end{pgfscope}%
\begin{pgfscope}%
\definecolor{textcolor}{rgb}{0.000000,0.000000,0.000000}%
\pgfsetstrokecolor{textcolor}%
\pgfsetfillcolor{textcolor}%
\pgftext[x=1.886022in,y=1.445787in,left,base]{\color{textcolor}{\sffamily\fontsize{10.000000}{12.000000}\selectfont\catcode`\^=\active\def^{\ifmmode\sp\else\^{}\fi}\catcode`\%=\active\def%{\%}NC}}%
\end{pgfscope}%
\begin{pgfscope}%
\pgfsetrectcap%
\pgfsetroundjoin%
\pgfsetlinewidth{1.003750pt}%
\definecolor{currentstroke}{rgb}{0.944006,0.377643,0.365136}%
\pgfsetstrokecolor{currentstroke}%
\pgfsetdash{}{0pt}%
\pgfpathmoveto{\pgfqpoint{1.497133in}{1.290541in}}%
\pgfpathlineto{\pgfqpoint{1.636022in}{1.290541in}}%
\pgfpathlineto{\pgfqpoint{1.774911in}{1.290541in}}%
\pgfusepath{stroke}%
\end{pgfscope}%
\begin{pgfscope}%
\pgfsetbuttcap%
\pgfsetroundjoin%
\definecolor{currentfill}{rgb}{0.944006,0.377643,0.365136}%
\pgfsetfillcolor{currentfill}%
\pgfsetlinewidth{1.003750pt}%
\definecolor{currentstroke}{rgb}{0.944006,0.377643,0.365136}%
\pgfsetstrokecolor{currentstroke}%
\pgfsetdash{}{0pt}%
\pgfsys@defobject{currentmarker}{\pgfqpoint{-0.020833in}{-0.020833in}}{\pgfqpoint{0.020833in}{0.020833in}}{%
\pgfpathmoveto{\pgfqpoint{0.000000in}{-0.020833in}}%
\pgfpathcurveto{\pgfqpoint{0.005525in}{-0.020833in}}{\pgfqpoint{0.010825in}{-0.018638in}}{\pgfqpoint{0.014731in}{-0.014731in}}%
\pgfpathcurveto{\pgfqpoint{0.018638in}{-0.010825in}}{\pgfqpoint{0.020833in}{-0.005525in}}{\pgfqpoint{0.020833in}{0.000000in}}%
\pgfpathcurveto{\pgfqpoint{0.020833in}{0.005525in}}{\pgfqpoint{0.018638in}{0.010825in}}{\pgfqpoint{0.014731in}{0.014731in}}%
\pgfpathcurveto{\pgfqpoint{0.010825in}{0.018638in}}{\pgfqpoint{0.005525in}{0.020833in}}{\pgfqpoint{0.000000in}{0.020833in}}%
\pgfpathcurveto{\pgfqpoint{-0.005525in}{0.020833in}}{\pgfqpoint{-0.010825in}{0.018638in}}{\pgfqpoint{-0.014731in}{0.014731in}}%
\pgfpathcurveto{\pgfqpoint{-0.018638in}{0.010825in}}{\pgfqpoint{-0.020833in}{0.005525in}}{\pgfqpoint{-0.020833in}{0.000000in}}%
\pgfpathcurveto{\pgfqpoint{-0.020833in}{-0.005525in}}{\pgfqpoint{-0.018638in}{-0.010825in}}{\pgfqpoint{-0.014731in}{-0.014731in}}%
\pgfpathcurveto{\pgfqpoint{-0.010825in}{-0.018638in}}{\pgfqpoint{-0.005525in}{-0.020833in}}{\pgfqpoint{0.000000in}{-0.020833in}}%
\pgfpathlineto{\pgfqpoint{0.000000in}{-0.020833in}}%
\pgfpathclose%
\pgfusepath{stroke,fill}%
}%
\begin{pgfscope}%
\pgfsys@transformshift{1.636022in}{1.290541in}%
\pgfsys@useobject{currentmarker}{}%
\end{pgfscope}%
\end{pgfscope}%
\begin{pgfscope}%
\definecolor{textcolor}{rgb}{0.000000,0.000000,0.000000}%
\pgfsetstrokecolor{textcolor}%
\pgfsetfillcolor{textcolor}%
\pgftext[x=1.886022in,y=1.241929in,left,base]{\color{textcolor}{\sffamily\fontsize{10.000000}{12.000000}\selectfont\catcode`\^=\active\def^{\ifmmode\sp\else\^{}\fi}\catcode`\%=\active\def%{\%}NC++}}%
\end{pgfscope}%
\end{pgfpicture}%
\makeatother%
\endgroup%

        \caption{$n_{\Omega}=40$}
        \label{fig:5-experiments-haydock-convergence-m-nv40}
    \end{subfigure}
    \begin{subfigure}[b]{0.49\columnwidth}
        %% Creator: Matplotlib, PGF backend
%%
%% To include the figure in your LaTeX document, write
%%   \input{<filename>.pgf}
%%
%% Make sure the required packages are loaded in your preamble
%%   \usepackage{pgf}
%%
%% Also ensure that all the required font packages are loaded; for instance,
%% the lmodern package is sometimes necessary when using math font.
%%   \usepackage{lmodern}
%%
%% Figures using additional raster images can only be included by \input if
%% they are in the same directory as the main LaTeX file. For loading figures
%% from other directories you can use the `import` package
%%   \usepackage{import}
%%
%% and then include the figures with
%%   \import{<path to file>}{<filename>.pgf}
%%
%% Matplotlib used the following preamble
%%   \def\mathdefault#1{#1}
%%   \everymath=\expandafter{\the\everymath\displaystyle}
%%   
%%   \makeatletter\@ifpackageloaded{underscore}{}{\usepackage[strings]{underscore}}\makeatother
%%
\begingroup%
\makeatletter%
\begin{pgfpicture}%
\pgfpathrectangle{\pgfpointorigin}{\pgfqpoint{2.759413in}{2.574073in}}%
\pgfusepath{use as bounding box, clip}%
\begin{pgfscope}%
\pgfsetbuttcap%
\pgfsetmiterjoin%
\definecolor{currentfill}{rgb}{1.000000,1.000000,1.000000}%
\pgfsetfillcolor{currentfill}%
\pgfsetlinewidth{0.000000pt}%
\definecolor{currentstroke}{rgb}{1.000000,1.000000,1.000000}%
\pgfsetstrokecolor{currentstroke}%
\pgfsetdash{}{0pt}%
\pgfpathmoveto{\pgfqpoint{0.000000in}{0.000000in}}%
\pgfpathlineto{\pgfqpoint{2.759413in}{0.000000in}}%
\pgfpathlineto{\pgfqpoint{2.759413in}{2.574073in}}%
\pgfpathlineto{\pgfqpoint{0.000000in}{2.574073in}}%
\pgfpathlineto{\pgfqpoint{0.000000in}{0.000000in}}%
\pgfpathclose%
\pgfusepath{fill}%
\end{pgfscope}%
\begin{pgfscope}%
\pgfsetbuttcap%
\pgfsetmiterjoin%
\definecolor{currentfill}{rgb}{1.000000,1.000000,1.000000}%
\pgfsetfillcolor{currentfill}%
\pgfsetlinewidth{0.000000pt}%
\definecolor{currentstroke}{rgb}{0.000000,0.000000,0.000000}%
\pgfsetstrokecolor{currentstroke}%
\pgfsetstrokeopacity{0.000000}%
\pgfsetdash{}{0pt}%
\pgfpathmoveto{\pgfqpoint{0.721913in}{0.549073in}}%
\pgfpathlineto{\pgfqpoint{2.659413in}{0.549073in}}%
\pgfpathlineto{\pgfqpoint{2.659413in}{2.474073in}}%
\pgfpathlineto{\pgfqpoint{0.721913in}{2.474073in}}%
\pgfpathlineto{\pgfqpoint{0.721913in}{0.549073in}}%
\pgfpathclose%
\pgfusepath{fill}%
\end{pgfscope}%
\begin{pgfscope}%
\pgfsetbuttcap%
\pgfsetroundjoin%
\definecolor{currentfill}{rgb}{0.000000,0.000000,0.000000}%
\pgfsetfillcolor{currentfill}%
\pgfsetlinewidth{0.803000pt}%
\definecolor{currentstroke}{rgb}{0.000000,0.000000,0.000000}%
\pgfsetstrokecolor{currentstroke}%
\pgfsetdash{}{0pt}%
\pgfsys@defobject{currentmarker}{\pgfqpoint{0.000000in}{-0.048611in}}{\pgfqpoint{0.000000in}{0.000000in}}{%
\pgfpathmoveto{\pgfqpoint{0.000000in}{0.000000in}}%
\pgfpathlineto{\pgfqpoint{0.000000in}{-0.048611in}}%
\pgfusepath{stroke,fill}%
}%
\begin{pgfscope}%
\pgfsys@transformshift{0.921272in}{0.549073in}%
\pgfsys@useobject{currentmarker}{}%
\end{pgfscope}%
\end{pgfscope}%
\begin{pgfscope}%
\definecolor{textcolor}{rgb}{0.000000,0.000000,0.000000}%
\pgfsetstrokecolor{textcolor}%
\pgfsetfillcolor{textcolor}%
\pgftext[x=0.921272in,y=0.451851in,,top]{\color{textcolor}{\rmfamily\fontsize{12.000000}{14.400000}\selectfont\catcode`\^=\active\def^{\ifmmode\sp\else\^{}\fi}\catcode`\%=\active\def%{\%}$\mathdefault{10^{2}}$}}%
\end{pgfscope}%
\begin{pgfscope}%
\pgfsetbuttcap%
\pgfsetroundjoin%
\definecolor{currentfill}{rgb}{0.000000,0.000000,0.000000}%
\pgfsetfillcolor{currentfill}%
\pgfsetlinewidth{0.803000pt}%
\definecolor{currentstroke}{rgb}{0.000000,0.000000,0.000000}%
\pgfsetstrokecolor{currentstroke}%
\pgfsetdash{}{0pt}%
\pgfsys@defobject{currentmarker}{\pgfqpoint{0.000000in}{-0.048611in}}{\pgfqpoint{0.000000in}{0.000000in}}{%
\pgfpathmoveto{\pgfqpoint{0.000000in}{0.000000in}}%
\pgfpathlineto{\pgfqpoint{0.000000in}{-0.048611in}}%
\pgfusepath{stroke,fill}%
}%
\begin{pgfscope}%
\pgfsys@transformshift{1.952643in}{0.549073in}%
\pgfsys@useobject{currentmarker}{}%
\end{pgfscope}%
\end{pgfscope}%
\begin{pgfscope}%
\definecolor{textcolor}{rgb}{0.000000,0.000000,0.000000}%
\pgfsetstrokecolor{textcolor}%
\pgfsetfillcolor{textcolor}%
\pgftext[x=1.952643in,y=0.451851in,,top]{\color{textcolor}{\rmfamily\fontsize{12.000000}{14.400000}\selectfont\catcode`\^=\active\def^{\ifmmode\sp\else\^{}\fi}\catcode`\%=\active\def%{\%}$\mathdefault{10^{3}}$}}%
\end{pgfscope}%
\begin{pgfscope}%
\pgfsetbuttcap%
\pgfsetroundjoin%
\definecolor{currentfill}{rgb}{0.000000,0.000000,0.000000}%
\pgfsetfillcolor{currentfill}%
\pgfsetlinewidth{0.602250pt}%
\definecolor{currentstroke}{rgb}{0.000000,0.000000,0.000000}%
\pgfsetstrokecolor{currentstroke}%
\pgfsetdash{}{0pt}%
\pgfsys@defobject{currentmarker}{\pgfqpoint{0.000000in}{-0.027778in}}{\pgfqpoint{0.000000in}{0.000000in}}{%
\pgfpathmoveto{\pgfqpoint{0.000000in}{0.000000in}}%
\pgfpathlineto{\pgfqpoint{0.000000in}{-0.027778in}}%
\pgfusepath{stroke,fill}%
}%
\begin{pgfscope}%
\pgfsys@transformshift{0.761511in}{0.549073in}%
\pgfsys@useobject{currentmarker}{}%
\end{pgfscope}%
\end{pgfscope}%
\begin{pgfscope}%
\pgfsetbuttcap%
\pgfsetroundjoin%
\definecolor{currentfill}{rgb}{0.000000,0.000000,0.000000}%
\pgfsetfillcolor{currentfill}%
\pgfsetlinewidth{0.602250pt}%
\definecolor{currentstroke}{rgb}{0.000000,0.000000,0.000000}%
\pgfsetstrokecolor{currentstroke}%
\pgfsetdash{}{0pt}%
\pgfsys@defobject{currentmarker}{\pgfqpoint{0.000000in}{-0.027778in}}{\pgfqpoint{0.000000in}{0.000000in}}{%
\pgfpathmoveto{\pgfqpoint{0.000000in}{0.000000in}}%
\pgfpathlineto{\pgfqpoint{0.000000in}{-0.027778in}}%
\pgfusepath{stroke,fill}%
}%
\begin{pgfscope}%
\pgfsys@transformshift{0.821322in}{0.549073in}%
\pgfsys@useobject{currentmarker}{}%
\end{pgfscope}%
\end{pgfscope}%
\begin{pgfscope}%
\pgfsetbuttcap%
\pgfsetroundjoin%
\definecolor{currentfill}{rgb}{0.000000,0.000000,0.000000}%
\pgfsetfillcolor{currentfill}%
\pgfsetlinewidth{0.602250pt}%
\definecolor{currentstroke}{rgb}{0.000000,0.000000,0.000000}%
\pgfsetstrokecolor{currentstroke}%
\pgfsetdash{}{0pt}%
\pgfsys@defobject{currentmarker}{\pgfqpoint{0.000000in}{-0.027778in}}{\pgfqpoint{0.000000in}{0.000000in}}{%
\pgfpathmoveto{\pgfqpoint{0.000000in}{0.000000in}}%
\pgfpathlineto{\pgfqpoint{0.000000in}{-0.027778in}}%
\pgfusepath{stroke,fill}%
}%
\begin{pgfscope}%
\pgfsys@transformshift{0.874079in}{0.549073in}%
\pgfsys@useobject{currentmarker}{}%
\end{pgfscope}%
\end{pgfscope}%
\begin{pgfscope}%
\pgfsetbuttcap%
\pgfsetroundjoin%
\definecolor{currentfill}{rgb}{0.000000,0.000000,0.000000}%
\pgfsetfillcolor{currentfill}%
\pgfsetlinewidth{0.602250pt}%
\definecolor{currentstroke}{rgb}{0.000000,0.000000,0.000000}%
\pgfsetstrokecolor{currentstroke}%
\pgfsetdash{}{0pt}%
\pgfsys@defobject{currentmarker}{\pgfqpoint{0.000000in}{-0.027778in}}{\pgfqpoint{0.000000in}{0.000000in}}{%
\pgfpathmoveto{\pgfqpoint{0.000000in}{0.000000in}}%
\pgfpathlineto{\pgfqpoint{0.000000in}{-0.027778in}}%
\pgfusepath{stroke,fill}%
}%
\begin{pgfscope}%
\pgfsys@transformshift{1.231746in}{0.549073in}%
\pgfsys@useobject{currentmarker}{}%
\end{pgfscope}%
\end{pgfscope}%
\begin{pgfscope}%
\pgfsetbuttcap%
\pgfsetroundjoin%
\definecolor{currentfill}{rgb}{0.000000,0.000000,0.000000}%
\pgfsetfillcolor{currentfill}%
\pgfsetlinewidth{0.602250pt}%
\definecolor{currentstroke}{rgb}{0.000000,0.000000,0.000000}%
\pgfsetstrokecolor{currentstroke}%
\pgfsetdash{}{0pt}%
\pgfsys@defobject{currentmarker}{\pgfqpoint{0.000000in}{-0.027778in}}{\pgfqpoint{0.000000in}{0.000000in}}{%
\pgfpathmoveto{\pgfqpoint{0.000000in}{0.000000in}}%
\pgfpathlineto{\pgfqpoint{0.000000in}{-0.027778in}}%
\pgfusepath{stroke,fill}%
}%
\begin{pgfscope}%
\pgfsys@transformshift{1.413361in}{0.549073in}%
\pgfsys@useobject{currentmarker}{}%
\end{pgfscope}%
\end{pgfscope}%
\begin{pgfscope}%
\pgfsetbuttcap%
\pgfsetroundjoin%
\definecolor{currentfill}{rgb}{0.000000,0.000000,0.000000}%
\pgfsetfillcolor{currentfill}%
\pgfsetlinewidth{0.602250pt}%
\definecolor{currentstroke}{rgb}{0.000000,0.000000,0.000000}%
\pgfsetstrokecolor{currentstroke}%
\pgfsetdash{}{0pt}%
\pgfsys@defobject{currentmarker}{\pgfqpoint{0.000000in}{-0.027778in}}{\pgfqpoint{0.000000in}{0.000000in}}{%
\pgfpathmoveto{\pgfqpoint{0.000000in}{0.000000in}}%
\pgfpathlineto{\pgfqpoint{0.000000in}{-0.027778in}}%
\pgfusepath{stroke,fill}%
}%
\begin{pgfscope}%
\pgfsys@transformshift{1.542219in}{0.549073in}%
\pgfsys@useobject{currentmarker}{}%
\end{pgfscope}%
\end{pgfscope}%
\begin{pgfscope}%
\pgfsetbuttcap%
\pgfsetroundjoin%
\definecolor{currentfill}{rgb}{0.000000,0.000000,0.000000}%
\pgfsetfillcolor{currentfill}%
\pgfsetlinewidth{0.602250pt}%
\definecolor{currentstroke}{rgb}{0.000000,0.000000,0.000000}%
\pgfsetstrokecolor{currentstroke}%
\pgfsetdash{}{0pt}%
\pgfsys@defobject{currentmarker}{\pgfqpoint{0.000000in}{-0.027778in}}{\pgfqpoint{0.000000in}{0.000000in}}{%
\pgfpathmoveto{\pgfqpoint{0.000000in}{0.000000in}}%
\pgfpathlineto{\pgfqpoint{0.000000in}{-0.027778in}}%
\pgfusepath{stroke,fill}%
}%
\begin{pgfscope}%
\pgfsys@transformshift{1.642170in}{0.549073in}%
\pgfsys@useobject{currentmarker}{}%
\end{pgfscope}%
\end{pgfscope}%
\begin{pgfscope}%
\pgfsetbuttcap%
\pgfsetroundjoin%
\definecolor{currentfill}{rgb}{0.000000,0.000000,0.000000}%
\pgfsetfillcolor{currentfill}%
\pgfsetlinewidth{0.602250pt}%
\definecolor{currentstroke}{rgb}{0.000000,0.000000,0.000000}%
\pgfsetstrokecolor{currentstroke}%
\pgfsetdash{}{0pt}%
\pgfsys@defobject{currentmarker}{\pgfqpoint{0.000000in}{-0.027778in}}{\pgfqpoint{0.000000in}{0.000000in}}{%
\pgfpathmoveto{\pgfqpoint{0.000000in}{0.000000in}}%
\pgfpathlineto{\pgfqpoint{0.000000in}{-0.027778in}}%
\pgfusepath{stroke,fill}%
}%
\begin{pgfscope}%
\pgfsys@transformshift{1.723835in}{0.549073in}%
\pgfsys@useobject{currentmarker}{}%
\end{pgfscope}%
\end{pgfscope}%
\begin{pgfscope}%
\pgfsetbuttcap%
\pgfsetroundjoin%
\definecolor{currentfill}{rgb}{0.000000,0.000000,0.000000}%
\pgfsetfillcolor{currentfill}%
\pgfsetlinewidth{0.602250pt}%
\definecolor{currentstroke}{rgb}{0.000000,0.000000,0.000000}%
\pgfsetstrokecolor{currentstroke}%
\pgfsetdash{}{0pt}%
\pgfsys@defobject{currentmarker}{\pgfqpoint{0.000000in}{-0.027778in}}{\pgfqpoint{0.000000in}{0.000000in}}{%
\pgfpathmoveto{\pgfqpoint{0.000000in}{0.000000in}}%
\pgfpathlineto{\pgfqpoint{0.000000in}{-0.027778in}}%
\pgfusepath{stroke,fill}%
}%
\begin{pgfscope}%
\pgfsys@transformshift{1.792882in}{0.549073in}%
\pgfsys@useobject{currentmarker}{}%
\end{pgfscope}%
\end{pgfscope}%
\begin{pgfscope}%
\pgfsetbuttcap%
\pgfsetroundjoin%
\definecolor{currentfill}{rgb}{0.000000,0.000000,0.000000}%
\pgfsetfillcolor{currentfill}%
\pgfsetlinewidth{0.602250pt}%
\definecolor{currentstroke}{rgb}{0.000000,0.000000,0.000000}%
\pgfsetstrokecolor{currentstroke}%
\pgfsetdash{}{0pt}%
\pgfsys@defobject{currentmarker}{\pgfqpoint{0.000000in}{-0.027778in}}{\pgfqpoint{0.000000in}{0.000000in}}{%
\pgfpathmoveto{\pgfqpoint{0.000000in}{0.000000in}}%
\pgfpathlineto{\pgfqpoint{0.000000in}{-0.027778in}}%
\pgfusepath{stroke,fill}%
}%
\begin{pgfscope}%
\pgfsys@transformshift{1.852693in}{0.549073in}%
\pgfsys@useobject{currentmarker}{}%
\end{pgfscope}%
\end{pgfscope}%
\begin{pgfscope}%
\pgfsetbuttcap%
\pgfsetroundjoin%
\definecolor{currentfill}{rgb}{0.000000,0.000000,0.000000}%
\pgfsetfillcolor{currentfill}%
\pgfsetlinewidth{0.602250pt}%
\definecolor{currentstroke}{rgb}{0.000000,0.000000,0.000000}%
\pgfsetstrokecolor{currentstroke}%
\pgfsetdash{}{0pt}%
\pgfsys@defobject{currentmarker}{\pgfqpoint{0.000000in}{-0.027778in}}{\pgfqpoint{0.000000in}{0.000000in}}{%
\pgfpathmoveto{\pgfqpoint{0.000000in}{0.000000in}}%
\pgfpathlineto{\pgfqpoint{0.000000in}{-0.027778in}}%
\pgfusepath{stroke,fill}%
}%
\begin{pgfscope}%
\pgfsys@transformshift{1.905450in}{0.549073in}%
\pgfsys@useobject{currentmarker}{}%
\end{pgfscope}%
\end{pgfscope}%
\begin{pgfscope}%
\pgfsetbuttcap%
\pgfsetroundjoin%
\definecolor{currentfill}{rgb}{0.000000,0.000000,0.000000}%
\pgfsetfillcolor{currentfill}%
\pgfsetlinewidth{0.602250pt}%
\definecolor{currentstroke}{rgb}{0.000000,0.000000,0.000000}%
\pgfsetstrokecolor{currentstroke}%
\pgfsetdash{}{0pt}%
\pgfsys@defobject{currentmarker}{\pgfqpoint{0.000000in}{-0.027778in}}{\pgfqpoint{0.000000in}{0.000000in}}{%
\pgfpathmoveto{\pgfqpoint{0.000000in}{0.000000in}}%
\pgfpathlineto{\pgfqpoint{0.000000in}{-0.027778in}}%
\pgfusepath{stroke,fill}%
}%
\begin{pgfscope}%
\pgfsys@transformshift{2.263117in}{0.549073in}%
\pgfsys@useobject{currentmarker}{}%
\end{pgfscope}%
\end{pgfscope}%
\begin{pgfscope}%
\pgfsetbuttcap%
\pgfsetroundjoin%
\definecolor{currentfill}{rgb}{0.000000,0.000000,0.000000}%
\pgfsetfillcolor{currentfill}%
\pgfsetlinewidth{0.602250pt}%
\definecolor{currentstroke}{rgb}{0.000000,0.000000,0.000000}%
\pgfsetstrokecolor{currentstroke}%
\pgfsetdash{}{0pt}%
\pgfsys@defobject{currentmarker}{\pgfqpoint{0.000000in}{-0.027778in}}{\pgfqpoint{0.000000in}{0.000000in}}{%
\pgfpathmoveto{\pgfqpoint{0.000000in}{0.000000in}}%
\pgfpathlineto{\pgfqpoint{0.000000in}{-0.027778in}}%
\pgfusepath{stroke,fill}%
}%
\begin{pgfscope}%
\pgfsys@transformshift{2.444732in}{0.549073in}%
\pgfsys@useobject{currentmarker}{}%
\end{pgfscope}%
\end{pgfscope}%
\begin{pgfscope}%
\pgfsetbuttcap%
\pgfsetroundjoin%
\definecolor{currentfill}{rgb}{0.000000,0.000000,0.000000}%
\pgfsetfillcolor{currentfill}%
\pgfsetlinewidth{0.602250pt}%
\definecolor{currentstroke}{rgb}{0.000000,0.000000,0.000000}%
\pgfsetstrokecolor{currentstroke}%
\pgfsetdash{}{0pt}%
\pgfsys@defobject{currentmarker}{\pgfqpoint{0.000000in}{-0.027778in}}{\pgfqpoint{0.000000in}{0.000000in}}{%
\pgfpathmoveto{\pgfqpoint{0.000000in}{0.000000in}}%
\pgfpathlineto{\pgfqpoint{0.000000in}{-0.027778in}}%
\pgfusepath{stroke,fill}%
}%
\begin{pgfscope}%
\pgfsys@transformshift{2.573591in}{0.549073in}%
\pgfsys@useobject{currentmarker}{}%
\end{pgfscope}%
\end{pgfscope}%
\begin{pgfscope}%
\definecolor{textcolor}{rgb}{0.000000,0.000000,0.000000}%
\pgfsetstrokecolor{textcolor}%
\pgfsetfillcolor{textcolor}%
\pgftext[x=1.690663in,y=0.248148in,,top]{\color{textcolor}{\rmfamily\fontsize{12.000000}{14.400000}\selectfont\catcode`\^=\active\def^{\ifmmode\sp\else\^{}\fi}\catcode`\%=\active\def%{\%}$m$}}%
\end{pgfscope}%
\begin{pgfscope}%
\pgfsetbuttcap%
\pgfsetroundjoin%
\definecolor{currentfill}{rgb}{0.000000,0.000000,0.000000}%
\pgfsetfillcolor{currentfill}%
\pgfsetlinewidth{0.803000pt}%
\definecolor{currentstroke}{rgb}{0.000000,0.000000,0.000000}%
\pgfsetstrokecolor{currentstroke}%
\pgfsetdash{}{0pt}%
\pgfsys@defobject{currentmarker}{\pgfqpoint{-0.048611in}{0.000000in}}{\pgfqpoint{-0.000000in}{0.000000in}}{%
\pgfpathmoveto{\pgfqpoint{-0.000000in}{0.000000in}}%
\pgfpathlineto{\pgfqpoint{-0.048611in}{0.000000in}}%
\pgfusepath{stroke,fill}%
}%
\begin{pgfscope}%
\pgfsys@transformshift{0.721913in}{1.083102in}%
\pgfsys@useobject{currentmarker}{}%
\end{pgfscope}%
\end{pgfscope}%
\begin{pgfscope}%
\definecolor{textcolor}{rgb}{0.000000,0.000000,0.000000}%
\pgfsetstrokecolor{textcolor}%
\pgfsetfillcolor{textcolor}%
\pgftext[x=0.303703in, y=1.025231in, left, base]{\color{textcolor}{\rmfamily\fontsize{12.000000}{14.400000}\selectfont\catcode`\^=\active\def^{\ifmmode\sp\else\^{}\fi}\catcode`\%=\active\def%{\%}$\mathdefault{10^{-2}}$}}%
\end{pgfscope}%
\begin{pgfscope}%
\pgfsetbuttcap%
\pgfsetroundjoin%
\definecolor{currentfill}{rgb}{0.000000,0.000000,0.000000}%
\pgfsetfillcolor{currentfill}%
\pgfsetlinewidth{0.803000pt}%
\definecolor{currentstroke}{rgb}{0.000000,0.000000,0.000000}%
\pgfsetstrokecolor{currentstroke}%
\pgfsetdash{}{0pt}%
\pgfsys@defobject{currentmarker}{\pgfqpoint{-0.048611in}{0.000000in}}{\pgfqpoint{-0.000000in}{0.000000in}}{%
\pgfpathmoveto{\pgfqpoint{-0.000000in}{0.000000in}}%
\pgfpathlineto{\pgfqpoint{-0.048611in}{0.000000in}}%
\pgfusepath{stroke,fill}%
}%
\begin{pgfscope}%
\pgfsys@transformshift{0.721913in}{1.672099in}%
\pgfsys@useobject{currentmarker}{}%
\end{pgfscope}%
\end{pgfscope}%
\begin{pgfscope}%
\definecolor{textcolor}{rgb}{0.000000,0.000000,0.000000}%
\pgfsetstrokecolor{textcolor}%
\pgfsetfillcolor{textcolor}%
\pgftext[x=0.303703in, y=1.614229in, left, base]{\color{textcolor}{\rmfamily\fontsize{12.000000}{14.400000}\selectfont\catcode`\^=\active\def^{\ifmmode\sp\else\^{}\fi}\catcode`\%=\active\def%{\%}$\mathdefault{10^{-1}}$}}%
\end{pgfscope}%
\begin{pgfscope}%
\pgfsetbuttcap%
\pgfsetroundjoin%
\definecolor{currentfill}{rgb}{0.000000,0.000000,0.000000}%
\pgfsetfillcolor{currentfill}%
\pgfsetlinewidth{0.803000pt}%
\definecolor{currentstroke}{rgb}{0.000000,0.000000,0.000000}%
\pgfsetstrokecolor{currentstroke}%
\pgfsetdash{}{0pt}%
\pgfsys@defobject{currentmarker}{\pgfqpoint{-0.048611in}{0.000000in}}{\pgfqpoint{-0.000000in}{0.000000in}}{%
\pgfpathmoveto{\pgfqpoint{-0.000000in}{0.000000in}}%
\pgfpathlineto{\pgfqpoint{-0.048611in}{0.000000in}}%
\pgfusepath{stroke,fill}%
}%
\begin{pgfscope}%
\pgfsys@transformshift{0.721913in}{2.261097in}%
\pgfsys@useobject{currentmarker}{}%
\end{pgfscope}%
\end{pgfscope}%
\begin{pgfscope}%
\definecolor{textcolor}{rgb}{0.000000,0.000000,0.000000}%
\pgfsetstrokecolor{textcolor}%
\pgfsetfillcolor{textcolor}%
\pgftext[x=0.395525in, y=2.203226in, left, base]{\color{textcolor}{\rmfamily\fontsize{12.000000}{14.400000}\selectfont\catcode`\^=\active\def^{\ifmmode\sp\else\^{}\fi}\catcode`\%=\active\def%{\%}$\mathdefault{10^{0}}$}}%
\end{pgfscope}%
\begin{pgfscope}%
\pgfsetbuttcap%
\pgfsetroundjoin%
\definecolor{currentfill}{rgb}{0.000000,0.000000,0.000000}%
\pgfsetfillcolor{currentfill}%
\pgfsetlinewidth{0.602250pt}%
\definecolor{currentstroke}{rgb}{0.000000,0.000000,0.000000}%
\pgfsetstrokecolor{currentstroke}%
\pgfsetdash{}{0pt}%
\pgfsys@defobject{currentmarker}{\pgfqpoint{-0.027778in}{0.000000in}}{\pgfqpoint{-0.000000in}{0.000000in}}{%
\pgfpathmoveto{\pgfqpoint{-0.000000in}{0.000000in}}%
\pgfpathlineto{\pgfqpoint{-0.027778in}{0.000000in}}%
\pgfusepath{stroke,fill}%
}%
\begin{pgfscope}%
\pgfsys@transformshift{0.721913in}{0.671410in}%
\pgfsys@useobject{currentmarker}{}%
\end{pgfscope}%
\end{pgfscope}%
\begin{pgfscope}%
\pgfsetbuttcap%
\pgfsetroundjoin%
\definecolor{currentfill}{rgb}{0.000000,0.000000,0.000000}%
\pgfsetfillcolor{currentfill}%
\pgfsetlinewidth{0.602250pt}%
\definecolor{currentstroke}{rgb}{0.000000,0.000000,0.000000}%
\pgfsetstrokecolor{currentstroke}%
\pgfsetdash{}{0pt}%
\pgfsys@defobject{currentmarker}{\pgfqpoint{-0.027778in}{0.000000in}}{\pgfqpoint{-0.000000in}{0.000000in}}{%
\pgfpathmoveto{\pgfqpoint{-0.000000in}{0.000000in}}%
\pgfpathlineto{\pgfqpoint{-0.027778in}{0.000000in}}%
\pgfusepath{stroke,fill}%
}%
\begin{pgfscope}%
\pgfsys@transformshift{0.721913in}{0.775127in}%
\pgfsys@useobject{currentmarker}{}%
\end{pgfscope}%
\end{pgfscope}%
\begin{pgfscope}%
\pgfsetbuttcap%
\pgfsetroundjoin%
\definecolor{currentfill}{rgb}{0.000000,0.000000,0.000000}%
\pgfsetfillcolor{currentfill}%
\pgfsetlinewidth{0.602250pt}%
\definecolor{currentstroke}{rgb}{0.000000,0.000000,0.000000}%
\pgfsetstrokecolor{currentstroke}%
\pgfsetdash{}{0pt}%
\pgfsys@defobject{currentmarker}{\pgfqpoint{-0.027778in}{0.000000in}}{\pgfqpoint{-0.000000in}{0.000000in}}{%
\pgfpathmoveto{\pgfqpoint{-0.000000in}{0.000000in}}%
\pgfpathlineto{\pgfqpoint{-0.027778in}{0.000000in}}%
\pgfusepath{stroke,fill}%
}%
\begin{pgfscope}%
\pgfsys@transformshift{0.721913in}{0.848716in}%
\pgfsys@useobject{currentmarker}{}%
\end{pgfscope}%
\end{pgfscope}%
\begin{pgfscope}%
\pgfsetbuttcap%
\pgfsetroundjoin%
\definecolor{currentfill}{rgb}{0.000000,0.000000,0.000000}%
\pgfsetfillcolor{currentfill}%
\pgfsetlinewidth{0.602250pt}%
\definecolor{currentstroke}{rgb}{0.000000,0.000000,0.000000}%
\pgfsetstrokecolor{currentstroke}%
\pgfsetdash{}{0pt}%
\pgfsys@defobject{currentmarker}{\pgfqpoint{-0.027778in}{0.000000in}}{\pgfqpoint{-0.000000in}{0.000000in}}{%
\pgfpathmoveto{\pgfqpoint{-0.000000in}{0.000000in}}%
\pgfpathlineto{\pgfqpoint{-0.027778in}{0.000000in}}%
\pgfusepath{stroke,fill}%
}%
\begin{pgfscope}%
\pgfsys@transformshift{0.721913in}{0.905796in}%
\pgfsys@useobject{currentmarker}{}%
\end{pgfscope}%
\end{pgfscope}%
\begin{pgfscope}%
\pgfsetbuttcap%
\pgfsetroundjoin%
\definecolor{currentfill}{rgb}{0.000000,0.000000,0.000000}%
\pgfsetfillcolor{currentfill}%
\pgfsetlinewidth{0.602250pt}%
\definecolor{currentstroke}{rgb}{0.000000,0.000000,0.000000}%
\pgfsetstrokecolor{currentstroke}%
\pgfsetdash{}{0pt}%
\pgfsys@defobject{currentmarker}{\pgfqpoint{-0.027778in}{0.000000in}}{\pgfqpoint{-0.000000in}{0.000000in}}{%
\pgfpathmoveto{\pgfqpoint{-0.000000in}{0.000000in}}%
\pgfpathlineto{\pgfqpoint{-0.027778in}{0.000000in}}%
\pgfusepath{stroke,fill}%
}%
\begin{pgfscope}%
\pgfsys@transformshift{0.721913in}{0.952433in}%
\pgfsys@useobject{currentmarker}{}%
\end{pgfscope}%
\end{pgfscope}%
\begin{pgfscope}%
\pgfsetbuttcap%
\pgfsetroundjoin%
\definecolor{currentfill}{rgb}{0.000000,0.000000,0.000000}%
\pgfsetfillcolor{currentfill}%
\pgfsetlinewidth{0.602250pt}%
\definecolor{currentstroke}{rgb}{0.000000,0.000000,0.000000}%
\pgfsetstrokecolor{currentstroke}%
\pgfsetdash{}{0pt}%
\pgfsys@defobject{currentmarker}{\pgfqpoint{-0.027778in}{0.000000in}}{\pgfqpoint{-0.000000in}{0.000000in}}{%
\pgfpathmoveto{\pgfqpoint{-0.000000in}{0.000000in}}%
\pgfpathlineto{\pgfqpoint{-0.027778in}{0.000000in}}%
\pgfusepath{stroke,fill}%
}%
\begin{pgfscope}%
\pgfsys@transformshift{0.721913in}{0.991865in}%
\pgfsys@useobject{currentmarker}{}%
\end{pgfscope}%
\end{pgfscope}%
\begin{pgfscope}%
\pgfsetbuttcap%
\pgfsetroundjoin%
\definecolor{currentfill}{rgb}{0.000000,0.000000,0.000000}%
\pgfsetfillcolor{currentfill}%
\pgfsetlinewidth{0.602250pt}%
\definecolor{currentstroke}{rgb}{0.000000,0.000000,0.000000}%
\pgfsetstrokecolor{currentstroke}%
\pgfsetdash{}{0pt}%
\pgfsys@defobject{currentmarker}{\pgfqpoint{-0.027778in}{0.000000in}}{\pgfqpoint{-0.000000in}{0.000000in}}{%
\pgfpathmoveto{\pgfqpoint{-0.000000in}{0.000000in}}%
\pgfpathlineto{\pgfqpoint{-0.027778in}{0.000000in}}%
\pgfusepath{stroke,fill}%
}%
\begin{pgfscope}%
\pgfsys@transformshift{0.721913in}{1.026022in}%
\pgfsys@useobject{currentmarker}{}%
\end{pgfscope}%
\end{pgfscope}%
\begin{pgfscope}%
\pgfsetbuttcap%
\pgfsetroundjoin%
\definecolor{currentfill}{rgb}{0.000000,0.000000,0.000000}%
\pgfsetfillcolor{currentfill}%
\pgfsetlinewidth{0.602250pt}%
\definecolor{currentstroke}{rgb}{0.000000,0.000000,0.000000}%
\pgfsetstrokecolor{currentstroke}%
\pgfsetdash{}{0pt}%
\pgfsys@defobject{currentmarker}{\pgfqpoint{-0.027778in}{0.000000in}}{\pgfqpoint{-0.000000in}{0.000000in}}{%
\pgfpathmoveto{\pgfqpoint{-0.000000in}{0.000000in}}%
\pgfpathlineto{\pgfqpoint{-0.027778in}{0.000000in}}%
\pgfusepath{stroke,fill}%
}%
\begin{pgfscope}%
\pgfsys@transformshift{0.721913in}{1.056150in}%
\pgfsys@useobject{currentmarker}{}%
\end{pgfscope}%
\end{pgfscope}%
\begin{pgfscope}%
\pgfsetbuttcap%
\pgfsetroundjoin%
\definecolor{currentfill}{rgb}{0.000000,0.000000,0.000000}%
\pgfsetfillcolor{currentfill}%
\pgfsetlinewidth{0.602250pt}%
\definecolor{currentstroke}{rgb}{0.000000,0.000000,0.000000}%
\pgfsetstrokecolor{currentstroke}%
\pgfsetdash{}{0pt}%
\pgfsys@defobject{currentmarker}{\pgfqpoint{-0.027778in}{0.000000in}}{\pgfqpoint{-0.000000in}{0.000000in}}{%
\pgfpathmoveto{\pgfqpoint{-0.000000in}{0.000000in}}%
\pgfpathlineto{\pgfqpoint{-0.027778in}{0.000000in}}%
\pgfusepath{stroke,fill}%
}%
\begin{pgfscope}%
\pgfsys@transformshift{0.721913in}{1.260407in}%
\pgfsys@useobject{currentmarker}{}%
\end{pgfscope}%
\end{pgfscope}%
\begin{pgfscope}%
\pgfsetbuttcap%
\pgfsetroundjoin%
\definecolor{currentfill}{rgb}{0.000000,0.000000,0.000000}%
\pgfsetfillcolor{currentfill}%
\pgfsetlinewidth{0.602250pt}%
\definecolor{currentstroke}{rgb}{0.000000,0.000000,0.000000}%
\pgfsetstrokecolor{currentstroke}%
\pgfsetdash{}{0pt}%
\pgfsys@defobject{currentmarker}{\pgfqpoint{-0.027778in}{0.000000in}}{\pgfqpoint{-0.000000in}{0.000000in}}{%
\pgfpathmoveto{\pgfqpoint{-0.000000in}{0.000000in}}%
\pgfpathlineto{\pgfqpoint{-0.027778in}{0.000000in}}%
\pgfusepath{stroke,fill}%
}%
\begin{pgfscope}%
\pgfsys@transformshift{0.721913in}{1.364125in}%
\pgfsys@useobject{currentmarker}{}%
\end{pgfscope}%
\end{pgfscope}%
\begin{pgfscope}%
\pgfsetbuttcap%
\pgfsetroundjoin%
\definecolor{currentfill}{rgb}{0.000000,0.000000,0.000000}%
\pgfsetfillcolor{currentfill}%
\pgfsetlinewidth{0.602250pt}%
\definecolor{currentstroke}{rgb}{0.000000,0.000000,0.000000}%
\pgfsetstrokecolor{currentstroke}%
\pgfsetdash{}{0pt}%
\pgfsys@defobject{currentmarker}{\pgfqpoint{-0.027778in}{0.000000in}}{\pgfqpoint{-0.000000in}{0.000000in}}{%
\pgfpathmoveto{\pgfqpoint{-0.000000in}{0.000000in}}%
\pgfpathlineto{\pgfqpoint{-0.027778in}{0.000000in}}%
\pgfusepath{stroke,fill}%
}%
\begin{pgfscope}%
\pgfsys@transformshift{0.721913in}{1.437713in}%
\pgfsys@useobject{currentmarker}{}%
\end{pgfscope}%
\end{pgfscope}%
\begin{pgfscope}%
\pgfsetbuttcap%
\pgfsetroundjoin%
\definecolor{currentfill}{rgb}{0.000000,0.000000,0.000000}%
\pgfsetfillcolor{currentfill}%
\pgfsetlinewidth{0.602250pt}%
\definecolor{currentstroke}{rgb}{0.000000,0.000000,0.000000}%
\pgfsetstrokecolor{currentstroke}%
\pgfsetdash{}{0pt}%
\pgfsys@defobject{currentmarker}{\pgfqpoint{-0.027778in}{0.000000in}}{\pgfqpoint{-0.000000in}{0.000000in}}{%
\pgfpathmoveto{\pgfqpoint{-0.000000in}{0.000000in}}%
\pgfpathlineto{\pgfqpoint{-0.027778in}{0.000000in}}%
\pgfusepath{stroke,fill}%
}%
\begin{pgfscope}%
\pgfsys@transformshift{0.721913in}{1.494793in}%
\pgfsys@useobject{currentmarker}{}%
\end{pgfscope}%
\end{pgfscope}%
\begin{pgfscope}%
\pgfsetbuttcap%
\pgfsetroundjoin%
\definecolor{currentfill}{rgb}{0.000000,0.000000,0.000000}%
\pgfsetfillcolor{currentfill}%
\pgfsetlinewidth{0.602250pt}%
\definecolor{currentstroke}{rgb}{0.000000,0.000000,0.000000}%
\pgfsetstrokecolor{currentstroke}%
\pgfsetdash{}{0pt}%
\pgfsys@defobject{currentmarker}{\pgfqpoint{-0.027778in}{0.000000in}}{\pgfqpoint{-0.000000in}{0.000000in}}{%
\pgfpathmoveto{\pgfqpoint{-0.000000in}{0.000000in}}%
\pgfpathlineto{\pgfqpoint{-0.027778in}{0.000000in}}%
\pgfusepath{stroke,fill}%
}%
\begin{pgfscope}%
\pgfsys@transformshift{0.721913in}{1.541431in}%
\pgfsys@useobject{currentmarker}{}%
\end{pgfscope}%
\end{pgfscope}%
\begin{pgfscope}%
\pgfsetbuttcap%
\pgfsetroundjoin%
\definecolor{currentfill}{rgb}{0.000000,0.000000,0.000000}%
\pgfsetfillcolor{currentfill}%
\pgfsetlinewidth{0.602250pt}%
\definecolor{currentstroke}{rgb}{0.000000,0.000000,0.000000}%
\pgfsetstrokecolor{currentstroke}%
\pgfsetdash{}{0pt}%
\pgfsys@defobject{currentmarker}{\pgfqpoint{-0.027778in}{0.000000in}}{\pgfqpoint{-0.000000in}{0.000000in}}{%
\pgfpathmoveto{\pgfqpoint{-0.000000in}{0.000000in}}%
\pgfpathlineto{\pgfqpoint{-0.027778in}{0.000000in}}%
\pgfusepath{stroke,fill}%
}%
\begin{pgfscope}%
\pgfsys@transformshift{0.721913in}{1.580862in}%
\pgfsys@useobject{currentmarker}{}%
\end{pgfscope}%
\end{pgfscope}%
\begin{pgfscope}%
\pgfsetbuttcap%
\pgfsetroundjoin%
\definecolor{currentfill}{rgb}{0.000000,0.000000,0.000000}%
\pgfsetfillcolor{currentfill}%
\pgfsetlinewidth{0.602250pt}%
\definecolor{currentstroke}{rgb}{0.000000,0.000000,0.000000}%
\pgfsetstrokecolor{currentstroke}%
\pgfsetdash{}{0pt}%
\pgfsys@defobject{currentmarker}{\pgfqpoint{-0.027778in}{0.000000in}}{\pgfqpoint{-0.000000in}{0.000000in}}{%
\pgfpathmoveto{\pgfqpoint{-0.000000in}{0.000000in}}%
\pgfpathlineto{\pgfqpoint{-0.027778in}{0.000000in}}%
\pgfusepath{stroke,fill}%
}%
\begin{pgfscope}%
\pgfsys@transformshift{0.721913in}{1.615019in}%
\pgfsys@useobject{currentmarker}{}%
\end{pgfscope}%
\end{pgfscope}%
\begin{pgfscope}%
\pgfsetbuttcap%
\pgfsetroundjoin%
\definecolor{currentfill}{rgb}{0.000000,0.000000,0.000000}%
\pgfsetfillcolor{currentfill}%
\pgfsetlinewidth{0.602250pt}%
\definecolor{currentstroke}{rgb}{0.000000,0.000000,0.000000}%
\pgfsetstrokecolor{currentstroke}%
\pgfsetdash{}{0pt}%
\pgfsys@defobject{currentmarker}{\pgfqpoint{-0.027778in}{0.000000in}}{\pgfqpoint{-0.000000in}{0.000000in}}{%
\pgfpathmoveto{\pgfqpoint{-0.000000in}{0.000000in}}%
\pgfpathlineto{\pgfqpoint{-0.027778in}{0.000000in}}%
\pgfusepath{stroke,fill}%
}%
\begin{pgfscope}%
\pgfsys@transformshift{0.721913in}{1.645148in}%
\pgfsys@useobject{currentmarker}{}%
\end{pgfscope}%
\end{pgfscope}%
\begin{pgfscope}%
\pgfsetbuttcap%
\pgfsetroundjoin%
\definecolor{currentfill}{rgb}{0.000000,0.000000,0.000000}%
\pgfsetfillcolor{currentfill}%
\pgfsetlinewidth{0.602250pt}%
\definecolor{currentstroke}{rgb}{0.000000,0.000000,0.000000}%
\pgfsetstrokecolor{currentstroke}%
\pgfsetdash{}{0pt}%
\pgfsys@defobject{currentmarker}{\pgfqpoint{-0.027778in}{0.000000in}}{\pgfqpoint{-0.000000in}{0.000000in}}{%
\pgfpathmoveto{\pgfqpoint{-0.000000in}{0.000000in}}%
\pgfpathlineto{\pgfqpoint{-0.027778in}{0.000000in}}%
\pgfusepath{stroke,fill}%
}%
\begin{pgfscope}%
\pgfsys@transformshift{0.721913in}{1.849405in}%
\pgfsys@useobject{currentmarker}{}%
\end{pgfscope}%
\end{pgfscope}%
\begin{pgfscope}%
\pgfsetbuttcap%
\pgfsetroundjoin%
\definecolor{currentfill}{rgb}{0.000000,0.000000,0.000000}%
\pgfsetfillcolor{currentfill}%
\pgfsetlinewidth{0.602250pt}%
\definecolor{currentstroke}{rgb}{0.000000,0.000000,0.000000}%
\pgfsetstrokecolor{currentstroke}%
\pgfsetdash{}{0pt}%
\pgfsys@defobject{currentmarker}{\pgfqpoint{-0.027778in}{0.000000in}}{\pgfqpoint{-0.000000in}{0.000000in}}{%
\pgfpathmoveto{\pgfqpoint{-0.000000in}{0.000000in}}%
\pgfpathlineto{\pgfqpoint{-0.027778in}{0.000000in}}%
\pgfusepath{stroke,fill}%
}%
\begin{pgfscope}%
\pgfsys@transformshift{0.721913in}{1.953122in}%
\pgfsys@useobject{currentmarker}{}%
\end{pgfscope}%
\end{pgfscope}%
\begin{pgfscope}%
\pgfsetbuttcap%
\pgfsetroundjoin%
\definecolor{currentfill}{rgb}{0.000000,0.000000,0.000000}%
\pgfsetfillcolor{currentfill}%
\pgfsetlinewidth{0.602250pt}%
\definecolor{currentstroke}{rgb}{0.000000,0.000000,0.000000}%
\pgfsetstrokecolor{currentstroke}%
\pgfsetdash{}{0pt}%
\pgfsys@defobject{currentmarker}{\pgfqpoint{-0.027778in}{0.000000in}}{\pgfqpoint{-0.000000in}{0.000000in}}{%
\pgfpathmoveto{\pgfqpoint{-0.000000in}{0.000000in}}%
\pgfpathlineto{\pgfqpoint{-0.027778in}{0.000000in}}%
\pgfusepath{stroke,fill}%
}%
\begin{pgfscope}%
\pgfsys@transformshift{0.721913in}{2.026711in}%
\pgfsys@useobject{currentmarker}{}%
\end{pgfscope}%
\end{pgfscope}%
\begin{pgfscope}%
\pgfsetbuttcap%
\pgfsetroundjoin%
\definecolor{currentfill}{rgb}{0.000000,0.000000,0.000000}%
\pgfsetfillcolor{currentfill}%
\pgfsetlinewidth{0.602250pt}%
\definecolor{currentstroke}{rgb}{0.000000,0.000000,0.000000}%
\pgfsetstrokecolor{currentstroke}%
\pgfsetdash{}{0pt}%
\pgfsys@defobject{currentmarker}{\pgfqpoint{-0.027778in}{0.000000in}}{\pgfqpoint{-0.000000in}{0.000000in}}{%
\pgfpathmoveto{\pgfqpoint{-0.000000in}{0.000000in}}%
\pgfpathlineto{\pgfqpoint{-0.027778in}{0.000000in}}%
\pgfusepath{stroke,fill}%
}%
\begin{pgfscope}%
\pgfsys@transformshift{0.721913in}{2.083791in}%
\pgfsys@useobject{currentmarker}{}%
\end{pgfscope}%
\end{pgfscope}%
\begin{pgfscope}%
\pgfsetbuttcap%
\pgfsetroundjoin%
\definecolor{currentfill}{rgb}{0.000000,0.000000,0.000000}%
\pgfsetfillcolor{currentfill}%
\pgfsetlinewidth{0.602250pt}%
\definecolor{currentstroke}{rgb}{0.000000,0.000000,0.000000}%
\pgfsetstrokecolor{currentstroke}%
\pgfsetdash{}{0pt}%
\pgfsys@defobject{currentmarker}{\pgfqpoint{-0.027778in}{0.000000in}}{\pgfqpoint{-0.000000in}{0.000000in}}{%
\pgfpathmoveto{\pgfqpoint{-0.000000in}{0.000000in}}%
\pgfpathlineto{\pgfqpoint{-0.027778in}{0.000000in}}%
\pgfusepath{stroke,fill}%
}%
\begin{pgfscope}%
\pgfsys@transformshift{0.721913in}{2.130428in}%
\pgfsys@useobject{currentmarker}{}%
\end{pgfscope}%
\end{pgfscope}%
\begin{pgfscope}%
\pgfsetbuttcap%
\pgfsetroundjoin%
\definecolor{currentfill}{rgb}{0.000000,0.000000,0.000000}%
\pgfsetfillcolor{currentfill}%
\pgfsetlinewidth{0.602250pt}%
\definecolor{currentstroke}{rgb}{0.000000,0.000000,0.000000}%
\pgfsetstrokecolor{currentstroke}%
\pgfsetdash{}{0pt}%
\pgfsys@defobject{currentmarker}{\pgfqpoint{-0.027778in}{0.000000in}}{\pgfqpoint{-0.000000in}{0.000000in}}{%
\pgfpathmoveto{\pgfqpoint{-0.000000in}{0.000000in}}%
\pgfpathlineto{\pgfqpoint{-0.027778in}{0.000000in}}%
\pgfusepath{stroke,fill}%
}%
\begin{pgfscope}%
\pgfsys@transformshift{0.721913in}{2.169860in}%
\pgfsys@useobject{currentmarker}{}%
\end{pgfscope}%
\end{pgfscope}%
\begin{pgfscope}%
\pgfsetbuttcap%
\pgfsetroundjoin%
\definecolor{currentfill}{rgb}{0.000000,0.000000,0.000000}%
\pgfsetfillcolor{currentfill}%
\pgfsetlinewidth{0.602250pt}%
\definecolor{currentstroke}{rgb}{0.000000,0.000000,0.000000}%
\pgfsetstrokecolor{currentstroke}%
\pgfsetdash{}{0pt}%
\pgfsys@defobject{currentmarker}{\pgfqpoint{-0.027778in}{0.000000in}}{\pgfqpoint{-0.000000in}{0.000000in}}{%
\pgfpathmoveto{\pgfqpoint{-0.000000in}{0.000000in}}%
\pgfpathlineto{\pgfqpoint{-0.027778in}{0.000000in}}%
\pgfusepath{stroke,fill}%
}%
\begin{pgfscope}%
\pgfsys@transformshift{0.721913in}{2.204017in}%
\pgfsys@useobject{currentmarker}{}%
\end{pgfscope}%
\end{pgfscope}%
\begin{pgfscope}%
\pgfsetbuttcap%
\pgfsetroundjoin%
\definecolor{currentfill}{rgb}{0.000000,0.000000,0.000000}%
\pgfsetfillcolor{currentfill}%
\pgfsetlinewidth{0.602250pt}%
\definecolor{currentstroke}{rgb}{0.000000,0.000000,0.000000}%
\pgfsetstrokecolor{currentstroke}%
\pgfsetdash{}{0pt}%
\pgfsys@defobject{currentmarker}{\pgfqpoint{-0.027778in}{0.000000in}}{\pgfqpoint{-0.000000in}{0.000000in}}{%
\pgfpathmoveto{\pgfqpoint{-0.000000in}{0.000000in}}%
\pgfpathlineto{\pgfqpoint{-0.027778in}{0.000000in}}%
\pgfusepath{stroke,fill}%
}%
\begin{pgfscope}%
\pgfsys@transformshift{0.721913in}{2.234145in}%
\pgfsys@useobject{currentmarker}{}%
\end{pgfscope}%
\end{pgfscope}%
\begin{pgfscope}%
\pgfsetbuttcap%
\pgfsetroundjoin%
\definecolor{currentfill}{rgb}{0.000000,0.000000,0.000000}%
\pgfsetfillcolor{currentfill}%
\pgfsetlinewidth{0.602250pt}%
\definecolor{currentstroke}{rgb}{0.000000,0.000000,0.000000}%
\pgfsetstrokecolor{currentstroke}%
\pgfsetdash{}{0pt}%
\pgfsys@defobject{currentmarker}{\pgfqpoint{-0.027778in}{0.000000in}}{\pgfqpoint{-0.000000in}{0.000000in}}{%
\pgfpathmoveto{\pgfqpoint{-0.000000in}{0.000000in}}%
\pgfpathlineto{\pgfqpoint{-0.027778in}{0.000000in}}%
\pgfusepath{stroke,fill}%
}%
\begin{pgfscope}%
\pgfsys@transformshift{0.721913in}{2.438402in}%
\pgfsys@useobject{currentmarker}{}%
\end{pgfscope}%
\end{pgfscope}%
\begin{pgfscope}%
\definecolor{textcolor}{rgb}{0.000000,0.000000,0.000000}%
\pgfsetstrokecolor{textcolor}%
\pgfsetfillcolor{textcolor}%
\pgftext[x=0.248148in,y=1.511573in,,bottom,rotate=90.000000]{\color{textcolor}{\rmfamily\fontsize{12.000000}{14.400000}\selectfont\catcode`\^=\active\def^{\ifmmode\sp\else\^{}\fi}\catcode`\%=\active\def%{\%}$L^1$ relative error}}%
\end{pgfscope}%
\begin{pgfscope}%
\pgfpathrectangle{\pgfqpoint{0.721913in}{0.549073in}}{\pgfqpoint{1.937500in}{1.925000in}}%
\pgfusepath{clip}%
\pgfsetrectcap%
\pgfsetroundjoin%
\pgfsetlinewidth{1.003750pt}%
\definecolor{currentstroke}{rgb}{0.537255,0.647059,0.760784}%
\pgfsetstrokecolor{currentstroke}%
\pgfsetdash{}{0pt}%
\pgfpathmoveto{\pgfqpoint{0.809982in}{1.948150in}}%
\pgfpathlineto{\pgfqpoint{1.108820in}{1.523005in}}%
\pgfpathlineto{\pgfqpoint{1.401255in}{1.298028in}}%
\pgfpathlineto{\pgfqpoint{1.694529in}{1.297587in}}%
\pgfpathlineto{\pgfqpoint{1.986285in}{1.297587in}}%
\pgfpathlineto{\pgfqpoint{2.278958in}{1.297587in}}%
\pgfpathlineto{\pgfqpoint{2.571345in}{1.297587in}}%
\pgfusepath{stroke}%
\end{pgfscope}%
\begin{pgfscope}%
\pgfpathrectangle{\pgfqpoint{0.721913in}{0.549073in}}{\pgfqpoint{1.937500in}{1.925000in}}%
\pgfusepath{clip}%
\pgfsetbuttcap%
\pgfsetroundjoin%
\definecolor{currentfill}{rgb}{0.537255,0.647059,0.760784}%
\pgfsetfillcolor{currentfill}%
\pgfsetlinewidth{1.003750pt}%
\definecolor{currentstroke}{rgb}{0.537255,0.647059,0.760784}%
\pgfsetstrokecolor{currentstroke}%
\pgfsetdash{}{0pt}%
\pgfsys@defobject{currentmarker}{\pgfqpoint{-0.020833in}{-0.020833in}}{\pgfqpoint{0.020833in}{0.020833in}}{%
\pgfpathmoveto{\pgfqpoint{0.000000in}{-0.020833in}}%
\pgfpathcurveto{\pgfqpoint{0.005525in}{-0.020833in}}{\pgfqpoint{0.010825in}{-0.018638in}}{\pgfqpoint{0.014731in}{-0.014731in}}%
\pgfpathcurveto{\pgfqpoint{0.018638in}{-0.010825in}}{\pgfqpoint{0.020833in}{-0.005525in}}{\pgfqpoint{0.020833in}{0.000000in}}%
\pgfpathcurveto{\pgfqpoint{0.020833in}{0.005525in}}{\pgfqpoint{0.018638in}{0.010825in}}{\pgfqpoint{0.014731in}{0.014731in}}%
\pgfpathcurveto{\pgfqpoint{0.010825in}{0.018638in}}{\pgfqpoint{0.005525in}{0.020833in}}{\pgfqpoint{0.000000in}{0.020833in}}%
\pgfpathcurveto{\pgfqpoint{-0.005525in}{0.020833in}}{\pgfqpoint{-0.010825in}{0.018638in}}{\pgfqpoint{-0.014731in}{0.014731in}}%
\pgfpathcurveto{\pgfqpoint{-0.018638in}{0.010825in}}{\pgfqpoint{-0.020833in}{0.005525in}}{\pgfqpoint{-0.020833in}{0.000000in}}%
\pgfpathcurveto{\pgfqpoint{-0.020833in}{-0.005525in}}{\pgfqpoint{-0.018638in}{-0.010825in}}{\pgfqpoint{-0.014731in}{-0.014731in}}%
\pgfpathcurveto{\pgfqpoint{-0.010825in}{-0.018638in}}{\pgfqpoint{-0.005525in}{-0.020833in}}{\pgfqpoint{0.000000in}{-0.020833in}}%
\pgfpathlineto{\pgfqpoint{0.000000in}{-0.020833in}}%
\pgfpathclose%
\pgfusepath{stroke,fill}%
}%
\begin{pgfscope}%
\pgfsys@transformshift{0.809982in}{1.948150in}%
\pgfsys@useobject{currentmarker}{}%
\end{pgfscope}%
\begin{pgfscope}%
\pgfsys@transformshift{1.108820in}{1.523005in}%
\pgfsys@useobject{currentmarker}{}%
\end{pgfscope}%
\begin{pgfscope}%
\pgfsys@transformshift{1.401255in}{1.298028in}%
\pgfsys@useobject{currentmarker}{}%
\end{pgfscope}%
\begin{pgfscope}%
\pgfsys@transformshift{1.694529in}{1.297587in}%
\pgfsys@useobject{currentmarker}{}%
\end{pgfscope}%
\begin{pgfscope}%
\pgfsys@transformshift{1.986285in}{1.297587in}%
\pgfsys@useobject{currentmarker}{}%
\end{pgfscope}%
\begin{pgfscope}%
\pgfsys@transformshift{2.278958in}{1.297587in}%
\pgfsys@useobject{currentmarker}{}%
\end{pgfscope}%
\begin{pgfscope}%
\pgfsys@transformshift{2.571345in}{1.297587in}%
\pgfsys@useobject{currentmarker}{}%
\end{pgfscope}%
\end{pgfscope}%
\begin{pgfscope}%
\pgfpathrectangle{\pgfqpoint{0.721913in}{0.549073in}}{\pgfqpoint{1.937500in}{1.925000in}}%
\pgfusepath{clip}%
\pgfsetrectcap%
\pgfsetroundjoin%
\pgfsetlinewidth{1.003750pt}%
\definecolor{currentstroke}{rgb}{0.184314,0.270588,0.360784}%
\pgfsetstrokecolor{currentstroke}%
\pgfsetdash{}{0pt}%
\pgfpathmoveto{\pgfqpoint{0.809982in}{2.386573in}}%
\pgfpathlineto{\pgfqpoint{1.108820in}{2.316113in}}%
\pgfpathlineto{\pgfqpoint{1.401255in}{2.312248in}}%
\pgfpathlineto{\pgfqpoint{1.694529in}{2.049891in}}%
\pgfpathlineto{\pgfqpoint{1.986285in}{1.642081in}}%
\pgfpathlineto{\pgfqpoint{2.278958in}{1.392423in}}%
\pgfpathlineto{\pgfqpoint{2.571345in}{1.392810in}}%
\pgfusepath{stroke}%
\end{pgfscope}%
\begin{pgfscope}%
\pgfpathrectangle{\pgfqpoint{0.721913in}{0.549073in}}{\pgfqpoint{1.937500in}{1.925000in}}%
\pgfusepath{clip}%
\pgfsetbuttcap%
\pgfsetroundjoin%
\definecolor{currentfill}{rgb}{0.184314,0.270588,0.360784}%
\pgfsetfillcolor{currentfill}%
\pgfsetlinewidth{1.003750pt}%
\definecolor{currentstroke}{rgb}{0.184314,0.270588,0.360784}%
\pgfsetstrokecolor{currentstroke}%
\pgfsetdash{}{0pt}%
\pgfsys@defobject{currentmarker}{\pgfqpoint{-0.020833in}{-0.020833in}}{\pgfqpoint{0.020833in}{0.020833in}}{%
\pgfpathmoveto{\pgfqpoint{0.000000in}{-0.020833in}}%
\pgfpathcurveto{\pgfqpoint{0.005525in}{-0.020833in}}{\pgfqpoint{0.010825in}{-0.018638in}}{\pgfqpoint{0.014731in}{-0.014731in}}%
\pgfpathcurveto{\pgfqpoint{0.018638in}{-0.010825in}}{\pgfqpoint{0.020833in}{-0.005525in}}{\pgfqpoint{0.020833in}{0.000000in}}%
\pgfpathcurveto{\pgfqpoint{0.020833in}{0.005525in}}{\pgfqpoint{0.018638in}{0.010825in}}{\pgfqpoint{0.014731in}{0.014731in}}%
\pgfpathcurveto{\pgfqpoint{0.010825in}{0.018638in}}{\pgfqpoint{0.005525in}{0.020833in}}{\pgfqpoint{0.000000in}{0.020833in}}%
\pgfpathcurveto{\pgfqpoint{-0.005525in}{0.020833in}}{\pgfqpoint{-0.010825in}{0.018638in}}{\pgfqpoint{-0.014731in}{0.014731in}}%
\pgfpathcurveto{\pgfqpoint{-0.018638in}{0.010825in}}{\pgfqpoint{-0.020833in}{0.005525in}}{\pgfqpoint{-0.020833in}{0.000000in}}%
\pgfpathcurveto{\pgfqpoint{-0.020833in}{-0.005525in}}{\pgfqpoint{-0.018638in}{-0.010825in}}{\pgfqpoint{-0.014731in}{-0.014731in}}%
\pgfpathcurveto{\pgfqpoint{-0.010825in}{-0.018638in}}{\pgfqpoint{-0.005525in}{-0.020833in}}{\pgfqpoint{0.000000in}{-0.020833in}}%
\pgfpathlineto{\pgfqpoint{0.000000in}{-0.020833in}}%
\pgfpathclose%
\pgfusepath{stroke,fill}%
}%
\begin{pgfscope}%
\pgfsys@transformshift{0.809982in}{2.386573in}%
\pgfsys@useobject{currentmarker}{}%
\end{pgfscope}%
\begin{pgfscope}%
\pgfsys@transformshift{1.108820in}{2.316113in}%
\pgfsys@useobject{currentmarker}{}%
\end{pgfscope}%
\begin{pgfscope}%
\pgfsys@transformshift{1.401255in}{2.312248in}%
\pgfsys@useobject{currentmarker}{}%
\end{pgfscope}%
\begin{pgfscope}%
\pgfsys@transformshift{1.694529in}{2.049891in}%
\pgfsys@useobject{currentmarker}{}%
\end{pgfscope}%
\begin{pgfscope}%
\pgfsys@transformshift{1.986285in}{1.642081in}%
\pgfsys@useobject{currentmarker}{}%
\end{pgfscope}%
\begin{pgfscope}%
\pgfsys@transformshift{2.278958in}{1.392423in}%
\pgfsys@useobject{currentmarker}{}%
\end{pgfscope}%
\begin{pgfscope}%
\pgfsys@transformshift{2.571345in}{1.392810in}%
\pgfsys@useobject{currentmarker}{}%
\end{pgfscope}%
\end{pgfscope}%
\begin{pgfscope}%
\pgfpathrectangle{\pgfqpoint{0.721913in}{0.549073in}}{\pgfqpoint{1.937500in}{1.925000in}}%
\pgfusepath{clip}%
\pgfsetrectcap%
\pgfsetroundjoin%
\pgfsetlinewidth{1.003750pt}%
\definecolor{currentstroke}{rgb}{0.976471,0.505882,0.145098}%
\pgfsetstrokecolor{currentstroke}%
\pgfsetdash{}{0pt}%
\pgfpathmoveto{\pgfqpoint{0.809982in}{2.253783in}}%
\pgfpathlineto{\pgfqpoint{1.108820in}{2.151075in}}%
\pgfpathlineto{\pgfqpoint{1.401255in}{1.968481in}}%
\pgfpathlineto{\pgfqpoint{1.694529in}{1.732526in}}%
\pgfpathlineto{\pgfqpoint{1.986285in}{1.256169in}}%
\pgfpathlineto{\pgfqpoint{2.278958in}{0.702808in}}%
\pgfpathlineto{\pgfqpoint{2.571345in}{0.636573in}}%
\pgfusepath{stroke}%
\end{pgfscope}%
\begin{pgfscope}%
\pgfpathrectangle{\pgfqpoint{0.721913in}{0.549073in}}{\pgfqpoint{1.937500in}{1.925000in}}%
\pgfusepath{clip}%
\pgfsetbuttcap%
\pgfsetroundjoin%
\definecolor{currentfill}{rgb}{0.976471,0.505882,0.145098}%
\pgfsetfillcolor{currentfill}%
\pgfsetlinewidth{1.003750pt}%
\definecolor{currentstroke}{rgb}{0.976471,0.505882,0.145098}%
\pgfsetstrokecolor{currentstroke}%
\pgfsetdash{}{0pt}%
\pgfsys@defobject{currentmarker}{\pgfqpoint{-0.020833in}{-0.020833in}}{\pgfqpoint{0.020833in}{0.020833in}}{%
\pgfpathmoveto{\pgfqpoint{0.000000in}{-0.020833in}}%
\pgfpathcurveto{\pgfqpoint{0.005525in}{-0.020833in}}{\pgfqpoint{0.010825in}{-0.018638in}}{\pgfqpoint{0.014731in}{-0.014731in}}%
\pgfpathcurveto{\pgfqpoint{0.018638in}{-0.010825in}}{\pgfqpoint{0.020833in}{-0.005525in}}{\pgfqpoint{0.020833in}{0.000000in}}%
\pgfpathcurveto{\pgfqpoint{0.020833in}{0.005525in}}{\pgfqpoint{0.018638in}{0.010825in}}{\pgfqpoint{0.014731in}{0.014731in}}%
\pgfpathcurveto{\pgfqpoint{0.010825in}{0.018638in}}{\pgfqpoint{0.005525in}{0.020833in}}{\pgfqpoint{0.000000in}{0.020833in}}%
\pgfpathcurveto{\pgfqpoint{-0.005525in}{0.020833in}}{\pgfqpoint{-0.010825in}{0.018638in}}{\pgfqpoint{-0.014731in}{0.014731in}}%
\pgfpathcurveto{\pgfqpoint{-0.018638in}{0.010825in}}{\pgfqpoint{-0.020833in}{0.005525in}}{\pgfqpoint{-0.020833in}{0.000000in}}%
\pgfpathcurveto{\pgfqpoint{-0.020833in}{-0.005525in}}{\pgfqpoint{-0.018638in}{-0.010825in}}{\pgfqpoint{-0.014731in}{-0.014731in}}%
\pgfpathcurveto{\pgfqpoint{-0.010825in}{-0.018638in}}{\pgfqpoint{-0.005525in}{-0.020833in}}{\pgfqpoint{0.000000in}{-0.020833in}}%
\pgfpathlineto{\pgfqpoint{0.000000in}{-0.020833in}}%
\pgfpathclose%
\pgfusepath{stroke,fill}%
}%
\begin{pgfscope}%
\pgfsys@transformshift{0.809982in}{2.253783in}%
\pgfsys@useobject{currentmarker}{}%
\end{pgfscope}%
\begin{pgfscope}%
\pgfsys@transformshift{1.108820in}{2.151075in}%
\pgfsys@useobject{currentmarker}{}%
\end{pgfscope}%
\begin{pgfscope}%
\pgfsys@transformshift{1.401255in}{1.968481in}%
\pgfsys@useobject{currentmarker}{}%
\end{pgfscope}%
\begin{pgfscope}%
\pgfsys@transformshift{1.694529in}{1.732526in}%
\pgfsys@useobject{currentmarker}{}%
\end{pgfscope}%
\begin{pgfscope}%
\pgfsys@transformshift{1.986285in}{1.256169in}%
\pgfsys@useobject{currentmarker}{}%
\end{pgfscope}%
\begin{pgfscope}%
\pgfsys@transformshift{2.278958in}{0.702808in}%
\pgfsys@useobject{currentmarker}{}%
\end{pgfscope}%
\begin{pgfscope}%
\pgfsys@transformshift{2.571345in}{0.636573in}%
\pgfsys@useobject{currentmarker}{}%
\end{pgfscope}%
\end{pgfscope}%
\begin{pgfscope}%
\pgfsetrectcap%
\pgfsetmiterjoin%
\pgfsetlinewidth{0.803000pt}%
\definecolor{currentstroke}{rgb}{0.000000,0.000000,0.000000}%
\pgfsetstrokecolor{currentstroke}%
\pgfsetdash{}{0pt}%
\pgfpathmoveto{\pgfqpoint{0.721913in}{0.549073in}}%
\pgfpathlineto{\pgfqpoint{0.721913in}{2.474073in}}%
\pgfusepath{stroke}%
\end{pgfscope}%
\begin{pgfscope}%
\pgfsetrectcap%
\pgfsetmiterjoin%
\pgfsetlinewidth{0.803000pt}%
\definecolor{currentstroke}{rgb}{0.000000,0.000000,0.000000}%
\pgfsetstrokecolor{currentstroke}%
\pgfsetdash{}{0pt}%
\pgfpathmoveto{\pgfqpoint{2.659413in}{0.549073in}}%
\pgfpathlineto{\pgfqpoint{2.659413in}{2.474073in}}%
\pgfusepath{stroke}%
\end{pgfscope}%
\begin{pgfscope}%
\pgfsetrectcap%
\pgfsetmiterjoin%
\pgfsetlinewidth{0.803000pt}%
\definecolor{currentstroke}{rgb}{0.000000,0.000000,0.000000}%
\pgfsetstrokecolor{currentstroke}%
\pgfsetdash{}{0pt}%
\pgfpathmoveto{\pgfqpoint{0.721913in}{0.549073in}}%
\pgfpathlineto{\pgfqpoint{2.659413in}{0.549073in}}%
\pgfusepath{stroke}%
\end{pgfscope}%
\begin{pgfscope}%
\pgfsetrectcap%
\pgfsetmiterjoin%
\pgfsetlinewidth{0.803000pt}%
\definecolor{currentstroke}{rgb}{0.000000,0.000000,0.000000}%
\pgfsetstrokecolor{currentstroke}%
\pgfsetdash{}{0pt}%
\pgfpathmoveto{\pgfqpoint{0.721913in}{2.474073in}}%
\pgfpathlineto{\pgfqpoint{2.659413in}{2.474073in}}%
\pgfusepath{stroke}%
\end{pgfscope}%
\begin{pgfscope}%
\pgfsetbuttcap%
\pgfsetmiterjoin%
\definecolor{currentfill}{rgb}{1.000000,1.000000,1.000000}%
\pgfsetfillcolor{currentfill}%
\pgfsetfillopacity{0.800000}%
\pgfsetlinewidth{1.003750pt}%
\definecolor{currentstroke}{rgb}{0.800000,0.800000,0.800000}%
\pgfsetstrokecolor{currentstroke}%
\pgfsetstrokeopacity{0.800000}%
\pgfsetdash{}{0pt}%
\pgfpathmoveto{\pgfqpoint{1.392965in}{1.643518in}}%
\pgfpathlineto{\pgfqpoint{2.542747in}{1.643518in}}%
\pgfpathquadraticcurveto{\pgfqpoint{2.576080in}{1.643518in}}{\pgfqpoint{2.576080in}{1.676852in}}%
\pgfpathlineto{\pgfqpoint{2.576080in}{2.357406in}}%
\pgfpathquadraticcurveto{\pgfqpoint{2.576080in}{2.390739in}}{\pgfqpoint{2.542747in}{2.390739in}}%
\pgfpathlineto{\pgfqpoint{1.392965in}{2.390739in}}%
\pgfpathquadraticcurveto{\pgfqpoint{1.359632in}{2.390739in}}{\pgfqpoint{1.359632in}{2.357406in}}%
\pgfpathlineto{\pgfqpoint{1.359632in}{1.676852in}}%
\pgfpathquadraticcurveto{\pgfqpoint{1.359632in}{1.643518in}}{\pgfqpoint{1.392965in}{1.643518in}}%
\pgfpathlineto{\pgfqpoint{1.392965in}{1.643518in}}%
\pgfpathclose%
\pgfusepath{stroke,fill}%
\end{pgfscope}%
\begin{pgfscope}%
\pgfsetrectcap%
\pgfsetroundjoin%
\pgfsetlinewidth{1.003750pt}%
\definecolor{currentstroke}{rgb}{0.537255,0.647059,0.760784}%
\pgfsetstrokecolor{currentstroke}%
\pgfsetdash{}{0pt}%
\pgfpathmoveto{\pgfqpoint{1.426299in}{2.265739in}}%
\pgfpathlineto{\pgfqpoint{1.592965in}{2.265739in}}%
\pgfpathlineto{\pgfqpoint{1.759632in}{2.265739in}}%
\pgfusepath{stroke}%
\end{pgfscope}%
\begin{pgfscope}%
\pgfsetbuttcap%
\pgfsetroundjoin%
\definecolor{currentfill}{rgb}{0.537255,0.647059,0.760784}%
\pgfsetfillcolor{currentfill}%
\pgfsetlinewidth{1.003750pt}%
\definecolor{currentstroke}{rgb}{0.537255,0.647059,0.760784}%
\pgfsetstrokecolor{currentstroke}%
\pgfsetdash{}{0pt}%
\pgfsys@defobject{currentmarker}{\pgfqpoint{-0.020833in}{-0.020833in}}{\pgfqpoint{0.020833in}{0.020833in}}{%
\pgfpathmoveto{\pgfqpoint{0.000000in}{-0.020833in}}%
\pgfpathcurveto{\pgfqpoint{0.005525in}{-0.020833in}}{\pgfqpoint{0.010825in}{-0.018638in}}{\pgfqpoint{0.014731in}{-0.014731in}}%
\pgfpathcurveto{\pgfqpoint{0.018638in}{-0.010825in}}{\pgfqpoint{0.020833in}{-0.005525in}}{\pgfqpoint{0.020833in}{0.000000in}}%
\pgfpathcurveto{\pgfqpoint{0.020833in}{0.005525in}}{\pgfqpoint{0.018638in}{0.010825in}}{\pgfqpoint{0.014731in}{0.014731in}}%
\pgfpathcurveto{\pgfqpoint{0.010825in}{0.018638in}}{\pgfqpoint{0.005525in}{0.020833in}}{\pgfqpoint{0.000000in}{0.020833in}}%
\pgfpathcurveto{\pgfqpoint{-0.005525in}{0.020833in}}{\pgfqpoint{-0.010825in}{0.018638in}}{\pgfqpoint{-0.014731in}{0.014731in}}%
\pgfpathcurveto{\pgfqpoint{-0.018638in}{0.010825in}}{\pgfqpoint{-0.020833in}{0.005525in}}{\pgfqpoint{-0.020833in}{0.000000in}}%
\pgfpathcurveto{\pgfqpoint{-0.020833in}{-0.005525in}}{\pgfqpoint{-0.018638in}{-0.010825in}}{\pgfqpoint{-0.014731in}{-0.014731in}}%
\pgfpathcurveto{\pgfqpoint{-0.010825in}{-0.018638in}}{\pgfqpoint{-0.005525in}{-0.020833in}}{\pgfqpoint{0.000000in}{-0.020833in}}%
\pgfpathlineto{\pgfqpoint{0.000000in}{-0.020833in}}%
\pgfpathclose%
\pgfusepath{stroke,fill}%
}%
\begin{pgfscope}%
\pgfsys@transformshift{1.592965in}{2.265739in}%
\pgfsys@useobject{currentmarker}{}%
\end{pgfscope}%
\end{pgfscope}%
\begin{pgfscope}%
\definecolor{textcolor}{rgb}{0.000000,0.000000,0.000000}%
\pgfsetstrokecolor{textcolor}%
\pgfsetfillcolor{textcolor}%
\pgftext[x=1.892965in,y=2.207406in,left,base]{\color{textcolor}{\rmfamily\fontsize{12.000000}{14.400000}\selectfont\catcode`\^=\active\def^{\ifmmode\sp\else\^{}\fi}\catcode`\%=\active\def%{\%}Haydock}}%
\end{pgfscope}%
\begin{pgfscope}%
\pgfsetrectcap%
\pgfsetroundjoin%
\pgfsetlinewidth{1.003750pt}%
\definecolor{currentstroke}{rgb}{0.184314,0.270588,0.360784}%
\pgfsetstrokecolor{currentstroke}%
\pgfsetdash{}{0pt}%
\pgfpathmoveto{\pgfqpoint{1.426299in}{2.033332in}}%
\pgfpathlineto{\pgfqpoint{1.592965in}{2.033332in}}%
\pgfpathlineto{\pgfqpoint{1.759632in}{2.033332in}}%
\pgfusepath{stroke}%
\end{pgfscope}%
\begin{pgfscope}%
\pgfsetbuttcap%
\pgfsetroundjoin%
\definecolor{currentfill}{rgb}{0.184314,0.270588,0.360784}%
\pgfsetfillcolor{currentfill}%
\pgfsetlinewidth{1.003750pt}%
\definecolor{currentstroke}{rgb}{0.184314,0.270588,0.360784}%
\pgfsetstrokecolor{currentstroke}%
\pgfsetdash{}{0pt}%
\pgfsys@defobject{currentmarker}{\pgfqpoint{-0.020833in}{-0.020833in}}{\pgfqpoint{0.020833in}{0.020833in}}{%
\pgfpathmoveto{\pgfqpoint{0.000000in}{-0.020833in}}%
\pgfpathcurveto{\pgfqpoint{0.005525in}{-0.020833in}}{\pgfqpoint{0.010825in}{-0.018638in}}{\pgfqpoint{0.014731in}{-0.014731in}}%
\pgfpathcurveto{\pgfqpoint{0.018638in}{-0.010825in}}{\pgfqpoint{0.020833in}{-0.005525in}}{\pgfqpoint{0.020833in}{0.000000in}}%
\pgfpathcurveto{\pgfqpoint{0.020833in}{0.005525in}}{\pgfqpoint{0.018638in}{0.010825in}}{\pgfqpoint{0.014731in}{0.014731in}}%
\pgfpathcurveto{\pgfqpoint{0.010825in}{0.018638in}}{\pgfqpoint{0.005525in}{0.020833in}}{\pgfqpoint{0.000000in}{0.020833in}}%
\pgfpathcurveto{\pgfqpoint{-0.005525in}{0.020833in}}{\pgfqpoint{-0.010825in}{0.018638in}}{\pgfqpoint{-0.014731in}{0.014731in}}%
\pgfpathcurveto{\pgfqpoint{-0.018638in}{0.010825in}}{\pgfqpoint{-0.020833in}{0.005525in}}{\pgfqpoint{-0.020833in}{0.000000in}}%
\pgfpathcurveto{\pgfqpoint{-0.020833in}{-0.005525in}}{\pgfqpoint{-0.018638in}{-0.010825in}}{\pgfqpoint{-0.014731in}{-0.014731in}}%
\pgfpathcurveto{\pgfqpoint{-0.010825in}{-0.018638in}}{\pgfqpoint{-0.005525in}{-0.020833in}}{\pgfqpoint{0.000000in}{-0.020833in}}%
\pgfpathlineto{\pgfqpoint{0.000000in}{-0.020833in}}%
\pgfpathclose%
\pgfusepath{stroke,fill}%
}%
\begin{pgfscope}%
\pgfsys@transformshift{1.592965in}{2.033332in}%
\pgfsys@useobject{currentmarker}{}%
\end{pgfscope}%
\end{pgfscope}%
\begin{pgfscope}%
\definecolor{textcolor}{rgb}{0.000000,0.000000,0.000000}%
\pgfsetstrokecolor{textcolor}%
\pgfsetfillcolor{textcolor}%
\pgftext[x=1.892965in,y=1.974999in,left,base]{\color{textcolor}{\rmfamily\fontsize{12.000000}{14.400000}\selectfont\catcode`\^=\active\def^{\ifmmode\sp\else\^{}\fi}\catcode`\%=\active\def%{\%}NC}}%
\end{pgfscope}%
\begin{pgfscope}%
\pgfsetrectcap%
\pgfsetroundjoin%
\pgfsetlinewidth{1.003750pt}%
\definecolor{currentstroke}{rgb}{0.976471,0.505882,0.145098}%
\pgfsetstrokecolor{currentstroke}%
\pgfsetdash{}{0pt}%
\pgfpathmoveto{\pgfqpoint{1.426299in}{1.800925in}}%
\pgfpathlineto{\pgfqpoint{1.592965in}{1.800925in}}%
\pgfpathlineto{\pgfqpoint{1.759632in}{1.800925in}}%
\pgfusepath{stroke}%
\end{pgfscope}%
\begin{pgfscope}%
\pgfsetbuttcap%
\pgfsetroundjoin%
\definecolor{currentfill}{rgb}{0.976471,0.505882,0.145098}%
\pgfsetfillcolor{currentfill}%
\pgfsetlinewidth{1.003750pt}%
\definecolor{currentstroke}{rgb}{0.976471,0.505882,0.145098}%
\pgfsetstrokecolor{currentstroke}%
\pgfsetdash{}{0pt}%
\pgfsys@defobject{currentmarker}{\pgfqpoint{-0.020833in}{-0.020833in}}{\pgfqpoint{0.020833in}{0.020833in}}{%
\pgfpathmoveto{\pgfqpoint{0.000000in}{-0.020833in}}%
\pgfpathcurveto{\pgfqpoint{0.005525in}{-0.020833in}}{\pgfqpoint{0.010825in}{-0.018638in}}{\pgfqpoint{0.014731in}{-0.014731in}}%
\pgfpathcurveto{\pgfqpoint{0.018638in}{-0.010825in}}{\pgfqpoint{0.020833in}{-0.005525in}}{\pgfqpoint{0.020833in}{0.000000in}}%
\pgfpathcurveto{\pgfqpoint{0.020833in}{0.005525in}}{\pgfqpoint{0.018638in}{0.010825in}}{\pgfqpoint{0.014731in}{0.014731in}}%
\pgfpathcurveto{\pgfqpoint{0.010825in}{0.018638in}}{\pgfqpoint{0.005525in}{0.020833in}}{\pgfqpoint{0.000000in}{0.020833in}}%
\pgfpathcurveto{\pgfqpoint{-0.005525in}{0.020833in}}{\pgfqpoint{-0.010825in}{0.018638in}}{\pgfqpoint{-0.014731in}{0.014731in}}%
\pgfpathcurveto{\pgfqpoint{-0.018638in}{0.010825in}}{\pgfqpoint{-0.020833in}{0.005525in}}{\pgfqpoint{-0.020833in}{0.000000in}}%
\pgfpathcurveto{\pgfqpoint{-0.020833in}{-0.005525in}}{\pgfqpoint{-0.018638in}{-0.010825in}}{\pgfqpoint{-0.014731in}{-0.014731in}}%
\pgfpathcurveto{\pgfqpoint{-0.010825in}{-0.018638in}}{\pgfqpoint{-0.005525in}{-0.020833in}}{\pgfqpoint{0.000000in}{-0.020833in}}%
\pgfpathlineto{\pgfqpoint{0.000000in}{-0.020833in}}%
\pgfpathclose%
\pgfusepath{stroke,fill}%
}%
\begin{pgfscope}%
\pgfsys@transformshift{1.592965in}{1.800925in}%
\pgfsys@useobject{currentmarker}{}%
\end{pgfscope}%
\end{pgfscope}%
\begin{pgfscope}%
\definecolor{textcolor}{rgb}{0.000000,0.000000,0.000000}%
\pgfsetstrokecolor{textcolor}%
\pgfsetfillcolor{textcolor}%
\pgftext[x=1.892965in,y=1.742592in,left,base]{\color{textcolor}{\rmfamily\fontsize{12.000000}{14.400000}\selectfont\catcode`\^=\active\def^{\ifmmode\sp\else\^{}\fi}\catcode`\%=\active\def%{\%}NC++}}%
\end{pgfscope}%
\end{pgfpicture}%
\makeatother%
\endgroup%

        \caption{$n_{\Omega}=160$}
        \label{fig:5-experiments-haydock-convergence-m-nv160}
    \end{subfigure}
    \caption{Behavior with $m$ for $\sigma=0.05$}
    \label{fig:5-experiments-haydock-convergence-m}
\end{figure}

\todo{Note how approximation not as good as Gaussian due to larger Frobenius norm?}

\begin{table}[ht]
    \caption{Runtime comparison}
    \label{tab:5-experiments-timing-haydock}
   \centering
\renewcommand{\arraystretch}{1.2}
\begin{tabular}{@{}lcccc@{}}
\toprule
 & \shortstack[c]{$m=800$ \\ $n_{\Omega} + n_{\Psi}=40$} & \shortstack[c]{$m=2400$ \\ $n_{\Omega} + n_{\Psi}=40$} & \shortstack[c]{$m=800$ \\ $n_{\Omega} + n_{\Psi}=160$} & \shortstack[c]{$m=2400$ \\ $n_{\Omega} + n_{\Psi}=160$}\\
\midrule
Haydock & $5.804$ $\pm$ $0.024$ & $12.611$ $\pm$ $0.136$ & $23.408$ $\pm$ $0.150$ & $48.933$ $\pm$ $0.394$ \\
NC & $1.044$ $\pm$ $0.013$ & $3.116$ $\pm$ $0.043$ & $4.396$ $\pm$ $0.034$ & $13.019$ $\pm$ $0.097$ \\
NC++ & $0.918$ $\pm$ $0.008$ & $2.761$ $\pm$ $0.048$ & $3.091$ $\pm$ $0.070$ & $9.391$ $\pm$ $0.186$ \\
\bottomrule
\end{tabular}

\end{table}

%%%%%%%%%%%%%%%%%%%%%%%%%%%%%%%%%%%%%%%%%%%%%%%%%%%%%%%%%%%%%%%%%%%%%%%%%%%%%%%%

\section{Application to various matrices}
\label{sec:5-experiments-various-matrices}

\todo{[Same plots for multiple matrices]}

\todo{Maybe comparison with SLQ, KPM, ...}

%\begin{figure}[ht]
%    \centering
%    %% Creator: Matplotlib, PGF backend
%%
%% To include the figure in your LaTeX document, write
%%   \input{<filename>.pgf}
%%
%% Make sure the required packages are loaded in your preamble
%%   \usepackage{pgf}
%%
%% Also ensure that all the required font packages are loaded; for instance,
%% the lmodern package is sometimes necessary when using math font.
%%   \usepackage{lmodern}
%%
%% Figures using additional raster images can only be included by \input if
%% they are in the same directory as the main LaTeX file. For loading figures
%% from other directories you can use the `import` package
%%   \usepackage{import}
%%
%% and then include the figures with
%%   \import{<path to file>}{<filename>.pgf}
%%
%% Matplotlib used the following preamble
%%   \def\mathdefault#1{#1}
%%   \everymath=\expandafter{\the\everymath\displaystyle}
%%   
%%   \makeatletter\@ifpackageloaded{underscore}{}{\usepackage[strings]{underscore}}\makeatother
%%
\begingroup%
\makeatletter%
\begin{pgfpicture}%
\pgfpathrectangle{\pgfpointorigin}{\pgfqpoint{2.759413in}{2.586282in}}%
\pgfusepath{use as bounding box, clip}%
\begin{pgfscope}%
\pgfsetbuttcap%
\pgfsetmiterjoin%
\definecolor{currentfill}{rgb}{1.000000,1.000000,1.000000}%
\pgfsetfillcolor{currentfill}%
\pgfsetlinewidth{0.000000pt}%
\definecolor{currentstroke}{rgb}{1.000000,1.000000,1.000000}%
\pgfsetstrokecolor{currentstroke}%
\pgfsetdash{}{0pt}%
\pgfpathmoveto{\pgfqpoint{0.000000in}{0.000000in}}%
\pgfpathlineto{\pgfqpoint{2.759413in}{0.000000in}}%
\pgfpathlineto{\pgfqpoint{2.759413in}{2.586282in}}%
\pgfpathlineto{\pgfqpoint{0.000000in}{2.586282in}}%
\pgfpathlineto{\pgfqpoint{0.000000in}{0.000000in}}%
\pgfpathclose%
\pgfusepath{fill}%
\end{pgfscope}%
\begin{pgfscope}%
\pgfsetbuttcap%
\pgfsetmiterjoin%
\definecolor{currentfill}{rgb}{1.000000,1.000000,1.000000}%
\pgfsetfillcolor{currentfill}%
\pgfsetlinewidth{0.000000pt}%
\definecolor{currentstroke}{rgb}{0.000000,0.000000,0.000000}%
\pgfsetstrokecolor{currentstroke}%
\pgfsetstrokeopacity{0.000000}%
\pgfsetdash{}{0pt}%
\pgfpathmoveto{\pgfqpoint{0.721913in}{0.549073in}}%
\pgfpathlineto{\pgfqpoint{2.659413in}{0.549073in}}%
\pgfpathlineto{\pgfqpoint{2.659413in}{2.474073in}}%
\pgfpathlineto{\pgfqpoint{0.721913in}{2.474073in}}%
\pgfpathlineto{\pgfqpoint{0.721913in}{0.549073in}}%
\pgfpathclose%
\pgfusepath{fill}%
\end{pgfscope}%
\begin{pgfscope}%
\pgfsetbuttcap%
\pgfsetroundjoin%
\definecolor{currentfill}{rgb}{0.000000,0.000000,0.000000}%
\pgfsetfillcolor{currentfill}%
\pgfsetlinewidth{0.803000pt}%
\definecolor{currentstroke}{rgb}{0.000000,0.000000,0.000000}%
\pgfsetstrokecolor{currentstroke}%
\pgfsetdash{}{0pt}%
\pgfsys@defobject{currentmarker}{\pgfqpoint{0.000000in}{-0.048611in}}{\pgfqpoint{0.000000in}{0.000000in}}{%
\pgfpathmoveto{\pgfqpoint{0.000000in}{0.000000in}}%
\pgfpathlineto{\pgfqpoint{0.000000in}{-0.048611in}}%
\pgfusepath{stroke,fill}%
}%
\begin{pgfscope}%
\pgfsys@transformshift{1.771566in}{0.549073in}%
\pgfsys@useobject{currentmarker}{}%
\end{pgfscope}%
\end{pgfscope}%
\begin{pgfscope}%
\definecolor{textcolor}{rgb}{0.000000,0.000000,0.000000}%
\pgfsetstrokecolor{textcolor}%
\pgfsetfillcolor{textcolor}%
\pgftext[x=1.771566in,y=0.451851in,,top]{\color{textcolor}{\rmfamily\fontsize{12.000000}{14.400000}\selectfont\catcode`\^=\active\def^{\ifmmode\sp\else\^{}\fi}\catcode`\%=\active\def%{\%}$\mathdefault{10^{2}}$}}%
\end{pgfscope}%
\begin{pgfscope}%
\pgfsetbuttcap%
\pgfsetroundjoin%
\definecolor{currentfill}{rgb}{0.000000,0.000000,0.000000}%
\pgfsetfillcolor{currentfill}%
\pgfsetlinewidth{0.602250pt}%
\definecolor{currentstroke}{rgb}{0.000000,0.000000,0.000000}%
\pgfsetstrokecolor{currentstroke}%
\pgfsetdash{}{0pt}%
\pgfsys@defobject{currentmarker}{\pgfqpoint{0.000000in}{-0.027778in}}{\pgfqpoint{0.000000in}{0.000000in}}{%
\pgfpathmoveto{\pgfqpoint{0.000000in}{0.000000in}}%
\pgfpathlineto{\pgfqpoint{0.000000in}{-0.027778in}}%
\pgfusepath{stroke,fill}%
}%
\begin{pgfscope}%
\pgfsys@transformshift{0.839681in}{0.549073in}%
\pgfsys@useobject{currentmarker}{}%
\end{pgfscope}%
\end{pgfscope}%
\begin{pgfscope}%
\pgfsetbuttcap%
\pgfsetroundjoin%
\definecolor{currentfill}{rgb}{0.000000,0.000000,0.000000}%
\pgfsetfillcolor{currentfill}%
\pgfsetlinewidth{0.602250pt}%
\definecolor{currentstroke}{rgb}{0.000000,0.000000,0.000000}%
\pgfsetstrokecolor{currentstroke}%
\pgfsetdash{}{0pt}%
\pgfsys@defobject{currentmarker}{\pgfqpoint{0.000000in}{-0.027778in}}{\pgfqpoint{0.000000in}{0.000000in}}{%
\pgfpathmoveto{\pgfqpoint{0.000000in}{0.000000in}}%
\pgfpathlineto{\pgfqpoint{0.000000in}{-0.027778in}}%
\pgfusepath{stroke,fill}%
}%
\begin{pgfscope}%
\pgfsys@transformshift{1.074450in}{0.549073in}%
\pgfsys@useobject{currentmarker}{}%
\end{pgfscope}%
\end{pgfscope}%
\begin{pgfscope}%
\pgfsetbuttcap%
\pgfsetroundjoin%
\definecolor{currentfill}{rgb}{0.000000,0.000000,0.000000}%
\pgfsetfillcolor{currentfill}%
\pgfsetlinewidth{0.602250pt}%
\definecolor{currentstroke}{rgb}{0.000000,0.000000,0.000000}%
\pgfsetstrokecolor{currentstroke}%
\pgfsetdash{}{0pt}%
\pgfsys@defobject{currentmarker}{\pgfqpoint{0.000000in}{-0.027778in}}{\pgfqpoint{0.000000in}{0.000000in}}{%
\pgfpathmoveto{\pgfqpoint{0.000000in}{0.000000in}}%
\pgfpathlineto{\pgfqpoint{0.000000in}{-0.027778in}}%
\pgfusepath{stroke,fill}%
}%
\begin{pgfscope}%
\pgfsys@transformshift{1.241022in}{0.549073in}%
\pgfsys@useobject{currentmarker}{}%
\end{pgfscope}%
\end{pgfscope}%
\begin{pgfscope}%
\pgfsetbuttcap%
\pgfsetroundjoin%
\definecolor{currentfill}{rgb}{0.000000,0.000000,0.000000}%
\pgfsetfillcolor{currentfill}%
\pgfsetlinewidth{0.602250pt}%
\definecolor{currentstroke}{rgb}{0.000000,0.000000,0.000000}%
\pgfsetstrokecolor{currentstroke}%
\pgfsetdash{}{0pt}%
\pgfsys@defobject{currentmarker}{\pgfqpoint{0.000000in}{-0.027778in}}{\pgfqpoint{0.000000in}{0.000000in}}{%
\pgfpathmoveto{\pgfqpoint{0.000000in}{0.000000in}}%
\pgfpathlineto{\pgfqpoint{0.000000in}{-0.027778in}}%
\pgfusepath{stroke,fill}%
}%
\begin{pgfscope}%
\pgfsys@transformshift{1.370225in}{0.549073in}%
\pgfsys@useobject{currentmarker}{}%
\end{pgfscope}%
\end{pgfscope}%
\begin{pgfscope}%
\pgfsetbuttcap%
\pgfsetroundjoin%
\definecolor{currentfill}{rgb}{0.000000,0.000000,0.000000}%
\pgfsetfillcolor{currentfill}%
\pgfsetlinewidth{0.602250pt}%
\definecolor{currentstroke}{rgb}{0.000000,0.000000,0.000000}%
\pgfsetstrokecolor{currentstroke}%
\pgfsetdash{}{0pt}%
\pgfsys@defobject{currentmarker}{\pgfqpoint{0.000000in}{-0.027778in}}{\pgfqpoint{0.000000in}{0.000000in}}{%
\pgfpathmoveto{\pgfqpoint{0.000000in}{0.000000in}}%
\pgfpathlineto{\pgfqpoint{0.000000in}{-0.027778in}}%
\pgfusepath{stroke,fill}%
}%
\begin{pgfscope}%
\pgfsys@transformshift{1.475791in}{0.549073in}%
\pgfsys@useobject{currentmarker}{}%
\end{pgfscope}%
\end{pgfscope}%
\begin{pgfscope}%
\pgfsetbuttcap%
\pgfsetroundjoin%
\definecolor{currentfill}{rgb}{0.000000,0.000000,0.000000}%
\pgfsetfillcolor{currentfill}%
\pgfsetlinewidth{0.602250pt}%
\definecolor{currentstroke}{rgb}{0.000000,0.000000,0.000000}%
\pgfsetstrokecolor{currentstroke}%
\pgfsetdash{}{0pt}%
\pgfsys@defobject{currentmarker}{\pgfqpoint{0.000000in}{-0.027778in}}{\pgfqpoint{0.000000in}{0.000000in}}{%
\pgfpathmoveto{\pgfqpoint{0.000000in}{0.000000in}}%
\pgfpathlineto{\pgfqpoint{0.000000in}{-0.027778in}}%
\pgfusepath{stroke,fill}%
}%
\begin{pgfscope}%
\pgfsys@transformshift{1.565047in}{0.549073in}%
\pgfsys@useobject{currentmarker}{}%
\end{pgfscope}%
\end{pgfscope}%
\begin{pgfscope}%
\pgfsetbuttcap%
\pgfsetroundjoin%
\definecolor{currentfill}{rgb}{0.000000,0.000000,0.000000}%
\pgfsetfillcolor{currentfill}%
\pgfsetlinewidth{0.602250pt}%
\definecolor{currentstroke}{rgb}{0.000000,0.000000,0.000000}%
\pgfsetstrokecolor{currentstroke}%
\pgfsetdash{}{0pt}%
\pgfsys@defobject{currentmarker}{\pgfqpoint{0.000000in}{-0.027778in}}{\pgfqpoint{0.000000in}{0.000000in}}{%
\pgfpathmoveto{\pgfqpoint{0.000000in}{0.000000in}}%
\pgfpathlineto{\pgfqpoint{0.000000in}{-0.027778in}}%
\pgfusepath{stroke,fill}%
}%
\begin{pgfscope}%
\pgfsys@transformshift{1.642363in}{0.549073in}%
\pgfsys@useobject{currentmarker}{}%
\end{pgfscope}%
\end{pgfscope}%
\begin{pgfscope}%
\pgfsetbuttcap%
\pgfsetroundjoin%
\definecolor{currentfill}{rgb}{0.000000,0.000000,0.000000}%
\pgfsetfillcolor{currentfill}%
\pgfsetlinewidth{0.602250pt}%
\definecolor{currentstroke}{rgb}{0.000000,0.000000,0.000000}%
\pgfsetstrokecolor{currentstroke}%
\pgfsetdash{}{0pt}%
\pgfsys@defobject{currentmarker}{\pgfqpoint{0.000000in}{-0.027778in}}{\pgfqpoint{0.000000in}{0.000000in}}{%
\pgfpathmoveto{\pgfqpoint{0.000000in}{0.000000in}}%
\pgfpathlineto{\pgfqpoint{0.000000in}{-0.027778in}}%
\pgfusepath{stroke,fill}%
}%
\begin{pgfscope}%
\pgfsys@transformshift{1.710561in}{0.549073in}%
\pgfsys@useobject{currentmarker}{}%
\end{pgfscope}%
\end{pgfscope}%
\begin{pgfscope}%
\pgfsetbuttcap%
\pgfsetroundjoin%
\definecolor{currentfill}{rgb}{0.000000,0.000000,0.000000}%
\pgfsetfillcolor{currentfill}%
\pgfsetlinewidth{0.602250pt}%
\definecolor{currentstroke}{rgb}{0.000000,0.000000,0.000000}%
\pgfsetstrokecolor{currentstroke}%
\pgfsetdash{}{0pt}%
\pgfsys@defobject{currentmarker}{\pgfqpoint{0.000000in}{-0.027778in}}{\pgfqpoint{0.000000in}{0.000000in}}{%
\pgfpathmoveto{\pgfqpoint{0.000000in}{0.000000in}}%
\pgfpathlineto{\pgfqpoint{0.000000in}{-0.027778in}}%
\pgfusepath{stroke,fill}%
}%
\begin{pgfscope}%
\pgfsys@transformshift{2.172907in}{0.549073in}%
\pgfsys@useobject{currentmarker}{}%
\end{pgfscope}%
\end{pgfscope}%
\begin{pgfscope}%
\pgfsetbuttcap%
\pgfsetroundjoin%
\definecolor{currentfill}{rgb}{0.000000,0.000000,0.000000}%
\pgfsetfillcolor{currentfill}%
\pgfsetlinewidth{0.602250pt}%
\definecolor{currentstroke}{rgb}{0.000000,0.000000,0.000000}%
\pgfsetstrokecolor{currentstroke}%
\pgfsetdash{}{0pt}%
\pgfsys@defobject{currentmarker}{\pgfqpoint{0.000000in}{-0.027778in}}{\pgfqpoint{0.000000in}{0.000000in}}{%
\pgfpathmoveto{\pgfqpoint{0.000000in}{0.000000in}}%
\pgfpathlineto{\pgfqpoint{0.000000in}{-0.027778in}}%
\pgfusepath{stroke,fill}%
}%
\begin{pgfscope}%
\pgfsys@transformshift{2.407676in}{0.549073in}%
\pgfsys@useobject{currentmarker}{}%
\end{pgfscope}%
\end{pgfscope}%
\begin{pgfscope}%
\pgfsetbuttcap%
\pgfsetroundjoin%
\definecolor{currentfill}{rgb}{0.000000,0.000000,0.000000}%
\pgfsetfillcolor{currentfill}%
\pgfsetlinewidth{0.602250pt}%
\definecolor{currentstroke}{rgb}{0.000000,0.000000,0.000000}%
\pgfsetstrokecolor{currentstroke}%
\pgfsetdash{}{0pt}%
\pgfsys@defobject{currentmarker}{\pgfqpoint{0.000000in}{-0.027778in}}{\pgfqpoint{0.000000in}{0.000000in}}{%
\pgfpathmoveto{\pgfqpoint{0.000000in}{0.000000in}}%
\pgfpathlineto{\pgfqpoint{0.000000in}{-0.027778in}}%
\pgfusepath{stroke,fill}%
}%
\begin{pgfscope}%
\pgfsys@transformshift{2.574248in}{0.549073in}%
\pgfsys@useobject{currentmarker}{}%
\end{pgfscope}%
\end{pgfscope}%
\begin{pgfscope}%
\definecolor{textcolor}{rgb}{0.000000,0.000000,0.000000}%
\pgfsetstrokecolor{textcolor}%
\pgfsetfillcolor{textcolor}%
\pgftext[x=1.690663in,y=0.248148in,,top]{\color{textcolor}{\rmfamily\fontsize{12.000000}{14.400000}\selectfont\catcode`\^=\active\def^{\ifmmode\sp\else\^{}\fi}\catcode`\%=\active\def%{\%}$n_{\Omega} + n_{\Psi}$}}%
\end{pgfscope}%
\begin{pgfscope}%
\pgfsetbuttcap%
\pgfsetroundjoin%
\definecolor{currentfill}{rgb}{0.000000,0.000000,0.000000}%
\pgfsetfillcolor{currentfill}%
\pgfsetlinewidth{0.803000pt}%
\definecolor{currentstroke}{rgb}{0.000000,0.000000,0.000000}%
\pgfsetstrokecolor{currentstroke}%
\pgfsetdash{}{0pt}%
\pgfsys@defobject{currentmarker}{\pgfqpoint{-0.048611in}{0.000000in}}{\pgfqpoint{-0.000000in}{0.000000in}}{%
\pgfpathmoveto{\pgfqpoint{-0.000000in}{0.000000in}}%
\pgfpathlineto{\pgfqpoint{-0.048611in}{0.000000in}}%
\pgfusepath{stroke,fill}%
}%
\begin{pgfscope}%
\pgfsys@transformshift{0.721913in}{0.913333in}%
\pgfsys@useobject{currentmarker}{}%
\end{pgfscope}%
\end{pgfscope}%
\begin{pgfscope}%
\definecolor{textcolor}{rgb}{0.000000,0.000000,0.000000}%
\pgfsetstrokecolor{textcolor}%
\pgfsetfillcolor{textcolor}%
\pgftext[x=0.303703in, y=0.855462in, left, base]{\color{textcolor}{\rmfamily\fontsize{12.000000}{14.400000}\selectfont\catcode`\^=\active\def^{\ifmmode\sp\else\^{}\fi}\catcode`\%=\active\def%{\%}$\mathdefault{10^{-6}}$}}%
\end{pgfscope}%
\begin{pgfscope}%
\pgfsetbuttcap%
\pgfsetroundjoin%
\definecolor{currentfill}{rgb}{0.000000,0.000000,0.000000}%
\pgfsetfillcolor{currentfill}%
\pgfsetlinewidth{0.803000pt}%
\definecolor{currentstroke}{rgb}{0.000000,0.000000,0.000000}%
\pgfsetstrokecolor{currentstroke}%
\pgfsetdash{}{0pt}%
\pgfsys@defobject{currentmarker}{\pgfqpoint{-0.048611in}{0.000000in}}{\pgfqpoint{-0.000000in}{0.000000in}}{%
\pgfpathmoveto{\pgfqpoint{-0.000000in}{0.000000in}}%
\pgfpathlineto{\pgfqpoint{-0.048611in}{0.000000in}}%
\pgfusepath{stroke,fill}%
}%
\begin{pgfscope}%
\pgfsys@transformshift{0.721913in}{1.418359in}%
\pgfsys@useobject{currentmarker}{}%
\end{pgfscope}%
\end{pgfscope}%
\begin{pgfscope}%
\definecolor{textcolor}{rgb}{0.000000,0.000000,0.000000}%
\pgfsetstrokecolor{textcolor}%
\pgfsetfillcolor{textcolor}%
\pgftext[x=0.303703in, y=1.360489in, left, base]{\color{textcolor}{\rmfamily\fontsize{12.000000}{14.400000}\selectfont\catcode`\^=\active\def^{\ifmmode\sp\else\^{}\fi}\catcode`\%=\active\def%{\%}$\mathdefault{10^{-4}}$}}%
\end{pgfscope}%
\begin{pgfscope}%
\pgfsetbuttcap%
\pgfsetroundjoin%
\definecolor{currentfill}{rgb}{0.000000,0.000000,0.000000}%
\pgfsetfillcolor{currentfill}%
\pgfsetlinewidth{0.803000pt}%
\definecolor{currentstroke}{rgb}{0.000000,0.000000,0.000000}%
\pgfsetstrokecolor{currentstroke}%
\pgfsetdash{}{0pt}%
\pgfsys@defobject{currentmarker}{\pgfqpoint{-0.048611in}{0.000000in}}{\pgfqpoint{-0.000000in}{0.000000in}}{%
\pgfpathmoveto{\pgfqpoint{-0.000000in}{0.000000in}}%
\pgfpathlineto{\pgfqpoint{-0.048611in}{0.000000in}}%
\pgfusepath{stroke,fill}%
}%
\begin{pgfscope}%
\pgfsys@transformshift{0.721913in}{1.923385in}%
\pgfsys@useobject{currentmarker}{}%
\end{pgfscope}%
\end{pgfscope}%
\begin{pgfscope}%
\definecolor{textcolor}{rgb}{0.000000,0.000000,0.000000}%
\pgfsetstrokecolor{textcolor}%
\pgfsetfillcolor{textcolor}%
\pgftext[x=0.303703in, y=1.865515in, left, base]{\color{textcolor}{\rmfamily\fontsize{12.000000}{14.400000}\selectfont\catcode`\^=\active\def^{\ifmmode\sp\else\^{}\fi}\catcode`\%=\active\def%{\%}$\mathdefault{10^{-2}}$}}%
\end{pgfscope}%
\begin{pgfscope}%
\pgfsetbuttcap%
\pgfsetroundjoin%
\definecolor{currentfill}{rgb}{0.000000,0.000000,0.000000}%
\pgfsetfillcolor{currentfill}%
\pgfsetlinewidth{0.803000pt}%
\definecolor{currentstroke}{rgb}{0.000000,0.000000,0.000000}%
\pgfsetstrokecolor{currentstroke}%
\pgfsetdash{}{0pt}%
\pgfsys@defobject{currentmarker}{\pgfqpoint{-0.048611in}{0.000000in}}{\pgfqpoint{-0.000000in}{0.000000in}}{%
\pgfpathmoveto{\pgfqpoint{-0.000000in}{0.000000in}}%
\pgfpathlineto{\pgfqpoint{-0.048611in}{0.000000in}}%
\pgfusepath{stroke,fill}%
}%
\begin{pgfscope}%
\pgfsys@transformshift{0.721913in}{2.428411in}%
\pgfsys@useobject{currentmarker}{}%
\end{pgfscope}%
\end{pgfscope}%
\begin{pgfscope}%
\definecolor{textcolor}{rgb}{0.000000,0.000000,0.000000}%
\pgfsetstrokecolor{textcolor}%
\pgfsetfillcolor{textcolor}%
\pgftext[x=0.395525in, y=2.370541in, left, base]{\color{textcolor}{\rmfamily\fontsize{12.000000}{14.400000}\selectfont\catcode`\^=\active\def^{\ifmmode\sp\else\^{}\fi}\catcode`\%=\active\def%{\%}$\mathdefault{10^{0}}$}}%
\end{pgfscope}%
\begin{pgfscope}%
\definecolor{textcolor}{rgb}{0.000000,0.000000,0.000000}%
\pgfsetstrokecolor{textcolor}%
\pgfsetfillcolor{textcolor}%
\pgftext[x=0.248148in,y=1.511573in,,bottom,rotate=90.000000]{\color{textcolor}{\rmfamily\fontsize{12.000000}{14.400000}\selectfont\catcode`\^=\active\def^{\ifmmode\sp\else\^{}\fi}\catcode`\%=\active\def%{\%}$L^1$ relative error}}%
\end{pgfscope}%
\begin{pgfscope}%
\pgfpathrectangle{\pgfqpoint{0.721913in}{0.549073in}}{\pgfqpoint{1.937500in}{1.925000in}}%
\pgfusepath{clip}%
\pgfsetrectcap%
\pgfsetroundjoin%
\pgfsetlinewidth{1.003750pt}%
\definecolor{currentstroke}{rgb}{0.001462,0.000466,0.013866}%
\pgfsetstrokecolor{currentstroke}%
\pgfsetdash{}{0pt}%
\pgfpathmoveto{\pgfqpoint{0.809982in}{2.055459in}}%
\pgfpathlineto{\pgfqpoint{1.180017in}{2.021849in}}%
\pgfpathlineto{\pgfqpoint{1.530977in}{1.995031in}}%
\pgfpathlineto{\pgfqpoint{1.877132in}{1.962044in}}%
\pgfpathlineto{\pgfqpoint{2.222805in}{1.922940in}}%
\pgfpathlineto{\pgfqpoint{2.571345in}{1.907338in}}%
\pgfusepath{stroke}%
\end{pgfscope}%
\begin{pgfscope}%
\pgfpathrectangle{\pgfqpoint{0.721913in}{0.549073in}}{\pgfqpoint{1.937500in}{1.925000in}}%
\pgfusepath{clip}%
\pgfsetbuttcap%
\pgfsetroundjoin%
\definecolor{currentfill}{rgb}{0.001462,0.000466,0.013866}%
\pgfsetfillcolor{currentfill}%
\pgfsetlinewidth{1.003750pt}%
\definecolor{currentstroke}{rgb}{0.001462,0.000466,0.013866}%
\pgfsetstrokecolor{currentstroke}%
\pgfsetdash{}{0pt}%
\pgfsys@defobject{currentmarker}{\pgfqpoint{-0.020833in}{-0.020833in}}{\pgfqpoint{0.020833in}{0.020833in}}{%
\pgfpathmoveto{\pgfqpoint{0.000000in}{-0.020833in}}%
\pgfpathcurveto{\pgfqpoint{0.005525in}{-0.020833in}}{\pgfqpoint{0.010825in}{-0.018638in}}{\pgfqpoint{0.014731in}{-0.014731in}}%
\pgfpathcurveto{\pgfqpoint{0.018638in}{-0.010825in}}{\pgfqpoint{0.020833in}{-0.005525in}}{\pgfqpoint{0.020833in}{0.000000in}}%
\pgfpathcurveto{\pgfqpoint{0.020833in}{0.005525in}}{\pgfqpoint{0.018638in}{0.010825in}}{\pgfqpoint{0.014731in}{0.014731in}}%
\pgfpathcurveto{\pgfqpoint{0.010825in}{0.018638in}}{\pgfqpoint{0.005525in}{0.020833in}}{\pgfqpoint{0.000000in}{0.020833in}}%
\pgfpathcurveto{\pgfqpoint{-0.005525in}{0.020833in}}{\pgfqpoint{-0.010825in}{0.018638in}}{\pgfqpoint{-0.014731in}{0.014731in}}%
\pgfpathcurveto{\pgfqpoint{-0.018638in}{0.010825in}}{\pgfqpoint{-0.020833in}{0.005525in}}{\pgfqpoint{-0.020833in}{0.000000in}}%
\pgfpathcurveto{\pgfqpoint{-0.020833in}{-0.005525in}}{\pgfqpoint{-0.018638in}{-0.010825in}}{\pgfqpoint{-0.014731in}{-0.014731in}}%
\pgfpathcurveto{\pgfqpoint{-0.010825in}{-0.018638in}}{\pgfqpoint{-0.005525in}{-0.020833in}}{\pgfqpoint{0.000000in}{-0.020833in}}%
\pgfpathlineto{\pgfqpoint{0.000000in}{-0.020833in}}%
\pgfpathclose%
\pgfusepath{stroke,fill}%
}%
\begin{pgfscope}%
\pgfsys@transformshift{0.809982in}{2.055459in}%
\pgfsys@useobject{currentmarker}{}%
\end{pgfscope}%
\begin{pgfscope}%
\pgfsys@transformshift{1.180017in}{2.021849in}%
\pgfsys@useobject{currentmarker}{}%
\end{pgfscope}%
\begin{pgfscope}%
\pgfsys@transformshift{1.530977in}{1.995031in}%
\pgfsys@useobject{currentmarker}{}%
\end{pgfscope}%
\begin{pgfscope}%
\pgfsys@transformshift{1.877132in}{1.962044in}%
\pgfsys@useobject{currentmarker}{}%
\end{pgfscope}%
\begin{pgfscope}%
\pgfsys@transformshift{2.222805in}{1.922940in}%
\pgfsys@useobject{currentmarker}{}%
\end{pgfscope}%
\begin{pgfscope}%
\pgfsys@transformshift{2.571345in}{1.907338in}%
\pgfsys@useobject{currentmarker}{}%
\end{pgfscope}%
\end{pgfscope}%
\begin{pgfscope}%
\pgfpathrectangle{\pgfqpoint{0.721913in}{0.549073in}}{\pgfqpoint{1.937500in}{1.925000in}}%
\pgfusepath{clip}%
\pgfsetrectcap%
\pgfsetroundjoin%
\pgfsetlinewidth{1.003750pt}%
\definecolor{currentstroke}{rgb}{0.445163,0.122724,0.506901}%
\pgfsetstrokecolor{currentstroke}%
\pgfsetdash{}{0pt}%
\pgfpathmoveto{\pgfqpoint{0.809982in}{2.386573in}}%
\pgfpathlineto{\pgfqpoint{1.180017in}{2.349814in}}%
\pgfpathlineto{\pgfqpoint{1.530977in}{2.283901in}}%
\pgfpathlineto{\pgfqpoint{1.877132in}{2.162841in}}%
\pgfpathlineto{\pgfqpoint{2.222805in}{1.817480in}}%
\pgfpathlineto{\pgfqpoint{2.571345in}{0.636573in}}%
\pgfusepath{stroke}%
\end{pgfscope}%
\begin{pgfscope}%
\pgfpathrectangle{\pgfqpoint{0.721913in}{0.549073in}}{\pgfqpoint{1.937500in}{1.925000in}}%
\pgfusepath{clip}%
\pgfsetbuttcap%
\pgfsetroundjoin%
\definecolor{currentfill}{rgb}{0.445163,0.122724,0.506901}%
\pgfsetfillcolor{currentfill}%
\pgfsetlinewidth{1.003750pt}%
\definecolor{currentstroke}{rgb}{0.445163,0.122724,0.506901}%
\pgfsetstrokecolor{currentstroke}%
\pgfsetdash{}{0pt}%
\pgfsys@defobject{currentmarker}{\pgfqpoint{-0.020833in}{-0.020833in}}{\pgfqpoint{0.020833in}{0.020833in}}{%
\pgfpathmoveto{\pgfqpoint{0.000000in}{-0.020833in}}%
\pgfpathcurveto{\pgfqpoint{0.005525in}{-0.020833in}}{\pgfqpoint{0.010825in}{-0.018638in}}{\pgfqpoint{0.014731in}{-0.014731in}}%
\pgfpathcurveto{\pgfqpoint{0.018638in}{-0.010825in}}{\pgfqpoint{0.020833in}{-0.005525in}}{\pgfqpoint{0.020833in}{0.000000in}}%
\pgfpathcurveto{\pgfqpoint{0.020833in}{0.005525in}}{\pgfqpoint{0.018638in}{0.010825in}}{\pgfqpoint{0.014731in}{0.014731in}}%
\pgfpathcurveto{\pgfqpoint{0.010825in}{0.018638in}}{\pgfqpoint{0.005525in}{0.020833in}}{\pgfqpoint{0.000000in}{0.020833in}}%
\pgfpathcurveto{\pgfqpoint{-0.005525in}{0.020833in}}{\pgfqpoint{-0.010825in}{0.018638in}}{\pgfqpoint{-0.014731in}{0.014731in}}%
\pgfpathcurveto{\pgfqpoint{-0.018638in}{0.010825in}}{\pgfqpoint{-0.020833in}{0.005525in}}{\pgfqpoint{-0.020833in}{0.000000in}}%
\pgfpathcurveto{\pgfqpoint{-0.020833in}{-0.005525in}}{\pgfqpoint{-0.018638in}{-0.010825in}}{\pgfqpoint{-0.014731in}{-0.014731in}}%
\pgfpathcurveto{\pgfqpoint{-0.010825in}{-0.018638in}}{\pgfqpoint{-0.005525in}{-0.020833in}}{\pgfqpoint{0.000000in}{-0.020833in}}%
\pgfpathlineto{\pgfqpoint{0.000000in}{-0.020833in}}%
\pgfpathclose%
\pgfusepath{stroke,fill}%
}%
\begin{pgfscope}%
\pgfsys@transformshift{0.809982in}{2.386573in}%
\pgfsys@useobject{currentmarker}{}%
\end{pgfscope}%
\begin{pgfscope}%
\pgfsys@transformshift{1.180017in}{2.349814in}%
\pgfsys@useobject{currentmarker}{}%
\end{pgfscope}%
\begin{pgfscope}%
\pgfsys@transformshift{1.530977in}{2.283901in}%
\pgfsys@useobject{currentmarker}{}%
\end{pgfscope}%
\begin{pgfscope}%
\pgfsys@transformshift{1.877132in}{2.162841in}%
\pgfsys@useobject{currentmarker}{}%
\end{pgfscope}%
\begin{pgfscope}%
\pgfsys@transformshift{2.222805in}{1.817480in}%
\pgfsys@useobject{currentmarker}{}%
\end{pgfscope}%
\begin{pgfscope}%
\pgfsys@transformshift{2.571345in}{0.636573in}%
\pgfsys@useobject{currentmarker}{}%
\end{pgfscope}%
\end{pgfscope}%
\begin{pgfscope}%
\pgfpathrectangle{\pgfqpoint{0.721913in}{0.549073in}}{\pgfqpoint{1.937500in}{1.925000in}}%
\pgfusepath{clip}%
\pgfsetrectcap%
\pgfsetroundjoin%
\pgfsetlinewidth{1.003750pt}%
\definecolor{currentstroke}{rgb}{0.944006,0.377643,0.365136}%
\pgfsetstrokecolor{currentstroke}%
\pgfsetdash{}{0pt}%
\pgfpathmoveto{\pgfqpoint{0.809982in}{2.091909in}}%
\pgfpathlineto{\pgfqpoint{1.180017in}{2.024101in}}%
\pgfpathlineto{\pgfqpoint{1.530977in}{1.943923in}}%
\pgfpathlineto{\pgfqpoint{1.877132in}{1.822271in}}%
\pgfpathlineto{\pgfqpoint{2.222805in}{1.716867in}}%
\pgfpathlineto{\pgfqpoint{2.571345in}{1.409788in}}%
\pgfusepath{stroke}%
\end{pgfscope}%
\begin{pgfscope}%
\pgfpathrectangle{\pgfqpoint{0.721913in}{0.549073in}}{\pgfqpoint{1.937500in}{1.925000in}}%
\pgfusepath{clip}%
\pgfsetbuttcap%
\pgfsetroundjoin%
\definecolor{currentfill}{rgb}{0.944006,0.377643,0.365136}%
\pgfsetfillcolor{currentfill}%
\pgfsetlinewidth{1.003750pt}%
\definecolor{currentstroke}{rgb}{0.944006,0.377643,0.365136}%
\pgfsetstrokecolor{currentstroke}%
\pgfsetdash{}{0pt}%
\pgfsys@defobject{currentmarker}{\pgfqpoint{-0.020833in}{-0.020833in}}{\pgfqpoint{0.020833in}{0.020833in}}{%
\pgfpathmoveto{\pgfqpoint{0.000000in}{-0.020833in}}%
\pgfpathcurveto{\pgfqpoint{0.005525in}{-0.020833in}}{\pgfqpoint{0.010825in}{-0.018638in}}{\pgfqpoint{0.014731in}{-0.014731in}}%
\pgfpathcurveto{\pgfqpoint{0.018638in}{-0.010825in}}{\pgfqpoint{0.020833in}{-0.005525in}}{\pgfqpoint{0.020833in}{0.000000in}}%
\pgfpathcurveto{\pgfqpoint{0.020833in}{0.005525in}}{\pgfqpoint{0.018638in}{0.010825in}}{\pgfqpoint{0.014731in}{0.014731in}}%
\pgfpathcurveto{\pgfqpoint{0.010825in}{0.018638in}}{\pgfqpoint{0.005525in}{0.020833in}}{\pgfqpoint{0.000000in}{0.020833in}}%
\pgfpathcurveto{\pgfqpoint{-0.005525in}{0.020833in}}{\pgfqpoint{-0.010825in}{0.018638in}}{\pgfqpoint{-0.014731in}{0.014731in}}%
\pgfpathcurveto{\pgfqpoint{-0.018638in}{0.010825in}}{\pgfqpoint{-0.020833in}{0.005525in}}{\pgfqpoint{-0.020833in}{0.000000in}}%
\pgfpathcurveto{\pgfqpoint{-0.020833in}{-0.005525in}}{\pgfqpoint{-0.018638in}{-0.010825in}}{\pgfqpoint{-0.014731in}{-0.014731in}}%
\pgfpathcurveto{\pgfqpoint{-0.010825in}{-0.018638in}}{\pgfqpoint{-0.005525in}{-0.020833in}}{\pgfqpoint{0.000000in}{-0.020833in}}%
\pgfpathlineto{\pgfqpoint{0.000000in}{-0.020833in}}%
\pgfpathclose%
\pgfusepath{stroke,fill}%
}%
\begin{pgfscope}%
\pgfsys@transformshift{0.809982in}{2.091909in}%
\pgfsys@useobject{currentmarker}{}%
\end{pgfscope}%
\begin{pgfscope}%
\pgfsys@transformshift{1.180017in}{2.024101in}%
\pgfsys@useobject{currentmarker}{}%
\end{pgfscope}%
\begin{pgfscope}%
\pgfsys@transformshift{1.530977in}{1.943923in}%
\pgfsys@useobject{currentmarker}{}%
\end{pgfscope}%
\begin{pgfscope}%
\pgfsys@transformshift{1.877132in}{1.822271in}%
\pgfsys@useobject{currentmarker}{}%
\end{pgfscope}%
\begin{pgfscope}%
\pgfsys@transformshift{2.222805in}{1.716867in}%
\pgfsys@useobject{currentmarker}{}%
\end{pgfscope}%
\begin{pgfscope}%
\pgfsys@transformshift{2.571345in}{1.409788in}%
\pgfsys@useobject{currentmarker}{}%
\end{pgfscope}%
\end{pgfscope}%
\begin{pgfscope}%
\pgfsetrectcap%
\pgfsetmiterjoin%
\pgfsetlinewidth{0.803000pt}%
\definecolor{currentstroke}{rgb}{0.000000,0.000000,0.000000}%
\pgfsetstrokecolor{currentstroke}%
\pgfsetdash{}{0pt}%
\pgfpathmoveto{\pgfqpoint{0.721913in}{0.549073in}}%
\pgfpathlineto{\pgfqpoint{0.721913in}{2.474073in}}%
\pgfusepath{stroke}%
\end{pgfscope}%
\begin{pgfscope}%
\pgfsetrectcap%
\pgfsetmiterjoin%
\pgfsetlinewidth{0.803000pt}%
\definecolor{currentstroke}{rgb}{0.000000,0.000000,0.000000}%
\pgfsetstrokecolor{currentstroke}%
\pgfsetdash{}{0pt}%
\pgfpathmoveto{\pgfqpoint{2.659413in}{0.549073in}}%
\pgfpathlineto{\pgfqpoint{2.659413in}{2.474073in}}%
\pgfusepath{stroke}%
\end{pgfscope}%
\begin{pgfscope}%
\pgfsetrectcap%
\pgfsetmiterjoin%
\pgfsetlinewidth{0.803000pt}%
\definecolor{currentstroke}{rgb}{0.000000,0.000000,0.000000}%
\pgfsetstrokecolor{currentstroke}%
\pgfsetdash{}{0pt}%
\pgfpathmoveto{\pgfqpoint{0.721913in}{0.549073in}}%
\pgfpathlineto{\pgfqpoint{2.659413in}{0.549073in}}%
\pgfusepath{stroke}%
\end{pgfscope}%
\begin{pgfscope}%
\pgfsetrectcap%
\pgfsetmiterjoin%
\pgfsetlinewidth{0.803000pt}%
\definecolor{currentstroke}{rgb}{0.000000,0.000000,0.000000}%
\pgfsetstrokecolor{currentstroke}%
\pgfsetdash{}{0pt}%
\pgfpathmoveto{\pgfqpoint{0.721913in}{2.474073in}}%
\pgfpathlineto{\pgfqpoint{2.659413in}{2.474073in}}%
\pgfusepath{stroke}%
\end{pgfscope}%
\begin{pgfscope}%
\pgfsetbuttcap%
\pgfsetmiterjoin%
\definecolor{currentfill}{rgb}{1.000000,1.000000,1.000000}%
\pgfsetfillcolor{currentfill}%
\pgfsetfillopacity{0.800000}%
\pgfsetlinewidth{1.003750pt}%
\definecolor{currentstroke}{rgb}{0.800000,0.800000,0.800000}%
\pgfsetstrokecolor{currentstroke}%
\pgfsetstrokeopacity{0.800000}%
\pgfsetdash{}{0pt}%
\pgfpathmoveto{\pgfqpoint{0.838580in}{0.632406in}}%
\pgfpathlineto{\pgfqpoint{1.865967in}{0.632406in}}%
\pgfpathquadraticcurveto{\pgfqpoint{1.899300in}{0.632406in}}{\pgfqpoint{1.899300in}{0.665739in}}%
\pgfpathlineto{\pgfqpoint{1.899300in}{1.346294in}}%
\pgfpathquadraticcurveto{\pgfqpoint{1.899300in}{1.379627in}}{\pgfqpoint{1.865967in}{1.379627in}}%
\pgfpathlineto{\pgfqpoint{0.838580in}{1.379627in}}%
\pgfpathquadraticcurveto{\pgfqpoint{0.805247in}{1.379627in}}{\pgfqpoint{0.805247in}{1.346294in}}%
\pgfpathlineto{\pgfqpoint{0.805247in}{0.665739in}}%
\pgfpathquadraticcurveto{\pgfqpoint{0.805247in}{0.632406in}}{\pgfqpoint{0.838580in}{0.632406in}}%
\pgfpathlineto{\pgfqpoint{0.838580in}{0.632406in}}%
\pgfpathclose%
\pgfusepath{stroke,fill}%
\end{pgfscope}%
\begin{pgfscope}%
\pgfsetrectcap%
\pgfsetroundjoin%
\pgfsetlinewidth{1.003750pt}%
\definecolor{currentstroke}{rgb}{0.001462,0.000466,0.013866}%
\pgfsetstrokecolor{currentstroke}%
\pgfsetdash{}{0pt}%
\pgfpathmoveto{\pgfqpoint{0.871913in}{1.254627in}}%
\pgfpathlineto{\pgfqpoint{1.038580in}{1.254627in}}%
\pgfpathlineto{\pgfqpoint{1.205247in}{1.254627in}}%
\pgfusepath{stroke}%
\end{pgfscope}%
\begin{pgfscope}%
\pgfsetbuttcap%
\pgfsetroundjoin%
\definecolor{currentfill}{rgb}{0.001462,0.000466,0.013866}%
\pgfsetfillcolor{currentfill}%
\pgfsetlinewidth{1.003750pt}%
\definecolor{currentstroke}{rgb}{0.001462,0.000466,0.013866}%
\pgfsetstrokecolor{currentstroke}%
\pgfsetdash{}{0pt}%
\pgfsys@defobject{currentmarker}{\pgfqpoint{-0.020833in}{-0.020833in}}{\pgfqpoint{0.020833in}{0.020833in}}{%
\pgfpathmoveto{\pgfqpoint{0.000000in}{-0.020833in}}%
\pgfpathcurveto{\pgfqpoint{0.005525in}{-0.020833in}}{\pgfqpoint{0.010825in}{-0.018638in}}{\pgfqpoint{0.014731in}{-0.014731in}}%
\pgfpathcurveto{\pgfqpoint{0.018638in}{-0.010825in}}{\pgfqpoint{0.020833in}{-0.005525in}}{\pgfqpoint{0.020833in}{0.000000in}}%
\pgfpathcurveto{\pgfqpoint{0.020833in}{0.005525in}}{\pgfqpoint{0.018638in}{0.010825in}}{\pgfqpoint{0.014731in}{0.014731in}}%
\pgfpathcurveto{\pgfqpoint{0.010825in}{0.018638in}}{\pgfqpoint{0.005525in}{0.020833in}}{\pgfqpoint{0.000000in}{0.020833in}}%
\pgfpathcurveto{\pgfqpoint{-0.005525in}{0.020833in}}{\pgfqpoint{-0.010825in}{0.018638in}}{\pgfqpoint{-0.014731in}{0.014731in}}%
\pgfpathcurveto{\pgfqpoint{-0.018638in}{0.010825in}}{\pgfqpoint{-0.020833in}{0.005525in}}{\pgfqpoint{-0.020833in}{0.000000in}}%
\pgfpathcurveto{\pgfqpoint{-0.020833in}{-0.005525in}}{\pgfqpoint{-0.018638in}{-0.010825in}}{\pgfqpoint{-0.014731in}{-0.014731in}}%
\pgfpathcurveto{\pgfqpoint{-0.010825in}{-0.018638in}}{\pgfqpoint{-0.005525in}{-0.020833in}}{\pgfqpoint{0.000000in}{-0.020833in}}%
\pgfpathlineto{\pgfqpoint{0.000000in}{-0.020833in}}%
\pgfpathclose%
\pgfusepath{stroke,fill}%
}%
\begin{pgfscope}%
\pgfsys@transformshift{1.038580in}{1.254627in}%
\pgfsys@useobject{currentmarker}{}%
\end{pgfscope}%
\end{pgfscope}%
\begin{pgfscope}%
\definecolor{textcolor}{rgb}{0.000000,0.000000,0.000000}%
\pgfsetstrokecolor{textcolor}%
\pgfsetfillcolor{textcolor}%
\pgftext[x=1.338580in,y=1.196294in,left,base]{\color{textcolor}{\rmfamily\fontsize{12.000000}{14.400000}\selectfont\catcode`\^=\active\def^{\ifmmode\sp\else\^{}\fi}\catcode`\%=\active\def%{\%}DGC}}%
\end{pgfscope}%
\begin{pgfscope}%
\pgfsetrectcap%
\pgfsetroundjoin%
\pgfsetlinewidth{1.003750pt}%
\definecolor{currentstroke}{rgb}{0.445163,0.122724,0.506901}%
\pgfsetstrokecolor{currentstroke}%
\pgfsetdash{}{0pt}%
\pgfpathmoveto{\pgfqpoint{0.871913in}{1.022220in}}%
\pgfpathlineto{\pgfqpoint{1.038580in}{1.022220in}}%
\pgfpathlineto{\pgfqpoint{1.205247in}{1.022220in}}%
\pgfusepath{stroke}%
\end{pgfscope}%
\begin{pgfscope}%
\pgfsetbuttcap%
\pgfsetroundjoin%
\definecolor{currentfill}{rgb}{0.445163,0.122724,0.506901}%
\pgfsetfillcolor{currentfill}%
\pgfsetlinewidth{1.003750pt}%
\definecolor{currentstroke}{rgb}{0.445163,0.122724,0.506901}%
\pgfsetstrokecolor{currentstroke}%
\pgfsetdash{}{0pt}%
\pgfsys@defobject{currentmarker}{\pgfqpoint{-0.020833in}{-0.020833in}}{\pgfqpoint{0.020833in}{0.020833in}}{%
\pgfpathmoveto{\pgfqpoint{0.000000in}{-0.020833in}}%
\pgfpathcurveto{\pgfqpoint{0.005525in}{-0.020833in}}{\pgfqpoint{0.010825in}{-0.018638in}}{\pgfqpoint{0.014731in}{-0.014731in}}%
\pgfpathcurveto{\pgfqpoint{0.018638in}{-0.010825in}}{\pgfqpoint{0.020833in}{-0.005525in}}{\pgfqpoint{0.020833in}{0.000000in}}%
\pgfpathcurveto{\pgfqpoint{0.020833in}{0.005525in}}{\pgfqpoint{0.018638in}{0.010825in}}{\pgfqpoint{0.014731in}{0.014731in}}%
\pgfpathcurveto{\pgfqpoint{0.010825in}{0.018638in}}{\pgfqpoint{0.005525in}{0.020833in}}{\pgfqpoint{0.000000in}{0.020833in}}%
\pgfpathcurveto{\pgfqpoint{-0.005525in}{0.020833in}}{\pgfqpoint{-0.010825in}{0.018638in}}{\pgfqpoint{-0.014731in}{0.014731in}}%
\pgfpathcurveto{\pgfqpoint{-0.018638in}{0.010825in}}{\pgfqpoint{-0.020833in}{0.005525in}}{\pgfqpoint{-0.020833in}{0.000000in}}%
\pgfpathcurveto{\pgfqpoint{-0.020833in}{-0.005525in}}{\pgfqpoint{-0.018638in}{-0.010825in}}{\pgfqpoint{-0.014731in}{-0.014731in}}%
\pgfpathcurveto{\pgfqpoint{-0.010825in}{-0.018638in}}{\pgfqpoint{-0.005525in}{-0.020833in}}{\pgfqpoint{0.000000in}{-0.020833in}}%
\pgfpathlineto{\pgfqpoint{0.000000in}{-0.020833in}}%
\pgfpathclose%
\pgfusepath{stroke,fill}%
}%
\begin{pgfscope}%
\pgfsys@transformshift{1.038580in}{1.022220in}%
\pgfsys@useobject{currentmarker}{}%
\end{pgfscope}%
\end{pgfscope}%
\begin{pgfscope}%
\definecolor{textcolor}{rgb}{0.000000,0.000000,0.000000}%
\pgfsetstrokecolor{textcolor}%
\pgfsetfillcolor{textcolor}%
\pgftext[x=1.338580in,y=0.963887in,left,base]{\color{textcolor}{\rmfamily\fontsize{12.000000}{14.400000}\selectfont\catcode`\^=\active\def^{\ifmmode\sp\else\^{}\fi}\catcode`\%=\active\def%{\%}NC}}%
\end{pgfscope}%
\begin{pgfscope}%
\pgfsetrectcap%
\pgfsetroundjoin%
\pgfsetlinewidth{1.003750pt}%
\definecolor{currentstroke}{rgb}{0.944006,0.377643,0.365136}%
\pgfsetstrokecolor{currentstroke}%
\pgfsetdash{}{0pt}%
\pgfpathmoveto{\pgfqpoint{0.871913in}{0.789813in}}%
\pgfpathlineto{\pgfqpoint{1.038580in}{0.789813in}}%
\pgfpathlineto{\pgfqpoint{1.205247in}{0.789813in}}%
\pgfusepath{stroke}%
\end{pgfscope}%
\begin{pgfscope}%
\pgfsetbuttcap%
\pgfsetroundjoin%
\definecolor{currentfill}{rgb}{0.944006,0.377643,0.365136}%
\pgfsetfillcolor{currentfill}%
\pgfsetlinewidth{1.003750pt}%
\definecolor{currentstroke}{rgb}{0.944006,0.377643,0.365136}%
\pgfsetstrokecolor{currentstroke}%
\pgfsetdash{}{0pt}%
\pgfsys@defobject{currentmarker}{\pgfqpoint{-0.020833in}{-0.020833in}}{\pgfqpoint{0.020833in}{0.020833in}}{%
\pgfpathmoveto{\pgfqpoint{0.000000in}{-0.020833in}}%
\pgfpathcurveto{\pgfqpoint{0.005525in}{-0.020833in}}{\pgfqpoint{0.010825in}{-0.018638in}}{\pgfqpoint{0.014731in}{-0.014731in}}%
\pgfpathcurveto{\pgfqpoint{0.018638in}{-0.010825in}}{\pgfqpoint{0.020833in}{-0.005525in}}{\pgfqpoint{0.020833in}{0.000000in}}%
\pgfpathcurveto{\pgfqpoint{0.020833in}{0.005525in}}{\pgfqpoint{0.018638in}{0.010825in}}{\pgfqpoint{0.014731in}{0.014731in}}%
\pgfpathcurveto{\pgfqpoint{0.010825in}{0.018638in}}{\pgfqpoint{0.005525in}{0.020833in}}{\pgfqpoint{0.000000in}{0.020833in}}%
\pgfpathcurveto{\pgfqpoint{-0.005525in}{0.020833in}}{\pgfqpoint{-0.010825in}{0.018638in}}{\pgfqpoint{-0.014731in}{0.014731in}}%
\pgfpathcurveto{\pgfqpoint{-0.018638in}{0.010825in}}{\pgfqpoint{-0.020833in}{0.005525in}}{\pgfqpoint{-0.020833in}{0.000000in}}%
\pgfpathcurveto{\pgfqpoint{-0.020833in}{-0.005525in}}{\pgfqpoint{-0.018638in}{-0.010825in}}{\pgfqpoint{-0.014731in}{-0.014731in}}%
\pgfpathcurveto{\pgfqpoint{-0.010825in}{-0.018638in}}{\pgfqpoint{-0.005525in}{-0.020833in}}{\pgfqpoint{0.000000in}{-0.020833in}}%
\pgfpathlineto{\pgfqpoint{0.000000in}{-0.020833in}}%
\pgfpathclose%
\pgfusepath{stroke,fill}%
}%
\begin{pgfscope}%
\pgfsys@transformshift{1.038580in}{0.789813in}%
\pgfsys@useobject{currentmarker}{}%
\end{pgfscope}%
\end{pgfscope}%
\begin{pgfscope}%
\definecolor{textcolor}{rgb}{0.000000,0.000000,0.000000}%
\pgfsetstrokecolor{textcolor}%
\pgfsetfillcolor{textcolor}%
\pgftext[x=1.338580in,y=0.731480in,left,base]{\color{textcolor}{\rmfamily\fontsize{12.000000}{14.400000}\selectfont\catcode`\^=\active\def^{\ifmmode\sp\else\^{}\fi}\catcode`\%=\active\def%{\%}NC++}}%
\end{pgfscope}%
\end{pgfpicture}%
\makeatother%
\endgroup%

%    \caption{$n_{\Omega}=160$}
%    \label{fig:5-experiments-multi-matrix-convergence}
%\end{figure}
