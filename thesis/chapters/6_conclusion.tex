\chapter{Conclusion}
\label{chp:6-conclusion}

In our work we have developed a family of methods
designed for the randomized computation of the spectral density of large matrices,
which are closely related to the methods from \cite{lin2017randomized}.\\

We were able to
significantly improve many algorithmic aspects of these methods. The development
of an alternative expansion framework in \refsec{sec:2-chebyshev-interpolation}
allowed us to vastly simplify the Chebyshev expansion stage, which is common to all the
studied methods, while obtaining provable accuracy of the expansion, all this in
addition to making this stage orders of magnitude faster in most cases. 
The studied methods were made more robust through a series of well founded
implementation strategies \refsec{subsec:3-nystrom-implementation-details}.
Furthermore, we give theoretical error guarantees for all the encountered methods.
All these developments are illustrated and verified in multiple numerical experiments
which are all provably reproducible.\\

What we have noticed is that the degree of the Chebyshev expansion usually needs
to be quite high to achieve a decent interpolation accuracy, which slows down
our methods considerably. Looking out into the future, alternative ways of
computing products of matrix functions with random vectors should be taken into
consideration \cite{cortinovis2023speeding,ubaru2017lanczos}. Also, different ways
of evaluating \reflin{lin:3-nystrom-pseudo-inverse} in \refalg{alg:3-nystrom-nystrom-chebyshev}
may still improve the accuracy and stability of our methods. For example
\cite[algorithm~5.6]{tropp2023randomized} may help, but a way of making this
compatible with the fast interpolation framework has not been found yet.\\

On the theoretical side, the incorporation of the Chebyshev expansion into the
error bound analogous to \refthm{thm:2-delta-gauss-chebyshev} is of high priority.
What hinders such a unification for now is the fact that the Chebyshev expansion
is not guaranteed to be non-negative, which in turn breaks the \gls{PSD} property,
which is fundamental to most theoretical results on the Nystr\"om approximation.
A fix could be achieved by finding a way of restricting the Chebyshev expansion
to only positive values. Otherwise, the theory will have to be extended to
indefinite matrices.
